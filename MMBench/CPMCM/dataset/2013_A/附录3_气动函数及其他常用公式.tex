\section*{附录3 气动函数及其他常用公式}

\subsection*{1 气动函数}
\[
q(\lambda) = \frac{\lambda}{\sqrt{1 + \frac{\gamma-1}{2}\lambda^2}}, \quad
\pi(\lambda) = \left(1 + \frac{\gamma-1}{2}\lambda^2\right)^{\frac{\gamma}{\gamma-1}}, \quad
\tau(\lambda) = 1 + \frac{\gamma-1}{2}\lambda^2, \quad
f(\lambda) = \frac{q(\lambda)}{\pi(\lambda)}
\]
其中,$\lambda$ 表示速度系数,$\lambda > 0$。

$\lambda$ 与马赫数 $Ma$ 的关系为:
\[
\lambda^2 = \frac{\gamma Ma^2}{2 + (\gamma - 1)Ma^2}
\]
气体绝热指数:空气 $\gamma = 1.4$,燃气 $\gamma = 1.33$。

---

\subsection*{2 气体常数 $R$}
计算方法:$R$ 与空气湿度和油气比 $f$ 有关。

湿燃气的 $R$:
\[
bm = 28.9644(1 - 0.0308764f), \quad
mg = \frac{(1+f)}{(1+f)/d + f/18.01534}, \quad
R = \frac{8314.298}{mg}
\]
其中,$f$ 为湿燃气的油气比,$d$ 为湿度。

简化计算:在进气道、风扇、压气机、CDFS和各个涵道中,$f=0$,因此 $R=287\ \mathrm{J/(kg \cdot K)}$;在燃烧室之后的部件中,$R=287.31\ \mathrm{J/(kg \cdot K)}$。

---

\subsection*{3 流量系数 $k_m$}
流量公式:
\[
\dot W_a = k_m A \frac{p^*}{\sqrt{T^*}} q(\lambda)
\]

流量系数:
\[
k_m = \sqrt{\frac{\gamma}{R}\left(\frac{2}{\gamma+1}\right)^{\frac{\gamma+1}{\gamma-1}}}
\]

空气:$\gamma=1.4, R=287 \quad \Rightarrow k_m=0.0404$

燃气:$\gamma=1.33, R=287.31 \quad \Rightarrow k_m=0.0397$

---

\subsection*{4 基本概念}
\textbf{压气机增压比}:该级出口气流的总压与进口气流的总压之比。

\textbf{压气机效率}:相同条件下,达到同样的增压比时,等熵压缩所需功与实际消耗功之比。

\textbf{压气机压比函数值}:
设某换算转速对应的增压比数据最大值 $pr_{max}$,最小值 $pr_{min}$,则定义:
\[
zz = \frac{pr - pr_{min}}{pr_{max} - pr_{min}}
\]

\textbf{涡轮压比函数值}:
设某换算转速对应的落压比最大值 $pr_{max}$,最小值 $pr_{min}$,则定义:
\[
zz = \frac{pr - pr_{min}}{pr_{max} - pr_{min}}
\]

\textbf{涵道比}:
\[
BPR = \frac{\text{总外涵道空气流量}}{\text{内涵道空气流量}}
\]

\textbf{前涵道比}:
\[
FPR = \frac{\text{CDFS空气流量}}{\text{副外涵道空气流量}}
\]

---

\subsection*{5 涡轮风扇发动机与涡喷发动机}
涡轮风扇发动机。涡轮风扇发动机简称为涡扇发动机。涡扇发动机的突出特点是气体在发动机中的流动部分地或全部地经历内,外两个通道,又称内涵和外涵。其中流过外涵的空气流量与流过内涵的空气流量之比称为涵道比。在涡扇发动机中,空气经进气系统首先进入风扇(又称为低压压气机)增压,而后分成内、外两股气流。外股气流进入外涵道;内股气流进入内涵道,经历与涡喷发动机类似的过程。
涡轮喷气发动机简称涡喷发动机。发动机工作时,外界空气经进气系统引入发动机,经压气机增压后进入燃烧室,在燃烧室中与供给的燃料混合并燃烧,形成高温高压的燃气,燃气在涡轮中膨胀,推动涡轮旋转,从而驱动压气机工作。燃气发生器燃气的可用能量全部用于在排气系统中增加燃气的动能,使燃气以很高的速度排出,以产生推力。
在涡轮后带有复燃加力燃烧室的涡轮喷气发动机称为复燃加力式涡轮喷气式发动机,简称加力涡喷发动机。
