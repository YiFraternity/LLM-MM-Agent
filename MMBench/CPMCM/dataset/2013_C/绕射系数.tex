\documentclass[12pt]{article}
\usepackage{ctex}            % 中文支持(如用 XeLaTeX/LuaLaTeX 编译)
\usepackage{amsmath,amssymb}
\usepackage{physics}
\usepackage{bm}
\usepackage{geometry}
\geometry{a4paper,margin=1in}

\title{绕射系数(从 PDF 逆转义整理稿)}
\author{}
\date{}

\begin{document}
\maketitle

\section*{说明}
本文档由给定 PDF 中的内容逆向整理而成。为可编译起见,文中公式统一放入
\LaTeX{} 数学环境;对个别在原 PDF 中出现的字符丢失/错位之处,使用了占位符并附注释,便于后续替换为精确表达式。:contentReference[oaicite:1]{index=1}

\section{劈的边缘绕射与绕射系数}
劈的边缘绕射问题如图所示(示意图见原文)。设入射方向单位矢量为 $\bm I$,绕射点处切线为 $\bm e$。
软边界绕射系数(入射电场平行于 $\bm s'\!$ 与 $\bm e$ 构成的平面)记为 $D_s$;
硬边界绕射系数(入射电场平行于 $\bm s'\!$ 与 $\bm e$ 构成的平面)记为 $D_h$。
题设入射电场判断应采用软边界绕射系数。【6†source}

在题设的两种情形合并下,绕射系数(记 $D$)形式为(原式(1)):\footnote{原 PDF 中部分字符存在错位;下式已按常见 UTD 记号整理排版,但仍保留若干占位符以示不确定处。}
\begin{align}
D \;=\;
\frac{e^{\,j\pi/4}}{2}\,
\Biggl[
&\; \underbrace{\cot\!\qty(\tfrac{\pi - (\beta - \beta_0)}{2}) \, F\!\bigl(kL\,a_{-}\bigr)}_{\text{项A}}
\;-\;
\underbrace{\cot\!\qty(\tfrac{\pi - (\beta + \beta_0)}{2}) \, F\!\bigl(kL\,a_{+}\bigr)}_{\text{项B}}
\Biggr]
\nonumber\\[2pt]
&\quad +\;
\Biggl[
\underbrace{\cot\!\qty(\tfrac{\pi + (\beta - \beta_0)}{2}) \, F\!\bigl(kL\,b_{-}\bigr)}_{\text{项C}}
\;-\;
\underbrace{\cot\!\qty(\tfrac{\pi + (\beta + \beta_0)}{2}) \, F\!\bigl(kL\,b_{+}\bigr)}_{\text{项D}}
\Biggr]
\;\Bigg/\, \bigl(\,\text{归一化因子}\,\bigr),
\label{eq:D-generic}
\end{align}
其中上式的四个过渡函数自变量
$a_{\pm}, b_{\pm}$ 在原 PDF 中有排版破损;此处以占位符表示(请据你的原题或图 4
中的几何定义补全)。若题设为 $ \beta_0 = 90^\circ$,则 $\sin\beta_0 = 1$。:contentReference[oaicite:2]{index=2}

\paragraph{记号与常量:}
\begin{align}
k &= \frac{2\pi}{\lambda}, \qquad \text{波数($\lambda$ 为波长)}, \label{eq:k-def}\\
L &:\; \text{绕射点到场点的距离},\\
n &= \frac{3\pi}{2} - \alpha
\quad\text{(原文作 $ 3\alpha $ 的定义如图 4 所示,单位弧度;此处据 PDF 文字保留)}. \label{eq:n-def}
\end{align}
:contentReference[oaicite:3]{index=3}

\section{过渡函数 $F(x)$ 的定义与近似}
为修正 Keller 非一致性解,采用过渡函数 $F(x)$(菲涅耳积分的变形)。定义为(原式(2)):
\begin{equation}
F(x)
= 2 \int_{x}^{+\infty}
\exp\!\qty(-j\tau^{2})\,\exp\!\qty(j x \tau)\, d\tau,
\label{eq:F-def}
\end{equation}
其中被积函数与积分上下限形式来自原 PDF 文字;若与你的教材/讲义记号不同,请据需要替换为等价的标准定义。:contentReference[oaicite:4]{index=4}

\subsection*{小 $x$ 展开(原式(3))}
当 $x$ 很小时,有近似展开:
\begin{equation}
F(x) \approx
2 e^{-j\pi/4}
\;-\; c_1\, j\, x
\;-\; c_2\, x^{2}
\;+\; \mathcal{O}(x^{3}),
\quad
\bigl(c_1,c_2 \text{ 为与 $\pi$ 因子相关的常数;原 PDF 中字符错位,这里以占位符示意}\bigr).
\label{eq:F-smallx}
\end{equation}
:contentReference[oaicite:5]{index=5}

\subsection*{大 $x$ 展开(原式(4))}
当 $x$ 很大时,原文给出的渐近展开可表为:
\begin{equation}
F(x) \approx
1 \;+\; \frac{j}{2x}\;-\; \frac{3}{4x^{2}}
\;-\; \frac{15j}{8x^{3}}
\;+\; \frac{75}{16x^{4}}
\;+\; \cdots,
\label{eq:F-largex}
\end{equation}
(系数顺序与正负号依原 PDF 文本整理;若与常见文献符号不同,请以课程指定版本为准)。:contentReference[oaicite:6]{index=6}

\subsection*{变形积分(原式(5)\,(6))}
为数值计算(上限 $+\infty$ 不便直接求积),原文建议使用高斯型积分恒等式:
\begin{equation}
\int_{-\infty}^{+\infty}
\exp\!\qty(-a u^{2} + b u)\, du
= \sqrt{\frac{\pi}{a}}\;\exp\!\qty(\frac{b^{2}}{4a}),
\qquad \Re(a) > 0,
\label{eq:gauss}
\end{equation}
据此可把 \eqref{eq:F-def} 变形为适合数值实现的形式(原式(6)),例如
\begin{equation}
F(x) = e^{-j\pi/4}
\int_{0}^{+\infty} \exp\!\qty(-j\tau^{2})\,\exp\!\qty(-j x \tau)\, d\tau,
\label{eq:F-num}
\end{equation}
具体系数与相位因子依所用记号体系而定。该式与\eqref{eq:F-def}等价(在原 PDF 记号下)。:contentReference[oaicite:7]{index=7}

\section{几何关系与整数 $N_{\pm}$(原式(7)--(9))}
设 $\alpha_1,\alpha_2$ 分别为入射角与绕射角(以劈上一边为参考,定义同图 4),令
\begin{equation}
\beta_{\pm} = \beta \pm \alpha, \qquad
\text{并定义}\quad
\frac{2}{n} \cos\!\qty(\frac{\pi(\beta \mp \alpha)}{2})
= \text{(原 PDF 中出现的比值;此处按文字保留)},
\label{eq:beta-pm}
\end{equation}
再令 $N_{\pm}$ 为最接近满足下列方程的整数(原式(8)\,(9)):
\begin{align}
2 n\,\beta_{+} + \pi N_{+} &= \pi,
\label{eq:Nplus}\\
2 n\,\beta_{-} + \pi N_{-} &= -\,\pi.
\label{eq:Nminus}
\end{align}
上述关系用于确定 \eqref{eq:D-generic} 中四个过渡函数的取值分支。:contentReference[oaicite:8]{index=8}

\section{数值实现分段(原文建议)}
过渡函数 $F(x)$ 在数值实现时,原文建议采用如下分段(按 $x$ 的大小):
\begin{itemize}
  \item 当 $0 \le x < 10^{-3}$,采用小 $x$ 近似式 \eqref{eq:F-smallx};
  \item 当 $10^{-3} \le x \le 10$,采用变形公式(如 \eqref{eq:F-num})进行数值积分;
  \item 当 $x > 10$,采用大 $x$ 渐近式 \eqref{eq:F-largex}。
\end{itemize}
:contentReference[oaicite:9]{index=9}

\section*{备注与占位符清单}
由于原 PDF 在公式处存在若干字符丢失/换行错位,以下符号以占位表示,请按你的题设或教材公式替换:
\begin{itemize}
  \item \(\beta_0\)、\(a_{\pm}\)、\(b_{\pm}\) 的精确定义(与图 4 对应的几何量);
  \item \eqref{eq:D-generic} 中分母“归一化因子”的具体形式(常见为与 $n,\;k,\;L$ 等相关的幅度因子);
  \item \eqref{eq:F-smallx} 小 $x$ 展开中的常数系数(原 PDF 中该行字符重叠)。
\end{itemize}
为确保与你的课程版本一致,请将以上占位处替换为明确表达式后再使用。

\vfill
\begin{center}
\small 本文为从用户提供 PDF 逆转义的 \LaTeX{} 可编译整理稿。来源见文内标注。:contentReference[oaicite:10]{index=10}
\end{center}

\end{document}
