\begin{center}
\textbf{第十届华为杯全国研究生数学建模竞赛}
\end{center}

\begin{tabular}{l l}
学校 & 华中农业大学 \\
\hline
参赛队号 & 10504008 \\
\hline
队员姓名 & 1. 李海燕 \\
 & 2. 胡曹园 \\
 & 3. 黄小明 \\
\hline
\end{tabular}

\begin{flushright}
参赛密码 \underline{\hspace{2cm}} \\
(由组委会填写)
\end{flushright}

\begin{center}
\textbf{参赛密码} \underline{\hspace{3cm}} \\
(由组委会填写)
\end{center}

\begin{center}
\includegraphics[width=0.3\textwidth]{image1.png} \quad
\includegraphics[width=0.3\textwidth]{image2.png} \quad
\includegraphics[width=0.3\textwidth]{image3.png}
\end{center}

\begin{center}
\textbf{第十届华为杯全国研究生数学建模竞赛}
\end{center}

\begin{flushleft}
\textbf{题目} \quad 基于二维多箱模型的PM2.5扩散问题研究
\end{flushleft}

\begin{center}
\textbf{摘要:}
\end{center}

随着我国新修订的《环境空气质量标准》的实施和PM2.5对人民生活健康影响越来越大,人们也越来越关注PM2.5。由于目前对PM2.5的研究较少,需要我们千方百计利用现有的资源开展探索性研究。

本文首先就西安市空气中PM2.5建立非稳态二维多箱扩散衰减模型,分析PM2.5的扩散规律。然后对PM2.5突发情况进行分析,建立了高斯扩散衰减模型。最后根据实际问题分析,给出武汉市未来五年PM2.5治理计划。

对于问题一,我们从相关性与独立性两方面出发,利用秩和检验和独立性检验,对PM2.5、$SO_{2}$、$NO_{2}$、PM10、CO、$O_{3}$六个污染物两两之间的相关性和独立性进行检验;并利用逐步回归得到PM2.5与其他各污染物之间的函数关系,得到PM2.5与$NO_{2}$、PM10、CO呈正相关,与$O_{3}$呈负相关。额外的,我们从附件1中发现日均最高温、最低温与PM2.5具有强相关性。

对于问题二,我们先从PM2.5浓度单日变化、月份变化、季节变化和空间位置规律四个层面对其时空分布规律进行分析与污染评价。

其次,将西安市均匀分为多个子箱,综合考虑风向、天气等因素的影响,建立非稳态二维多箱扩散衰减模型。然后结合实际情况对模型进行定性分析。以西安市某一天PM2.5的浓度数据为输入样本,根据模型模拟得到PM2.5的扩散衰减结果。模拟结果的平均相对误差为8.7%,说明模型能够较好预测西安市PM2.5扩散情况。

再次,对市人民体育场突发情形下PM2.5的污染扩散进行预测与评估,建立高斯点源扩散衰减模型。以该地区PM2.5监测数据最高的一天为例,得到在突发情况下经开区、小寨等八个区为重污染区,临潼等四个地区相对安全。

最后对非稳态二维多箱扩散衰减模型预测PM2.5浓度与实测浓度进行比较,得到平均相对误差为25.2%,说明模型较为合理。

对于问题三,根据污染物浓度越低越难治理、以及PM2.5的复杂性等实际情况,

在修正指数模型下得出武汉市未来 5 年 PM2.5 年终平均浓度降至 $35 \mu \mathrm{g} / \mathrm{m}^{3}$ 的治理方案,每年年终平均治理指标分别为 $0 \mu \mathrm{g} / \mathrm{m}^{3}$、$127 \mu \mathrm{g} / \mathrm{m}^{3}$、$66 \mu \mathrm{g} / \mathrm{m}^{3}$、$34 \mu \mathrm{g} / \mathrm{m}^{3}$、$18 \mu \mathrm{g} / \mathrm{m}^{3}$,其中第一年为项目前期准备阶段,PM2.5 浓度无下降。

在考虑完成治理目标、费用最少且方案具有可行性等条件下,建立数学优化模型,得到武汉市未来 5 年内 PM2.5 年终平均下降浓度分别为 $0 \mu \mathrm{g} / \mathrm{m}^{3}$、$121 \mu \mathrm{g} / \mathrm{m}^{3}$、$71 \mu \mathrm{g} / \mathrm{m}^{3}$、$40 \mu \mathrm{g} / \mathrm{m}^{3}$、$13 \mu \mathrm{g} / \mathrm{m}^{3}$,综合治理费用为 245 百万元,专项治理费用为 107.255 百万元,最后阐述了该方案的合理性。

本文最后对模型的优缺点做出了客观评价,并提出了模型改进思路。

关键字:PM2.5;二维多箱模型;高斯扩散衰减模型;目标优化

\section*{目录}

\section*{1 问题来源}
\dotfill -5-

\section*{2 模型假设}
\dotfill -6-

\section*{3 符号说明}
\dotfill -6-

\section*{4 第一问 模型建立与求解}
\dotfill -7-

\subsection*{4.0 问题分析}
\dotfill -7-

\subsection*{4.1 PM2.5 与 $SO_{2}$、$NO_{2}$、可吸入颗粒物、CO、$O_{3}$ 之间的相关性}
\dotfill -7-

\subsection*{4.2 PM2.5 与 $SO_{2}$、$NO_{2}$、可吸入颗粒物、CO、$O_{3}$ 之间的独立性}
\dotfill -8-

\subsection*{4.3 PM2.5 与 $SO_{2}$、$NO_{2}$、可吸入颗粒物、CO、$O_{3}$ 的回归模型验证}
\dotfill -8-

\subsection*{4.4 PM2.5 强相关的要素}
\dotfill -9-

\section*{5 第二问 模型建立与求解}
\dotfill -11-

\subsection*{5.0 问题分析}
\dotfill -11-

\subsection*{5.1 PM2.5 的时空分布规律及污染评估}
\dotfill -12-

\subsubsection*{5.1.1 PM2.5 的时间变化特征}
\dotfill -13-

\subsubsection*{5.1.2 PM2.5 的空间分布特征}
\dotfill -16-

\subsubsection*{5.1.3 PM2.5 时空变化规律及污染评估}
\dotfill -18-

\subsection*{5.2 基于非稳态二维多箱的 PM2.5 浓度扩散模型}
\dotfill -19-

\subsubsection*{5.2.1 PM2.5 的发生机理与转换关系}
\dotfill -19-

\subsubsection*{5.2.2 PM2.5 的扩散衰减模型}
\dotfill -20-

\subsubsection*{5.2.3 基于非稳态二维多箱模型的 PM2.5 扩散衰减分析}
\dotfill -25-

\subsection*{5.3 突发情形下的 PM2.5 污染扩散预测模型}
\dotfill -28-

\subsubsection*{5.3.1 高斯点源扩散衰减模型}
\dotfill -28-

\subsubsection*{5.3.2 点污染源的选取与假设}
\dotfill -29-

\subsubsection*{5.3.3 突发情形下高斯点源扩散模型对 PM2.5 的预测}
\dotfill -30-

\subsection*{5.4 PM2.5 污染扩散预测模型的合理性检验}
\dotfill -31-

\section*{6 第三问 模型建立与求解}
\dotfill -33-

\subsection*{6.0 问题分析}
\dotfill -33-

\subsection*{6.1 未来五年减少武汉市 PM2.5 浓度的治理方案}
\dotfill -33-

\subsubsection*{6.1.1 降低 PM2.5 浓度的主要方法及存在缺陷}
\dotfill -33-

\subsubsection*{6.1.2 降低 PM2.5 浓度的主要治理途径}
\dotfill -34-

\subsection*{6.2 武汉市未来 5 年治理方案投入的经费预算}
\dotfill -35-

\subsubsection*{6.2.1 武汉市降低 PM2.5 浓度的目标优化}
\dotfill -35-

\subsubsection*{6.2.2 目标优化模型求解}
\dotfill -36-

\subsubsection*{6.2.3 治理方案有效性分析}
\dotfill -37-

\section*{7 模型检验与评价}
\dotfill -37-

\subsection*{7.1 模型的优点}
\dotfill -38-

\subsection*{7.2 模型的改进}
\dotfill -38-

\section*{参考文献}
\dotfill -38-

\section*{基于二维多箱模型的 PM2.5 扩散问题研究}

\section*{1 问题来源}

空气质量是衡量一个城市生活水平的重要指标。近年来,随着经济的快速发展,文化的不断繁荣,市区内机动车数量的不断增加,化石燃料的持续燃烧,以及道路和建筑施工扬尘、工业粉尘、餐饮油烟等气体污染源的肆意排放,直接导致空气中的粉尘颗粒数量的不断增加,灰霾程度和持续的天数也在不断攀升。以 2013 年年初华北平原、黄淮、江淮、江汉、江南、华南北部等地区的灰霾天气为例,持续的天数、灰霾的严重程度,以及对周围环境、百姓生活的影响都是极其严重的。此次会买天气波及国内 $1 / 4$ 的土地,受影响人口达 6 亿人。由此可看,灰霾等空气质量问题已经成为中央政府、环境保护部门以及百姓自身热切关注的重要问题。

全国大范围灰霾天气的主要成因是空气中颗粒粉尘等物质的含量过高。为了进一步提高对空气质量的监控、预报和控制,中央和地方政府不断调整出台响应的政策法规,并于 2012 年 2 月 29 日公布了重新修订的《环境空气质量标准》(GB3095—2012),新的空气质量标准对环境空气的功能区进行了重新分类,将之前的空气污染指数(API)改为了新的空气质量指数(AQI),在二氧化硫、二氧化氮和可吸入颗粒物 PM10 等基本指标的基础上,新增细颗粒物 PM2.5、臭氧和一氧化碳 CO 三项指标,并在日常空气质量检测中增设了细微颗粒(PM2.5)浓度限值和臭氧 8 小时平均浓度限值,并调整了可吸入颗粒、二氧化氮、铅和苯并(a)芘(黑碳)等指标的浓度限值。从空气质量标准的修订结果来看,环保部门首次将导致灰霾天气产生——对人类健康危害极大的细颗粒物 PM2.5 作为空气质量监测的指标。细微颗粒(PM2.5)具体指的是空气动力学当量直径小于或等于 $2.5 \mu \mathrm{m}$ 的大气颗粒物,由于其对人类的健康和植物的生长影响巨大,早已被纳入了美国、加拿大、日本、印度等国的常规监测体系。

由于细颗粒物 PM2.5 才纳入到空气质量检测中来,公众的认知时限较短,在学术界的研究也比较滞后,相关研究和监测的数据较少,导致对细微颗粒的成因、变化及一般规律的研究也不够深入。但是它的危害有目共睹,对生态环境和人类健康的影响不可避免,所以需要借助有限的数据去对 PM2.5 进行如下分析。首先,进行相关因素分析,尽可能找出现有 AQI 检测指标之外易于观测,且与 PM2.5 强相关的成分要素;其次,分析已有数据中 PM2.5 的时空分布规律,评估 PM2.5 的污染程度。综合考虑风力、湿度等天气和季节因素的影响,分析 PM2.5 的发生和演变规律,建立相关扩散和衰减模型。针对 PM2.5 的浓度短期激增,延续数小时等突发状况,给出污染扩散与评价方法,再对已有评价结论,给出进一步恶劣情况的预防和安置方案,并检验模型的预测效果。最后,合理的给出能够降低某地区 PM2.5 浓度的治理方案,在尽量减少综合治理费用和专项治理费的前提下,输出投入经费的总预算和分预算,并有效合理的解释方案。

\section{模型假设}

(1) 每个站点假设成为一个单箱,每个单箱的浓度与每个站点周围的浓度相一致;

(2) PM2.5 浓度是均匀分布在每个单箱内的;

(3) PM2.5 污染物在每个二维多箱内的子箱中混合均匀;

(4) 忽略每个子箱中的化学反应以及 PM2.5 的干、湿沉积;

(5) 某一时刻的风矢量不随水平位置的不同而改变;

(6) 题目中所列数据真实可靠且具有较强的代表性;

(7) 不考虑软件工具在数据处理及图形绘制中的误差;

\section{符号说明}

\begin{tabular}{ll}
\hline 符号 & 符号说明 \\
\hline $Q$ & 污染源强度 \\
$u_{x}$ & 子箱 $b_{i,j}$ 在 $X$ 轴方向上的平均风速分量 \\
$u_{y}$ & 子箱 $b_{i,j}$ 在 $Y$ 轴方向上的平均风速分量 \\
$d$ & 污染源距离一个站点的距离 \\
$e$ & 平方向上的 PM2.5 扩散系数 \\
$C_{0}$ & 输入箱体的 PM2.5 浓度 \\
$C$ & 箱内 PM2.5 的浓度 \\
$W$ & 子箱宽度,定义西安城区的东西方向为款 \\
$L$ & 子箱 $b_{i,j}$ 的长度,定义西安城区的南北方向为长 \\
\hline
\end{tabular}

\section*{4 第一问 模型建立与求解}

\section*{问题一}

(1) 依据附件 1 或附件 2 中的数据或自行采集数据,利用或建立适当的数学模型,对 AQI 中 6 个基本监测指标的相关与独立性进行定量分析,尤其是对其中 PM2.5 浓度与其它 5 项分指标及其对应污染物浓度之间的相关性及其关系进行分析。

(2) 如果能够发现 AQI 基本监测指标以外的、可监测的、且与 PM2.5 强相关的成分要素,请陈述发现的方法、定量分析的结果、数据及来源。

\subsection*{4.0 问题分析}

对空气质量指数中监测指标的相关性与独立性进行分析的方法是多种多样的,从思路上,我们引用秩和检验分析,来测量 6 个变量两两组合后的相关关系。再运用独立性检验来对 PM2.5 浓度与其它 5 项基本监测指标之间的独立性进行检验。

灰霾天气除了有 PM2.5 之外,还受到如温度、湿度、风向、PM2.5 距离地面高度等多种因素的影响。在对 AQI 基本监测指标的相关性与独立性进行定量分析之后,依据已有的、和可查到的天气数据,尝试着寻找与 PM2.5 强相关的(可监测的)要素。

\section*{数据处理:}

利用附件 1,武汉一个监测的数据为例,监测日期是 2013.01.01 至 2013.08.26。首先对样本数据进行统计,如果某一天空气质量很好,首要污染源为空的,则标记为 0。

\subsection*{4.1 PM2.5 与 $SO_{2}$、$NO_{2}$、可吸入颗粒物、CO、$O_{3}$ 之间的相关性}

利用秩和检验对 PM2.5 浓度与其它 5 项 AQI 基本监测指标进行相关关系分析。利用 SPSS 软件,以武汉市 PM2.5 浓度数据为例,秩和检验可以针对不服从正态分布的样本数据的相关性进行检验,保证检验结果的有效性。检验结果如下表 4-1 所示:

\begin{table}[h]
\centering
\caption{武汉市内 PM2.5 与其他变量的相关系数}
\begin{tabular}{c c c c c c c}
\hline
相关系数 & $SO_{2}$ & $NO_{2}$ & 可吸入颗粒物 & CO & $O_{3}$ & PM2.5 \\
\hline
$SO_{2}$ & 1.0000 & 0.7986 & 0.7646 & 0.6957 & -0.1708 & 0.7233 \\
$NO_{2}$ & 0.7986 & 1.0000 & 0.8099 & 0.7271 & -0.1218 & 0.7838 \\
可吸入颗粒物 & — & — & 1.0000 & 0.6767 & -0.0510 & 0.8389 \\
CO & — & — & — & 1.0000 & -0.4129 & 0.8755 \\
$O_{3}$ & — & — & — & — & 1.0000 & -0.3565 \\
PM2.5 & — & — & — & — & — & 1.0000 \\
\hline
\end{tabular}
\end{table}

表 4-1 中所有变量之间的相关系数校验值均小于 0.001。其中,PM2.5 浓度与基本监测指标中 CO 浓度的正向相关性最高,为 0.8755;其次是可吸入颗粒、NO2、SO2,

\section*{4.2 PM2.5 与 $SO_{2}$、$NO_{2}$、可吸入颗粒物、CO、$O_{3}$ 之间的独立性}

利用卡方检验对 AQI 中 6 项基本监测指标之间的独立性进行检验,仍以武汉市某监测点数据为例,时间为 2013.01.01 至 2013.08.26,得到独立性检验的结果如表 4-2 所示:

\begin{table}[h]
\centering
\caption{武汉市内 6 项空气质量监测指标的独立性检验}
\begin{tabular}{l c c c c c c}
\hline
独立性检验 & $SO_{2}$ & $NO_{2}$ & 可吸入颗粒物 & CO & $O_{3}$ & PM2.5 \\
\hline
卡方 & 120.992 & 83.882 & 125.025 & 191.731 & 108.286 & 104.126 \\
df & 59 & 71 & 99 & 53 & 100 & 122 \\
渐进显著性 & 0.000 & 0.141 & 0.040 & 0.000 & 0.269 & 0.877 \\
\hline
\end{tabular}
\end{table}

从卡方独立性检验结果来看,变量 $SO_{2}$、可吸入颗粒物、CO 的检验值在 0.05 水平下显著,说明这三个变量之间相互不独立的,而变量 $NO_{2}$、$O_{3}$ 和 PM2.5 之间则是相互独立的。

\section*{4.3 PM2.5 与 $SO_{2}$、$NO_{2}$、可吸入颗粒物、CO、$O_{3}$ 的回归模型验证}

由于 PM2.5 与 SO2、NO2、可吸入颗粒物、CO、O3 等指标之间具有一定的相关性和独立性,所以我们建立回归模型验证它们之间的函数关系。以武汉市 2013.01.01 至 2013.08.26 期间某监测点数据为例,因变量为 PM2.5 浓度($Y$),自变量为 SO2($X_{1}$)、NO2($X_{2}$)、可吸入颗粒物($X_{3}$)、CO($X_{4}$)、O3($X_{5}$),构造多元回归方程。结果如表 4-3 所示:

\begin{table}[h]
\centering
\caption{武汉市 PM2.5 浓度的逐步回归模型}
\begin{tabular}{l c c c c}
\hline
Parameter & Standard & & & \\
\hline
变量 & 系数 & 误差 & t 值 & $Pr > |t|$ \\
\hline
常系数 & -32.9559 & 7.1738 & -4.59 & $<.0001$ \\
$SO_{2}$ & 0.1733 & 0.2047 & 0.85 & 0.3981 \\
$NO_{2}$ & 0.4418 & 0.1547 & 2.86 & 0.0047 \\
可吸入颗粒物 & 0.8820 & 0.0921 & 9.58 & $<.0001$ \\
CO & 2.2071 & 0.2135 & 10.34 & $<.0001$ \\
$O_{3}$ & -0.3339 & 0.0662 & -5.05 & $<.0001$ \\
\hline
\end{tabular}
\end{table}

把回归方程的 t 检验结果来看,SO2 对 PM2.5 浓度的影响并不显著,采用逐步回归方法,提出不显著变量,得到剩余 4 个变量对 PM2.5 浓度的影响,均在 1\% 的水平下显著,则最优回归方程为:

\begin{equation}
Y = -32.9559 + 0.4418X_{2} + 0.8820X_{3} + 2.2071X_{4} - 0.3339X_{5}
\tag{4-1}
\end{equation}

多元回归方程的 $F$ 统计量满足 $F > F_{0.05}$,因此回归方程在 0.05 的 $\alpha = 0.05$ 显著水平下成立。

\section*{4.4 PM2.5 强相关的要素}

PM2.5 成分比较复杂,形成机理比较繁琐,来源也是多种多样,土壤扬尘、森林火灾、火山爆发、化石燃料燃烧、工业尾气排放、机动车排放等均有可能会造成高浓度灰霾天气出现。由此可见,气温、降水、风向、风速、气压等天气因素具有可能导致 PM2.5 浓度的上升。

介于调查数据的局限性,本文只验证了除 AQI 基础监测指标之外的,与 PM2.5 浓度强相关的,易于观测的一项指标——气温。

已知数据中只有西安地区给出了气象数据,时间是 2011 年 1 月 1 日至 2013 年 4 月 28 日。以西安市内全市 PM2.5 数据和平均气温数据为例,比较分析日均气温中的最高温度、最低温度和 PM2.5 之间的相关性和独立性检验,结果如下表 4-4、表 4-5 所示:其中日均最高温和最低温与 PM2.5 之间均呈显著负相关,在 0.05 水平下显著,对应相关系数为 -0.6780 和 -0.6811。而最高气温与最低气温之间具有显著正相关。

\begin{table}[h]
\centering
\caption{西安市,最高温、最低温与 PM2.5 浓度的相关性检验}
\begin{tabular}{c c c c}
\hline
相关系数 & 最高温 & 最低温 & PM2.5 \\
\hline
最高温 & 1.0000 & 0.9496 & -0.6780 \\
最低温 & 0.9496 & 1.0000 & -0.6811 \\
PM2.5 & -0.6780 & -0.6811 & 1.0000 \\
\hline
\end{tabular}
\end{table}

日均最高温和最低温与 PM2.5 浓度之间具有显著负相关。这表明,日最高温和最低温越高,则 PM2.5 浓度越低。夏季日均气温高,则 PM2.5 扩散速度较快,PM2.5 浓度会偏低,冬季日均气温较低,城市处在采暖区,燃煤取暖、机动车尾气、建筑业扬尘等活动增加了 PM2.5 的浓度,气温低抑制了 PM2.5 的扩散速度,最终污染物 PM2.5 聚集在城市上空,导致 PM2.5 浓度严重超标。

对西安市 2011 年 1 月 1 日至 2013 年 4 月 28 日的全市 PM2.5 浓度与日均最高温和最低温之间做独立性检验。

\begin{table}[h]
\centering
\caption{西安市,最高温、最低温与 PM2.5 浓度的独立性检验}
\begin{tabular}{c c c c}
\hline
独立性检验 & 最高温 & 最低温 & PM2.5 \\
\hline
卡方 & 78.825 & 57.600 & 87.658 \\
df & 38 & 35 & 136 \\
渐近显著性 & 0.000 & 0.009 & 1.000 \\
\hline
\end{tabular}
\end{table}

从独立性检验结果来看,PM2.5 与最高温、最低温的独立性检验 P 值均小于 0.05,说明 PM2.5 与这两个因素不是相互独立的,即 PM2.5 与这两个因素存在相关性。

气温数据是天气预报中最常见的检测指标,具有易于观测、与 PM2.5 强相关的显著特点,所以利用日均最高温和最低温与 PM2.5 浓度的关系来检测预报为来的天气状况,方便快捷。

通过上述分析,第一问中得到以下结论:

(1) 采用秩和检验模型对 PM2.5 与 SO2、NO2、可吸入颗粒物、CO、O3 等指标的相关性进行分析,结果 PM2.5 指数与 CO、可吸入颗粒、NO2、SO2 指数依次呈正向相关,与 O3 呈典型负相关;

(2) 利用独立性检验来对 PM2.5 与 SO2、NO2、可吸入颗粒物、CO、O3 等指标之间的独立性进行检验。其中,变量 $SO_2$、可吸入颗粒物、CO 的检验值在 0.05 水平下显著,说明这三个变量之间相互不独立的,而变量 $NO_2$、O3 和 PM2.5 之间则是相互独立的;

(3) 为了进一步验证 PM2.5 与 SO2、NO2、可吸入颗粒物、CO、O3 等指标的函数关系,我们采用逐步回归来验证。逐步回归的结果显示,CO 对 PM2.5 的正向影响最大,其次是可吸入颗粒、NO2,O3 的反向影响最显著,SO2 对 PM2.5 的影响则不显著;

(4) 通过西安市日均最高气温、日均最低气温与 PM2.5 数据的分析,可知:日均最高气温、日均最低气温与 PM2.5 呈显著强相关,且日均气温数据在空气质量观测中可监测,易于观察,操作方便,可以用作显示 PM2.5 信息的指标。

\section*{5 第二问 模型建立与求解}

\section*{问题二:}

1. 描述已知地区内 PM2.5 的时空分布及其规律,并结合新修订的《环境空气质量标准》进行污染评估。

2. 建立能够刻画已知地区 PM2.5 的扩散与衰减规律的数学模型,综合考虑风力、湿度等天气和季节因素的影响,并利用已知数据进行定量与定性分析。

3. 假设已知地区某监测点处的 PM2.5 的浓度突然增至数倍,且延续数小时,建立针对这种突发情形的污染扩散预测与评估方法。并以该地区 PM2.5 监测数据最高的一天为例,在全地区 PM2.5 浓度最高点处的浓度增至 2 倍,持续 2 小时,利用建立的模型进行预测评估,给出重度污染和可能安全区域。

4. 采用适当方法检验所建立模型和方法的合理性,并根据已有研究成果探索 PM2.5 的成因、演变等一般性规律。

\subsection*{5.0 问题分析}

第二问描述 PM2.5 的分布与演变规律,给出合理的应急处理政策。

首先,对已知数据进行有效的补充,并从时间和空间上对 PM2.5 的分布规律进行定性和定量的描述。

其次,建立非稳态二维多箱模型,综合考虑了风向、季节因素对 PM2.5 浓度扩散变化的影响,对 PM2.5 的扩散和衰减分布进行模拟预测,主要针对面源污染扩散的预测。并以西安市 13 个站点任意一天的 PM2.5 浓度数据为例,进行数据模拟,得到 PM2.5 的扩散预测结果,并对同一天的实际观测数据和模拟预测数据之间的相对误差进行检验,评价模型的有效性。

再次,针对 PM2.5 浓度激增的突发情况建立扩散预测与评价模型,并利用所建模型,对已知 PM2.5 浓度最高点处的浓度增至 2 倍的未知情况,进行预测评估。

最后,对所建评价模型的合理性进行评价,并分析 PM2.5 的成因、演变等一般性规律。

\section*{数据处理:}

附件中 PM2.5 的数据相对较少,只有 2013 年 1 月 1 日到 2013 年 4 月 26 日期间西安市 13 个不同监测站 6 项基础监测指标的信息,信息含量非常少,不利于问题的了分析。本文从附录网站上获得最新更新的数据,将西安市全市 PM2.5 浓度数据、13 个监测站点的 PM2.5 浓度数据均更新到 2013 年 9 月 19 日。

针对 2013 年 1 月-9 月的数据中,缺省值采用拉格朗日差值对其进行补充,对已有的这部分数据进行统计,发现其中部分 PM2.5 指数存在异常值,但是为了保证 PM2.5 指数在时间上的连续型,利用小波分析进行去噪,保证监测点数据整体的有效性。

第二问中 PM2.5 扩散衰减模型用到的数据均是 PM2.5 指数转化成浓度后的数据,单位为 $\mu \mathrm{g} / \mathrm{m}^{3}$,程序见附录 1。

西安市 2013 年 2 月份数据存在错误,显示为 2012 年 2 月份的数据,所以这部分天气时仍按原有题目的用意,按 2013 年 2 月份西安地区气象数据来处理。

\section*{5.1 PM2.5 的时空分布规律及污染评估}

西安市 13 个监测站点的分布图,如图 5-1 红色圆圈所示:对照西安市交通路线图可知,位于西安市内的监测站点有 9 个,分别为二环西的高压开关厂、二环南路的小寨、二环东的兴庆小区,二环中部的市人民体育场、三环北的经开区、三环东北的广运潭、三环东的纺织城、三环南的曲江文化集团和位于三环西南的高新区;靠近西安市三环以外的监测站点共有 5 个,其中阎良区位于显示区的东北,距离较远,图 5-1 中未标示,三环外东北为临潼区、三环外正南为长安区、三环外西北是草滩。

\begin{figure}[h]
    \centering
    \includegraphics[width=\textwidth]{image.png}
    \caption{西安市空气质量监测站点地理位置示意图}
    \label{fig:5-1}
\end{figure}

环境保护部新修订的《环境空气质量标准》中规定空气质量指数及相关信息如表 5-1 所示,其中 AQI 在 0-50、51-100 和 100-150 内分属一级、二级和三级,对人类健康的影响相对较少,其余为三级以上,长期接触对正常人群的健康状况影响较大,所以在对 PM2.5 进行研究的同时,重点比照 3 级及以下的情况。

\begin{table}[h]
    \centering
    \caption{空气质量指数及相关信息}
    \label{tab:5-1}
    \begin{tabular}{c c c c}
        \hline
        空气质量指数 & 空气质量等级 & 空气质量类别及颜色 & 对健康的影响 \\
        \hline
        0-50 & 一级 & 优 & 绿色 \\
        & & & 空气质量良好,基本无污染 \\
        51-100 & 二级 & 良 & 黄色 \\
        & & & 空气质量可接受,轻微污染 \\
        101-150 & 三级 & 轻度污染 & 橙色 \\
        & & & 长期接触,易感人群出现症状 \\
        151-200 & 四级 & 中度污染 & 红色 \\
        & & & 长期接触,健康人群出现症状 \\
        201-300 & 五级 & 重度污染 & 紫色 \\
        & & & 健康人群普遍出现症状 \\
        >300 & 六级 & 严重污染 & 褐红色 \\
        & & & 健康人群出现明显症状 \\
        \hline
    \end{tabular}
\end{table}

为了能够正确区分 PM2.5 浓度在不同范围之内的分类,列出空气质量分级指数的级别与对应污染物浓度的限值进行参照,如图 5-2 所示,其中 PM2.5 浓度单项位于 [0-35]、[36-75]、[76-115] 之间时,对应的 AQI 的 [0-50]、[51-100]、[100-150],即空气质量等级的 1 级、2 级和 3 级。

\begin{table}
\centering
\caption{空气质量分指数与预定的污染物项目浓度限值}
\begin{tabular}{c c c c c c c c c c}
\hline
AQI 限值 & $SO_{2}$ & $NO_{2}$ & PM10 & CO & $O3_{1}$ & PM2.5 & AQI & PM2.5 & 类别 \\
\hline
0 & 0 & 0 & 0 & 0 & 0 & 0 & 0-50 & $\leq=35$ & 1级 \\
50 & 50 & 40 & 50 & 2 & 160 & 35 & 51-100 & 36-75 & 2级 \\
100 & 150 & 80 & 150 & 4 & 200 & 75 & 101-150 & 76-115 & 3级 轻度污染 \\
150 & 475 & 180 & 250 & 14 & 300 & 115 & 151-200 & 116-150 & 4级 中度污染 \\
200 & 800 & 280 & 350 & 24 & 400 & 150 & 201-300 & 151-250 & 5级 重度污染 \\
300 & 1600 & 565 & 420 & 36 & 800 & 250 & $>$300 & 251-350 & 6级 严重污染 \\
400 & 2100 & 750 & 500 & 48 & 1000 & 350 & & 350-500 & \\
500 & 2620 & 940 & 600 & 60 & 1200 & 500 & & & \\
\hline
\end{tabular}
\end{table}

\subsection{5.1.1 PM2.5 的时间变化特征}

\subsubsection{(1) 西安市 PM2.5 每天变化特征}

总体上看,西安市 2013 年 1 月 1 日到 2013 年 9 月 19 日的 PM2.5 浓度变化规律如图 5-2 所示。其中,在 1 月初至 2 月中旬 PM2.5 呈上升趋势,在 2 月 10 日达到最高点,PM2.5=500 单位,之后开始缓慢下降,在 4 月底下降速度变得平缓,之后 PM2.5 的浓度稳定在 50-150 单位之间。

由图 5-2 可知,西安市 2013 年 1 月-2013 年 9 月期间,共 261 天,PM2.5 浓度超出 150 单位的天数有 107 天,占到已有数据样本的 40.99%,所以西安市在 2013 年年初至今,共有近 6 成的天气是 PM2.5 浓度的达到 1、2、3 级标准的,剩余天气中 PM2.5 的浓度均不达标,处在中度污染以上,长期生活在这样的环境下,不利于正常人群的身心健康。

\begin{figure}[h]
\centering
\includegraphics[width=\textwidth]{image.png}
\caption{西安市 2013 年 1 月-9 月每天 PM2.5 浓度分布曲线图}
\end{figure}

对西安市 2013 年 1 月-9 月的首要污染源进行统计可知(图 5-3),261 天中 PM2.5 是首要污染源的天数为 147 天,占到总数的 56.11%,其次是 PM10、O3-8h。由此可知 PM2.5 已经成为日常空气污染源中最主要的因素,一半以上的空气污染首要来源来自 PM2.5。

\begin{figure}[h]
    \centering
    \includegraphics[width=\textwidth]{image1.png}
    \caption{西安市 1-9 月首要污染源的来源分布}
    \label{fig:5.3}
\end{figure}

\subsubsection{西安市 PM2.5 的月份变化特征}

从各地区来看,首先统计 13 个监测站点 PM2.5 浓度在 [0-35](1 级)之内的频数,如图 \ref{fig:5.4} 所示。图中 PM2.5 的浓度频数呈现规则变化,同一月份不同监测站点之间的频数也存在显著差异。

其中,1-3 月整个西安市分布的 13 个站点显示的 PM2.5 浓度在 [0-35] 之内的频数为 0,说明 1 级天气出现的次数为 0,PM2.5 的浓度严重超标。随后 4 月份 PM2.5 浓度开始上升,5 月份开始下降到 0。6-7 月又开始上升,7 月份 PM2.5 为 1 级的频数达到最大值,说明此时出现好天气的天数最多。

4 月份 13 个监测站点中 PM2.5 频数由低到高的顺序为:长安区 > 小寨 > 曲江文化集团 > 临潼区 > 兴庆小区 > 市人民体育场 > 高新西区 > 阎良区 > 纺织城 > 经开区 > 广运潭 > 草滩 > 高压开关厂,说明 4 月份西安市内长安区周围出现好天气的天数最多,而高压开关厂周围出现好天气的天数最少。

7 月份 13 个站点 PM2.5 频数由低到高的顺序为:曲江文化集团 > 小寨 > 兴庆小区 > 临潼区 > 长安区 > 市人民体育场 > 高新西区 > 经开区 > 广运潭 > 纺织城 > 阎良区 > 草滩 > 高压开关厂,说明 7 月份西安市内曲江文化集团周围出现好天气的天数最多,而高压开关厂周围出现好天气的天数最少。

\begin{figure}[h]
    \centering
    \includegraphics[width=\textwidth]{image2.png}
    \caption{西安市 13 个分站点 1 月-9 月 PM2.5 浓度优等的频数统计}
    \label{fig:5.4}
\end{figure}

\begin{table}
\centering
\caption{西安地区1-9月PM2.5各级出现的频数(部分)}
\begin{tabular}{c c c c c c c c c c c c c c}
\hline
月份 & 等级 & 高压开关厂 & 兴庆小区 & 纺织城 & 小寨 & 市人民体育场 & 高新西区 & 经开区 & 长安区 & 阎良区 & 临潼区 & 曲江文化集团 & 广运潭 & 草滩 \\
\hline
1月 & 1 & 0 & 0 & 0 & 0 & 0 & 0 & 0 & 0 & 0 & 0 & 0 & 0 & 0 \\
 & 2 & 3 & 2 & 2 & 3 & 3 & 2 & 3 & 2 & 3 & 3 & 2 & 2 & 2 \\
 & 3 & 1 & 3 & 5 & 2 & 2 & 2 & 1 & 3 & 1 & 4 & 5 & 5 & 0 \\
\hline
4月 & 1 & 0 & 0 & 0 & 0 & 0 & 0 & 0 & 0 & 0 & 0 & 0 & 0 & 0 \\
 & 2 & 2 & 8 & 5 & 11 & 8 & 7 & 5 & 12 & 7 & 10 & 11 & 4 & 4 \\
 & 3 & 10 & 11 & 14 & 7 & 7 & 6 & 11 & 8 & 11 & 7 & 8 & 6 & 14 \\
\hline
7月 & 1 & 3 & 7 & 4 & 7 & 1 & 3 & 1 & 9 & 4 & 6 & 10 & 3 & 1 \\
 & 2 & 14 & 12 & 12 & 12 & 10 & 14 & 13 & 10 & 13 & 10 & 13 & 12 & 13 \\
 & 3 & 12 & 9 & 9 & 7 & 9 & 8 & 10 & 7 & 12 & 9 & 8 & 3 & 8 \\
\hline
9月 & 1 & 0 & 0 & 0 & 1 & 0 & 1 & 0 & 1 & 0 & 0 & 1 & 0 & 0 \\
 & 2 & 4 & 5 & 6 & 5 & 2 & 4 & 4 & 4 & 5 & 2 & 8 & 5 & 3 \\
 & 3 & 8 & 8 & 9 & 9 & 4 & 9 & 6 & 9 & 6 & 4 & 9 & 6 & 10 \\
\hline
\end{tabular}
\end{table}

在1-9月份之间,高压开关厂在7月份出现1级、2级或3级好天气的概率最大,如表5-4所示,为93.55\%;兴庆小区在4月份和7月份出现好天气的概率最大,为90.32\%;纺织城出现好天气的最大概率是90.32\%,8月份;小寨、市人民体育场、高新西区、经开区、长安区、阎良区、临潼区、曲江文化集团、广运潭、草滩出现1、2、3级好天气的最大概率分别93.33\%、70\%、96.67\%、77.42\%、96.67\%、96.67\%、93.33\%、96.67\%、96.77\%、86.67\%,主要出现在6月份,长安区、曲江文化集团的最好天气出现在7月和5月。

\begin{table}
\centering
\caption{西安市PM2.5出现1级、2级、3级的总概率}
\begin{tabular}{l c c c c c c c c c}
\hline
月份 & 1月 & 2月 & 3月 & 4月 & 5月 & 6月 & 7月 & 8月 & 9月 \\
\hline
高压开关厂 & 12.9 & 3.57 & 22.58 & 40 & 74.19 & 73.33 & 93.55 & 83.87 & 63.16 \\
兴庆小区 & 16.13 & 3.57 & 22.58 & 63.33 & 90.32 & 83.33 & 90.32 & 83.87 & 68.42 \\
纺织城 & 22.58 & 10.71 & 6.45 & 63.33 & 77.42 & 90 & 80.65 & 90.32 & 78.95 \\
小寨 & 16.13 & 10.71 & 45.16 & 60 & 77.42 & 93.33 & 83.87 & 83.87 & 78.95 \\
市人民体育场 & 16.13 & 0 & 22.58 & 50 & 51.61 & 70 & 64.52 & 35.48 & 31.58 \\
高新西区 & 12.9 & 7.14 & 25.81 & 43.33 & 74.19 & 96.67 & 80.65 & 64.52 & 73.68 \\
经开区 & 12.9 & 10.71 & 22.58 & 53.33 & 74.19 & 90 & 77.42 & 77.42 & 52.63 \\
长安区 & 16.13 & 3.57 & 38.71 & 66.67 & 77.42 & 96.67 & 83.87 & 77.42 & 73.68 \\
阎良区 & 12.9 & 10.71 & 25.81 & 60 & 87.1 & 96.67 & 93.55 & 90.32 & 57.89 \\
临潼区 & 22.58 & 7.14 & 19.35 & 56.67 & 70.97 & 93.33 & 80.65 & 70.97 & 31.58 \\
曲江文化集团 & 22.58 & 28.57 & 35.48 & 63.33 & 96.77 & 96.67 & 100 & 93.55 & 94.74 \\
广运潭 & 22.58 & 7.14 & 16.13 & 33.33 & 83.87 & 96.67 & 58.06 & 80.65 & 57.89 \\
草滩 & 6.45 & 7.14 & 16.13 & 60 & 70.97 & 86.67 & 70.97 & 80.65 & 68.42 \\
\hline
\end{tabular}
\end{table}

\begin{table}
\centering
\caption{西安市13个监测点季节性PM2.5指数}
\begin{tabular}{c c c c c c c}
\hline
站点 & 高压开关厂 & 兴庆小区 & 纺织城 & 小寨 & 市人民体育场 & 高新西区 & 经开区 \\
\hline
冬季 & 261.75 & 237.08 & 227.25 & 222.37 & 243.59 & 264.29 & 243.00 \\
春季 & 137.33 & 114.27 & 119.37 & 107.90 & 132.30 & 128.78 & 116.38 \\
夏季 & 78.13 & 69.54 & 72.41 & 66.93 & 100.79 & 78.22 & 79.98 \\
秋季 & 104.11 & 99.68 & 91.89 & 94.84 & 121.37 & 99.84 & 114.79 \\
\hline
站点 & 长安区 & 阎良区 & 临潼区 & 曲江文化集团 & 广运潭 & 草滩 \\
\hline
冬季 & 221.54 & 236.53 & 217.47 & 228.98 & 245.93 & 264.07 \\
春季 & 110.61 & 115.35 & 111.66 & 106.20 & 126.77 & 130.29 \\
夏季 & 69.85 & 68.33 & 76.47 & 57.14 & 80.61 & 79.82 \\
秋季 & 93.21 & 102.58 & 131.84 & 78.58 & 106.11 & 106.79 \\
\hline
\end{tabular}
\end{table}

\begin{figure}[h]
\centering
\includegraphics[width=0.8\textwidth]{image.png}
\caption{2013年西安市1-9月PM2.5月平均变化曲线}
\end{figure}

知,位于西安市二环以内的监测站点有4个,分别为二环西的高压开关厂、二环南路的小寨、二环东的兴庆小区,以及二环中部的市人民体育场;内部靠近三环的监测站点有5个,分别为三环北的经开区、三环东北的广运潭、三环东的纺织城、三环南的曲江文化集团和位于三环西南的高新区;三环以外共有5个监测站点,其中阎良区位于显示区的东北,距离较远,图5-6中未标示,三环外东北为临潼区、三环外正南为长安区、三环外西北是草滩。

\begin{figure}[h]
    \centering
    \includegraphics[width=\textwidth]{image.png}
    \caption{西安市空气质量监测站点地理位置示意图}
    \label{fig:5-6}
\end{figure}

从地理空间分布上对西安市的13个监测站点的PM2.5年平均值进行统计,如下表5-6所示。可知,PM2.5的年平均最高值出现在二环内的高压开关厂,年平均最小值出现在三环边上的曲江文化集团。虽然监测点数量有限,但整体上仍可以看出二环内的站点PM2.5平均比三环内边上的站点值高,三环内边上的PM2.5平均值比三环外站点的值高。空间分布上具有从城区到郊区PM2.5浓度指数逐步降低的特点。

\begin{table}[h]
\centering
\caption{西安市PM2.5的空间分布状况}
\label{tab:5-6}
\begin{tabular}{|c|c|c|c|c|c|c|}
\hline
\multirow{2}{*}{二环内} & 高压开关厂 & 兴庆小区 & 市人民体育场 & 小寨 & \multicolumn{1}{c|}{—} & 年平均PM2.5 \\ \cline{2-7} 
 & 142 & 125 & 146 & 118 & \multicolumn{1}{c|}{—} & 133 \\ \hline
\multirow{2}{*}{三环内边} & 高新西区 & 经开区 & 纺织城 & 曲江文化集团 & 广运潭 & 年平均PM2.5 \\ \cline{2-7} 
 & 139 & 132 & 125 & 115 & 136 & 129 \\ \hline
\multirow{2}{*}{三环外} & 长安区 & 阎良区 & 临潼区 & 草滩 & \multicolumn{1}{c|}{—} & 年平均PM2.5 \\ \cline{2-7} 
 & 120 & 125 & 125 & 141 & \multicolumn{1}{c|}{—} & 128 \\ \hline
\end{tabular}
\end{table}

数据处理环节,利用西安全市和各站点的PM2.5数据计算扩散区域内各网格采样点的污染数值据,建立站点坐标(图中红点)与PM2.5之间的对应关系,利用Matlab软件画出等值线来体现该城区PM2.5浓度指数的空间分布。西安市PM2.5的在1月份和6月份的平均分布等值曲线如下图5-7、图5-8所示。

\begin{figure}[h]
    \centering
    \includegraphics[width=\textwidth]{image1.png}
    \caption{西安市 PM2.5 浓度指数 1 月平均分布等值图}
    \label{fig:5-7}
\end{figure}

\begin{figure}[h]
    \centering
    \includegraphics[width=\textwidth]{image2.png}
    \caption{西安市 PM2.5 浓度指数 6 月平均分布等值图}
    \label{fig:5-8}
\end{figure}

选用西安市 1 月份和 6 月份的原因在于这两个月份的浓度值差别最限制,1 月份的 PM2.5 指数最高,6 月份的 PM2.5 指数相对最低。即 1 月份的空气质量最差,6 月份的空气质量最好。

\subsection{PM2.5 时空变化规律及污染评估}

本节利用西安市 2013 年 1 月-9 月共 261 天的数据,从全市整体和 13 个分站点角度对 PM2.5 的时空分布规律进行统计分析,结果表明:

(1)每天变化,西安市总体在 2013 年 1 月-2013 年 9 月内,PM2.5 浓度指数超标天数有 107 天,占到总天数的 40.99\%。PM2.5 指数变化趋势是:1、2 月份最高,3、5 月份降低,4、6、7 月份回升,其中 7 月份 PM2.5 出现 1 级的频数和频率最大,此时好天气的天数最多。

(2) 月份变化,西安市 13 个分监测站点中,1-3 月份 PM2.5 指数全部超标。4-9 月中 6 月份 PM2.5 指数最高,整体趋势在降低。

(3) 每月出现 1 级天气的频数为例,4 月份长安区 PM2.5 出现 1 级的频数最高,高压开关厂出现的频数最低;即 4 月份中长安区出现 1 级天气的天数最多,高压开关厂出现 1 级天气的天数最少。7 月份曲江文化集团 PM2.5 出现 1 级的频数最高,高压开关厂出现的频数最低;即 7 月份曲江文化集团出现 1 级天气的天数最多,高压开关厂出现 1 级天气的天数最少。所以,高压开关厂周围 PM2.5 指数是 13 个站点中最差的。

(4) 在 1-9 月份之间,6 月份出现 1、2、3 级好天气的概率最大。13 个站点中高新区、长安区、阎良区、曲江文化集团、广运潭出现 1、2、3 级好天气的概率最大,为 96.67%,市人民体育场出现 1、2、3 级好天气的概率最小,仅为 70%。

(5) 季节变化,西安市 PM2.5 指数整体上具有典型的季节特征,冬季的 PM2.5 为 246.89,是 3 级标准的 2.14 倍,超标严重,且明显高于其他三个季度。春季、夏季、秋季(9 月)的 PM2.5 指数为 115.62、70.60、100.84,三级指数均达标。

(6) 西安市 13 个监测站点的春、夏、秋、冬四季节的 PM2.5 指数,同样具有显著的季节特性。其中冬季采暖期的 PM2.5 显著高于非采暖期的春、夏、秋季节。其中,冬季采暖期中,高新区和高压开关厂的 PM2.5 指数最高;春季高压开关厂、市人民体育场和草滩的 PM2.5 指数最高;夏季、秋季均是市人民体育场的 PM2.5 指数最高。

(7) 空间变化,依据西安市主要交通干线分布,将 13 个监测站点划分为二环内、三环内、三环外三个空间等级。PM2.5 年平均最高值出现在二环内的高压开关厂,年平均最小值出现在三环边上的曲江文化集团。

(8) 整体上看,二环内的站点 PM2.5 指数平均值比三环内边上的站点值高,三环内边上的 PM2.5 平均值比三环外站点的值高。空间分布上 PM2.5 浓度指数具有从城区到郊区逐步降低的特点。

\section*{5.2 基于非稳态二维多箱的 PM2.5 浓度扩散模型}

\subsection*{5.2.1 PM2.5 的发生机理与转换关系}

(1) PM2.5 的发生机理

PM2.5 是粒径小于 2.5 纳米的空气中的悬浮物,由微小的可溶性固体、可溶性气体构成,是空气质量监测中重要的指标,由 PM2.5、硫化物、氮氧化物等物质组成的含水的微小液滴或颗粒,是形成灰霾的主要原因。

PM2.5 分布的成因又与天气有着非常重要的关系,通常风速、气温、湿度等季节因素会对 PM2.5 的发生和衰减产生影响。其中,风速是加速 PM2.5 扩散和稀释速度的,而气温和湿度则是控制 PM2.5 分布浓度的主要因素。

例如:相对湿度为 90%,温度在 \(5^\circ \mathrm{C}\) 时,可溶性固体溶入水中降低了水的蒸汽压,微小水滴因为曲率半径变小其蒸汽压又得到提高,形成了微小的溶液液滴与气体的动态平衡。这些微小液滴的粒径从纳米至微米级别不等,PM2.5 就集中于此,在这个范围上的微观化学反应更为突出,形成很多亚稳态的有毒物质。加之有液滴的存在,形成了光的吸收、折射与散射现象,从而降低了空气的浓度,集聚了更多的 PM2.5。

(2) PM2.5 的转换关系

PM2.5 $\longleftrightarrow$ PM10 $\longleftrightarrow$ TSP: 细微颗粒、可吸入颗粒与空气总悬浮物之间的转换关系,大于 PM2.5 粒径的悬浮物主要受空气相对湿度的影响,当湿度增加时,污染物表面富含水分,通过相互凝聚成较大颗粒发生沉降;当相对湿度变小时,污染物表面吸附力小,不易凝聚沉降,风力作用下已凝聚的颗粒会分解,由于粒径变小更易悬浮造成持续的污染。

PM2.5 $\longleftrightarrow$ 光化烟雾: PM2.5 对视觉的影响较大,特别是 0.1-1 $\mu$m(相当于 PM1)之间的颗粒是形成灰霾的祸首,在这一级别上,电化学、纳米结构、表面化学等微观现象更为突出,极易形成光的折射、散射等现象,大幅度降低大气的光感觉。PM2.5 的毒性更强,其易侵入人体的肺泡,在呼吸系统上产生堆积,诱发哮喘和肺癌,有毒的可溶物进入血液,诱发多种疾病。湿度、温度对大气的影响更为突出,湿度下降时,易形成光化学烟雾,湿度上升时形成 PM2.5 的灰霾天。由于空气湿度处处不同,所以 PM2.5 不是一个固定的形态和质量,而是一个动态的过程。

\subsection*{5.2.2 PM2.5 的扩散衰减模型}

(1) PM2.5 扩散衰减模型的概述

利用空气质量预测模型对 PM2.5 的扩散衰减模型进行模拟,给出有效的预测分析,能够快速提供污染的分布结果,令环保部门迅速做出反应,采取防治和削减措施,这在环境规划与污染防治中也是具有非常重要作用的。

目前针对 PM2.5 等大气污染物的质量预测模型种类繁多,在不同的条件限制下选用不同的模型[1-2]。其中,可用于对 PM2.5 的扩散与衰减进行预测的代表性模型主要有三个:第一个是高斯扩散模型;第二代是光化学氧化模型,例如 UAM 模型等;第三代有代表性的空气质量扩散预测模型是一个大气模型系统,如箱式模型,MODEL 3。对点源的计算,应用的最广的是高斯模型。但当排放污染源数量较多,排放高度又不是很高时。高斯模型计算的工作量不仅会很大,而且计算所需的气象联合概率频率矩阵在许多地区是难以得出的,因此会存在较大的预测误差。由此,有必要建立一种适合排放源较多,且排放量较大的城市型空气污染扩散与衰减估算模型。箱式模型的计算量远远小于高斯模型,且能有效减少系统误差,提高预测精度,因此在研究气体环境污染与扩散预测等方面得到广泛的应用。

(2) 单箱模型

箱式模型是最简单的一种模型,具有简单易行、操作简便的优点。箱模型包括单箱模型和多箱模型。其中,单箱假设是将每个站点假设成为一个单箱,每个单箱的浓度与每个站点周围的浓度相一致,单箱模型现已被广泛地应用在城市大气环境质量预测中[3-5]。

单箱模型的主要假设:

(1) 单箱假设是将每个站点假设成为一个单箱,每个单箱的浓度与每个站点周围的浓度相一致,即假设 PM2.5 是均匀分布在每个单箱内的;

(2) PM2.5 浓度在箱中是完全均匀;

(3) PM2.5 浓度随排放量、时间改变而变化,且 PM2.5 从箱内的输出项,既包括通过垂直方向上的清除过程(沉降、雨除、冲洗),也包括混合层内平均风速 $u$ 的水平输送作用。

(4) 大量的 PM2.5 以 $Q$ 速率排放有害物质,这些 PM2.5 向上的扩散受到单箱高度 $h$ 上的逆温层限制;

根据 PM2.5 的输入、输出关系,可以写出质量平衡方程
\begin{equation}
dClhb = ubh(C_0 - C) + lbQ
\tag{5-1}
\end{equation}

式 (5-1) 中,$Q$ 为污染源强度,$u$ 为平均风速,$mg/s$;$l$ 为箱的长度,$m$;$b$ 为箱子的宽度,$m$;$h$ 为箱子的高度,$m$;$C_0$ 为输入浓度,$mg/m^3$;$C$ 为箱内浓度,$mg/m^3$;$d$ 为污染源距离一个站点的距离,$km$;详情见下图 5-9 所示:

\begin{figure}[h]
\centering
\includegraphics[width=\textwidth]{image.png}
\caption{单箱体模型示意图}
\end{figure}

当持续时间较长,可以得到平衡浓度为:
\begin{equation}
C = C_0 + \frac{dQl}{uh} \left[ 1 - e^{-\frac{ut}{l}} \right]
\tag{5-2}
\end{equation}

上式表明浓度与源强、箱体的长度呈正比,而与风速和混合层厚度呈反比。该公式还表明任何给定的浓度取决于排放量和气象条件。

\begin{equation}
C = C_0 + \frac{Ql}{uh}
\tag{5-3}
\end{equation}

\begin{equation}
Q = \frac{C - C_0}{d} \times u \times h / l
\tag{5-4}
\end{equation}

依据上述公式,在只考虑下风向的时候,设高压开关厂为污染源,小寨为受影响的单箱。由于每个站点在日常生活中都有可能是 PM2.5 浓度最高的污染源,所以为了求出污染源强度 $Q$,先以这两个站点为例,利用补充后的西安市 1-9 月每天的 PM2.5 的数据,求得污染源强度 $Q = 68.96 \, mg/s$,并将 $Q = 68.96 \, mg/s$ 带回方程 (5-2),求得:
\begin{equation}
C = C_0 + \frac{68.96dl}{uh} \left[ 1 - e^{-\frac{ut}{l}} \right]
\tag{5-5}
\end{equation}

此时公式 5-5 含有时间变量 $t$,可依据时间的变化求得不同时刻 $t$ 对应的污染源相对浓度;所以依据不同污染源和被影响站点,就可以求得被污染的污染浓度,即箱内的单箱浓度;

(3) 多箱模型

多箱模型是在单箱模型基础上的一种改进,主要包括二维多箱模型和三维多箱模型 [6-7]。适合于小污染源较多的城市地区,或对尺度很大的关于污染地区进行扩散预测,如果要预测一个城市或广域地区的污染,可以将调查区域看成是矩形的箱体,根据箱内污染物质的流出和流入的情况计算箱内污染物的质量浓度。其中单个箱体污染物流入、流出规律见图 5-9,二维多箱模型的示意图如图 5-10 所示,其中箱体 1-9 内污染物的流入流出状况如图 5-9 所示:

\begin{figure}[h]
    \centering
    \includegraphics[width=0.8\textwidth]{image.png}
    \caption{二维多箱模型示意图}
    \label{fig:2d_box_model}
\end{figure}

(4) 非稳态二维多箱模型(上风向和下风向之间的转换)

本文以二维稳态多箱模型 [7-10] 为基础,考虑到非稳态模型的特点,构建了非稳态二维多箱模型,主要假设有:

(1) PM2.5 污染物在每个子箱中混合均匀;

(2) 忽略每个子箱中的化学反应以及 PM2.5 的干、湿沉积;

(3) 某一时刻的风矢量不随水平位置的不同而改变。

定义西安市南北向为长、东西向为宽,在高度方向上将混合层高度分成高度为 \( h \) 的 \( p \) 个子箱子,在长度方向上将西安城市长度分成长度为 \( L \) 的 \( m \) 个子箱,在宽度方向上将城市宽度分成宽度为 \( W \) 的 \( n \) 个子箱,这样共组成 \( p \times m \times n \) 个子箱。

将风矢量分解到 \( X \)、\( Y \) 轴,规定顺轴为正,逆轴为负,对子箱 \( b_{ij} \) 建立此风向坐标系下的质量平衡方程。

上风向 \( X \) 方向上输入的 PM2.5 量:
- 当风向为 \( X \) 轴正方向时为 \( C_{i-1,j} \times u_x \times L \times h \);
- 当风向为 \( X \) 轴负方向时为 \( C_{i+1,j} \times u_x \times L \times h \);

上风向 \( Y \) 方向上输出的 PM2.5 量:
- 当风向为 \( Y \) 轴正方向时为 \( C_{i,j-1} \times u_y \times W \times h \);
- 当风向为 \( Y \) 轴负方向时为 \( C_{i,j+1} \times u_x \times W \times h \);

下风向 \( X \) 方向输出的 PM2.5 量:
\begin{equation}
C_{i,j} \times u_x \times L \times h
\tag{5-6-1}
\end{equation}

下风向 \( Y \) 方向输出的 PM2.5 量:

\begin{equation}
C_{i,j} \times u_y \times W \times h
\tag{5-6-2}
\end{equation}

由子箱 $b_{i-1,j}$ 水平扩展到 $b_{i,j}$ 的 PM2.5 量:
\begin{equation}
e \times h \times L \times (C_{i-1,j} - C_{i,j}) / W
\tag{5-6-3}
\end{equation}

由子箱 $b_{i,j}$ 水平扩展到 $b_{i+1,j}$ 的 PM2.5 量:
\begin{equation}
e \times h \times L \times (C_{i,j} - C_{i+1,j}) / W
\tag{5-6-4}
\end{equation}

由子箱 $b_{i,j-1}$ 水平扩展到 $b_{i,j}$ 的 PM2.5 量:
\begin{equation}
e \times h \times W \times (C_{i,j-1} - C_{i,j}) / L
\tag{5-6-5}
\end{equation}

由子箱 $b_{i,j}$ 水平扩展到 $b_{i,j+1}$ 的 PM2.5 量:
\begin{equation}
e \times h \times W \times (C_{i,j} - C_{i,j+1}) / L
\tag{5-6-6}
\end{equation}

污染源排入子箱 $b_{i,j}$ 的 PM2.5 量:
\begin{equation}
Q_{i,j} \times W \times L
\tag{5-7}
\end{equation}

根据质量守恒原理列式得:
\begin{equation}
\frac{dC_{i,j}}{dt} LhW = u_x C_{i+1,j} Lh - u_x C_{i,j} Lh + u_y C_{i,j+1} Wh - u_y C_{i,j} Wh + Q_{i,j} WL
\tag{5-8}
\end{equation}
\begin{equation}
+ eLh(C_{i-1,j} - C_{i,j}) / W - eLh(C_{i,j} - C_{i+1,j}) / W
\end{equation}
\begin{equation}
+ eWh(C_{i,j-1} - C_{i,j}) / L - eWh(C_{i,j} - C_{i,j+1}) / L
\end{equation}

其中, $C_{i,j}$ 为子箱 $b_{i,j}$ 的 PM2.5 总量, $i = 1, 2, \ldots n$, $j = 1, 2, \ldots m$;

上风向 $X$ 方向上输入的 PM2.5 量:
当风向为 $X$ 轴正方向时为 $C_{i-1,j} \times u_x \times L \times h$; 当风向为 $X$ 轴负方向时为 $C_{i+1,j} \times u_x \times L \times h$; 上风向 $Y$ 方向上输出的 PM2.5 量: 当风向为 $Y$ 轴正方向时为 $C_{i,j-1} \times u_y \times W \times h$; 当风向为 $Y$ 轴负方向时为 $C_{i,j+1} \times u_x \times W \times h$;

其中:
- $e$ 为水平方向上的 PM2.5 扩散系数;
- $h$ 为二维箱子 $b_{i,j}$ 高度;
- $L$ 为子箱 $b_{i,j}$ 的长度 ($m$), 定义西安城区的南北方向为长;
- $Q_{i,j}$ 为子箱 $b_{i,j}$ 的排放源强度;
- $u_x$ 为子箱 $b_{i,j}$ 在 $X$ 轴方向上的平均风速分量;
- $u_y$ 为子箱 $b_{i,j}$ 在 $Y$ 轴方向上的平均风速分量;
- $W$ 为子箱宽度, 定义西安城区的东西方向为宽;

本文还假设在子箱的边界处有:
\begin{equation}
C_{0,j} = C_{n+1,j} = 0
\tag{5-9}
\end{equation}
\begin{equation}
C_{i,0} = C_{i,m+1} = 0
\end{equation}

(5) 非稳态二维多箱模型风向说明

气象预报模型中是以风向方位角 $\theta$ 表示风向的, 以北风为零度, 顺时针旋转。由此, 将风向坐标转化到本坐标系下, 并将风向分解为 $X$ 轴分量和 $Y$ 轴分量。

\begin{equation}
\begin{aligned}
u_{x} & = -\sin(\theta)u \\
u_{y} & = -\cos(\theta)u
\end{aligned}
\tag{5-10}
\end{equation}

(6) 非稳态二维多箱模型的算法流程

计算之前已经对迭代的步长进行了选取、筛选和分析,具体步骤如下图 5-11 所示:

\begin{figure}[h]
\centering
\includegraphics[width=0.8\textwidth]{image.png}
\caption{非稳态二维多箱扩散模型流程图}
\label{fig:5-11}
\end{figure}

(7) 非稳态二维多箱模型扩散衰减的定性分析

非稳态二维多箱模型反映了西安市 PM2.5 的扩散情况,在该模型中不仅考虑了距离对于模型本身的影响,还考虑了风向、风速以及天气情况的影响。非稳态二维多箱模型以西安市为一个系统,不考虑系统与外界物质的输入域输出,仅考虑系统中物质浓度的变化过程。

模型中各个箱子之间均存在直接或间接的联系,如(1,2)和(3,2)虽然不存在直接联系,但是(1,2)可以通过(1,3)和(3,2)间接发生作用,所以该模型不仅系统的表现出西安市 PM2.5 浓度变化与周围区域的关系,还展现出箱体系统内部物质的扩散分布方式。

非稳态二维多箱模型中引入箱体本身产生的 PM2.5 强度 $Q_{i,j}$,$Q_{i,j}$ 表示各个地区原有的 PM2.5 的含量和污染扩散衰减之后的浓度总和,即:
\begin{equation}
Q_{i,j} = \text{产生的 PM2.5} - \text{雨水衰减 PM2.5}
\end{equation}

\section*{5.2.3 基于非稳态二维多箱模型的PM2.5扩散衰减分析}

数据处理环节,利用西安全市和各站点的PM2.5数据计算扩散区域内各网格采样点的污染数值据,建立站点坐标与PM2.5之间的对应关系,利用Matlab软件画出等值线来体现该城区PM2.5浓度指数的空间分布。全西安市PM2.5的平均分布等值曲线如下图5-12所示。

\begin{figure}[h]
    \centering
    \includegraphics[width=\textwidth]{image.png}
    \caption{西安市PM2.5平均分布等值图}
    \label{fig:5-12}
\end{figure}

\subsection*{(1) 西安市区13个箱体站点的划分}

从西安市PM2.5的平均分布等值曲线来看,各站点之间PM2.5的浓度指数存在一定差异。利用非稳态二维多箱的方法将13个站点进行分割,在地图上如5-13所示:

\begin{figure}[h]
    \centering
    \includegraphics[width=\textwidth]{image.png}
    \caption{西安市箱体站点的划分}
    \label{fig:5-13}
\end{figure}

依据西安市监测站点的分布规律,以及西安市的经济状况、城市布局以及各站点污染源的分布密度情况,规划非稳定的二维多箱模型,拟定箱体划分如图所示,箱体长宽全部为 $7.7\,\text{km} \times 4.75\,\text{km}$,高度方向上只限定一个相同的高度。

\subsubsection{基于非稳态二维多箱模型的 PM2.5 扩散分析}

首先利用西安市各站点的 PM2.5 浓度数据为例,带入非稳态二维多箱模型,对 PM2.5 的扩散与衰减状况进行分析。由于面源数据(西安市全部站点)的时间(2013.1 月-9 月)较长,带入模型的运算量较大,为了简化计算,从西安市 2013 年 1 月-9 月中随机抽取 2013 年 2 月 10 日的 PM2.5 浓度数据(表 5-7)为实例,进行实例分析。利用 matlab 软件对 2 月 10 日的 PM2.5 浓度分空间分布展示如图 5-14 所示。

\begin{table}[h]
    \centering
    \caption{西安市 2013 年 2 月 10 日-11 日 PM2.5 浓度的初始分布}
    \label{tab:5-7}
    \begin{tabular}{c c c c c c c c}
        \hline
        站点 & 高压开关厂 & 兴庆小区 & 纺织城 & 小寨 & 市人民体育场 & 高新西区 & 经开区 \\
        \hline
        2 月 10 日 & 361 & 94 & 446 & 101 & 499 & 500 & 500 \\
        实测数据 & 长安区 & 阎良区 & 临潼区 & 曲江文化集团 & 广运潭 & 草滩 & \\
        & 500 & 494 & 500 & 500 & 373 & 458 & \\
        \hline
    \end{tabular}
\end{table}

以西安市 2013 年 2 月 10 日 PM2.5 的初始浓度分布情况为例,代入非稳态二维多箱模型后,得到 1 天后(2013 年 2 月 11 日)各站点 PM2.5 浓度分布情况(表 5-8),PM2.5 浓度分布图如图 5-14 所示。

\begin{figure}[h]
    \centering
    \includegraphics[width=\textwidth]{image1.png}
    \caption{西安市 2013 年 2 月 10 日 PM2.5 初始空间分布图}
    \label{fig:5-14}
\end{figure}

\begin{figure}[h]
    \centering
    \includegraphics[width=\textwidth]{image2.png}
    \caption{基于非稳态二维多箱的 PM2.5 扩散分布图}
    \label{fig:5-15}
\end{figure}

由图 \ref{fig:5-13} 和图 \ref{fig:5-14} 可知,2013 年 2 月 11 日的 PM2.5 浓度分布比 2 月 10 日 PM2.5 的浓度分布更均匀。空间上看,图 \ref{fig:5-15} 比图 \ref{fig:5-14} 的空间趋势更平缓,即 PM2.5 经过一天的扩散后,实际浓度分布如 2 月 11 日所示。

\subsubsection{非稳态二维多箱-PM2.5 扩散模型的有效性检验}

为了验证非稳态二维多箱模型对 PM2.5 浓度扩散和衰减预测的精度,利用非稳态二维多箱模型给出的 2 月 11 日 PM2.5 浓度的预测数据分布状况与实际附件数据中给出的西安市各站点在 2 月 11 日给出的 PM2.5 的实测数据之间进行比较检验。

\begin{table}
\centering
\caption{西安市2013年2月11日PM2.5浓度的实测值与模拟值}
\begin{tabular}{c c c c c c c c}
站点 & 高压开关厂 & 兴庆小区 & 纺织城 & 小寨 & 市人民体育场 & 高新西区 & 经开区 \\
\hline
2月11日 & 258 & 242 & 249 & 100.6 & 262 & 255 & 257 \\
实测数据 & 长安区 & 阎良区 & 临潼区 & 曲江文化集团 & 广运潭 & 草滩 & \\
 & 257 & 226 & 225 & 253 & 187 & 244 & \\
\hline
站点 & 高压开关厂 & 兴庆小区 & 纺织城 & 小寨 & 市人民体育场 & 高新西区 & 经开区 \\
2月11日 & 215 & 198 & 246 & 220 & 242 & 251 & 257 \\
预测数据 & 长安区 & 阎良区 & 临潼区 & 曲江文化集团 & 广运潭 & 草滩 & \\
 & 217 & 225 & 213 & 256 & 221 & 211 & \\
\end{tabular}
\end{table}

利用公式5-11对2月11日的PM2.5实测数据与预测数据之间的有效性进行检验,得到2月11日PM2.5实测数据与PM2.5模拟数据的相对误差,如表5-9所示。

\begin{equation}
\text{相对误差} = \frac{\text{实测数据} - \text{预测数据}}{\text{实测数据}} \tag{5-11}
\end{equation}

\begin{table}
\centering
\caption{2月11日PM2.5实测数据与PM2.5预测数据的相对误差}
\begin{tabular}{c c c c c c c c}
站点 & 高压开关厂 & 兴庆小区 & 纺织城 & 小寨 & 市人民体育场 & 高新西区 & 经开区 \\
\hline
相对误差 & 0.1667 & 0.1818 & 0.0120 & -1.1869 & 0.0763 & 0.0157 & 0.0000 \\
 & 长安区 & 阎良区 & 临潼区 & 曲江文化集团 & 广运潭 & 草滩 & \\
 & 0.1556 & 0.0044 & 0.0533 & -0.0119 & -0.1818 & 0.1352 & \\
\end{tabular}
\end{table}

从表5-9的相对误差结果来看,除了小寨的预测数据与实测数据之间偏差有点大外,剩余12个站点的PM2.5浓度的预测数据与实测数据之间的相对误差绝对值均控制在0.2以内,由此可见,非稳态二维多箱模型在预测西安市PM2.5浓度扩散与衰减分布结果的精度较高,可以有效的预测出PM2.5的扩散衰减规律。

\section{突发情形下的PM2.5污染扩散预测模型}

基于非稳态二维多箱的PM2.5浓度扩散模型适合于大范围的面源污染预测,而突发情形下PM2.5浓度在短时间小空间内急剧上升,不能将其看成一个面源污染,而是应该看成点污染。我们对其污染扩散的规律进行预测,所以上一问中非稳态二维多箱并不合适点源污染的扩散预测,所以本问选用对点源污染扩散预测最经典的——高斯模型,对突发情形下的PM2.5的污染扩散进行预测和模拟。

\subsection{高斯点源扩散衰减模型}

设有1扩散源,某物质从此扩散源向四周扩散,沿 \(x, y, z\) 三个方向的扩散系数分别为常数,衰减减少的质量与浓度成正比,扩散前周围空间此物质的浓度为零,估计物质的分布。

设 \(u(x, y, z, t)\) 是 \(t\) 时刻点 \((x, y, z)\) 处某物质的浓度。任取一个闭曲面 \(S\),它所围的区域是 \(\Omega\),由于扩散,从 \(t\) 到 \(t + \Delta t\) 时刻这段时间内,通过 \(S\) 流入 \(\Omega\) 的质量为:

\begin{equation}
M_1 = \int_t^{t+\Delta t} \oiint_S \left( a^2 \frac{\partial u}{\partial x} \cos \alpha + b^2 \frac{\partial u}{\partial y} \cos \beta + c^2 \frac{\partial u}{\partial z} \cos \gamma \right) dS dt \tag{5-12}
\end{equation}

其中 \(a^2\),\(b^2\),\(c^2\) 分别是沿 \(x\),\(y\),\(z\) 方向的扩散系数。

由高斯公式
\begin{equation}
M_{1}=\int_{t}^{t+\Delta t} \iiint_{\Omega}\left(a^{2} \frac{\partial^{2} u}{\partial x^{2}}+b^{2} \frac{\partial^{2} u}{\partial y^{2}}+c^{2} \frac{\partial^{2} u}{\partial z^{2}}\right) d x d y d z d t
\tag{5-13}
\end{equation}
由于衰减, $\Omega$ 内的质量减少为:
\begin{equation}
M_{2}=\int_{t}^{t+\Delta t} \iiint_{\Omega} k^{2} u d x d y d z d t
\tag{5-14}
\end{equation}
其中 $k_{2}$ 为衰减系数。

由质量守恒定律,从 $t$ 到 $t+\Delta t$ 时刻间 $\Omega$ 内由于扩散与衰减的共同作用,积存于 $\Omega$ 内的质量为 $M_{1}-M_{2}$。

换一个角度看,在 $t$ 到 $t+\Delta t$ 时刻间 $\Omega$ 内由于浓度的变化引起的质量增加为:
\begin{equation}
\begin{aligned}
M_{3} & =\iiint_{\Omega}\left[u(x, y, z, t+\Delta t)-u(x, y, z)\right] d x d y d z \\
& =\int_{t}^{t+\Delta t} \iiint_{\Omega} \frac{\partial u}{\partial t} d x d y d z d t
\end{aligned}
\tag{5-15}
\end{equation}
显然, $M_{3}=M_{1}-M_{2}$,即:
\begin{equation}
\begin{aligned}
& \int_{t}^{t+\Delta t} \iiint_{\Omega} \frac{\partial u}{\partial t} d x d y d z d t \\
& =\int_{t}^{t+\Delta t} \iiint_{\Omega}\left(a^{2} \frac{\partial^{2} u}{\partial x^{2}}+b^{2} \frac{\partial^{2} u}{\partial y^{2}}+c^{2} \frac{\partial^{2} u}{\partial z^{2}}-k^{2} u\right) d x d y d z d t
\end{aligned}
\tag{5-16}
\end{equation}
由 $\Delta t$,$t$,$\Omega$ 的任意性得:
\begin{equation}
\frac{\partial u}{\partial t}=a^{2} \frac{\partial^{2} u}{\partial x^{2}}+b^{2} \frac{\partial^{2} u}{\partial y^{2}}+c^{2} \frac{\partial^{2} u}{\partial z^{2}}-k^{2} u
\tag{5-16}
\end{equation}
上述方程是常系数线性抛物型方程,它就是有衰减扩散过程的数学模型。

另设扩散源在点 $(x_{0}, y_{0}, z_{0})$ 处,则此扩散问题满足 Cauchy 问题:
\begin{equation}
\left\{\begin{array}{l}
\frac{\partial u}{\partial t}=a^{2} \frac{\partial^{2} u}{\partial x^{2}}+b^{2} \frac{\partial^{2} u}{\partial y^{2}}+c^{2} \frac{\partial^{2} u}{\partial z^{2}}-k^{2} u \\
u(x, y, z, 0)=M \delta(x-x_{0}) \delta(y-y_{0}) \delta(z-z_{0})
\end{array}\right.
\tag{5-17}
\end{equation}
其中 $M$ 为扩散源的质量。用傅立叶变换可求得 Cauchy 问题的解析解为:

\begin{equation}
u(x, y, z, t)=\frac{M}{8 \pi t a b c \sqrt{\pi t}} \exp \left\{-\frac{\left(x-x_{0}\right)^{2}}{4 a t^{2}}-\frac{\left(y-y_{0}\right)^{2}}{4 b t^{2}}-\frac{\left(z-z_{0}\right)^{2}}{4 c t^{2}}-k t^{2}\right\}
\tag{5-18}
\end{equation}

其中 $a^{2}$,$b^{2}$,$c^{2}$ 分别是沿 $x$,$y$,$z$ 方向的扩散系数,$k_{2}$ 为衰减系数,排放源点的坐标为 $(x_{0}, y_{0}, z_{0})$。

\subsection*{5.3.2 点污染源的选取与假设}

由题意可知,随机选取西安市某一观测站点为突发情况下的污染源,建立针对西安市某监测点 PM2.5 浓度突然增至数倍,延续数小时,所在突发情形的污染扩散预测与评估方法。

本文随机抽取一天,得到该天中污染源浓度最高的站点——市人民体育馆,作为此

次突发情形的污染源,这一问主要研究点源污染的衰减扩散情形,所以此时其他站点的 PM2.5 污染排放情况可被视为 0,主要关注市人民体育场上空 PM2.5 的衰减扩散情况。以该地区 PM2.5 浓度最高值为参照,当 PM2.5 浓度突然增至 2 倍,持续 2 小时,利用所建的高斯点源扩散衰减模型,进行突发情形的预测评估,模型本身不仅考虑了 PM2.5 的扩散距离,还将 PM2.5 本身的衰减考虑在内。所以根据模型参数,可以估测 PM2.5 以某一强度持续排放数小时后,在该排放强度下某一时间某一地点的污染浓度。

\subsection{5.3.3 突发情形下高斯点源扩散模型对 PM2.5 的预测}

根据要求,我们选取市人民体育场为 PM2.5 的污染排放点。无风理想情况下,PM2.5 沿水平方向、竖直方向的扩散系数可由表 5-10 查的,已经求得沿 x 轴方向、y 轴方向、z 轴方向 PM2.5 的扩散系数分别为 0.93、0.93、1.97。已知,小风和静风时的扩散参数为大气稳定度与扩散时间的函数,当稳定度仍然用 PS 方法分为 6 类是,扩散参数表示为 $\sigma_{x}=\sigma_{y}=\gamma_{1}T, \sigma_{z}=\gamma_{2}T$,$T$ 的取值为秒,$\gamma_{1}、\gamma_{2}$ 参照表 5-10 选取。

\begin{table}[h]
\centering
\caption{小风和静风时的扩散参数}
\begin{tabular}{c c c c c}
\hline
稳定度 & \multicolumn{1}{c}{$\gamma_{1}$} & \multicolumn{1}{c}{$\gamma_{2}$} \\
\hline
 & 0-0.5 m/s & 0.5-1.0 m/s & 1.0-1.5 m/s & 1.5-2.0 m/s \\
\hline
A & 0.93 & 0.76 & 1.57 & 1.57 \\
B & 0.76 & 0.56 & 0.47 & 0.47 \\
C & 0.55 & 0.35 & 0.21 & 0.21 \\
D & 0.47 & 0.27 & 0.12 & 0.12 \\
E & 0.44 & 0.24 & 0.07 & 0.07 \\
F & 0.44 & 0.24 & 0.05 & 0.05 \\
\hline
\end{tabular}
\end{table}

已知 PM2.5 的最高浓度限制为 $500 \mu \mathrm{g} / \mathrm{m}^{3}$。所以,依据题意,需要在市人民体育馆周围以 $1000 \mu \mathrm{g} / \mathrm{m}^{3}$ 的浓度标准持续排放两个小时。将上述参数带入 matlab,进行编程,得到此种突发情形下 PM2.5 的扩散衰减示意图,如图 5-13 所示。

从图 5-13 中可以看到,污染物 PM2.5 以市人民体育场为中心不断向四周扩散,随着扩散距离的增加,PM2.5 的浓度也在逐渐减少,说明 PM2.5 的浓度得到了稀释。

根据附件里的《中华人民共和国环境保护标准》,当空气质量指数大于 200 时,空气质量类别为重度污染,将空气质量指数转化为 PM2.5 浓度,即 PM2.5 浓度大于 $75 \mu \mathrm{g} / \mathrm{m}^{3}$ 时为重度污染。所以由上图 5-13 可知,我们以市中心体育场的 PM2.5 监测数据最高的一天为例,在全地区 PM2.5 浓度最高点处的浓度增至 2 倍,持续 2 小时,重度污染的监测点有:经开区、兴庆小区、小寨、曲江、长安、高新区、高压开关厂、广运谭,可能安全的区域有:草滩、纺织城、临潼区、阎良区,详见表 5-11;由此可知,该模型可推广到多种天气情况下的突发扩散预测。

\section*{5-11 市人民体育场周围重度污染监测点与可能的安全区域}

\begin{tabular}{|c|c|c|c|c|c|c|c|}
\hline
\multicolumn{8}{|c|}{ 重度污染监测站点 } \\
\hline
经开区 & 兴庆小区 & 小寨 & 曲江 & 长安 & 高新区 & 高压开关厂 & 广运谭 \\
\hline
\multicolumn{8}{|c|}{ 可能安全的监测站点 } \\
\hline
草滩 & 纺织城 & 临潼区 & 阎良区 & & & & \\
\hline
\end{tabular}

\begin{figure}[h]
\centering
\includegraphics[width=\textwidth]{image.png}
\caption{市人民体育场在突发情形下 PM2.5 的扩散衰减示意图}
\end{figure}

\section*{5.4 PM2.5 污染扩散预测模型的合理性检验}

模型预测的合理性与有效性一定程度上反映了模型的好坏,为了进一步验证二维多样扩散模型在预测 PM2.5 污染扩散时的有效性,我们在单个天数预测数据与实测数据之间进行比较检验的基础之上,对西安市 PM2.5 的总体预测属于与实测数据进行对比分析,依据相对误差的精度来估计模型的准确性。

首先,我们以某一天各地区 PM2.5 浓度作为输入数据,迭代后第二天的 PM2.5 浓度为预测数据。根据预测数据与当天实测数据对比,计算了从一月一日至八月一日的 PM2.5 的预测值,并与实测数据进行比较。统计得到表 5-12

\begin{table}[h]
\centering
\caption{PM2.5 预测与实测数据对比}
\begin{tabular}{|c|c|c|c|c|c|c|c|}
\hline
\multicolumn{8}{|c|}{ 表 5-12 } \\
\hline
\end{tabular}
\end{table}

\begin{table}
\centering
\begin{tabular}{c c c c c c c c c c c}
\hline
站点 & 高压 & 兴庆 & 纺织 & 市人 & 高新 & 经开 & 长安 & 阎良 & 临潼 & 曲江 \\
 & 开关 & 小区 & 城 & 民体 & 西区 & 区 & 区 & 区 & 区 & 文化 \\
 & 厂 & & & 育场 & & & & & & 集团 \\
\hline
相对误差 & 0.24 & 0.36 & 0.21 & 0.37 & 0.17 & 0.24 & 0.18 & 0.38 & 0.22 & 0.19 \\
 & & & & & & & & & & 0.18 \\
 & & & & & & & & & & 0.20 \\
 & & & & & & & & & & 0.24 \\
\hline
\end{tabular}
\end{table}

根据相对误差的结果可知,对西安市13个站点的PM2.5浓度预测模型的效果很好,可以用于西安市PM2.5扩散预测。

\section*{6 第三问 模型建立与求解}

\section*{问题三:}

1. 已知该地区现有 PM2.5 的年平均浓度为 $280 \mu \mathrm{g} / \mathrm{m}^{3}$,请给出合理的治理方案,将 5 年后 PM2.5 的年终平均浓度降低至 $35 \mu \mathrm{g} / \mathrm{m}^{3}$;

2. 依据综合治理费用和专项治理费用的投入标准,给武汉地区设计一份有效的专业治理方案,并给出未来五年投入总经费与逐年经费投入预算计划,论述方案的合理性。

\subsection*{6.0 问题分析:}

第三问要给出 5 年内降低 PM2.5 年平均浓度的方案,依据综合治理费用和专项治理费用的投入标准,结合武汉市的治理方案,在保证两项治理费用最低的前提下,给出经费投入预算表,并论述方案的合理性。

\subsection*{6.1 未来五年减少武汉市 PM2.5 浓度的治理方案}

\subsubsection{6.1.1 降低 PM2.5 浓度的主要方法及存在缺陷}

目前对 PM2.5 的治理方法有很多种,现行主要的治理方法有如下几种:

1. 采用水喷雾系统,降低 PM2.5 浓度的步骤:
   - \textbf{Step1:} 布置 PM2.5、PM10、TSP 监控网,布置风速、温度、湿度等气象监测网;
   - \textbf{Step2:} 在广场、小区建筑密集区域架设喷雾设备,喷雾设备可以采用荷电水雾方法。景观区域采用雾粒粒径可调的设备,水雾平均粒径从 $5 \mu \mathrm{m}-50 \mu \mathrm{m}$ 可调,粒径越大,能耗越小,粒径 $20 \mu \mathrm{m}$ 以上时采用自来水压力即可实现喷雾;
   - \textbf{Step3:} 喷雾设备装载 GSM、CDMA 等无线数据控制设备和 GPS 等空间定位设备,对喷雾量、粒径、喷雾影响范围实现实时监控;
   - \textbf{Step4:} 建立区域的计算控制系统和数据库系统,通过合理的程序自动控制,编造人工影响模型;

2. 火电厂污染源控制:燃煤电厂基本上已全部安装脱硫除尘装置,若进一步提升排放标准,将大幅度提高火电成本,而且此时火电厂所导致的城市 PM2.5 污染已是很小的一部分;

3. 农作物的燃烧处理:农作物的燃烧会导致空气中 PM2.5、PM10、TSP 浓度的显著上升,且在适当温度和湿度条件下,污染源之间会相互转换,导致污染状况的加剧,限制农作物燃烧又会增加政府的机会成本;

4. 工业排放污染的控制:工业区的外迁和控制工业污染虽然能在一定程度上解决集中污染问题,但又导致一个区域到另一个区域污染更加分散,因此可借助技术改进来控制污染源;

5. 汽车污染源的控制:汽车污染已成为城市 PM2.5 的主要污染源,可以通过逐步淘汰旧汽车的方式提升排放标准,但这使得三-五年内汽车污染源控制达标难

以实现;

6、植树种草:灰霾天气主要发生在湿度相对较低的冬春季,但此时植物的自净化能力很有限;

\subsection*{6.1.2 降低 PM2.5 浓度的主要治理途径}

根据上述分析可知,武汉市政府对于 PM2.5 浓度的治理需要多方面入手,包括治理项目的计划、融资、建设、评估以及项目的投入使用。依据武汉市一般环境治理项目的建设周期可知,治理方案的主要步骤如下:

\textbf{Step1:} 设第一年为项目准备阶段,这一年完成项目的策划、融资、建设和方案评估,虽然有治理资金的投入,但是 PM2.5 浓度并没有降低;

\textbf{Step2:} 设 PM2.5 起始浓度为 \( y \),则 \( y_1 = 280 \, \mu g/m^3 \),治理年限为 \( t \) 年。第二年开始启动治理方案,由于 PM2.5 具起始浓度较高,且项目方案的治理效果明显,所以 PM2.5 浓度下降明显;

\textbf{Step3:} PM2.5 的治理难度符合修正指数曲线模型的规律,即初期增长迅速,随着时间的增加,增长率逐渐降低,即越到后期,PM2.5 浓度越低,资金投入越大,治理难度越大,效果越不明显。

\textbf{Step4:} 根据上述分析结果,可知治理方面安排如下:在第一年末,PM2.5 浓度为 \( Y_1 = 280 \, \mu g/m^3 \);PM2.5 浓度在第二年年初开始显著降低,到第五年末,PM2.5 浓度降低到 \( 35 \, \mu g/m^3 \);带入修正指数曲线模型(6-1):

\begin{equation}
Y = \alpha + \beta_0 \times \beta_1^t
\tag{6-1}
\end{equation}

已知 \( Y_1 = 280 \, \mu g/m^3 \),\( t \leq 1 \),得到最终的治理方案模型(6-2):

\begin{equation}
Y = 15.4 + 580.19 \times 0.52^t
\tag{6-2}
\end{equation}

由此得到武汉市未来 5 年 PM2.5 浓度的治理指标的分布规律如下图 6-1 所示:

\begin{figure}[h]
\centering
\includegraphics[width=\textwidth]{image.png}
\caption{未来 5 年 PM2.5 浓度的下降指标}
\end{figure}

由模型(6-2)的求得未来 5 年内 PM2.5 年终平均浓度如表 6-1 所示:

\begin{table}
\centering
\caption{未来5年内PM2.5年终平均浓度}
\begin{tabular}{l c c c c c}
\hline
年份 & 第1年 & 第2年 & 第3年 & 第4年 & 第5年 \\
\hline
PM2.5年终平均浓度($\mu$g/m$^3$) & 280 & 153 & 87 & 53 & 35 \\
\hline
年份 & 第1年 & 第2年 & 第3年 & 第4年 & 第5年 \\
\hline
PM2.5浓度变化率($\mu$g/m$^3$) & 0 & 127 & 66 & 34 & 18 \\
\hline
\end{tabular}
\end{table}

根据附件一武汉市的监测数据可知,武汉市2013.01.01-08.26所有数据中PM2.5浓度的最低值为15.4$\mu$g/m$^3$,假定PM2.5=15.4$\mu$g/m$^3$是武汉市PM2.5浓度的最小值,对武汉市未来10年PM2.5治理方案的效果进行估计,如图6-2所示:

\begin{figure}[h]
\centering
\includegraphics[width=\textwidth]{image.png}
\caption{未来10年内武汉市PM2.5浓度达标所需时间}
\end{figure}

从图6-2可知,武汉市想要在未来十年内实现PM2.5平均浓度下降到15.4$\mu$g/m$^3$以下,需要参照治理方案中的修正指数曲线模型,严格执行治理方案的同时,保证资金、技术等各方面的投入,才能在10年之后实现PM2.5年末平均浓度降低到15.4$\mu$g/m$^3$以下。

\subsection{武汉市未来5年治理方案投入的经费预算}

\subsubsection{武汉市降低PM2.5浓度的目标优化}

武汉市未来5年在降低PM2.5浓度方面需要资金的持续投入,资金的费用既要考虑综合治理费用,又要考虑专项治理费用。其中,综合治理费用的标准为每减少$N$浓度单位的PM2.5,需要花费当年$N$个单位(百万元)的投入费用。专项治理费用为降低$N\mu$g/m$^3$的PM2.5浓度需要花费$N^2\times0.005$百万的专项资金。

因此本题可转化为以修正指数曲线为基础的目标优化,既要让武汉市未来5年PM2.5年平均浓度达到预定指标,又要使综合治理指标和专项治理指标的费用

最少,在上一问结论的基础上,假定第一年 PM2.5 浓度降低量为 $0 \, \mu \mathrm{g} / \mathrm{m}^3$,剩余 4 年 PM2.5 浓度降低的减少量分别为 $x_2, x_3, x_4, x_5 \, \mu \mathrm{g} / \mathrm{m}^3$,则目标函数是综合费用与专项治理费用均满足最低:

目标函数: $\min (x_2^2 + x_3^2 + x_4^2 + x_5^2) \times 0.005$

约束条件中,由于综合治理费用与 PM2.5 减少量之间呈线性相关,且未来 5 年武汉市 PM2.5 浓度的减少量的要求是在第 5 年年末达到 $35 \, \mu \mathrm{g} / \mathrm{m}^3$,所以 PM2.5 浓度的减少量为 $(280-35) \, \mu \mathrm{g} / \mathrm{m}^3$,即 $245 \, \mu \mathrm{g} / \mathrm{m}^3$,所以减少量满足目标:

\begin{equation}
x_2 + x_3 + x_4 + x_5 > (280-35)
\tag{6-3}
\end{equation}

依据武汉市 2013.01.01-08.26 数据中 PM2.5 浓度的最低值,假定武汉市未来 10 年 PM2.5 浓度的最小值是 PM2.5=15.4 $\mu \mathrm{g} / \mathrm{m}^3$,所以满足:

\begin{equation}
x_2 + x_3 + x_4 + x_5 \leq (280-15.4)
\tag{6-4}
\end{equation}

修正指数曲线模型中,初期增长迅速,随着时间的增加,增长率逐渐降低,所以有:

\begin{equation}
x_2 > x_3 > x_4 > x_5
\tag{6-5}
\end{equation}

综上所述,建立目标优化模型:

目标函数:$\min (x_2^2 + x_3^2 + x_4^2 + x_5^2) \times 0.005$

约束条件
\begin{equation}
\text{s.t. } \begin{cases}
x_2 + x_3 + x_4 + x_5 > (280-35) \\
x_2 + x_3 + x_4 + x_5 \leq (280-15.4) \\
x_2 > x_3 > x_4 > x_5
\end{cases}
\tag{6-6}
\end{equation}

\subsection*{6.2.2 目标优化模型求解}

针对上述目标优化模型,利用 matlab 编程搜索算法(程序见附录),求解所有满足条件的约束方案,由公式 6-6 算的武汉市 5 年 PM2.5 年平均浓度的减少量分别为 $63 \, \mu \mathrm{g} / \mathrm{m}^3$、$62 \, \mu \mathrm{g} / \mathrm{m}^3$、$61 \, \mu \mathrm{g} / \mathrm{m}^3$、$60 \, \mu \mathrm{g} / \mathrm{m}^3$,对应五年总的专项费用投资金额为 75.67 百万元,虽然满足专项投资金额最小的目标约束,但是不难发现费用最少的方案中 PM2.5 每年的减少量是大致相等的,这样的治理方案并不符合实际情况,所以需要对已有的目标规划进行优化,增加约束条件:

\begin{equation}
Y = 280, t \leq 1
\end{equation}

\begin{equation}
Y = 15.4 + 580.18 \times 0.52^t, \quad t \geq 2
\tag{6-7}
\end{equation}

由 6-6、6-7 可以求得 $x_2 = 126 \, \mu \mathrm{g} / \mathrm{m}^3$、$x_3 = 66 \, \mu \mathrm{g} / \mathrm{m}^3$、$x_4 = 35 \, \mu \mathrm{g} / \mathrm{m}^3$、$x_5 = 18 \, \mu \mathrm{g} / \mathrm{m}^3$,此结果是在不考虑费用约束最小的计算结果,考虑到综合治理费用和专业治理费用的约束后,则需要对目标规划进行修正,经过修正后的目标优化结果如表 6-2 所示:

\begin{table}[h]
\centering
\caption{未来 5 年内 PM2.5 年终浓度变化量}
\begin{tabular}{c c c c c c}
\hline \hline
年份 & 第 1 年 & 第 2 年 & 第 3 年 & 第 4 年 & 第 5 年 \\
\hline
PM2.5 浓度变化率 ($\mu \mathrm{g} / \mathrm{m}^3$) & 0 & 121 & 71 & 40 & 13 \\
\hline \hline
\end{tabular}
\end{table}

武汉市未来 5 年内,每年年末 PM2.5 的减少量分别为 $0 \, \mu \mathrm{g} / \mathrm{m}^3$、$121 \, \mu \mathrm{g} / \mathrm{m}^3$、

\section*{6.2.3 治理方案有效性分析}

首先在技术上提出了多种手段,多管齐下综合治理 PM2.5 的方案。采用在市区设喷泉、加大绿化面积,限制高耗能高排放企业等多手段,而且要治理 PM2.5 不是单方面去消除 PM2.5,而是在消减 PM2.5 时同时消减其他污染物,如种植植物可以同时吸收多种污染物。

在消减 PM2.5 的年指标上,综合考虑到项目的滞后性,即项目的产生到项目实施投入的时间差,认为第一年无消减目标。在往后的消减指标中,由于浓度大越易于处理,所以每年的治理指标较高。但是随着浓度的降低,治理就越困难,所以指标也相应减少。

在资金投入上,相比于最优(但是不符合实际)的资金投入多投入 42%,为 107.47 百万元,在武汉市目前的发展条件下是完全可以的。

\section{模型检验与评价}

\subsection{模型的优点}

1. 对于所给数据的整行或整段缺失情况,我们并没有直接删除,而是对其做了插值处理,保证了数据在时间上的连续性。

2. 针对不同条件下的 PM2.5 扩散衰减情况,我们建立了较为合适的模型,以二维多箱模型对应面源扩散、高斯扩散模型对应点源扩散,充分体现了我们考虑问题的严谨性。

3. 由于数据量和变量因素较多,我们在处理问题的时候一般先用某个样本数据作为研究对象,考虑简单模型,再逐步深入研究,难度由浅入深。

\subsection{模型的缺点}

1. 由于 PM2.5 产生的复杂性及受多种因素的影响,模型只能在短期时间的预测有效。

2. 对于问题三,由于约束条件过少以及对问题分析不够细致,结果与实际偏差可能较大。

3. 在单箱大气质量模型中没有考虑风场随高度的变化,也没有考虑到研究区域内大气污染物分布的不均匀性。因此,单箱大气质量模型预测大气中污染物浓度分布会存在一定的误差.

\section{参考文献}

[1] L. Ortolano 主编,华南环境科学研究所译,环境规划与决策,北京:中国环境科学出版社,1988,98-104

[2] 程声通,环境系统分析,北京:高等教育出版社,1990,442-453

[3] 刘天齐主编,城市环境规划规范及方法指南,北京,中国环境科学出版社,1991,36-43

[4] 夏青主编,大气环境综合整治规划,北京:中国环境科学出版社,1990,89-92

[5] 曲绍清,单层多箱模型及其应用,上海环境科学,1991,10(5):38-41

[6] 程水源等,二维多箱模型预测大气环境方法的研究,环境科学,1998,19(2):16-19

[7] Mcdonald, K.M., Cheng, L., and Angle, R.P. A comparison of box and plume model calculations for sculpture deposition and flux in Alberta. Atmospheric Environment, 1996,30(17):2969-2980

[8] 程水源等,城市工业区大气污染物浓度场确定方法的研究,化工环保,1997(17):114-118

[9] 程水源等,对几种大气环境预测方法的评估,环境科学,1991,12(3):85-88

[10] 程水源等,关于大气环境预测的新探讨,环境科学丛刊,1991,12(2):34-38

[11] 谷清,李云生,《大气环境模式计算方法》,2002.3:1-40

\begin{table}
\centering
\caption{西安地区PM2.5各级出现的频数(2013年1月至9月)}
\begin{tabular}{c c c c c c c c c c c c c}
\hline
月份 & 等级 & 高压开关厂 & 兴庆小区 & 纺织城 & 小寨 & 市人民体育场 & 高新西区 & 经开区 & 长安区 & 阎良区 & 临潼区 & 曲江文化集团 & 广运潭 & 草滩 \\
\hline
1 & 1 & 0 & 0 & 0 & 0 & 0 & 0 & 0 & 0 & 0 & 0 & 0 & 0 & 0 \\
1 & 2 & 3 & 2 & 2 & 3 & 3 & 2 & 3 & 2 & 3 & 3 & 2 & 2 & 2 \\
1 & 3 & 1 & 3 & 5 & 2 & 2 & 2 & 1 & 3 & 1 & 4 & 5 & 5 & 0 \\
1 & 0 & 0 & 0 & 0 & 0 & 0 & 0 & 0 & 0 & 0 & 0 & 0 & 0 & 0 \\
2 & 2 & 0 & 0 & 0 & 0 & 0 & 0 & 1 & 0 & 0 & 0 & 1 & 0 & 0 \\
2 & 3 & 1 & 1 & 3 & 3 & 0 & 2 & 2 & 1 & 3 & 2 & 7 & 2 & 2 \\
1 & 0 & 0 & 0 & 0 & 0 & 0 & 0 & 0 & 0 & 0 & 0 & 0 & 0 & 0 \\
3 & 2 & 1 & 1 & 0 & 1 & 0 & 1 & 1 & 0 & 1 & 1 & 1 & 1 & 1 \\
3 & 3 & 6 & 6 & 2 & 13 & 7 & 7 & 6 & 12 & 7 & 5 & 10 & 4 & 4 \\
1 & 0 & 0 & 0 & 0 & 0 & 0 & 0 & 0 & 0 & 0 & 0 & 0 & 0 & 0 \\
4 & 2 & 2 & 8 & 5 & 11 & 8 & 7 & 5 & 12 & 7 & 10 & 11 & 4 & 4 \\
4 & 3 & 10 & 11 & 14 & 7 & 7 & 6 & 11 & 8 & 11 & 7 & 8 & 6 & 14 \\
1 & 1 & 2 & 5 & 2 & 2 & 1 & 2 & 1 & 3 & 1 & 1 & 3 & 2 & 1 \\
5 & 2 & 12 & 15 & 13 & 12 & 9 & 10 & 8 & 14 & 16 & 9 & 16 & 11 & 6 \\
5 & 3 & 10 & 11 & 6 & 10 & 6 & 11 & 14 & 7 & 10 & 12 & 11 & 13 & 15 \\
1 & 2 & 8 & 4 & 4 & 10 & 5 & 5 & 5 & 6 & 3 & 7 & 14 & 5 & 2 \\
6 & 2 & 16 & 13 & 19 & 15 & 12 & 16 & 15 & 18 & 22 & 15 & 13 & 17 & 16 \\
6 & 3 & 4 & 4 & 4 & 3 & 4 & 8 & 7 & 5 & 4 & 6 & 2 & 7 & 8 \\
1 & 3 & 7 & 4 & 7 & 7 & 1 & 3 & 1 & 9 & 4 & 6 & 10 & 3 & 1 \\
7 & 2 & 14 & 12 & 12 & 12 & 10 & 14 & 13 & 10 & 13 & 10 & 13 & 12 & 13 \\
7 & 3 & 12 & 9 & 9 & 7 & 9 & 8 & 10 & 7 & 12 & 9 & 8 & 3 & 8 \\
1 & 1 & 3 & 2 & 4 & 4 & 0 & 3 & 0 & 3 & 2 & 2 & 4 & 1 & 1 \\
8 & 2 & 15 & 15 & 21 & 16 & 6 & 10 & 14 & 14 & 19 & 14 & 18 & 13 & 16 \\
8 & 3 & 10 & 8 & 5 & 6 & 5 & 7 & 10 & 7 & 6 & 6 & 7 & 11 & 8 \\
1 & 0 & 0 & 0 & 0 & 1 & 0 & 1 & 0 & 1 & 0 & 0 & 1 & 0 & 0 \\
9 & 2 & 4 & 5 & 6 & 5 & 2 & 4 & 4 & 4 & 5 & 2 & 8 & 5 & 3 \\
9 & 3 & 8 & 8 & 9 & 9 & 4 & 9 & 6 & 9 & 6 & 4 & 9 & 6 & 10 \\
\hline
\end{tabular}
\end{table}

\section*{附录:}

\section*{附录1. \%将PM2.5指数转化为浓度}

\section*{\%将PM2.5指数转化为浓度}

\section*{PM=[];\%输入各站点每天的pm2.5指标}

\begin{verbatim}
C=[];%输出各站点每天PM2.5浓度
[n,m]=size(PM);
for i=1:n
    for j=1:m
        if PM(i,j)<=50
            MINpm=0;
            MAXpm=50;
            MINbp=0;
            MAXbp=35;
        end
        if PM(i,j)>50&&PM(i,j)<=100
            MINpm=50;
            MAXpm=100;
            MINbp=35;
            MAXbp=75;
        end
        if PM(i,j)>100&&PM(i,j)<=150
            MINpm=100;
            MAXpm=150;
            MINbp=75;
            MAXbp=115;
        end
        if PM(i,j)>150&&PM(i,j)<=200
            MINpm=150;
            MAXpm=200;
            MINbp=115;
            MAXbp=150;
        end
        if PM(i,j)>200&&PM(i,j)<=300
\end{verbatim}

\begin{verbatim}
MINpm=200;
MAXpm=300;
MINbp=150;
MAXbp=250;
end

if PM(i,j)>300&&PM(i,j)<=400
    MINpm=300;
    MAXpm=400;
    MINbp=250;
    MAXbp=350;
end

if PM(i,j)>400&&PM(i,j)<=500
    MINpm=400;
    MAXpm=500;
    MINbp=350;
    MAXbp=500;
end

C(i,j)=((PM(i,j)-MINpm)*(MAXbp-MINbp))/(MAXpm-MINpm)+MINbp; %根据附件里的PM指标计算公式反推出浓度
end
end
C
\end{verbatim}

\section*{附录2. \%求各站点随污染源变化的坐标程序}

\begin{verbatim}
PM=[
数据略
];
[n,m]=size(PM);
p=[];
for i=1:n
\end{verbatim}

\begin{verbatim}
temp=0;
for j=1:m
    if PM(i,j)>temp
        p(i)=j;%记录每天PM2.5指标最大的站点,并假设以该站点为污染源
        temp=PM(i,j);
    end
end
end
p;

JW=[
数据略
];%输入各站点的经纬度;
D=[];
c=length(p);
for k=1:c
    for i=1:13
        d=6371*(acos(sin(JW(i,1)/180*pi)*sin(JW(p(k),1)/180*pi)+cos(JW(i,1)/180*pi)*cos(JW(p(k),1)/180*pi)*cos(JW(i,2)/180*pi-JW(p(k),2)/180*pi)));%将两点的经纬度转化为两点之间的距离
        D(k,i)=d;
    end
end
D;

W1=[];
for k=1:c
    for i=1:13
        W1(i,k)=JW(i,1)-JW(p(k),1);
        J1(i,k)=JW(i,2)-JW(p(k),2);
    end
end
\end{verbatim}

\begin{verbatim}
end
end
W1;
J1;

C=[];
for i=1:13
    for j=1:c
        C(i,j)=W1(i,j)/J1(i,j);
    end
end
C;
L1=[];
for i=1:13
    for j=1:c
        L1(i,j)=sqrt(C(i,j)^2+1);
    end
end
L1;

for i=1:c
    for j=1:13
        Y(j,i)=D(i,j)/L1(j,i);
    end
end
Y;%每天污染源的y坐标;
for i=1:13
    for j=1:c
        X(i,j)=C(i,j)*Y(i,j);
    end
end
\end{verbatim}

\begin{verbatim}
end
end
X;%每天污染源的X坐标

附录3.%PM2.5浓度扩散图(2-2)

%浓度扩散图
xyz=[
经纬度坐标略
];
x=xyz(:,1);
y=xyz(:,2);
z1=[
数据略
];
z=z1(1,:);
x1=34.14:0.005:34.67;
y1=108.87:0.005:109.24;
[x1,y1]=meshgrid(x1,y1);
z1=griddata(x,y,z,x1,y1,'v4');
subplot(1,1,1)
mesh(x1,y1,z1);
hold on
[c,h]=contour(x1,y1,z1);
clabel(c,h)
xlabel('纬度');
ylabel('经度');
hold on
title('西安市突发情况下扩散图')
plot(x,y,'r*')
\end{verbatim}

\begin{verbatim}
附录4. 突发情形下的PM2.5污染扩散预测模型

%这是主程序,命名为lianxuwuran.m
function Value=lianxuwuran_2(t)
%连续污染扩散模型
global xx yy zz;
xmin=-20;dx=1;xmax=50;ymin=-10;dy=1;ymax=30;zmin=0;dz=1;zmax=10;
Cxyz_t((xmax-xmin)/dx+1,(ymax-ymin)/dy+1,(zmax-zmin)/dz+1)=0;
ii=0;jj=0;kk=0;
for zz=zmin:dz:zmax;kk=kk+1;
    for yy=ymin:dy:ymax;jj=jj+1;
        for xx=xmin:dx:xmax;ii=ii+1;
            Cxyz_t(ii,jj,kk)=quadl(@fun3D,1,t);
        end
        ii=0;
    end
    jj=0;
end
Cxyz_tt2=Cxyz_t;
M=200000;K=0.01;
Dx=0.93^2;Dy=0.93^2;Dz=1.57^2;ux=0;uy=0;uz=0;
[x,y,z]=meshgrid(xmin:dx:xmax,ymin:dy:ymax,zmin:dz:zmax);
C_Point=M./(8.*(pi.*t).^(3./2).*sqrt(Dx.*Dy.*Dz)).*exp(-(x-ux).^2./(4.*Dx.*t)-(y-uy).^2./(4.*Dy.*t)-(z-uz).^2./(4.*Dz.*t)).*exp(-K.^2.*t);
if (t<=30&t>0)
figure(1);

subplot(1,1,1);[c,h]=contour(Cxyz_tt2(:,:,11),10);clabel(c,h);grid on;
title('污染源中心扩散示意图');
xlabel('x轴坐标/(km)'); ylabel('y轴坐标/(km)');
else t>30
figure(1);
subplot(1,1,1);[c,h]=contour(Cxyz_tt2(:,:,11),8);grid on;
clabel(c,h);axis tight;title('污染源中心扩散示意图');
xlabel('x轴坐标/(km)'); ylabel('y轴坐标/(km)');

end
%%这是子程序,命名为:fun3D.m
function CPoint=fun3D(t)
global xx yy zz;
Q=60000;Cq=12;K=4.2/(24*60*6);
Dx=50;Dy=5;Dz=2;ux=1.5;uy=0.2;uz=0.1;
CPoint=Cq*Q./(8.*(pi.*t).^(3./2).*sqrt(Dx.*Dy.*Dz)).*exp(-(xx-ux.*t).^2./(4.*Dx.*t)-(yy-uy.*t).^2./(4.*Dy.*t)-(zz-uz.*t).^2./(4.*Dz.*t)).*exp(-K.*t);
\end{verbatim}

\begin{verbatim}
附录5.%PM2.5整治计划

x=0:5;
n=length(x);
for i=1:n;
    if i==1;
        y(i)=280;
    else
        y(i)=15.4+507.188*(0.5217^(x(i)));
    end
end
plot(x,y,'b-')
\end{verbatim}