\begin{center}
\textbf{\Large 全国第四届研究生数学建模竞赛}
\end{center}

\begin{center}
\includegraphics[width=0.4\textwidth]{image.png} \quad
\begin{tabular}{|c|}
\hline
题号 C \\
\hline
\end{tabular}
\end{center}

\textbf{题目} \underline{基于提高高速公路路面质量改进方案的探讨}

\section{摘 要:}

本文探讨了提高高速公路路面质量的改进方案。对于问题 1,在对四个指标两两进行相关性分析后得出它们之间的关系是非线性的,在分析了各种非线性回归方法后,我们采用多元二项式回归方法求得了它们之间的数学模型,对回归方程进行检验后,剩余回归标准差为 2.9,比较接近于 0,认为模型的回归效果比较好;同时由于智能算法具有很强的非线性数据处理能力,我们也采用了 BP 神经网络模型对问题 1 进行分析和模型检验,认为 BP 神经网络模型能够建立比较精确的四个指标之间的数量关系。对于问题 2,我们用多因素方差分析的方法分别求出了对四个指标影响显著的因素,然后用多元回归的方法分别建立了它们之间的数学模型,模型的剩余回归标准差为 3 左右。对于问题 3,本文采用了建立立方体单元模型方案和多元线性回归分析方案分别用理论模型和数值分析的方法分析了集料的筛孔通过率与路面压实度上界的数量关系,多元线性回归分析的相关系数 $R^2$ 为 0.96。对于问题 4,我们采用单因素方差分析法得出不同型号的沥青和碎石质量对路面的性能影响比较显著,其中 16 型号沥青对路面性能影响最大,采用双因素方差分析法得出厂家、产地不同的沥青和碎石对路面性能影响不大。结合对数据分析的结果,本文提出应增加路面的使用寿命与损害情况的调查、交通流量调查以及超载现象、施工情况的调查等调查数据的测试项目,从提高高速公路路面低温性能、高温性能、抗水损害性能三个方面提出了一些建议。

关键词:路面质量;回归分析;BP 神经网络;方差分析;立方体单元模型

\section{问题的重述}

高速公路路面的寿命对降低高速公路的运行成本、保障运输安全有着极其重要的意义。由于我国建设高速公路的历史不长,目前我们对高速公路路面寿命的客观规律的了解还不太全面、深入。为此,我国高速公路的设计者和建设者为延长高速公路路面寿命做了大量的试验和调查研究,研究发现对高速公路路面的质量有较大影响或有影响的因素的数据中所涉及的指标主要有油石比、筛孔通过率、VV(空隙率)、VMA(矿料间隙率)、VFA(饱和度)、DP(粉胶比)、毛体积密度、最大理论密度、\%Gmm(最初压实度)、\%Gmm(最大压实度)九种。而关于高速公路路面质量的试验指标主要有以下四种:

1) TSR (\%) ——冻融劈裂强度比,反映混合料抗水损害性能的指标,越大越好。

2) SO (\%) ——浸水马歇尔稳定度比,反映混合料抗水损害性能的指标,越大越好。

3) 车辙——动稳定度,为车辙试验中试件变形稳定时,荷载作用次数与变形深度的比值,单位是“次/mm”,反映混合料的抗车辙性能(高温性能),越大越好。

4) 弯拉应变,为低温小梁弯曲试验时的破坏应变。反映混合料的低温变形能力,越大越好。

现在我们要探讨的问题主要有以下几个:

1) 描述高速公路路面质量的抗水损害性能、高温性能、低温性能的四个指标之间有没有数量关系?如果有数量关系,那么它们之间的数学模型又是什么呢?

2) 您认为影响高速公路路面质量的最重要的和比较重要的因素主要有哪些,选择这些因素的理由是什么?建立它们和描述高速公路路面的抗水损害性能、高温性能、低温性能的四个质量指标之间比较精确的数学模型,根据模型您认为采用什么样的方案可以提高高速公路路面的质量。

3) 若从理论上探讨集料的筛孔通过率(即级配组成,指混合料中粗细不同集料的构成比例)与路面压实度的上界(数学意义下的上界)之间的数量关系,您有何见解?

4) 沥青(在数据表格中混合料类型中含25是普通沥青,是其他数字如13、16或20的是改性沥青)、碎石质量对高速公路路面抗水损害性能、高温性能、低温性能究竟有没有影响,您的观点是什么?根据是什么?不同厂家、不同产地的但型号相同、类型相同的沥青、碎石质量对高速公路路面抗水损害性能、高温性能、低温性能究竟有没有影响,您的观点是什么?根据又是什么?

5) 根据您对数据分析的结果,现在测试高速公路路面质量的试验项目中是否有重要的遗漏?您对高速公路建设部门有什么建议?

\section{问题分析}

由于该问题的数据是在多种试验条件下对不同的来自不同石料产地、不同石料岩性、不同沥青型号及厂家、不同混合料类型的路面材料而得到的,而且有的试验数据也不够全面,因此我们首先要做的就是对数据进行筛选和分类,确保我们的试验数据具有可比性。

\section{3. 模型的假设}

1) 不考虑气候等因素对试验结果的影响。

2) 假定矿料基本颗粒为规则的球体。

3) 假定各级颗粒粒径均相等,

4) 假设路面结构由立方体单元结构组成,每个立方体单元结构由若干个直径不同的圆球组成,圆球与圆球之间,圆球与立方体单元之间均相切接触,立方体的厚度为圆球的直径,截面为正方形;

5) 假设通过同一筛孔的粒料由同一型号的立方体单元机构组成,该型号的立方体单元

\section{4. 符号说明}

\begin{align*}
x_1 &: \text{TSR(%)} \\
x_2 &: \text{车辙} \\
x_3 &: \text{弯拉应变} \\
y &: \text{SO(%)} \\
a &: \text{立方体单元结构的边长} \\
P_i &: \text{集料的筛孔通过率} \\
P_v &: \text{路面压实度}
\end{align*}

\section{5. 模型的建立和求解}

\subsection{5.1 问题1的求解}

\subsubsection{5.1.1 问题1的多元回归方法分析}

由于该问题的数据是在多种试验条件下对不同的来自不同石料产地、不同石料岩性、不同沥青型号及厂家、不同混合料类型的路面材料而得到的,而且有的试验数据也不够全面。因此首先我们根据实际情况对这些数据进行分析和处理,我们的处理原则是:

1) 在“通过筛孔百分率”中,根据理论知识和实际情况我们分析认为“31.5”、“26.5”下面没有数据的地方均补齐为100\%。

2) 对于一行数据中出现多列数据为“\textbackslash”或空白,我们认为该项试验做的不完整或者不成功,试验结果具有不可信性,我们做删除处理。

3) 对于一些很明显的数据误差或者错误,我们认为这是试验者记录或其它原因造成的误差,做删除处理。

在此处理原则下,为了结果的准确性,我们分了多种情况下探讨 TSR(%)、SO(%)、车辙、弯拉应变四个指标的数量关系:

情形1:不考虑石料岩性、产地、沥青型号及厂家、混合材料类型等因素等试验结果的影响

我们对 TSR(%)、SO(%)、车辙、弯拉应变四个指标进行相关系数分析,发现它们之间的相关系数如下表所示:

\begin{table}[h]
\centering
\caption{在多种条件作用下的四个指标的相关性分析}
\begin{tabular}{|c|c|c|c|c|}
\hline
 & TSR(%) & SO(%) & 车辙 & 弯拉应变 \\
\hline
TSR(\%) & 1 &  &  &  \\
\hline
SO(\%) & 0.300754 & 1 &  &  \\
\hline
车辙 & 0.215147 & 0.237973 & 1 &  \\
\hline
弯拉应变 & 0.150972 & 0.19959 & 0.334516 & 1 \\
\hline
\end{tabular}
\end{table}

\textbf{情形2:将石料岩性分为石灰岩、和玄武岩两类来分析四个指标之间的相关系数}


\begin{table}
\centering
\caption{石料岩性为石灰岩条件下的四个指标之间的相关系数}
\begin{tabular}{|c|c|c|c|c|}
\hline
 & TSR(\%) & SO(\%) & 车辙 & 弯拉应变 \\
\hline
TSR(\%) & 1 &  &  &  \\
\hline
SO(\%) & 0.248381 & 1 &  &  \\
\hline
车辙 & 0.286548 & 0.251511 & 1 &  \\
\hline
弯拉应变 & -0.0055 & 0.028091 & 0.18149 & 1 \\
\hline
\end{tabular}
\end{table}


\begin{table}
\centering
\caption{石料岩性为玄武岩条件下的四个指标之间的相关系数}
\begin{tabular}{|c|c|c|c|c|}
\hline
 & TSR(\%) & SO(\%) & 车辙 & 弯拉应变 \\
\hline
TSR(\%) & 1 &  &  &  \\
\hline
SO(\%) & 0.34631 & 1 &  &  \\
\hline
车辙 & 0.19811 & 0.284314 & 1 &  \\
\hline
弯拉应变 & -0.00222 & -0.08861 & 0.123968 & 1 \\
\hline
\end{tabular}
\end{table}

\textbf{情形3:将混合材料类型分为SUP系列、SMA系列、AC系列、25系列来分析四个指标之间的相关系数}


\begin{table}
\centering
\caption{混合材料为AC系列时,四个指标之间的相关系数}
\begin{tabular}{|c|c|c|c|c|}
\hline
 & TSR(\%) & SO(\%) & 车辙 & 弯拉应变 \\
\hline
TSR(\%) & 1 &  &  &  \\
\hline
SO(\%) & 0.478143 & 1 &  &  \\
\hline
车辙 & 0.277501 & 0.372992 & 1 &  \\
\hline
弯拉应变 & 0.137498 & 0.341922 & 0.468064 & 1 \\
\hline
\end{tabular}
\end{table}


\begin{table}
\centering
\caption{混合材料为SMA系列时,四个指标之间的相关系数}
\begin{tabular}{|c|c|c|c|c|}
\hline
 & TSR(\%) & SO(\%) & 车辙 & 弯拉应变 \\
\hline
TSR(\%) & 1 &  &  &  \\
\hline
SO(\%) & 0.387657 & 1 &  &  \\
\hline
车辙 & 0.063659 & -0.06649 & 1 &  \\
\hline
弯拉应变 & -0.06183 & -0.18721 & 0.026383 & 1 \\
\hline
\end{tabular}
\end{table}


\begin{table}
\centering
\caption{混合材料为SUP系列时,四个指标之间的相关系数}
\begin{tabular}{|c|c|c|c|c|}
\hline
 & TSR(\%) & SO(\%) & 车辙 & 弯拉应变 \\
\hline
TSR(\%) & 1 &  &  &  \\
\hline
SO(\%) & 0.078724 & 1 &  &  \\
\hline
车辙 & 0.149324 & 0.414563 & 1 & \\
\hline
弯拉应变 & -0.04277 & -0.03808 & 0.03734 & 1 \\
\hline
\end{tabular}
\end{table}

而当混合型材料均为25系列时,弯拉应变这一列数据为空。

情形 4:根据以上的相关性分析,选取两个相关性系数比较大的石料岩性为玄武岩条件、混合材料为AC系列进行综合分析


\begin{table}[h]
\centering
\caption{石料岩性为玄武岩条件、混合材料为AC系列下四个指标之间的相关系数}
\begin{tabular}{|c|c|c|c|c|}
\hline
 & TSR(\%) & SO(\%) & 车辙 & 弯拉应变 \\
\hline
TSR(\%) & 1 & & & \\
\hline
SO(\%) & 0.000786 & 1 & & \\
\hline
车辙 & 0.022175 & -0.08588 & 1 & \\
\hline
弯拉应变 & 0.193657 & 0.24552 & 0.064995 & 1 \\
\hline
\end{tabular}
\end{table}

选取正相关系数最大的 0.478,即 TSR(\%)和 S0(\%)进行线性回归,绘出它们之间的线性拟合图:


\begin{figure}[h]
\centering
\includegraphics[width=0.8\textwidth]{image.png} \% 替换为实际图像文件名
\caption{相关系数最大的两个指标之间的线性拟合图}
\end{figure}


通过以上相关性分析,我们可以很清楚的发现,在各种情况下,影响高速公路路面质量的四个试验指标之间的相关系数最大只有0.48,小于0.5。因此我们可以认为四个试验指标之间的相关系数很小,即它们之间没有线性相关性。因此,我们取四个变量中的一个变量为因变量,另外三个变量为自变量,建立它们之间的数学模型。

首先,我们分别选取任意一个变量作为因变量,其它三个变量作为自变量,采用多种多元非线性回归方法分析后,我们选择多元二项式回归方法进行回归分析。下表是我们分别采用变量TSR(\%)、S0(\%)、车辙、弯拉应变四个指标为因变量,其它三个变量为自变量时的回归剩余标准差,在理论上来说,回归剩余标准差越趋近于0,回归模型效
果越好


\begin{table}[h]
\centering
\caption{分别选取任意一个变量为因变量时回归剩余残差表}
\begin{tabular}{|p{0.3\textwidth}|p{0.6\textwidth}|}
\hline
因变量 & 回归剩余标准差 \\
\hline
TSR(\%) & 3.4435 \\
\hline
SO(\%) & 2.9152 \\
\hline
车辙 & 1.3543e+003 \\
\hline
弯拉应变 & 2.2319e+003 \\
\hline
\end{tabular}
\end{table}

回归剩余残差最小,此时,模型的回归效果也是最好的,所得到的模型也是最精确的,因此,我们采用 $S0(\%)$ 作为因变量,$TSR(\%)$、车辙、弯拉应变三个变量为自变量,建立它们之间的数学模型关系。我们把 $TSR(\%)$ 设为 $x_1$,车辙设为 $x_2$,弯拉应变设为 $x_3$,$S0(\%)$ 设为 $y$。用 MATLAB 软件做多元二项式回归,具体程序参见附录 1。

所得到的数学模型为:
\begin{align*}
y &= -124.828928743796 + 4.603391118663x_1 + 0.001656876237x_2 + 0.000832093552x_3 \\
&\quad - 0.025310896815x_1^2 - 0.000000126093x_2^2 - 0.000000027793x_3^2
\end{align*}

\begin{figure}[h]
\centering
\includegraphics[width=\textwidth]{image.png}
\caption{多元二项式回归曲线图}
\end{figure}

上图是用 matlab 软件进行回归时,得到的一个反映关于四个因素之间关系的交互式画面,有三幅图形,其中左边的一幅是 $x_2$、$x_3$ 固定时的曲线 $y(x_1)$ 及其置信区间,中间是 $x_1$、$x_3$ 固定时的曲线 $y(x_2)$ 及其置信区间,右边是 $x_1$、$x_2$ 固定时的曲线 $y(x_3)$ 及其置信区间。

经过回归分析,发现其模型的回归剩余残差为 2.9152,接近于 0,故认为回归的效果比较显著,模型比较精确。

\subsubsection{5.1.2 问题 1 的 BP 神经网络模型求解}

BP 神经网络是在人类对大脑神经网络认识理解的基础上人工构建的能够实现某种功能的数学模型,它由大量简单单元件相互连接而成的复杂网络,具有高度的并行性、高度的非线性全局作用、良好的容错性和联想记忆功能、十分强大的自适应、自学习能力。在人工神经网络的实际应用中,BP 网络广泛应用于函数逼近、模式识别与分类、数据压缩等,80~90\% 的人工神经网络模型是采用 BP 网络或它的变化形式,它是前馈网络的核心部分,体现了人工神经网络最精华的部分。由于这些突出的优点,结合高速公路路面质量的抗水损害性能、高温性能、低温性能的四个指标之间存在的非线性问题,本文认为 BP 神经网络在分析四个指标之间的数量关系中的应用是可行的。

\begin{figure}[h]
    \centering
    \includegraphics[width=0.8\textwidth]{bp_neural_network_diagram.png}
    \caption{BP 神经网络模型图解}
    \label{fig:bp_neural_network}
\end{figure}

BP 神经网络即为误差反向传播(Back Propagation)神经网络,是神经网络中应用最为广泛的一种学习算法。主要过程为:

1) 输入已知学习样本,通过设置的网络结构和前一次迭代的权值和阈值,从网络的第一层向后计算各神经元的输出;

2) 对权值和阈值进行修改,从最后一层向前计算各权值和阈值对总误差的影响(梯度),据此对各权值和阈值进行修改。

以上两个过程反复交替,直到达到收敛为止。

1) 确定网络层数:如果样本空间过大,应建立两个以上的隐层;由于隐层增多会影响计算速度,一般情况下,只需一个隐层,本实验采用一个隐层。

2) 输入层节点数在四个指标中选择三个指标,输出层的节点数为四个指标中剩下的一个指标。这样就有四种输入输出层方案,本文从中选取一种方案,即将 SO 指标、车辙指标、弯拉应变指标作为输入层节点数,将 TSR 指标作为输出层节点数。

3) 隐层的节点数:根据经验公式确定隐层节点个数。

4) 传输函数 BP 网络中的传输函数通常采用 S(sigmoid)型函数如下式:
\begin{equation}
f(x) = \frac{1}{1 + e^{-x}}
\end{equation}

BP 网络是一种单向传播的多层前向网络,分为输入层、隐含层和输出层,层与层之间多采用全连接方式,同一层单元之间不存在相互连接。其中隐含层可为多层。数学上已经证明过 3 层网络即可模拟任何非线性问题,BP 学习算法的学习过程有正向传播和反向传播两部分组成,在正向传播过程中,输入模式从输入层经过隐层神经元的处理后,传向输出层,每一层神经元的状态只影响下一层神经元状态。如果在输出层得不到期望的输出,则转入反向传播,此时误差信号从输出层向输入层传播并沿途调整各层间连接权值和阈值,以使误差不断减小,直到达到精度要求。

\begin{equation}
Y_j = f\left(\sum_{i=1}^{l} W_{kj} X_i - \theta_j\right) \tag{1}
\end{equation}
隐含层:

输出层:
\begin{equation}
Y_{k} = f\left( \sum_{j=1}^{m} W_{kj} Y_{j} - \gamma_{k} \right)
\tag{2}
\end{equation}

式中:$X_{i}$ 表示网络的输入;$Y_{j}$ 表示隐层单元和 $j$ 个节点的输出;$Y_{k}$ 表示输出层第 $k$ 个节点的实际输出;$f(x)$ 表示激活函数;$w_{ji}$、$w_{kj}$ 分别表示输入层到隐层和隐层到输出层之间的连接权向量;$\theta_{j}$、$\gamma_{k}$ 分别表示隐层和输出层的阈值。

定义误差函数为:
\begin{equation}
E_{p} = \frac{1}{2} \sum_{k} \left( O_{pk} - Y_{pk} \right)^{2}
\end{equation}

式中:$O_{pk}$ 为第 $P$ 个输入模式的期望输出;$Y_{pk}$ 为网络的实际计算输出。

按梯度下降法求权值、节点阈值,调整值如下:
\begin{align}
\Delta w_{ij}(t+1) &= \eta \delta_{j} Y_{i} + a \Delta w_{ij}(t) \\
\Delta \theta_{ij}(t+1) &= \eta \delta_{j} + a \Delta \theta_{j}(t)
\end{align}

式中:$Y_{i}$ 是 $i$ 神经元在 $t$ 时刻的实际输出;$0 < \eta < 1$ 为学习效率;$a$ 为动量因子;$\delta_{j}$ 为节点的误差,根据节点的形式不同分别计算。当为隐含层节点时:
\begin{equation}
\delta_{j} = Y_{j} (1 - Y_{j}) \sum_{k} \left( \delta_{k} w_{kj} \right)
\end{equation}

为输出节点时:
\begin{equation}
\delta_{j} = Y_{j} (1 - Y_{j}) (O_{j} - Y_{j})
\end{equation}

式中:$Y_{j}$ 是神经元 $j$ 在 $t$ 时刻的实际输出;$k$ 为神经元 $j$ 的下一层神经元编号。

BP 学习算法对网络进行训练,其学习步骤如下:
1. 权值和神经元阈值初始化;
2. 输入样本模式,指定输出层各神经元的期望值;
3. 按式 (1) (2) 依次计算每层神经元的实际输出;
4. 计算输出误差;
5. 若误差满足要求,运算结束;反之则反向逐层计算网络各节点的偏差,隐含层按式 (6) 计算,输出层按式 (7) 计算,按式 (4) (5) 修正网络权值和节点阈值;
6. 转式 (1)

本问题的 BP 神经网络实验设计:
1) 样本数据选择:
根据《附件一:配合比报告统计试验数据》中提供的四种实验指标分析,由于有些数据不全,本文仅选择其中四个实验指标均有的情况进行分析。一共有 167 组合格的数据,本文采用 100 组数据作为学习样本,67 组作为测试样本。

2) 源程序:
试验用 Matlab 7.0 自带的神经网络工具箱建立 BP 神经网络模型进行学习和测试。源程序如下:

\begin{verbatim}
close all;
clear;
P=xlsread('D:\MATLAB7\work\youhuadata.xls', 1, 'B3:D102');
P=P';
T=xlsread('D:\MATLAB7\work\youhuadata.xls', 1, 'A3:A102');
T=0.01*T';
[r, c]=size(P);
net=newff(minmax(P), [r, 16, 1], {'tansig', 'tansig', 'purelin'}, 'traingdx', 'learnngdm', 'mse');
net.trainParam.show=100;
net.trainParam.lr=0.0005;
net.trainParam.epochs=2500;
net.trainParam.goal=0.002;
net=train(net, P, T);

A=sim(net, P);
E=T-A;
MSE=mse(E);

PP=xlsread('D:\MATLAB7\work\youhuadata.xls', 1, 'B103:D169');
PP=PP';
TT=xlsread('D:\MATLAB7\work\youhuadata.xls', 1, 'A103:A169');
TT=0.01*TT';

AA=sim(net, PP);
EE=TT-AA;
MSE=mse(EE);
figure(2);
title('BP 神经网络模型预测与检验');
xlabel('预测样本');
ylabel('TSR 值');
X=1:67;
plot(X, TT, '-r', X, AA, '--b');
legend('实测值', '预测值');
figure(3);
title('误差图');
xlabel('预测样本');
ylabel('TSR 误差值');
X=1:67;
plot(X, EE, '-*');
legend('误差值');
\end{verbatim}

3. 学习BP神经网络模型

本问题的BP神经网络模型参数应该选取样本数为100、学习率为0.0005、隐层节点单元数为16个的三层BP神经网络模型。为了验证模型的合理性和科学性,选取了67组样本数据进行测试。

\begin{figure}[h]
    \centering
    \includegraphics[width=\textwidth]{image.png}
    \caption{训练样本的收敛图}
    \label{fig:training_convergence}
\end{figure}

\section{用模型进行预测和检验}

\begin{figure}[h]
    \centering
    \includegraphics[width=\textwidth]{image1.png}
    \caption{模型的预测值和实际值}
    \label{fig:5}
\end{figure}

\begin{figure}[h]
    \centering
    \includegraphics[width=\textwidth]{image2.png}
    \caption{BP 神经网络测试结果}
    \label{fig:6}
\end{figure}

图 6 为测试结果,横轴为测试样本序号,纵轴为误差值,可见误差基本在 $[-0.1134, 0.1016]$ 之间,测试误差平均值为 0.001466,误差方差为 0.0016595,误差满足所要求的精度。可见我们建立的BP神经网络模型是符合要求的。

\subsection{5.2 问题2的求解}

对于问题2,首先我们分别考虑九个影响因素对四个指标中的一个指标用SPSS软件进行多因素方差分析,从而找出分别对TSR(\%)、S0(\%)、车辙、弯拉应变四个指标影响比较大的因素,再建立它们之间比较精确的数学模型。

\subsubsection{5.2.1 各指标与影响因素之间的多因素方差分析}

设因素油石比(\%)、通过筛孔百分率(\%)19.0、13.2、9.5、4.75、2.36、1.18、0.60、
0.30、0.15、0.075、VV、VMA、VFA、DP、毛体积密度、最大理论密度、\%Gmm(最初)、\%Gmm(最大)分别为V1、V21、V3、V4、V5、V6、V7、V8、V9、V10、V11、V12、V13、V14、V15、V16、V17、V18、V19。我们对所选取的比较完整的数据用 SPSS 软件分别对其进行多因素方差分析,得出它们的方差如下表所示:

\begin{table}
\centering
\caption{多因素方差分析表}
\begin{tabular}{|c|c|c|c|c|c|}
\hline
\multirow{2}{*}{\begin{tabular}{c} 相伴概率$<$0.05 \\ 指标 \end{tabular}} & \multirow{2}{*}{\begin{tabular}{c} 因素 \end{tabular}} & \multirow{2}{*}{\begin{tabular}{c} TSR(\%) \end{tabular}} & \multirow{2}{*}{\begin{tabular}{c} SO(\%) \end{tabular}} & \multirow{2}{*}{\begin{tabular}{c} 车辙 \end{tabular}} & \multirow{2}{*}{\begin{tabular}{c} 弯拉应变 \end{tabular}} \\
\cline{1-6}
 & & & & & \\
\hline
油石比(\%) & V1 & 0.0572 & 0.175 & 0.537 & 0.584 \\
\hline
\multirow{10}{*}{\begin{tabular}{c} 通过 \\ 筛孔 \\ 百分率(\%) \end{tabular}} & 19.0 & V2 & 0.080 & 0.317 & 0.953 \\
\cline{2-6}
 & 13.2 & V3 & 0.361 & 0.071 & 0.977 \\
\cline{2-6}
 & 9.5 & V4 & 0.411 & 0.011 & 0.472 \\
\cline{2-6}
 & 4.75 & V5 & 0.577 & 0.027 & 0.080 \\
\cline{2-6}
 & 2.36 & V6 & 0.553 & 0.157 & 0.707 \\
\cline{2-6}
 & 1.18 & V7 & 0.029 & 0.698 & 0.022 \\
\cline{2-6}
 & 0.60 & V8 & 0.809 & 0.593 & 0.009 \\
\cline{2-6}
 & 0.30 & V9 & 0.018 & 0.709 & 0.043 \\
\cline{2-6}
 & 0.15 & V10 & 0.019 & 0.516 & 0.036 \\
\cline{2-6}
 & 0.075 & V11 & 0.832 & 0.486 & 0.437 \\
\hline
VV & V12 & 0.101 & 0.023 & 0.657 & 0.174 \\
\hline
VMA & V13 & 0.938 & 0.011 & 0.027 & 0.967 \\
\hline
VFA & V14 & 0.499 & 0.009 & 0.077 & 0.917 \\
\hline
DP & V15 & 0.894 & 0.766 & 0.939 & 0.060 \\
\hline
毛体积密度 & V16 & 0.203 & 0.942 & 0.832 & 0.286 \\
\hline
最大理论密度 & V17 & 0.258 & 0.656 & 0.763 & 0.390 \\
\hline
\%Gmm(最初) & V18 & 0.178 & 0.263 & 0.879 & 0.977 \\
\hline
\%Gmm(最大) & V19 & 0.881 & 0.362 & 0.938 & 0.970 \\
\hline
\end{tabular}
\end{table}

\begin{table}
\centering
\begin{tabular}{|c|c|c|c|c|c|}
\hline
Source & Type III Sum of Squares & df & Mean Square & F & Sig. \\
\hline
Corrected Model & 742.389(a) & 57 & 13.024 & 43.056 & .023 \\
Intercept & 372415.338 & 1 & 372415.338 & 1231125.085 & .000 \\
V9 & 425.019 & 26 & 16.347 & 54.039 & .018 \\
V10 & 268.652 & 17 & 15.803 & 52.241 & .019 \\
V9*V10 & 19.360 & 1 & 19.360 & 64.000 & .015 \\
Error & .605 & 2 & .303 & & \\
Total & 443368.520 & 60 & & & \\
Corrected Total & 742.994 & 59 & & & \\
\hline
\end{tabular}
\caption{对 TSR(\%) 影响最大的三组因素进行方差分析}
\end{table}

\begin{table}
\centering
\begin{tabular}{|c|c|c|c|c|c|}
\hline
Source & Type III Sum of Squares & df & Mean Square & F & Sig. \\
\hline
Corrected Model & 653.182(a) & 44 & 14.845 & 2.479 & .029 \\
Intercept & 397664.145 & 1 & 397664.145 & 66416.340 & .000 \\
V7 & 653.182 & 44 & 14.845 & 2.479 & .029 \\
Error & 89.812 & 15 & 5.987 & & \\
Total & 443368.520 & 60 & & & \\
Corrected Total & 742.994 & 59 & & & \\
\hline
\end{tabular}
\caption{对 TSR(\%) 影响最大的三组因素进行方差分析}
\end{table}

\begin{table}
\centering
\begin{tabular}{c c c c c}
\hline
Source & Type III Sum of Squares & df & Mean Square & F & Sig. \\
\hline
Corrected Model & 560.763(a) & 49 & 11.444 & 3.878 & .013 \\
Intercept & 305591.596 & 1 & 305591.596 & 103566.9 & .000 \\
V13 & 127.749 & 9 & 14.194 & 4.811 & .011 \\
V14 & 317.034 & 24 & 13.210 & 4.477 & .009 \\
V13*V14 & 62.695 & 8 & 7.837 & 2.656 & .075 \\
Error & 29.507 & 10 & 2.951 & & \\
Total & 492548.420 & 60 & & & \\
Corrected Total & 590.270 & 59 & & & \\
\hline
\end{tabular}
\caption{对S0(\%)影响最大的四组因素进行方差分析表}
\end{table}

\begin{table}
\centering
\begin{tabular}{c c c c c}
\hline
Source & Type III Sum of Squares & df & Mean Square & F & Sig. \\
\hline
Corrected Model & 458.608(a) & 41 & 11.186 & 1.529 & .166 \\
Intercept & 259827.306 & 1 & 259827.306 & 35522.04 & .000 \\
V11 & 149.304 & 20 & 7.465 & 4 & .486 \\
V12 & 88.920 & 3 & 29.640 & 1.021 & .023 \\
V11*V12 & 168.488 & 18 & 9.360 & 4.052 & .303 \\
Error & 131.662 & 18 & 7.315 & 1.280 & \\
Total & 492548.420 & 60 & & & \\
Corrected Total & 590.270 & 59 & & & \\
\hline
\end{tabular}
\caption{对S0(\%)影响最大的四组因素进行方差分析表}
\end{table}

\begin{table}
\centering
\begin{tabular}{c c c c c}
\hline
Source & Type III Sum of Squares & df & Mean Square & F & Sig. \\
\hline
\end{tabular}
\caption{对S0(\%)影响最大的四组因素进行方差分析表}
\end{table}

\begin{table}
\centering
\begin{tabular}{c c c c c}
\hline
Corrected & 548.045(a) & 48 & 11.418 & 2.974 \\
Model & & & & .027 \\
Intercept & 444106.694 & 1 & 444106.694 & 115693.8 \\
V5 & 548.045 & 48 & 11.418 & 69 \\
Error & 42.225 & 11 & 3.839 & 2.974 \\
Total & 492548.420 & 60 & & .000 \\
Corrected & 590.270 & 59 & & .027 \\
Total & & & & \\
\hline
\end{tabular}
\end{table}

经过计算分析可以得到V13和V14对S0(\%) 的影响最大,也即VMA矿料间隙率和VFA饱和度对S0(\%) 的影响最大。

3) 车辙的多因素方差分析

对车辙影响最大的四组因素进行分析结果如下表:

\begin{table}
\centering
\begin{tabular}{c c c c c}
\hline
Source & Type III & df & Mean Square & F \\
& Sum of & & & Sig. \\
& Squares & & & \\
\hline
Corrected & 142297296. & 57 & 2496443.800 & 109.488 \\
Model & 583(a) & & & .009 \\
Intercept & 1049252292 & 1 & 1049252292. & 46017.81 \\
V7 & .729 & & 729 & 9 \\
V8 & 14517167.5 & 14 & 1036940.536 & 45.478 \\
V7*V8 & 00 & & & .022 \\
Error & 31625399.1 & 13 & 2432723.013 & 106.694 \\
Total & 67 & & & .009 \\
Corrected & .000 & 0 & & \\
Total & 45602.000 & 2 & 22801.000 & \\
& 1287676249 & 60 & & \\
& .000 & 59 & & \\
& 142342898. & 583 & & \\
\hline
\end{tabular}
\caption{对车辙影响最大的四组因素进行方差分析表}
\end{table}

经过计算分析可以得到V7和V8对车辙的影响最大,也即通过筛孔百分率1.18、0.60对车辙的影响最大。比较重要的影响因素为VMA矿料间隙率和VFA饱和度。

4) 弯拉应变的多因素方差分析

对弯拉应变最大的四组因素进行分析结果如下表:
\begin{table}
\centering
\begin{tabular}{c c c c c}
\hline
 & Type III & & & \\
Source & Sum of & df & Mean Square & F \\
 & Squares & & & Sig. \\
\hline
Corrected & 4943036.34 & 48 & 102979.924 & 3.196 \\
Model & 8(a) & & & .020 \\
Intercept & 476465609. & 1 & 476465609.3 & 14785.82 \\
V5 & 310 & & 10 & 4 \\
 & 4943036.34 & 48 & 102979.924 & 3.196 \\
 & 8 & & & .000 \\
Error & 354469.378 & 11 & 32224.489 & \\
Total & 530663221. & 60 & & \\
 & 620 & & & .020 \\
Corrected & 5297505.72 & 59 & & \\
Total & 6 & & & \\
\hline
\end{tabular}
\caption{对弯拉应变影响最大的四组因素进行方差分析表}
\end{table}

\begin{table}
\centering
\begin{tabular}{c c c c c}
\hline
 & Type III & & & \\
Source & Sum of & df & Mean Square & F \\
 & Squares & & & Sig. \\
\hline
Corrected & 5287859.12 & 57 & 92769.458 & 19.234 \\
Model & 1(a) & & & .051 \\
Intercept & 488107097. & 1 & 488107097.9 & 101197.6 \\
V7 & 955 & & 55 & 96 \\
V8 & 628503.450 & 14 & 44893.104 & 9.308 \\
 & 2567331.83 & 13 & 197487.064 & 40.944 \\
 & 0 & 0 & & \\
V7*V8 & .000 & 2 & 4823.303 & \\
Error & 9646.605 & & & \\
Total & 530663221. & 60 & & \\
 & 620 & & & .000 \\
Corrected & 5297505.72 & 59 & & .024 \\
Total & 6 & & & \\
\hline
\end{tabular}
\caption{对弯拉应变影响最大的四组因素进行方差分析表}
\end{table}

a $R$ Squared = .998 (Adjusted $R$ Squared = .946)

经过计算分析,V5,V8即通过筛孔百分率(\%) 0.577,0.809对对弯拉应变的影响最大。

\subsubsection{5.2.2 各指标与重要和比较重要影响因素之间的数学模型的建立}

1) 对 TSR(\%) 和通过筛孔百分率(\%) 0.30、0.15 建立多元回归模型

设 TSR(\%) 为 $y_{1}$,通过筛孔百分率(\%) 0.30、0.15 分别为 $x_{1a}$,$x_{1b}$ 用 matlab 软件对其进行多元回归得到其模型为(具体程序见附录 2)

\[
y_{a} = 88.9258 + 0.9139x_{1a} - 1.6979x_{1b} - 0.0720x_{1a}^{2} + 0.1195x_{1b}^{2}
\]

模型的剩余回归标准差为 3.3982,比较接近于 0,故认为模型的回归效果比较好。

2) 对 S0(\%) 和 VMA 矿料间隙率和 VFA 饱和度建立多元回归模型

设 S0(\%) 为 $y_{b}$,VMA 矿料间隙率为 $x_{2a}$,VFA 饱和度为 $x_{2b}$,用 matlab 软件对其进行多元回归得到其模型为(具体程序见附录 2)

\[
y_{b} = 222.1100 - 1.2730x_{2a} - 3.5730x_{2b} + 0.0365x_{2a}^{2} + 0.0262x_{2b}^{2}
\]

模型的剩余回归标准差为 3.3255,比较接近于 0,故认为模型的回归效果比较好。

\subsubsection{5.2.3 提高高速公路路面质量的方法}

根据以上分析结果,我们可以得出以下提高高速公路路面质量的方法:

1) 沥青混合料的高温黏结性和矿料级配的嵌挤作用对沥青的高温稳定性能有着很大的影响。同时我们也应该采用坚硬、颗粒接近立方体的集料,这样沥青的混合料的高温性能就比较好。因此,我们应该适当减少沥青用量,加大压实功,使混合料充分嵌挤,又没有留下大的空隙率是提高沥青路面高温稳定性的重要措施。

2) 由上面分析知道,沥青用量和矿料组成级配对高速公路路面的低温性能有着较大的影响,因此在提高高速公路路面的抗低温性能时,我们应主要从以下几个方面改进:沥青的油源和温度敏感性、沥青混合料的组成、矿粉的细度等。

3) 由数据分析知道,沥青的水稳定性主要取决于矿料的性质、沥青与矿料之间相互作用的性质、沥青混合料的空隙率、沥青膜的厚度等。因此提高沥青混合料的空隙率和矿料的性质是提高沥青的抗水稳定性的重要措施。

\subsection{5.3 问题 3 的求解}

本文对这个问题设计了两种方案:

\subsubsection{5.3.1 方案一:建立理论模型进行分析求解}

\paragraph{一、模型假设:}

当前级配理论都基于两个基本假设:1) 假设基本颗粒为规则球体;2) 假设各级颗粒粒径均相等;其理论基础是堆积理论,认为集料排列状态对空隙率的影响较大,而空隙率的大小与粒径的大小并无关系。当集料嵌入适当粒径的粒料,可使空隙率下降,

但是如果下降到一定程度,就会破坏粒料的密实状况,这就是级配填充理论。而现在实际常用的连续级配,并不是按填充理论进行设计的,而是采用了粒径较大的次级料来使上一级料的颗粒与颗粒之间不完全接触,相对地增加了填料的用量,这种级配理论称为干涉理论。综合上述理论,结合本问题的要求,本文提出如下假设:

(1) 假设基本颗粒为规则球体;

(2) 假设各级颗粒粒径均相等;

(3) 假设路面结构由立方体单元结构组成,每个立方体单元结构由若干个直径不同的圆球组成,圆球与圆球之间,圆球与立方体单元之间均相切接触,立方体的厚度为圆球的直径,截面为正方形;

(4) 假设通过同一筛孔的粒料由同一型号的立方体单元机构组成,该型号的立方体单元中圆球的直径与该筛孔直径相同;

(5) 假设粒料密度相同,各型号的立方体单元结构的数量之间的比例与各筛余质量的比例一致。

\paragraph{二、模型构建。}

(一) 立方体单元结构的孔隙率分析与计算

假设立方体单元结构的边长为 \( a \);

\begin{figure}[h]
    \centering
    \includegraphics[width=0.9\textwidth]{image.png}
    \caption{单元截面结构图}
    \label{fig:unit_structure}
\end{figure}

如果立方体单元结构由 1 个圆球填充,如图 7 左所示:则圆球的直径为 \( a \),结构的孔隙率计算为:
\[
V_{1} = a^{3} - \frac{4}{3} \pi R_{1}^{3} = a^{3} - \frac{1}{6} \pi a_{1}^{3} = a^{3} \left( 1 - \frac{1}{6} \pi \right)
\]

如果立方体单元结构由 8 个圆球填充,如图 7 中所示:则圆球的直径为 \( a/2 \),结构的孔隙率计算为:
\[
V_{2} = a^{3} - 8 \times \frac{4}{3} \pi R_{2}^{3} = a^{3} - 8 \times \frac{4}{3} \pi \left( \frac{a}{4} \right)^{3} = a^{3} \left( 1 - \frac{1}{6} \pi \right)
\]

如果立方体单元结构由 \( n^{3} \) 个圆球填充,如图 7 右所示:则圆球的直径为 \( a/n \),结构的孔隙率计算为:
\[
V_{n} = a^{3} - n^{3} \times \frac{4}{3} \pi R_{n}^{3} = a^{3} - n^{3} \times \frac{4}{3} \pi \left( \frac{a}{2n} \right)^{3} = a^{3} \left( 1 - \frac{1}{6} \pi \right)
\]

(二) 路面结构的模拟

根据集料的筛孔通过率 $P_{i}$,可以计算集料各粒径的筛余 $P_{i}^{1}$ 的构成比例,由假设条件(5),可以得知各粒径对应的立方体单元结构的数量比例 $P_{i}^{2}$,路面压实度 $P_{v}$ 即为所有的各型号立方体单元结构的空隙率之和的累积。

\begin{align*}
\sum_{i=1}^{n} P_{i} &= 1 \\
P_{i}^{1} &= 1 - P_{i} \\
P_{i}^{2} &= P_{i}^{1} \\
P_{v} &= \sum_{i=1}^{n} P_{i}^{2} V_{i} = \left(1 - \frac{\pi}{6}\right) \sum_{i=1}^{n} P_{i}^{2} = 1 - \frac{\pi}{6} = 0.4764 = 47.64\%
\end{align*}

(三)模型分析与评价

模型能够模拟集料的不同粒径的粒料的构成比例,得出的数学意义下的路面压实度的上界为 $47.64\%$,如果用混合料试件成型时达最大压实状态时的压实度来近似估计路面压实度的上界,由于路面最大压实状态时的压实度包含沥青的体积,导致其值稍大于路面压实度的上界,根据《附件一:配合比报告统计试验数据》中混合料试件成型时达最大压实状态时的压实度一般为 $95\%$ 左右,油石比数据一般在 $5\%$ 左右,而沥青的密度为集料密度的一半不到,所以路面最大压实状态时的压实度比路面压实度的上界大概大 $10\%$ 左右,所以本文认为单纯用集料来模拟路面压实度能够达到 $85\%$ 就很不错了,虽然本文的模型模拟的误差还较大差距,但是考虑到在工程实践中,集料的粒径不可能全部是圆球形的,且同一筛余中的集料粒径是完全不一样,其值在两个相邻的筛孔尺寸之间波动,这两个假设都过于理想化。总之,本文的模型是基本可以模拟集料的筛孔通过率与路面压实度的上界的数量关系的。

\paragraph{三、模型的改进}

(一)立方体单元结构存在的问题

(1)筛孔通过率不影响路面压实度的上界的结果,即不管筛孔通过率如何变化,路面压实度的上界均为 $73.82\%$,没有直接反映二者之间的数量关系;

(2)路面压实度的上界不收敛,为一个常数;

(3)立方体单元结构形式过于简单,不能模拟集颗粒径形状不规则的实际情况。

(二)立方体单元结构的改进

由于立方体单元结构的 8 个顶点处还有大量的空间,可以在这 8 个角落增加 8 个圆球,各大球与小球以及立方体单元结构的接触均相切。

\begin{figure}[h]
    \centering
    \includegraphics[width=0.8\textwidth]{image.png}
    \caption{单元截面结构图}
    \label{fig:unit_cross_section}
\end{figure}

\begin{align*}
R_2 + \sqrt{3}R_2 + \frac{a}{2} &= \frac{a}{2} \times \sqrt{3} \\
R_2 &= \frac{a}{2} \times (2 - \sqrt{3}) \\
V_1 &= a^3 - \frac{4}{3}\pi R_1^3 - 8 \times \frac{4}{3}\pi R_2^3 \\
&= a^3 - \frac{1}{6}\pi a^3 - 8 \times \frac{4}{3}\pi \left(\frac{a}{2}\right)^3 (2 - \sqrt{3})^3 \\
&= a^3 \left(1 - \frac{1}{6}\pi - \frac{4}{3}\pi (2 - \sqrt{3})^3\right) \\
V_1 &= 0.557a^3
\end{align*}

这说明增加 8 个小圆球可以提高模型的准确性,由此可以联想在剩下的空间里继续增加相切小圆球的数量,模型的准确性可以得到进一步的提高。如果一个立方体单元可以放 3 类相同直径的小圆球,每类小圆球的数量(由)分别为 1,8,24。由于 $R_2/R_1$ 接近 1:5,$R_3/R_1$ 接近 1:25,基本可以模拟筛孔尺寸。

(三) 路面结构的模拟

根据集料的筛孔通过率 $P_i$,可以计算集料各粒径的筛余 $P_i^1$ 的构成比例,由假设条件(5),可以得知各粒径对应的立方体单元结构的数量比例 $P_i^2$,同时将大尺寸筛孔对应的立方体单元结构的数量比例 $P_i^2$ 中包含的小尺寸筛孔的立方体单元结构的数量比例 $P_i^2$ 扣去,得到调整后的立方体单元结构的数量比例 $P_i^{'2}$,各路面压实度 $P_V$ 即为所有的各型号立方体单元结构的空隙率之和的累积。

\begin{equation}
P_V = \sum_{i=1}^{n} P_i^{'2} V_i
\end{equation}

\subsubsection{方案二:基于多元线性回归分析的经验数量关系}
由于本文涉及的实际情况过于复杂,建立理论模型分析过于困难。根据附件一中的大量数据样本,本文尝试采用多元线性回归分析方法,探索集料的筛孔通过率与路面压实度上界之间的数量关系。由于筛孔通过率为 100,各号筛的累计筛余百分比定义为该号筛及大于该号筛的各号筛的分计筛余百分比之和。但需注意,0.075\,mm 筛不参与累计筛余计算。各号筛上的分计筛余百分比为该筛上的筛余量除以试样总质量 $M$ 的百分比。由此可见,各号筛的分计筛余百分比仅与相邻两号筛的筛孔尺寸有关,与其他筛孔的分计筛余无显著线性关系。因此,本文尝试以各号筛的分计筛余百分比作为自变量,分析其与路面压实度上界之间的数量关系。若存在显著的线性关系,则可通过数学变换,进一步转化为集料的筛孔通过率与路面压实度上界之间的经验关系模型。

\begin{table}
\centering
\caption{表 14 各筛孔的分计筛余百分率数据表}
\begin{tabular}{|c|c|c|c|c|c|c|c|c|c|c|c|c|c|}
\hline
编号 & 31.5 & 26.5 & 19 & 13.2 & 9.5 & 4.75 & 2.36 & 1.18 & 0.6 & 0.3 & 0.15 & 0.075 & 小于 0.075 \\
\hline
1 & 1 & 1 & 16.7 & 18.4 & 6.8 & 19.2 & 14 & 7.6 & 4 & 4.2 & 2 & 1.3 & 3.8 \\
\hline
2 & 1 & 1.6 & 11.7 & 19.9 & 11.1 & 17.8 & 16 & 7.8 & 3.4 & 3.5 & 1.6 & 1.5 & 3.1 \\
\hline
3 & 1 & 3.1 & 15 & 15.5 & 13.3 & 16.2 & 11.3 & 9.8 & 5.1 & 3.8 & 1.2 & 0.9 & 3.8 \\
\hline
4 & 1 & 1.6 & 12.4 & 16.5 & 16.5 & 18.5 & 11.5 & 7.3 & 4.3 & 4 & 1.5 & 1.4 & 3.5 \\
\hline
5 & 1 & 0.3 & 15.4 & 19.6 & 9.2 & 11.7 & 10.4 & 11.6 & 9.9 & 5.9 & 1.3 & 0.8 & 2.9 \\
\hline
6 & 1 & 0 & 17.1 & 15.9 & 12.7 & 16.3 & 14.7 & 8.7 & 3.3 & 3.1 & 1.2 & 1.5 & 4.5 \\
\hline
7 & 1 & 0.9 & 22.4 & 13.8 & 10 & 15.9 & 10.4 & 8.3 & 5.6 & 2.8 & 2.8 & 2 & 4.1 \\
\hline
8 & 1 & 1.7 & 13.9 & 12.7 & 16 & 19.3 & 11.4 & 7.4 & 4.7 & 4.6 & 1.9 & 1.5 & 3.9 \\
\hline
9 & 1 & 2.7 & 16.9 & 15.1 & 8.8 & 16.4 & 14.8 & 7 & 5.7 & 3.8 & 1.7 & 2.3 & 3.8 \\
\hline
10 & 1 & 0.3 & 14.5 & 18.9 & 11.7 & 18.2 & 11.5 & 8 & 6.6 & 3.9 & 0.9 & 0.7 & 3.8 \\
\hline
11 & 1 & 1.9 & 12.3 & 12.8 & 13.1 & 22.4 & 12.7 & 7.1 & 6.7 & 3.8 & 1.5 & 1 & 3.7 \\
\hline
12 & 1 & 0 & 19.9 & 25.4 & 6.6 & 11.5 & 11.4 & 8.3 & 6.1 & 3.3 & 1.5 & 1 & 4 & 97.7 \\
\hline
13 & 1 & 0.9 & 15.4 & 14.4 & 13.7 & 17.9 & 12.6 & 6.8 & 5.6 & 4 & 1.8 & 1.7 & 4.2 & 97.8 \\
\hline
14 & 1 & 5.2 & 14.6 & 13.3 & 15.4 & 13.7 & 8 & 8.3 & 8.9 & 4.9 & 2.1 & 0.7 & 3.9 & 98 \\
\hline
\end{tabular}

从附件一中随机选取了 14 组数据,一共有 13 个因素,用混合料试件成型时达最大压实状态时的压实度来近似估计路面压实度的上界,在 matlab 中运行,源程序见附录 3:即可得到多元线性回归方程:$Y = Xb$

其中 $Y$ 为路面压实度,$X$ 为各筛孔的尺寸因素,$b$ 为系数矩阵。

\begin{equation}
Y = \sum_{i=1}^{n} X_i b_i = 0.9297X_2 + 0.7484X_3 + 1.0177X_4 + 0.9405X_5 + 0.9698X_6 + 0.7342X_7
\end{equation}

\begin{equation}
+ 1.0203X_8 + 1.3121X_9 + 0.5785X_{10} + 0.5776X_{11} + 2.1347X_{12} + 2.4187X_{13}
\end{equation}

(其中 $X_i$ 为各筛孔尺寸对应的分计筛余百分比:)

\begin{figure}[h]
    \centering
    \includegraphics[width=\textwidth]{residual_case_order_plot.png}
    \caption{残差分析图}
    \label{fig:residual_analysis}
\end{figure}

Bint 矩阵为回归系数的区间估计;,$r$ 与 $right$ 为残差及其置信区间;stats 中第一个数据为相关系数 $R^2$,由于 $R^2 = 0.96406$,与 1 十分接近,故回归方程的很显著,stats 中第二个数据为 F 检验中的 F 值,其值为 $F = 4.877$,stats 中第二个数据为与 F 值对应的概率 $p$,因为 $p = 0.18234 < 0.05$,所以本文认为回归方程有显著线性关系,回归模型成立。

由于分计筛余百分比与筛孔通过率之间存在线性关系,可以用如下公式进行转化,

\begin{equation}
P_{i}=100-A_{i}=100-\left(a_{1}+a_{2}+....+a_{i}\right)=100-\sum_{i=1}^{i} \frac{a_{j}}{M}
\end{equation}

所以集料筛孔通过率与路面压实度的上界存在显著线性关系。

\begin{equation}
Y=\sum_{i=1}^{n} a_{i} b_{i}=\sum_{i=1}^{n} P_{i} b_{i}
\end{equation}

\subsection{5.4. 问题4的求解}

\subsubsection{5.4.1 沥青的种类对高速公路路面抗水损害性能、高温性能、低温性能的影响分析}

要分析沥青的种类对高速公路路面抗水损害性能、高温性能、低温性能有没有影响,我们可以采用方差分析的方法来鉴别各因素对试验结果影响程度。在方差分析试验中,我们把沥青的种类分别对高速公路路面抗水损害性能、高温性能、低温性能有没有影响称为试验指标,在本题中把沥青的种类对 TSR(\%)、S0(\%)、车辙、弯拉应变四个指标的影响分别作为 4 个试验指标,把沥青看成因子,把沥青的含 25 的普通沥青和含 13、16 或 20 的改性沥青作为水平,建立在不同沥青型号下 TSR(\%)、S0(\%)、车辙、弯拉应变的数据关系,分别对 TSR(\%)、S0(\%)、车辙、弯拉应变四个指标进行单因素方差分析。单因素方差的数学模型一般为

\begin{equation}
\left\{\begin{array}{l}
Y_{i j}=\overline{\mu}+\partial_{i}+e_{i j},(i=1,2, \ldots, r, j=1,2, \ldots, n_{i}) \\
e_{i j} \sim N\left(0, \alpha^{2}\right), \text { 且诸 } e_{i j} \text { 相互独立; } \\
\sum_{i=1}^{r} n_{i} \partial_{i}=0
\end{array}\right.
\end{equation}

设 $H_{0}: \mu_{1}=\mu_{2}=\ldots=\mu_{r}, H_{1}$:存在至少一对 $\mu_{i} \neq \mu_{j}$,现在我们用单因素方差分析表分别对 TSR(\%)、S0(\%)、车辙、弯拉应变四个指标进行单因素方差分析。建立单因素方差分析表为:

\begin{table}[h]
\centering
\caption{单因素方差分析表}
\begin{tabular}{|c|c|c|c|c|}
\hline
方差来源 & DF(自由度) & $S^{2}$(平方和) & $\overline{S}^{2}$(均方差) & $F$ \\
\hline
因素A & $r-1$ & $S_{A}^{2}$ & $\overline{S}_{A}^{2}$ & $F=\frac{\overline{S}_{A}^{2}}{\overline{S}_{E}^{2}}$ \\
\hline
随机误差 & $n-r$ & $S_{E}^{2}$ & $\overline{S}_{E}^{2}$ & \\
\hline
总和 & $n-1$ & $S_{T}^{2}$ & & \\
\hline
\end{tabular}
\end{table}

1) 沥青的种类对 TSR(\%) 显著性影响分析

构建沥青的种类对 TSR(\%) 的影响的数据表格如下表所示

\begin{table}[h]
\centering
\caption{沥青的种类对 TSR(\%) 的影响的单因素分析表格}
\begin{tabular}{|c|c|}
\hline
沥青种类 & 相应的 TSR(\%) 数值 \\
\hline
13 & $y_{11}, y_{12}, \ldots, y_{1n_1}$ \\
\hline
16 & $y_{21}, y_{22}, \ldots, y_{2n_2}$ \\
\hline
20 & $y_{31}, y_{32}, \ldots, y_{3n_3}$ \\
\hline
25 & $y_{41}, y_{42}, \ldots, y_{4n_4}$ \\
\hline
\end{tabular}
\end{table}

根据数据表格分析我们建立的沥青的种类对TSR(\%) 的影响的单因素方差分析表格为:


\begin{table}[h]
\centering
\caption{沥青的种类对TSR(\%) 的影响的单因素方差分析表}
\begin{tabular}{|c|c|c|c|c|}
\hline
方差来源 & $DF$ (自由度) & $S^2$ (平方和) & $\overline{S}^2$ (均方差) & $F$ \\
\hline
因素A & 3 & 148.6877997 & 49.56259989 & 4.109576963 \\
\hline
随机误差 & 224 & 2701.5 & 12.06026809 & \\
\hline
总和 & 227 & & & \\
\hline
\end{tabular}
\end{table}

当 $\alpha=0.05$ 时,在 $F$ 分布表中查得分位数值 $F_{0.95}(3, 225) = 2.60$,$F = 4.109576963 > F_{0.95}(3, 225) = 2.60$,所以拒绝 $H_0$,即认为沥青的种类不同对 TSR(\%) 的影响是显著的。现在分析哪种沥青对 TSR(\%) 的影响最大,根据公式

\[
\overline{\partial}_i = \overline{Y}_i - \overline{Y} \ (i = 1, 2, \ldots, r),
\]

算出因素 TSR(\%) 下的各沥青种类的效应值为:

\[
\overline{\partial}_1 = 85.2 - 85.1 = 0.1, \ \overline{\partial}_2 = 82.3 - 85.1 = -2.8, \ \overline{\partial}_3 = 85.4 - 85.1 = 0.3,
\]
\[
\overline{\partial}_4 = 85.6 - 85.1 = 0.5,
\]

计算结果表明,$\overline{\partial}_2$ 也即 16 型号的沥青对 TSR(\%) 的影响是最大的。

\textbf{2) 沥青的种类对S0(\%)显著性影响分析}

同上所述,构建沥青的种类对S0(\%) 的影响的数据表格如下表所示

根据数据表格分析我们建立的沥青的种类对S0(\%) 的影响的单因素方差分析表格为:


\begin{table}[h]
\centering
\caption{沥青的种类对S0(\%) 的影响的单因素方差分析表}
\begin{tabular}{|c|c|c|c|c|}
\hline
方差来源 & $DF$ (自由度) & $S^2$ (平方和) & $\overline{S}^2$ (均方差) & $F$ \\
\hline
因素A & 3 & 249.0865246 & 83.02884152 & 7.947525 \\
\hline
随机误差 & 264 & 2758.042853 & 10.44713202 & \\
\hline
总和 & 267 & & & \\
\hline
\end{tabular}
\end{table}

当 $\partial=0.05$ 时,在 $F$ 分布表中查得分位数值 $F_{0.95}(3,264)=2.60$,\\
$F=7.947525 > F_{0.95}(3,264)=2.60$,所以拒绝 $H_0$,即认为沥青的种类不同对 $S0(\%)$ 的影响是显著的。现在分析哪种沥青对 $S0(\%)$ 的影响最大,根据公式 $\overline{\partial}_i = \overline{Y}_i - \overline{Y} (i=1,2,\dots,r)$,算出因素 $S0(\%)$ 下的各沥青种类的效应值为:
\begin{equation}
\overline{\partial}_1 = 90.7 - 90.2 = 0.5, \quad \overline{\partial}_2 = 86.7 - 90.2 = -3.5, \quad \overline{\partial}_3 = 90.6 - 90.2 = 0.4, \quad \overline{\partial}_4 = 89.8 - 90.2 = -0.4;
\end{equation}
计算结果表明,$\overline{\partial}_2$ 也即 16 型号的沥青对 $S0(\%)$ 的影响是最大的。

\textbf{3) 沥青的种类对车辙显著性影响分析}

同上所述,构建沥青的种类对车辙的影响的数据表格如下表所示

根据数据表格分析我们建立的沥青的种类对车辙的影响的单因素方差分析表格为:

\begin{table}
\centering
\caption{沥青的种类对车辙的影响的单因素方差分析表}
\begin{tabular}{|c|c|c|c|c|}
\hline
方差来源 & $DF$ (自由度) & $S^2$ (平方和) & $\overline{S^2}$ (均方差) & $F$ \\
\hline
因素A & 3 & 256767595.6 & 85589198.54 & 40.95598856 \\
\hline
随机误差 & 217 & 453483281.3 & 2089784.707 & \\
\hline
总和 & 220 & & & \\
\hline
\end{tabular}
\end{table}

当 $\partial=0.05$ 时,在 $F$ 分布表中查得分位数值 $F_{0.95}(3,217)=2.60$,\\
$F=40.95598856 >> F_{0.95}(3,217)=2.60$,所以拒绝 $H_0$,由此认为沥青的种类不同对车辙的影响是显著的,而且这种影响比前两种影响更为显著。现在分析哪种沥青对车辙的影响最大,根据公式 $\overline{\partial}_i = \overline{Y}_i - \overline{Y} (i=1,2,\dots,r)$,算出因素车辙下的各沥青种类的效应值为:
\begin{align}
\overline{\partial}_1 &= 5036.8 - 4139 = 897.8, \\
\overline{\partial}_2 &= 3126.8 - 4139 = -1012.2, \\
\overline{\partial}_3 &= 4635.3 - 4139 = 226.3, \\
\overline{\partial}_4 &= 2090.1 - 4139 = -2048.9
\end{align}
计算结果表明,$\overline{\partial}_4$ 也即 25 型沥青对车辙的影响最大。

\textbf{4) 沥青的种类对弯拉应变显著性影响分析}

同上所述,构建沥青的种类对弯拉应变的影响的数据表格如下表所示,由于 25 型沥青对弯拉应变的数据全部为空,因此我们只考虑13、16、20型沥青对弯拉应变的影响。

根据数据表格分析我们建立的沥青的种类对弯拉应变的影响的单因素方差分析表格为

\begin{table}
\centering
\caption{沥青的种类对弯拉应变的影响的单因素方差分析表}
\begin{tabular}{|c|c|c|c|c|}
\hline
方差来源 & $DF$ (自由度) & $S^2$ (平方和) & $\overline{S^2}$ (均方差) & $F$ \\
\hline
因素A & 3 & 2452107.02 & 1226053.51 & 8.51184015 \\
\hline
随机误差 & 166 & 24054838.03 & 144040.9463 & \\
\hline
总和 & 169 & & & \\
\hline
\end{tabular}
\end{table}

当 $\alpha=0.05$ 时,在 $F$ 分布表中查得分位数值 $F_{0.95}(3,166)=2.65$,$F=8.51184015 > F_{0.95}(3,166)=2.65$,所以拒绝 $H_0$,由此认为沥青的种类不同对弯拉应变的影响是显著的。现在分析哪种沥青对弯拉应变的影响最大,根据公式 $\overline{\partial_i} = \overline{Y_i} - \overline{Y} (i=1,2,\dots,r)$,算出因素弯拉应变下的各沥青种类的效应值为:
\begin{align*}
\overline{\partial_1} &= 3149 - 3042.5 = -93.5, \\
\overline{\partial_2} &= 2733.6 - 3042.5 = -308.9, \\
\overline{\partial_3} &= 2952.1 - 3042.5 = -90.4
\end{align*}

计算结果表明,$\overline{\partial_2}$ 也即16型号的沥青对弯拉应变的影响最大。

综合以上结果我们可以建立以下综合分析表格
\begin{table}[h]
\centering
\caption{表19 沥青种类对路面影响的综合分析表格}
\begin{tabular}{|c|c|c|c|}
\hline
 & 影响种类 & 影响程度 & 影响最大的沥青种类 \\
\hline
沥青的种类对TSR(\%)显著性影响 & 抗水损害性能 & 有显著性影响 (4.109576963-2.60) /2.60=58\% & 16型号 \\
\hline
沥青的种类对S0(\%)显著性影响 & 抗水损害性能 & 有显著性影响 (7.947525-2.60) /2.60=201\% & 16型号 \\
\hline
沥青的种类对车辙显著性影响 & 高温性能 & 有显著性影响 (40.95598856-2.60) /2.60=1475\% & 25型号 \\
\hline
沥青的种类对弯拉应变显著性影响 & 低温性能 & 有显著性影响 (8.51184015-2.60) & 16型号 \\
\hline
\end{tabular}
\end{table}

从上表我们可以很清楚地知道,沥青的种类对车辙(即高速公路路面的高温性能)影响最为明显,其次是低温性能和抗水损害性能。其中,16 号型沥青对沥青混合料的低温性能和抗水损害性能影响最大。

\subsubsection{碎石质量对高速公路路面抗水损害性能、高温性能、低温性能的影响分析}
关于碎石质量,本文将其理解为沥青种类与不同级配共同作用下的综合结果,具体在表格数据中反映为混合料类型。因此,本文分别对 AC-13、AC-16、AC-20、AC-25、SUP-13、SMA-13、SUP-20、SUP-25 这八种混合料类型(代表不同碎石质量)对高速公路路面的抗水损害性能、高温性能和低温性能的影响进行分析。

同样,采用单因素方差分析法(One-way ANOVA)来求解该问题。选取 TSR(\%)、$S_0$ (\%)、车辙深度(动稳定度)、弯拉应变作为评价指标(因变量),将上述八种混合料类型作为不同水平进行显著性分析。

1)碎石质量对 TSR(\%) 的显著性影响分析

构建混合料类型对 TSR(\%) 影响的数据表格如下表所示:

\begin{table}[h]
\centering
\caption{沥青的种类对TSR(\%) 的影响的单因素分析表格}
\begin{tabular}{|l|l|}
\hline 碎石质量种类 & 相应的TSR(\%) 数值 \\
\hline AC-13 & $y_{11}, y_{12}, \ldots, y_{1n_1}$ \\
\hline SMA-13 & $y_{21}, y_{22}, \ldots, y_{2n_2}$ \\
\hline SUP-13 & $y_{31}, y_{32}, \ldots, y_{3n_3}$ \\
\hline AC-16 & $y_{41}, y_{42}, \ldots, y_{4n_4}$ \\
\hline AC-20 & $y_{51}, y_{52}, \ldots, y_{5n_5}$ \\
\hline SUP-20 & $y_{61}, y_{62}, \ldots, y_{6n_6}$ \\
\hline AC-25 & $y_{71}, y_{72}, \ldots, y_{7n_7}$ \\
\hline SUP-25 & $y_{81}, y_{82}, \ldots, y_{8n_8}$ \\
\hline
\end{tabular}
\end{table}

根据数据表格分析我们建立的碎石质量的种类对TSR(\%)的影响的单因素方差分析表格为:
\begin{table}[h]
\centering
\caption{碎石质量的种类对TSR(\%) 的影响的单因素方差分析表}
\begin{tabular}{|l|l|l|l|l|}
\hline 方差来源 & $DF$ (自由度) & $S^2$ (平方和) & $\overline{S^2}$ (均方差) & $F$ \\
\hline 因素A & 7 & 178 & 25.42857 & 2.199902 \\
\hline
随机误差 & 211 & 2438.94 & 11.55896 & \\
\hline
总和 & 218 & & & \\
\hline
\end{tabular}

当 $\partial=0.05$ 时,在 $F$ 分布表中查得分位数值 $F_{0.95}(7,211)=2.01$,\\
$F=2.199902>F_{0.95}(7,211)=2.01$,所以拒绝 $H_0$,由此认为碎石质量不同对 TSR(\%) 的影响\\
是显著的。现在分析哪种碎石种类对 TSR(\%) 的影响最大,根据公式\\
$\overline{\partial}_i=\overline{Y}_i-\overline{Y}(i=1,2,\ldots,r)$,算出因素 TSR(\%) 下的各碎石种类的效应值为:\\
$\overline{\partial}_1=84.8-85.2=-0.4$,$\overline{\partial}_2=85.4-85.2=0.2$,$\overline{\partial}_3=85.1-85.2=-0.1$,$\overline{\partial}_4=82.4-85.2=-2.8$,\\
$\overline{\partial}_5=84.7-85.2=-0.5$,$\overline{\partial}_6=86.1-85.2=0.9$,$\overline{\partial}_7=86.9-85.2=1.7$,\\
$\overline{\partial}_8=84.9-85.2=-0.3$,通过计算得知 $\overline{\partial}_4$ 即碎石种类为 AC-16 时,对 TSR(\%) 的影响最显著。

2)碎石质量不同对 S0(\%) 显著性影响分析


\begin{table}[h]
\centering
\caption{碎石质量的种类对S0(\%) 的影响的单因素方差分析表}
\begin{tabular}{|c|c|c|c|c|}
\hline
方差来源 & $DF$(自由度) & $S^2$(平方和) & $\overline{S}^2$(均方差) & $F$ \\
\hline
因素A & 7 & 263.343 & 37.62043 & 3.691191 \\
\hline
随机误差 & 205 & 2089.3 & 10.19195 & \\
\hline
总和 & 212 & & & \\
\hline
\end{tabular}
\end{table}

当 $\partial=0.05$ 时,在 $F$ 分布表中查得分位数值 $F_{0.95}(7,205)=2.01$,而\\
$F=3.691191>F_{0.95}(7,205)=2.01$,所以拒绝 $H_0$,由此认为碎石质量不同对 S0(\%) 的影响是\\
显著的。

现在分析哪种碎石种类对 S0(\%) 的影响最大,根据公式 $\overline{\partial}_i=\overline{Y}_i-\overline{Y}(i=1,2,\ldots,r)$,算出\\
因素 S0(\%) 下的各碎石种类的效应值为:\\
$\overline{\partial}_1=91.4-89.9=1.5$,$\overline{\partial}_2=90.6-89.9=0.7$,$\overline{\partial}_3=88.8-89.9=-1.1$,\\
$\overline{\partial}_4=87.0-89.9=-2.9$,$\overline{\partial}_5=90.2-89.9=0.3$,$\overline{\partial}_6=90.4-89.9=0.5$,$\overline{\partial}_7=89.9-89.9=0$,\\
$\overline{\partial}_8=88.6-89.9=-1.3$

通过计算我们知道,$\overline{\partial}_{1}$ 即 AC-13 碎石种类对 S0(\%) 的影响最大。

3) 碎石质量不同对车辙显著性影响分析

\begin{table}[h]
\centering
\caption{碎石质量的种类对车辙的影响的单因素方差分析表}
\begin{tabular}{|c|c|c|c|c|}
\hline 方差来源 & $DF$ (自由度) & $S^{2}$ (平方和) & $\overline{S^{2}}$ (均方差) & $F$ \\
\hline 因素A & 7 & 378308831 & 54044118.72 & 35.42961 \\
\hline 随机误差 & 202 & 308129654.1 & 1525394.327 & \\
\hline 总和 & 209 & & & \\
\hline
\end{tabular}
\end{table}

当 $\partial=0.05$ 时,在 $F$ 分布表中查得分位数值 $F_{0.95}(7,202)=2.01$,$F=35.42961>F_{0.95}(7,202)=2.01$,因此拒绝 $H_{0}$,由此认为碎石质量不同对车辙的影响是显著的。

现在分析哪种碎石种类对车辙的影响最大,根据公式 $\overline{\partial}_{i}=\overline{Y}_{i}-\overline{Y}(i=1,2,\ldots,r)$,算出因素 S0(\%) 下的各碎石种类的效应值为:
\[
\overline{\partial}_{1}=4783.8-4160.8=623, \overline{\partial}_{2}=6179.5-4160.8=2018.7, \overline{\partial}_{3}=3180.7-4160.8=-980.2,
\]
\[
\overline{\partial}_{4}=2728.9-4160.8=-1431.9, \overline{\partial}_{5}=4343.7-4160.8=182.9,
\]
\[
\overline{\partial}_{6}=4484.3-4160.8=323.5, \overline{\partial}_{7}=2337.8-4160.8=-1823, \overline{\partial}_{8}=2001-4160.8=-2159.8
\]
通过计算分析 $\overline{\partial}_{8}$ 即 SUP-25 对车辙的影响最大。

4) 碎石质量不同对弯拉应变的显著性影响分析

由于 25 系列对弯拉应变均没有数据,故选择六个水平进行分析。


\begin{table}[h]
\centering
\caption{碎石种类对弯拉应变的影响的单因素方差分析表}
\begin{tabular}{|c|c|c|c|c|}
\hline 方差来源 & $DF$ (自由度) & $S^{2}$ (平方和) & $\overline{S^{2}}$ (均方差) & $F$ \\
\hline 因素A & 7 & 41071295.2 & 8214259.04 & 1.623102794 \\
\hline 随机误差 & 153 & 784429770.1 & 5060837.227 & \\
\hline 总和 & 160 & & & \\
\hline
\end{tabular}
\end{table}

当 $\partial=0.05$ 时,在 $F$ 分布表中查得分位数值 $F_{0.95}(7,153)=2.06$,

\begin{table}[h]
\centering
\caption{碎石种类对路面影响的综合分析表格}
\begin{tabular}{|c|c|c|c|}
\hline
 & 影响种类 & 影响程度 & 影响最大的碎石种类 \\
\hline
碎石的种类对TSR(\%)显著性影响 & 抗水损害性能 & 有显著性影响 (2.199902 -2.01)/2.01=9\% & AC-16 \\
\hline
碎石的种类对S0(\%)显著性影响 & 抗水损害性能 & 有显著性影响 (3.691191 -2.01)/2.01=83.6\% & AC-13 \\
\hline
碎石的种类对车辙显著性影响 & 高温性能 & 有显著性影响 (35.4296-2.01)/2.01=1663\% & SUP-25 \\
\hline
碎石的种类对弯拉应变显著性影响 & 低温性能 & 无显著性影响 & \\
\hline
\end{tabular}
\end{table}

\subsection{5.4.3 不同产地、不同厂家但型号相同、类型相同的沥青、碎石质量对路面性能影响分析}

对于这个问题,我们采用双因素方差分析方法来解决这个问题,分别把厂家和产地作为两个试验因素。由于不同的厂家和不同的产地这两个因素对指标的效应不是简单的叠加,而是它们两个因素有“交互效应”,因此我们采用“有交互作用的双因素方差分析”表来分析它们对路面性能的影响。在这种情况下,数据结构如下表所示。

\begin{table}[h]
\centering
\caption{有交互作用的双因素方差分析数据结构表}
\begin{tabular}{|c|c|c|c|c|c|}
\hline
 & & \multicolumn{4}{c|}{因素B} \\
\hline
 & & $B_{1}$ & $B_{2}$ & $\cdots$ & $B_{s}$ \\
\hline
\multirow{4}{*}{因素A} & $A_{1}$ & $y_{111}$, $y_{112}$, $\cdots$, $y_{11n}$ & $y_{121}$, $y_{122}$, $y_{12n}$ & & $y_{1s1}$, $y_{1s2}$, $y_{1sn}$ \\
\hline
 & $A_{2}$ & $y_{211}$, $y_{212}$, $\cdots$, $y_{21n}$ & $y_{221}$, $y_{222}$, $y_{22n}$ & & $y_{2s1}$, $y_{2s2}$, $y_{2sn}$ \\
\hline
 & $\cdots$ & $\cdots\cdots$ & $\cdots\cdots$ & & $\cdots\cdots$ \\
\hline
 & $A_{r}$ & $y_{r11}$, $y_{r12}$, $y_{r1n}$ & $y_{r21}$, $y_{r22}$, $y_{r2n}$ & & $y_{rs1}$, $y_{rs2}$, $y_{rsn}$ \\
\hline
\end{tabular}
\end{table}

其中,$y_{ijk}$ 表示因素A、B在第$i$,$j$个水平状态下第$k$个样本观测值。

其模型如下:

\begin{equation}
\begin{cases}
y_{ijk} = \overline{\mu} + \partial_{i} + \beta_{j} + \gamma_{ij} + e_{ijk} \\
e_{ijk} \sim N(0, \sigma^2), \text{且各 } e_{ijk} \text{ 相互独立} \\
\sum_{i=1}^{r} \partial_{i} = 0, \sum_{j=1}^{s} \beta_{j} = 0, \sum_{i=1}^{r} \sum_{j=1}^{s} \gamma_{ij} = 0
\end{cases}
\end{equation}

其中 $\partial_{i}, \beta_{j}$ 分别表示因素 A、B 对试验指标所起的作用,$\gamma_{ij} = (\mu_{ij} - \overline{\mu}) - \partial_{i} - \beta_{j}$,$(\mu_{ij} - \overline{\mu})$ 反映水平组合 $(A_{i}, B_{j})$ 对试验指标的总效应,$\gamma_{ij}$ 等于总效应减去 $A_{i}$ 的效应 $\partial_{i}$ 及 $B_{j}$ 的效应 $\beta_{j}$,所以 $\gamma_{ij}$ 表示 $A_{i}$ 与 $B_{j}$ 对试验指标的交互效应,需要检验以下假设:

对因素 A,提出:“因素 A 对试验指标影响不显著”等价于
\[ H_{01}: \partial_{1} = \partial_{2} = \ldots = \partial_{r} = 0, \]

对因素 B,提出:“因素 A 对试验指标影响不显著”等价于
\[ H_{02}: \beta_{1} = \beta_{2} = \ldots = \beta_{s} = 0, \]

对交互效应 A*B,提出:“A*B 对试验指标影响不显著”等价于
\[ H_{03}: \gamma_{ij} = 0 (i = 1, 2, \ldots, r; j = 1, 2, \ldots, s) \]

在本问中,我们把不同厂家设为因素 A,把不同产地设为因素 B,用 SPSS 软件分析不同厂家、不同产地但类型、型号相同的沥青对路面质量的影响,结果如下:


\begin{table}[h]
\centering
\caption{各个型号的不同产地对指标的相伴概率}
\begin{tabular}{|c|c|c|c|c|c|c|}
\hline
\textbf{指标} & \textbf{Sup-13} & \textbf{Sup-25} & \textbf{SMA-13} & \textbf{AC-13} & \textbf{AC-16} & \textbf{AC-20} \\
\hline
TSR(\%) & 0.689 & 0.182 & 0.551 & 0.108 & 0.442 & 0.042 \\
\hline
SO(\%) & 0.011 & 0.078 & 0.424 & 0.311 & 0.509 & 0.707 \\
\hline
车辙 & 0.977 & 0.126 & 0.107 & 0.466 & 0.647 & 0.212 \\
\hline
弯拉应变 & 0.026 & \textasciitilde\textasciitilde\textasciitilde & 0.224 & 0.065 & 0.240 & 0.747 \\
\hline
\end{tabular}
\end{table}


\begin{table}[h]
\centering
\caption{各个型号的厂家对指标的相伴关概率}
\begin{tabular}{|c|c|c|c|c|c|c|}
\hline
\textbf{指标} & \textbf{Sup-13} & \textbf{Sup-25} & \textbf{SMA-13} & \textbf{AC-13} & \textbf{AC-16} & \textbf{Ac-20} \\
\hline
TSR(\%) & \textasciitilde\textasciitilde\textasciitilde & 0.186 & 0.517 & 0.441 & 0.113 & 0.018 \\
\hline
SO(\%) & \textasciitilde\textasciitilde\textasciitilde & 0.325 & 0.952 & 0.564 & 0.309 & 0.364 \\
\hline
车辙 & \textasciitilde\textasciitilde\textasciitilde & 0.321 & 0.317 & 0.977 & 0.150 & 0.001 \\
\hline
弯拉应变 & \textasciitilde\textasciitilde\textasciitilde & \textasciitilde\textasciitilde\textasciitilde & 0.080 & 0.225 & 0.251 & 0.147 \\
\hline
\end{tabular}
\end{table}


\begin{table}
\centering
\caption{各个型号的不同厂家和产地共同对指标的相伴概率}
\begin{tabular}{|c|c|c|c|c|c|c|}
\hline
指标 & Sup-13 & Sup-25 & SMA-13 & AC-13 & AC-16 & Ac-20 \\
\hline
TSR(\%) & $\sim\sim\sim$ & $\sim\sim\sim$ & 0.836 & $\sim\sim\sim$ & $\sim\sim\sim$ & $\sim\sim\sim$ \\
\hline
SO(\%) & $\sim\sim\sim$ & $\sim\sim\sim$ & 0.475 & $\sim\sim\sim$ & $\sim\sim\sim$ & $\sim\sim\sim$ \\
\hline
车辙 & $\sim\sim\sim$ & $\sim\sim\sim$ & 0.163 & $\sim\sim\sim$ & $\sim\sim\sim$ & $\sim\sim\sim$ \\
\hline
弯拉应变 & $\sim\sim\sim$ & $\sim\sim\sim$ & 0.144 & $\sim\sim\sim$ & $\sim\sim\sim$ & $\sim\sim\sim$ \\
\hline
\end{tabular}
\end{table}

注:当相伴概率$<0.05$即为有显著性影响。

通过上述双因素分析表我们可以知道,不同厂家、不同产地的但型号相同、类型相同的沥青、碎石质量对高速公路路面性能没有显著影响。

\subsection{5.5 问题5的求解}

\subsubsection{5.5.1 测试高速公路的试验项目中重要的遗漏分析}

(1) 增加沥青性能指标的试验资料:

在附件一中仅提供油石比、沥青型号及厂家就试图分析出其与路面使用性能之间的关系,太过于简单,与实际情况相差很远,沥青的高温、低温、抗水损害性能对沥青路面的使用性能有非常重要的影响,仅靠油石比、沥青型号及厂家很难反映沥青的因素对高速公路路面质量的使用性能。

(2) 增加集料质量、矿粉质量的试验资料:

在附件一中仅提供集料的级配、空隙率情况来分析路面使用性能,而集料的一些重要指标没有提供,在分析中很难找出这些因素与路面使用性能的相互作用关系,比如粗集料的压碎值、磨耗值、吸水率等重要指标。

(3) 路面的使用寿命与损害情况的调查:

本课题仅仅提供四个试验指标来分析高速公路路面的使用性能,这四个试验指标能在较大程度上反映高速公路的路面质量,但是只能说明路面建设初期的路面使用性能,而路面完全暴露在大自然中,受天气、气候等的影响较大,在长期的使用过程中,各种因素的后期作用效果和影响是否在多大程度上与四个试验指标相符?而高速公路的使用性能是在后期的使用过程中才能体现出来的,所以建议提供道路使用寿命与损害情况的调查数据资料,以分析其与质量影响因素之间的关系,这种分析更加容易发现问题和符合路面的实际情况。

(4) 交通流量调查以及超载现象的分析研究:

高速公路的交通流量对车辙性能有很大的影响,不同的交通流量对路面的破坏性是不一样的,路面的寿命也就不一样。同样超载现象严重的道路对路面的破坏性也更大,因此要增加这两项内容。

(5) 施工情况的调查:

本课题提供的背景情况是高速公路的建设工艺比较简单,在附件一中提供的290组路面试验样本数据中,不能保证100\%的施工质量均不存在问题,在数据分析中发现对路面质量有影响的10个因素几乎相同的情况下,四个指标值还是有较大差距,这说明了施工质量的影响是存在的。所以建议增加施工情况的分析指标,这样可以更加准确的分析问题;

(6) 基层质量的影响分析

沥青路面结构由面层、基层、垫层、土基组成,路面使用性能受到其他层的影响较大,尤其是基层的影响。自 1988 年,我国第一条高速公路建成通车以来,到 2006 年底,高速公路总里程已超过 4 万公里。在这些高速公路的路面结构中,超过 80\% 的基层采用的是半刚性基层。半刚性基层具有很好的整体性,有较大的扩散荷载的能力。然而,十几年的使用表明,铺筑半刚性基层的沥青路面普遍存在早期损坏现象。一种早期损坏是在沥青路面建成不久,在当年或者 2-3 年沥青路面就发生不同程度的车辙、坑槽、网裂等早期损坏。另一种类型是普遍达不到路面设计要求的设计年限(我国沥青路面设计年限大部分是 15 年,少数是 20 年),充其量 7-8 年,或者 10 年左右就必须进行大修。这不能不引起我国工程技术人员对半刚性基层沥青路面的反思。第一,半刚性基层必然发生干缩裂缝和温缩裂缝是个无法改变的事实,这必然导致沥青路面反射裂缝的问题;第二,半刚性基层沥青路面结构的不相容性也是导致其发生各种病害的根源。由于半刚性基层强度大、模量高,在交通荷载的作用下,使得沥青层处于非常不利的受力状态。研究表明,基层刚度较大时,沥青层将承受较大的竖向压应力和剪应力,这将加速沥青路面损坏的发生。第三,半刚性基层本身非常致密,几乎成为完全不透水的层次。来自沥青层的水将积存在基层表面,无法通过基层排走,在车辆荷载的作用下,基层表面将不断受到动水压力的冲刷,逐渐形成灰浆,半刚性基层的灰浆逐渐充满下面层的空隙,并在行车荷载产生的动水压力作用下,产生通常所说的唧浆。此外,动水不断冲刷沥青膜,造成沥青混合料松散,成为沥青面层的水损害破坏的重要原因。

\subsubsection{5.5.2 对高速公路建设部门的建议}

(一) 提高低温性能的建议:

(1) 增加测试沥青低温性能的指标:SHRP 研究成果证明了沥青性能对低温问题的直接贡献率为 80\%,采用方差分析和回归分析等数学手段建立沥青低温性能指标与路面使用性能之间的数量关系,对如何改善沥青的低温性对提高路面使用性能具有重要的意义;

(2) 沥青混合料的低温抗裂性能主要取决于强度和变形能力,抗裂性能高的材料一般具有较高的强度和较大的变形能力,故应综合考虑应力和变形两个影响因素来评价沥青混合料的低温抗裂性,建议用劈裂强度试验测定劈裂抗拉强度和破坏劲度模量;

(3) 测试沥青的抗老化性能:老化是沥青路面在生产、使用过程中必然发生的现象,老化使得沥青混合料的柔性和粘附性降低,脆性增加,低温变形能力降低,是沥青路面出现开裂的重要影响因素,对于建立沥青抗老化性能与路面使用性能之间的关系具有重要意义。

(二) 提高高温性能的建议:

(1) 密度变化对沥青混合料高温稳定性有显著影响,用提高沥青混合料密度来增加

沥青路面抗车辙能力是非常有效的手段;

(2) 分析表明粗集料越多,高温抗车辙能力并不一定越强,粉胶比也有最佳值,过大或过小的粉胶比也将导致抗车辙能力的下降。

(3) 改性沥青的抗车辙性能要比普通沥青的效果要好,其中SBS改性沥青的效果明显。

(三) 提高抗水损害性能的建议:

(1) 使用优质的碱性石料和性能良好的沥青胶结料;

(2) 采用密实性混凝土级配,保证压实度,减少空隙率;

(3) 对沥青进行改性,使用改性沥青;

(4) 测试集料—沥青界面粘附力,探索改善沥青在集料表明粘附性的方法。

\section{六. 模型的评价与改进}

关于问题 1:多元非线性回归模型的评价与改进:

问题 1 采用多元二项式回归,虽然回归的剩余标准差为 2.9152,比较趋近于 0,可以认为回归效果比较好,但应该有更好的回归方法,使其结果更精确。同时,由于有效样本数据有限,对 BP 神经网络模型的学习和预测精度有一定的影响,另外,BP 神经网络模型还存在初始权值难以确定,局部最优等缺陷,还可以采用遗传算法等智能算法来改进 BP 神经网络模型。

关于问题 2:多因素方差模型和多元回归模型的评价与改进:

由于在进行多因素方差分析时,涉及到的因素比较多,因此我们只选取了对指标影响最为显著的两个因素建立多元回归模型,但实际上对该指标影响的因素有很多,所以模型的精度不一定很高。同时,本文认为得出的结论与实际情况存在一定的差异,需要结合更多的高速公路路面专业知识进行分析和调整,对路面质量的影响还有很多其他重要的因素没有涉及,比如:沥青性能指标、集料质量和矿粉质量指标、交通流量调查以及超载现象等,如果能够得到这些数据,本文认为采用方差分析和回归分析等数学方法,应该可以得出更加令人信服的结论和数量关系。

关于问题 3:立方体单元模型的改进和多元线性回归分析方法的改进:

立方体单元模型存在解决了各种筛孔通过率数量的模拟,但是存在如下三个问题:

(1) 筛孔通过率不影响路面压实度的上界的结果,即不管筛孔通过率如何变化,路面压实度的上界均为 \(47.64\%\),没有直接反映二者之间的数量关系;

(2) 路面压实度的上界不收敛,为一个常数;

(3) 立方体单元结构形式过于简单,不能模拟集颗粒径形状不规则的实际情况。但是在立方体单元结构的 8 个顶点处还有大量的空间,可以在这 8 个角落增加 8 个圆球,各大球与小球以及立方体单元结构的接触均相切。同样的思路,可以在剩下的空间里继续增加相切小球的类型和数量,从理论上可以建立立方体单元直径以及数量比例与各种筛孔通过率之间的线性关系,各种类型立方体单元模型与空隙率之间存在确定的数量关系,通过求解线性方程组能够比较精确的建立各筛孔通过率与路面

压实度上界之间的数量关系,在此基础上,结合高速公路路面专业知识和级配实验可以提出最优级配。

对于本文通过多元线性回归分析方法得到的各筛孔通过率与路面压实度上界之间的数量关系,是基于随即抽取的 14 组数据样本得到的,这种显著的线性关系是否在更大样本数据中也成立,还需要时间去验证,即便成立,这种线性关系是否能用在高速公路建设的实际中还需要分析和验证,同时如何利用这种数量关系寻找最优级配方案还值得探讨。

关于问题 4 单因素方差分析和双因素方差分析模型的评价与改进

在对不同厂家、不同产地的但型号相同、类型相同的沥青碎石质量对高速公路路面影响的双因素回归分析时,由于样本数据的不足,在进行分析时,结果可能有一定的误差。但根据专业知识和实际情况,这种影响应该是很小的。

\section{七. 参考文献}

[1] 杨虎,刘琼荪,钟波编著. 数理统计 [M]. 高等教育出版社,2006

[2] 赵静,但琦. 数学建模与数学实验 [M]. 高等教育出版社,2000

[3] 朱建平,范霄文编著. Excel 在统计工作中的应用 [M]. 清华大学出版社,2007-

[4] 余建英,何旭红编著. 数据统计分析与 SPSS 应用 [M]. 人民邮电出版社,2003

[5] 孙敬水主编. 计量经济学 [M]. 清华大学出版社,2006

[6] 张登良编著. 沥青路面 [M]. 人民交通出版社,1999

[7] 沈金安著. 沥青及沥青混合料路用性能 [M]. 人民交通出版社,2006

[8] 赵新坡. 密级配沥青稳定碎石基层材料与性能研究 [D]. 长安大学硕士学位论文,2006

[9] 周开利,康耀红编著. 神经网络模型及其 MATLAB 仿真程序设计 [M]. 清华大学出版社,2005

[10] 飞思科技产品研发中心编著. 神经网络理论与 MATLAB7 实现 [M]. 电子工业出版社,2005

[11] 阎平凡,张长永编著. 人工神经网络与模拟进化计算 [M]. 清华大学出版社,2005

\section{八. 附录}

\subsection{附录 1:用 matlab 软件求四个因素之间的数量关系}

Matlab 源程序

\begin{verbatim}
> x1=xlsread('C:\Documents and Settings\anyone\桌面\数模\数据分析, 付鹏\第一题\指标数据', 'Sheet1', 'A1:A167');
>> y=xlsread('C:\Documents and Settings\anyone\桌面\数模\数据分析, 付鹏\第一题\指标数据', 'Sheet1', 'B1:B167');
>> x2=xlsread('C:\Documents and Settings\anyone\桌面\数模\数据分析, 付鹏\第一题\指标数据', 'Sheet1', 'C1:C167');
>> x3=xlsread('C:\Documents and Settings\anyone\桌面\数模\数据分析, 付鹏\第一题\指标数据', 'Sheet1', 'D1:D167');
>> x=[x1, x2, x3];
>> sptool(x, y)
>> beta6, rmse6
beta6 =
   1.0e+002 *
   -1.24828928743796
    0.04603391118663
    0.00001656876237
\end{verbatim}

\begin{align*}
0.0000832093552 \\
-0.00025310896815 \\
-0.0000000126093 \\
-0.00000000027793 \\
\text{rmse6} = \\
2.91521924089186
\end{align*}

\section{附录 2:问题 2 的多元回归方法建立四个指标和比较显著影响因素之间的数学模型源程序}

1. 对 TSR (\%) 和通过筛孔百分率 (\%) 0.30、0.15 建立多元回归模型

\begin{verbatim}
a1=xlsread('C:\Documents and Settings\anyone\桌面\数模\数据分析, 付鹏\第二题\模型建立\指标一.xls', 'Sheet1', 'A1:A225');
a2=xlsread('C:\Documents and Settings\anyone\桌面\数模\数据分析, 付鹏\第二题\模型建立\指标一.xls', 'Sheet1', 'B1:B225');
a=[a1,a2];
b=xlsread('C:\Documents and Settings\anyone\桌面\数模\数据分析, 付鹏\第二题\模型建立\指标一.xls', 'Sheet1', 'C1:C225');
rstool(a, b, 'purequadratic')
\end{verbatim}

\begin{figure}[h]
    \centering
    \includegraphics[width=\textwidth]{image.png}
    \caption{图示}
\end{figure}

\begin{align*}
\text{beta1} = \\
88.9258 \\
0.9139 \\
-1.6979 \\
-0.0720 \\
0.1195
\end{align*}

\begin{verbatim}
>> rmse
rmse =
3.3982
b=88.9258+ 0.9139 a1 -1.6979a2 -0.0720 a1^2 + 0.1195a2^2
\end{verbatim}

2. 对 S0(\%) 和 VMA 矿料间隙率和 VFA 饱和度建立多元回归模型

设 $V9, V10$ 分别为 $a1, a2$,指标 2 为 $b$;

\begin{verbatim}
a1=xlsread('C:\Documents and Settings\anyone\桌面\数模\数据分析, 付鹏\第二题\模型建立\指标二.xls', 'Sheet1', 'A1:A269');
a2=xlsread('C:\Documents and Settings\anyone\桌面\数模\数据分析, 付鹏\第二题\模型建立\指标二.xls', 'Sheet1', 'B1:B269');
a=[a1,a2];
b=xlsread('C:\Documents and Settings\anyone\桌面\数模\数据分析, 付鹏\第二题\模型建立\指标二.xls', 'Sheet1', 'C1:C269');
rstool(a, b, 'purequadratic')
\end{verbatim}

\begin{figure}[h]
    \centering
    \includegraphics[width=\textwidth]{image.png}
    \caption{Pure Quadratic Model}
\end{figure}

\begin{verbatim}
beta =
  222.1100
  -1.2730
  -3.5730
   0.0365
   0.0262
>> rmse
rmse =
3.3255
b=222.1100 -1.2730 a1 -3.5730 a2+ 0.0365a1^2+ 0.0262a2^2
\end{verbatim}

附录 3 问题 3 的多元线性回归源程序

\begin{verbatim}
Y=xlsread('D:\MATLAB7\work\C3\15.xls', 1, 'B20:N33'); //读取因素数据
X=xlsread('D:\MATLAB7\work\C3\15.xls', 1, 'G3:G16'); //读取压实度数据
[b, bint, r, right, stats] = regress(Y, X); //做多元线性回归分析
b, bint, r, right, stats //输出结果及参数
rcoplot(r, right); //做残差分析图
\end{verbatim}

在显著性水平为0.05时,运行结果如下:

\begin{align*}
b &= \\
& 0 \\
& 0.9297 \\
& 0.7484 \\
& 1.0177 \\
& 0.9405 \\
& 0.9698 \\
& 0.7342 \\
& 1.0203 \\
& 1.3121 \\
& 0.5785 \\
& 0.5776 \\
& 2.1347 \\
& 2.4187 \\
\text{bint} &= \\
& 0 \quad 0 \\
& 0.5206 \quad 1.3389 \\
& -1.8581 \quad 3.3550 \\
& 0.4985 \quad 1.5369 \\
& -0.2468 \quad 2.1277 \\
& 0.5823 \quad 1.3573 \\
& -1.2368 \quad 2.7051 \\
& -1.5050 \quad 3.5457 \\
& 0.8937 \quad 1.7305 \\
& -1.3334 \quad 2.4904 \\
& -2.0437 \quad 3.1988 \\
& -6.1866 \quad 10.4559 \\
& -7.0376 \quad 11.8749 \\
r &= \\
& -0.1022
\end{align*}

\begin{align*}
-0.0138 \\
0.0070 \\
-0.0872 \\
0.0018 \\
-0.0033 \\
-0.0454 \\
0.2464 \\
0.0249 \\
-0.1467 \\
0.1496 \\
0.1692 \\
-0.1391 \\
-0.0613
\end{align*}

\begin{align*}
\text{right} = \\
-0.4138 & \quad 0.2094 \\
-0.5715 & \quad 0.5439 \\
-0.1962 & \quad 0.2102 \\
-1.0689 & \quad 0.8945 \\
-0.1715 & \quad 0.1751 \\
-0.4070 & \quad 0.4005 \\
-0.2069 & \quad 0.1160 \\
-0.4430 & \quad 0.9359 \\
0.0101 & \quad 0.0397 \\
-1.0119 & \quad 0.7186 \\
-0.4343 & \quad 0.7335 \\
-0.0190 & \quad 0.3575 \\
-0.3606 & \quad 0.0823 \\
-0.1026 & \quad -0.0199
\end{align*}