\documentclass{article}
\usepackage{ctex}
\usepackage{amsmath}
\begin{abstract}
本文首先分析了该表中存在的不合理数据,有两类不合理的数据,包括:使用的轿车数据还没有反馈,轿车销售需要一定时间。随后对该表的制作方法提出建议。第二问中,对于0205批次使用月数18时的千车故障数预测,我们假设该模型服从Weibull分布函数,并利用该批次使用月数0至12的数据进行拟合,得到拟合函数,再由拟合函数得到0205批次使用18个月的千车故障数,结果为44.08。对于0306批次使用月数为9时的千车故障数预测,由于每批次的千车故障数符合Weibull分布,因此采用多元线性回归模型,利用前面1至2个月的数据对第9个月的故障数进行回归计算,结果为18.1114。对于0310批次使用月数12的预测,则根据数据反馈滞后与质量管理关系模型从纵向进行预测,结果为6.44。第三问,从销售服务过程质量管理的角度,提出了7个方面的质量管理决策与咨询,并给出相应的数学模型。第四问,分别从生产组织管理和配送及物流链管理两个方面提出了想法与建议。
\end{abstract}

\section{第一问}:
\begin{table}
\centering
\begin{tabular}{|c|c|c|c|c|c|c|c|c|c|c|c|c|c|}
\hline
使用月数 & 12 & 11 & 10 & 9 & 8 & 7 & 6 & 5 & 4 & 3 & 2 & 1 & 0 \\
\hline
生产 & 制表时 & & & & & & & & & & & & \\
月份 & 销售量 & & & & & & & & & & & & \\
\hline
0201 & 2457 & 4.88 & 4.88 & 4.48 & 4.07 & 4.07 & 3.66 & 2.44 & 2.44 & 1.22 & 1.22 & 0.41 & 0.41 \\
\hline
0202 & 1670 & 5.99 & 5.99 & 5.39 & 5.39 & 5.39 & 4.19 & 4.19 & 3.59 & 2.99 & 2.4 & 1.8 & 0 \\
\hline
0203 & 1580 & 4.43 & 3.8 & 3.8 & 3.8 & 3.8 & 3.16 & 2.53 & 2.53 & 1.27 & 0.63 & 0 & 0 \\
\hline
0204 & 3704 & 13.77 & 12.15 & 11.61 & 9.72 & 9.18 & 7.83 & 6.75 & 5.67 & 4.32 & 1.35 & 0.54 & 0 \\
\hline
0205 & 3806 & 36.78 & 34.68 & 31.53 & 29.43 & 27.06 & 25.22 & 23.12 & 21.81 & 18.13 & 16.55 & 13.4 & 3.94 \\
\hline
0206 & 2910 & 41.58 & 39.18 & 36.08 & 32.99 & 31.62 & 28.87 & 24.74 & 23.02 & 18.9 & 15.46 & 13.4 & 4.47 \\
\hline
0207 & 1614 & 72.49 & 69.39 & 62.58 & 54.52 & 47.71 & 43.99 & 40.27 & 34.7 & 30.36 & 26.64 & 22.3 & 3.72 \\
\hline
0208 & 1985 & 75.57 & 71.54 & 69.02 & 64.48 & 56.93 & 52.9 & 45.34 & 36.78 & 28.21 & 20.65 & 13.6 & 1.51 \\
\hline
0209 & 2671 & 112.32 & 110.45 & 108.57 & 104.08 & 95.84 & 84.61 & 74.88 & 65.89 & 52.04 & 42.31 & 27.33 & 1.87 \\
\hline
0210 & 2107 & 121.97 & 119.6 & 116.28 & 115.33 & 107.74 & 96.35 & 84.48 & 69.29 & 54.11 & 39.39 & 22.78 & 2.85 \\
\hline
0211 & 1399 & 95.78 & 95.78 & 94.35 & 92.21 & 85.78 & 82.2 & 72.19 & 61.47 & 47.18 & 40.03 & 25.73 & 3.57 \\
\hline
0212 & 403 & 101.74 & 101.74 & 94.29 & 91.81 & 89.33 & 84.37 & 81.89 & 67 & 52.11 & 44.67 & 32.26 & 7.44 \\
\hline
0301 & 6450 & 122.79 & 122.79 & 122.48 & 121.55 & 119.84 & 115.5 & 108.06 & 98.29 & 82.64 & 66.98 & 44.96 & 3.72 \\
\hline
0302 & 2522 & 143.93 & 143.93 & 143.93 & 143.93 & 141.95 & 139.57 & 135.21 & 125.69 & 106.66 & 84.46 & 62.25 & 1.59 \\
\hline
0303 & 2900 & & 60.34 & 60.34 & 60.34 & 60.34 & 60 & 58.28 & 55.86 & 51.72 & 46.21 & 33.1 & 1.03 \\
\hline
0304 & 1127 & & & 18.63 & 18.63 & 18.63 & 18.63 & 18.63 & 16.86 & 15.97 & 13.31 & 7.99 & 2.66 & 0 \\
\hline
0305 & 818 & & & & 14.67 & 14.67 & 14.67 & 14.67 & 13.45 & 13.45 & 13.45 & 11 & 8.56 & 1.22 \\
\hline
0306 & 1199 & & & & & 5.84 & 5.84 & 5.84 & 5.84 & 5.84 & 5.84 & 5 & 1.67 & 0 \\
\hline
0307 & 1831 & & & & & & 13.65 & 13.65 & 13.65 & 13.65 & 13.11 & 10.38 & 7.1 & 0.55 \\
\hline
0308 & 1754 & & & & & & & 5.7 & 5.7 & 5.7 & 5.7 & 5.7 & 4.56 & 1.71 & 0 \\
\hline
0309 & 2163 &  &  &  &  &  &  & 0.92 & 0.92 & 0.92 & 0.92 & 0.46 & 0.46 &  \\
\hline
0310 & 2389 &  &  &  &  &  &  & 0 & 0 & 0 & 0 & 0 & 0 &  \\
\hline
0311 & 2434 &  &  &  &  &  &  & 0 & 0 & 0 & 0 & 0 & 0 &  \\
\hline
0312 & 1171 &  &  &  &  &  &  & 0 & 0 & 0 & 0 & 0 & 0 &  \\
\hline
\end{tabular}
\caption{}
\end{table}

该表存在不合理数据,即不适合用于预测的数据。

该表为2004年4月1日从数据库中整理出来的某个部件的千车故障数表,但表中2004年1月,2月,3月的数据没有变化,例如0311批次使用了0,1,2,3个月的千车故障数都是0,又如0302批次使用了9,10,11,12个月的千车故障数都是143.93,这是因为已售出的轿车使用几个月后的保修情况还没有数据反馈。所以在2004年1,2,3月发生的故障数还没得到统计,即左下方斜对角的数据的后三组数据(表中标黄色的)都是不合理数据。这也说明了实际上该表只统计到03年的12月份。

表中如0306批次使用了3,4,5个月的千车故障数都是5.84,这是因为前面的车还没卖完,特别是03年01月生长了6450辆,这使得接下来几个月的车售出推迟了,因此在该表统计时0306批次几乎没有能达到这个使用月份的车,也就不会有故障。再如0309批次使用了0,1,2个月的千车故障数只有0.46,0.46,0.92,这都是因为销售要一个工程,进入当时统计范围的汽车并没有制表是的2163辆,而是一个要小的多的车数。表中因为此而偏小的数据有:0211批次使用了12月,0301使用了9,10月,0302使用了8,9个月,0303使用了6,7,8个月,0304使用了5,6,7个月,0305使用了4,5,6个月,0306使用了4,5个月,0307使用了3,4个月,0308使用了2,3个月,0309使用了0,1,2个月,0310使用了0,1个月,0311使用了0个月的千车故障数(表中以黑体标出)。

所以有两种原因使得表中存在不合理数据,一为使用了的轿车数据还没有反馈,一为轿车销售需要一定时间,这两方面都导致了数据上的滞后。当然随着时间的推移,轿车不断地销售出去,已售出轿车使用一段时间后的千车故障数也能不断自动更新,再打印出的表中数据也将会变化,当一批车基本卖出后再经过一段足够长的时间,该表才能准确的反映其千车故障数。

制表方法建议:使用月数最好由0到12从左至右排序,这样便于统计,也更容易看。从千车故障数的数据表中比较难看出信息延迟的情况,故我们在制作新的表时,可以将各个批次单独形成一个表,在每个表中应该包含这个批次的部件出售时间,相应的数量,和跟使用月数有关的千车故障率,如下:

\begin{table}
\centering
\begin{tabular}{|c|c|c|c|c|c|c|c|c|c|c|c|c|c|c|}
\hline
使用月数 & 0 & 1 & 2 & 3 & 4 & 5 & 6 & 7 & 8 & 9 & 10 & 11 & 12 &  \\
\hline
出售 & 数量 &  &  &  &  &  &  &  &  &  &  &  &  &  \\
时间 &  &  &  &  &  &  &  &  &  &  &  &  &  &  \\
\hline
0201 &  &  &  &  &  &  &  &  &  &  &  &  &  &  \\
\hline
\end{tabular}
\caption{}
\end{table}

\section{第二问:}

\subsection{第(1)小问}

\paragraph{问题的提出:}
利用表中的数据预测:0205批次使用月数18时的千车故障数

\paragraph{模型假设:}
(1) 该批次汽车售出已久,所有车使用了12月以上并已记于表中,即其千车故障数为准确信息,没有滞后。
(2) 部件发生故障修理后,该车又投入使用,没有后继的影响。

\paragraph{符号说明:}
\begin{itemize}
    \item $t$:汽车使用月数;
    \item $m$:威布尔函数形状参数
    \item $\eta$:威布尔函数尺度参数
    \item $\sigma$:样本标准差
    \item $\mu$:样本均值
    \item $F(t)$:威布尔函数
    \item $C$:变异系数
\end{itemize}

\paragraph{问题与模型的分析建立和求解:}
由于汽车部件在制造及运行过程中是受各种随机因素的影响,所以它的寿命是随机的。因此通过实验数据的统计处理和理论分析,可以确定部件故障的分布函数。根据汽车该部件故障数据的特点及分布函数的性质,可假设该部件的累积故障概率函数为二参数威布尔分布(Weibull Distribution),即:

\[
F(t) = 1 - \exp\left[-\left(\frac{t}{\eta}\right)^m\right] \quad \cdots (1)
\]

0205批次汽车的千车故障表:

\begin{table}[h]
\centering
\begin{tabular}{|c|c|c|c|c|c|c|c|c|c|c|c|c|}
\hline
使用月数 & 0 & 1 & 2 & 3 & 4 & 5 & 6 & 7 & 8 & 9 & 10 & 11 & 12 \\ \hline
千车故障数 & 3.94 & 8.93 & 13.4 & 16.55 & 18.13 & 21.81 & 23.12 & 25.22 & 27.06 & 29.43 & 31.53 & 34.68 & 36.78 \\ \hline
\end{tabular}
\caption{表(3)}
\end{table}

对于累积故障分布函数为威布尔分布的情形,其分布函数的参数可用变异系数法进行估计,变异系数的表达式为:

\[
C = \frac{\sigma}{\mu} = \left[\frac{\Gamma(1 + \frac{2}{m})}{\Gamma^2(1 + \frac{1}{m})} - 1\right]^{1/2} \quad \cdots (2)
\]

因此给定 $C$ 值,即可以由(2)式计算出相应的 $m$ 值。实际上 $m$ 与 $C$ 之间可以查表求得,同理由相同数表可以确定 $\frac{\mu}{\eta}$ 和 $\frac{\sigma}{\eta}$。由此求出 $\eta$ 的值。

利用 Matlab 对其进行求解(程序 twbr.m),Weibull Distribution 函数的拟合情况如下图:

\begin{figure}[h]
    \centering
    \includegraphics[width=\textwidth]{image.png}
    \caption{0205 批次千车故障数的威布尔拟合}
    \label{fig:1}
\end{figure}

由上图可以看出,威布尔曲线确实很好的拟合了 0205 批次的千车故障数;利用所得的 Weibull Distribution 函数对其进行预测得到 0205 批次的使用月数 18 时的千车故障数为 44.08。

\paragraph{累计故障 Weibull Distribution 函数的假设检验:}

累计故障分布函数是否准确反映样本的分布规律,可以按照数理统计的方法进行假设检验。在可靠性工程中应用较多的有两种方法:其一是 $\chi^2$ 检验,其二是柯尔莫哥洛夫-斯米尔诺夫(Kolmogorov-Smirnov)检验。本模型采用 Kolmogorov-Smirnov 检验法。该检验为拟合优度检验,假设 $F_0(x)$ 是一已知的分布,$F_n(x)$ 是未知分布函数的一个较优的估计,取检验统计量 $D = \max \left| F_n(x) - F_0(x) \right|$

则样本服从指定分布(即 $F(x) = F_0(x)$)时,D 的观测值应该较小,如果 D 的观测值较大,则原假设可能不成立。利用 Matlab 调用函数 H=kstest(X,cdf,alpha,tail); 取 alpha=0.05,检验得该批次的千车故障数服从 Weibull Distribution。也即我们所建模型预测到 0205 批次的使用了 18 个月份的千车故障数(44.08)具有一定的可信度。

\subsection{第(2)小问}

由上面的讨论可知,千车故障数 s(t) 除以 1000 后服从

\begin{equation}
F(t) = 1 - \exp \left[ - \left( \frac{t}{\eta} \right)^m \right] \quad \cdots \cdots (1)
\end{equation}

(t=0,1,2,\ldots,)

\begin{align*}
\therefore \ln(-\ln(1-F(t)) &= m\ln(t) - m\ln(\eta) \\
\text{令 } K(t) &= \ln(-\ln(1-F(t)); \\
\text{可得 } K(1) &= -m\ln(\eta); \\
K(2) &= m\ln(2) - m\ln(\eta); \\
K(3) &= m\ln(3) - m\ln(\eta); \\
K(9) &= 2m\ln(3) - m\ln(\eta); \\
\text{所以有:} \\
K(9) &= 2K(3) - K(1) \tag{3} \\
K(9) &= \frac{2\ln 3}{\ln 2} \cdot (K(2) - K(1)) + K(1); \tag{4} \\
K(9) &= K(3) + \frac{\ln(3)}{\ln(2)}(K(2) - K(1)) \tag{5}
\end{align*}

理论上来说,只要知道前两个数据 \(K(1)\) 和 \(K(2)\),即只需要知道 \(F(1), F(2)\),即可预测出九月份的千车故障率,代入(3)式得 \(K(9) = -3.8861\);即 0306 千车故障率 \(s(9) = 20.3160\)。

同理,利用(4)式可得 \(s(9) = 52.8845\)。

利用(5)式,可得 \(s(9) = 32.8419\)。

可见,不同的表达式算出来的值相差很大,其原因是数据本身,每一批次的车数不够,统计的数量不够大,造成了数值的不够稳定,这样就不能直接用理论的表达式来计算。但我们从上面可以看出,千车故障率 \(s(9)\) 与前面的 \(s(1), s(2), s(3)\) 大致成线性关系:

\[
s(9) = b(0) + b(1) \cdot s(1) + b(2) \cdot s(2) + b(3) \cdot s(3) + \varepsilon
\]

\[
\varepsilon \sim N(0, \sigma^2)
\]

使用 0201 到 0301 的数据,其中去掉 0203,0209,0210 这三组数据,因为在实际计算中无法取到对数。

在 matlab6.5 上可以计算出(程序 treg.m):

\[
b(0) = 10.1945; b(1) = 3.4527, \, b(2) = -2.4145, \, b(3) = -0.1046
\]

图表如下:

\begin{figure}[h]
    \centering
    \includegraphics[width=\textwidth]{image.png}
    \caption{图(2)}
\end{figure}

所以有0306的9月份的千车故障率 \( s(9) = 18.1114 \)

这个数据跟用 \( s(3) \) 式得出的数据相差不大,应该是相对准确的。但在回归的过程中发现,结果受数据选取的影响还是比较大的。在实际操作过程中,可以通过多选取一些数据来解决。

\subsection{第(3)小题}

\subsubsection{问题的提出:}

利用这个表的数据预测0310批次使用月数12时的千车故障数

\subsubsection{模型假设:}

(1) 假设每个批次生产的车在其出厂后的3个月内全部卖出

(2) 假设每月的销售量基本稳定(其平均销售值为2211辆/月)

\subsubsection{符号说明:无}

\subsubsection{问题与模型的分析建立和求解:}

由于销售的需要一个过程,假设每月的销售量为平均销售额2211辆,由于厂家总希望将生产年份早的车卖出,所以可假定每批车在生产后的3个月内会卖掉,根据销售量对那些统计是并没有卖出那么多车辆的千车故障数进行修正。

由上面模型知没批次均服从Weibull Distribution,因此使用了12个月份的千车故障数基本上线性依赖于前面的月份,所以可以根据多元线形回归对那些使用月数还没到达12个月的进行计算,得出其使用了12个月的千车故障数得到0210至0309批次使用12个月的千车故障数:

\begin{table}[h]
    \centering
    \begin{tabular}{|c|c|c|c|c|c|c|c|c|c|c|c|}
    \hline
    0201 & 0202 & 0203 & 0204 & 0205 & 0206 & 0207 & 0208 & 0209 & 0210 & 0211 & 0212 \\ \hline
    4.88 & 5.99 & 4.43 & 13.77 & 36.78 & 41.58 & 72.49 & 75.57 & 112.32 & 121.97 & 95.78 & 112.01 \\ \hline
    & & & & & & & & & & & \\ \hline
    0301 & 0302 & 0303 & 0304 & 0305 & 0306 & 0307 & 0308 & 0309 & & & \\ \hline
    164.7 & 185.34 & 90.301 & 28.388 & 33.856 & 14.052 & 37.856 & 15.209 & 10.67 & & & \\ \hline
    \end{tabular}
    \caption{表(4)}
\end{table}

图(3)略

质量控制对部件的故障有着及其关键的影响,只有合理地进行质量管理,量,部件的故障率才可以控制在一定的范围内。

当某部件故障率很低时,生产商就精力放于其它方面:如增加产量,降低成本,或着重于另外的部件,这就会导致该部件质量的降低,于是故障率逐渐上升,同时考虑到信息反馈的滞后,所以从统计表中发现故障数过高到提高质量需要一个过程。于是故障数不断的上升,当生产商关注该问题并实施质量改进,这也好一个过程,所以其故障数发展趋势如下图所示:

\begin{figure}[h]
\centering
\includegraphics[width=0.8\textwidth]{image.png}
\caption{故障数与生产批次(生产月份)之间关系}
\label{fig:4}
\end{figure}

由表中可知在 0302 批次中其千车故障数达到最大。接着质量得到控制,故障数不断很快下降。

根据此分析的进行数据拟合可得 0310 批次使用月数 12 时的千车故障数是:6.44。

\section{第三问:}

本题是属于一个销售服务过程质量管理。

服务质量管理的主要目标是用户满意度,提高企业生产质量和降低生产成本,通过提高用户的满意度来赢得消费者。服务质量管理内容包括服务的态度、服务工作的完好度(维修是否及时,一次性维修成功率)顾客投诉处理与顾客访问,服务质量管理措施计划制定的依据是所有千车故障数的数据表。

根据这些表格,删除无用的数据,可以得出下列的决策。

\paragraph{1、各种部件的维修率比较:}

对各种部件人维修率作出统计,重点关注那些返修率较高的部件,对其进行分类。

返修率较高零件的分类(5)

\begin{table}[h]
\centering
\begin{tabular}{|c|c|}
\hline
\diagbox{分类}{返修率} & \\
\hline
制造工艺 & \\
\hline
技术设计 & \\
\hline
维护问题 & \\
\hline
\end{tabular}
\caption{返修率较高零件的分类}
\label{tab:5}
\end{table}

并将用户需求反馈给制造商和厂家。

\paragraph{2、部件成本效益比较:}
统计返修部件的供应商和供应价格,并比较同类供应商的价格,通过调整供应商降低成本。

\paragraph{3、对车的部件返修率进行排序,并对部件返修率与顾客对该部件的重视程度进行矩阵分析。(6)}

\begin{figure}[h]
    \centering
    \includegraphics[width=0.8\textwidth]{image1.png}
    \caption{返修率与顾客重视程度矩阵}
    \label{fig:matrix}
\end{figure}

得到四个区域

\begin{table}[h]
    \centering
    \begin{tabular}{|c|c|}
        \hline
        I & II \\
        \hline
        III & IV \\
        \hline
    \end{tabular}
    \caption{四个区域划分}
    \label{tab:regions}
\end{table}

I(维修率高,顾客重视程度高) \\
II(维修率高,顾客重视程度低) \\
III(维修率低,顾客重视程度高) \\
IV(维修率低,顾客重视程度低)

关注和重视I类的部件,并将意见反馈类生产商,通过改进技术和提高质量管理,降低部件的返修率,提高顾客的满意程度。

采用双向逼近的产品设计方法:一方面考虑用户的质量、功能需求,一方面考虑制造商和可制造性需求,从满足消费者的角度出发,产品的设计应尽可能逼近用户的需求,这是质量设计过程中向外逼近过程;同时,产品的质量设计也是满足制造系统可制造性要求,这是向内逼近的要求。

\begin{figure}[h]
    \centering
    \includegraphics[width=0.8\textwidth]{image2.png}
    \caption{双向逼近的产品设计方法}
    \label{fig:design_method}
\end{figure}

\paragraph{4、统计维护时间,维护成本,维护工人类用等,降低维护成本。}
维护成本是指销售服务过程中的质量成本,即销售过程中为保证产品或服务质量的支出。

和费用,以及未达到产品或服务质理标准而发生的损失费用,包括索赔、保修等。从所有部千车故障数的数据表,可以统计出关于维护成本的以下图形。

\paragraph{5. 根据用户的信息,统计返修部件意见最多的用户和他们的职业,年龄、性别、区域等特征。并找出特征和部件维修之间的相关性,待征之间的相关性,作出统计分析,加强对用户的培训和对轿车使用的介绍。}

\paragraph{6. 利用矩阵分析法对部件与部件之间进行交叉分析。}

\begin{figure}[h]
\centering
\includegraphics[width=0.8\textwidth]{image.png}
\caption{零件 1 和零件 2 的关系图}
\end{figure}

重视协方差较大的两个零件,并分析部件同时返修的成因,交由厂商进行处理。

\paragraph{7. 统计同一部件返修时间段的频率分布,并得出所有部件返修时间段的频率分布。}

总之,掌握轿车所有部件千车故障数的数据表,可以为质量管理提供决策。

(1) 提高服务质量,降低服务费用。

(2) 跟踪轿车用户对产品的使用情况,减少索赔退货等造成质量成本损失。

(3) 增加用户信息反馈,使企业不断改进产品设计,制造质量,降低服务维修费用。

\section{第四问}

\subsubsection{生产组织管理}

1. 产品开发与工艺设计

产品开发与工艺设计对产品质量产生重大的影响,也从一定程度上造成汽车零部件的缺陷和瑕疵。从零部件故障的归因分析来看,如果产品质量所蕴含的技术价值下降到一定程度,必须通过产品开发来提高其质量,倘若产品的工艺设计不符合要求,产品质量也会出现问题。另外,由于产业升级的需要,即便是高科技产业也面临着巨大挑战和威胁。产品开发与工艺设计具有重要意义。

2. 综合生产计划

综合生产计划是一种中期的生产计划,规定某一年度内企业主要经济指标,如品种、产量、产值等,它一个非常重要的内容是进行生产计划与能力的平衡。分析生产能力需求包括如何进行生产能力的核算、如何提高生产能力的利用率、如何进行生产能力的长期和短期调整等。

3. 作业计划与生产控制

作业计划是根据物料需求计划下达的生产任务,结合车间的情况,安排本车间各工段、工作中心、轮班的作业任务。作为作业计划的基础,作业排序具有关键作用。

\subsubsection{配送及物流链管理}

发展全方位的汽车零部件配送系统和物流链管理服务,有利于培养企业的核心能力,从而提高企业在市场竞争力,获得持续的竞争优势。

\paragraph{1. 改善提高零部件物流服务}

\paragraph{2. 更新改革零部件经营观念}

企业如果想要提高顾客的品牌忠诚度,必须及时变换经营理念,调整僵滞机制;给企业和产品重新定位。改革陈旧的汽车零部件的经营方式,开拓和重新组合汽车零部件销售服务

渠道,是极其重要的根基。为了在激烈的市场竞争中不输给其它竞争对手,企业必须令其每一位职员都坚决彻底改变陈旧落伍的汽车零部件交易经营观念。首先,必须让公司大多数高级管理人员确信,汽车售后服务、汽车零部件销售和快速递送,以及汽车备件物流链管理都是公司的利润中心之一,而决不是以往的偏见:即成本或者费用是低人一等的差使。汽车产品的销售,或者新型汽车产品的销售果然需要竞争,但是新车需要的零部件不多。各种牌号汽车厂商之间最激烈的竞争往往不是在汽车本身;而是在售后服务及其可以互相替代使用的零部件销售额上。如果消费者到其汽车零部件特许专卖店的人数存在逐年减少的状况,汽车公司的整体销售额就会下跌,成本相应会上升,公司经济效益必然下滑。如果汽车售后服务管理人员缺乏敬业精神;官僚主义严重,销售渠道不畅,销售方式单一,缺乏灵活变通,不贴近市场,把消费者看成是麻烦制造者等等,这种观念有可能影响到公司的经营业绩。

\paragraph{3. 建立统一零部件中心枢纽}

公司必须根据地理环境以及市场的变化,以及汽车用户的数量和销售商的经营情况,把市场区域划分成不同的汽车零部件配送中心和物流链管理区。当然这决不是一成不变的,将随时根据市场需求,及时进行调整,其唯一动力就是千方百计满足客户的要求和逐步提高公司的经济效益。在改革的同时,还必须竞争对手的活动。其次,在汽车零部件物流链管理和汽车售后服务上,应该做好全面筹划和部署。既要重点开发发展目标客户集中的区域市场,又要兼顾满足消费者急需的其它汽车售后服务地区,比如设立汽车零部件特许转卖点,专门为客户提供紧急援助等。再次,通过一系列的改革措施,包括缩短汽车备件和各种零部件产品从设计到成品最后到消费者之间的时间,减少成本,向管理要经济效益,可以达到重组改革的效果。汽车零部件的物流链服务不仅促进售后服务显著改善,而且又鼓励汽车公司上层及时作好决策,采取各种切实措施降低包装、物流链各个关节之间的联结点以及保养检修的成本,因为竞争焦点最后还是集中在经营成本上。此外在零部件的供应链上普遍采用电子商务方式;提高工作效率,减员增效。为了进一步提高物流链配送服务的精确性,提高其成品

\paragraph{4. 构建立体零部件服务网络体系}

在传统的配送体系里,对于不常用的配件,公司的配送时间往往落后于消费者愿意等待的时间。现在通过物流链的网络向各个中心和汽配供应商发出指令,可以很好地解决这个问题。

先进的汽车物流信息管理系统能够为企业零部件物流配送提供技术支持。在运输、仓储管理方面的信息系统化和条形码化管理,可以提升企业服务水平、降低经营成本。企业还可以借助汽车物流信息管理系统的 GPS 车辆监控系统,来有效控制并降低运输成本、提高运输效率。

\section{总结:}

本文通过对千车故障率的数据表的分析,得到了故障率的分布模型,并在此基础上得出了相应的预测,再从质量管理的角度,提出了决策和咨询的建议。

\section{参考文献}

1. 《数学模型与数学建模》 刘来福,曾文艺

2. 《概率论与数理统计》 余锦华等

3. 《经济计量分析》第五版 Geeene

4. 《数学建模的理论与实践》吴孟达 等编著

5. 《数学模型》第二版 姜启源编