\begin{center}
\includegraphics[width=0.2\textwidth]{image1.png} \quad
\includegraphics[width=0.2\textwidth]{image2.png} \quad
\includegraphics[width=0.2\textwidth]{image3.png} \quad
\includegraphics[width=0.2\textwidth]{image4.png}
\end{center}

\begin{center}
\textbf{中国研究生创新实践系列大赛} \\
\textbf{“华为杯”第十八届中国研究生} \\
\textbf{数学建模竞赛}
\end{center}

\begin{table}[h]
\centering
\begin{tabular}{l l}
学校 & 常州大学 \\
\hline
参赛队号 & 21102920005 \\
\hline
队员姓名 & \begin{tabular}{l l}
1. & 陈思炜 \\
2. & 许浩 \\
3. & 黄宁
\end{tabular}
\end{tabular}
\end{table}

\begin{center}
\textbf{中国研究生创新实践系列大赛}
\end{center}

\begin{center}
\textbf{“华为杯”第十八届中国研究生数学建模竞赛}
\end{center}

\begin{center}
\textbf{题目} \quad 航空公司机组优化排班问题
\end{center}

\section*{摘 要:}

机组分配问题可以表述如下,航空公司必须在给定时间段内运营多个不同起点和目的地的航班段,为了运营所有已计划的飞行段,公司必须向其分配适当的机组成员(数量和资格),同时需要考虑各种工作规定和集体协议。我们主要结果如下:

针对问题一:可将原问题划分为航班规划和机组排班两个子问题,以尽可能多的航班可以满足最低机组人员配置为主要目标,在最短连接时间等限制条件下,建立单目标的加权 0-1 整数规划模型。首先利用广度优先搜索方法寻找所有满足限制条件的航班信息,组成适合的航段执勤组;随后使用启发式搜索方法对完成规划的航班进行机组人员排班;结果如下:A 类数据有(1)206 趟航班均满足机组配置,(2)0 趟航班不满足机组配置,其中(3)机组人员总体乘机次数为 8 次,(4)替补资格使用次数为 0 次。B 类数据有(1)13922 趟航班满足机组配置,(2)32 趟航班不满足机组配置,其中(3)机组人员总体乘机次数为 386 次,(4)替补资格使用次数为 450 次。

针对问题二:在问题一的基础上引进执勤概念,在保证总体执勤成本最低和所有机组人员执勤时长平衡的目标下,仍建立加权 0-1 整数规划模型。首先是航班规划问题,同样可以基于广度优先搜索方法来解决,需要在问题一的基础上引入执勤所需约束条件。而对于机组排班问题,结合实际情况,我们考虑尽可能多的使用工资较低的机组人员进行优先排班,并运用贪心算法来解决该问题。最后计算得到问题二中,A 类数据(1)全部 206 趟航班均满足机组配置,(2)0 趟航班不满足机组配置,其中(3)机组人员总体乘机次数为 8 次,(4)替补资格使用次数为 0 次,(5)机组总体利用率为 82.27\%,(6)总体执勤成本为 55.5866 万元,(7)机组人员执勤时长基本满足均衡。B 类数据有(1)13922 趟航班均满足机组配置,(2)32 趟航班不满足机组配置,其中(3)机组人员总体乘机次数为 386 次,(4)替补资格使用次数为 450 次,(5)机组总体利用率为 72.23\%,(6)总体执勤成本为 3690.84 万元,(7)机组人员执勤时长基本满足均衡。

针对问题三:在前两问的模型进行推广,加入任务环作为约束,在保证总体任务环成本最低和每个机组人员之间的任务环时长尽可能平衡为目标,建立加权 0-1 整数规划模型。基于问题一、二的求解方案,添加相关约束条件进行求解。最后计算求得问题三中,A 类数据(1)206 趟航班均满足机组配置,(2)0 趟航班不满足机组配置,其中(3)机组人员总体乘机次数为 8 次,(4)替补资格使用次数为 0 次,(5)机组总体利用率为 76.06\%,(6)总体执勤成本为 57.8893 万元,(7)执勤时长分布在 1 小时至 11 小时之间,满足均衡性稍差,(8)总任务环成本为 62.7653 万元,(9)任务环在一天、二天、三天、四天的数量分布分别为 12、6、4、10。B 类数据(1)10935 趟航班均满足机组配置,(2)3020 趟航班不满足机组配置,其中(3)机组人员总体乘机次数为 298 次,(4)替补资格使用次数为 450 次,(5)总体利用率为 69.78\%,(6)总体执勤成本为 3827.7 万元,(7)执

勤时长分布在1小时至11小时之间,满足均衡性稍差,(8)总任务环成本为4143.78万元,(9)任务环在一天、二天、三天、四天的数量分布分别为201、389、605、526。

关键词:机组排班;机组任务配对;启发式算法;贪心算法

\section*{目录}
\begin{itemize}
    \item 一、问题重述 \dotfill 4
    \begin{itemize}
        \item 1.1 问题背景 \dotfill 4
        \item 1.2 问题提出 \dotfill 4
    \end{itemize}
    \item 二、问题分析 \dotfill 5
    \begin{itemize}
        \item 2.1 针对问题一 \dotfill 5
        \item 2.2 针对问题二 \dotfill 5
        \item 2.3 针对问题三 \dotfill 5
    \end{itemize}
    \item 三、模型假设 \dotfill 5
    \item 四、符号说明 \dotfill 6
    \item 五、模型建立与求解 \dotfill 6
    \begin{itemize}
        \item 5.1 问题一 \dotfill 6
        \begin{itemize}
            \item 5.1.1 模型建立 \dotfill 7
            \item 5.1.2 模型求解 \dotfill 8
            \item 5.1.3 结果分析 \dotfill 10
        \end{itemize}
        \item 5.2 问题二 \dotfill 13
        \begin{itemize}
            \item 5.2.1 模型建立 \dotfill 13
            \item 5.2.2 模型求解 \dotfill 15
            \item 5.2.3 结果分析 \dotfill 16
        \end{itemize}
        \item 5.3 问题三 \dotfill 18
        \begin{itemize}
            \item 5.3.1 模型建立 \dotfill 18
            \item 5.3.2 模型求解 \dotfill 21
            \item 5.3.3 结果分析 \dotfill 22
        \end{itemize}
    \end{itemize}
    \item 六、模型评价与推广 \dotfill 25
    \begin{itemize}
        \item 6.1 模型评价 \dotfill 25
        \item 6.2 模型推广 \dotfill 25
    \end{itemize}
    \item 七、参考文献 \dotfill 26
    \item 附录 MATLAB 程序 \dotfill 27
\end{itemize}

\section{问题重述}

\subsection{问题背景}

航空公司的机组人员调度问题很复杂,对于航空行业来说,机组成本仅次于燃料成本。因此,即使是在计划表上进行微小改进也能带来显著的经济效益。航空机组排班问题从人员配对和人员分配两方面进行,旨在建立优化模型获取最高利润。

\subsection{问题提出}

1. 建立线性规划模型给 A 套和 B 套航班数据分配机组人员,依编号次序满足目标:①尽可能多的航班满足机组配置;④尽可能少的总体乘机次数;⑦尽可能少使用替补资格。同时给出约束条件如下:每个机组人员初始从基地出发并最终回到基地;每个机组人员的下一航段的起飞机场必须和上一航段的到达机场一致;每个机组人员相邻两个航段之间的连接时间不小于 $MinCT$ 分钟。

2. 引进执勤概念。假定每个机组人员每单位小时执勤成本给定。本问题除了需要满足问题 1 的所有目标外,还需满足如下目标:②机组人员的总体执勤成本最低;⑤机组人员之间的执勤时长尽可能平衡。同时给出约束条件如下:每个机组人员每天至多只能执行一个执勤;每次执勤的飞行时间最多不超过 $MaxBlk$ 分钟;每次执勤的时长最多不超过 $MaxDP$ 分钟;每个机组人员下一执勤的起始机场必须和上一执勤的结束机场一致;每个机组人员相邻两个执勤之间的休息时间不小于 $MinRest$ 分钟。

3. 编制排班计划。假定每个机组人员的每单位小时任务环成本给定(注:不包括执勤成本,可以设想为出差补贴)。本问题除了需要满足问题 1 和 2 的所有目标外,还需满足如下目标:③机组人员的总体任务环成本最低;⑥机组人员之间的任务环时长尽可能平衡。同时给出约束条件如下:每个机组人员每个排班周期的任务环总时长不超过 $MaxTAFB$ 分钟;每个机组人员相邻两个任务环之间至少有 $MinVacDay$ 天休息;每个机组人员连续执勤天数不超过 $MaxSuccOn$ 天。

\section*{二、问题分析}

本题要求根据不同的条件约束对不同的航班人员排班,通过编写程序算法对 A 类和 B 类数据进行验证,最后输出相应计算结果。具体问题分析如下:

\subsection*{2.1 针对问题一}

问题一的主要目标是在保证连续飞行任务之间最短的休息时间来追求尽可能多的满足起飞机组配置的航班。本文将问题一分为两个子问题:航班任务配对问题和机组人员指派问题。其中第一个子问题主要是对任务进行配对,发现一组往返飞行航线并覆盖所有航段;第二个子问题是任务的分配,主要是将已组合完成的机组任务派遣给各组机组人员执行。考虑解决航班配对问题,在满足任务一的条件下寻找从基地出发最终回到基地,并且连接时间不小于 $MinCT$ 分钟的航班;其次,将机组人员进行配对,匹配到规划好的航班;最终得到航班分配的机组人员信息。

\subsection*{2.2 针对问题二}

问题二在问题一的基础上引入执勤概念,同样是将排班分配问题分成两个子问题;首先确定每天航班线路,之后根据每天确定的航班线路来分配机组人员。本文选择与问题一同样的广度优先搜索方法来对满足约束条件的航班进行连接;然后在机组人员分配任务上,问题二需要机组人员总体执勤成本最低并且执勤时长要尽可能平衡,因此本文将贪心算法思想运用于排班问题,尽可能选择在机组人员中每单位小时执勤成本较低的机组人员,最终得到航班分配的机组人员信息。

\subsection*{2.3 针对问题三}

问题三是在问题一、二的基础上进行扩展,只是将问题二生成的一系列执勤进行排列得到任务环,并将多个任务环进行连接组成排班计划。具体解题步骤与问题二基本类似,只是需要添加连续执勤天数不得超过 $MaxSuccOn$ 和相邻任务环之间至少有 $MinVacDay$ 天为约束条件,最终基于问题二的模型求解得到航班分配的机组人员信息。

\section*{三、模型假设}

1. 机组人员只要空闲就可随时上岗,不会无故缺勤;
2. A 套和 B 套数据中的航班信息不会更改;
3. 机组人员之间可以任意组合;
4. 允许存在因为无法满足最低机组资格配置而不能起飞的航班;
5. 不满足最低机组资格配置的航班不能配置任何机组人员;
6. 机组人员可以乘机摆渡,即实际机组配置可以超过最低配置要求,乘机机组人员的航段时间计入执勤时间,但不计入飞行时间。

\section*{四、符号说明}

\begin{tabular}{l l}
\hline
符号 & 符号说明 \\
\hline
$f_{i}$ & 表示航班,索引为 $i$ \\
$x_{j}$ & 表示机组人员,索引为 $j$ \\
$A$ & 表示所有正机长人员集合 \\
$B$ & 表示所有具有副机长资格人员集合 \\
$PT_{i}$ & 表示航班 $f_{i}$ 的飞行时间 \\
$Dptrtime_{i}^{dp}$ & 表示航班 $f_{i}$ 的起飞时刻(日期+时间) \\
$Arrvtime_{i}^{arr}$ & 表示航班 $f_{i}$ 的到达时刻(日期+时间) \\
$BASE_{j}$ & 表示机组人员 $x_{j}$ 的基地 \\
$AirP_{i}^{dp}$ & 表示航班 $f_{i}$ 的出发机场 \\
$AirP_{i}^{arr}$ & 表示航班 $f_{i}$ 的到达机场 \\
$\delta_{i}$ & 0-1 变量,表示航班 $f_{i}$ 是否达到最低机组配置 \\
$cf_{ij}$ & 0-1 变量,表示机组人员 $x_{j}$ 是否分配到航班 $f_{i}$ 上 \\
$captain_{ij}$ & 0-1 变量,表示主资格为正机长的 $x_{j}$ 是否以主资格分配在航班 $f_{i}$ 上 \\
$officer_{ij}$ & 0-1 变量,表示具有副机长资格的 $x_{j}$ 是否以副机长分配在航班 $f_{i}$ 上 \\
$dh_{ij}$ & 0-1 变量,表示机组人员 $x_{j}$ 是否作为乘机人员分配在航班 $f_{i}$ 上 \\
$DZCost_{j}$ & 表示机组成员 $x_{j}$ 单位执勤成本 \\
$TZ_{j}^{duty}$ & 表示机组成员 $x_{j}$ 总的执勤时间 \\
$Date_{i}^{dp}$ & 表示航班 $f_{i}$ 的起飞日期 \\
$Date_{i}^{arr}$ & 表示航班 $f_{i}$ 的到达日期 \\
$PRCost_{j}$ & 表示机组成员 $x_{j}$ 单位小时任务环成本 \\
$T_{j}^{Ring}$ & 表示机组成员 $x_{j}$ 总的任务环时间 \\
$MinCT$ & 表示两个航段之间最小连接时间 \\
$MaxBlk$ & 表示一次执勤飞行的最大时长 \\
$MaxDP$ & 表示执勤的最大时长 \\
$MinRest$ & 表示相邻执勤之间的最短休息时间 \\
$MaxTAFB$ & 表示排班周期单个机组人员最长任务环总时长 \\
$MaxSuccOn$ & 表示连续执勤的最长天数 \\
$MinVacDay$ & 表示相邻两个任务环之间至少休息时间 \\
\hline
\end{tabular}

\section*{五、模型建立与求解}

\subsection{5.1 问题一}

根据题目所述规则,本题需要我们建立一个线性规划模型来给航班分配机组人员,使得尽可能多的航班都能满足起飞条件。

\subsection{5.1.1 模型建立}

针对问题一题目要求,考虑建立加权的 0-1 整数规划模型并为航班分配机组人员,依次使得尽可能多的航班满足机组配置,同时尽可能少的总体乘机次数,并且尽可能少使用替补资格这三个目标,为了更好的通过数学语言描述问题,我们建立如下 0-1 变量:

\begin{equation}
\delta_{i} =
\begin{cases}
1, & \text{若航班 } f_{i} \text{ 满足最低机组配置} \\
0, & \text{否则}
\end{cases}
\tag{1}
\end{equation}

\begin{equation}
cf_{ij} =
\begin{cases}
1, & \text{若机组人员 } x_{j} \text{ 分配在航班 } f_{i} \text{ 上} \\
0, & \text{否则}
\end{cases}
\tag{2}
\end{equation}

\begin{equation}
captain_{ij} =
\begin{cases}
1, & \text{若主资格为正机长的 } x_{j} \text{ 作为正机长分配在航班 } f_{i} \text{ 上} \\
0, & \text{否则}
\end{cases}
\tag{3}
\end{equation}

\begin{equation}
officer_{ij} =
\begin{cases}
1, & \text{若具有副机长资格的 } x_{j} \text{ 作为副机长分配在 } f_{i} \text{ 上} \\
0, & \text{否则}
\end{cases}
\tag{4}
\end{equation}

\begin{equation}
dh_{ij} =
\begin{cases}
1, & \text{若 } x_{j} \text{ 作为乘机人员分配在航班 } f_{i} \text{ 上} \\
0, & \text{否则}
\end{cases}
\tag{5}
\end{equation}

其中,$x_{j}$ 为机组人员,$j=1,2,\dots,n$ 为其下标;$f_{i}$ 表示第 $i$ 个航班,其中 $i=1,2,\dots,m$,并且默认出发时间早的航班下标小于出发时间晚的航班下标;若航班同时出发,则按照附件机组排班次序排序。

① 在目标一中,要使得尽可能多的航班满足机组配置,我们建立如下目标函数:

\begin{equation}
\max \sum_{i=1}^{m} \delta_{i}, \text{ 即等价于 } \min -\sum_{i=1}^{m} \delta_{i}
\tag{6}
\end{equation}

② 同时为了尽可能的减少总体乘机次数(人次),我们有:

\begin{equation}
\min \sum_{i=1}^{m} \sum_{j=1}^{n} dh_{ij}
\tag{7}
\end{equation}

③ 最后一个目标为了尽可能的少使用替补资格(人次),当 $x_{j} \in A \cap B$ 且 $officer_{ij} = 1$,则替补人次加一,则有:

\begin{equation}
\min \sum_{i=1}^{m} \sum_{j \in A \cap B} officer_{ij}
\tag{8}
\end{equation}

综上,根据题目要求建立加权的线性规划模型,得到最终目标函数为:

\begin{equation}
\min -\lambda_{1} \sum_{i=1}^{m} \delta_{i} + \lambda_{2} \sum_{i=1}^{m} \sum_{j=1}^{n} dh_{ij} + \lambda_{3} \sum_{i=1}^{m} \sum_{j \in A \cap B} officer_{ij}
\tag{9}
\end{equation}

其中 $\lambda_{1} > 0$,$\lambda_{2} > 0$,$\lambda_{3} > 0$ 分别表示各个目标函数权重。

进一步,将题目中的非线性约束转换成线性约束,为了便于表达,考虑机组人员 $x_{j}$ 所在航班集合为 $FlightSet_{j} = \{i \mid cf_{ij} \neq 0, i=1,\dots,m\}$。假设集合 $FlightSet_{j}$ 包含 $K_{j}$ 个元素,不妨设航班下标从小到大排序 $j_{1}, j_{2}, \dots, j_{K_{j}}$。根据题目要求,上述目标函数还需有以下约束条件:

(1) 若航班 \( f_i \) 达到最低配置,则 \( \delta_i = 1 \),则至少分配一个正机长和一个副机长,于是我们有如下约束条件:
\begin{equation}
\begin{aligned}
\sum_{j \in A} captain_{ij} &\geq \delta_i \\
\sum_{j \in B} officer_{ij} &\geq \delta_i
\end{aligned}
\tag{10}
\end{equation}

(2) 考虑机组人员 \( x_j \) 能否在航班 \( f_i \) 成为正机长、副机长、或者乘机人员,取决于机务人员 \( x_j \) 是否在航班 \( f_i \),并且只能有一种身份,因此:
\begin{equation}
\begin{aligned}
cf_{ij} &\geq captain_{ij}, \ cf_{ij} \geq officer_{ij}, \ cf_{ij} \geq dh_{ij} \\
cf_{ij} &= captain_{ij} + officer_{ij} + dh_{ij}
\end{aligned}
\tag{11}
\end{equation}

(3) 考虑每个机组人员需要从初始基地出发并最终回到初始基地,即需要机务人员 \( x_j \) 所乘航班初始航班的出发地和 \( x_j \) 基地相同,并且最后机务人员出发航班的到达机场必须和初始基地相同,于是我们可以得到:
\begin{equation}
AirP_{j_1}^{dp} = BASE_j, \ AirP_{j_{K_j}}^{arr} = BASE_j
\tag{12}
\end{equation}

(4) 根据题目要求,机组人员 \( x_j \) 的上一班航段的到达机场必须和下一班航段的起飞机场一致,即
\begin{equation}
AirP_{j_{h+1}}^{dp} = AirP_{j_h}^{arr}, h = 1, 2, 3, \ldots, K_j - 1
\tag{13}
\end{equation}

(5) 同时每个机组人员两个相邻航段之间的连接连接时间不能小于输入 \( MinCT \) 分钟,可得:
\begin{equation}
Dptrtime_{j_{h+1}}^{dp} - Arrvtime_{j_h}^{arr} \geq MinCT, h = 1, 2, 3, \ldots, K_j - 1
\tag{14}
\end{equation}

综上所述,我们可以得到问题一的加权单目标函数模型,如下所示:
\begin{equation}
\begin{aligned}
\min & \ -\lambda_1 \sum_{i=1}^m \delta_i + \lambda_2 \sum_{i=1}^m \sum_{j=1}^n dh_{ij} + \lambda_3 \sum_{i=1}^m \sum_{j \in A \cap B} officer_{ij} \\
s.t. & \begin{cases}
\sum_{j \in A} captain_{ij} \geq \delta_i \\
\sum_{j \in B} officer_{ij} \geq \delta_i \\
captain_{ij} = 0, \text{当 } j \notin A \\
officer_{ij} = 0, \text{当 } j \notin B \\
cf_{ij} \geq captain_{ij}, \ cf_{ij} \geq officer_{ij}, \ cf_{ij} \geq dh_{ij} \\
cf_{ij} = captain_{ij} + officer_{ij} + dh_{ij} \\
AirP_{j_1}^{dp} = BASE_j, \ AirP_{j_{K_j}}^{arr} = BASE_j \\
AirP_{j_{h+1}}^{dp} = AirP_{j_h}^{arr}, h = 1, 2, 3, \ldots, K_j - 1 \\
Dptrtime_{j_{h+1}}^{dp} - Arrvtime_{j_h}^{arr} \geq MinCT, h = 1, 2, 3, \ldots, K_j - 1 \\
\lambda_1 > 0, \ \lambda_2 > 0, \ \lambda_3 > 0
\end{cases}
\end{aligned}
\tag{15}
\end{equation}

\subsection{5.1.2 模型求解}

\textbf{步骤 1:数据预处理}

根据题目所给数据,首先使用 Excel 表格对 A 套数据所有航班信息按照起飞时间进行

\begin{table}
\centering
\caption{8月11日数据预处理结果}
\begin{tabular}{c c c c c c c c}
\hline
出发 & 出发 & 到达 & 到达 & 休息 & 出发 & 出发 & 到达 \\
时间 & 机场 & 时间 & 机场 & 时间 & 时间 & 机场 & 时间 \\
\hline
8:00 & NKX & 9:30 & PGX & 40 & 10:10 & PGX & 11:40 \\
11:30 & NKX & 13:50 & XGS & 40 & 14:30 & XGS & 16:50 \\
12:20 & NKX & 14:05 & PDK & 45 & 14:50 & PDK & 16:40 \\
\hline
\end{tabular}
\end{table}

\section*{步骤2:数据分析}

将A套数据中8月11日、12日以及15日需要摆渡的乘机人员的当天所有飞行安排提取出来,以日期为纵坐标,飞机起飞降落时间为横坐标,航班号为标签绘制甘特图如图5.1所示:

\begin{figure}[h]
\centering
\includegraphics[width=\textwidth]{image.png}
\caption{飞机日程表}
\end{figure}

从上图可以发现,绘制的3天航班信息里每架飞机每天执行的飞行任务基本一致,同样的飞行任务消耗的时间也是一样的,但每天会有部分航班变动。同时每天的航班大部分都存在往返,最终所有的机组人员都可以回到基地。而对照B类数据可以发现,每天同样包含很多重复航班,航班信息一致(起飞时间、起飞机场、到达时间、到达机场等),所以我们将所有重复航班进行整理,单独给一天航班信息进行机组排班,剩余天数的机组排班与这一天一致,这样可以减少算法计算的复杂性。结合步骤一数据预处理后的结果,同样可以发现A套数据在8月11、12、15日出现了特殊情况,主要原因是前期只考虑执行

飞行任务的机组,导致后续将有航班无法满足最低机组资格配置,所以需要针对此问题考虑在合适的航班里添加乘机机组来执行后续飞行任务。

\section*{步骤 3:数据求解}

针对问题一的约束条件,建立加权的 0-1 整数规划模型,将问题一分为两个子问题,第一个子问题需要解决航班规划问题,第二个子问题则需要解决机组人员排班问题。本文基于启发式算法,构造了求解两个子问题的方法。

\textbf{方法思想:} 对于求解航班规划问题,根据问题一所要求的全部约束条件(如相邻两个航班之间的连接时间需要大于 40 分钟),对每天当中每个航班进行线路规划,确定一天当中的每一条航线。首先循环遍历每天的所有航班信息,在每一天中,搜索航班起飞机场、起飞时间以及到达机场、到达时间,对于同一天中起飞机场和到达机场航班相同,并且到达与起飞时间的差值大于等于 40 分钟的航班,保存到完成航线集合中。在航线确定后,解决机组人员排班问题,首先确定机组人员需要摆渡的航班,其次考虑从实际意义出发,根据正副机长搭配对每个航班线路进行排班,确保尽可能多的航班满足机组配置,同时尽可能的减少乘机次数以及使用替补资格的次数。

对航班线路进行优化,减少连接时间的同时,满足航段之间最小连接时间 $MinCT=40$ 分钟,有如下算法:

\begin{enumerate}
    \item 第一步,将所有航班按出发日期为第一次序、出发时间为第二次序、出发机场为第三次序进行排序,并将所有航班放入到航班集合 $fli$ 中。
    \item 第二步,运用广度优先搜索算法,逐步搜索出航班路线。
    \begin{enumerate}
        \item 从航班集合 $fli$ 中搜索出发机场为基地的航班,并把这些航班从航班集合 $fli$ 中剔除、放入航班集合 $F_{Base}^{dp}$ 中,作为下一层搜索的条件,并将这些航班记为第一层航班。
        \item 再从航班集合 $fli$ 搜索出发机场与航班集合 $F_{Base}^{dp}$ 中航班的到达机场相同的航班,将满足最短连接时间约束的航班与第一层航班连接,并将这些航班记为第二层航班。将已连接的第一层航班从航班集合 $F_{Base}^{dp}$ 中除去,将已连接的第二层航班从集合 $fli$ 中除去、并加入到航班集合 $F_{Base}^{dp}$ 中。若第二层航班的到达机场为基地,则该条路线为一条完整的执勤路线,将其从航班集合 $F_{Base}^{dp}$ 中除去,保存到执勤路线中。
        \item 重复上述搜索,直到航班集合 $fli$ 中没有满足连接条件的航班为止。计算出的执勤路线中包含的航班为满足机组配置的航班。
        \item 将未满足机组配置的航班,再判断是否可以通过机组人员乘机来满足。在执勤路线中搜索包含未满足航班的出发机场的路线、再搜索包含未满足航班的到达机场的路线,将两条路线与该航班拼接为一条执勤路线,若该执勤路线时间在一天之内,则该条路线满足要求,保存到执勤路线中。
        \item 剩下的航班为不满足机组配置的航班。
        \item 最后,进行机组人员配置。将机组人员尽可能均等地分为两组,分别为正机长组和副机长组。首先按执勤路线开始时间顺序排列,正机长和副机长按顺序循环排列给每一条执勤路线。
    \end{enumerate}
\end{enumerate}

\subsection*{5.1.3 结果分析}

\subsubsection{不满足机组配置航班数}

根据上述方法计算表明,A 类数据最终所有航班均可满足最低机组配置,而 B 类数据有 32 架航班不满足最低机组配置,具体取消航班信息如下表所示:

\begin{table}[h]
\centering
\caption{B 类数据未满足最低机组需求的航班信息}
\begin{tabular}{c}
\hline
\end{tabular}
\end{table}

\begin{tabular}{llllllll}
航班号 & 出发日期 & 出发时间 & 出发机场 & 到达日期 & 到达时间 & 到达机场 & 机组安排 \\
\hline
FB8567 & 8/1/2019 & 0:35 & XXJ & 8/1/2019 & 2:25 & SXA & C1F1 \\
FB240 & 8/1/2019 & 7:00 & FBX & 8/1/2019 & 8:40 & TGD & C1F1 \\
FB292 & 8/1/2019 & 7:05 & OOJ & 8/1/2019 & 8:20 & TGD & C1F1 \\
FB312 & 8/1/2019 & 7:10 & OSY & 8/1/2019 & 8:15 & TGD & C1F1 \\
FB394 & 8/1/2019 & 7:10 & UYS & 8/1/2019 & 8:15 & TGD & C1F1 \\
FB652 & 8/1/2019 & 7:10 & SXJ & 8/1/2019 & 8:25 & TGD & C1F1 \\
FB382 & 8/1/2019 & 7:20 & YKJ & 8/1/2019 & 8:40 & TGD & C1F1 \\
FB582 & 8/1/2019 & 7:25 & MYJ & 8/1/2019 & 8:55 & TGD & C1F1 \\
FB432 & 8/1/2019 & 7:30 & FBX & 8/1/2019 & 8:45 & GKS & C1F1 \\
FB50 & 8/1/2019 & 7:30 & NOU & 8/1/2019 & 9:05 & TGD & C1F1 \\
FB402 & 8/1/2019 & 7:30 & AXO & 8/1/2019 & 8:35 & TGD & C1F1 \\
FB522 & 8/1/2019 & 7:35 & XMJ & 8/1/2019 & 8:40 & HOM & C1F1 \\
FB532 & 8/1/2019 & 7:35 & OAX & 8/1/2019 & 8:55 & TGD & C1F1 \\
FB562 & 8/1/2019 & 7:35 & XNZ & 8/1/2019 & 8:55 & TGD & C1F1 \\
FB632 & 8/1/2019 & 7:35 & HWX & 8/1/2019 & 9:10 & TGD & C1F1 \\
FB672 & 8/1/2019 & 7:35 & THJ & 8/1/2019 & 9:00 & TGD & C1F1 \\
FB792 & 8/1/2019 & 7:35 & SHO & 8/1/2019 & 9:10 & TGD & C1F1 \\
FB492 & 8/1/2019 & 7:40 & GKS & 8/1/2019 & 8:55 & TGD & C1F1 \\
FB692 & 8/1/2019 & 7:40 & BCJ & 8/1/2019 & 9:20 & TGD & C1F1 \\
FB732 & 8/1/2019 & 7:50 & UDJ & 8/1/2019 & 9:10 & HOM & C1F1 \\
FB1702 & 8/1/2019 & 8:00 & FBX & 8/1/2019 & 9:05 & HOM & C1F1 \\
FB1271 & 8/1/2019 & 8:00 & THJ & 8/1/2019 & 10:00 & NOU & C1F1 \\
FB242 & 8/1/2019 & 8:00 & FBX & 8/1/2019 & 9:45 & TGD & C1F1 \\
FB620 & 8/1/2019 & 8:00 & XSJ & 8/1/2019 & 9:45 & TGD & C1F1 \\
FB602 & 8/1/2019 & 8:05 & XMH & 8/1/2019 & 9:45 & TGD & C1F1 \\
FB662 & 8/1/2019 & 8:15 & G KU & 8/1/2019 & 10:05 & TGD & C1F1 \\
FB642 & 8/1/2019 & 8:25 & XMJ & 8/1/2019 & 10:05 & TGD & C1F1 \\
FB52 & 8/1/2019 & 8:30 & NOU & 8/1/2019 & 10:05 & TGD & C1F1 \\
FB301 & 8/1/2019 & 8:40 & GKS & 8/1/2019 & 10:55 & SXA & C1F1 \\
FB1203 & 8/1/2019 & 8:45 & FBX & 8/1/2019 & 10:30 & SXA & C1F1 \\
FB1892 & 8/1/2019 & 9:00 & TAN & 8/1/2019 & 9:55 & TGD & C1F1 \\
FB1146 & 8/1/2019 & 11:30 & XBT & 8/1/2019 & 13:30 & HOM & C1F1 \\
\end{tabular}

从上述数据我们可以发现,B类数据里不满足起飞条件的航班都集中在8月1号,它们的起飞地点都是基地以外的地区,结合B类所有数据分析可以发现,在上述航班起飞时间之前,并没有从基地出发的航班到达该地区,所以可以验证我们的模型算法能够计算出准确的航班信息。

\section*{2. 机组人员总体乘机次数}

利用上述算法,我们可以得到问题一中A类数据中需要摆渡的航班有4个,B类数据中需要摆渡的航班有193个,而每个航班需要有两名机组人员才能起飞,所以最终的乘机次数需要乘以2,具体需要摆渡的航班信息如下表所示:

表5.3 A类所有需要摆渡的航班信息

\begin{tabular}{lllllll}
FA681 & 8/12/2021 & 10:10 & PGX & 8/12/2021 & 11:40 & NKX \\
\end{tabular}

\begin{table}
\centering
\begin{tabular}{l l l l l l l}
FA3 & 8/12/2021 & 10:25 & PGX & 8/12/2021 & 11:40 & NKX \\
 & & & & & & C1F1 \\
FA891 & 8/12/2021 & 10:30 & XGS & 8/12/2021 & 12:50 & NKX \\
 & & & & & & C1F1 \\
FA891 & 8/15/2021 & 10:30 & XGS & 8/15/2021 & 12:50 & NKX \\
 & & & & & & C1F1 \\
\end{tabular}
\end{table}

\begin{table}
\centering
\begin{tabular}{l l l l l l l}
\multicolumn{7}{c}{表5.4 B类部分需要摆渡的航班信息} \\
FB8560 & 8/1/2019 & 6:35 & SXA & 8/1/2019 & 8:30 & HOM \\
 & & & & & & C1F1 \\
FB1200 & 8/1/2019 & 7:05 & SXA & 8/1/2019 & 8:40 & FBX \\
 & & & & & & C1F1 \\
FB460 & 8/1/2019 & 8:00 & SXA & 8/1/2019 & 10:25 & TGD \\
 & & & & & & C1F1 \\
FB434 & 8/1/2019 & 9:10 & FBX & 8/1/2019 & 10:25 & GKS \\
 & & & & & & C1F1 \\
FB1606 & 8/1/2019 & 11:45 & XNZ & 8/1/2019 & 12:30 & HOM \\
 & & & & & & C1F1 \\
FB982 & 8/1/2019 & 12:15 & TUK & 8/1/2019 & 14:00 & TGD \\
 & & & & & & C1F1 \\
FB1644 & 8/1/2019 & 15:50 & MYJ & 8/1/2019 & 16:40 & HOM \\
 & & & & & & C1F1 \\
FB770 & 8/1/2019 & 16:50 & SXA & 8/1/2019 & 18:55 & HOM \\
 & & & & & & C1F1 \\
FB72 & 8/1/2019 & 17:00 & NOU & 8/1/2019 & 18:35 & TGD \\
 & & & & & & C1F1 \\
FB264 & 8/1/2019 & 17:35 & FBX & 8/1/2019 & 19:25 & TGD \\
 & & & & & & C1F1 \\
FB700 & 8/1/2019 & 17:40 & BCJ & 8/1/2019 & 19:20 & TGD \\
 & & & & & & C1F1 \\
FB1104 & 8/1/2019 & 17:40 & HWJ & 8/1/2019 & 19:20 & TGD \\
 & & & & & & C1F1 \\
FB780 & 8/1/2019 & 17:55 & NOU & 8/1/2019 & 19:45 & HOM \\
 & & & & & & C1F1 \\
FB308 & 8/1/2019 & 18:00 & SXA & 8/1/2019 & 20:10 & GKS \\
 & & & & & & C1F1 \\
FB474 & 8/1/2019 & 18:05 & SXA & 8/1/2019 & 20:35 & TGD \\
 & & & & & & C1F1 \\
\end{tabular}
\end{table}

\begin{figure}[h]
\centering
\includegraphics[width=0.4\textwidth]{image1.png}
\caption{FA884}
\end{figure}

\begin{figure}[h]
\centering
\includegraphics[width=0.4\textwidth]{image2.png}
\caption{FA891}
\end{figure}

\begin{table}
\centering
\begin{tabular}{l l l}
\multicolumn{3}{c}{表5.5 问题一结果汇总} \\
结果指标 & A类数据 & B类数据 \\
不满足机组配置航班数 & 0 & 32 \\
 & 12 & \\
\end{tabular}
\end{table}

\begin{tabular}{l c c}
\hline
满足机组配置航班数 & 206 & 13922 \\
机组人员总体乘机次数 & 8 & 386 \\
替补资格使用次数 & 0 & 450 \\
程序运行分钟数 & 0.002 分钟 & 0.25 分钟 \\
\hline
\end{tabular}

从上述数据结果我们可以发现,随着航班数据的大幅度增多,排班难度也随之增加。A 类数据在经过 8 次乘机后,所有航班都能满足最低机组配置要求;B 类数据在经过 286 次乘机后,还是出现 32 次航班不满足最低机组配置要求。

\subsection{5.2 问题二}

\subsubsection{5.2.1 模型建立}

相比问题一,问题二加入了执勤概念,并且要求机组人员的总执勤成本最低,每个机组人员之间的执勤时长要尽可能平衡。所以在问题一的基础上添加了机组人员的单位执勤成本 $DZCost_{j}$ 和总的执勤时间 $TZ_{j}^{duty}$。

对于问题二新增的两个目标函数建立如下:

① 机组人员的总体执勤成本最低,考虑人员 $x_{j}$ 的成本为 $DZCost_{j} \times TZ_{j}^{duty}$,建立目标函数如下:
\begin{equation}
\min \sum_{j=1}^{n} DZCost_{j} \times TZ_{j}^{duty}
\tag{16}
\end{equation}

② 机组人员之间的执勤时长要尽可能平衡,建立目标函数如下:
\begin{equation}
\min \{ \max \{ TZ_{j}^{duty} \} - \min \{ TZ_{j}^{duty} \} \}
\tag{17}
\end{equation}

其中上述函数可以通过增加约束的方法改写成线性问题:
\begin{equation}
\min \{ t_{\max} - t_{\min} \}
\tag{18}
\end{equation}
\begin{equation}
s.t. \begin{cases}
TZ_{j}^{duty} \leq t_{\max}, j=1,2,\ldots,n \\
TZ_{j}^{duty} \geq t_{\min}, j=1,2,\ldots,n
\end{cases}
\end{equation}

综上,问题二的总目标函数可写为:
\begin{equation}
\begin{aligned}
\min & -\alpha_{1} \sum_{i=1}^{m} \delta_{i} + \alpha_{2} \sum_{i=1}^{m} \sum_{j=1}^{n} dh_{ij} + \alpha_{3} \sum_{i=1}^{m} \sum_{j \in A \cap B} officer_{ij} + \alpha_{4} \sum_{j=1}^{n} DZCost_{j} \times TZ_{j}^{duty} \\
& + \alpha_{5} (t_{\max} - t_{\min})
\end{aligned}
\tag{19}
\end{equation}

为了进一步便于表达,我们在上一问假设机组人员 $x_{j}$ 所在航班集合为 $FlightSet_{j} = \{ i | cf_{ij} \neq 0, i=1,\ldots,m \}$。假设集合 $FlightSet_{j}$ 包含 $K_{j}$ 个元素。不妨设航班下标从小到大排序 $j_{1}, j_{2}, \ldots, j_{K_{j}}$。由题目知,同一天起飞的航班为同一次执勤,并且这种划分是唯一的,用集合 $\{ j_{1}, j_{2}, \ldots, j_{duty_{1}} \}$ 表示第一次执勤所分配的航班集合,以此类推。
\begin{equation}
\{ j_{1}, j_{2}, \ldots, j_{duty_{1}} \}, \{ j_{(duty_{1}+1)}, j_{(duty_{1}+2)}, \ldots, j_{duty_{2}} \}, \ldots, \{ j_{(duty_{(s_{j}-1)}+1)}, j_{(duty_{(s_{j}-1)}+2)}, \ldots, j_{duty_{s_{j}}} \}
\tag{20}
\end{equation}

显然序列 $1, 2, \ldots, duty_{1}, duty_{1}+1, duty_{1}+2, \ldots, duty_{2}, \ldots, duty_{(s_{j}-1)}+1, duty_{(s_{j}-1)}+2, \ldots, duty_{s_{j}}$ 和序列 $1, 2, 3, 4, \ldots, K_{j}-1, K_{j}$ 相同。

此外,问题二在满足问题一所有的约束条件外,上述目标函数还有以下约束条件:

① 每个机组人员一次执勤起始时间从当天所执行的第一趟航班的起飞时间算起,结束时间按最后一趟航班的到达时间计算,即

\begin{equation}
TZ_{j}^{duty} = (Arrvtime_{j_{duty_{1}}}^{arr} - Dptrtime_{j_{1}}^{dp}) + (Arrvtime_{j_{duty_{2}}}^{arr} - Dptrtime_{j_{(duty_{1}+1)}}^{dp}) + \ldots
\end{equation}
\begin{equation}
+ (Arrvtime_{j_{duty_{s}}}^{arr} - Dptrtime_{j_{(duty_{(s-1)}+1)}}^{dp})
\end{equation}

② 每个机组人员每天至多只能执行一个执勤;即 24 小时内只能完整的执行一个执勤,也就是相邻的两个执勤之间的跨度时间需要超过 24 小时,即
\begin{equation}
Arrvtime_{j_{duty_{2}}}^{arr} - Dptrtime_{j_{1}}^{dp} > 24h,
\end{equation}
\begin{equation}
Arrvtime_{j_{duty_{3}}}^{arr} - Dptrtime_{j_{(duty_{1}+1)}}^{dp} > 24h, \ldots, Arrvtime_{j_{duty_{s_{j}}}}^{arr} - Dptrtime_{j_{(duty_{(s_{j}-2)}+1)}}^{dp} > 24h
\end{equation}

③ 每次执勤的飞行时间不得超过 $MaxBlk$ 分钟,即
\begin{equation}
(1-dh_{j_{1},j})PT_{j_{1}} + (1-dh_{j_{2},j})PT_{j_{2}} + \ldots + (1-dh_{j_{duty_{1},j}})PT_{j_{duty_{1}}} \leq MaxBlk
\end{equation}
\begin{equation}
(1-dh_{j_{(duty_{(s_{j}-1)}+1)},j})PT_{j_{(duty_{(s_{j}-1)}+1)}} + (1-dh_{j_{(duty_{(s_{j}-1)}+2),j}})PT_{j_{(duty_{s_{j}-1}+2)}} + \ldots + (1-dh_{j_{duty_{s_{j}},j}})PT_{j_{duty_{s_{j}}}} \leq MaxBlk
\end{equation}

④ 每次执勤时长不得超过 $MaxDP$,每个机组成员 $x_{j}$ 在第 h 执勤的第一个出发航班下标为 $j_{duty_{(h-1)+1}}$,最后出发的航班下标为 $j_{duty_{h}}$,最后执勤时长为
\begin{equation}
Arrvtime_{j_{duty_{1}}}^{arr} - Dptrtime_{j_{1}}^{dp} \leq MaxDP,
\end{equation}
\begin{equation}
Arrvtime_{j_{duty_{2}}}^{arr} - Dptrtime_{j_{(duty_{1}+1)}}^{dp} \leq MaxDP,
\end{equation}
\begin{equation}
\ldots
\end{equation}
\begin{equation}
Arrvtime_{j_{duty_{s_{j}}}}^{arr} - Dptrtime_{j_{(duty_{(s_{j}-1)}+1)}}^{dp} \leq MaxDP,
\end{equation}

⑤ 每个机组人员下一次执勤的起始机场必须和上一执勤的结束机场一致;由问题一的约束条件可知,该约束显然成立。

⑥ 每个机组人员的相邻两个执勤之间的休息时间大于等于 $MinRest$ 分钟,即
\begin{equation}
Dptrtime_{j_{(duty_{1}+1)}}^{dp} - Arrvtime_{j_{duty_{1}}}^{arr} \geq MinRest,
\end{equation}
\begin{equation}
Dptrtime_{j_{(duty_{2}+1)}}^{dp} - Arrvtime_{j_{duty_{2}}}^{arr} \geq MinRest,
\end{equation}
\begin{equation}
\ldots
\end{equation}
\begin{equation}
Dptrtime_{j_{(duty_{(s_{j}-1)}+1)}}^{dp} - Arrvtime_{j_{duty_{(s_{j}-1)}}}^{arr} \geq MinRest,
\end{equation}

据此,我们得到问题二的数学模型如下:

\begin{equation}
\begin{aligned}
\min & -\alpha_{1} \sum_{i=1}^{m} \delta_{i} + \alpha_{2} \sum_{i=1}^{m} \sum_{j=1}^{n} d h_{i j} + \alpha_{3} \sum_{i=1}^{m} \sum_{j \in A \cap B} officer_{i j} + \alpha_{4} \sum_{j=1}^{n} D Z C o s t_{j} \times T_{j}^{d u t y} \\
& + \alpha_{5}\left(t_{\max }-t_{\min }\right)
\end{aligned}
\end{equation}

\begin{equation}
\begin{aligned}
& TZ_{j}^{d u t y}=\sum_{k=1}^{s_{j}}\left(Arrvtime_{J_{d u t y_{k}}}^{a r r}-Dptrtime_{J_{(d u t y(k-1)+1)}}^{d p}\right), \text { 若 } k=1, j_{d u t y_{0}+1}=j_{1} \\
& Arrvtime_{J_{d u t y_{2}}}^{a r r}-Dptrtime_{j_{1}}^{d p}>24 h, \ldots, Arrvtime_{J_{d u t y_{s_{j}}}}^{a r r}-Dptrtime_{J_{(d u t y(s_{j}-2)+1)}}^{d p}>24 h \\
& \sum_{k=1}^{d u t y_{1}}\left(1-d h_{j_{k}, j}\right) P T_{j_{k}} \leq MaxBlk, \ldots, \sum_{k=d u t y_{(s_{j}-1)+1}}^{d u t y_{s_{j}}}\left(1-d h_{j_{k}, j}\right) P T_{j_{k}} \leq MaxBlk \\
& Arrvtime_{J_{d u t y_{1}}}^{a r r}-Dptrtime_{j_{1}}^{d p} \leq MaxDP, \ldots, Arrvtime_{J_{d u t y_{s_{j}}}}^{a r r}-Dptrtime_{J_{(d u t y(s_{j}-1)+1)}}^{d p} \leq MaxDP \\
& Dptrtime_{J_{(d u t y_{1}+1)}}^{d p}-Arrvtime_{J_{d u t y_{1}}}^{a r r} \geq MinRest, \ldots, Dptrtime_{J_{(d u t y(s_{j}-1)+1)}}^{d p}-Arrvtime_{J_{d u t y(s_{j}-1)}}^{a r r} \geq MinRest \\
& \sum_{j \in A} captain_{i j} \geq \delta_{i} \\
& \sum_{j \in B} officer_{i j} \geq \delta_{i} \\
& captain_{i j}=0, \text { 当 } j \notin A \\
& officer_{i j}=0, \text { 当 } j \notin B \\
& c f_{i j} \geq captain_{i j}, \quad c f_{i j} \geq officer_{i j}, \quad c f_{i j} \geq d h_{i j} \\
& c f_{i j}=captain_{i j}+officer_{i j}+d h_{i j} \\
& AirP_{j_{1}}^{d p}=BASE_{j}, \quad AirP_{j_{K_{j}}}^{a r r}=BASE_{j} \\
& AirP_{j_{h+1}}^{d p}=AirP_{j_{h}}^{a r r}, h=1,2,3, \ldots, K_{j}-1 \\
& Dptrtime_{j_{h+1}}^{d p}-Arrvtime_{j_{h}}^{a r r} \geq \text { MinCT }, h=1,2,3, \ldots, K_{j}-1 \\
& \lambda_{1}>0, \lambda_{2}>0, \lambda_{3}>0
\end{aligned}
\tag{26}
\end{equation}

\subsection{5.2.2 模型求解}

问题二与问题一的目标都是保证航班尽可能多的满足最低机组配置,所不同的是问题二加入了执勤概念,每个机组人员需要在满足执勤约束条件下追求总体执勤成本最低和执勤时长尽可能平衡。

根据问题二约束条件要求,其航班规划与问题一的约束条件基本一致,主要是起飞机场、起飞时间以及到达机场、到达时间的规划问题。所以其求解思想与问题一是一致的,首先是循环遍历每天的航班信息,将每天重复的航班排列在另外一个数据表,对其进行单独排序可以减少算法复杂性,随后在每一天中,搜索航班起飞机场、起飞时间以及到达机场、到达时间,对于同一天中起飞机场和到达机场相同,并且到达时间与起飞时间的差值大于等于 40 分钟的航班,最后保存到完成航线集合中。基于此,结合问题一的排班方法,我们可以求得问题二的方法如下:

\textbf{第一步:确定航班规划}

根据问题一思路确定航班规划

\textbf{第二步:组合执勤}

(1) 将航班集合 \( F \) 按日期划分到各个组,从第一天开始,一直到最后一天,按天组合航班,订制执勤计划。

(2) 从第一天的航班中,选择一趟出发机场为基地的航班,作为一次执勤的第一趟航班,再从剩余航班集合 \( F \) 中,选择一趟出发机场与第一趟航班到达机场相同的、且连接时间大于 \( MinCT \) 的航班,与该航班连接。

(3) 重复上述连接操作,若到达机场为出发基地,则构成一次执勤,保存到执勤集合 \( W \) 中。若无可连接航班、且到机场不为出发基地,则不构成一次执勤。

(4) 重复上述操作组合执勤,直到没有可以组成的一次执勤的航班为止。

### 第三步:确定机组分配

在航线确定之后,开始对机组人员进行排班问题求解,问题二是在问题一基础上引入执勤,添加了执勤所需的约束条件。确定机组分配时首先是确定需要摆渡的航班,根据正副机长搭配对每个航班线路进行排班,确保每个机组人员每天只能执行一个执勤,在问题一的目标下,尽可能让机组人员的总体执勤成本最低,且每个机组人员的执勤时长要平衡。所以本文采用贪心算法来求解最低执勤成本,按照工资最低的机组人员优先排班原则进行求解,具体实现过程如下:

1. 首先每个机组人员每天至多只能执行一个执勤,分析 A 类、B 类数据可以发现,并无跨夜的航班任务,所以该条件是无需作任何约束;同时不需要约束的还有上一执勤的机场必须和下一执勤的起始机场一致,因为该条件在航班规划已经解决。其次,每个机组人员每次执勤的飞行时间最多不超过 10 个小时,我们根据机组人员所负责航班的起飞、降落时间来计算飞行时间,添加 10 个小时的最大限制条件。然后规定每次执勤时长最多为 12 小时,该部分约束只需在飞行时间的基础上加上每个航班的间隔连接时间。最后是每个机组人员的相邻执勤之间休息时间需要大于 11 小时,上一天最后一个航班的降落时间与第二天第一个航班的起飞时间之间的间隔要在 11 小时以上。

2. 将机组排班问题分解为若干小问题,如求解机组人员的执勤成本问题和机组人员的执勤时间问题,同时还包括尽可能多的航班能满足最低机组配置,求解过程中,按照工资最低的机组人员优先选择的原则进行排班。

3. 对 2) 每一个子问题进行求解,得到子问题的局部最优解。

4. 将 3) 的局部最优解组合成问题二最终所需的机组排班解。

#### 5.2.3 结果分析

1. **不满足机组配置航班数**

根据上述方法计算表明,问题二中不满足机组配置航班数与问题一一致,具体的航班信息也是一样,所以在本问不作展示。

2. **机组人员总体乘机次数**

利用上述方法,我们最终求得问题二 A 类数据中需要摆渡的航班有 4 个,与问题一一致,B 类数据中需要摆渡的航班有 193 个,而每个航班需要有两名机组人员才能起飞,所以最终的乘机次数需要乘以 2,具体需要摆渡的航班信息如下表所示:

\begin{table}[h]
\centering
\caption{B 类部分需要摆渡的航班信息}
\begin{tabular}{c c c c c c c}
\hline
FB298 & 8/1/2019 & 18:30 & OOJ & 8/1/2019 & 19:55 & TGD \\
FB380 & 8/1/2019 & 18:35 & NOU & 8/1/2019 & 20:30 & SXJ \\
FB458 & 8/1/2019 & 18:35 & TUK & 8/1/2019 & 20:20 & TGD \\
FB647 & 8/5/2019 & 18:00 & TGD & 8/5/2019 & 19:45 & XMJ \\
\hline
\end{tabular}
\end{table}

\begin{tabular}{lllllll}
FB75 & 8/5/2019 & 18:00 & TGD & 8/5/2019 & 19:35 & NOU \\
FB285 & 8/9/2019 & 16:40 & TGD & 8/9/2019 & 17:55 & OXU \\
FB1647 & 8/9/2019 & 17:50 & HOM & 8/9/2019 & 18:40 & MYJ \\
FB75 & 8/10/2019 & 18:00 & TGD & 8/10/2019 & 19:35 & NOU \\
FB537 & 8/10/2019 & 17:30 & TGD & 8/10/2019 & 18:45 & OAX \\
FB1647 & 8/13/2019 & 17:50 & HOM & 8/13/2019 & 18:40 & MYJ \\
FB1647 & 8/14/2019 & 17:50 & HOM & 8/14/2019 & 18:40 & MYJ \\
FB1647 & 8/17/2019 & 17:50 & HOM & 8/17/2019 & 18:40 & MYJ \\
FB1097 & 8/19/2019 & 17:15 & TGD & 8/19/2019 & 19:45 & SXA \\
FB267 & 8/19/2019 & 18:00 & TGD & 8/19/2019 & 20:00 & FBX \\
FB267 & 8/21/2019 & 18:00 & TGD & 8/21/2019 & 20:00 & FBX \\
\end{tabular}

\section*{3. 机组总体利用率}

机组总体利用率为最低配置总体飞行时长除以总体执勤时长,其中最低配置总体飞行时长为总体执勤时长减去所有航段之间的连接时长和所有乘机人员的乘机时长,最终根据上述方法解答,得到结果如下表所示:

\textbf{表5.7 机组总体利用率}

\begin{tabular}{lll}
结果指标(注) & A类数据 & B类数据 \\
最低配置总体飞行时长 & 44090分钟 & 2634700分钟 \\
总体执勤时长 & 53590分钟 & 3647654分钟 \\
机组总体利用率 & 82.27\% & 72.23\% \\
\end{tabular}

\section*{4. 总体执勤成本}

每个机组人员一天只能完成一个执勤,因此一天的执勤时长为执勤中最后一趟航班的到达时刻减去第一趟航班的出发时刻。再将机组人员每天的执勤时间加和,得到总的执勤时长。每个机组人员的执勤成本,通过总执勤时长乘每单位小时执勤成本得到。最后,将每个机组人员的执勤成本求和,得到总体执勤成本,通过计算得到,A、B类数据总体执勤成本如下表所示:

\textbf{表5.8 总体执勤成本}

\begin{tabular}{lll}
结果指标(注) & A类数据 & B类数据 \\
总体执勤成本(万元) & 55.5866 & 3690.84 \\
\end{tabular}

\section*{5. 执勤时长平衡性}

在衡量机组人员的执勤时长平衡性时,通过比较他们一次执勤的最大、最小以及平均执勤时长,还有每个机组人员的执勤天数、以及每个机组人员在整个排版周期的执勤总时长的均方差来进行比较,通过计算得到,A、B组数据平衡性结果如下表所示,A类机组人员的执勤时长柱状图如图3所示:

\textbf{表5.9 数据平衡性}

\begin{tabular}{lll}
结果指标(注) & A类数据 & B类数据 \\
执勤时长均方差 & 676.63 & 526.28 \\
最小/平均/最大一次执勤飞行时间 & 1小时30分/3小时26分/4小时40分 & 1小时15分/4小时32分/5小时42分 \\
最小/平均/最大一次执勤时 & 1小时30分/4小时15分/5 & 1小时55分/5小时15分/8 \\
\end{tabular}

\begin{table}
\centering
\begin{tabular}{l l l}
\hline
结果指标(注) & A类数据 & B类数据 \\
\hline
不满足机组配置航班数 & 0 & 32 \\
满足机组航班配置数 & 206 & 13922 \\
机组人员总体乘机次数 & 8 & 386 \\
替补资格使用次数 & 0 & 450 \\
机组总体利用率 & 82.27\% & 72.23\% \\
最小/平均/最大一次执勤飞行时间 & 1小时05分/3小时26分/4小时40分 & 1小时15分/4小时32分/5小时42分 \\
最小/平均/最大一次执勤时长 & 2小时55分/4小时15分/5小时20分 & 1小时55分/5小时15分/8小时20分 \\
最小/平均/最大机组人员执勤天数 & 3/10/15 & 2/12/19 \\
总体执勤成本(万元) & 55.5866 & 3690.84 \\
程序运行分钟数 & 0.05分钟 & 0.4分钟 \\
\hline
\end{tabular}
\caption{问题二结果汇总}
\end{table}

通过上述数据处理可以发现,A类航班数量较少,在经过8次乘机后所有航班均可满足最低机组要求,且最终的机组利用率高达82.27\%,总体执勤成本为55.5866万元,同时A类最终的机组排班平衡性较好;随着航班数量的大幅度上升,B类数据出现了32次航班不满足最低机组人员配置要求,且机组总体利用率也下降至72.23\%,最终的总体执勤成本为3690.84万元。

\subsection*{5.3 问题三}

\subsubsection{5.3.1 模型建立}

问题三在前两个问题的基础上,添加了任务环作为考虑因素,目标函数由原先的执勤

成本最低,改成总体任务环成本最低,同时也要求机组人员之间的任务环时长尽可能平衡。据此,在原先假设的基础上,问题三的两个目标函数定义如下:

① 机组人员的总体任务环成本最低:
\begin{equation}
\min \sum_{j=1}^{n} PRCost_{j} \times T_{j}^{Ring}
\tag{27}
\end{equation}

② 机组人员之间的任务环时长尽可能平衡:
\begin{equation}
\min \{\max \{T_{j}^{Ring}\} - \min \{T_{j}^{Ring}\}\}
\tag{28}
\end{equation}

其中上述函数可以通过增加约束的方法改写成线性问题
\begin{equation}
\min \{l_{\max} - l_{\min}\}
\tag{29}
\end{equation}
\begin{equation}
s.t.
\begin{cases}
T_{j}^{duty} \leq l_{\max}, j = 1, 2, \ldots, n \\
T_{j}^{duty} \geq l_{\min}, j = 1, 2, \ldots, n
\end{cases}
\end{equation}

综上,问题三的总目标函数可写为:
\begin{equation}
\min -\alpha_{1} \sum_{i=1}^{m} \delta_{i} + \alpha_{2} \sum_{i=1}^{m} \sum_{j=1}^{n} dh_{ij} + \alpha_{3} \sum_{i=1}^{m} \sum_{j \in A \cap B} officer_{ij} + \alpha_{4} \sum_{j=1}^{n} DZCost_{j} \times TR_{j}^{duty}
\tag{30}
\end{equation}
\begin{equation}
+ \alpha_{5} (t_{\max} - t_{\min}) + \alpha_{6} \sum_{j=1}^{n} PRCost_{j} \times T_{j}^{Ring} + \alpha_{7} (l_{\max} - l_{\min})
\end{equation}

进一步为了便于表达,我们在前两问所作假设之外,进一步给出如下假设:$Dutyset_{1}^{j}$ 代表 $x_{j}$ 第一次执勤的航班集合,以此类推:
\begin{equation}
Dutyset_{1}^{j} = \{j_{1}, j_{2}, \ldots, j_{duty_{1}}\},
\end{equation}
\begin{equation}
Dutyset_{2}^{j} = \{j_{(duty_{1}+1)}, j_{(duty_{1}+2)}, \ldots, j_{duty_{2}}\}
\tag{31}
\end{equation}
\begin{equation}
\ldots,
\end{equation}
\begin{equation}
Dutyset_{s_{j}}^{j} = \{j_{(duty_{(s_{j}-1)}+1)}, j_{(duty_{(s_{j}-1)}+2)}, \ldots, j_{duty_{s_{j}}}\}
\end{equation}

由题目中任务环的说明,若干个顺序执行的执勤构成一个任务环,因此执勤 $Dutyset_{1}^{j}, Dutyset_{2}^{j}, \ldots, Dutyset_{s}^{j}$ 可以按照顺序划分成若干个任务环,这种划分方法可能不唯一。将任务环记为如下集合:
\begin{equation}
Ring_{1}^{j} = \{Dutyset_{1}^{j}, Dutyset_{2}^{j}, \ldots, Dutyset_{r_{1}}^{j}\},
\end{equation}
\begin{equation}
Ring_{2}^{j} = \{Dutyset_{(r_{1}+1)}^{j}, Dutyset_{(r_{1}+2)}^{j}, \ldots, Dutyset_{r_{2}}^{j}\}
\tag{32}
\end{equation}
\begin{equation}
\ldots,
\end{equation}
\begin{equation}
Ring_{q_{j}}^{j} = \{Dutyset_{(r_{(q_{j}-1)}+1)}^{j}, Dutyset_{(r_{(q_{j}-1)}+2)}^{j}, \ldots, Dutyset_{r_{q_{j}}}^{j}\}
\end{equation}

显然序列 $1, 2, \ldots, r_{1}, r_{1}+1, \ldots, r_{2}, \ldots, r_{q_{j}-1}+1, \ldots, r_{q_{j}}$ 和序列 $1, 2, \ldots, s$ 相同易知,任务环 $Ring_{h}^{j}, h = 1, 2, \ldots, q_{j}$ 的最后一个执勤 $Dutyset_{r_{h}}^{j}$ 的最后一个航班下标为 $j_{duty_{r_{h}}}$,第一个执勤 $Dutyset_{(r_{(h-1)}+1)}^{j}$ 的第一个航班下标为 $j_{(duty_{r_{(h-1)}}+1)}$。则该任务环时长为

\[
Arrtime_{j_{duty_{r_h}}}^{arr} - Dptrtime_{j_{(duty_{r_{(h-1)}}+1)}}^{dp}
\]

与问题二类似,问题三也需要保证总体执勤成本最低,同时问题二的约束条件在问题三的情况下也是适用的。除此之外,问题三的目标函数还有如下约束条件:

① 每个机组人员每个排班周期的任务环总时长不超过 $MaxTAFB$ 分钟,题目要求中给出了 $MaxTAFB$ 为 14400 分钟,即 10 天,而排班周期最长为 31 天,因此该约束理解为单个任务时长不超过 10 天,任务环 $Ring_h^j, h=1,2,...,q_j$ 的最后一个执勤 $Dutyset_{r_h}^j$ 的最后一个航班下标为 $j_{duty_{r_h}}$;第一个执勤 $Dutyset_{(r_{(h-1)}+1)}^j$ 的第一个航班下标为 $j_{(duty_{r_{(h-1)}}+1)}$。则该任务环时长为 $Arrtime_{j_{duty_{r_h}}}^{arr} - Dptrtime_{j_{(duty_{r_{(h-1)}}+1)}}^{dp}$。则有约束条件为:
\begin{equation}
\begin{aligned}
Arrtime_{j_{duty_{r_h}}}^{arr} - Dptrtime_{j_{(duty_{r_{(h-1)}}+1)}}^{dp} & \leq MaxTAFB \\
Arrtime_{j_{duty_{r_1}}}^{arr} - Dptrtime_{j_1}^{dp} & \leq MaxTAFB
\end{aligned}
\tag{33}
\end{equation}
$$h=1,2,...,q_j, j=1,2,...,n$$

② 每个机组人员相邻两个任务环之间至少有 $MinVacDay$ 天休息。考虑任务环 $Ring_h^j, h=1,2,...,(q_j-1)$ 的最后一个执勤 $Dutyset_{r_h}^j$ 的最后一个航班下标为 $j_{duty_{r_h}}$;$Ring_{(h+1)}^j$ 第一个执勤 $Dutyset_{(r_h+1)}^j$ 的第一个航班下标为 $j_{(duty_{r_h}+1)}$。综上有约束条件如下:
\begin{equation}
Dptrtime_{j_{(duty_{r_h}+1)}}^{dp} - Arrtime_{j_{duty_{r_h}}}^{arr} \geq MinVacDay
\tag{34}
\end{equation}
$$h=1,2,...,(q_j-1), \ j=1,2,...,n$$

③ 每个机组人员连续执勤天数不超过 $MaxSuccOn$ 天,我们认为两个相邻执勤休息时间小于 24 小时,则认为是连续执勤。那么 $x_j$ 所在航班 $j_1, j_2, ..., j_{K_j}$ 可以根据相邻两个航班之间休息时间是小于 24 小时,来判断两个航班是否是连续执勤。假设集合 $Dutyset-continu = \{(j_{(con_k-1)}, j_{con_k}) \big| j_1, ..., j_{K_j}, \text{若航班} j_{con_k} \text{和} j_{(con_k+1)} \text{之间休息时间} > 24 \text{小时}\}$。显然航班 $j_{(con_k-1)}, j_{con_k}$ 不连续,分别属于不同的连续执勤时间段。假设 $set-continu$ 含有 $Q_j$ 个元素,不妨设为:$(j_{(con_1-1)}, j_{con_1}), (j_{(con_2-1)}, j_{con_2}), ..., (j_{(con_{Q_j}-1)}, j_{con_{Q_j}})$,并且按照航班出发先后顺序排序。那么第一个连续执勤时间为航班 $j_{con_1}$ 到达时间减去 $j_1$ 的出发时间;第二个连续执勤时长为航班 $j_{con_2}$ 的到达时间减去航班 $j_{(con_1+1)}$ 的出发时间,以此类推,由此我们得到约束条件为:
\begin{equation}
\begin{aligned}
Arrtime_{j_{(con_k-1)}}^{arr} - Dptrtime_{j_{con(k-1)}}^{dp} & \leq MaxSuccOn \\
Arrtime_{j_{K_j}}^{arr} - Dptrtime_{j_{con_{Q_j}}}^{dp} & \leq MaxSuccOn
\end{aligned}
\tag{35}
\end{equation}
$$k=1,2,3,...,Q_j$$

据此,我们得到问题三的数学模型如下:

\begin{equation}
\begin{aligned}
\min & -\alpha_{1} \sum_{i=1}^{m} \delta_{i} + \alpha_{2} \sum_{i=1}^{m} \sum_{j=1}^{n} d h_{ij} + \alpha_{3} \sum_{i=1}^{m} \sum_{j \in A \cap B} officer_{ij} + \alpha_{4} \sum_{j=1}^{n} DZCost_{j} \times TR_{j}^{duty} \\
& + \alpha_{5} (t_{\max} - t_{\min}) + \alpha_{6} \sum_{j=1}^{n} PRCost_{j} \times T_{j}^{Ring} + \alpha_{7} (l_{\max} - l_{\min})
\end{aligned}
\end{equation}

\begin{equation}
\begin{aligned}
Arrvtime_{j_{duty_{r_{h}}}}^{arr} - Dptrtime_{j_{(duty_{r_{(h-1)}}+1)}}^{dp} & \leq \text{MaxTAFB} \\
Arrvtime_{j_{duty_{1}}}^{arr} - Dptrtime_{j_{1}}^{dp} & \leq \text{MaxTAFB} \\
h & = 1, 2, \ldots, q_{j}, \, j = 1, 2, \ldots, n \\
Dptrtime_{j_{(duty_{r_{h}}+1)}}^{dp} - Arrvtime_{j_{duty_{r_{h}}}}^{arr} & \geq \text{MinVacDay} \\
h & = 1, 2, \ldots, (q_{j} - 1), \, j = 1, 2, \ldots, n \\
Arrvtime_{j_{(con_{k-1})}}^{arr} - Dptrtime_{j_{con(k-1)}}^{dp} & \leq \text{MaxSuccOn} \\
Arrvtime_{j_{K_{j}}}^{arr} - Dptrtime_{j_{conQ_{j}}}^{dp} & \leq \text{MaxSuccOn} \\
k & = 1, 2, 3, \ldots, Q_{j} \\
TZ_{j}^{duty} & = \sum_{k=1}^{s_{j}} (Arrvtime_{j_{duty_{k}}}^{arr} - Dptrtime_{j_{(duty_{(k-1)}+1)}}^{dp}), \, \text{若 } k = 1, \, j_{duty_{0}+1} = j_{1} \\
Arrvtime_{j_{duty_{2}}}^{arr} - Dptrtime_{j_{1}}^{dp} & > 24h, \ldots, Arrvtime_{j_{duty_{s_{j}}}}^{arr} - Dptrtime_{j_{(duty_{(s_{j}-2)}+1)}}^{dp} > 24h \\
\sum_{k=1}^{duty_{1}} (1 - d h_{j_{k},j}) PT_{j_{k}} & \leq \text{MaxBlk}, \ldots, \sum_{k=duty_{(s_{j}-1)}+1}^{duty_{s_{j}}} (1 - d h_{j_{k},j}) PT_{j_{k}} \leq \text{MaxBlk} \\
Arrvtime_{j_{duty_{1}}}^{arr} - Dptrtime_{j_{1}}^{dp} & \leq \text{MaxDP}, \ldots, Arrvtime_{j_{duty_{s_{j}}}}^{arr} - Dptrtime_{j_{(duty_{(s_{j}-1)}+1)}}^{dp} \leq \text{MaxDP} \\
Dptrtime_{j_{(duty_{1}+1)}}^{dp} - Arrvtime_{j_{duty_{1}}}^{arr} & \geq \text{MinRest}, \ldots, Dptrtime_{j_{(duty_{(s_{j}-1)}+1)}}^{dp} - Arrvtime_{j_{duty_{(s_{j}-1)}}}^{arr} \geq \text{MinRest} \\
\sum_{j \in A} captain_{ij} & \geq \delta_{i} \\
\sum_{j \in B} officer_{ij} & \geq \delta_{i} \\
captain_{ij} & = 0, \, \text{当 } j \notin A \\
officer_{ij} & = 0, \, \text{当 } j \notin B \\
cf_{ij} & \geq captain_{ij}, \, cf_{ij} \geq officer_{ij}, \, cf_{ij} \geq d h_{ij} \\
cf_{ij} & = captain_{ij} + officer_{ij} + d h_{ij} \\
AirP_{j_{1}}^{dp} & = BASE_{j}, \, AirP_{j_{K_{j}}}^{arr} = BASE_{j} \\
AirP_{j_{h+1}}^{dp} & = AirP_{j_{h}}^{arr}, \, h = 1, 2, 3, \ldots, K_{j} - 1 \\
Dptrtime_{j_{h+1}}^{dp} - Arrvtime_{j_{h}}^{arr} & \geq \text{MinCT}, \, h = 1, 2, 3, \ldots, K_{j} - 1 \\
\lambda_{1} & > 0, \, \lambda_{2} > 0, \, \lambda_{3} > 0
\end{aligned}
\tag{36}
\end{equation}

\subsection{5.3.2 模型求解}

与问题二的思路基本一致,不同的是,它将问题二中一连串的执勤和休息时间构成一个任务环,多个任务环构成一个机组人员的排班周期。本问在前两问的约束条件下,需要构建模型保证机组人员总体任务环成本最低,且机组人员之间任务环时长尽可能平衡。

问题三的航班规划是将问题二生成的一连串执勤进行配对构成单个任务环,执勤的配对同样基于问题一、二的航班规划原则,首先要保证上一执勤最后飞机降落机场与下一执勤开始飞机起飞机场保持一致;其次是优先配对两个时间间隔短的执勤;然后是要确保连续执勤的天数不能超过 4 天(如果某个机组人员在当天最后一班航班降落时刻与他第二天第一班航班起飞时刻相差 24 小时,则连续执勤天数重新计算);最后是执勤的天数和休息的天数加起来不得超过 10 天。

根据上述的四个原则可以构建一个合格的单位任务环,但每个机组人员的排班周期是由多个单位任务环构成,其构成条件同样是基于起落机场一致原则,在满足两个任务环之间至少有 2 天休息时间即可构建一个完整的排班周期。基于此,我们基于问题一、二的求解思路下,可以建立以下方法:

\textbf{第一步:确定航班规划}

根据问题一思路规划航班路线。

\textbf{第二步:组合执勤}

根据问题二思路组合执勤。

\textbf{第三步:构造任务环}

(1) 在所有执勤集合 \( W \) 中选择一条出发机场为基地的执勤,作为任务环的第一条执勤,再从剩余执勤集合 \( W \) 中选择一条出发机场与第一条执勤到达机场对应、出发时间与第一条执勤到达时间大于 \( MinRest \)、且连接时间最短的执勤,与第一条执勤连接。

(2) 再从剩余执勤集合 \( W \) 中选择一条符合上述连接条件的执勤,若连接到的执勤数量为 3 的倍数,则下一条执勤的出发时间需与当前执勤的到达时间相差 24 小时,

(3) 重复从剩余执勤集合 \( W \) 中选择满足连接条件、且连接时间最短的执勤,与上一条执勤连接。若任务环时长超过 \( MaxSuccOn/2 \),且当前连接执勤到达机场为出发基地,则任务环构造完毕,保存到任务环集合 \( T \) 中。

(4) 重复上述过程构造任务环,直到无法再构造出新的任务环。

\textbf{第四步:确定机组分配}

确定一个完整的排班周期后,开始对其分配合适的机组人员。为了追求任务环成本最低,本问同样基于贪心算法进行求解,以工资低的机组人员优先排班为原则。首先找出数据中包含的基地信息,找出有多少个基地,然后将所有机组人员划分到各个基地,

\subsection*{5.3.3 结果分析}

1. 不满足机组配置航班数

根据上述方法计算表明,问题三中 A 类数据不满足机组配置航班数与前两问一致,具体的航班信息也是一样,所以在本问不作展示。B 类数据由于数据量大,又加上任务环等条件约束,造成不满足机组配置的航班数急剧上升,达到了 3020 架次。

2. 机组人员总体乘机次数

利用上述方法,我们最终求得问题三 A 类数据中需要摆渡的航班有 4 个,与前两问一致,B 类数据中需要摆渡的航班有 149 个,相较于问题二下降许多,主要原因是问题三中 B 类数据不满足机组配置的航班数有大幅度提升。

3. 机组总体利用率

计算表明,在问题三的约束条件下,A、B 类数据的机组利用率如表 5.11 所示,部分 A、B 类航班信息具体排班信息如表 5.12、5.13 所示:

\begin{table}
\caption{部分A类航班信息排班表}
\begin{tabular}{cccccccc}
机组人员 & 航班号 & 出发日期 & 出发时间 & 出发机场 & 到达日期 & 到达时间 & 到达机场 & 资格 \\
\hline
A0001 & FA680 & 8/11/2021 & 8:00 & NKX & 8/11/2021 & 9:30 & PGX & C \\
A0001 & FA681 & 8/11/2021 & 10:10 & PGX & 8/11/2021 & 11:40 & NKX & C \\
A0001 & FA812 & 8/11/2021 & 12:20 & NKX & 8/11/2021 & 14:05 & PDK & C \\
A0001 & FA813 & 8/11/2021 & 14:50 & PDK & 8/11/2021 & 16:40 & NKX & C \\
A0001 & FA872 & 8/12/2021 & 7:55 & NKX & 8/12/2021 & 9:00 & PLM & C \\
A0001 & FA873 & 8/12/2021 & 9:40 & PLM & 8/12/2021 & 10:50 & NKX & C \\
A0001 & FA884 & 8/12/2021 & 11:30 & NKX & 8/12/2021 & 13:50 & XGS & C \\
A0001 & FA885 & 8/12/2021 & 14:30 & XGS & 8/12/2021 & 16:50 & NKX & C \\
A0001 & FA680 & 8/13/2021 & 8:00 & NKX & 8/13/2021 & 9:30 & PGX & C \\
A0001 & FA681 & 8/13/2021 & 10:10 & PGX & 8/13/2021 & 11:40 & NKX & C \\
A0001 & FA812 & 8/13/2021 & 12:20 & NKX & 8/13/2021 & 14:05 & PDK & C \\
A0001 & FA813 & 8/13/2021 & 14:50 & PDK & 8/13/2021 & 16:40 & NKX & C \\
A0001 & FA680 & 8/14/2021 & 8:00 & NKX & 8/14/2021 & 9:30 & PGX & C \\
A0001 & FA681 & 8/14/2021 & 10:10 & PGX & 8/14/2021 & 11:40 & NKX & C \\
A0001 & FA812 & 8/14/2021 & 12:20 & NKX & 8/14/2021 & 14:05 & PDK & C \\
A0001 & FA813 & 8/14/2021 & 14:50 & PDK & 8/14/2021 & 16:40 & NKX & C \\
\multicolumn{8}{c}{休息} \\
A0001 & FA872 & 8/18/2021 & 7:55 & NKX & 8/18/2021 & 9:00 & PLM & C \\
A0001 & FA873 & 8/18/2021 & 9:40 & PLM & 8/18/2021 & 10:50 & NKX & C \\
A0001 & FA884 & 8/18/2021 & 11:30 & NKX & 8/18/2021 & 13:50 & XGS & C \\
A0001 & FA885 & 8/18/2021 & 14:30 & XGS & 8/18/2021 & 16:50 & NKX & C \\
A0001 & FA872 & 8/19/2021 & 7:55 & NKX & 8/19/2021 & 9:00 & PLM & C \\
A0001 & FA873 & 8/19/2021 & 9:40 & PLM & 8/19/2021 & 10:50 & NKX & C \\
\end{tabular}
\end{table}

\begin{table}
\caption{部分B类航班信息排班表}
\begin{tabular}{cccccccc}
机组人员 & 航班号 & 出发日期 & 出发时间 & 出发机场 & 到达日期 & 到达时间 & 到达机场 & 主要资格 \\
\hline
B0081 & FB13 & 8/1/2019 & 7:00 & TGD & 8/1/2019 & 8:05 & HOM & C \\
B0081 & FB18 & 8/1/2019 & 9:00 & HOM & 8/1/2019 & 10:10 & TGD & C \\
B0081 & FB663 & 8/1/2019 & 11:00 & TGD & 8/1/2019 & 12:45 & G KU & C \\
B0081 & FB666 & 8/1/2019 & 13:40 & G KU & 8/1/2019 & 15:30 & TGD & C \\
B0081 & FB667 & 8/1/2019 & 16:10 & TGD & 8/1/2019 & 17:55 & G KU & C \\
B0081 & FB662 & 8/2/2019 & 8:15 & G KU & 8/2/2019 & 10:05 & TGD & C \\
B0081 & FB575 & 8/2/2019 & 10:45 & TGD & 8/2/2019 & 12:15 & HWJ & C \\
\end{tabular}
\end{table}

\begin{tabular}{lllllll}
B0081 & FB576 & 8/2/2019 & 12:55 & HWJ & 8/2/2019 & 14:30 \\
 & & & & & & TGD \\
B0081 & FB1097 & 8/2/2019 & 17:15 & TGD & 8/2/2019 & 19:45 \\
 & & & & & & SXA \\
B0081 & FB1200 & 8/3/2019 & 7:05 & SXA & 8/3/2019 & 8:40 \\
 & & & & & & FBX \\
B0081 & FB246 & 8/3/2019 & 9:25 & FBX & 8/3/2019 & 11:10 \\
 & & & & & & TGD \\
B0081 & FB623 & 8/3/2019 & 11:55 & TGD & 8/3/2019 & 13:40 \\
 & & & & & & XSJ \\
B0081 & FB626 & 8/3/2019 & 14:25 & XSJ & 8/3/2019 & 16:10 \\
 & & & & & & TGD \\
B0081 & FB73 & 8/3/2019 & 17:00 & TGD & 8/3/2019 & 18:35 \\
 & & & & & & NOU \\
B0081 & FB52 & 8/4/2019 & 8:30 & NOU & 8/4/2019 & 10:05 \\
 & & & & & & TGD \\
B0081 & FB677 & 8/4/2019 & 10:45 & TGD & 8/4/2019 & 12:10 \\
 & & & & & & THJ \\
B0081 & FB678 & 8/4/2019 & 13:00 & THJ & 8/4/2019 & 14:25 \\
 & & & & & & TGD \\
B0081 & FB613 & 8/4/2019 & 17:40 & TGD & 8/4/2019 & 19:20 \\
 & & & & 休息 & & XMH \\
B0081 & FB618 & 8/5/2019 & 20:05 & XMH & 8/5/2019 & 21:45 \\
 & & & & & & TGD \\
B0081 & FB273 & 8/6/2019 & 19:45 & TGD & 8/6/2019 & 21:40 \\
 & & & & & & FBX \\
B0081 & FB1203 & 8/7/2019 & 8:45 & FBX & 8/7/2019 & 10:30 \\
 & & & & & & SXA \\
B0081 & FB1206 & 8/7/2019 & 11:10 & SXA & 8/7/2019 & 12:55 \\
 & & & & & & FBX \\
B0081 & FB2144 & 8/7/2019 & 13:40 & FBX & 8/7/2019 & 15:45 \\
 & & & & & & TGD \\
B0081 & FB263 & 8/7/2019 & 16:25 & TGD & 8/7/2019 & 18:15 \\
 & & & & & & FBX \\
B0081 & FB268 & 8/7/2019 & 19:10 & FBX & 8/7/2019 & 21:00 \\
 & & & & & & TGD \\
B0081 & FB15 & 8/8/2019 & 8:00 & TGD & 8/8/2019 & 9:05 \\
 & & & & & & HOM \\
B0081 & FB20 & 8/8/2019 & 10:00 & HOM & 8/8/2019 & 11:15 \\
 & & & & & & TGD \\
B0081 & FB91 & 8/8/2019 & 11:55 & TGD & 8/8/2019 & 14:45 \\
 & & & & & & HUK \\
B0081 & FB92 & 8/8/2019 & 15:30 & HUK & 8/8/2019 & 18:25 \\
 & & & & & & TGD \\
B0081 & FB1891 & 8/9/2019 & 7:30 & TGD & 8/9/2019 & 8:25 \\
 & & & & & & TAN \\
B0081 & FB1896 & 8/9/2019 & 17:20 & TAN & 8/9/2019 & 18:20 \\
 & & & & & & TGD \\
\end{tabular}

\section*{4. 总体任务环成本}

总体任务环成本为总体执勤成本加出差补贴两部分之和,出差补贴用每个机组人员的出差补贴加和计算,单个机组人员的出差补贴为任务环时长乘单位小时任务环成本。每个机组人员在排版周期内有多个任务环,单个任务环的总时长为该任务环最后一趟航班(到达基地)的到达时刻减去第一趟航班的起飞时刻,通过计算,A、B类数据的任务环成本如下表所示:

\begin{tabular}{ccc}
\hline 结果指标(注) & A类数据 & B类数据 \\
\hline 任务环成本 & 62.7653 & 4143.78 \\
\hline
\end{tabular}

\section*{5. 任务环时长平衡性}

任务环平衡性的分析,通过统计出每类数据的排班周期中,执行一个、二个、三个、四个任务环的机组人员数量来得出,A、B类数据的任务环分布如下表所示:

\begin{tabular}{ccc}
\hline 结果指标(注) & A类数据 & B类数据 \\
\hline 一天 & 12 & 201 \\
二天 & 6 & 389 \\
三天 & 24 & \\
\hline
\end{tabular}

\begin{table}
\centering
\caption{表5.16 问题三结果汇总}
\begin{tabular}{l c c}
\hline
结果指标(注) & A类数据 & B类数据 \\
\hline
不满足机组配置航班数 & 0 & 3020 \\
满足机组航班配置数 & 206 & 10935 \\
机组人员总体乘机次数 & 8 & 298 \\
替补资格使用次数 & 0 & 450 \\
机组总体利用率 & 76.06\% & 69.78\% \\
最小/平均/最大一次执勤飞行时间 & 1小时05分/5时48分/8小时10分 & 1小时05分/5时24分/8小时00分 \\
最小/平均/最大一次执勤时长 & 1小时30分/7小时32分/11小时30分 & 1小时15分/7小时46分/11小时30分 \\
最小/平均/最大机组人员执勤天数 & 4/6/9 & 2/12/21 \\
一/二/三/四天任务环数量分布 & 12/6/4/10 & 201/389/605/526 \\
总体执勤成本(万元) & 57.8893 & 3827.7 \\
总体任务环成本(万元) & 62.7653 & 4143.78 \\
程序运行分钟数 & 0.0667分钟 & 0.5833分钟 \\
\hline
\end{tabular}
\end{table}

通过上述数据处理可以发现,随着更多约束条件的添加,机组的总体利用率在逐步下降,飞行员在执勤过程中的休息时间变的更多,导致总体执勤成本增加。

\section*{六、模型评价与推广}

\subsection*{6.1 模型评价}

机组排班问题是航空公司生产经营活动中必须解决的生产计划问题,同时它也是非常有挑战的工作。本文针对航空公司机组排班问题所涉及的三个目标,建立加权0-1整数规划模型来求解最优排班问题。但随着问题约束条件越来越与实际需求接近,问题的求解也变得越来越复杂,很难找到一个合适的算法来求得所有最优的机组排班表。为此本文基于启发式算法来求解最优解,将机组排班分为两个子问题,分别是航班规划和机组排班,通过对两个子问题进行求解,可以得到一个相对优化的航班计划。

\subsection*{6.2 模型推广}

由于航班优化问题本质上是一个NP-Hard问题,在短时间内很难获得最优解。尽管题目的三个问题已经尽可能的满足实际需求,但还是忽略了很多问题,比如航班延误、乘客座位安排等问题,所以我们仍需要针对这些问题在本文的模型上作进一步完善、改进。

\section*{七、参考文献}

1. Xiaodong Luo, Yogesh Dashora, Tina Shaw (2015) Airline Crew Augmentation: Decades of Improvements from Sabre. Interfaces 45(5): 409-424.

2. Saeed Saemi, Alireza Rashidi Komijan, Reza Tavakkoli-Moghaddam and Mohammad Fallah, A new mathematical model to cover crew pairing and rostering problems simultaneously, Journal of Engg. Research Vol. 9 No. (2) June 2021 pp. 218-233

3. Mohamed Haouari, Farah Zeghal Mansour, Hanif D. Sherali (2019) A New Compact Formulation for the Daily Crew Pairing Problem. Transportation Science 53(3): 811-828

4. 张米. 航空公司机组排班模型研究[D]. 清华大学, 2014.

5. 邵俊. 基于遗传算法的机组任务配对研究 [D][D]. 南京航空航天大学, 2006.