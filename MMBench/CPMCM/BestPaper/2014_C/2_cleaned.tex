\documentclass{article}
\usepackage{amsmath}
\usepackage{amssymb}
\usepackage{graphicx}

\title{无线通信中的快时变信道建模}
\author{}
\date{}

\begin{document}

\maketitle

\begin{abstract}
本文主要围绕无线信道参数估计的问题,在合理假设的基础上,分析处理某信道的测试数据以及不同速度条件引起信道参数的变化,采取广泛运用的基扩展方法,建立较为精确的快时变信道参数估计模型,并基于最小二乘法对模型参数进行求解,利用 NMSE 评价所建模型的准确度。此外,本文还对服从特定分布的多径时变传输信道建模与仿真,并分析了真实信号在信道传输过程中的调制解调过程。

对于问题 1:数据文件 1 给出了来自不同路径的 9 组数据,要求通过建模用部分测试数据预测整体数据。本文采用了基扩展方法建模,用四种不同的基函数模型预估仿真分析,尤其是创新性提出并尝试 Chebyshev-BEM 模型。对比仿真结果发现 Legendre-BEM 模型与 Chebyshev-BEM 模型对信道估计结果较好,并且计算了算法复杂度,讨论了基函数阶数与算法复杂度之间的关系。

对于问题 2A:数据文件 2、3、4 给出不同速度下信道测试数据,本文根据多普勒效应的具体推导,在问题 1 的基础上探究不同速度对模型准确性的影响,推导得出随着速度的增大,信道估计值与实际值的贴合度逐渐下降,归一化均方误差 NMSE 迅速增大的结论。

对于问题 2B:要求建立新的信道仿真模型,产生多径衰落信道相互独立、幅度服从瑞利分布、相位服从均匀分布的信道数据。本文引出较为典型的 Zheng 模型,在此基础上得到多个不相关 Rayleigh 波形生成方法;利用改进的 Zheng 模型进行信道的仿真,并讨论了径数和多径延迟对信道的影响;最后,通过比较得出多径数 \( L \) 取值为 8~10 时对信道估计产生的影响最大以及确定速度下单段信道长度 \( N \) 越大、有效信号传输的比例越大的结论。

对于问题 3:本文先勾勒了信号传输过程的流程图,分析了信号先调制成复数序列,通过信道传输,再在接收端对信号进行过滤后解调为二进制序列的过程。详细推导了基带滤波求逆运算的运算方法,随后运用问题 1 所建模型,具体阐述逆运算还原信号的过程,通过仿真发现可以较好地还原信号,并且有效地控制了算法的简洁度。

本文的亮点在于:在综合现有文献方法的基础上,提出创新性的想法并大胆地尝试,例如由 Legendre-BEM 模型推广到 Chebyshev-BEM 模型,取得了很好的仿真效果。此外,本文在模型准确度和算法简洁度方面进行了大量对比试验分析,选出的模型具有较强的工程实用性,基本完成了题目所给出的任务。

关键词:信道估计 基函数 瑞利波形 调制解调
\end{abstract}

\begin{itemize}
    \item 一、问题背景 \dotfill 1
        \begin{itemize}
            \item 1.1 问题的引出 \dotfill 1
            \item 1.2 国内外研究现状分析 \dotfill 2
        \end{itemize}
    \item 二、问题描述 \dotfill 4
    \item 三、专业术语及符号说明 \dotfill 5
        \begin{itemize}
            \item 3.1 专业术语 \dotfill 5
            \item 3.2 符号说明 \dotfill 6
        \end{itemize}
    \item 四、问题一 \dotfill 7
        \begin{itemize}
            \item 4.1 问题分析 \dotfill 7
            \item 4.2 模型建立 \dotfill 7
                \begin{itemize}
                    \item 4.2.1 基函数选择 \dotfill 7
                    \item 4.2.2 多普勒效应 \dotfill 10
                    \item 4.2.3 数据分段处理 \dotfill 10
                \end{itemize}
            \item 4.3 实例求解 \dotfill 10
            \item 4.4 本章小结 \dotfill 16
        \end{itemize}
    \item 五、问题二 \dotfill 17
        \begin{itemize}
            \item 5.1 问题分析 \dotfill 17
            \item 5.2 速度变更 \dotfill 17
            \item 5.3 模型建立 \dotfill 18
                \begin{itemize}
                    \item 5.3.1 单信道正弦波叠加 Rayleigh 仿真方法 \dotfill 19
                    \item 5.3.2 Zheng 模型 \dotfill 20
                    \item 5.3.3 多个不相关 Rayleigh 波形生成 \dotfill 20
                \end{itemize}
            \item 5.4 实例求解 \dotfill 21
            \item 5.5 本章小结 \dotfill 22
        \end{itemize}
    \item 六、问题三 \dotfill 23
        \begin{itemize}
            \item 6.1 问题分析 \dotfill 23
            \item 6.2 模型建立 \dotfill 23
                \begin{itemize}
                    \item 6.2.1 调制与解调 \dotfill 23
                    \item 6.2.2 16QAM \dotfill 24
                    \item 6.2.3 信道传输与求逆 \dotfill 25
                \end{itemize}
            \item 6.3 实例求解 \dotfill 27
            \item 6.4 本章小结 \dotfill 28
        \end{itemize}
    \item 七、总结及展望 \dotfill 29
    \item 八、参考文献 \dotfill 30
    \item 九、致谢 \dotfill 32
\end{itemize}

\section{问题背景}

\subsection{问题的引出}

随着信息化进程的加快, 人们对数据信息的交换需求促进了宽带业务迅猛发展, 数据的高速传输更是未来无线通信的趋势。宽带移动通信传输正在改变着人们的生活, 更为快速和准确的传递信息是其基本需求。据预测, 到 2020 年, 数以千亿的“物”, 包括汽车、计量表、医疗设备和家电等都将连入移动通信网络, 人们的移动数字生活也将更加美好。由于移动通信网络连接环境复杂多变, 对实现高速宽带数据传递提出了更高的要求和挑战。分析现有通信模型的不足, 建立新的数学模型, 对提升信道容量、增加信息传输速率和降低误码率会有很好的促进作用。

在通信系统中, 发送端通过信道传输信号到接收端, 在传输过程中, 不可避免地要引入干扰噪声。接收端对包含噪声的信号进行合理解码, 得到正确的信息, 完成信息传输过程, 原理用图 1-1 表示。

\begin{figure}[h]
\centering
\includegraphics[width=0.8\textwidth]{image.png}
\caption{通信基本模型示意图}
\end{figure}

通信过程的数学模型可以表示为:
\begin{align}
c_{i,n} &= \sigma_0 \sqrt{2/N} \tag{5-3a} \\
f_{1,n} &= f_d \cos\left(\frac{2\pi n - \pi + \phi}{4N}\right) \tag{5-3b} \\
f_{2,n} &= f_d \sin\left(\frac{2\pi n - \pi + \phi}{4N}\right) \tag{5-3c}
\end{align}

从式 (1-1) 可以看出, 在已知接收端信号 $Y$ 的情况下, 要得知发送端的信号 $X$, 还需要知道信道变量 $H$ 和噪声 $W$ 的统计特征。$W$ 可视为加性高斯白噪声 AWGN (Additive White Gaussian Noise), 因此问题的关键就是对 $H$ 规律的探索。

此外, 在无线信道中, 发送和接收之间通常存在多于一条的信号传播路径。多径的存在是因为发射机和接收机之间建筑物和其他物体的反射、绕射、散射等引起的。由于环境的复杂性, 信号传播途径也复杂多变, 需要对其进行简化和抽象, 建立描述、估计信道传播的数学模型。当信号在无线信道传播时, 多径反射和衰减的变化将使信号经历随机波动。无线多径传输系统的时间离散形式的数学表达式为 \cite{ref1}:
\begin{align}
\widetilde{c}_{k,i}(t) &= \sum_{n=1}^N c_{k,i,n} \cos(2\pi f_{k,i,n} t + \theta_{k,i,n}), i = 1, 2 \tag{5-4} \\
c_{k,i,n} &= \sigma_0 \sqrt{2/N} \\
f_{k,1,n} &= f_d \cos\left(\frac{2\pi n - \pi + \phi}{4N} + \frac{2\pi}{4PN}\right) \tag{5-5} \\
f_{k,2,n} &= f_d \sin\left(\frac{2\pi n - \pi + \phi}{4N} + \frac{2\pi}{4PN}\right)
\end{align}

式中 $L$ 为信道的多径数, $K$ 为传输信号的长度, $w[n]$ 可视为 AWGN, $h_l[n]$

就是信道参数。

可以看出,由于多径效应的存在,接收端接收到的信号相比于实际发送的信号在时域上被展宽,称为时延扩展。移动台与基站间的相对运动带来的多普勒效应使信道特性随时间变化 \({}^{[2-3]}\) 。运动速度越快,信道变化越快。如果我们准确地从接收端得到发端的信号,必须准确地对无线信道 \(H\) 进行估计,常用的方法包括使用训练序列(导频),即在发送端插入训练序列,在接收端根据已知导频可以估计信道。由于信道是时变的,需要周期性地插入训练信号和进行信道参数估计。在慢衰落信道情况下,使用导频是一种比较准确经济的方法,但在高速运动的快时变信道情况下,就需要频繁地增加训练信号(开销),在接收端增加相同的信道估计次数。由于导频不承载有用信息,过密的导频插入将会占用过多的传输资源,降低有用信息的传输速率,因此在快时变的信道中,希望发现新的数学模型来估计快变的信道参数,降低导频的插入频率。

减少信道参数估计的方法有插值等方法,而基扩展 BEM(Basis Expansion Model) 方法是最近研究得比较活跃的一种方法 \({}^{[1-2]}\),有可能以后在实际通信系统获得应用并提升性能。其主要是利用有限个基函数的线性组合来描述一定时间内的时变信道,可以模拟有多普勒效应的快时变信道,减少信道参数直接估计的次数。所以我们要建立新的模型,并对模型准确度和算法复杂度作出评估。

\subsection{国内外研究现状分析}

信道估计的方法通常可分为三类:第一类是基于导频信号和插值技术的信道估计方法,根据插入的导频符号在 IFFT 之前还是之后,分为时域导频符号插入法和频域导频符号插入法;第二类是基于判决反馈的方法,假设信道是慢变化的,即至少在两个 OFDM(Orthogonal Frequency Division Multiplexing) 码字内是恒定的,并将接收的上一个码字解调后获得的符号估计,作为已知信息符号对信道进行估计;第三类是基于被传输的信息符号的有限字符特性和其统计特性的信道盲估计方法。在实际的运用中,可以根据不同的系统要求选择适当的算法,而导频辅助的信道估计方法是目前应用得最为广泛的一类方法。

B. Lindoff \({}^{[4]}\) 等人用 RAKE 接收机联合信道插值的方法实现时变信道的估计。对于每一个 RAKE 接收的分支,先用已知的导频信号获得初始的信道估计,然后利用插值的方法得到最终的信道估计,最后用最大比合并 RAKE 接收机的分支。H.N.Lee \({}^{[5]}\) 在估计快速时变信道时,充分利用发射整形滤波器的冲激响应已知这一条件,减少待估计的信道参数个数以及相应的训练序列长度。这样,在快变信道条件下,信道估计误差减小,增加了精确度。在这个方法中,可用最小二乘估计、最大似然估计、最大后验概率估计等准则来估计初始的信道参数,然后利用 \(\sin x / x\) 函数内插的方法实现快变信道的跟踪。

Che-ShenYeh 等人提出了时域导频时域互相关时变估计算法 \({}^{[6]}\) 。将一个 OFDM 符号周期分成多个(如 16 或 32)等长度的子符号时隙,并假定每个子符号

时隙内信道的幅度和相位是恒定不变的,用接收的导频符号与本地的导频符号的互相关来估计信道的时域冲激响应。

目前,针对大的时延和多普勒扩展的时间和频率双选择性衰落信道,最精确的近似是采用基扩展 BEM 模型。根据所用基向量的差异,常用的基扩展信道模型可以分为:采用复指数基函数来建模的 CE-BEM(Complex Exponential BEM)模型\cite{ref7},采用多项式作为基函数的 P-BEM (Polynomial BEM)\cite{ref8,ref9},采用椭圆基序列的 DPS-BEM(Discrete prolate sequences BEM)\cite{ref10},采用余弦变换模型的 DCT-BEM(Discrete Cosine Transform BEM)\cite{ref11}等。BEM 通过线性的合并确定的基函数来近似信道的抽头系数,选择合适的基函数,对于模型的精确性有至关重要的作用,尤其是在高的多普勒扩展情况下,如 DKL-BEM(Discrete Karhuen-Loeve BEM)\cite{ref12},DPS-BEM 提供了很好的近似,这两种模型利用信道的统计特性构造 BEM 基,即用信道自相关函数的特征向量作为基函数向量,而 CE-BEM 的近似度则较差。但以 CE-BEM 作为模型的频域信道响应时近似带状的,因此在信道估计中得到了广泛的应用,也有很多 CE-BEM 的修正模型来提高建模性能\cite{ref13}。

作为时频双选信道的最佳模型\cite{ref14},BEM 模型采用变化的基函数与不随时间变化的 BEM 系数来表征信道。与 AR 模型不同的是,BEM 模型可以描述一个传输块内的信道响应,而 AR 模型是根据符号更新的,这样基于 BEM 模型的信道估计问题就可以转化为线性参数估计的问题。

相比于其他估计算法,基于 BEM 模型的信道估计方法降低了算法的复杂度,特别是基于 CE-BEM 模型的信道估计。CE-BEM 信道模型的频域响应具有严格带状的特点,此结构的特殊性使得基于 CE-BEM 模型的信道估计技术在以往的研究中得到了广泛的关注\cite{ref15}。但是随着终端移动速度的提高,由于 CE-BEM 模型的边界频率溢出效应,模型误差增大,估计误差随之增大。

针对要解决的第三个问题,我们发现目前研究 QAM 调制解调技术的文献很多,于风云、张平就数字微波系统的全数字 QAM 调制解调方式,以 ADS 为仿真平台对 16QAM 的全数字调制与解调过程进行了仿真,理论上论述 QAM 调制的优劣\cite{ref16}。曾长华、杨震等提出了一种新颖的、基于软件无线电的 QAM 调制信号的同步算法\cite{ref17}。该算法采用载波相位补偿频率偏移思想,利用软件编程实现。这些研究大多是传统的理论分析与仿真。侯立正、李维英等,用 Verilog 硬件描述语言完成了 16QAM 调制系统主要部分的底层设计,并且在 FPGA 上成功实现了主要部分的设计\cite{ref18}。其中大部分研究主要应用于数字电视的情况,而且其研究的焦点侧重于解调算法的载波恢复或定时恢复部分,但对 QAM 调制解调的整个系统进行完整的论述的文章较少,大多数文章停留在仿真层面,在 FPGA、CPLD 上硬件基于 FPGA 的 16QAM 调制解调系统的研究实现的文章不多。而且大多只实现系统的某个环节,将整个系统在 FPGA 开发板上实现的还没有。

\section{问题描述}

根据题目的要求,本文需完成以下三方面的工作:

1. 根据给出的某信道测试参数,建立数学模型,在保持一定的准确度的情况下,把测试数据中的部分数据通过所建模型计算获得,从而减少实际数据的测试量。用图表方式展示原始数据与计算结果的误差,并分析模型所用算法的复杂度;

2. 根据信道在不同速度 $90\mathrm{Km/h}$、$270\mathrm{Km/h}$、$450\mathrm{Km/h}$ 时的测试数据(信道采样频率是 $200\mathrm{KHz}$)。对这些数据进行分析,探索运动速度对所建模型准确度影响的规律。描述信道建模的过程,并利用所建信道模型产生的仿真数据,验证所建模型在减少测试数据方面的效果;

3. 根据实际信道受噪声影响的情况,对涉及的信道增加 AWGN 噪声,SNR 的取值参考范围从 0 到 $40\mathrm{dB}$。自行定义任意输入信号,进行数字调制及解调,信道参数采用前面所建减少信道数据测试频度的模型,分析 SNR 与 BER 之间的关系。

\section{专业术语及符号说明}

\subsection{专业术语}

\begin{itemize}
    \item \textbf{AWGN}
    \begin{itemize}
        \item 加性高斯白噪声 AWGN(Additive White Gaussian Noise)是最基本的噪声与干扰模型,它的幅度分布服从高斯分布,而功率谱密度服从均匀分布。
    \end{itemize}
    \item \textbf{BEM}
    \begin{itemize}
        \item 基扩展模型 BEM(Basis Expansion Model)是利用有限个基函数的线性组合来描述一定时间内的时变信道,可以模拟有多普勒效应的快时变信道,减少信道参数直接估计的次数。
    \end{itemize}
    \item \textbf{OFDM}
    \begin{itemize}
        \item 正交频分复用 OFDM(Orthogonal Frequency Division Multiplexing)是 MCM Multi-Carrier Modulation 多载波调制的一种,是将信道分成若干正交子信道,将高速数据信号转换成并行的低速子数据流,调制到在每个子信道上进行传输。
    \end{itemize}
    \item \textbf{QAM}
    \begin{itemize}
        \item 正交振幅调制技术 QAM(Quadrature Amplitude Modulation)就是用两个调制信号对频率相同、相位正交的两个载波进行调幅,然后将已调信号加在一起进行传输或发射。
    \end{itemize}
    \item \textbf{NMSE}
    \begin{itemize}
        \item 归一化均方误差 (Normalized Mean Square Error, NMSE) 表征计算精度,其表达式为:
        \begin{align}
Y &= F y \\
&= F \left( \sum_{m=0}^{M-1} P_m F^H \widetilde{D}_m F \right) \widetilde{x} + W \\
&= F \left( \sum_{m=0}^{M-1} P_m F^H \widetilde{D}_m \right) F F^H D_0 F x + W \\
&= F \left( \sum_{m=0}^{M-1} P_m F^H \widetilde{D}_m \right) D_0 A + W
\tag{6-12}
\end{align}
    \end{itemize}
    \item \textbf{瑞利分布}
    \begin{itemize}
        \item 瑞利分布(Rayleigh distribution)当一个随机二维向量的两个分量呈独立的、有着相同的方差的正态分布时,这个向量的模呈瑞利分布。瑞利分布是最常见的用于描述平坦衰落信号接收包络或独立多径分量接受包络统计时变特性的一种分布类型。两个正交高斯噪声信号之和的包络服从瑞利分布。瑞利分布的概率密度函数是:
        \begin{equation}
            f(x, \sigma) = \frac{x}{\sigma^2} e^{-x^2 / 2\sigma^2}, \quad x \geq 0
            \tag{3-2}
        \end{equation}
    \end{itemize}
\end{itemize}

16QAM: 一种二进制数据转为复数的方法

\subsection{符号说明}

\begin{itemize}
    \item 信道冲击响应: $h_i[n]$
    \item 采样周期: $T_s$
    \item 基函数: $B_m$
    \item 多普勒频移: $f_d$
    \item 载波频率: $f_c$
    \item Doppler 系数: $c_{i,n}$
    \item 离散 Doppler 频率: $f_{i,n}$
    \item Doppler 相位: $\theta_{i,n}$
\end{itemize}

\section{问题一}

\subsection{问题分析}

数据文件 1 给出了该信道的测试参数,共有 9 组数据,分别来自不同路径。题目的要求是建立数学模型,在保持一定的准确度的情况下,把测试数据中的部分数据通过所建模型计算获得,从而减少实际数据的测试量。最后,用图表方式展示原始数据与计算结果的误差,并分析模型所用算法的复杂度。我们的思路是采用基扩展 BEM 模型,利用有限个基函数的线性组合来描述一定时间内的时变信道,模拟有多普勒效应的快时变信道,减少信道参数直接估计的次数,所以问题就转化为基函数的选择以及如何对数据进行分段处理。

\subsection{模型建立}

\subsubsection{基函数选择}

基扩展 BEM 的数学模型为 \cite{ref19}:

\begin{equation}
h_l[n] = \sum_{m=0}^{M-1} b_{lm} B_m[n], \qquad l=0,\cdots,L-1
\tag{4-1}
\end{equation}

式中 $b_{lm}$ 是第 $l$ 个路径第 $m$ 个基系数,在一定时间周期 $T$ 内不随时间 $n$ 变化,$B_m$ 是第 $m$ 个基函数矢量,变量是时间 $n$,通过上式,把时变量 $h_l[n]$ 转化为一定时间周期 $T$ 内非时变量 $b_{lm}$ 和另一时变量 $B_m[n]$(是时间 $n$ 的函数,但函数形式不变)的表达式,即在 $T$ 内估计一次 $b_{lm}$ 即可实现对快时变信道参数 $h_l[n]$ 的估算。

式 (4-1) 是用 BEM 模型描述时变信道的通用形式,可以选择不同的基函数建立不同的基扩展模型。下面分别就几种基函数进行讨论。

\begin{itemize}
    \item 复指数 BEM 模型 (CE-BEM)
\end{itemize}

复指数 BEM (CE-BEM) 采用傅立叶基来作为基函数,是比较常用的一类的 BEM,此时的基函数为

\begin{equation}
B_m[n] = e^{j\omega_m n}
\tag{4-2}
\end{equation}

其中,$\omega_m = 2\pi(m-M/2)/N$。因此,信道的冲击响应可以表示为:

\begin{equation}
h_l[n] = \sum_{m=0}^{M-1} b_{lm} e^{j\omega_m n}, \quad l=0,\ldots,L-1
\tag{4-3}
\end{equation}

CE-BEM 优点是具有简单的基函数,并且用 CE-BEM 模型描述的信道在频域的响应是近似带状的,这种结构在信道估计的时候有非常重要的意义,很多算法都是利用 CE-BEM 模型的结构特殊性的有点。当然此模型的缺点也比较明显,由于在建模时分辨率达不到要求导致建模时的误差较大,从而导致信道的

频谱泄漏,引起吉布斯效应。

\begin{itemize}
    \item 离散卡-洛基函数 BEM 模型(DKL-BEM)
\end{itemize}

DKL-BEM 模型是在均方误差(Minimum Mean Square Error, MMSE)准则下的最优模型,要想得到 DKL-BEM 模型的基函数,首先要构造一个核矩阵 $C$,矩阵 $C$ 是任一个信道抽头的归一化协方差矩阵。

\begin{equation}
C_{l} = E\left\{h_{l}h_{l}^{H}\right\}/\sigma_{l}^{2}
\tag{4-4}
\end{equation}

基函数 $B_{m}$ 是矩阵 $C$ 的第 $m$ 个特征向量,其中特征值按照从大到小的顺序排列。

\begin{equation}
CB_{m} = \lambda_{m}B_{m}
\tag{4-5}
\end{equation}

DKL 模型实际上是多普勒谱的降秩分解,对 $C$ 进行奇异值分解,得到

\begin{equation}
C_{l} = U_{l}\Lambda_{l}V_{l}
\tag{4-6}
\end{equation}

\begin{equation}
\Lambda_{l} =
\begin{bmatrix}
\lambda_{0,l} & & & \\
& \lambda_{1,l} & & \\
& & \ddots & \\
& & & \lambda_{N,l}
\end{bmatrix}
\tag{4-7}
\end{equation}

其中,$\Lambda_{l}$ 是由矩阵 $C_{l}$ 的特征值按照从大到小的顺序组成的对角阵,$U_{l}$ 为各特征值对应的特征向量所组成的矩阵,基函数矩阵为

\begin{equation}
B = U_{l}(:,1:(M+1))
\tag{4-8}
\end{equation}

在使用 DKL-BEM 信道建模时必须已知信道相关矩阵,也就是需要知道信道的统计特性。Jakes 模型为最常用的假设,可知其相关函数为

\begin{equation}
E\{h(n,l)h^{*}(n-x,l)\} = \sigma_{l}^{2}J_{0}(2\pi f_{\max}t_{s})
\tag{4-9}
\end{equation}

其中 $\sigma_{l}^{2}$ 为第 $l$ 条路径的方差,$J_{0}(\cdot)$ 为零阶贝塞尔函数。

\begin{itemize}
    \item 椭圆基函数 BEM 模型(DPS-BEM)
\end{itemize}

根据上面的介绍,DKL-BEM 模型在均方误差意义上是最优的,但这种方法需要信道的多普勒功率谱满足特定的形状,如果实际信道不能满足此条件时,则会引起性能的降低。DPS-BEM 近似适用于所有信道特性,由 Zemen 和 Mecklenbrauker 于 2005 年首次提出,它采用矩形功率谱来达到次优的性能,其 $N \times N$ 阶矩阵 $C$ 为:

\begin{equation}
[C]_{n,x} = \frac{\sin(2\pi(n-x)f_{\max}t_{s})}{\pi(n-x)}
\tag{4-10}
\end{equation}

基函数 $B_{m}$ 同样满足式(4-5),DPS 序列是一系列正交的球函数,有很强的能量集中性。利用 DPS-BEM 模型来描述时变信道,无论是在多普勒频移较高或者是多普勒频移低的情况下,都有比较好的性能。

\begin{itemize}
    \item 多项式 BEM 模型(P-BEM)
\end{itemize}

多项式 BEM(P-BEM) 由泰勒级数展开并近似得到,其基函数为

\[
B_{m}[n] = (n - N/2)^{m}
\tag{4-11}
\]

式(3-4) 介绍的 P-BEM 模型,采用单一的多项式,在信道变化比较平缓的时候模型误差较小。P-BEM 模型对多普勒扩展比较敏感,在高的多普勒扩展下性能下降比较明显。后面会介绍到 Legendre-BEM 模型,这种模型也属于 P-BEM 模型的范畴,利用 Legendre 多项式来描述信道,相比于单一多项式 P-BEM 来说,性能更好。

在 Legendre-BEM 模型中,基函数为

\[
B_{m}[n] = B_{m}(nT_{s})
\tag{4-12}
\]

其中,$T_{s}$ 是一个 OFDM 符号周期(不包括循环前缀)。

\[
B_{m}(t) = P_{m}(2t/NT_{s} - 1)
\tag{4-13}
\]

$P_{m}(t)$ 是 $m$ 阶 Legendre 多项式

\[
P_{m}(x) = \frac{1}{2^{m}m!} \frac{d^{m}}{dx^{m}}[(x^{2} - 1)^{m}]
\tag{4-14}
\]

$m = 0, 1, 2$ 时,$m$ 阶 Legendre 多项式如下所示:

\[
P_{0}(x) = 1
\tag{4-15}
\]

\[
P_{1}(x) = x
\tag{4-16}
\]

\[
P_{2}(x) = \frac{3}{2}x^{2} - \frac{1}{2}
\tag{4-17}
\]

此外,我们还创新性地尝试了一种新的基函数模型,即 Chebyshev 多项式。基函数为:

\[
B_{m}(t) = T_{m}(\frac{2t}{NT_{s}} - 1)
\tag{4-18}
\]

其中,$T_{m}(t)$ 是 $m$ 阶 Chebyshev 多项式。

\[
T_{0}(x) = 1
\tag{4-19}
\]

\begin{equation}
T_{1}(x)=x \tag{4-20}
\end{equation}

\begin{equation}
T_{m+1}(x)=2xT_{m}(x)-T_{m-1}(x),(n\geq1) \tag{4-21}
\end{equation}

在后面的计算中,分别代入这几种基函数,并进行对比,以找到最优的基函数模型。

\subsubsection{多普勒效应}

移动台与基站间的相对运动会引起频率的偏移,即多普勒频移。许多散射的路径来自不同的方向,导致了不同的多普勒频移,形成多普勒扩展。多普勒频移与移动台的移动速度 \(v\),运动方向与接收机多径波的入射角 \(\theta\) 有关。多普勒频移 \(f_{d}\) 示为如下形式:

\begin{equation}
f_{d}(t)=\frac{v}{\lambda}\cos\theta=\frac{vf_{c}}{c}\cos\theta(t),(n\geq1) \tag{4-22}
\end{equation}

其中,\(c\) 为光速,\(f_{c}\) 为载波频率。多普勒频移随着角度 \(\theta\) 的变化而发生变化,对于一定的速度 \(v\),定义最大的多普勒频移为:

\begin{equation}
f_{\text{max}}=\frac{vf_{c}}{c} \tag{4-23}
\end{equation}

\subsubsection{数据分段处理}

数据 1 给出了 9 个不同路径的信号数据,分组处理的实质就是在系统中插入导频。而导频的插入要考虑一下两个方面,意识导频见的间隔要尽可能小,以保证估计出的信道响应能够很好地对抗信道的时间选择性衰落以及频率选择性衰落,而是由于导频出的时频资源不能传输有用数据信息,应尽量减少导频数量来减小系统开销,提高资源利用率 \({}^{[19-20]}\)。一般采用二倍过采样,其时域的采样间隔为:

\begin{equation}
N\leq\frac{1}{2f_{\text{max}}T} \tag{4-24}
\end{equation}

其中,\(f_{\text{max}}\) 为最大多普勒频移,\(T\) 为信道采用周期。

\subsection{实例求解}

由题目给出的测试参数信息,运动速度 \(180\text{Km/h}\),载波频率 \(3\text{GHz}\),信道采样频率 \(200\text{KHz}\),由式 (4-23) 与 (4-24) 可得:

\begin{equation}
N\leq\frac{1}{2\frac{vf_{c}}{c}T}=\frac{c}{2vf_{c}T}=\frac{3\times10^{8}}{2\times180/3.6\times3\times10^{9}\times\frac{1}{200\times10^{3}}}=200 \tag{4-25}
\end{equation}

则有信道数据在 200 个点以内可以认为信道参数 \(b\) 为常值。实际计算取值 \(N=140\);用于辨识该段信道参数的点数 \(n=36\);基函数矩阵 \(B\) 的计算阶次

$M=6$。分别比较采用复指数 BEM 模型和多项式 BEM 模型(包括 P-BEM,Legendre-BEM 和 Chebyshev-BEM 三种)对信道参数进行估计,并分别进行分段和整体的估计信道误差 (NSME) 进行对比。

\begin{figure}[h]
    \centering
    \includegraphics[width=\textwidth]{image.png}
    \caption{单段估计各 BEM 模型第一径道的实部的信道参数估计对比(横轴代表时间/s)}
    \label{fig:4-1}
\end{figure}

\begin{table}[h]
    \centering
    \caption{各 BEM 模型归一化均方误差 (NMSE) 对比}
    \label{tab:4-1}
    \begin{tabular}{|c|c|c|c|c|}
    \hline
    NSME & 复指数 BEM & P-BEM & L-BEM & C-BEM \\ \hline
    单段 & 2.39105E-02 & 2.77569E+03 & 4.32686E-05 & 4.32614E-05 \\ \hline
    整体 & 3.42214E-02 & 2.65921E+03 & 6.29667E-05 & 6.29627E-05 \\ \hline
    \end{tabular}
\end{table}

\begin{figure}[h]
    \centering
    \includegraphics[width=\textwidth]{image.png}
    \caption{整体估计各 BEM 模型第一径道的实部的信道参数估计对比 (横轴代表时间/s)}
    \label{fig:4-2}
\end{figure}

各 BEM 模型计算结果如图 4-1、图 4-2(横轴代表时间序列) 和表 4-1 所示。由单段估计图 4-1 和总体估计图 4-2 可以看出,在确定单段长度 \( N \)、辨识点数 \( n \) 和基函数的阶次的条件下,复指数 BEM 和 P-BEM 两组模型在超出辨识点区域后的估计值段迅速偏离实际信道值;而两组正交多项式 (Legendre 和 Chebyshev) 为基的信道估计值则与实际值较好的吻合。由 NMSE 的对比表 4-1 可知,复指数-BEM 和 P-BEM 两组模型的 NMSE 值很大,对信道估计偏离较远;L-BEM 和 C-BEM 两组模型的 NMSE 值很小,故能对原信道进行有效的估计。比较各组的单段估计与整体估计的结果可知,单段估计与整体估计的 NMSE 相差很小,即为整体估计的各分段存在着相近的 NMSE。因此,可以用单段信道估计结果作为整体信道估计的近似值。

进一步考虑辨识点数 \( n \) 和基函数 \( M \) 的阶次对估算结果的影响。由于在相同条件下 Legendre-BEM 信道估计方法对信道估计结果较好 (Chebyshev-BEM 结果较之稍好,但相差不大),这里采用 Legendre-BEM 方法来研究辨识点数 \( n \) 和基函数阶次 \( M \) 对信道辨识结果的影响。取 \( N=160 \),\( n \) 占总点数 \( N \) 的比例分别为
\[
\left[\frac{1}{2} \quad \frac{1}{4} \quad \frac{1}{8} \quad \frac{1}{10} \quad \frac{1}{16}\right],
\]
基函数阶次 \( M \) 分别为 3、5、7、13、15、19 时,L-BEM

模型归一化均方误差(NMSE)的单段与整体估计结果比较。

\begin{figure}[h]
    \centering
    \includegraphics[width=\textwidth]{image.png}
    \caption{$R_{n}=\frac{1}{8}$时,单段估计第一径道的实部的信道参数估计对比(横轴代表点的序列)}
    \label{fig:4-3}
\end{figure}

\begin{figure}[h]
    \centering
    \includegraphics[width=\textwidth]{image.png}
    \caption{$M=5$ 时,单段估计第一径道的实部的信道估计对比(横轴代表点的序列)}
    \label{fig:4-4}
\end{figure}

考虑在 $n$ 占总点数 $N$ 的比例 $R_{n}$ 和基函数阶次 $M$ 的不同组合下单段信道的结算结果如图 \ref{fig:4-3}、图 \ref{fig:4-4} 和表 4-2 所示。由图 \ref{fig:4-3} 可以看出,随着 $M$ 的增大,估计信道曲线与实际信道曲线并不是随之更加贴合(在前面的辨识段,贴合程度增大,但在估计段则迅速偏离);其主要原因是估计段没有了约束点束缚,出现高次震荡。由图 \ref{fig:4-4} 可以看出,随着 $R_{n}$ 的减小,估计信道曲线与实际信道曲线

\begin{table}
\centering
\begin{tabular}{c|c|c|c|c|c|c|c}
\hline
 & \multicolumn{2}{c|}{NMSE} & \multicolumn{5}{c}{辨识点数$n$比例$R_{n}$} \\
\hline
 & \multicolumn{2}{c|}{} & 1/2 & 1/4 & 1/8 & 1/10 & 1/16 \\
\hline
\multirow{6}{*}{单段} & \multirow{12}{*}{$M$} & 3 & 1.64396E-02 & 4.68910E-02 & 6.99690E-02 & 7.52857E-02 & 8.37362E-02 \\
\cline{3-8}
 & & 5 & 2.56925E-04 & 1.53886E-03 & 3.13739E-03 & 3.58169E-03 & 4.41011E-03 \\
\cline{3-8}
 & & 7 & 1.09292E-06 & 1.43638E-05 & 2.04364E-01 & 5.82772E+00 & 8.69098E-01 \\
\cline{3-8}
 & & 13 & 2.88192E-01 & 4.65518E+01 & 4.65178E+00 & 3.98783E-01 & 4.83222E-01 \\
\cline{3-8}
 & & 15 & 2.54769E-01 & 1.84095E-01 & 4.89405E+01 & 4.49973E-01 & 5.32512E-01 \\
\cline{3-8}
 & & 19 & 1.00826E+02 & 2.71829E-01 & 5.71706E+01 & 2.50149E+00 & 8.67797E-01 \\
\hline
\multirow{6}{*}{整体} & & 3 & 3.00739E-02 & 1.02956E-01 & 1.69439E-01 & 1.85866E-01 & 2.12693E-01 \\
\cline{3-8}
 & & 5 & 4.40783E-04 & 3.10964E-03 & 6.89664E-03 & 8.00049E-03 & 1.01414E-02 \\
\cline{3-8}
 & & 7 & 1.79602E-06 & 2.79173E-05 & 2.31055E-01 & 5.43397E+00 & 1.01595E+00 \\
\cline{3-8}
 & & 13 & 3.98748E-01 & 6.52913E+01 & 5.26362E+00 & 5.45588E-01 & 6.68586E-01 \\
\cline{3-8}
 & & 15 & 3.39955E-01 & 2.66500E-01 & 5.63357E+01 & 6.83588E-01 & 6.83672E-01 \\
\cline{3-8}
 & & 19 & 8.15618E+01 & 3.24641E-01 & 5.65113E+01 & 3.14243E+00 & 1.02201E+00 \\
\hline
\end{tabular}
\caption{归一化均方误差(NMSE) 随辨识点数$n$比例和基函数阶次$M$的变化}
\end{table}

\begin{figure}[h]
\centering
\includegraphics[width=\textwidth]{image1.png}
\caption{归一化均方误差(NMSE) 随辨识点数$n$比例和基函数阶次$M$的变化 (R1-R5分别代表$R_{n}$为1/2, 1/4, 1/8, 1/10, 1/16)}
\end{figure}

以下比较各算法进行计算复杂度的对比。主要考虑乘法在各算法的比较,由于各多项式的方法的乘法次数主要与多项式的阶次相关,故P-BEM、L-BEM和C-BEM三种方法的计算量相当。这里考虑L-BEM模型的乘法次数进行计算。

复杂度估算。

由式(4-1)可以看出,对信道参数进行估计,主要可以分为两个部分:(1) 通过不同的基函数形成矩阵 $\boldsymbol{B}_{M \times N}$,这也是不同基函数选择造成复杂不同的主要差异,(2) 及对式子进行最小二乘求解得出 $\boldsymbol{b}_{L \times M}$。对于 L-BEM 情况,形成 $\boldsymbol{B}_{M \times N}$ 大致需要 $M^2N$ 次乘法。由于信道的各路径相互独立,故可以分开计算对应通道的 $\boldsymbol{b}_{l \times M}$,采用最小二乘法求得伪逆需要 $\frac{1}{3}M^3 + M^2 - \frac{M}{3} + 2M^2N$ 次乘法(考虑高斯求逆过程),各通道求得 $\boldsymbol{b}_{L \times M}$ 需要 $MNL$ 次乘法。

故信道估计的计算过程乘法总的次数约为 $\frac{1}{3}M^3 + M^2 - \frac{M}{3} + 3M^2N + MNL$ 次。可以看出由于基函数的选择对 $M$ 的大小影响很大,而 $M$ 的阶次与乘法次数成三次关系。一般 $M$ 的取值较小(4~6 次即可满足),而 $N$ 的取值较大,$M$ 的增大会显著增加计算乘法次数。

\subsection{本章小结}

本章从题干出发,充分挖掘题目中所含信息,通过对比分析找到相对较好的预估算法模型。首先,在查阅文献的基础上给出了 BEM 建模的推导过程,并详细介绍了多种常用的 BEM 基函数,Legendre-BEM 模型的启发创新性地提出了 Chebyshev-BEM 模型;其次,分别对各种 BEM 模型预估仿真分析,将复指数 BEM 模型、一般多项式 BEM 模型、Legendre-BEM 模型以及 Chebyshev-BEM 模型的仿真结果进行对比,发现 Legendre-BEM 模型与 Chebyshev-BEM 模型对信道估计结果较好,并且当模型参数 $M=6$ 时,估计精度较高;最后,以 Legendre-BEM 模型为例计算了该算法的复杂度,并得出 $M$ 的增大会显著增加复杂度的结论。

\section{问题二}

\subsection{问题分析}

对于问题二的前半部分,在 4.2.2 节,我们讨论了多普勒频移的具体形式,数据文件 2、3、4 分别是载波频率为 3GHz 时,信道在不同速度 $90\mathrm{Km/h}$、$270\mathrm{Km/h}$、$450\mathrm{Km/h}$ 时的测试数据。在问题一的理论基础上变更速度大小来探究速度对所建模型准确性的影响。对于问题二的后半部分,则需要建立新的信道模型,来产生多径衰落信道相互独立,幅度服从瑞利 (Rayleigh) 分布,相位服从均匀分布的信道数据。本节利用改进的 Zheng 模型产生多个不相关 Rayleigh 波形。并利用所建信道模型产生的仿真数据,验证问题一所建的模型在减少测试数据方面的效果。

\subsection{速度变更}

由题目分析可知,当信道处于的速度增大时,多普勒效应的影响逐渐增大,使得信道从慢衰落信道转换为快时变信道,需要更加增加插入导频的频率。由问题一分析可知,速度增大导致最大多普勒频移 $f_{\mathrm{max}}$ 增大,进而使得信道的时域间隔 $N$ 减小。分别带入题目中的三组速度,可得不同速度下的时域间隔 $N$ 的上限取值,如表 5-1 所示。

\begin{table}[h]
\centering
\caption{不同速度下的时域间隔 $N$}
\begin{tabular}{|c|c|c|c|}
\hline
$V/(\mathrm{Km/h})$ & 90 & 270 & 450 \\
\hline
$N$ & 400 & 133 & 80 \\
\hline
\end{tabular}
\end{table}

为方便比较不同速度对第一问中模型准确度(由归一化均方误差 NMSE 表示)的影响,对三个不同速度取相同的时间间隔 $N=64$,用于信道估算的点为 $n=16$。由此计算得出三个速度下,进行单段信道和整体信道估计的 NSME 值,如表 5-2 所示。

\begin{table}[h]
\centering
\caption{不同速度下的归一化均方误差 NMSE}
\begin{tabular}{|c|c|c|c|}
\hline
$V/(\mathrm{Km/h})$ & 90 & 270 & 450 \\
\hline
单段 & $1.13357 \mathrm{E}-12$ & $5.41020 \mathrm{E}-07$ & $2.15959 \mathrm{E}-04$ \\
\hline
整体 & $1.04337 \mathrm{E}-12$ & $7.43613 \mathrm{E}-07$ & $3.26984 \mathrm{E}-04$ \\
\hline
\end{tabular}
\end{table}

\begin{figure}[h]
    \centering
    \includegraphics[width=\textwidth]{image.png}
    \caption{在不同速度下的单段信道估计对比(R 为实部图像, Z 为幅值图像)由图 5-1 和表 5-2 可得,随着速度的增大,信道估计值与实际值的贴合度逐渐下降;归一化均方误差 NMSE 迅速增大。}
    \label{fig:5-1}
\end{figure}

\subsection{模型建立}

Rayleigh 衰落信道是无线信道仿真的基础,对于单个平坦 Rayleigh 衰落信道,仿真方法可以分为白噪声滤波法和正弦波叠加法(SoS),其中滤波法根据成形滤波器的不同又分为 FIR 滤波法和 AR 滤波法。罗志年等先用统一的表达式把基于正弦波叠加法的平坦 Rayleigh 仿真模型分为四大类,对各种类型性能进

进行了的研究比较,最后根据分析提出了一种高效的多个独立 Rayleigh 波形的仿真模型。

\subsubsection{单信道正弦波叠加 Rayleigh 仿真方法}

基于 SoS 的 Rayleigh 信道仿真模型是根据 Clarke 散射信道模型,通过有限数目的正弦波叠加实现。如式(5-1)所示:

\begin{equation}
\widetilde{c}(t) = \widetilde{c}_1(t) + j\widetilde{c}_2(t)
\tag{5-1a}
\end{equation}

\begin{equation}
\widetilde{c}_i(t) = \sum_{n=1}^N c_{i,n} \cos(2\pi f_{i,n} t + \theta_{i,n}), \quad i=1,2
\tag{5-1b}
\end{equation}

其中,$c_{i,n}$、$f_{i,n}$ 和 $\theta_{i,n}$ 分别称为 Doppler 系数、离散 Doppler 频率和 Doppler 相位;$\widetilde{c}_1(t)$、$\widetilde{c}_2(t)$ 分别为同相分量和正交分量;$N$ 为正弦波的数目。

对于给定的正弦波数目 $N$,正弦波叠加仿真方法取决于三个参数,即幅度、Doppler 频率和相位。由传播理论可知,到达波的相位在 $[0, 2\pi)$ 之间均匀分布,对于散射环境来说,由于散射能量很小,不存在强度很大的来波分量,接收信号是所有散射波的矢量叠加,因此信道的衰落主要是由相位的随机特性决定的。为此,我们把基于正弦波叠加的 Rayleigh 平坦衰落仿真模型分为四大类:

I. $c_{i,n}$、$f_{i,n}$ 和 $\theta_{i,n}$ 都是确定变量;

II. $c_{i,n}$ 和 $f_{i,n}$ 是确定变量,$\theta_{i,n}$ 是随机变量;

III. $c_{i,n}$ 是确定变量,$f_{i,n}$ 和 $\theta_{i,n}$ 都是随机变量;

IV. $c_{i,n}$、$f_{i,n}$ 和 $\theta_{i,n}$ 都是随机变量。

除了类型 I,其他 3 类都至少存在一个随机变量,后 3 种类型的实部和虚部自相关与互相关统计特性可以表示为:

\begin{equation}
\widetilde{R}_{c_ic_i}(\Delta t) = E\left\{\sum_{n=1}^N \sum_{m=1}^N c_{i,n} c_{i,m} \cos(2\pi f_{i,m} t + \theta_{i,m}) \cos[2\pi f_{i,n} (t+\Delta t) + \theta_{i,n}]\right\}
\tag{5-2a}
\end{equation}

\begin{equation}
\widetilde{R}_{c_1c_2}(\Delta t) = E\left\{\sum_{n=1}^N \sum_{m=1}^N c_{1,n} c_{2,m} \cos(2\pi f_{2,m} t + \theta_{2,m}) \cos[2\pi f_{1,n} (t+\Delta t) + \theta_{1,n}]\right\}
\tag{5-2b}
\end{equation}

其中,随机相位 $\theta_{i,n}$ 均匀分布于 $[0, 2\pi)$,对于二阶相关特性具有决定性作用,随机相位有三种定义方式:

方式 1:$\theta_{1,n}$、$\theta_{2,n}$ 相互独立,共有 $2N$ 个独立的随机相位。

方式 2:$\theta_{1,n} = \theta_{2,n} = \theta_n$,共有 $N$ 个独立的随机相位。

方式 3:$\theta_{1,n} = \theta_{2,n} = \theta$,只有 1 个独立的随机相位。

类型 I 是一个确定过程,其实就是 Jakes 仿真器,它不能产生多个不相关 Rayleigh 衰落波形,研究类型 I 的平稳和各态历经性没有什么意义,但我们可以计算它的时间平均和时间相关等性质。

类型 III 具有固定幅度,随机 Doppler 频率和相位。类型 III 有三种典型的仿真模型,分别为 Hoehler 模型,相位定义为方式 1;Zheng 模型,相位定义为方式 1;Xiao 模型相位定义为方式 2,这里着重介绍 Zheng 模型。

\subsubsection{Zheng 模型}

\begin{align}
c_{i,n} &= \sigma_0 \sqrt{2/N} \tag{5-3a} \\
f_{1,n} &= f_d \cos\left(\frac{2\pi n - \pi + \phi}{4N}\right) \tag{5-3b} \\
f_{2,n} &= f_d \sin\left(\frac{2\pi n - \pi + \phi}{4N}\right) \tag{5-3c}
\end{align}

其中,$\phi$ 在 $(-\pi, \pi]$ 均匀分布。

\subsubsection{多个不相关 Rayleigh 波形生成}

对单信道仿真结果分析可以得到如下结论:

1) 类型 II 模型具有各态历经性,但当正弦波数 $N$ 为有限的时候,相关特性与参考模型有一定的偏离。

2) 类型 III 除了 MCM 方法,统计特性与参考模型的统计特性最接近,但不具有各态历经性。

3) 当周期足够长且正弦波数量足够大时,类型 IV 和类型 III 趋于一致。但在仿真中,$N$ 一般不会很大,所以类型 IV 不是一个很好的 Rayleigh 仿真模型。

综上所述,Zheng 模型和 Xiao 模型是高效的仿真模型,经过改进 Zheng 模型更适合于多个不相关 Rayleigh 波形的产生。设生成 $P$ 个不相关的波形,第 $k$ 个波形为 $\widetilde{c}_k(t) = \widetilde{c}_{k1}(t) + j\widetilde{c}_{k2}(t), k = 1, \cdots, P$,其中:

\begin{align}
\widetilde{c}_{k,i}(t) &= \sum_{n=1}^N c_{k,i,n} \cos(2\pi f_{k,i,n} t + \theta_{k,i,n}), i = 1, 2 \tag{5-4} \\
c_{k,i,n} &= \sigma_0 \sqrt{2/N} \\
f_{k,1,n} &= f_d \cos\left(\frac{2\pi n - \pi + \phi}{4N} + \frac{2\pi}{4PN}\right) \tag{5-5} \\
f_{k,2,n} &= f_d \sin\left(\frac{2\pi n - \pi + \phi}{4N} + \frac{2\pi}{4PN}\right)
\end{align}

这样我们得到了一个高效的可生成多个不相关 Rayleigh 波形的仿真模型。

\subsection{实例求解}

本文中实际采用改进的 Zheng 模型进行不相关 Rayleigh 衰落信道仿真模型 \cite{zheng2000} 进行信道的仿真。在式 (5-4) 和 (5-5) 中,设定随机相位 $\theta$ 和 $\phi$ 为均匀分布(在 Matlab 中可以使用 unifrnd 函数产生),则可以得出 $P$ 个不相关的 Rayleigh 波形,再将多个不同的波形叠加在不同信道多径上,则由 $\widetilde{c}_k(t) = \widetilde{c}_{k_1}(t) + j\widetilde{c}_{k_2}(t)$ 可得最终的仿真信道数据。

其中,信道仿真用到的参数分别设定为:信道采样频率 $200\mathrm{KHz}$,载波频率 $3\mathrm{GHz}$,不相关波形的数目为 $10$,信道的采样点数为 $100000$ 个。分别考虑信道多径数和多径延迟对信道的影响,其中多径延迟主要与信道的速度相关。以下分别考虑这两个因素变化的对信道参数的影响。

取速度 $V=180\mathrm{Km/h}$,采用 Legendre-BEM 模型对信道参数进行估计。单段信道长度 $N=100$(估算得出上限值为 $200$),导频信道长度 $n=20$,基函数的阶次 $M=6$。分别考虑信道多径数为 $4, 6, 8, 10, 12$ 时,信道估计的归一化均方误差 NMSE 的变化,如表 5-3 所示,可以看出对于单段和整体信道估计的 NMSE 均在 $8\sim10$ 径道时取到最大值。

\begin{table}[h]
\centering
\caption{不同多径数的归一化均方误差 NMSE 表}
\begin{tabular}{|c|c|c|c|c|c|}
\hline
$L$ & 4 & 6 & 8 & 10 & 12 \\
\hline
单段 & $1.19579\mathrm{E}-06$ & $4.10611\mathrm{E}-07$ & $2.65676\mathrm{E}-05$ & $1.76481\mathrm{E}-06$ & $3.89813\mathrm{E}-07$ \\
\hline
整体 & $1.70729\mathrm{E}-06$ & $1.76252\mathrm{E}-06$ & $1.76839\mathrm{E}-06$ & $1.77889\mathrm{E}-06$ & $1.76678\mathrm{E}-06$ \\
\hline
\end{tabular}
\end{table}

\begin{figure}[h]
\centering
\includegraphics[width=0.8\textwidth]{image.png}
\caption{不同多径数的归一化均方误差 NMSE 图取速度多径数 $L=9$,采用 Legendre-BEM 模型对信道参数进行估计。基函数的阶次 $M=6$,分别考虑速度为 $50\mathrm{Km/h}, 150\mathrm{Km/h}, 250\mathrm{Km/h}, 350\mathrm{Km/h}$ 和 $450\mathrm{Km/h}$ 下,信道估计的归一化均方误差 NMSE 的变化。以最高速 $450\mathrm{Km/h}$ 作为单段信道长度 $N=64$ 的取值上限(估算得出上限值为 $80$),导频信道长度 $n=16$。由此可得,信道估计的归一化均方误差 NMSE 的变化,如表 5-3 所示。可以看出,随着速度的线性增加,}
\end{figure}

\begin{table}[h]
\centering
\caption{不同速度下的归一化均方误差 NMSE 表}
\begin{tabular}{|c|c|c|c|c|c|}
\hline
$V/(Km/h)$ & 50 & 150 & 250 & 350 & 450 \\
\hline
单段 & $2.36741E-15$ & $2.12191E-09$ & $3.96545E-07$ & $2.52540E-05$ & $7.81177E-06$ \\
\hline
整体 & $9.47191E-16$ & $6.69309E-10$ & $2.98920E-07$ & $1.65733E-05$ & $3.32947E-04$ \\
\hline
\end{tabular}
\end{table}

\begin{figure}[h]
\centering
\includegraphics[width=0.8\textwidth]{image.png} % 替换为实际图像文件名
\caption{不同速度下的归一化均方误差 NMSE 图}
\end{figure}

图 5-3 不同速度下的归一化均方误差 NMSE 图综上,多径数目 $L$ 的变化将对信道估计产生一定的影响,并在取值为 8~10 时对信道估计产生的影响最大;速度变化对于信道估计的影响很大。$N$ 的取值越大,则导频插入的频率越低;导频所占的比例越低,则用于有效信号传输的比例增大。因此在确定速度下运动时,应尽量取到 $N$ 的极限值,并在满足一定信道估计精度的条件下减小导频所占 $N$ 的比例。

\subsection{本章小结}

本章主要分为两个部分,前一部分通过改变信道速度探究其对模型准确度影响的规律,发现随着速度的增大,信道估计值与实际值的贴合度逐渐下降,归一化均方误差 NMSE 迅速增大。后一部分根据题干要求,对特定的多径时变传输信道建模。首先,介绍了单信道正弦波叠加 Rayleigh 仿真方法,并引出较为典型的 Zheng 模型,在此基础上得到多个不相关 Rayleigh 波形生成方法;然后,利用改进的 Zheng 模型进行信道的仿真,并讨论了径数和多径延迟对信道的影响;最后,通过比较得出多径数 $L$ 取值为 8~10 时对信道估计产生的影响最大以及确定速度下单段信道长度 $N$ 越大有效信号传输的比例越大的结论。

\section{问题三}

\subsection{问题分析}

问题三提出了与前两问不同的新概念,即信号在传递过程中的调制与解调过程。以 16QAM 星座图的方法为例,并且要考虑实际信道受噪声影响,解决问题的思路是将信号调制成复数序列,通过信道传输,在接收端对信号进行过滤后解调为二进制序列。

\begin{figure}[h]
\centering
\begin{tikzpicture}[node distance=2cm, auto, >=latex']
    \node [block] (signal) {信号};
    \node [block, below of=signal] (qam_mod) {QAM 调制};
    \node [block, below of=qam_mod] (channel) {通过 信道};
    \node [block, right of=qam_mod, xshift=4cm] (add_carrier) {加入 导频};
    \node [block, right of=channel, xshift=4cm] (noise) {噪声};
    \node [block, right of=noise, xshift=4cm] (filter) {滤波};
    \node [block, right of=filter, xshift=4cm] (qam_demod) {QAM 解调};
    \node [block, above of=qam_demod] (recovered_signal) {恢复 信号};
    \node [block, below of=qam_demod] (received_signal) {接收 信号};

    \draw[->] (signal) -- (qam_mod);
    \draw[->] (qam_mod) -- (channel);
    \draw[->] (channel) -- (add_carrier);
    \draw[->] (add_carrier) -- (noise);
    \draw[->] (noise) -- (filter);
    \draw[->] (filter) -- (qam_demod);
    \draw[->] (qam_demod) -- (recovered_signal);
    \draw[->] (filter) -- (received_signal);
\end{tikzpicture}
\caption{QAM 信道传输流程图如图 6-2 所示,为进行信号传输过程的流程图。其主要可以分为以下几个部分:(1) 信号调制,包括信号的转制为复数列、载波、调制和加入导频;(2) 信号传输,包括信号通过信道、引入噪声和信号接收;(3) 信号处理,包括噪声污染信号的滤波、信道估计和解调过程。通过以上的步骤即可完成整个信号传输的过程。为简单计,考虑已知信道信息已知条件下,单段信号的传输。(由此可以省略加入导频、信道估计的过程)}
\end{figure}

\subsection{模型建立}

\subsubsection{调制与解调}

调制在通信系统中的作用至关重要。所谓调制,就是把信号转换成适合在信道中传输的形式的一种过程。广义的调制分为基带调制和载波调制。调制信号是指来自信源的消息信号,这些信号可以是模拟的,也可以是数字的。在无线传输中,信号是以电磁波的形式通过天线辐射到空间的。为了获得较高的辐射效率,天线的尺寸必须与发射信号波长相比拟,而基带信号包含的较低频率分量的波长较长,只是天线过长而难以实现。但若通过调制,把基带信号的频谱搬至较高的载波频率上,是已调信号的频谱与信道的带通特性相匹配,这样就可以提高传输性能,以较小的发送功率与较短的天线来辐射电磁波。

解调(也称检波)则是调制的逆过程,其作用是将已调信号中的调制信号

恢复出来。解调的方法可分为两类:相干解调和非相干解调(包络检波)。相干解调时,为了无失真地恢复原基带信号,接收端必须提供一个与接收的已调载波严格同步的本地载波。

\subsubsection{16QAM}

正交振幅调制(Quadrature Amplitude Modulation, QAM)是一种振幅和相位联合键控。在多进制键控体制中,相位键控的带宽和功率占用方面都具有优势,即带宽占用小和比特信噪比要求低。因此,MPSK 和 MDPSK 体制为人们所喜用。但是,在 MPSK 体制中,随着 \( M \) 的增大,相邻相位的距离逐渐减小,使噪声容限随之减小,误码率难于保证。为了改善在 \( M \) 大时的噪声容限,发展出了 QAM 体制。在 QAM 体制中,信号的振幅和相位作为两个独立的参量同时受到调制。这种信号的一个码元可以表示为

\begin{equation}
s_k(t) = A_k \cos(\omega_0 t + \theta_k), \quad kT < t \leq (k+1)T
\tag{6-1}
\end{equation}

其中,\( k \) 为整数;\( A_k \) 和 \( \theta_k \) 分别可以取多个离散值。式(6-1)可以转化成

\begin{equation}
s_k(t) = A_k \cos\theta_k \cos\omega_0 t - A_k \sin\theta_k \sin\omega_0 t
\tag{6-2}
\end{equation}

令 \( X_k = A_k \cos\theta_k \),\( Y_k = -A_k \sin\theta_k \),则式(6-1)变为

\begin{equation}
s_k(t) = X_k \cos\omega_0 t + Y_k \sin\omega_0 t
\tag{6-3}
\end{equation}

其中,\( X_k \) 和 \( Y_k \) 也是可以取多个离散值的变量。

QAM 是用两路独立的基带信号对两个相互正交的同频载波进行抑制载波双边带调幅,利用这种已调信号的频谱在同一带宽内的正交性,实现两路并行的数字信息的传输。该调制方式通常有二进制 QAM(4QAM)、四进制 QAM(16QAM)、八进制 QAM(64QAM)等,对应的空间信号矢量端点分布图称为星座图,分别有 4、16、64 个矢量端点。目前 QAM 最高已达到 1024QAM。样点数目越多,其传输效率越高。但并不是样点数目越多越好,随着样点数目的增加,QAM 系统的误码率会逐渐增大,所以在对可靠性要求较高的环境,不能使用较多点数目的 QAM。图 6-2 是 16QAM 编码图。

\begin{figure}[h]
    \centering
    \includegraphics[width=0.8\textwidth]{image1.png}
    \caption{16QAM 编码图}
    \label{fig:16QAM}
\end{figure}

16QAM 信号的产生方法主要有两种。第一种是正交调幅法,即用两路独立的正交 4ASK 信号叠加,形成 16QAM 信号,如图 \ref{fig:16QAM}(a) 所示。第二种方法是复合相移法,它用两路独立的 QPSK 信号叠加,形成 16QAM 信号,如图 \ref{fig:16QAM}(b) 所示。图中虚线大圆上的 4 个大黑点表示第一个 QPSK 信号矢量的位置。在这 4 个位置上可以叠加上第二个 QPSK 矢量,后者的位置用虚线小圆上的 4 个小黑点表示。

\begin{figure}[h]
    \centering
    \includegraphics[width=0.8\textwidth]{image2.png}
    \caption{正交调幅 16QAM (a) 和复合相移 16QAM (b)}
    \label{fig:16QAM_methods}
\end{figure}

\subsubsection{信道传输与求逆}

实际信道会受到噪声的影响,并且存在多径延迟、多径衰减增益等问题。这些问题在问题一和问题二中都有提及,此处不再赘述。接收信号要先进行噪声过滤,然后利用问题一中建立的模型进行求逆运算,得到所需的信号。

由式 (1-2) 和式 (4-1) 可得:
\begin{equation}
y[n] = \sum_{l=0}^{L-1} \left( \sum_{m=0}^{M-1} b_m B_m[n] \right) x[n-l] + w[n] \quad n=0,\dots,K-1
\tag{6-4}
\end{equation}

因此,信道变量 $H$ 可以表示为:

\begin{equation}
H = \sum_{m=0}^{M-1} P_m C_m
\tag{6-5}
\end{equation}

其中,$P_m$ 是对角矩阵,且 $P_m(i,i) = B_m(i)$,$C_m$ 是循环矩阵。

引入新变量:
\begin{equation}
\widetilde{x} = C_0 x
\tag{6-6}
\end{equation}

将其代入式 (1-1) 可得:
\begin{equation}
Y = H \cdot X + W = H C_0^{-1} C_0 x + w = H C_0^{-1} \widetilde{x} + w = \widetilde{H} \widetilde{x} + w
\tag{6-7}
\end{equation}

其中,
\begin{equation}
\widetilde{H} = H C_0^{-1}
\tag{6-8}
\end{equation}

将式 (6-5) 代入式 (6-8),可得:
\begin{equation}
\widetilde{H} = \sum_{m=0}^{M-1} P_m C_m C_0^{-1} = \sum_{m=0}^{M-1} P_m \widetilde{C}_m
\tag{6-9}
\end{equation}

其中,$C_m = F^H D_m F$,$(m=0,\cdots,M-1)$,$F$ 为离散傅里叶变换矩阵,$D_m$ 是对角矩阵。

\begin{equation}
\widetilde{x} = C_0 x = F^H D_0 F x
\tag{6-10}
\end{equation}

将式 (6-9) 代入式 (6-7):
\begin{equation}
y = \left( \sum_{m=0}^{M-1} P_m F^H \widetilde{D}_m F \right) \widetilde{x} + w
\tag{6-11}
\end{equation}

在频域里,式 (6-11) 有如下形式:
\begin{align}
Y &= F y \\
&= F \left( \sum_{m=0}^{M-1} P_m F^H \widetilde{D}_m F \right) \widetilde{x} + W \\
&= F \left( \sum_{m=0}^{M-1} P_m F^H \widetilde{D}_m \right) F F^H D_0 F x + W \\
&= F \left( \sum_{m=0}^{M-1} P_m F^H \widetilde{D}_m \right) D_0 A + W
\tag{6-12}
\end{align}

其中
\begin{equation}
A = F x = D_0^{-1} F \widetilde{x}
\tag{6-13}
\end{equation}

求解步骤:
\begin{enumerate}
    \item 先根据 $C_m = F^H D_m F$ 求出 $D_m$;
    \item 由 $D_m(i,i) = \left( F[b_m, 0, \ldots, 0]^T \right)(i)$ 求出 $\widetilde{D}_m$;
    \item 求解方程式 $y = \left( \sum_{m=0}^{M-1} P_m F^H \widetilde{D}_m F \right) \widetilde{x} + w$,求出 $\widetilde{x}$;
    \item 由 \( A = Fx = D_0^{-1} F \tilde{x} \) 求出 \( A \),进一步确定出 \( x \)。
\end{enumerate}

矩阵的求逆过程可以通过最小二乘法、奇异值分解的伪逆法等方法进行。

将以上的调制与解调,导频进行的信号估计,滤波等过程按照如图 6-1 的流程图串联起来,即可完成一次信号的传输与接收过程。

\subsection{实例求解}

采用本文的方法,通过信号产生、调制、通过信道、加入白噪声、滤波、解调等过程可得如下的计算结果,如图 6-4 和图 6-5 所示。可以看出随着 SNR 的增大,BER 逐渐减小。在图 6-5 中,在高噪声情况下 BER 均为零,故在对数图中无法显示。

\begin{figure}[h]
\centering
\includegraphics[width=\textwidth]{image1.png}
\caption{已调信号 (In: red, Qn: green)}
\end{figure}

\begin{figure}[h]
\centering
\includegraphics[width=\textwidth]{image2.png}
\caption{产生信号的调制、频率及星座图}
\end{figure}

\begin{figure}[h]
\centering
\includegraphics[width=\textwidth]{image3.png}
\caption{16QAM 在不同信噪比下的误码率}
\end{figure}

\subsection{本章小结}

对于问题三,题目提出信号在传递过程中的调制与解调过程,因此本章先勾勒了信号传输过程的流程图,分析了信号先调制成复数序列,通过信道传输,再在接收端对信号进行过滤后解调为二进制序列的过程。详细推导了基带滤波求逆运算的运算方法,随后运用问题 1 所建模型,具体阐述逆运算还原信号的过程,通过仿真发现可以较好地还原信号,并且有效地控制了算法的简洁度。

\section{总结及展望}

随着信息化进程的加快,人们对数据信息的交换需求促进了宽带业务迅猛发展,数据的高速传输更是未来无线通信的趋势。由于移动通信网络连接环境复杂多变,对实现高速宽带数据传递提出了更高的要求和挑战。分析现有通信模型的不足,建立新的数学模型,对提升信道容量、增加信息传输速率和降低误码率会有很好的促进作用。

本文的主要工作和创新如下:

\begin{enumerate}
    \item 主要工作有:分析处理某信道的测试数据以及不同速度条件引起信道参数的变化,采用运用广泛的基扩展方法,建立较为精确的快时变信道参数估计模型,并基于最小二乘法对模型参数进行求解,利用 NMSE 评价所建模型的准确度。此外,本文还对服从特定分布的多径时变传输信道建模,并分析了真实信号在信道传输过程中的调制解调过程;
    \item 在前面工作的基础上对基函数的形式进行了更深入的探究,由 Legendre-BEM 模型推广到 Chebyshev-BEM 模型,取得了很好的仿真效果。得出了 Legendre-BEM 模型和 Chebyshev-BEM 模型优于其他模型的结论;
    \item 在模型准确度和算法简洁度方面进行了大量对比试验分析,选出的模型具有较强的工程实用性。
\end{enumerate}

本课题的一些工作还可以继续深入开展,主要有以下几个方面的展望:

\begin{enumerate}
    \item 基扩展之外的建模方法也有很多值得借鉴的地方,由于时间限制,未对它们进行深入的发掘研究;
    \item 更大程度上改善算法,提高运行效率,增强算法的工程实用性;
    \item 寻找更为科学的判别算法复杂度的方法,能够快速判断模型的复杂程度,提高选择模型的效率。
\end{enumerate}

\section{八、参考文献}

\begin{enumerate}
    \item Tomasz Hrycak, etc. Low Complexity Equalization for Doubly Selective Channels Modeled by a Basis Expansion. IEEE Trans. Signal Processing, 2010, 58(11): 5706-5719.
    \item Saptarshi Das. Mathematical Methods for Wireless Channel Estimation and Equalization. Dissertation, University of Vienna, 2009.
    \item 吴伟凌等,移动通信原理(第 2 版),电子工业出版社,2009.1.
    \item B. Lindoff, C. Ostberg, H. Eriksson. Channel estimation for the WCDMA system, performance and robustness analysis from a terminal perspective. IEEE VTC, Sept. 1999, 2: 1565-1569.
    \item H. N. Lee and G. J. Pottle. Fast Adaptive Equalization/Diversity Combining For Time-Varying Dispersive Channels. IEEE Trans. On Communications, September, 1998, vol. 46, No. 9: 1146-1162.
    \item Che-Shen Yeh and Yinyi Lin. Channel Estimation Using Pilot Tones in OFDM Systems. IEEE Transactions on Broadcasting, Dec. 1999, 45(4): 400-409.
    \item Giannakis G B and Tepedelenlioglu C. Basis expansion models and diversity techniques for blind identification and equalization of time-varying channels. Proceedings of the IEEE. 1998, vol. 86, pp. 1969-1986.
    \item Borah D. K. and Hart B. D. Frequency-selective fading channel estimation with a polynomial time-varying channel model. IEEE Transactions on Communications. June 1999, vol. 47, pp. 862-873.
    \item Borah D. K. and Hart B. D. A robust receiver structure for time-varying, frequency flat Raleigh fading channels. IEEE Transactions on Communications. Mar. 1999, vol. 47, pp. 862-873.
    \item Zemen T and Mecklenbrauker C.F. Time-variant channel estimation using discrete prolate spherical sequences. IEEE Transactions on Signal Processing. Sept. 2005, vol. 53, pp. 3597-3607.
    \item Panayirci E, Senol. H, Poor. H.V. Joint Channel Estimation, Equalization, and Data Detection for OFDM Systems in the Presence of Very High Mobility [J]. IEEE Transactions on Signal Processing. 2010, vol. 58(8), pp. 4225-4238.
    \item Yip K.-W. and Ng T.-S. Karhunen-loeve expansion of the WSSUS channel output and its application to efficient simulation. IEEE Journal on Selected Areas in Communications, May 1997, vol. 15, pp. 640-646.
    \item 代光发,陈少平. 一种改进的快变信道展开模型[J]. 电子学报,2010, 38(07): 1500-1504.
    \item Elham Z., Omidi M.J., Faezeh M. and Somayeh M. Oversampled Legendre basis expansion model for doubly-selective channels [A]. 2011 19th Iranian Conference on Electrical Engineering (ICEE), Tehran, 2011: pp. 1-5.
    \item Ma X.L., Giannakis G B., and Ohno S. Optimal training for block transmissions over doubly selective channels. IEEE Transactions on Signal Processing. May 2003, vol. 51, no. 5, pp. 1351-1366.
\end{enumerate}

\begin{enumerate}
    \item[16] 于风云,张平. QAM 调制与解调的全数字实现. 现代电子技术,2005,(3): 53-55.
    \item[17] 曾长华,杨震,郑清. 一种新的基于软件无线电的 QAM 调制信号的同步算法. 2006,(4): 45-49.
    \item[18] 侯立正,李维英,王彩玉. 16-QAM 调制系统的 FPGA 实现. 2006,(12): 26-29.
    \item[19] 李昕. 基于基扩展快时变信道模型的 OFDM 系统信道估计[D]. 西南交通大学, 2013.DOI:10.7666/d.Y2319599.
    \item[20] 罗志年,张文军,管云峰. 不相关 Rayleigh 衰落信道仿真模型. 系统仿真学报. 2009, 21(13): 3872-3875.
\end{enumerate}

\section{九、致谢}

尊敬的评委老师:

您好!

非常感谢老师您百忙之中的悉心评阅,我们期待老师您的点拨,谢谢!您辛苦了!

\hfill 祝好!

\end{document}

% Missing placeholders restored
\begin{table}[h]
\centering
\begin{tabular}{l l}
学校 & 上海交通大学 \\
\hline
参赛队号 & 10248046 \\
\hline
队员姓名 & \\
1. & 王靖森 \\
2. & 刘涛 \\
3. & 张译文 \\
\hline
\end{tabular}
\end{table}