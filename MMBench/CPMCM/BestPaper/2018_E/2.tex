\begin{center}
\includegraphics[width=0.3\textwidth]{image1.png} \quad
\includegraphics[width=0.3\textwidth]{image2.png} \quad
\includegraphics[width=0.3\textwidth]{image3.png} \\
\includegraphics[width=0.3\textwidth]{image4.png} \quad
\includegraphics[width=0.3\textwidth]{image5.png} \quad
\includegraphics[width=0.3\textwidth]{image6.png}
\end{center}

\begin{center}
\textbf{“华为杯”第十五届中国研究生数学建模竞赛}
\end{center}

\begin{table}[h]
\centering
\begin{tabular}{l l}
学校 & 上海师范大学 \\
\hline
参赛队号 & 18102700005 \\
\hline
队员姓名 & 1. 武育宏 \\
 & 2. 谢菁华 \\
 & 3. 魏怿琛 \\
\end{tabular}
\end{table}

\begin{center}
\textbf{“华为杯”第十五届中国研究生数学建模竞赛}
\end{center}

\begin{center}
\textbf{题目} \quad 多无人机对组网雷达的协同干扰
\end{center}

\begin{center}
\textbf{摘 \quad 要:}
\end{center}

组网雷达系统相比单部雷达,具有空间覆盖范围更广、目标检测概率更高和抗干扰能力更强的特点,因而在军事特别是现代战争中得到广泛的应用。如何对组网雷达实施行之有效的干扰,已成为当前世界电子对抗界面临的一个重大问题。

本文主要考虑多架无人机对组网雷达系统的协同干扰问题,即采用适当的协同策略,通过协同控制多架无人机的飞行轨迹,在敌方组网雷达系统中形成一条或者多条虚拟目标航迹,从而达到欺骗的目的。根据无人机协同干扰组网雷达的工作机理,我们提出了合理假设,建立起多无人机协同干扰组网雷达的优化模型,给出了每架无人机的运动规律和协同策略,此外还分析了虚假航迹的运动规律和合理性。

针对问题一,按照题目给定的虚假目标航迹,无人机做匀速直线运动,航向、航速和飞行高度都在允许范围内,本文提出了一系列合理假设,通过对无人机飞行规律的深入分析,建立了多无人机协同干扰组网雷达的优化模型;然后利用枚举法对该模型进行求解;进而设计多无人机飞行方案以确保用最少架次的无人机实现附件 1 要求的虚假目标航迹,注意到,枚举法利用了计算机具有运算速度快、计算精确度高的特点,可考虑所有可能情况,从中找出符合要求的无人机运动规律,即可得到全局最优解。计算结果表明,安排 31 架无人机协同飞行,可以产生题目附件 1 所要求的虚假目标航迹。此外,我们还分析了每架无人机的运动规律和协同策略(见表 4-3)。相应的计算结果已经按照规定格式存入附件 2。

针对问题二,需要在实现问题一所要求的虚假轨迹条件下,优化无人机编队的飞行轨迹使得所形成的虚假航迹最多。在问题一的求解基础上,我们进一步分析了无人机的飞行规律,在保证转弯半径、飞行高度和速度都在合理范围内的情况下,考虑了 9 架无人机组成的编队在 5 分钟内的飞行情况,建立了协同航迹欺骗的最优控制模型。计算机仿真结果表明,按照该模型的协同策略,除了能产生附件 1 要求的虚假航迹外,还可另外产生 3 条虚假目标航迹。与此同时,我们还

\section*{分析了协同飞行的每架无人机的运动规律及其合理性。相应的计算结果已经按照规定格式存入附件 3。}

针对问题三,当组网雷达系统中的某部雷达受到压制或其他因素干扰时,我们采用题目给定的航迹维持策略,在满足无人机飞行高度和速度等限定条件下,基于问题二,重新建立了多无人机协同干扰组网雷达的目标分配模型,运用竞标算法,通过在雷达可探测跟踪的工作区域内选择目标,建立目标的竞标信息,无人机彼此之间实行信息传递,并完成目标任务的分配。我们的计算及分析结果表明,由 9 架无人机组成的编队在 5 分钟内,除了完成附件 1 要求的虚假航迹外,至多还可产生 9 条虚假航迹。此外,我们还发现该算法既可实现目标无冲突分配,也可保证无人机执行任务的效率。

\textbf{关键词:多无人机 组网雷达 优化模型 枚举法 竞标算法}

\section*{目录}

\section*{一、问题重述}
\subsection*{1.1 问题背景} \dotfill 4
\subsection*{1.2 需要解决的问题} \dotfill 5

\section*{二、模型假设} \dotfill 6

\section*{三、符号说明} \dotfill 6

\section*{四、问题一模型的建立和求解} \dotfill 7
\subsection*{4.1 问题一的分析} \dotfill 7
\subsection*{4.2 数据处理} \dotfill 7
\subsection*{4.3 模型的建立} \dotfill 9
\subsubsection*{4.3.1 多无人机协同干扰组网雷达的优化模型} \dotfill 9
\subsubsection*{4.3.2 基于枚举法的模型的目标优化} \dotfill 10
\subsection*{4.4 模型的求解} \dotfill 10
\subsubsection*{4.4.1 模型求解结果} \dotfill 13

\section*{五、问题二模型的建立和求解} \dotfill 16
\subsection*{5.1 问题二的分析} \dotfill 16
\subsection*{5.2 虚假目标和无人机运动模型} \dotfill 16
\subsection*{5.3 协同航迹欺骗的最优控制模型} \dotfill 18
\subsection*{5.4 模型的求解} \dotfill 20
\subsubsection*{5.4.1 附件 1 虚假航迹的产生} \dotfill 20
\subsubsection*{5.4.2 其它虚假航迹的产生} \dotfill 21

\section*{六、问题三模型的建立和求解} \dotfill 23
\subsection*{6.1 问题三的分析} \dotfill 23
\subsection*{6.2 模型的建立} \dotfill 23
\subsubsection*{6.2.1 多无人机协同产生多个虚假航迹} \dotfill 23
\subsubsection*{6.2.2 基于竞标算法的模型} \dotfill 24
\subsection*{6.3 模型的求解} \dotfill 25

\section*{七、模型的改进和评价} \dotfill 28
\subsection*{7.1 模型的算法和复杂度分析} \dotfill 28
\subsection*{7.2 模型的推广} \dotfill 28

\section*{八、参考文献} \dotfill 29

\section*{附录} \dotfill 30

\section*{一、问题重述}

\subsection*{1.1 问题背景}

组网雷达系统是应用两部或两部以上空间位置互相分离而覆盖范围互相重叠的雷达的观测或判断来实施搜索、跟踪和识别目标的系统,综合应用了多种抗干扰措施,具有较强的抗干扰能力,因而在军事中得到了广泛应用。如何对组网雷达实施行之有效的干扰,是当今电子对抗界面临的一个重大问题。

诸多干扰方式中较为有效的是欺骗干扰,包括距离欺骗、角度欺骗、速度欺骗以及多参数欺骗等。现考虑距离假目标欺骗,干扰机基于侦察到的敌方雷达发射电磁波的信号特征,对其进行相应处理后,延迟(或导前)一定时间后再发射出去,使雷达接收到一个或多个比该目标真实距离靠后(或靠前)的回波信号。

在组网雷达探测跟踪下,真目标和有源假目标在空间状态(如位置、速度等)上表现出显著的差异:对于真目标,其空间状态与雷达部署位置无关,在统一坐标系中,各雷达探测出的真目标空间状态是基本一致的,可以认为它们是源自于同一个目标(同源);对于有源假目标(为方便记做假目标),它们存在于雷达与干扰机连线以及延长线上,其空间状态由干扰机和雷达部署位置共同决定,不同雷达量测到的有源假目标的空间状态一般是不一致的,有理由认为其来自于不同目标(非同源),利用这种不一致性就可以在组网雷达信息融合中心将假目标有效剔除。这种利用真假目标在组网雷达观测下的空间状态差异来进行假目标鉴别的思想简称为“同源检验”,它是组网雷达对真假目标甄别的理论依据。

为了能对组网雷达实施有效干扰,现可利用多架无人机对组网雷达协同干扰。无人机搭载的干扰设备对接收到的雷达信号进行相应处理后转发回对应的雷达,雷达接收到转发回的干扰信号形成目标航迹点信息,传输至组网雷达信息融合中心。由于多无人机的协同飞行,因此在融合中心就会出现多部雷达在统一坐标系的同一空间位置上检测到目标信号,基于一定的融合规则就会判断为一个合理的目标航迹点,多个连续的合理目标航迹点就形成了目标航迹,即实现了一条虚假航迹。通过协同控制无人机的飞行航迹,可在敌方的组网雷达系统中形成一条或多条欺骗干扰航迹,迫使敌方加强空情处置,达到欺骗目的。

文献[1]针对组网雷达抗欺骗干扰能力强的特点,分析了利用多机协同对组网雷达进行航迹欺骗的可行性;文献[2]为了有效地对组网雷达实施干扰,根据航迹欺骗干扰原理和无人机任务特点,对无人机实施航迹欺骗干扰模型进行了简化;从几何关系上建立了定高飞行模式下假目标与无人机的运动模型,给出了模型求解流程;最后利用仿真得出了各无人机的运动参数,分析了无人机间协同的速度大小关系。仿真结果验证了这一干扰模型的可行性与有效性。文献[3]研究了多机协同航迹欺骗的耦合关系,分析了多机协同欺骗的约束条件,给出了代价函数,建立了多机协同航迹欺骗最优控制模型。

某组网雷达系统由5部雷达组成,雷达最大作用距离均为150km,也就是只能对距雷达150km范围内的目标进行有效检测。5部雷达的地理位置坐标分别为雷达1(80,0,0),雷达2(30,60,0),雷达3(55,110,0),雷达4(105,110,0),雷达5(130,60,0)(单位:统一为km)。雷达将检测到的回波信号经过处理后形成航迹点状态信息传输到融合中心,融合中心对5部雷达获取的目标状态信息进行“同源检验”,只要有3部以上雷达的航迹点状态信息通过了同源检验,即至少

有3部雷达同一时刻解算出的目标空间位置是相同的,融合中心就将其确定为一个合理的航迹点,20个连续的经融合中心确认的航迹点形成的合理航迹,将被组网雷达系统视为一条真实的目标航迹。所谓合理的航迹是要满足相应的目标运动规律,无论是运动速度还是转弯半径等均应在合理的范围内。

现考虑多架无人机对组网雷达系统的协同干扰问题。无人机的飞行速度控制在 $120\,\mathrm{km/h} \sim 180\,\mathrm{km/h}$,飞行高度控制在 $2000\,\mathrm{m} \sim 2500\,\mathrm{m}$,最大加速度不超过 $10\,\mathrm{m/s^2}$。由于安全等因素的考虑,无人机间距需控制在 $100\,\mathrm{m}$ 以上。鉴于无人机的 RCS 较小,也采用了若干隐身技术 \cite{ref10},在距雷达一定距离飞行时,真实目标产生的回波不能被雷达有效检测;干扰设备产生的欺骗干扰信号经过了放大增强环节,能保证被雷达有效检测到。每架无人机均搭载有干扰设备,可独立工作。同一时刻一架无人机只能干扰一部雷达,但可在该部雷达接收机终端(雷达屏幕上)产生多个目标点,这些目标点均位于雷达与无人机连线以及延长线上,距雷达距离超过 $150\,\mathrm{km}$ 的假目标信息直接被雷达系统删除;同一时刻多架无人机可以干扰同一部雷达。雷达同一时刻接收的多个目标点的状态信息均同时传送到信息融合中心。每架无人机不同时刻可干扰不同雷达。同一条航迹不同时刻的航迹点,可以由组网雷达系统中不同的三部雷达检测确定。

\section{需要解决的问题}

问题一:现给出一条拟产生的虚假目标航迹数据,该虚假航迹数据包含 20 个时刻的虚假目标位置坐标信息,时间间隔为 10 秒。为实现较好的干扰效果,现限定每架无人机在该空域均做匀速直线运动,航向、航速和飞行高度可在允许范围内根据需要确定。为完成虚假目标航迹,最少需要多少架无人机?具体分析每一架无人机的运动规律和相应的协同策略。

问题二:对雷达实施有源假目标欺骗干扰时,干扰设备可同时转发至多 7 个假目标信息,但它们均存在于雷达与无人机连线以及延长线上,延迟(或导前)的时间可根据实际需要确定。该组网雷达系统的每一部雷达的数据更新率为 10 秒(即每间隔 10 秒获得一批目标的空间状态数据,无人机转发回对应雷达的假目标信息能及时获取)。协同无人机编队可产生出多条虚假航迹,以实现更好的干扰效果。实际中无人机可机动飞行,但为控制方便,无人机尽可能少做转弯、爬升、俯冲等机动动作,转弯半径不小于 $250\,\mathrm{m}$。则 9 架无人机组成的编队在 5 分钟内完成附件 1 要求的虚假航迹的同时,至多还可产生出多少条虚假的航迹?请给出每一架无人机的运动规律,并分析每一条虚假航迹的运动规律和合理性。

问题三:当组网雷达系统中的某部雷达受到压制干扰或其它因素的干扰时,可能在某些时刻无法正常获取回波信号,此时组网雷达系统信息融合中心可以采用下面的航迹维持策略:若之前与受干扰的雷达联合检测到目标的另 2 部雷达没有受到干扰,正常检测到回波信号,那么在融合中心就对这两部雷达检测的目标航迹点信息进行同源检验,若通过亦视为是合理的目标航迹点;若一条航迹中这类航迹点的个数不超过 3 个时(该航迹的其余航迹点仍需通过前面规定的“同源检验”),该航迹就被继续保留。针对上述航迹维持策略,协同无人机编队的飞行,有可能产生更多的虚假航迹。该组网雷达系统的每一部雷达的数据更新率仍为 10 秒。重新讨论由 9 架无人机组成的编队在 5 分钟内,完成附件 1 要求的虚假航迹的同时,至多还可产生出多少条虚假的航迹。给出每一架无人机的运动规律和协同策略,分析每一条虚假航迹的运动规律和合理性。

\section{模型假设}

假设一:假设无人机及其搭载的干扰设备在执行任务时,不受台风、暴风雨等天气问题的影响;

假设二:假设无人机在执行干扰任务时,互不干扰且不存在故障原因$^{[9]}$;

假设三:假设无人机具备较强隐身能力,不易被组网雷达检测到。无人机有隐身技术,才能提高干扰的成功率,否则组网雷达将通过假目标和无人机航迹的相关性辨别出无人机和假目标,达不到预期的干扰效果;

假设四:假设无人机从接收雷达信号到发射回波信号的过程中不存在异常状态;

假设五:假设无人机搭载的干扰设备具备延时(或导前)信号的能力,且逼真度高,能成功欺骗组网雷达。

\section{符号说明}

\begin{tabular}{c c}
\hline \hline
符号 & 符号说明 \\
\hline
$t_{i}$ & 时刻$i$ \\
$R_{k}$ & 第$k$个雷达 \\
$P_{i}$ & 第$i$个虚假目标点 \\
$\boldsymbol{u}_{i-k}$ & 在时刻$i$对雷达$k$进行干扰的无人机 \\
$h_{\max}/h_{\min}$ & 无人机最高/最低飞行高度 \\
$\boldsymbol{v}_{i,j}^{R_{k}}$ & 由$t_{i}$到达$t_{j}$且对$R_{k}$进行干扰的无人机的飞行速度 \\
$h_{i}^{R_{k}}$ & $P_{i}$与$R_{k}$连线与高度为$h$的水平面的交点 \\
$y$ & 转弯半径 \\
$S$ & 虚假航迹数量 \\
$i-k$ & 虚假目标点$P_{i}$与雷达$R_{k}$连线上无人机的位置 \\
\hline \hline
\end{tabular}

\section*{四、问题一模型的建立和求解}

\subsection*{4.1 问题一的分析}

问题一要求多架无人机对组网雷达系统进行协同干扰,题中规定无人机飞行高度控制在 $2000m \sim 2500m$,为实现较好的干扰效果,不考虑无人机做俯冲、爬升等机动动作。问题一给出了 20 个时刻的虚假目标的位置坐标信息,其时间间隔为 10 秒,同时给出了 5 部雷达的位置坐标信息。每一时刻虚假目标点都位于雷达与无人机位置连线的延长线上,这样若给定无人机的飞行高度,则延长线与该高度的水平面相交于一点(下文统一称其为定高点),则可以由公式计算得到无人机的具体位置坐标。通过枚举法计算出某一架无人机在速度容许范围内做匀速直线运动可达到的定高点,之后建立基于多无人机协同干扰的目标优化模型,运用 Matlab 编程求解,可得在实现虚假目标的情况下所需无人机的最少数量。

在建立目标优化模型时,需考虑以下几个约束条件:

(1) 出于对安全、天气等因素的考虑,任意两架无人机的间距需控制在 $100m$ 以上;

(2) 任意一架无人机的飞行高度必须控制在 $2000m \sim 2500m$ 之间,且飞行速度控制在 $120km/h \sim 180km/h$ 范围内;

(3) 无人机在整个干扰过程中做匀速直线运动,同时无人机不做俯冲、爬升等机动动作;

(4) 在同一时刻有 3 部以上雷达的航迹点状态信息通过了同源检验,即至少有 3 部雷达在同一时刻解算出的虚假目标空间位置是相同的,这样融合中心才能将其确定为一个合理的航迹点;

\subsection*{4.2 数据处理}

首先运用 Matlab 对附件 1 中的 20 个虚假目标位置的坐标信息进行处理,得到虚假目标航迹图,如图 4-1 所示。

\begin{figure}[h]
    \centering
    \includegraphics[width=0.45\textwidth]{image1.png}
    \caption{虚假目标航迹示意图}
    \label{fig:4-1}
\end{figure}
\begin{figure}[h]
    \centering
    \includegraphics[width=0.45\textwidth]{image2.png}
    \caption{雷达有效检测范围平面示意图}
    \label{fig:4-2}
\end{figure}

雷达有效检测范围是以该雷达为中心,半径为 $150km$ 的半球,运用 Matlab 得其在 $xoy$ 平面上的投影,如图 4-2 所示。无人机只有在图示范围内才能对组网

雷达进行有效干扰。

无人机飞行高度为 $2000m \sim 2500m$,而虚假目标航迹高度处于 $6000m$ 及以上,可知无人机搭载的干扰设备对接收到的雷达信号进行处理后延迟转发回对应的雷达,则无人机位置位于虚假目标与雷达位置连线上。经 Matlab 处理,虚假目标点与 5 部雷达的连线情况及 5 部雷达的位置分布如图 4-3 所示,图 4-4 中两平面中间的三维区域即为无人机允许活动范围。

\begin{figure}[h]
    \centering
    \includegraphics[width=\textwidth]{image1.png}
    \caption{各时刻假目标与雷达位置连线示意图}
    \label{fig:4-3}
\end{figure}

\begin{figure}[h]
    \centering
    \includegraphics[width=\textwidth]{image2.png}
    \caption{无人机飞行高度范围示意图}
    \label{fig:4-4}
\end{figure}

\section{模型的建立}

\subsection{多无人机协同干扰组网雷达的优化模型}

本问题基于多无人机协同干扰组网雷达,形成附件 1 要求的虚假目标航迹。通过合理安排无人机飞行航迹,使得每个时刻均有三架无人机对三个雷达进行有效干扰,在保证形成虚假目标航迹的前提下,通过使某一架无人机在速度容许范围内做匀速直线运动时达到尽可能多的定高点,从而实现减少无人机数量这一目的。

\subsubsection{确定目标函数}

本题主要研究的是在完成所要求的虚假目标航迹前提下,最少需要多少架无人机。在规划无人机飞行方案时,若在假目标点 $P_j$ 与雷达 $R_k$ 连线上放置一架无人机,则记 $l_{jk}=1$,若不放置则记为 $l_{jk}=0$,目标函数为:
\[
\min \sum_{j=1}^{20} \sum_{k=1}^{5} l_{jk}.
\]

\subsubsection{确定约束条件}

\textbf{约束条件一:} 无人机飞行速度 $v$ 满足 $120 \leq v \leq 180$(单位为 km/h);

\textbf{约束条件二:} 无人机飞行高度 $h$ 满足 $2000 \leq h \leq 2500$(单位为 m);

\textbf{约束条件三:} 无人机间距控制在 100m 以上,即:
\[
\sqrt{(x_1^i - x_2^i)^2 + (y_1^i - y_2^i)^2 + (z_1^i - z_2^i)^2} > 100,
\]
其中,在 $t_i$ 时刻,这两架无人机坐标分别简记为 $(x_1^i, y_1^i, z_1^i)$ 和 $(x_2^i, y_2^i, z_2^i)$。

\textbf{约束条件四:} 无人机做匀速直线运动,即同一架无人机在不同时刻速度相同,
\[
v^{i+1} = v^i,
\]
其中 $v^i$ 表示该无人机在 $t_i$ 时刻的速度,$v^{i+1}$ 表示该无人机在 $t_{i+1}$ 时刻的速度。

\textbf{约束条件五:} 无人机飞行航向角恒定,即 $\frac{y_u^{i+1} - y_u^i}{x_u^{i+1} - x_u^i}$ 为定值。

其中,该无人机 $u$ 在 $t_i$ 和 $t_{i+1}$ 时刻的坐标分别记为 $(x_u^i, y_u^i)$ 和 $(x_u^{i+1}, y_u^{i+1})$。

\textbf{约束条件六:} 在同一时刻有 3 部以上雷达的航迹点状态信息通过了同源检验,即至少有 3 部雷达在同一时刻解算出的虚假目标空间位置是相同的,这样融合中心才能将其确定为一个合理的航迹点,即满足下式:

\[
\sum_{k=1}^{5} l_{jk} \geq 3, \, j=1 \sim 20, \, k=1 \sim 5,
\]

综上所述,建立基于多无人机对组网雷达的协同干扰优化模型为:

\[
\min \sum_{j=1}^{20} \sum_{k=1}^{5} l_{jk}
\]

满足

\[
\begin{cases}
120 \leq v \leq 180 \\
2000 \leq h \leq 2500 \\
\sqrt{(x_1^i - x_2^i)^2 + (y_1^i - y_2^i)^2 + (z_1^i - z_2^i)^2} > 100 \\
v^{i+1} = v^i \quad (i=1, 2, \dots, 19) \\
\frac{y_u^{i+1} - y_u^i}{x_u^{i+1} - x_u^i} = C \, (C \text{为定值}) \\
\sum_{k=1}^{5} l_{jk} \geq 3, \, j=1 \sim 20, \, k=1 \sim 5.
\end{cases}
\]

### 4.3.2 基于枚举法的模型的目标优化

本文在对多无人机协同干扰多部雷达时的整个工作流程的充分理解的基础上,建立了多无人机协同干扰组网雷达的优化模型 [2]。虚假目标点 $P_i$ 与雷达 $R_k$ 的连线与高度为 $h (h_{\min} \leq h \leq h_{\max})$ 的水平面相交于一点,为描述方便,将该点记作定高点 $h_i^{R_k}$。模型的优化流程如下:

1. 确定雷达和 20 个假目标点的位置坐标信息;
2. 确定虚假目标点与雷达的所有位置连线及假目标点确定的虚假目标航迹;
3. 计算无人机的最优飞行高度 $h (h_{\min} \leq h \leq h_{\max})$;
4. 由式(1),(3)确定无人机的安排位置;
5. 将 1 架无人机依次从定高点 $h_i^{R_k}$ 飞至 $h_j^{R_k} \, (j \geq i+1, i+1, \cdots, 20)$,计算无人机飞至其他定高点的速度;
6. 判断上述速度是否在无人机速度控制范围内 $(120 \leq v_{i-j}^{R_k} \leq 180)$ (单位为 $km/h$)。若满足,转(5)并记录下该位置的信息;否则不予考虑;
7. 由(6)可得无人机的最少数量,并分析每一架无人机的运动规律;
8. 完成多无人机协同干扰雷达 $R_k$ 的任务。

详细了解优化过程的操作流程,可以安全、高效的完成对组网雷达的有效干扰,同时无人机飞行路径的安排也是模型中的重要步骤,它详细地说明了在路径安排的过程中具体操作的先后顺序,它对于能否更有效减少无人机的数量有着极为重要的作用。

### 4.4 模型的求解

本文选用枚举法对多无人机协同干扰组网雷达的优化模型进行求解。枚举法利用了计算机运算速度快、精确度高的特点,对要解决问题的所有可能情况,一个不漏地进行检验,从中找出符合要求的无人机运动规律,因此利用枚举法可保证找到全局最优解。

下面求解若要使一架无人机在某两个时刻对同一部雷达进行干扰,需要满足怎样的飞行要求。算法流程图如下:

\begin{figure}[h]
\centering
\includegraphics[width=0.8\textwidth]{algorithm_flowchart.png}
\caption{模型算法流程图}
\end{figure}

假设无人机分别处于 \( t_i \) 时刻和 \( t_j \) 时刻 \((1 \leq i, j \leq 20, i < j)\),相应的假目标 \( P_i \) 和 \( P_j \) 的坐标位置分别为 \((x_{P_i}, y_{P_i}, z_{P_i})\), \((x_{P_j}, y_{P_j}, z_{P_j})\)。该无人机干扰雷达 \( R_k \) \((1 \leq k \leq 5)\),雷达 \( R_k \) 坐标位置为 \((x_{R_k}, y_{R_k}, z_{R_k})\)。在三维坐标系中将虚假目标坐标与雷达坐标连线,得到两条直线,分别是

假目标 \( P_i \) 和雷达 \( R_k \) 形成的直线方程:
\begin{equation}
\frac{x - x_{R_k}}{x_{P_i} - x_{R_k}} = \frac{y - y_{R_k}}{y_{P_i} - y_{R_k}} = \frac{z - z_{R_k}}{z_{P_i} - z_{R_k}},
\tag{1}
\end{equation}

假目标 \( P_j \) 和雷达 \( R_k \) 形成的直线方程:
\begin{equation}
\frac{x - x_{R_k}}{x_{P_j} - x_{R_k}} = \frac{y - y_{R_k}}{y_{P_j} - y_{R_k}} = \frac{z - z_{R_k}}{z_{P_j} - z_{R_k}},
\tag{2}
\end{equation}

因此,这里固定无人机飞行高度为 \( h \),两条直线分别与高度为 \( h \) 的水平平面形成两个交点,分别是

直线(1)与平面 \( z = h \) 的交点:
\begin{equation}
\begin{cases}
x_{u_{i-k}} = \frac{(h - z_{R_k})(x_{R_k} - x_{P_i})}{z_{R_k} - z_{P_i}} + x_{R_k} \\
y_{u_{i-k}} = \frac{(h - z_{R_k})(x_{R_k} - y_{P_i})}{z_{R_k} - z_{P_i}} + y_{R_k} \\
z_{u_{i-k}} = h
\end{cases}
\tag{3}
\end{equation}

直线(2)与平面 \( z = h \) 的交点:
\begin{equation}
\left\{
\begin{aligned}
x_{u_{i-k}} &= \frac{(h - z_{R_k})(x_{R_k} - x_{P_j})}{z_{R_k} - z_{P_i}} + x_{R_k} \\
y_{u_{i-k}} &= \frac{(h - z_{R_k})(x_{R_k} - y_{P_j})}{z_{R_k} - z_{P_j}} + y_{R_k}, \\
z_{u_{i-k}} &= h
\end{aligned}
\right.
\tag{4}
\end{equation}

两交点连线即该无人机由 \( t_i \) 时刻至 \( t_j \) 时刻做匀速直线运动的飞行路径。易得

飞行路径长度: \( d = \sqrt{(x_{u_{i-k}} - x_{u_{j-k}})^2 + (y_{u_{i-k}} - y_{u_{j-k}})^2 + (z_{u_{i-k}} - z_{u_{j-k}})^2} \),

飞行时间: \( t = t_j - t_i \),

由于做匀速直线运动,则无人机在这个运动过程中的航速(m/s)为:
\[
v_{i,j}^{R_k} = \frac{d}{t}.
\]

无人机飞行高度范围为 \( 2000m \sim 2500m \),当高度增加时,如图 4-5 所示,无人机利用率为两次的概率增加。因此,从优化的角度分析,无人机最佳飞行高度应为 \( 2500m \),即固定上述 \( h = h_{\max} \),则上述两直线与平面的交点即为定高点 \( h_i^{R_k} \) 和定高点 \( h_j^{R_k} \),无人机由定高点 \( h_i^{R_k} \) 飞至定高点 \( h_j^{R_k} \)。

\begin{figure}[h]
\centering
\includegraphics[width=\textwidth]{image.png}
\caption{无人机飞行高度与可选方案的数量关系示意图}
\end{figure}

将航速范围 \( 120km/h \sim 180km/h \) 进行单位换算,即为 \( 33.33m/s \sim 50m/s \),若所得航速处于允许范围内,说明无人机可从定高点 \( h_i^{R_k} \) 以速度 \( v_{i,j-k} \) 沿直线飞至定高点 \( h_j^{R_k} \),即在定高点 \( h_i^{R_k} \) 和定高点 \( h_j^{R_k} \) 安排同一架无人机;若所得航速不在允许范围内,说明无人机无法在定高点 \( h_i^{R_k} \) 和定高点 \( h_j^{R_k} \) 之间做匀速直线运动,则需安排两架不同的无人机分别在 \( t_i \) 时刻和 \( t_j \) 时刻对雷达 \( R_k \) 进行干扰。

以无人机干扰雷达 \( R_2 \) 和雷达 \( R_3 \) 为例,假设无人机在 \( t_1 \) 时刻对雷达 \( R_2 \) 和雷达

\section*{表4-1 干扰雷达 \(R_{2}\) 的无人机由 \(t_{1}\) 时刻做匀速直线运动至其他时刻的速度对应表}

\begin{tabular}{|c|c|c|c|c|c|c|c|c|c|c|}
\hline 时刻 & 1 & 2 & 3 & 4 & 5 & 6 & 7 & 8 & 9 & 10 \\
\hline 速度 & 0 & 20.53 & 21.16 & 21.80 & 22.45 & 23.12 & 23.81 & 24.52 & 25.26 & 26.03 \\
\((m/s)\) & & & & & & & & & & \\
\hline 时刻 & 11 & 12 & 13 & 14 & 15 & 16 & 17 & 18 & 19 & 20 \\
\hline 速度 & 26.84 & 27.69 & 28.57 & 29.47 & 30.43 & 31.44 & 32.53 & 33.71 & 35.00 & 36.40 \\
\((m/s)\) & & & & & & & & & & \\
\hline
\end{tabular}

\section*{表4-2 干扰雷达 \(R_{3}\) 的无人机由 \(t_{1}\) 时刻做匀速直线运动至其他时刻的速度对应表}

\begin{tabular}{|c|c|c|c|c|c|c|c|c|c|c|}
\hline 时刻 & 1 & 2 & 3 & 4 & 5 & 6 & 7 & 8 & 9 & 10 \\
\hline 速度 & 0 & 19.46 & 19.89 & 20.41 & 21.02 & 21.70 & 22.48 & 23.34 & 24.29 & 25.33 \\
\((m/s)\) & & & & & & & & & & \\
\hline 时刻 & 11 & 12 & 13 & 14 & 15 & 16 & 17 & 18 & 19 & 20 \\
\hline 速度 & 26.46 & 27.69 & 29.00 & 30.39 & 31.88 & 33.48 & 35.21 & 37.09 & 39.14 & 41.38 \\
\((m/s)\) & & & & & & & & & & \\
\hline
\end{tabular}

由表4-1中数据可得,在速度控制范围内,无人机可从定高点 \(h_{1}^{R_{2}}\) 做匀速直线运动至定高点 \(h_{8}^{R_{2}}\),或定高点 \(h_{9}^{R_{2}}\),或定高点 \(h_{20}^{R_{2}}\) 为使从定高点 \(h_{1}^{R_{2}}\) 出发的无人机利用率最高,该无人机有以上三条飞行路径可供选择。具体选择哪条路径,需协同考虑其他定高点出发的无人机的飞行路径。同理,由表4-2中数据,为使从定高点 \(h_{1}^{R_{3}}\) 出发的无人机利用率最高,该无人机有以上五条飞行路径可供选择。

现有20个时刻分别对应5部雷达,若一架无人机可在速度范围内做匀速直线运动,且在某两个时刻对同一雷达进行干扰,则将该架无人机飞行路径记入附录表格中。由数据可知,当无人机以某一速度从一时刻飞至另一时刻时,不可能再以该速度从第二个时刻飞至其他时刻,即一架做匀速直线运动的无人机至多可利用两次。

\subsection*{4.4.1 模型求解结果}

根据表4-1,4-2及附录所得速度,在保证无人机利用率最高的前提下,安排每一架无人机可行的飞行路径。经分析发现,无人机在不同时刻只能干扰同一部雷达。为实现附件1的虚假航迹最少需要31架无人机,具体无人机分配情况如表4-3所示。

\section*{表4-3 无人机飞行安排}

\begin{tabular}{|c|c|c|c|c|c|}
\hline 起始位置 & 飞离位置 & 速度 \((m/s)\) & 起始位置 & 飞离位置 & 速度 \((m/s)\) \\
\hline 1-2 & 20-2 & 36.40 & 6-2 & 15-2 & 34.56 \\
1-3 & 20-3 & 41.38 & 6-3 & 15-3 & 38.80 \\
1-4 & / & 50.00 & 6-4 & 16-4 & 34.74 \\
\hline
\end{tabular}

\begin{tabular}{|c|c|c|c|c|c|}
\hline
2-2 & 19-2 & 35.86 & 7-2 & 14-2 & 34.40 \\
2-3 & 19-3 & 40.63 & 7-3 & 14-3 & 38.58 \\
2-4 & 20-4 & 37.44 & 7-4 & 15-4 & 34.37 \\
\hline
3-2 & 18-2 & 35.42 & 8-2 & 13-2 & 34.31 \\
3-3 & 18-3 & 40.02 & 8-3 & 13-3 & 38.43 \\
3-4 & 19-4 & 36.55 & 8-4 & 14-4 & 34.10 \\
\hline
4-2 & 17-2 & 35.06 & 9-2 & 12-2 & 34.27 \\
4-3 & 17-3 & 39.51 & 9-3 & 12-3 & 38.35 \\
4-4 & 18-4 & 35.82 & 9-4 & 13-4 & 33.93 \\
\hline
5-2 & 16-2 & 34.77 & 10-2 & 11-2 & 34.22 \\
5-3 & 16-3 & 39.11 & 10-3 & 11-3 & 38.26 \\
5-4 & 17-4 & 35.22 & 10-4 & 12-4 & 33.83 \\
\hline
11-4 & / & 50.00 & & & \\
\hline
\end{tabular}

(注:表中 $i-j$ 表示无人机位于定高点 $h_{i}^{R_{j}}$。)

经验证,所有无人机在飞行过程中两两间隔超过 100m,满足题设要求。首先对无人机进行编号:$u_{i-k}$ ($i=1,2,\ldots,11,k=2,3,4$) 表示在 $t_{i}$ 时刻对雷达 $R_{k}$ 进行干扰的无人机。

然后对表中数据解释说明:起始时刻 1-2,飞离时刻 20-2,速度 36.40m/s 表示无人机在 $t_{1}$ 时刻位于定高点 $h_{1}^{R_{2}}$,在 $t_{20}$ 时刻位于定高点 $h_{20}^{R_{2}}$,无人机在 $t_{1}$ 时刻至 $t_{20}$ 时刻之间从定高点 $h_{1}^{R_{2}}$ 做匀速直线运动飞至定高点 $h_{20}^{R_{2}}$,速度为 36.40m/s。做类似推理可知其它 30 架无人机的飞行轨迹。

最后,由表 4-3 得:

(1) 无人机 $u_{1-4}$ 无法在可允许速度范围内飞至其它定高点,因此该架无人机利用率只有一次;

(2) 只有两架无人机从 $t_{10}$ 时刻运动至 $t_{11}$ 时刻,因此需增加一架无人机在 $t_{11}$ 时刻飞行至定高点 $h_{11}^{R_{4}}$,编号为 $u_{11-4}$。

这样一来,共需 $10 \times 3 + 1 = 31$ 架无人机,无人机的协同飞行策略便已确定。

\begin{figure}[h]
    \centering
    \includegraphics[width=\textwidth]{image.png}
    \caption{$u_{2-2}$,$u_{2-3}$,$u_{3-4}$ 三架无人机的飞行轨迹}
    \label{fig:4-6}
\end{figure}

在高度为2500m的水平面上,31架无人机的飞行轨迹如图4-7所示。

\begin{figure}[h]
    \centering
    \includegraphics[width=\textwidth]{image1.png}
    \caption{31架无人机的飞行轨迹示意图}
    \label{fig:4-7}
\end{figure}

将图4-7中圆圈标记处进行局部放大,可以清晰地看出该区域无人机的飞行轨迹,见图4-8。

\begin{figure}[h]
    \centering
    \includegraphics[width=\textwidth]{image2.png}
    \caption{飞行轨迹局部放大示意图}
    \label{fig:4-8}
\end{figure}

\section{问题二模型的建立和求解}

\subsection{问题二的分析}

问题二要求无人机编队对组网雷达系统进行协同干扰,产生多条虚假航迹,以实现更好的干扰效果。本题给定无人机数量为 9 架,无人机编队在 5 分钟内完成附件 1 要求的虚假航迹的同时,还可产生多条虚假航迹。

本题首先考虑如何产生附件 1 要求的虚假航迹。可安排三架无人机分别对三个雷达进行干扰(本题假定安排无人机 $A_1, A_2, A_3$ 分别干扰雷达 $R_2, R_3, R_4$),且每一架无人机在 $t_1$ 时刻至 $t_2$ 时刻之间只对一部雷达进行干扰。当固定每一架无人机在 $t_1$ 时刻的飞行高度和飞行速度后,初始位置的坐标信息就确定,可根据雷达与虚假目标点的连线方程求得下一时刻的位置坐标,那么这一架无人机的飞行路径就可以确定。

再来考虑如何产生其它虚假航迹。在无人机 $A_1, A_2, A_3$ 中选取一架(例如选择 $A_1$),再安排两架新的无人机 $A_4$ 和 $A_5$,让这三架无人机分别干扰雷达 $R_2, R_1, R_5$,在每一个时刻产生一个虚假目标点,从而协同产生一条新的虚假航迹。

在建立目标优化模型时,需考虑以下几个约束条件:

(1) 出于对安全、天气等因素的考虑,任意两架无人机的间距需控制在 $100m$ 以上;

(2) 任意一架无人机的飞行高度必须控制在 $2000m \sim 2500m$ 之间,且飞行速度控制在 $120km/h \sim 180km/h$ 范围内;

(3) 无人机在整个干扰过程中可机动飞行,但尽可能少做转弯、俯冲、爬升等机动动作,转弯半径不小于 $250m$;

(4) 对雷达实施有源假目标欺骗干扰时,每一架无人机同一时刻至多产生 7 个假目标信息,均位于雷达与无人机连线以及延长线上(无人机携带的干扰设备可延迟或导前发射回波信号的时间可根据实际情况需要确定);

\subsection{虚假目标和无人机运动模型}

无人机编队由多架无人机组成,在分析编队模型之前,首先对单架无人机模型进行分析 \cite{ref1,ref8}。建立三维坐标系,给出雷达 $R_k$、虚假目标 $P$ 和无人机 $u$ 的运动学变量关系如图 5-1 所示。三者位于同一直线上,其中 $r$ 和 $R$ 分别表示无人机与虚假目标到雷达的径向距离,$(x_0, y_0, z_0), (x_u, y_u, z_u), (x_p, y_p, z_p)$ 分别表示雷达、无人机与虚假目标在三维直角坐标系中的位置,$\varphi$ 和 $\theta$ 表示雷达到无人机与虚假目标的俯仰角与方位角,$V_u$ 和 $V_p$ 分别表示无人机和虚假目标的速度矢量,$\alpha$ 和 $\beta$ 则表示速度矢量的俯仰角和方位角。

\begin{figure}[h]
    \centering
    \includegraphics[width=0.8\textwidth]{image.png}
    \caption{雷达、无人机与虚假目标的运动学变量关系示意图}
    \label{fig:5-1}
\end{figure}

图 \ref{fig:5-1} 中雷达 $R_{k}$、虚假目标 $P$ 和无人机 $u$ 之间的空间位置关系 [4] 可表示为

\begin{equation}
x_{u} = x_{0} + m(x_{P} - x_{0}).
\end{equation}

其中,$m = \frac{r}{R}$ 为轨迹控制参数,表示无人机与雷达距离和虚假目标与雷达距离之比。当 $m < 1$ 时无人机对雷达产生前置虚假目标,当 $m > 1$ 时无人机对雷达产生后置虚假目标。

观察图 \ref{fig:5-1} 中雷达 $R_{k}$、虚假目标 $P$ 和无人机 $u$ 的几何关系可知,无人机与虚假目标的基本运动学模型可表示为

\begin{align}
\dot{x}_{i} &= V_{i} \cos \alpha_{i} \cos \beta_{i}, \tag{1} \\
\dot{y}_{i} &= V_{i} \cos \alpha_{i} \sin \beta_{i}, \tag{2} \\
\dot{z}_{i} &= V_{i} \sin \alpha_{i}, \tag{3}
\end{align}

其中,$i = P$ 时表示虚假目标,$i = u$ 时表示无人机。

\section{虚假目标运动模型}

为方便观察和计算,利用式 (4) 可以将直角坐标系下对应的虚假目标运动模型转换到相应的以其对应雷达为原点的球面坐标系中

\begin{equation}
\begin{bmatrix}
\dot{R} \\
R \cos \varphi \dot{\theta} \\
R \dot{\varphi}
\end{bmatrix}
=
\begin{bmatrix}
\cos \varphi \cos \theta & \cos \varphi \sin \theta & \sin \varphi \\
-\sin \theta & \cos \theta & 0 \\
-\sin \varphi \cos \theta & -\sin \varphi \cos \theta & \cos \varphi
\end{bmatrix}
\begin{bmatrix}
\dot{x}_{P} \\
\dot{y}_{P} \\
\dot{z}_{P}
\end{bmatrix},
\tag{4}
\end{equation}

将式 (1)~(3) 代入式 (4) 展开并化简得虚假目标在所设球面坐标系下的位置微分方程组:

\begin{equation}
\dot{R} = V_{P} (\cos \varphi \cos \alpha_{P} \cos (\beta_{P} - \theta) + \sin \varphi \sin \alpha_{P}),
\tag{5}
\end{equation}

\begin{equation}
\dot{\theta} = \frac{V_p \cos \alpha_p \sin (\beta_p - \theta)}{R \cos \varphi},
\tag{6}
\end{equation}

\begin{equation}
\dot{\varphi} = \frac{V_p}{R} \left( -\sin \varphi \cos \alpha_p \cos (\beta_p - \theta) + \cos \varphi \sin \alpha_p \right).
\tag{7}
\end{equation}

\section*{2. 无人机运动模型}

将无人机对应的运动模型进行变换,得其相应的位置微分方程组:

\begin{equation}
\dot{r} = V_u \left( \cos \varphi \cos \alpha_u \cos (\beta_u - \theta) + \sin \varphi \sin \alpha_u \right),
\tag{8}
\end{equation}

\begin{equation}
\dot{\theta} = \frac{V_u \cos \alpha_u \sin (\beta_u - \theta)}{r \cos \varphi},
\tag{9}
\end{equation}

\begin{equation}
\dot{\varphi} = \frac{V_u}{r} \left( -\sin \varphi \cos \alpha_u \cos (\beta_u - \theta) + \cos \varphi \sin \alpha_u \right).
\tag{10}
\end{equation}

\section*{3. 控制模型}

观察图 5-1 中雷达、无人机和虚假目标的位置关系可知,无人机必须位于雷达和虚假目标连线及其延长线上,因此由式 (9) 和 (10) 可推导得出

\begin{equation}
\tan \alpha_u = \frac{\dot{\theta} \cos \varphi \sin \varphi \cos (\beta_u - \theta) + \dot{\varphi} \sin (\beta_u - \theta)}{\dot{\theta} \cos^2 \varphi}.
\tag{11}
\end{equation}

将式 (11) 和式 (9) 代入式 (8) 中可以得到

\begin{equation}
\dot{r} = r \dot{\theta} \left( \cos^2 \varphi \cot (\beta_u - \theta) + \tan \alpha_u \frac{\cos \varphi \sin \varphi}{\sin (\beta_u - \theta)} \right).
\tag{12}
\end{equation}

观察式 (11) 和 (12) 可知,无人机运动方程受到 $\beta_u$ 的控制。

由式 (8)~(10) 可推得无人机速度表达式如下

\begin{equation}
V_u = \sqrt{(\dot{r})^2 + (r \dot{\theta} \cos \varphi)^2 + (r \dot{\varphi})^2}.
\tag{13}
\end{equation}

结合式 (13) 与上述各式可知,无人机速度与假目标速度之间存在一定关系,达到了相互调制的效果。

\section*{5.3 协同航迹欺骗的最优控制模型}

本问题基于多无人机组对网雷达进行协同干扰,通过协同安排 9 架无人机的飞行路径,使得每个时刻均有多架无人机对至少三个雷达进行有效干扰 [6],且雷达在同一时刻解算出的目标空间位置相同,从而产生一个合理的航迹点。本题讨论在 5 分钟内形成附件 1 要求的虚假目标航迹的同时,至多还可产生多少条虚假航迹,并确定每架无人机的飞行路径。

\section*{1. 确定目标函数}

本题主要研究的是在完成附件 1 中虚假目标航迹前提下,至多还可产生多少

条虚假航迹。记虚假航迹数量为 \( S \),目标函数为:
\[
\max \left( S + \frac{1}{a_u^2} \right)
\]
其中,\( a_u^2 \) 表示无人机飞行轨迹曲率,根据题中要求应尽可能的小。

### 2. 确定约束条件

**约束条件一:** 无人机飞行速度 \( \nu \) 满足 \( 120 \leq \nu \leq 180 \)(速度单位为 km/h);

**约束条件二:** 无人机飞行高度 \( h \) 满足 \( 2000 \leq h \leq 2500 \)(高度单位为 m);

**约束条件三:** 无人机间距控制在 100m 以上,即:
\[
\sqrt{(x_1^i - x_2^i)^2 + (y_1^i - y_2^i)^2 + (z_1^i - z_2^i)^2} > 100,
\]
其中,在 \( t_i \) 时刻,这两架无人机坐标分别简记为 \( (x_1^i, y_1^i, z_1^i) \) 和 \( (x_2^i, y_2^i, z_2^i) \);

**约束条件四:** 无人机转弯半径 \( y \) 不小于 250m;

**约束条件五:** 在同一时刻有 3 部以上雷达的航迹点状态信息通过了同源检验,即至少有 3 部雷达在同一时刻解算出的虚假目标空间位置是相同的,这样融合中心才能将其确定为一个合理的航迹点。在规划无人机飞行方案时,若在假目标点 \( P_j \) 与雷达 \( R_k \) 连线上有一架无人机,则记 \( l_{jk} = 1 \),若没有则记为 \( l_{jk} = 0 \),即满足下式:
\[
\sum_{k=1}^5 l_{jk} \geq 3, \, j = 1 \sim 20, \, k = 1 \sim 5,
\]

综上所述,建立基于多无人机对组网雷达的协同干扰优化模型为:
\[
\max \left( S + \frac{1}{a_u^2} \right)
\]
\[
\begin{cases}
120 \leq \nu \leq 180 \\
2000 \leq h \leq 2500 \\
\sqrt{(x_1^i - x_2^i)^2 + (y_1^i - y_2^i)^2 + (z_1^i - z_2^i)^2} > 100. \\
y \geq 250 \\
\sum_{k=1}^5 l_{jk} \geq 3
\end{cases}
\]

\section{模型的求解}

\subsection{附件 1 虚假航迹的产生}

首先对 9 架无人机进行编号,记作 $A_i (i=1,2,\ldots,9)$,任取 3 架无人机(这里假定取 $A_1, A_2, A_3$ 这三架无人机),让它们分别对应干扰 3 部雷达(这里假定取 $R_2, R_3, R_4$ 这 3 部雷达)。在每一个时刻,这 3 架无人机协同干扰这 3 部雷达,形成一个合理的虚假目标点。假定无人机从 $t_1$ 时刻开始飞行,直至 $t_2$ 时刻,则可形成附件 1 所需的虚假航迹。

现在来说明无人机 $A_1$ 从 $t_1$ 时刻起,做怎样的飞行直至 $t_2$ 时刻。首先需知无人机 $A_1$ 在 20 个时刻均对雷达 $R_2$ 进行干扰。假定无人机 $A_1$ 在 $t_1$ 时刻从固定高度开始飞行,考虑到无人机可做爬升、俯冲等机动动作,且经计算发现若初始高度太低会导致无人机在飞行过程中并非一直在允许范围内,考虑无人机初始飞行高度适当的低,此处假定为 2100m。

虚假目标点 $P_1$ 与雷达 $R_2$ 的连线 $l_1$ 方程:
\begin{equation}
\frac{x-x_{P_1}}{x_{R_2}-x_{P_1}} = \frac{y-y_{P_1}}{y_{R_2}-y_{P_1}} = \frac{z-z_{P_1}}{z_{R_2}-z_{P_1}},
\tag{1}
\end{equation}

虚假目标点 $P_2$ 与雷达 $R_2$ 的连线 $l_2$ 方程:
\begin{equation}
\frac{x-x_{P_2}}{x_{R_2}-x_{P_2}} = \frac{y-y_{P_2}}{y_{R_2}-y_{P_2}} = \frac{z-z_{P_2}}{z_{R_2}-z_{P_2}},
\tag{2}
\end{equation}

其中各点坐标为:$P_1(x_{P_1}, y_{P_1}, z_{P_1})$,$P_2(x_{P_2}, y_{P_2}, z_{P_2})$,$R_2(x_{R_2}, y_{R_2}, z_{R_2})$ 均已确定。

无人机在 $t_1$ 时刻位于 $l_1$ 上,当固定高度后,无人机坐标可确定,记为 $Q_{P_1-R_2}(x_0, y_0, z_0)$。下面说明如何求解该无人机在 $t_2$ 时刻的坐标,记为 $Q_{P_2-R_2}(x_1, y_1, z_1)$。

首先可知 $t_1$ 时刻与 $t_2$ 时刻的时间间隔为 10 秒,无人机在 $t_1$ 时刻以某一俯仰角开始飞行,为控制无人机始终处于高度范围内,假定速度取最小速度,那么,$Q_{P_1-R_2}$ 与 $Q_{P_2-R_2}$ 的距离为 $33.33 \times 10 = 333.3 \, \text{m}$。

又可知 $Q_{P_2-R_2}(x_1, y_1, z_1)$ 位于直线 $l_2$ 上,即满足方程 (2),结合两点距离为 333.3m,可求出 $Q_{P_2-R_2}(x_1, y_1, z_1)$。

那么,无人机 $A_1$ 在 $t_1$ 时刻至 $t_2$ 时刻间,从 $Q_{P_1-R_2}(x_0, y_0, z_0)$ 以速度 $33.33 \, \text{m/s}$ 飞至 $Q_{P_2-R_2}(x_1, y_1, z_1)$,即无人机 $A_1$ 在 $t_1$ 时刻至 $t_2$ 时刻间的飞行路径确定。同理可计算无人机 $A_1$ 在 $t_2$ 时刻至 $t_3$ 时刻,……,$t_{19}$ 时刻至 $t_{20}$ 时刻的飞行路径,从而确定从 $t_1$ 时刻至 $t_{20}$ 时刻的飞行路径。

上述为无人机 $A_1$ 的飞行路径,同理得到无人机 $A_2$ 和 $A_3$ 的飞行路径,由这三架无人机就可确定附件 1 所需的虚假航迹,将这三架无人机的飞行路径反映在同一图中,见图 5-2。

\begin{figure}[h]
    \centering
    \includegraphics[width=\textwidth]{image1.png}
    \caption{无人机 $A_{1}, A_{2}, A_{3}$ 协同干扰雷达 $R_{2}, R_{3}, R_{4}$ 的飞行路径}
    \label{fig:5-2}
\end{figure}

\subsection{其它虚假航迹的产生}

在雷达 $R_{2}, R_{3}, R_{4}$ 中任意选取一部雷达,在它和对应无人机的连线轨迹上寻找另一个虚假目标信息点 $M$,再安排两架新的无人机 $A_{4}$ 和 $A_{5}$ 分别对雷达 $R_{1}$ 和 $R_{5}$ 进行干扰,使这两架无人机产生的虚假目标信息点与 $M$ 重合,则产生了一个合理的虚假目标点,每个时刻都会产生一个虚假目标点,从而产生一条新的虚假航迹。经计算可得,这样的虚假航迹有三条,如图 \ref{fig:5-3} 所示。

\begin{figure}[h]
    \centering
    \includegraphics[width=\textwidth]{image2.png}
    \caption{三条虚假航迹}
    \label{fig:5-3}
\end{figure}

他们分别是这样产生的:
\begin{enumerate}
    \item 安排无人机 $A_{1}$ 和两架新无人机 $A_{4}, A_{5}$,分别对雷达 $R_{2}, R_{1}, R_{5}$ 进行干扰;
    \item 安排无人机 $A_{2}$ 和两架新无人机 $A_{6}, A_{7}$,分别对雷达 $R_{3}, R_{1}, R_{2}$ 进行干扰;
    \item 安排无人机 $A_{3}$ 和两架新无人机 $A_{8}, A_{9}$,分别对雷达 $R_{4}, R_{2}, R_{3}$ 进行干扰。
\end{enumerate}
它们的具体飞行路径见图 5-4,5-5,5-6。

\begin{figure}[h]
    \centering
    \includegraphics[width=\textwidth]{image1.png}
    \caption{无人机 $A_{1}, A_{4}, A_{5}$ 协同干扰雷达 $R_{2}, R_{1}, R_{5}$ 的飞行路径}
    \label{fig:5-4}
\end{figure}

\begin{figure}[h]
    \centering
    \includegraphics[width=\textwidth]{image2.png}
    \caption{无人机 $A_{2}, A_{6}, A_{7}$ 协同干扰雷达 $R_{3}, R_{1}, R_{2}$ 的飞行路径}
    \label{fig:5-5}
\end{figure}

\begin{figure}[h]
    \centering
    \includegraphics[width=\textwidth]{image3.png}
    \caption{无人机 $A_{3}, A_{8}, A_{9}$ 协同干扰雷达 $R_{4}, R_{2}, R_{3}$ 的飞行路径}
    \label{fig:5-6}
\end{figure}

\section{六、问题三模型的建立和求解}

\subsection{6.1 问题三的分析}

当组网雷达系统中的某部雷达受到压制干扰或天气、油耗等其它因素的干扰时,可能无法正常获取某些时刻的回波信号。此时组网雷达系统信息融合中心会采取以下航迹维持策略:

(1) 若之前与受干扰的雷达联合检测到目标的另外两部雷达没有受到干扰,可正常检测到回波信号,融合中心即对这两部雷达检测的目标航迹点信息进行同源检验,若通过亦将其视为合理目标航迹点;

(2) 若一条航迹中此类航迹点个数不超过 3 个时(该航迹的其余航迹点仍需通过之前规定的“同源检验”),该航迹即被保留。

针对上述航迹维持策略,协同无人机编队的飞行有可能产生更多的虚假航迹。该组网雷达系统的每一部雷达的数据更新率仍为 10 秒。

本问题要求制定最佳的多无人机协同飞行策略,包括每一架无人机的使用效率、飞行时间,航迹以及相互之间的配合。针对上述航迹维持的策略,重新讨论由 9 架无人机组成的编队在 5 分钟内,完成附件 1 要求的虚假航迹的同时,至多还可产生出多少条虚假的航迹。通过协同所有无人机的协同策略,可得每一架无人机的运动规律,并分析每一条虚假航迹的运动规律及其合理性。

\subsection{6.2 模型的建立}

\subsubsection{6.2.1 多无人机协同产生多个虚假航迹}

该模型可描述为:当组网雷达系统中的某一部雷达收到压制干扰时,9 架无人机仍须在 5 分钟内完成附件 1 的虚假轨迹,与此同时,产生尽可能多的虚假航迹,对组网雷达进行干扰\cite{ref5,ref7}。且满足以下约束条件前提下,达到虚假航迹数量的最优目标。

(1) 时间限制在 $300s$ 以内,且必须形成附件 1 的虚假航迹;

(2) 组网雷达可能会受到压制干扰或者天气、油耗等因素的干扰;

(3) 无人机转弯半径不低于 $250m$,且无人机间距不小于 $100m$;

(4) 由于无人机体积较小,携带燃料少,因此无人机在飞行过程中能耗应维持在最小范围内;

(5) 无人机受材质限制,出于对无人机利用效率的考虑,其过载应在一定范围内;

(6) 为保证形成较为稳定的虚假目标航迹,无人机的飞行轨迹曲率应该尽可能小;

记最优决策变量为 $J$,为使目标函数在约束条件下最优化,即:

\[
\min J = c_1 V_u^2 + c_2 \left( \frac{\dot{r}}{V_u} \right) [1 + (\dot{\varphi})] + c_3 a_u^2,
\]

其中,$c_1, c_2, c_3, c_4$ 为各项系数,式中三部分各自对应上述条件 (4)~(6)。

\section*{6.2.2 基于竞标算法的模型}

满足
\[
\begin{cases}
120 \leq v \leq 180(km/h), \\
2000 \leq h \leq 2500(m), \\
\sum_{i} u_{i} = 9, \\
0 \leq t \leq 300(s), \\
y \geq 250(m),
\end{cases}
\]

当一架无人机 $u_{i-k}$ 位于多个雷达 $R_{k}$ 可探测跟踪的工作区域内,需对这些雷达分析竞标该目标(无人机同一时刻只能对一部雷达进行干扰,而同一时刻同一部雷达可以接受多个虚假目标点 $P_{i}$),决定由哪一架无人机来回应雷达发出的探测信号。竞标算法具体如下:

\section*{1. 建立竞标信息}

\begin{align*}
A_{ik} &= i, \\
B_{ik} &= c_{i}, \\
T_{ik} &= \tau_{ik}, \\
H_{ik} &= h_{ij}.
\end{align*}

其中,$A_{ik} = i$ 表示无人机 $u_{i-k}$ 只能对雷达 $R_{k}$ 发出信号;

$B_{ik} = c_{ik}$ 表示无人机 $u_{i-k}$ 与雷达 $R_{k}$ 的距离;

$T_{ik} = \tau_{ik}$ 表示无人机 $u_{i-k}$ 在 $t_{i}$ 时刻对雷达 $R_{k}$ 进行干扰的时刻点;

$H_{ik} = h$ 表示无人机飞行的高度。

\section*{2. 竞标目标}

假定无人机 $u_{i-k}$ 位于雷达 $R_{k}$ 的工作范围中,则这些无人机竞标这些目标,多架无人机搭载的干扰设备对接收到的雷达信号进行相应处理后转发回对应的雷达,雷达接收到转发回的干扰信号形成目标航迹点信息,传输至组网雷达信息融合中心。

假设雷达 $R_{1}$ 在无人机 $u_{i-k}$ 和无人机 $u_{j-k}$ 干扰的附近区域,无人机 $u_{i-k}$ 产生的回波信号和无人机 $u_{j-k}$ 产生的回波信号同时被雷达 $R_{k}$ 接收。两架无人机均主动比较各自的干扰优势,记作 $c_{ij}$。若 $c_{ij} < c_{jj}$,则无人机 $u_{i-k}$ 更新数据,即 $A_{ik} = R_{k}$,$B_{ik} = c_{ik}$,$T_{ik} = \tau_{ik}$,$H_{ik} = h_{ij}$。当 $c_{ij} = c_{jj}$ 时,再比较时刻点 $\tau_{ik} < \tau_{jk}$,则无人机 $u_{j-k}$ 更新信息。

\section*{3. 中标目标}

假设在步骤(2)中,竞标获胜的无人机为 $u_{i-k}$,则令 $x_{ij} = 1$,即把无人机 $u_{i-k}$ 分配给雷达 $R_{k}$ 来实行干扰任务。假设雷达 $R_{k}$ 周围的无人机均不能满足 $x_{ij}^{k} \leq J$,则

从相邻的区域分派无人机对雷达进行干扰。

\section*{6.3 模型的求解}

通过对模型的分析和计算,以及上述问题二中所求出的最佳虚假航迹,在组网雷达系统中的某部雷达受到压制干扰或其它因素的干扰的前提下,为完成附件 1 中要求的虚假航迹的同时,重新规划 9 架无人机在 5 分钟内编队组成尽量多的虚假轨迹。本问的多无人机规划方案为优先选择不受其他因素影响的雷达(即假设雷达可以一直进行接收信号并作出回应)。9 架无人机在 5 部雷达可影响的工作区域内进行协同飞行,本题求解出 9 架无人机在 5 分钟内最多可能产生出 9 条虚假轨迹以供欺骗雷达。

\section*{1. 具体 9 架无人机飞行轨迹如下:}

为了保证所有无人机在 5 分钟内形成附件 1 的虚假航线,且考虑到组网雷达系统中的某部雷达受到压制干扰时,可能在某些时刻无法正常获取回波信号。现规定,所有无人机均在第一个时刻 \( t_1 \) 从固定高度 \( h \) 飞出,通过进行爬升、俯冲等机动动作,使其在规定时间内尽量与其他所有的无人机在不发生碰撞的情况下,融合更多的虚假目标点,从而形成更多的虚假轨迹。

如下图所示,不难看出,9 架无人机均做出了上升与俯冲这两类机动动作,且其飞行高度均控制在 \( 2000m \sim 2500m \) 内。且随着时刻 \( t \) 的变化,9 架无人机的飞行轨迹,航行角均发生改变。值得一提的是、在本问题中,为了方便控制无人机的飞行轨迹,模拟出更多的虚假航迹,不妨控制每架无人机的飞行速度。

\begin{figure}[h]
    \centering
    \includegraphics[width=0.48\textwidth]{image1.png}
    \caption{1-3 号无人机飞行轨迹}
    \label{fig:6-1}
\end{figure}
\begin{figure}[h]
    \centering
    \includegraphics[width=0.48\textwidth]{image2.png}
    \caption{4-5 号无人机飞行轨迹}
    \label{fig:6-2}
\end{figure}

\begin{figure}[h]
    \centering
    \includegraphics[width=0.48\textwidth]{image3.png}
    \caption{6-7 号无人机飞行轨迹}
    \label{fig:6-3}
\end{figure}
\begin{figure}[h]
    \centering
    \includegraphics[width=0.48\textwidth]{image4.png}
    \caption{8-9 号无人机飞行轨迹}
    \label{fig:6-4}
\end{figure}

\section*{2. 由三架无人机形成的虚假航迹坐标位置信息如下:}

本文中给出了 5 部雷达的坐标位置信息, 由图 4-2 可知, 5 部雷达之间存在坐标对称等信息。不难计算出雷达 $R_{i}$ 和雷达 $R_{j}$ 两两之间的距离, 其中 $d_{12}=d_{15}$, $d_{13}=d_{14}$, $d_{23}=d_{45}$, $d_{24}=d_{35}$, 这为安排无人机的运动规律提供了一定的参考。以下给出 9 条轨迹线中的 3 条轨迹线的具体坐标位置信息。示例如下:

\begin{table}[h]
\centering
\caption{第一条虚假航迹 20 个目标点对应的位置信息}
\begin{tabular}{|c|c|c|c|c|}
\hline
时刻 & 1 & 2 & 3 & 4 \\
\hline
坐标 (km) & (6.12, 7.02, 0.81) & (6.17, 7.01, 0.81) & (6.23, 7.00, 0.81) & (6.29, 6.98, 0.80) \\
\hline
时刻 & 5 & 6 & 7 & 8 \\
\hline
坐标 (km) & (6.3, 6.97, 0.80) & (6.41, 6.95, 0.79) & (6.47, 6.93, 0.79) & (6.52, 6.90, 0.78) \\
\hline
时刻 & 9 & 10 & 11 & 12 \\
\hline
坐标 (km) & (6.58, 6.87, 0.77) & (6.63, 6.84, 0.76) & (6.68, 6.80, 0.75) & (6.73, 6.76, 0.74) \\
\hline
时刻 & 13 & 14 & 15 & 16 \\
\hline
坐标 (km) & (6.78, 6.73, 0.73) & (6.82, 668, 71) & (6.86, 6.64, 0.70) & (6.89, 6.59, 0.68) \\
\hline
时刻 & 17 & 18 & 19 & 20 \\
\hline
坐标 (km) & (6.93, 6.55, 0.67) & (6.96, 6.50, 0.65) & (6.99, 6.54, 0.63) & (7.01, 6.40, 0.61) \\
\hline
\end{tabular}
\end{table}

\begin{table}[h]
\centering
\caption{第二条虚假航迹对应的 20 个目标点的位置信息}
\begin{tabular}{|c|c|c|c|c|}
\hline
时刻 & 1 & 2 & 3 & 4 \\
\hline
坐标 (km) & (7.53, 1.72, .20) & (7.54, 1.75, 20) & (7.55, 1.78, .20) & (7.55, 1.82, 0.21) \\
\hline
时刻 & 5 & 6 & 7 & 8 \\
\hline
坐标 (km) & (7.56, 1.85, .21) & (7.57, 1.88, .21) & (7.57, 1.91, 0.21) & (7.58, 1.95, 0.22) \\
\hline
时刻 & 9 & 10 & 11 & 12 \\
\hline
坐标 (km) & (7.59, 1.98, .22) & (7.59, 2.01, .22) & (7.60, 2.04, 0.22) & (7.61, 2.08, 0.22) \\
\hline
时刻 & 13 & 14 & 15 & 16 \\
\hline
坐标 (km) & (7.61, 2.11, .22) & (7.62, 2.14, .23) & (7.62, 2.17, 0.23) & (7.63, 2.21, 0.23) \\
\hline
时刻 & 17 & 18 & 19 & 20 \\
\hline
坐标 (km) & (7.63, 2.24, .23) & (7.63, 2.27, .22) & (7.63, 2.31, 0.22) & (7.63, 2.34, 0.22) \\
\hline
\end{tabular}
\end{table}

\begin{table}
\centering
\caption{表6-3 第二条虚假航迹对应的20个目标点的位置信息}
\begin{tabular}{c c c c c}
\hline
时刻 & 1 & 2 & 3 & 4 \\
\hline
坐标 & (11.31, 6.24, 20) & (11.27, 625, 20) & (11.24, 626, 21) & (11.21, 626, 21) \\
(km) & & & & \\
\hline
时刻 & 5 & 6 & 7 & 8 \\
\hline
坐标 & (11.11, 6.26, 22) & (11.14, 626, 22) & (11.11, 626, 22) & (11.08, 626, 23) \\
(km) & & & & \\
\hline
时刻 & 9 & 10 & 11 & 12 \\
\hline
坐标 & (11.04, 6.26, 23) & (11.01, 626, 23) & (10.98, 625, 24) & (10.94, 625, 24) \\
(km) & & & & \\
\hline
时刻 & 13 & 14 & 15 & 16 \\
\hline
坐标 & (10.91, 6.24, 24) & (10.88, 623, 24) & (10.85, 622, 24) & (10.82, 621, 24) \\
(km) & & & & \\
\hline
时刻 & 17 & 18 & 19 & 20 \\
\hline
坐标 & (10.79, 6.19, 24) & (10.76, 618, 24) & (10.73, 617, 23) & (10.70, 615, 23) \\
(km) & & & & \\
\hline
\end{tabular}
\end{table}

\section{七、模型的改进和评价}

组网雷达在战场上可以构成全方位、立体化、多层次的战斗体系,能够扩展时间、空间的探测范围,具有很强的抗干扰能力,在军事中得到了广泛应用。因此,如何对组网雷达实施行之有效的干扰,是亟待解决的重要问题。诸多干扰方式中较为有效的是欺骗干扰,用多架无人机携带干扰装置对组网雷达进行协同干扰,是现今运用最为广泛的方式之一。

\subsection{7.1 模型的算法和复杂度分析}

针对问题一,在预设虚假目标航迹的前提下,对无人机运动轨迹的设计问题进行了研究。从最优控制的角度出发,采用多无人机协同干扰组网雷达的优化模型,研究动力学约束和几何约束下的多无人机协同干扰飞行航迹设计方案。利用枚举法对模型进行求解,具体得到每一架无人机的飞行航迹,即相应的协同策略。

针对问题二,采用了协同航迹欺骗的最优控制模型和无人机飞行路径优化模型。这两个模型将题中所给信息转化为数据信息,并在求解过程中进行数据存储。对无人机飞行航迹进行坐标分析后,运用计算机仿真得到虚假目标航迹。

针对问题三,在问题二的基础上加入某些不确定性因素,如能耗问题、材料问题、轨迹曲率问题等。对于目标模型运用竞标算法,通过在雷达可探测跟踪的工作区域内选择目标,建立目标的竞标信息,无人机之间通过彼此之间的信息传递完成目标任务的分配问题。

\subsection{7.2 模型的推广}

本文中采用的枚举法利用了计算机运算速度快、精确度高的特点,可考虑所有可能情况,从中得到全局最优解。但枚举法无法对数据量过大的信息进行处理,这是它的局限所在。

计算机仿真分析可与理论推导相结合,考虑到干扰设计方法的有效性和可行性,可达到对组网雷达系统航迹欺骗干扰的目的,但当系统中的逻辑判断环节较多时,仿真较为困难。

竞标算法不仅能够实现目标的无冲突分配,同时也保证无人机执行任务的效率。但此算法要求较高,它需要保证目标周边区域的无人机之间竞标信息有效传递。

\section*{八、参考文献}

[1] 孙琳, 李小波, 毛云祥, 徐旭宇. 基于多机协同的组网雷达欺骗干扰策略[J]. 电子信息对抗技术, 2016,31(03):51-54+63.

[2] 郭淑芬, 余国文, 熊鑫, 徐鹏. 基于无人机协同的航迹欺骗干扰方法研究[J]. 空军预警学院学报, 2018,32(01):44-47+54.

[3] 李飞, 周中良, 苟新禹, 陈中起. 基于多机协同航迹欺骗的组网雷达突防技术[J]. 系统工程与电子技术,2013,35(11):2309-2313.

[4] 朱宇. 对组网雷达的多机协同航迹欺骗干扰技术[J]. 2015 年第 4 期 2015,No.4.

[5] 宗思光, 刘涛, 梁善永. 基于改进遗传算法的干扰资源分配问题研究[J]. 电光与控制, 2018,25(05):41-45.

[6] 赵明. 多无人机系统的协同目标分配和航迹规划方法研究[D]. 哈尔滨工业大学, 2016.

[7] 张远, 方青, 曲成华. 基于遗传算法的组网雷达优化部署[J]. 雷达科学与技术, 2014,12(01):76-80.

[8] 袁天, 陶建锋, 李兴成. 飞行器编队协同的虚假航迹产生仿真研究[J]. 计算机仿真, 2017,34(08):100-104.

[9] McClain T W, Chandler P R, Pachter M. Decomposition Strategy for Optimal Coordination of Unmanned Air Vehicles [C] // Proceedings of the 2000 American Control Conference, 2000, 6: 691-695.

[10] Jian Chen, Darren M. Dawson, Mohammad Salah, Timothy Burg. Cooperative control of multiple vehicles with limited sensing[J]. International Journal of Adaptive Control and Signal Processing, 2007, 21.