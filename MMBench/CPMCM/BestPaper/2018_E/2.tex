\documentclass{article}
\usepackage{titling}

\title{多无人机对组网雷达的协同干扰}
\date{}

\begin{document}

\maketitle

\begin{abstract}
组网雷达系统相比单部雷达,具有空间覆盖范围更广、目标检测概率更高和抗干扰能力更强的特点,因而在军事特别是现代战争中得到广泛的应用。如何对组网雷达实施行之有效的干扰,已成为当前世界电子对抗界面临的一个重大问题。

本文主要考虑多架无人机对组网雷达系统的协同干扰问题,即采用适当的协同策略,通过协同控制多架无人机的飞行轨迹,在敌方组网雷达系统中形成一条或者多条虚拟目标航迹,从而达到欺骗的目的。根据无人机协同干扰组网雷达的工作机理,我们提出了合理假设,建立起多无人机协同干扰组网雷达的优化模型,给出了每架无人机的运动规律和协同策略,此外还分析了虚假航迹的运动规律和合理性。

针对问题一,按照题目给定的虚假目标航迹,无人机做匀速直线运动,航向、航速和飞行高度都在允许范围内,本文提出了一系列合理假设,通过对无人机飞行规律的深入分析,建立了多无人机协同干扰组网雷达的优化模型;然后利用枚举法对该模型进行求解;进而设计多无人机飞行方案以确保用最少架次的无人机实现附件 1 要求的虚假目标航迹,注意到,枚举法利用了计算机具有运算速度快、计算精确度高的特点,可考虑所有可能情况,从中找出符合要求的无人机运动规律,即可得到全局最优解。计算结果表明,安排 31 架无人机协同飞行,可以产生题目附件 1 所要求的虚假目标航迹。此外,我们还分析了每架无人机的运动规律和协同策略(见表 4-3)。相应的计算结果已经按照规定格式存入附件 2。

针对问题二,需要在实现问题一所要求的虚假轨迹条件下,优化无人机编队的飞行轨迹使得所形成的虚假航迹最多。在问题一的求解基础上,我们进一步分析了无人机的飞行规律,在保证转弯半径、飞行高度和速度都在合理范围内的情况下,考虑了 9 架无人机组成的编队在 5 分钟内的飞行情况,建立了协同航迹欺骗的最优控制模型。计算机仿真结果表明,按照该模型的协同策略,除了能产生附件 1 要求的虚假航迹外,还可另外产生 3 条虚假目标航迹。与此同时,我们还
\end{abstract}

\section{分析了协同飞行的每架无人机的运动规律及其合理性}
相应的计算结果已经按照规定格式存入附件 3。

针对问题三,当组网雷达系统中的某部雷达受到压制或其他因素干扰时,我们采用题目给定的航迹维持策略,在满足无人机飞行高度和速度等限定条件下,基于问题二,重新建立了多无人机协同干扰组网雷达的目标分配模型,运用竞标算法,通过在雷达可探测跟踪的工作区域内选择目标,建立目标的竞标信息,无人机彼此之间实行信息传递,并完成目标任务的分配。我们的计算及分析结果表明,由 9 架无人机组成的编队在 5 分钟内,除了完成附件 1 要求的虚假航迹外,至多还可产生 9 条虚假航迹。此外,我们还发现该算法既可实现目标无冲突分配,也可保证无人机执行任务的效率。

\textbf{关键词:多无人机 组网雷达 优化模型 枚举法 竞标算法}

\section*{目录}

\section{一、问题重述}
\subsection{1.1 问题背景} \dotfill 4
\subsection{1.2 需要解决的问题} \dotfill 5

\section{二、模型假设} \dotfill 6

\section{三、符号说明} \dotfill 6

\section{四、问题一模型的建立和求解} \dotfill 7
\subsection{4.1 问题一的分析} \dotfill 7
\subsection{4.2 数据处理} \dotfill 7
\subsection{4.3 模型的建立} \dotfill 9
\subsubsection{4.3.1 多无人机协同干扰组网雷达的优化模型} \dotfill 9
\subsubsection{4.3.2 基于枚举法的模型的目标优化} \dotfill 10
\subsection{4.4 模型的求解} \dotfill 10
\subsubsection{4.4.1 模型求解结果} \dotfill 13

\section{五、问题二模型的建立和求解} \dotfill 16
\subsection{5.1 问题二的分析} \dotfill 16
\subsection{5.2 虚假目标和无人机运动模型} \dotfill 16
\subsection{5.3 协同航迹欺骗的最优控制模型} \dotfill 18
\subsection{5.4 模型的求解} \dotfill 20
\subsubsection{5.4.1 附件 1 虚假航迹的产生} \dotfill 20
\subsubsection{5.4.2 其它虚假航迹的产生} \dotfill 21

\section{六、问题三模型的建立和求解} \dotfill 23
\subsection{6.1 问题三的分析} \dotfill 23
\subsection{6.2 模型的建立} \dotfill 23
\subsubsection{6.2.1 多无人机协同产生多个虚假航迹} \dotfill 23
\subsubsection{6.2.2 基于竞标算法的模型} \dotfill 24
\subsection{6.3 模型的求解} \dotfill 25

\section{七、模型的改进和评价} \dotfill 28
\subsection{7.1 模型的算法和复杂度分析} \dotfill 28
\subsection{7.2 模型的推广} \dotfill 28

\includegraphics[width=0.3\textwidth]{image1.png} \quad
\includegraphics[width=0.3\textwidth]{image2.png} \quad
\includegraphics[width=0.3\textwidth]{image3.png} \\
\includegraphics[width=0.3\textwidth]{image4.png} \quad
\includegraphics[width=0.3\textwidth]{image5.png} \quad
\includegraphics[width=0.3\textwidth]{image6.png}

\begin{table}[h]
\centering
\begin{tabular}{l l}
学校 & 上海师范大学 \\
\hline
参赛队号 & 18102700005 \\
\hline
队员姓名 & 1. 武育宏 \\
 & 2. 谢菁华 \\
 & 3. 魏怿琛 \\
\end{tabular}
\end{table}

[REFERENCES:1]

\end{document}