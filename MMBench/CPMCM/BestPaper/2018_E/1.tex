\begin{center}
\textbf{“华为杯”第十五届中国研究生数学建模竞赛}
\end{center}

\begin{center}
\textbf{题目} \quad 多无人机对组网雷达的协同干扰
\end{center}

\begin{center}
\textbf{摘 \quad 要:}
\end{center}

文章研究多无人机协同干扰的规划问题,求解协同策略和飞行方案,以对组网雷达实施有效的欺骗干扰。多无人机中任何一架无人机的航迹规划一旦差之毫厘、“同源检验”就有可能失之交臂。这就要求对每一架无人机的飞行路径和它们之间的协同策略进行周密的安排和精确的计算。使用直角坐标系直接描述多无人机的动态过程、空间关系及协同策略是十分困难的。本文采用拓扑学的思想搭建立体几何模型,简化了问题的求解过程。

拓扑学告诉我们,空间物体在形变的过程中,物体上点和点之间存在一些相对稳定的次序比邻关系,这些关系不随它们之间距离和方向的变化而变化。类似地,根据无人机协同工作的原理,可以推断出虚假航迹点、信号发射器角色、雷达点之间总会保持一些稳定的几何关系。让这些关系保持稳定,就是无人机合作的协同策略。找出这些关系,配合每个时刻假目标航迹点坐标的具体情况,可以求解出扮演信号发射器角色的无人机在每个时刻所需要到达的位置坐标,这就把随时间变化的“几何角色”转化成“何时赶往何处”的四维任务。在给无人机分配任务的时候,需要用优化统筹的思路来决定由哪架无人机扮演哪个角色,找到可行的航迹规划方案。通过以上步骤,可以规划出每一架无人机、每一条虚假航迹的合适坐标。文章给出了全部相关坐标,参见附件。

问题一要求规划一个协同方案,用尽可能少的无人机产生出给定的虚假航迹路线。通过处理数据可以发现,安排全部无人机在高度 $2.5 \mathrm{~km}$ 平面飞行,可以用较少的无人机完成全部的干扰工作。结合“距离欺骗”的基本原理,我们发现扮演干扰信号发射器角色的无人机需要在雷达收发信号的时刻出现在“雷达 - 假目标”连线与高度 $2.5 \mathrm{~km}$ 平面的交点上。这个几何性质是不随“假目标”坐标的变换而变化的。这就可以解出,要产生这样的假目标,5 个雷达在 20 个时刻所对应的无人机的具体坐标,也就得到 100 个“何时须到何处”的具体任务。删减多余的任务,并把可由同一架飞机完成的任务做合并处理,最终求解出使用最少无人机完成任务的多无人机协同干扰方案。研究发现,在平面匀速直线运动的条

件约束下,产生全部虚假航迹点最少需要 31 架无人机。它们分工协作,在 20 个时刻里对 2 号雷达、3 号雷达和 4 号雷达实施干扰。

问题二允许九架无人机在 5 分钟内机动飞行,要求在产生指定航迹的基础上,尽可能地产生更多有效虚假航迹路线。通过计算和比较,我们选择对 1、2、5 三个雷达实施干扰,让全部无人机在高度 2.0km 平面飞行。针对该问题,我们构造了以“规定航迹点”和 1、2、5 三台雷达为顶点的四面体模型。根据棱边、侧面、平面等一系列的几何关系,可以求解出信号发射器之间协同关系的约束条件和协同策略。通过分析几何模型可以发现,在产生成指定的 1 条虚假航迹线基础上,最多还可以生成 3 条新的虚假航迹线。通过优化计算,可以求解出较好的任务分配方案,这样就可以确定第 1-20 时刻每一架无人机的相应坐标,同时也就确定了虚假航迹点在每个时刻的具体坐标。

问题三允许航迹路线存在瑕疵,即连续的 20 个航迹点当中,若不少于 17 个点满足“三雷达同源检验”,且剩余点满足“双雷达同源检验”,雷达就可将该航迹路线视为有效航迹路线。针对这样的条件,我们在前一个四面体模型的基础上,寻找可暂时离岗的无人机。借助模型研究发现,在 12-20 时刻期间,每期可安排一架无人机离岗。离岗无人机配合其它无人机,能在这些时刻产生出一连串新的航迹点。有了这些新的航迹点,多无人机在完成指定的 1 条虚假航迹线基础上,最多还可以产生 4 条新的有效航迹线。这 1 条多出来的航迹线从 11 时刻开始产生,持续到第 30 时刻。通过几何分析,可以描述出每一个“发射器角色”的协同策略;通过优化计算,可以把角色合理地分配给无人机,进而计算出无人机坐标和虚假航迹点坐标。

每回答一个问题,我们都根据题目要求,对无人机间距、加速情况、拐弯情况是否满足要求做检验。我们还以主流第四代战斗机相关参数为标准,对假目标的瞬时加速度、机动能力等指标做了检查,确保虚假航迹路线是可信的。最后,作为本文的一个重要创新,文章专门编写程序重复抽样,对虚假航迹点能否通过“同源检验”做了检查。研究发现,计算结论与模型推理相一致,误差可控、结论可靠。由此可见,借鉴拓扑学思想可以极大地简化协同问题的规划,该思路有广泛的应用价值。

\textbf{关键词:电子对抗 协同干扰 立体几何 运筹优化}

\begin{itemize}
    \item[1.] 问题重述 \dotfill 4
    \item[2.] 模型假设 \dotfill 6
        \begin{itemize}
            \item[2.1] 题目对模型给出的假设 \dotfill 6
            \item[2.2] 为简化模型的求解而追加的假设 \dotfill 6
            \item[2.3] 符号说明 \dotfill 6
        \end{itemize}
    \item[3.] 问题一的建模与求解 \dotfill 7
        \begin{itemize}
            \item[3.1] 对问题的可视化分析 \dotfill 7
            \item[3.2] 问题的解答 \dotfill 9
            \item[3.3] 无人机的运动合理性检验 \dotfill 13
        \end{itemize}
    \item[4.] 问题二的建模与求解 \dotfill 14
        \begin{itemize}
            \item[4.1] 无人机机动能力的可视化分析 \dotfill 14
            \item[4.2] 问题的解答 \dotfill 14
                \begin{itemize}
                    \item[4.2.1] 产生规定航迹点 \dotfill 15
                    \item[4.2.2] 研究四面体模型归纳无人机与假目标的几何特征 \dotfill 16
                    \item[4.2.3] 计算无人机和假目标在各时刻的实际坐标 \dotfill 20
                \end{itemize}
            \item[4.3] 无人机和假目标航迹的运动合理性检验 \dotfill 22
        \end{itemize}
    \item[5.] 问题三的建模与求解 \dotfill 24
        \begin{itemize}
            \item[5.1] 问题的解答 \dotfill 24
                \begin{itemize}
                    \item[5.1.1] 寻找最优路径产生规定航迹点 \dotfill 24
                    \item[5.1.2] 无人机位置的几何特征 \dotfill 24
                    \item[5.1.3] 无人机及假目标航迹的航迹优化和坐标求解 \dotfill 25
                \end{itemize}
            \item[5.2] 无人机和假目标航迹的运动合理性检验 \dotfill 28
        \end{itemize}
    \item[6.] 计算结果的精度评价 \dotfill 29
    \item[7.] 结论 \dotfill 30
    \item[] 参考文献 \dotfill 30
    \item[] 附录 \dotfill 31
\end{itemize}

\section{问题重述}

距离观测的基本原理是根据电磁波从发射到反射所用的时间,来判断目标与雷达间的距离。而距离假目标欺骗的原理是,干扰机对雷达信号进行处理之后,再反射给对应的雷达。通过控制电子波返回雷达的时间,可以给雷达对距离的判断带来误导。这种干扰有一个基本的特征,假目标一定在无人机与雷达的连线上。

\begin{figure}[h]
    \centering
    \includegraphics[width=0.8\textwidth]{image1.png}
    \caption{对雷达实施距离多假目标欺骗干扰示意图(图片来自原题题干)}
\end{figure}

组网雷达系统具有良好的抗电子干扰、抗低空突防能力,得到越来越广泛的应用\cite{ref1}。对敌方雷达实施有效干扰是十分困难的,原因在于一旦敌方使用多个雷达,不同雷达所发现的某目标空间状态可能会不一致,进而识破无人机的欺骗。这种剔除虚假目标的思想简称为“同源检验”\cite{ref2}。根据题目要求,至少让三个雷达监控到一致的航迹点时,才算是虚假航迹通过了“同源检验”,进而被融合中心视为一个真实的航迹点。为了同时欺骗多台雷达,就需要利用多架无人机的协同欺骗,让多个雷达看到同一个虚假目标,这样就达到了欺骗敌方雷达的效果\cite{ref3}。

\begin{figure}[h]
    \centering
    \includegraphics[width=0.8\textwidth]{image2.png}
    \caption{多无人机协同干扰组网雷达系统示意图(图片来自原题题干)}
\end{figure}

问题一要求产生指定航线。题目给出虚假目标航迹路线,讨论如何以最少数量的无人机在匀速直线飞行的约束性,产生出这些虚假目标航迹点,并分析每一架无人机的运动规律和相应的协同策略。

问题二要求求解,使用九架飞机机动飞行,如何在产生规定航线的同时再产生尽可能多的有效航迹路线。题目要求给出每一架无人机的运动规律,并分析每

\newpage

\section{问题分析}

\section{模型假设}

\section{符号说明}

\section{模型建立与求解}

\section{模型评价与改进}

\section{参考文献}

\begin{thebibliography}{99}
    \bibitem{ref1} 参考文献1
    \bibitem{ref2} 参考文献2
    \bibitem{ref3} 参考文献3
\end{thebibliography}

一条虚假航迹的运动规律和合理性。

问题三中的雷达改变了航迹维持策略,要求求解在新的航迹约束下,重新讨论由 9 架无人机组成的编队在 5 分钟内,完成附件 1 要求的虚假航迹的同时,至多还可产生出多少条虚假的航迹。题目要求给出每一架无人机的运动规律和协同策略,并分析每一条虚假航迹的运动规律和合理性。

\section{模型假设}

\subsection{题目对模型给出的假设}

1. 敌方雷达系统已知,且只要有 3 部以上雷达的航迹点状态信息通过了同源检验,敌方融合中心就将其确定为一个合理的航迹点。

2. 无人机的飞行速度在 $120 \mathrm{~km} / \mathrm{h}-180 \mathrm{~km} / \mathrm{h}$,飞行高度控制在 $2000 \mathrm{~m}-2500 \mathrm{~m}$,最大加速度不超过 $10 \mathrm{~m} / \mathrm{s}^{2}$。

3. 无人机间距控制在 $100 \mathrm{~m}$ 以上,足以保证安全、避免相撞。

4. 真实目标产生的回波不会被监测到,而干扰设备产生的欺骗干扰信号一定能被有效地监测到。

5. 同一时刻一架无人机只能干扰一部雷达,但可以多次干扰该雷达,产生的目标点位于雷达与无人机的连线及延长线上。

6. 同一时刻多架无人机可以干扰同一部雷达,不同时刻可以干扰不同雷达。

\subsection{为简化模型的求解而追加的假设}

7. 在问题 1 中,无人机在该空域指定平面做匀速直线运动,不妨假设全部飞机均在统一平面飞行,只有在濒临相撞的时刻做高度调整。

8. 无人机可以瞬间完成加速。

9. 无人机可以瞬间完成 $20.6^{\circ}$ 以内的方向调整,而 $20.6^{\circ}$ 以上的方向调整对无人机在一个周期内的飞行轨迹具有较大的影响,需要单独考察。

10. 假设电磁波在空气中运动所花费的时间为 0。

11. 虚假航迹路线所描述的假目标飞行路径连续,最大速度、瞬时加速等性能不超过主流四代战斗机的一般性能。即最大速度在 2.5 倍音速以内,加速度在 4 倍重力加速度以内,否则视为航迹不连续。

\subsection{符号说明}

\begin{tabular}{ll} 
符号 & 说明 \\
\hline 
$j=\{\text{``A''}, \ldots, \text{``E''}\}$ & 用字母给雷达编号 \\
$i=\{\mathrm{P}_{\mathrm{p}}, \mathrm{Q}\}$ & 用字母给虚假航迹编号,其中规定航迹点编为 $Q$,优化生成的编为 $\mathrm{P}_{\mathrm{p}}$ \\
$M_{j i}$ & 为信号发射器的角色编号,向 $j$ 雷达发射信号,产生 $i$ 航迹点 \\
$D(t, x, y, z)$ & 在 $t$ 时刻出现在坐标 $(x, y, z)$ 处的飞行任务。 \\
$k$ & 无人机的编号 \\
$t$ & 时刻编号,令雷达在 1 时刻观测到第一个假目标航迹点 \\
$v_{k}$ & 无人机的速度 \\
$(x, y, z)$ & 平面直角坐标系下的坐标标记 \\
$T$ & 雷达发射周期,本题中为常数 $10 \mathrm{~s}$ \\
\end{tabular}

\section{问题一的建模与求解}

\subsection{对问题的可视化分析}

根据附件给出的假目标航迹点,可以绘制出假目标的航迹图如下:

\begin{figure}[h]
    \centering
    \includegraphics[width=0.8\textwidth]{image1.png}
    \caption{题目中虚假目标航迹线轨迹图}
    \label{fig:1}
\end{figure}

图 3 题目给出的假目标航迹路线示意图

通过观察数据发现,相邻的两个虚假航迹点,在距离上十分接近。倘若无人机所飞行的平面较低,则会导致任务点与任务点之间的距离太近。而无人机需要保持高速直线飞行,若要想让两个任务交给同一架飞机来完成,就要求两个任务在空间上有稍长的距离。为此,提高无人机的飞行平面,规定干扰机均在高度 2.5 千米处实施干扰。

不妨将干扰 1 号雷达的航空区域称为 A 区域,1 号雷达简称为 A 雷达,所在的坐标定义为 A 点;干扰雷达 2 的航空区域称为 B 区域,2 号雷达简称为 B 雷达,所在的坐标定义为 B 点;干扰雷达 3 的航空区域成为 C 区域,3 号雷达简称为 C 雷达,所在的坐标定义为 C 点;干扰雷达 4 的航空区域称为 D 区域,4 号雷达简称为 D 雷达,所在的坐标定义为 D 点;干扰雷达 5 的航空区域成为 E 区域,5 号雷达简称为 E 雷达,所在的坐标定义为 E 点。设题目规定的虚拟航线点为 Q 点。

俯视高度 2.5km 的平面,需要有一台信号发射器扮演这样的角色 $M_{AQ}$,向 A 雷达发射来自虚假目标 Q 的信号。$M_{AQ}$ 的空间坐标或许一直在变化,但它与周边图形的关系是稳定的,该点一直都在 AQ 连线与高度 2.5km 平面的交汇处。

\begin{figure}[h]
    \centering
    \includegraphics[width=0.8\textwidth]{image2.png}
    \caption{多无人机协同干扰组网雷达系统示意图}
    \label{fig:2}
\end{figure}

定义 $\mathbf{M}_{\mathrm{AQ}}$ 坐标为 $(x_{\mathrm{t}}, y_{\mathrm{t}}, 2.5)$,$\mathbf{M}_{\mathrm{AQ}}$ 与雷达之间的距离为 $l_{\mathrm{a}}$,投影到 $x$ 轴上的长度为 $x_{\mathrm{a}}$,投影到 $y$ 轴的长度为 $y_{\mathrm{a}}$,投影到 $z$ 轴的长度为 $z_{\mathrm{a}}$;另,让假目标与交点的距离为 $l_{\mathrm{b}}$,投影到 $x$ 轴上的长度为 $x_{\mathrm{b}}$,投影到 $y$ 轴的长度为 $y_{\mathrm{b}}$,投影到 $z$ 轴的长度为 $z_{\mathrm{b}}$。根据 $z=2.5$ 平面与 $z=0$ 平面相平行,可以根据相似三角形原理得出如下数量关系:

\[
\frac{l_{\mathrm{a}}}{l_{\mathrm{b}}} = \frac{x_{\mathrm{a}}}{x_{\mathrm{b}}} = \frac{y_{\mathrm{a}}}{y_{\mathrm{b}}} = \frac{z_{\mathrm{a}}}{z_{\mathrm{b}}}.
\tag{3.1}
\]

已知 1 时刻假目标的坐标为 $(60.600, 69.982, 7.995)$,1 号雷达的坐标为 $(80.00, 0, 0)$,为求干扰的任务点坐标,设方程组如下:

\[
x_{1} = 80 + x_{\mathrm{a}} / (x_{\mathrm{a}} + x_{\mathrm{b}})(60.6 - 80) = 80 + z_{\mathrm{a}} / (z_{\mathrm{a}} + z_{\mathrm{b}})(60.6 - 80),
\]
\[
y_{1} = 0 + x_{\mathrm{a}} / (x_{\mathrm{a}} + x_{\mathrm{b}})(69.982 - 0) = z_{\mathrm{a}} / (z_{\mathrm{a}} + z_{\mathrm{b}})(69.982 - 0),
\]
\[
z_{\mathrm{a}} / z_{\mathrm{b}} = 2.5 / (7.995 - 2.5).
\]

解得,在 1 时刻,无人机出现在坐标 $(79.39971, 21.88305, 2.5)$ 处,可以给 A 雷达带来干扰,使得雷达误以为目标在坐标为 $(60.600, 69.982, 7.995)$ 的位置。为了在 1 时刻造成统一的假象,B 雷达也应收到来自同样虚假信号源的信号,类似地可以求解 1 时刻 $\mathbf{M}_{\mathrm{BQ}}$ 坐标为 $(39.56848, 63.12133, 2.5)$,对应 C 雷达的 $\mathbf{M}_{\mathrm{CQ}}$ 在 1 时刻坐标为 $(56.75109, 97.48655, 2.5)$。类似地,我们可以全部 20 个时点,用来干扰 A-E 雷达的无人机坐标。

\begin{figure}[h]
\centering
\includegraphics[width=0.8\textwidth]{image.png}
\caption{20 个时刻 50 个干扰任务的坐标图}
\end{figure}

按照要求,需要在 200 秒的时间内完成全部干扰工作。经过计算发现,短短 200s 的时间并不足以让一架飞机从 A 区域飞到 B-E 等其它区域。也就是说,每架飞机只能干扰所在区域对应的雷达,而不能去干扰其它雷达。这个发现简化了问题的运算。

本题的解答分为如下几个步骤:

- 第一步:首先观察无人机所能飞行的最高($z=2.5\mathrm{km}$)极端高度,关注对应的区域 A,计算从 1 时刻到 20 时刻期间,恰好完成 T 周期时,干扰机所应当对应坐标。定义无人机所需要执行的干扰任务。
- 第二步:计算 $2.5\mathrm{km}$ 平面上,B 区域、C 区域、D 区域和 E 区域的飞机航线。
- 第三步:将飞行任务进行合并,寻找这样的飞机,可以满足在 $t=t_{\mathrm{p}}$ 时刻在该时刻指定的干扰位置,形成假目标 $p$,同时在 $t=t_{\mathrm{q}}$ 能恰好赶到 $t_{\mathrm{q}}$ 所指定的干扰位

\begin{table}
\centering
\caption{z=2.5km上干扰A雷达的干扰任务清单 (单位:千米)}
\begin{tabular}{c c c c}
时刻 & $x$ & $y$ & $z$ \\
\hline
1 & 73.93370857 & 21.88305191 & 2.50 \\
2 & 74.10933584 & 21.90726817 & 2.50 \\
3 & 74.27718416 & 21.94783155 & 2.50 \\
4 & 74.43718434 & 22.00536616 & 2.50 \\
5 & 74.58888889 & 22.08031746 & 2.50 \\
6 & 74.73241688 & 22.17359335 & 2.50 \\
7 & 74.86718246 & 22.28594455 & 2.50 \\
8 & 74.99283854 & 22.41861979 & 2.50 \\
9 & 75.10895326 & 22.57307439 & 2.50 \\
10 & 75.21533333 & 22.751 & 2.50 \\
11 & 75.31135903 & 22.95469912 & 2.50 \\
12 & 75.39629121 & 23.18612637 & 2.50 \\
13 & 75.46680643 & 23.45073375 & 2.50 \\
14 & 75.52207977 & 23.75249288 & 2.50 \\
15 & 75.56218182 & 24.09563636 & 2.50 \\
16 & 75.58556548 & 24.48549107 & 2.50 \\
17 & 75.59191457 & 24.92791762 & 2.50 \\
18 & 75.57954545 & 25.42985893 & 2.50 \\
19 & 75.5472155 & 26.00040355 & 2.50 \\
20 & 75.49333333 & 26.64875 & 2.50 \\
\end{tabular}
\end{table}

\begin{figure}[h]
    \centering
    \includegraphics[width=0.8\textwidth]{image1.png}
    \caption{M_{A0}在20个时刻对应的干扰任务}
    \label{fig:1}
\end{figure}

至此,为了干扰1号雷达,我们定义了20个任务,需要无人机在指定时刻到达指定位置。根据同样的原理,可以依次绘制出干扰2至5号雷达在高度2.5km处的任务点。

\begin{figure}[h]
    \centering
    \includegraphics[width=0.45\textwidth]{image2.png}
    \includegraphics[width=0.45\textwidth]{image3.png}
    \includegraphics[width=0.45\textwidth]{image4.png}
    \includegraphics[width=0.45\textwidth]{image5.png}
    \caption{其它区域20个时刻对应的干扰任务}
    \label{fig:2}
\end{figure}

结合上节,我们得到了100个任务点。考虑到,每一个假目标只需要三个无人机执行干扰任务,这就需要将上述任务进行整合。在整合任务之前,首先需要定义两个概念。

定义1:用符号D(t,x,y,z)就可以定义这样一个干扰任务,在t时刻无人机必须到达坐标(x,y,z)处。

定义2:在任务D_{1}和任务D_{2}之间,存在某种广义距离,记为DD_{1,2},单位为km/h,计算过程如下:

\begin{equation}
DD_{1,2} = \frac{\sqrt{(x_1 - x_2)^2 + (y_1 - y_2)^2 + (z_1 - z_2)^2}}{\left| t_1 - t_2 \right|}
\tag{3.2}
\end{equation}

假定两个任务点的时间差距为 \(25s\),空间距离为 \(1km\),那么两个任务间的“广义距离”为 \(144km/h\),那么可以用一架无人机以 \(144km/h\) 的速度进行平面匀速直线运动,在前一个时刻完成任务一,在 \(25s\) 后恰好到达空间距离为 \(1km\) 的任务二处,顺便完成任务二。因此,如果任务 \(\mathbf{D}_1\) 和任务 \(\mathbf{D}_2\) 之间的广义距离介于 \(120km/h-150km/h\) 之间,那么认定两个任务可由同一架无人机来完成。可以通过计算任务间的“广义距离”,判断是否可以用一架飞机同时完成两个任务,实现任务的合并。

将任务点根据时刻编号,下面分别 A 区域的任务合并图:

\begin{table}[h]
\centering
\caption{A 区域各任务的广义距离是否介于 \(120-180km/h\)}
\begin{tabular}{|c|c|c|c|c|c|c|c|c|c|c|c|c|c|c|c|c|c|c|c|}
\hline
\multicolumn{21}{|c|}{A 区域内伪装虚假航迹点的任务编号} \\
\hline
 & 1 & 2 & 3 & 4 & 5 & 6 & 7 & 8 & 9 & 10 & 11 & 12 & 13 & 14 & 15 & 16 & 17 & 18 & 19 & 20 \\
\hline
1 & 否 & 否 & 否 & 否 & 否 & 否 & 否 & 否 & 否 & 否 & 否 & 否 & 否 & 否 & 否 & 否 & 否 & 否 & 否 & 否 \\
\hline
2 & 否 & 否 & 否 & 否 & 否 & 否 & 否 & 否 & 否 & 否 & 否 & 否 & 否 & 否 & 否 & 否 & 否 & 否 & 否 & 否 \\
\hline
3 & 否 & 否 & 否 & 否 & 否 & 否 & 否 & 否 & 否 & 否 & 否 & 否 & 否 & 否 & 否 & 否 & 否 & 否 & 否 & 否 \\
\hline
4 & 否 & 否 & 否 & 否 & 否 & 否 & 否 & 否 & 否 & 否 & 否 & 否 & 否 & 否 & 否 & 否 & 否 & 否 & 否 & 否 \\
\hline
5 & 否 & 否 & 否 & 否 & 否 & 否 & 否 & 否 & 否 & 否 & 否 & 否 & 否 & 否 & 否 & 否 & 否 & 否 & 否 & 否 \\
\hline
6 & 否 & 否 & 否 & 否 & 否 & 否 & 否 & 否 & 否 & 否 & 否 & 否 & 否 & 否 & 否 & 否 & 否 & 否 & 否 & 否 \\
\hline
7 & 否 & 否 & 否 & 否 & 否 & 否 & 否 & 否 & 否 & 否 & 否 & 否 & 否 & 否 & 否 & 否 & 否 & 否 & 否 & 是 \\
\hline
8 & 否 & 否 & 否 & 否 & 否 & 否 & 否 & 否 & 否 & 否 & 否 & 否 & 否 & 否 & 否 & 否 & 否 & 否 & 否 & 是 \\
\hline
9 & 否 & 否 & 否 & 否 & 否 & 否 & 否 & 否 & 否 & 否 & 否 & 否 & 否 & 否 & 否 & 否 & 否 & 否 & 是 & 是 \\
\hline
10 & 否 & 否 & 否 & 否 & 否 & 否 & 否 & 否 & 否 & 否 & 否 & 否 & 否 & 否 & 否 & 否 & 否 & 是 & 是 & 是 \\
\hline
11 & 否 & 否 & 否 & 否 & 否 & 否 & 否 & 否 & 否 & 否 & 否 & 否 & 否 & 否 & 否 & 否 & 否 & 是 & 是 & 是 \\
\hline
12 & 否 & 否 & 否 & 否 & 否 & 否 & 否 & 否 & 否 & 否 & 否 & 否 & 否 & 否 & 否 & 否 & 是 & 是 & 是 & 是 \\
\hline
13 & 否 & 否 & 否 & 否 & 否 & 否 & 否 & 否 & 否 & 否 & 否 & 否 & 否 & 否 & 是 & 是 & 是 & 是 & 是 & 是 \\
\hline
14 & 否 & 否 & 否 & 否 & 否 & 否 & 否 & 否 & 否 & 否 & 否 & 否 & 否 & 是 & 是 & 是 & 是 & 是 & 是 & 是 \\
\hline
15 & 否 & 否 & 否 & 否 & 否 & 否 & 否 & 否 & 否 & 否 & 否 & 否 & 否 & 是 & 是 & 是 & 是 & 是 & 是 & 否 \\
\hline
16 & 否 & 否 & 否 & 否 & 否 & 否 & 否 & 否 & 否 & 否 & 否 & 否 & 是 & 是 & 是 & 是 & 是 & 否 & 是 & 否 \\
\hline
17 & 否 & 否 & 否 & 否 & 否 & 否 & 否 & 否 & 否 & 否 & 否 & 是 & 是 & 是 & 是 & 是 & 是 & 否 & 否 & 否 \\
\hline
18 & 否 & 否 & 否 & 否 & 否 & 否 & 否 & 否 & 否 & 否 & 是 & 是 & 是 & 是 & 是 & 是 & 是 & 否 & 否 & 否 \\
\hline
19 & 否 & 否 & 否 & 否 & 否 & 否 & 否 & 否 & 否 & 是 & 是 & 是 & 是 & 是 & 是 & 是 & 是 & 否 & 否 & 否 \\
\hline
20 & 否 & 否 & 否 & 否 & 否 & 否 & 否 & 否 & 是 & 是 & 是 & 是 & 是 & 是 & 是 & 是 & 是 & 否 & 否 & 否 \\
\hline
\end{tabular}
\end{table}

图 8:A 区域的任务合并图

在 A 区域当中,拟合 20 个虚假航迹点,需要定义 20 个任务。如果两个任务的“广义距离”在 \(120km/h-180km/h\) 之间的检验结果为“是”,则可以认定两个任务可由同一架无人机完成。考虑到没有三个点恰好在同一直线的情况,我们仅考虑任务点间的两两合并。以上图为例,第 7 个任务与第 20 个任务的交叉点为“是”,也就是说可以用一架飞机以两个任务间的“广义距离”所表示的速度来匀速飞行,在完成对应第 7 个虚假航迹点的干扰欺骗任务之后,飞到指定位置完成对应第 20 个的虚假航迹点的干扰欺骗。

类似地,可以绘制 B-E 区域相应的广义矩里范围判断图。

\begin{table}[h]
\centering
\caption{B 区域各任务的广义距离是否介于 \(120-180km/h\)}
\begin{tabular}{|c|c|c|c|c|c|c|c|c|c|c|c|c|c|c|c|c|c|c|c|}
\hline
\multicolumn{21}{|c|}{B 区域内伪装虚假航迹点的任务编号} \\
\hline
 & 1 & 2 & 3 & 4 & 5 & 6 & 7 & 8 & 9 & 10 & 11 & 12 & 13 & 14 & 15 & 16 & 17 & 18 & 19 & 20 \\
\hline
1 & 否 & 否 & 否 & 否 & 否 & 否 & 否 & 否 & 否 & 否 & 否 & 否 & 否 & 否 & 否 & 否 & 否 & 否 & 否 & 是 \\
\hline
2 & 否 & 否 & 否 & 否 & 否 & 否 & 否 & 否 & 否 & 否 & 否 & 否 & 否 & 否 & 否 & 否 & 否 & 否 & 是 & 是 \\
\hline
3 & 否 & 否 & 否 & 否 & 否 & 否 & 否 & 否 & 否 & 否 & 否 & 否 & 否 & 否 & 否 & 否 & 否 & 是 & 是 & 是 \\
\hline
4 & 否 & 否 & 否 & 否 & 否 & 否 & 否 & 否 & 否 & 否 & 否 & 否 & 否 & 否 & 否 & 否 & 是 & 是 & 是 & 是 \\
\hline
5 & 否 & 否 & 否 & 否 & 否 & 否 & 否 & 否 & 否 & 否 & 否 & 否 & 否 & 否 & 否 & 是 & 是 & 是 & 是 & 是 \\
\hline
6 & 否 & 否 & 否 & 否 & 否 & 否 & 否 & 否 & 否 & 否 & 否 & 否 & 否 & 否 & 是 & 是 & 是 & 是 & 是 & 是 \\
\hline
7 & 否 & 否 & 否 & 否 & 否 & 否 & 否 & 否 & 否 & 否 & 否 & 否 & 否 & 是 & 是 & 是 & 是 & 是 & 是 & 是 \\
\hline
8 & 否 & 否 & 否 & 否 & 否 & 否 & 否 & 否 & 否 & 否 & 否 & 否 & 是 & 是 & 是 & 是 & 是 & 是 & 是 & 是 \\
\hline
9 & 否 & 否 & 否 & 否 & 否 & 否 & 否 & 否 & 否 & 否 & 否 & 是 & 是 & 是 & 是 & 是 & 是 & 是 & 是 & 是 \\
\hline
10 & 否 & 否 & 否 & 否 & 否 & 否 & 否 & 否 & 否 & 否 & 是 & 是 & 是 & 是 & 是 & 是 & 是 & 是 & 是 & 是 \\
\hline
11 & 否 & 否 & 否 & 否 & 否 & 否 & 否 & 否 & 否 & 是 & 是 & 是 & 是 & 是 & 是 & 是 & 是 & 是 & 是 & 是 \\
\hline
12 & 否 & 否 & 否 & 否 & 否 & 否 & 否 & 否 & 是 & 是 & 是 & 是 & 是 & 是 & 是 & 是 & 是 & 是 & 是 & 是 \\
\hline
13 & 否 & 否 & 否 & 否 & 否 & 否 & 否 & 是 & 是 & 是 & 是 & 是 & 是 & 是 & 是 & 是 & 是 & 是 & 是 & 是 \\
\hline
14 & 否 & 否 & 否 & 否 & 否 & 否 & 是 & 是 & 是 & 是 & 是 & 是 & 是 & 是 & 是 & 是 & 是 & 是 & 是 & 是 \\
\hline
15 & 否 & 否 & 否 & 否 & 否 & 是 & 是 & 是 & 是 & 是 & 是 & 是 & 是 & 是 & 是 & 是 & 是 & 是 & 是 & 是 \\
\hline
16 & 否 & 否 & 否 & 否 & 是 & 是 & 是 & 是 & 是 & 是 & 是 & 是 & 是 & 是 & 是 & 是 & 是 & 是 & 是 & 是 \\
\hline
17 & 否 & 否 & 否 & 是 & 是 & 是 & 是 & 是 & 是 & 是 & 是 & 是 & 是 & 是 & 是 & 是 & 是 & 是 & 是 & 是 \\
\hline
18 & 否 & 否 & 是 & 是 & 是 & 是 & 是 & 是 & 是 & 是 & 是 & 是 & 是 & 是 & 是 & 是 & 是 & 是 & 是 & 是 \\
\hline
19 & 否 & 是 & 是 & 是 & 是 & 是 & 是 & 是 & 是 & 是 & 是 & 是 & 是 & 是 & 是 & 是 & 是 & 是 & 是 & 是 \\
\hline
20 & 是 & 是 & 是 & 是 & 是 & 是 & 是 & 是 & 是 & 是 & 是 & 是 & 是 & 是 & 是 & 是 & 是 & 是 & 是 & 是 \\
\hline
\end{tabular}
\end{table}

\begin{table}[h]
\centering
\caption{C 区域各任务的广义距离是否介于 \(120-180km/h\)}
\begin{tabular}{|c|c|c|c|c|c|c|c|c|c|c|c|c|c|c|c|c|c|c|c|}
\hline
\multicolumn{21}{|c|}{C 区域内伪装虚假航迹点的任务编号} \\
\hline
 & 1 & 2 & 3 & 4 & 5 & 6 & 7 & 8 & 9 & 10 & 11 & 12 & 13 & 14 & 15 & 16 & 17 & 18 & 19 & 20 \\
\hline
1 & 否 & 否 & 否 & 否 & 否 & 否 & 否 & 否 & 否 & 否 & 否 & 否 & 否 & 否 & 否 & 否 & 否 & 否 & 否 & 是 \\
\hline
2 & 否 & 否 & 否 & 否 & 否 & 否 & 否 & 否 & 否 & 否 & 否 & 否 & 否 & 否 & 否 & 否 & 否 & 否 & 是 & 是 \\
\hline
3 & 否 & 否 & 否 & 否 & 否 & 否 & 否 & 否 & 否 & 否 & 否 & 否 & 否 & 否 & 否 & 否 & 否 & 是 & 是 & 是 \\
\hline
4 & 否 & 否 & 否 & 否 & 否 & 否 & 否 & 否 & 否 & 否 & 否 & 否 & 否 & 否 & 否 & 否 & 是 & 是 & 是 & 是 \\
\hline
5 & 否 & 否 & 否 & 否 & 否 & 否 & 否 & 否 & 否 & 否 & 否 & 否 & 否 & 否 & 否 & 是 & 是 & 是 & 是 & 是 \\
\hline
6 & 否 & 否 & 否 & 否 & 否 & 否 & 否 & 否 & 否 & 否 & 否 & 否 & 否 & 否 & 是 & 是 & 是 & 是 & 是 & 是 \\
\hline
7 & 否 & 否 & 否 & 否 & 否 & 否 & 否 & 否 & 否 & 否 & 否 & 否 & 否 & 是 & 是 & 是 & 是 & 是 & 是 & 是 \\
\hline
8 & 否 & 否 & 否 & 否 & 否 & 否 & 否 & 否 & 否 & 否 & 否 & 否 & 是 & 是 & 是 & 是 & 是 & 是 & 是 & 是 \\
\hline
9 & 否 & 否 & 否 & 否 & 否 & 否 & 否 & 否 & 否 & 否 & 否 & 是 & 是 & 是 & 是 & 是 & 是 & 是 & 是 & 是 \\
\hline
10 & 否 & 否 & 否 & 否 & 否 & 否 & 否 & 否 & 否 & 否 & 是 & 是 & 是 & 是 & 是 & 是 & 是 & 是 & 是 & 是 \\
\hline
11 & 否 & 否 & 否 & 否 & 否 & 否 & 否 & 否 & 否 & 是 & 是 & 是 & 是 & 是 & 是 & 是 & 是 & 是 & 是 & 是 \\
\hline
12 & 否 & 否 & 否 & 否 & 否 & 否 & 否 & 否 & 是 & 是 & 是 & 是 & 是 & 是 & 是 & 是 & 是 & 是 & 是 & 是 \\
\hline
13 & 否 & 否 & 否 & 否 & 否 & 否 & 否 & 是 & 是 & 是 & 是 & 是 & 是 & 是 & 是 & 是 & 是 & 是 & 是 & 是 \\
\hline
14 & 否 & 否 & 否 & 否 & 否 & 否 & 是 & 是 & 是 & 是 & 是 & 是 & 是 & 是 & 是 & 是 & 是 & 是 & 是 & 是 \\
\hline
15 & 否 & 否 & 否 & 否 & 否 & 是 & 是 & 是 & 是 & 是 & 是 & 是 & 是 & 是 & 是 & 是 & 是 & 是 & 是 & 是 \\
\hline
16 & 否 & 否 & 否 & 否 & 是 & 是 & 是 & 是 & 是 & 是 & 是 & 是 & 是 & 是 & 是 & 是 & 是 & 是 & 是 & 是 \\
\hline
17 & 否 & 否 & 否 & 是 & 是 & 是 & 是 & 是 & 是 & 是 & 是 & 是 & 是 & 是 & 是 & 是 & 是 & 是 & 是 & 是 \\
\hline
18 & 否 & 否 & 是 & 是 & 是 & 是 & 是 & 是 & 是 & 是 & 是 & 是 & 是 & 是 & 是 & 是 & 是 & 是 & 是 & 是 \\
\hline
19 & 否 & 是 & 是 & 是 & 是 & 是 & 是 & 是 & 是 & 是 & 是 & 是 & 是 & 是 & 是 & 是 & 是 & 是 & 是 & 是 \\
\hline
20 & 是 & 是 & 是 & 是 & 是 & 是 & 是 & 是 & 是 & 是 & 是 & 是 & 是 & 是 & 是 & 是 & 是 & 是 & 是 & 是 \\
\hline
\end{tabular}
\end{table}

\begin{table}
\centering
\begin{tabular}{c|c c c c c c c c c c c c c c c c c c c}
\hline
 & \multicolumn{20}{c}{D区域内伪装虚假航迹点的任务编号} \\
\hline
 & 1 & 2 & 3 & 4 & 5 & 6 & 7 & 8 & 9 & 10 & 11 & 12 & 13 & 14 & 15 & 16 & 17 & 18 & 19 & 20 \\
\hline
1 & 否 & 否 & 否 & 否 & 否 & 否 & 否 & 否 & 否 & 否 & 否 & 否 & 否 & 否 & 否 & 否 & 否 & 否 & 是 & 是 \\
2 & 否 & 否 & 否 & 否 & 否 & 否 & 否 & 否 & 否 & 否 & 否 & 否 & 否 & 否 & 否 & 否 & 否 & 否 & 是 & 是 \\
3 & 否 & 否 & 否 & 否 & 否 & 否 & 否 & 否 & 否 & 否 & 否 & 否 & 否 & 否 & 否 & 否 & 否 & 否 & 是 & 是 \\
4 & 否 & 否 & 否 & 否 & 否 & 否 & 否 & 否 & 否 & 否 & 否 & 否 & 否 & 否 & 否 & 否 & 否 & 否 & 是 & 是 \\
5 & 否 & 否 & 否 & 否 & 否 & 否 & 否 & 否 & 否 & 否 & 否 & 否 & 否 & 否 & 否 & 否 & 否 & 否 & 是 & 是 \\
6 & 否 & 否 & 否 & 否 & 否 & 否 & 否 & 否 & 否 & 否 & 否 & 否 & 否 & 否 & 否 & 否 & 否 & 否 & 是 & 是 \\
7 & 否 & 否 & 否 & 否 & 否 & 否 & 否 & 否 & 否 & 否 & 否 & 否 & 否 & 否 & 否 & 否 & 否 & 否 & 是 & 是 \\
8 & 否 & 否 & 否 & 否 & 否 & 否 & 否 & 否 & 否 & 否 & 否 & 否 & 否 & 否 & 否 & 否 & 否 & 否 & 是 & 是 \\
9 & 否 & 否 & 否 & 否 & 否 & 否 & 否 & 否 & 否 & 否 & 否 & 否 & 否 & 否 & 否 & 否 & 否 & 否 & 是 & 是 \\
10 & 否 & 否 & 否 & 否 & 否 & 否 & 否 & 否 & 否 & 否 & 否 & 否 & 否 & 否 & 否 & 否 & 否 & 否 & 是 & 是 \\
11 & 否 & 否 & 否 & 否 & 否 & 否 & 否 & 否 & 否 & 否 & 否 & 否 & 否 & 否 & 否 & 否 & 否 & 否 & 是 & 是 \\
12 & 否 & 否 & 否 & 否 & 否 & 否 & 否 & 否 & 否 & 否 & 否 & 否 & 否 & 否 & 否 & 否 & 否 & 否 & 是 & 是 \\
13 & 否 & 否 & 否 & 否 & 否 & 否 & 否 & 否 & 否 & 否 & 否 & 否 & 否 & 否 & 否 & 否 & 否 & 否 & 是 & 是 \\
14 & 否 & 否 & 否 & 否 & 否 & 否 & 否 & 否 & 否 & 否 & 否 & 否 & 否 & 否 & 否 & 否 & 否 & 否 & 是 & 是 \\
15 & 否 & 否 & 否 & 否 & 否 & 否 & 否 & 否 & 否 & 否 & 否 & 否 & 否 & 否 & 否 & 否 & 否 & 否 & 是 & 是 \\
16 & 否 & 否 & 否 & 否 & 否 & 否 & 否 & 否 & 否 & 否 & 否 & 否 & 否 & 否 & 否 & 否 & 否 & 否 & 是 & 是 \\
17 & 否 & 否 & 否 & 否 & 否 & 否 & 否 & 否 & 否 & 否 & 否 & 否 & 否 & 否 & 否 & 否 & 否 & 否 & 是 & 是 \\
18 & 否 & 否 & 否 & 否 & 否 & 否 & 否 & 否 & 否 & 否 & 否 & 否 & 否 & 否 & 否 & 否 & 否 & 否 & 是 & 是 \\
19 & 否 & 否 & 否 & 否 & 否 & 否 & 否 & 否 & 否 & 否 & 否 & 否 & 否 & 否 & 否 & 否 & 否 & 否 & 是 & 是 \\
20 & 是 & 是 & 是 & 是 & 是 & 是 & 是 & 是 & 是 & 是 & 是 & 是 & 是 & 是 & 是 & 是 & 是 & 是 & 是 & 是 \\
\hline
\end{tabular}
\quad
\begin{tabular}{c|c c c c c c c c c c c c c c c c c c c}
\hline
 & \multicolumn{20}{c}{E区域内伪装虚假航迹点的任务编号} \\
\hline
 & 1 & 2 & 3 & 4 & 5 & 6 & 7 & 8 & 9 & 10 & 11 & 12 & 13 & 14 & 15 & 16 & 17 & 18 & 19 & 20 \\
\hline
1 & 否 & 否 & 否 & 否 & 否 & 否 & 否 & 否 & 否 & 否 & 否 & 否 & 否 & 否 & 否 & 否 & 否 & 否 & 否 & 否 \\
2 & 否 & 否 & 否 & 否 & 否 & 否 & 否 & 否 & 否 & 否 & 否 & 否 & 否 & 否 & 否 & 否 & 否 & 否 & 否 & 否 \\
3 & 否 & 否 & 否 & 否 & 否 & 否 & 否 & 否 & 否 & 否 & 否 & 否 & 否 & 否 & 否 & 否 & 否 & 否 & 否 & 否 \\
4 & 否 & 否 & 否 & 否 & 否 & 否 & 否 & 否 & 否 & 否 & 否 & 否 & 否 & 否 & 否 & 否 & 否 & 否 & 否 & 否 \\
5 & 否 & 否 & 否 & 否 & 否 & 否 & 否 & 否 & 否 & 否 & 否 & 否 & 否 & 否 & 否 & 否 & 否 & 否 & 否 & 否 \\
6 & 否 & 否 & 否 & 否 & 否 & 否 & 否 & 否 & 否 & 否 & 否 & 否 & 否 & 否 & 否 & 否 & 否 & 否 & 否 & 否 \\
7 & 否 & 否 & 否 & 否 & 否 & 否 & 否 & 否 & 否 & 否 & 否 & 否 & 否 & 否 & 否 & 否 & 否 & 否 & 否 & 否 \\
8 & 否 & 否 & 否 & 否 & 否 & 否 & 否 & 否 & 否 & 否 & 否 & 否 & 否 & 否 & 否 & 否 & 否 & 否 & 否 & 是 \\
9 & 否 & 否 & 否 & 否 & 否 & 否 & 否 & 否 & 否 & 否 & 否 & 否 & 否 & 否 & 否 & 否 & 否 & 否 & 否 & 是 \\
10 & 否 & 否 & 否 & 否 & 否 & 否 & 否 & 否 & 否 & 否 & 否 & 否 & 否 & 否 & 否 & 否 & 否 & 否 & 否 & 是 \\
11 & 否 & 否 & 否 & 否 & 否 & 否 & 否 & 否 & 否 & 否 & 否 & 否 & 否 & 否 & 否 & 否 & 否 & 否 & 否 & 是 \\
12 & 否 & 否 & 否 & 否 & 否 & 否 & 否 & 否 & 否 & 否 & 否 & 否 & 否 & 否 & 否 & 否 & 否 & 否 & 是 & 是 \\
13 & 否 & 否 & 否 & 否 & 否 & 否 & 否 & 否 & 否 & 否 & 否 & 否 & 否 & 否 & 否 & 否 & 否 & 否 & 是 & 是 \\
14 & 否 & 否 & 否 & 否 & 否 & 否 & 否 & 否 & 否 & 否 & 否 & 否 & 否 & 否 & 否 & 否 & 否 & 否 & 是 & 是 \\
15 & 否 & 否 & 否 & 否 & 否 & 否 & 否 & 否 & 否 & 否 & 否 & 否 & 否 & 否 & 否 & 否 & 否 & 否 & 是 & 是 \\
16 & 否 & 否 & 否 & 否 & 否 & 否 & 否 & 否 & 否 & 否 & 否 & 否 & 否 & 否 & 否 & 否 & 否 & 否 & 是 & 是 \\
17 & 否 & 否 & 否 & 否 & 否 & 否 & 否 & 否 & 否 & 否 & 否 & 否 & 否 & 否 & 否 & 否 & 否 & 否 & 是 & 是 \\
18 & 否 & 否 & 否 & 否 & 否 & 否 & 否 & 否 & 否 & 否 & 否 & 否 & 否 & 否 & 否 & 否 & 否 & 否 & 是 & 是 \\
19 & 否 & 否 & 否 & 否 & 否 & 否 & 否 & 否 & 否 & 否 & 否 & 否 & 否 & 否 & 否 & 否 & 否 & 否 & 是 & 是 \\
20 & 否 & 否 & 否 & 否 & 否 & 否 & 否 & 否 & 否 & 否 & 否 & 否 & 否 & 否 & 否 & 否 & 否 & 否 & 是 & 是 \\
\hline
\end{tabular}
\caption{其它区域的任务合并图}
\end{table}

根据题图可见,B 区域、C 区域和 D 区域当中,任务与任务之间的广义距离较为恰当,有较多的重叠机会。而成功欺骗三台雷达,就可以实现无人机的协同干扰。在平面匀速直线运动的严格假定下,最少使用 10 架无人机,可以完成 B 区域内全部 20 个任务。类似地,使用 10 架飞机可以完成 C 区域内全部 20 个干扰任务;可以用 11 架飞机完成 D 区域内全部 20 个干扰任务。飞行任务的安排大致可以描述为:

在 B 区域,用一架飞机先后执行两个可合并的任务,分别是伪装虚拟航迹点 1 和虚拟航迹点 20,飞机编号 1,类似地,该区域中航迹点 2 和航迹点 19 由 2 号飞机完成伪装,3 号对应 18 号,4 号对应 17 号,5 号对应 16 号,6 号与 15 号合并,7 号与 14 号合并,8 号同 13 号合并,9 号同 12 号合并,10 号同 11 号合并。在 C 区域中的任务合并安排与 B 区域相同。D 区域当中,1 号同 20 号合并,2 号同 19 号合并,3 号同 18 号合并,4 号同 17 号合并,5 号由一架飞机单独完成,6 号同 16 号合并,7 号同 15 号合并,8 号同 14 号合并,9 号同 13 号合并,10 号同 12 号合并,11 号由一架飞机单独完成。有 29 个任务需要在两个指定时刻以匀速直线运动经过两个定点,执行这些任务的无人机已经有了确定的直线航迹和速度限制。至于另外两个单独的任务,我们规定他们的航线路程与邻近的航线平行,避免相撞。在这里还有一点需要补充,并没有三个点在同一条直线上。因此,不考虑一台无人机仅通过平面匀速直线运动,完成三个任务的情况。无人机航迹示意图如下,具体坐标安排参加附件。

\begin{figure}[h]
\centering
\includegraphics[width=\textwidth]{image.png}
\caption{31 架无人机的航迹示意图}
\end{figure}

至此,我们已经完成了对三台雷达伪装出 20 个相互一致的虚假航迹点。考虑 A 和 E 两个区域,并不能帮助我们减少无人机的使用。为了操作的方便,我们初步决定,在本问当中不在 A 和 E 两个区域安排干扰任务。

\subsection{3.3 无人机的运动合理性检验}

通过统计这些飞行数据可以发现,31 架无人机之间的距离在每个时刻都是安全的。因此,不需要对它们做调整高度的错层调整,也没有更多任务得到合并的可能。

\section*{4. 问题二的建模与求解}

\subsection{4.1 无人机机动能力的可视化分析}

首先考察无人机的机动飞行技能。可以转弯,使得无人机的活动范围得到扩大。根据题目的要求,最小拐弯半径为 250m,可以用几何方法求解在 10s 内所能行驶的最短直线为下图中弦的长度。

\begin{figure}[h]
    \centering
    \includegraphics[width=0.6\textwidth]{image.png} % 替换为实际图片路径
    \caption{无人机转弯求弦示意图}
    \label{fig:drone_turn}
\end{figure}

图 \ref{fig:drone_turn} 中,红弧长为:
\[
\text{红弧长} = 120 \, \text{km/h} \times \frac{1000 \, \text{m/km}}{3600 \, \text{s/h}} \times 10 \, \text{s}
\]

弧长为 333.34 米,对应圆心角为 $75.39^\circ$,对应弦长为 309.19 米。因此,309.2 米为无人机在 10s 的时间里所能走过的最短距离。

类似地,可以计算飞行器在 20s 的时间,可以走过 1000 米的弧长,对应圆心角为 $229.2^\circ$,对应弦长为 454.65 米,即无人机在 20s 的时间里所能走过的最短距离为 454.65 米;飞行器在 30s 的时间里可以走过 1500 米的弧长,对应弦长为 70.56 米,即无人机在转弯半径不小于 250 米的约束下,30s 的时间里最短可以走过 70.56 米的距离。考虑到无人机可以在 32 秒时完成掉头动作,因此在 32s 以上的时间里,无人机可以通过调整转弯半径,实现回到起点的操作。在一个相对长的时间里,无人机可以移动的最短距离为 0 米。对应的,无人机在相应时间所能实现的最远飞行距离,就是以 $180 \, \text{km/h}$ 的速度匀速直线运动所能走过的距离。无人机的机动能力得到较大的增强。

\subsection{4.2 问题的解答}

使用 9 架无人机实施干扰,需要让无人机尽可能以较少的移动实现较大的影响。为了尽量的减少无人机的俯冲以及爬升等动作,文章拟将无人机定在同一高度进行干扰。考虑到 9 架无人机要产生给定的虚假航迹,并且要尽可能的产生多的虚假航迹,而相应的研究中发现 2500 米高度无法满足要求,在此本文将无人机的飞行高度定在 2000 米。接下来对无人机的具体坐标位置进行了分析,使得无人机能够尽可能产生多的虚假航迹,实现更好的干扰效果 \cite{ref4}。本问题的解决步骤如下:

\begin{enumerate}
    \item 第一步:用 9 架飞机在前 20 个时刻营造出附件 1 中要求的虚假航迹;
    \item 第二步:每个时点上有三架无人机在工作,同时六架在休息。使用六架无人机营造出更多航迹;
    \item 第三步:计算新的虚假航迹和无人机飞行坐标;
    \item 第四步:检查是否有过快加速、过急拐弯的情况,和过近距离等情况,对无人机坐标做适当调整。
\end{enumerate}

\subsection*{4.2.1 产生规定航迹点}

根据无人机机动飞行的特点,定义新的任务可合并的条件:

若任务 $D_1$ 和任务 $D_2$ 之间两个任务的相关指标满足如下四个条件之一,则称两个任务可以合并完成。

条件 1: $(309.2m < \sqrt{(x_1 - x_2)^2 + (y_1 - y_2)^2 + (z_1 - z_2)^2} < 500m) \cap (t_1 - t_2 = 1)$,

条件 2: $(454.7m < \sqrt{(x_1 - x_2)^2 + (y_1 - y_2)^2 + (z_1 - z_2)^2} < 1000m) \cap (t_1 - t_2 = 2)$,

条件 3: $(70.6m < \sqrt{(x_1 - x_2)^2 + (y_1 - y_2)^2 + (z_1 - z_2)^2} < 1500m) \cap (t_1 - t_2 = 3)$,

条件 4: $(\sqrt{(x_1 - x_2)^2 + (y_1 - y_2)^2 + (z_1 - z_2)^2} < (t_1 - t_2) * 500m) \cap (t_1 - t_2 > 3)$.

下面使用 Excle 软件计算任意两个任务间的距离,并使用 if 函数绘制任务合并图。值得一提的是,在本问的分析当中,飞行的先后顺序十分重要。飞机在第 10 个时刻完成了干扰任务,就已经不可能再飞回到第 8 个时刻完成下一个任务。因此,下列图中的上半角均为否。做 A 区域的任务合并图如下:

\begin{table}[h]
\centering
\begin{tabular}{|c|c|c|c|c|c|c|c|c|c|c|c|c|c|c|c|c|c|c|c|}
\hline
\multicolumn{21}{|c|}{A 区域各任务的是否满足合并条件} \\
\hline
\multicolumn{2}{|c|}{} & \multicolumn{19}{c|}{A 区域内伪装虚假航迹点的任务编号} \\
\hline
 &  & 1 & 2 & 3 & 4 & 5 & 6 & 7 & 8 & 9 & 10 & 11 & 12 & 13 & 14 & 15 & 16 & 17 & 18 & 19 & 20 \\
\hline
1 & 否 & 否 & 否 & 否 & 否 & 否 & 否 & 否 & 否 & 否 & 否 & 否 & 否 & 否 & 否 & 否 & 否 & 否 & 否 & 否 \\
\hline
2 & 否 & 否 & 否 & 否 & 否 & 否 & 否 & 否 & 否 & 否 & 否 & 否 & 否 & 否 & 否 & 否 & 否 & 否 & 否 & 否 \\
\hline
3 & 否 & 否 & 否 & 否 & 否 & 否 & 否 & 否 & 否 & 否 & 否 & 否 & 否 & 否 & 否 & 否 & 否 & 否 & 否 & 否 \\
\hline
4 & 是 & 否 & 否 & 否 & 否 & 否 & 否 & 否 & 否 & 否 & 否 & 否 & 否 & 否 & 否 & 否 & 否 & 否 & 否 & 否 \\
\hline
5 & 是 & 是 & 否 & 否 & 否 & 否 & 否 & 否 & 否 & 否 & 否 & 否 & 否 & 否 & 否 & 否 & 否 & 否 & 否 & 否 \\
\hline
6 & 是 & 是 & 是 & 否 & 否 & 否 & 否 & 否 & 否 & 否 & 否 & 否 & 否 & 否 & 否 & 否 & 否 & 否 & 否 & 否 \\
\hline
7 & 是 & 是 & 是 & 是 & 否 & 否 & 否 & 否 & 否 & 否 & 否 & 否 & 否 & 否 & 否 & 否 & 否 & 否 & 否 & 否 \\
\hline
8 & 是 & 是 & 是 & 是 & 是 & 否 & 否 & 否 & 否 & 否 & 否 & 否 & 否 & 否 & 否 & 否 & 否 & 否 & 否 & 否 \\
\hline
9 & 是 & 是 & 是 & 是 & 是 & 是 & 否 & 否 & 否 & 否 & 否 & 否 & 否 & 否 & 否 & 否 & 否 & 否 & 否 & 否 \\
\hline
10 & 是 & 是 & 是 & 是 & 是 & 是 & 是 & 否 & 否 & 否 & 否 & 否 & 否 & 否 & 否 & 否 & 否 & 否 & 否 & 否 \\
\hline
11 & 是 & 是 & 是 & 是 & 是 & 是 & 是 & 是 & 否 & 否 & 否 & 否 & 否 & 否 & 否 & 否 & 否 & 否 & 否 & 否 \\
\hline
12 & 是 & 是 & 是 & 是 & 是 & 是 & 是 & 是 & 是 & 否 & 否 & 否 & 否 & 否 & 否 & 否 & 否 & 否 & 否 & 否 \\
\hline
13 & 是 & 是 & 是 & 是 & 是 & 是 & 是 & 是 & 是 & 是 & 否 & 否 & 否 & 否 & 否 & 否 & 否 & 否 & 否 & 否 \\
\hline
14 & 是 & 是 & 是 & 是 & 是 & 是 & 是 & 是 & 是 & 是 & 是 & 否 & 否 & 否 & 否 & 否 & 否 & 否 & 否 & 否 \\
\hline
15 & 是 & 是 & 是 & 是 & 是 & 是 & 是 & 是 & 是 & 是 & 是 & 是 & 否 & 否 & 否 & 否 & 否 & 否 & 否 & 否 \\
\hline
16 & 是 & 是 & 是 & 是 & 是 & 是 & 是 & 是 & 是 & 是 & 是 & 是 & 是 & 否 & 否 & 否 & 否 & 否 & 否 & 否 \\
\hline
17 & 是 & 是 & 是 & 是 & 是 & 是 & 是 & 是 & 是 & 是 & 是 & 是 & 是 & 是 & 否 & 否 & 否 & 否 & 否 & 否 \\
\hline
18 & 是 & 是 & 是 & 是 & 是 & 是 & 是 & 是 & 是 & 是 & 是 & 是 & 是 & 是 & 是 & 否 & 否 & 否 & 否 & 否 \\
\hline
19 & 是 & 是 & 是 & 是 & 是 & 是 & 是 & 是 & 是 & 是 & 是 & 是 & 是 & 是 & 是 & 是 & 否 & 否 & 否 & 否 \\
\hline
20 & 是 & 是 & 是 & 是 & 是 & 是 & 是 & 是 & 是 & 是 & 是 & 是 & 是 & 是 & 是 & 是 & 是 & 否 & 否 & 否 \\
\hline
\end{tabular}
\caption{A 区域任务合并示意图}
\end{table}

将无人机群安排在 A 区域,可以用较少的飞机完成较多的飞行任务。A 区域使用三架飞机就可以持续产生虚假目标的航迹信号,一架先后完成任务 1、任务 4、任务 7、任务 10、任务 13、任务 15、任务 16、任务 17、任务 18、任务 19;另一架先后完成任务 2、任务 5、任务 8、任务 11、任务 14、任务 20;最后一架先后完成任务 3、任务 6、任务 9、任务 12;依次编号为 1-3.

类似地,可以做出其它区域的任务合并图。

\begin{table}
\centering
\begin{tabular}{|c|c c c c c c c c c c c c c c c c c c c|}
\hline
\multicolumn{21}{|c|}{B区域各任务的是否满足合并条件} \\
\hline
\multicolumn{21}{|c|}{B区域内伪装虚假航迹点的任务编号} \\
\hline
 & 1 & 2 & 3 & 4 & 5 & 6 & 7 & 8 & 9 & 10 & 11 & 12 & 13 & 14 & 15 & 16 & 17 & 18 & 19 & 20 \\
\hline
1 & 否 & 否 & 否 & 否 & 否 & 否 & 否 & 否 & 否 & 否 & 否 & 否 & 否 & 否 & 否 & 否 & 否 & 否 & 否 & 否 \\
2 & 否 & 否 & 否 & 否 & 否 & 否 & 否 & 否 & 否 & 否 & 否 & 否 & 否 & 否 & 否 & 否 & 否 & 否 & 否 & 否 \\
3 & 否 & 否 & 否 & 否 & 否 & 否 & 否 & 否 & 否 & 否 & 否 & 否 & 否 & 否 & 否 & 否 & 否 & 否 & 否 & 否 \\
4 & 是 & 否 & 否 & 否 & 否 & 否 & 否 & 否 & 否 & 否 & 否 & 否 & 否 & 否 & 否 & 否 & 否 & 否 & 否 & 否 \\
5 & 是 & 是 & 否 & 否 & 否 & 否 & 否 & 否 & 否 & 否 & 否 & 否 & 否 & 否 & 否 & 否 & 否 & 否 & 否 & 否 \\
6 & 是 & 是 & 是 & 否 & 否 & 否 & 否 & 否 & 否 & 否 & 否 & 否 & 否 & 否 & 否 & 否 & 否 & 否 & 否 & 否 \\
7 & 是 & 是 & 是 & 是 & 否 & 否 & 否 & 否 & 否 & 否 & 否 & 否 & 否 & 否 & 否 & 否 & 否 & 否 & 否 & 否 \\
8 & 是 & 是 & 是 & 是 & 是 & 否 & 否 & 否 & 否 & 否 & 否 & 否 & 否 & 否 & 否 & 否 & 否 & 否 & 否 & 否 \\
9 & 是 & 是 & 是 & 是 & 是 & 是 & 否 & 否 & 否 & 否 & 否 & 否 & 否 & 否 & 否 & 否 & 否 & 否 & 否 & 否 \\
10 & 是 & 是 & 是 & 是 & 是 & 是 & 是 & 否 & 否 & 否 & 否 & 否 & 否 & 否 & 否 & 否 & 否 & 否 & 否 & 否 \\
11 & 是 & 是 & 是 & 是 & 是 & 是 & 是 & 是 & 否 & 否 & 否 & 否 & 否 & 否 & 否 & 否 & 否 & 否 & 否 & 否 \\
12 & 是 & 是 & 是 & 是 & 是 & 是 & 是 & 是 & 是 & 否 & 否 & 否 & 否 & 否 & 否 & 否 & 否 & 否 & 否 & 否 \\
13 & 是 & 是 & 是 & 是 & 是 & 是 & 是 & 是 & 是 & 是 & 否 & 否 & 否 & 否 & 否 & 否 & 否 & 否 & 否 & 否 \\
14 & 是 & 是 & 是 & 是 & 是 & 是 & 是 & 是 & 是 & 是 & 是 & 否 & 否 & 否 & 否 & 否 & 否 & 否 & 否 & 否 \\
15 & 是 & 是 & 是 & 是 & 是 & 是 & 是 & 是 & 是 & 是 & 是 & 是 & 否 & 否 & 否 & 否 & 否 & 否 & 否 & 否 \\
16 & 是 & 是 & 是 & 是 & 是 & 是 & 是 & 是 & 是 & 是 & 是 & 是 & 是 & 否 & 否 & 否 & 否 & 否 & 否 & 否 \\
17 & 是 & 是 & 是 & 是 & 是 & 是 & 是 & 是 & 是 & 是 & 是 & 是 & 是 & 是 & 否 & 否 & 否 & 否 & 否 & 否 \\
18 & 是 & 是 & 是 & 是 & 是 & 是 & 是 & 是 & 是 & 是 & 是 & 是 & 是 & 是 & 是 & 否 & 否 & 否 & 否 & 否 \\
19 & 是 & 是 & 是 & 是 & 是 & 是 & 是 & 是 & 是 & 是 & 是 & 是 & 是 & 是 & 是 & 是 & 否 & 否 & 否 & 否 \\
20 & 是 & 是 & 是 & 是 & 是 & 是 & 是 & 是 & 是 & 是 & 是 & 是 & 是 & 是 & 是 & 是 & 是 & 是 & 是 & 否 \\
\hline
\end{tabular}
\caption{B区域各任务的是否满足合并条件}
\end{table}

\begin{table}
\centering
\begin{tabular}{|c|c c c c c c c c c c c c c c c c c c c|}
\hline
\multicolumn{21}{|c|}{C区域各任务的是否满足合并条件} \\
\hline
\multicolumn{21}{|c|}{C区域内伪装虚假航迹点的任务编号} \\
\hline
 & 1 & 2 & 3 & 4 & 5 & 6 & 7 & 8 & 9 & 10 & 11 & 12 & 13 & 14 & 15 & 16 & 17 & 18 & 19 & 20 \\
\hline
1 & 否 & 否 & 否 & 否 & 否 & 否 & 否 & 否 & 否 & 否 & 否 & 否 & 否 & 否 & 否 & 否 & 否 & 否 & 否 & 否 \\
2 & 否 & 否 & 否 & 否 & 否 & 否 & 否 & 否 & 否 & 否 & 否 & 否 & 否 & 否 & 否 & 否 & 否 & 否 & 否 & 否 \\
3 & 否 & 否 & 否 & 否 & 否 & 否 & 否 & 否 & 否 & 否 & 否 & 否 & 否 & 否 & 否 & 否 & 否 & 否 & 否 & 否 \\
4 & 是 & 是 & 否 & 否 & 否 & 否 & 否 & 否 & 否 & 否 & 否 & 否 & 否 & 否 & 否 & 否 & 否 & 否 & 否 & 否 \\
5 & 是 & 是 & 是 & 否 & 否 & 否 & 否 & 否 & 否 & 否 & 否 & 否 & 否 & 否 & 否 & 否 & 否 & 否 & 否 & 否 \\
6 & 是 & 是 & 是 & 是 & 否 & 否 & 否 & 否 & 否 & 否 & 否 & 否 & 否 & 否 & 否 & 否 & 否 & 否 & 否 & 否 \\
7 & 是 & 是 & 是 & 是 & 是 & 否 & 否 & 否 & 否 & 否 & 否 & 否 & 否 & 否 & 否 & 否 & 否 & 否 & 否 & 否 \\
8 & 是 & 是 & 是 & 是 & 是 & 是 & 否 & 否 & 否 & 否 & 否 & 否 & 否 & 否 & 否 & 否 & 否 & 否 & 否 & 否 \\
9 & 是 & 是 & 是 & 是 & 是 & 是 & 是 & 否 & 否 & 否 & 否 & 否 & 否 & 否 & 否 & 否 & 否 & 否 & 否 & 否 \\
10 & 是 & 是 & 是 & 是 & 是 & 是 & 是 & 是 & 否 & 否 & 否 & 否 & 否 & 否 & 否 & 否 & 否 & 否 & 否 & 否 \\
11 & 是 & 是 & 是 & 是 & 是 & 是 & 是 & 是 & 是 & 否 & 否 & 否 & 否 & 否 & 否 & 否 & 否 & 否 & 否 & 否 \\
12 & 是 & 是 & 是 & 是 & 是 & 是 & 是 & 是 & 是 & 是 & 否 & 否 & 否 & 否 & 否 & 否 & 否 & 否 & 否 & 否 \\
13 & 是 & 是 & 是 & 是 & 是 & 是 & 是 & 是 & 是 & 是 & 是 & 否 & 否 & 否 & 否 & 否 & 否 & 否 & 否 & 否 \\
14 & 是 & 是 & 是 & 是 & 是 & 是 & 是 & 是 & 是 & 是 & 是 & 是 & 否 & 否 & 否 & 否 & 否 & 否 & 否 & 否 \\
15 & 是 & 是 & 是 & 是 & 是 & 是 & 是 & 是 & 是 & 是 & 是 & 是 & 是 & 否 & 否 & 否 & 否 & 否 & 否 & 否 \\
16 & 是 & 是 & 是 & 是 & 是 & 是 & 是 & 是 & 是 & 是 & 是 & 是 & 是 & 是 & 否 & 否 & 否 & 否 & 否 & 否 \\
17 & 是 & 是 & 是 & 是 & 是 & 是 & 是 & 是 & 是 & 是 & 是 & 是 & 是 & 是 & 是 & 否 & 否 & 否 & 否 & 否 \\
18 & 是 & 是 & 是 & 是 & 是 & 是 & 是 & 是 & 是 & 是 & 是 & 是 & 是 & 是 & 是 & 是 & 否 & 否 & 否 & 否 \\
19 & 是 & 是 & 是 & 是 & 是 & 是 & 是 & 是 & 是 & 是 & 是 & 是 & 是 & 是 & 是 & 是 & 是 & 否 & 否 & 否 \\
20 & 是 & 是 & 是 & 是 & 是 & 是 & 是 & 是 & 是 & 是 & 是 & 是 & 是 & 是 & 是 & 是 & 是 & 是 & 是 & 否 \\
\hline
\end{tabular}
\caption{C区域各任务的是否满足合并条件}
\end{table}

\begin{table}
\centering
\begin{tabular}{|c|c c c c c c c c c c c c c c c c c c c|}
\hline
\multicolumn{21}{|c|}{D区域各任务的是否满足合并条件} \\
\hline
\multicolumn{21}{|c|}{D区域内伪装虚假航迹点的任务编号} \\
\hline
 & 1 & 2 & 3 & 4 & 5 & 6 & 7 & 8 & 9 & 10 & 11 & 12 & 13 & 14 & 15 & 16 & 17 & 18 & 19 & 20 \\
\hline
1 & 否 & 否 & 否 & 否 & 否 & 否 & 否 & 否 & 否 & 否 & 否 & 否 & 否 & 否 & 否 & 否 & 否 & 否 & 否 & 否 \\
2 & 否 & 否 & 否 & 否 & 否 & 否 & 否 & 否 & 否 & 否 & 否 & 否 & 否 & 否 & 否 & 否 & 否 & 否 & 否 & 否 \\
3 & 否 & 否 & 否 & 否 & 否 & 否 & 否 & 否 & 否 & 否 & 否 & 否 & 否 & 否 & 否 & 否 & 否 & 否 & 否 & 否 \\
4 & 是 & 否 & 否 & 否 & 否 & 否 & 否 & 否 & 否 & 否 & 否 & 否 & 否 & 否 & 否 & 否 & 否 & 否 & 否 & 否 \\
5 & 是 & 是 & 否 & 否 & 否 & 否 & 否 & 否 & 否 & 否 & 否 & 否 & 否 & 否 & 否 & 否 & 否 & 否 & 否 & 否 \\
6 & 是 & 是 & 是 & 否 & 否 & 否 & 否 & 否 & 否 & 否 & 否 & 否 & 否 & 否 & 否 & 否 & 否 & 否 & 否 & 否 \\
7 & 是 & 是 & 是 & 是 & 否 & 否 & 否 & 否 & 否 & 否 & 否 & 否 & 否 & 否 & 否 & 否 & 否 & 否 & 否 & 否 \\
8 & 是 & 是 & 是 & 是 & 是 & 否 & 否 & 否 & 否 & 否 & 否 & 否 & 否 & 否 & 否 & 否 & 否 & 否 & 否 & 否 \\
9 & 是 & 是 & 是 & 是 & 是 & 是 & 否 & 否 & 否 & 否 & 否 & 否 & 否 & 否 & 否 & 否 & 否 & 否 & 否 & 否 \\
10 & 是 & 是 & 是 & 是 & 是 & 是 & 是 & 否 & 否 & 否 & 否 & 否 & 否 & 否 & 否 & 否 & 否 & 否 & 否 & 否 \\
11 & 是 & 是 & 是 & 是 & 是 & 是 & 是 & 是 & 否 & 否 & 否 & 否 & 否 & 否 & 否 & 否 & 否 & 否 & 否 & 否 \\
12 & 是 & 是 & 是 & 是 & 是 & 是 & 是 & 是 & 是 & 否 & 否 & 否 & 否 & 否 & 否 & 否 & 否 & 否 & 否 & 否 \\
13 & 是 & 是 & 是 & 是 & 是 & 是 & 是 & 是 & 是 & 是 & 否 & 否 & 否 & 否 & 否 & 否 & 否 & 否 & 否 & 否 \\
14 & 是 & 是 & 是 & 是 & 是 & 是 & 是 & 是 & 是 & 是 & 是 & 否 & 否 & 否 & 否 & 否 & 否 & 否 & 否 & 否 \\
15 & 是 & 是 & 是 & 是 & 是 & 是 & 是 & 是 & 是 & 是 & 是 & 是 & 否 & 否 & 否 & 否 & 否 & 否 & 否 & 否 \\
16 & 是 & 是 & 是 & 是 & 是 & 是 & 是 & 是 & 是 & 是 & 是 & 是 & 是 & 否 & 否 & 否 & 否 & 否 & 否 & 否 \\
17 & 是 & 是 & 是 & 是 & 是 & 是 & 是 & 是 & 是 & 是 & 是 & 是 & 是 & 是 & 否 & 否 & 否 & 否 & 否 & 否 \\
18 & 是 & 是 & 是 & 是 & 是 & 是 & 是 & 是 & 是 & 是 & 是 & 是 & 是 & 是 & 是 & 否 & 否 & 否 & 否 & 否 \\
19 & 是 & 是 & 是 & 是 & 是 & 是 & 是 & 是 & 是 & 是 & 是 & 是 & 是 & 是 & 是 & 是 & 否 & 否 & 否 & 否 \\
20 & 是 & 是 & 是 & 是 & 是 & 是 & 是 & 是 & 是 & 是 & 是 & 是 & 是 & 是 & 是 & 是 & 是 & 是 & 是 & 否 \\
\hline
\end{tabular}
\caption{D区域各任务的是否满足合并条件}
\end{table}

\begin{table}
\centering
\begin{tabular}{|c|c c c c c c c c c c c c c c c c c c c|}
\hline
\multicolumn{21}{|c|}{E区域各任务的是否满足合并条件} \\
\hline
\multicolumn{21}{|c|}{E区域内伪装虚假航迹点的任务编号} \\
\hline
 & 1 & 2 & 3 & 4 & 5 & 6 & 7 & 8 & 9 & 10 & 11 & 12 & 13 & 14 & 15 & 16 & 17 & 18 & 19 & 20 \\
\hline
1 & 否 & 否 & 否 & 否 & 否 & 否 & 否 & 否 & 否 & 否 & 否 & 否 & 否 & 否 & 否 & 否 & 否 & 否 & 否 & 否 \\
2 & 否 & 否 & 否 & 否 & 否 & 否 & 否 & 否 & 否 & 否 & 否 & 否 & 否 & 否 & 否 & 否 & 否 & 否 & 否 & 否 \\
3 & 否 & 否 & 否 & 否 & 否 & 否 & 否 & 否 & 否 & 否 & 否 & 否 & 否 & 否 & 否 & 否 & 否 & 否 & 否 & 否 \\
4 & 是 & 否 & 否 & 否 & 否 & 否 & 否 & 否 & 否 & 否 & 否 & 否 & 否 & 否 & 否 & 否 & 否 & 否 & 否 & 否 \\
5 & 是 & 是 & 否 & 否 & 否 & 否 & 否 & 否 & 否 & 否 & 否 & 否 & 否 & 否 & 否 & 否 & 否 & 否 & 否 & 否 \\
6 & 是 & 是 & 是 & 否 & 否 & 否 & 否 & 否 & 否 & 否 & 否 & 否 & 否 & 否 & 否 & 否 & 否 & 否 & 否 & 否 \\
7 & 是 & 是 & 是 & 是 & 否 & 否 & 否 & 否 & 否 & 否 & 否 & 否 & 否 & 否 & 否 & 否 & 否 & 否 & 否 & 否 \\
8 & 是 & 是 & 是 & 是 & 是 & 否 & 否 & 否 & 否 & 否 & 否 & 否 & 否 & 否 & 否 & 否 & 否 & 否 & 否 & 否 \\
9 & 是 & 是 & 是 & 是 & 是 & 是 & 否 & 否 & 否 & 否 & 否 & 否 & 否 & 否 & 否 & 否 & 否 & 否 & 否 & 否 \\
10 & 是 & 是 & 是 & 是 & 是 & 是 & 是 & 否 & 否 & 否 & 否 & 否 & 否 & 否 & 否 & 否 & 否 & 否 & 否 & 否 \\
11 & 是 & 是 & 是 & 是 & 是 & 是 & 是 & 是 & 否 & 否 & 否 & 否 & 否 & 否 & 否 & 否 & 否 & 否 & 否 & 否 \\
12 & 是 & 是 & 是 & 是 & 是 & 是 & 是 & 是 & 是 & 否 & 否 & 否 & 否 & 否 & 否 & 否 & 否 & 否 & 否 & 否 \\
13 & 是 & 是 & 是 & 是 & 是 & 是 & 是 & 是 & 是 & 是 & 否 & 否 & 否 & 否 & 否 & 否 & 否 & 否 & 否 & 否 \\
14 & 是 & 是 & 是 & 是 & 是 & 是 & 是 & 是 & 是 & 是 & 是 & 否 & 否 & 否 & 否 & 否 & 否 & 否 & 否 & 否 \\
15 & 是 & 是 & 是 & 是 & 是 & 是 & 是 & 是 & 是 & 是 & 是 & 是 & 否 & 否 & 否 & 否 & 否 & 否 & 否 & 否 \\
16 & 是 & 是 & 是 & 是 & 是 & 是 & 是 & 是 & 是 & 是 & 是 & 是 & 是 & 否 & 否 & 否 & 否 & 否 & 否 & 否 \\
17 & 是 & 是 & 是 & 是 & 是 & 是 & 是 & 是 & 是 & 是 & 是 & 是 & 是 & 是 & 否 & 否 & 否 & 否 & 否 & 否 \\
18 & 是 & 是 & 是 & 是 & 是 & 是 & 是 & 是 & 是 & 是 & 是 & 是 & 是 & 是 & 是 & 否 & 否 & 否 & 否 & 否 \\
19 & 是 & 是 & 是 & 是 & 是 & 是 & 是 & 是 & 是 & 是 & 是 & 是 & 是 & 是 & 是 & 是 & 否 & 否 & 否 & 否 \\
20 & 是 & 是 & 是 & 是 & 是 & 是 & 是 & 是 & 是 & 是 & 是 & 是 & 是 & 是 & 是 & 是 & 是 & 是 & 是 & 否 \\
\hline
\end{tabular}
\caption{E区域各任务的是否满足合并条件}
\end{table}

图 12 B-E 区域任务合并示意图

根据上图可以看出,在 A、B、E 三个区域,可以用较少的飞机完成较多的飞行任务。

B 区域使用三架飞机,一架先后完成任务 1、任务 4、任务 7、任务 9、任务 11、任务 13、任务 14、任务 15、任务 16、任务 17、任务 18、任务 19 和任务 20;另一架飞机先后完成任务 2、任务 5、任务 8、任务 10、任务 12;最后一架飞机先后完成任务 3、任务 6;依次编号为 4-6.

E 区域使用三架飞机,一架先后完成任务 1、任务 4、任务 7、任务 10、任务 13、任务 15、任务 16、任务 17、任务 18;另一架飞机先后完成任务 2、任务 5、任务 8、任务 11、任务 14、任务 19;最后一架飞机先后完成任务 3、任务 6、任务 9、任务 12、任务 20;依次编号为 7-9.

具体坐标安排参加附件。经检查,在 A、C 区域安排无人机,或安排无人机进行跨区域干扰(如使用无人机在 A 区域飞行的同时干扰 2 号雷达),并不能更好地完成任务。因此放弃在 A 和 C 两个区域安排无人机,放弃安排无人机跨区域干扰。

\subsection{研究四面体模型归纳无人机与假目标的几何特征}

根据上节的研究可知,ABE 三台雷达各由 3 架无人机进行交替干扰。不妨将第一问中要求的虚假目标机航迹点规定为 Q 点,ABEQ 四个点可以组成一个四面体。Q 点的坐标尽管随时点变动,但在每一个时点都有唯一 Q 点,也就有唯一的四面体。执行规定任务的无人机机会飞到四面体的棱边上,以便三条连线可以顺利交于该时刻的 Q 点。在棱边 AQ 上,总有一架无人机针对 A 雷达生成 Q 航迹,命名为 \( M_{AQ} \),一架无人机针对 B 雷达生成 Q 航迹,命名为 \( M_{BQ} \),一架无人机

机针对 E 雷达生成 Q 航迹,命名为 $\mathbf{M}_{\mathrm{EQ}}$。倘若让这三架执行任务中的无人机均在飞行过程中先后发出两个信号,在棱边上形成新的虚假目标机航迹点,那么每个多出来的信号,都需要另外两架无人机做配合,才能通过 “同源检验”。绘制示意图如下,$\mathbf{M}_{\mathrm{AQ}}$ 无人机在棱边 $AQ$ 上面飞行,生成了两个虚假信号,$Q$ 信号已经有 $\mathbf{M}_{\mathrm{BQ}}$ 和 $\mathbf{M}_{\mathrm{EQ}}$ 两台无人机配合,可以同时骗过 3 个雷达;然而第二个信号,却需要 $B$ 区域与 $E$ 区域当中有更多的无人机予以配合,用来欺骗 $B$ 雷达和 $E$ 雷达。

\begin{figure}[h]
    \centering
    \includegraphics[width=0.8\textwidth]{AQ_tetrahedron.png}
    \caption{AQ 棱边生成新的虚假航迹点示意图}
\end{figure}

不妨将棱边 $AQ$ 上的虚假航迹点命名为 $P_{\mathrm{A}}$,将棱边 $BQ$ 上的虚假航迹点命名为 $P_{\mathrm{B}}$,将棱边 $EQ$ 上的虚假航迹点命名为 $P_{\mathrm{E}}$。除了当期时刻正在棱边工作的无人机 $\mathbf{M}_{\mathrm{AQ}}$、$\mathbf{M}_{\mathrm{AQ}}$ 与 $\mathbf{M}_{\mathrm{AQ}}$ 之外,每个对应雷达还有两台闲置无人机可供使用,安排如下:安排无人机针对 $A$ 雷达生成 $P_{\mathrm{B}}$ 航迹,命名为 $\mathbf{M}_{\mathrm{AB}}$;针对 $A$ 雷达生成 $P_{\mathrm{E}}$ 航迹,命名为 $\mathbf{M}_{\mathrm{AE}}$;针对 $B$ 雷达生成 $P_{\mathrm{A}}$ 航迹,命名为 $\mathbf{M}_{\mathrm{BA}}$;针对 $B$ 雷达生成 $P_{\mathrm{E}}$ 航迹,命名为 $\mathbf{M}_{\mathrm{BE}}$;针对 $E$ 雷达生成 $P_{\mathrm{A}}$ 航迹,命名为 $\mathbf{M}_{\mathrm{EA}}$;针对 $E$ 雷达生成 $P_{\mathrm{B}}$ 航迹,命名为 $\mathbf{M}_{\mathrm{EB}}$。至此,九架无人机安排完毕,每个时刻可以用来形成四条虚假航迹点。因为虚假航迹点要么在四面体的顶点,要么在四面体的棱边上,因此可以推导出一个重要的结论:

伪装这些虚假航迹点的无人机,要么在动态四面体的棱边上,要么在四面体的面上。

至此,9 架无人机得到了充分的利用。并不能在不增加无人机的情况下,在同一个时刻伪装出更多的航迹点。在上节中假设无人机在高度 $2.0\,\mathrm{km}$ 的平面当中飞行,下面的计算延续该假设。

首先查看 $A$ 区域的飞机情况。$\mathbf{M}_{\mathrm{AQ}}(x_{\mathrm{AQ}}, y_{\mathrm{AQ}}, z_{\mathrm{AQ}})$ 的坐标需要同时满足的条件有:

条件 1:在 $AQ$ 连线上。

条件 2:在 $2.0\,\mathrm{km}$ 高度的平面上。

因为点和直线不平行,且交点可以唯一确定,因此可以容易地确定 \( M_{AQ} \) 的坐标。类似地,可以求解 \( M_{BQ} \) 和 \( M_{EQ} \) 的坐标。 \( M_{AQ} \)、\( M_{BQ} \) 和 \( M_{EQ} \) 的坐标可以直接求解。

A 区域中 \( M_{AB} \) 的坐标需要同时满足的条件有:

条件 1:点在面 \( ABQ \) 上面。

条件 2:点在 2.0km 高度的平面上,即在 \( M_{BQ} \) 和 \( M_{AQ} \) 的连线上。

条件 3:点在 \( A \) 与 \( P_B \) 的直线上,倘若 \( P_B \) 的坐标确定,则 \( M_{AB} \) 的坐标可以唯一确定。

条件 4:所能选择的 \( P_B \) 同时在面 \( ABQ \) 和面 \( EBQ \) 上,在二者相交的棱边上。

E 区域中 \( M_{EB} \) 的坐标需要同时满足的条件有:

条件 1:点在面 \( EBQ \) 上面。

条件 2:点在 2.0km 高度的平面上,即在 \( M_{BQ} \) 和 \( M_{EQ} \) 的连线上。

条件 3:点在 \( E \) 与 \( P_B \) 的直线上,倘若 \( P_B \) 的坐标确定,则 \( M_{EB} \) 的坐标可以唯一确定。

条件 4:所能选择的 \( P_B \) 同时在面 \( ABQ \) 和面 \( EBQ \) 上,在二者相交的棱边上。

从上面的分析过程可知,当干扰达到最大效果时,除了无人机 \( M_{AQ} \)、\( M_{BQ} \)、\( M_{EQ} \) 的坐标确定了以外,其他无人机的坐标必须满足一定的条件,为了更清晰的反应无人机与虚假航迹目标的关系以及方便文章后面的优化求解,我们将其转换成方程约束的形式 \({}^{[5]}\)。

无人机 \( M_{AB}(x_{AB}, y_{AB}, 2) \)、\( M_{BE}(x_{BA}, y_{BA}, 2) \) 点位于 \( M_{AQ}M_{BQ} \) 直线上则满足:
\begin{equation}
\left\{
\begin{aligned}
\frac{(x_{AB} - x_{AQ})}{(x_{BQ} - x_{AQ})} &= \frac{y_{AB} - y_{AQ}}{y_{BQ} - y_{AQ}} = \frac{z_{AB} - z_{AQ}}{z_{BQ} - z_{AQ}} \\
\frac{(x_{BA} - x_{AQ})}{(x_{BQ} - x_{AQ})} &= \frac{y_{BA} - y_{AQ}}{y_{BQ} - y_{AQ}} = \frac{z_{BA} - z_{AQ}}{z_{BQ} - z_{AQ}}
\end{aligned}
\right.
\tag{4.1}
\end{equation}

无人机 \( M_{AE}(x_{AE}, y_{AE}, 2) \)、\( M_{EA}(x_{EA}, y_{EA}, 2) \) 点位于 \( M_{AQ}M_{EQ} \) 直线上则满足:
\begin{equation}
\left\{
\begin{aligned}
\frac{(x_{AE} - x_{AQ})}{(x_{EQ} - x_{AQ})} &= \frac{y_{AE} - y_{AQ}}{y_{EQ} - y_{AQ}} = \frac{z_{AE} - z_{AQ}}{z_{EQ} - z_{AQ}} \\
\frac{(x_{EA} - x_{AQ})}{(x_{EQ} - x_{AQ})} &= \frac{y_{EA} - y_{AQ}}{y_{EQ} - y_{AQ}} = \frac{z_{EA} - z_{AQ}}{z_{EQ} - z_{AQ}}
\end{aligned}
\right.
\tag{4.2}
\end{equation}

无人机 \( M_{BE}(x_{BE}, y_{BE}, 2) \)、\( M_{EB}(x_{EB}, y_{EB}, 2) \) 点位于 \( M_{EQ}M_{BQ} \) 直线上则满足:
\begin{equation}
\left\{
\begin{aligned}
\frac{(x_{BE} - x_{EQ})}{(x_{BQ} - x_{EQ})} &= \frac{y_{BE} - y_{EQ}}{y_{BQ} - y_{EQ}} = \frac{z_{BE} - z_{EQ}}{z_{BQ} - z_{EQ}} \\
\frac{(x_{EB} - x_{EQ})}{(x_{BQ} - x_{EQ})} &= \frac{y_{EB} - y_{EQ}}{y_{BQ} - y_{EQ}} = \frac{z_{EB} - z_{EQ}}{z_{BQ} - z_{EQ}}
\end{aligned}
\right.
\tag{4.3}
\end{equation}

接下来在第一时刻对各虚假目标与各对应飞机的坐标位置之间的关系进行求解,由于 \( P_B \) 点为 \( AM_{AE} \)、\( BM_{BE} \) 两条直线的交点上并且 \( P_B \) 在 \( BQ \) 直线上,则 \( P_B(x_{PB}, y_{PB}, z_{PB}) \) 必须满足如下方程组:

\begin{equation}
\left\{
\begin{aligned}
\frac{x_{PB} - x_A}{x_{AE} - x_A} &= \frac{y_{PB} - y_A}{y_{AE} - y_A} = \frac{z_{PB} - z_A}{z_{AE} - z_A} \\
\frac{x_{PB} - x_B}{x_{BE} - x_B} &= \frac{y_{PB} - y_B}{y_{BE} - y_B} = \frac{z_{PB} - z_B}{z_{BE} - z_B} \\
\frac{x_{PB} - x_Q}{x_B - x_Q} &= \frac{y_{PB} - y_Q}{y_B - y_Q} = \frac{z_{PB} - z_Q}{z_B - z_Q}
\end{aligned}
\right.
\tag{4.4}
\end{equation}

由于 \( P_A \) 点为 \( BM_{BA} \)、\( EM_{EA} \) 两条直线的交点上并且 \( P_A \) 在 \( AQ \) 直线上,则 \( P_A(x_{PA}, y_{PA}, z_{PA}) \) 必须满足如下方程组:

\begin{equation}
\left\{
\begin{aligned}
\frac{x_{PA} - x_B}{x_{BA} - x_B} &= \frac{y_{PA} - y_B}{y_{BA} - y_B} = \frac{z_{PA} - z_B}{z_{BA} - z_B} \\
\frac{x_{PA} - x_E}{x_{EA} - x_E} &= \frac{y_{PA} - y_E}{y_{EA} - y_E} = \frac{z_{PA} - z_E}{z_{EA} - z_E} \\
\frac{x_{PA} - x_Q}{x_A - x_Q} &= \frac{y_{PA} - y_Q}{y_A - y_Q} = \frac{z_{PA} - z_Q}{z_A - z_Q}
\end{aligned}
\right.
\tag{4.5}
\end{equation}

由于 \( P_E \) 点为 \( AM_{AB} \)、\( EM_{EB} \) 两条直线的交点上并且 \( P_E \) 在 \( EQ \) 直线上,则 \( P_E(x_{PE}, y_{PE}, z_{PE}) \) 必须满足如下方程组:

\begin{equation}
\left\{
\begin{aligned}
\frac{x_{PE} - x_A}{x_{AB} - x_A} &= \frac{y_{PE} - y_A}{y_{AB} - y_A} = \frac{z_{PE} - z_A}{z_{AB} - z_A} \\
\frac{x_{PE} - x_E}{x_{EB} - x_E} &= \frac{y_{PE} - y_E}{y_{EB} - y_E} = \frac{z_{PE} - z_E}{z_{EB} - z_E} \\
\frac{x_{PE} - x_Q}{x_E - x_Q} &= \frac{y_{PE} - y_Q}{y_E - y_Q} = \frac{z_{PE} - z_Q}{z_E - z_Q}
\end{aligned}
\right.
\tag{4.6}
\end{equation}

由于无人机都在高度为 \( 2 \) km 的平面上飞行,因此对于无人机 \( M_{AB} \)、\( M_{AE} \)、\( M_{BA} \)、\( M_{BE} \)、\( M_{EA} \) 及 \( M_{EB} \) 的纵坐标必将满足如下条件:

\begin{equation}
z_{AB} = z_{AE} = z_{BA} = z_{BE} = z_{EA} = z_{EB} = 2
\tag{4.7}
\end{equation}

根据上面的条件可知,我们可以通过调整 \( P_B \) 的位置,在约束条件下求解出最优的 \( P_B \) 的坐标使得 \( M_{AB} \) 与 \( M_{EB} \) 的所对应的无人机拐弯角度最缓、加速幅度最小。

类似地,我们可以通过调整 \( P_A \) 的位置,在约束条件下,求解出最优的 \( P_A \) 的坐标使得 \( M_{BA} \) 与 \( M_{EA} \) 的所对应的无人机拐弯角度最缓、加速幅度最小;在约束条件下求解出最优的 \( P_E \) 的坐标使得 \( M_{AE} \) 与 \( M_{BE} \) 的所对应的无人机拐弯角度最缓、加速幅度最小。

至此,我们将 \( M_{AB} \)、\( M_{AE} \)、\( M_{BA} \)、\( M_{BE} \)、\( M_{EA} \)、\( M_{EB} \) 坐标位置的求解问题,转化成一个高度 \( 2.0 \) km 高度平面内的优化问题。以 3 号飞机为例,在时刻 3、时刻 6、时刻 9 和时刻 12 是有固定的位置的,需要在 \( M_{AQ} \) 的坐标处,而其它时间目前是闲置的,可以在 \( M_{AB} \) 坐标处,也可以在 \( M_{AE} \) 坐标处。假定 3 号无人机在时刻 3 时必需出现在 \( M_{AQ} \) 处,定义该点为 \( O \) 点,查看 3 号无人机能否在时刻 4 出

现在 4 时刻的 \( \mathbf{M}_{BQ} \) 和 \( \mathbf{M}_{AQ} \) 的连线上,定义该线段为 1。下面的示意图可以描述出无人机飞行的一些限制条件。

\begin{figure}[h]
    \centering
    \includegraphics[width=0.7\textwidth]{image.png} % 替换为实际图片路径
    \caption{无人机急转弯示意图}
    \label{fig:drone_turn}
\end{figure}

不难发现,当 O 点与线段 1 相距太近,一定要急转弯才能确保 10s 后恰好到达线段上。否则,如果不到 10s 就出现在线段上,且无人机的掉头速度特别慢,那么就不能在规定的时间点上飞抵 \( \mathbf{M}_{AB} \) 或者 \( \mathbf{M}_{AE} \) 这样的点上。在完成坐标求解之后,文章专门安排章节查看无人机的拐弯角度是否超过自身的能力。

\subsubsection{计算无人机和假目标在各时刻的实际坐标}

经过上面的分析可知,拟合四条虚拟航迹是 9 架无人机的能力的极限,而航迹点坐标与无人机坐标的解并不是唯一的,这就需要计算机优化算法。根据上面分析给出的约束条件,通过 MATLAB 软件计算求解,我们得到了一个初步的优化结果。1-9 号无人机分别在不同时刻扮演着不同的角色:

\begin{table}[h]
    \centering
    \caption{不同时刻下扮演对应角色的无人机编号}
    \label{tab:drone_roles}
    \begin{tabular}{c c c c c c c c c c c c c c c c c c c c c}
        \hline
        \multirow{2}{*}{角色} & \multicolumn{1}{c}{\multirow{2}{*}{对应}} & \multicolumn{20}{c}{时刻} \\
        & \multicolumn{1}{c}{} & 1 & 2 & 3 & 4 & 5 & 6 & 7 & 8 & 9 & 10 & 11 & 12 & 13 & 14 & 15 & 16 & 17 & 18 & 19 & 20 \\
        \hline
        \( \mathbf{M}_{AQ} \) & Q & 1 & 2 & 3 & 1 & 2 & 3 & 1 & 2 & 3 & 1 & 2 & 3 & 1 & 2 & 1 & 1 & 1 & 1 & 1 & 2 \\
        \( \mathbf{M}_{AQ} \) & \( P_A \) & 1 & 2 & 3 & 1 & 2 & 3 & 1 & 2 & 3 & 1 & 2 & 3 & 1 & 2 & 1 & 1 & 1 & 1 & 1 & 2 \\
        \( \mathbf{M}_{AB} \) & \( P_B \) & 3 & 1 & 2 & 3 & 1 & 2 & 3 & 1 & 2 & 3 & 1 & 2 & 3 & 3 & 3 & 3 & 3 & 3 & 3 & 3 \\
        \( \mathbf{M}_{AE} \) & \( P_E \) & 2 & 3 & 1 & 2 & 3 & 1 & 2 & 3 & 1 & 2 & 3 & 1 & 2 & 1 & 2 & 2 & 2 & 2 & 2 & 1 \\
        \( \mathbf{M}_{BQ} \) & Q & 4 & 5 & 6 & 4 & 5 & 6 & 4 & 5 & 4 & 5 & 4 & 5 & 4 & 4 & 4 & 4 & 4 & 4 & 4 & 4 \\
        \( \mathbf{M}_{BA} \) & \( P_A \) & 5 & 6 & 4 & 5 & 6 & 4 & 5 & 4 & 5 & 4 & 5 & 4 & 5 & 5 & 5 & 5 & 5 & 5 & 5 & 5 \\
        \( \mathbf{M}_{BQ} \) & \( P_B \) & 4 & 5 & 6 & 4 & 5 & 6 & 4 & 5 & 4 & 5 & 4 & 5 & 4 & 4 & 4 & 4 & 4 & 4 & 4 & 4 \\
        \( \mathbf{M}_{BE} \) & \( P_E \) & 6 & 4 & 5 & 6 & 4 & 5 & 6 & 6 & 6 & 6 & 6 & 6 & 6 & 6 & 6 & 6 & 6 & 6 & 6 & 6 \\
        \( \mathbf{M}_{EQ} \) & Q & 7 & 8 & 9 & 7 & 8 & 9 & 7 & 8 & 9 & 7 & 8 & 9 & 7 & 8 & 7 & 7 & 7 & 7 & 8 & 9 \\
        \( \mathbf{M}_{EA} \) & \( P_A \) & 9 & 7 & 8 & 9 & 7 & 8 & 9 & 7 & 8 & 9 & 7 & 8 & 9 & 9 & 9 & 9 & 9 & 9 & 9 & 8 \\
        \( \mathbf{M}_{EB} \) & \( P_B \) & 8 & 9 & 7 & 8 & 9 & 7 & 8 & 9 & 7 & 8 & 9 & 7 & 8 & 7 & 8 & 8 & 8 & 8 & 7 & 7 \\
        \( \mathbf{M}_{EQ} \) & \( P_E \) & 7 & 8 & 9 & 7 & 8 & 9 & 7 & 8 & 9 & 7 & 8 & 9 & 7 & 8 & 7 & 7 & 7 & 7 & 8 & 9 \\
        \hline
    \end{tabular}
\end{table}

每一个航迹点,都有三台无人机扮演相应角色向对应雷达发射信号,就可以通过雷达中心设置的同源检验,实现无人机之间的有效协同。为了更加清楚的反应无人机在各区域内的相对位置,下面给出在 20 个时刻产生题目所要求的虚假航迹时,在 2km 高空各自区域中无人机的具体位置情况如下图所示:

\begin{figure}[h]
    \centering
    \includegraphics[width=0.7\textwidth]{image2.png} % 替换为实际图片路径
    \caption{无人机在各区域内的相对位置}
    \label{fig:drone_positions}
\end{figure}

\begin{figure}[h]
    \centering
    \includegraphics[width=\textwidth]{image1.png}
    \caption{雷达A监测区域高度2km平面散点图}
\end{figure}

\begin{figure}[h]
    \centering
    \includegraphics[width=\textwidth]{image2.png}
    \caption{雷达B监测区域高度2km平面散点图}
\end{figure}

\begin{figure}[h]
    \centering
    \includegraphics[width=\textwidth]{image3.png}
    \caption{雷达E监测区域高度2km平面散点图}
\end{figure}

图15 A、B、E区域2km高度无人机位置示意图

其中,1-20标号表示20个不同的时刻,在各区域中,颜色相同的点表示相同的无人机,从图中可以发现,在前些时刻两架无人机相叉飞行,在后续时刻基本上一架无人机就可以完成任务,其他两架获得了自由,从而有产生更多虚拟航迹的可能。

文章提供的优化算法以及确定的无人机飞行路线能够产生四条虚假航线(包括题目所要求的虚假航迹),其虚假航迹的飞行路线如下:

\begin{figure}[h]
    \centering
    \includegraphics[width=\textwidth]{image4.png}
    \caption{四条虚假航机的飞行轨迹图}
\end{figure}

图16 四条虚假航线飞行轨迹图

从四条虚假航迹的轨迹可以看出,其中除了有一条航迹路线有些许弯曲以外,其他的航迹路线相对平滑,具有很好的干扰效果。九架无人机为了给某些雷

\section*{九架无人机在各区域内的飞行轨迹图}

\begin{figure}[h]
    \centering
    \includegraphics[width=\textwidth]{image.png}
    \caption{九架无人机在各区域内的飞行轨迹图}
    \label{fig:17}
\end{figure}

从图中可以发现,在各区域中都有三架飞机对该区域的雷达进行干扰,并且在虚假目标点,经常是不同的无人机进行交叉干扰。为了能够产生更多的虚假航线,其中存在少数无人机发生较大的拐弯。从整体上看,本文求解的无人机路线是相对合理的,能够取得较好的干扰效果。

\subsection{无人机和假目标航迹的运动合理性检验}

对于无人机和对应航迹的运动合理性检验分为三个主要部分,具体包括:(a)针对无人机运行轨迹上相邻无人机间距的检验,(b)针对无人机飞行轨迹方向变化的检验以及(c)对虚假航线的检验。下面内容主要对这三种检验角度加以详细叙述(本检验部分的代码均由 R 语言进行编译,所有代码均为作者独立编写)。

\subsubsection{a. 针对相邻无人机间距的检验}

根据前文的分析我们已经得知,相邻两航迹点的间隔在 309.2 米到 500 米之间时,可以保证无人机可以顺利的从一个航迹点到达另一个需要到达的相邻航迹点。因此我们以此为检验标准对相邻无人机的间距进行检验。

具体检验步骤如下所示:

(1)利用每架无人机航迹在每次雷达更新数据时得到的航迹点坐标位置,计算相邻两个航迹点间的位移距离;

(2)将计算出来的相邻航迹点间的位移距离与预先设定的检验范围(309.2m-500m)进行比较;

(3)如若计算出的相邻航迹点间的位移距离在预先设定的检验范围内,就可以说明无人机可以恰好用时 10s 到达相邻的航迹点;如若没有通过针对航迹点位移距离的检验,就需要对无人机飞行轨迹进行优化甚至是重新规划,以确保无人机在雷达接收空间状态数据时恰好到达特定的航迹点。

\subsubsection{b. 针对无人机飞行轨迹方向变化的检验}

由于题目中要求无人机尽量少做转弯、爬升以及俯冲动作,因为题目 2 中我们已经将无人机的飞行高度固定在 2000 米,所以不用考虑爬升以及俯冲的过程,尽量减少无人机在飞行轨迹中的转弯即可对此问题进行优化。还是利用已经知道的坐标可以计算出从某一个雷达数据更新区间结束至后一个雷达数据更新区间结束时,无人机飞行位移的大致方向,即无人机飞行轨迹在虚拟坐标系中的斜率,

利用这些斜率的变化程度可以近似完成对无人机飞行轨迹方向变化的检验。首先为了使无人机在飞行的过程中尽量减少转弯动作,并确保方向改变所需的时间尽可能不会影响到无人机飞行所需的时间,其次考虑到题目中规定无人机的最小转弯半径为 \(250 \, \text{m}\),无人机在 \(1 \, \text{s}\) 内依照最快速度可以完成的最小偏移角度为 \(20.6^\circ\),也就是说在此题目中应确保无人机不做偏移角度小于 \(20.6^\circ\) 的曲线运动,故将检验评价的标准设定为无人机飞行角度变化量至少为 \(20.6^\circ\) 的范围内。具体检验步骤如下所示:

(1)利用每架无人机航迹在每次雷达更新数据时得到的坐标位置,计算相邻两个位置间的斜率;

(2)利用反正切函数对斜率对应的方向角度进行求解;

(3)计算从某一次更新数据时所在的位置到邻近位置所需要发生的平均偏移角度;

(4)将计算出来的平均偏移角度与预先制定的标准进行比较即可得到对于无人机轨迹方向变化检验的相关结果;

(5)如若无人机飞行轨迹中的最小偏移角度大于 \(20.6^\circ\),那就是说此条航线通过了针对无人机角度的检验;如若没有通过针对无人机角度的检验,就需要对无人机飞行的航线继续进行优化,避免其航线轨迹中出现小于 \(20.6^\circ\) 的偏移角度。

c. 针对虚假航线真实性的检验

虚假航迹路线所描述的假目标飞行路径连续,最大速度、瞬时加速等性能不超过主流四代战斗机的一般性能。即最大速度在 \(2.5\) 倍音速以内,加速度在 \(4\) 倍重力加速度以内,否则视为航迹不连续。具体检验步骤如下所示:

(1)利用虚假航迹上的每个点的坐标,可以计算出虚拟航迹上相邻的两个点间的空间距离;

(2)利用雷达的数据更新率(\(10 \, \text{s}\)),可以计算出虚拟航迹上某个点到相邻的下一个点的平均速度,并求出平均速度的最大值;

(3)再利用雷达的数据更新率(\(10 \, \text{s}\))以及已经计算出来的平均速度数据,可以大致计算出虚拟航线的平均加速度,并计算出平均加速度的最大值;

(4)将计算出的指标与主流四代战斗机的一般性能进行比较,即最大速度在 \(2.5\) 倍音速以内,加速度在 \(4\) 倍重力加速度以内;

(5)如若在最大速度小于 \(2.5\) 倍音速(\(850 \, \text{m/s}\))并且最大加速度小于 \(4\) 倍重力加速度(\(39.2 \, \text{m/s}^2\)),就可以认为虚假航线符合实际情况;如若计算出的两个结果中存在一个不满足预先设定的标度,就认为虚假航线是不符合实际情况的,需要重新对无人机的航线进行优化。

问题二中的检验结论

(1)问题二中的 9 架无人机在五分钟内完成问题一中提及的虚假航线并发现新的虚假航线时,所处的飞行轨迹可以通过对于无人机间距的检验。

(2)问题二中的 9 架无人机在五分钟内完成问题一中提及的虚假航线并发现新的虚假航线时,所处的飞行轨迹可以通过对于无人机飞行轨迹方向变化的检验。

(3)问题二中产生的另外三条虚假航线轨迹均可以通过对虚假航线真实性的检验。

\section{问题三的建模与求解}

\subsection{问题的解答}

由第二问的求解过程可知,若要保持四条虚假航迹点稳定航行,在每一个时刻点都需要有无人机来扮演 $M_{AB}$、$M_{AE}$、$M_{BA}$、$M_{BE}$、$M_{EA}$、$M_{EB}$、$M_{AQ}$、$M_{BQ}$ 和 $M_{EQ}$ 的角色。然而,根据第三问的描述可知,不必在每个时刻都有三架无人机待命等待雷达系统的“同源检验”。雷达系统有时会自我怀疑,以至于放松对“同源检验”的要求。即便有时只有两个雷达传来信号通过了“同源检验”,只要在整条航迹当中这种情况不超过 3 次,雷达系统仍会认可整条航迹。无人机的协同规划可以变得更为灵活。然而,在规划之前,首先要确保无人机要顺利的产生问题 1 中的航迹点。

\subsubsection{寻找最优路径产生规定航迹点}

在构造新的虚拟航迹点之前,需要求解哪些时刻下扮演哪些角色的无人机可以暂时离岗,执行其它任务。离岗无人机所对应的角色需要满足如下条件:

条件 1:对应同一虚假航迹路线的角色,岗位空置次数不可大于 3 次。这就确保了上一问中的所有航迹在新的情况下仍被视为有效航迹。

条件 2:当 A 区域有无人机离岗,B 区域和 C 区域的对应无人机必需在岗,否则对应航迹点连“双雷达同源检验”都无法通过,航迹路线将失效。

条件 3:不难计算全部无人机的离岗机会为 9 次,原则上同一时刻只安排一架无人机离岗,这样就可以凑出 9 个时刻有飞机离岗,为产生新的有效虚假航迹路线创造条件。

条件 4:所有离岗无人机对应同一个雷达,这样可以确保虚拟航迹路线的较为平缓;飞机离岗和返岗的速度范围需要满足规定,尽可能减少急拐弯。

经过优化,确定离岗飞机为:12、14、17-20 时刻的 1 号机及 13、15-16 时刻的 3 号机。除了这些情况之外,其它情况在 1-20 时刻与第二问中的结果相同。

\subsubsection{无人机位置的几何特征}

令新的航迹路线在第 11 个时刻开始出现,其中 11 时刻只能通过“双雷达同源检验”,航迹线持续到第 30 个时刻。不妨将第 5 条虚假航迹点命名为 $P_5$,让其出现在面 $BEQ$ 上(这是三个可能的待选平面之一,但我们检查了其它情况,发现并不能产生更多的航迹点,且三个平面具有等价性,因此随意选了一个平面来帮助后面的计算)。已知 $M_{BE}$ 与 $M_{EB}$ 两架无人机在面 $BEQ$ 上有一个交点。由于该点在 $B$ 与 $M_{BE}$ 的连线上,无人机 $M_{BE}$ 可以持续地向 $B$ 雷达发信号,宣称该点有一个目标。类似地,无人机 $M_{EB}$ 可以持续地向 $E$ 雷达发信号,宣称该点有一个目标。离岗无人机可以在 12-20 的 9 个时刻里,向 $A$ 雷达持续地发现信号,宣传该点有一个目标。在前 11 个时刻,雷达系统对该点的航迹识别中,只有两个信号通过了“同源检验”;然而,在第 12 个时刻之后,一直到第 30 个时刻(原题要求无人机可实施长达 5 分钟的干扰,而每隔 10 秒定义一个时刻,共计 30 个时刻),都可以有第三个源头来支持该点的“同源检验”。该虚假航迹有 18 个时刻通过了三雷达同源检验,提前的两个时刻通过了双雷达同源检验。因此,在该位置可以有第四条有效航迹。

虚假航迹点需要满足如下条件:

条件 1:$P_5$ 在面 $QBE$ 上,  
条件 2:$P_5$ 在点 $B$ 与点 $M_{BE}$ 的连线上。  
条件 3:$P_5$ 在点 $E$ 与点 $M_{EB}$ 的连线上。

\subsection*{5.1.3 无人机及假目标航迹的航迹优化和坐标求解}

通过原题给出的条件,$B$ 点与 $E$ 点的坐标是已知的;通过第二问的计算,我们已经确定 $M_{BE}$ 和 $M_{EB}$ 在 10-20 时刻的全部交点。在一个平面上,两条确定直线若有交点,其交点一定是唯一的。因此,在第 10-20 时刻,$P_5$ 的坐标可以看成是给定的。而在第 21-30 个时刻,$P_5$ 的坐标不是唯一的,这就需要对剩余航迹路线进行优化。

给出 $P_5$ 剩余虚假航迹点如下约束条件:

条件 1:$P_5$ 剩余虚假航迹路线全程的加速度数值和方向具有连续性,进而速度和路线都具有连续性。$P_5$ 剩余虚假航迹路线尽管与其它虚假航迹线的坐标不同,但不同虚假目标的飞行习惯要有一定的相似性,即虚假目标的加速度等基本一致。

条件 2:第 10-20 时刻当中,用全部九架飞机协同配合,完成 $P_5$ 剩余虚假航迹路线,确保每个时刻都有三个雷达信号,且没有不可飞抵的任务点(无人机临近的任务点之间间隔过近或过远都被视为不可飞抵)。

条件 3:在这个过程中,尽可能避免无人机的拐弯、俯冲、变速。

在上面条件的约束下,我们求解出一条合适的 $P_5$ 虚假航迹路线。详细坐标见附件,在此绘制示意图如下:

\begin{figure}[h]
\centering
\includegraphics[width=0.8\textwidth]{image.png}
\caption{新产生的一条虚假航迹飞行轨迹图}
\end{figure}

在确定 $P_5$ 虚假航迹路线之后,还需要求解如何组织九架无人机协同配合,最终完成路线的伪装。下面给出一组新的任务定义,向 $A$ 雷达传递信号来伪装 $P_5$ 虚假航迹的无人机角色为 $M_{A5}$,向 $B$ 雷达传递信号来伪装 $P_5$ 虚假航迹的无人机角色为 $M_{B5}$,向 $E$ 雷达传递信号来伪装 $P_5$ 虚假航迹的无人机角色为 $M_{E5}$。在 11-20 时刻,扮演 $M_{B5}$ 与 $M_{BE}$ 两个角色无人机是重合的,扮演 $M_{E5}$ 与 $M_{EB}$ 两个无人机的角色是重合,而 $M_{A5}$ 的角色由 $A$ 区域里面的离岗无人机来扮演。下面给不同编号的无人机分配对应角色:

\begin{table}
\centering
\caption{不同时刻下扮演对应角色的无人机编号}
\begin{tabular}{c c c c c c c c c c c c}
\hline
角色 & 对应航迹点 & \multicolumn{10}{c}{时刻} \\
\cline{3-12}
 & & 11 & 12 & 13 & 14 & 15 & 16 & 17 & 18 & 19 & 20 \\
\hline
$M_{A5}$ & $P_5$ & 无 & 1 & 3 & 1 & 3 & 3 & 1 & 1 & 1 & 1 \\
$M_{B5}$ & $P_5$ & 6 & 6 & 6 & 6 & 6 & 6 & 6 & 6 & 6 & 6 \\
$M_{E5}$ & $P_5$ & 9 & 7 & 8 & 7 & 8 & 8 & 8 & 8 & 7 & 7 \\
\hline
\end{tabular}
\end{table}

$M_{B5}$ 与 $M_{E5}$ 的对应的无人机坐标与前一问一致,是给定的。在这些时刻扮演 $M_{A5}$ 角色的无人机全部是“离岗”无人机,尽管也有明确的编号,坐标已经与前一问当中有所不同。$M_{A5}$ 角色对应的坐标需要满足下面的约束条件:

条件 1:$M_{A5}$ 点在 A 与 $P_5$ 的连线上。

条件 2:$M_{A5}$ 的坐标与对应无人机前一个时刻及后一个时刻角色的坐标,距离不超过 500m,且不低于 310 米,确保对应无人机可以顺利在规定时间内飞达。

条件 3:要尽可能地避免无人机转弯、俯冲,减少无人机的变速。

可以优化出 1 号无人机、3 号无人机的飞行轨迹,具体坐标见附件,在此绘制示意图如下:

\begin{figure}[h]
\centering
\includegraphics[width=0.8\textwidth]{image.png}
\caption{1、3 号无人机飞行轨迹优化图}
\end{figure}

根据 $P_5$ 在 21-30 时刻的坐标,并规定无人机在高度 2.0km 的平面上飞行,可以求解出全部的无人机任务点,将各区域的任务点按照执行任务的时刻编号。通过计算任务点之间坐标的距离,可以计算出哪些任务可以合并。分布绘制 A、B 和 E 三个区域的任务合并图。

\begin{table}
\centering
\begin{tabular}{|c|c|c|c|c|c|c|c|c|c|c|}
\hline
\multicolumn{11}{|c|}{A区域各任务是否满足合并条件} \\
\hline
\multicolumn{1}{|c|}{} & \multicolumn{10}{c|}{A区域内伪装虚假航迹点的任务编号} \\
\hline
\multicolumn{1}{|c|}{} & 21 & 22 & 23 & 24 & 25 & 26 & 27 & 28 & 29 & 30 \\
\hline
21 & 否 & 否 & 否 & 否 & 否 & 否 & 否 & 否 & 否 & 否 \\
\hline
22 & 否 & 否 & 否 & 否 & 否 & 否 & 否 & 否 & 否 & 否 \\
\hline
23 & \textcolor{red}{是} & 否 & 否 & 否 & 否 & 否 & 否 & 否 & 否 & 否 \\
\hline
24 & \textcolor{red}{是} & \textcolor{red}{是} & 否 & 否 & 否 & 否 & 否 & 否 & 否 & 否 \\
\hline
25 & \textcolor{red}{是} & \textcolor{red}{是} & \textcolor{red}{是} & 否 & 否 & 否 & 否 & 否 & 否 & 否 \\
\hline
26 & \textcolor{red}{是} & \textcolor{red}{是} & \textcolor{red}{是} & \textcolor{red}{是} & 否 & 否 & 否 & 否 & 否 & 否 \\
\hline
27 & \textcolor{red}{是} & \textcolor{red}{是} & \textcolor{red}{是} & \textcolor{red}{是} & \textcolor{red}{是} & 否 & 否 & 否 & 否 & 否 \\
\hline
28 & \textcolor{red}{是} & \textcolor{red}{是} & \textcolor{red}{是} & \textcolor{red}{是} & \textcolor{red}{是} & 否 & 否 & 否 & 否 & 否 \\
\hline
29 & \textcolor{red}{是} & \textcolor{red}{是} & \textcolor{red}{是} & \textcolor{red}{是} & \textcolor{red}{是} & \textcolor{red}{是} & 否 & 否 & 否 & 否 \\
\hline
30 & \textcolor{red}{是} & \textcolor{red}{是} & \textcolor{red}{是} & \textcolor{red}{是} & \textcolor{red}{是} & \textcolor{red}{是} & \textcolor{red}{是} & 否 & 否 & 否 \\
\hline
\end{tabular}
\caption{A区域各任务是否满足合并条件}
\end{table}

\begin{table}
\centering
\begin{tabular}{|c|c|c|c|c|c|c|c|c|c|c|}
\hline
\multicolumn{11}{|c|}{B区域各任务是否满足合并条件} \\
\hline
\multicolumn{1}{|c|}{} & \multicolumn{10}{c|}{B区域内伪装虚假航迹点的任务编号} \\
\hline
\multicolumn{1}{|c|}{} & 21 & 22 & 23 & 24 & 25 & 26 & 27 & 28 & 29 & 30 \\
\hline
21 & 否 & 否 & 否 & 否 & 否 & 否 & 否 & 否 & 否 & 否 \\
\hline
22 & 否 & 否 & 否 & 否 & 否 & 否 & 否 & 否 & 否 & 否 \\
\hline
23 & 否 & 否 & 否 & 否 & 否 & 否 & 否 & 否 & 否 & 否 \\
\hline
24 & \textcolor{red}{是} & 否 & 否 & 否 & 否 & 否 & 否 & 否 & 否 & 否 \\
\hline
25 & \textcolor{red}{是} & \textcolor{red}{是} & 否 & 否 & 否 & 否 & 否 & 否 & 否 & 否 \\
\hline
26 & \textcolor{red}{是} & \textcolor{red}{是} & \textcolor{red}{是} & 否 & 否 & 否 & 否 & 否 & 否 & 否 \\
\hline
27 & \textcolor{red}{是} & \textcolor{red}{是} & \textcolor{red}{是} & \textcolor{red}{是} & 否 & 否 & 否 & 否 & 否 & 否 \\
\hline
28 & \textcolor{red}{是} & \textcolor{red}{是} & \textcolor{red}{是} & \textcolor{red}{是} & \textcolor{red}{是} & 否 & 否 & 否 & 否 & 否 \\
\hline
29 & \textcolor{red}{是} & \textcolor{red}{是} & \textcolor{red}{是} & \textcolor{red}{是} & \textcolor{red}{是} & \textcolor{red}{是} & 否 & 否 & 否 & 否 \\
\hline
30 & \textcolor{red}{是} & \textcolor{red}{是} & \textcolor{red}{是} & \textcolor{red}{是} & \textcolor{red}{是} & \textcolor{red}{是} & \textcolor{red}{是} & 否 & 否 & 否 \\
\hline
\end{tabular}
\caption{B区域各任务是否满足合并条件}
\end{table}

\begin{table}
\centering
\begin{tabular}{|c|c|c|c|c|c|c|c|c|c|c|}
\hline
\multicolumn{11}{|c|}{E区域各任务是否满足合并条件} \\
\hline
\multicolumn{1}{|c|}{} & \multicolumn{10}{c|}{E区域内伪装虚假航迹点的任务编号} \\
\hline
\multicolumn{1}{|c|}{} & 21 & 22 & 23 & 24 & 25 & 26 & 27 & 28 & 29 & 30 \\
\hline
21 & 否 & 否 & 否 & 否 & 否 & 否 & 否 & 否 & 否 & 否 \\
\hline
22 & 否 & 否 & 否 & 否 & 否 & 否 & 否 & 否 & 否 & 否 \\
\hline
23 & \textcolor{red}{是} & 否 & 否 & 否 & 否 & 否 & 否 & 否 & 否 & 否 \\
\hline
24 & \textcolor{red}{是} & \textcolor{red}{是} & 否 & 否 & 否 & 否 & 否 & 否 & 否 & 否 \\
\hline
25 & \textcolor{red}{是} & \textcolor{red}{是} & \textcolor{red}{是} & 否 & 否 & 否 & 否 & 否 & 否 & 否 \\
\hline
26 & \textcolor{red}{是} & \textcolor{red}{是} & \textcolor{red}{是} & \textcolor{red}{是} & 否 & 否 & 否 & 否 & 否 & 否 \\
\hline
27 & \textcolor{red}{是} & \textcolor{red}{是} & \textcolor{red}{是} & \textcolor{red}{是} & \textcolor{red}{是} & 否 & 否 & 否 & 否 & 否 \\
\hline
28 & \textcolor{red}{是} & \textcolor{red}{是} & \textcolor{red}{是} & \textcolor{red}{是} & \textcolor{red}{是} & \textcolor{red}{是} & 否 & 否 & 否 & 否 \\
\hline
29 & \textcolor{red}{是} & \textcolor{red}{是} & \textcolor{red}{是} & \textcolor{red}{是} & \textcolor{red}{是} & \textcolor{red}{是} & \textcolor{red}{是} & 否 & 否 & 否 \\
\hline
30 & \textcolor{red}{是} & \textcolor{red}{是} & \textcolor{red}{是} & \textcolor{red}{是} & \textcolor{red}{是} & \textcolor{red}{是} & \textcolor{red}{是} & \textcolor{red}{是} & 否 & 否 \\
\hline
\end{tabular}
\caption{E区域各任务是否满足合并条件}
\end{table}

图 20 A、B、E 区域的任务合并图

类似于第二问中利用无人机在完成任务时对 A、B、E 三个区域中航迹点合并任务的处理,我们利用 21-30 时刻的航迹点间距离以及 5.1 节中提及的任务合并的四个条件,除此之外,在这里的分析中时间的先后顺序同样重要。为了完成优化后得到的 P5 在 21-30 的虚假航线的生成,综合考虑上述因素,利用 Excel 绘制出上述任务合并图。

经过对合并图的分析及优化,可以大致阐述出对 9 架无人机的飞行航线安排,具体安排如下所示:

- 区域 A 为了生成 P5 在 21-30 时刻的虚拟航线,需要两架无人机的共同协作。其中一架无人机按顺序完成任务 21、任务 23、任务 25、任务 27 以及任务 29;而另一架无人机则需要按顺序处理任务 22、任务 24、任务 26、任务 28 和任务 30。

- 区域 B 为了生成此条虚拟航线,需要的无人机架次为三架次。其中一架无人机按顺序完成任务 21、任务 24、任务 27 以及任务 30;而另一架无人机按顺序完成任务 22、任务 25 以及任务 28;那么最后一架无人机只需要按顺序完成任务 23、任务 26 和任务 29。

- 区域 E 为了生成 P5 在 21-30 时刻的虚拟航线,需要使用两架无人机。其中一架按顺序完成任务 21、任务 23、任务 25、任务 27 以及任务 29;而另一架无人机需要按照先后顺序完成任务 22、任务 24、任务 26、任务 28 以及任务 30。

不同于第二问中可以任意对航线进行分配,21-30 时刻的虚拟航线 P5 是 11-20 时刻虚拟航迹的延续,具体由几号无人机去生成 21-30 时刻的虚拟航线还需要进一步的判断,因为在第二问中我们已经计算了第 20 时刻 9 架无人机所在的具体位置,所以仍利用 5.1 节中提及的任务合并的四个条件对航线分配进行判断,比较第 20 时刻 9 架无人机的具体位置以及 21-30 时刻航迹点所在位置,得出结论如下:

A 区域中 1 号和 3 号无人机在 20 时刻后存在完成任务可能性的最近时刻是 23 时刻,而 2 号无人机的最近时刻是 21 时刻。同理,也可以得到 B 区域中 4 号无人机在 20 时刻后存在完成任务可能性的最近时刻是 21 时刻,而 5 号、6 号无人机的最近时刻为 23 时刻。而 E 区域中 7 号、8 号无人机都可以完成 21 时刻以及 22 时刻的任务,另外 9 号无人机在 20 时刻后存在完成任务可能性的最近时刻是 22 时刻。再利用题目中给出的 “双雷达同源检验”,可以得到不同编号的无人机分配对应角色:

\begin{table}[h]
\centering
\caption{不同时刻下扮演对应角色的无人机编号}
\begin{tabular}{c c c c c c c c c c c}
\hline \hline
角色 & 对应航迹点 & \multicolumn{9}{c}{时刻} \\
 & & 21 & 22 & 23 & 24 & 25 & 26 & 27 & 28 & 29 & 30 \\
\hline
$M_{A5}$ & $P_5$ & 无 & 2 & 3 & 2 & 3 & 2 & 3 & 2 & 3 & 2 \\
$M_{B5}$ & $P_5$ & 4 & 无 & 6 & 4 & 5 & 6 & 4 & 5 & 6 & 4 \\
$M_{E5}$ & $P_5$ & 7 & 8 & 7 & 8 & 7 & 8 & 7 & 8 & 7 & 8 \\
\hline \hline
\end{tabular}
\end{table}

因此对应上文提到的航线顺序,A 区域中的两架无人机分别是 3 号和 2 号无人机,B 区域中的三架无人机分别是 4 号、5 号以及 6 号无人机,E 区域中的两家无人机分别为 7 号以及 8 号无人机。同时第 21 时刻和第 22 时刻所对应的航迹点就是题中提及的 “双雷达同源检验” 航迹点。

九架无人机在 11 到 30 时刻的飞行轨迹示意图如下所示:

\begin{figure}[h]
\centering
\caption{九架无人机在各区域内的飞行轨迹图}
\includegraphics[width=0.8\textwidth]{image.png}
\end{figure}

图 21 九架无人机在 11 到 30 时刻的飞行轨迹示意图

\subsection{5.2 无人机和假目标航迹的运动合理性检验}

再次利用问题二中关于无人机和对应航迹的运动合理性检验,仍旧从三个角度:(1)针对无人机运行轨迹上相邻无人机间距的检验、(2)针对无人机飞行轨迹方向变化的检验以及(3)对虚假航线的检验对问题三中无人机的飞行轨迹以及虚假航迹进行考察。检验结果如下:

(1)问题三中的 9 架无人机在降低 “同源性检验” 要求的情况下,在五分钟内完成问题一中虚假航线并再次产生出新的虚假航线时,所处的飞行轨迹可以通过对于无人机间距的检验。

(2)问题三中的 9 架无人机在降低 “同源性检验” 要求的情况下,在五分

钟内完成问题一中虚假航线并再次产生出新的虚假航线时,所处的飞行轨迹可以通过对于无人机飞行轨迹方向变化的检验。

(3)问题三中新产生的一条虚假航线轨迹可以通过对虚假航线真实性的检验。

\section*{6. 计算结果的精度评价}

在前面的分析过程中,文章通过优化的方法得出了无人机的行走坐标路线,使得能够产生足够多的虚假航迹,对雷达造成干扰。但是求解以及运算过程中必然会产生误差,我们所求的虚假航迹与真实的虚假航迹是否完全一样是值得思考与探究的问题。本节对文章计算结果的精度进行了分析。

由于虚假目标必然是无人机与雷达连线的交点,所以若其中存在一架无人机与对应雷达的连线并不与虚假目标相交,则说明存在一定的误差或者计算错误。因此,本文通过计算无人机与雷达的连线与虚假目标所在平面的交点和虚假目标的距离大小来判断计算结果的精度问题。如果其交点与虚假目标的距离接近于 0,则说明计算结果的精度较高;反之,计算结果具有较差的精度。

文章采取 bootstrap 随机抽样的方法对计算结果的精度进行了检验评价。由于一架无人机只能影响一个雷达的传输信号,所以文章通过随机抽样的方法随机选取无人机、随机时刻、随机假目标航迹点,通过求解无人机和相应雷达连线与对应面的交点坐标及其与虚假目标点的距离,从而对计算精度进行评价。在重复抽样 500 次的情况下,得到其距离差别的大小情况如下:

\begin{figure}[h]
    \centering
    \includegraphics[width=\textwidth]{image.png}
    \caption{虚假轨迹误差检验随机抽样分析结果}
\end{figure}

从检验结果可以看出,无人机与雷达的连线与对应虚假目标所在平面的交点(计算得到的虚假目标点)和虚假目标的距离基本上在 0 附近很小的范围内变动(相对于航线坐标的数量级)。使用本文方法规划出的各无人机协同干扰方案所产生的虚假目标坐标位置,具有较高的精度,模型具有良好的应用价值。

\section{结论}

本文研究了无人机的协同执行任务规划问题,运用立体几何模型和优化运筹方法,完成了生成指定虚假航迹、产生尽可能多的虚拟航迹等一系列的任务。解决问题的一般步骤是,首先根据无人机的协同条件求解几何性质,然后根据几何性质及坐标信息设计优化运筹,最终实现多无人机的统筹规划和飞行规律的计算。

问题 1 的结论为:在匀速直线运动的约束下,产生指定航迹点最少需要 31 架无人机。无人机必需保持协同的状态,在每个时刻点总有 3 架飞机分别出现在虚假航迹点与 2 号雷达、3 号雷达和 4 号雷达的连线上。所有无人机可以保持在高度 $2.5 \mathrm{~km}$ 的平面上飞行,飞行规律参加附件。

问题 2 的结论:9 架机动飞行的无人机,在产生指定航迹点路线的基础上,最多可以再产生 3 条有效航迹路线。无人机集中干扰 1 号、2 号和 5 号雷达,各无人机保持在高度 $2.0 \mathrm{~km}$ 平面上飞行,飞行规律参加附件。最终计算出的虚假航迹路线与 1 中给定的路线有一定的相似性,具体在各时刻的坐标值参见附件。

问题 3 的结论:针对放松后的航迹维持策略,最多可以再多生成 1 条有效的航迹路线。新生成的虚假目标航迹路线是从第 11 时刻开始生成的,并首先被 2 号雷达及 5 号雷达监测到,该航迹路线可以一直持续到 30 时刻,最终对组网雷达的欺骗。新航迹路线在各时刻的坐标见附件。

\section*{参考文献}

[1] 马亚涛, 赵国庆, 徐晨. 现有技术条件下对组网雷达的航迹欺骗[J]. 电子信息对抗技术, 28(2):34-37, 2013.

[2] 孙琳, 李小波, 毛云祥, 等. 基于多机协同的组网雷达欺骗干扰策略[J]. 电子信息对抗技术, 31(3):51-54, 2016.

[3] 郭淑芬, 余国文, 熊鑫, 等. 基于无人机协同的航迹欺骗干扰方法研究[J]. 空军预警学院学报, 32(1):44-48,2018.

[4] Mclain T W, Chandler P R, Pachter M. A decomposition strategy for optimal coordination of unmanned air vehicles[C]. American Control Conference, 1(6):369-373, 2002.

[5] Song S, Zhang K, Guan S. A Trajectory Planning Method Based on Hierarchy Decomposition Strategy for Coordination of Multiple Unmanned Air Vehicles[J]. Tactical Missile Technology, 2004.