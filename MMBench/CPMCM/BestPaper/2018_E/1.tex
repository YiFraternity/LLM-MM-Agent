\documentclass{article}
\usepackage{amsmath}
\usepackage{amssymb}

\title{多无人机对组网雷达的协同干扰}
\author{}
\date{}

\begin{document}

\maketitle

\begin{abstract}
文章研究多无人机协同干扰的规划问题,求解协同策略和飞行方案,以对组网雷达实施有效的欺骗干扰。多无人机中任何一架无人机的航迹规划一旦差之毫厘、“同源检验”就有可能失之交臂。这就要求对每一架无人机的飞行路径和它们之间的协同策略进行周密的安排和精确的计算。使用直角坐标系直接描述多无人机的动态过程、空间关系及协同策略是十分困难的。本文采用拓扑学的思想搭建立体几何模型,简化了问题的求解过程。

拓扑学告诉我们,空间物体在形变的过程中,物体上点和点之间存在一些相对稳定的次序比邻关系,这些关系不随它们之间距离和方向的变化而变化。类似地,根据无人机协同工作的原理,可以推断出虚假航迹点、信号发射器角色、雷达点之间总会保持一些稳定的几何关系。让这些关系保持稳定,就是无人机合作的协同策略。找出这些关系,配合每个时刻假目标航迹点坐标的具体情况,可以求解出扮演信号发射器角色的无人机在每个时刻所需要到达的位置坐标,这就把随时间变化的“几何角色”转化成“何时赶往何处”的四维任务。在给无人机分配任务的时候,需要用优化统筹的思路来决定由哪架无人机扮演哪个角色,找到可行的航迹规划方案。通过以上步骤,可以规划出每一架无人机、每一条虚假航迹的合适坐标。文章给出了全部相关坐标,参见附件。

问题一要求规划一个协同方案,用尽可能少的无人机产生出给定的虚假航迹路线。通过处理数据可以发现,安排全部无人机在高度 $2.5 \mathrm{~km}$ 平面飞行,可以用较少的无人机完成全部的干扰工作。结合“距离欺骗”的基本原理,我们发现扮演干扰信号发射器角色的无人机需要在雷达收发信号的时刻出现在“雷达 - 假目标”连线与高度 $2.5 \mathrm{~km}$ 平面的交点上。这个几何性质是不随“假目标”坐标的变换而变化的。这就可以解出,要产生这样的假目标,5 个雷达在 20 个时刻所对应的无人机的具体坐标,也就得到 100 个“何时须到何处”的具体任务。删减多余的任务,并把可由同一架飞机完成的任务做合并处理,最终求解出使用最少无人机完成任务的多无人机协同干扰方案。研究发现,在平面匀速直线运动的条件约束下,产生全部虚假航迹点最少需要 31 架无人机。它们分工协作,在 20 个时刻里对 2 号雷达、3 号雷达和 4 号雷达实施干扰。

问题二允许九架无人机在 5 分钟内机动飞行,要求在产生指定航迹的基础上,尽可能地产生更多有效虚假航迹路线。通过计算和比较,我们选择对 1、2、5 三个雷达实施干扰,让全部无人机在高度 2.0km 平面飞行。针对该问题,我们构造了以“规定航迹点”和 1、2、5 三台雷达为顶点的四面体模型。根据棱边、侧面、平面等一系列的几何关系,可以求解出信号发射器之间协同关系的约束条件和协同策略。通过分析几何模型可以发现,在产生成指定的 1 条虚假航迹线基础上,最多还可以生成 3 条新的虚假航迹线。通过优化计算,可以求解出较好的任务分配方案,这样就可以确定第 1-20 时刻每一架无人机的相应坐标,同时也就确定了虚假航迹点在每个时刻的具体坐标。

问题三允许航迹路线存在瑕疵,即连续的 20 个航迹点当中,若不少于 17 个点满足“三雷达同源检验”,且剩余点满足“双雷达同源检验”,雷达就可将该航迹路线视为有效航迹路线。针对这样的条件,我们在前一个四面体模型的基础上,寻找可暂时离岗的无人机。借助模型研究发现,在 12-20 时刻期间,每期可安排一架无人机离岗。离岗无人机配合其它无人机,能在这些时刻产生出一连串新的航迹点。有了这些新的航迹点,多无人机在完成指定的 1 条虚假航迹线基础上,最多还可以产生 4 条新的有效航迹线。这 1 条多出来的航迹线从 11 时刻开始产生,持续到第 30 时刻。通过几何分析,可以描述出每一个“发射器角色”的协同策略;通过优化计算,可以把角色合理地分配给无人机,进而计算出无人机坐标和虚假航迹点坐标。

每回答一个问题,我们都根据题目要求,对无人机间距、加速情况、拐弯情况是否满足要求做检验。我们还以主流第四代战斗机相关参数为标准,对假目标的瞬时加速度、机动能力等指标做了检查,确保虚假航迹路线是可信的。最后,作为本文的一个重要创新,文章专门编写程序重复抽样,对虚假航迹点能否通过“同源检验”做了检查。研究发现,计算结论与模型推理相一致,误差可控、结论可靠。由此可见,借鉴拓扑学思想可以极大地简化协同问题的规划,该思路有广泛的应用价值。

关键词:电子对抗 协同干扰 立体几何 运筹优化 
\end{abstract}

\tableofcontents

\section{问题重述}

距离观测的基本原理是根据电磁波从发射到反射所用的时间,来判断目标与雷达间的距离。而距离假目标欺骗的原理是,干扰机对雷达信号进行处理之后,再反射给对应的雷达。通过控制电子波返回雷达的时间,可以给雷达对距离的判断带来误导。这种干扰有一个基本的特征,假目标一定在无人机与雷达的连线上。

\begin{figure}[h]
    \centering
    \includegraphics[width=0.8\textwidth]{image1.png}
    \caption{对雷达实施距离多假目标欺骗干扰示意图(图片来自原题题干)}
\end{figure}

组网雷达系统具有良好的抗电子干扰、抗低空突防能力,得到越来越广泛的应用\cite{ref1}。对敌方雷达实施有效干扰是十分困难的,原因在于一旦敌方使用多个雷达,不同雷达所发现的某目标空间状态可能会不一致,进而识破无人机的欺骗。这种剔除虚假目标的思想简称为“同源检验”\cite{ref2}。根据题目要求,至少让三个雷达监控到一致的航迹点时,才算是虚假航迹通过了“同源检验”,进而被融合中心视为一个真实的航迹点。为了同时欺骗多台雷达,就需要利用多架无人机的协同欺骗,让多个雷达看到同一个虚假目标,这样就达到了欺骗敌方雷达的效果\cite{ref3}。

\begin{figure}[h]
    \centering
    \includegraphics[width=0.8\textwidth]{image2.png}
    \caption{多无人机协同干扰组网雷达系统示意图(图片来自原题题干)}
\end{figure}

问题一要求产生指定航线。题目给出虚假目标航迹路线,讨论如何以最少数量的无人机在匀速直线飞行的约束性,产生出这些虚假目标航迹点,并分析每一架无人机的运动规律和相应的协同策略。

问题二要求求解,使用九架飞机机动飞行,如何在产生规定航线的同时再产生尽可能多的有效航迹路线。题目要求给出每一架无人机的运动规律,并分析每一条虚假航迹的运动规律和合理性。 
问题三中的雷达改变了航迹维持策略,要求求解在新的航迹约束下,重新讨
论由9 架无人机组成的编队在5 分钟内,完成附件1 要求的虚假航迹的同时,至
多还可产生出多少条虚假的航迹。题目要求给出每一架无人机的运动规律和协同
策略,并分析每一条虚假航迹的运动规律和合理性。

\newpage

\section{问题分析}

\section{模型假设}

\subsection{题目对模型给出的假设}

\subsection{为简化模型的求解而追加的假设}

\subsection{符号说明}

\section{问题一的建模与求解}

\subsection{对问题的可视化分析}

\subsection{问题的解答}

\subsection{无人机的运动合理性检验}

\section{问题二的建模与求解}

\subsection{无人机机动能力的可视化分析}

\subsection{问题的解答}

\subsubsection{产生规定航迹点}

\subsubsection{研究四面体模型归纳无人机与假目标的几何特征}

\subsubsection{计算无人机和假目标在各时刻的实际坐标}

\subsection{无人机和假目标航迹的运动合理性检验}

\section{问题三的建模与求解}

\subsection{问题的解答}

\subsubsection{寻找最优路径产生规定航迹点}

\subsubsection{无人机位置的几何特征}

\subsubsection{无人机及假目标航迹的航迹优化和坐标求解}

\subsection{无人机和假目标航迹的运动合理性检验}

\section{计算结果的精度评价}

\section{结论}

\textbf{参考文献}

[REFERENCES:n]

\textbf{附录}

[APPENDIX:n]

[REFERENCES:1]

\end{document}