\section*{全国第五届研究生数学建模竞赛}

\begin{center}
\includegraphics[width=0.4\textwidth]{image.png}
\end{center}

\textbf{题目} 基于能量模型的交通信号实时优化

\section*{摘要:}

本文提出了交叉路口的能量模型,将交叉路口节点看作一个具有能量属性的系统,而把车辆的等待延时看作车辆流对节点能量的积累作用,将求解车辆等待时间最短问题转化单位时间系统能量的最小问题,从而达到对交通信号灯的最优控制。

对于单个交叉路口:建立节点能量模型,引入和定义了交叉路口的节点能量来度量车辆等待时间。首先对黄灯时长进行优化,并用概率模型产生 Poisson 仿真数据流,再利用遗传算法和模拟退火算法求解模型。得出当周期固定时和不固定时,实时控制方案的车辆等待时间分别小于固定配时方案 10.83\% 和 33.18\%,并且当车流量越大,分布越不均匀时,该模型和算法的优越性越明显。

对于线状区域多个交叉路口:在节点能量模型的基础上建立耦合能量模型,引入和定义耦合能量来度量相邻交叉路口的相互影响大小,并定量计算出节点耦合能量。利用遗传算法求得实时控制方案比固定配时方案的延误平均降低 14.59\%。最后,根据实际数值验证了耦合能量模型的合理性,并将此模型在两节点的算法扩展到线状区域的任意节点。

对于网状区域多个交叉路口:在节点能量模型和耦合能量模型的基础上建立网格能量模型。利用三角形划分平面区域,引入网格耦合能量来度量垂直的两条道路的相互影响,利用遗传算法求得实时控制方案的延误平均降低 24.20\%。最后,将以上三个模型综合并推广到任意网格,得出:网格能量 = 节点能量 + 节点耦合能量 + 网格耦合能量。

本文还详细地给出了遗传算法和模拟退火算法在求解以上每个模型时的可计算性和时间、空间复杂度。最后,根据节点能量模型和优化算法给出了预估的信号实时控制方法建议。

\textbf{本文特色}

1. 建立的能量模型从系统层面深刻刻画了交叉路口节点的行为和本质,类比物理概念解决数学问题。

2. 在讨论线状和网状区域时引入节点耦合能量和网格耦合能量,一方面从宏观考虑,有效避免了节点内部的复杂计算;另一方面使模型便于扩展,可以推广到一般线状和网络区域。

\textbf{关键字:} 能量模型 \quad 耦合 \quad 网格 \quad 遗传算法

\textbf{参赛队号} 1069812

\begin{tabular}{|c|}
\hline 参赛密码 \_\_\_\_\_\_\_\_\_\_\_\_\_\_\_\_\_\_\_\_\_\_\_\_\_\_\_\_\_\_\_\_\_\_\_\_\_\_\_\_\_\_\_\_\_\_\_\_\_\_\_\_\_\_\_\_\_\_\_\_\_\_\_\_\_\_\_\_\_\_\_\_\_\_\_\_\_\_\_\_\_\_\_\_\_\_\_\_\_\_\_\_\_\_\_\_\_\_\_\_\_\_\_\_\_\_\_\_\_\_\_\_\_\_\_\_\_\_\_\_\_\_\_\_\_\_\_\_\_\_\_\_\_\_\_\_\_\_\_\_\_\_\_\_\_\_\_\_\_\_\_\_\_\_\_\_\_\_\_\_\_\_\_\_\_\_\_\_\_\_\_\_\_\_\_\_\_\_\_\_\_\_\_\_\_\_\_\_\_\_\_\_\_\_\_\_\_\_\_\_\_\_\_\_\_\_\_\_\_\_\_\_\_\_\_\_\_\_\_\_\_\_\_\_\_\_\_\_\_\_\_\_\_\_\_\_\_\_\_\_\_\_\_\_\_\_\_\_\_\_\_\_\_\_\_\_\_\_\_\_\_\_\_\_\_\_\_\_\_\_\_\_\_\_\_\_\_\_\_\_\_\_\_\_\_\_\_\_\_\_\_\_\_\_\_\_\_\_\_\_\_\_\_\_\_\_\_\_\_\_\_\_\_\_\_\_\_\_\_\_\_\_\_\_\_\_\_\_\_\_\_\_\_\_\_\_\_\_\_\_\_\_\_\_\_\_\_\_\_\_\_\_\_\_\_\_\_\_\_\_\_\_\_\_\_\_\_\_\_\_\_\_\_\_\_\_\_\_\_\_\_\_\_\_\_\_\_\_\_\_\_\_\_\_\_\_\_\_\_\_\_\_\_\_\_\_\_\_\_\_\_\_\_\_\_\_\_\_\_\_\_\_\_\_\_\_\_\_\_\_\_\_\_\_\_\_\_\_\_\_\_\_\_\_\_\_\_\_\_\_\_\_\_\_\_\_\_\_\_\_\_\_\_\_\_\_\_\_\_\_\_\_\_\_\_\_\_\_\_\_\_\_\_\_\_\_\_\_\_\_\_\_\_\_\_\_\_\_\_\_\_\_\_\_\_\_\_\_\_\_\_\_\_\_\_\_\_\_\_\_\_\_\_\_\_\_\_\_\_\_\_\_\_\_\_\_\_\_\_\_\_\_\_\_\_\_\_\_\_\_\_\_\_\_\_\_\_\_\_\_\_\_\_\_\_\_\_\_\_\_\_\_\_\_\_\_\_\_\_\_\_\_\_\_\_\_\_\_\_\_\_\_\_\_\_\_\_\_\_\_\_\_\_\_\_\_\_\_\_\_\_\_\_\_\_\_\_\_\_\_\_\_\_\_\_\_\_\_\_\_\_\_\_\_\_\_\_\_\_\_\_\_\_\_\_\_\_\_\_\_\_\_\_\_\_\_\_\_\_\_\_\_\_\_\_\_\_\_\_\_\_\_\_\_\_\_\_\_\_\_\_\_\_\_\_\_\_\_\_\_\_\_\_\_\_\_\_\_\_\_\_\_\_\_\_\_\_\_\_\_\_\_\_\_\_\_\_\_\_\_\_\_\_\_\_\_\_\_\_\_\_\_\_\_\_\_\_\_\_\_\_\_\_\_\_\_\_\_\_\_\_\_\_\_\_\_\_\_\_\_\_\_\_\_\_\_\_\_\_\_\_\_\_\_\_\_\_\_\_\_\_\_\_\_\_\_\_\_\_\_\_\_\_\_\_\_\_\_\_\_\_\_\_\_\_\_\_\_\_\_\_\_\_\_\_\_\_\_\_\_\_\_\_\_\_\_\_\_\_\_\_\_\_\_\_\_\_\_\_\_\_\_\_\_\_\_\_\_\_\_\_\_\_\_\_\_\_\_\_\_\_\_\_\_\_\_\_\_\_\_\_\_\_\_\_\_\_\_\_\_\_\_\_\_\_\_\_\_\_\_\_\_\_\_\_\_\_\_\_\_\_\_\_\_\_\_\_\_\_\_\_\_\_\_\_\_\_\_\_\_\_\_\_\_\_\_\_\_\_\_\_\_\_\_\_\_\_\_\_\_\_\_\_\_\_\_\_\_\_\_\_\_\_\_\_\_\_\_\_\_\_\_\_\_\_\_\_\_\_\_\_\_\_\_\_\_\_\_\_\_\_\_\_\_\_\_\_\_\_\_\_\_\_\_\_\_\_\_\_\_\_\_\_\_\_\_\_\_\_\_\_\_\_\_\_\_\_\_\_\_\_\_\_\_\_\_\_\_\_\_\_\_\_\_\_\_\_\_\_\_\_\_\_\_\_\_\_\_\_\_\_\_\_\_\_\_\_\_\_\_\_\_\_\_\_\_\_\_\_\_\_\_\_\_\_\_\_\_\_\_\_\_\_\_\_\_\_\_\_\_\_\_\_\_\_\_\_\_\_\_\_\_\_\_\_\_\_\_\_\_\_\_\_\_\_\_\_\_\_\_\_\_\_\_\_\_\_\_\_\_\_\_\_\_\_\_\_\_\_\_\_\_\_\_\_\_\_\_\_\_\_\_\_\_\_\_\_\_\_\_\_\_\_\_\_\_\_\_\_\_\_\_\_\_\_\_\_\_\_\_\_\_\_\_\_\_\_\_\_\_\_\_\_\_\_\_\_\_\_\_\_\_\_\_\_\_\_\_\_\_\_\_\_\_\_\_\_\_\_\_\_\_\_\_\_\_\_\_\_\_\_\_\_\_\_\_\_\_\_\_\_\_\_\_\_\_\_\_\_\_\_\_\_\_\_\_\_\_\_\_\_\_\_\_\_\_\_\_\_\_\_\_\_\_\_\_\_\_\_\_\_\_\_\_\_\_\_\_\_\_\_\_\_\_\_\_\_\_\_\_\_\_\_\_\_\_\_\_\_\_\_\_\_\_\_\_\_\_\_\_\_\_\_\_\_\_\_\_\_\_\_\_\_\_\_\_\_\_\_\_\_\_\_\_\_\_\_\_\_\_\_\_\_\_\_\_\_\_\_\_\_\_\_\_\_\_\_\_\_\_\_\_\_\_\_\_\_\_\_\_\_\_\_\_\_\_\_\_\_\_\_\_\_\_\_\_\_\_\_\_\_\_\_\_\_\_\_\_\_\_\_\_\_\_\_\_\_\_\_\_\_\_\_\_\_\_\_\_\_\_\_\_\_\_\_\_\_\_\_\_\_\_\_\_\_\_\_\_\_\_\_\_\_\_\_\_\_\_\_\_\_\_\_\_\_\_\_\_\_\_\_\_\_\_\_\_\_\_\_\_\_\_\_\_\_\_\_\_\_\_\_\_\_\_\_\_\_\_\_\_\_\_\_\_\_\_\_\_\_\_\_\_\_\_\_\_\_\_\_\_\_\_\_\_\_\_\_\_\_\_\_\_\_\_\_\_\_\_\_\_\_\_\_\_\_\_\_\_\_\_\_\_\_\_\_\_\_\_\_\_\_\_\_\_\_\_\_\_\_\_\_\_\_\_\_\_\_\_\_\_\_\_\_\_\_\_\_\_\_\_\_\_\_\_\_\_\_\_\_\_\_\_\_\_\_\_\_\_\_\_\_\_\_\_\_\_\_\_\_\_\_\_\_\_\_\_\_\_\_\_\_\_\_\_\_\_\_\_\_\_\_\_\_\_\_\_\_\_\_\_\_\_\_\_\_\_\_\_\_\_\_\_\_\_\_\_\_\_\_\_\_\_\_\_\_\_\_\_\_\_\_\_\_\_\_\_\_\_\_\_\_\_\_\_\_\_\_\_\_\_\_\_\_\_\_\_\_\_\_\_\_\_\_\_\_\_\_\_\_\_\_\_\_\_\_\_\_\_\_\_\_\_\_\_\_\_\_\_\_\_\_\_\_\_\_\_\_\_\_\_\_\_\_\_\_\_\_\_\_\_\_\_\_\_\_\_\_\_\_\_\_\_\_\_\_\_\_\_\_\_\_\_\_\_\_\_\_\_\_\_\_\_\_\_\_\_\_\_\_\_\_\_\_\_\_\_\_\_\_\_\_\_\_\_\_\_\_\_\_\_\_\_\_\_\_\_\_\_\_\_\_\_\_\_\_\_\_\_\_\_\_\_\_\_\_\_\_\_\_\_\_\_\_\_\_\_\_\_\_\_\_\_\_\_\_\_\_\_\_\_\_\_\_\_\_\_\_\_\_\_\_\_\_\_\_\_\_\_\_\_\_\_\_\_\_\_\_\_\_\_\_\_\_\_\_\_\_\_\_\_\_\_\_\_\_\_\_\_\_\_\_\_\_\_\_\_\_\_\_\_\_\_\_\_\_\_\_\_\_\_\_\_\_\_\_\_\_\_\_\_\_\_\_\_\_\_\_\_\_\_\_\_\_\_\_\_\_\_\_\_\_\_\_\_\_\_\_\_\_\_\_\_\_\_\_\_\_\_\_\_\_\_\_\_\_\_\_\_\_\_\_\_\_\_\_\_\_\_\_\_\_\_\_\_\_\_\_\_\_\_\_\_\_\_\_\_\_\_\_\_\_\_\_\_\_\_\_\_\_\_\_\_\_\_\_\_\_\_\_\_\_\_\_\_\_\_\_\_\_\_\_\_\_\_\_\_\_\_\_\_\_\_\_\_\_\_\_\_\_\_\_\_\_\_\_\_\_\_\_\_\_\_\_\_\_\_\_\_\_\_\_\_\_\_\_\_\_\_\_\_\_\_\_\_\_\_\_\_\_\_\_\_\_\_\_\_\_\_\_\_\_\_\_\_\_\_\_\_\_\_\_\_\_\_\_\_\_\_\_\_\_\_\_\_\_\_\_\_\_\_\_\_\_\_\_\_\_\_\_\_\_\_\_\_\_\_\_\_\_\_\_\_\_\_\_\_\_\_\_\_\_\_\_\_\_\_\_\_\_\_\_\_\_\_\_\_\_\_\_\_\_\_\_\_\_\_\_\_\_\_\_\_\_\_\_\_\_\_\_\_\_\_\_\_\_\_\_\_\_\_\_\_\_\_\_\_\_\_\_\_\_\_\_\_\_\_\_\_\_\_\_\_\_\_\_\_\_\_\_\_\_\_\_\_\_\_\_\_\_\_\_\_\_\_\_\_\_\_\_\_\_\_\_\_\_\_\_\_\_\_\_\_\_\_\_\_\_\_\_\_\_\_\_\_\_\_\_\_\_\_\_\_\_\_\_\_\_\_\_\_\_\_\_\_\_\_\_\_\_\_\_\_\_\_\_\_\_\_\_\_\_\_\_\_\_\_\_\_\_\_\_\_\_\_\_\_\_\_\_\_\_\_\_\_\_\_\_\_\_\_\_\_\_\_\_\_\_\_\_\_\_\_\_\_\_\_\_\_\_\_\_\_\_\_\_\_\_\_\_\_\_\_\_\_\_\_\_\_\_\_\_\_\_\_\_\_\_\_\_\_\_\_\_\_\_\_\_\_\_\_\_\_\_\_\_\_\_\_\_\_\_\_\_\_\_\_\_\_\_\_\_\_\_\_\_\_\_\_\_\_\_\_\_\_\_\_\_\_\_\_\_\_\_\_\_\_\_\_\_\_\_\_\_\_\_\_\_\_\_\_\_\_\_\_\_\_\_\_\_\_\_\_\_\_\_\_\_\_\_\_\_\_\_\_\_\_\_\_\_\_\_\_\_\_\_\_\_\_\_\_\_\_\_\_\_\_\_\_\_\_\_\_\_\_\_\_\_\_\_\_\_\_\_\_\_\_\_\_\_\_\_\_\_\_\_\_\_\_\_\_\_\_\_\_\_\_\_\_\_\_\_\_\_\_\_\_\_\_\_\_\_\_\_\_\_\_\_\_\_\_\_\_\_\_\_\_\_\_\_\_\_\_\_\_\_\_\_\_\_\_\_\_\_\_\_\_\_\_\_\_\_\_\_\_\_\_\_\_\_\_\_\_\_\_\_\_\_\_\_\_\_\_\_\_\_\_\_\_\_\_\_\_\_\_\_\_\_\_\_\_\_\_\_\_\_\_\_\_\_\_\_\_\_\_\_\_\_\_\_\_\_\_\_\_\_\_\_\_\_\_\_\_\_\_\_\_\_\_\_\_\_\_\_\_\_\_\_\_\_\_\_\_\_\_\_\_\_\_\_\_\_\_\_\_\_\_\_\_\_\_\_\_\_\_\_\_\_\_\_\_\_\_\_\_\_\_\_\_\_\_\_\_\_\_\_\_\_\_\_\_\_\_\_\_\_\_\_\_\_\_\_\_\_\_\_\_\_\_\_\_\_\_\_\_\_\_\_\_\_\_\_\_\_\_\_\_\_\_\_\_\_\_\_\_\_\_\_\_\_\_\_\_\_\_\_\_\_\_\_\_\_\_\_\_\_\_\_\_\_\_\_\_\_\_\_\_\_\_\_\_\_\_\_\_\_\_\_\_\_\_\_\_\_\_\_\_\_\_\_\_\_\_\_\_\_\_\_\_\_\_\_\_\_\_\_\_\_\_\_\_\_\_\_\_\_\_\_\_\_\_\_\_\_\_\_\_\_\_\_\_\_\_\_\_\_\_\_\_\_\_\_\_\_\_\_\_\_\_\_\_\_\_\_\_\_\_\_\_\_\_\_\_\_\_\_\_\_\_\_\_\_\_\_\_\_\_\_\_\_\_\_\_\_\_\_\_\_\_\_\_\_\_\_\_\_\_\_\_\_\_\_\_\_\_\_\_\_\_\_\_\_\_\_\_\_\_\_\_\_\_\_\_\_\_\_\_\_\_\_\_\_\_\_\_\_\_\_\_\_\_\_\_\_\_\_\_\_\_\_\_\_\_\_\_\_\_\_\_\_\_\_\_\_\_\_\_\_\_\_\_\_\_\_\_\_\_\_\_\_\_\_\_\_\_\_\_\_\_\_\_\_\_\_\_\_\_\_\_\_\_\_\_\_\_\_\_\_\_\_\_\_\_\_\_\_\_\_\_\_\_\_\_\_\_\_\_\_\_\_\_\_\_\_\_\_\_\_\_\_\_\_\_\_\_\_\_\_\_\_\_\_\_\_\_\_\_\_\_\_\_\_\_\_\_\_\_\_\_\_\_\_\_\_\_\_\_\_\_\_\_\_\_\_\_\_\_\_\_\_\_\_\_\_\_\_\_\_\_\_\_\_\_\_\_\_\_\_\_\_\_\_\_\_\_\_\_\_\_\_\_\_\_\_\_\_\_\_\_\_\_\_\_\_\_\_\_\_\_\_\_\_\_\_\_\_\_\_\_\_\_\_\_\_\_\_\_\_\_\_\_\_\_\_\_\_\_\_\_\_\_\_\_\_\_\_\_\_\_\_\_\_\_\_\_\_\_\_\_\_\_\_\_\_\_\_\_\_\_\_\_\_\_\_\_\_\_\_\_\_\_\_\_\_\_\_\_\_\_\_\_\_\_\_\_\_\_\_\_\_\_\_\_\_\_\_\_\_\_\_\_\_\_\_\_\_\_\_\_\_\_\_\_\_\_\_\_\_\_\_\_\_\_\_\_\_\_\_\_\_\_\_\_\_\_\_\_\_\_\_\_\_\_\_\_\_\_\_\_\_\_\_\_\_\_\_\_\_\_\_\_\_\_\_\_\_\_\_\_\_\_\_\_\_\_\_\_\_\_\_\_\_\_\_\_\_\_\_\_\_\_\_\_\_\_\_\_\_\_\_\_\_\_\_\_\_\_\_\_\_\_\_\_\_\_\_\_\_\_\_\_\_\_\_\_\_\_\_\_\_\_\_\_\_\_\_\_\_\_\_\_\_\_\_\_\_\_\_\_\_\_\_\_\_\_\_\_\_\_\_\_\_\_\_\_\_\_\_\_\_\_\_\_\_\_\_\_\_\_\_\_\_\_\_\_\_\_\_\_\_\_\_\_\_\_\_\_\_\_\_\_\_\_\_\_\_\_\_\_\_\_\_\_\_\_\_\_\_\_\_\_\_\_\_\_\_\_\_\_\_\_\_\_\_\_\_\_\_\_\_\_\_\_\_\_\_\_\_\_\_\_\_\_\_\_\_\_\_\_\_\_\_\_\_\_\_\_\_\_\_\_\_\_\_\_\_\_\_\_\_\_\_\_\_\_\_\_\_\_\_\_\_\_\_\_\_\_\_\_\_\_\_\_\_\_\_\_\_\_\_\_\_\_\_\_\_\_\_\_\_\_\_\_\_\_\_\_\_\_\_\_\_\_\_\_\_\_\_\_\_\_\_\_\_\_\_\_\_\_\_\_\_\_\_\_\_\_\_\_\_\_\_\_\_\_\_\_\_\_\_\_\_\_\_\_\_\_\_\_\_\_\_\_\_\_\_\_\_\_\_\_\_\_\_\_\_\_\_\_\_\_\_\_\_\_\_\_\_\_\_\_\_\_\_\_\_\_\_\_\_\_\_\_\_\_\_\_\_\_\_\_\_\_\_\_\_\_\_\_\_\_\_\_\_\_\_\_\_\_\_\_\_\_\_\_\_\_\_\_\_\_\_\_\_\_\_\_\_\_\_\_\_\_\_\_\_\_\_\_\_\_\_\_\_\_\_\_\_\_\_\_\_\_\_\_\_\_\_\_\_\_\_\_\_\_\_\_\_\_\_\_\_\_\_\_\_\_\_\_\_\_\_\_\_\_\_\_\_\_\_\_\_\_\_\_\_\_\_\_\_\_\_\_\_\_\_\_\_\_\_\_\_\_\_\_\_\_\_\_\_\_\_\_\_\_\_\_\_\_\_\_\_\_\_\_\_\_\_\_\_\_\_\_\_\_\_\_\_\_\_\_\_\_\_\_\_\_\_\_\_\_\_\_\_\_\_\_\_\_\_\_\_\_\_\_\_\_\_\_\_\_\_\_\_\_\_\_\_\_\_\_\_\_\_\_\_\_\_\_\_\_\_\_\_\_\_\_\_\_\_\_\_\_\_\_\_\_\_\_\_\_\_\_\_\_\_\_\_\_\_\_\_\_\_\_\_\_\_\_\_\_\_\_\_\_\_\_\_\_\_\_\_\_\_\_\_\_\_\_\_\_\_\_\_\_\_\_\_\_\_\_\_\_\_\_\_\_\_\_\_\_\_\_\_\_\_\_\_\_\_\_\_\_\_\_\_\_\_\_\_\_\_\_\_\_\_\_\_\_\_\_\_\_\_\_\_\_\_\_\_\_\_\_\_\_\_\_\_\_\_\_\_\_\_\_\_\_\_\_\_\_\_\_\_\_\_\_\_\_\_\_\_\_\_\_\_\_\_\_\_\_\_\_\_\_\_\_\_\_\_\_\_\_\_\_\_\_\_\_\_\_\_\_\_\_\_\_\_\_\_\_\_\_\_\_\_\_\_\_\_\_\_\_\_\_\_\_\_\_\_\_\_\_\_\_\_\_\_\_\_\_\_\_\_\_\_\_\_\_\_\_\_\_\_\_\_\_\_\_\_\_\_\_\_\_\_\_\_\_\_\_\_\_\_\_\_\_\_\_\_\_\_\_\_\_\_\_\_\_\_\_\_\_\_\_\_\_\_\_\_\_\_\_\_\_\_\_\_\_\_\_\_\_\_\_\_\_\_\_\_\_\_\_\_\_\_\_\_\_\_\_\_\_\_\_\_\_\_\_\_\_\_\_\_\_\_\_\_\_\_\_\_\_\_\_\_\_\_\_\_\_\_\_\_\_\_\_\_\_\_\_\_\_\_\_\_\_\_\_\_\_\_\_\_\_\_\_\_\_\_\_\_\_\_\_\_\_\_\_\_\_\_\_\_\_\_\_\_\_\_\_\_\_\_\_\_\_\_\_\_\_\_\_\_\_\_\_\_\_\_\_\_\_\_\_\_\_\_\_\_\_\_\_\_\_\_\_\_\_\_\_\_\_\_\_\_\_\_\_\_\_\_\_\_\_\_\_\_\_\_\_\_\_\_\_\_\_\_\_\_\_\_\_\_\_\_\_\_\_\_\_\_\_\_\_\_\_\_\_\_\_\_\_\_\_\_\_\_\_\_\_\_\_\_\_\_\_\_\_\_\_\_\_\_\_\_\_\_\_\_\_\_\_\_\_\_\_\_\_\_\_\_\_\_\_\_\_\_\_\_\_\_\_\_\_\_\_\_\_\_\_\_\_\_\_\_\_\_\_\_\_\_\_\_\_\_\_\_\_\_\_\_\_\_\_\_\_\_\_\_\_\_\_\_\_\_\_\_\_\_\_\_\_\_\_\_\_\_\_\_\_\_\_\_\_\_\_\_\_\_\_\_\_\_\_\_\_\_\_\_\_\_\_\_\_\_\_\_\_\_\_\_\_\_\_\_\_\_\_\_\_\_\_\_\_\_\_\_\_\_\_\_\_\_\_\_\_\_\_\_\_\_\_\_\_\_\_\_\_\_\_\_\_\_\_\_\_\_\_\_\_\_\_\_\_\_\_\_\_\_\_\_\_\_\_\_\_\_\_\_\_\_\_\_\_\_\_\_\_\_\_\_\_\_\_\_\_\_\_\_\_\_\_\_\_\_\_\_\_\_\_\_\_\_\_\_\_\_\_\_\_\_\_\_\_\_\_\_\_\_\_\_\_\_\_\_\_\_\_\_\_\_\_\_\_\_\_\_\_\_\_\_\_\_\_\_\_\_\_\_\_\_\_\_\_\_\_\_\_\_\_\_\_\_\_\_\_\_\_\_\_\_\_\_\_\_\_\_\_\_\_\_\_\_\_\_\_\_\_\_\_\_\_\_\_\_\_\_\_\_\_\_\_\_\_\_\_\_\_\_\_\_\_\_\_\_\_\_\_\_\_\_\_\_\_\_\_\_\_\_\_\_\_\_\_\_\_\_\_\_\_\_\_\_\_\_\_\_\_\_\_\_\_\_\_\_\_\_\_\_\_\_\_\_\_\_\_\_\_\_\_\_\_\_\_\_\_\_\_\_\_\_\_\_\_\_\_\_\_\_\_\_\_\_\_\_\_\_\_\_\_\_\_\_\_\_\_\_\_\_\_\_\_\_\_\_\_\_\_\_\_\_\_\_\_\_\_\_\_\_\_\_\_\_\_\_\_\_\_\_\_\_\_\_\_\_\_\_\_\_\_\_\_\_\_\_\_\_\_\_\_\_\_\_\_\_\_\_\_\_\_\_\_\_\_\_\_\_\_\_\_\_\_\_\_\_\_\_\_\_\_\_\_\_\_\_\_\_\_\_\_\_\_\_\_\_\_\_\_\_\_\_\_\_\_\_\_\_\_\_\_\_\_\_\_\_\_\_\_\_\_\_\_\_\_\_\_\_\_\_\_\_\_\_\_\_\_\_\_\_\_\_\_\_\_\_\_\_\_\_\_\_\_\_\_\_\_\_\_\_\_\_\_\_\_\_\_\_\_\_\_\_\_\_\_\_\_\_\_\_\_\_\_\_\_\_\_\_\_\_\_\_\_\_\_\_\_\_\_\_\_\_\_\_\_\_\_\_\_\_\_\_\_\_\_\_\_\_\_\_\_\_\_\_\_\_\_\_\_\_\_\_\_\_\_\_\_\_\_\_\_\_\_\_\_\_\_\_\_\_\_\_\_\_\_\_\_\_\_\_\_\_\_\_\_\_\_\_\_\_\_\_\_\_\_\_\_\_\_\_\_\_\_\_\_\_\_\_\_\_\_\_\_\_\_\_\_\_\_\_\_\_\_\_\_\_\_\_\_\_\_\_\_\_\_\_\_\_\_\_\_\_\_\_\_\_\_\_\_\_\_\_\_\_\_\_\_\_\_\_\_\_\_\_\_\_\_\_\_\_\_\_\_\_\_\_\_\_\_\_\_\_\_\_\_\_\_\_\_\_\_\_\_\_\_\_\_\_\_\_\_\_\_\_\_\_\_\_\_\_\_\_\_\_\_\_\_\_\_\_\_\_\_\_\_\_\_\_\_\_\_\_\_\_\_\_\_\_\_\_\_\_\_\_\_\_\_\_\_\_\_\_\_\_\_\_\_\_\_\_\_\_\_\_\_\_\_\_\_\_\_\_\_\_\_\_\_\_\_\_\_\_\_\_\_\_\_\_\_\_\_\_\_\_\_\_\_\_\_\_\_\_\_\_\_\_\_\_\_\_\_\_\_\_\_\_\_\_\_\_\_\_\_\_\_\_\_\_\_\_\_\_\_\_\_\_\_\_\_\_\_\_\_\_\_\_\_\_\_\_\_\_\_\_\_\_\_\_\_\_\_\_\_\_\_\_\_\_\_\_\_\_\_\_\_\_\_\_\_\_\_\_\_\_\_\_\_\_\_\_\_\_\_\_\_\_\_\_\_\_\_\_\_\_\_\_\_\_\_\_\_\_\_\_\_\_\_\_\_\_\_\_\_\_\_\_\_\_\_\_\_\_\_\_\_\_\_\_\_\_\_\_\_\_\_\_\_\_\_\_\_\_\_\_\_\_\_\_\_\_\_\_\_\_\_\_\_\_\_\_\_\_\_\_\_\_\_\_\_\_\_\_\_\_\_\_\_\_\_\_\_\_\_\_\_\_\_\_\_\_\_\_\_\_\_\_\_\_\_\_\_\_\_\_\_\_\_\_\_\_\_\_\_\_\_\_\_\_\_\_\_\_\_\_\_\_\_\_\_\_\_\_\_\_\_\_\_\_\_\_\_\_\_\_\_\_\_\_\_\_\_\_\_\_\_\_\_\_\_\_\_\_\_\_\_\_\_\_\_\_\_\_\_\_\_\_\_\_\_\_\_\_\_\_\_\_\_\_\_\_\_\_\_\_\_\_\_\_\_\_\_\_\_\_\_\_\_\_\_\_\_\_\_\_\_\_\_\_\_\_\_\_\_\_\_\_\_\_\_\_\_\_\_\_\_\_\_\_\_\_\_\_\_\_\_\_\_\_\_\_\_\_\_\_\_\_\_\_\_\_\_\_\_\_\_\_\_\_\_\_\_\_\_\_\_\_\_\_\_\_\_\_\_\_\_\_\_\_\_\_\_\_\_\_\_\_\_\_\_\_\_\_\_\_\_\_\_\_\_\_\_\_\_\_\_\_\_\_\_\_\_\_\_\_\_\_\_\_\_\_\_\_\_\_\_\_\_\_\_\_\_\_\_\_\_\_\_\_\_\_\_\_\_\_\_\_\_\_\_\_\_\_\_\_\_\_\_\_\_\_\_\_\_\_\_\_\_\_\_\_\_\_\_\_\_\_\_\_\_\_\_\_\_\_\_\_\_\_\_\_\_\_\_\_\_\_\_\_\_\_\_\_\_\_\_\_\_\_\_\_\_\_\_\_\_\_\_\_\_\_\_\_\_\_\_\_\_\_\_\_\_\_\_\_\_\_\_\_\_\_\_\_\_\_\_\_\_\_\_\_\_\_\_\_\_\_\_\_\_\_\_\_\_\_\_\_\_\_\_\_\_\_\_\_\_\_\_\_\_\_\_\_\_\_\_\_\_\_\_\_\_\_\_\_\_\_\_\_\_\_\_\_\_\_\_\_\_\_\_\_\_\_\_\_\_\_\_\_\_\_\_\_\_\_\_\_\_\_\_\_\_\_\_\_\_\_\_\_\_\_\_\_\_\_\_\_\_\_\_\_\_\_\_\_\_\_\_\_\_\_\_\_\_\_\_\_\_\_\_\_\_\_\_\_\_\_\_\_\_\_\_\_\_\_\_\_\_\_\_\_\_\_\_\_\_\_\_\_\_\_\_\_\_\_\_\_\_\_\_\_\_\_\_\_\_\_\_\_\_\_\_\_\_\_\_\_\_\_\_\_\_\_\_\_\_\_\_\_\_\_\_\_\_\_\_\_\_\_\_\_\_\_\_\_\_\_\_\_\_\_\_\_\_\_\_\_\_\_\_\_\_\_\_\_\_\_\_\_\_\_\_\_\_\_\_\_\_\_\_\_\_\_\_\_\_\_\_\_\_\_\_\_\_\_\_\_\_\_\_\_\_\_\_\_\_\_\_\_\_\_\_\_\_\_\_\_\_\_\_\_\_\_\_\_\_\_\_\_\_\_\_\_\_\_\_\_\_\_\_\_\_\_\_\_\_\_\_\_\_\_\_\_\_\_\_\_\_\_\_\_\_\_\_\_\_\_\_\_\_\_\_\_\_\_\_\_\_\_\_\_\_\_\_\_\_\_\_\_\_\_\_\_\_\_\_\_\_\_\_\_\_\_\_\_\_\_\_\_\_\_\_\_\_\_\_\_\_\_\_\_\_\_\_\_\_\_\_\_\_\_\_\_\_\_\_\_\_\_\_\_\_\_\_\_\_\_\_\_\_\_\_\_\_\_\_\_\_\_\_\_\_\_\_\_\_\_\_\_\_\_\_\_\_\_\_\_\_\_\_\_\_\_\_\_\_\_\_\_\_\_\_\_\_\_\_\_\_\_\_\_\_\_\_\_\_\_\_\_\_\_\_\_\_\_\_\_\_\_\_\_\_\_\_\_\_\_\_\_\_\_\_\_\_\_\_\_\_\_\_\_\_\_\_\_\_\_\_\_\_\_\_\_\_\_\_\_\_\_\_\_\_\_\_\_\_\_\_\_\_\_\_\_\_\_\_\_\_\_\_\_\_\_\_\_\_\_\_\_\_\_\_\_\_\_\_\_\_\_\_\_\_\_\_\_\_\_\_\_\_\_\_\_\_\_\_\_\_\_\_\_\_\_\_\_\_\_\_\_\_\_\_\_\_\_\_\_\_\_\_\_\_\_\_\_\_\_\_\_\_\_\_\_\_\_\_\_\_\_\_\_\_\_\_\_\_\_\_\_\_\_\_\_\_\_\_\_\_\_\_\_\_\_\_\_\_\_\_\_\_\_\_\_\_\_\_\_\_\_\_\_\_\_\_\_\_\_\_\_\_\_\_\_\_\_\_\_\_\_\_\_\_\_\_\_\_\_\_\_\_\_\_\_\_\_\_\_\_\_\_\_\_\_\_\_\_\_\_\_\_\_\_\_\_\_\_\_\_\_\_\_\_\_\_\_\_\_\_\_\_\_\_\_\_\_\_\_\_\_\_\_\_\_\_\_\_\_\_\_\_\_\_\_\_\_\_\_\_\_\_\_\_\_\_\_\_\_\_\_\_\_\_\_\_\_\_\_\_\_\_\_\_\_\_\_\_\_\_\_\_\_\_\_\_\_\_\_\_\_\_\_\_\_\_\_\_\_\_\_\_\_\_\_\_\_\_\_\_\_\_\_\_\_\_\_\_\_\_\_\_\_\_\_\_\_\_\_\_\_\_\_\_\_\_\_\_\_\_\_\_\_\_\_\_\_\_\_\_\_\_\_\_\_\_\_\_\_\_\_\_\_\_\_\_\_\_\_\_\_\_\_\_\_\_\_\_\_\_\_\_\_\_\_\_\_\_\_\_\_\_\_\_\_\_\_\_\_\_\_\_\_\_\_\_\_\_\_\_\_\_\_\_\_\_\_\_\_\_\_\_\_\_\_\_\_\_\_\_\_\_\_\_\_\_\_\_\_\_\_\_\_\_\_\_\_\_\_\_\_\_\_\_\_\_\_\_\_\_\_\_\_\_\_\_\_\_\_\_\_\_\_\_\_\_\_\_\_\_\_\_\_\_\_\_\_\_\_\_\_\_\_\_\_\_\_\_\_\_\_\_\_\_\_\_\_\_\_\_\_\_\_\_\_\_\_\_\_\_\_\_\_\_\_\_\_\_\_\_\_\_\_\_\_\_\_\_\_\_\_\_\_\_\_\_\_\_\_\_\_\_\_\_\_\_\_\_\_\_\_\_\_\_\_\_\_\_\_\_\_\_\_\_\_\_\_\_\_\_\_\_\_\_\_\_\_\_\_\_\_\_\_\_\_\_\_\_\_\_\_\_\_\_\_\_\_\_\_\_\_\_\_\_\_\_\_\_\_\_\_\_\_\_\_\_\_\_\_\_\_\_\_\_\_\_\_\_\_\_\_\_\_\_\_\_\_\_\_\_\_\_\_\_\_\_\_\_\_\_\_\_\_\_\_\_\_\_\_\_\_\_\_\_\_\_\_\_\_\_\_\_\_\_\_\_\_\_\_\_\_\_\_\_\_\_\_\_\_\_\_\_\_\_\_\_\_\_\_\_\_\_\_\_\_\_\_\_\_\_\_\_\_\_\_\_\_\_\_\_\_\_\_\_\_\_\_\_\_\_\_\_\_\_\_\_\_\_\_\_\_\_\_\_\_\_\_\_\_\_\_\_\_\_\_\_\_\_\_\_\_\_\_\_\_\_\_\_\_\_\_\_\_\_\_\_\_\_\_\_\_\_\_\_\_\_\_\_\_\_\_\_\_\_\_\_\_\_\_\_\_\_\_\_\_\_\_\_\_\_\_\_\_\_\_\_\_\_\_\_\_\_\_\_\_\_\_\_\_\_\_\_\_\_\_\_\_\_\_\_\_\_\_\_\_\_\_\_\_\_\_\_\_\_\_\_\_\_\_\_\_\_\_\_\_\_\_\_\_\_\_\_\_\_\_\_\_\_\_\_\_\_\_\_\_\_\_\_\_\_\_\_\_\_\_\_\_\_\_\_\_\_\_\_\_\_\_\_\_\_\_\_\_\_\_\_\_\_\_\_\_\_\_\_\_\_\_\_\_\_\_\_\_\_\_\_\_\_\_\_\_\_\_\_\_\_\_\_\_\_\_\_\_\_\_\_\_\_\_\_\_\_\_\_\_\_\_\_\_\_\_\_\_\_\_\_\_\_\_\_\_\_\_\_\_\_\_\_\_\_\_\_\_\_\_\_\_\_\_\_\_\_\_\_\_\_\_\_\_\_\_\_\_\_\_\_\_\_\_\_\_\_\_\_\_\_\_\_\_\_\_\_\_\_\_\_\_\_\_\_\_\_\_\_\_\_\_\_\_\_\_\_\_\_\_\_\_\_\_\_\_\_\_\_\_\_\

\section*{目录}

\begin{itemize}
    \item[1.] 问题重述 \dotfill 4
    \item[2.] 模型准备 \dotfill 4
        \begin{itemize}
            \item[2.1] 基本假设 \dotfill 4
            \item[2.2] 符号说明 \dotfill 5
        \end{itemize}
    \item[3.] 问题分析 \dotfill 5
    \item[4.] 单交叉路口节点能量模型建立及求解 \dotfill 6
        \begin{itemize}
            \item[4.1] 节点能量模型建立 \dotfill 6
                \begin{itemize}
                    \item[4.1.1] 节点能量的定义与物理意义 \dotfill 7
                    \item[4.1.2] 模型建立 \dotfill 7
                \end{itemize}
            \item[4.2] 模型实时优化算法 \dotfill 12
                \begin{itemize}
                    \item[4.2.1] 遗传算法 \dotfill 12
                    \item[4.2.2] 模拟退火算法 \dotfill 13
                \end{itemize}
            \item[4.3] 模型求解及信号配时方案 \dotfill 14
                \begin{itemize}
                    \item[4.3.1] Poisson 数据流生成 \dotfill 15
                    \item[4.3.2] 黄灯时长优化 \dotfill 15
                    \item[4.3.3] 最优信号配时方案 \dotfill 16
                \end{itemize}
            \item[4.4] 模型分析与评价 \dotfill 17
                \begin{itemize}
                    \item[4.4.1] 与固定配时方案的比较 \dotfill 17
                    \item[4.4.2] 模型分析 \dotfill 21
                \end{itemize}
            \item[4.5] 模型的进一步讨论和求解 \dotfill 22
        \end{itemize}
    \item[5.] 线状交叉路口耦合能量模型建立及求解 \dotfill 23
        \begin{itemize}
            \item[5.1] 耦合能量模型建立 \dotfill 23
                \begin{itemize}
                    \item[5.1.1] 耦合的引入与物理意义 \dotfill 23
                    \item[5.1.2] 耦合能量计算 \dotfill 24
                    \item[5.1.3] 衰减因子参数的确定 \dotfill 25
                \end{itemize}
            \item[5.2] 模型实时优化算法 \dotfill 26
            \item[5.3] 模型求解及信号配时方案 \dotfill 27
            \item[5.4] 模型分析与评价 \dotfill 27
                \begin{itemize}
                    \item[5.4.1] 与固定配时的比较 \dotfill 27
                    \item[5.4.2] 模型分析 \dotfill 28
                \end{itemize}
        \end{itemize}
    \item[6.] 网状多交叉路口网格能量模型及求解 \dotfill 29
        \begin{itemize}
            \item[6.1] 网格能量模型建立 \dotfill 30
                \begin{itemize}
                    \item[6.1.1] 网格能量的引入 \dotfill 30
                    \item[6.1.2] 网格能量定义与计算 \dotfill 30
                    \item[6.1.3] 网格能量模型建立 \dotfill 31
                \end{itemize}
            \item[6.2] 模型实时优化算法 \dotfill 31
            \item[6.3] 模型求解及信号配时方案 \dotfill 32
            \item[6.4] 模型分析与评价 \dotfill 32
                \begin{itemize}
                    \item[6.4.1] 与固定配时的比较 \dotfill 32
                    \item[6.4.2] 模型分析 \dotfill 32
                \end{itemize}
        \end{itemize}
    \item[7.] 模型实时优化算法比较与分析 \dotfill 34
        \begin{itemize}
            \item[7.1] 单节点模型优化算法分析比较 \dotfill 34
        \end{itemize}
\end{itemize}

\begin{itemize}
    \item[7.1.1] 遗传算法分析 \dotfill 34
    \item[7.1.2] 模拟退火算法 \dotfill 34
    \item[7.1.3] 两种算法在求解单交叉口优化问题中的结果比较 \dotfill 35
    \item[7.2] 线状节点模型优化算法可计算性和复杂性分析 \dotfill 36
    \item[7.3] 网状节点模型优化算法可计算性和复杂性分析 \dotfill 36
    \item[8.] 模型应用建议 \dotfill 37
    \item 参考文献 \dotfill 37
    \item 附录 \dotfill 38
\end{itemize}

\section{问题重述}

为提高城市道路服务功能,需要根据不同交叉路口、不同相位、不同方向、不同时段的交通流量(标准车辆数),以及信号灯的转换与车辆的起动的损失时间,合理的配置各路口的信号灯的周期长,以及同一周期内的红、绿、黄信号的响应时间,使全体车辆在所有道口的等待的时间最短。本题要求分别设计单个交叉路口,线状交叉路口,网状多交叉路口的实时配置方案,具体需要解决以下问题:

\begin{enumerate}
    \item 构造单个交叉路口(十字路口或丁字路口)交通信号实时控制的点控制数学模型,并给出相应的实时算法。为检验此模型和算法,根据城市交通流分布规律(一般理解为 Poisson 分布),设计一种实时产生交通流序列的方案,根据设计的算法和产生的交通流数据,计算并给出单交叉路口点控制的实时信号配时方案(分为周期固定和周期不固定两种情形考虑),并与固定配时方案比较,说明实时配时方案的效果和优势。
    \item 构造多个交叉路口(线状区域)交通信号实时控制的线控制(至少 2 个交叉口)数学模型,并给出相应的实时算法和具体配置方案,同时分析模型算法的可计算性、算法的复杂性。
    \item 构造多个交叉路口(网络区域)交通信号实时控制的面控制(至少 5 个交叉口)数学模型,并设计相应的实时算法和具体配置方案,同时分析模型算法的可计算性、算法的复杂性。
    \item 给交通管理部门提出应用你所得结果的咨询和建议(例如:流量预测方法、数据处理方法,软件设计、实现步骤等)。
\end{enumerate}

\section{模型准备}

\subsection{基本假设}

\begin{enumerate}
    \item 不考虑行人和交通事故造成的堵车现象所带来的延误时间。
    \item 车辆在相邻两个交叉路口之间的公路上行驶时,如果未遇到红绿灯,均以匀速行驶且不停车。
    \item 右转的车辆可以实时通过,并且在右转后不影响直行车辆。
    \item 在绿灯时期,每个车道等待的车辆首先以饱和流量流出,直至此路口的堵塞消失。
    \item 假设所有十字路口的相位顺序采用题目中所给的次序,即不对相位的出现顺序进行优化。
    \item 任意两个相邻交叉路口之间的公路的距离大于交叉路口前等待的车队列长度。
\end{enumerate}

\subsection{2.2 符号说明}

\begin{itemize}
    \item $E$: 交叉路口节点能量
    \item $E_{i}^{in}$: 第 $i$ 相位流入节点的能量 ($i=1,2,3,4$)
    \item $E_{i}^{out}$: 第 $i$ 相位流出节点的能量 ($i=1,2,3,4$)
    \item $Q(t)$: $t$ 时刻位于交叉路口的车辆数量
    \item $F$: 车流到达速率矢量
    \item $f_{j}=F(j)$: 第 $j$ 车道车辆到达速率 ($j=1,2,\dots,8$)
    \item $R$: 到达车辆的数目矢量
    \item $r_{j}=R(j)$: 第 $j$ 个车道的到达车辆数目 ($j=1,2,\dots,8$)
    \item $S$: 车道的饱和流量矢量
    \item $s_{j}=S(j)$: 第 $j$ 个车道的饱和流量 ($j=1,2,\dots,8$)
    \item $N_{i}$: 第 $i$ 个相位结束时节点的车辆数目 ($i=1,2,3,4$)
    \item $t_{i}$: 第 $i$ 个相位的绿灯停止时间 ($i=1,2,3,4$)
    \item $\Delta t_{y}$: 黄灯的持续时间
    \item $\Delta t_{s}$: 启动延误时间
    \item $\Delta t_{d}$: 黄灯期间损失时间
    \item $T$: 周期时长, $T=t_{4}+\Delta t_{y}$
    \item $E_{AB}$: 节点 A 和节点 B 的节点耦合能量
    \item $q_{AB}$: 在一个周期内从节点 A 和节点 B 之间流过的车辆个数
    \item $E_{ABD}$: 三角形 ABD 的网格耦合能量
\end{itemize}

\section{3. 问题分析}

题目要求对交叉路口建立控制数学模型。车流在交叉路口的进入和流出是一个及其离散和不确定的过程, 这意味着对车流的局部描述是无法达到很好的效果

的。交叉路口节点也可看作一个系统,这个系统具有输入、输出和积累,分别对应进入节点的车流、离开节点的车流和原来积累的车流。对这个系统建立模型,实际上是要对这个系统的本质行为进行一个评价和模式化,而这个评价正可以作为指标来对节点的行为做出调整。

交叉路口节点具有对车流进行调配、控制的属性,从外部来看,即具有吸收和释放车辆的能力。进入系统的车流对系统的某种性质是一个不断积累的过程,而离开的车流是一个释放的过程,因此交叉路口节点的行为可以通过其输入、输出和积累的量来衡量,即车流的量来衡量。另一方面车辆在交叉路口节点处的等待时间也是描述节点行为和属性的一个重要因素,因为对不等待的车辆来说,交叉路口和直行路没有本质区别。正是在交叉路口等待的车辆和其等待的时间深刻反应了交叉路口的本质。因此,对交叉路口的建模将紧紧围绕交叉路口处车流量和车辆的等待时间来进行。

问题 1 要求对单交叉路口控制建模。单交叉路口建模的复杂性在于车流的不确定性和信号控制的突变性。这就要求必须从交叉路口和信号控制的整体性上解决问题,这样就可以从系统层面描述车流的输入、输出和滞留。这个模型是一个强非线性模型,因此,一般的代数优化方法无法求解次模型,因此可以考虑非线性优化算法如遗传算法、模拟退火算法、神经网络算法等等。

问题 2 与问题 1 的联系在于线状多交叉路口仍是由单交叉路口节点构成的,因此单交叉路口的模型仍可以部分沿用至多交叉路口。线状多交叉路口与单交叉路口的本质区别在于线状交叉路口的各个相邻单节点之间存在耦合关系,他们不是独立的节点。其求解算法和单节点的算法并无本质区别,都是非线性优化,不同的是优化的变量个数和目标不同,因此,优化算法可以沿用单节点模型的。

问题 3 中网状多交叉路口的模型的不同之处除了相邻节点之间的耦合关系外,还存在网络中的耦合,即不相邻节点也有可能存在耦合关系,因此需要解决这种网状耦合关系。求解算法与前面的算法无大异,只是随着节点数目的增大,其可解性未必能保证,需要进一步讨论。

根据模型和优化算法即可得到问题 6 中的预测和建议。

\section*{4. 单交叉路口节点能量模型建立及求解}

单交叉路口节点是城市交通信号控制的基本单元,在城市交通信号智能控制中起着基础和核心的作用,但由于节点处车流的强非线性和难描述性,对单交叉路口节点建立模型是一个非常复杂的问题。黄崇超[1]等提出了一种基于车辆等待总延误的优化模型,计算了一个周期内所有车辆等待的总时间。本文引入节点能量来描述单交叉路口节点总延误。

\subsection*{4.1 节点能量模型建立}

本文将从能量模型的引入、能量的物理意义、利用能量建立模型等几个方面讨论节点能量模型。

\subsection{4.1.1 节点能量的定义与物理意义}

车流在交叉路口的进入和流出是一个及其离散和不确定的过程,这意味着对车流的局部描述是无法达到很好的效果的。另一方面,本文将交叉路口节点看作一个系统,这个系统具有输入、输出和积累,分别对应进入节点的车流、离开节点的车流和原来积累的车流。对这个系统来说,所有的输入、输出和积累从侧面反应了系统的行为。对这个系统建立模型,实际上是要对这个系统的本质行为进行一个评价和模式化,而这个评价正可以作为指标来对节点的行为做出调整。

进入系统的车流对系统的某种性质是一个不断积累的过程,而离开的车流是一个释放的过程,本文可以将这个节点看作一个对称的弹性体(不一定是线性弹性体),流入的车辆不断“挤压”这个弹性体,而流出的车辆不断疏松和释放这种挤压效果。在一定的强度和范围内,对弹性体的挤压增加了弹性体反弹的趋势,而对弹性体的释放和疏松减少了弹性体反弹的趋势。因此,类比于物理学中的定义,本文定义这种节点反弹的趋势为节点能量。这个节点能量首先与节点中车辆的数量有关,车辆的数目越多,对弹性体的挤压就越强烈,其反弹的趋势也就越强烈;其次,这个节点能量与车辆的等待时间有关系,车辆等待越久,对弹性体的挤压效果越久,由于积累的作用,弹性体反弹的趋势也越强。定义 1 给出了节点能量的严格定义。

\textbf{定义 1:交叉路口节点能量}

车流对交叉路口节点的积累作用称为交叉路口节点能量,表达为

\begin{equation}
E = \int_{0}^{t} Q(\tau) d\tau
\tag{4-1}
\end{equation}

其中,$E$ 表示交叉路口节点在 $[0, t]$ 区间内的能量,$Q$ 表示节点的车辆数量。

有了节点能量定义,本文就可以对交叉路口节点处车辆等待延时等行为进行描述。下面,本文讨论节点能量的物理含义。

从零时刻起,在 $[0, \Delta t]$ 内,进入交叉路口的车辆数为 $Q(\Delta t)$,则这些车辆的的等待时间为 $Q(\Delta t)\Delta t$,则在 $[0, t]$ 内,所有车辆在交叉路口处的等待的总时间为 $\int_{0}^{t} Q(\tau) d\tau$,即本文定义的节点能量。

评价交叉路口信号控制能力的一个重要指标是全体车辆在所有道口的等待时间,等待时间越短,交叉路口信号控制能力越优。因此,在本文的问题中,所有车辆在节点的等待时间转化交叉交叉路口节点能量,使全体车辆在所有道口的等待时间最短转化为交叉路口节点能量最小。下面本文基于能量的概念建立单节点交叉路口数学模型。

\subsection{4.1.2 模型建立}

本文以十字路口为建模对象,丁字路口的模型可类似得出,并且不对交叉路口相位次序进行讨论,即严格遵守题目中给出的十字交叉路口相位设计。

设 \( i=1,2,3,4 \),\( \Delta t_{y} \) 表示黄灯时长,\( t_{i} \) 表示第 \( i \) 个相位绿灯结束时刻,则第 \( i \) 个相位结束时刻为 \( t_{i}+\Delta t_{y} \);\( \Delta t_{s} \) 表示车辆起动延误,\( \Delta t_{d} \) 表示黄灯损失时间;这些时间关系如图 4-1 所示。

\begin{figure}[h]
\centering
\includegraphics[width=\textwidth]{image.png}
\caption{红绿灯时间关系}
\end{figure}

设 \( E_{i} \) 表示第 \( i \) 个相位结束时节点的能量,\( E_{i}^{in} \) 表示第 \( i \) 个相位流入节点的能量,\( E_{i}^{out} \) 表示第 \( i \) 个相位流出的能量,则

\begin{equation}
E_{i}=E_{i}^{in}-E_{i}^{out}+E_{i-1}
\tag{4-2}
\end{equation}

其中 \( i=1,2,3,4 \)。

由于假设右转车流不受信号灯的限制,因此右转车流对节点的能量无影响,因此,这里考虑每个路口有左转和直行两条道路。十字路口的车道编号如图 4-2。

\begin{figure}[h]
    \centering
    \includegraphics[width=0.8\textwidth]{image.png}
    \caption{十字路口车道编号}
    \label{fig:crossroad_lanes}
\end{figure}

设 $8 \times 1$ 向量 $L_{i}$ 为红绿灯状态向量,表示第 $i$ 个相位的红绿灯状态,即
\[
L_{i} = \begin{bmatrix} l_{1} & l_{2} & l_{3} & l_{4} & l_{5} & l_{6} & l_{7} & l_{8} \end{bmatrix}^{T},
\]
$l_{j}$ 表示第 $j$ 个车道的红绿灯状态,其取值如式 (4-3):
\begin{equation}
l_{j} = 
\begin{cases} 
1 & \text{绿灯} \\
0 & \text{红灯}
\end{cases} \quad j = 1, 2, \dots, 8
\tag{4-3}
\end{equation}

$F = \begin{bmatrix} f_{1} & f_{2} & f_{3} & f_{4} & f_{5} & f_{6} & f_{7} & f_{8} \end{bmatrix}^{T}$ 表示车流到达速率,$f_{j}$ ($j = 1, 2, \dots, 8$) 表示第 $j$ 个车道的车流到达率;$R = \begin{bmatrix} r_{1} & r_{2} & r_{3} & r_{4} & r_{5} & r_{6} & r_{7} & r_{8} \end{bmatrix}^{T}$ 表示到达车辆的数目,$r_{j}$ ($j = 1, 2, \dots, 8$) 表示第 $j$ 个车道的到达车辆数目;$S = \begin{bmatrix} s_{1} & s_{2} & s_{3} & s_{4} & s_{5} & s_{6} & s_{7} & s_{8} \end{bmatrix}^{T}$ 表示车道的饱和流量,$s_{j}$ ($j = 1, 2, \dots, 8$) 表示第 $j$ 个车道的饱和流量;$N_{i}$ ($i = 1, 2, 3, 4$) 表示第 $i$ 个相位结束时节点的车辆数目;$I = \begin{bmatrix} 1 & 1 & 1 & 1 & 1 & 1 & 1 & 1 \end{bmatrix}^{T}$ 为一个辅助单位向量。

下面计算节点一个周期的能量。

\textbf{第一相位:}

第一相位的红绿灯状态向量为 $L_{1} = \begin{bmatrix} 1 & 0 & 0 & 0 & 1 & 0 & 0 & 0 \end{bmatrix}^{T}$。

\textbf{第一相位时到达的车辆数目 $R_{1}$:}

\begin{equation}
R_{1} = \int_{0}^{t_{1} + \Delta t_{y}} F(t) dt
\tag{4-4}
\end{equation}

则流入节点的能量为:

\begin{equation}
E_{1}^{in} = \int_{0}^{t_{1} + \Delta t_{y}} \langle R_{1}, I \rangle dt
\tag{4-5}
\end{equation}

流出节点的能量:

\begin{equation}
E_{1}^{out} = \int_{\Delta t_{s}}^{t_{1} + \Delta t_{y} - \Delta t_{d}} t < S, L_{1} > dt
\tag{4-6}
\end{equation}

假设节点初始能量 \( E_{0} = 0 \),如果 \( E_{0} \) 不为 0,则只是给节点能量加上一个常数,不影响模型的本质和正确性。则

\begin{align}
E_{1} &= E_{1}^{in} - E_{1}^{out} \\
&= \int_{0}^{t_{1} + \Delta t_{y}} < R, I > dt - \int_{\Delta t_{s}}^{t_{1} + \Delta t_{y} - \Delta t_{d}} t < S, L_{1} > dt
\tag{4-7}
\end{align}

第一相位结束时节点中车的数量 \( N_{1} \):

\begin{align}
N_{1} &= < R_{1}, I > - \int_{\Delta t_{s}}^{t_{1} + \Delta t_{y} - \Delta t_{d}} < S, L_{1} > dt \\
&= \int_{0}^{t_{1} + \Delta t_{y}} < F(t), I > dt - \int_{\Delta t_{s}}^{t_{1} + \Delta t_{y} - \Delta t_{d}} < S, L_{1} > dt
\tag{4-8}
\end{align}

第二相位:

第二相位的红绿灯状态向量为 \( L_{2} = [0 \quad 1 \quad 0 \quad 0 \quad 0 \quad 1 \quad 0 \quad 0]^{T} \)。第二相位时到达的车辆数目 \( R_{2} \):

\begin{equation}
R_{2} = \int_{t_{1} + \Delta t_{y}}^{t_{2} + \Delta t_{y}} F(t) dt
\tag{4-9}
\end{equation}

则进入节点的能量为:

\begin{equation}
E_{2}^{in} = \int_{t_{1} + \Delta t_{y}}^{t_{2} + \Delta t_{y}} (< R_{2}, I > + N_{1}) dt
\tag{4-10}
\end{equation}

流出节点的能量:

\begin{equation}
E_{2}^{out} = \int_{t_{1} + \Delta t_{y} + \Delta t_{s}}^{t_{2} + \Delta t_{y} - \Delta t_{d}} t < S, L_{2} > dt
\tag{4-11}
\end{equation}

则第二相位结束时节点中的能量

\begin{align}
E_{2} &= E_{2}^{in} - E_{2}^{out} + E_{1} \\
&= \int_{t_{1} + \Delta t_{y}}^{t_{2} + \Delta t_{y}} (< R_{2}, I > + N_{1}) dt - \int_{t_{1} + \Delta t_{y} + \Delta t_{s}}^{t_{2} + \Delta t_{y} - \Delta t_{d}} t < S, L_{2} > dt + E_{1}
\tag{4-12}
\end{align}

第二相位结束时节点中车的数量 \( N_{2} \):

\begin{align}
N_{2} &= < R_{2}, I > - \int_{t_{1} + \Delta t_{y} + \Delta t_{s}}^{t_{2} + \Delta t_{y} - \Delta t_{d}} < S, L_{2} > dt + N_{1} \\
&= \int_{t_{1} + \Delta t_{y}}^{t_{2} + \Delta t_{y}} < F(t), I > dt - \int_{t_{1} + \Delta t_{y} + \Delta t_{s}}^{t_{2} + \Delta t_{y} - \Delta t_{d}} < S, L_{2} > dt + N_{1}
\tag{4-13}
\end{align}

第三相位:

第三相位的红绿灯状态向量为 $L_{3}=[0 \quad 0 \quad 1 \quad 0 \quad 0 \quad 1 \quad 0]^{T}$。第三相位时到达的车辆数目 $R_{3}$:

\begin{equation}
R_{3}=\int_{t_{2}+\Delta t_{y}}^{t_{3}+\Delta t_{y}} F(t) dt
\tag{4-14}
\end{equation}

则进入节点的能量为:

\begin{equation}
E_{3}^{in}=\int_{t_{2}+\Delta t_{y}}^{t_{3}+\Delta t_{y}}\left(<R_{3}, I>+N_{2}\right) dt
\tag{4-15}
\end{equation}

流出节点的能量:

\begin{equation}
E_{3}^{out}=\int_{t_{2}+\Delta t_{y}+\Delta t_{s}}^{t_{3}+\Delta t_{y}-\Delta t_{d}} t<S, L_{3}>dt
\tag{4-16}
\end{equation}

第三相位结束时节点的能量

\begin{equation}
\begin{aligned}
E_{3} & =E_{3}^{in}-E_{3}^{out}+E_{2} \\
& =\int_{t_{2}+\Delta t_{y}}^{t_{3}+\Delta t_{y}}\left(<R_{3}, I>+N_{2}\right) dt-\int_{t_{2}+\Delta t_{y}+\Delta t_{s}}^{t_{3}+\Delta t_{y}-\Delta t_{d}} t<S, L_{3}>dt+E_{2}
\end{aligned}
\tag{4-17}
\end{equation}

第三相位结束时节点中车的数量 $N_{3}$:

\begin{equation}
\begin{aligned}
N_{3} & =<R_{3}, I>-\int_{t_{2}+\Delta t_{y}+\Delta t_{s}}^{t_{3}+\Delta t_{y}-\Delta t_{d}}<S, L_{3}>dt+N_{2} \\
& =\int_{t_{2}+\Delta t_{y}}^{t_{3}+\Delta t_{y}}<F(t), I>dt-\int_{t_{2}+\Delta t_{y}+\Delta t_{s}}^{t_{3}+\Delta t_{y}-\Delta t_{d}}<S, L_{3}>dt+N_{2}
\end{aligned}
\tag{4-18}
\end{equation}

第四相位:

第四相位的红绿灯状态向量为 $L_{4}=[0 \quad 0 \quad 0 \quad 1 \quad 0 \quad 0 \quad 1]^{T}$。第四相位时到达的车辆数目 $R_{4}$:

\begin{equation}
R_{4}=\int_{t_{3}+\Delta t_{y}}^{t_{4}+\Delta t_{y}} F(t) dt
\tag{4-19}
\end{equation}

则进入节点的能量为:

\begin{equation}
E_{4}^{in}=\int_{t_{3}+\Delta t_{y}}^{t_{4}+\Delta t_{y}}\left(<R_{4}, I>+N_{3}\right) dt
\tag{4-20}
\end{equation}

流出节点的能量:

\begin{equation}
E_{4}^{out}=\int_{t_{3}+\Delta t_{y}+\Delta t_{s}}^{t_{4}+\Delta t_{y}-\Delta t_{d}} t<S, L_{4}>dt
\tag{4-21}
\end{equation}

则第四相位结束时节点中的能量

\begin{equation}
\begin{aligned}
E_{4} & =E_{4}^{in}-E_{4}^{out}+E_{3} \\
& =\int_{t_{3}+\Delta t_{y}}^{t_{4}+\Delta t_{y}}\left(<R_{4}, I>+N_{3}\right) dt-\int_{t_{3}+\Delta t_{y}+\Delta t_{s}}^{t_{4}+\Delta t_{y}-\Delta t_{d}} t<S, L_{4}>dt+E_{3}
\end{aligned}
\tag{4-22}
\end{equation}

第四相位结束时节点中车的数量 $N_{4}$ :

\begin{equation}
\begin{aligned}
N_{4} & =<R_{4}, I>-\int_{t_{3}+\Delta t_{y}+\Delta t_{s}}^{t_{4}+\Delta t_{y}-\Delta t_{d}}<S, L_{4}>dt+N_{3} \\
& =\int_{t_{3}+\Delta t_{y}}^{t_{4}+\Delta t_{y}}<F(t), I>dt-\int_{t_{3}+\Delta t_{y}+\Delta t_{s}}^{t_{4}+\Delta t_{y}-\Delta t_{d}}<S, L_{4}>dt+N_{3}
\end{aligned}
\tag{4-23}
\end{equation}

$E_{4}$ 就是交叉路口节点在一个周期内的总能量,即从所有车辆的总延时方面描述了交叉路口节点在一个周期内的行为。这就是单交叉路口节点能量模型。使全体车辆在所有道口的等待时间最短就转化为使节点能量最小。为了求解优化模型,将节点能量转化为单位时间内的能量,并用 $E(t_{1}, t_{2}, t_{3}, t_{4})$ 表示 $E_{4}$,即基于能量的单节点交叉路口优化模型为:

\begin{equation}
\begin{cases}
\min E(t_{1}, t_{2}, t_{3}, t_{4})/T \\
s.t \quad T=t_{4}+\Delta t_{y}
\end{cases}
\tag{4-24}
\end{equation}

上式就是节点能量优化模型。下文将利用优化算法求解模型以使节点单位时间内能量最小从而达到全体车辆在所有道口等待时间最短的目标。

### 4.2 模型实时优化算法

求解非线性目标规化 $\begin{cases}\min E(t_{1}, t_{2}, t_{3}, t_{4})/T, \\ s.t \quad T=t_{4}+\Delta t_{y}\end{cases}$,如果采用传统的经典算法,在有限的时间内,很难计算出来,因此,必须对算法降阶,因此,我们选择了遗传算法和模拟退火算法。

#### 4.2.1 遗传算法

本文对非线性目标规化方程进行了离散化处理,取 $\Delta t$ 为 1 秒。这在实际中,也是有意义的,现存的计时器都是精确到秒的。离散化后,问题的求解有了很大的简化。

(1) 相关系数的选择

\begin{table}[h]
\centering
\caption{相关系数}
\begin{tabular}{|c|c|}
\hline
交叉概率 & 0.3 \\
\hline
变异概率 & 0.2 \\
\hline
初始种群 & 100 \\
\hline
遗传代数 & 30 \\
\hline
\end{tabular}
\end{table}

(2) 个体的编码方法

把每一种配时方案看作是一个染色体,则染色体分为 4 段,分别对应每种配时方案中各个相位的开始时记刻 $t_{1}$,$t_{2}$,$t_{3}$,$t_{4}$。

(3) 处理约束条件

约束 1. 相位开始时刻的约束转化为: $t_{1}<t_{2}, \ t_{2}<t_{3}, \ t_{3}<t_{4}$

约束 2. 周期的范围约束转化为: $T_{\min }<t_{4}<T_{\max }$

约束 3. 各相位的持续时间约束约化为: $T1_{\min }<t_{1}<T1_{\max }$, $T2_{\min }<t_{2}-t_{1}<T2_{\max }$, $T3_{\min }<t_{3}-t_{2}<T3_{\max }$, $T4_{\min }<t_{4}-t_{3}<T4_{\max }$。

约束 4. 若周期固定为常数 $c$,转化为: $t_{4}=c$

(4) 交叉操作的定义

染色体 $a$ 和染色体 $b$ 进行交叉,即对染色体的四个信息段进行融合。记 $Cti$ 为染色体 $C$ 的第 $i$ 个染色体,则遗传操作 $c=a \otimes b$ 表示。

$Cti=ati*(1-r)+r*bti \quad (i=1,2,3,4)$

(5) 变异的定义

染色体 $C$ 的变异操作 $\sim C$ 定义为,$Cti=\text{RandU}(Ti_{\min }, Ti_{\max })$,即 $C$ 的某个染色体在其定义域内随机取值。

(6) 适值函数

选取要优化为的目标函数为适值函数,即适值函数 $fit$ 定义为:

$fit(t_{1}, t_{2}, t_{3}, t_{4})=\frac{E(t_{1}, t_{2}, t_{3}, t_{4})}{T}$

下面给出了遗传算法。

\textbf{遗传算法:}

\textbf{Step1:} 产生模型为 100 的种群,初始化表 1 中各项参数,置当标记 $i=0$

\textbf{Step2:} 若 $i>30$,则转 Step7,否则,转 Step3

\textbf{Step3:} 计算种群中各个染色体的适值函数值,并用它们的值初始化一个轮赌盘,适值函数越小的染色体在轮赌盘上占有的扇区越小。

\textbf{Step4:} 对种群中的个体进行杂交,子代的双亲节点由轮赌盘上的节点随机产生,随机值满足从 0 到轮赌盘量程 $S$ 间的均匀分布,使适值函数小的值具有更高的被选择到的概率。

\textbf{Step5:} 对种群中的个体以变异概率进行突变,从而改变其某个染色体的值。

\textbf{Step6:} 对种群交叉和变异后的种群进行更新,转 Step2。

\textbf{Step7:} 对种群中所有存活个体进行遍历,适值函数最小值作为结果返回。

\subsection*{4.2.2 模拟退火算法}

同 4.2.1 节相比,本节将用模拟退火算法求解节点能量优化模型。

(1) 状态空间的定义

目标对象的状态空间为四维空间 $(t_{1}, t_{2}, t_{3}, t_{4})$ 的子集,其中,子空间满足限定条件,$t_{1}<t_{2}$,$t_{2}<t_{3}$,$t_{3}<t_{4}$,以及,$T1_{\min }<t_{1}<T1_{\max }$,$T2_{\min }<t_{2}-t_{1}<T2_{\max }$,$T3_{\min }<t_{3}-t_{2}<T3_{\max }$,$T4_{\min }<t_{4}-t_{3}<T4_{\max }$。

状态空间的移动定义:目标对象 $(t1, t2, t3, t4)$,可以在状态空间中沿任意方向,任意路径移动,这样保证了,不会把结果局限于目标对象的局部最优结果,从而求出全局最优解。

(2) 低温状态和高温状态的定义

利用要优化的目标函数 \( fit(t_1, t_2, t_3, t_4) = \frac{E(t_1, t_2, t_3, t_4)}{T} \),对对象的温度状态进行定义。其中,定义函数值小的对象处于低温状态,函数值大的对象处于高温状态。并通过 Metropolis 准则进行移动。

\begin{figure}[h]
    \centering
    \includegraphics[width=0.8\textwidth]{image.png}
    \caption{模拟退火算法流程图}
    \label{fig:4-3}
\end{figure}

(3) 算法中相应参数

本文对模拟退火算法的相应参数定义见下表 4-2:

\begin{table}[h]
    \centering
    \caption{参数列表}
    \label{tab:4-2}
    \begin{tabular}{|c|c|}
        \hline
        最低温度 & 2 摄氏度 \\
        \hline
        最高温度 & 100 摄氏度 \\
        \hline
        温度间隔 & 1 摄氏度 \\
        \hline
        最平衡持续周期 & 500 \\
        \hline
    \end{tabular}
\end{table}

图 4-3 给出了模拟退火算法的流程图。

\section*{4.3 模型求解及信号配时方案}

上文已经建立了节点的数学模型并确定了求解算法,下文将利用这个模型和

优化算法求解模型并得出最优信号配时方案,并利用优化结果对优化算法进行比较分析。

\subsection*{4.3.1 Poisson 数据流生成}

为了求解模型和给出最优信号配时方案,必须有数据基础。依题意,交通流分布规律一般为服从 Possion 分布,我们利用算法生成一组 Possion 流。

Possion 概率分布为:
\begin{equation}
P[N(t) = K] = \frac{(\lambda t)^K}{K!} e^{-\lambda t}, \quad K = 0, 1, 2, \ldots
\tag{4-25}
\end{equation}

本文对文献 [2] 中的 Possion 数据生成算法进行了改进,以使其更适合于道路交通流的模拟。

设车道的饱和流量为 \( s \),则到达率(单位:车/秒)应小于 \( s \),一般要求到达率小于 \( s/2 \),否则该车流不符合泊松分布。算法 1 是改进的期望为 \( \lambda \) 的 Possion 流生成算法,其中 \( RandU(0, 1) \) 是生成 0-1 均匀分布序列的函数,\( temp \) 是一个辅助变量。

\textbf{算法:}
\begin{itemize}
    \item \textbf{Step 1:} \( K = 0 \), \( temp = 1 \)
    \item \textbf{Step 2:} 产生服从 0-1 均匀分布的随机数: \( u = RandU(0, 1) \)
    \item \textbf{Step 3:} \( temp = temp * u \)
    \item \textbf{Step 4:} \( K = K + 1 \)
    \item \textbf{Step 5:} 转到 Step 2 计算,直到 \( temp < e^{-\lambda} \)
    \item \textbf{Step 6:} 若 \( K > s \),则转 Step 1
    \item \textbf{Step 7:} 返回 \( K - 1 \)
\end{itemize}

附录 1 中给出了生成的部分 Possion 数据流。

\subsection*{4.3.2 黄灯时长优化}

为了求解模型,需要给出黄灯时长。由于黄灯时长主要由道路本身结构、车速等固定因素决定,与节点的车流量关系较小,因此,黄灯时长一般取为常数。虽然如此,选择一个合适的黄灯时长不仅有利于车辆通行,更是道路车辆安全的保证,因此有必要研究黄灯时长的优化问题。文献 [3] 中给出了一种求解黄灯时长的方法,本文对其做了改进得到了黄灯时长优化方法。

黄灯时长首先必须使以常规速度行驶的车辆通过十字路口,其次必须使黄灯开始时处在停车线以内的车辆也可以在红灯开始之前通过十字路口。设交叉路口长度为 \( X \),车的平均长度为 \( y \),平均停车距离为 \( z \),车的常规速度为 \( v \),刹车力为 \( W \),\( u \) 为车辆质量。黄灯时长 \( \Delta t_y \) 为:
\begin{equation}
\Delta t_y = \frac{x + y + z}{v}
\tag{4-26}
\end{equation}

根据力学原理:

\begin{equation}
w = -u \frac{d^2 z}{dt^2}
\end{equation}
\begin{equation}
\left. \frac{dz}{dt} \right|_{t=0} = v
\end{equation}
\begin{equation}
z(0) = 0
\end{equation}

解上述方程得到:
\begin{equation}
z = \frac{uv^2}{2w}
\end{equation}

则黄灯时长为:
\begin{equation}
\Delta t_y = \frac{x + y + z}{v}
\end{equation}
\begin{equation}
= \frac{x + y}{v} + \frac{uv}{2w}
\end{equation}

上式中交叉路口长度 \(x\)、车的平均长度 \(y\)、车的平均质量 \(u\) 和刹车力 \(w\) 都可以根据经验值确定,带入上式便得到了黄灯时长 \(\Delta t_y\)。

\subsection*{4.3.3 最优信号配时方案}

利用遗传算法对节点能量模型进行求解,分别在固定周期和非固定周期约束条件下对四个相位的配时进行优化,得到了所有车辆单位等待时间最短意义下最优配时方案。

首先固定周期,在此约束下针对十组不同的 Possion 车辆流得到了最优相位配时,为了比较,本文给出了固定周期固定配时情况下的单位等待时间,如表 4-3 所示。其中各个参数取值为:饱和流量 \(S = [2, 2, 2, 2, 2, 2, 2]\),8 个车道 Possion 流的均值向量 \(\lambda = [1, 1, 1, 1, 2, 2, 2, 2]\),固定周期 \(T = 120\),固定配时的四个相位的时长为 \(30s\),黄灯时长 \(\Delta t_y = 3\),起动延时 \(\Delta t_s = 2\),黄灯损失时间 \(\Delta t_d = 1\)。

\begin{table}[h]
\centering
\caption{固定周期相位优化配时方案}
\begin{tabular}{|c|c|c|c|c|c|c|c|}
\hline
序号 & 第 1 相位(s) & 第 2 相位(s) & 第 3 相位(s) & 第 4 相位(s) & 优化配时方案单位等待时间(s) & 固定配时方案单位等待时间(s) & 等待时间减少比率 \\
\hline
1 & 24 & 30 & 30 & 36 & 589.593 & 607.74 & 2.99\% \\
\hline
2 & 15 & 27 & 18 & 60 & 547.992 & 597.894 & 8.35\% \\
\hline
3 & 10 & 15 & 54 & 41 & 575.659 & 645.569 & 10.83\% \\
\hline
4 & 11 & 37 & 26 & 46 & 537.667 & 600.52 & 10.47\% \\
\hline
5 & 19 & 30 & 30 & 38 & 601.967 & 635.951 & 5.34\% \\
\hline
6 & 24 & 31 & 29 & 36 & 593.065 & 610.593 & 2.87\% \\
\hline
7 & 24 & 27 & 35 & 34 & 622.431 & 640.016 & 2.75\% \\
\hline
8 & 16 & 34 & 35 & 35 & 625.789 & 669.618 & 6.55\% \\
\hline
9 & 22 & 35 & 30 & 33 & 555.61 & 578.024 & 3.88\% \\
\hline
10 & 20 & 30 & 35 & 35 & 560.683 & 590.699 & 5.08\% \\
\hline
\end{tabular}
\end{table}

其次,不固定优化配时方案的周期,在此约束下针对十组不同的 Possion 车辆流得到了最优相位配时,为了比较,给出了固定周期固定配时情况下的单位等待时间,如表 4-4 所示。其中各个参数取值为:饱和流量 \( S = [4, 4, 4, 4, 4, 4, 4, 4] \),8 个车道 Possion 流的均值向量 \( \lambda = [1, 2, 3, 4, 4, 3, 2, 1] \),固定配时的周期 \( T = 120 \),固定配时的四个相位的时长为 \( 30 \, \text{s} \),黄灯时长 \( \Delta t_y = 3 \),起动延时 \( \Delta t_s = 2 \),黄灯损失时间 \( \Delta t_d = 1 \)。

\begin{table}[h]
\centering
\caption{固定周期相位优化配时方案}
\begin{tabular}{|c|c|c|c|c|c|c|c|}
\hline
序号 & 第 1 相位 (s) & 第 2 相位 (s) & 第 3 相位 (s) & 第 4 相位 (s) & 优化配时方案单位等待时间 (s) & 固定配时方案单位等待时间 (s) & 等待时间减少比率 \\
\hline
1 & 15 & 23 & 21 & 34 & 676.896 & 944.13 & 28.30\% \\
\hline
2 & 22 & 25 & 29 & 15 & 755.053 & 966.919 & 21.91\% \\
\hline
3 & 19 & 22 & 26 & 17 & 697.862 & 989.919 & 29.50\% \\
\hline
4 & 22 & 24 & 22 & 20 & 697.862 & 898.154 & 22.30\% \\
\hline
5 & 20 & 17 & 17 & 26 & 654.265 & 945.268 & 30.79\% \\
\hline
6 & 31 & 17 & 16 & 16 & 720.096 & 977.041 & 26.30\% \\
\hline
7 & 16 & 26 & 24 & 21 & 657.289 & 938.707 & 29.98\% \\
\hline
8 & 15 & 32 & 18 & 15 & 619.506 & 927.138 & 33.18\% \\
\hline
9 & 20 & 27 & 25 & 15 & 691.533 & 943.333 & 26.69\% \\
\hline
\end{tabular}
\end{table}

由上面两个表可明显看出,优化配时方案的单位等待时间明显小于固定配时的等待时间。下文中会进一步比较分析优化方案和固定配时方案。

\subsection*{4.4 模型分析与评价}
\subsubsection{4.4.1 与固定配时方案的比较}
由 4.3.3 的结果可以看出,本文基于节点能量模型的优化信号配时方案与固定配时方案相比具有很大的优越性。实际中首先要考虑算法优越性的稳定,即在尽可能多的情况下优化配时方案比固定配时方案优越;其次,大车流量情况对交叉口信号控制是一个严重考验,因此希望在大车流量情况下优化配时方案对固定配时方案的优越性比在小车流情况下更优。下文首先分析优化配时方案优越性的稳定,然后在各个道口车流不均匀和各种不同大小车流量的情况下比较优化配时方案和固定配时方案。

\section*{I. 固定周期优化配时方案和固定配时方案比较}
(1) 模型优越性和算法优越性稳定的比较

固定周期下,针对十次不同的 Possion 车辆流得到了最优相位配时和固定周期固定配时情况下的单位等待时间,如图 4-4 所示。其中各个参数取值为:饱和流量 \( S = [2, 2, 2, 2, 2, 2, 2, 2] \),8 个车道 Possion 流的均值向量 \( \lambda = [1, 1, 1, 1, 2, 2, 2, 2] \),固定周期 \( T = 120 \),固定配时的四个相位的时长为 \( 30 \, \text{s} \),黄灯时长 \( \Delta t_y = 3 \),起动延时 \( \Delta t_s = 2 \),黄灯损失时间 \( \Delta t_d = 1 \)。图中等待时间单位为:秒 (s)。

\begin{figure}[h]
    \centering
    \includegraphics[width=\textwidth]{image1.png}
    \caption{优化配时方案与固定配时方案的比较}
    \label{fig:4-4}
\end{figure}

图 4-4 固定周期下优化配时方案和固定配时方案比较

由图 3 可以看出,固定周期下多组实验结果都表明优化配时方案的全体车辆等待时间比固定配时方案的车辆等待时间小,这不仅证明了本文节点能量模型和优化算法的优越性,而且显示了本文的模型和算法具有较好的稳定性。

\subsubsection{在车流量增加情况下优化配时方案和固定配时方案的比较}

固定周期下,增加 Possion 车辆流的大小,得到了最优相位配时和固定周期固定配时情况下的单位等待时间,如图 4-5 所示。其中各个参数取值为:饱和流量 $S=[4,4,4,4,4,4,4]$,固定周期 $T=120$,固定配时的四个相位的时长为 $30s$,黄灯时长 $\Delta t_{y}=3$,起动延时 $\Delta t_{s}=2$,黄灯损失时间 $\Delta t_{d}=1$。图中等待时间单位为:秒(s)。

\begin{figure}[h]
    \centering
    \includegraphics[width=\textwidth]{image2.png}
    \caption{优化配时方案与固定配时方案随车流增加的比较}
    \label{fig:4-5}
\end{figure}

图 4-5 固定周期下优化配时方案和固定配时方案随车流增加的比较

由上图可以看出,在车流量增加的情况下,优化配时方案下车辆的等待时间

比固定配时方案下的等待时间少。实际中期望在大车流量情况下优化配时方案对固定配时方案的优越性强,但由上图看出,优化配时方案对固定配时方案的优越性在大车流量和小车流情况下差别并不是很大,这主要是由于在固定周期情况下,优化算法只能达到局部最优解,而无法达到全局最优解。下面将看到在周期不固定情况下大车流量情况下优化配时方案对固定配时方案的优越性很明显。

\section*{II. 不固定周期优化配时方案和固定配时方案比较}

(1) 模型优越性和算法优越性稳定的比较

不固定优化配时方案的周期,针对十组不同的 Possion 车辆流得到了最优相位配时,如图 4-6 所示。其中各个参数取值为:饱和流量 $S=[4,4,4,4,4,4,4]$,8 个车道 Possion 流的均值向量 $\lambda=[1,2,3,4,4,3,2,1]$,固定配时的周期 T=120,固定配时的四个相位的时长为 $30 \mathrm{~s}$,黄灯时长 $\Delta t_{y}=3$,起动延时 $\Delta t_{s}=2$,黄灯损失时间 $\Delta t_{d}=1$。图中等待时间单位为:秒(s)。

由图 5 可以看出,不固定周期下多组实验结果都表明优化配时方案的全体车辆等待时间比固定配时方案的车辆等待时间小很多,这不仅证明了本文节点能量模型和优化算法的优越性,而且显示了本文的模型和算法具有较好的稳定性。

\begin{figure}[h]
\centering
\includegraphics[width=\textwidth]{image.png}
\caption{不固定周期优化配时方案与固定配时方案比较}
\label{fig:4-6}
\end{figure}

图 4-6 不固定周期优化配时方案和固定配时方案比较

(2) 在车流量增加情况下优化配时方案和固定配时方案的比较

不固定优化配时方案的周期,增加 Possion 车辆流量的大小,得到了最优相位配时和固定周期固定配时情况下的单位等待时间,如图 4-7 所示。其中各个参数取值为:饱和流量 $S=[4,4,4,4,4,4,4]$,固定周期 T=120,固定配时的四个相位的时长为 $30 \mathrm{~s}$,黄灯时长 $\Delta t_{y}=3$,起动延时 $\Delta t_{s}=2$,黄灯损失时间 $\Delta t_{d}=1$。图中等待时间单位为:秒(s)。

\begin{figure}[h]
    \centering
    \includegraphics[width=\textwidth]{image.png}
    \caption{不固定周期优化配时方案与固定配时方案随车流增加的比较}
    \label{fig:4-7}
\end{figure}

图 4-7 不固定周期优化配时方案和固定配时方案随车流增加的比较

由上图可以看出,在车流量增加的情况下,优化配时方案下车辆的等待时间比固定配时方案下的等待时间少。在大车流量情况下优化配时方案对固定配时方案的优越性更强。这表明了本文的模型和优化算法在大车流量严峻情况下具有很强的优越性,这也与实践期望相一致。

(3)8 个车道流量不均衡性情况下优化配时方案和固定配时方案的比较

实践情况中期望在路口各个车道流量不均衡(即各个车道流量差别很不均匀)情况下交叉口信号控制方案能有较强的调度和控制能力,这中不均衡性也是对模型和算法的一个严峻考验。不固定优化配时方案的周期,改变 8 个车道流量的组合即改变车道流量的均衡性,得到了最优相位配时固定周期固定配时情况下的单位等待时间,如图 4-8 所示。其中通过改变各个车道的 Possion 流的均值来调整车流量的均衡性,5 次实验下 8 路车流均值分别为:$\lambda_{1}=[2,2,2,2,2,2,2,2]$,$\lambda_{2}=[1,3,2,2,2,2,2,2]$,$\lambda_{3}=[3,1,3,1,2,2,2,2]$,$\lambda_{4}=[3,1,3,1,2,3,2,1]$,$\lambda_{5}=[3,1,3,1,3,1,3,1]$,其他参数取值为:饱和流量 $S=[4,4,4,4,4,4,4,4]$,固定周期 $T=120$,固定配时的四个相位的时长为 $30s$,黄灯时长 $\Delta t_{y}=3$,起动延时 $\Delta t_{s}=2$,黄灯损失时间 $\Delta t_{d}=1$。图中等待时间单位为:秒(s)。

\begin{figure}[h]
    \centering
    \includegraphics[width=\textwidth]{image.png}
    \caption{不固定周期优化配时方案与固定配时方案随均衡性变换的比较}
    \label{fig:4-8}
\end{figure}

图 4-8 不固定周期优化配时方案和固定配时方案随车流增加的比较

由上图可以看出,优化配时方案对固定配时方案的优越性在车流量不均衡的情况下更明显,这表明车道的车流越不均衡,即车流情况越复杂,本文模型和优化方案的优越性更强。这也与实际期望相一致。

\subsection{4.4.2 模型分析}

由 4.4.1 的分析和比较可以看出,本文的节点能量模型和优化算法具有很强的优越性,下面从模型的合理性、可扩展性等方面分析模型。

\subsubsection{合理性}

本文模型的一个核心思想是将交叉路口节点看作一个具有能量属性的系统,而将车辆的等待延时作用看作车辆流对节点能量的作用。自然界的能量表示一个物体或者系统具有能作用的属性,这个作用可以是释放也可以是吸收,即具有容纳的能力。交叉路口节点具有对车流进行调配、控制的属性,从外部来看,即具有吸收和释放车辆的能力。正因此,交叉路口节点可以具有能量,至于怎样衡量这个能量的大小,这与其他因素有关系。这是可以将交叉路口节点看作具有能量的合理的地方。

其次,车辆流是直接作用于交叉路口节点的因素,节点的一切行为都会反映在车辆流上。所以,这个节点能量首先与节点中车辆的数量有关,车辆的数目越多,节点吸收的就越多,认为其自身能向外作用的能力就越大,所以节点能量与车辆数目呈正相关关系;其次,这个节点能量与车辆的等待时间有关系,车辆等待越久,由于积累的作用,节点向外作用的趋势就越明显,即节点能量与车辆的等待时间呈正相关关系。这是用车辆流和车两流等待时间衡量节点能量大小的合理性之所在。

最后,利用本文模型得到的信号优化方案与传统的固定配时方案相比具有明显的优势,4.4.1 中的数据说明了这一点,这也从侧面反应了本文模型的合理性和正确性。

\subsubsection{优点}

综合上面的数据和分析,可以得出本文的模型具有以下优点。

(1) 本文的模型从整体上对单节点交叉路口进行了描述,从能量角度较为深

刻的刻画了交叉路口的行为特征。

(2) 可扩展性较好。本文模型是描述单节点的,但能量的概念却是对所有节点系统来说的。由于节点能量是一个相对独立又和外界交流的概念,因此,本文的模型具有较好的扩展性。下文将看到,这一概念和模型将被扩展到线状和网格状多交叉路口节点的情况。

**缺点:**

本文的模型具有以下几个缺点。

(1) 本文的模型只对各个相位的配时进行了优化而没有对相位差进行优化,这样不利于全面和综合的解决问题。

(2) 本文的模型主要从系统层面描述问题,对局部和微观的道路行为无法深入的描述,因此在某些需要考虑微观的情况下可能会有很大误差。

### 4.5 模型的进一步讨论和求解

本文所提出的节点能量模型中,在一般车流量较大的时候能够很好的减小所有车辆的平均等待时间,但在车流量很小的情况下可能会有误差,这里做出进一步讨论。

当车流量很小的时候,绿灯期间的初始时期,所有车辆以饱和流量流过交叉路口,但是当等待的车辆全部通过交通路口时,如果信号灯仍然为绿灯,则车辆以进入交叉路口的流速流出,因此,在每一个相位中,流出交叉路口的车辆数因分为两部分计算,下面以第一相位为例说明计算过程,其余相位的计算完全类似。

在第一相位中,1 车道和 5 车道放行,设 $t_{p1}$ 和 $t_{p5}$ 分别为 1 车道和 5 车道绿灯期间所有车流全部流完的时刻,在此之后车辆以进入速率无等待流出,因此 $E_{1}^{out}$ 应当修正为:

\begin{equation}
E_{1}^{out} = \int_{\Delta t_{s}}^{t_{p1}} t S(1) dt + \int_{t_{p1}}^{t_{1} + \Delta t_{y} - \Delta t_{d}} t f_{1} dt
\tag{4-30}
\end{equation}
\begin{equation}
+ \int_{\Delta t_{s}}^{t_{p5}} t S(5) dt + \int_{t_{p5}}^{t_{1} + \Delta t_{y} - \Delta t_{d}} t f_{5} dt
\end{equation}

其中,$t_{1}$ 为第一相位的绿灯停止时间,$\Delta t_{y}$ 为黄灯的持续时间,$\Delta t_{s}$ 为启动延误时间,$\Delta t_{d}$ 为黄灯期间损失时间。要求 $E_{1}^{out}$,首先需要求出 $t_{p1}$ 和 $t_{p5}$。根据 $t_{p1}$ 和 $t_{p5}$ 的定义,易知它们满足:

\begin{equation}
\begin{cases}
N_{0} + \int_{0}^{t_{p1}} f_{1}(t) dt = \int_{\Delta t_{s}}^{t_{p1}} S(1) dt \\
N_{0} + \int_{0}^{t_{p5}} f_{5}(t) dt = \int_{\Delta t_{s}}^{t_{p5}} S(5) dt
\end{cases}
\tag{4-31}
\end{equation}

将求得的 $t_{p1}$ 和 $t_{p5}$ 代入定义式即可得到 $E_{1}^{out}$ 的修正值。其余相位的改进与第一相位类似。

\section{线状交叉路口耦合能量模型建立及求解}

线状交叉路口相对于单节点交叉路口,本质区别就是各个节点之间具有相互的影响,即一个交叉路口的红绿灯的配时所产生的能量会直接影响与其相邻的节点的能量。基于此点考虑,本文将建立 2 个相邻节点的耦合能量模型并且根据模型求解出实时配时方案。

\subsection{耦合能量模型建立}

与节点能量模型相同,本文将从耦合能量的引入与物理意义、能量的计算以及参数的确定等几个方面讨论耦合能量模型。

\subsubsection{耦合的引入与物理意义}

考虑如图 5-1 所示的两个相邻交叉路口 A, B,本文的目标是求出节点 A 和 B 的总体能量的最小值。然而,由于节点 A 的配时方案会对节点 B 的能量产生影响,同样的,节点 B 的配时也会对节点 A 的能量造成影响,因此,全局能量取得的最优值时的信号灯配时方案就不是各个节点分别取得各自能量最小值的配时方案。

\begin{figure}[h]
    \centering
    \includegraphics[width=0.8\textwidth]{image.png}
    \caption{两个相邻节点示意图}
\end{figure}

要求出全局配时方案的最优解,必须量化 A 与 B 之间的影响,若要按照传统方法精确计算出此部分的值,必须分不同相位,不同车道讨论,但是由于交通系统具有很强的非线性和不确定性,这部分很难计算,因此本文换个角度考虑此问题。

事实上,可以将 A 的配时方案对 B 的能量的影响等效为节点 A 在 B 处产生了一定的能量,这很类似于电磁学中两个电磁线圈由于耦合产生的互感,因此将这个能量命名为 A 对 B 的“耦合能量”,这个能量随着 A 与 B 的距离增加而衰减;同样的,B 的配时方案对 A 的能量的影响也可等效为产生了一个 B 对 A “耦合能量”。这两个耦合能量表明了节点 A 与 B 相互影响的强弱。从后面的分析和计算,可以充分说明本文所定义的耦合能量是合理性,而且在复杂的系统中具有很大的优越性。

\textbf{定义 2:} 一个节点 A 的配时方案对其相邻节点产生 B 的等效能量影响,称为节点 A 对节点的 B 的耦合能量,用 $E_{A \to B}$ 表示,节点 B 的配时方案对其相邻节点产生 A 的等效能量影响,称为节点 B 对节点的 A 的耦合能量,用 $E_{B \to A}$ 表示。

定义 3: 节点 A 和节点 B 之间的耦合能量 $E_{AB}$ 为节点 A 对节点的 B 的耦合能量与节点 B 对节点的 A 的耦合能量之和, 即:
\begin{equation}
E_{AB} = E_{A \to B} + E_{B \to A}
\tag{5-1}
\end{equation}

定义 4: 节点 A 和节点 B 组成的系统的总能量 $E$ 为 A 和 B 分别看作孤立单节点时的各自节点能量与 A 和 B 的耦合能量之和, 即:
\begin{equation}
E = E_A + E_B + E_{AB}
\tag{5-2}
\end{equation}

其中, $E_A$ 和 $E_B$ 分别为将 A 和 B 看作孤立节点, 即流入 A 和 B 的车流均为 Poisson 流时节点能量。

从定义 4 可以看出, 当两个节点相邻时, 系统的总能量由两部分组成, 一部分为 A, B 分别看成孤立节点时各自的能量, 这与单节点的情形相同, 另一部分为两个节点由于相互影响而产生的耦合能量。仔细观察可以发现, 这个总的能量很类似于相邻两个线圈产生的电磁感应, 第一部分对应于两个线圈的自感, 第二部分对应于它们之间的互感。这样从宏观上定义的耦合能量, 不仅在物理涵义易于理解, 也避免了微观上需要分不同相位不同车道时进行的繁琐计算。同时, 对于节点较多的线状区域具有很强的可扩展性, 而且节点个数越多, 模型优越性越明显。

以上给出了耦合能量的定义和物理意义, 下面将通过对耦合能量 $E_{AB}$ 的计算, 严格的说明本模型的合理性。

### 5.1.2 耦合能量计算

首先, 耦合能量与 A 与 B 之间的公路上的车流量有关。设两个节点具有相同的周期 $T$, 在 $[0, T]$ 内, 从 A 节点到 B 节点流入的车辆分为通过右转, 直行和左转进入 A 与 B 之间的公路。下面分这三种情况讨论:

#### (1) 通过右转从 A 流入 B 的数量

右转的车辆从 A 节点的 12 车道流入 B, 由于这部分车辆不受到信号灯的限制, 相当于在整个周期内以 Poisson 流 $P_{A12}(t)$ 进入 B, 因此在 $[0, T]$ 内从 A 节点 12 车道进入 B 节点数量 $q_{A12}$ 为:
\begin{equation}
q_{A12} = \int_0^T P_{A12}(t) dt
\tag{5-3}
\end{equation}

#### (2) 通过直行从 A 流入 B 的数量

直行的车辆从 A 节点的 2 车道流入 B, 流量为该车道的饱和流量 $S_A(2)$, 由于仅仅在第二相位时 2 车道为绿灯, 因此在 $[0, T]$ 内从 A 节点 2 车道进入 B 节点数量 $q_{A2}$ 为:
\begin{equation}
q_{A2} = \int_{t_{A1} + \Delta t_s}^{t_{A2} + \Delta t_y - \Delta t_d} S_A(2) dt = S_A(2) (t_{A2} + \Delta t_y - \Delta t_d - t_{A1} - \Delta t_s)
\tag{5-4}
\end{equation}

其中, $t_{A1}$ 和 $t_{A2}$ 分别表示 A 节点第一相位和第二相位的绿灯停止时间, $\Delta t_y$ 为

黄灯的持续时间,$\Delta t_{s}$ 为启动延误时间,$\Delta t_{d}$ 为黄灯期间损失时间。

(3) 通过左转从 A 流入 B 的数量

左转的车辆从 A 节点的 4 车道流入 B,流量为该车道的饱和流量 $S_{A}(4)$,由于仅仅在第三相位时 4 车道为绿灯,因此在 $[0,T]$ 内从 A 节点 4 车道进入 B 节点数量 $q_{A4}$ 为:

\[
q_{A4} = \int_{t_{A2} + \Delta t_{s}}^{t_{A3} + \Delta t_{y} - \Delta t_{d}} S_{A}(4) dt = S_{A}(4)(t_{A3} + \Delta t_{y} - \Delta t_{d} - t_{A2} - \Delta t_{s})
\tag{5-5}
\]

其中,$t_{A2}$ 和 $t_{A3}$ 分别表示 A 节点第二相位和第三相位的绿灯停止时间,$\Delta t_{y}$ 为黄灯持续时间,$\Delta t_{s}$ 为启动延误时间,$\Delta t_{d}$ 为黄灯期间损失时间。

因此,在一个周期 $[0,T]$ 内,从 A 流入 B 的车辆数 $q_{A}$ 为:$q_{A} = q_{A12} + q_{A2} + q_{A4}$

同理,在一个周期 $[0,T]$ 内,从 B 流入 A 的车辆数 $q_{B}$ 为:$q_{B} = q_{B6} + q_{B8} + q_{B10}$

$q_{B6}$,$q_{B8}$,$q_{B10}$ 分别表示通过右转,直行和左转从 B 进入 A 的车辆数量,其计算如下:

\[
q_{B6} = \int_{0}^{T} P_{B6}(t) dt
\tag{5-6}
\]

\[
q_{B8} = \int_{t_{B1} + \Delta t_{s}}^{t_{B2} + \Delta t_{y} - \Delta t_{d}} S_{B}(8) dt = S_{B}(8)(t_{B2} + \Delta t_{y} - \Delta t_{d} - t_{B1} - \Delta t_{s})
\tag{5-7}
\]

\[
q_{B10} = \int_{t_{B2} + \Delta t_{s}}^{t_{B3} + \Delta t_{y} - \Delta t_{d}} S_{B}(10) dt = S_{B}(10)(t_{B3} + \Delta t_{y} - \Delta t_{d} - t_{B2} - \Delta t_{s})
\tag{5-8}
\]

其中,$t_{B1}$,$t_{B2}$ 和 $t_{B3}$ 分别表示 A 节点第一相位和第二相位的绿灯停止时间,$\Delta t_{y}$ 为黄灯的持续时间,$\Delta t_{s}$ 为启动延误时间,$\Delta t_{d}$ 为黄灯期间损失时间。

A、B 之间的实际车量 $q_{AB}$ 为 $q_{AB} = q_{A} + q_{B}$。

由于计算节点 A 和 B 的节点能量 $E_{A}$ 和 $E_{B}$ 时,已经假设他们为孤立的节点,即 A、B 节点上的所有车道上的车流量均为 Poisson 流,而在实际中,A 与 B 之间公路上的输入车道中,如图 5-1,A 节点的 7、8 车道和 B 节点的 1、2 车道,车辆总数完全是之前求出的 $q_{A}$ 和 $q_{B}$,不包括他们各自作为孤立节点的 Poisson 流。

因此在计算总的耦合能量时,应当减去前两部分多计算的 Poisson 流 $q'$,即:$q_{AB} = q_{A} + q_{B} - q'$,而多计算的 Poisson 流 $q'$ 为:

\[
q' = \int_{t_{A1} + \Delta t_{s}}^{t_{A3} + \Delta t_{y} - \Delta t_{d}} \left[ P_{A}(7) + P_{A}(8) \right] dt + \int_{t_{B1} + \Delta t_{s}}^{t_{B3} + \Delta t_{y} - \Delta t_{d}} \left[ P_{B}(1) + P_{B}(2) \right] dt
\tag{5-9}
\]

因此,根据本文能量的定义,A、B 之间车辆对于系统所产生的耦合能量为:

\[
E_{AB} = \int_{0}^{T} q_{AB}(t) dt
\tag{5-10}
\]

5.1.3 衰减因子参数的确定

然而,如果按照上面的方法,相当于 $q_{AB}$ 中所有的车辆都对系统产生能量,

但是实际上并非如此,因为只有在节点附近的车才处于等待状态,才“挤压弹性体”,也就是才产生能量。因此应该进行修正。

显然,$E_{AB}$ 与 A 和 B 之间的距离 $d$ 有关,记作 $E_{AB}(d)$,可以这样考虑:

(1) 当 $d$ 增大时,$E_{AB}$ 减小,表示当两个交叉路口 A 和 B 的距离越远,他们之间的配置方案对于对方的影响也越弱。

(2) 当 $d$ 趋于无穷大时,A 和 B 相距无限远,此时 A, B 均相当于孤立节点,即此时:
\begin{equation}
E_{AB}(\infty) = 0
\tag{5-11}
\end{equation}

(3) 当 $E_{AB}$ 增大的时候,即 A, B 之间的耦合能量增强时,$E_{AB}$ 随着 $d$ 的变化率在下降。从物理涵义上解释,当车流量上升时,所有的车流量在整个道路上分布相对均匀,因此它 $d$ 的变化率在下降。

根据以上三点分析,可以列出 $E_{AB}(d)$ 满足的微分方式:
\begin{equation}
\begin{cases}
\frac{\mathrm{d}E_{AB}}{\mathrm{d}d} = -\frac{1}{\alpha}E_{AB} \\
E_{AB}(\infty) = 0
\end{cases}
\tag{5-12}
\end{equation}

解得:
\begin{equation}
E_{AB} = Ce^{\frac{1}{\alpha}d}
\tag{5-13}
\end{equation}

结合式 5-10 可得,
\begin{equation}
E_{AB} = e^{\frac{1}{\alpha}d}\int_{0}^{T}q_{AB}(t)\mathrm{d}t
\tag{5-14}
\end{equation}

在式中,$\alpha$ 表示耦合能量随着距离的衰减快慢程度,当两个交通灯相距 $\alpha$ 米时,他们之间的耦合能量衰减为初始的 $\frac{1}{e}$。

此时,节点 A 和 B 组成的系统的能量为:$E = E_{A} + E_{B} + E_{AB}$

至此,一个线状多交叉路口的交通信号实时控制问题即转化为单目标规划问题:
\begin{equation}
\begin{cases}
\min E(t_{Ai}, t_{Bi})/T = (E_{A} + E_{B} + E_{AB})/T \\
s.t \begin{cases}
T = t_{A4} + \Delta t_{y} = t_{B4} + \Delta t_{y} \\
i = 1, 2, 3, 4
\end{cases}
\end{cases}
\tag{5-15}
\end{equation}

其中,$t_{Ai}, t_{Bi}$ ($i = 1, 2, 3, 4$) 分别为节点 A 和 B 的第 $i$ 个相位的绿灯截至时刻。

\section*{5.2 模型实时优化算法}

基于在求解单交叉路口配时方案时对遗传算法和模型退火算法的比较,本文更注意算法的稳定性,而对于空间复杂度的要求不是很高。因此,在求解线形多路口配时问题和网状多路口配时问题时,本文均选择了遗传算法对其求解,这也是具有一定合理性的。

同 4.2.1 中所述的算法相比,在算法的流程上基本相同,只是在种群建立和个体交叉和遗传中存在一些区别。

首先,对染色体的信息段进行扩展,由单交叉口的 $(t_{1}, t_{2}, t_{3}, t_{4})$ 扩展为

\((t_1, t_2, t_3, t_4, t_5, t_6, t_7, t_8)\),这里 \(t_4 = t_8\)。

其次,在交叉过程时要求对个体的 8 个信息段进行随机交叉遗传给子代。对于变异操作,和单节点模型不同的是,对于 t4 和 t8 的变异要保持同步性,即对改变其中一个的值,则必须同时更新另一个的值,以保证两个红绿灯周期的一致性。

另外,染色体的适应度函数的求法也发生改变,其值根据公式 \(E = E_1 + E_2 - E_m\) 求。\(E_1\) 和 \(E_2\) 的值同单节点中的求解方法相同。

\subsection*{5.3 模型求解及信号配时方案}

利用遗传算法对耦合能量模型进行求解,对于如图 5-1 所示的两个交叉路口 A 和 B,针对 5 组不同的 Possion 车辆流,分别对 A 和 B 每个节点的四个相位的配时进行优化,得到了所有车辆单位等待时间最短意义下最优配时方案,如表 5-1 和 5-2。其中各个参数取值为:饱和流量 \(S_A = [4, 4, 4, 4, 4, 4, 4, 4]\),\(S_B = [3, 3, 3, 3, 3, 2, 2, 2, 2, 2]\),每个节点 8 个车道 Possion 流的均值向量 \(\lambda = [2, 2, 2, 2, 2, 2, 2, 2, 2, 2]\),A 与 B 之间距离 \(d_{AB} = 600 \, \text{m}\),衰减因子 \(\alpha = 1500 \, \text{m}\),在 \(t = 0\) 时刻,A 和 B 的第一相位同时开始。黄灯时长 \(\Delta t_y = 3\),起动延时 \(\Delta t_s = 2\),黄灯损失时间 \(\Delta t_d = 1\)。

\begin{table}[h]
\centering
\caption{A 节点的优化配时方案}
\begin{tabular}{|c|c|c|c|c|c|}
\hline
序号 & 第 1 相位 (s) & 第 2 相位 (s) & 第 3 相位 (s) & 第 4 相位 (s) & 周期 (s) \\
\hline
1 & 25 & 46 & 20 & 11 & 102 \\
\hline
2 & 23 & 49 & 26 & 12 & 110 \\
\hline
3 & 19 & 45 & 22 & 11 & 97 \\
\hline
4 & 15 & 58 & 20 & 11 & 104 \\
\hline
5 & 29 & 32 & 19 & 29 & 109 \\
\hline
\end{tabular}
\end{table}

\begin{table}[h]
\centering
\caption{B 节点的优化配时方案}
\begin{tabular}{|c|c|c|c|c|c|}
\hline
序号 & 第 1 相位 (s) & 第 2 相位 (s) & 第 3 相位 (s) & 第 4 相位 (s) & 周期 (s) \\
\hline
1 & 28 & 37 & 13 & 24 & 102 \\
\hline
2 & 10 & 54 & 26 & 20 & 110 \\
\hline
3 & 34 & 23 & 14 & 26 & 97 \\
\hline
4 & 36 & 17 & 25 & 26 & 104 \\
\hline
5 & 20 & 44 & 22 & 23 & 109 \\
\hline
\end{tabular}
\end{table}

\subsection*{5.4 模型分析与评价}

\subsubsection{5.4.1 与固定配时的比较}

经过试验,得出实时配时和固定配时的单位时间总延误如表 5-3:

\begin{table}[h]
\centering
\caption{实时配置和固定配置的比较}
\begin{tabular}{|c|c|c|c|}
\hline
序号 & 实时配置 & 固定配置 & 优化 \\
\hline
1 & 1536.54 & 1777.8 & 13.57\% \\
\hline
2 & 1500.48 & 1740.21 & 13.78\% \\
\hline
3 & 1524.07 & 1814.91 & 16.03\% \\
\hline
4 & 1471.41 & 1703.38 & 13.62\% \\
\hline
5 & 1459.92 & 1737.18 & 15.96\% \\
\hline
\end{tabular}
\end{table}

从表5-3可以看出实时配置方案比较稳定的优于固定配置方案,这5次试验中单位时间的延误平均下降了14.59\%。

\subsection*{5.4.2 模型分析}

\subsubsection{合理性}

从直观上看,由于相邻两个交叉路口的分配方案会产生相互影响,相互制约,系统的总能量应当大于两个节点分别取各自最小值时的能量,而耦合能量模型充分考虑它们之间的相互影响,加上了一个等效的耦合项,用来描述他们之间相互制约的强弱。在实际的耦合能量的计算中,本文分步讨论了各个因素对耦合能量的影响,并且利用单节点所定义的能量,给出了出耦合能量的具体计算解析式,并求出节点耦合能量与节点能量之间的数值大小关系,如表5-4,从数据中也可以看出合理性。

\begin{table}[h]
\centering
\caption{节点能量和节点耦合能量的大小关系}
\begin{tabular}{|c|c|c|c|c|}
\hline
序号 & $E_{A}$ & $E_{B}$ & $E_{AB}$ & $\frac{E_{AB}}{\sqrt{E_{A}E_{B}}}$ \\
\hline
1 & 607.941 & 747.638 & 180.986 & 0.27 \\
\hline
2 & 621.257 & 686.442 & 192.382 & 0.29 \\
\hline
3 & 602.23 & 691.92 & 229.92 & 0.36 \\
\hline
4 & 542.467 & 723.159 & 205.788 & 0.33 \\
\hline
5 & 611.67 & 675.311 & 172.943 & 0.27 \\
\hline
\end{tabular}
\end{table}

该表显示,节点耦合能量 $E_{AB}$ 小于节点能量 $E_{A}$ 和 $E_{B}$,这与实际是相符合的,因为现实中两个交叉路口之间的影响对于他们的分配方案仅仅起到辅助作用,起决定作用的还是流入路口的车流数。

类似于电磁感应中的互感系数,本文引入节点耦合系数 $\rho_{AB}$ 刻画两个节点之间的耦合强弱:

\begin{equation}
\rho_{AB} = \frac{E_{AB}}{\sqrt{E_{A}E_{B}}}
\tag{5-16}
\end{equation}

从试验数据看,本题中 $\rho_{AB}$ 约为 0.303。

\subsubsection{可扩展性}

该模型具有很强的可扩展性,基于该模型,可以解决任意线状区域的交通信号灯优化。考虑如图5-2具有n个节点的线状区域,系统总能量为将各自看成孤立节点是的能量与所有相邻节点之间耦合能量之和:

\begin{equation}
E = \sum_{i=1}^{n} E_{i} + \sum_{j=1}^{n-1} E_{j,j+1}
\tag{5-17}
\end{equation}

其中,$E_{i} (i=1,2,\dots n)$ 为第 $i$ 个节点看作孤立节点时的节点能量,$E_{j,j+1} (j=1,2,\dots n-1)$ 为第 $j$ 个节点和第 $j+1$ 个节点的耦合能量。

\begin{figure}[h]
    \centering
    \includegraphics[width=0.8\textwidth]{image.png}
    \caption{线状多节点路口示意图}
    \label{fig:5-2}
\end{figure}

\textbf{优越性:}

该模型与传统的控制优化模型相比具有许多优越性,首先,本模型从整体入手,避免了各个节点之间的细节计算。其次,该模型在扩展的过程中,随着节点个数的增多复杂度的增加速度远远小于其他算法,体现出很大的优越性。

\textbf{缺点:}

(1) 本模型由于是从整体考虑系统的总能量,因此,只有在讨论耦合能量的时候只有在相邻两个交叉路口的信号灯周期相同的情况下才适用。

(2) 本模型中的参数的选取是根据经验值确定的,因此可能会产生一定误差,但始终会优于固定配时情形。

(3) 由于本模型优化的变量是不同节点在不同相位的绿灯时长,因此不存在对相位差的优化。

\section{网状多交叉路口网格能量模型及求解}

网状区域交叉路口相对于单节点交叉路口和线状区域交叉路口,不仅在各个节点之间具有相互的影响,而且在相邻的两条路上也具有相互的影响。考虑如图 6-1 所示的交通示意图,AB 节点构成的线状系统的配时方案会对由 AD 节点构成的系统的能量产生影响,同理,AD 节点构成的线状系统的配时方案会对由 AB 节点构成的系统的能量产生影响,这些配时方案会相互制约,因此需要全局考虑。类似于从节点能量模型到线状的耦合能量模型的扩展,本文将建立一个网格能量模型,并且根据模型求解出网状区域实时配时方案。

\begin{figure}[h]
    \centering
    \includegraphics[width=0.8\textwidth]{image1.png}
    \caption{五节点网状交叉路口示意图}
    \label{fig:5-node-grid}
\end{figure}

\subsection{网格能量模型建立}

与节点能量模型和耦合能量模型相同,本文将从网格能量的引入、定义、以及计算来建立网格能量模型。

\subsubsection{网格能量的引入}

由于当两条道路距离比较远时,道路之间的相互影响可以忽略,因此本文之考虑相邻的两个互相垂直道路之间的配时方案所带来的能量影响。而任意两个相邻且互相垂直的道路可以构成一个三角形,因此,本文首先将需要讨论的网络区域(图 \ref{fig:grid-triangle}) 用直角三角形划分,三角形的顶点为三个交叉路口,三角形的直角边为相邻的两条道路。此时,直角三角形是网络的基本单元,通过对每个三角形中总能量的求解即可获得整个网状区域的能量。因此,本文首先讨论与计算直角三角形单元的能量,将其称之为网格能量。

\begin{figure}[h]
    \centering
    \includegraphics[width=0.8\textwidth]{image2.png}
    \caption{网格三角剖分示意图}
    \label{fig:grid-triangle}
\end{figure}

以 $\triangle ABD$ 为例,网格能量包括三部分,第一部分是将三个节点分别看作孤立节点时的各自节点能量,第二部分是任意有路相连的两个节点之间的节点耦合能量,取值与线状区域时的耦合能量取值相同,第三部分为整个三角形区域的网格耦合能量,这部分的能量是由相邻的两条直角边 $AB$ 和 $AD$ 上的节点耦合能量引起的。下面将通过严格的计算推导出网格能量,并说明本文引入网格能量的合理性和优越性。

\subsubsection{网格能量定义与计算}

定义 5:在一个直角三角形网格内,每条直角边的节点耦合能量对整个网格产生的等效能量影响,称之为网格耦合能量。

在 $\triangle ABD$ 中,网格能量为 $E_{ABD}$,因此对于一个网格,总的能量 $E$ 可以表示为:

\begin{equation}
E = E_{A} + E_{B} + E_{D} + E_{AB} + E_{AD} + E_{ABD}
\tag{6-1}
\end{equation}

其中, $E_{A} + E_{B} + E_{D}$ 为节点能量, $E_{AB} + E_{AD}$ 为节点耦合能量, $E_{ABD}$ 为网格耦合能量, 因此, 对于任意一个三角形网络构成的系统, 都应当满足:

系统总能量 = 节点能量 + 节点耦合能量 + 网格耦合能量

首先, $\Delta ABD$ 的网格耦合能量应该与节点耦合能量 $E_{AB}$ 和 $E_{AD}$ 有关, 并且随着 $\Delta ABD$ 的面积的增大而衰减, 与节点耦合能量的讨论完全类似, 本文直接给出一个网格耦合能量的合理表达式, 并且通过具体的分配方案的计算验证其合理性。

\begin{equation}
E_{ABD} = e^{-\frac{1}{\beta} S_{\Delta ABD}} \int_{0}^{T} q_{AB}(t) + q_{AD}(t) dt = e^{-\frac{1}{\gamma} S_{\Delta ABD}} (E_{AB} + E_{AD})
\tag{6-2}
\end{equation}

$S_{\Delta ABD}$ 为 $\Delta ABD$ 的面积, $q_{AB}(t)$ 为一个周期内在 $AB$ 路上的车流数量的总和, $q_{AD}(t)$ 为一个周期内在 $AD$ 路上的车流数量的总和, $E_{AB}$ 为 $A$ 与 $B$ 的节点耦合能量, $E_{AD}$ 为 $A$ 与 $D$ 的节点耦合能量。$\beta$ 和 $\gamma$ 为修正参数, 其中, $\gamma$ 衡量了相邻的两条路 $AB$ 和 $AD$ 的能量影响随 $\Delta ABD$ 面积的增大的衰减快慢程度, 当三个交通灯构成的直角三角形的面积为 $\gamma$ 时, 他们之间的网格耦合能量衰减为初始的 $\frac{1}{e}$。

\subsection*{6.1.3 网格能量模型建立}

对于图 5-1 所示的 5 个节点 $A, B, C, D, E$, 系统的总能量可以表示为:

\begin{equation}
E = E_{A} + E_{B} + E_{A} + E_{D} + E_{AB} + E_{AD} + E_{BC} + E_{BE} + E_{DE} + E_{ABD} + E_{BDE} + E_{BCE}
\end{equation}

其中, $E_{A} + E_{B} + E_{A} + E_{A} + E_{D}$ 为节点能量和, $E_{AB} + E_{AD} + E_{BC} + E_{BE} + E_{DE}$ 为节点耦合能量和, $E_{ABD} + E_{BDE} + E_{BCE}$ 为网格耦合能量和。

通过此模型, 多节点网状区域的交通信号灯控制问题可以转化为求解如下问题的解:

\begin{equation}
\left\{
\begin{aligned}
& \min E(t_{Ai}, t_{Bi}, t_{Ci}, t_{Di}, t_{Ei}) / T \\
& s.t \left\{
\begin{aligned}
& T = t_{A4} + \Delta t_{y} = t_{B4} + \Delta t_{y} = t_{C4} + \Delta t_{y} = t_{D4} + \Delta t_{y} = t_{E4} + \Delta t_{y} \\
& i = 1, 2, 3, 4
\end{aligned}
\right.
\end{aligned}
\right.
\tag{6-3}
\end{equation}

其中, $t_{Ai}, t_{Bi}, t_{Ci}, t_{Di}, t_{Ei} \ (i = 1, 2, 3, 4)$ 分别为各节点的第 $i$ 个相位的绿灯截至时刻。

\subsection*{6.2 模型实时优化算法}

同 5.2, 将群体中的个体定义进行扩展即可完成对网状节点模型的求解。和前文的扩展方式相同, 继续将种群个体中染色体的信息段加长, 即把网络中要优化的所有节点的配时 $t_{1}, \ t_{2}, \ t_{3}, \ t_{4}$ 都加入个体中, 则可完成对染色体的编码。

由于个体的适应值函数变为 $E = E_{A} + E_{B} + E_{D} + E_{AB} + E_{AD} + E_{ABD}$, 其中节点能量和单交叉路口优化问题中的能量计算公式相同, 两个节点间的耦合能量同线性交叉口优化问题中的能量计算公式一致, 网格耦合能量根据公式, 也是可以线多项式时间内可以计算的。算法运行的时间复杂度同线性交叉口的配时优化算法的

时间复杂度是相同的。

\section*{6.3 模型求解及信号配时方案}

利用遗传算法对网格能量模型进行求解,对于如图6-1所示的五个交叉路口 A,B,C,D 和 E,每个节点的四个相位的配时进行优化,得到了所有车辆单位等待时间最短意义下最优配时方案,如表6-1。

其中各个参数取值为:饱和流量 $S_{A}=[4,4,4,4,4,4,4,4]$,$S_{B}=[3,3,3,3,3,2,2,2,2,2]$,每个节点 8 的个车道 Possion 流的均值向量 $\lambda=[2,2,2,2,2,2,2,2,2,2]$,A 与 B 之间距离 $d_{AB}=200 \mathrm{~m}$,衰减因子 $\alpha=1500 \mathrm{~m}$,在 $t=0$ 时刻,A 和 B 的第一相位同时开始。黄灯时长 $\Delta t_{y}=3$,起动延时 $\Delta t_{s}=2$,黄灯损失时间 $\Delta t_{d}=1$。

\begin{table}[h]
\centering
\caption{各个节点的优化配时方案}
\begin{tabular}{|c|c|c|c|c|c|}
\hline
节点 & 第 1 相位 (s) & 第 2 相位 (s) & 第 3 相位 (s) & 第 4 相位 (s) & 周期 (s) \\
\hline
A & 24 & 30 & 30 & 11 & 95 \\
\hline
B & 17 & 37 & 29 & 12 & 95 \\
\hline
C & 16 & 32 & 34 & 13 & 95 \\
\hline
D & 18 & 31 & 31 & 15 & 95 \\
\hline
E & 17 & 32 & 31 & 15 & 95 \\
\hline
\end{tabular}
\end{table}

\section*{6.4 模型分析与评价}

\subsection*{6.4.1 与固定配时的比较}

重复进行 5 次试验,得到了实时配置和固定配时的单位时间总延误如表 6-2:

\begin{table}[h]
\centering
\caption{实时配置和固定配置的比较}
\begin{tabular}{|c|c|c|c|}
\hline
序号 & 实时配置 & 固定配置 & 优化 \\
\hline
1 & 3395.81 & 4503.23 & 24.59\% \\
\hline
2 & 3407.75 & 4323.37 & 21.18\% \\
\hline
3 & 3380.85 & 4405.48 & 23.26\% \\
\hline
4 & 3185.28 & 4415.77 & 27.87\% \\
\hline
5 & 3334.29 & 4393.39 & 24.11\% \\
\hline
\end{tabular}
\end{table}

可以看出实时配置方案比较稳定的优于固定配置方案,这 5 次试验中单位时间的延误平均下降了 24.20\%。

\subsection*{6.4.2 模型分析}

合理性:

同耦合能量模型相同,由于相邻两条道路的分配方案会产生相互影响,相互制约,因此,直角三角形系统的总能量应当大于按照两个线状区域算出的能量之和,而网格能量模型充分考虑它们之间的相互影响,加上了一个等效的网格耦合

\begin{table}[h]
\centering
\caption{节点能量和节点耦合能量的大小关系}
\begin{tabular}{|c|c|c|c|}
\hline
节点能量 & 节点耦合能量 & 网格耦合能量 & 网格耦合系数 \\
\hline
$E_{A}=568.827$ & $E_{AB}=116.25$ & $E_{ABD}=30.6688$ & $\rho_{ABD}=0.2726$ \\
\hline
$E_{B}=550.582$ & $E_{BC}=140.898$ & $E_{BDE}=34.1539$ & $\rho_{BDE}=0.2715$ \\
\hline
$E_{C}=501.663$ & $E_{AD}=110.364$ & $E_{ABD}=34.7514$ & $\rho_{BCE}=0.2719$ \\
\hline
$E_{D}=502.663$ & $E_{DE}=136.483$ & & \\
\hline
$E_{E}=539.908$ & $E_{BE}=115.882$ & & \\
\hline
\end{tabular}
\end{table}

该表显示,网格耦合能量 $E_{ABD}$ 小于节点耦合能量 $E_{AB}$ 和 $E_{AB}$,而后者又小于节点能量 $E_{A}$,$E_{B}$ 和 $E_{D}$,这与实际是相符合的,因为现实中决定交通信号灯分配方案的最主要因素为流入各个路口的车流数,其次才是相邻两个交叉路口之间的影响,而最后才考虑相邻两条垂直道路之间的影响。

与节点耦合相同,引入网格耦合系数 $\rho_{ABD}$ 刻画两个节点之间的耦合强弱:
\begin{equation}
\rho_{ABD}=\frac{E_{ABD}}{\sqrt{E_{AB}E_{AD}}}
\tag{6-4}
\end{equation}

在本实验中,$\rho_{ABD}$ 约为 0.27。

\textbf{可扩展性:}

网格能量模型具有很强的可扩展性,基于该模型,可以解决任意网状区域的交通信号灯优化。考虑如图 6-3 具有多个节点的网状区域,由于网格之间的距离很远,并且网格之间的影响可以忽略,因此系统总能量为将各自网格能量之和:
\begin{equation}
E=\sum_{i=1}^{n}E_{i}=\sum_{i=1}^{n}(E_{i}^{1}+E_{i}^{2}+E_{i}^{3})
\tag{6-5}
\end{equation}

其中,$E_{i}(i=1,2,...n)$ 为第 $i$ 网格的能量,$E_{i}^{1}$ 为第 $i$ 个网格的节点能量之和,$E_{i}^{2}$ 为第 $i$ 个网格的节点耦合能量之和,$E_{i}^{3}$ 为第 $i$ 个网格的网格耦合能量之和。

\begin{figure}[h]
\centering
\includegraphics[width=0.5\textwidth]{grid_network.png}
\caption{一般网络区域}
\end{figure}

优越性:

与耦合能量模型相同,网格能量模型与传统的控制优化模型相比也避免了各个节点之间的细节计算,并且随着节点个数的增多,复杂度的增加速度远远小于传统模型,体现出很大的优越性。

缺点:

(1) 本模型由于是从整体考虑系统的总能量,因此,只有在讨论耦合能量的时候只有在相邻两个交叉路口的信号灯周期相同的情况下才适用。

(2) 本模型中的参数的选取是根据经验值确定的,因此可能会产生一定误差,但始终会优于固定配时情形。

(3) 由于本模型优化的变量是不同节点在不同相位的绿灯时长,因此不存在对相位差的优化。

\section*{7. 模型实时优化算法比较与分析}

\subsection{7.1 单节点模型优化算法分析比较}

遗传算法与模拟退火算法都属于非经典算法,都可以对 NPC 问题进行降阶处理。在求解本问题中,两种算法表现出了各自自身的优缺点。

\subsubsection{7.1.1 遗传算法分析}

(1) 时间复杂度

遗传算法的外围框架的时间复杂度为种群大小 \( m \) 与迭代次数 \( n \) 的乘积。设内部的时间复杂度的为 \( t \),则总的时间复杂度为 \( O(t * m * n) \)。

对于内部的时间复杂度,主要集中在染色体的适应度函数上,即本文中的公式 \( fit(t_1, t_2, t_3, t_4) = \frac{E(t_1, t_2, t_3, t_4)}{T} \) 中。由模型公式可知内层求解适应值函数最差情况下的时间复杂度为 \( O(n_1 * n_2) \),其中 \( n_1 \) 为车流数,\( n_2 \) 为离散后的最大周期数。

(2) 空间复杂度

遗传算法的空间复杂度为群种数与个体所占空间的乘积。一般来说,种群个体编码后所占的内存空间常常是定值,而遗传算法为了减少迭代的次数往往牺牲空间复杂度,即减大种群中样本个数。这样就增大了程序运行的空间复杂度。

算法的寻优性能

遗传算法由于本身寻优性能的优势,稳定性较好,只要能保证种群数目与迭代次数,常常能收敛于全局最优解的附近,而且很接近最优解,因此具有很好的寻优性能。

\subsubsection{7.1.2 模拟退火算法}

(1) 时间复杂度

模拟退火算法框架的时间复杂度为 \( O(t * N_o * N_i) \),其中 \( N_o \) 为外循环次数,则

\begin{equation}
N_{o} = \frac{K_{\max} - K_{\min}}{\Delta T},
\end{equation}
内层循环 \(N_{i}\) 为热平衡的持续时间,若 \(N_{i}\) 较小,则降温过程过快,算法常常收敛于局部最优解,而 \(N_{i}\) 很大又会影响模拟退火算法的执行时间。 \(t\) 为算法的时间内层执行时间,在本问题中,和遗传算法的内层时间复杂度相同,也为 \(O(n_{1} * n_{2})\),其中 \(n_{1}\) 为车流数,\(n_{2}\) 为离散后的最大周期数。

(2) 空间复杂度

与遗传算法相比,模拟退火算法牺牲算法的寻优性能来获得较低的时间复杂度。因此,在运行过程中,内存中只保存着一个对象个体的副本,因此,占用内存较小。

\section*{7.1.3 两种算法在求解单交叉口优化问题中的结果比较}

\textbf{表 7-1 遗传算法和模拟退火算法结果对比}

\begin{table}[h]
\centering
\begin{tabular}{|c|c|c|c|c|c|c|c|c|c|c|c|c|}
\hline
 & 相位 & 第一组 & 第二组 & 第三组 & 第四组 & 第五组 & 第六组 & 第七组 & 第八组 & 第九组 & 第十组 & 最小值 & 平均值 \\
\hline
\multirow{5}{*}{遗传算法} & 1 & 11 & 8 & 13 & 13 & 14 & 17 & 6 & 12 & 15 & 12 & & \\
\cline{2-14}
 & 2 & 21 & 29 & 18 & 24 & 12 & 22 & 43 & 23 & 21 & 24 & & \\
\cline{2-14}
 & 3 & 22 & 15 & 17 & 36 & 19 & 12 & 31 & 16 & 8 & 22 & & \\
\cline{2-14}
 & 4 & 7 & 10 & 10 & 8 & 12 & 12 & 15 & 10 & 8 & 9 & & \\
\cline{2-14}
 & E/T & 635 & 631 & 649 & 827 & 554 & 682 & 897 & 643 & 583 & 569 & 554 & 667 \\
\hline
\multirow{5}{*}{模拟退火} & 1 & 40 & 20 & 9 & 23 & 16 & 49 & 13 & 13 & 19 & 7 & & \\
\cline{2-14}
 & 2 & 48 & 8 & 59 & 7 & 32 & 7 & 21 & 27 & 8 & 15 & & \\
\cline{2-14}
 & 3 & 12 & 19 & 22 & 21 & 25 & 14 & 12 & 32 & 7 & 22 & & \\
\cline{2-14}
 & 4 & 25 & 47 & 9 & 13 & 13 & 12 & 21 & 41 & 27 & 21 & & \\
\cline{2-14}
 & E/T & 1321 & 908 & 891 & 741 & 841 & 923 & 690 & 1067 & 657 & 571 & 571 & 861 \\
\hline
\end{tabular}
\end{table}

本文实现了两种算法对单交叉口相位时间优化问题。在求解中,两种算法均能逼近问题的最优解。相比之下,遗传算法的稳定性较好,多次求解相同泊松流分布下的配时结果变化不大,而模拟退火算法常常局部于一个局部最优解。因此,在本问题的求解中,对它进行多次计算,取最优值,从求解的结果可以看出,这个最优值和遗传算法得到的最优值是相符的。这两种算法求解单交叉口相位时间优化问题的结果如下表,表中的时间单位都为秒(s)。图 7-1 给出了这两种算法下等待时间的曲线图,从图中可明显看出,模拟退火方法的变化较大,稳定性不如遗传算法,但经过多次迭代其最优值也可达到与遗传算法相当的最优值上。

\begin{figure}[h]
    \centering
    \includegraphics[width=\textwidth]{image.png}
    \caption{遗传算法和模拟退火算法求解单交叉口模型配时优化比较}
    \label{fig:comparison}
\end{figure}

\section{线状节点模型优化算法可计算性和复杂性分析}

如 5.2.1 中所述,本文利用遗传算法求解线状节点模型。由于能量模型把节点间的相交作用当作是两个能量节点的藕和作用,从宏观上来考虑他们的相交作用,大大减化了问题的分析。同时,利用该模型,把配时优化问题转化为系统能量的最小值,从而对问题建立了显式的非线性目标模型。

由于 $E_{A}$ 和 $E_{B}$ 和单节点模型中的适应度函数相同,因此是可以求出的。对于 $E_{AB}$,由式 $E_{AB} = e^{-\frac{1}{\alpha}d} \int_{0}^{T} q_{AB}(t) dt$ 确定。因此,算法的可计算性就等同于 $E_{AB}$ 的可计算性。公式中的积分是对 $t$ 的定积分,流量 $q_{AB}(t)$ 是程序中生成的泊松流,一个周期内的定积分,即是对 $0$ 到 $t_{4}$ 的积分。以 $1$ 秒为单位,对公式进行离散化,得 $E_{AB} = e^{-\frac{1}{\alpha}d} \sum_{0}^{t_{4}} q_{AB} * 1$,这样便求出了 $E_{AB}$ 的近似值。经过以上分析可以可知,本文中求线性多节点的配时优化算法是可计算的。

由前文中对遗传算法性能的分析知,算法总的时间复杂度是外围的时间复杂度 $m*n$ 乘以内部的时间复杂度,其中 $m$ 为种群大小,$n$ 为迭代次数。要求解 EAB,要求 $q_{AB}$ 进行求和,其复杂度为 $T^{2}$,其中 $T$ 为周期数。可知利用本模型求解线性多路口的总时间复杂度为 $O(T^{2}mn)$。

\section{网状节点模型优化算法可计算性和复杂性分析}

算法运行的时间复杂度同线性交叉口的配时优化算法的时间复杂度是相同的。

虽然,在求解线性和网格能量模型中,遗传算法的时间复杂度为多项式时间,

\section*{8. 模型应用建议}

本文的模型和方法提供了一种根据过去和现在的数据为将来决策提供依据的模式,即定时监测车流量,根据模型和算法计算出适合于当前车流的交通信号控制方案,然后将其应用于下一阶段的控制中。然后经过一定的时间间隔,再根据当前车流量进行一次决策,如此循环进行下去。这样就能保证在每一时刻的交通信号控制策略都是接近最优的。监测时间间隔和决策间隔越小,控制策略越优。

因此我们可以设计一种交通信号实时决策控制系统。首先,在交叉路口安装实时监测设备,设定监测间隔 \(\Delta T\),将监测到的实时数据通过网络或其他媒介传入中央决策系统,中央决策系统根据本文的模型和优化算法实时决策得到信号控制方案,然后将方案输出至终端即交叉路口就实现了实时控制。

但无论时间间隔多小或中央处理系统的处理速度多快,中央决策方案输出的信号控制策略相对于其用来决策的数据都是过时的,因为输出的信号控制策略都只是相对于本时刻的数据来说是最优的,而对其要控制的下一时刻的车流量未必是最优的。为了解决这个问题,我们可以根据经验数据进行预测。一般来说,交叉路口每天各个时刻车流量的分布都是相似的,因此中央决策系统在得到由监测设备传来的实时数据时,可以根据历史数据首先计算监测数据和历史同时刻数据的相似性的大小,然后计算历史同一时刻数据到历史下一时刻的数据的映射关系,利用相似性的大小对这个映射关系做出修正,利用这个修正的映射关系得到监测数据的映射,即得到当前下一时刻的预测数据,中央决策系统根据这个预测数据进行决策,得到信号控制方案,输出至交叉路口,即可以得到更优的控制策略。

\section*{参考文献}

[1] 黄崇超 于刚,交叉口延误函数与实时信号配时优化模型,武汉大学学报(工学版),第 39 卷第 5 期,2006 年。

[2] 刘宝碇,赵瑞清,王刚,不确定规划及应用,北京,清华大学出版社,2003 年。

[3] 熊启才,数学模型方法及应用,重庆,重庆大学出版社,2005 年。

\section*{附录}

利用概率算法生成的8组均值为3泊松流,模拟8个车道的车流

\begin{verbatim}
1 1 4 4 1 3 3 2
2 1 6 2 2 4 4 5
1 4 1 3 2 3 4 5
5 6 3 3 1 3 3 3
3 3 2 2 6 4 3 2
3 2 2 5 5 3 3 3
5 6 5 3 2 0 2 1
3 4 6 3 3 1 3 3
1 3 3 8 0 2 3 1
7 5 1 1 2 3 1 2
4 1 0 3 2 2 4 2
1 2 1 3 3 5 4 1
4 3 3 6 5 6 1 4
6 3 1 1 4 5 1 5
4 5 3 2 2 2 3 5
2 2 4 3 2 4 6 3
2 2 3 3 1 6 5 3
1 6 1 1 2 4 6 4
2 6 5 3 1 1 2 3
7 6 2 2 3 1 3 5
4 4 4 4 2 2 4 6
3 2 5 1 0 3 6 2
8 4 4 2 1 3 3 0
3 2 3 3 9 2 2 4
4 4 0 5 1 0 4 3
0 6 3 6 3 2 0 6
3 4 4 2 0 1 4 2
0 3 2 3 3 4 3 2
1 1 3 5 3 1 2 3
5 1 6 1 4 5 2 2
1 3 2 1 1 1 2 5
1 5 1 3 5 3 3 2
1 5 3 6 3 6 2 3
4 4 2 6 2 3 0 3
4 1 1 2 4 4 1 2
2 8 2 3 2 3 1 2
4 1 1 5 2 1 2 4
1 4 3 3 7 5 4 4
4 5 3 7 3 3 4 3
2 2 3 2 3 0 2 1
2 3 1 1 5 2 5 2
\end{verbatim}

\begin{verbatim}
4 3 6 4 3 2 4 0
3 3 2 4 1 4 1 3
3 3 4 2 6 4 5 3
5 1 1 5 8 2 2 2
2 5 3 3 4 2 3 2
6 4 1 6 3 2 2 6
1 4 4 4 5 0 1 2
2 3 4 6 8 2 1 1
4 0 6 4 7 3 2 2
2 4 8 2 1 0 7 3
1 0 5 3 2 6 3 6
5 2 3 6 3 2 3 3
6 3 2 9 2 3 2 2
2 3 1 2 9 2 4 4
6 4 4 3 6 2 2 8
4 5 1 0 3 3 5 3
4 3 2 1 5 2 2 3
2 2 3 2 4 4 3 2
3 5 6 3 0 2 5 4
2 1 3 3 0 1 5 2
2 5 4 7 2 5 6 2
5 0 1 1 4 5 2 2
1 2 4 2 2 3 2 5
4 3 5 4 1 0 4 1
1 2 4 2 6 2 0 2
3 5 1 0 6 3 3 5
4 3 4 2 4 4 4 4
2 2 3 3 4 3 2 2
6 1 3 2 1 2 3 2
1 4 4 6 3 3 3 2
7 3 6 3 6 2 1 2
7 6 4 2 5 3 2 5
2 4 3 2 0 3 4 2
4 4 2 2 3 8 4 4
1 1 3 3 4 3 4 2
1 1 2 2 1 5 2 6
5 3 2 2 1 7 4 2
2 3 2 2 1 6 4 3
4 3 1 4 6 2 1 4
7 2 4 2 3 3 3 3
4 1 0 2 10 2 3 2
3 3 3 2 4 5 3 1
2 5 1 4 3 3 0 8
0 4 1 3 2 4 2 4
\end{verbatim}

\begin{verbatim}
3 3 5 3 5 2 5 3
4 5 6 3 2 3 4 2
2 7 7 3 3 2 3 3
3 3 1 4 1 1 6 2
1 3 3 3 1 6 3 4
4 5 2 1 4 4 3 2
2 6 4 5 0 2 3 3
3 2 3 4 2 4 3 5
5 3 1 2 5 1 3 0
7 2 7 6 3 5 3 9
3 3 6 5 1 5 0 0
3 1 2 3 2 3 6 4
5 3 3 3 2 2 8 2
6 4 3 4 1 7 3 1
2 1 6 2 6 4 0 3
6 1 4 4 1 6 3 1
2 4 2 3 2 4 3 5
5 5 2 7 6 0 2 5
5 2 6 3 4 2 4 1
6 5 2 4 0 1 1 2
5 1 1 1 1 2 5 5
1 5 4 2 2 5 5 1
4 3 4 9 3 5 1 4
3 0 2 3 3 4 5 5
2 6 1 1 1 5 3 4
6 2 3 3 4 4 5 2
2 5 5 0 5 7 2 2
6 4 1 5 4 5 1 4
2 4 3 5 2 1 4 3
3 5 3 4 5 2 3 1
3 4 1 3 3 5 1 4
2 3 7 1 4 2 2 3
4 4 2 3 0 2 1 3
1 1 3 2 2 1 0 1
1 2 4 6 4 9 1 4
4 2 4 5 3 1 4 1
\end{verbatim}