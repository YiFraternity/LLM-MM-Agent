\title{城市道路交通信号实时控制数学模型及微观仿真}
\maketitle

\section{全国第五届研究生数学建模竞赛}

\begin{center}
\includegraphics[width=0.5\textwidth]{image.png}
\end{center}

\textbf{题目} 城市道路交通信号实时控制数学模型及微观仿真

\begin{abstract}
本题是关于城市道路交通信号控制问题。有效的信号控制优化方法可以用于提高城市交通效率、改善城市交通状况。

本文首先对点、线状和区域交通网络特征进行了深入的分析。其次,我们确定了交通信号方式为四相位轮放方式。通过对题意的分析,我们确定目标是使得所有车辆的总延误时间最小。

我们在韦伯斯特延误模型基础上,建立了以所有车辆总延误时间最小为目标的交叉路口实时控制模型,融合基于混沌理论的短期交通流预测技术,设计了遗传算法,可以依据交通流的动态变化对信号周期以及各相位绿信比进行调整。我们对元胞自动机交通仿真原理进行分析,这是一个带信号控制的双向网络交通的微观仿真模型。然后利用该仿真方法对预测的交通流数据进行模拟仿真实现实时控制。

对于单个十字路口的点控制模型仿真中,我们采用元胞长度 7.5 m,模拟时间步长 1s,随机慢化概率 0.05,车仿真时间为 120 分钟对其仿真求解。结果得到定时配时方案的平均总延误时间为 57.6 小时,平均车辆延误时间为 35.3 秒;实时配时方案总延误时间为 47.7 小时,平均车辆延误时间为 29.7 秒。实时配时方案在减少车辆延误时间上取得了显著的效果,总延误时间减少了 17%,平均车辆延误减少了 16%。

对于线状区域和网络区域问题中,我们采用绿波控制的方法来优化干线交叉口的信号控制。为了实现干线的双向绿波,我们采用了轮放的信号控制方法。从理论上讲,双向完全绿波控制配时方案可能不存在。因而,我们构建了一个以期得到最接近双向完全绿波的优化模型。我们利用元胞自动机对其仿真,得到绿波控制的线控制配时方案比独立设置的交叉口控制方案性能有很大改善,绿波控制方案与之相比,总延误时间减少了 13%,平均停车率减少 7%,而平均速度则提高了 7%。

通过对线区域的信号控制方法的推广,我们构建了区域信号优化模型。区域实时配时方案比相同交通区域面积情况下,用两组实时线控制方案的总延误时间降低了 20.5%。该算法可行且适应性强,算法简单易实现。给交通管理部门提出了可行性建议。

关键词:实时信号控制,绿波控制,元胞自动机,延误时间
\end{abstract}

\section*{目录}

\section{一 问题的提出}
\subsection{1.1 背景} \dotfill 2
\subsection{1.2 问题} \dotfill 2

\section{二 基本假设} \dotfill 2

\section{三 符号说明} \dotfill 3

\section{四 问题分析} \dotfill 3

\section{五 模型的建立与求解} \dotfill 5
\subsection{5.1 模型的准备} \dotfill 5
\subsubsection{5.1.1 交通信号方式的确定——四相位轮放方式} \dotfill 5
\subsubsection{5.1.2 延误时间的确定} \dotfill 6
\subsubsection{5.1.3 平均车速的确定} \dotfill 7
\subsubsection{5.1.4 交通流数据的产生——Poisson 分布} \dotfill 7
\subsubsection{5.1.5 饱和度的计算公式} \dotfill 8
\subsubsection{5.1.6 各相位的有效绿灯时间} \dotfill 8
\subsection{5.2 单交叉路口点控制的数学模型、算法设计和结果分析} \dotfill 9
\subsubsection{5.2.1 模型基础} \dotfill 9
\subsubsection{5.2.2 单交叉路口定时点控制模型} \dotfill 10
\subsubsection{5.2.3 单交叉路口实时点控制优化模型} \dotfill 11
\subsubsection{5.2.4 单交叉路口实时点控制算法设计} \dotfill 12
\subsubsection{5.2.5 短时交通流预测方法} \dotfill 14
\subsubsection{5.2.6 元胞自动机交通仿真模型} \dotfill 15
\subsubsection{5.2.7 点控制的算例仿真及结果分析} \dotfill 18
\subsection{5.3 线控制的数学模型、算法设计和结果分析} \dotfill 21
\subsubsection{5.3.1 线控制的数学模型} \dotfill 21
\subsubsection{5.3.2 线控制的实时控制算法设计} \dotfill 23
\subsubsection{5.3.3 线控制的算例仿真及其结果分析} \dotfill 24
\subsection{5.4 区域控制的数学模型、算法设计和结果分析} \dotfill 27
\subsubsection{5.4.1 区域控制的数学模型} \dotfill 27
\subsubsection{5.4.2 区域控制的算例仿真及其结果分析} \dotfill 28

\section{六 算法的稳定性分析} \dotfill 29

\section{七 模型的改进与拓展} \dotfill 30

\section{八 模型的评价} \dotfill 30

\section{九 给交通管理部门的建议} \dotfill 31

\section{参考文献} \dotfill 32

\section{问题的提出}

\subsection{背景}

人类进入21世纪以来,经济和科学技术迅猛发展,交通运输已经成为社会经济生活中的重要方面,并对保证社会经济体系的正常运转发挥着越来越大的作用,实现交通现代化已经被各国政府日益重视。

城市交通管理问题的基本问题之一,是城市道路交通信号控制问题。即根据不同交叉路口、不同相位、不同方向、不同时段的交通流量(标准车辆数),合理的配置各路口的信号灯的周期长,以及同一周期内的红、绿、黄信号的响应时间。以前大多采用固定周期,固定信号比的配时控制方案。为提高道路服务功能,设计实时优化的配时方案对道路畅通和应急决策管理具有重要意义。实时配时方案,即根据交通流量的实时大小,实时配置信号灯的周期长、各种色灯的响应时间,同时考虑信号灯的转换与车辆的起动的损失时间,使全体车辆在所有道口的延误的时间最短[1-3]。

\subsection{问题}

1. 构造单个交叉路口(十字路口或丁字路口)交通信号实时控制的点控制数学模型,并给出相应的实时算法。

2. 构造多个交叉路口(线状区域,)交通信号实时控制的线控制(至少2个交叉口)数学模型,并给出相应的实时算法。

3. 构造多个交叉路口(网络区域)交通信号实时控制的面控制(至少5个交叉口)数学模型,并设计相应的实时算法。

4. 根据城市交通流分布规律(一般理解为Poisson分布),设计一种实时产生交通流序列的方案。并根据你的算法和产生的交通流数据,计算并给出单交叉路口点控制的实时信号配时方案(分为周期固定和周期不固定两种情形考虑),并与固定配时方案比较,说明实时配时方案的效果和优势。

5. 对多路口信号配时的情况,根据你产生的交通流数据和相应的实时算法,分别给出线状区域、网络区域实时配时方案,并比较和评价你所得到的结果。同时分析模型算法的可计算性、算法的复杂性。

6. 给交通管理部门提出应用你所得结果的咨询和建议(例如:流量预测方法、数据处理方法,软件设计、实现步骤等)。

\section{基本假设}

1. 假设车辆到达服从泊松分布;

2. 所有车辆都严格遵守交通规则,不闯红灯;

3. 信号灯的转换与车辆的起动的损失时间为定值;

4. 右转相位不受信号灯的信号控制,即可以实时放行;

5. 假设所有的司机和行人都遵守现行的交通规则,不会出现意外交通事故;

6. 通过停止线的车辆遇上黄灯时可继续前行,所以作为绿灯的特殊情况来处理;

7. 只考虑双向四车道、四相位的交通系统,且采用各方向轮放的信号控制方法。

\section{三 符号说明}

为了便于描述问题和表达,我们对文中的变量符号进行解释说明,如表1所示:

\begin{table}[h]
\centering
\caption{符号说明一览表}
\begin{tabular}{c c l c}
\hline
序号 & 符号 & 意义 & 单位 \\
\hline
1 & $g_j$ & 第$j$相位的有效绿灯时间 & $S$ \\
2 & $C_j$ & 某车道第$j$个信号周期长 & $S$ \\
3 & $\Delta t$ & 车辆启动的平均损失时间 & $S$ \\
4 & $t_i^d$ & 第$i$辆车的延误时间 & $S$ \\
5 & $t_i^z$ & 第$i$辆车的滞留时间 & $S$ \\
6 & $t_i$ & 第$i$辆车的延误时间 & $S$ \\
7 & $T_j$ & 第$j$个交叉口的总延误时间 & $S$ \\
8 & $t_{i,j}$ & 第$i$个和第$j$个交叉路口之间的平均行驶时间 & $S$ \\
9 & $L_0$ & 平均每辆车的车身长度 & $m$ \\
10 & $L_{i,j}$ & 第$i$个和第$j$个交叉路口之间的距离 & $m$ \\
11 & $\overline{v}$ & 车辆的平均行驶速度 & $m/s$ \\
12 & $v_i$ & 第$i$个辆车的行驶速度 & $m/s$ \\
13 & $j$ & 相位,分别取东南西北四个方向 & $-$ \\
14 & $y_i$ & 第$i$个车道的饱和度 & $1$ \\
15 & $\lambda_i$ & 第$i$个周期的绿信比 & $1$ \\
16 & $\overline{h}_t$ & 车辆的平均车头时距 & $s/m$ \\
17 & $q_i$ & 第$i$个车道的交通流量 & 辆$/h$ \\
18 & $CAP_i$ & 第$i$个车道的通行能力 & 辆$/h$ \\
19 & $N_p$ & 第$i$条机动车道的可能通过能力 & 辆$/h$ \\
20 & $Q_i$ & 第$i$个车道的饱和流量 & 辆$/h$ \\
\hline
\end{tabular}
\end{table}

\section{四 问题分析}

现代城市中交通问题已经成为了制约城市发展和经济增长的重大问题。人口的快速增长,土地资源紧缺,私家车数量激增等各方面的因素进一步加重了城市交通的负担。纵观城市交通问题,主要分为“大动脉”和“微循环”两大部分。“大动脉”是指城市道路中的主干道,“微循环”则是各个主干道之间的连接线和道路间的结点。只有“大动脉”和“微循环”都畅通起来,才能有良好的现代化城市交通体系。因此道路结点的优化设计对城市交通系统的整体畅通起到了非常基础的作用,是城市交通系统的重要环节。

\begin{figure}[h]
    \centering
    \includegraphics[width=\textwidth]{image.png}
    \caption{交叉路口通用物理模型示意图}
    \label{fig:intersection_model}
\end{figure}

尽管实际城市道路交叉路口的物理空间布局可能很复杂,我们可以将单交叉路口的情况简化为图 \ref{fig:intersection_model} 所示的物理模型。在单个交叉路口物理模型示意图中,我们可以看到东西南北四个方向上,都有直行、右转、左转的车辆进入交叉口。如果不对车辆的行驶加以控制,车辆在交叉口内很有可能发生冲突。这样就必须有效控制交通信号,让每个方向的车轮流通过。由于在问题的假设中,考虑右转的车辆不受信号灯限制,所以我们只需要考察直行和左转的控制。

问题 1,需要我们构造单个交叉路口:十字路口或者丁字路口交通信号实时控制的点控制数学模型。我们采取双向四车道,轮放的方式进行控制。根据历史数据来预测在未来 $t$ 时间内的流量,根据当前交通流量的大小,来配置下一周期的信号灯周期长以及各种色灯的相应时间。模型的目标是使得全体车辆在所有道口的延误时间最短。

问题 2,将问题 1 中的若干个单交叉路口串联在一起,形成线状区域并实现实时线控制。此时模型的目标不变,但是限制条件变了。由于相邻两个交叉路口存在距离产生的时差,必然会导致两者之间的相位差。问题转化为在固定的距离时间差情况下,调控不同交叉路口的周期和相位大小、相序,使得目标函数值最小。

问题 3,网络区域可以由单交叉路口和线状区域组合而成。即是在线状区域的基础上,考虑若干个单交叉路口与之组合的情况。模型的目标函数依然不变,但是增加了相应的面控约束。在水平和垂直方向上均有交叉路口的时差和相位差,这时周期的实时配置会变得很复杂。

问题 4 中,我们将交通流分布规律假定为 Poisson 分布是合理可行的。根据数据预测的原理和方法,可以采用神经网络、时间序列法等对产生的历史数据进行处理分析,并给出未来 $t$ 时间内车辆流量。利用这些数据根据问题 1 的模型和算法,对固定周期和不定周期分别进行求解,对配时方案和效果进行比较分析。

问题 5,根据 Poisson 分布产生的数据以及预测的车流量数据,对线状区域、网络区域分别进行实时配时,对结果进行分析比较。模型算法的可计算性和复杂性进行评价。

问题 6,根据问题 4 和 5 计算的模型结果,就流量预测方法、数据处理方法、软件设计和实现步骤几个方面,给交通管理部门提出可行性的建议。

\section{五 模型的建立与求解}

\subsection{5.1 模型的准备}

\subsubsection{5.1.1 交通信号方式的确定——四相位轮放方式}

如图 2 所示,我们研究一个双向四车道构成的交通系统,交叉口的信号控制的方式为四相位轮放方式。在路段出口处,两条车道分别设定为直左车道和直右车道。

\begin{figure}[h]
\centering
\includegraphics[width=0.8\textwidth]{image.png}
\caption{交叉路口通用物理模型图}
\end{figure}

交叉路口有四相位组成,分别由四个信号灯来指示控制。

\textbf{相位 1}

如图 2 所示,当信号灯 1 显示绿灯时,其他信号灯 2、3 和 4 均显示红色。此时由南往北方向的车辆可以直行,右转和左转,其他方向的车辆禁止通行(右转车辆除外)。

\textbf{相位 2}

如图 2 所示,当信号灯 2 显示绿灯时,其他信号灯 1、3 和 4 均显示红色。此时由东往西方向的车辆可以直行,右转和左转,其他方向禁止通行(右转车辆除外)。

\textbf{相位 3}

如图 2 所示,当信号灯 3 显示绿灯时,其他信号灯 1、2 和 4 显示红色。此时由北往南方向的车可以直行,右转和左转,其他方向禁止通行(右转车辆除外)。

\textbf{相位 4}

如图 2 所示,当信号灯 4 显示绿灯时,其他信号灯 2、3 和 1 均显示红色。此时由南往北方向的车可以直行,右转和左转,其他方向禁止通行(右转车辆除外)。

\begin{figure}[h]
    \centering
    \includegraphics[width=\textwidth]{image.png}
    \caption{交叉路口双向四车道四相位轮放图}
    \label{fig:traffic_light_schedule}
\end{figure}

\subsection{总延误时间的确定}

从图 \ref{fig:traffic_light_schedule} 中,我们可以看到任意方向的信号周期是红灯、绿灯和黄灯组成的。车辆进入信号灯控制区域分为以下几种情况:

情况 1:车辆进入信号灯控制区域时,受控信号显示绿灯;当车辆到达停止线时,信号灯依然显示绿灯,则车辆正常通过交叉口。此时车辆没有延误时间和滞留时间,延误时间为 0;

情况 2:车辆进入信号灯控制区域时,受控信号显示绿灯;当车辆到达停止线时,信号灯显示红灯,此时车辆无法正常通过交叉口。此时进入下一周期,滞留时间为此方向的红灯时间;当信号灯显示绿灯时,车辆起动,车辆有起动损失时间。车辆延误时间为延误时间和滞留时间之和;

情况 3:车辆进入信号灯控制区域时,受控信号显示红灯;当车辆到达停止线时,信号灯显示绿灯,这时车辆正常通过。此时车辆的延误时间为前面车辆启动所引起的延误时间;

情况 4:车辆进入信号灯控制区域时,受控信号显示红灯;当车辆到达停止线时,信号灯依然显示红灯,车辆将无法正常通过交叉口。滞留时间为绿灯开始时刻与车辆到达时刻之差。当信号灯指示绿灯时,车辆起动,车辆有起动损失时间。车辆延误时间为延误时间和滞留时间之和;

结论:1. 车辆总延误时间等于因无法正常通过的滞留时间与信号灯的转换、车辆的起动的损失时间之和。

2. 交叉路口的车辆总延误时间等于所有到达此交叉路口车辆的延误时间之和。

\subsubsection{滞留时间确定}

滞留时间 = 绿灯开始指示时刻 — 车辆到达时刻

即: \( t_{i}^{z} = T_{\text{绿}} - T_{i} \)

\subsubsection{延误时间确定}

由于信号转换和车辆启动引起的延误时间示意图如图 4 所示,当信号由红灯转变为绿灯的时候,车辆行驶速度由 0 开始加速直至正常行驶速度 \(\overline{v}\),这段时间车辆的平均速度小于 \(\overline{v}\),这样会引起一定的时间损失,我们称为前损失时间。当信号指示由绿灯转变成黄灯的时候,车速由正常行驶速度 \(\overline{v}\) 减速为 0,这段时间损失称为后损失时间。

因此,由于信号转换和车辆启动引起的延误时间等于前损失时间加上后损失时间。在实际过程中,延误时间变化范围不大,我们将延误时间设定为定值 \(2s\)。

\begin{figure}[h]
\centering
\includegraphics[width=\textwidth]{image.png}
\caption{由于信号转换和车辆启动引起的延误时间示意图}
\end{figure}

\subsection{5.1.3 平均车速的确定}

定义:车辆行程与相应时间之比,成为车速。

即为:车辆通过某路程与所用总时间之比,总时间包括有效时间和无效损失时间,但是不含车辆启动,制动和掉头的时间,也不包括途中无理损失时间。

用公式表达为:
\begin{equation}
\bar{v} = \frac{L_{i,j}}{t_{i,j}}
\tag{1}
\end{equation}

根据历史数据,我们设定车辆平均速度为 \(81 \, \text{km/h}\)。

\subsection{5.1.4 交通流数据的产生—Poisson 分布}

我们假定在单位时间内到达的车辆符合泊松分布:
\begin{equation}
P[N(t) = K] = \frac{(\lambda t)^K}{K!} e^{-\lambda t}, \quad K = 0, 1, 2, \ldots;
\tag{2}
\end{equation}

\begin{equation}
f(t) =
\begin{cases}
\lambda e^{-\lambda t}, & t \geq 0 \\
0, & t < 0
\end{cases}
\tag{3}
\end{equation}

例如我们要在 \(60 \, \text{s}\) 时间范围内产生均值为 \(15\) 的泊松分布,在 MATLAB 中使用泊松分布函数,可以产生概率如图 5 所示。

\begin{figure}[h]
\centering
\includegraphics[width=\textwidth]{image2.png}
\caption{泊松分布概率示意图}
\end{figure}

\begin{figure}[h]
    \centering
    \includegraphics[width=\textwidth]{image.png}
    \caption{Poisson分布产生}
    \label{fig:poisson}
\end{figure}

本文交通需求的产生方法是:利用泊松分布产生每分钟到达的车辆数,产生的车辆随机选择该分钟内的任意时刻进入交通系统。

\subsection{饱和度的计算公式}

饱和度是各车道实际到达交通量与该车道通行能力之比,它的大小直接影响着延误计算模型的选取,单纯的靠限制饱和度的大小不能从根本上解决问题。因此要通过饱和度的计算来确定计算模型。饱和度计算公式为

\begin{equation}
y_{i} = \frac{q_{i}}{CAP_{i}}
\tag{4}
\end{equation}

式中:$q_{i}$ 为第 $i$ 车道的实际到达交通量;$CAP_{i}$ 为第 $i$ 车道的通行能力。按照我国城市道路设计规范的规定,在不受平面交叉口影响时,一条机动车道的可能通过能力计算公式如下:

\begin{equation}
N_{p} = \frac{3600}{\overline{h}_{t}}
\tag{5}
\end{equation}

式中:$\overline{h}_{t}$ 为连续车辆平均车头时距,单位为 $s/Veh$。

\subsection{各相位的有效绿灯时间}

在所有信号相位之间按各相位的最大流量比值进行分配,得各相位有效绿灯时间:

\begin{equation}
g_{j} = G_{e} \frac{\max\left[\psi_{1}, \psi_{2}, \ldots\right]}{\psi}
\tag{6}
\end{equation}

式中,$\psi = \sum \max\left[\psi_{i}, \psi_{i}^{'}, \ldots\right]$,$G_{e} = c - L$。

\section{5.2 单交叉路口点控制的数学模型、算法设计和结果分析}

\subsection{5.2.1 模型基础}

模型的基础主要是根据饱和度的值来确定延误时间计算模型。

\subsubsection{韦伯斯特模型}

适用条件:饱和度小于 0.9 的单独路口进口道的车辆信号情况。

国际上经典的交通工程著作介绍的是 Webster 发表在 “Traffic Signal Settings” [4]。Webster 交叉口进口车辆的平均延误时间计算公式如下:

\begin{equation}
t = \frac{C(1-\lambda)^2}{2(1-\lambda x)} + \frac{x^2}{2q(1-x)} - 0.65 \left[ \frac{C}{q^2} \right]^{\frac{1}{3}} x^{(2+5\lambda)}
\tag{7}
\end{equation}

式中,$t$ 为每辆车的平均延误时间;$C$ 为信号周期时长;$\lambda$ 为绿信比,即有效绿等时间与信号周期时间的比率;$q$ 为车流量,$pcu/s$;$s$ 为饱和流量,$pcu/s$;$x$ 为饱和度,观测最大流量与信号交叉口进口道的通行能力之比,$\frac{q}{\lambda s}$。这个表达式的第一项是均匀车辆到达率所产生的延误时间,第二项是根据车辆随机到达服从泊松分布,利用 M/D/1 模型计算出来的车辆的随机性延误,第三项是从车流模拟试验得到的,没有经过世界上其他国家和城市的检验,为了保持严谨性,本文不考虑第三项。因此,简化后的 Webster 模型如下所示:

\begin{equation}
t = \frac{C(1-\lambda)^2}{2(1-\lambda x)} + \frac{x^2}{2q(1-x)}
\tag{8}
\end{equation}

考虑使车辆总延误时间最少,用近似解决该模型,得到定时信号最佳周期时长

\begin{equation}
C = \frac{1.5L + 5}{1 + \psi}
\tag{9}
\end{equation}

式中:$L$ 为每个周期的总损失时间包括信号灯的转换时间和车辆的起动损失时间;$\psi$ 为组成周期的全部信号相位最大的饱和度 $\psi_i$ 之和,$\psi = \sum \max[\psi_i, \psi'_i, \dots]$;

该模型虽然应用广泛,但是其局限性也很明显,当饱和度较大,逐渐趋向于 1.0 时,按照该公式计算出来的延误值明显偏大。

\subsubsection{瞬态延误计算模型}

适用条件:饱和度大于 0.9 的单独路口进口道的车辆信号情况。

该模型是致力于如何减少由于饱和度过大而造成延误明显偏大的问题的。我们采用了非饱和流与饱和流之间的过渡函数延误模型。

\begin{equation}
t' =
\begin{cases}
C(1-g/C)^2 & x < 1 \\
(C-g)/2 & x \geq 1
\end{cases}
+ \frac{Q_0}{C}
\tag{10}
\end{equation}

式中大括号内为均衡相位延误(分为饱和与非饱和两种情况),随机延误和过饱和延误是在 \( Q_0 \) 中计算的。饱和度较大时,它弥补了以往被忽视的随机延误。 \( Q_0 \) 为 \((0, \mathrm{T})\) 时间段内平均饱和排队长度的过渡函数,其计算公式为

\[
Q_0 =
\begin{cases}
0 & x \leq x_0 \\
\frac{CT}{4} \left[ (x-1) + \sqrt{(x-1)^2 + \frac{12(x-x_0)}{CT}} \right] & x > x_0
\end{cases}
\tag{11}
\]

式中:\( x_0 \) 为对应车道的饱和流率(辆/s),\( x_0 = 0.67 + \frac{s_g}{600} \);\( C \) 为该车道的通行能力(辆/s);T 为观察是时段的长度(S)。

### 5.2.2 单交叉路口定时点控制模型

#### 目标函数的确定

由问题的分析和模型的预备可知,模型的目标是使得交叉路口所有车辆的延误时间最小。车辆延误时间等于因无法正常通过的滞留时间与信号灯的转换、车辆的起动的损失时间之和。

#### 单交叉路口信号实时控制模型

以总延误时间最少为目标函数对其进行实时优化配置。

根据饱和度的大小选择对应的模型。设定 \( \eta \) 为延误时间模型选择因子,选择规则定义为:当相位 \( i \) 的各车道平均饱和度 \( \leq 0.9 \) 时 \( \eta \) 为 1,否则 \( \eta \) 为 0;数学表达式为:

\[
\eta =
\begin{cases}
1 & \frac{\sum\limits_{j=1}^k x_{ij}}{k} \leq 0.9 \\
0 & \text{其他}
\end{cases}
\tag{12}
\]

总延误时间为每条车道上车辆的平均延误时间与该车道到达交通量的乘积之和,即

\[
T = \sum_i \sum_j (\eta t_i + (1-\eta) t_i') q_{ij}
\tag{13}
\]

式中:\( q_{ij} \) 为第 \( i \) 相位中第 \( j \) 进口道的车流辆(pcu/s);\( t_i \) 为第 \( i \) 相位依据 Webster 延迟模型计算得各进口道的车辆延误之和(s);\( t_i' \) 为第 \( i \) 相位依据瞬态延误 Akcelik 模型计算得各进口道的车辆延误之和(s);

#### 约束条件的确定

选定目标函数以后,需要寻找一个最优的信号周期时长 \( C \) 和有效绿灯时间 \( g_e \)。目前普遍采用的是先由最佳周期时长 \( C \) 的计算公式(10)近似求得 \( C \),再根据 \( C \leq 60N \) 时增大周期时长,可提高通行能力的原则,调整 \( C \),最后以有效绿灯时间为自变量将其优化。显然此方法不能保证得到使目标函数值最小的 \( C \),而且该方法的适用性有限,当饱和度逐渐增加到 1 时,车辆的总延误的计算公式将不 \( C \) 适用,导致计算结果不正确。

为了克服以上弊端,本文采用对周期离散变量正反逐步双向调整的策略,同时对周期和绿信比进行了优化。周期正向调整从 $C_{\min }$ 开始逐步递增到 $C_{\max }$,计算出最小的总延误时间 $T_{c_{\min 1}}(x)$;反向调整思路与正向调整思路相同,从 $C_{\max }$ 开始逐步递增到 $C_{\min }$,计算出最小的总延误 $T_{c_{\min 2}}(x)$。将两最小总延误值进行比较,较小的值所在的周期和绿信比即位模型的最优周期和绿信比,即最佳配时方案。

考虑信号控制的实际约束条件,将约束条件表述如下:

\begin{equation}
\begin{aligned}
& x_{i}-a_{i} \geq 0 \\
& \sum_{i=1}^{n}\left(x_{i}+\Delta t_{i}\right)=C \\
& 0.75 \leq \frac{y_{i} C}{x_{i}} \leq 1.2 \\
& C_{\min } \leq C \leq C_{\max } f \\
& \bar{v}=75 \mathrm{~km} / \mathrm{h} \\
& 1 \leq i \leq N
\end{aligned}
\tag{14}
\end{equation}

式中:$a_{i}$ 为第 $i$ 相位最小有效绿灯时间;$\Delta t_{i}$ 为第 $i$ 相位损失时间;约束条件3是对交叉口的饱和度(流量与进口道通行能力之比,通行能力等于饱和流量与该流向所在相位的绿信比之积)的限制,0.75用来避免在饱和度过小,即通行能力远大于交通需求时,无谓的增加车辆延误和停车次数,1.2用来避免饱和度过大而造成拥挤,其值可变。约束条件5是对信号周期的约束,目前普遍采用的公式为 $C_{\min }=20 N, C_{\max }=60 N$,$N$ 为相位总数。由此可得,单交叉路口定时点控制模型如下所示:

\begin{equation}
\begin{aligned}
& \text { obj: } \operatorname{Min} T=\sum_{i} \sum_{j}\left(\eta t_{i}+(1-\eta) t_{i}^{\prime}\right) q_{i j} \\
& s.t. \quad \begin{cases}x_{i}-a_{i} \geq 0 \\ \sum_{i=1}^{n}\left(x_{i}+\Delta t_{i}\right)=C \\ 0.75 \leq \frac{y_{i} C}{x_{i}} \leq 1.2 \\ C_{\min } \leq C \leq C_{\max } \\ \bar{v}=75 \mathrm{~km} / \mathrm{h} \\ 1 \leq i \leq N\end{cases}
\end{aligned}
\tag{15}
\end{equation}

\subsection{5.2.3 单交叉路口实时点控制优化模型}

首先根据当前时刻之前的部分历史数据序列,应用基于混沌理论的交通流预测模型,根据第 $t$ 个周期之前的 4 个历史周期的交通流量 $\left\{q_{i}^{C-4}, q_{i}^{C-3}, q_{i}^{C-2}, q_{i}^{C-1}\right\}$ 预测出交通流量 $q_{i}^{C+1}$;然后根据上述饱和度计算公式(4)得到相位 $i$ 的最大饱和度 $\psi_{i}$ 并依据其大小选择相应的车辆延误计算模型。最后可以得到单交叉口的下一个周期 $C+1$ 的优化模型,如下所述:

\begin{equation}
\begin{aligned}
\text{obj: } & \text{Min } T = \sum_{i} (\eta(t_i + 1) + (1 - \eta)(t_i' + 1))q_i^{C+1} \\
\text{s.t. } &
\begin{cases}
x_i - a_i \geq 0 \\
\sum_{i=1}^{n} (x_i + \Delta t_i) = C \\
0.75 \leq \frac{y_i C}{x_i} \leq 1.2 \\
C_{\min} \leq C \leq C_{\max} \\
\nu = 75 \, \text{km/h} \\
1 \leq i \leq N
\end{cases}
\end{aligned}
\tag{16}
\end{equation}

\subsection{5.2.4 单交叉路口实时点控制算法设计}

由于交叉路口配时问题是一个随机过程,采用传统的算法往往难以获得比较满意的最优解,本文采用遗传算法进行模型求解。

算法的适应度函数为模型的目标函数,采用二进制编码方式,用二进制码串 $[d_{n/4}, \dots, d_1, c_{n/3}, \dots, c_1, b_{n/3}, \dots, b_1, a_{n/3}, \dots, a_1]$ 表示一条染色体,$n$ 为染色体长度。二进制码串 $[d_{n/4}, \dots, d_1], [c_{n/4}, \dots, c_1], [b_{n/4}, \dots, b_1], [a_{n/4}, \dots, a_1]$ 分别对应相位 1、相位 2、相位 3、相位 4 的配时 $g_1, g_2, g_3, g_4$,且满足关系式:

\begin{equation}
\begin{aligned}
g_1 &= e + (C - 4e) \sum_{i=1}^{n/4} \frac{a_i 2^i}{2^{n/4+1} - 1} \\
g_2 &= e + (C - 4e) \sum_{i=1}^{n/4} \frac{b_i 2^i}{2^{n/4+1} - 1} \\
g_3 &= e + (C - 4e) \sum_{i=1}^{n/4} \frac{c_i 2^i}{2^{n/4+1} - 1} \\
g_4 &= e + (C - 4e) \sum_{i=1}^{n/4} \frac{d_i 2^i}{2^{n/4+1} - 1}
\end{aligned}
\tag{17}
\end{equation}

式中 $e$ 为相位转换损失时间。

同时,在生成初始种群和经交叉、变异算子生成新的部分种群时,必须考虑模型的约束条件。本文选用的 Smith 等于 1987 年提出的循环交叉操作交叉,循环交叉的执行以父种群的特征作为参考,使每个子代种群在约束条件下进行重组,变异策略选用对等变换策略。

交叉概率 $P_c$ 和变异概率 $P_m$ 是影响遗传算法行为和性能的关键所在,直接影响算法的收敛性。本文采用自适应遗传算法来确定,公式如下:

\begin{equation}
P_c =
\begin{cases}
P_{c1} - \frac{(P_{c1} - P_{c2})(f' - f_{\text{avg}})}{f_{\text{max}} - f_{\text{avg}}}, & f \geq f_{\text{avg}} \\
P_{c1}, & f < f_{\text{avg}}
\end{cases}
\tag{18}
\end{equation}

\begin{equation}
P_{m} =
\begin{cases}
P_{m1} - \dfrac{(P_{m1} - P_{m2})(f_{\max} - f_{\text{avg}})}{f_{\max} - f_{\text{avg}}}, & f \geq f_{\text{avg}} \\
P_{m1}, & f < f_{\text{avg}}
\end{cases}
\tag{19}
\end{equation}

式中,$f_{\max}$ 为群体中最大的适应度;$f_{\text{avg}}$ 为每代群体的平均适应度值;$f'$ 为要交叉的两个个体中较大的适应度值;$f$ 为要变异个体的适应度值;$P_{c1}=0.9$,$P_{c2}=0.6$,$P_{m1}=0.1$,$P_{m2}=0.001$。

综上,算法规则和流程图如下所述:

Step1:初始化,设定种群数目、染色体长度、迭代总代数、复制、杂交变异概率;

Step2:采用二进制编码,在可行域内随机产生种群数目大小的染色体;

Step3:计算种群适应值,并按适应值大小排序;

Step4:按复制概率复制具有适应值较高的染色体到下一代;

Step5:按交叉、变异概率生成除复制染色体以外满足种群数目的染色体到新一代;

Step6:判断是否到了迭代总代数,若没有,则转Step3;

Step7:按最优适应值计算各相位配时;

Step8:根据历史车流量数据,根据预测模型,预估下一周期车流量,转Step2,进行下一个周期的循环。

\begin{figure}[h]
    \centering
    \includegraphics[width=\textwidth]{flowchart.png}
    \caption{实时交叉路口点控制遗传算法流程图}
    \label{fig:flowchart}
\end{figure}

\subsection{5.2.5 短时交通流预测方法}

国内外学者对短时交通流预测模型做了大量的工作,最早有学者提出的是基于解析数学方法的模型,其特点是模型建立在数理统计中回归和时间序列预测的基础上,用解析数学方法来描述状态变量的变化趋势,但是由于这些模型无法反映交通流的不确定性与非线性特征,尤其是随着预测时间间隔的缩短,流量的规律性越发不明显,不确定性越来越强,故预测的精度不高,且抗干扰能力差。

本文采用混沌理论来进行短时交通流量预测,混沌理论研究非线性动力学系统随时间变化的规律。基于混沌理论,可以不必建立预测数学模型,而直接根据交通流序列计算出的客观规律进行预测,这样可提高预测的精度和可信度,实现交通流的短时预测需求。

在建立基于混沌理论的短时交通流预测模型之前,对相空间重构理论做一个简单的介绍。

相空间重构的基本原理是 F. Taken 和 R. Mane 的延迟嵌入理论,即只要适当选取嵌入维数 $m$ 和延迟时间 $\tau$,原混沌动力系统的几何特征与重构的 $m$ 维状态空间的几何特征便是等价的,它们具有相同的拓扑结构。这意味着可把预测问题转化为相空间里的一个短的演化过程来讨论,从而为混沌时间序列的预测奠定了坚实的理论基础。

我们假设已知交叉口某个相位 $i$ 的交通流量历史数据为 $q_i(t)$,$t=1,2,\ldots,N$。重构时间延迟 $\tau$ 和重构相空间维数 $m$,根据互信息法和虚假邻点也随之可以确定下来。由 $q_i(t)$ 构成的重构相空间中状态变量为行向量 $Y(t) = (q_i(t), q_i(t+\tau), \ldots, q_i(t+(m-1)\tau)) \in R^m$,$(t=1,2,\ldots,N-(m-1)\tau)$

其中,每一行向量表示重构相空间中的一个状态点,各状态点依次相连构成吸引子。设反映系统动力学行为的映射为 $F$,对 $n=1,2,\ldots,M-1$,有 $Y(n+1) = F(Y(n))$。

尽管我们不知道 $F$ 的具体形式,但是从重构相空间中状态点演化过程中仍然可以得到对 $F$ 的直观了解。用神经网络的网络映射功能,通过学习可以学得到 $F$ 的映射能力。这样,给出一个观测序列 $d(k)$,$k=1,2,\ldots,k$,其中

预测的具体步骤如下:

(1) 确定交通流序列 $q_i(t)$ 的嵌入维数 $m$ 和延迟时间 $\tau$;为了便于讨论,本文取 $\tau=8$,$m=10$;

(2) 对原始时间序列 $q_i(t)$,$t=1,2,\ldots,N$ 进行归一化;

(3) 重构交通流动力系统的相空间,一维的交通流时间序列转化为相空间的相点 $Y_i$,$i$ 从 $1$ 至 $N-(m-1)\tau$;

(4) 确定当前状态点,即中心点 $Y_m$,$M=N-(m-1)\tau$,寻找中心点 $Y_m$ 的 $k$ 个最近邻点:$Y_{j_1}, Y_{j_2}, \ldots, Y_{j_k}$,并限制短暂分离;

(5) 用 RBF 神经网络进行逼近:从 $k$ 个邻点中找出 $k-h$ 个相点及这些相点经过一步演化成的相点作为学习样本,剩余的 $h$ 个相点作为检验样本;

(6) 预测,以中心点作为输入向量,预测交通系统的下一个状态。

\subsection{5.2.6 元胞自动机交通仿真模型}

我们采用元胞自动机模型对交通流进行仿真。该理论建模始于20世纪90年代 \cite{ref5, ref6}。随后,元胞自动机交通仿真模型迅猛发展成为一种新的交通流动力学模型 \cite{ref7, ref8}。人们把元胞自动机理论应用于交通流的研究,采用离散的时空和状态变量,规定车辆运动的演化规则,通过大量的样本平均,来揭示交通规律。由于交通元素从本质上来说是离散的,用元胞自动机理论来研究交通,就避免了离散-连续-离散的近似过程,因此有其独特的优越性。目前,元胞自动机已被广泛应用于单车道交通、多车道交通、双向交通以及网络交通 \cite{ref9}。

本论文采用元胞自动机模型对具有信号控制的双向网络交通进行建模。该仿真模型将直接应用于评价本文提出在信号控制方法。由于以往对网络交通的建模主要在单向交通网络中进行,因而本文需要建立一个新的元胞自动机模型。本文采用的交通网络为双向四车道格子网络,我们可以引入双车道元胞自动机模型用于描述车辆在路段上的行驶过程。由于交叉口处采用各方向轮放的信号控制方法,车辆在交叉口处的行驶不会和其他方向的车流产生冲突。因而,车辆都可以按照自身的行驶路径顺利通过交叉口。

在元胞自动机交通流模型中,道路被划分为等距格子,每个格子表示一个元胞。在某个时刻,元胞或者是空的,或者被一辆车占据。每一辆车的速度可以取 $0, 1, 2, \ldots, V_{\text{max}}$,$V_{\text{max}}$ 为最大速度。在 $t \to t+1$ 的时间步里,根据如下规则对系统的状态进行更新:

step1:加速,$v_n \to \min(v_n + 1, v_{\text{max}})$;对应于现实中司机期望以最大速度行驶的特性。

step2:减速,$v_n \to \min(v_n, d_n)$;驾驶员为了避免和前车发生碰撞而采取减速的措施。

step3:随机慢化,以概率 $p$;$v_n \to \max(v_n - 1, 0)$;由各种不确定因素(如路面状况不好、驾驶员的不同心态等)造成的车辆减速。

step4:运动,$x_n \to x_n + v_n$;车辆按照调整后的速度向前行驶。

这里,$x_n, v_n$ 分别表示 $n$ 车的位置和速度,$d_n = x_{n+1} - x_n - l_{\text{veh}}$ 表示 $n$ 车和前车 $n+1$ 之间空的元胞数,$l_{\text{veh}}$ 表示车辆长度(在计算 $d_n$ 时特指车辆 $n+1$ 的长度),该演化过程可以用图7表示。

\begin{table}[h]
\centering
\caption{参考时间 $t$}
\begin{tabular}{|c|c|c|c|c|c|c|c|}
\hline
\textcircled{2} & & \textcircled{1} & & & \textcircled{1} & \textcircled{0} & \\
\hline
\multicolumn{8}{|c|}{step1:加速} \\
\hline
\textcircled{2} & & \textcircled{2} & & & \textcircled{2} & \textcircled{1} & \\
\hline
\multicolumn{8}{|c|}{step2:减速} \\
\hline
\textcircled{1} & & \textcircled{2} & & & \textcircled{ } & \textcircled{1} & \\
\hline
\multicolumn{8}{|c|}{step3:随机慢化} \\
\hline
\textcircled{0} & & \textcircled{2} & & & \textcircled{ } & \textcircled{1} & \\
\hline
\multicolumn{8}{|c|}{step4:运动} \\
\hline
\textcircled{0} & & & & \textcircled{2} & \textcircled{0} & & \textcircled{1} \\
\hline
\end{tabular}
\end{table}

图7 NaSch模型中的演化过程示意图

本文的元胞自动机模型具有三个特征:

特征 1:首先采用的是并行更新的方法,而不是顺序更新,这是因为:同顺序更新相比,并行更新可以引起一连串的过渡反应。假设有一辆车因随机慢化作用而减速,如果这时道路的车辆密度足够大,就有可能造成后面的跟驰车辆在减速步进行减速。另外,如果 \( p > 0 \),那么该跟驰车有可能由于随机慢化而进一步进行减速,最终导致静止车辆的出现,进而产生交通堵塞,相对顺序更新而言,并行更新中自发堵塞产生的机理更加真实。

特征 2:除了更新方式以外,本文的元胞自动机模型采用的是开口边界条件。在开口边界条件中,假设道路最左边的元胞对应于 \( x = 1 \),并且道路的入口端包含 \( v_{\max} \) 个元胞,也就是说,车辆可以从元胞 \((0, 1, 2, \ldots, V_{\max})\) 进入到道路中。在 \( t \to t+1 \) 时刻,当道路上的车辆更新完成后,监测道路上的头车和尾车的位置 \( x_{\text{lead}} \) 和 \( X_{\text{last}} \),如果 \( X_{\text{last}} > V_{\max} \),则一辆速度为 \( V_{\max} \) 的车将以概率 \( \alpha \) 进入元胞 \(\min[X_{\text{last}} - V_{\max}, V_{\max}]\)。在道路的出口处,如果 \( X_{\text{lead}} > L_{\text{road}} \),那么道路上的头车以概率 \( \beta \) 驶出路段,而紧跟其后的第二辆车成为新的头车。

特征 3:最后一个特征是本文所采用的对称换道规则。由于本文假设的都是双车道的交叉口,所以有必要引入一套车辆对称换道规则构造一个对称的双车道元胞自动机模型,具体换道规则如下所示:

(1) 换道动机:
\[
d_u < \min(v_u + 1, v_{\max}) \text{ and } d_{u,\text{other}} > d_u
\]

(2) 安全条件:
\[
d_{u,\text{back}} > d_{\text{safe}}
\]

式中,\( x_u, v_u \) 分别为第 \( u \) 辆车的位置和速度;\( d_u = x_{u+1} - x_u - l_{\text{veh}} \) 是第 \( u \) 辆车与前车之间的空的元胞的数目;\( l_{\text{veh}} \) 为车长,没有特殊说明均取 1;\( d_{n,\text{other}} \) 是第 \( u \) 辆车与旁道上的前车之间的空格数;\( d_{u,\text{back}} \) 是第 \( u \) 辆车与旁道上的后车之间的空格数;\( d_{\text{safe}} \) 则是确保不会发生撞车的安全距离;\( v_{\max} \) 是车辆的最大速度。\( d_u < \min(v_u + 1, v_{\max}) \) 表示车辆在本道上不能按期望的速度行驶;\( d_{u,\text{other}} > d_u \) 则表示旁道上的行驶条件要求本道好。

当车辆行驶至交叉口处时,直行和左转车辆要受到信号控制。当信号控制为红灯时,这两类车辆需要减速停车;而信号控制为绿灯时,直行车辆只需要根据前车的行驶情况,调整速度行驶,左转车辆由于在交叉口要完成转弯运动,设其以恒定速度 1 通过交叉口。右转车辆不受信号控制,只要能够确保安全,它可以以速度 1 实现转弯。图 8 给出了各类车辆在交叉口内的行驶路径。

\begin{figure}[h]
    \centering
    \includegraphics[width=\textwidth]{image1.png}
    \caption{车辆在交叉口内的行驶路径}
    \label{fig:8}
\end{figure}

\begin{figure}[h]
    \centering
    \includegraphics[width=\textwidth]{image2.png}
    \caption{单交叉口元胞自动机模型仿真图}
    \label{fig:9}
\end{figure}

\section{表2 单交叉口定时配时信号配时方案仿真结果}

\begin{tabular}{c c c c c}
\hline
路段需求 & 信号周期 & 总车辆 & 总延误时间 & 平均 \\
(辆/分) & (分) & & (小时) & 停车率 \\
\hline
10 & 74 & 4560 & 34.97861 & 0.301821 \\
11 & 86 & 5016 & 40.21342 & 0.311231 \\
12 & 98 & 5472 & 45.91778 & 0.321202 \\
13 & 116 & 5928 & 51.98843 & 0.338258 \\
14 & 120 & 6384 & 71.90954 & 0.396952 \\
15 & 120 & 6840 & 100.7078 & 0.46092 \\
\hline
\end{tabular}

注:总车辆数是根据路段需求数和仿真时间确定的,其公式为 $4 \times$ 路段需求 $\times (120-6)$

\section{表3 单交叉口实时配时信号配时方案仿真结果}

\begin{tabular}{c c c c c}
\hline
路段需求 & 信号周期 & 总车辆 & 总延误时间 & 平均 \\
(辆/分) & (分) & & (小时) & 停车率 \\
\hline
10 & 74 & 4560 & 31.72454 & 0.279942 \\
11 & 86 & 5016 & 39.80392 & 0.282323 \\
12 & 98 & 5472 & 40.18102 & 0.291235 \\
13 & 116 & 5928 & 46.71722 & 0.313235 \\
14 & 120 & 6384 & 58.22065 & 0.344073 \\
15 & 120 & 6840 & 69.94306 & 0.366581 \\
\hline
\end{tabular}

注:总车辆数是根据路段需求数和仿真时间确定的,其公式为 $4 \times$ 路段需求 $\times (120-6)$

\section{总延误时间:总延误时间是指在仿真的时间内,所有车辆的滞留时间和启动损失时间之和,单位为小时。}

\begin{figure}[h]
    \centering
    \includegraphics[width=\textwidth]{image.png}
    \caption{实时配时与定时配时交叉口延误比较(差额为定时减去实时)}
    \label{fig:delay_comparison}
\end{figure}

不同交通需求下,定时配时方案的平均总延误时间为 57.6 小时,平均车辆延误时间为 35.3 秒;实时配时方案总延误时间为 47.7 小时,平均车辆延误时间为 29.7 秒。实时配时方案在减少车辆延误时间上取得了显著的效果,数据表明,平均总延误时间减少了 17%,平均车辆延误减少了 16%。

通过查看图 \ref{fig:delay_comparison},我们可以发现,定时与实时配时方案在改善交叉口延误时间的效果差异是一条先下降后上升的曲线,曲线的最低点是各路段需求量为 11(辆/分) 时。在该点之前定时配时与实时配时的显著差异逐渐下降,到该点达到了最低,随后显著差异稳步增加,当各路段需求量为 15(辆/分),差额达到了 30\% 之多。这表明存在一个点使得这两种配时方案差异最小,随着需求量的增大,定时配时方案逐渐不能适应,此时我们应该考虑对其进行改造优化,使用实时配时方案。

平均停车率:表示的是在仿真的时间内,平均每个时步 $t$ 中最大速度为 0 的车辆数占当前系统中总车辆数的比例。公式表示为:

\begin{equation}
\sum_{t=1}^{T} \frac{\sum_{u=1}^{U^t} \boldsymbol{\omega}}{U^t}, \boldsymbol{\omega} =
\begin{cases}
1 & v_u^{\max} = 0 \\
0 & \text{else}
\end{cases}
\tag{20}
\end{equation}

式中,$U^t$ 表示在时步 $t$ 上时,系统得所有车辆数;$T$ 表示系统的仿真时间。

\begin{figure}[h]
\centering
\includegraphics[width=\textwidth]{image1.png}
\caption{实时配时与定时配时交叉口延误比较(差额为定时减去实时)}
\label{fig:11}
\end{figure}

不同交通需求下,定时配时方案的平均停车率为 0.355,实时配时方案平均停电率为 0.313。实时配时方案在减少车辆平均停车率上取得了一定的效果,数据表明,平均停车率减少了 12%。

通过查看图 \ref{fig:11},我们可以发现,定时与实时配时方案在改善交叉口停车率的效果差异是一条抛物线后直线上升的曲线,曲线的最低点是各路段需求量为 10 和 13(辆/分)时。在这两点之间时,差异不是很显著,持平稳状态。随后当路段需求量超过 13(辆/分)时,差异呈直线急剧上升势。从相差 7\% 增加到 20\% 之多。这表明平均停车率有一个临界点,在此临界点左侧两种方法的差异不显著,而在另一侧差异显著增加。当路段需求量增大到 13(辆/分)时,定时配时方案逐渐不能适应,此时我们应该考虑对其进行改造优化,使用实时配时方案。

平均速度:在仿真的时间内,车辆的平均行驶速度。

\begin{figure}[h]
\centering
\includegraphics[width=\textwidth]{image.png}
\caption{实时配时与定时配时交叉口平均速度比较(差额为实时减去定时)}
\label{fig:12}
\end{figure}

不同交通需求下,定时配时方案的平均速度为 $48.8\,\text{km/h}$,实时配时方案平均速度为 $52\,\text{km/h}$。实时配时方案在减少车辆平均停车率上取得了一定的效果,数据表明,平均速度增加了 $6\%$。

通过查看图 \ref{fig:12},我们可以发现,定时与实时配时方案在改善交叉口停车率的效果差异是一条抛物线后直线上升的曲线,曲线的最低点是各路段需求量为 $10$ 和 $13$(辆/分)时。在这两点之间时,差异不是很显著,持平稳状态。随后当路段需求量超过 $13$(辆/分)时,差异呈直线急剧上升势。从相差 $3\%$ 增加到 $18\%$ 之多。这表明平均速度有一个临界点,在此临界点左侧两种方法的差异不显著,而在另一侧差异显著增加。当路段需求量增大到 $13$(辆/分)时,定时配时方案逐渐不能适应,此时我们应该考虑对其进行改造优化,使用实时配时方案。

这里,我们注意到一个现象,平均速度和平均停车率两者有很相似的特性,差异曲线很相似,临界点都是在路段需求 $13$(辆/分)上。这说明平均速度和平均停车率有其内在的相似性。

\subsection{5.3 线控制的数学模型、算法设计和结果分析}

\subsubsection{5.3.1 线控制的数学模型}

采用“绿波”来实现对干线交叉口的信号控制,所谓绿波就是在指定的道路上,当归定好路段的平均车速后,要求信号控制机根据路段距离,把该车流的绿灯起始时间做相应的调整,这样一来,以确保该车流到达每个路口时,正好遇到“绿灯”。值得注意的是,实现绿波控制的交叉口具有相同的信号周期。当各交叉口的理论最佳周期不相等时,选择其中最大的周期作为各个交叉口的共有周期。

理论上,单向绿波较容易实现,而双向绿波控制的实现需要相对苛刻的条件 \cite{ref10}。当路段长度为车流在一个信号周期内行驶的距离的整数倍时很容易完全实现双向绿波,但是现实中的路段往往不能满足这个条件。通常而言,要同时实现双向绿波控制,需要对各交叉口的相序做出合适的调整。相序的调整很大程度增加了双向绿波控制的实现难度。最为糟糕的情形是,无论如何设置相序都无法完全实现双向绿波控制。

依据双向绿波信号控制的特点,本文将设计一种新的的干线交叉口信号联合控制方法。该方法要实现的目标是:得到的信号配时方法最大程度接近双向绿波控制。在满足基本配时约束的前提下,配时得到的干线方向相邻交叉口的相位差与理论相位差之间的误差最小。我们用相位差的总误差来评定配时方案与双向绿波控制的接近程度。

不妨设交叉口的相位集合为 $B = \{1, 2, 3, 4\}$,分别用于表示东西向、南北、西东、北南四个信号相位。考虑东西向和西东两个方向实现绿波控制,则可定义绿波控制相位集合 $I = \{1, 3\}$。

如图 \ref{fig:13} 所示,我们考虑由东向西方向的两个相邻交叉口 $n$ 和 $n+1$。由东向西方向的车流受相位 $i=1$ 的信号控制,定义 $\Delta t^n$ 为交叉口 $n$ 与交叉口 $n+1$ 对于相位 $i$ 的绿波控制理论相差。$\Delta t^n$ 的取值受到交叉口的排队长度,相位损失时间,路段平均行驶时间约束。

经过分析可以得到 $\Delta t^n = \frac{p_n - p_{n+1} + s}{\bar{v}} + l_{\text{补}}$,其中 $p_n = V_n \bar{l}_o$,$p_{n+1} = V_{n+1} \bar{l}_o$,$s$ 为路段 $(n, n+1)$ 的距离,$\bar{v}$ 为路段 $(n, n+1)$ 的平均速度,$l_{\text{补}}$ 为补偿系数,该补偿系数是对于司机反映时间、相位转换时间和黄灯时间的补偿。

\begin{figure}[h]
    \centering
    \includegraphics[width=\textwidth]{image.png}
    \caption{两交叉口的绿波控制理论相位差}
    \label{fig:13}
\end{figure}

设 $t_i^n$ 和 $t_i^{n+1}$ 分别表示交叉口 $n$ 与交叉口 $n+1$ 的第 $i$ 个相位的开始时间,我们可以推算出相邻交叉口的相位差与理论相位差之间的误差:
\begin{equation}
    e_i^n = \left| t_i^n + \Delta t^n - t_i^{n+1} + k_i^n C \right|
    \tag{21}
\end{equation}
这里,$C$ 是信号周期,$k_i^n$ 是一个能够使得 $e_i^n$ 最小的整数。

可以采用如下模型去求解干线多交叉口实时控制优化问题:

\begin{equation}
\begin{aligned}
\text{obj: } & \min f(x) = \sum_{n=1}^{N-1} (t_1^n + \Delta t^n - t_1^{n+1} + k_1^n C)^2 \\
& + \sum_{n=1}^{N-1} (t_3^{n+1} + \Delta t^n - t_3^n + k_3^n C)^2
\end{aligned}
\end{equation}

\begin{equation}
\begin{aligned}
\text{s.t. } & \begin{cases}
g_{i,\min}^n \leq g_i^n & \forall n \in A, \forall i \in B \tag{1} \\
t_i^n + g_i^n + Y = \sum_{j \in B \cap j \neq i} x_{ij}^n t_j^n & \forall n \in A, \forall i \in B \tag{2} \\
\sum_{i \in B(i \neq j))} x_{ij}^n = 1 & \forall n \in A, \forall j \in B \tag{3} \\
\sum_{j \in B(j \neq i))} x_{ji}^n = 1 & \forall n \in A, \forall i \in B \tag{4} \\
\sum_{i \in B} g_i^n = C - l & \forall n \in A \tag{5} \\
x_{ij}^n \in \{1, 0\}, k_i \in \mathbb{Z} & \forall n \in A, \forall i \in B, \forall j \in B \tag{6}
\end{cases}
\end{aligned}
\end{equation}

式中:$C$ 表示交叉口的周期,本文各交叉口的周期相同;$\gamma$ 表示相位转换时间和黄灯时间之和;$g_i^n$ 表示交叉口 $n$ 的相位 $i$ 的有效绿灯时间,$g_{i,\min}^n$ 表示交叉口 $n$ 的相位 $i$ 的有效绿灯时间的最小值,它由单点信号配时情况决定,$l$ 表示单个周期的黄灯损失时间,$N$ 表示交叉口数量。$x_{ij}^n$ 为 0-1 变量,如果交叉口 $n$ 的第 $i$ 个相位切换到第 $j$ 个相位则 $x_{ij}^n = 1$,否则 $x_{ij}^n = 0$。

约束条件 (1) 描述交叉口 $n$ 的第 $i$ 相位最小取值约束,受到总周期时长 $C$ 的约束;条件 (2) 表示如果存在 $i \to j$,那么 $t_i$ 与 $t_j$ 存在条件 (2) 的转换关系;条件 (3) 表示有且只有一个相位 $i$ 直接转换到相位 $j$;条件 (4) 表示有且只有一个相位就直接转换到相位 $j$;条件 (5) 表示交叉口的各个相位的有效绿灯时间之和的约束。

在求解以上模型之前,需要采用单点信号控制模型对各个交叉口分别进行优化配时。配时完成以后,选取所有交叉口中最大的信号周期作为绿波协调控制周期。需要说明的是,以上模型虽然是静态实现干线绿波控制,但该模型可以借助于短期的交通流量预测,动态地计算交叉口的绿波协调控制的配时方案,以达到动态信号控制的目的。

\subsection{5.3.2 线控制的实时控制算法设计}

与单交叉口实时控制算法相同,本文依然采用遗传算法来解决干线多交叉口实时控制。该算法的适应度函数为 $\sum_{i \in I} \sum_{n \in A \cap n \neq \max(A)} (t_i^n + \Delta t^n - t_i^{n+1} + k_i C)^2$,选用顺序编码方式,二进制字符串 $\{i_4^n, i_3^n, i_2^n, i_1^n, \dots, i_4^1, i_3^1, i_2^1, i_1^1\}$ 表示一条染色体,$i_4^n, i_3^n, i_2^n, i_1^n$ 为交叉口的 $n$ 的相位安排顺序,如 $\{1324\}$;交叉、变异时都只在相同的交叉口内进行,两个交叉口直接不进行。设计步骤如下:

\textbf{Step1:} 根据每一交叉口的平面布局及计算交通量,按单点定时控制的配时方法,确定每一个交叉口所需的周期时长并以所需周期最大的交叉口为关键交叉口,最大的周

\section{5.3.3 线控制的算例仿真及其结果分析}

我们以两个交叉口为例,应用干线控制模型对其进行绿波控制。

\begin{figure}[h]
\centering
\includegraphics[width=\textwidth]{image1.png}
\caption{干线控制的两交叉口基本输入数据}
\end{figure}

如图14所示,根据泊松分布,在不同交叉口的不同相位设置相应的 $\lambda$ 数值。将 $\lambda$ 转换为 Veh/h 单位,可知该干线路段基本输入数据为

\begin{table}[h]
\centering
\caption{干线路段基本输入数据}
\begin{tabular}{c c c c c c}
\hline
\diagbox{相位}{交叉口} & 相邻距离 & 第一相位 & 第二相位 & 第三相位 & 第四相位 \\
& /平均车速 & (Veh/h) & (Veh/h) & (Veh/h) & (Veh/h) \\
\hline
交叉口1 & 600m, 81km/h & 774 & 840 & 720 & 600 \\
交叉口2 & & 780 & 660 & 714 & 900 \\
\hline
\end{tabular}
\end{table}

\begin{figure}[h]
\centering
\includegraphics[width=\textwidth]{image2.png}
\caption{交叉口的独立信号控制方案}
\end{figure}

首先,根据单交叉口信号控制模型,可以得到交叉口1和交叉口2的信号配时方案如图15所示。

交叉口2周期为111s,交叉口1的周期为102s。根据干线控制算法可知交叉口2为关键交叉口,其周期长111s为线控系统的共用周期。根据干线控制模型,计算研究发现在交叉口1的3相位绿灯开始时间延迟于交叉口2的3相位 $600 \times 3.6 / 81 = 27 \, \text{s}$ 时可以完全实现双向绿波,即满足 $t_3^1 - t_3^2 = 27$;

\begin{figure}[h]
\centering
\includegraphics[width=\textwidth]{image1.png}
\caption{(a) 交叉口1信号配时方案}
\end{figure}

\begin{figure}[h]
\centering
\includegraphics[width=\textwidth]{image2.png}
\caption{(b) 交叉口2信号配时方案}
\end{figure}

图16 干线双向绿波控制方案

图16显示,交叉口1的相位转换顺序为 $1 \to 3 \to 4 \to 2$,交叉口2的相位转换顺序为 $1 \to 4 \to 2 \to 3$,黄灯时间为3s;运用本文建立的元胞自动机交通仿真模型,在vs.net上进行仿真实现。假设每个元胞单元格表示7.5米的实际道路长,路段上车辆平均速度为3格/秒,则仿真结果如图17所示。

仿真时间为120分钟,前6分钟为预热时间,从第7分钟开始正式的干线控制仿真。

\begin{table}[h]
\centering
\caption{干线控制绿波前后结果比较}
\begin{tabular}{l c c c}
\hline
\diagbox{指标}{方法} & 绿波控制 & 不设置相位差的控制 & 各交叉口独立设置 \\
\hline
总延误(h) & 115.9431 & 145.8295 & 132.7616 \\
平均停车率 & 0.455355 & 0.51413 & 0.489954 \\
平均速度(km/h) & 40.58293 & 35.90057 & 37.77879 \\
\hline
\end{tabular}
\end{table}

从表5可以看出,绿波控制的干线控制配时方案无论是总延误、平均停车率还是平均速度相比其他两种都是最优的。不设置信号延迟的控制方案是指两个交叉口用同样的周期时长,但是不设置相位差的干线控制方案,由表可知该种方法的三个指标都是最差的,交叉口独立设置的干线控制方案性能居中,绿波控制方案与之相比,总延误减少了13%之多,平均停车率减少了7%,而平均速度则提高了7%。

这给我们的启示是在干线控制中,绿波控制能够取得很好的提升道路交通性能的效果,在条件允许的情况下,尽量设置绿波控制配时方案。在一些无法实现绿波控制的干线控制中,共用一个关键交叉口周期时长的配时方案往往没有各交叉口独立设置方案优,此时可以考虑采用各交叉口独立配时方案。

\begin{figure}[h]
    \centering
    \includegraphics[width=\textwidth]{image1.png}
    \caption{干线控制元胞自动机交通仿真实现图}
    \label{fig:17}
\end{figure}

\begin{figure}[h]
    \centering
    \includegraphics[width=\textwidth]{image2.png}
    \caption{三种方法的信号配时效果比较}
    \label{fig:18}
\end{figure}

(横坐标中1表示绿波控制、2表示无相位差的控制、3表示独立控制)

\section{5.4 区域控制的数学模型、算法设计和结果分析}

\subsection{5.4.1 区域控制的数学模型}

\subsubsection{区域多交叉口信号控制优化模型}

目前,国内外对于区域性的信号优化控制方法的研究还不够深入,没有形成一定的理论体系。本文对区域的交叉口信号控制采用的方法是:在区域内最大程度地实现绿波控制。与干线的绿波控制相比,区域绿波信号控制的复杂性提升了很多。不过,我们仍然可以采用干线绿波控制的思想,即绿波控制方向相邻交叉口的相位差与理论相位差之间的误差最小。

我们假设研究区域内的交通网络为方格型网络,网络东西向有 $M$ 条街道,南北向有 $N$ 条街道。对于东西向的任意街道 $m$,可以采用以上的干线信号优化模型中目标函数的计算方法去计算街道 $m$ 的绿波控制相位误差;同理也可以计算南北向任意街道 $n$ 的相位误差。与干线多交叉口模型类似,可以得出如下区域信号控制优化模型:

\begin{equation}
\begin{aligned}
\text{obj: } & \min f(x) = \sum_{m=1}^{M} \sum_{n=1}^{N-1} \left( t_1^{m,n} + \Delta t_{EW}^{m,n} - t_1^{m,n+1} + k_1^{m,n} C \right)^2 \\
& + \sum_{m=1}^{M} \sum_{n=1}^{N-1} \left( t_3^{m,n+1} + \Delta t_{EW}^{m,n} - t_3^{m,n} + k_3^{m,n} C \right)^2 \\
& + \sum_{n=1}^{N} \sum_{m=1}^{M-1} \left( t_2^{m,n} + \Delta t_{SN}^{m,n} - t_2^{m,n+1} + k_2^{m,n} C \right)^2 \\
& + \sum_{n=1}^{N} \sum_{m=1}^{M-1} \left( t_4^{m,n+1} + \Delta t_{SN}^{m,n} - t_4^{m,n} + k_4^{m,n} C \right)^2
\end{aligned}
\end{equation}

\begin{equation}
\begin{aligned}
\text{s.t. } & \begin{cases}
g_{i,\min}^{m,n} \leq g_i^{m,n} & \forall i, m, n \\
t_i^{m,n} + g_i^{m,n} + Y = \sum_{j \in B \cap j \neq i} x_{ij}^{m,n} t_j^{m,n} & \forall i, m, n \\
\sum_{i \in B(i \neq j)} x_{ij}^{m,n} = 1 & \forall j, m, n \\
\sum_{j \in B(j \neq i)} x_{ji}^{m,n} = 1 & \forall i, m, n \\
\sum_{i \in B} g_i^{m,n} = C - l & \forall m, n \\
x_{ji}^{m,n} \in \{1, 0\}, k_i \in \mathbb{Z} & \forall i, j, m, n
\end{cases}
\end{aligned}
\end{equation}

以上模型中,$t_i^{m,n}$ 表示交叉口 $(m, n)$ 的第 $i$ 个相位的起始时刻;$\Delta t_{EW}^{m,n}$ 表示交叉口 $(m, n)$ 与交叉口 $(m, n+1)$ 之间的理论相差,$\Delta t_{SN}^{m,n}$ 表示交叉口 $(m, n)$ 与交叉口 $(m+1, n)$ 之间的理论相差;$g_i^{m,n}$ 表示交叉口 $(m, n)$ 的第 $i$ 个相位的有效绿灯时间;$x_{ij}^{m,n}$ 为 0-1 变量,如果交叉口 $(m, n)$ 的第 $i$ 个相位切换到第 $j$ 个相位则 $x_{ij}^{m,n} = 1$,否则 $x_{ij}^{m,n} = 0$。

以上优化模型与干线多交叉口信号优化模型相比,其目标函数仅增加了南北向信号协调误差,而约束条件没有太大的变化。由于约束条件数量以及变量数的增加,区域信号控制模型更难求解。在求解以上模型之前,同样也需要采用单点信号控制模型对各个交叉口分别进行优化配时。配时完成以后,选取所有交叉口中最大的信号周期作为绿波协调控制周期。该模型可以借助于短期的交通流量预测,动态地计算交叉口的绿波协调控制的配时方案,以达到动态信号控制的效果。

\subsection{5.4.2 区域控制的算例仿真及其结果分析}

我们以六个交叉口为例,应用区域控制模型对其进行绿波控制。

\begin{figure}[h]
\centering
\includegraphics[width=\textwidth]{image1.png}
\caption{区域控制的多交叉口基本输入数据}
\end{figure}

根据本文上述的区域交叉口信号控制模型,求解出各个信号的独立配时方案,在此基础上,运用本文建立的元胞自动机交通仿真模型,在 vs.net 上进行仿真实现。假设每个元胞单元格表示 7.5 米的实际道路长,路段上车辆平均速度为 3 格/秒,则仿真结果如图 20 所示:

\begin{figure}[h]
\centering
\includegraphics[width=\textwidth]{image2.png}
\caption{区域控制元胞自动机交通仿真实现图}
\end{figure}

仿真时间为 120 分钟,前 6 分钟为预热时间,从第 7 分钟开始正式的干线控制仿真。

\begin{table}
\centering
\caption{干线控制绿波前后结果比较}
\begin{tabular}{c c c c}
\hline
\diagbox{指标}{方法} & 单独控制 & 两主干道绿波(东西向) & 区域协调控制 \\
\hline
总延误(h) & 292.848 & 251.3281 & 232.7045 \\
平均停车率 & 0.467354 & 0.427246 & 0.406538 \\
平均速度(km/h) & 39.54703 & 42.69822 & 44.25063 \\
\hline
\end{tabular}
\end{table}

从表 6 可以看出,区域协调控制的多交叉口配时方案无论是总延误、平均停车率还是平均速度相比其他两种都是最优的。两主干道绿波是指在东西向两条主干道上实现绿波控制,南北相位不进行绿波控制,由表可知该种方法的三个指标都居中,各交叉口的独立控制的配时方案是差的。该三种方案相互比较结果如下表 7 所示:

\begin{table}
\centering
\caption{三种控制策略两俩差异比较(A 表示区域协调控制、B 为两主干道绿波、C 为单独控制)}
\begin{tabular}{c c c c}
\hline
\diagbox{指标}{方法} & A-C & A-B & B-C \\
\hline
总延误(h) & 20.5\% & 7.4\% & 14\% \\
平均停车率 & 13\% & 4.8\% & 8.5\% \\
平均速度(km/h) & 11.8\% & 3.6\% & 7.9\% \\
\hline
\end{tabular}
\end{table}

由表 7 可知区域协调控制与单独控制的差异性最大,两主干道绿波控制策略与单独控制策略的差异适中,区域协调制与两主干道绿波控制的差异最小。这表明两主干道绿波能够取得一定层次的效果,由于没有完全在次干道实现绿波控制,所以其改善效果相比区域协调控制差。在一些难以实现区域控制的地方,可以将两主干道绿波控制作为其较优控制策略。

\section{六 算法的稳定性分析}

\subsection{算法的可计算性}

本文建立的微观交叉口信号控制模型为非线性整数规划问题,且规模比较小。表现形式为一个带约束的优化模型。考虑交通流的特性,本文采用了启发式算法中的遗传算法对其进行了求解,遗传算法的寻优过程不需要对象的特定知识,也不需要对象的搜索空间是连续可微的,并且具有全局寻优的能力,因此可以很好的解决该类模型。从理论上证明本文设计的算法是可行的。

\subsection{算法的复杂性}

算法的复杂性和交叉口的数量具有很大的关系,交叉口的数量小于一定数目时,建立的微观交叉口信号控制模型的自变量数目较小,理论上的解析算法复杂度也是较小的。本文设计的点控、线控和面控交叉口的数量都是有一定限制的,所以其算法复杂性不会很高。

\section{七 模型的改进与拓展}

在前面的模型中,只以延误时间最小为目标,确定最佳周期长和绿信比。它忽略了交叉口的排队长度和停车率指标,这两项对于提高交叉口的行车效率和效益都是相当重要的。尽管澳大利亚学者Akcelik引入了“停车补偿系数”,也是一种补偿行为,没有从根本上考虑停车率等一些相关指标与信号周期之间的具体关系。

由于排队长度与停车率直接相关,为了简化模型本文只考虑排队长度这一因素。利用加权系数将双目标延误时间和排队长度加权组合后转化为单目标,得到优化模型的目标函数,如下:

\begin{equation}
\begin{aligned}
\text{obj: } & \min f(x) = \rho_1 T + \rho_2 \sum_{i=1}^n H_i \\
\text{s.t. } &
\begin{cases}
x_i - a_i \geq 0 \\
\sum_{i=1}^n (x_i + \Delta t_i) = C \\
0.75 \leq \frac{y_i C}{x_i} \leq 1.2 \\
c_{\min} \leq c \leq c_{\max} \\
1 \leq i \leq N
\end{cases}
\end{aligned}
\tag{24}
\end{equation}

\section{八 模型的评价}

\subsection{模型的优点}

(1) 提高了模型的对饱和度适应能力:考虑了饱和度过高时,Webster 延迟模型将失效的状况,引入了 Akcelik 模型来克服这种状况,模型会根据当前路段的饱和度自动匹配对应的延迟模型。

(2) 采用对周期离散变量正反逐步双向调整的策略,实现了周期和绿信比进行了优化。

(3) 改善交通流量预测的精度:将混沌理论引入到短时交通流预测中,避免了时间序列算法预测精度不高,且抗干扰能力差等缺点以及卡尔曼滤波预测模型难以实现在线预测等弊端。

(4) 通过建立与双向绿波差值平方最小模型,从理论上探讨了干道实现双向绿波的可行性,通过实例验证发现,该模型是较优的。

(5) 将绿波带的思想推广到整个区域,建立了区域协调控制模型,该模型用最小的自变量和约束考虑了充足的约束,在保证模型可能性和健壮性的同时,减弱了模型的复杂性,为区域实现实时控制提供了前提。

(6) 利用元胞自动机交通运输模型,并在 VS 上实现了仿真,对我们的模型、模型的算法设计进行了检验,弥补了无法将模型应用于实际交叉口的缺陷。

\subsection{模型的缺点}

但是,我们的模型还存在一定的不足。我们只是通过预测数据来实现实时控制,而在实际的道路上是通过传感器等交通状态获取设备来获得实时地的交通数据,从而进行实时控制。但由于知识水平不足,没有给出如何运用传感器的来更好的实现实时控制。

\section{九 给交通管理部门的建议}

最早的交通信号控制是英国 20 世纪使用气动橡皮管来进行信号控制,随着微电子技术、通信技术、自动化技术的发展,交通信号控制技术也逐渐成熟起来。现在很难想象没有交通信号控制的交通将会是何种形态。

目前,交通信号控制技术可以分为定时和实时控制两种控制方式,两者都有其自身的特点。定时信号控制是指交叉口的信号周期固定的一种信号控制方式,其具有资金投入少、实施简单等优点,同时由于其信号配时方式依据交通流历史统计规律,造成其信号配时具有一定的延迟性和不确定性。实时信号控制与定时信号控制相反,信号周期不是固定的,其特点具有实时性、可变性、信号配时策略根据当前时刻交通流特性而变化。它需要相关传感器及传感器网络技术、计算机技术等支持。

本文对定时信号控制和实时信号控制分别进行了建模分析研究,得到了一些研究结果如下:

(1) 定时信号控制方式有其最大能力限定,当需求达到一定时,其效果会急剧下降,车辆在交叉口的延误、停车率和平均运行速度指标都下降。

(2) 相比定时信号控制方式,实时信号控制方式优化改善效果明显。条件允许时、成熟时,可以考虑对定时信号控制方式进行升级改造为实时信号控制方式。

(3) 在城市的主要干道,条件允许时可以采用线控,实现多交叉口的协调控制,线控产生的绿波带可以极大的改善主干道的道路服务水平。

(4) 在实现绿波控制时,优先考虑使用双向控制,当干道理论上无法实现双向绿波控制时,才考虑使用单向控制。

(5) 对于一些主要的、重要的区域,可以考虑使用面控,实现多条干道的协调控制,产生的多个绿波带,可以极大的改善整个区域的道路服务水平。

(6) 当区域控制受条件所限难以实现时,可以考虑将两主干道绿波控制作为其较优控制策略。

在本文进行实时控制时,采用了基于混沌理论的短时交通流预测方法,研究结果表明,将混沌理论应用于短时交通流预测是正确的选择,其混沌的特性正好与道路交通流混沌的特性不谋而合,相比其他方法,更能符合交通流的特性。

交通信号控制是整个智能交通系统核心组成部分,其以智能交通系统其他环节有紧密的联系。其通常是布置在上位机上,通过接收下位机传来的交通流数据进行数据融合、分析、后用于信号控制。其软件设计总体框架如图 21 所示:

\begin{figure}[h]
\centering
\includegraphics[width=0.8\textwidth]{image.png}
\caption{交通信号控制软件总体框架}
\end{figure}

其中:
主控模块式整个程序的框架,负责整个程序流程的控制与组织;
通信模块进行数据通信,负责接收与发送数据;
数据处理模块对通信模块接收到的数据进行处理,计算车流信息,判断信号机的现行状态。

数据组织模块式根据信号机的现行状态与步法号以及优化模块的运算结果对信号机的步伐及方案数据进行组织。

优化模块根据数据处理模块的处理结果,结合交叉路口当前的车流量状况和信号机的现行状态对信号机的配时方案进行优化,计算出最适合当前交通流的信号灯来控制配时方案。

显示模块将当前信号机的运行状态及通信端口的控制状态以最适合的方式在计算机屏幕上显示。

中国在这改革改放的 30 年之间得到了飞速的发展,神 7 的成功发射就是一个铁的见证。而智能交通系统(ITS)在中国的这 10 多年来,也取得了蓬勃的发展,其已经实实在在的嵌入到成百上千的城市生活中,数以亿计的人享受着他所带来的快捷和便利,奥运智能交通保障系统的成功就是一个生动的例子。而交通信号控制作为智能交通系统的核心环节,其自身也在不断的完善,我们相信中国的交通信号控制、智能交通系统的明天会更好。

\section{参考文献}

[1] 徐勋倩,黄卫,单路口交通信号多相位实时控制模型及其算法,控制理论与应用,2005年03期

[2] 陈琳,刘翔,孙优贤,单交叉路口交通流的通用多相位智能控制策略,浙江大学学报(工学版),2006年11期

[3] 李建斌,高成修,城市道路网络多交叉路口交通信号实时优化控制模型与算法,系统工程,2004年10期

[4] Road Res. Tech. Paper, No.39, Road Research Laboratory, London, 1958

[5] Nagel, K., Schreckenberg, M., A cellular automaton model for freeway traffic. Journal of Physique I, 1992, 2 (12): 2221–2228.

[6] Biham, O., Middleton, A.A., Levine, D., Self-organization and dynamical transition in traffic-flow models, Physical Review A, 1992, 46(10-15): 6124–6127.

[7] Jia, B., Jiang, R., Wu, Q.S., The traffic behaviors near an off-ramp in the cellular automaton traffic model, Physical Review E, 2004, 69 (5): 056105.

[8] Chung, K.H., Hui, P.M., Gu, G.Q., Two-dimensional traffic flow problems with faulty traffic lights, Physical Review E, 1995, 51(1): 772–774.

[9] 贾斌,高自友,李克平,李新刚. 基于元胞自动机在交通系统建模与模拟. 科学出版社. 2007.

[10] 翟忠民. 道路交通组织优化. 人民交通出版社. 2004.