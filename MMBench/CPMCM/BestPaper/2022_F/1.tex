\begin{center}
\includegraphics[width=0.2\textwidth]{image1.png} \quad
\includegraphics[width=0.2\textwidth]{image2.png} \quad
\includegraphics[width=0.2\textwidth]{image3.png} \quad
\includegraphics[width=0.2\textwidth]{image4.png}
\end{center}

\begin{center}
\textbf{中国研究生创新实践系列大赛} \\
\textbf{中国光谷·“华为杯”第十九届中国研究生} \\
\textbf{数学建模竞赛}
\end{center}

\begin{table}[h]
\centering
\begin{tabular}{ll}
学校 & 辽宁工程技术大学 \\
\hline
参赛队号 & 22101470006 \\
\hline
队员姓名 & 1. 高营辉 \\
& 2. 苑祎菲 \\
& 3. 贾策 \\
\end{tabular}
\end{table}

\begin{center}
\textbf{中国研究生创新实践系列大赛}
\end{center}

\begin{center}
\textbf{中国光谷·“华为杯”第十九届中国研究生}
\end{center}

\begin{center}
\textbf{数学建模竞赛}
\end{center}

\begin{flushleft}
题目 \quad COVID-19 疫情期间生活物资的科学管理问题
\end{flushleft}

\begin{center}
\textbf{摘 \quad 要:}
\end{center}

近年来,新冠疫情突发公共卫生事件发生的频率越来越高。为减轻疫情带来的损害,防止疫情扩散,对于大规模居家隔离人群的科学管理变得尤为重要,与此同时,需要积极开展应急救援工作,为事发区域有效及时的提供大量的应急物资。本文利用数据驱动的机器学习方法,分别构建多目标最大覆盖选址模型和基于储备中心-集散点-小区三级拓扑网络的生活物资多周期分配优化模型,从而在全面提高疫情防控效率的同时,努力节约相关部门的人员投入与经费支出,为疫情期间生活物资的科学管理问题提供有力的策略,具有一定的现实意义。

针对问题一,对附件数据以及查阅的资料整理并可视化,对实行发放蔬菜包前后效果进行分析、判别。首先利用 Origin 软件对长春市感染人数、蔬菜包的储存量、出库量、进货量以及价格进行分析,从而初步判断长春市各区蔬菜包的投放时间为 3 月 28 日。其次,考虑投放蔬菜包的影响,构建 LSTM 循环神经网络模型对未投放蔬菜包时感染人数进行预测,并与实际数据对比验证了蔬菜包投放时间对感染人数的影响。最后,利用 ploy2D 非线性曲面拟合蔬菜包总储存量、单价与感染人数的关系式,并利用 LM 最小二乘算法不断迭代优化参数,经 F 值和残差分验证了拟合效果显著,进而得到投放蔬菜包对疫情防控具有积极作用。

针对问题二,基于数据驱动的机器学习方法,构建多目标最大覆盖选址模型。首先结合附件 3、4 中数据,建立影响投放点数量的评价指标体系,并基于熵权-模糊综合评价模型得到长春市 9 个区投放点数量分布合理性的评分,进而评价 9 个区投放点数量设置的合理性。针对不合理的区域,提出比例-熵权法对投放点数量进行优化调整。其次,采用系统聚类算法分别从长春市各分区以及整个长春市角度分析确定长春市设置储备中心个数,累计为 36 个聚类中心。最后,考虑未来疫情、自然灾害等不确定因素,建立多目标最大覆盖选址模型,并利用遗传算法求解模型,获得了 20 个最优的政府大规模储备中心点的位置、选址半径、管辖范围小区个数、管辖范围内人口数以及其潜在的备用场所位置。

针对问题三,分析蔬菜包需求、发放规律,结合小区位置、人口信息等影响因素,给出最优蔬菜包供应方案。首先,利用 Origin 软件将 3 月 26 日-5 月 1 日长春市 9 个区的蔬菜包库存量、接收与投放量进行可视化,并结合安全库存理论分析各区蔬菜包接收、发放规律。其次,结合附件 3 信息利用数据包络分析法建立基于决策偏好的超效率 DEA

评价模型,对长春市蔬菜包分配方案进行评价,评价结果为“不合理”。最后,加入“投影”约束条件,建立基于投影分析的 DEA 反馈调整模型对 4 月 10 日至 4 月 15 日长春市 9 个区的蔬菜包供应方案进行调整。

针对问题四,考虑不同约束下,建立基于三级拓扑网络的生活物资多周期分配优化模型。首先,构建储备中心-集散点-小区的三级有序拓扑网络,并在第二、三问的基础上,建立基于三级拓扑网络的生活物资多周期分配优化模型。其次,考虑不同蔬菜品种集和最优集散地数量确定基于最短距离的多周期动态物资供应方案。接着,考虑长春市真实街道情形,构建基于蚁群算法的物资配送路径规划模型,利用自适应遗传算法求解模型得到各区储备中心到各小区的三级最优路径,并计算最优行驶路径长度,将其应用于多周期分配优化模型得到合理的物资供应方案。最后,考虑最小配送时间的目标,基于不同卡车承载规模约束,利用“顺路”原则将物资零散配送到多个需求小区得到不同物资分配方案,并给出多周期动态物资供应方案的优越性。

最后,对模型的优缺点进行了评价和推广,并对重要参数进行了灵敏度分析,我们发现储备中心服务半径的大小、储备中心的个数、单位运输成本等都会影响疫情控制效果。当因素发生扰动时,模型能够保证鲁棒性,进一步说明本文的合理性与有效性。

关键词:LSTM 循环神经网络;熵权-模糊综合评价模型;超效率 DEA 评价模型;投影反馈调整模型;多目标最大覆盖选址模型;蚁群优化算法;自适应遗传算法

\section*{一、问题重述}

\subsection{1.1 问题背景}

自 2022 年以来全国范围内接连出现多起较大规模疫情爆发事件。我国主要采取封闭式管理方式来控制疫情的发展,从而实现疫情清零。因此,疫情期间各类人群的科学管理变得尤为重要。目前,疫情期间对确诊病例、无症状感染、密切接触者采用集中治疗或隔离处理手段,对于此类人群容易实现科学管理。但是,对于大量分散居家隔离人员统一管理成为疫情防控管理工作中的新难点问题。例如:疫情期间封闭管理,居民的生活物质如何科学合理的发放,发放过程中可能存在疫情二次传播。对于行动不便、信息化水平较低的特殊群体的管理会消耗大量的人力资源。在疫情期间援助与医务人员的服务对疫情的防控与治疗起到无可替代的作用。因此,解决疫情期间生活物资的科学管理问题,在全面提高疫情防控效率的同时,努力节约相关部门的人员投入与经费支出,并为今后的应急管理方案打下坚实的理论与实践基础,具有十分重要的意义。

\subsection{1.2 问题提出}

基于上述问题背景与提供的附件数据,本文需要研究解决以下问题:

\textbf{问题一:各种生活物资的大规模流动方式对疫情的影响}

根据附件 1 中所提供的长春市 COVID-19 疫情期间病毒感染人数数据分析发展规律,并对长春市实行发放蔬菜包前后效果进行判别与分析。

\textbf{问题二:生活物资投放点数量与位置问题}

根据附件 3、附件 4 中相关数据,讨论附件 2 中提供的长春市不同区域投放点数量分布的合理性。考虑未来疫情、自然灾害等特殊事件,对于政府储备物资和大规模物资分拣场所的位置与数量规模进行合理规划,并提出最优的选址数量、规模及其潜在的备用场所位置。

\textbf{问题三:生活物资的科学发放问题}

根据附件 5 分析蔬菜包需求、发放规律,并根据附件 3 中的各小区位置与人口信息,评价并调整 4 月 10 日至 4 月 15 日蔬菜包供应方案。

\textbf{问题四:为长春市做好大规模封控情况下居民生活物资有序发放预案}

基于第二、三问的基础上,根据附件 3 给出的长春市街道和小区情况的表格,做出特殊时期保障居民生活物资供应的详细预案(有序网络图)。

\section*{二、问题分析}

\subsection{2.1 针对问题一}

本题首先利用 Origin 软件对长春市感染人数、蔬菜包的储存量、出库量、进货量以及价格进行分析,从而初步判断蔬菜包的投放时间。其次,考虑投放蔬菜包的影响,基于 LSTM 循环神经网络算法对未投放蔬菜包时感染人数进行预测,并与实际数据对比验证蔬菜包投放时间和对感染人数的影响。最后,利用 ploy2D 非线性曲面拟合蔬菜包总储存量、单价与感染人数的关系式,并利用 LM 最小二乘算法不断迭代优化参数,进而得到投放蔬菜包对疫情防控具有积极作用。

\subsection{2.2 针对问题二}

本题首先结合附件 3、4 中数据,构建熵权-模糊综合评价模型对长春市 9 个区的投放点数量合理性进行评价,并提出比例-熵权法对投放点数量进行优化调整。其次,采用系

统聚类算法确定长春市 9 个区设置储备中心个数。最后,考虑未来疫情、自然灾害等因素,建立多目标最大覆盖选址模型,并利用遗传算法求解模型,获得最优的选址数量、规模及其潜在的备用场所位置。

\subsection{2.3 针对问题三}

本题首先利用 Origin 软件将 3 月 26 日-5 月 1 日长春市 9 个区的蔬菜包库存量、接收与投放量进行可视化,并结合安全库存理论分析各区蔬菜包接收、发放规律。其次,结合附件 3 信息利用数据包络分析法建立基于决策偏好的超效率 DEA 评价模型,对长春市蔬菜包分配方案进行评价。最后,加入“投影”约束条件,建立基于投影分析的 DEA 反馈调整模型对 4 月 10 日至 4 月 15 日长春市 9 个区的蔬菜包供应方案进行调整。

\subsection{2.4 针对问题四}

本题首先构建储备中心-集散点-小区的三级有序拓扑网络,并在第二、三问的基础上,建立基于三级拓扑网络的生活物资多周期分配优化模型。其次,考虑不同蔬菜品种集和最优集散地数量确定基于最短距离的多周期动态物资供应方案。接着,考虑长春市真实街道情形,构建基于蚁群算法的物资配送路径规划模型,利用自适应遗传算法求解模型得到各区储备中心到各小区的三级最优路径,并计算最优行驶路径长度,将其应用于多周期分配优化模型得到合理的物资供应方案。最后,考虑最小配送时间目标,基于不同卡车承载规模约束,利用“顺路”原则将物资零散配送到多个需求小区得到不同物资分配方案。

\section{三、模型假设}

考虑到实际问题,本文做如下假设:

\begin{itemize}
    \item 假设附件提供的数据是真实可靠的;
    \item 假设不考虑出生率,死亡率以及人口流动对需求造成的影响;
    \item 假设数据中个别缺失数据对结果不会产生重大影响
\end{itemize}

\section{四、符号说明}

在本文中,我们在模型构造中使用了下表中的术语,其他非常用的符号将在使用后被引入。

\begin{table}[h]
\centering
\begin{tabular}{c c l}
\hline
序号 & 符号 & 符号说明 \\
\hline
1 & $W_{f}$ & 遗忘门的权重矩阵 \\
2 & $b_{f}$ & 遗忘门的偏置项 \\
3 & $W_{i}$ & 输入门的权重矩阵 \\
4 & $b_{i}$、$b_{c}$ & 输入门的偏置项 \\
5 & $f_{t}$ & 遗忘门的输出 \\
6 & $p$ & 新冠感染人数 \\
7 & $p'$ & 感染人数归一化 \\
8 & $X_{\text{max}}$ & 新冠感染人数最大值 \\
9 & $E$ & 模型损失度 \\
10 & $Q_{e}$ & 平方的误差和 \\
11 & $Q_{r}$ & 平方的回归和 \\
12 & $w_{j}$ & 各因素评价权重 \\
13 & $x_{k}^{\text{min}}$ & 第 $j$ 个指标下在不同评价对象下的最小值 \\
\hline
\end{tabular}
\end{table}

\begin{tabular}{c c l}
14 & $x_{k}^{\text{max}}$ & 第j个指标下在不同评价对象下的最大值 \\
15 & $d_{ij}$ & 储备中心$i\in I$到集散地$j\in J$的最短距离 \\
16 & $d_{jm}$ & 集散地$i\in I$到小区$j\in J$的最短距离 \\
17 & $I$ & 物资储备中心集合 \\
18 & $J$ & 物资集散点集合 \\
19 & $M$ & 小区集合 \\
20 & $N$ & 应急蔬菜物资品种集合 \\
\end{tabular}

\section*{五、问题一模型的建立与求解}

随着新冠疫情在全球爆发,城市疫情防控工作尤为重要。有人提出一种观点认为城市疫情的发展或被控制扑灭与生活物资发放方式有关。问题一本质是数据挖掘与分析,要求结合附件 1 中所提供的长春市 COVID-19 疫情期间病毒感染人数数据对长春市实行发放蔬菜包前后效果进行判别与分析,得到蔬菜包投放与城市新冠感染人数的关系为今后城市疫情防控工作提供帮助。首先,利用 Origin 软件分析长春市和 9 个区的新冠感染人数及无症状感染人数的发展规律。其次,分析了蔬菜包总储存量、日总存储量、日出库量、日进货量三者的发展规律,确定了蔬菜包投放的时间为 3 月 28 日。基于 LSTM 循环神经网络对未投放蔬菜包时新冠感染人数进行了预测,确定了蔬菜包投放对感染人数存在影响。最终,利用 ploy2D 函数对蔬菜包总储存量、单价与感染人数进行非线性曲线拟合,利用 Levenberg-Marquardt 最小二乘优化算法进行不断迭代得到最优参数。根据 F 值与残差分析验证了拟合效果显著。

\begin{figure}[h]
    \centering
    \includegraphics[width=\textwidth]{image1.png}
    \caption{问题一思路流程图}
    \label{fig:flowchart}
\end{figure}

\subsection{长春市新冠感染人数与蔬菜包数据分析}

根据附件 1 将长春市 3 月 4 日到 5 月 6 日期间的新增无症状感染人数与新增感染人数进行了统计,利用 Origin 软件进行了可视化处理,如图 5-2 所示。从图 5-2 可知,长春市的新冠感染人数在 3 月 12 日、3 月 14 日、3 月 22 日、4 月 1 日出现了峰值,分别为 831 人、1190 人、1979 人、1544 人。新增无症状感染人数在 4 月 2 日出现了最高值,人数达到 3100 人。由此可知,长春市新冠疫情突然爆发,在 3 月 12 日到 4 月 1 日期间出现了 4 次峰值。无症状感染人数突然爆发与新增感染人数爆发间隔了大约一周时间。

\begin{figure}[h]
    \centering
    \includegraphics[width=\textwidth]{image1.png}
    \caption{长春市新增感染人数、无症状感染人数与时间的关系}
    \label{fig:5-2}
\end{figure}

图 5-3 为长春市 9 区的新增感染人数与时间的关系。从图 5-3 可以看出,在 3 月 29 日到 4 月 1 日期间,9 个区的新冠感染人数出现倒 “V” 型。其中长春新区感染人数最多,感染人数达到了 382 人、716 人。

\begin{figure}[h]
    \centering
    \includegraphics[width=\textwidth]{image2.png}
    \caption{长春市 9 区的新增感染人数与时间的关系}
    \label{fig:5-3}
\end{figure}

将附件 4 中蔬菜总储备量、日出库量、日进货量进行统计,并利用 origin 软件制作成瀑布图,如图 5-4 所示。从图 5-4 可以看出,在 3 月 16 日到 3 月 26 日期间,蔬菜总储备量呈现下降的趋势,但是幅度不大。蔬菜日出库量、日进货量均呈现平稳的变化趋势。由于图 5-2、图 5-3 可知,长春市最早出现新冠疫情的时间为 3 月 12 日,在 3 月 29 到 4 月 1 日期间长春市出现了疫情最严重的时刻。长春市的蔬菜储备量针对前期小规模疫情爆发具有一定可控性。然而,在疫情爆发最严重阶段,在 3 月 26 日蔬菜总储备量突然下降,日出库量当天上升,日进货量过两天上升。由于疫情的爆发,市民大量采购蔬菜,导致长春市蔬菜总储备量出现突然下降,日出库量与日进货量分别在不同时间段发生上升。在 3 月 28 日到 4 月 21 日期间,蔬菜总储备量整体呈波动变化,上升-下降-上升的变化趋势。日出库量与日进货量也呈现相同的变化趋势。此时,长春市政府出台蔬

\begin{figure}[h]
    \centering
    \includegraphics[width=\textwidth]{image1.png}
    \caption{长春市蔬菜总存储量、日出库量、日进货量三者的关系}
    \label{fig:5-4}
\end{figure}

由于上述分析可知,蔬菜总存储量、日出库量、日进货量与疫情防控存在某种联系。为了深入探究这种关系我们将蔬菜包中各类蔬菜的价格进行了统计,如图 \ref{fig:5-5} 所示。从图 \ref{fig:5-5} 可以看出,在 3 月 26 日前蔬菜包中的 8 种蔬菜的价格均呈现平稳变化,价格增幅不超过 25\%。其中大白菜与青椒的价格波动比较大,由于两类蔬菜存放时间较短,因此价格波动较大。但是 8 种蔬菜价格整体呈现平稳趋势。在 3 月 26 日之后,8 种蔬菜的价格出现突然上升的的趋势。这种变化规律解释了,图 3 中的变化规律。根据蔬菜价格突然上升可以推测出长春市疫情的突然爆发,导致市民的恐慌对蔬菜的急切购买。长春市的蔬菜总储量急剧下降,市民需求量大,因此蔬菜单价上升。3 月 26 日到 4 月 21 日期间,由于长春市政府出台了疫情防控政策,对市民与蔬菜保供企业进行了管理,此时蔬菜呈现波动起伏的变化趋势。在 4 月 21 日之后 8 种蔬菜基本上回归与 3 月 28 日之前的价格,由此可见,长春市疫情控制得到了有效控制。

\begin{figure}[h]
    \centering
    \includegraphics[width=\textwidth]{image2.png}
    \caption{蔬菜包中各类蔬菜价格波动图}
    \label{fig:5-5}
\end{figure}

综上所述,根据图 5-4 可知,长春市蔬菜存储量在 3 月 26 号出现下降,由此推测长春市政府在 26 号开始准备投放蔬菜包。3 月 28 日长春市蔬菜存储量出现上升趋势,由此可以确定蔬菜包投放时间为 3 月 28 日。蔬菜包投放前,3 月 28 日前,根据图 5-2、图 5-3 可知,长春市新冠感染人数出现 4 次峰值的增加。在蔬菜包投放之后,长春市新冠感染人数上升的波动得到控制,在一段时间之后,新冠感染人数逐渐下降。最终,确定长春市新冠感染人数与蔬菜存储量、日出库量、日进货量、单价存在某种关系。

\subsection*{5.2 基于 LSTM 循环神经网络的长春市新冠感染人数预测}

根据图 5-2、图 5-3 可知,长春市新冠感染人数的变化随时间的变化而变化,并具有一定周期性,其数据类型属于时间序列数据。针对长春市新冠感染人数进行预测,本文采用深度学习算法中的 LSTM 循环神经网络。LSTM 循环神经网络以 RNN 循环神经网络为基础上提出的一种特殊的神经网络。相比较 RNN 传统神经网,解决了梯度消散与爆炸的问题。LSTM 主要针对 RNN 的隐藏层进行改进,设计了一个记忆结构,实现选择记忆功能。其结构包含 3 个控制门单元函数和 1 个神经元状态更新函数。LSTM 神经网络通过遗忘门 (Forget gate)、输入门 (Input gate) 和输出门 (Output gate) 控制信息,神经元状态更新函数协调控制更新整个神经网络中的信息的流向,最终实现网络功能,如图 5-6 所示。

\begin{figure}[h]
    \centering
    \includegraphics[width=\textwidth]{lstm_structure.png}
    \caption{LSTM 神经网络结构图}
    \label{fig:lstm_structure}
\end{figure}

遗忘门的作用是控制整个神经网络选择性舍弃信息:
\begin{equation}
f_{t} = \sigma \left( W_{f} \cdot \left[ h_{t-1}, x_{t} \right] + b_{f} \right)
\tag{5.1}
\end{equation}
式中:$W_{f}$ 为遗忘门的权重矩阵;$h_{t-1}$ 为 $t-1$ 时刻的输出向量;$x_{t}$ 为 $t$ 时刻输入向量;$b_{f}$ 为遗忘门的偏置项;$\sigma$ 为激活函数。

输入门的作用是有选择性的将神经元信息更新到细胞状态中,并储存在细胞状态:
\begin{equation}
i_{t} = \sigma \left( W_{i} \cdot \left[ h_{t-1}, x_{t} \right] + b_{i} \right)
\tag{5.2}
\end{equation}
\begin{equation}
\tilde{C}_{t} = \tanh \left( W_{C} \cdot \left[ h_{t-1}, x_{t} \right] + b_{C} \right)
\tag{5.3}
\end{equation}
式中:$W_{i}$ 为输入门的权重矩阵;$b_{i}、b_{c}$ 为输入门的偏置项;$\sigma、\tanh$ 为输入门的激活函数。

神经元状态作用是更新神经元状态,协助遗忘门与输入门流入信息:
\begin{equation}
C_{t} = f_{t} * C_{t-1} + i_{t} * \tilde{C}_{t}
\tag{5.4}
\end{equation}
式中:$f_{t}$ 为遗忘门的输出;$i_{t}$ 是流入的新信息;$\tilde{C}_{t}$ 是新数据形成的控制参数,最终实现状态的更新。

输出门的作用是根据细胞状态得到最终输出信息:

\begin{equation}
o_{t}=\sigma\left(W_{o}\left[h_{t-1}, x_{t}\right]+b_{o}\right)
\tag{5.5}
\end{equation}

\begin{equation}
h_{t}=o_{t} * \tanh \left(C_{t}\right)
\tag{5.6}
\end{equation}

式中:$W_{o}$ 为输出门的权重矩阵;$b_{o}$ 为输出门的偏置项;$h_{t}$ 为神经网络 t 时刻的输出值、t+1 时刻的输入值。

\subsection*{5.2.1 LSTM 模型构建}

本文以 LSTM 循环神经网络为基础对未发送蔬菜包时长春市新冠疫情影响人数预测模型进行搭建,具体流程图如图 5-7 所示。首先,确定 LSTM 循环神经网络的结构和初始参数。其次,将数据的 80% 划分为训练集,20% 作为预测集。为避免选取的输入数据的数量量纲不一致而导致模型在训练时小数量级的特征值对结果的影响程度被大数量级特征值弱化的情况,因此,针对数据进行归一化处理,max-min 归一化计算公式如下所示:

\begin{equation}
p^{\prime}=a+\frac{\left(p-X_{\min }\right) \cdot\left(b-a\right)}{X_{\max }-X_{\min }}
\tag{5.7}
\end{equation}

式中 $p$ 为新冠感染人数;$p^{\prime}$ 为感染人数归一化结果;$X_{\max }$ 为新冠感染人数最大值;$X_{\min }$ 为新冠感染人数最小值;a、b 为映射边界。

利用 Adam 梯度下降法计算并不断调整 LSTM 模型中的适应参数。Adam 梯度下降法不仅提高拟合率而且可以保障各神经元的权重与偏置不会陷入局部最优解。利用均方误差来评价预测结果与实际目标结果的损失度,其计算公式如下所示:

\begin{equation}
E=\frac{\sum_{i=1}^{n}\left(f(x)-y\right)^{2}}{n}
\tag{5.8}
\end{equation}

式中 $E$ 为模型损失度;$f(x)$ 为模型预测值;$y$ 为期望目标;$n$ 为模型自由度。

\begin{figure}[h]
    \centering
    \includegraphics[width=0.8\textwidth]{lstm_flowchart.png}
    \caption{LSTM 循环神经网络模型流程图}
    \label{fig:lstm_flowchart}
\end{figure}

\subsection*{5.2.2 结果分析}

通过 Matlab 软件中的 TrainNetwork 方法构建了 LSTM 循环神经网络,根据上述步骤将求解器设置为 Adam。输入的数据的维度为一维,训练了 1000 轮,迭代了 250 次。LSTM 网络设置了 25 个隐含单元数。经过 250 次迭代数据误差达到 0.1 左右,目标值损失值达到 0 左右,LSTM 循环神经网络训练过程如图 5-8 所示。

\begin{figure}[h]
    \centering
    \includegraphics[width=\textwidth]{image1.png}
    \caption{LSTM 网络训练过程}
    \label{fig:5-8}
\end{figure}

从图 5-9 可以看出,长春市未发送蔬菜包时新冠感染人数,呈先增加在降低的变化趋势。在 30 天左右出现峰值,在 100 天时感染人数降低为最小。发放蔬菜包后的新冠感染人数峰值在 1900 人左右,未发送蔬菜包时感染人数峰值为 2600 左右。由此可见,长春市政府发放蔬菜包可以降低新冠感染人数,降低幅度为 36.8\%。发放蔬菜包时新冠感染人数在投放蔬菜包之后,感染人数得到一定控制,但是存在一定时间的延迟。

\begin{figure}[h]
    \centering
    \includegraphics[width=\textwidth]{image2.png}
    \caption{未发放蔬菜包与发送蔬菜包人数对比图}
    \label{fig:5-9}
\end{figure}

\subsection{5.3 蔬菜包存储量、单价与感染人数的关系}

根据上述分析可知,蔬菜包的投放对长春市新冠感染人数存在某种关系。蔬菜包存储量与单价会影响蔬菜包的投放。因此利用 origin 软件采用 ploy2D 函数对蔬菜包存储量、单价与感染人数进行非线性曲面拟合,其函数公式为:

\begin{equation}
z = z0 + ax + by + cx^2 + dy^2 + fxy
\tag{5.9}
\end{equation}

利用 Levenberg-Marquardt 最小二乘优化算法进行不断迭代得到最优参数。LM 算法使用了一种带阻尼的高斯-牛顿方法。将变量行走的长度 \( h \) 控制在一定的信赖域之内,保证泰勒展开出现一个最优的近似效果。最终得到拟合函数和拟合曲线如下所示:

\[
z = -1975 + 0.0945x + 597.5y - x^2 - 47.45y^2 + 0.01088xy
\]

根据统计学计算原理,统计量 \( F \) 值测试拟合回归模型的显著性,并将计算出的 \( F \) 值与查找表获得的显著性水平 \( \alpha \) 的下临界值 \( F_{\alpha}(k, n-k-1) \) 进行比较。

\begin{equation}
F = \frac{Q_{\text{r}}/k}{Q_{\text{e}}/(n-k-1)}
\tag{5.10}
\end{equation}

其中 \( Q_{\text{r}} \) 是平方的回归和;\( Q_{\text{e}} \) 是平方的误差和。

根据表 5.1 数值可知,查阅 \( F \) 值,满足上述公式要求,拟合回归方差效果显著。

\begin{table}[h]
\centering
\caption{新冠感染人数方差分析表}
\begin{tabular}{l c c c c}
\hline
 & DF & 平方和 & 均方 & F 值 & 概率 > F \\
\hline
回归 & 8 & 1.63445E7 & 2.04306E6 & 15.12791 & 1.99394E-10 \\
残差 & 50 & 6.75263E6 & 135052.57522 & & \\
未修正的整体 & 58 & 2.30971E7 & & & \\
修正的整体 & 57 & 1.24579E7 & & & \\
\hline
\end{tabular}
\end{table}

图 5-10 为拟合函数的三维曲面图,从图 5-10 可以看出,随着蔬菜单价的增加,新增本土感染人数呈现增大的变化趋势。随着蔬菜存储量的增大,新增本土感染人数呈现降低的变化趋势。由此可见,蔬菜单价升高导致,市民恐慌,市民大量外出采购蔬菜导致感染人数呈现增加的变化趋势。市蔬菜储存量增加,城市蔬菜富足,可以控制管理好市民生活,本土感染人数发生下降。也由此可以推断,城市的蔬菜储量上升,疫情得到稳定控制。图 5-11 为原数据与拟合数据的三维散点图。从图 5-11 可以看出,在蔬菜储量 4000-12000 吨,蔬菜单价在 4-6 元时,拟合效果最优。

\begin{figure}[h]
\centering
\includegraphics[width=0.45\textwidth]{image1.png}
\caption{拟合曲面展示图}
\end{figure}
\begin{figure}[h]
\centering
\includegraphics[width=0.45\textwidth]{image2.png}
\caption{原数据与拟合数据三维散点图}
\end{figure}

为更好验证拟合效果,对拟合得到的残差进行分析,利用 Origin 软件做出残差正态 Q-Q 图,如图 5-12 所示。从图可以看出,期望值均分布在上下线百分位数之间。残差符合要去,更加验证拟合效果显著。

\begin{figure}[h]
    \centering
    \includegraphics[width=\textwidth]{image.png}
    \caption{残差正态Q-Q图}
    \label{fig:qq_plot}
\end{figure}

根据全市及9个区新冠感染人数、无症状感染人数的发展规律,长春市蔬菜总存储量、日出库量、日进货量、单价的变化规律,综合判断了蔬菜包投放时间在3月26日。针对蔬菜总存储量、单价与感染人数的关系进行了拟合,通过F值、残差确定了拟合效果显著。未今后疫情防控给出建议,蔬菜总存储量维持在4000-12000吨,蔬菜单价在4-6元时,对于长春市疫情防控效果最佳。

\section*{六、问题二模型的建立与求解}

考虑到疫情初期需要大量的人力资源又要降低人员流动接触,因此,疫情爆发初期长春市政府将居民生活物资放到投放点进行发放。问题二要求结合附件 3、4 中数据对投放点数量的合理性进行讨论,其问题本质属于评价类建模。因此,本节建立了熵权一模糊综合评价的模型对长春市 9 个重点区的投放点数量进行评价。在评价模型之后,提出比例-熵权法,对长春市 9 个重点区的投放点进行重新设计。问题二还要求对政策政府储备物资和大规模物资分拣场所的位置与数量规模进行合理规划,并提出最优的选址数量、规模及其潜在的备用场所位置。其问题本质是聚类,选址问题。首先,根据系统聚类确定了长春市 9 个重点区一共需要 36 个储量中心。其次,考虑未来疫情、自然灾害等因素,建立了遗传算法优化求解多目标函数模型,确定了长春市 9 个重点区一共需要 20 个储量中心,及备用储备中心的位置。

\begin{figure}[h]
    \centering
    \includegraphics[width=\textwidth]{image1.png}
    \caption{第二题思路流程图}
    \label{fig:flowchart}
\end{figure}

\subsection{评价指标因素集的建立}

根据附件 3 我们发现对于长春市不同区域投放点数量分布的合理性的影响主要分为三个方面,即人口、小区、道路。人口划分为隔离人数、小区人数、每日新增人数。小区划分为栋数、户数、位置。道路划分为长度、宽度、密度。使用双层模糊综合评价方法对长春市不同区域投放点数量分布的合理性进行评价,指标评价结构如图 6-2 所示。

\begin{figure}[h]
    \centering
    \includegraphics[width=\textwidth]{image.png}
    \caption{双层模糊综合评价结构}
    \label{fig:evaluation_structure}
\end{figure}

根据上述双层模糊综合评价结构,对指标进行建立。建立第一层因素集 \( U = \{U1, U2, U3\} = \{\text{人口}, \text{小区}, \text{道路}\} \)。建立第二次因素集 \( U1 = \{U11, U12, U13\} = \{\text{隔离人口数}, \text{感染人口数}, \text{小区人口数}\} \)、\( U2 = \{U21, U22, U23\} = \{\text{小区栋数}, \text{小区户数}, \text{小区位置}\} \)、\( U3 = \{U31, U32, U33\} = \{\text{道路宽度}, \text{道路长度}, \text{道路密度}\} \)。

\subsection{评价指标权重确定}

我们对长春市不同区域投放点数量分布的合理性采用5分制进行评价。如表\ref{tab:evaluation_terms}所示。设置对应的评价等级分别为 \( V1, V2, V3, V4, V5 \),其对应的得分分别为 \( 5, 4, 3, 2, 1 \)。即 \( V = \{V1, V2, V3, V4, V5\} = \{\text{非常满意}, \text{满意}, \text{一般}, \text{不满意}, \text{非常不满意}\} = \{5, 4, 3, 2, 1\} \)。

\begin{table}[h]
    \centering
    \caption{双层模糊综合评语及解释}
    \label{tab:evaluation_terms}
    \begin{tabular}{c c c}
        \hline
        评语等级 & 模糊评语 & 评语解释 \\
        \hline
        V1 & 非常合理 & 投放点数量非常合理 \\
        V2 & 合理 & 投放点数量合理 \\
        V3 & 一般 & 投放点数量适中 \\
        V4 & 不合理 & 投放点数量不合理 \\
        V5 & 非常不合理 & 投放点数量严重不合理 \\
        \hline
    \end{tabular}
\end{table}

对于长春市不同区域投放点数量分布的合理性邀请了5位业内专家对其进行打分,整个打分过程选用10分制打分,评分的基本原则为:10-9分为非常满意;8-7分为满意;6-5分为一般;4-3分为不满意;2-1分为非常不满意。得出的评分按表\ref{tab:expert_scores}的评价指标进行分类,利用熵权法对专家的评分进行计算,求出各个指标的权重。

\begin{table}[h]
    \centering
    \caption{二级指标专家打分表}
    \label{tab:expert_scores}
    \begin{tabular}{c c c c c c c c}
        \hline
        隔离人口数 & 感染人口数 & 小区人口数 & 小区栋数 & 小区户数 & 小区位置 & 道路宽度 & 道路长度 & 道路密度 \\
        \hline
        9 & 7 & 6 & 4 & 8 & 1 & 7 & 6 & 3 \\
        7 & 8 & 3 & 10 & 8 & 9 & 9 & 10 & 5 \\
        4 & 8 & 6 & 9 & 5 & 10 & 4 & 6 & 3 \\
        5 & 9 & 7 & 5 & 3 & 6 & 1 & 4 & 8 \\
        8 & 3 & 9 & 9 & 2 & 2 & 1 & 3 & 10 \\
        \hline
    \end{tabular}
\end{table}

(1) 定各指标的原始矩阵:

在嫡权法赋权的过程中,假设有 $m$ 个评价对象,$n$ 个评价指标,形成一个 $m*n$ 的原始矩阵 $X=(x_{ij})_{m\times n}$。为在第 $J$ 个评价对象下的第 $k$ 个评价指标的数值。将专家的评分作为原始矩阵进行计算。

\begin{equation}
x=\left(\begin{array}{ccc}
x_{11} & \cdots & x_{1n} \\
\vdots & \ddots & \vdots \\
x_{m1} & \cdots & x_{mn}
\end{array}\right), j=1,2,3,\cdots,m; i=1,2,3,\cdots,n
\tag{6.1}
\end{equation}

根据公式(1)及表 2 分别得到人口,小区,道路的二级原始矩阵。

\[
x_{1}=\left[\begin{array}{lll}
9 & 7 & 6 \\
7 & 8 & 3 \\
4 & 8 & 6 \\
5 & 9 & 7 \\
8 & 3 & 9
\end{array}\right] \quad x_{2}=\left[\begin{array}{lll}
4 & 8 & 1 \\
10 & 8 & 9 \\
9 & 5 & 10 \\
5 & 3 & 6 \\
9 & 2 & 2
\end{array}\right] \quad x_{3}=\left[\begin{array}{lll}
7 & 6 & 3 \\
9 & 10 & 5 \\
4 & 6 & 3 \\
1 & 4 & 8 \\
1 & 3 & 10
\end{array}\right]
\]

(2)矩阵归一化

因各评价指标之间在性质、数量级等方面存在差异,将原始矩阵进行归一化处理得到无量纲化矩阵 $Y=(y_{jk})_{m*n}$,可以有效消除这些特征之间的差异性。

\begin{equation}
y=\frac{x_{ik}-x_{k}^{\min }}{x_{k}^{\max }-x_{k}^{\min }}, y \in[0,1]
\tag{6.2}
\end{equation}

式中 $x_{k}^{\max }$ 是第 $j$ 个指标下在不同评价对象下的最大值;$x_{k}^{\min }$ 为第 $j$ 个指标下在不同评价对象下的最小值。

\begin{equation}
y=\left(\begin{array}{ccc}
y_{11} & \cdots & y_{1n} \\
\vdots & \ddots & \vdots \\
y_{m1} & \cdots & y_{mn}
\end{array}\right), j=1,2,3,\cdots,m; i=1,2,3,\cdots,n
\tag{6.3}
\end{equation}

由上述公式对上述进行归一化处理:

\[
y_{1}=\left[\begin{array}{ccc}
1 & 0.67 & 0.50 \\
0.60 & 0.83 & 0 \\
0 & 0.83 & 0.50 \\
0.20 & 1 & 0.67 \\
0.80 & 0 & 1
\end{array}\right] \quad y_{2}=\left[\begin{array}{ccc}
0 & 1 & 0 \\
1 & 1 & 0.8889 \\
0.8333 & 0.5000 & 1 \\
0.1667 & 0.1667 & 0.5556 \\
0.8333 & 0 & 0.1111
\end{array}\right] \quad y_{3}=\left[\begin{array}{ccc}
0.7500 & 0.4286 & 0 \\
1 & 1 & 0.2857 \\
0.3750 & 0.4286 & 0 \\
0 & 0.1429 & 0.7143 \\
0 & 0 & 1
\end{array}\right]
\]

(3)确定各评价指标的权重

\begin{equation}
p_{jk}=\frac{y_{jk}}{\sum_{j=1}^{m} y_{jk}}
\tag{6.4}
\end{equation}

由上述公式计算得到各指标的权重:

\begin{align*}
y_{1}^{'} &=
\begin{bmatrix}
0.3846 & 0.2000 & 0.1875 \\
0.2308 & 0.2500 & 0 \\
0 & 0.2500 & 0.1875 \\
0.0769 & 0.3000 & 0.2500 \\
0.3077 & 0 & 0.3750
\end{bmatrix}
\quad
y_{2}^{'} =
\begin{bmatrix}
0 & 0.3750 & 0 \\
0.3529 & 0.3750 & 0.3478 \\
0.2941 & 0.1875 & 0.3913 \\
0.0588 & 0.0625 & 0.2174 \\
0.2941 & 0 & 0.0435
\end{bmatrix}
\quad
y_{3}^{'} =
\begin{bmatrix}
0.3529 & 0.2143 & 0 \\
0.4706 & 0.5000 & 0.1429 \\
0.1765 & 0.2143 & 0 \\
0 & 0.0714 & 0.3571 \\
0 & 0 & 0.5000
\end{bmatrix}
\end{align*}

(4) 确定 j 评价指标的熵值

\begin{equation}
e_{j} = -k \sum_{j=1}^{m} p_{jk} \ln p_{jk}
\tag{6.5}
\end{equation}

式中 \( k = 1 / (\ln m) \),因 \( p_{jk} = 0 \) 对上式无意义,因此进一步修正:

\begin{equation}
f_{jk} = \frac{(1 + p_{jk})}{\sum_{j=1}^{m} (1 + p_{jk})}
\tag{6.6}
\end{equation}

\begin{equation}
E_{j} = \frac{\sum_{j=1}^{m} f_{jk} \ln f_{jk}}{\ln m}
\tag{6.7}
\end{equation}

根据上述公式计算各评价指标的熵值:

隔离人口数 \( A_{11} \) 的熵值:

\begin{align*}
E_{11} &= -\frac{1}{Ln(10)} (0.3846 * \ln(0.3846) + 0.2308 * \ln(0.2308) + \\
&\quad 0 * \ln(0) + 0.0769 * \ln(0.0769) + 0.3077 * \ln(0.3077)) = 0.7865
\end{align*}

感染人口数 \( A_{12} \) 的熵值:

\begin{align*}
E_{12} &= -\frac{1}{Ln(10)} (0.2000 * \ln(0.2000) + 0.2500 * \ln(0.2500) + \\
&\quad 0.2500 * \ln(0.2500) + 0.3000 * \ln(0.3000) + 0 * \ln(0)) = 0.8551
\end{align*}

小区人口数 \( A_{13} \) 的熵值:

\begin{align*}
E_{13} &= -\frac{1}{Ln(10)} (0.1875 * \ln(0.1875) + 0 * \ln(0) + \\
&\quad 0.1875 * \ln(0.1875) + 0.2500 * \ln(0.2500) + 0.3750 * \ln(0.3750)) = 0.8339
\end{align*}

小区栋数 \( A_{21} \) 的熵值:

\begin{align*}
E_{21} &= -\frac{1}{Ln(10)} (0 * \ln(0) + 0.3529 * \ln(0.3529) + \\
&\quad 0.2941 * \ln(0.2941) + 0.0588 * \ln(0.0588) + 0.2941 * \ln(0.2941)) = 0.7792
\end{align*}

小区户数 \( A_{22} \) 的熵值:

\begin{align*}
E_{22} &= -\frac{1}{\ln(10)}(0.3750 \ln(0.3750) + 0.3750 \ln(0.3750) + \\
&\quad 0.1875 \ln(0.1875) + 0.0625 \ln(0.0625) + 0 \ln(0)) = 0.7598
\end{align*}

小区位置 $A_{23}$ 的熵值:
\begin{align*}
E_{23} &= -\frac{1}{\ln(10)}(0 \ln(0) + 0.3478 \ln(0.3478) + \\
&\quad 0.3913 \ln(0.3913) + 0.2174 \ln(0.2174) + 0.0435 \ln(0.0435)) = 0.7472
\end{align*}

道路宽度 $A_{31}$ 的熵值:
\begin{align*}
E_{31} &= -\frac{1}{\ln(10)}(0.3529 \ln(0.3529) + 0.4706 \ln(0.4706) + \\
&\quad 0.1765 \ln(0.1765) + 0 \ln(0) + 0 \ln(0)) = 0.6390
\end{align*}

道路长度 $A_{32}$ 的熵值:
\begin{align*}
E_{32} &= -\frac{1}{\ln(10)}(0.2143 \ln(0.2143) + 0.5000 \ln(0.5000) + \\
&\quad 0.2143 \ln(0.2143) + 0.0714 \ln(0.0714) + 0 \ln(0)) = 0.7427
\end{align*}

道路密度 $A_{33}$ 的熵值:
\begin{align*}
E_{33} &= -\frac{1}{\ln(10)}(0 \ln(0) + 0.1429 \ln(0.1429) + \\
&\quad 0 \ln(0) + 0.3571 \ln(0.3571) + 0.5000 \ln(0.5000)) = 0.6165
\end{align*}

(5) 确定 $j$ 评价指标的熵权
\begin{equation}
w_{j} = \frac{d_{j}}{\sum_{j=1}^{n} d_{j}}, \, j = 1, 2, \dots, n
\tag{6.8}
\end{equation}

由上述公式确定了各二级指标的熵权:
隔离人口数,感染人口数,小区人口数的熵权分别为:
\begin{equation}
w_{11} = 0.4070, \quad w_{12} = 0.2763, \quad w_{13} = 0.3167
\end{equation}
因此,人口的各二级指标的权重向量为: $w_{1} = (0.4070, 0.2763, 0.3167)$

小区栋数,小区户数,小区位置的熵权分别为:
\begin{equation}
w_{21} = 0.3093, \quad w_{22} = 0.3366, \quad w_{23} = 0.3542
\end{equation}
因此,小区的各二级指标的权重向量为: $w_{2} = (0.3093, 0.3366, 0.3542)$

道路宽度,道路长度,道路密度的熵权分别为:
\begin{equation}
w_{31} = 0.3604, \quad w_{32} = 0.2569, \quad w_{33} = 0.3828
\end{equation}
因此,道路的各二级指标的权重向量为: $w_{3} = (0.3604, 0.2569, 0.3828)$

对于长春市不同区域投放点数量分布的合理性邀请了 5 位业内专家对其进行打分,

整个打分过程选用 10 分制打分,评分的基本原则为:10-9 分为非常满意;8-7 分为满意;6-5 分为一般;4-3 分为不满意;2-1 分为非常不满意。得出的评分按表 6.3 的评价指标进行分类,利用熵权法对专家的评分进行计算,求出各个指标的权重。

\begin{table}[h]
\centering
\caption{一级指标专家打分表}
\begin{tabular}{c c c}
\hline
人数 & 小区 & 道路 \\
\hline
6 & 4 & 3 \\
9 & 3 & 1 \\
4 & 6 & 8 \\
3 & 6 & 10 \\
4 & 9 & 5 \\
\hline
\end{tabular}
\end{table}

(1) 定各指标的原始矩阵:  
由公式(6.1)计算得到:
\[
x = 
\begin{bmatrix}
6 & 4 & 3 \\
9 & 3 & 1 \\
4 & 6 & 8 \\
3 & 6 & 10 \\
4 & 9 & 5
\end{bmatrix}
\]

(2) 矩阵归一化  
由公式(6.2)计算得到:
\[
y = 
\begin{bmatrix}
0.5000 & 0.1667 & 0.2222 \\
1 & 0 & 0 \\
0.1667 & 0.5000 & 0.7778 \\
0 & 0.5000 & 1 \\
0.1667 & 1 & 0.4444
\end{bmatrix}
\]

(3) 确定各评价指标的权重  
由公式(6.4)计算得到一级各指标的权重:
\[
y' = 
\begin{bmatrix}
0.2727 & 0.0769 & 0.0909 \\
0.5455 & 0 & 0 \\
0.0909 & 0.2308 & 0.3182 \\
0 & 0.2308 & 0.4091 \\
0.0909 & 0.4615 & 0.1818
\end{bmatrix}
\]

(4) 确定 j 评价指标的熵值  
由公式(6.5)-(6.7)计算得到一级各指标的熵值

隔离人口数 $A_1$ 的熵值:
\[
E_1 = -\frac{1}{Ln(10)}(0.2727 * \ln(0.2727) + 0.5455 * \ln(0.5455) + 0.0909 * \ln(0.0909) + 0 * \ln(0) + 0.0909 * \ln(0.0909)) = 0.6965
\]

感染人口数 $A_2$ 的熵值:

\begin{equation}
E_{2} = -\frac{1}{\ln(10)}(0.0769 \cdot \ln(0.0769) + 0 \cdot \ln(0) + 0.2308 \cdot \ln(0.2308) + 0.2308 \cdot \ln(0.2308) + 0.4615 \cdot \ln(0.4615)) = 0.7648
\end{equation}

小区人口数 \(A_{3}\) 的熵值:

\begin{equation}
E_{3} = -\frac{1}{\ln(10)}(0.0909 \cdot \ln(0.0909) + 0 \cdot \ln(0) + 0.3182 \cdot \ln(0.3182) + 0.4091 \cdot \ln(0.4091) + 0.1818 \cdot \ln(0.1818)) = 0.7816
\end{equation}

(5) 确定 \(j\) 评价指标的熵权

由公式 (2.8) 计算出一级各指标的熵权

人口,小区,道路的熵权分别为:

\begin{equation}
w_{1} = 0.40090, \quad w_{2} = 0.3106, \quad w_{3} = 0.2885
\end{equation}

因此,人口的各二级指标的权重向量为: \(w = (0.4009, 0.3106, 0.2885)\)

根据上述计算分别得到一级和二级各类指标的权重值,如图 6-3 所示。利用 Origin 软件将权重制作成三维饼状图,将各类指标占比最大值的突出显示。从图 6-3 可以看出,在一级指标中人口 U1 权重最大,权重为 0.4009。人口二级权重指标中隔离人口数 U11 权重最大,权重为 0.407。小区二级权重指标中小区位置 U23 权重最大,权重为 0.3542。道路二级权重指标中道路密度 U33 权重最大,权重为 0.3828。由此可见,人口、隔离人数、小区位置、道路密度评价指标对长春市蔬菜包投放点数量分布影响比较大。

\begin{figure}[h]
    \centering
    \includegraphics[width=0.45\textwidth]{一级指标权重.png}
    \caption{一级指标权重}
\end{figure}

\begin{figure}[h]
    \centering
    \includegraphics[width=0.45\textwidth]{人口二级指标权重.png}
    \caption{人口二级指标权重}
\end{figure}

\begin{figure}[h]
    \centering
    \includegraphics[width=0.45\textwidth]{小区二级指标权重.png}
    \caption{小区二级指标权重}
\end{figure}

\begin{figure}[h]
    \centering
    \includegraphics[width=0.45\textwidth]{道路二级指标权重.png}
    \caption{道路二级指标权重}
\end{figure}

图 6-3 各等级各类指标权重

\section*{6.3 确定模糊评价矩阵}

针对 9 个二级评价指标分别对长春市九个区域设置投放点数量的合理性做相应的问卷调查,根据统计结果对各个指标等级进行评价,并以朝阳区为例进行展示,结果见表 6.4。

\begin{table}
\centering
\caption{投放点数量合理性评价统计}
\begin{tabular}{l c c c c c}
\hline
二级指标 & 非常合理 & 比较合理 & 一般 & 不合理 & 非常不合理 \\
\hline
U11 & 0.05 & 0.15 & 0.35 & 0.25 & 0.20 \\
U12 & 0.10 & 0.18 & 0.27 & 0.30 & 0.15 \\
U13 & 0.08 & 0.24 & 0.29 & 0.34 & 0.05 \\
U21 & 0.10 & 0.21 & 0.23 & 0.26 & 0.20 \\
U22 & 0.10 & 0.30 & 0.20 & 0.25 & 0.15 \\
U23 & 0.15 & 0.30 & 0.16 & 0.27 & 0.12 \\
U31 & 0.22 & 0.27 & 0.20 & 0.17 & 0.14 \\
U32 & 0.11 & 0.20 & 0.27 & 0.36 & 0.06 \\
U33 & 0.09 & 0.35 & 0.24 & 0.20 & 0.12 \\
\hline
\end{tabular}
\end{table}

根据表4调查的投放点数量合理性结果,建立模糊关系的评价矩阵,具体如下:
\begin{align*}
RA_{1} &= \begin{bmatrix}
0.05 & 0.15 & 0.35 & 0.25 & 0.20 \\
0.10 & 0.18 & 0.27 & 0.30 & 0.15 \\
0.08 & 0.24 & 0.29 & 0.34 & 0.05
\end{bmatrix} \\
RA_{2} &= \begin{bmatrix}
0.10 & 0.21 & 0.23 & 0.26 & 0.20 \\
0.10 & 0.30 & 0.20 & 0.25 & 0.15 \\
0.15 & 0.30 & 0.16 & 0.27 & 0.12
\end{bmatrix} \\
RA_{3} &= \begin{bmatrix}
0.22 & 0.27 & 0.20 & 0.17 & 0.14 \\
0.11 & 0.20 & 0.27 & 0.36 & 0.06 \\
0.09 & 0.35 & 0.24 & 0.20 & 0.12
\end{bmatrix}
\end{align*}

\subsection{6.4 模糊综合评价}

首先二级指标对设置投放点数量合理性进行模糊综合评价,将各单因素评价矩阵分别与权重集进行模糊变换,计算公式如下:
\begin{equation}
B_{i} = W_{i} \circ R_{A_{i}} = (b_{1}, b_{2}, \cdots b_{n})
\tag{6.9}
\end{equation}
其中,$W_{i}$为权重矩阵,“$\circ$”表示$W_{i}$与$R_{A_{i}}$的广义的合成运算,即模糊算子的组合。基于表3、表4,利用上述公式进行模糊运算,分别求得各二级指标的评价向量为:
\begin{align*}
B_{1} &= W_{1} \circ R_{A_{1}} = (0.0792 \quad 0.2402 \quad 0.3719 \quad 0.2161 \quad 0.0926); \\
B_{2} &= W_{2} \circ R_{A_{2}} = (0.0944 \quad 0.1960 \quad 0.2756 \quad 0.3238 \quad 0.1102); \\
B_{3} &= W_{3} \circ R_{A_{3}} = (0.1132 \quad 0.2684 \quad 0.3732 \quad 0.1387 \quad 0.1065);
\end{align*}

为了得到各指标对长春市朝阳区投放点数量合理性的评价结果,我们给评语集进行赋分,如:非常合理=10分、比较合理=8分、一般=6分、不合理=4分,非常不合理=2分,最后通过赋分求和,我们可以分别得到人口、小区、道路交通对长春市朝阳区设置投放点数量的综合评分为:
\begin{align*}
A_{1} &= 0.0792 \times 10 + 0.2402 \times 8 + 0.3719 \times 6 + 0.2161 \times 4 + 0.0926 \times 2 = 2.1822, \\
A_{2} &= 0.0944 \times 10 + 0.1960 \times 8 + 0.2756 \times 6 + 0.3238 \times 4 + 0.1102 \times 2 = 3.6812, \\
A_{3} &= 0.1132 \times 10 + 0.2684 \times 8 + 0.3732 \times 6 + 0.1387 \times 4 + 0.1065 \times 2 = 4.1822.
\end{align*}

由上述评分结果看出,在人口指标方面对于朝阳区设置投放点数量评价结果为“非

常不合理”,在小区指标方面对于朝阳区设置投放点数量评价结果为“不合理”,在道路指标方面对于朝阳区设置投放点数量评价结果为“一般”。

其次,对一级指标对设置投放点数量合理性进行模糊综合评价,根据上面的计算得到一级指标的矩阵为:
\[
R_{0}=\left[\begin{array}{l}
A_{1} \\
A_{2} \\
A_{3}
\end{array}\right]=\left[\begin{array}{lllll}
0.0792 & 0.2402 & 0.3719 & 0.2161 & 0.0926 \\
0.0944 & 0.1960 & 0.2756 & 0.3238 & 0.1102 \\
0.1132 & 0.2684 & 0.3732 & 0.1387 & 0.1065
\end{array}\right]
\]
并根据公式 $B=W_{1} \cdot R_{1}$ 可得
\[
B=W_{0} \cdot R_{0}=(0.1205,0.1978,0.2532,0.3417,0.0868)
\]
接着,根据公式 $B=W * Y$ 对长春市朝阳区设置投放点数量的合理性综合评分结果如下:
\[
G_{A}=0.1205 \times 10+0.1978 \times 8+0.2532 \times 6+0.3417 \times 4+0.0868 \times 2=3.2156
\]
通过该综合评分可以得到“非常合理”为 0.1205,“比较合理”为 0.1978,“一般”为 0.2532,“不合理”为 0.3417,“非常不合理”为 0.0868,结合评语集计算可得朝阳区设置的投放点数量合理性综合评分为 3.2156。根据评语集可以判断朝阳区设置投放数量的合理性综合评价结果为“不合理”,整体上该投放点数量不能保障隔离区人口的日常生活物资需求。

类似于上述过程,采用同样的模型方法对长春市其他 8 个区进行熵权-模糊综合评价,评价结果见表 6.5。

\begin{table}[h]
\centering
\caption{长春市各区设置投放点数量合理性熵权-模糊综合评价结果}
\begin{tabular}{l c c c c c c c}
\hline
评价对象 & 非常合理 & 比较合理 & 一般 & 不合理 & 非常不合理 & 赋分求和 & 评价结果 \\
\hline
朝阳区 & 0.1205 & 0.1978 & 0.2532 & 0.3417 & 0.0868 & 3.2156 & 不合理 \\
南关区 & 0.1523 & 0.3456 & 0.2747 & 0.1713 & 0.0561 & 7.1967 & 比较合理 \\
宽城区 & 0.3826 & 0.2596 & 0.1829 & 0.1335 & 0.0414 & 8.7864 & 非常合理 \\
绿园区 & 0.0199 & 0.1058 & 0.2319 & 0.2985 & 0.3339 & 1.3298 & 非常不合理 \\
二道区 & 0.0193 & 0.1432 & 0.2301 & 0.3576 & 0.2498 & 3.5824 & 不合理 \\
长春新区 & 0.0102 & 0.1962 & 0.4213 & 0.2478 & 0.1245 & 4.8246 & 一般 \\
经开区 & 0.0321 & 0.1271 & 0.3892 & 0.2981 & 0.1529 & 4.4317 & 一般 \\
净月区 & 0.0534 & 0.1596 & 0.1987 & 0.3542 & 0.2341 & 3.9562 & 不合理 \\
汽开区 & 0.0880 & 0.1378 & 0.1825 & 0.3398 & 0.2519 & 3.0781 & 不合理 \\
\hline
\end{tabular}
\end{table}

根据熵权-模糊综合评价方法针对长春市蔬菜包投放点数量合理性进行了评价。由表 6.5 可知,长春市 9 个区的蔬菜包投放点均呈现不同程度的不合理。因此我们提出比例熵权法对长春市 9 个区域的蔬菜包投放点进行重新分配。假设投放点的总数量保持不变,计算公式如下所示:
\begin{equation}
\sum_{i,j=1}^{n} \frac{x_{ij}}{\sum_{i=1}^{n} x_{ij}} \times w_{j}
\tag{6.10}
\end{equation}
式中 $w_{j}$ 为各因素评价权重;$x_{ij}$ 长春市各指标数据。

根据上述公式计算出长春市 9 个区域的权重及蔬菜包投放点调整前后数据量,具体

\begin{table}
\centering
\caption{表6.6 长春市各区权重及蔬菜包投放点数量}
\begin{tabular}{c c c c}
\hline
区域名称 & 权重 & 调整前 & 调整后 \\
\hline
朝阳区 & 17\% & 94 & 261 \\
南关区 & 14\% & 261 & 225 \\
宽城区 & 11\% & 181 & 167 \\
绿园区 & 12\% & 470 & 187 \\
二道区 & 13\% & 9 & 198 \\
长春新区 & 10\% & 215 & 153 \\
经开区 & 7\% & 37 & 114 \\
净月区 & 8\% & 279 & 122 \\
汽开区 & 8\% & 10 & 129 \\
\hline
\end{tabular}
\end{table}

图6-4展示了长春市9个区的权重,从图中可以看出,朝阳区和南关区权重占比尤为突出,朝阳区权重为17\%,南关区权重为14\%。经开区权重最小,权重为7\%。净月区与汽开区权重一样为8\%。图6-5表示了长春市9个区蔬菜包投放点数量调整前后对比。由于朝阳区权重较大,调整前朝阳区数量蔬菜包投放点仅有94。根据公式(2.9)计算出来的权重进行了重新分配将其调整为261。绿园区调整前的蔬菜包投放点数量为470,但是其权重为12\%,从新计算重新分配为187。本章节根据提出的比例熵权法计算出长春市各区重新分配的蔬菜包投放点数量,与熵权-模糊综合评价结果得到了相互验证。

\begin{figure}[h]
\centering
\includegraphics[width=0.45\textwidth]{placeholder_radar_chart.png}
\caption{图6-4 长春市9个区的权重雷达图}
\end{figure}
\begin{figure}[h]
\centering
\includegraphics[width=0.45\textwidth]{placeholder_bar_chart.png}
\caption{图6-5 长春市9个区蔬菜包投放点数量}
\end{figure}

\subsection{6.5 长春市物资储备中心的选址模型构建}

在疫情防控期间,需要为长春市各小区提供物资。因此,选择防疫物资储备中心,不应花费较长的时间重新建造用于储备防疫物资的仓库。防疫物资选址的相关研究,大部分研究者在建立模型时仅仅考虑一种影响因素作为单一目标考虑。然而,防疫物资储备中心的选取需要多目标综合考虑。本节考虑了储备中心到小区的距离、储备中心的最大储存量、储备中心到小区的运输成本这三点因素。根据附件3提供的长春市9个小区的坐标位置构建系统聚类数学模型得到储备中心最优的选址数量、规模及其潜在的备用场所位置。

\subsubsection{6.5.1 模型假设}

本文做出如下假设:
\begin{enumerate}
    \item 本文研究的为静态问题;
\end{enumerate}

(2) 设施点和需求点呈离散型点状分布;  
(3) 防疫储备中心设施点的建造和运营的费用均可由估算得出;  
(4) 最大储存量和运输成本之间呈正比。  

\subsection*{6.5.2 模型参数}

(1) $Cap_{k}$:储备中心 $i$ 的最大储存量;  
(2) $d_{ij}$:储备中心 $i$ 到小区 $j$ 的距离;  
(3) $q_{ki}$:储备中心提供的最大储量;  
(4) $C_{ki}$:储备中心到小区的运输成本;  
(5) $p$:储备中心个数。  

需要对长春市物资储备中心个数 $P$ 值进行科学的决策。本节提出系统聚类模型对各需求点进行系列聚类,选择较为合适的聚类中心作为长春市物资储备中心个数。系统聚类属于机器学习中的无监督算法,相比 K-means 聚类算法,该算法可以很好的解决 K-means 聚类算法中不好确定聚类中心个数的问题,该算法的计算步骤具体流程见图 6-6。

\begin{figure}[h]
\centering
\includegraphics[width=0.8\textwidth]{image.png} % 替换为实际图片路径
\caption{系统聚类模型流程图}
\label{fig:6-6}
\end{figure}

利用 Matlab 软件根据图 6-7 进行系统聚类求解。系统聚类模型尤为重要的步骤是如何判断合理聚类中心数,本节采用肘部法则判断合适的长春市物资储备中心的聚类中心数。其原理是假设有 $n$ 个样本被划分到 $k$ 个类中,其中 $k \leq n-1$。一类中至少有两个样本,用 $C_{k}$ 表示第 $k$ 类,其中 $k=1, 2, \cdots, k$,且该类中心为 $u_{k}$。那么第 $k$ 类的畸变系数为 $\sum\limits_{i \in c_{k}} |x_{i} - u_{k}|^{2}$,所有类的总畸变系数为 $J = \sum\limits_{k=1}^{k} \sum\limits_{i \in c_{k}} |x_{i} - u_{k}|^{2}$。图 2.5 为朝阳区畸变系数与聚类中心数的点线图。从图 6.6 可以看出朝阳区聚类中心数 4-5 时为最优聚类数。其他 8 个区的聚类中心数求解同理可得。

\begin{figure}[h]
    \centering
    \includegraphics[width=\textwidth]{image1.png}
    \caption{朝阳区畸变系数点线图}
    \label{fig:6-7}
\end{figure}

根据上述原理确定了 9 个区的最优聚类中心数,将 9 个区的聚类中心数具体位置展示出来,具体如图 \ref{fig:6-8} 所示。图中红色五角星代表的是最优聚类位置,其余颜色的“星”型为长春市 9 个区的小区的地理位置。

\begin{figure}[h]
    \centering
    \begin{subfigure}[t]{0.3\textwidth}
        \centering
        \includegraphics[width=\textwidth]{image2a.png}
        \caption{朝阳区聚类中心位置}
        \label{fig:6-8a}
    \end{subfigure}
    \hfill
    \begin{subfigure}[t]{0.3\textwidth}
        \centering
        \includegraphics[width=\textwidth]{image2b.png}
        \caption{二道区聚类中心位置}
        \label{fig:6-8b}
    \end{subfigure}
    \hfill
    \begin{subfigure}[t]{0.3\textwidth}
        \centering
        \includegraphics[width=\textwidth]{image2c.png}
        \caption{经开区聚类中心位置}
        \label{fig:6-8c}
    \end{subfigure}
    \hfill
    \begin{subfigure}[t]{0.3\textwidth}
        \centering
        \includegraphics[width=\textwidth]{image2d.png}
        \caption{净月区聚类中心位置}
        \label{fig:6-8d}
    \end{subfigure}
    \hfill
    \begin{subfigure}[t]{0.3\textwidth}
        \centering
        \includegraphics[width=\textwidth]{image2e.png}
        \caption{宽城区聚类中心位置}
        \label{fig:6-8e}
    \end{subfigure}
    \hfill
    \begin{subfigure}[t]{0.3\textwidth}
        \centering
        \includegraphics[width=\textwidth]{image2f.png}
        \caption{绿园区聚类中心位置}
        \label{fig:6-8f}
    \end{subfigure}
    \caption{各区域聚类中心位置}
    \label{fig:6-8}
\end{figure}

\begin{figure}[h]
    \centering
    \includegraphics[width=\textwidth]{image.png}
    \caption{长春市9个区域聚类中心位置图}
    \label{fig:cluster_centers}
\end{figure}

将长春市9个区域的聚类中心数量及坐标位置进行统计,如表\ref{tab:cluster_centers}所示。采用系统聚类模型可以计算出长春市存储中心选址位置累计为36。

\begin{table}[h]
    \centering
    \caption{长春市9个区的聚类中心数量与坐标}
    \label{tab:cluster_centers}
    \begin{tabular}{c c c c c c c c}
        \hline
        名称 & 数量 & 聚类坐标 & 名称 & 数量 & 聚类坐标 & 名称 & 数量 \\
        \hline
        & & & & & & & \\
        & 42.889,33.543 & & & 68.499,21.580 & & & 49.288,16.794 \\
        & 51.700,46.035 & 净月 & 3 & 81.447,9.124 & 南关 & 3 & 58.697,28.415 \\
        朝阳 & 5 & 51.792,30.300 & & 84.811,21.178 & & & 60.384,44.207 \\
        & 54.928,39.540 & & & 61.260,54.994 & & & 42.291,41.735 \\
        & 47.454,38.380 & 宽城 & 4 & 50.358,64.423 & 汽开 & 3 & 25.477,30.597 \\
        & 69.004,57.709 & & & 54.635,52.303 & & & 33.329,36.742 \\
        二道 & 4 & 69.309,47.072 & & 57.809,62.133 & & & 38.010,23.334 \\
        & 71.378,40.619 & & & 47.325,56.080 & 长春 & 4 & 31.759,14.838 \\
        & 66.948,40.834 & & & 42.800,51.871 & & & 38.532,9.479 \\
        & 74.456,32.399 & 绿园 & 6 & 36.524,48.973 & & & 44.002,28.165 \\
        经开 & 4 & 67.089,33.745 & & 29.231,41.852 & & & \\
        & 81.632,43.840 & & & 44.035,45.224 & & & \\
        & 75.522,41.338 & & & 48.609,48.436 & & & \\
        \hline
    \end{tabular}
\end{table}

\subsection{6.5.3 模型的建立}

根据上述问题分析,在模型基本假设的前提下建立了如下的多目标模型:

\begin{equation}
    \max Z_1 = \sum_{i=1}^{m} \sum_{j=1}^{n} m_i * q_{ij} * x_{ij}
    \tag{6.11}
\end{equation}

\begin{equation}
    q_{ij} = \frac{\max\{d_{ij}\} - d_{ij}}{\max\{d_{ij}\}}
    \tag{6.12}
\end{equation}

\begin{equation}
    \min Z_2 = \sum_{i=1}^{m} \sum_{j=1}^{n} C_{ij} * d_{ij} * L_{ij} + \sum_{i=1}^{n} (F_j + C_j) * y_i (C = C_{ij} * d_{ij} * L_{ij})
    \tag{6.13}
\end{equation}

\begin{equation}
s.t.
\begin{cases}
X_{ij} \leq y_{j}, \forall i, j \\
\sum_{j=1}^{n} x_{ij} \leq 1, \forall i \\
\sum_{j=1}^{n} L_{ij} \geq L_{i} \\
L_{ij} \geq 0 (i=1,2,3 \cdots m; i=1,2,3 \cdots n) \\
\sum_{j=1}^{n} y_{i} = 36 \\
x_{ij} \in \{0,1\} (i=1,2,3, \cdots m; j=1,2,3 \cdots n) \\
y_{j} \in \{0,1\} (j=1,2,3 \cdots n)
\end{cases}
\end{equation}

上述模型中目标函数(6.11)为储备中心的最大储存量满足长春市需求;目标函数(6.13)是指储备中心到小区的运输成本最少。

约束条件分别表示使选择的储备中心点 $j$ 覆盖各小区 $i$;表示确保每个小区需求点至多只能由一个储备中心点为其提供服务;表示应急物资满足各小区的需求量;表示指定被建立的储备中心数量为 36 个;表示选址决策为 0-1 约束,即建立或者不建立。

\subsection{6.5.4 模型的算法求解}

构建的长春市储备中心选址模型属于典型的 NP-hard 问题,根据多目标模型进行求解。在模型求解的过程中采用遗传算法进行优化求解。遗传算法 (Genetic Algorithms, GA) 是一种基于自然界遗传与生物进化论的并行随机搜索最优方法。该方法的基本要素包括个体基因的编码、适应度函数、遗传操作及运行参数。选择、交叉、突变为遗传操作。初始种群大小、最大遗传代数、交配及变异概率为运行参数。在算法优化时,首先,根据实际问题确定目标函数。其次,建立原始种群进行遗传操作。最终,得到最优的个体,具体流程见图 6-9。

\begin{figure}[h]
\centering
\includegraphics[width=0.8\textwidth]{genetic_algorithm_flowchart.png}
\caption{遗传算法流程图}
\end{figure}

多目标函数中有三个目标是储备中心的最大储存量覆盖长春市疫情防控需求,储存中心储备中心到小区的距离与储备中心到小区的运输成本最小。为了方便模型求解,可

以采用线性加权法来处理,将多目标模型转化为单目标模型来进行求解。针对三个目标给出各自权重系数,可表示为 $\lambda_1, \lambda_2, \lambda_3$,且满足 $\lambda_1 + \lambda_2 + \lambda_3 = 1$。则将长春市储备中心选址模型多目标函数变为一个新的单目标函数 $U = \min(\lambda_1 Z_1 + \lambda_2 Z_2 + \lambda_3 Z_3)$。具体操作步骤如下所示:

\begin{enumerate}
    \item \textbf{步骤 1:} 对基础数据进行编码,并且构建物资储量中心位置点和小区之间的距离矩阵 $C = (d_{ij})_{m \times n}$;
    \item \textbf{步骤 2:} 对遗传算法进行初始参数进行设置,设置初始种群规模,最大迭代次数;
    \item \textbf{步骤 3:} 设置初始种群为 $p(0)$,令 $t = 1, p(t) = p(0)$;
    \item \textbf{步骤 4:} 将目标函数 $U = \min(\lambda_1 Z_1 + \lambda_2 Z_2 + \lambda_3 Z_3)$ 作为适应度函数,求解种群 $p(t)$ 的适应度值;
    \item \textbf{步骤 5:} 随机选择出两个父个体 $p_1, p_2$ 进行遗传操作开始选择、交叉、变异,并将得到的新基因插入到个体中得到子种群 $Q(t)$
    \item \textbf{步骤 6:} 合并初始种群 $p(t)$ 和子种群 $Q(t)$,并产生一个新的种群 $N(t)$
    \item \textbf{步骤 7:} 对新的种群 $N(t)$ 进行优劣排序,得到 $p(t+1)$;
    \item \textbf{步骤 8:} 用 $t = t + 1$ 判断是否满足输出要求,若满足则运算结束。若不满足在进行适应度计算。
\end{enumerate}

利用 Matlab 软件实现遗传算法求解多目标模型,最终迭代了 500 次,目标函数达到 $1.2 \times 10^6$,满足要求计算结束,如图 \ref{fig:6-10} 所示。

\begin{figure}[h]
    \centering
    \includegraphics[width=\textwidth]{image.png}
    \caption{迭代次数与目标函数折线图}
    \label{fig:6-10}
\end{figure}

通过遗传算法求解多目标模型。图 6-11 展示了经度在 25-50,纬度在 0-35 范围内聚类中心位置,红色五角星为储备中心,绿色为备选中心。红线表示储备中心覆盖的小区范围。图 6-12 表示长春市 9 个区的储备中心数量与位置,从图中可以见出聚类数量为 20 个。比系统聚类得到的数量减少了 16 个。采用遗传算法优化多目标求解,在满足储备中心的最大储存量覆盖长春市疫情防控需求,储存中心储备中心到小区的距离与储备中心到小区的运输成本最小前提下,减少了聚类中心数量,模型得到了优化。

\begin{figure}[h]
    \centering
    \includegraphics[width=\textwidth]{image1.png}
    \caption{图6-11 局部聚类中心展示}
\end{figure}

\begin{figure}[h]
    \centering
    \includegraphics[width=\textwidth]{image2.png}
    \caption{图6-12 长春市9个区聚类中心展示}
\end{figure}

表6.8展示了长春市9个区的防疫物资储量中心的的位置,选址半径,管辖小区的个数,覆盖人口数。

\begin{table}[h]
\centering
\caption{表6.8 长春市9个区的聚类中心位置}
\begin{tabular}{c c c c c c}
\hline
数量 & 选址位置 & 所属区域 & 选址半径 & 管辖范围小区个数 & 选取范围人口数 \\
\hline
1 & (69.089,37.982) & 经开区 & 125 & 102 & 80684 \\
2 & (37.245,22.796) & 长春新区 & 112 & 42 & 93959 \\
3 & (49.664,16.915) & 南关区 & 80 & 35 & 107383 \\
4 & (52.873,49.293) & 宽城区 & 149 & 139 & 52827 \\
5 & (54.240,63.121) & 宽城区 & 127 & 90 & 83854 \\
6 & (83.933,12.452) & 净月区 & 95 & 32 & 104745 \\
7 & (35.945,10.836) & 长春新区 & 112 & 26 & 145966 \\
8 & (45.721,55.546) & 绿园区 & 101 & 39 & 88654 \\
9 & (79.670,26.378) & 净月区 & 91 & 44 & 140345 \\
10 & (37.509,49.118) & 绿园区 & 134 & 79 & 94773 \\
11 & (66.111,57.677) & 二道区 & 135 & 89 & 70751 \\
12 & (44.709,33.099) & 朝阳区 & 165 & 102 & 132605 \\
13 & (57.261,28.816) & 南关区 & 122 & 65 & 130684 \\
14 & (57.702,41.262) & 南关区 & 103 & 73 & 85488 \\
15 & (33.357,38.006) & 汽开区 & 136 & 58 & 190506 \\
16 & (78.615,43.211) & 经开区 & 97 & 48 & 137564 \\
17 & (66.571,47.500) & 二道区 & 152 & 130 & 174849 \\
18 & (44.924,43.122) & 汽开区 & 174 & 89 & 94795 \\
19 & (25.722,31.049) & 汽开区 & 103 & 67 & 89375 \\
20 & (67.783,21.191) & 净月区 & 143 & 60 & 920018 \\
\hline
\end{tabular}
\end{table}

\section*{七、问题三模型的建立与求解}

在疫情期间生活物资的发放过程中蔬菜作为一类人们急需且保质期较短的特殊产品,其分配的效果比较重要。问题三要求根据附件 5 分析蔬菜包需求、发放规律。问题的本质是数据挖掘与分析,利用 Origin 软件将 3 月 26 日-5 月 1 日长春市 9 个区的蔬菜包库存量、接收与投放进行了可视化分析。从 “供与求” 的角度分析了 9 个区蔬菜包的分配方案。要求根据附件 3 中的各小区位置与人口信息,评价并调整 4 月 10 日至 4 月 15 日蔬菜包供应方案。蔬菜包科学合理分配属于多投入、多产出的复制系统。因此采用数据包络分析法(DEA)进行建立评价指标体系。基于决策偏好的超效率 DEA 评价模型,对长春市蔬菜包分配方案合理进行分析。在此模型模型基础上添加了 “投影” 约束条件,建立了基于投影分析的 DEA 反馈调整模型对不合理的蔬菜包分配进行了调整。最终,得到 4 月 10 日至 4 月 15 日长春市 9 个区的蔬菜包供应方案。

\begin{figure}[h]
    \centering
    \includegraphics[width=\textwidth]{image1.png}
    \caption{问题三思路流程图}
    \label{fig:flowchart}
\end{figure}

\subsection{长春市 9 个区的蔬菜安全库储存量分析}

根据附件 5 统计了长春市 9 个区蔬菜包库存量与时间变化规律,为方便分析其规律利用 Origin 软件制作成了桥图,如图 7-2 所示。从图 7-2 中可以看出,朝阳区、南关区、二道区、宽城区、经开区蔬菜包库存量在 3 月 26 日开始均匀增加。在 3 月 26 日到 5 月 1 日期间蔬菜包库存量出现了不同程度的减小。然而,库存量没有出来负值,整体库存量可以维持各自区域的供给。净月区、绿园区、汽开区、长春新区的蔬菜包库存量出现负值,这 4 个区的蔬菜包库存量出现了亏损,表明蔬菜包分配不合理。宽城区的蔬菜包库存量持续增加,仅有几天是减小的。由此可知,宽城区的蔬菜包不断的储存,但是没有发放。蔬菜包具有一定的保质期,宽城区蔬菜包过量囤积会导致蔬菜包超过保质期而损失。绿园区、经开区蔬菜包最大储存量累计达到了 80000 与 70000(箱/袋)。然而在 3 月 26 日到 5 月 1 日期间仅有少量减小,由此可以判断,这两个区的蔬菜包会出现过剩。综上所述,蔬菜包需要合理的储存,存储量不宜过大,也不能出现连续大量供给。

\begin{figure}[h]
    \centering
    \includegraphics[width=\textwidth]{image1.png}
    \caption{长春市9个区蔬菜包库存量变化规律}
    \label{fig:7-2}
\end{figure}

根据附件5统计了长春市9个区蔬菜包库接收与投放的变化规律,如图\ref{fig:7-3}所示。从图7.2可以看出,9个区的蔬菜接收与投放峰值集中在4月10日左右。朝阳区、绿园区、长春新区、汽开区可以明显的看出蔬菜包接收与供给呈增大-减小的变化趋势。二道区、南关区、宽城区、经开区、净月区出现了两次波动,4月10日与3月30日左右,其中4月10日峰值最大。朝阳区、二道区、南关区接收与投放数量保持了“收支平衡”的现象。汽开区、经开区在4月10日之后投放数量明显大于接收数量。由此可见,为保障防疫物资有效合理的方法,应该需要对蔬菜包数量、接收、投放作出科学合理的分配方案。

\begin{figure}[h]
    \centering
    \includegraphics[width=\textwidth]{image2.png}
    \caption{长春市9个区蔬菜包接收与投放变化规律}
    \label{fig:7-3}
\end{figure}

\section*{7.2 长春市 9 个区的蔬菜包科学分配方案评价指标体系构建}

对长春市 9 个区的蔬菜包进行科学合理的分配。考虑了人口、小区、蔬菜包等指标对蔬菜包科学分配进行重新评价。选择数据包络法进行分析,相对于模糊综合评价法,本节筛选的的指标均可完成量化指标,做出的方案可以用于投入产出的效率研究,为长春市防控物资决策提供依据。蔬菜包科学合理分配属于多投入、多产出的复制系统,指标之间无法用具体的数学表达式表示,符合数据包络分析法(DEA)黑箱系统操作。DEA 可以有效避免主观因素过度影响,从数学理论与真实数据出发更加客观可靠。分析各决策单元的劣势所在,并提供改进方向,为开行方案调整优化提供依据。因此,本节选用数据包络分析法(DEA)进行分析。建立评价指标体系,如图 7-4 所示。目标层为长春市蔬菜包科学合理分配方案。一级指标为人口、小区、蔬菜包。人口的二级指标为隔离人数、感染人数、小区人数。小区的二级指标为小区栋数、小区户数、小区位置。蔬菜包的二级指标为蔬菜包价格、保质期、库存量。

\begin{figure}[h]
\centering
\includegraphics[width=\textwidth]{image.png}
\caption{基于数据包络分析法的评价指标体系}
\end{figure}

\subsection*{7.2.1 基于决策偏好的超效率 DEA 评价模型特点}

基于传统 DEA 模型进行优化决策。由于篇幅有限,本节没有介绍传统 DEA 模型,主要阐述超效率 DEA 评价模型。超效率 DEA 评价模型是结合长春市蔬菜包分配决策偏好,提出基于决策偏好的超效率 DEA 评价模型,从而提高了模型分区能力。其模型的核心是在对蔬菜包分配单元进行效率评价时,将其决策单元排除后进行评价。当效率 $\theta^{*}=1$ 的 DEA 有效决策单元数值大于 1。当 $\theta^{*}<1$ 的 DEA 无效决策单元相对效率数值不变。

\begin{figure}[h]
\centering
\includegraphics[width=0.8\textwidth]{image2.png}
\end{figure}

图 7-5 超效率 DEA 示意图

从图 7-5 可以看出对决策单元 $DMU_{B}$ 进行效率评价,传统 $DEA$ 模型中前沿面为 $ABCDE$。然而,本节提出的超效率模型将决策单元 $DMU_{B}$ 排除掉,前沿面为 $ACDE$。在评价时,对于 $DEA$ 无效的决策单元 $DMU_{F}, DMU_{G}$ 超效率模型与传统模型的相对效率值没有发生变化。前沿面导致 $DEA$ 有效单元后移,超效率模型相对效率值较传统模型增大。图 7-5 中 $BB_{1}$ 表示决策单元 $B$ 投入量可以发生增幅,其超效率评价值 $\frac{OB_{1}}{OB} > 1$。实现了 $DEA$ 有效单元的决策排序功能。因此,利用超效率 $DEA$ 模型,可以扩大效率评价的取值范围,实现了蔬菜包科学分配。

\subsection*{7.2.2 基于决策偏好的超效率 DEA 评价模型构建}

运用超效率 $DEA$ 理论,以 $C^{2}R$ 为基本模型。

\begin{equation}
\max h_{j_{0}} = \frac{u^{T}Y_{j_{0}}}{v^{T}X_{j_{0}}}
\tag{7.1}
\end{equation}

\begin{equation}
s.t.
\begin{cases}
\frac{u^{T}Y_{j}}{v^{T}X_{j}} \leq 1, j = 1, 2, 3 \cdots, n \\
v = (v_{1}, v_{2}, v_{3} \cdots, v_{m})^{T} \geq 0 \\
u = (u_{1}, u_{2}, u_{3} \cdots, u_{q})^{T} \geq 0
\end{cases}
\tag{7.2}
\end{equation}

结合以长春市蔬菜包分配决策偏好的指标重要度排序,对蔬菜包分配方案的超效率 $DEA$ 模型进行建立。假设评价蔬菜包分配的方案数量为 $n$ 个,即 $DUM_{j} (j = 1, 2, 3 \cdots, n)$。其中每一个 $DMU$ 均有 $m$ 个投入指标,即 $X_{j} (X_{1j}, X_{2j}, X_{3j} \cdots, X_{mj})^{T}$。$q$ 个产出指标,即 $Y_{j} (Y_{1j}, Y_{2j}, Y_{3j} \cdots, Y_{qj})^{T}$ 当前要进行评价的方案为 $DMU$。基于问题 2 的熵权-模糊综合评价模型计算评价指标的权重,投入指标与产出指标对应的权重向量分别为:

\begin{equation}
\omega^{T} = (\omega_{1}, \omega_{2}, \cdots, \omega_{m})
\tag{7.3}
\end{equation}

\begin{equation}
\mu^{T} = (\mu_{1}, \mu_{2}, \cdots, \mu_{m})
\end{equation}

根据指标排序对 $DEA$ 模型添加约束,将长春市 9 个区的蔬菜包的分配方案建立超效率 $DEA$ 模型:

\begin{equation}
s.t.
\begin{cases}
\mu^{T}Y_{j} - \omega^{T}X_{j} \leq 0, j = 1, 2, 3 \cdots, n, j \neq k \\
\omega^{T}X_{j_{0}} = 1 \\
\omega_{i} - \omega_{i+1} \geq 0, i = 1, 2, 3 \cdots, m-1 \\
\mu_{r} - \mu_{r+1} \geq 0, i = 1, 2, 3 \cdots, q-1 \\
\omega \geq 0 \\
\mu \geq 0
\end{cases}
\tag{7.4}
\end{equation}

对上述模型转换成对偶问题,并进一步引入松弛变量 $s_{i}^{-}$ 与剩余变量 $s_{r}^{+}$ 如下所示:

\begin{equation}
\min \theta - \varepsilon \sum (s^{-} + s^{+})
\tag{7.5}
\end{equation}

\begin{equation}
s \cdot t \cdot
\begin{cases}
\sum_{\substack{j=1 \\ j \neq k}}^{n} \lambda_{j} x_{ij} + \lambda_{n+1} + s_{i}^{-} = \theta x_{ik}, i=1 \\
\sum_{\substack{j=1 \\ j \neq k}}^{n} \lambda_{j} x_{ij} + \lambda_{n+i} - \lambda_{n+i-1} + s_{i}^{-} = \theta x_{ik}, i=2,3,\dots,m-1 \\
\sum_{\substack{j=1 \\ j \neq k}}^{n} \lambda_{j} x_{ij} - \lambda_{n+m-1} + s_{i}^{-} = \theta x_{ik}, i=m \\
\sum_{\substack{j=1 \\ j \neq k}}^{n} \lambda_{j} y_{rj} - \lambda_{n+m} - s_{i}^{+} = y_{rk}, r=1
\end{cases}
\tag{7.6}
\end{equation}

\subsection{7.2.3 模型求解结果}

(1) 模型整体效率计算

利用上述构建的基于决策偏好的超效率 DEA 评价模型,对长春市蔬菜包分配方案合理进行分析。设置评价模型中的松弛变量为 $s_{1}^{-}, s_{2}^{-}, s_{3}^{-}, s_{4}^{-}$,剩余变量为 $s_{1}^{+}, s_{2}^{+}, s_{3}^{+}, s_{4}^{+}$,用 Matlab 软件进行编程,求解结果见表 7.1。

\begin{table}[h]
\centering
\caption{评价模型中的松弛变量和剩余变量}
\begin{tabular}{c c c c c c c c c c}
\hline
DMU & $\theta 1$ & s1- & s2- & s3- & s4- & s1+ & s2+ & s3+ & s4+ & 有效性 \\
\hline
1 & 0.784 & 0 & 0 & 0.3 & 0.1 & 1.1 & 0 & 0 & 0 & 无效 \\
2 & 1.000 & 0 & 0 & 0 & 0 & 0 & 0 & 0 & 0 & 有效 \\
3 & 1.000 & 0 & 0 & 0 & 0 & 0 & 0 & 0 & 0 & 有效 \\
\hline
\end{tabular}
\end{table}

从表 7.1 计算结果可以看出,长春市蔬菜包分配方案 2,3 整体有效。即方案 2,3 产出已经达到最佳化状态,整体情况达到最优。方案 1 整体无效,从松弛变量与剩余变量可以看出,投入值可以降低,存在输入数据冗余的状态。

(2) 方案技术有效性与规模有效性计算

利用上述构建的 $DEA, BC^{2}$ 模型,对方案技术的有效性与规模有效性进行分析,求解结果见表 7.2。

\begin{table}[h]
\centering
\caption{方案技术有效性与规模有效性计算结果}
\begin{tabular}{c c c c c c}
\hline
DMU & $\theta_{2}$ & 人口有效性 & $\theta_{3}$ & 小区有效性 & $\theta_{4}$ & 蔬菜包有效性 \\
\hline
1 & 0.937 & 无效 & 0.837 & 无效 & 0.731 & 无效 \\
2 & 1.000 & 有效 & 1.000 & 有效 & 0.938 & 无效 \\
3 & 1.000 & 有效 & 1.000 & 有效 & 1.000 & 有效 \\
\hline
\end{tabular}
\end{table}

从表 7.2 可以看出,$DMU_{1}$ 的人口、小区、蔬菜包均无效。$DMU_{3}$ 整体有效性,$DMU_{2}$ 人口、小区有效性,蔬菜包无效性。由此可见 $DMU_{1}$ 三者均不有效则方案 1 整体无效。

(3) 超效率排序

由上述分析可知,将长春市 9 个区的超效率进行排序计算,计算结果见表 7.3。

\begin{table}[h]
\centering
\caption{9 个区的超效率计算结果}
\begin{tabular}{c c c c}
\hline
DMU & 区域名称 & $\theta_{5}$ & 方案排序 \\
\hline
1 & 经开区 & 1.028 & 4 \\
2 & 朝阳区 & 1.293 & 1 \\
3 & 长春新区 & 0.974 & 7 \\
\hline
\end{tabular}
\end{table}

\begin{tabular}{c c c c}
4 & 南关区 & 1.092 & 2 \\
5 & 宽城区 & 1.003 & 5 \\
6 & 绿园区 & 0.991 & 6 \\
7 & 净月区 & 0.833 & 9 \\
8 & 汽开区 & 0.890 & 8 \\
9 & 二道区 & 1.034 & 3 \\
\end{tabular}

\subsection*{7.2.4 基于投影分析的 DEA 反馈调整模型}

对于蔬菜包合理分配方案,根据评价模型对分配方案进行评价不是最重要的目的,而需要给出优化调整方案,为长春市 4 月 10 日至 4 月 15 日蔬菜包供应最优方案。仅对分配方案进行评价,对后续防疫工作的帮助有限,需要给出反馈方案。因此,基于 DEA 投影原理,对投影模型进行针对性改进,进而对 DEA 整体无效的蔬菜包分配方案给出具体的方案反馈调整方法。

对于 DEA 模型不合理的蔬菜包分配方案,添加约束条件进行反馈调整。本节考虑了分配方案的整体效率、经济效率、规模效率。

\begin{equation}
\begin{cases}
\text{蔬菜包使用效率} = \text{蔬菜包接收数量} / \text{蔬菜包投出数量} \\
\text{经济效率} = \text{蔬菜包单价} - \text{成本价格}
\end{cases}
\end{equation}

考虑实际情况,主要考虑使用效率,保证蔬菜包在保质期内投放,对蔬菜包分配方案进行反馈调整,使经济效益与运输效率得到综合性提升。蔬菜包分配方案反馈调整目的与流程如图 7-6 所示。

\begin{figure}[h]
\centering
\includegraphics[width=0.8\textwidth]{image.png}
\caption{投影分析的 DEA 反馈调整模型流程图}
\label{fig:7-6}
\end{figure}

\subsection*{7.2.5 模型参数}

(1) $\theta$:决策单元 $DUM$ 的效率评价指数;

(2) $\lambda_{j}$:对偶问题中,$DUM$ 的线性组合权重系数;

(3) $s_{i}^{-}$: 松弛变量;

(4) $s_{j}^{+}$: 剩余变量;

(5) $DMU$: 评价指标。

\subsection{7.2.6 模型构建}

根据上述流程图原理,针对 $DEA$ 模型的有效决策单元进行了有效反馈调整。添加了 “投影” 约束条件,投影模型如下所示:

\begin{equation}
\left\{
\begin{aligned}
X_{k}^{*} &= \theta_{k}^{*} X_{k} - s^{-} = \sum_{k=1}^{s} \lambda_{k} X_{k} \\
Y_{k}^{*} &= Y_{k} + s^{+} = \sum_{k=1}^{s} \lambda_{k} X_{k}
\end{aligned}
\right.
\tag{7.7}
\end{equation}

当 $DEA$ 模型的整体无效时,对原始指标进行改进添加 “投影”:

\begin{equation}
\left\{
\begin{aligned}
X_{k}^{*} &= \theta_{k}^{*} X_{k} - s^{-} \\
Y_{k}^{*} &= Y_{k} + s^{+}
\end{aligned}
\right.
\tag{7.8}
\end{equation}

上述投影分析可以对无效蔬菜包分配方案进行一定调整,可以实现蔬菜包的合理分配。

\subsection{7.2.7 模型求解结果}

经过分析,方案中投入指标均为可变动指标,调整人口、小区、蔬菜包的指标。将小区指标提高。根据实际情况,在蔬菜包合理分配时,小区人数对蔬菜包需求比较高,导致冗余度较大。因此,综合考虑下进行指标幅度调整,见表 7.4。

\begin{table}[h]
\centering
\caption{指标最大调整幅度}
\begin{tabular}{c c c c}
\hline
指标 & $x_{1}$ & $x_{2}$ & $x_{3}$ \\
\hline
最大调整幅度 & $10\%$ & $20\%$ & $10\%$ \\
\hline
\end{tabular}
\end{table}

确定投入指标调整幅度后,运用调整模型进行计算,计算结果见表 7.5。

\begin{table}[h]
\centering
\caption{反馈调整模型计算结果表}
\begin{tabular}{c c c c c c c c c}
\hline
参数 & $s_{1}^{-}$ & $s_{2}^{-}$ & $s_{3}^{-}$ & $s_{4}^{-}$ & $s_{1}^{+}$ & $s_{2}^{+}$ & $s_{3}^{+}$ & $\theta$ \\
\hline
计算结果 & 0.393 & 3.289 & 0.992 & 0.000 & 14.474 & 10.284 & 5.836 & 2.736 & 0.953 \\
\hline
\end{tabular}
\end{table}

综上,根据计算结果,对平日开行方案进行反馈调整。投入指标调整过程如下:

\begin{table}[h]
\centering
\caption{调整后接收+自采蔬菜包数量结果}
\begin{tabular}{c c c c c c c c c c}
\hline
日期 & 朝阳区 & 南关区 & 二道区 & 宽城区 & 绿园区 & 长春新区 & 经开区 & 净月区 & 汽开区 \\
\hline
4.10 & 21876 & 19098 & 20944 & 45838 & 19304 & 43988 & 30765 & 8653 & 3039 \\
4.11 & 18394 & 21975 & 18449 & 33987 & 17139 & 38475 & 27569 & 8747 & 3933 \\
4.12 & 15931 & 15722 & 14038 & 24563 & 16383 & 25582 & 22196 & 5755 & 2837 \\
4.13 & 9388 & 8795 & 7983 & 29675 & 18930 & 12928 & 16347 & 7931 & 3038 \\
4.14 & 10374 & 4563 & 3840 & 18753 & 18833 & 19373 & 24795 & 3421 & 3940 \\
4.15 & 12039 & 6099 & 2830 & 15397 & 17303 & 10875 & 19074 & 2979 & 13294 \\
\hline
\end{tabular}
\end{table}

图 7-7 为长春市 9 个区调整后接收+自采蔬菜包数量三维柱状图。从图 7.6 可以看出,对于汽开区 4 月 10 日到 4 月 14 日蔬菜包数量变化较平稳,在 4 月 15 日,蔬菜包数量增加到 13294。宽城区与长春新区是在 4 月 10 日需要大量蔬菜包储备,在 4 月 10 日后,数量逐渐减小。

\begin{figure}[h]
    \centering
    \includegraphics[width=\textwidth]{image1.png}
    \caption{接收+自采蔬菜包数量三维柱状图}
    \label{fig:7-7}
\end{figure}

\begin{table}[h]
    \centering
    \caption{调整后投放蔬菜包数量结果}
    \label{tab:7-7}
    \begin{tabular}{c c c c c c c c c c}
        \textbf{日期} & \textbf{朝阳区} & \textbf{南关区} & \textbf{二道区} & \textbf{宽城区} & \textbf{绿园区} & \textbf{长春新区} & \textbf{经开区} & \textbf{净月区} & \textbf{汽开区} \\
        \hline
        4.10 & 20688 & 20285 & 24908 & 62226 & 10393 & 60874 & 24360 & 2594 & 2012 \\
        4.11 & 15076 & 23223 & 9122 & 33251 & 15393 & 31854 & 46424 & 3462 & 1039 \\
        4.12 & 10125 & 17567 & 20698 & 62800 & 13092 & 20937 & 24963 & 2863 & 439 \\
        4.13 & 12476 & 2000 & 10637 & 62737 & 19967 & 15255 & 23994 & 2522 & 139 \\
        4.14 & 11874 & 2070 & 6190 & 23875 & 20356 & 17913 & 20224 & 1948 & 503 \\
        4.15 & 15285 & 8040 & 6748 & 15244 & 22380 & 18731 & 20886 & 1416 & 1102 \\
    \end{tabular}
\end{table}

图 \ref{fig:7-8} 为长春市 9 个区投放蔬菜包数量三维柱状图。从图 \ref{tab:7-7} 可以看出,绿园区和朝阳区在 4 月 10 日蔬菜包投放需求量较大。宽城区、长春新区整体蔬菜包投放量变化平稳,需求量较低。

\begin{figure}[h]
    \centering
    \includegraphics[width=\textwidth]{image2.png}
    \caption{调整后投放蔬菜包数量三维柱状图}
    \label{fig:7-8}
\end{figure}

\section*{八、问题四模型的建立与求解}

考虑不同周期下不同生活物资种类投放量与接收量的关系及其动态变化特性,构建储备中心-集散点-小区的三级有序拓扑网络,并在第二、三问的基础上,建立基于三级拓扑网络的生活物资多周期分配优化模型。其次,考虑不同蔬菜品种集和最优集散地数量确定基于最短距离的多周期动态物资供应方案。接着,考虑长春市真实街道情形,构建基于蚁群算法的物资配送路径规划模型,利用自适应遗传算法求解模型得到各区储备中心到各小区的三级最优路径,并计算最优行驶路径长度,将其应用于多周期分配优化模型得到合理的物资供应方案。最后,考虑最小配送时间目标,基于不同卡车承载规模约束,利用“顺路”原则将物资零散配送到多个需求小区得到不同物资分配方案。

\begin{figure}[h]
    \centering
    \includegraphics[width=\textwidth]{image1.png}
    \caption{第四问思路流程图}
    \label{fig:flowchart}
\end{figure}

\subsection{基于三级拓扑网络的生活物资多周期分配优化模型}

长春市发生疫情后,政府将会采取大规模封控的措施进行防控疫情,在此背景下,给出保障居民生活物资供应的有序发放方案对高效应对疫情是非常必要的。但是,各居民小区的应急生活物资数量有限,需要及时的从长春市大规模的储备中心向各个集散点调运,并且再从各集散点分配至各个居民小区。

\subsection{构建储备中心-集散点-小区的三级有序拓扑网络}

基于此,为满足众多小区的多周期生活物资需求,基于储备中心-集散点-小区的三级生活物资配送网络,考虑不同周期下不同生活物资种类投放量与接收量的关系及其动态变化特性,即从储备中心调运至集散点的生活物资数量不能超过储备中心在该时期内剩余可用物资数量;当前周期的实际生活物资需求的数量等于该周期的需求量与上一周期的生活物资缺少量之和;当前周期的实际生活物资投放数量等于该周期内可投放数量与上一周期的剩余数量之和。在满足一定的约束条件下实现对所有小区的生活物资多周期的最优分配目标,构建基于储备中心-集散点-小区的三级有序拓扑网络如下图所示。

\begin{figure}[h]
    \centering
    \includegraphics[width=\textwidth]{image.png}
    \caption{基于储备中心-集散地-小区的三级有序拓扑网络图}
    \label{fig:network}
\end{figure}

\subsection{模型假设}

(1) 假设应急蔬菜物资在分配过程当中运输车辆充足,道路通畅;

(2) 以一天(24h)为一个周期对应急蔬菜物资进行分配;

(3) 不同品种应急蔬菜物资(如青椒、土豆、白菜等)可以混装进行分配,但不同同品种应急蔬菜物资之间不存在替代效用;

(4) 假设长春市每个区的储备中心位置均为各区的聚类中心点。

\subsection{参数定义}

在构建模型过程中相关参数和变量的数学符号进行如下描述:

\begin{itemize}
    \item $I$: 物资储备中心集合,$i \in I$;
    \item $J$: 物资集散点集合,$j \in J$;
    \item $M$: 小区集合,$m \in M$;
    \item $N$: 应急蔬菜物资品种集合,$n \in N$;
    \item $K$: 应急蔬菜物资分配的规划周期集合,$k \in K$;
    \item $x_{ijn}^k$: 在 $k$ 周期时储备中心 $i \in I$ 分配应急蔬菜物资 $n \in N$ 到集散点 $j \in J$ 的数量;
    \item $x_{jmn}^k$: 在 $k$ 周期时集散点 $j \in J$ 分配应急蔬菜物资 $n \in N$ 到小区 $m \in M$ 的数量;
    \item $\gamma_{mn}^k$: 应急蔬菜物资不足的延迟损失函数;
    \item $d_{ij}$: 从储备中心 $i \in I$ 到集散点 $j \in J$ 的最短距离;
    \item $d_{jm}$: 从集散点 $i \in I$ 到小区 $j \in J$ 的最短距离;
\end{itemize}

\subsection{模型构建}

首先利用比例短缺的延迟损失度量公平性。在 $k \in K$ 周期内小区 $m \in M$ 对于应急蔬菜物资 $n \in N$ 的比例短缺值利用如下公式计算:

\begin{equation}
P_{mn}^k = \frac{D_{mn}^{'k} - \sum_{j \in J} x_{jmn}^k}{D_{mn}^{'k}}
\tag{8.1}
\end{equation}

其中,$D_{mn}^{'k}$ 为 $k \in K$ 周期内小区 $m \in M$ 对于蔬菜物资 $n \in N$ 的实际发放量;$\sum_{j \in J} x_{jmn}^k$ 为 $k \in K$ 周期从集散点 $j \in J$ 分配蔬菜物资 $n \in N$ 到小区 $m \in M$ 的总数量。

根据上面的分析,构建基于储备中心-集散点-小区的三级拓扑网络应急蔬菜物资多周期分配优化模型:

(1) 目标函数
\begin{align}
\min G_1 &= \sum_{m \in M} \sum_{n \in N} \sum_{k \in K} \gamma_{mn}^k (P_{mn}^k) \tag{8.2} \\
\min G_2 &= \sum_{i \in I} \sum_{j \in J} \sum_{n \in N} \sum_{k \in K} c_{ijn}^k x_{ijn}^k d_{ij} + \sum_{i \in I} \sum_{j \in J} \sum_{n \in N} \sum_{k \in K} c_{jmn}^k x_{jmn}^k d_{jm} \tag{8.3} \\
\min G_3 &= \sum_{i \in I} \sum_{j \in J} \sum_{n \in N} \sum_{k \in K} a_m x_{ijn}^k d_{ij} + \sum_{j \in J} \sum_{m \in M} \sum_{n \in N} \sum_{k \in K} a_m x_{jmn}^k d_{jm} \tag{8.4}
\end{align}

其中,目标函数 $G_1$ 表示最小化所有应急周期所有小区所有物资短缺的延迟损失;目标函数 $G_2$ 表示最小化所有应急周期物资分配的总成本,包括从储备中心到集散点以及从集散点到小区两部分;目标函数 $G_3$ 表示最小化所有应急周期物资分配的工作量,按运输里程与小区居民人数乘积计算。

(2) 约束条件
\begin{equation}
s.t.
\begin{cases}
\sum_{j \in J} x_{jmn}^k + P_{mn}^k = D_{mn}^k + P_{mn}^{k-1} & \forall m \in M, n \in N, k \in K \\
\sum_{m \in M} x_{jmn}^k + Q_{jn}^k = \sum_{i \in I} x_{ijn}^k + Q_{jn}^{k-1} & \forall j \in J, n \in N, k \in K \\
\sum_{j \in J} x_{ijn}^k + Q_{in}^{'k} = Q_{in}^k + Q_{in}^{'k-1} & \forall i \in I, n \in N, k \in K \\
\sum_{n \in N} x_{ijn}^k \leq Q_{jn}^k \cdot z_{ij}^k & \forall i \in I, j \in J, k \in K \\
\sum_{n \in N} x_{jmn}^k \leq Q_{jm}^k \cdot z_{jm}^k & \forall j \in J, m \in M, k \in K \\
z_{ij}^k \in \{0, 1\} & \forall i \in I, j \in J, k \in K \\
z_{jm}^k \in \{0, 1\} & \forall j \in J, m \in M, k \in K \\
x_{ijn}^k \geq 0 & \forall i \in I, j \in J, n \in N, k \in K \\
x_{jmn}^k \geq 0 & \forall j \in J, m \in M, n \in N, k \in K \\
P_{mn}^k \geq 0 & \forall m \in M, n \in N, k \in K
\end{cases} \tag{8.5}
\end{equation}

上述约束条件分别表示平衡每个小区的物资不足情况,即不可以在物资需求出现之前预先分配蔬菜物资;表示为集散点的蔬菜物资供应量守恒约束,确保在给定的周期内接收和自采的量在某一周期超过投放的部分可转入到未来周期中予以分配;表示为储备中心的蔬菜物资供应量守恒约束,能够确保从储备中心到集散点的蔬菜物资投放量不超过储备中心在该周期内的可用蔬菜物资数量;表示从储备中心到集散点以及从集散点到小区分配应急蔬菜物资的运输能力限制;表示为二进制变量,并且取值必须为 0 或 1;最后表示决策变量的非负性。

\subsection*{8.2.4 模型求解}

由于问题二中介绍的传统遗传算法在研究储备中心选址与生活物资调配相结合的问题上存在一些局限性。针对该问题,我们从需求点和生活物资科学发放视角出发,提出自适应遗传算法(AGA),结合储备中心实际情况利用具有方向性的初始群体生成法来提高算法的搜索速度,设计自适应交叉和变异算子使得 AGA 算法在进化速度与解的质量之间能够进行权衡,并得到全局最优解。

将交叉概率与变异概率均设计成与进化代数 \( t \) 有关的函数,并将自适应交叉、变异参数设置为 \( p_{c0} = 0.1 \), \( p_{c1} = 0.6 \), \( p_{c2} = 0.9 \), \( p_{m0} = 0.001 \), \( p_{m1} = 0.1 \), \( p_{m2} = 0.5 \),算法中其余参数均与问题二中遗传算法参数设置一致。

在相同参数下进行 100 次的进化迭代寻优,得到 AGA 算法的适应度进化曲线如下图所示:

\begin{figure}[h]
    \centering
    \includegraphics[width=0.8\textwidth]{image1.png}
    \caption{AGA 算法适应度进化曲线}
    \label{fig:aga_curve}
\end{figure}

由上图可以看出,由于初始种群生成的方向性操作,使得 AGA 算法能够在迭代前期就达到很高的适应度,并且在迭代 70 次时可以有效的求得全局 Pareto 最优解集在三维空间分布如下图所示。

\begin{figure}[h]
    \centering
    \includegraphics[width=0.8\textwidth]{image2.png}
    \caption{Pareto 最优解集在三维空间分布}
    \label{fig:pareto_distribution}
\end{figure}

\subsection{8.2.5 结果分析}

利用 Matlab 软件将上述模型求解,仅考虑直线距离,结果如图 8-5 所示。图 8-5 展示了储备中心与集散点位置及运输路径图,属于三级拓扑配送网络中的一级到二级单位配送路径。图 8-5 中绿色三角形表示储备中心,红色五角星代表集散中心,其余为小区位置。此时仅考虑直线最短距离,求解得到了 9 个储备中心,26 个集散点。可以覆盖整体长春市的防疫物资配送。

\begin{figure}[h]
    \centering
    \includegraphics[width=\textwidth]{image1.png}
    \caption{储备中心到集散地的配送网络图}
    \label{fig:8-5}
\end{figure}

图 8-6 为集散地到小区的配送网络图,属于三级拓扑配送网络中的二级单位到三级单位的配送网络。

\begin{figure}[h]
    \centering
    \includegraphics[width=\textwidth]{image2.png}
    \caption{集散地到小区的配送网络图}
    \label{fig:8-6}
\end{figure}

将长春市 9 个区的储备中心和集散地的坐标制作成表 8.1。

\begin{table}
\caption{储备中心和储备中心坐标}
\begin{tabular}{l l l}
\hline
区域名称 & 储备中心 & 集散地 \\
\hline
长春新区 & (38.621, 20.448) & (43.469, 27.819) \\
 & & (35.945, 10.836) \\
 & & (36.276, 22.427) \\
 & & (25.478, 30.597) \\
汽开区 & (34.200, 36.742) & (42.291, 41.735) \\
 & & (33.330, 36.743) \\
 & & (58.698, 28.415) \\
南关区 & (57.475, 33.624) & (60.385, 44.208) \\
 & & (49.289, 16.795) \\
 & & (34.841, 46.968) \\
绿园区 & (42.409, 49.014) & (45.683, 46.951) \\
 & & (45.103, 54.839) \\
 & & (50.431, 64.395) \\
宽城区 & (55.550, 59.817) & (58.767, 60.906) \\
 & & (57.503, 52.344) \\
 & & (84.812, 21.178) \\
净月区 & (76.427, 18.143) & (68.499, 21.581) \\
 & & (81.448, 9.125) \\
 & & (79.289, 43.006) \\
经开区 & (74.290, 37.437) & (67.089, 33.76) \\
 & & (74.443, 33.004) \\
 & & (69.148, 40.58) \\
二道区 & (69.154, 46.687) & (69.005, 57.709) \\
 & & (69.280, 46.919) \\
 & & (45.349, 33.892) \\
朝阳区 & (49.255, 37.779) & (54.556, 37.432) \\
 & & (50.655, 44.301) \\
\hline
\end{tabular}
\end{table}

由于图8-6数据密度,因此将长春新区的储备中心和集散地网络配送关系进行展示,如图8-7所示。从8-7可以看出,储备中心直线到达集散地。集散地在到达小区,三个集散地分别服务三个区域的小区,将其用绿色、粉色、蓝色线进行区别。由此可见,选择的储备中心与集散地满足服务长春新区。

\begin{figure}[h]
    \centering
    \includegraphics[width=\textwidth]{image.png}
    \caption{长春市储备中心和集散点的网络配送图}
    \label{fig:network_delivery}
\end{figure}

根据上述设置网络配送图,我们展示了长春市长春新区一个周期内的投放蔬菜包方式,具体方案见表\ref{tab:transport_allocation}。

\begin{table}[h]
    \centering
    \caption{各小区一周之间的运输分配方案}
    \label{tab:transport_allocation}
    \begin{tabular}{c c c c c c}
        \hline
        小区 & 类别 & 第1天 & 第2天 & \dots & 第7天 & 总分配量(吨) \\
        \hline
        & 大白菜 & 2.5 & 1.3 & \dots & & 13.8 \\
        & 土豆 & & 1.3 & \dots & 0.9 & 10.4 \\
        小区1 & 西红柿 & 1.2 & & \dots & 1.6 & 9.2 \\
        & 黄瓜 & 0.8 & & \dots & 2.7 & 12.6 \\
        & 青椒 & & 0.7 & \dots & & 8.1 \\
        & 圆葱 & 0.6 & & \dots & 1.1 & 14.6 \\
        & 白萝卜 & 1.5 & 1.7 & \dots & & 10.3 \\
        小区2 & 西红柿 & & & \dots & 0.8 & 9.6 \\
        & 土豆 & & 1.2 & \dots & 3.1 & 8.4 \\
        & 生菜 & 4.0 & 2.4 & \dots & & 15.9 \\
        \dots & \dots & \dots & \dots & \dots & \dots & \dots \\
        & 西红柿 & & 3.7 & \dots & 0.7 & 14.3 \\
        & 生菜 & 0.8 & & \dots & & 9.7 \\
        小区153 & 青椒 & 1.1 & & \dots & 1.0 & 5.8 \\
        & 圆葱 & 3.2 & 0.3 & \dots & & 10.8 \\
        & 黄瓜 & & 4.2 & \dots & 3.6 & 12.1 \\
        \hline
    \end{tabular}
\end{table}

长春新区包含153个小区,发放的蔬菜种类包含大白菜、土豆、西红柿等十余种菜类。我们的蔬菜发放预案设定每七天为一个周期,每个周期结束后会对小区发放的蔬菜种类进行灵活变更。小区每天都进行蔬菜包的发放:随机发放三种蔬菜,蔬菜吨数不定。

物资数量符合长春市民基本物资生活保障标准。

\section{基于蚁群算法的物资配送路径规划模型}

针对生活物资配送的最优路径规划问题,考虑长春市真实街道信息,对从储备中心出发经过集散点最后到达各小区的最优行驶路线进行规划,并通过附件 3 的道路长度以及道路宽度信息,计算获得从储备中心 $i \in I$ 到集散点 $j \in J$ 的最短距离 $d_{ij}$ 以及从集散点 $i \in I$ 到小区 $j \in J$ 的最短距离 $d_{jm}$,并应用到三级拓扑网络的生活物资多周期分配优化模型中进行科学发放物资。

\subsection{蚁群算法}

蚁群算法是模仿蚂蚁觅食行为的一种算法,蚂蚁在觅食过程中会留下信息素,后期跟进的蚂蚁则根据路上的信息素浓度高低来寻找正确且适宜的路径进行抉择,原理如下图 8-8 所示。

\begin{figure}[h]
    \centering
    \begin{subfigure}[b]{0.3\textwidth}
        \includegraphics[width=\textwidth]{ant_search_early.png}
        \caption{蚂蚁寻路前期}
    \end{subfigure}
    \hfill
    \begin{subfigure}[b]{0.3\textwidth}
        \includegraphics[width=\textwidth]{ant_search_mid.png}
        \caption{蚂蚁寻路中期}
    \end{subfigure}
    \hfill
    \begin{subfigure}[b]{0.3\textwidth}
        \includegraphics[width=\textwidth]{ant_search_late.png}
        \caption{蚂蚁寻路后期}
    \end{subfigure}
    \caption{蚁群算法原理图}
    \label{fig:ant_algorithm}
\end{figure}

利用蚁群算法在真实街道中对分配应急蔬菜物资的路径进行规划,要求蚂蚁从储备中心开始寻找最优路径抵达各个集散点,并从集散点寻找最短路径将蔬菜物资运送到各个小区。蚁群在进行节点路径的选择中,根据信息素浓度计算周围自由节点的转移概率,并根据如下式子进行信息素更新:

\begin{equation}
\begin{cases}
\tau_{ij}(t+1) = (1-\rho)\tau_{ij}(t) + \Delta\tau_{ij}(t, t+1) \\
\Delta\tau_{ij}(t, t+1) = \sum_{k=1}^{m} \Delta\tau_{ij}^{k}(t, t+1)
\end{cases}
\tag{8.6}
\end{equation}

在该过程中,每个蚂蚁即代表了每个车辆,每只蚂蚁所走的路径即为车辆配送物资的一个可行性路径,将这其中找到最优的路径进行保留,其结果即为长春市九个区的储备中心到各个小区的最优路径,算法的具体流程如下图 8-9 所示。

\begin{figure}[h]
    \centering
    \includegraphics[width=0.8\textwidth]{ant_algorithm_flowchart.png}
    \caption{蚁群算法流程图}
    \label{fig:ant_algorithm_flowchart}
\end{figure}

\begin{figure}[h]
    \centering
    \includegraphics[width=\textwidth]{ant_colony_flowchart.png}
    \caption{蚁群算法流程图}
    \label{fig:ant_colony_flowchart}
\end{figure}

图 8-10 为考虑实际街道下长春新区的配送网络图。橘黄色三角形为储量中心,紫色五角星为集散地。根据蚁群算法优化模型得到最优配送路径。由于考虑实际街道,街道存在道路交叉口、拐角等复杂位置。从图中可以看出,储量中心达到集散地主要考虑的是直线街道,单方向行走路线。集散地到达小区的配送路径,选择的道路多为可以覆盖多个小区。散个集散地分别服务三个不同区域的小区。从图中可以看出,左上方的集散地主要服务粉色小区,路径为粉色。中间偏上的集散地主要服务绿色小区。左下集散地主要服务与蓝色区域的小区。综上所述,集散地可以服务长春新区整个小区对防疫物资配送。

\begin{figure}[h]
    \centering
    \includegraphics[width=\textwidth]{image.png}
    \caption{考虑实际街道下配送网络图}
    \label{fig:delivery_network}
\end{figure}

我们展示了长春市长春新区在考虑了道路节点后的物资调运方式。每个小区的调运方式服从三级拓扑网络图:储备中心-集散地-小区,并回到集散地进行下一次的物资调运,调运的蔬菜包总量即为当前小区需要投放的物资数量之和,见表\ref{tab:delivery_paths}。

\begin{table}[h]
    \centering
    \caption{路径方案与卡车配送能力(考虑实际道路)}
    \label{tab:delivery_paths}
    \begin{tabular}{l l r}
        \hline
        小区 & 调运路径(道路编号) & 调运蔬菜包总量(吨) \\
        \hline
        小区1 & 68-50-18399-6393-9404-20-50 & 4.3 \\
        小区2 & 68-50-292-140-2184-7593-73-38-50 & 3.6 \\
        小区3 & 68-50-18399-6393-40-393-8930-20-50 & 2.7 \\
        小区4 & 68-50-392-383-938-50 & 5.1 \\
        小区5 & 68-50-28-849-193-11-50 & 3.9 \\
        \dots & \dots & \dots \\
        小区150 & 68-50-32-50 & 2.2 \\
        小区151 & 68-1939-149-30-1939 & 3.5 \\
        小区152 & 68-50-1030-19-40-18494-50 & 2.6 \\
        小区153 & 68-1939-832-1092-334-345-103-1939 & 2.9 \\
        \hline
    \end{tabular}
\end{table}

考虑最小配送时间目标,基于不同卡车承载规模约束,利用“顺路”原则将物资零散配送到多个需求小区得到不同物资分配方案,计算结果如图\ref{fig:delivery_results}所示。从图\ref{fig:delivery_results}可以看出,橘黄色三角形为储量中心,紫色五角星为集散地。增加了多卡车不同承载能力影

\begin{figure}[h]
    \centering
    \includegraphics[width=\textwidth]{image.png}
    \caption{考虑多卡车不同承载能力的配送网络图}
    \label{fig:delivery_network}
\end{figure}

在考虑了卡车调运的约束条件后,一辆大卡车可携带的物资数量足以满足多个小区的物资需求。这样相邻近的小区即可被优先送达,具体方案见表\ref{tab:delivery_plan}。

\begin{table}[h]
    \centering
    \caption{路径方案与卡车配送能力(考虑卡车承载能力)}
    \label{tab:delivery_plan}
    \begin{tabular}{l l l}
        \hline
        小区 & 调运路径 & 调运蔬菜包总量(吨) \\
        \hline
        小区1、49、129、148 & 50-18399-6393-8488-44-49-50 & 33.4 \\
        小区2、88、135 & 50-292-140-2184-7593-73-38-50 & 48.2 \\
        小区3、23、43、78 & 50-75-394-43-50 & 32.9 \\
        小区4、78、153 & 50-392-383-938-758-50 & 25.7 \\
        小区5、9、80、121、135 & 50-28-466-99-50 & 36.3 \\
        \dots & \dots & \dots \\
        小区57、93、103、120 & 1939-832-1092-3503-194-12-1939 & 20.3 \\
        小区76、78、90 & 50-233-38-3875-188-50 & 35.7 \\
        \hline
    \end{tabular}
\end{table}

\section*{九、灵敏度分析}

为了分析在不同参数的设定下对模型求解结果的影响,本节将会分别从改变服务半径的大小、储备中心的个数、单位运输成本来进行灵敏度分析。

\subsubsection{(1) 储存中心规模半径灵敏度分析}

现将储存中心规模半径设置为 \(120 \mathrm{~m} 、 125 \mathrm{~m} 、 130 \mathrm{~m} 、 135 \mathrm{~m}\),并得到与目标函数之间的关系,如图9-1所示。从图9-1可以看出随着半径的增大,目标函数整体增大的变化趋势。在4个半径数值中,半径 \(120 \mathrm{~m}\) 时目标函数最小。

\begin{figure}[h]
    \centering
    \includegraphics[width=\textwidth]{image1.png}
    \caption{储存中心规模半径灵敏度分析}
    \label{fig:9-1}
\end{figure}

\subsubsection{(2) 储存中心个数灵敏度分析}

现将储存中心个数设置为 \(3 、 4 、 5 、 6\),并得到与目标函数之间的关系,如图9-2所示。随着存储中心数量增大,目标函数整体呈现降低的变化趋势。运输成本和时效受到的影响最大。

\begin{figure}[h]
    \centering
    \includegraphics[width=\textwidth]{image2.png}
    \caption{储存中心个数灵敏度分析}
    \label{fig:9-2}
\end{figure}

(3) 运输成本灵敏度分析

现将单位运输成本设置为 0.072、0.076、0.08、0.084,并得到与目标函数之间的关系,如图 9-3 所示。随着单位运输成本增大,目标函数整体呈现上升的变化趋势。运输成本和时效受到的影响最大。

\begin{figure}[h]
\centering
\includegraphics[width=\textwidth]{image.png}
\caption{单位运输成本灵敏度分析}
\label{fig:9-3}
\end{figure}

\section{十、模型的评价}

\subsection{10.1 模型的优点}

\begin{itemize}
    \item 建立了熵权-模糊综合评价模型对长春市 9 个区投放点数量分布合理性的进行评分
    \item 建立了投影分析的 DEA 反馈调整模型对蔬菜包分配不合理的方案进行反馈
    \item 提出了储备中心-集散点-小区的三级有序拓扑网络
    \item 建立了基于蚁群算法的物资配送路径规划模型,利用自适应遗传算法求解模型得到各区储备中心到各小区的三级最优路径
\end{itemize}

\subsection{10.2 模型的缺点}

\begin{itemize}
    \item 使用 DEA 投影分析进行方案反馈调整,虽然可以从改进方向与改进幅度上为长春市蔬菜包分配提供决策调整依据,但实际调整工作会受诸多因素限制,很难一概而论,因此反馈调整模型有待进一步的优化完善。
    \item 使用遗传算法对多目标选址模型进行了求解,此算法也存在着容易进入局部最优问题。
\end{itemize}

\subsection{10.3 模型的改进及推广}

\begin{itemize}
    \item 模型适用于解决疫情期间生活物资的科学管理问题。
    \item 模型不仅适用于新冠疫情物资储存中心选址,还适用于各类突发事件的救援物资选址问题。
\end{itemize}

\section{十一、参考文献}

[1] 王琦. 成品粮应急储备库的选址研究[D]. 河南工业大学, 2017.

[2] 王紫岩. 新冠疫情背景下的防疫物资储备中心选址-分配研究[D]. 东华大学, 2022. DOI:10.27012/d.cnki.gdhuu.2022.001621.