\begin{center}
\includegraphics[width=0.2\textwidth]{image1.png} \quad
\includegraphics[width=0.2\textwidth]{image2.png} \quad
\includegraphics[width=0.2\textwidth]{image3.png} \quad
\includegraphics[width=0.2\textwidth]{image4.png}
\end{center}

\begin{center}
\textbf{中国研究生创新实践系列大赛}
\end{center}

\begin{center}
\textbf{中国光谷·“华为杯”第十九届中国研究生}
\end{center}

\begin{center}
\textbf{数学建模竞赛}
\end{center}

\begin{table}[h]
\centering
\begin{tabular}{ll}
学 & 校 \\
 & 上海海事大学 \\
\end{tabular}
\end{table}

\begin{table}[h]
\centering
\begin{tabular}{ll}
参赛队号 & 22102540295 \\
\end{tabular}
\end{table}

\begin{table}[h]
\centering
\begin{tabular}{ll}
队员姓名 & 1. 栗语璇 \\
 & 2. 马睿 \\
 & 3. 李姚娜 \\
\end{tabular}
\end{table}

\begin{center}
\textbf{中国研究生创新实践系列大赛}
\end{center}

\section*{中国光谷·“华为杯”第十九届中国研究生数学建模竞赛}

\section*{题目 COVID-19 疫情期间生活物资的科学管理问题建模与优化}

\section*{摘 要:}

2020 年爆发的新冠肺炎疫情,时至今日国内和国际仍呈现多发频发态势。受国际环境更趋复杂严峻和国内疫情冲击明显的超预期影响,我国经济下行的压力进一步增大。面对复杂的防疫局势,建立规范化的疫情快速清零机制十分必要,特别是亟需解决居民生活物资发放的科学管理问题。生活物资发放是否科学,不仅影响隔离居民的基本生活,而且影响着疫情的二次传播。与此同时,应急物资管理本身就是一项复杂的系统工程,具有非线性、不确定性、动态性,给抗疫过程中的生活物资发放带来了巨大挑战,以长春市抗疫过程中的数据为依托,利用数据挖掘、数据分析、统计建模、运筹优化等方法探索一疫情下兼顾防疫安全与效率的科学物资管理方案既具现实意义又具研究价值。

问题一的主要目的探究生活物资的大规模流动方式对疫情的影响。对长春市 2022 年 3 月 26 日实行发放蔬菜包前后防控工作的效果进行判别与分析。本部分的研究分为三部分,包括可视化分析、统计分析(统计特征指标和时间序列统计特征指标)、SEIR 预测对比分析(4.2 节)。分析结果显示:(1)疫情的发展或控制与生活物资发放方式有关。长春市实行的蔬菜包方式是疫情期间为保障居民生活和减少疫情传播的有效发放方法;(2)在发放初期由于经验不足、备货力度不够、保供人员缺少、供货源不稳定等问题,使感染人数不减反升;(3)蔬菜包的发放时间间隔越大,能够减少人员接触频率,减轻保供工作人员的工作强度,减少疫情传播的风险。

问题二的主要目的解决生活物资投放点数量与位置问题。子问题一是要求讨论投放点数量的合理性并优化,子问题二和子问题三是在子问题一的基础上对应急物资供应链上游的大规模物资分拣场所与政府储备物资的规划。针对子问题一,本文首先对隔离人口数和生活物资投放点数量做 Person 相关系数,发现隔离人口数与生活物资投放点数量之间是极低的正相关,表明投放点的数量与分配方案不合理(5.2 节)。其次,采用层次分析法计算长春市生活物资投放点数量与分配的影响因素指标权重,综合评价投放点数量的合理性(5.2 节),发现投放点的数量严重冗余、分配不合理。在此基础上,修正附件 2 投放点数量得到各区优化调整的投放数量(表 5-8)。针对子问题二,本文首先使用平均法和数据挖掘估算出各区生活必需品的需求情况,依次制定政府储备物资安全库存(表 5-9)。针对子问题三,本文首先建立采用 K-means 聚类算法和遗传算法求解政府储备物资和大规模物资分拣场所的位置、数量规模与所服务的区域并提出最优的选址数量、规模(表 5-13)。

问题三的主要目的是优化生活物资的发放效果。子问题一是分析蔬菜包需求和发放规律,子问题二是评价和调整 4 月 10 日至 4 月 15 日蔬菜包供应方案。针对子问题一,本文采用文本挖掘技术,统计了 2022 年 3 月 26 日到 5 月 1 日九区总体和分区的每日蔬菜包供给量和需求量。分别从蔬菜包的供应规律、需求规律和供需规律三方面切入,运用描述性统计、正态性检验、相关性分析的方法结合新冠肺炎感染人数变化(6.2 节),多层次分

析蔬菜包的供需规律。针对问题二,本文采用 TOPSIS 综合评价方法,建立包含:各区本土感染人数和无症状感染者人数,各区小区栋数、小区户数和小区人口数、街道数等指标构成的评价体系,计算各指标权重,获得各区的得分与排名(表 6-5)。为评价蔬菜包供应方案的调整效果,本文利用数据挖掘技术统计出 4 月 10 日至 4 月 15 日,各区每天的蔬菜包接收数目和发放数目(表 6-9)。本文将蔬菜包浪费或缺货数目作为蔬菜包发放效果的评价指标,计算出各区蔬菜包供应的优化方案与调整幅度(表 6-10)。从各区调整效果对比(表 6-11)结果显示,采用此优化方案九个区的蔬菜包浪费量或缺货问题得到明显改善。

问题四的目的是为长春市做好大规模封控情况下居民生活物资有序发放预案。子问题一是对上游物资来源地进行合理选址,子问题二是对中游物资集散地进行合理分配与选址,子问题三是在考虑车辆限重的情况下,将“工作量”(工作量 = 运输里程 × 小区居民人数)的最小化合理规划末端配送路径以及配送量。针对子问题一,采用射线法“由点连面”判定交通路口节点的区域划分,同时能够过滤掉冗余的交通路口节点信息。在此基础上,采用枚举法在保留下来的交通路口节点中寻找与所有小区的“工作量”指标之和最小的位置。每个区重复上述方法便可得到各区的中心点坐标,即为九个上游物资来源地位置(表 7-2)。针对子问题二,本文采用轴辐射网络对中游的集散地进行选址和分配(表 7-4)。针对子问题三,本文考虑病毒传播和车量容量限制的路径优化模型(7.4 节)。鉴于此问题属于 NP-hard 问题,本文采用大规模邻域算法(ALNS)和模拟退火算法(SA)构成的混合启发式算法进行求解,最终得到由上游物资来源地-中游货物集散地-下游小区的多级有序物流网络(图 7-4)、最短配送路径以及合理的配送车辆数量与种类(7.5 节)和有序网络可视化(7.6 节)。最后对模型的改进与推广提出展望。

关键词:多层物流网络优化;选址与路径优化;数据挖掘;K-means 聚类;传染病模型;综合评价方法;相关性分析;时间序列分析;混合启发式算法;NP-hard;应急物流

\section*{目录}
\begin{itemize}
    \item[] 目录 \dotfill 3
    \item[] 一、问题重述 \dotfill 5
    \begin{itemize}
        \item[] 1.1. 问题背景 \dotfill 5
        \item[] 1.2. 问题提出 \dotfill 6
    \end{itemize}
    \item[] 二、模型假设 \dotfill 6
    \item[] 三、符号说明 \dotfill 7
    \item[] 四、问题一分析与求解 \dotfill 7
    \begin{itemize}
        \item[] 4.1. 问题一分析:生活物资发放方式对疫情的影响分析 \dotfill 7
        \item[] 4.2. 评判发放蔬菜包对疫情的影响 \dotfill 8
        \begin{itemize}
            \item[] 4.2.1. 可视化分析 \dotfill 8
            \item[] 4.2.2. 统计分析 \dotfill 11
            \item[] 4.2.3. 预测对比分析 \dotfill 12
        \end{itemize}
        \item[] 4.3. 结果分析 \dotfill 13
    \end{itemize}
    \item[] 五、问题二分析与求解 \dotfill 14
    \begin{itemize}
        \item[] 5.1. 问题二分析:投放点合理性评价与优化 \dotfill 14
        \item[] 5.2. 子问题1:投放点数量的合理性评价与优化——Person、层次分析法 \dotfill 15
        \begin{itemize}
            \item[] 5.2.1. 投放点数量的评价 \dotfill 15
            \item[] 5.2.2. 投放点数量的优化 \dotfill 19
        \end{itemize}
        \item[] 5.3. 子问题2:政府储备物资规划——平均法和数据挖掘 \dotfill 20
        \item[] 5.4. 子问题3:大规模物资分拣场所选址与优化——K-means聚类和遗传算法的综合算法 \dotfill 21
        \begin{itemize}
            \item[] 5.4.1. 基于K-means的大规模物资分拣场所选址 \dotfill 21
            \item[] 5.4.2. 基于遗传算法的选址优化模型 \dotfill 25
            \item[] 5.4.3. 计算结果 \dotfill 27
        \end{itemize}
    \end{itemize}
    \item[] 六、问题三分析与求解 \dotfill 28
    \begin{itemize}
        \item[] 6.1. 问题三分析 \dotfill 28
        \item[] 6.2. 子问题1:蔬菜包的供应规律和需求规律——数据挖掘和数据分析 \dotfill 29
        \begin{itemize}
            \item[] 6.2.1. 蔬菜包的供应规律 \dotfill 29
            \item[] 6.2.2. 蔬菜包的需求规律 \dotfill 35
            \item[] 6.2.3. 蔬菜包的供需规律 \dotfill 39
        \end{itemize}
    \end{itemize}
\end{itemize}

\begin{itemize}
    \item 6.3. 子问题2:调整蔬菜包供应方案——TOPSIS综合评价与优化 \dotfill 40
    \item 6.3.1. TOPSIS介绍 \dotfill 40
    \item 6.3.2. 评价指标体系 \dotfill 42
    \item 6.3.3. 评价结果 \dotfill 42
    \item 6.3.4. 结果分析 \dotfill 45
    \item 七、问题四分析与求解 \dotfill 49
    \item 7.1. 问题分析 \dotfill 49
    \item 7.2. 上游物资来源地的选址——射线法、枚举法 \dotfill 50
    \item 7.2.1. 划分交通路口点的区域 \dotfill 50
    \item 7.2.2. 确定上游物资来源地在各区的选址 \dotfill 51
    \item 7.3. 中游集散点的辐射网络模型 \dotfill 51
    \item 7.4. 考虑病毒传播和车量容量限制的路径优化模型 \dotfill 53
    \item 7.4.1 基于复杂网络的病毒传播模型 \dotfill 53
    \item 7.4.2 考虑节省人力,减少人员的直接、间接接触的配送路径优化模型 \dotfill 54
    \item 7.4.3 混合元启发式算法 \dotfill 56
    \item 7.5. 结果分析 \dotfill 59
    \item 7.6. 有序网络结果可视化 \dotfill 61
    \item 八、模型的改进与推广 \dotfill 62
    \item 8.1. 模型的优点 \dotfill 62
    \item 8.2. 模型的缺点 \dotfill 62
    \item 8.3. 模型的改进与推广 \dotfill 63
    \item 参考文献 \dotfill 64
    \item 附录 Python 程序 \dotfill 65
    \item 问题一算法程序 \dotfill 65
    \item 问题二算法程序 \dotfill 66
    \item 问题三算法程序 \dotfill 71
    \item 问题四算法程序 \dotfill 73
\end{itemize}

\section{问题重述}

\subsection{问题背景}

2020年爆发的新冠肺炎疫情,时至今日国内和国际仍呈现多发频发态势。2022年以来由于新冠病毒变异株奥密克戎 BA.2 的高传播性、高传染性和高隐匿性,新一轮新冠肺炎疫情在我国多地散发,陆续发生多次大规模疫情爆发危机。凭借我国的制度优势,在疫情大规模爆发期间采取封闭式管理方式实现了疫情的快速清零。我国现行新冠肺炎疫情防控政策来看,涉及到封闭式管理方式的情况主要有两种,一是在静态管理情况下的“三个暂停”,即除参与防疫人员外,全市所有行政事业单位人员全部居家办公;除相关重点企业、保障民生的公共服务类企业外,所有经营性场所暂停营业;除具备条件的集散点、药店、医疗机构等外,其他商户暂停营业;出租车、网约车等车辆停运。以及三个“不”,即:指居民不聚集、不流动、不出门\cite{ref1}。二是对在风险区的划定与管控中划分为高风险区的居住小区(村)实行封控措施。实行“足不出户、上门服务”的政策,只有当连续7天无新增感染者,且第7天风险区域内所有人员完成一轮核酸筛查均为阴性时,才能降为中风险地区。连续3天无新增感染者时,才能降为低风险区\cite{ref2}。由此可见,封闭式管理的目的是减少城市人口流动,同时开展大规模核酸检测,实现以最快速度排查潜在感染者,从而切断疫情传播链。

但是,受国际环境更趋复杂严峻和国内疫情冲击明显的超预期影响,我国经济下行的压力进一步增大。面对复杂的防疫局势,从宏观调控的角度来看,需要政府高效统筹疫情防控和经济社会发展成效。从微观规制的角度来看,则急需建立规范化的疫情快速清零机制。特别是,亟需解决居民生活物资发放的科学管理问题。生活物资发放是否科学,不仅影响隔离居民的基本生活,而且很有可能在物资发放过程中造成疫情的二次传播。从学术界现有研究来看,学者们普遍认可了封闭式管理对快速抗击疫情的重要作用,认为采用封闭式应急管理模式是医院应对突发新型冠状病毒肺炎疫情的有效方式\cite{ref3}。但是,关于封闭管理下居民生活物资的发放问题的研究,目前主要聚焦在研讨居民生活物资的采购与配送方式,如刘志强等提出物业代购、电商配送等采购方式,陈镜羽等总结分析出“小区团购+集中取货”“门店共享+非接触式自提”“O2O平台+共享配送员”等末端物流配送模式\cite{ref4}。

然而,由于疫情的超传播力往往超出预期,同时应急工作的时效性、安全性和不计成本性的特点,疫情期间生活物资的管理是一项复杂的系统工程,往往需要政府居于主导地位进行统筹管理。数据表明,在疫情期间政府完全有能力调动足够数量的生活物资,只是生活物资发放给小区居民的渠道不畅\cite{ref5}。从不同城市大规模疫情的控制工作实践来看,采用蔬菜包的做法是疫情期间为居民发放生活物资的有效方式。但是,实现蔬菜包科学供应十分困难,一是居民对新鲜蔬菜需求频率高,因而配送频率也较高,容易造成人员间密切接触\cite{ref6}。同时,米、面、油及肉等基本生活物资,由于保质期较长且主要由政府储备,因此其配送任务一般穿插到蔬菜包的配送过程。另外,水果也是居民们普遍需要的,虽属于非必须食品,但如果生产出来却无法销售也对经济发展不利。

因此,蔬菜无论在数量、频次、影响等方面都更突出,故本赛题聚焦蔬菜的科学供应问题。本文致力于研究在疫情封控期间,政府如果制定科学的物资管理方案,以高效应对疫情居民生活物资发放的管理问题,旨在全面提高疫情防控效率,同时节约人员投入和经费支出。

\section*{1.2. 问题提出}

为制定科学的物资管理方案,保证蔬菜的科学供应,全面提高疫情防控效率的同时节约人员投入和经费支出。本文利用数学建模和数据挖掘技术依次解决如下问题。

\textbf{问题一:生活物资发放方式对疫情的影响分析}

结合附件 1 中所提供的长春市 COVID-19 疫情期间病毒感染人数数据及其它附件数据或团队能搜集到的数据对长春市实行发放蔬菜包前后效果进行判别与分析,为今后的防控工作提供理论依据。

\textbf{问题二:生活物资投放点数量评价与优化、物资储备与选址规划}

结合附件 2 中提供的当时长春市不同区域投放点数量分布结果。同时,结合附件 3 中和附件 4 中有关数据,讨论投放点数量的合理性,并通过数学建模进行适当的优化。另外,请充分考虑未来疫情、自然灾害等特殊事件,对于政府储备物资和大规模物资分拣场所的位置与数量规模进行合理规划,并提出最优的选址数量、规模及其潜在的备用场所位置。并将相关结果以表格的形式放置在正文中,需要包含选址位置、所属区域、选址半径、管辖范围小区个数、管辖范围内人口数等关键信息。

\textbf{问题三:生活物资的供需关系分析与分配优化}

结合附件 5 分析蔬菜包需求、发放规律,并根据附件 3 中的各小区位置与人口信息,评价并调整 4 月 10 日至 4 月 15 日蔬菜包供应方案。

\textbf{问题四:生活物资发放的多层次物流网络规划与评价}

在第二、三问的基础上,结合附件 3 给出的长春市街道和小区情况的表格,做出特殊时期保障居民生活物资供应的详细预案(有序网络图)。网络上游是各项物资来源(每个区选一个地点,参赛队可自行根据坐标选择),中游是各项物资的集散地(集散地数量自行选择,可以先按附件 2 设置,再调整优化),网络下游是长春市所有小区。物流是一个周期内各天通过网络各条边所运输的各项生活物资的数量(开始可以只考虑蔬菜,不同日期可以发送不同品种蔬菜以增加居民的蔬菜品种)。在开始时,网络的各条边可以不使用真实的街道,允许采用两点之间最短路连接。后来,可以选择少数行政区按真实街道选择路线,直至全市。考虑到特殊时期,所以节省人力是最重要的指标,同时希望减少人员的直接、间接接触。其中,工作量指标按运输里程与小区居民人数乘积计算。在完成有序网络图后需要进一步考虑用卡车运送物资,大卡车每辆可装 10 吨,小卡车每辆可装 4 吨,观察预案有无显著不同。要求在正文以图形或表格的方式精简地将相关结果表述出来,并请对照指标分析与评价所给出的发放预案的优势。

\section*{二、模型假设}

假设 1:假设题目附件数据真实可靠,适用于做数学建模、数据挖掘和数据分析;

假设 2:假设题目附件数据中的缺失值对结果不会造成重大影响;

假设 3:假设每个投送点向各小区配送的速度相同,不考虑由车的性能、天气等因素造成的配送速度变化;

假设 4:假设每个投送点向各小区配送的过程中不考虑恶劣天气、自然灾害、交通拥堵、红绿灯等意外情况;

假设 5:由于题目中没有限制货车的派车成本、装卸货成本,因此假设上述成本忽略不计;

假设 6:假设货车数量充足且不会超载行驶;

假设 7:假设每辆货车工作时间不大于 10 小时;

假设 8:假设物资投放点的容量相同;

假设 9:假设蔬菜包隔夜存放第二天会变质不能再继续发放给隔离群众;

\section*{三、符号说明}

\begin{tabular}{c l l}
\hline
No. & 符号 & 含义 \\
\hline
1 & $N$ & 2022年长春市总人口 \\
2 & $\lambda_{1}$ & 患病者每天有效接触的易感者的平均人数 \\
3 & $\lambda_{2}$ & 潜伏者每天有效接触的易感者的平均人数 \\
4 & $\delta$ & 日发病率,每天发病成为患病者的潜伏者占潜伏者总数的比例 \\
5 & $\mu$ & 日治愈率,每天治愈的患病者人数占患病者总数的比例 \\
6 & $\sigma$ & 传染期接触数 ($\sigma = \lambda_{1}/\mu$) \\
7 & $t_{End}$ & 预测日期长度 \\
8 & $i_{0}$ & 患病者比例的初值 \\
9 & $e_{0}$ & 潜伏者比例的初值 \\
10 & $s_{0}$ & 易感者比例的初值 \\
11 & $Y_{0}$ & 微分方程组的初值 \\
12 & $\overline{G}$ & 平均增长率 \\
13 & $R$ & 极差 \\
14 & $S$ & 标准差 \\
15 & $\theta_{i}$ & 交付的病毒传播风险 \\
16 & $N_{P}$ & 小区居民人数 \\
17 & $H$ & 需要选取的集散点个数 \\
18 & $Q$ & 集散点容量的集合 \\
19 & $EO, EI$ & 起始路线编号和终止路线编号 \\
20 & $F(i)$ & 路线编号终点的前序节点集合 \\
21 & $A(i)$ & 路线编号起点的后序节点集合 \\
22 & $P$ & 需要选取的集散点数量 \\
23 & $K$ & 车辆集合 \\
\hline
\end{tabular}

\section*{四、问题一分析与求解}

\subsection*{4.1. 问题一分析:生活物资发放方式对疫情的影响分析}

根据问题一要求,本题需要对长春市实行发放蔬菜包前后效果进行判别与分析,为今后的防控工作提供理论依据。首先,从定性的角度,依据附件1中所提供的长春市COVID-19疫情期间2022年3月4日到2022年5月23日的全市和九个区的新增本土感染者人数、新增无症状感染者人数,分别绘制时间序列折线图,通过折线随时间的波动变化和发展趋势来对比3月26日开始发放蔬菜包前后新增本土感染者人数和新增无症状感染人数的变化。其次,计算极差、标准差、均值、Hurst指数、平均增长率(初始值到峰值)等指标,对比蔬菜包发放前后新冠肺炎疫情时间序列统计特征的变化。最后,参考现有文献中新冠病毒奥密克戎亚型变异株BA.2的传染率[7]、恢复率[8]、潜伏期[7],运用SEIR传染病预测模型,将2022年3月25日的新增本土感染者人数作为感染者的初始人数,将同日的新增

无症状感染人数作为潜伏者的人数,通过查询长春市卫生健康委员会网站,收集同日的累计治愈人数作为治愈者的初始人数,将全市人口数减去感染者人数、潜伏者人数和治愈者人数后得到易感染者的人数。由此预测在不采用蔬菜包物资发放方式的情况下,2022年3月26日至2022年5月23日长春市每天新增本土感染人数和无症状感染人数。从而,将此预测结果与附件1的数据进行对比,量化评价蔬菜包对疫情影响。

\begin{figure}[h]
\centering
\includegraphics[width=\textwidth]{image.png}
\caption{问题一的思路流程图}
\label{fig:flowchart}
\end{figure}

\subsection{评判发放蔬菜包对疫情的影响}

\subsubsection{可视化分析}

从定性的角度,依据附件1中所提供的长春市COVID-19疫情期间2022年3月4日到2022年5月23日的全市和九个区的新增本土感染者人数、新增无症状感染者人数,分别绘制时间序列折线图,通过折线随时间的波动变化和发展趋势来对比3月26日开始发放蔬菜包前后新增本土感染者人数和新增无症状感染人数的变化。具体分析如下:

首先,计算长春市自2022年3月4日至2022年5月23日新冠肺炎疫情发展的总体情况,包括感染人数、无症状感染人数和总计感染人数,如图\ref{fig:4-2}、图\ref{fig:4-3}和图\ref{fig:4-4}所示。从图可看出自2月26日开始发放蔬菜包后,随后两天内新增无症状感染人数和总计新增人数(累计新增人数)都呈倒“V”下降趋势,新增本土感染人数也有所下降,但自3月28号后,新冠疫情总体呈上升态势并在4月2日达到顶峰,这可能是由于3月29日开始长春市东北亚粮油、海吉星两大蔬菜批发市场突发疫情而临时关闭\cite{reference}, 加上全市疫情形势严峻,蔬菜包的备货力度不足,保供人员上岗严重不足,导致市民大部分市民开始采取网上订购的方式,这一定程度上增加了在网购蔬菜的配送过程的感染风险,体现了采用蔬菜包发放方式的必要性。

随着长春市努力尽快释放蔬菜包的末端配送队伍,强化配送力量、多渠道稳定蔬菜包货源、多点位通物流和多层次保供应,自4月28日开始有序恢复生产生活秩序,全市疫

\begin{figure}[h]
    \centering
    \includegraphics[width=\textwidth]{image1.png}
    \caption{长春市全市新冠肺炎疫情发展的总体情况(2022.03.04-2022.05.23)}
    \label{fig:4-2}
\end{figure}

\begin{figure}[h]
    \centering
    \includegraphics[width=\textwidth]{image2.png}
    \caption{采用蔬菜包后长春市全市新增无症状感染者情况对比(2022.03.04-2022.05.23)}
    \label{fig:4-3}
\end{figure}

\begin{figure}[h]
    \centering
    \includegraphics[width=\textwidth]{image1.png}
    \caption{采用蔬菜包后长春市全市新增本土感染者情况对比(2022.03.04-2022.05.23)}
    \label{fig:4-4}
\end{figure}

\begin{figure}[h]
    \centering
    \includegraphics[width=\textwidth]{image2.png}
    \caption{采用蔬菜包后长春市全市新增感染者情况对比(2022.03.04-2022.05.23)}
    \label{fig:4-5}
\end{figure}

其次,分别计算长春市九个区的2022年3月4日至2022年5月23日开始新增本土感染者人数和新增无症状感染人数的变动趋势,分别如图\ref{fig:4-6}和图\ref{fig:4-7}所示。由此可看出,自3月26日开始发放蔬菜包,长春市九区的新增本土病例有所下降,虽然4月份疫情有所反弹,但整体最终趋势趋于零,体现了蔬菜包对减少疫情传播的有效性。

\begin{figure}[h]
    \centering
    \includegraphics[width=\textwidth]{image3.png}
    \caption{新增本土感染者人数变动趋势(2022.03.04-2022.05.23)}
    \label{fig:4-6}
\end{figure}

\section*{图 4-6. 长春市九区新冠肺炎新增本土病例的情况(2022.03.04-2022.05.23)}

\begin{figure}[h]
    \centering
    \includegraphics[width=\textwidth]{image.png}
    \caption{长春市九区新冠肺炎新增本土病例的情况(2022.03.04-2022.05.23)}
\end{figure}

\section*{图 4-7. 长春市九区新冠肺炎新增无症状病例的情况(2022.03.04-2022.05.23)}

\subsection{4.2.2. 统计分析}

各统计指标的介绍如下:

\subsubsection{(1) 统计特征指标}

1) 均值:是表示数据集中趋势的量数,用于描述新增确诊病例的时间序列集中趋势。其计算公式如式(4-1)所示。

\begin{equation}
\bar{x} = \frac{\sum_{i}^{n} x_{i}}{n}
\tag{4-1}
\end{equation}

2) 极差:是标志值变动的最大范围,也是测定标志变动的最简单的指标。新冠疫情时间序列统计指标中发挥着评价新增确诊病例的离散度的作用,其计算公式如式(4-2)所示:

\begin{equation}
R = x_{max} - x_{min}
\tag{4-2}
\end{equation}

3) 标准差:是描述时间序列全局结构的统计量,用于评价新增确诊病例时间序列的变化程度\cite{ref10},其计算公式如式(4-3)所示。其中,$i$ 表示新增确诊病例时间序列的时长;$x_{i}$ 表示每日新增确诊病例的均值;$n$ 代表每日新增确诊病例总时长。

\begin{equation}
s = \sqrt{\frac{\sum_{i}^{n} (x_{i} - \bar{x})^{2}}{n - 1}}
\tag{4-3}
\end{equation}

\subsubsection{(2) 时域特征指标}

Hurst 指数常用于分析时间序列的分形特征和长期记忆过程,目前在时间序列变化趋势的持续性或反持续性强度判断方面得到广泛引用。新冠疫情的 Hurst 指数可以定量描述疫情将来的发展趋势与现有趋势的关系。Hurst 指数越大表示时间序列趋势的延续性最高。

本文采用 R/S 分析即重新标度的极差分析(rescaled range analysis),计算 Hurst 指数。计算公式如式(4-4)所示。其中 $H$ 表示新增确诊病例时间序列的 Hurst 指数;$R$ 表示每日新增确诊病例每个时间序列段的最大距离;$s$ 表示每日新增确诊病例每个事件段的标准差;$K$ 为正常人口数;$t$ 表示每日新增确诊时间序列段的时间窗口长度。当 Hurst 指数为 0.5 时,时间序列记忆表现为随机游走形式,无法判定走向;当 Hurst 指数为 $[0.5, 1]$ 时,时间序列数据存在长期持续性。当 Hurst 指数为 $[0, 0.5]$ 时,时间序列数据为反持续性。

\begin{equation}
\log \left( \left( \frac{R}{s} \right)_{t} \right) = \log (K) + H \log (t)
\tag{4-4}
\end{equation}

(3) 疫情特征指标

平均增长率,可以用来反映研究期内从发现病例的第一天到新增病例达到最大值的速度,可以用来表示疫情的蔓延速度 \({}^{[10]}\),同时也能用于反映国家的防控措施是否有效。计算公式如式 (4-5) 所示。其中,\(N\) 代表每日新增确诊病例时间序列初始值到峰值的时长;\(x_{max}\) 代表每日新增确诊病例的最大值。\(x_{0}\) 代表第一个不为零的每日新增确诊病例数。

\[
\overline{G} = \sqrt[N]{\frac{x_{max}}{x_{0}}} - 1
\]

统计分析的计算结果如表 4-1 所示。从长春市新冠肺炎新增本土病例的均值来看,在 3 月 26 日使用蔬菜包之前每日平均新增本土病例为 556.59 例,而使用蔬菜包之后平均每日新增本土病例为 220.28 例,使用蔬菜包后每日平均新增本土感染人数下降 60.42%。从极差指标来看,在 3 月 26 日使用蔬菜包之前每日新增本土病例的最大值为 1997 例,最小值为 2,极差为 1977。而使用蔬菜包之后最大值为 1544,最小值为 0,极差为 1544。使用蔬菜包后,新增本土感染人数极大值降低了 21.98%,极小值出现了 0 表明疫情趋缓出现了 “零” 增长,极差降低了 21.9%。从 Hurst 指数来看,指数从 \([0.5, 1.0]\) 区间变为 \([0, 0.5]\) 区间,表明疫情的持续性有所减弱。从平均增长率来看,使用蔬菜包之前每日平均新增本土病例的平均增长率为 37%,使用蔬菜包之后平均增长率为 16%,表明使用蔬菜包之后疫情的蔓延速度明显减缓。

\textbf{表 4-1. 新冠疫情时间序列统计指标}

\begin{tabular}{c c c c c}
\hline
NO. & 统计特征分类 & 统计指标 & 计算结果 & 结果对比 \\
\hline
 & & & 3 月 4 日-3 月 25 日 & 3 月 26 日-5 月 23 日 \\
\hline
1 & 统计特征 & 均值 \((\bar{x})\) & 556.59 & 220.28 & -60.42\% \\
2 & & 极差 \((R)\) & 1977 & 1544 & -21.90\% \\
3 & & 标准差 \((s)\) & 563.76 & 345.37 & -38.74\% \\
4 & & 极大值 \((x_{max})\) & 1979 & 1544 & -21.98\% \\
5 & & 极小值 \((x_{min})\) & 2 & 0 & -2 \\
6 & 时域特征 & Hurst 指数 & 0.55 & 0.413 & -24.90\% \\
7 & 疫情特征 & 平均增长率 \((\overline{G})\) & 0.37 & 0.16 & -56.76\% \\
\hline
\end{tabular}

\subsection*{4.2.3. 预测对比分析}

运用 SEIR 传染病预测模型,将 2022 年 3 月 25 日的新增本土感染者人数作为感染者的初始人数,将同日的新增无症状感染人数作为潜伏者的人数,通过查询长春市卫生健康委员会网站,收集同日的累计治愈人数作为治愈者的初始人数,将全市人口数减去感染者人数、潜伏者人数和治愈者人数后得到易感染者的人数。模型参数如表 4-2 所示,计算结果如图 4-8 所示。预测在不采用蔬菜包物资发放方式的情况下 2022 年 3 月 26 日至 2022 年 5 月 23 日长春市每天新增本土感染人数累计 298.69 万人约占全市总人口的 35%,无症状感染人数累计达 170.68 万人约占全市总人口的 20%。与附件 1 的数据进行对比,采用蔬菜包后 2022 年 3 月 26 日至 2022 年 5 月 23 日长春市累计新增本土感染人数降低了 77.5%,无症状感染者降低了近 30%。由此可见,使用蔬菜包的方式能够有效减少感染新冠病毒的风险,减少病毒的传播,是实现快速 “清零” 和保障隔离居民生活物资的有效方式。

\textbf{表 4-2. 考虑潜伏期传染性的 SEIR 新冠感染人数预测模型参数}

\begin{tabular}{c c c}
\hline
NO. & 符号 & 含义 & 数值 \\
\hline
1 & \(N\) & 2022 年长春市总人口 & 8534000 \\
\hline
\end{tabular}

\begin{table}
\centering
\begin{tabular}{c c l c}
\hline
2 & $\lambda_{1}$ & 患病者每天有效接触的易感者的平均人数 & 1.00 \\
3 & $\lambda_{2}$ & 潜伏者每天有效接触的易感者的平均人数 & 0.25 \\
4 & $\delta$ & 日发病率,每天发病成为患病者的潜伏者占潜伏者总数的比例 & 0.05 \\
5 & $\mu$ & 日治愈率,每天治愈的患病者人数占患病者总数的比例 & 0.05 \\
6 & $\sigma$ & 传染期接触数 ($\sigma = \lambda_{1}/\mu$) & 20 \\
7 & $t_{End}$ & 预测日期长度 & 60 \\
\hline
\end{tabular}
\caption{采用蔬菜包和不采用蔬菜包的新冠感染人数对比}
\end{table}

\begin{table}
\centering
\begin{tabular}{l c c c}
\hline
 & 采用蔬菜包(万人) & 不采用蔬菜包(万人) & 结果对比 \\
\hline
累计新增本土感染人数 & 1.28 & 5.69 & -77.5\% \\
累计新增无症状感染人数 & 2.23 & 8.68 & -28.92\% \\
\hline
\end{tabular}
\end{table}

\begin{figure}[h]
\centering
\includegraphics[width=\textwidth]{image.png}
\caption{SEIR新冠感染人数预测结果可视化}
\end{figure}

\subsection{结果分析}

经过可视化分析、统计分析和预测对比分析,结果显示:疫情的发展或被控制扑灭与生活物资发放方式有关,发放蔬菜包的做法是疫情期间为居民发放生活物资的有效方式,能够有效的保证居民生活蔬菜肉蛋奶等基本生活需求,蔬菜包的发放时间间隔越大,能够减少人员接触频率,减轻保供工作人员的工作强度,减少疫情传播的风险。

但是,在蔬菜包发放的初期由于经验不足导致蔬菜包备货力度不够、保供人员缺少、供货源不稳定等问题,使感染人数有所增加。未来,为保证蔬菜包对疫情的正向控制作用,一是需要政府努力释放保供力量,包括释放蔬菜包的末端配送队伍,强化配送力量、多渠道稳定蔬菜包货源、多点位通物流和多层次保供应。二是蔬菜包等生活物资时一定要采取无接触配送的方式,不要和配送人员接触。三是蔬菜包的外包装一定要消毒处理,尽量在静置2至3个小时以后再取用,并且在冲洗时也应该佩戴一次性手套并一用一扔。

\section*{五、问题二分析与求解}

\subsection*{5.1. 问题二分析:投放点合理性评价与优化}

根据问题二要求,此题可以分为两个子问题。子问题一是要求讨论投放点数量的合理性,并通过数学建模进行适当的优化,此问题可看作是数学建模中的综合评价问题,可以通过建立评价指标体系及相应的权重系数,构造综合评价模型,计算各系统的综合评价值并得出综合评价结果,此外还需结合附件3和附件4的数据对评价对象进行优化。子问题二是要求充分考虑未来疫情、自然灾害等特殊事件,对于政府储备物资和大规模物资分拣场所的位置与数量规模进行合理规划,并提出最优的选址数量、规模及其潜在的备用场所位置。两题均要求将相关结果需要包含选址位置、所属区域、选址半径、管辖范围小区个数、管辖范围内人口数等关键信息。子问题2分解两小问,第一小问是政府储备物资的规划即对物资的库存优化,第二小问是对大规模物资分拣场所投放点进行选址规划。因此,子问题2可看作是在子问题1的基础上对投放点上游的仓库和库存进行合理规划的问题。

针对子问题一,首先对附件2的隔离人口数和生活物资投放点数量做Person相关系数分析结果为0.136,说明各区隔离人口数与生活物资投放点数量存在极低的正相关关系,隔离人口数越多投放点数量不一定多。同时也发现附件二物资投放点数量的不合理问题包括:没有考虑隔离人口数、交通路口和交通线路密度、投放点选址不明确、数据为截面数据不能体现投放点的数量随日期的动态变化情况。其次,采用层次分析法分析长春市生活物资投放点数量分布的影响因素,综合评价附件2中投放点数量的合理性。结合层次分析法的得分和修正附件2投放点数量得到各区合理的投放数量。

针对子问题二的第一小问政府储备物资的规划问题,首先采用平均法对附件2的投放点服务能力进行估算,得到平均每个投放点可以服务0.207万人。其次,依照附件4中《长春市重点民生商品供应情况表》以及商务部印发的《生活必需品市场供应保障工作手册》(2021版)中规定,每人每日消费粮食325克,食用油30克,蔬菜400克,肉类115克以及附件3中的小区人口数量,运用平均法估算出了每个区的生活必需品需求情况作为储备物资的安全库存。针对子问题二的第二小问大规模物资分拣场所进行选址规划的问题,建立采用K-means聚类算法和遗传算法求解政府储备物资和大规模物资分拣场所的位置、数量规模与所服务的区域,并提出最优的选址数量、规模及其潜在的备用场所位置等关键信息。

\begin{figure}[h]
    \centering
    \includegraphics[width=\textwidth]{image.png}
    \caption{问题二思路流程图}
    \label{fig:flowchart}
\end{figure}

\subsection{子问题 1:投放点数量的合理性评价与优化——Person、层次分析法}

\subsubsection{投放点数量的评价}

针对子问题一,首先对附件2的隔离人口数和生活物资投放点数量做Person相关系数分析结果为0.136,表5-1相关程度度量表来看,这表明各区隔离人口数与生活物资投放点数量存在极低的正相关关系,隔离人口数越多投放点数量不一定多。这与实际生活经验相违背,体现了附件2中投放点数量设置的不合理性。因此,本文假设隔离人口数对生活物资投放点数量要求越大。

\begin{table}[h]
    \centering
    \caption{相关程度度量表}
    \label{tab:correlation}
    \begin{tabular}{l l}
        \hline
        相关性 & 相关系数 \\
        \hline
        极强相关 & 0.8~1.0 \\
        强相关 & 0.6~0.8 \\
        中相关 & 0.4~0.6 \\
        弱相关 & 0.2~0.4 \\
        极弱或无相关 & 0.0~0.2 \\
        \hline
    \end{tabular}
\end{table}

采用层次分析法分析长春市生活物资投放点数量分布的影响因素,综合评价附件2中投放点数量的合理性。层次分析法(Analytic Hierarchy Process, AHP)的基本思想是将问题层次化,依照问题的性质和总目标,将其划分为不同的组成要素,并依据这些要素间的

\begin{table}[h]
\centering
\caption{层次分析法的操作步骤}
\begin{tabular}{ll}
\hline
步骤 & 操作 \\
\hline
Step 1 & 找准各因素之间的隶属度关系,建立递阶层次结构 \\
Step 2 & 构造判断矩阵并赋值 \\
Step 3 & 层次单排序(计算权向量)与检验(一致性检验) \\
Step 4 & 层次总排序(组合权向量)与检验(一致性检验) \\
Step 5 & 评价结果分析 \\
\hline
\end{tabular}
\end{table}

\subsubsection{建立层次结构模型}

\begin{figure}[h]
\centering
\includegraphics[width=\textwidth]{image.png} % 替换为实际的图像文件名
\caption{投放点数量评价的层次结构模型}
\end{figure}

本文构建的蔬菜包投放点数量合理性评价的层级分析法结构模型如图 5-2 所示。具体做法是,依据附件 1 汇总 4 月 10 日到 4 月 15 日的各区本土感染人数和无症状感染者人数,提取并统计附件 3 中的各区的小区栋数、小区户数(户)和小区人口数(人),此外我们考虑到交通状况也是影响物资供应方案的重要因素,因此我们根据附件 3 中的街道编号统计出了各区的街道数量。由此得到的评价指标数据如表 5-2 所示。

\begin{table}[h]
\centering
\caption{蔬菜包投放点数量合理性评价的层次分析指标数据}
\begin{tabular}{c c c c c c c}
\hline
\multirow{2}{*}{ 时间 } & \multirow{2}{*}{ 区 } & \multicolumn{5}{c}{ 评价指标 } \\
\cline{3-7}
& & 小区栋数 & 小区户数 (户) & 小区人口数 (人) & 街道数 (个) & 新增本土感染人数 (例) & 新增无症状感染人数 (例) \\
\hline
4 月 10 & 朝阳区 & 3740 & 219874 & 548179 & 69 & 3 & 32 \\
4 月 10 & 南关区 & 2336 & 183877 & 462146 & 68 & 7 & 55 \\
\vdots & \vdots & \vdots & \vdots & \vdots & \vdots & \vdots & \vdots \\
\hline
\end{tabular}
\end{table}

\begin{table}
\centering
\begin{tabular}{c c c c c c c}
4月11 & 宽城区 & 1524 & 128899 & 306143 & 58 & 34 \\
4月11 & 绿园区 & 2198 & 155261 & 372541 & 57 & 6 \\
& & & & & & 158 \\
4月12 & 二道区 & 1820 & 166428 & 405784 & 50 & 8 \\
4月12 & 长春新区 & 2116 & 141780 & 338274 & 29 & 6 \\
& & & & & & 55 \\
4月13 & 净月区 & 1507 & 85748 & 204056 & 59 & 7 \\
4月13 & 汽开区 & 1171 & 81675 & 193462 & 54 & 5 \\
& & & & & & 9 \\
4月14 & 朝阳区 & 3740 & 219874 & 548179 & 69 & 7.2 \\
4月14 & 南关区 & 2336 & 183877 & 462146 & 68 & 18.72 \\
& & & & & & 5 \\
4月15 & 净月区 & 1507 & 85748 & 204056 & 59 & 4 \\
4月15 & 汽开区 & 1171 & 81675 & 193462 & 54 & 10 \\
\end{tabular}
\end{table}

(2) 构建的判断矩阵

表5-3. 层次分析法的判断矩阵

\begin{table}
\centering
\begin{tabular}{c c c c c c}
指标 & 小区栋数 & 小区户数(户) & 小区人口数(人) & 街道数目 & 新增本土新增无症状感染者 \\
小区栋数 & 1 & 1 & 1 & 1 & 0.333 \\
小区户数(户) & 1 & 1 & 1 & 1 & 0.333 \\
小区人口数(人) & 1 & 1 & 1 & 1 & 0.333 \\
街道数目 & 1 & 1 & 1 & 1 & 0.2 \\
新冠疫情人数 & 3 & 3 & 3 & 5 & 1 \\
无症状感染者人数 & 3 & 3 & 3 & 5 & 1 \\
\end{tabular}
\end{table}

(3) AHP层次分析结果

根据表5-4的AHP层次分析结果,获得各项指标的权重值,基于层次分析法(方根法)的权重计算结果显示,小区栋数的权重为9.637%,小区户数(户)的权重为9.637%,小区人口数(人)的权重为9.637%,街道数目的权重为8.128%,新冠疫情人数的权重为31.48%,无症状感染者人数的权重为31.48%。

表5-4.AHP层次分析结果

\begin{table}
\centering
\begin{tabular}{c c c c c}
项 & 特征向量 & 权重值(%) & 最大特征根 & CI值 \\
小区栋数 & 0.693 & 9.637 & & \\
小区户数(户) & 0.693 & 9.637 & 6.044 & 0.009 \\
小区人口数(人) & 0.693 & 9.637 & & \\
街道数目 & 0.585 & 8.128 & & \\
\end{tabular}
\end{table}

\begin{table}[h]
\centering
\caption{表5-5.AHP的一致性检验结果}
\begin{tabular}{c c c c}
\hline
最大特征根 & CI值 & RI值 & CR值 \\
\hline
6.044 & 0.009 & 1.25 & 0.007 \\
\hline
\end{tabular}
\end{table}

\begin{table}[h]
\centering
\caption{表5-6.附件2:长春市9个区隔离人口数量与生活物资投放点数量}
\begin{tabular}{c c c}
\hline
区域名称 & 隔离人口数(万人) & 生活物资投放点数量 \\
\hline
朝阳区 & 57.8 & 94 \\
南关区 & 48.9 & 261 \\
宽城区 & 32.6 & 181 \\
绿园区 & 38.5 & 470 \\
二道区 & 42.6 & 9 \\
长春新区(高新) & 36.8 & 215 \\
经开区 & 20.3 & 37 \\
净月区 & 22.8 & 279 \\
汽开区 & 21.7 & 10 \\
总计 & 322 & 1556 \\
\hline
\end{tabular}
\end{table}

\begin{table}[h]
\centering
\caption{表5-7.长春市九个区的小区数据汇总}
\begin{tabular}{c c c c}
\hline
所属区域 & 小区栋数 & 小区户数(户) & 小区人口数(人) \\
\hline
二道区 & 1820 & 166428 & 405784 \\
净月区 & 1507 & 85748 & 204056 \\
南关区 & 2336 & 183877 & 462146 \\
宽城区 & 1524 & 128899 & 306143 \\
\hline
\end{tabular}
\end{table}

\begin{table}
\centering
\begin{tabular}{l r r r r r}
\hline
\textbf{所属区域} & \textbf{得分} & \textbf{总得分} & \textbf{占比} & \textbf{修正的附件2} & \textbf{优化结果} \\
 & & & & \textbf{投放总量} & \textbf{(个)} \\
\hline
二道区 & 56123.43464 & & 0.132918083 & & 172.7935084 \\
净月区 & 28594.39379 & & 0.067720588 & & 88.03676493 \\
南关区 & 63610.77547 & & 0.15065048 & & 195.8456237 \\
宽城区 & 44877.04046 & & 0.106283057 & & 138.1679741 \\
朝阳区 & 75464.63773 & 422240.8 & 0.178724183 & 1300 & 232.341438 \\
汽开区 & 26983.50788 & & 0.0639055 & & 83.07714994 \\
经开区 & 26819.0921 & & 0.063516111 & & 82.57094465 \\
绿园区 & 52546.75656 & & 0.124447376 & & 161.7815887 \\
长春新区(高新) & 47221.13722 & & 0.111834621 & & 145.3850076 \\
\hline
\end{tabular}
\caption{优化后的投放点数量}
\end{table}

\subsection*{5.2.2. 投放点数量的优化}

综合考虑各区的小区人口数、小区户数、街道数目、新冠肺炎本土感染人数以及无症状感染人数,运用层次分析法结合修正后的投放点总量,计算得到各区的物资投放点的数量。各区优化前后的对比情况如图5-3所示。其中调整幅度最大的是二道区(1819.93\%),其次是汽开区(730.77\%)、再者是经开区(123.16\%)。调整方案中,较原来需要增设投放点数量最多的是朝阳区(需要增加78.79个),较原来需要减少投放点数量最多的是绿园区(需要减少308.22个)。

\begin{figure}[h]
    \centering
    \includegraphics[width=\textwidth]{image1.png}
    \caption{投放点数量的优化前后对比图}
    \label{fig:5-3}
\end{figure}

\begin{figure}[h]
    \centering
    \includegraphics[width=\textwidth]{image2.png}
    \caption{投放点数量的调整幅度变化图}
    \label{fig:5-4}
\end{figure}

\subsection{子问题 2:政府储备物资规划——平均法和数据挖掘}

通过对附件 4 中生活物资相关数据集文件,我们可以得知在 3 月 18 日到 5 月 23 日的物资供应情况,按照商务部印发的《生活必需品市场供应保障工作手册》(2021 版)中规定,每人每日消费粮食 325 克,食用油 30 克,蔬菜 400 克,肉类 115 克。按长春市城区 450 万人口计算,长春市城区每日消耗粮食 1462 吨、食用油 135 吨、蔬菜 1800 吨、肉类 518 吨。通过该项数据以及附件 3 中的小区人口数量我们可以大致估算出每个区的生活必需品需求情况。根据国家发改委、商务部《新冠肺炎疫情防控生活物资保障工作指南(试

\begin{figure}[h]
    \centering
    \includegraphics[width=0.8\textwidth]{placeholder_image.png} % 替换为实际图像文件名
    \caption{蔬菜库存存储情况(2022.03.18-2022.05.13)}
    \label{fig:vegetable_storage}
\end{figure}

\begin{table}[h]
    \centering
    \caption{政府储备安全库存}
    \label{tab:reserve_inventory}
    \begin{tabular}{l c c c c}
        \hline
        区域名称 & 粮食 & 食用油 & 蔬菜 & 肉类 \\
        & (325克/人/日) & (30克/人/日) & (400克/人/日) & (115克/人/日) \\
        \hline
        朝阳区 & 178.158175 & 16.44537 & 219.2716 & 63.040585 \\
        二道区 & 131.8798 & 12.17352 & 162.3136 & 46.66516 \\
        经开区 & 61.503 & 5.6772 & 75.696 & 21.7626 \\
        净月区 & 66.3182 & 6.12168 & 81.6224 & 23.46644 \\
        宽城区 & 99.496475 & 9.18429 & 122.4572 & 35.206445 \\
        绿园区 & 121.075825 & 11.17623 & 149.0164 & 42.842215 \\
        南关区 & 150.19745 & 13.86438 & 184.8584 & 53.14679 \\
        汽开区 & 62.87515 & 5.80386 & 77.3848 & 22.24813 \\
        长春新区 & 109.93905 & 10.14822 & 135.3096 & 38.90151 \\
        总计 & 981.443125 & 90.59475 & 1207.93 & 347.279875 \\
        \hline
    \end{tabular}
\end{table}

5.4. 子问题3:大规模物资分拣场所选址与优化——K-means聚类和遗传算法的综合算法

5.4.1. 基于K-means的大规模物资分拣场所选址

21

针对大规模物资分拣场所的选址问题,本问根据附件 3 提供的小区坐标,通过聚类的方式获得类中心作为选址位置。K-means 聚类属于无监督学习,是一种基于距离的聚类算法,将距离作为相似性的评价指标,该算法认为簇是由距离相近的对象构成的,即两对象的距离越近,其相似度就越大,因此把距离相近且独立的簇作为目标。考虑到小区数量规模达 1409 个属于大规模的数据,而 K-means 可以基于欧式距离快速聚类,所求的类中心就是质心位置,与本题所求的大规模物资分拣场所选址的位置相对应。

### 建立基于 K-means 聚类的选址模型

设模型样本集为 \(\{X\}\) 有 \(n\) 个样本以及 \(k\) 个类 \(\{1, 2, \ldots, K\}\),模型的目标是每个类中的样本到类中心的距离之和最小。建立基于聚类的选址问题数学模型如式子(5-1)到式(5-4)所示。其中,式子(5-1)表示目标函数即欧式距离最短。式(2)表示各类只能分配到一个类中心中;式(5-3)是均值向量公式。式(5-4)中 \(y_{ij}\) 为 0-1 变量。

\begin{equation}
\min \sum_{j=1}^{k} \sum_{X \in S_j} y_{ij} \|X - m_j\|
\tag{5-1}
\end{equation}

约束条件:

\begin{equation}
\sum_{j=1}^{n} y_{ij} = 1 (i = 1, 2, \cdots, n)
\tag{5-2}
\end{equation}

\begin{equation}
m_j = \frac{1}{\sum_{i=1}^{n} y_{ij}} \sum_{i=1}^{n} y_{ij} X_i (j = 1, 2, \cdots, K)
\tag{5-3}
\end{equation}

\begin{equation}
y_{ij} =
\begin{cases}
1, \text{样本 } i \text{ 聚到到聚类中心 } j \text{ 中} \\
0, \text{否则}
\end{cases}
\tag{5-4}
\end{equation}

聚类结果如表 5-10 所示,其中聚类中心位置即为大规模物资分拣场所的选址位置。选址结果可视化如图 5-6 和图 5-7 所示。

\begin{table}[h]
\centering
\caption{基于 K-means 聚类的选址结果}
\begin{tabular}{c c c c c c c c c}
\hline \hline
小区编号 & 小区栋数 & 小区户数(户) & 小区人口数(人) & 小区横坐标 & 小区纵坐标 & 所属街道编号 & 聚类中心横坐标 & 聚类中心纵坐标 \\
\hline
1 & 11 & 1060 & 2417 & 57.90 & 62.13 & C0035 & 13 & 57.56 & 59.55 \\
2 & 37 & 4171 & 9080 & 51.79 & 61.61 & C0048 & 5 & 51.09 & 64.86 \\
3 & 3 & 480 & 1333 & 54.80 & 54.88 & C0010 & 13 & 57.56 & 59.55 \\
\hline \hline
\end{tabular}
\end{table}

\begin{table}
\centering
\begin{tabular}{c c c c c c c c c}
\hline
4 & 19 & 1093 & 2658 & 54.22 & 67.84 & C0053 & 5 & 51.09 & 64.86 \\
\hline
5 & 15 & 1330 & 3033 & 56.73 & 61.48 & C0035 & 13 & 57.56 & 59.55 \\
\hline
\multicolumn{1}{c}{\dots} & \multicolumn{1}{c}{\dots} & \multicolumn{1}{c}{\dots} & \multicolumn{1}{c}{\dots} & \multicolumn{1}{c}{\dots} & \multicolumn{1}{c}{\dots} & \multicolumn{1}{c}{\dots} & \multicolumn{1}{c}{\dots} & \multicolumn{1}{c}{\dots} & \multicolumn{1}{c}{\dots} \\
\hline
1405 & 5 & 231 & 908 & 25.51 & 34.51 & I0032 & 9 & 25.72 & 31.05 \\
\hline
1406 & 4 & 267 & 744 & 35.65 & 40.53 & I0008 & 3 & 33.17 & 37.72 \\
\hline
1407 & 2 & 185 & 727 & 26.07 & 36.63 & I0008 & 9 & 25.72 & 31.05 \\
\hline
1408 & 3 & 115 & 320 & 33.63 & 36.87 & I0008 & 3 & 33.17 & 37.72 \\
\hline
1409 & 1 & 48 & 134 & 45.32 & 42.28 & I0007 & 11 & 46.29 & 38.45 \\
\hline
\end{tabular}
\end{table}

\begin{figure}[h]
\centering
\includegraphics[width=0.8\textwidth]{image.png}
\caption{大规模物资分拣场所选址与服务小区范围}
\end{figure}

\begin{figure}[h]
    \centering
    \includegraphics[width=\textwidth]{image.png}
    \caption{小区聚类的类属性可视化}
    \label{fig:5-7}
\end{figure}

关于 $K$(类别数量)的选择,本文采用轮廓系数、肘部法则来最终确定最佳的 $K$ 值。其中,轮廓系数表示聚类后各类样本间的紧密程度和各类之间的离散程度。同一类中样本间的距离越小,异类间样本距离越大,则轮廓系数的值越大,聚类效果越好,因此常被用作评价聚类结果的性能指标。轮廓系数的计算公式如式(5-5)所示,式中 $a_i$ 代表样本内点的内聚度,其计算公式如(5-6)所示。其中 $j$ 代表样本 $i$ 在同一类内的其他样本点,$distance$ 代表了求 $i$ 与 $j$ 的距离,所以 $a(i)$ 越小说明该类越紧密。$b(i)$ 的计算需要遍历其他类簇得到多个值从中选择最小的作为最终的结果。通过计算当 $K=25$ 时

\begin{equation}
S_i = \frac{b(i) - a(i)}{max\{a(i), b(i)\}}
\tag{5-5}
\end{equation}

\begin{equation}
a(i) = \frac{1}{n-1} \sum_{j \neq 1}^n distance(i, j)
\tag{5-6}
\end{equation}

肘部法则的原理是成本函数,即类别畸变程度之和。每个类别的畸变程度等于每个变量点到其类别中心的位置距离平方和。当类内部越紧凑时类的畸变程度越小。在选择 $K$ 值时,肘部法则会刻画出不同值的成本函数值。随着值的增大,每个类包含的样本数会减少,样本距离其类中心会更近且畸变程度会减小。伴随值继续增大,畸变程度的改善效果将递减,此时畸变程度的改善效果下降幅度最大的位置对应的值就是肘部。肘部法则的结果如图 5-8 所示。由此本文将 $K=25$ 作为选址的类数较为合适。

\begin{figure}[h]
    \centering
    \includegraphics[width=0.8\textwidth]{image.png} % 替换为实际图像文件名
    \caption{肘部法则的结果可视化}
    \label{fig:elbow}
\end{figure}

\subsection{基于遗传算法的选址优化模型}

本文在聚类确定选址位置的基础上构建基于遗传算法的选址优化模型,目的是以配送距离最短为目标,确定每个类中心(即大规模物资分拣场所选址)所服务的小区。

遗传算法的(Genetic Algorithm,简称GA)是一种自适应随机搜索启发式算法,是模拟达尔文生物进化论的自然选择和遗传学机理的生物进化过程的计算模型。遗传算法的主要思想是采用概率寻优方法自动获取和指导优化的搜索空间,并能够自适应地调整搜索方向。遗传算法步骤如图\ref{fig:elbow}所示。核心步骤如下所示:

\begin{table}[h]
    \centering
    \caption{遗传算法的核心步骤}
    \label{tab:ga_steps}
    \begin{tabular}{l p{12cm}}
        \hline
        步骤 & 操作 \\
        \hline
        Step 1 & 编码:在进行寻优前将解空间的解数据用遗传空间的基因型串结构数据来表示,通过不同的组合构成不同的点; \\
        Step 2 & 生成初始群体:随机产生初始数据作为初始点开始迭代。设置迭代计数器,设置最大迭代次数; \\
        Step 3 & 评价适应度值; \\
        Step 4 & 选择:将选择算子作用于群体; \\
        Step 5 & 交叉:将交叉算子作用于群体; \\
        Step 6 & 变异:将变异算子作用于群体; \\
        Step 7 & 生成下一代群体; \\
        Step 8 & 终止条件判断:具有最大适应度的个体作为最优解输出,终止运算。 \\
        \hline
    \end{tabular}
\end{table}

\begin{figure}[h]
    \centering
    \includegraphics[width=0.8\textwidth]{genetic_algorithm_flowchart.png}
    \caption{遗传算法基本步骤}
    \label{fig:genetic_algorithm_flowchart}
\end{figure}

通过上述分析,以大规模物资分拣场到居民小区或者投放点距离最小为目标,建立的最短路径模型如下所示:

\begin{table}[h]
    \centering
    \caption{大规模物资分拣场的选址和路径优化模型}
    \label{tab:optimization_model}
    \begin{tabular}{l p{12cm}}
        \hline
        \textbf{集合} & \\
        \hline
        $V$ & 顶点集合,$V = N \cup \{O\}$。其中,$O$ 代表配送中心。 \\
        $E$ & 路线集合,$E = \{(i, j), \forall i, j \in N, i \neq j\}$ \\
        $EO, EI$ & 起始路线编号和终止路线编号 \\
        $F(i)$ & 路线编号终点的前序节点集合 \\
        $A(i)$ & 路线编号起点的后序节点集合 \\
        $K$ & 车辆集合,$K = \{1, 2, \dots, |K|\}$ \\
        \hline
        \textbf{参数} & \\
        \hline
        $C_{ij}$ & 路线距离,$(i, j) \in E$ \\
        $D_{ij}$ & 边上需求,$(i, j) \in E$ \\
        $Q$ & 车的容量 \\
        \hline
        \textbf{决策变量} & \\
        \hline
        $x_{kij}$ & 0-1 变量:若车辆 $k$ 经过边 $(i, j)$ 时为 1,否则为 0。$(i, j) \in E$ \\
        $u_{ki}$ & 整数:车辆 $k$ 离开节点 $i$ 的装载量,$\forall k \in K, \forall i \in V$ \\
        \hline
        \textbf{目标函数:} & \\
        \hline
        & $\min \sum_{k \in K} \sum_{(i, j) \in E} C_{ij} \cdot x_{kij}$ \\
        \hline
    \end{tabular}
\end{table}

约束条件:
\begin{align}
\sum_{(i,j)\in EO} x_{kij} &= 1, \forall k \in K \tag{5-8} \\
\sum_{(i,j)\in EI} x_{kij} &= 1, \forall k \in K \tag{5-9} \\
\sum_{k \in K} \sum_{(i,j)\in EO} x_{kij} &= |K| \tag{5-10} \\
\sum_{j \in F(i)} x_{kji} &= \sum_{j \in A(i)} x_{kij}, \forall i \in N, \forall k \in K \tag{5-11} \\
\sum_{k \in K} D_{ij} \cdot x_{kij} &\leq Q, \forall (i,j) \in E \tag{5-12} \\
u_{kj} &\geq u_{ki} + \left(x_{kij} - 1\right) \cdot Q + D_{ij}, \forall (i,j) \in E \text{ and } j \neq 0, \forall k \in K \tag{5-13} \\
x_{kij} &\in \{0,1\}, \forall (i,j) \in E, \forall k \in K \tag{5-14} \\
u_{ki} &\geq 0, \forall i \in N, \forall k \in K \tag{5-15}
\end{align}

\subsection{5.4.3. 计算结果}

由于此问题是 NP-hard 问题,为保证求解的质量和效率采用遗传算法进行求解(求解代码请见附件),计算结果如表 5-13 所示。

\begin{table}[h]
\centering
\caption{表 5-13. 大规模物资分拣场所选址与优化问题结果}
\begin{tabular}{c c c c c c c}
\hline \hline
No. & $x$ 坐标 & $y$ 坐标 & 管辖范围小区个数 & 管辖范围内人口数 & 总服务距离 & 选址半径 & 所属行政区 \\
\hline
0 & 35.944714 & 10.835988 & 28 & 147343 & 141.16 & 5.04 & 长春新区 \\
1 & 50.116419 & 47.424687 & 83 & 185812 & 256.47 & 3.09 & 绿园区 \\
2 & 75.333454 & 29.647553 & 38 & 57373 & 164.05 & 4.32 & 经开区 \\
3 & 33.171931 & 37.717431 & 58 & 98891 & 192.60 & 3.32 & 汽开区 \\
4 & 69.229461 & 46.756211 & 57 & 151390 & 166.78 & 2.93 & 二道区 \\
5 & 51.085767 & 64.857218 & 58 & 110980 & 195.35 & 3.37 & 宽城区 \\
6 & 57.384641 & 40.61421 & 92 & 209895 & 293.83 & 3.19 & 南关区 \\
7 & 82.247632 & 8.72149 & 34 & 32365 & 150.23 & 4.42 & 净月区 \\
8 & 56.473728 & 29.202943 & 58 & 122016 & 203.91 & 3.52 & 南关区 \\
9 & 25.722411 & 31.049356 & 45 & 94921 & 235.01 & 5.22 & 汽开区 \\
10 & 65.375926 & 25.069579 & 61 & 98668 & 271.11 & 4.44 & 南关区 \\
11 & 46.286134 & 38.451977 & 75 & 130886 & 224.98 & 3.00 & 朝阳区 \\
12 & 79.12181 & 42.936327 & 50 & 98753 & 190.90 & 3.82 & 经开区 \\
13 & 57.563096 & 59.545922 & 68 & 114498 & 252.90 & 3.72 & 宽城区 \\
14 & 49.695732 & 16.72146 & 42 & 80277 & 154.15 & 3.67 & 长春新区 \\
\hline \hline
\end{tabular}
\end{table}

\begin{table}
\centering
\begin{tabular}{c c c c c c c}
15 & 85.918443 & 20.252905 & 35 & 32226 & 148.77 & 4.25 & 净月区 \\
16 & 45.679303 & 54.744489 & 47 & 86432 & 141.84 & 3.02 & 绿园区 \\
17 & 44.069407 & 31.157538 & 73 & 203616 & 273.36 & 3.74 & 朝阳区 \\
18 & 68.519579 & 39.019632 & 107 & 181750 & 367.96 & 3.44 & 经开区 \\
19 & 67.83679 & 58.149913 & 50 & 142176 & 179.01 & 3.58 & 二道区 \\
20 & 35.885269 & 49.83894 & 39 & 106757 & 137.10 & 3.52 & 绿园区 \\
21 & 70.224214 & 16.55781 & 28 & 72936 & 121.55 & 4.34 & 净月区 \\
22 & 36.99186 & 22.565549 & 34 & 141379 & 152.49 & 4.48 & 长春新区 \\
23 & 42.013396 & 44.762825 & 82 & 113287 & 228.49 & 2.79 & 绿园区 \\
24 & 60.760388 & 48.59209 & 67 & 205198 & 224.92 & 3.36 & 二道区 \\
\end{tabular}
\end{table}

\section*{六、问题三分析与求解}

\subsection*{6.1. 问题三分析}

根据问题三要求,此题可以分为两个子问题,子问题一是依据附件5分析蔬菜包需求和发放规律。子问题二是根据附件3中的各小区位置与人口信息,评价和调整4月10日至4月15日蔬菜包供应方案。针对子问题一,本文采用文本挖掘技术,统计了附件5当中从2022年3月26日到5月1日九区总体和分区的每日蔬菜包的供给量和需求量。分别从蔬菜包的供应规律、需求规律和供需规律三方面切入,运用描述性统计、正态性检验、相关性分析的方法结合新冠肺炎感染人数变化,多层次分析蔬菜包的供需规律。针对问题二,本文采用TOPSIS综合评价方法,建立包含:各区本土感染人数和无症状感染者人数,各区小区栋数、小区户数和小区人口数、街道数等指标构成的评价体系,计算各指标权重,获得各区的得分与排名作为优化依据。

\begin{figure}[h]
\centering
\includegraphics[width=0.8\textwidth]{image.png}
\caption{问题三的思路流程图}
\end{figure}

\begin{table}
\centering
\begin{tabular}{|l|c|c|c|c|c|c|c|c|c|}
\hline
变量名 & 样本量 & 最大值 & 最小值 & 平均值 & 标准差 & 中位数 & 方差 & 峰度 & 偏度 变异系数(CV) \\
\hline
朝阳区 & 37 & 40355 & 0 & 10939.622 & 9207.865 & 8846 & 84784778.742 & 3.994 & 1.904 0.841 \\
\hline
南关区 & 37 & 27671 & 0 & 7636.459 & 9058.974 & 3200 & 82065010.977 & -0.882 & 0.837 1.186 \\
\hline
宽城区 & 37 & 88666 & 0 & 16820.568 & 20663.588 & 9852 & 426983866.474 & 5.806 & 2.427 1.228 \\
\hline
绿园区 & 37 & 118057 & 0 & 17759.973 & 24029.334 & 16356 & 577408891.027 & 7.72 & 2.368 1.353 \\
\hline
二道区 & 37 & 54933 & 0 & 11734.946 & 15226.859 & 5661 & 231857237.219 & 1.479 & 1.463 1.299 \\
\hline
长春新区 & 37 & 44826 & 0 & 8433.297 & 9487.222 & 4824 & 90007375.048 & 5.855 & 2.213 1.125 \\
\hline
经开区 & 37 & 51347 & 0 & 10032.541 & 13410.316 & 2600 & 179836572.589 & 1.393 & 1.415 1.337 \\
\hline
净月区 & 37 & 23333 & 0 & 6618.811 & 6805.284 & 4953 & 46311888.269 & 0.19 & 1.007 1.028 \\
\hline
莲花山 & 37 & 16376 & 0 & 1509.676 & 2891.45 & 890 & 8360484.503 & 19.927 & 4.084 1.915 \\
\hline
汽开区 & 37 & 38854 & 0 & 6241.297 & 9495.845 & 0 & 90171080.604 & 3.339 & 1.836 1.521 \\
\hline
九台区 & 37 & 13588 & 0 & 991.486 & 2941.79 & 0 & 8654128.535 & 11.271 & 3.348 2.967 \\
\hline
中韩示范区 & 37 & 2625 & 0 & 206.081 & 569.186 & 0 & 323972.41 & 9.119 & 2.979 2.762 \\
\hline
九区蔬菜包接收数量总计 & 37 & 293518 & 7100 & 98924.757 & 85080.114 & 74113 & 7238625718.967 & 0.16 & 1.069 0.860 \\
\hline
\end{tabular}
\end{table}

\begin{table}
\centering
\begin{tabular}{l r r r r r r r r r r r r r r r r r r r r r r r r r r r r r r r r r r r r r r r r r r r r r r r r r r r r r r r r r r r r r r r r r r r r r r r r r r r r r r r r r r r r r r r r r r r r r r r r r r r r r r r r r r r r r r r r r r r r r r r r r r r r r r r r r r r r r r r r r r r r r r r r r r r r r r r r r r r r r r r r r r r r r r r r r r r r r r r r r r r r r r r r r r r r r r r r r r r r r r r r r r r r r r r r r r r r r r r r r r r r r r r r r r r r r r r r r r r r r r r r r r r r r r r r r r r r r r r r r r r r r r r r r r r r r r r r r r r r r r r r r r r r r r r r r r r r r r r r r r r r r r r r r r r r r r r r r r r r r r r r r r r r r r r r r r r r r r r r r r r r r r r r r r r r r r r r r r r r r r r r r r r r r r r r r r r r r r r r r r r r r r r r r r r r r r r r r r r r r r r r r r r r r r r r r r r r r r r r r r r r r r r r r r r r r r r r r r r r r r r r r r r r r r r r r r r r r r r r r r r r r r r r r r r r r r r r r r r r r r r r r r r r r r r r r r r r r r r r r r r r r r r r r r r r r r r r r r r r r r r r r r r r r r r r r r r r r r r r r r r r r r r r r r r r r r r r r r r r r r r r r r r r r r r r r r r r r r r r r r r r r r r r r r r r r r r r r r r r r r r r r r r r r r r r r r r r r r r r r r r r r r r r r r r r r r r r r r r r r r r r r r r r r r r r r r r r r r r r r r r r r r r r r r r r r r r r r r r r r r r r r r r r r r r r r r r r r r r r r r r r r r r r r r r r r r r r r r r r r r r r r r r r r r r r r r r r r r r r r r r r r r r r r r r r r r r r r r r r r r r r r r r r r r r r r r r r r r r r r r r r r r r r r r r r r r r r r r r r r r r r r r r r r r r r r r r r r r r r r r r r r r r r r r r r r r r r r r r r r r r r r r r r r r r r r r r r r r r r r r r r r r r r r r r r r r r r r r r r r r r r r r r r r r r r r r r r r r r r r r r r r r r r r r r r r r r r r r r r r r r r r r r r r r r r r r r r r r r r r r r r r r r r r r r r r r r r r r r r r r r r r r r r r r r r r r r r r r r r r r r r r r r r r r r r r r r r r r r r r r r r r r r r r r r r r r r r r r r r r r r r r r r r r r r r r r r r r r r r r r r r r r r r r r r r r r r r r r r r r r r r r r r r r r r r r r r r r r r r r r r r r r r r r r r r r r r r r r r r r r r r r r r r r r r r r r r r r r r r r r r r r r r r r r r r r r r r r r r r r r r r r r r r r r r r r r r r r r r r r r r r r r r r r r r r r r r r r r r r r r r r r r r r r r r r r r r r r r r r r r r r r r r r r r r r r r r r r r r r r r r r r r r r r r r r r r r r r r r r r r r r r r r r r r r r r r r r r r r r r r r r r r r r r r r r r r r r r r r r r r r r r r r r r r r r r r r r r r r r r r r r r r r r r r r r r r r r r r r r r r r r r r r r r r r r r r r r r r r r r r r r r r r r r r r r r r r r r r r r r r r r r r r r r r r r r r r r r r r r r r r r r r r r r r r r r r r r r r r r r r r r r r r r r r r r r r r r r r r r r r r r r r r r r r r r r r r r r r r r r r r r r r r r r r r r r r r r r r r r r r r r r r r r r r r r r r r r r r r r r r r r r r r r r r r r r r r r r r r r r r r r r r r r r r r r r r r r r r r r r r r r r r r r r r r r r r r r r r r r r r r r r r r r r r r r r r r r r r r r r r r r r r r r r r r r r r r r r r r r r r r r r r r r r r r r r r r r r r r r r r r r r r r r r r r r r r r r r r r r r r r r r r r r r r r r r r r r r r r r r r r r r r r r r r r r r r r r r r r r r r r r r r r r r r r r r r r r r r r r r r r r r r r r r r r r r r r r r r r r r r r r r r r r r r r r r r r r r r r r r r r r r r r r r r r r r r r r r r r r r r r r r r r r r r r r r r r r r r r r r r r r r r r r r r r r r r r r r r r r r r r r r r r r r r r r r r r r r r r r r r r r r r r r r r r r r r r r r r r r r r r r r r r r r r r r r r r r r r r r r r r r r r r r r r r r r r r r r r r r r r r r r r r r r r r r r r r r r r r r r r r r r r r r r r r r r r r r r r r r r r r r r r r r r r r r r r r r r r r r r r r r r r r r r r r r r r r r r r r r r r r r r r r r r r r r r r r r r r r r r r r r r r r r r r r r r r r r r r r r r r r r r r r r r r r r r r r r r r r r r r r r r r r r r r r r r r r r r r r r r r r r r r r r r r r r r r r r r r r r r r r r r r r r r r r r r r r r r r r r r r r r r r r r r r r r r r r r r r r r r r r r r r r r r r r r r r r r r r r r r r r r r r r r r r r r r r r r r r r r r r r r r r r r r r r r r r r r r r r r r r r r r r r r r r r r r r r r r r r r r r r r r r r r r r r r r r r r r r r r r r r r r r r r r r r r r r r r r r r r r r r r r r r r r r r r r r r r r r r r r r r r r r r r r r r r r r r r r r r r r r r r r r r r r r r r r r r r r r r r r r r r r r r r r r r r r r r r r r r r r r r r r r r r r r r r r r r r r r r r r r r r r r r r r r r r r r r r r r r r r r r r r r r r r r r r r r r r r r r r r r r r r r r r r r r r r r r r r r r r r r r r r r r r r r r r r r r r r r r r r r r r r r r r r r r r r r r r r r r r r r r r r r r r r r r r r r r r r r r r r r r r r r r r r r r r r r r r r r r r r r r r r r r r r r r r r r r r r r r r r r r r r r r r r r r r r r r r r r r r r r r r r r r r r r r r r r r r r r r r r r r r r r r r r r r r r r r r r r r r r r r r r r r r r r r r r r r r r r r r r r r r r r r r r r r r r r r r r r r r r r r r r r r r r r r r r r r r r r r r r r r r r r r r r r r r r r r r r r r r r r r r r r r r r r r r r r r r r r r r r r r r r r r r r r r r r r r r r r r r r r r r r r r r r r r r r r r r r r r r r r r r r r r r r r r r r r r r r r r r r r r r r r r r r r r r r r r r r r r r r r r r r r r r r r r r r r r r r r r r r r r r r r r r r r r r r r r r r r r r r r r r r r r r r r r r r r r r r r r r r r r r r r r r r r r r r r r r r r r r r r r r r r r r r r r r r r r r r r r r r r r r r r r r r r r r r r r r r r r r r r r r r r r r r r r r r r r r r r r r r r r r r r r r r r r r r r r r r r r r r r r r r r r r r r r r r r r r r r r r r r r r r r r r r r r r r r r r r r r r r r r r r r r r r r r r r r r r r r r r r r r r r r r r r r r r r r r r r r r r r r r r r r r r r r r r r r r r r r r r r r r r r r r r r r r r r r r r r r r r r r r r r r r r r r r r r r r r r r r r r r r r r r r r r r r r r r r r r r r r r r r r r r r r r r r r r r r r r r r r r r r r r r r r r r r r r r r r r r r r r r r r r r r r r r r r r r r r r r r r r r r r r r r r r r r r r r r r r r r r r r r r r r r r r r r r r r r r r r r r r r r r r r r r r r r r r r r r r r r r r r r r r r r r r r r r r r r r r r r r r r r r r r r r r r r r r r r r r r r r r r r r r r r r r r r r r r r r r r r r r r r r r r r r r r r r r r r r r r r r r r r r r r r r r r r r r r r r r r r r r r r r r r r r r r r r r r r r r r r r r r r r r r r r r r r r r r r r r r r r r r r r r r r r r r r r r r r r r r r r r r r r r r r r r r r r r r r r r r r r r r r r r r r r r r r r r r r r r r r r r r r r r r r r r r r r r r r r r r r r r r r r r r r r r r r r r r r r r r r r r r r r r r r r r r r r r r r r r r r r r r r r r r r r r r r r r r r r r r r r r r r r r r r r r r r r r r r r r r r r r r r r r r r r r r r r r r r r r r r r r r r r r r r r r r r r r r r r r r r r r r r r r r r r r r r r r r r r r r r r r r r r r r r r r r r r r r r r r r r r r r r r r r r r r r r r r r r r r r r r r r r r r r r r r r r r r r r r r r r r r r r r r r r r r r r r r r r r r r r r r r r r r r r r r r r r r r r r r r r r r r r r r r r r r r r r r r r r r r r r r r r r r r r r r r r r r r r r r r r r r r r r r r r r r r r r r r r r r r r r r r r r r r r r r r r r r r r r r r r r r r r r r r r r r r r r r r r r r r r r r r r r r r r r r r r r r r r r r r r r r r r r r r r r r r r r r r r r r r r r r r r r r r r r r r r r r r r r r r r r r r r r r r r r r r r r r r r r r r r r r r r r r r r r r r r r r r r r r r r r r r r r r r r r r r r r r r r r r r r r r r r r r r r r r r r r r r r r r r r r r r r r r r r r r r r r r r r r r r r r r r r r r r r r r r r r r r r r r r r r r r r r r r r r r r r r r r r r r r r r r r r r r r r r r r r r r r r r r r r r r r r r r r r r r r r r r r r r r r r r r r r r r r r r r r r r r r r r r r r r r r r r r r r r r r r r r r r r r r r r r r r r r r r r r r r r r r r r r r r r r r r r r r r r r r r r r r r r r r r r r r r r r r r r r r r r r r r r r r r r r r r r r r r r r r r r r r r r r r r r r r r r r r r r r r r r r r r r r r r r r r r r r r r r r r r r r r r r r r r r r r r r r r r r r r r r r r r r r r r r r r r r r r r r r r r r r r r r r r r r r r r r r r r r r r r r r r r r r r r r r r r r r r r r r r r r r r r r r r r r r r r r r r r r r r r r r r r r r r r r r r r r r r r r r r r r r r r r r r r r r r r r r r r r r r r r r r r r r r r r r r r r r r r r r r r r r r r r r r r r r r r r r r r r r r r r r r r r r r r r r r r r r r r r r r r r r r r r r r r r r r r r r r r r r r r r r r r r r r r r r r r r r r r r r r r r r r r r r r r r r r r r r r r r r r r r r r r r r r r r r r r r r r r r r r r r r r r r r r r r r r r r r r r r r r r r r r r r r r r r r r r r r r r r r r r r r r r r r r r r r r r r r r r r r r r r r r r r r r r r r r r r r r r r r r r r r r r r r r r r r r r r r r r r r r r r r r r r r r r r r r r r r r r r r r r r r r r r r r r r r r r r r r r r r r r r r r r r r r r r r r r r r r r r r r r r r r r r r r r r r r r r r r r r r r r r r r r r r r r r r r r r r r r r r r r r r r r r r r r r r r r r r r r r r r r r r r r r r r r r r r r r r r r r r r r r r r r r r r r r r r r r r r r r r r r r r r r r r r r r r r r r r r r r r r r r r r r r r r r r r r r r r r r r r r r r r r r r r r r r r r r r r r r r r r r r r r r r r r r r r r r r r r r r r r r r r r r r r r r r r r r r r r r r r r r r r r r r r r r r r r r r r r r r r r r r r r r r r r r r r r r r r r r r r r r r r r r r r r r r r r r r r r r r r r r r r r r r r r r r r r r r r r r r r r r r r r r r r r r r r r r r r r r r r r r r r r r r r r r r r r r r r r r r r r r r r r r r r r r r r r r r r r r r r r r r r r r r r r r r r r r r r r r r r r r r r r r r r r r r r r r r r r r r r r r r r r r r r r r r r r r r r r r r r r r r r r r r r r r r r r r r r r r r r r r r r r r r r r r r r r r r r r r r r r r r r r r r r r r r r r r r r r r r r r r r r r r r r r r r r r r r r r r r r r r r r r r r r r r r r r r r r r r r r r r r r r r r r r r r r r r r r r r r r r r r r r r r r r r r r r r r r r r r r r r r r r r r r r r r r r r r r r r r r r r r r r r r r r r r r r r r r r r r r r r r r r r r r r r r r r r r r r r r r r r r r r r r r r r r r r r r r r r r r r r r r r r r r r r r r r r r r r r r r r r r r r r r r r r r r r r r r r r r r r r r r r r r r r r r r r r r r r r r r r r r r r r r r r r r r r r r r r r r r r r r r r r r r r r r r r r r r r r r r r r r r r r r r r r r r r r r r r r r r r r r r r r r r r r r r r r r r r r r r r r r r r r r r r r r r r r r r r r r r r r r r r r r r r r r r r r r r r r r r r r r r r r r r r r r r r r r r r r r r r r r r r r r r r r r r r r r r r r r r r r r r r r r r r r r r r r r r r r r r r r r r r r r r r r r r r r r r r r r r r r r r r r r r r r r r r r r r r r r r r r r r r r r r r r r r r r r r r r r r r r r r r r r r r r r r r r r r r r r r r r r r r r r r r r r r r r r r r r r r r r r r r r r r r r r r r r r r r r r r r r r r r r r r r r r r r r r r r r r r r r r r r r r r r r r r r r r r r r r r r r r r r r r r r r r r r r r r r r r r r r r r r r r r r r r r r r r r r r r r r r r r r r r r r r r r r r r r r r r r r r r r r r r r r r r r r r r r r r r r r r r r r r r r r r r r r r r r r r r r r r r r r r r r r r r r r r r r r r r r r r r r r r r r r r r r r r r r r r r r r r r r r r r r r r r r r r r r r r r r r r r r r r r r r r r r r r r r r r r r r r r r r r r r r r r r r r r r r r r r r r r r r r r r r r r r r r r r r r r r r r r r r r r r r r r r r r r r r r r r r r r r r r r r r r r r r r r r r r r r r r r r r r r r r r r r r r r r r r r r r r r r r r r r r r r r r r r r r r r r r r r r r r r r r r r r r r r r r r r r r r r r r r r r r r r r r r r r r r r r r r r r r r r r r r r r r r r r r r r r r r r r r r r r r r r r r r r r r r r r r r r r r r r r r r r r r r r r r r r r r r r r r r r r r r r r r r r r r r r r r r r r r r r r r r r r r r r r r r r r r r r r r r r r r r r r r r r r r r r r r r r r r r r r r r r r r r r r r r r r r r r r r r r r r r r r r r r r r r r r r r r r r r r r r r r r r r r r r r r r r r r r r r r r r r r r r r r r r r r r r r r r r r r r r r r r r r r r r r r r r r r r r r r r r r r r r r r r r r r r r r r r r r r r r r r r r r r r r r r r r r r r r r r r r r r r r r r r r r r r r r r r r r r r r r r r r r r r r r r r r r r r r r r r r r r r r r r r r r r r r r r r r r r r r r r r r r r r r r r r r r r r r r r r r r r r r r r r r r r r r r r r r r r r r r r r r r r r r r r r r r r r r r r r r r r r r r r r r r r r r r r r r r r r r r r r r r r r r r r r r r r r r r r r r r r r r r r r r r r r r r r r r r r r r r r r r r r r r r r r r r r r r r r r r r r r r r r r r r r r r r r r r r r r r r r r r r r r r r r r r r r r r r r r r r r r r r r r r r r r r r r r r r r r r r r r r r r r r r r r r r r r r r r r r r r r r r r r r r r r r r r r r r r r r r r r r r r r r r r r r r r r r r r r r r r r r r r r r r r r r r r r r r r r r r r r r r r r r r r r r r r r r r r r r r r r r r r r r r r r r r r r r r r r r r r r r r r r r r r r r r r r r r r r r r r r r r r r r r r r r r r r r r r r r r r r r r r r r r r r r r r r r r r r r r r r r r r r r r r r r r r r r r r r r r r r r r r r r r r r r r r r r r r r r r r r r r r r r r r r r r r r r r r r r r r r r r r r r r r r r r r r r r r r r r r r r r r r r r r r r r r r r r r r r r r r r r r r r r r r r r r r r r r r r r r r r r r r r r r r r r r r r r r r r r r r r r r r r r r r r r r r r r r r r r r r r r r r r r r r r r r r r r r r r r r r r r r r r r r r r r r r r r r r r r r r r r r r r r r r r r r r r r r r r r r r r r r r r r r r r r r r r r r r r r r r r r r r r r r r r r r r r r r r r r r r r r r r r r r r r r r r r r r r r r r r r r r r r r r r r r r r r r r r r r r r r r r r r r r r r r r r r r r r r r r r r r r r r r r r r r r r r r r r r r r r r r r r r r r r r r r r r r r r r r r r r r r r r r r r r r r r r r r r r r r r r r r r r r r r r r r r r r r r r r r r r r r r r r r r r r r r r r r r r r r r r r r r r r r r r r r r r r r r r r r r r r r r r r r r r r r r r r r r r r r r r r r r r r r r r r r r r r r r r r r r r r r r r r r r r r r r r r r r r r r r r r r r r r r r r r r r r r r r r r r r r r r r r r r r r r r r r r r r r r r r r r r r r r r r r r r r r r r r r r r r r r r r r r r r r r r r r r r r r r r r r r r r r r r r r r r r r r r r r r r r r r r r r r r r r r r r r r r r r r r r r r r r r r r r r r r r r r r r r r r r r r r r r r r r r r r r r r r r r r r r r r r r r r r r r r r r r r r r r r r r r r r r r r r r r r r r r r r r r r r r r r r r r r r r r r r r r r r r r r r r r r r r r r r r r r r r r r r r r r r r r r r r r r r r r r r r r r r r r r r r r r r r r r r r r r r r r r r r r r r r r r r r r r r r r r r r r r r r r r r r r r r r r r r r r r r r r r r r r r r r r r r r r r r r r r r r r r r r r r r r r r r r r r r r r r r r r r r r r r r r r r r r r r r r r r r r r r r r r r r r r r r r r r r r r r r r r r r r r r r r r r r r r r r r r r r r r r r r r r r r r r r r r r r r r r r r r r r r r r r r r r r r r r r r r r r r r r r r r r r r r r r r r r r r r r r r r r r r r r r r r r r r r r r r r r r r r r r r r r r r r r r r r r r r r r r r r r r r r r r r r r r r r r r r r r r r r r r r r r r r r r r r r r r r r r r r r r r r r r r r r r r r r r r r r r r r r r r r r r r r r r r r r r r r r r r r r r r r r r r r r r r r r r r r r r r r r r r r r r r r r r r r r r r r r r r r r r r r r r r r r r r r r r r r r r r r r r r r r r r r r r r r r r r r r r r r r r r r r r r r r r r r r r r r r r r r r r r r r r r r r r r r r r r r r r r r r r r r r r r r r r r r r r r r r r r r r r r r r r r r r r r r r r r r r r r r r r r r r r r r r r r r r r r r r r r r r r r r r r r r r r r r r r r r r r r r r r r r r r r r r r r r r r r r r r r r r r r r r r r r r r r r r r r r r r r r r r r r r r r r r r r r r r r r r r r r r r r r r r r r r r r r r r r r r r r r r r r r r r r r r r r r r r r r r r r r r r r r r r r r r r r r r r r r r r r r r r r r r r r r r r r r r r r r r r r r r r r r r r r r r r r r r r r r r r r r r r r r r r r r r r r r r r r r r r r r r r r r r r r r r r r r r r r r r r r r r r r r r r r r r r r r r r r r r r r r r r r r r r r r r r r r r r r r r r r r r r r r r r r r r r r r r r r r r r r r r r r r r r r r r r r r r r r r r r r r r r r r r r r

\begin{figure}[h]
    \centering
    \includegraphics[width=\textwidth]{image1.png}
    \caption{(a) 正态性检验直方图 \hspace{2cm} (b) 正态性检验 P-P 图}
\end{figure}

\begin{figure}[h]
    \centering
    \includegraphics[width=\textwidth]{image2.png}
    \caption{(c) 正态性检验 Q-Q 图}
\end{figure}

图 6-2. 蔬菜包供应量的正态性检验

各区本土新增与无症状感染者的汇总如图 6-4 所示。蔬菜包供应量的时序图如图 6-5 所示,蔬菜包供给量与新冠感染人数的时序变化折线图如图 6-6 所示。蔬菜包供应量与新冠感染人数的Spearman相关性检验热力图如图 6-7 所示。Spearman相利用单调方程评价两个统计变量的相关性,适用于评价非正态分布数据的相关性。对于样本容量为$n$的样本,$n$个原始数据被转换成等级数据,Spearman相关系数$\rho$为式(6-1)所示。

\begin{equation}
\rho = \frac{\sum_{i}(x_{i}-\bar{x})(y_{i}-\bar{y})}{\sqrt{\sum_{i}(x_{i}-\bar{x})^{2}\sum_{i}(y_{i}-\bar{y})^{2}}}.
\tag{6-1}
\end{equation}

\begin{table}
\centering
\begin{tabular}{c c c c c c c c c c}
 & 朝阳区 & 南关区 & 宽城区 & 绿园区 & 二道区 & 长春新区(高新) & 经开区 & 净月区 & 汽开区 \\
\hline
3月4日 & 0 & 0 & 0 & 0 & 0 & 0 & 0 & 0 & 0 \\
3月5日 & 0 & 0 & 0 & 0 & 0 & 0 & 0 & 0 & 0 \\
3月6日 & 0 & 0 & 0 & 1 & 0 & 0 & 1 & 1 & 0 \\
3月7日 & 1 & 0 & 0 & 0 & 0 & 0 & 1 & 2 & 0 \\
3月8日 & 0 & 0 & 0 & 0 & 0 & 0 & 0 & 0 & 0 \\
3月9日 & 0 & 0 & 1 & 2 & 0 & 4 & 0 & 2 & 0 \\
3月10日 & 0 & 0 & 0 & 0 & 0 & 0 & 4 & 0 & 0 \\
3月11日 & 0 & 7 & 1 & 2 & 0 & 35 & 4 & 0 & 1 \\
3月12日 & 5 & 30 & 3 & 10 & 4 & 31 & 4 & 154 & 1 \\
3月13日 & 10 & 14 & 4 & 7 & 2 & 7 & 12 & 33 & 2 \\
3月14日 & 101 & 31 & 13 & 27 & 24 & 43 & 37 & 30 & 21 \\
3月15日 & 22 & 12 & 3 & 4 & 0 & 14 & 26 & 2 & 1 \\
3月16日 & 42 & 10 & 5 & 5 & 3 & 8 & 15 & 9 & 2 \\
3月17日 & 29 & 54 & 16 & 51 & 28 & 66 & 67 & 32 & 21 \\
3月18日 & 40 & 50 & 12 & 32 & 9 & 36 & 72 & 110 & 13 \\
3月19日 & 107 & 109 & 39 & 33 & 26 & 69 & 70 & 13 & 18 \\
3月20日 & 90 & 73 & 29 & 35 & 43 & 49 & 100 & 44 & 30 \\
3月21日 & 63 & 63 & 36 & 91 & 25 & 100 & 67 & 34 & 31 \\
3月22日 & 183 & 98 & 135 & 92 & 99 & 94 & 111 & 63 & 61 \\
3月23日 & 119 & 64 & 158 & 140 & 80 & 90 & 96 & 50 & 85 \\
3月24日 & 55 & 46 & 63 & 61 & 46 & 80 & 88 & 10 & 11 \\
3月25日 & 54 & 70 & 153 & 63 & 72 & 89 & 57 & 87 & 67 \\
3月26日 & 70 & 48 & 45 & 36 & 81 & 45 & 64 & 41 & 38 \\
3月27日 & 49 & 51 & 77 & 70 & 54 & 126 & 146 & 74 & 29 \\
3月28日 & 78 & 77 & 83 & 109 & 28 & 66 & 43 & 19 & 44 \\
3月29日 & 67 & 42 & 457 & 155 & 41 & 41 & 64 & 36 & 23 \\
3月30日 & 75 & 39 & 352 & 72 & 185 & 145 & 178 & 21 & 50 \\
3月31日 & 101 & 61 & 413 & 327 & 129 & 39 & 76 & 61 & 62 \\
4月1日 & 223 & 213 & 1031 & 250 & 54 & 71 & 189 & 39 & 37 \\
4月2日 & 375 & 299 & 833 & 503 & 542 & 151 & 262 & 89 & 97 \\
4月3日 & 342 & 737 & 386 & 377 & 140 & 222 & 214 & 115 & 104 \\
4月4日 & 206 & 165 & 488 & 236 & 136 & 122 & 165 & 89 & 64 \\
4月5日 & 269 & 151 & 916 & 313 & 112 & 122 & 312 & 74 & 71 \\
4月6日 & 192 & 252 & 621 & 353 & 110 & 106 & 306 & 80 & 38 \\
4月7日 & 188 & 296 & 580 & 233 & 155 & 80 & 235 & 89 & 42 \\
4月8日 & 48 & 91 & 333 & 63 & 61 & 61 & 51 & 19 & 12 \\
4月9日 & 41 & 74 & 354 & 214 & 27 & 32 & 75 & 9 & 3 \\
4月10日 & 35 & 62 & 312 & 241 & 62 & 21 & 48 & 22 & 2 \\
4月11日 & 36 & 60 & 185 & 164 & 32 & 18 & 52 & 30 & 3 \\
4月12日 & 84 & 69 & 375 & 133 & 59 & 61 & 98 & 23 & 18 \\
4月13日 & 35 & 49 & 384 & 152 & 72 & 45 & 55 & 33 & 14 \\
\hline
区感染人数 & & & & & & & & & \\
计(本土+无 & & & & & & & & & \\
症状) & 3434 & 3567 & 8896 & 4656 & 2541 & 2389 & 3463 & 1636 & 1116 \\
\end{tabular}
\end{table}

每天感染人数总计(本土+无症状)

\begin{tabular}{c}
0 \\
0 \\
3 \\
4 \\
0 \\
9 \\
4 \\
50 \\
242 \\
91 \\
327 \\
84 \\
99 \\
364 \\
374 \\
484 \\
493 \\
510 \\
936 \\
882 \\
460 \\
712 \\
468 \\
676 \\
547 \\
926 \\
1117 \\
1269 \\
2107 \\
3151 \\
2637 \\
1671 \\
2340 \\
2058 \\
1898 \\
739 \\
829 \\
805 \\
580 \\
920 \\
839 \\
31698 \\
\end{tabular}

\begin{figure}[h]
    \centering
    \includegraphics[width=\textwidth]{image.png}
    \caption{蔬菜包供应量与新冠感染人数的Spearman相关性检验热力图}
    \label{fig:heatmap}
\end{figure}

综合来看,九个区蔬菜包的供应规律如下:

\textbf{规律1:} 蔬菜包供给量随疫情变化具有阶段性特征。一是新冠肺炎疫情快速扩散阶段(3月26日到4月2日),新冠病毒感染人数持续上涨并在4月2日达到顶峰,此时九区的总供给量随着上涨。从九个区总体供给量的时序变化趋势来看,呈现倒“V”的形状,说明随着感染人数的增加,政府加大了对各区物资的供给。二是新冠肺炎疫情小幅反弹阶段(4月3日到4月12日),此阶段新增新冠感染人数较第一阶段有所下降但仍存在小幅度反弹的现象,但总体有所好转。此阶段蔬菜包供应量呈现上升趋势并且在4月11日达到最高水平(293518包),主要分配给绿园区(69828包)和经开区(51347包),结合4月11日当天的疫情状况来看新增感染人数580例,而主要新增的区是宽城区(185例)和绿园区(164例),结合经开区当天的46424.5包的物资发放量来看,经开区存在4922包的物资浪费。三是新冠肺炎疫情下疫情的快速清零阶段(4月14日到5月1日),九区自4月14日到5月1日保持了零新增,特别是长春市自4月28日起逐步解除社会管控有序恢复正常生产生活秩序,蔬菜包供应量逐步减少,直到5月1日的7100包供应给宽城区(4200包)、莲山区(2100包)和净月区(800包)。

\begin{figure}[h]
    \centering
    \includegraphics[width=\textwidth]{image1.png}
    \caption{各区蔬菜包供应量的时序图}
    \label{fig:6-5}
\end{figure}

\begin{figure}[h]
    \centering
    \includegraphics[width=\textwidth]{image2.png}
    \caption{供给量与新冠感染人数的时序变化折线图}
    \label{fig:6-6}
\end{figure}

规律2:蔬菜包的供应量近似正态分布。根据正态性检验直方图、P-P图、Q-Q图都表明虽然数据不是绝对正态,但基本可接受为正态分布。

规律3:蔬菜包的供应量与各区新增病例的关系是低相关,只有汽开区是中度相关(0.477113)这表明疫情新增人数是不是影响蔬菜包供应量的主要因素。

规律4:蔬菜包供给量与疫情新增人数发展变化来看,两者发展趋势相似,但是蔬菜包供给量的变化具有滞后性。

\subsection{蔬菜包的需求规律}

本文将附件5中的将九区的蔬菜包每日实际发放量认定为蔬菜包的需求量,表示各区每日对蔬菜包的需求。从需求的角度分析九个区的蔬菜包整体与分区的需求规律。

\begin{table}
\centering
\begin{tabular}{c c c c c c c c c c}
\hline
日期 & 朝阳区 & 南关区 & 宽城区 & 绿园区 & 二道区 & 长春新区 & 经开区 & 净月区 & 汽开区 & 总量 \\
\hline
4月4日 & 8292 & 2411 & 11327 & 32970 & 7000 & 4000 & 0 & 4934 & 0 & 70934 \\
4月5日 & 10482 & 860 & 16270 & 25379 & 16000 & 4869 & 1000 & 7329 & 3560 & 85749 \\
4月6日 & 6023 & 2455 & 12532 & 22285 & 9765 & 9194 & 9226 & 5083 & 5085 & 81648 \\
4月7日 & 10702 & 5142 & 20921 & 16619 & 15125 & 8015 & 8410 & 8871 & 2510 & 96315 \\
4月8日 & 17260 & 500 & 42821 & 15366 & 11259 & 16431 & 7100 & 10451 & 16876 & 138064 \\
4月9日 & 25808 & 19080 & 26557 & 9900 & 18000 & 14558 & 35414 & 8980 & 3603 & 161900 \\
4月10日 & 38244 & 1629 & 62226 & 32812 & 26908 & 17874 & 24360 & 11594 & 8777 & 224424 \\
4月11日 & 37189.5 & 19041 & 13251 & 57047 & 29122 & 21893.5 & 46424.5 & 7462 & 14722 & 246152.5 \\
4月12日 & 23125 & 21371 & 12800 & 74611 & 17698 & 31737 & 16663.5 & 19333 & 5862.5 & 223201 \\
4月13日 & 19475.5 & 25285 & 2737 & 39967 & 10637 & 25254.5 & 13994 & 24522 & 6044.5 & 167916.5 \\
4月14日 & 11874 & 23223.5 & 5385 & 16356 & 7142 & 17913 & 20000 & 18948 & 7500 & 128341.5 \\
4月15日 & 8285 & 14447 & 10244 & 22380 & 5748 & 8731 & 25000 & 14136 & 5000 & 113971 \\
4月16日 & 7560 & 17111.5 & 4788 & 17138 & 5725 & 7543 & 14214 & 7033 & 1800 & 82912.5 \\
4月17日 & 7918 & 5070 & 6500 & 0 & 1168 & 2000 & 4000 & 2680 & 0 & 29336 \\
4月18日 & 10244 & 8040 & 8761 & 19224 & 500 & 8854 & 0 & 4960 & 700 & 61283 \\
4月19日 & 9048 & 7887 & 3017 & 0 & 5098 & 10700 & 0 & 17714 & 676 & 54140 \\
4月20日 & 8575 & 0 & 6992 & 0 & 300 & 2801 & 0 & 1804 & 6754 & 27226 \\
4月21日 & 9971 & 2000 & 5000 & 0 & 4096 & 111 & 3300 & 3429 & 0 & 27907 \\
4月22日 & 9971 & 0 & 5000 & 0 & 4096 & 111 & 3300 & 3429 & 0 & 25907 \\
4月23日 & 9597 & 0 & 13990 & 0 & 1393 & 3260 & 0 & 1180 & 0 & 29420 \\
4月24日 & 9200 & 0 & 20686 & 0 & 0 & 3000 & 0 & 4953 & 0 & 37839 \\
4月25日 & 10725 & 0 & 22704 & 0 & 50 & 2645 & 0 & 5976 & 0 & 42100 \\
4月26日 & 8661 & 0 & 20510 & 0 & 0 & 2736 & 0 & 3869 & 0 & 35776 \\
4月27日 & 7234 & 0 & 20194 & 0 & 0 & 1800 & 0 & 2924 & 0 & 32152 \\
4月28日 & 4146 & 0 & 14048 & 0 & 0 & 1560 & 0 & 429 & 0 & 20183 \\
4月29日 & 858 & 0 & 8000 & 0 & 0 & 2951 & 0 & 730 & 0 & 12539 \\
4月30日 & 0 & 0 & 9500 & 0 & 0 & 0 & 0 & 700 & 0 & 10200 \\
5月1日 & 0 & 0 & 4200 & 0 & 0 & 0 & 0 & 800 & 0 & 5000 \\
\hline
\end{tabular}
\end{table}

\begin{table}
\centering
\begin{tabular}{c c c c c c c c c c}
\hline
变量名 & 样本量 & 最大值 & 最小值 & 平均值 & 标准差 & 中位数 & 方差 & 峰度 & 偏度 & 变异系数(CV) \\
\hline
朝阳区 & 28 & 38244 & 0 & 11802.429 & 9385.754 & 9398.5 & 88092383.495 & 2.734 & 1.639 & 0.795 \\
南关区 & 28 & 25285 & 0 & 6269.75 & 8522.972 & 1814.5 & 72641049.583 & -0.231 & 1.15 & 1.359 \\
宽城区 & 28 & 62226 & 2737 & 14677.179 & 12815.741 & 11929.5 & 164243228.819 & 6.696 & 2.331 & 0.873 \\
绿园区 & 28 & 74611 & 0 & 14359.071 & 19237.054 & 4950 & 370064262.884 & 2.644 & 1.608 & 1.339 \\
二道区 & 28 & 29122 & 0 & 7029.643 & 8332.783 & 4597 & 69435264.831 & 1.038 & 1.294 & 1.185 \\
长春新区 & 28 & 31737 & 0 & 8233.643 & 8439.917 & 4434.5 & 71232202.701 & 0.975 & 1.265 & 1.025 \\
经开区 & 28 & 46424.5 & 0 & 8300.214 & 12201.024 & 2150 & 148864993.897 & 2.668 & 1.725 & 1.470 \\
\hline
\end{tabular}
\end{table}

\begin{table}
\centering
\begin{tabular}{|c|c|c|c|c|c|c|c|c|c|}
\hline
净月区 & 28 & 24522 & 429 & 7294.75 & 6427.184 & 5021.5 & 41308690.935 & 0.798 & 1.211 & 0.881 \\
\hline
汽开区 & 28 & 16876 & 0 & 3195.357 & 4523.415 & 688 & 20461285.108 & 2.795 & 1.718 & 1.416 \\
\hline
总量 & 28 & 246152.5 & 5000 & 81162.036 & 69729.606 & 57711.5 & 4862218019.295 & 0.23 & 1.097 & 0.859 \\
\hline
\end{tabular}
\end{table}

综合来看,九个区蔬菜包的整体需求规律如下:

\textbf{规律1:} 蔬菜包需求量随疫情变化具有阶段性特征。结合图6-10各区蔬菜包需求应量的时序图和图6-11蔬菜包需求量与新冠感染人数的时序变化折线图。蔬菜包输球划分为四个阶段,一是新冠肺炎疫情快速反弹阶段(4月4日到4月8日,阶段1),在此阶段,居民对蔬菜包的需求量呈持续上升态势,主要需求地是绿园区和宽城区分别占此阶段九区总需求量的23.82\%和21.97\%。这一部分原因是受新冠新增感染人数的影响,新冠病毒感染人数持续上涨并在4月5日达到顶峰(新增感染2340人),这是自3月4日到4月13日期间的第三次较大规模反弹,主要发生在宽城区(新增感染916人)、绿园区(新增感染313人)、经开区(新增感染312人)。另一部分原因是绿园区小区人口数较多为372541人在九区中人口排名居于第四位。不过宽城区人口数为306143人在九区中人口排名居于第六位,人口占九区总人口的10.13\%,而总需求量占九区的23.82\%,这表明宽城区的新冠肺炎感染导致隔离的人数较其他区更多。二是新冠肺炎疫情快速缓解阶段(4月8日到4月11日)在此阶段,居民对蔬菜包的需求量仍持续上升直到4月11日达到顶峰为(246152.5包),主要分布在绿园区(57047包,23.18\%)、经开区(46424.5包,18.86\%)、朝阳区(37189.5,15.1\%)。三是新冠肺炎疫情局势趋于清零阶段(4月11日到4月17日),随着新增人数的减少直到4月14日实现零增长,蔬菜包的需求量呈下降趋势,在4月17日降低至4月4日以来需求量的最低点29336包。四是新冠肺炎疫情持续零增长阶段(4月17日到5月1日),九区的蔬菜包每日需求量小幅度上升,但上升幅度极小,相比于前三个阶段,一直保持较低的需求水平。

\begin{figure}[h]
\centering
\includegraphics[width=\textwidth]{image.png}
\caption{各区蔬菜包需求应量的时序图(2022.4.4-2022.5.1)}
\end{figure}

\begin{figure}[h]
    \centering
    \includegraphics[width=\textwidth]{image1.png}
    \caption{蔬菜包需求量与新冠感染人数的时序变化折线图(2022.4.4-2022.5.1)}
    \label{fig:6-11}
\end{figure}

规律2:蔬菜包的需求量近似正态分布。根据正态性检验直方图、P-P 图、Q-Q 图都表明虽然数据不是绝对正态,但基本可接受为正态分布。蔬菜包供应量的正态性检验如表 \ref{tab:6-4} 所示。蔬菜包需求量的正态性检验直方图如图 \ref{fig:6-9-a} 所示,展示了九区蔬菜包发放总计数据的正态性检验直方图,正态图基本上呈现出钟形(中间高,两端低),则说明数据虽然不是绝对正态,但基本可接受为正态分布。图 \ref{fig:6-9-b} 中展示了正态性检验 P-P 体现出观测的累计概率(P)与正态累计概率(P)的拟合情况,从拟合程度来看蔬菜包的供应量近似服从正态分布。6-9-(c)展示了正态性检验 Q-Q 的结果。从图中可看出,散点与直线重合度较高蔬菜包的需求近似服从正态分布。

\begin{table}[h]
    \centering
    \caption{蔬菜包需求量的正态性检验结果}
    \label{tab:6-4}
    \begin{tabular}{c c c c c c c c}
        \hline
        变量名 & 样本量 & 中位数 & 平均值 & 标准差 & 偏度 & 峰度 & S-W 检验 & K-S 检验 \\
        \hline
        28 & 57711.5 & 81162.036 & 69729.606 & 1.097 & 0.23 & 0.861(0.002***) & 0.177(0.309) & 28 \\
        \hline
    \end{tabular}
    \begin{flushleft}
        注:***、**、*分别代表 1\%、5\%、10\%的显著性水平
    \end{flushleft}
\end{table}

\begin{figure}[h]
    \centering
    \begin{subfigure}[b]{0.45\textwidth}
        \includegraphics[width=\textwidth]{image2.png}
        \caption{正态性检验直方图}
        \label{fig:6-9-a}
    \end{subfigure}
    \hfill
    \begin{subfigure}[b]{0.45\textwidth}
        \includegraphics[width=\textwidth]{image3.png}
        \caption{正态性检验 P-P 图}
        \label{fig:6-9-b}
    \end{subfigure}
\end{figure}

\begin{figure}[h]
    \centering
    \includegraphics[width=\textwidth]{image1.png}
    \caption{(c)正态性检验 Q-Q 图}
    \label{fig:6-9}
\end{figure}

图 6-9. 蔬菜包供应量的正态性检验

\section*{规律 3:蔬菜包的需求量与各区新增感染病例的关系是大部分是弱相关。如图 6-12 蔬菜包需求量与新冠感染人数的 Spearman 相关性检验热力图中绿园区和二道区的蔬菜包的需求量与各区新增感染病例是中度相关,Spearman 相关系数分别为 0.568426 和 0.556370。其他均为弱相关。}

\begin{figure}[h]
    \centering
    \includegraphics[width=\textwidth]{image2.png}
    \caption{蔬菜包需求量与新冠感染人数的 Spearman 相关性检验热力图}
    \label{fig:6-12}
\end{figure}

\section*{规律 4:蔬菜包的各区需求量与各小区信息、街道数和感染人数的具有相关性。由图 6-13 的 person 相关系数热力图来看,蔬菜包的各区需求量总量与新增感染人数强相关 (0.779),与街道数目极弱相关 (0.181),与小区栋数弱相关 (0.349)、与小区户数弱}

相关(0.334)。同时,也可反映出小区人口数与小区户数、小区栋数之间是极强相关,同时小区的相关信息都与街道数目中度相关。

\begin{table}[h]
\centering
\begin{tabular}{c|c|c|c|c|c|c}
\hline
 & 新增感染人数(本土+... & 0.779 & 0.146 & 0.164 & 0.000 & 0.213 \\
\hline
街道数目 & 0.181 & 0.444 & 0.424 & 0.453 & 1.000 & 0.213 \\
\hline
小区栋数 & 0.349 & 0.911 & 0.912 & 1.000 & 0.453 & 0.000 \\
\hline
小区户数(户) & 0.358 & 0.999 & 1.000 & 0.912 & 0.424 & 0.164 \\
\hline
小区人口数(人) & 0.334 & 1.000 & 0.999 & 0.911 & 0.444 & 0.146 \\
\hline
蔬菜包需求总量(4... & 1.000 & 0.334 & 0.358 & 0.349 & 0.181 & 0.779 \\
\hline
\end{tabular}
\end{table}

图6-13. 蔬菜包的各区需求量与各小区信息、街道数和感染人数的person相关性分析

\subsection*{6.2.3. 蔬菜包的供需规律}

对比4月4日到5月1日之间,九个区蔬菜包需求量、供给量的情况可以探究蔬菜包的供需规律。由于蔬菜包既有蔬菜又穿插肉、蛋、奶和水果,都属于易腐产品。若当日的供给量大于需求量,则会造成蔬菜包的浪费。若当日的供给量小于需求量,则会产生缺货问题不能满足隔离群众的基本生活需要。从图6-14蔬菜包的供需数量对比分析来看再4月4日到4月8日之间以及4月17日到4月24日之间存在蔬菜包的轻微浪费,这两个时间段处于疫情扩散和反弹阶段。而在4月8日到13日之间存在蔬菜包的较为严重的短缺问题。从图6-15各区蔬菜包的浪费和缺货情况来看,蔬菜包浪费量最大的区是绿园区(浪费94764包),其次是宽城区(浪费75122)。发生蔬菜包浪费问题频次最高的区是二道区发生了11次。蔬菜包缺货量最大的区是南关区(缺货68255包),其次是长春新区(缺货55248.5包),净月区(缺货46953包)。发生缺货问题频次最高的区是净月区发生了12次。发生浪费或缺货频次最高的区是二道区共发生了21次。

\begin{figure}[h]
    \centering
    \includegraphics[width=\textwidth]{image1.png}
    \caption{蔬菜包的供需数量对比分析}
    \label{fig:supply_demand}
\end{figure}

\begin{table}[h]
    \centering
    \begin{tabular}{c c c c c c c c c c}
        & 朝阳区 & 南关区 & 宽城区 & 绿园区 & 二道区 & 长春新区 & 经开区 & 净月区 & 汽开区 \\
        \hline
        4月4日 & 0 & -1620 & -2588 & 0 & 3000 & -4000 & 0 & 5810 & 0 \\
        4月5日 & -2323 & -260 & -4650 & 0 & 4000 & -45 & 300 & -5374 & -3560 \\
        4月6日 & 10000 & -135 & -850 & 0 & 14235 & -4384 & -6626 & -5083 & -5085 \\
        4月7日 & -5101 & 12768 & -2344 & 0 & -12125 & 5695 & 0 & -8871 & -2510 \\
        4月8日 & -100 & 19081 & 2927 & 0 & -5234 & 3737 & 0 & 0 & -16876 \\
        4月9日 & -37 & 1120 & 31245 & 8656 & 4673 & 30268 & 2400 & 0 & -3203 \\
        4月10日 & 2111 & 26042 & 21189 & 11960 & 97 & -8687 & 47 & 0 & -7817 \\
        4月11日 & 1792 & 3972 & 9875 & 12781 & -1165 & 11735 & 4922.5 & 0 & 1654 \\
        4月12日 & 5995 & 1297 & -4251 & 43446 & 13514 & -18128 & 18324.5 & 4000 & -4973 \\
        4月13日 & -7466 & -13428 & 99 & -13158 & 19644 & -5255 & -994 & -1396 & -6045 \\
        4月14日 & 0 & -18153.5 & -874 & 0 & 5767 & -12913 & 0 & -4992 & -2055 \\
        4月15日 & 561 & -6407 & 3223 & 0 & -87 & -1326 & 4000 & -2438 & -4000 \\
        4月16日 & 0 & -9224.5 & 5064 & 0 & -4635 & 4672 & 5200 & -3012 & -1800 \\
        4月17日 & -585 & -3070 & -6500 & 17921 & -1168 & 7764 & -4000 & 1496 & 1440 \\
        4月18日 & 0 & -6040 & -2438 & 0 & 1400 & 0 & 0 & -4960 & 613 \\
        4月19日 & 0 & -7887 & -1025 & 0 & 602 & -400 & 0 & -2165 & 1024 \\
        4月20日 & 0 & 0 & 0 & 0 & 1600 & 1899 & 0 & -1804 & -2354 \\
        4月21日 & 0 & -2000 & 0 & 0 & -296 & -111 & -3300 & -3429 & 0 \\
        4月22日 & -2106 & 0 & 1500 & 0 & -4096 & 1889 & -3300 & -3429 & 0 \\
        4月23日 & 520 & 0 & 0 & 0 & -1393 & 0 & 0 & 0 & 1140 \\
        4月24日 & 0 & 0 & 0 & 0 & 0 & 0 & 0 & 0 & 1360 \\
        4月25日 & 0 & 0 & 0 & 0 & -50 & 0 & 0 & 0 & 1550 \\
        4月26日 & 0 & 0 & 0 & 0 & 0 & 0 & 0 & 0 & 1544 \\
        4月27日 & 0 & 0 & 0 & 0 & 0 & 0 & 0 & 0 & 1500 \\
        4月28日 & 0 & 0 & 0 & 0 & 0 & 0 & 0 & 0 & 2400 \\
        4月29日 & 0 & 0 & 0 & 0 & 0 & 0 & 0 & 0 & 650 \\
        4月30日 & 0 & 0 & 0 & 0 & 0 & 0 & 0 & 0 & 1400 \\
        5月1日 & 0 & 0 & 0 & 0 & 0 & 0 & 0 & 0 & 2100 \\
    \end{tabular}
    \caption{各区蔬菜包的浪费和缺货情况}
    \label{tab:vegetable_waste_deficit}
\end{table}

\section{子问题2:调整蔬菜包供应方案——TOPSIS综合评价与优化}
\subsection{TOPSIS介绍}

\begin{tabular}{l l}
\hline
\textbf{步骤} & \textbf{操作} \\
\hline
Step 1 & 1) 定义各个维度的理想化目标; \\
 & 2) 计算各个维度距离理想化的距离 \\
 & 3) 保证各维度效用函数单调性 \\
\hline
Step 2 & TOPSIS 将数据指标分为四种类型 \\
 & 1) 极大型指标: 数值越大越好, 如收入、利润、成绩及肺活量等。 \\
 & 2) 极小型指标: 数值越小越好, 如成本、损失及污染程度等。 \\
 & 3) 中间型指标: 越接近某个固定值越好, 如水的 ph 值。 \\
 & 4) 区间型指标: 落在某个区间上最好, 如人的体温。 \\
\hline
Step 3 & 数据正向化: 为了统一指标维度, 需要将数据指标正向化, 即让其指标数值越大越好; \\
 & 1) 极大型指标不需要做正向化处理; \\
 & 2) 常用极小型指标转换公式。如式 (6-2) 所示。 \\
 & 3) 中间型指标转换, 计算 M 值 (即距离最好的固定值距离) 和转换公式。如式 (6-3) 和式 (6-4) 所示。 \\
 & 4) 区间型指标转换。假设最好的区间为 $[a, b]$ 同样计算 M 值 (即距离最好的区间值距离), 如公式 (6-5) 所示。转换公式如式 (6-6) 所示。 \\
\hline
Step 4 & 数据模标准化处理: 为了消除各数据指标之间的量纲影响, 对其进行模标准化, 公式如 (6-7) 所示。 \\
\hline
Step 5 & 构建评分: \\
 & 1) 找出最优数据指标点, 即各个指标维度的最大值, 计算当前样本距离该最优数据点的距离, 记为 $D+$; \\
 & 2) 找出最劣数据指标点, 即各个指标维度的最小值, 计算当前样本距离该最劣数据点的距离, 记为 $D-$; \\
 & 3) 构建评分 $C$ 并对样本进行排序, 得到最终结果。计算方法如式 (6-8) 所示。 \\
\hline
\end{tabular}

\begin{equation}
x_i =
\begin{cases}
max - x_i \text{(通用公式)} \\
\frac{1}{x}i \text{(数据指标均为正的情况下)}
\end{cases}
\tag{6-2}
\end{equation}

\begin{equation}
M = max\{|X - x_{\text{best}}|\}
\tag{6-3}
\end{equation}

\begin{equation}
x_i = 1 - \frac{(x_i - x_{\text{best}})}{M}
\tag{6-4}
\end{equation}

\begin{equation}
M = max\{a - min\{X\}, max\{X\} - b\}
\tag{6-5}
\end{equation}

\begin{equation}
x_{i}=
\begin{cases}
1-\frac{a-x_{i}}{M} & x<a \\
1 & a\leqslant x\leqslant b \\
1-\frac{x_{i}-b}{M} & x>b
\end{cases}
\tag{6-6}
\end{equation}

\begin{equation}
z_{i}=\frac{x_{i}}{\sqrt{\sum_{i=1}^{n}x_{i}^{2}}}
\tag{6-7}
\end{equation}

\begin{equation}
C=\frac{(D-)}{D_{+}+D_{-}}
\tag{6-8}
\end{equation}

Python 实现请见附录,4 月 10 日至 4 月 15 日 TOSIS 指标体系权重与各区得分和排名请见附件。

\subsection*{6.3.2. 评价指标体系}

本文构建的蔬菜包保供方案的 TOPSIS 评价指标体系如表 6-4 所示。具体做法是,依据附件 1 汇总 4 月 10 日到 4 月 15 日的各区本土感染人数和无症状感染者人数,提取并统计附件 3 中的各区的小区栋数、小区户数(户)和小区人口数(人),此外我们考虑到交通状况也是影响物资供应方案的重要因素,因此我们根据附件 3 中的街道编号统计出了各区的街道数量。

\begin{table}[h]
\centering
\caption{蔬菜包保供方案的 TOPSIS 评价指标体系}
\begin{tabular}{c c c c c c c}
\hline
\multirow{2}{*}{ 时间 } & \multirow{2}{*}{ 区 } & \multicolumn{5}{c}{ 评价指标 } \\
\cline{3-7}
& & 小区栋数 & 小区户数(户) & 小区人口数(人) & 街道数目(个) & 本土感染人数(例) & 无症状感染人数(例) \\
\hline
4 月 10 & 朝阳区 & 3740 & 219874 & 548179 & 69 & 3 & 32 \\
4 月 10 & 南关区 & 2336 & 183877 & 462146 & 68 & 7 & 55 \\
\cdots & \cdots & \cdots & \cdots & \cdots & \cdots & \cdots & \cdots \\
4 月 11 & 宽城区 & 1524 & 128899 & 306143 & 58 & 34 & 151 \\
4 月 11 & 绿园区 & 2198 & 155261 & 372541 & 57 & 6 & 158 \\
\cdots & \cdots & \cdots & \cdots & \cdots & \cdots & \cdots & \cdots \\
4 月 12 & 二道区 & 1820 & 166428 & 405784 & 50 & 8 & 51 \\
4 月 12 & 长春新区 & 2116 & 141780 & 338274 & 29 & 6 & 55 \\
\cdots & \cdots & \cdots & \cdots & \cdots & \cdots & \cdots & \cdots \\
4 月 13 & 净月区 & 1507 & 85748 & 204056 & 59 & 7 & 26 \\
4 月 13 & 汽开区 & 1171 & 81675 & 193462 & 54 & 5 & 9 \\
\cdots & \cdots & \cdots & \cdots & \cdots & \cdots & \cdots & \cdots \\
4 月 14 & 朝阳区 & 3740 & 219874 & 548179 & 69 & 7.2 & 10 \\
4 月 14 & 南关区 & 2336 & 183877 & 462146 & 68 & 18.72 & 5 \\
\cdots & \cdots & \cdots & \cdots & \cdots & \cdots & \cdots & \cdots \\
4 月 15 & 净月区 & 1507 & 85748 & 204056 & 59 & 4 & 17.81 \\
4 月 15 & 汽开区 & 1171 & 81675 & 193462 & 54 & 10 & 6.165 \\
\hline
\end{tabular}
\end{table}

\subsection*{6.3.3. 评价结果}

\begin{table}
\centering
\caption{表6-5. 蔬菜包保供方案的TOPSIS评价各区综合得分情况}
\begin{tabular}{c c c c}
\hline
时间 & 区 & 综合得分 & 求和 \\
\hline
4月10 & 朝阳区 & 0.46537736 & 3.11868048 \\
4月10 & 南关区 & 0.40893815 & 3.11868048 \\
4月10 & 宽城区 & 0.66279791 & 3.11868048 \\
4月10 & 绿园区 & 0.53561046 & 3.11868048 \\
4月10 & 二道区 & 0.34455529 & 3.11868048 \\
4月10 & 长春新区 & 0.26565694 & 3.11868048 \\
4月10 & 经开区 & 0.17950934 & 3.11868048 \\
4月10 & 净月区 & 0.14693533 & 3.11868048 \\
4月10 & 汽开区 & 0.1092997 & 3.11868048 \\
\hline
4月11 & 朝阳区 & 0.52775655 & 3.19815469 \\
4月11 & 南关区 & 0.44533184 & 3.19815469 \\
4月11 & 宽城区 & 0.63806236 & 3.19815469 \\
4月11 & 绿园区 & 0.497046 & 3.19815469 \\
4月11 & 二道区 & 0.34045462 & 3.19815469 \\
4月11 & 长春新区 & 0.28826481 & 3.19815469 \\
4月11 & 经开区 & 0.1529691 & 3.19815469 \\
4月11 & 净月区 & 0.19437041 & 3.19815469 \\
4月11 & 汽开区 & 0.113899 & 3.19815469 \\
\hline
4月12 & 朝阳区 & 0.51438545 & 3.07800004 \\
4月12 & 南关区 & 0.42111607 & 3.07800004 \\
4月12 & 宽城区 & 0.65760758 & 3.07800004 \\
4月12 & 绿园区 & 0.45946199 & 3.07800004 \\
4月12 & 二道区 & 0.32338452 & 3.07800004 \\
4月12 & 长春新区 & 0.26319514 & 3.07800004 \\
4月12 & 经开区 & 0.29535501 & 3.07800004 \\
4月12 & 净月区 & 0.14349428 & 3.07800004 \\
4月12 & 汽开区 & 0.10915525 & 3.07800004 \\
\hline
4月13 & 朝阳区 & 0.41640057 & 2.78678981 \\
4月13 & 南关区 & 0.36289706 & 2.78678981 \\
4月13 & 宽城区 & 0.69920281 & 2.78678981 \\
4月13 & 绿园区 & 0.4259294 & 2.78678981 \\
\hline
\end{tabular}
\end{table}

\begin{tabular}{llll}
4月13 & 二道区 & 0.30964448 & 2.78678981 \\
4月13 & 长春新区 & 0.22633662 & 2.78678981 \\
4月13 & 经开区 & 0.11533849 & 2.78678981 \\
4月13 & 净月区 & 0.13482233 & 2.78678981 \\
4月13 & 汽开区 & 0.09621805 & 2.78678981 \\
4月14 & 朝阳区 & 0.41693092 & 2.78817051 \\
4月14 & 南关区 & 0.36071757 & 2.78817051 \\
4月14 & 宽城区 & 0.69696051 & 2.78817051 \\
4月14 & 绿园区 & 0.42326617 & 2.78817051 \\
4月14 & 二道区 & 0.30783534 & 2.78817051 \\
4月14 & 长春新区 & 0.22579334 & 2.78817051 \\
4月14 & 经开区 & 0.10846292 & 2.78817051 \\
4月14 & 净月区 & 0.13490041 & 2.78817051 \\
4月14 & 汽开区 & 0.11330333 & 2.78817051 \\
4月15 & 朝阳区 & 0.4195516 & 2.80742699 \\
4月15 & 南关区 & 0.36077513 & 2.80742699 \\
4月15 & 宽城区 & 0.69460301 & 2.80742699 \\
4月15 & 绿园区 & 0.43217672 & 2.80742699 \\
4月15 & 二道区 & 0.30893402 & 2.80742699 \\
4月15 & 长春新区 & 0.22926816 & 2.80742699 \\
4月15 & 经开区 & 0.11377876 & 2.80742699 \\
4月15 & 净月区 & 0.13600009 & 2.80742699 \\
4月15 & 汽开区 & 0.1123395 & 2.80742699 \\
\end{tabular}

\textbf{表6-6. 蔬菜包保供方案的TOPSIS评价指标权重(以4月10日为例)}

\begin{tabular}{llll}
 & TOPSIS & & \\
评价指标 & 信息熵值e & 信息效用值d & 权重(\%) \\
小区栋数 & 0.876 & 0.124 & 12.499 \\
小区户数(户) & 0.832 & 0.168 & 16.921 \\
小区人口数(人) & 0.811 & 0.189 & 18.96 \\
感染人数 & 0.782 & 0.218 & 21.894 \\
无症状感染者 & 0.777 & 0.223 & 22.384 \\
街道数目 & 0.927 & 0.073 & 7.342 \\
\end{tabular}

\textbf{表6-7. 蔬菜包保供方案中各区的综合得分与排名(以4月10日为例)}

\begin{tabular}{lllll}
索引值 & 正理想解距离(D+) & 负理想解距离(D-) & 综合得分指数 & 排序 \\
朝阳区 & 0.43174588 & 0.48249842 & 0.52775655 & 2 \\
南关区 & 0.44869475 & 0.36024793 & 0.44533184 & 4 \\
宽城区 & 0.31381615 & 0.55322867 & 0.63806236 & 1 \\
绿园区 & 0.41786229 & 0.41295383 & 0.497046 & 3 \\
二道区 & 0.51620738 & 0.26646413 & 0.34045462 & 5 \\
长春新区 & 0.53799241 & 0.21789604 & 0.28826481 & 6 \\
经开区 & 0.63752883 & 0.11513418 & 0.1529691 & 8 \\
净月区 & 0.60478974 & 0.14591473 & 0.19437041 & 7 \\
\end{tabular}

\begin{table}
\caption{中间值计算结果(以4月10日为例)}
\begin{tabular}{c c c}
\hline
项 & 正理想解 & 负理想解 \\
\hline
小区栋数 & 0.71336706 & 2e-8 \\
小区户数(户) & 0.62073166 & 0 \\
小区人口数(人) & 0.6345996 & 0 \\
感染人数 & 0.91425388 & 0.00000269 \\
无症状感染者 & 0.67670733 & 4.4e-7 \\
街道数目 & 0.48126413 & 0.0000012 \\
\hline
\end{tabular}
\end{table}

\subsection*{6.3.4. 结果分析}

根据各区的每天的综合得分,按照每天各区的得分比重计算出每天合理的蔬菜包供应量。为评价蔬菜包供应方案的调整效果,本文利用数据挖掘技术总结和统计出4月10日至4月15日,各区每天的蔬菜包接收数目和发放数目如表6-9所示。其中,发放数目是指各区实际发放给管辖小区的蔬菜包数量,因此本文将其定义为小区居民每日蔬菜包的需求量。接收数目是指各区接收到的市州支援、市级集采和属地自保的蔬菜包,表示各区每日蔬菜包的供应能力,即每日可提供给其管辖小区的蔬菜包的数量。

由于蔬菜包属于易腐商品,特别是在包装或仓储条件不满足保鲜要求时极易在短时间内发生霉烂和变质。通过查阅资料我们发现,长春市的蔬菜包一般采用纸箱包装如图6-16所示,这意味着蔬菜包在不具保鲜作用的纸箱中隔夜存放很大概率会发生变质。因此,本文假设假设蔬菜包隔夜存放第二天会变质不能再继续发放给隔离群众。每区当日的蔬菜包接收量等价于每日每区的供应能力,每区当日的蔬菜包发放量等价于每日各区隔离群众的需求量。附件5中所存在的每日发放量大于接收量的情况视为缺货,反之视为浪费。将蔬菜包浪费或缺货数目进行对比作为供应方案调整效果的评价指标。各区蔬菜包供应方案的调整计算结果如表6-10所示,各区调整效果对比如表6-11所示。蔬菜包供应方案调整后九个区的蔬菜包浪费量或缺货问题得到明显改善。

\begin{table}
\caption{各区每天的蔬菜包接收数目和发放数目(2022.4.10-2022.4.15)}
\begin{tabular}{c c c c}
\hline
时间 & 区 & 发放数目(需求数目) & 接收数目(供应数目) \\
\hline
4月10 & 朝阳区 & 38244 & 40355 \\
4月10 & 南关区 & 25285 & 27671 \\
4月10 & 宽城区 & 62226 & 83415 \\
4月10 & 绿园区 & 32812 & 44772 \\
4月10 & 二道区 & 26908 & 27005 \\
4月10 & 长春新区 & 17874 & 9187 \\
4月10 & 经开区 & 24360 & 24407 \\
4月10 & 净月区 & 11594 & 11594 \\
4月10 & 汽开区 & 8777 & 17136 \\
4月11 & 朝阳区 & 37189.5 & 38981 \\
4月11 & 南关区 & 23223.5 & 23013 \\
4月11 & 宽城区 & 13251 & 23126 \\
4月11 & 绿园区 & 57047 & 69828 \\
4月11 & 二道区 & 29122 & 27957 \\
\hline
\end{tabular}
\end{table}

\begin{tabular}{l l l l}
4月11 & 长春新区 & 21893.5 & 33628 \\
4月11 & 经开区 & 46424.5 & 51347 \\
4月11 & 净月区 & 7462 & 7462 \\
4月11 & 汽开区 & 14722 & 16376 \\
4月12 & 朝阳区 & 23125 & 29120 \\
4月12 & 南关区 & 14447 & 22668 \\
4月12 & 宽城区 & 12800 & 8549 \\
4月12 & 绿园区 & 74611 & 118057 \\
4月12 & 二道区 & 17698 & 31212 \\
4月12 & 长春新区 & 31737 & 13609 \\
4月12 & 经开区 & 16663.5 & 34988 \\
4月12 & 净月区 & 19333 & 23333 \\
4月12 & 汽开区 & 5862.5 & 8417 \\
4月13 & 朝阳区 & 19475.5 & 12010 \\
4月13 & 南关区 & 17111.5 & 11857 \\
4月13 & 宽城区 & 2737 & 2836 \\
4月13 & 绿园区 & 39967 & 26809 \\
4月13 & 二道区 & 10637 & 30281 \\
4月13 & 长春新区 & 25254.5 & 20000 \\
4月13 & 经开区 & 13994 & 13000 \\
4月13 & 净月区 & 24522 & 23126 \\
4月13 & 汽开区 & 6044.5 & 5431 \\
4月14 & 朝阳区 & 11874 & 11874 \\
4月14 & 南关区 & 5070 & 5070 \\
4月14 & 宽城区 & 5385 & 4511 \\
4月14 & 绿园区 & 16356 & 16356 \\
4月14 & 二道区 & 7142 & 12909 \\
4月14 & 长春新区 & 17913 & 5000 \\
4月14 & 经开区 & 20000 & 20000 \\
4月14 & 净月区 & 18948 & 13956 \\
4月14 & 汽开区 & 7500 & 16127 \\
4月15 & 朝阳区 & 8285 & 8846 \\
4月15 & 南关区 & 8040 & 8040 \\
4月15 & 宽城区 & 10244 & 13467 \\
4月15 & 绿园区 & 22380 & 22380 \\
4月15 & 二道区 & 5748 & 5661 \\
4月15 & 长春新区 & 8731 & 7405 \\
4月15 & 经开区 & 25000 & 29000 \\
4月15 & 净月区 & 14136 & 11698 \\
4月15 & 汽开区 & 5000 & 5000 \\
\end{tabular}

数据来源:依据附件5整理而得。

\begin{figure}[h]
    \centering
    \includegraphics[width=\textwidth]{image.png}
    \caption{图6-16.长春市的蔬菜包的包装规格}
    \label{fig:vegetable_package}
\end{figure}

\textbf{图片来源:多种“蔬菜包”供选择,长春长青街道送菜忙 (baidu.com)}

\begin{table}[h]
    \centering
    \caption{表6-10. 各区蔬菜包供应的优化方案与调整幅度}
    \label{tab:vegetable_supply}
    \begin{tabular}{c c c c c}
        \hline
        时间 & 区 & 得分占比 & 调整后的供应方案(包) & 调整幅度 \\
        & & & 接收蔬菜包*得分占比 & \\
        \hline
        4月10 & 朝阳区 & 0.1492225 & 43573 & 7.97\% \\
        4月10 & 南关区 & 0.1311254 & 38289 & 38.37\% \\
        4月10 & 宽城区 & 0.2125251 & 62058 & -25.60\% \\
        4月10 & 绿园区 & 0.1717427 & 50149 & 12.01\% \\
        4月10 & 二道区 & 0.1104811 & 32261 & 19.46\% \\
        4月10 & 长春新区 & 0.0851825 & 24873 & 170.75\% \\
        4月10 & 经开区 & 0.0575594 & 16807 & -31.14\% \\
        4月10 & 净月区 & 0.0471146 & 13758 & 18.66\% \\
        4月10 & 汽开区 & 0.0350468 & 10234 & -40.28\% \\
        \hline
        4月11 & 朝阳区 & 0.1650191 & 48436 & 24.26\% \\
        4月11 & 南关区 & 0.1392465 & 40871 & 77.60\% \\
        4月11 & 宽城区 & 0.1995095 & 58560 & 153.22\% \\
        4月11 & 绿园区 & 0.1554165 & 45618 & -34.67\% \\
        4月11 & 二道区 & 0.1064535 & 31246 & 11.76\% \\
        4月11 & 长春新区 & 0.0901347 & 26456 & -21.33\% \\
        4月11 & 经开区 & 0.0478304 & 14039 & -72.66\% \\
        4月11 & 净月区 & 0.0607758 & 17839 & 139.06\% \\
        4月11 & 汽开区 & 0.035614 & 10453 & -36.17\% \\
        \hline
        4月12 & 朝阳区 & 0.1671168 & 48605 & 66.91\% \\
        4月12 & 南关区 & 0.1368148 & 39792 & 75.54\% \\
        4月12 & 宽城区 & 0.2136477 & 62138 & 626.84\% \\
        4月12 & 绿园区 & 0.1492729 & 43415 & -63.23\% \\
        4月12 & 二道区 & 0.1050632 & 30557 & -2.10\% \\
        \hline
    \end{tabular}
\end{table}

\begin{table}
\centering
\begin{tabular}{l l l l l}
\hline
4月12 & 长春新区 & 0.0855085 & 24870 & 82.74\% \\
4月12 & 经开区 & 0.0959568 & 27908 & -20.23\% \\
4月12 & 净月区 & 0.0466193 & 13559 & -41.89\% \\
4月12 & 汽开区 & 0.035463 & 10314 & 22.54\% \\
4月13 & 朝阳区 & 0.1494194 & 21718 & 80.83\% \\
4月13 & 南关区 & 0.1302205 & 18928 & 59.63\% \\
4月13 & 宽城区 & 0.250899 & 36468 & 1185.90\% \\
4月13 & 绿园区 & 0.1528387 & 22215 & -17.14\% \\
4月13 & 二道区 & 0.1111115 & 16150 & -46.67\% \\
4月13 & 长春新区 & 0.0812177 & 11805 & -40.98\% \\
4月13 & 经开区 & 0.0413876 & 6016 & -53.73\% \\
4月13 & 净月区 & 0.0483791 & 7032 & -69.59\% \\
4月13 & 汽开区 & 0.0345265 & 5018 & -7.60\% \\
4月14 & 朝阳区 & 0.1495357 & 17097 & 43.98\% \\
4月14 & 南关区 & 0.1293743 & 14791 & 191.75\% \\
4月14 & 宽城区 & 0.2499705 & 28579 & 533.55\% \\
4月14 & 绿园区 & 0.1518078 & 17356 & 6.12\% \\
4月14 & 二道区 & 0.1104076 & 12623 & -2.22\% \\
4月14 & 长春新区 & 0.0809826 & 9259 & 85.18\% \\
4月14 & 经开区 & 0.0389011 & 4448 & -77.76\% \\
4月14 & 净月区 & 0.0483831 & 5532 & -60.36\% \\
4月14 & 汽开区 & 0.0406372 & 4646 & -71.19\% \\
4月15 & 朝阳区 & 0.1494435 & 16999 & 92.16\% \\
4月15 & 南关区 & 0.1285074 & 14617 & 81.81\% \\
4月15 & 宽城区 & 0.2474162 & 28143 & 108.98\% \\
4月15 & 绿园区 & 0.1539405 & 17510 & -21.76\% \\
4月15 & 二道区 & 0.1100417 & 12517 & 121.11\% \\
4月15 & 长春新区 & 0.0816649 & 9289 & 25.44\% \\
4月15 & 经开区 & 0.0405278 & 4610 & -84.10\% \\
4月15 & 净月区 & 0.048443 & 5510 & -52.90\% \\
4月15 & 汽开区 & 0.0400151 & 4552 & -8.97\% \\
\hline
\end{tabular}
\end{table}

\textbf{蔬菜包保供方案调整效果对比如表6-11所示。本文基于TOSIS综合评价结果调整各区蔬菜保供应方案能有效减少蔬菜包每日浪费或缺货的问题,从改善效果来看能减轻13.84\%的浪费或缺货数量,降低14.8\%的浪费问题发生频次,降低25\%的缺货问题发生频次。综上,本文提出的蔬菜包供应调整方案有效且可行。}

\begin{table}
\centering
\begin{tabular}{l l l l l}
\hline
\multicolumn{2}{l}{ 效果评价指标 } & \multicolumn{1}{l}{ 浪费或缺货数量 } & \multicolumn{1}{l}{ 浪费发生的频次 } & \multicolumn{1}{l}{ 缺货发生的频次 } \\
日期 & 调整前 & 调整后 & 调整前 & 调整后 \\
\hline
2022/4/10 & 36540 & 36540 & 7 & 7 & 1 & 2 \\
2022/4/11 & 42635 & 40735 & 6 & 4 & 2 & 1 \\
2022/4/12 & 73717 & 54031.178 & 7 & 4 & 0 & 0 \\
2022/4/13 & -14621 & -12621 & 2 & 1 & 7 & 6 \\
\hline
\end{tabular}
\caption{表6-11.蔬菜包供应方案的调整效果评价结果}
\end{table}

\begin{table}
\centering
\begin{tabular}{c c c c c c}
\hline
2022/4/14 & 1949 & 1877 & 2 & 4 & 3 \\
2022/4/15 & 3388 & 3128 & 3 & 3 & 3 \\
总计 & 172850 & 148932.178 & 27 & 23 & 16 \\
总体优化效果 & 减轻13.84\% & & 降低14.81\% & & 降低25\% \\
\hline
\end{tabular}
\end{table}

\section{七、问题四分析与求解}

\subsection{7.1. 问题分析}

要求在第二、三问的基础上,结合附件3给出的中长春市街道和小区情况,制作特殊时期保障居民生活物资供应的详细预案(有序网络图)。该网络是一个多级物流网络由上游物资来源地、中游物资集散地和下游长春市所有小区,以及两种规格的卡车(大卡车限载10吨,小卡车限载4吨)组成。此问题根据有序网络的层级,可以分解为三个子问题:子问题一是对上游九个物资来源地进行合理选址,子问题二是对中游物资集散地进行合理的分配和选址,子问题三是在考虑车辆限重的情况下,将“工作量”(工作量=运输里程×小区居民人数)的最小化合理规划末端配送路径以及配送量的重要目标。子问题一的难点在于附件三并未给出交通路口节点的区域划。子问题二和子问题三的难点在于数据规模较大且所建立的数学模型属于混合0-1整数非线性问题难以快速找到全局最优解。

针对子问题一,根据附件3小区的坐标边界,采用射线法法“由点连面”,用于判定交通路口节点是否在小区群边界内,克服了附件3交通路口节点数据缺失区域划分的问题,同时能够过滤掉冗余的交通路口节点信息。采用枚举法在保留下来的交通路口节点中寻找到所有小区的“工作量”指标之和最小的位置。每个区重复上述方法便可得到各区的中心点坐标,即为九个上游物资来源地。

针对子问题二,本文采用轴辐射网络对中游的集散地进行选址和分配。针对子问题三,构建考虑病毒传播和车辆容量限制的路径优化模型。鉴于此问题属于NP-hard问题,本文采用ALNS和SA构成的混合启发式算法进行求解,最终得到由上游物资来源地-中游货物集散地-下游小区的多级有序物流网络、最短配送路径以及合理的配送车辆数量与种类。

\begin{figure}[h]
    \centering
    \includegraphics[width=\textwidth]{image.png}
    \caption{问题四的思路流程图}
    \label{fig:flowchart}
\end{figure}

\section{上游物资来源地的选址——射线法、枚举法}

\subsection{划分交通路口点的区域}

上游物资来源地的选址需要依据附件 3 的交通路口节点数据。但是由于该数据并未给出每个交通路口节点的区域划分,因此不能直接获取各区交通路口节点状况,也就不能保证每个区都有一个上游物资来源地的目标。针对此问题,本文首先采用射线法判断交通路口点是否位于由全区小区构成的不规则形状内,若在则该交通路口点划分到该区,否则从该区中剔除。通过此步骤能够克服附件 3 交通路口节点数据缺失区域划分的问题,同时能够过滤掉冗余的交通路口节点信息。射线法的一般思路如表 \ref{tab:ray_method} 所示。(实现代码请见附录-问题四算法程序)

\begin{table}[h]
    \centering
    \caption{射线法的一般思路}
    \label{tab:ray_method}
    \begin{tabular}{l l}
        \hline
        步骤 & 操作 \\
        \hline
        Step 1 & 从目标点出发朝 $x$ 轴正方向引一条射线,计算此射线与小区构成的多边形所有边的交点数目; \\
        Step 2 & 若交点个数为奇数,则目标点在小区构成的多边形内部; \\
               & 若交点个数为偶数,则目标点在小区构成的多边形外部。 \\
        \hline
    \end{tabular}
\end{table}

\subsection{7.2.2. 确定上游物资来源地在各区的选址}

采用枚举法在保留下来的交通路口节点中寻找到所有小区的“工作量”指标之和最小的位置,计算方法如式(7-1)所示。每个区重复上述方法便可得到各区的中心点坐标,即为九个上游物资来源地。九个上游来源地坐标结果如表7-2所示。

\begin{equation}
\sqrt{\sum_{i=1}^{n}(x_{i}-y_{i})^{2} * \text{各小区人口数}}
\tag{7-1}
\end{equation}

\begin{table}[h]
\centering
\caption{上游物资来源地位置与行政区域划分}
\begin{tabular}{c c c c c}
\hline
No. & 区 & 编号 & $x$ & $y$ \\
\hline
1 & 长春新区 & 3191 & 38.79708 & 20.7877 \\
2 & 汽开区 & 3849 & 34.29753 & 36.3538 \\
3 & 南关区 & 6797 & 57.00713 & 33.26233 \\
4 & 绿园区 & 761 & 42.16913 & 49.3732 \\
5 & 宽城区 & 7701 & 55.64605 & 59.49537 \\
6 & 净月区 & 7425 & 72.76918 & 17.93157 \\
7 & 经开区 & 2258 & 74.4338 & 37.50083 \\
8 & 二道区 & 2558 & 69.9437 & 46.98793 \\
9 & 朝阳区 & 988 & 49.3975 & 38.09937 \\
\hline
\end{tabular}
\end{table}

\section{7.3. 中游集散点的辐射网络模型}

轴辐式网络优化可以分为两层:一方面包括集散点的选取与运输网络的构建;另一方面包括需求点的分配与配送网络的构建,相比较于基础的选址模型,轴辐式网络具有更加复杂的网络结构,模型的求解也相对较为困难。本文中基于启发式算法求解轴辐式网络问题,首先可以将算法看做两个层级,第一层级用于集散点的选取,采用了随机生成的方式生成所需要的枢纽。第二层用于需求点的分配,结合附件3的路线实际距离计算节点到各个集散点的距离,将节点分配到最近的集散点。在构建模型时以最小工作量为目标函数(工作量:工作量按运输里程与小区居民人数乘积计算),对于集散点的数量和需求节点与集散点之间的连接进行了相关约束。

\begin{equation}
\min \left(\sum_{ik} P^{C} C_{ik} O_{i} x_{ik} + \sum_{ikh} P^{T} C_{kh} y_{ikh} + \sum_{ik} P^{D} C_{ki} D_{i} x_{ik}\right) \cdot N_{P}
\tag{7-2}
\end{equation}

\begin{equation}
\sum_{k} x_{ik} = 1, \forall i
\tag{7-3}
\end{equation}

\begin{equation}
x_{ik} \leq x_{kk}, \forall i, k
\tag{7-4}
\end{equation}

\begin{equation}
\sum_{k} x_{kk} = H
\tag{7-5}
\end{equation}

\begin{equation}
\sum_{h} y_{ikh} - \sum_{h} y_{ihk} = O_{i}x_{ik} - \sum_{j} W_{ij}x_{jk}, \forall i, k \tag{7-6}
\end{equation}

\begin{equation}
\sum_{h \neq k} y_{ikh} \leq O_{i}x_{ik}, \forall i, k \tag{7-7}
\end{equation}

\begin{equation}
x_{ik} \in \{0, 1\}, \forall i, k \tag{7-8}
\end{equation}

\begin{equation}
y_{ikh} \geq 0, \forall i, k, h \tag{7-9}
\end{equation}

(7-3)为约束条件,表示一个需求节点只分配一个集散点;(7-4)为约束条件,表示一个需求节点只能和一个集散点连接;(7-5)对集散点的数量进行限制;约束(7-7)确定每个需求节点在集散点路径上的流量;(7-8)表示$x_{ik}$为0-1整数变量;(7-9)表示$y_{ikh}$为大于0的连续型变量

\textbf{表7-3. 中游集散点的辐射网络模型的集合、数据与变量}

\begin{tabular}{c l}
\hline
\textbf{集合} & \\
\hline
$N_{P}$ & 小区居民人数 \\
$H$ & 需要选取的集散点个数 \\
$Q$ & 集散点容量的集合 \\
\hline
\textbf{数据} & \\
\hline
$P$ & 需要选取的集散点数量 \\
$K$ & 车辆集合$K \in \{1, 2, \dots, |K|\}$ \\
$P^{C}, P^{D}, P^{T}$ & 表示辐节点到集散点,集散点到集散点,集散点到辐节点的单位距离费用 \\
$W_{ij}$ & 需求节点之间的流量$i, j \in N$ \\
$O_{i} = \sum_{j} W_{ij}$ & 流出节点$i$的流量$i \in N$ \\
$D_{i} = \sum_{j} W_{ji}$ & 流入节点$i$的流量$j \in N$ \\
\hline
\textbf{变量} & \\
\hline
$x_{ik}$ & 表示需求节点$i$分配到集散点$k$, \\
& $x_{kk} = 1$ 表示$k$为集散点,$i, k \in N$ \\
$y_{ikh}$ & 表示需求节点$i$经过集散点$k$到达集散点$h$ \\
& 的流量$i, k, h \in N$ \\
$x_{ij} \in \{0, 1\}$ & 表示需求节点与集散点的服务关系,若需求 \\
& 节点$i$分配到集散点$j$,$x_{ij} = 1$,否则$x_{ij} = 0$ \\
$y_{i} \in \{0, 1\}$ & 表示集散点$i$是否开放 \\
\hline
\end{tabular}

\begin{table}
\centering
\begin{tabular}{c c c c}
No. & x坐标 & y坐标 & 行政区域划分 \\
\hline
1 & 88.89116 & 19.82388 & 净月区 \\
2 & 41.15211 & 46.26814 & 绿园区 \\
3 & 69.92888 & 40.99724 & 二道区 \\
4 & 38.22631 & 22.3927 & 长春新区(高新) \\
5 & 58.92209 & 62.45739 & 宽城区 \\
96 & 31.99186 & 37.44983 & 汽开区 \\
97 & 68.19788 & 52.67407 & 二道区 \\
98 & 50.22061 & 42.86536 & 朝阳区 \\
99 & 43.10932 & 43.9721 & 汽开区 \\
100 & 58.07429 & 48.57984 & 宽城区 \\
\end{tabular}
\caption{基于轴辐射网络的中游物资集散地选址结果}
\end{table}

\begin{figure}[h]
\centering
\includegraphics[width=\textwidth]{image.png}
\caption{轴辐射网络示意图}
\end{figure}

\section*{7.4. 考虑病毒传播和车量容量限制的路径优化模型}

\subsection*{7.4.1 基于复杂网络的病毒传播模型}

重要的生活物资运送服务将导致病毒在人与人之间传播。复杂网络模拟人群的分组和相互作用,因此能够很好地描述疫情传播过程。涉及流行病爆发阈值的设置以及受感染者

53

随时间的演变特征。复杂网络包含两个元素:节点和边。对于生活物资交付流程,节点表示小区、集散点。边缘表示小区、集散点之间的距离。

复杂网络用于描述由于生物物质输送而导致的新冠肺炎在小区之间传播的动态过程。一般来说,一个小区 \( i \) 包含了一些感染和疑似新冠肺炎病例的人群。考虑到这些 COVID 病例的不同传播能力,我们将其设置为不同的病毒传播风险。因此,邻域可以用初始 COVID 概率 \( r_i \) 描述为等式 (7-10),其中 \( P_c^i \)、\( P_s^i \) 表示确诊和无症状新冠肺炎病例的数量。\( \varepsilon 0 \) 和 \( \varepsilon 1 \) 是这两种情况的相应传输概率设置为 1 和 0.15。通过规范化小区的概率,我们将小区中的 COVID 感染风险 \( \beta_i \) 统一为等式 (7-11),其中 \( r_{min} \) 和 \( r_{max} \) 是城市所有小区的最小和最大风险值。新冠肺炎病例越多,\( \beta_i \) 值越大。

\begin{equation}
r_i = \varepsilon 0 \times P_c^i + \varepsilon 1 \times P_s^i
\tag{7-10}
\end{equation}

\begin{equation}
\beta_i = \frac{r_i - r_{min}}{r_{max} - r_{min}}
\tag{7-11}
\end{equation}

当送货员在城市送货时,由于人与人或人与物之间的必要接触,新冠肺炎可能被传播。一般来说,从集散点到小区的生活物资分布可以用有向图 \( G(V, E) \) 来描述,\( V = \{v_1, v_2, \ldots, v_n\} \) 是所有节点和 \( E = \{e_{12}, e_{23}, \ldots, e_{ij}\} \)(\( i, j \in V \))的是所有边的集合。在本文中,节点表示小区、集散点和物资来源点。边缘表示从一个小区到另一个小区的旅行路线段。

病毒在传播过程中分两种情况。一种情况是,当送货员分发生活物资时,新冠肺炎可能会从小区居民传播给送货员,具体取决于接触者和病毒预防水平。这里,交付的病毒传播风险由 \( \theta_i \) 表示交付中的接触方式。\( \theta_i \) 的范围在 0 和 1 之间。从形式上讲,一次发放物资行为的传递概率可以计算为 (7-12),其中 \( d_i \) 是小区 \( i \) 的保护水平,\( \Delta t \) 是送货员在小区 \( i \) 停留的时间,由装卸时间决定。装卸时间越长,生活物资、送货员和居民之间的接触越多,病毒传播概率 \( p_i \) 越高。在本文中,我们将 \( \theta_i \) 的值设为 1。

\begin{equation}
p_i = (1 - \exp(-d_i \times \Delta t)) \times \theta_i
\tag{7-12}
\end{equation}

另一种情况是病毒可能从送货员传播给居民。一般来说,送货员服务的小区越多,病毒传播风险就越高。因此,病毒传播的风险应该沿着送货员的路线累积。在一次传播后,邻居的感染风险 \( \beta_i^+ \) 等于感染前的风险 \( \beta_i^- \),以及该交付的附加风险 \( \gamma_{hi} \times p_i \),如式 (7-13) 所示。对于送货员来说,一次送货行动后的病毒传播风险 \( \gamma_{ij} \) 等于之前的风险加上访问节点的风险,如式 (7-14) 所示,其中 \( \gamma_{hi} \) 和 \( \gamma_{ij} \) 分别是边缘 \( e_{hi} \) 和 \( e_{ij} \) 的风险,\( h \) 是之前访问的节点。配送网络的风险最终由指标 \( \delta \) 表示,如式 (7-15) 所示,等于所有小区的累积风险和配送后送货员的累积风险。所有交付完成的时间用 \( T \) 表示。

\begin{equation}
\beta_i^+ = \beta_i^- + \gamma_{hi} \times p_i
\tag{7-13}
\end{equation}

\begin{equation}
\gamma_{ij} = \gamma_{hi} + \beta_i^- \times p_i
\tag{7-14}
\end{equation}

\begin{equation}
\delta = \sum_{i \in V} \beta_i^+(T) + \sum_{i, j \in V, j \neq i} \gamma_{ij}
\tag{7-15}
\end{equation}

\subsection{7.4.2 考虑节省人力,减少人员的直接、间接接触的配送路径优化模型}

我们构建了配送路径优化模型,目标是节省人力,并减少因必要的生活物资输送而导致的病毒传播风险。在这里,我们假设仓库将有足够的物流车辆。小区的生活用品订单被分配给最近的集散点。一辆车从物资来源地出发,在集散点领取生活用品,将其送到相应的物资投放点,然后返回出发的物资来源地。汽车在城市里四处奔波,运送生活用品包,直到物资投放点的所有订单都送达为止。模型定义如下:

\begin{tabular}{c c} 
\hline 
$D$ & 物资来源点集合 \\ 
$H=H_{1}+H_{2}$ & 车辆集合 \\ 
$M$ & 集散点集合 \\ 
$C$ & 小区集合 \\ 
\hline 
$d$ & 物资来源点 \\ 
\hline 
$H_{1}$ & 负载能力为4的车辆 \\ 
$H_{2}$ & 负载能力为10的车辆 \\ 
\hline 
$m$ & 集散点 \\ 
\hline 
\end{tabular}

假设所有车辆离开各自的物资来源点并最终返回物资来源点车辆负载不得超过其负载能力$H_{1}, H_{2}$, 送货员的工作时间不得超过规定的工作时间$MT$。

配送路径优化模型可以表示为有向图$G(V, E)$。图$V$的节点$=\{D, M, C\}$是物资来源点、集散点和小区的交集。边缘$E$表示从一个节点移动到另一个。每条边$e_{ij}$表示从节点$i$到节点$j$的一个路段,并带有行驶距离$d_{ij}$。

新冠病毒传播因素涉及以下两个部分:新冠肺炎传播风险和旅行距离。在模型中总传播风险包括所有小区的风险和所有送货员的风险。一般来说,从一家集散点直接运送到一个小区将产生最小的病毒传播风险,因为不存在从小区到小区的间接病毒传播。然而,这将需要更多数量的货车和送货员,加大了送货员接触新冠肺炎。因此,为时目标协调使用加权参数$\alpha$,考虑新冠病毒传播的配送路径优化模型转换为单目标问题,并构建了混合整数规划模型。模型如下所示:

\begin{tabular}{c c} 
\hline 
决策变量 & \\ 
$x_{ij}^{dk} \in (0,1)$ & 二进制变量,如果从物资来源点$d$出发的车辆$k$从节点$i$行驶到节点$j$,则等于1 \\ 
\hline 
$T_{i}^{dk}$ & $k$车在节点$i$离开物资来源点$d$的到达时间 \\ 
\hline 
$Q_{i}^{dk}$ & 物资来源点$d$离开的$k$车离开节点$i$时的载荷 \\ 
\hline 
$\beta_{i}^{+}(T), \beta_{i}^{-}(T)$ & $T$后和$T$前节点$i$的病毒传播风险 \\ 
\hline 
$\gamma_{ij}^{dk}$ & $k$车从$d$站从节点$i$行驶到节点$j$的病毒传播风险。 \\ 
\hline 
\end{tabular}

目标函数:
\begin{equation}
\min F = \alpha \times \left( \sum_{i \in V} \beta^{+}(T_{i}) \right. \tag{7-16}
\end{equation}
\begin{equation}
\left. + \sum_{d \in D} \sum_{k \in H_{d}} \sum_{i \in V} \sum_{j \in D} \gamma_{ij}^{dk} \right) + (1-\alpha) \times \sum_{d \in D} \sum_{k \in H_{d}} \sum_{i \in V} \sum_{j \in V, i \neq j} d_{ij} \times x_{ij}^{dk}
\end{equation}

约束条件:
\begin{equation}
\beta_{i}^{+}(T_{i}^{dk}) = \beta_{i}^{-}(T_{i}^{dk}) + x_{hi}^{dk} \times \gamma_{ij}^{dk} \times p_{i}, \forall d \in D, k \in H_{d} \tag{7-17}
\end{equation}
\begin{equation}
\gamma_{ij}^{dk} = \gamma_{hi}^{dk} \times x_{hi}^{dk} + \beta_{i}^{+}(T_{i}^{dk}) \times p_{i} \times x_{hi}^{dk}, \forall i, j \in V, d \in D, k \in H_{d} \tag{7-18}
\end{equation}

\begin{align}
\sum_{d \in D} \sum_{k \in H_d} \sum_{j \in V} x_{ji}^{dk} &= \sum_{d \in D} \sum_{k \in H_d} \sum_{j \in V} x_{ij}^{dk}, \forall i \in V \tag{7-19} \\
\sum_{i \in D} \sum_{j \in M} x_{ij}^{dk} &< |H_d|, \forall d \in D, k \in H_d \tag{7-20} \\
\sum_{j \in V} x_{dj}^{dk} &= \sum_{j \in V} x_{id}^{dk} = 1, \forall d \in D, k \in H_d \tag{7-21} \\
Q_i^{dk} &\leq MQ \leftarrow \sum_{j \in V} x_{ij}^{dk} = 1, \forall i \in V, d \in D, k \in H_d \tag{7-22} \\
T_j^{dk} &= T_i^{dk} + t_{ij} + s_i, \leftarrow x_{ij}^{dk} = 1, \forall i, j \in V, d \in D, k \in H_d \tag{7-23} \\
T_i^{dk} &\leq MT, \forall i \in V, d \in D, k \in H_d \tag{7-24} \\
T_i &= MAX\left(T_i^{dk}\right), \forall i \in V \tag{7-25}
\end{align}

式(8)的第一部分表示总病毒传播风险,包括所有小区和所有送货人的累积风险。式(8)的第二部分表示所有配送路线的总长度。$\alpha$ 和 $1-\alpha$ 是两个目标的相应权重。等式(9)和(10)表明,小区风险和送货员风险在一次生活材料运送行动后更新。等式(11)表明,小区 $i$ 的到达车辆数量等于左侧车辆数量。等式(12)要求从一个物资来源点出发的车辆数量小于物资来源点内的车辆数量 $|H_d|$。等式(13)规定,一辆车必须返回出发物资来源点。(14)确保一辆车交付的生活物资总量小于车辆容量 $MQ$。等式(15)规定,车辆到达节点 $j$ 的时间等于前一节点 $i$ 的到达时间加上节点 $i$ 的服务时间和节点 $i$ 到 $j$ 的行程时间 $MT$。等式(17)表示活性物质交付过程在节点 $i$,$T_i^{dk}$ 结束,等于最后一辆交付车辆的到达时间。

\subsection{7.4.3 混合元启发式算法}

当涉及数千个节点时,很难在合理的时间内得到全局最优解。在这里,我们开发了一种结合 ALNS(大规模邻域算法)和 SA(模拟退火算法)的混合元启发式算法来解决这个问题。

在大规模邻域搜索(LNS)中,邻域是由 destroy 和 repair 方法隐式定义的。destroy 方法会破坏当前解的一部分,repair 方法会对被破坏的解进行重建。destroy 方法通常包含随机性的元素,以便在每次调用 destroy 方法时破坏解的不同部分。那么,解 $x$ 的邻域 $N(x)$ 就可以定义为:首先通过利用 destroy 方法破坏解 $x$,然后利用 repair 方法重建解 $\bar{x}$,从而得到的一系列解的集合,图 1 展示了一种邻域生成方法。

ALNS 在 LSN 的基础上,允许在同一个搜索中使用多个 destroy 和 repair 方法来获得当前解的邻域。ALNS 会为每个 destroy 和 repair 方法分配一个权重,通过该权重从而控制每个 destroy 和 repair 方法在搜索期间使用的频率。在搜索的过程中,ALNS 会对各个 destroy 和 repair 方法的权重进行动态调整,以便获得更好的邻域和解。

邻域生成算子(operator)destroy/repair 成对出现,算法以“先删除后插入”的策略进行寻优;在寻优的同时,动态调整算子的被选概率。

\begin{figure}[h]
    \centering
    \includegraphics[width=\textwidth]{image.png}
    \caption{2-opt method (local search algorithm)}
    \label{fig:2-opt}
\end{figure}

在轴辐式网络问题求解中,可行解 $x$ 是枢纽序列,即 $x::=<i_1, i_2, \ldots, i_{HUBS}>$,则辐节点集合为 $\hat{x} = N - x$。下面设计 7 种领域生成算子。

\begin{table}[h]
    \centering
    \caption{邻域生算子}
    \label{tab:operators}
    \begin{tabular}{l l}
        \hline \hline
        Operator & Define \\
        \hline
        $\tilde{x} \leftarrow \text{ExchangeOne}(x)$ & Select one node between $x$ and $\hat{x}$ randomly, exchange them and obtain a new solution $\tilde{x}$ \\
        $\tilde{x} \leftarrow \text{ExchangeTwo}(x)$ & Select two nodes between $x$ and $\hat{x}$ randomly, exchange them and obtain a new solution $\tilde{x}$ \\
        $\tilde{x} \leftarrow \text{ExchangeThree}(x)$ & Select three nodes between $x$ and $\hat{x}$ randomly, exchange them and obtain a new solution $\tilde{x}$ \\
        $\tilde{x} \leftarrow \text{SwapTwo}(x)$ & Select two nodes randomly, swap them and obtain a new solution $\tilde{x}$ \\
        $\tilde{x} \leftarrow \text{New}(x)$ & Select HUBS nodes from $N$ randomly, and obtain a new solution $\tilde{x}$ \\
        $\tilde{x} \leftarrow \text{RandI0ne}(x)$ & Remove a node from $x$ randomly, and insert a new solution $\tilde{x}$ \\
        $\tilde{x} \leftarrow \text{Reverse}(x)$ & Select two nodes from $x$ randomly, reverse them and obtain a new solution $\tilde{x}$ \\
        \hline \hline
    \end{tabular}
\end{table}

算子集合 $\lambda \in O = \{o_1, o_2, \ldots, o_m\}$,算子的初始分数 $I(\lambda) = M$,算子改进当前解对应的分数增量为 $\Delta_\lambda = \{\Delta_1, \Delta_2, \ldots, \Delta_m\}$。定义算子权重 $\omega_\lambda = \frac{I(\lambda)}{S(\lambda)}$,$\lambda \in O$,$I(\lambda)$ 是算子 $\lambda$ 在最近一次迭代后的分数,$S(\lambda)$ 是已经被选中的所有算子当前的总分,$S(\lambda) = \sum_\lambda I(\lambda)$。采用轮盘赌选择策略,动态控制算子被选中的概率,直到满足停止准则,算法终止。算子被选中的概率 $P(\lambda) = \frac{I(\lambda)}{S(\lambda)} \cdot \frac{1}{\sum_{\lambda=1}^m \omega_\lambda}$,其中 $S(\lambda) \neq 0$。ALNS 的伪代码如表 \ref{tab:alns} 所示。

\begin{table}[h]
    \centering
    \caption{ALNS 的伪代码}
    \label{tab:alns}
    \begin{tabular}{l l}
        \hline \hline
        Algorithm & Adaptive Larger Neighborhood Search \\
        \hline
        Input: & 初始可行解 $\Pi$ \\
        Output: & 当前最优解 $\Gamma$ \\
        \hline
        Steps & \\
        Step 1 & 初始化:算子的初始分数 $I(\lambda) = M$,$\Gamma \leftarrow \Pi$,$\omega_\lambda = \frac{1}{m}$;分数增量 $\alpha, \beta$,$\alpha < \beta$ \\
        \hline \hline
    \end{tabular}
\end{table}

\begin{enumerate}
    \item[Step 2] 当满足停止准则,算法终止;否则,转到 Step3.
    \item[Step 3] 基于当前权重 $\omega_{\lambda}$,通过轮盘赌选择策略,依概率选择一个算子 $\lambda$,并计算得解 $\Pi^*$
    \item[Step 4] If $Makespan \ \Pi^* < Makespan \ \Pi$, then update $\Pi \leftarrow \Pi^*$; Set $I(\lambda) \leftarrow I(\lambda) + \alpha$
    \item[Step 5] If $Makespan \ \Pi^* < Makespan \ \Gamma$, then update $\Gamma \leftarrow \Pi^*$; Set $I(\lambda) \leftarrow I(\lambda) + \beta$
    \item[Step 6] 当连续 3 次迭代,$Makespan \ \Gamma$ 未改进,Clear $I(\lambda), S(\lambda)$, Set $\omega_{\lambda} = \frac{1}{m}$. 并返回 Step2.
\end{enumerate}

混合元启发式算法包括两个步骤:(1)构造一个好的初始解;(2)通过 ALNS 和 SA 对当前解决方案进行迭代改进。1)初始解决方案:初始解决方案生成方法迭代地将活性物质订单插入路线。工作流程如算法 1 所示。首先,为 H(1 号线)的所有车辆生成从物资来源点开始的初始空路线。然后,随机选择一个订单 $c_i$。将选择一条交货路线 $r_d$ 来插入此订单。路线 $r_d$ 中的最佳插入位置 $(i, j)$ 是插入成本最小的位置(第 3-9 行)。插入操作将重复,直到插入所有订单。

\textbf{Algorithm 1 Construct the Initial Solution}

\textbf{Input:} H: the vehicle set; C: the neighborhood demand order set

\textbf{Output:} the initial solution S

\begin{enumerate}
    \item $R \equiv \{r_1, r_2, ..., r_d\} \leftarrow$ initialize route for $h_d \in H$
    \item for all $c_i \in C$ do
    \item \quad $f_{min} \leftarrow \infty$
    \item \quad for all $r_d \in R$ do
    \item \quad \quad if $f_{min} >$ insert\_cost$(i, j)$ and $q_d + q_j < Q_d$ then
    \item \quad \quad \quad best\_insert = $(r_d, i, j)$
    \item \quad \quad \quad $f_{min} =$ insert\_cost$(i, j)$
    \item \quad \quad end if
    \item \quad end for
    \item \quad insert\_order$(r_d, i, j)$, $q_d += q_j$
    \item end for
\end{enumerate}

\textbf{Algorithm 2 ALNS}

\textbf{Input:} M Supermarkets set; C Neighborhoods set; Max\_Iter Maximum iterations; S initial solution.

\textbf{Output:} Best solution $S^*$

\begin{enumerate}
    \item Initialize parameters, set $S^* = S$
    \item for $k = 1$ to Max\_Iter do.
    \item \quad Choose a destroying operator $O_i$
    \item \quad Remove Q nodes from S by $O_i$
    \item \quad Reinsert Q nodes by the repair operator, produce a new solution $S'$,
    \item \quad Accept current solution $S'$ by simulated annealing
    \item \quad if If accept $S'$ then
    \item \quad \quad $S = S'$
    \item \quad \quad if $f(S') < f(S^*)$ then
\end{enumerate}

\begin{enumerate}
    \setcounter{enumi}{9}
    \item $S^* = S'$
    \item end if
    \item Update the weight of destroying operators $\pi_i$
    \item end if
    \item end for
\end{enumerate}

\subsection{结果分析}

本文选取所有小区中通过聚类得到的 100 个点作为集散中心,由 7.2 节求解的 9 个物资来源地进行配送服务。算法由 python 编译在 pycharm 平台实现,使用的电脑配置为 Windows 10 64 Bit Intel(R) Core(TM) i7-11510U CPU @ 1.8 GHz with Memory by 16GB。使用混合元启发式算法对考虑病毒传播和车量容量限制的路径优化模型模型进行求解,将最优结果展示如下:

\begin{figure}[h]
    \centering
    \includegraphics[width=\textwidth]{image.png}
    \caption{考虑病毒传播和车量容量限制的路径优化模型目标函数迭代图}
    \label{fig:7-3}
\end{figure}

\begin{figure}[h]
    \centering
    \includegraphics[width=0.8\textwidth]{image.png}
    \caption{考虑病毒传播和车量容量限制的有序网络路径优化结果}
    \label{fig:7-4}
\end{figure}

\begin{table}[h]
    \centering
    \caption{考虑病毒传播和车量容量限制的有序网络路径优化结果}
    \label{tab:7-5}
    \begin{tabular}{c c c c c c}
        \hline
        目标函数 & 感染风险 & 运输距离(km) & 运行时间(s) & \multicolumn{2}{c}{车辆数} \\
        \cline{5-6}
        & & & & $H_1$ & $H_2$ \\
        \hline
        891.05280610 & 362.5 & 552.87431 & 3780 & 8 & 24 \\
        \hline
    \end{tabular}
\end{table}

\begin{table}[h]
    \centering
    \caption{考虑病毒传播和车量容量限制的路径优化结果}
    \label{tab:7-6}
    \begin{tabular}{l l}
        \hline
        v1 & d7-64-38-5-54-2-d7 \\
        v2 & d1-3-20-76-48-90-d1 \\
        v3 & d6-27-31-0-66-46-d6 \\
        v4 & d7-10-86-68-d7 \\
        v5 & d8-37-74-29-51-53-d8 \\
        v6 & d4-1-98-44-16-d4 \\
        v7 & d8-96-35-d8 \\
        v8 & d1-50-36-19-81-43-55-d1 \\
        v9 & d2-25-93-88-d2 \\
        v10 & d5-47-72-62-18-41-d5 \\
        v11 & d5-4-13-42-73-79-d5 \\
        v12 & d2-22-70-6-d2 \\
        v13 & d8-21-61-99-94-30-d8 \\
        v14 & d8-17-45-80-77-12-d8 \\
        v15 & d6-52-9-34-89-d6 \\
        v16 & d2-95-67-24-d2 \\
        v17 & d9-40-11-56-97-d9 \\
        v18 & d4-59-78-28-84-65-d4 \\
        v19 & d8-75-71-92-83-8-d8 \\
        \hline
    \end{tabular}
\end{table}

\section*{7.6. 有序网络结果可视化}

\begin{figure}[h]
    \centering
    \includegraphics[width=\textwidth]{network_visualization.png}
    \caption{有序网络结果可视化}
    \label{fig:network_visualization}
\end{figure}

61

\begin{table}
\centering
\begin{tabular}{|c|c|c|}
\hline
\includegraphics[width=0.3\textwidth]{image1.png} & \includegraphics[width=0.3\textwidth]{image2.png} & \includegraphics[width=0.3\textwidth]{image3.png} \\
\hline
车辆1路径规划图 & 车辆2路径规划图 & 车辆3路径规划图 \\
\hline
\includegraphics[width=0.3\textwidth]{image4.png} & \includegraphics[width=0.3\textwidth]{image5.png} & \includegraphics[width=0.3\textwidth]{image6.png} \\
\hline
车辆4路径规划图 & 车辆5路径规划图 & 车辆6路径规划图 \\
\hline
\includegraphics[width=0.3\textwidth]{image7.png} & \includegraphics[width=0.3\textwidth]{image8.png} & \includegraphics[width=0.3\textwidth]{image9.png} \\
\hline
车辆7路径规划图 & 车辆8路径规划图 & 车辆9路径规划图 \\
\hline
\end{tabular}
\end{table}

\section*{八、模型的改进与推广}

\subsection{8.1. 模型的优点}

本文构建模型提出了一种有效的优化方法,并考虑了配送过程中可能发生的新冠病毒传播,构建了一个给予复杂网络的病毒传播模型,模型加入多种应急物资、转运中心容量限制、转运中心最大服务半径等诸多符合现实情况的条件和约束。在构建模型的过程中充分分析供给需求以及可能产生的相关成本,对于选择的指标通过综合评价给予相应权重,对于实际问题的研究及解决具有十分重要的现实意义。

\subsection{8.2. 模型的缺点}

在本文模型构建过程中,由于无法获取全部真实数据以及当前数据与实际数据可能存在一定误差,建模过程中对相关步骤研究不够严谨,假设过于理想化等原因,本文的研究内容及算法都有待进一步完善。

\section{模型的改进与推广}

由于时间有限,本篇文章的模型的构建还有很多不足之处,但得到的结构还是比较合理的。未来研究可以在此基础上加入对突发重大疫情应急物资需求预测与配送路径规划的研究,当前长春新区创新打造“配送单位—小区—物业公司”的物转模式,建立以小区书记为中枢,小区网格长、小区指挥长、物业、志愿者、消杀队伍共同参与的配送队伍体系,对管控区居民能够做到货品配送到单元门口,电话分批通知下楼取货;对封控区居民能够做到消杀队员着二级防护装备,将相关货品配送到居民家门口,打通服务居民“最后一米”。只有打通应急物资保障各环节壁垒,突发重大疫情下需求点的应急物资才能够及时、合理地得到保障,发挥应急物资的最大价值,切实降低群众生命及财产威胁。

\section*{参考文献}

[1] 澎湃新闻. 【防疫科普】静默管理的含义 [EB/OL]. (2022-08-28). \url{https://www.thepaper.cn/newsDetail_forward_19648281}

[2] 卫生健康委网站. 关于印发新型冠状病毒肺炎防控方案(第九版)的通知 [EB/OL]. (2022-06-28). \url{http://www.gov.cn/xinwen/2022-06/28/content_5698168.htm}

[3] 侯冷晨, 沈兵, 张成刚, 郭小毛, 姜艳, 盛伟琪, 钱明平, 胡龙军, 方秉华. 突发新型冠状病毒肺炎疫情后医院封闭式管理的探索与思考 [J]. 华西医学, 2022, 37(08): 1145-9.

[4] 陈镜羽, 张立. 疫情背景下应急生活保障物资末端物流配送模式研究 [J]. 物流科技, 2020, 43(10): 47-50.

[5] 姜凯凯, 高浥尘, 孙洁. 依托便利店构建生活物资应急配送终端体系——以日本便利店的灾后救援经验为例 [J]. 国际城市规划, 2021, 36(05): 121-8.

[6] 朱爱萍, 周应恒. 我国蔬菜市场需求分析 [J]. 华中农业大学学报(社会科学版), 2001, (03): 26-31.

[7] 王宇歌. Omicron BA.2 几乎完全不同于新冠病毒原始毒株 [EB/OL]. (2022-04-12). \url{https://new.qq.com/rain/a/20220402A08KLT00}

[8] 疫情新闻. 长春新冠肺炎疫情图 [EB/OL]. (2022). \url{http://m.sy72.com/covid19/covid20_202248.html}

[9] 吉林省人民政府新闻办公室. 长春就买菜难问题致歉 东北亚粮油、海吉星两大市场暂时关闭 [EB/OL]. (2022). \url{https://view.inews.qq.com/k/20220329A0BEGC00?web_channel=wap&openApp=false}

[10] 谢聪慧, 吴世新, 张晨, 孙文涛, 何海芳, 裴韬, 罗格平. 基于谱系聚类的全球各国新冠疫情时间序列特征分析 [J]. 地球信息科学学报, 2021, 23(02): 236-45.