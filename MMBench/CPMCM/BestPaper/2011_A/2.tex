\begin{center}
\textbf{全国第八届研究生数学建模竞赛}
\end{center}

\begin{tabular}{l l}
学校 & 复旦大学、上海交通大学、东南大学 \\
\hline
参赛队号 & 99999029 \\
\hline
队员姓名 & \begin{tabular}{c c c}
1. & 余 & 猷 \\
2. & 李平好 & \\
3. & 朱 & 莲 \\
\end{tabular} \\
\hline
\end{tabular}

\begin{center}
参赛密码 \_\_\_\_\_\_\_\_\_\_\_\_\_\_\_\_\_\_\_\_\_\_\_\_\_\_\_\_\_\_\_\_\_\_\_\_\_\_\_\_\_\_\_\_\_\_\_\_\_\_\_\_\_\_\_\_\_\_\_\_\_\_\_\_\_\_\_\_\_\_\_\_\_\_\_\_\_\_\_\_\_\_\_\_\_\_\_\_\_\_\_\_\_\_\_\_\_\_\_\_\_\_\_\_\_\_\_\_\_\_\_\_\_\_\_\_\_\_\_\_\_\_\_\_\_\_\_\_\_\_\_\_\_\_\_\_\_\_\_\_\_\_\_\_\_\_\_\_\_\_\_\_\_\_\_\_\_\_\_\_\_\_\_\_\_\_\_\_\_\_\_\_\_\_\_\_\_\_\_\_\_\_\_\_\_\_\_\_\_\_\_\_\_\_\_\_\_\_\_\_\_\_\_\_\_\_\_\_\_\_\_\_\_\_\_\_\_\_\_\_\_\_\_\_\_\_\_\_\_\_\_\_\_\_\_\_\_\_\_\_\_\_\_\_\_\_\_\_\_\_\_\_\_\_\_\_\_\_\_\_\_\_\_\_\_\_\_\_\_\_\_\_\_\_\_\_\_\_\_\_\_\_\_\_\_\_\_\_\_\_\_\_\_\_\_\_\_\_\_\_\_\_\_\_\_\_\_\_\_\_\_\_\_\_\_\_\_\_\_\_\_\_\_\_\_\_\_\_\_\_\_\_\_\_\_\_\_\_\_\_\_\_\_\_\_\_\_\_\_\_\_\_\_\_\_\_\_\_\_\_\_\_\_\_\_\_\_\_\_\_\_\_\_\_\_\_\_\_\_\_\_\_\_\_\_\_\_\_\_\_\_\_\_\_\_\_\_\_\_\_\_\_\_\_\_\_\_\_\_\_\_\_\_\_\_\_\_\_\_\_\_\_\_\_\_\_\_\_\_\_\_\_\_\_\_\_\_\_\_\_\_\_\_\_\_\_\_\_\_\_\_\_\_\_\_\_\_\_\_\_\_\_\_\_\_\_\_\_\_\_\_\_\_\_\_\_\_\_\_\_\_\_\_\_\_\_\_\_\_\_\_\_\_\_\_\_\_\_\_\_\_\_\_\_\_\_\_\_\_\_\_\_\_\_\_\_\_\_\_\_\_\_\_\_\_\_\_\_\_\_\_\_\_\_\_\_\_\_\_\_\_\_\_\_\_\_\_\_\_\_\_\_\_\_\_\_\_\_\_\_\_\_\_\_\_\_\_\_\_\_\_\_\_\_\_\_\_\_\_\_\_\_\_\_\_\_\_\_\_\_\_\_\_\_\_\_\_\_\_\_\_\_\_\_\_\_\_\_\_\_\_\_\_\_\_\_\_\_\_\_\_\_\_\_\_\_\_\_\_\_\_\_\_\_\_\_\_\_\_\_\_\_\_\_\_\_\_\_\_\_\_\_\_\_\_\_\_\_\_\_\_\_\_\_\_\_\_\_\_\_\_\_\_\_\_\_\_\_\_\_\_\_\_\_\_\_\_\_\_\_\_\_\_\_\_\_\_\_\_\_\_\_\_\_\_\_\_\_\_\_\_\_\_\_\_\_\_\_\_\_\_\_\_\_\_\_\_\_\_\_\_\_\_\_\_\_\_\_\_\_\_\_\_\_\_\_\_\_\_\_\_\_\_\_\_\_\_\_\_\_\_\_\_\_\_\_\_\_\_\_\_\_\_\_\_\_\_\_\_\_\_\_\_\_\_\_\_\_\_\_\_\_\_\_\_\_\_\_\_\_\_\_\_\_\_\_\_\_\_\_\_\_\_\_\_\_\_\_\_\_\_\_\_\_\_\_\_\_\_\_\_\_\_\_\_\_\_\_\_\_\_\_\_\_\_\_\_\_\_\_\_\_\_\_\_\_\_\_\_\_\_\_\_\_\_\_\_\_\_\_\_\_\_\_\_\_\_\_\_\_\_\_\_\_\_\_\_\_\_\_\_\_\_\_\_\_\_\_\_\_\_\_\_\_\_\_\_\_\_\_\_\_\_\_\_\_\_\_\_\_\_\_\_\_\_\_\_\_\_\_\_\_\_\_\_\_\_\_\_\_\_\_\_\_\_\_\_\_\_\_\_\_\_\_\_\_\_\_\_\_\_\_\_\_\_\_\_\_\_\_\_\_\_\_\_\_\_\_\_\_\_\_\_\_\_\_\_\_\_\_\_\_\_\_\_\_\_\_\_\_\_\_\_\_\_\_\_\_\_\_\_\_\_\_\_\_\_\_\_\_\_\_\_\_\_\_\_\_\_\_\_\_\_\_\_\_\_\_\_\_\_\_\_\_\_\_\_\_\_\_\_\_\_\_\_\_\_\_\_\_\_\_\_\_\_\_\_\_\_\_\_\_\_\_\_\_\_\_\_\_\_\_\_\_\_\_\_\_\_\_\_\_\_\_\_\_\_\_\_\_\_\_\_\_\_\_\_\_\_\_\_\_\_\_\_\_\_\_\_\_\_\_\_\_\_\_\_\_\_\_\_\_\_\_\_\_\_\_\_\_\_\_\_\_\_\_\_\_\_\_\_\_\_\_\_\_\_\_\_\_\_\_\_\_\_\_\_\_\_\_\_\_\_\_\_\_\_\_\_\_\_\_\_\_\_\_\_\_\_\_\_\_\_\_\_\_\_\_\_\_\_\_\_\_\_\_\_\_\_\_\_\_\_\_\_\_\_\_\_\_\_\_\_\_\_\_\_\_\_\_\_\_\_\_\_\_\_\_\_\_\_\_\_\_\_\_\_\_\_\_\_\_\_\_\_\_\_\_\_\_\_\_\_\_\_\_\_\_\_\_\_\_\_\_\_\_\_\_\_\_\_\_\_\_\_\_\_\_\_\_\_\_\_\_\_\_\_\_\_\_\_\_\_\_\_\_\_\_\_\_\_\_\_\_\_\_\_\_\_\_\_\_\_\_\_\_\_\_\_\_\_\_\_\_\_\_\_\_\_\_\_\_\_\_\_\_\_\_\_\_\_\_\_\_\_\_\_\_\_\_\_\_\_\_\_\_\_\_\_\_\_\_\_\_\_\_\_\_\_\_\_\_\_\_\_\_\_\_\_\_\_\_\_\_\_\_\_\_\_\_\_\_\_\_\_\_\_\_\_\_\_\_\_\_\_\_\_\_\_\_\_\_\_\_\_\_\_\_\_\_\_\_\_\_\_\_\_\_\_\_\_\_\_\_\_\_\_\_\_\_\_\_\_\_\_\_\_\_\_\_\_\_\_\_\_\_\_\_\_\_\_\_\_\_\_\_\_\_\_\_\_\_\_\_\_\_\_\_\_\_\_\_\_\_\_\_\_\_\_\_\_\_\_\_\_\_\_\_\_\_\_\_\_\_\_\_\_\_\_\_\_\_\_\_\_\_\_\_\_\_\_\_\_\_\_\_\_\_\_\_\_\_\_\_\_\_\_\_\_\_\_\_\_\_\_\_\_\_\_\_\_\_\_\_\_\_\_\_\_\_\_\_\_\_\_\_\_\_\_\_\_\_\_\_\_\_\_\_\_\_\_\_\_\_\_\_\_\_\_\_\_\_\_\_\_\_\_\_\_\_\_\_\_\_\_\_\_\_\_\_\_\_\_\_\_\_\_\_\_\_\_\_\_\_\_\_\_\_\_\_\_\_\_\_\_\_\_\_\_\_\_\_\_\_\_\_\_\_\_\_\_\_\_\_\_\_\_\_\_\_\_\_\_\_\_\_\_\_\_\_\_\_\_\_\_\_\_\_\_\_\_\_\_\_\_\_\_\_\_\_\_\_\_\_\_\_\_\_\_\_\_\_\_\_\_\_\_\_\_\_\_\_\_\_\_\_\_\_\_\_\_\_\_\_\_\_\_\_\_\_\_\_\_\_\_\_\_\_\_\_\_\_\_\_\_\_\_\_\_\_\_\_\_\_\_\_\_\_\_\_\_\_\_\_\_\_\_\_\_\_\_\_\_\_\_\_\_\_\_\_\_\_\_\_\_\_\_\_\_\_\_\_\_\_\_\_\_\_\_\_\_\_\_\_\_\_\_\_\_\_\_\_\_\_\_\_\_\_\_\_\_\_\_\_\_\_\_\_\_\_\_\_\_\_\_\_\_\_\_\_\_\_\_\_\_\_\_\_\_\_\_\_\_\_\_\_\_\_\_\_\_\_\_\_\_\_\_\_\_\_\_\_\_\_\_\_\_\_\_\_\_\_\_\_\_\_\_\_\_\_\_\_\_\_\_\_\_\_\_\_\_\_\_\_\_\_\_\_\_\_\_\_\_\_\_\_\_\_\_\_\_\_\_\_\_\_\_\_\_\_\_\_\_\_\_\_\_\_\_\_\_\_\_\_\_\_\_\_\_\_\_\_\_\_\_\_\_\_\_\_\_\_\_\_\_\_\_\_\_\_\_\_\_\_\_\_\_\_\_\_\_\_\_\_\_\_\_\_\_\_\_\_\_\_\_\_\_\_\_\_\_\_\_\_\_\_\_\_\_\_\_\_\_\_\_\_\_\_\_\_\_\_\_\_\_\_\_\_\_\_\_\_\_\_\_\_\_\_\_\_\_\_\_\_\_\_\_\_\_\_\_\_\_\_\_\_\_\_\_\_\_\_\_\_\_\_\_\_\_\_\_\_\_\_\_\_\_\_\_\_\_\_\_\_\_\_\_\_\_\_\_\_\_\_\_\_\_\_\_\_\_\_\_\_\_\_\_\_\_\_\_\_\_\_\_\_\_\_\_\_\_\_\_\_\_\_\_\_\_\_\_\_\_\_\_\_\_\_\_\_\_\_\_\_\_\_\_\_\_\_\_\_\_\_\_\_\_\_\_\_\_\_\_\_\_\_\_\_\_\_\_\_\_\_\_\_\_\_\_\_\_\_\_\_\_\_\_\_\_\_\_\_\_\_\_\_\_\_\_\_\_\_\_\_\_\_\_\_\_\_\_\_\_\_\_\_\_\_\_\_\_\_\_\_\_\_\_\_\_\_\_\_\_\_\_\_\_\_\_\_\_\_\_\_\_\_\_\_\_\_\_\_\_\_\_\_\_\_\_\_\_\_\_\_\_\_\_\_\_\_\_\_\_\_\_\_\_\_\_\_\_\_\_\_\_\_\_\_\_\_\_\_\_\_\_\_\_\_\_\_\_\_\_\_\_\_\_\_\_\_\_\_\_\_\_\_\_\_\_\_\_\_\_\_\_\_\_\_\_\_\_\_\_\_\_\_\_\_\_\_\_\_\_\_\_\_\_\_\_\_\_\_\_\_\_\_\_\_\_\_\_\_\_\_\_\_\_\_\_\_\_\_\_\_\_\_\_\_\_\_\_\_\_\_\_\_\_\_\_\_\_\_\_\_\_\_\_\_\_\_\_\_\_\_\_\_\_\_\_\_\_\_\_\_\_\_\_\_\_\_\_\_\_\_\_\_\_\_\_\_\_\_\_\_\_\_\_\_\_\_\_\_\_\_\_\_\_\_\_\_\_\_\_\_\_\_\_\_\_\_\_\_\_\_\_\_\_\_\_\_\_\_\_\_\_\_\_\_\_\_\_\_\_\_\_\_\_\_\_\_\_\_\_\_\_\_\_\_\_\_\_\_\_\_\_\_\_\_\_\_\_\_\_\_\_\_\_\_\_\_\_\_\_\_\_\_\_\_\_\_\_\_\_\_\_\_\_\_\_\_\_\_\_\_\_\_\_\_\_\_\_\_\_\_\_\_\_\_\_\_\_\_\_\_\_\_\_\_\_\_\_\_\_\_\_\_\_\_\_\_\_\_\_\_\_\_\_\_\_\_\_\_\_\_\_\_\_\_\_\_\_\_\_\_\_\_\_\_\_\_\_\_\_\_\_\_\_\_\_\_\_\_\_\_\_\_\_\_\_\_\_\_\_\_\_\_\_\_\_\_\_\_\_\_\_\_\_\_\_\_\_\_\_\_\_\_\_\_\_\_\_\_\_\_\_\_\_\_\_\_\_\_\_\_\_\_\_\_\_\_\_\_\_\_\_\_\_\_\_\_\_\_\_\_\_\_\_\_\_\_\_\_\_\_\_\_\_\_\_\_\_\_\_\_\_\_\_\_\_\_\_\_\_\_\_\_\_\_\_\_\_\_\_\_\_\_\_\_\_\_\_\_\_\_\_\_\_\_\_\_\_\_\_\_\_\_\_\_\_\_\_\_\_\_\_\_\_\_\_\_\_\_\_\_\_\_\_\_\_\_\_\_\_\_\_\_\_\_\_\_\_\_\_\_\_\_\_\_\_\_\_\_\_\_\_\_\_\_\_\_\_\_\_\_\_\_\_\_\_\_\_\_\_\_\_\_\_\_\_\_\_\_\_\_\_\_\_\_\_\_\_\_\_\_\_\_\_\_\_\_\_\_\_\_\_\_\_\_\_\_\_\_\_\_\_\_\_\_\_\_\_\_\_\_\_\_\_\_\_\_\_\_\_\_\_\_\_\_\_\_\_\_\_\_\_\_\_\_\_\_\_\_\_\_\_\_\_\_\_\_\_\_\_\_\_\_\_\_\_\_\_\_\_\_\_\_\_\_\_\_\_\_\_\_\_\_\_\_\_\_\_\_\_\_\_\_\_\_\_\_\_\_\_\_\_\_\_\_\_\_\_\_\_\_\_\_\_\_\_\_\_\_\_\_\_\_\_\_\_\_\_\_\_\_\_\_\_\_\_\_\_\_\_\_\_\_\_\_\_\_\_\_\_\_\_\_\_\_\_\_\_\_\_\_\_\_\_\_\_\_\_\_\_\_\_\_\_\_\_\_\_\_\_\_\_\_\_\_\_\_\_\_\_\_\_\_\_\_\_\_\_\_\_\_\_\_\_\_\_\_\_\_\_\_\_\_\_\_\_\_\_\_\_\_\_\_\_\_\_\_\_\_\_\_\_\_\_\_\_\_\_\_\_\_\_\_\_\_\_\_\_\_\_\_\_\_\_\_\_\_\_\_\_\_\_\_\_\_\_\_\_\_\_\_\_\_\_\_\_\_\_\_\_\_\_\_\_\_\_\_\_\_\_\_\_\_\_\_\_\_\_\_\_\_\_\_\_\_\_\_\_\_\_\_\_\_\_\_\_\_\_\_\_\_\_\_\_\_\_\_\_\_\_\_\_\_\_\_\_\_\_\_\_\_\_\_\_\_\_\_\_\_\_\_\_\_\_\_\_\_\_\_\_\_\_\_\_\_\_\_\_\_\_\_\_\_\_\_\_\_\_\_\_\_\_\_\_\_\_\_\_\_\_\_\_\_\_\_\_\_\_\_\_\_\_\_\_\_\_\_\_\_\_\_\_\_\_\_\_\_\_\_\_\_\_\_\_\_\_\_\_\_\_\_\_\_\_\_\_\_\_\_\_\_\_\_\_\_\_\_\_\_\_\_\_\_\_\_\_\_\_\_\_\_\_\_\_\_\_\_\_\_\_\_\_\_\_\_\_\_\_\_\_\_\_\_\_\_\_\_\_\_\_\_\_\_\_\_\_\_\_\_\_\_\_\_\_\_\_\_\_\_\_\_\_\_\_\_\_\_\_\_\_\_\_\_\_\_\_\_\_\_\_\_\_\_\_\_\_\_\_\_\_\_\_\_\_\_\_\_\_\_\_\_\_\_\_\_\_\_\_\_\_\_\_\_\_\_\_\_\_\_\_\_\_\_\_\_\_\_\_\_\_\_\_\_\_\_\_\_\_\_\_\_\_\_\_\_\_\_\_\_\_\_\_\_\_\_\_\_\_\_\_\_\_\_\_\_\_\_\_\_\_\_\_\_\_\_\_\_\_\_\_\_\_\_\_\_\_\_\_\_\_\_\_\_\_\_\_\_\_\_\_\_\_\_\_\_\_\_\_\_\_\_\_\_\_\_\_\_\_\_\_\_\_\_\_\_\_\_\_\_\_\_\_\_\_\_\_\_\_\_\_\_\_\_\_\_\_\_\_\_\_\_\_\_\_\_\_\_\_\_\_\_\_\_\_\_\_\_\_\_\_\_\_\_\_\_\_\_\_\_\_\_\_\_\_\_\_\_\_\_\_\_\_\_\_\_\_\_\_\_\_\_\_\_\_\_\_\_\_\_\_\_\_\_\_\_\_\_\_\_\_\_\_\_\_\_\_\_\_\_\_\_\_\_\_\_\_\_\_\_\_\_\_\_\_\_\_\_\_\_\_\_\_\_\_\_\_\_\_\_\_\_\_\_\_\_\_\_\_\_\_\_\_\_\_\_\_\_\_\_\_\_\_\_\_\_\_\_\_\_\_\_\_\_\_\_\_\_\_\_\_\_\_\_\_\_\_\_\_\_\_\_\_\_\_\_\_\_\_\_\_\_\_\_\_\_\_\_\_\_\_\_\_\_\_\_\_\_\_\_\_\_\_\_\_\_\_\_\_\_\_\_\_\_\_\_\_\_\_\_\_\_\_\_\_\_\_\_\_\_\_\_\_\_\_\_\_\_\_\_\_\_\_\_\_\_\_\_\_\_\_\_\_\_\_\_\_\_\_\_\_\_\_\_\_\_\_\_\_\_\_\_\_\_\_\_\_\_\_\_\_\_\_\_\_\_\_\_\_\_\_\_\_\_\_\_\_\_\_\_\_\_\_\_\_\_\_\_\_\_\_\_\_\_\_\_\_\_\_\_\_\_\_\_\_\_\_\_\_\_\_\_\_\_\_\_\_\_\_\_\_\_\_\_\_\_\_\_\_\_\_\_\_\_\_\_\_\_\_\_\_\_\_\_\_\_\_\_\_\_\_\_\_\_\_\_\_\_\_\_\_\_\_\_\_\_\_\_\_\_\_\_\_\_\_\_\_\_\_\_\_\_\_\_\_\_\_\_\_\_\_\_\_\_\_\_\_\_\_\_\_\_\_\_\_\_\_\_\_\_\_\_\_\_\_\_\_\_\_\_\_\_\_\_\_\_\_\_\_\_\_\_\_\_\_\_\_\_\_\_\_\_\_\_\_\_\_\_\_\_\_\_\_\_\_\_\_\_\_\_\_\_\_\_\_\_\_\_\_\_\_\_\_\_\_\_\_\_\_\_\_\_\_\_\_\_\_\_\_\_\_\_\_\_\_\_\_\_\_\_\_\_\_\_\_\_\_\_\_\_\_\_\_\_\_\_\_\_\_\_\_\_\_\_\_\_\_\_\_\_\_\_\_\_\_\_\_\_\_\_\_\_\_\_\_\_\_\_\_\_\_\_\_\_\_\_\_\_\_\_\_\_\_\_\_\_\_\_\_\_\_\_\_\_\_\_\_\_\_\_\_\_\_\_\_\_\_\_\_\_\_\_\_\_\_\_\_\_\_\_\_\_\_\_\_\_\_\_\_\_\_\_\_\_\_\_\_\_\_\_\_\_\_\_\_\_\_\_\_\_\_\_\_\_\_\_\_\_\_\_\_\_\_\_\_\_\_\_\_\_\_\_\_\_\_\_\_\_\_\_\_\_\_\_\_\_\_\_\_\_\_\_\_\_\_\_\_\_\_\_\_\_\_\_\_\_\_\_\_\_\_\_\_\_\_\_\_\_\_\_\_\_\_\_\_\_\_\_\_\_\_\_\_\_\_\_\_\_\_\_\_\_\_\_\_\_\_\_\_\_\_\_\_\_\_\_\_\_\_\_\_\_\_\_\_\_\_\_\_\_\_\_\_\_\_\_\_\_\_\_\_\_\_\_\_\_\_\_\_\_\_\_\_\_\_\_\_\_\_\_\_\_\_\_\_\_\_\_\_\_\_\_\_\_\_\_\_\_\_\_\_\_\_\_\_\_\_\_\_\_\_\_\_\_\_\_\_\_\_\_\_\_\_\_\_\_\_\_\_\_\_\_\_\_\_\_\_\_\_\_\_\_\_\_\_\_\_\_\_\_\_\_\_\_\_\_\_\_\_\_\_\_\_\_\_\_\_\_\_\_\_\_\_\_\_\_\_\_\_\_\_\_\_\_\_\_\_\_\_\_\_\_\_\_\_\_\_\_\_\_\_\_\_\_\_\_\_\_\_\_\_\_\_\_\_\_\_\_\_\_\_\_\_\_\_\_\_\_\_\_\_\_\_\_\_\_\_\_\_\_\_\_\_\_\_\_\_\_\_\_\_\_\_\_\_\_\_\_\_\_\_\_\_\_\_\_\_\_\_\_\_\_\_\_\_\_\_\_\_\_\_\_\_\_\_\_\_\_\_\_\_\_\_\_\_\_\_\_\_\_\_\_\_\_\_\_\_\_\_\_\_\_\_\_\_\_\_\_\_\_\_\_\_\_\_\_\_\_\_\_\_\_\_\_\_\_\_\_\_\_\_\_\_\_\_\_\_\_\_\_\_\_\_\_\_\_\_\_\_\_\_\_\_\_\_\_\_\_\_\_\_\_\_\_\_\_\_\_\_\_\_\_\_\_\_\_\_\_\_\_\_\_\_\_\_\_\_\_\_\_\_\_\_\_\_\_\_\_\_\_\_\_\_\_\_\_\_\_\_\_\_\_\_\_\_\_\_\_\_\_\_\_\_\_\_\_\_\_\_\_\_\_\_\_\_\_\_\_\_\_\_\_\_\_\_\_\_\_\_\_\_\_\_\_\_\_\_\_\_\_\_\_\_\_\_\_\_\_\_\_\_\_\_\_\_\_\_\_\_\_\_\_\_\_\_\_\_\_\_\_\_\_\_\_\_\_\_\_\_\_\_\_\_\_\_\_\_\_\_\_\_\_\_\_\_\_\_\_\_\_\_\_\_\_\_\_\_\_\_\_\_\_\_\_\_\_\_\_\_\_\_\_\_\_\_\_\_\_\_\_\_\_\_\_\_\_\_\_\_\_\_\_\_\_\_\_\_\_\_\_\_\_\_\_\_\_\_\_\_\_\_\_\_\_\_\_\_\_\_\_\_\_\_\_\_\_\_\_\_\_\_\_\_\_\_\_\_\_\_\_\_\_\_\_\_\_\_\_\_\_\_\_\_\_\_\_\_\_\_\_\_\_\_\_\_\_\_\_\_\_\_\_\_\_\_\_\_\_\_\_\_\_\_\_\_\_\_\_\_\_\_\_\_\_\_\_\_\_\_\_\_\_\_\_\_\_\_\_\_\_\_\_\_\_\_\_\_\_\_\_\_\_\_\_\_\_\_\_\_\_\_\_\_\_\_\_\_\_\_\_\_\_\_\_\_\_\_\_\_\_\_\_\_\_\_\_\_\_\_\_\_\_\_\_\_\_\_\_\_\_\_\_\_\_\_\_\_\_\_\_\_\_\_\_\_\_\_\_\_\_\_\_\_\_\_\_\_\_\_\_\_\_\_\_\_\_\_\_\_\_\_\_\_\_\_\_\_\_\_\_\_\_\_\_\_\_\_\_\_\_\_\_\_\_\_\_\_\_\_\_\_\_\_\_\_\_\_\_\_\_\_\_\_\_\_\_\_\_\_\_\_\_\_\_\_\_\_\_\_\_\_\_\_\_\_\_\_\_\_\_\_\_\_\_\_\_\_\_\_\_\_\_\_\_\_\_\_\_\_\_\_\_\_\_\_\_\_\_\_\_\_\_\_\_\_\_\_\_\_\_\_\_\_\_\_\_\_\_\_\_\_\_\_\_\_\_\_\_\_\_\_\_\_\_\_\_\_\_\_\_\_\_\_\_\_\_\_\_\_\_\_\_\_\_\_\_\_\_\_\_\_\_\_\_\_\_\_\_\_\_\_\_\_\_\_\_\_\_\_\_\_\_\_\_\_\_\_\_\_\_\_\_\_\_\_\_\_\_\_\_\_\_\_\_\_\_\_\_\_\_\_\_\_\_\_\_\_\_\_\_\_\_\_\_\_\_\_\_\_\_\_\_\_\_\_\_\_\_\_\_\_\_\_\_\_\_\_\_\_\_\_\_\_\_\_\_\_\_\_\_\_\_\_\_\_\_\_\_\_\_\_\_\_\_\_\_\_\_\_\_\_\_\_\_\_\_\_\_\_\_\_\_\_\_\_\_\_\_\_\_\_\_\_\_\_\_\_\_\_\_\_\_\_\_\_\_\_\_\_\_\_\_\_\_\_\_\_\_\_\_\_\_\_\_\_\_\_\_\_\_\_\_\_\_\_\_\_\_\_\_\_\_\_\_\_\_\_\_\_\_\_\_\_\_\_\_\_\_\_\_\_\_\_\_\_\_\_\_\_\_\_\_\_\_\_\_\_\_\_\_\_\_\_\_\_\_\_\_\_\_\_\_\_\_\_\_\_\_\_\_\_\_\_\_\_\_\_\_\_\_\_\_\_\_\_\_\_\_\_\_\_\_\_\_\_\_\_\_\_\_\_\_\_\_\_\_\_\_\_\_\_\_\_\_\_\_\_\_\_\_\_\_\_\_\_\_\_\_\_\_\_\_\_\_\_\_\_\_\_\_\_\_\_\_\_\_\_\_\_\_\_\_\_\_\_\_\_\_\_\_\_\_\_\_\_\_\_\_\_\_\_\_\_\_\_\_\_\_\_\_\_\_\_\_\_\_\_\_\_\_\_\_\_\_\_\_\_\_\_\_\_\_\_\_\_\_\_\_\_\_\_\_\_\_\_\_\_\_\_\_\_\_\_\_\_\_\_\_\_\_\_\_\_\_\_\_\_\_\_\_\_\_\_\_\_\_\_\_\_\_\_\_\_\_\_\_\_\_\_\_\_\_\_\_\_\_\_\_\_\_\_\_\_\_\_\_\_\_\_\_\_\_\_\_\_\_\_\_\_\_\_\_\_\_\_\_\_\_\_\_\_\_\_\_\_\_\_\_\_\_\_\_\_\_\_\_\_\_\_\_\_\_\_\_\_\_\_\_\_\_\_\_\_\_\_\_\_\_\_\_\_\_\_\_\_\_\_\_\_\_\_\_\_\_\_\_\_\_\_\_\_\_\_\_\_\_\_\_\_\_\_\_\_\_\_\_\_\_\_\_\_\_\_\_\_\_\_\_\_\_\_\_\_\_\_\_\_\_\_\_\_\_\_\_\_\_\_\_\_\_\_\_\_\_\_\_\_\_\_\_\_\_\_\_\_\_\_\_\_\_\_\_\_\_\_\_\_\_\_\_\_\_\_\_\_\_\_\_\_\_\_\_\_\_\_\_\_\_\_\_\_\_\_\_\_\_\_\_\_\_\_\_\_\_\_\_\_\_\_\_\_\_\_\_\_\_\_\_\_\_\_\_\_\_\_\_\_\_\_\_\_\_\_\_\_\_\_\_\_\_\_\_\_\_\_\_\_\_\_\_\_\_\_\_\_\_\_\_\_\_\_\_\_\_\_\_\_\_\_\_\_\_\_\_\_\_\_\_\_\_\_\_\_\_\_\_\_\_\_\_\_\_\_\_\_\_\_\_\_\_\_\_\_\_\_\_\_\_\_\_\_\_\_\_\_\_\_\_\_\_\_\_\_\_\_\_\_\_\_\_\_\_\_\_\_\_\_\_\_\_\_\_\_\_\_\_\_\_\_\_\_\_\_\_\_\_\_\_\_\_\_\_\_\_\_\_\_\_\_\_\_\_\_\_\_\_\_\_\_\_\_\_\_\_\_\_\_\_\_\_\_\_\_\_\_\_\_\_\_\_\_\_\_\_\_\_\_\_\_\_\_\_\_\_\_\_\_\_\_\_\_\_\_\_\_\_\_\_\_\_\_\_\_\_\_\_\_\_\_\_\_\_\_\_\_\_\_\_\_\_\_\_\_\_\_\_\_\_\_\_\_\_\_\_\_\_\_\_\_\_\_\_\_\_\_\_\_\_\_\_\_\_\_\_\_\_\_\_\_\_\_\_\_\_\_\_\_\_\_\_\_\_\_\_\_\_\_\_\_\_\_\_\_\_\_\_\_\_\_\_\_\_\_\_\_\_\_\_\_\_\_\_\_\_\_\_\_\_\_\_\_\_\_\_\_\_\_\_\_\_\_\_\_\_\_\_\_\_\_\_\_\_\_\_\_\_\_\_\_\_\_\_\_\_\_\_\_\_\_\_\_\_\_\_\_\_\_\_\_\_\_\_\_\_\_\_\_\_\_\_\_\_\_\_\_\_\_\_\_\_\_\_\_\_\_\_\_\_\_\_\_\_\_\_\_\_\_\_\_\_\_\_\_\_\_\_\_\_\_\_\_\_\_\_\_\_\_\_\_\_\_\_\_\_\_\_\_\_\_\_\_\_\_\_\_\_\_\_\_\_\_\_\_\_\_\_\_\_\_\_\_\_\_\_\_\_\_\_\_\_\_\_\_\_\_\_\_\_\_\_\_\_\_\_\_\_\_\_\_\_\_\_\_\_\_\_\_\_\_\_\_\_\_\_\_\_\_\_\_\_\_\_\_\_\_\_\_\_\_\_\_\_\_\_\_\_\_\_\_\_\_\_\_\_\_\_\_\_\_\_\_\_\_\_\_\_\_\_\_\_\_\_\_\_\_\_\_\_\_\_\_\_\_\_\_\_\_\_\_\_\_\_\_\_\_\_\_\_\_\_\_\_\_\_\_\_\_\_\_\_\_\_\_\_\_\_\_\_\_\_\_\_\_\_\_\_\_\_\_\_\_\_\_\_\_\_\_\_\_\_\_\_\_\_\_\_\_\_\_\_\_\_\_\_\_\_\_\_\_\_\_\_\_\_\_\_\_\_\_\_\_\_\_\_\_\_\_\_\_\_\_\_\_\_\_\_\_\_\_\_\_\_\_\_\_\_\_\_\_\_\_\_\_\_\_\_\_\_\_\_\_\_\_\_\_\_\_\_\_\_\_\_\_\_\_\_\_\_\_\_\_\_\_\_\_\_\_\_\_\_\_\_\_\_\_\_\_\_\_\_\_\_\_\_\_\_\_\_\_\_\_\_\_\_\_\_\_\_\_\_\_\_\_\_\_\_\_\_\_\_\_\_\_\_\_\_\_\_\_\_\_\_\_\_\_\_\_\_\_\_\_\_\_\_\_\_\_\_\_\_\_\_\_\_\_\_\_\_\_\_\_\_\_\_\_\_\_\_\_\_\_\_\_\_\_\_\_\_\_\_\_\_\_\_\_\_\_\_\_\_\_\_\_\_\_\_\_\_\_\_\_\_\_\_\_\_\_\_\_\_\_\_\_\_\_\_\_\_\_\_\_\_\_\_\_\_\_\_\_\_\_\_\_\_\_\_\_\_\_\_\_\_\_\_\_\_\_\_\_\_\_\_\_\_\_\_\_\_\_\_\_\_\_\_\_\_\_\_\_\_\_\_\_\_\_\_\_\_\_\_\_\_\_\_\_\_\_\_\_\_\_\_\_\_\_\_\_\_\_\_\_\_\_\_\_\_\_\_\_\_\_\_\_\_\_\_\_\_\_\_\_\_\_\_\_\_\_\_\_\_\_\_\_\_\_\_\_\_\_\_\_\_\_\_\_\_\_\_\_\_\_\_\_\_\_\_\_\_\_\_\_\_\_\_\_\_\_\_\_\_\_\_\_\_\_\_\_\_\_\_\_\_\_\_\_\_\_\_\_\_\_\_\_\_\_\_\_\_\_\_\_\_\_\_\_\_\_\_\_\_\_\_\_\_\_\_\_\_\_\_\_\_\_\_\_\_\_\_\_\_\_\_\_\_\_\_\_\_\_\_\_\_\_\_\_\_\_\_\_\_\_\_\_\_\_\_\_\_\_\_\_\_\_\_\_\_\_\_\_\_\_\_\_\_\_\_\_\_\_\_\_\_\_\_\_\_\_\_\_\_\_\_\_\_\_\_\_\_\_\_\_\_\_\_\_\_\_\_\_\_\_\_\_\_\_\_\_\_\_\_\_\_\_\_\_\_\_\_\_\_\_\_\_\_\_\_\_\_\_\_\_\_\_\_\_\_\_\_\_\_\_\_\_\_\_\_\_\_\_\_\_\_\_\_\_\_\_\_\_\_\_\_\_\_\_\_\_\_\_\_\_\_\_\_\_\_\_\_\_\_\_\_\_\_\_\_\_\_\_\_\_\_\_\_\_\_\_\_\_\_\_\_\_\_\_\_\_\_\_\_\_\_\_\_\_\_\_\_\_\_\_\_\_\_\_\_\_\_\_\_\_\_\_\_\_\_\_\_\_\_\_\_\_\_\_\_\_\_\_\_\_\_\_\_\_\_\_\_\_\_\_\_\_\_\_\_\_\_\_\_\_\_\_\_\_\_\_\_\_\_\_\_\_\_\_\_\_\_\_\_\_\_\_\_\_\_\_\_\_\_\_\_\_\_\_\_\_\_\_\_\_\_\_\_\_\_\_\_\_\_\_\_\_\_\_\_\_\_\_\_\_\_\_\_\_\_\_\_\_\_\_\_\_\_\_\_\_\_\_\_\_\_\_\_\_\_\_\_\_\_\_\_\_\_\_\_\_\_\_\_\_\_\_\_\_\_\_\_\_\_\_\_\_\_\_\_\_\_\_\_\_\_\_\_\_\_\_\_\_\_\_\_\_\_\_\_\_\_\_\_\_\_\_\_\_\_\_\_\_\_\_\_\_\_\_\_\_\_\_\_\_\_\_\_\_\_\_\_\_\_\_\_\_\_\_\_\_\_\_\_\_\_\_\_\_\_\_\_\_\_\_\_\_\_\_\_\_\_\_\_\_\_\_\_\_\_\_\_\_\_\_\_\_\_\_\_\_\_\_\_\_\_\_\_\_\_\_\_\_\_\_\_\_\_\_\_\_\_\_\_\_\_\_\_\_\_\_\_\_\_\_\_\_\_\_\_\_\_\_\_\_\_\_\_\_\_\_\_\_\_\_\_\_\_\_\_\_\_\_\_\_\_\_\_\_\_\_\_\_\_\_\_\_\_\_\_\_\_\_\_\_\_\_\_\_\_\_\_\_\_\_\_\_\_\_\_\_\_\_\_\_\_\_\_\_\_\_\_\_\_\_\_\_\_\_\_\_\_\_\_\_\_\_\_\_\_\_\_\_\_\_\_\_\_\_\_\_\_\_\_\_\_\_\_\_\_\_\_\_\_\_\_\_\_\_\_\_\_\_\_\_\_\_\_\_\_\_\_\_\_\_\_\_\_\_\_\_\_\_\_\_\_\_\_\_\_\_\_\_\_\_\_\_\_\_\_\_\_\_\_\_\_\_\_\_\_\_\_\_\_\_\_\_\_\_\_\_\_\_\_\_\_\_\_\_\_\_\_\_\_\_\_\_\_\_\_\_\_\_\_\_\_\_\_\_\_\_\_\_\_\_\_\_\_\_\_\_\_\_\_\_\_\_\_\_\_\_\_\_\_\_\_\_\_\_\_\_\_\_\_\_\_\_\_\_\_\_\_\_\_\_\_\_\_\_\_\_\_\_\_\_\_\_\_\_\_\_\_\_\_\_\_\_\_\_\_\_\_\_\_\_\_\_\_\_\_\_\_\_\_\_\_\_\_\_\_\_\_\_\_\_\_\_\_\_\_\_\_\_\_\_\_\_\_\_\_\_\_\_\_\_\_\_\_\_\_\_\_\_\_\_\_\_\_\_\_\_\_\_\_\_\_\_\_\_\_\_\_\_\_\_\_\_\_\_\_\_\_\_\_\_\_\_\_\_\_\_\_\_\_\_\_\_\_\_\_\_\_\_\_\_\_\_\_\_\_\_\_\_\_\_\_\_\_\_\_\_\_\_\_\_\_\_\_\_\_\_\_\_\_\_\_\_\_\_\_\_\_\_\_\_\_\_\_\_\_\_\_\_\_\_\_\_\_\_\_\_\_\_\_\_\_\_\_\_\_\_\_\_\_\_\_\_\_\_\_\_\_\_\_\_\_\_\_\_\_\_\_\_\_\_\_\_\_\_\_\_\_\_\_\_\_\_\_\_\_\_\_\_\_\_\_\_\_\_\_\_\_\_\_\_\_\_\_\_\_\_\_\_\_\_\_\_\_\_\_\_\_\_\_\_\_\_\_\_\_\_\_\_\_\_\_\_\_\_\_\_\_\_\_\_\_\_\_\_\_\_\_\_\_\_\_\_\_\_\_\_\_\_\_\_\_\_\_\_\_\_\_\_\_\_\_\_\_\_\_\_\_\_\_\_\_\_\_\_\_\

\section*{全国第八届研究生数学建模竞赛}

\begin{figure}[h]
    \centering
    \includegraphics[width=0.5\textwidth]{logo.png}
    \caption{全国研究生数学建模竞赛标志}
\end{figure}

\section*{题目 基于光的波粒二象性一种猜想的数学仿真}

\section*{摘要:}

本文紧密围绕题干提出的三个问题和两大关系,从现有的经典物理光学理论和专业实验结果出发,运用数学思维,综合光子理论,成功建立了基于光的波粒二象性一种猜想的数学模型。

针对龚祖同院士的光微子碰撞猜想,我们建立了基于光子碰撞后概率分布的模型
\[
\frac{I(p)}{I_0} = \frac{1}{\sqrt{2\pi\sigma}} \exp\left(-\frac{p^2}{2\sigma^2}\right) \cdot \left(\frac{\sin(\varepsilon p)}{\varepsilon p}\right)^2 + \frac{\sin \varepsilon p}{\mu \varepsilon p} \cdot \frac{\cos \pi p}{1 - 4p^2} \cdot \left(1 - \frac{\sin(\varepsilon p)}{\varepsilon p}\right),
\]
在最终模型的参数确定中,采用了最小二乘估计 RSS 准则,使得最终模型在满足与实验数据相吻合的前提下充分逼近玻恩理论公式(逼近误差以残差平方和计算,仅占理论结果的 0.54\%)。另外,模型的适应性很强,在模型提供的衍射公式的基础上稍作延伸便能得到与玻恩公式相匹配的干涉多缝衍射公式,有力地证明了龚院士的猜想可靠性。

针对北大学者提出的光子作为电磁场自我旋转的猜想,我们采用双管齐下的办法,分别从专业证明和数学模型分析方面建立了电磁场偏转模型和光子旋转模型,一方面利用基尔霍夫公式证明了北大学者提出的关于光子电磁场的诸多假设都是有意义的,另一方面也发现在他提出的光子自旋猜想下,双缝干涉的明纹暗纹区域满足条件 $r_1 + r_2 = (2k+1)\lambda/2$ 与 $r_1 + r_2 = k\lambda$,而不是经典理论中的 $r_1 - r_2 = k\lambda$ 与 $r_1 - r_2 = (2k+1)\lambda/2$,而且它的衍射光强分布也与玻恩公式有些许差别,可见,该猜想有待于提升和完善。

最后,经过定性、定量分析两大猜想各自对于大量物理实验的优缺性,我们建立了自己的猜想模型——光子蜂窝网络模型。该模型引入了“光子域”、“光子电力”、“光子磁力”、“光子键”等概念,从五个子模型出发,定性解释了四个光学现象,合理回答了题目提出的三大问题,并定量证明了衍射光强分布,使模型具有强大的说服力,可信度高。

\section*{关键词 基于光子碰撞后概率分布的模型 最小二乘估计 RSS 准则 电磁场偏转模型 电磁场偏转模型 基尔霍夫公式 光子蜂窝网络模型}

\section*{目录}

\section*{一、问题重述}
\dotfill 3

\section*{二、模型假设}
\dotfill 3

\section*{三、符号说明}
\dotfill 4

\section*{四、模型的分析、建立与求解}
\dotfill 4

\subsection*{4.1 模型一的分析、建立与求解}
\dotfill 4

\subsubsection*{4.1.0 模型假设}
\dotfill 4

\subsubsection*{4.1.1 符号说明}
\dotfill 4

\subsubsection*{4.1.2 模型分析}
\dotfill 4

\subsubsection*{4.1.3 模型建立}
\dotfill 5

\subsubsection*{4.1.4 参数确定}
\dotfill 7

\subsubsection*{4.1.5 模型一的检验}
\dotfill 8

\subsubsection*{4.1.6 模型一的适应度分析}
\dotfill 8

\subsection*{4.2 模型二——电磁场偏转模型(基尔霍夫衍射积分)的分析与证明}
\dotfill 10

\subsubsection*{4.2.0 模型假设}
\dotfill 10

\subsubsection*{4.2.1 模型分析}
\dotfill 10

\subsubsection*{4.2.2 公式推导}
\dotfill 10

\subsection*{4.3 模型三——光子旋转模型的的分析、建立与求解}
\dotfill 13

\subsubsection*{4.3.0 模型假设}
\dotfill 13

\subsubsection*{4.3.1 符号说明}
\dotfill 13

\subsubsection*{4.3.2 模型分析}
\dotfill 13

\subsubsection*{4.3.3 模型建立}
\dotfill 14

\subsubsection*{4.3.4 模型求解与分析}
\dotfill 16

\subsection*{4.4 模型优化之光子蜂窝网络模型}
\dotfill 18

\subsubsection*{4.4.0 模型假设}
\dotfill 18

\subsubsection*{4.4.1 模型的建立}
\dotfill 19

\subsubsection*{4.4.2 模型对于“三个问题”的解释}
\dotfill 22

\subsubsection*{4.4.3 模型定量分析}
\dotfill 25

\section*{五、模型的评价及推广}
\dotfill 28

\section*{六、参考文献}
\dotfill 28

\section{问题重述}

自从爱因斯坦 1905 年提出光子概念以来,光的波粒二象性始终困扰着绝大多数人,至今仍然无法令常人像对其他物理现象一样地信服。光与物质相互作用时象粒子一样服从力学定律,但光又有波的内在性质,如:衍射、双缝干涉、偏振以及光子湮灭。光既可以是粒子,又同时是波,但我们至今不能同时看到这两个方面,光究竟以粒子还是以波的形式出现,只决定于实验。

仔细分析可以发现,对光波的内在性质:衍射、双缝干涉、偏振及光子湮灭的解释需要解决三个问题。第一个问题是光通常是按直线前进的,为什么光经过细缝后前进方向会发生改变?为什么必须是光通过很窄的细缝才会使前进方向发生改变?第二个问题是若干个光子同时作用的最终效果如何确定?是否仅取决于光子的个数?第三个问题是光通过偏振片的真实过程是什么?

在爱因斯坦之后继续有许多物理学家致力于探索光子波粒二象性的奥秘,并且也取得一定的进展。如我国已故的龚祖同院士猜想,光子并非仅是一个质点,其构造可能与氢原子类似,最简单的形式是由一个负电荷和一个与其等量的正电荷组成,负电荷在一个球形的轨道上围绕正电荷旋转(见附件 1 的 11 页),因此光线通过一个小孔或一条细缝时,如果其中一个电荷与小孔或狭缝边沿发生弹性碰撞或非弹性碰撞或被边缘吸引,光子的运动方向就会改变(在微观情况下遮挡物也不再可以视为平面),由此产生衍射。如果缝比较宽,则与小孔或细缝边缘发生碰撞的光子比例就会大大减少,直至几乎看不出来衍射。最近,北京大学又有学者提出光子是一个按光速直线前进的电磁场的猜想(见附件 2 的 112 页),光子的速度都相等。这个电磁场按照由光的波长所决定的角速度围绕上述前进直线旋转,相位由电场在 \(x\) 方向的投影或磁场在 \(y\) 方向的投影所决定,其旋转方向也只有左旋、右旋两种,因此事实上单个光子只有左旋光子、右旋光子两种。但一个左旋光子与一个右旋光子可以组成一个线性偏振光子对。当一个左旋光子与一个运动方向完全相同的右旋光子相遇时,如果相位完全相反,则发生光子湮灭现象。如果相位完全相同,则增强,由于能量与电磁场的强度是平方关系,相当于有 4 个光子存在。由于从两条细缝到达屏幕上同一点的距离不同,分别通过两条缝来的光子就会有相位差,由此产生双缝干涉现象。猜想偏振镜片仅让相位均值方向与偏振镜片光轴平行的线性偏振光子对通过(见附件 2 的 116 和 117 页)。由于正常光线中包含处于各种相位的光子对,在同时与偏振镜片碰撞时,这时各种相位的光子对会重新组合,产生平行于、垂直于偏振镜光轴的光子对,平行于偏振镜光轴的光子对能够通过偏振镜,而垂直于偏振镜光轴的光子对则被偏振镜阻挡,由此大约 50\% 的光线可以通过偏振镜片。

上述关于光子波粒二象性的猜想在一定程度上能够自圆其说,但只是定性的,还应该经过定量计算的检验。如果经过数学模型的仿真,肯定无法得到和大量物理实验相吻合的结果,则可以否定该猜想,如果在一定条件下可以得到和大量物理实验相吻合的结果,则上述猜想就更进一步,至少目前不能轻易否定。当然既然是猜想就很可能有错误,尤其对这样难度空前的问题,能够完全解释倒是不正常的。你们的任务是建立仿真这些猜想的数学模型,例如:单缝衍射中衍射缝的尺寸以及光子模型中的尺度、频率等参数对衍射屏上光强的空间分布形式的影响;双缝干涉试验中两条缝之间的距离、缝的宽度与干涉光图样的光强空间分布形式的关系等,并将你们模型得到的结果与诺贝尔物理奖获得者玻恩的光学经典名著《光学原理》p374(见附件 4 的 2)中的光的衍射、双缝干涉、多缝干涉的强度近似公式进行比较。你们也完全可以提出改进,而且也可以提出自己的猜想,设法做出尽可能多的数学仿真。

\section{模型假设}

由于本篇论文的论题是一个难度空前的问题,在无假设条件下对于诸多现象能够完全解释是不正常的,因此针对每个模型尤其是猜想我们都会给出一些合理而必要的假设,为了突出假设的针对性,我们会在每个模型分析前给出对应的假设,在此便不再赘述。

\section*{三、符号说明}

由于本篇论文我们将给出四个模型,每个模型中均涉及到多个符号,为避免符号混淆的情况发生,我们将在每个模型之前给出相应的符号说明。

\section*{四、模型的分析、建立与求解}

\subsection{模型一的分析、建立与求解}

\subsubsection{模型假设}

1、在恭祖同院士的猜想中,光子结构尺寸为归一化的“1”,即当说明衍射缝的宽度为 \( s \) 时,意味着衍射缝的宽度相当于 \( s \) 个光子结构的大小;

2、衍射缝宽小到与光子结构尺寸相同时,光子中的正光微子和负光微子与狭缝边沿发生弹性碰撞或非弹性碰撞或被边缘吸引的概率为 1;

3、为了遵循经典物理学原理,我们认为光子的速度恒为 \( c \),这表明,即使光子与狭缝边沿发生了碰撞,其能量依然不会改变,发生改变的仅是它的运动方向,间接影响其所能到达的区域而已;

\subsubsection{符号说明}

\begin{itemize}
    \item \( N \):衍射缝(孔)的个数。
    \item \( M \):一个确定的光束中光子的个数。(在视觉可视度范围内,光子数目应足够大)
    \item \( Q \):\( M \) 中与狭缝边沿发生碰撞的光子个数。
    \item \( \Delta Q \):未发生碰撞的光子个数。
    \item \( R(p) \):观测屏上角度变化值为 \( p \) 的地方所接收到的光粒子数目。
    \item \( s \):衍射缝的宽度。
    \item \( d \):衍射缝的间距。
    \item \( p \):\( p = \sin \theta - \sin \theta_0 \)。
    \item \( \theta_0 \):入射光束与衍射光栅面的法线的夹角。
    \item \( \theta \):衍射光线与衍射光栅面的法线的夹角。
    \item \( k \):\( k = \frac{\omega}{c} \)。
    \item \( \omega \):光的圆频率。 \quad \( c \):真空光速。
    \item \( \frac{1}{s} \):衍射或干涉时观测屏中心的光强值。
    \item \( I_0 \):狭缝足够宽,几乎没有碰撞的条件下直接照射的光强。
\end{itemize}

\subsubsection{模型分析}

当这个单缝足够且适当地大,我们可以想象光将无障碍式自由通过“狭”缝,即 \( M \) 中与狭缝边沿发生碰撞的光子个数 \( Q \) 为 0,此时衍射光线与衍射光栅面的法线所成的夹角 \( \theta \) 最小,且光强最大。由此我们不难想象,随着狭缝宽度 \( s \) 的增加,\( Q \) 将相应地增多,而在碰撞个数增加的同时,碰撞的程度也不尽相同,我们完全可以认为,某个光子碰撞的程度就决定了其碰撞后的方向,也就是决定了夹角 \( \theta \)(或角度变化值 \( p \))的取值,进

而决定了它到达观测屏上的位置,而观测屏上某个位置最后接收到的光子个数的多少决定了该位置光强的高低,由此可见,按照这个思路所建立的模型既能解释为什么光经过细缝后前进方向会发生改变,也能解释为什么光必须是通过很窄的细缝才会使前进方向发生改变。考虑了模型的合理性,我们下面便开始建立基于光子碰撞后概率分布的模型。

在经典物理光学中,干涉和衍射时产生的光强大小都是通过波的合成幅度来求的,其具体数值涉及到光的波动性,而本题我们是要用光的粒子性来解释波动特征,所以我们无需计算光强的具体数值,我们仅考虑相对光强值 $I/I_{0}$。

\subsection*{4.1.3 模型建立}

根据上面的分析,我们所建立的模型应该需要考虑两个部分:

1. 光束通过狭缝时发生碰撞的粒子数目与总的光子数目呈什么关系;
2. 发生碰撞的粒子其碰撞程度各是怎样的,即 $Q$ 个碰撞粒子的角度变化值的概率分布是什么样的。

先看第 1 个问题,我们已经知道,发生碰撞的粒子数目随着狭缝宽度的缩小而增加,仔细分析又知,狭缝越窄,碰撞增加的机率越大,因此可以说,$Q$ 与狭缝宽度的倒数 $1/s$ 并不是呈简单的正比例关系。观察下面的示意图 4.1.1,再考虑相位差 $p$ 对碰撞关系的影响(不同位置处所涉及的碰撞粒子数是不同的,实际上,$p$ 与 $s$ 有着紧密联合性,例如,狭缝宽度为 $s_{1}$ 时在 $p_{1}$ 处光强为 A,则当狭缝宽度减小为 $s_{2}$ 时,衍射范围增大,衍射条纹变宽,设此时光强为 A 的地方其相位差为 $p_{2}$,必有 $p_{2} > p_{1}$)我们建立碰撞粒子数 $Q$ 与狭缝宽度 $s$ 的合理关系式为:

\begin{equation}
\frac{Q}{M} = \frac{\sin(\varepsilon sp)}{\varepsilon sp} \tag{式 4.1-1}
\end{equation}

其中,$\varepsilon$ 表示碰撞粒子数 $Q$ 与狭缝宽度 $s$ 之间关系的调节系数,我们之后会对其进行讨论。

\begin{figure}[h]
\centering
\includegraphics[width=\textwidth]{image.png}
\caption{碰撞粒子数 $Q$ 与狭缝宽度 $s$ 的两种极端关系示意图}
\end{figure}

再看第二个问题,根据常识,我们可以想象大多数碰撞后的光粒子都只会偏离本来运动方向微小方向,只有少数部分光粒子才会偏离较大朝着偏远的地方奔去,而且奔向两边的粒子数目应该呈对称关系,这是毋庸置疑的,具体示意图如下:

\begin{figure}[h]
    \centering
    \includegraphics[width=\textwidth]{image.png}
    \caption{碰撞后光微子的大致运动分布示意图}
    \label{fig:4.1.2}
\end{figure}

为此,简单考虑时我们可以假设当碰撞粒子数为 \( Q \) 时,角度变化值为 \( p \) 处的观测屏上从碰撞粒子中所“接收”到的光粒子数 \( R' \) 与 \( Q \) 的关系式呈正态分布,但正如上文所提到的那样,\( p \) 与 \( s \) 有着紧密联合性,\( s \) 的大小也会影响 \( p \) 处所“接收”的光粒子数目,综合考虑,我们认为合理的关系应该为:
\begin{equation}
\frac{R'(p)}{Q} = N(0, \sigma) = \frac{1}{\sqrt{2\pi\sigma}} \exp\left(-\frac{p^2}{2\sigma^2}\right) \bullet \frac{\sin(\varepsilon sp)}{\varepsilon sp} \tag{式 4.1-2}
\end{equation}

其中,\( \sigma \) 表示方差系数,具体数值我们之后会讨论。值得注意的是,对于未发生碰撞的光子,我们首先认为其直接增加了直射中心点及其附近区域的光强,设这个“增光带” \( \Delta \) 所增加的光子数为 \( \Delta Q = M - Q = M - \frac{\sin(\varepsilon sp)}{\varepsilon sp} \)。综上所述,结合已经得到的式 (4.1-1) 和式 (4.1-2),我们可以得出该种模型下的相对光强分布为:
\begin{equation}
\frac{I(p)}{I_0} = \frac{R(p)}{M} =
\begin{cases}
\frac{R'(p)}{Q} \bullet \frac{Q}{M} + \frac{\Delta Q}{M\Delta} = \frac{1}{\Delta} + \left(\frac{1}{\sqrt{2\pi\sigma}} \exp\left(-\frac{p^2}{2\sigma^2}\right) - \frac{1}{M\Delta}\right) \left(\frac{\sin(\varepsilon sp)}{\varepsilon sp}\right)^2 & p \leq \Delta \\
\frac{R(p)}{Q} \bullet \frac{Q}{M} = \frac{1}{\sqrt{2\pi\sigma}} \exp\left(-\frac{p^2}{2\sigma^2}\right) \bullet \left(\frac{\sin(\varepsilon sp)}{\varepsilon sp}\right)^2 & p > \Delta
\end{cases} \tag{式 4.1-3}
\end{equation}

上式显然是一个分段函数,且当 \( s \) 取定、参数 \( \varepsilon \) 确定后,其必然在 \( \pm\Delta \) 处不可导,而且无论 \( \Delta \) 怎么取,图形都是不光滑的分段曲线,这样从光学角度就相当于观测屏上出现了光断层现象,而且对于我们求取参数 \( \varepsilon \)、\( \sigma \) 也是一大障碍,因此我们要设法使上述函数变为光滑可微函数。当然,我们的思想还是要保证未发生碰撞的光子数 \( \Delta Q \) 集中分布在直射中心点及其紧邻区域。

在通信系统中,升余弦系统 \cite{ref1} (163-164) 在频域具有很好的低通特性,尤其是当滚降系数为 0 时便是理想低通系统,考虑到时域和频域的对称性,时域的低通性随着频域尖锐性的降低而提高,因此我们选择滚降系数为 1 时的时域升余弦函数作为 \( \Delta Q \) 在观测屏上的分布密度函数。即:
\begin{equation}
\frac{\Delta Q(p)}{\Delta Q} = \frac{\sin \varepsilon sp}{\mu \varepsilon sp} \bullet \frac{\cos \pi p}{1 - 4p^2} \tag{式 4.1-4}
\end{equation}

其中,参数 \( \mu \) 表示非碰撞分布系数,其选取要结合 \( \varepsilon \) 综合考虑,具体取值我们也在下一部分讨论。综上所述,我们得到该种模型下的含有参数的相对光强分布为:
\begin{equation}
\frac{I(p)}{I_0} = \frac{R'(p)}{Q} \bullet \frac{Q}{M} + \frac{\Delta Q(p)}{\Delta Q} \bullet \frac{\Delta Q}{M} = \frac{1}{\sqrt{2\pi\sigma}} \exp\left(-\frac{p^2}{2\sigma^2}\right) \bullet \left(\frac{\sin(\varepsilon sp)}{\varepsilon sp}\right)^2 + \frac{\sin \varepsilon sp}{\mu \varepsilon sp} \bullet \frac{\cos \pi p}{1 - 4p^2} \bullet \left(1 - \frac{\sin(\varepsilon sp)}{\varepsilon sp}\right) \tag{式 4.1-5}
\end{equation}

\subsection{4.1.4 参数确定}

首先可由 $I(0)/I_{0}=1$ 得到 $\sigma=1/\sqrt{2\pi}$。为确定参数 $\varepsilon$、$\mu$ 的值,我们需要找到一组专业性实验数据,然后将我们的模型理论值与实验值对比,采用最小二乘估计 $RSS(\mu,\varepsilon)$ 准则 \cite{ref2}[20-22],最终确定两个参数的值。

从《衍射光强的实验教案》中,我们得到一组可靠的 $p\sim\hat{I}(p)/\hat{I}_{0}$ 共 152 对数据,足以较准确确定参数 $\varepsilon$、$\mu$ 的值。实验中,$s=0.5\,mm$。由

\begin{equation}
RSS(\mu,\varepsilon)=\sum_{i=1}^{152}\left(\frac{I(p(i))}{I_{0}}-\frac{\hat{I}(p(i))}{\hat{I}_{0}}\right)^{2}
\tag{式 4.1-6}
\end{equation}

极小化 $RSS(\mu,\varepsilon)$ 时 \cite{ref3}[203-205],毫无疑问要对其进行偏微分,再令偏微分为零便能得到我们想要的 $\varepsilon$、$\mu$ 的值。即:

\begin{equation}
\left\{
\begin{aligned}
\frac{\partial RSS(\mu,\varepsilon)}{\partial\mu} &= 2\sum_{i=1}^{152}\left(\frac{I(p(i))}{I_{0}}-\frac{\hat{I}(p(i))}{\hat{I}_{0}}\right)\bullet\frac{\partial}{\partial\mu}\frac{I(p(i))}{I_{0}} = 0 \\
\frac{\partial RSS(\mu,\varepsilon)}{\partial\varepsilon} &= 2\sum_{i=1}^{152}\left(\frac{I(p(i))}{I_{0}}-\frac{\hat{I}(p(i))}{\hat{I}_{0}}\right)\bullet\frac{\partial}{\partial\varepsilon}\frac{I(p(i))}{I_{0}} = 0
\end{aligned}
\right.
\tag{式 4.1-7}
\end{equation}

将 (式 4.1-5) 代入上式 \cite{ref4}[225-240] 可得,在 $s=0.5\,mm$ 的条件下,$\mu=150$,$\varepsilon=2\pi\times10^{6}/3$。这时,$RSS(\mu,\varepsilon)=0.5636$,该残差平方和与相对光强总能量 $(\hat{I}(p(i))^{2}/\hat{I}_{0}^{2})$ 之比为 0.0275,误差已经很小,直观地看,我们的模型与实验光强分布的差别如下图所示:

\begin{figure}[h]
    \centering
    \includegraphics[width=\textwidth]{image.png}
    \caption{模型一与实验光强分布对比}
    \label{fig:4.1.3}
\end{figure}

\subsection{4.1.5 模型一的检验}

为了检验的模型一最终公式的可靠性,理论上我们应该看其与大量物理实验结果是否吻合,而我们已经知道,经过理论推导的玻恩公式已被证明与许多实验数据是吻合的,所以我们可以直接将我们的公式与玻恩公式 $I^{(0)}(p)=\left[\frac{\sin ksp/2}{ksp/2}\right]^{2}(k=\frac{w}{c}=\frac{2\pi}{\lambda})$ 进行对比,并以残差的平方和来衡量我们模型的可靠性,以 HeNe 激光(波长为 0.6328 微米)为例,我们先根据玻恩衍射公式自己产生仿真数据,然后将所需变量代入模型一计算输出结果,得到残差平方和为 $RSS(\mu,\varepsilon)=0.22961$,它与理论上相对光强总能量($\hat{I}(p(i))^2/\hat{I}_{0}^2$)之比为 0.0054,这个残差值是相当可观的。其中模型结果与理论结果对比如下:

\begin{figure}[h]
    \centering
    \includegraphics[width=\textwidth]{image1.png}
    \caption{单缝衍射时模型一与理论光强分布对比}
    \label{fig:4.1.4}
\end{figure}

\subsection{4.1.6 模型一的适应度分析}

针对龚祖同院士的波粒二象性的猜想,我们已经建立了基于碰撞情况分析的模型,并已经验证它对单缝衍射可以给出很好的定性定量解释,那么它对于双缝干涉和多缝衍射的适应度又如何呢?

双缝干涉中,将有两组独立的“光子团”分别经过缝 $S_{1}$ 和 $S_{2}$,此时在碰撞模型下,光子间的相互作用也显得尤为突出,即两组“光子团”在各自的狭缝内受到碰撞后又会在观测屏前的空间中发生碰撞,而且这种碰撞式的相互作用非常强烈,相当于两缝中间又多出了一个可以发生单缝衍射的“合成缝”$S_{3}$,为此,我们可以断定,最后观测屏上光子数目的分布是由这三个缝隙共同影响的,而且考虑到双缝把大约 50\% 的能量传递给了“合成缝”,所以“合成缝”会起主要作用。可令 $\frac{I(p)}{I_{0}}=\frac{I_{S_{3}}(p)}{I_{0}}+\frac{I_{S_{1}}(p)}{2I_{0}}+\frac{I_{S_{2}}(p)}{2I_{0}}$,不难想象,

当以合成缝产生的光强分布为标尺时,双缝产生的光强与其仅有相位上的差别,即 $I_{s_{1}}(p)=I_{s_{3}}(p-p_{0})$,$I_{s_{2}}(p)=I_{s_{3}}(p+p_{0})$,其中 $p_{0}=d/2L$,$L$ 为狭缝到观测屏的距离,$p_{0}$ 要尽可能的小。于是有,

\begin{equation}
\frac{I(p)}{I_{0}}=\frac{I_{s_{3}}(p)}{I_{0}}+\frac{I_{s_{3}}\left(p-\frac{d}{2L}\right)}{2I_{0}}+\frac{I_{s_{3}}\left(p+\frac{d}{2L}\right)}{2I_{0}}
\tag{式 4.1-8}
\end{equation}

将模型一表达式(1-5)代入上式便能得到完整的干涉光强分布表达式。当我们选取合适的缝距 $d$ 和缝屏距 $L$ 时便能得到满意的干涉图样。例如,当我们取 $d=10cm$,$L=50cm$ 时,调整两种情况下的自变量(即相位差)关系,模型所得结果和玻恩理论干涉光强分布的比较如下:

\begin{figure}[h]
\centering
\includegraphics[width=0.8\textwidth]{image1.png}
\caption{模型一的干涉光强分布的傅里叶变换图}
\end{figure}

\begin{figure}[h]
\centering
\includegraphics[width=0.8\textwidth]{image2.png}
\caption{玻恩理论光强分布的傅里叶变换频谱图}
\end{figure}

图 4.1.5 双缝干涉时模型一与玻恩理论公式的结果对比

图1-5的右侧是两种情况下光强分布的频谱图,变换之后的图像在原点平移之前四角是低频,最亮,平移之后中间部分是低频,最亮,亮度大说明低频的能量大(幅角比较大)。虽然上述仿真对于我们的干涉常识是种挑战,实际上从理论上说这是毋庸置疑的,观测屏上形成的条纹应是单缝衍射和双缝干涉的综合结果,因而各级双缝干涉条纹中心的光强自然要受到单缝衍射的限制,造成各级干涉条纹中心光强并不相等,双缝干涉条纹中心的强度被限制在单缝衍射的轮廓之下。不难看出,我们的模型在双缝干涉时也近乎逼近玻恩理论公式,这充分说明了模型的适应性,也从侧面说明了龚祖同院士猜想的合理性(只不过要选取合适的缝距 $d$ 和缝屏距 $L$)。类似的,我们完全可以类推,当考虑 $N$ 缝衍射时,我们可以将式(1-8)的干涉模型进一步推广为:

\begin{equation}
\frac{I(p)}{I_{0}}=\frac{I_{s_{N+1}}(p)}{I_{0}}+\alpha_{1}\frac{I_{s_{N+1}}(p-d/2L)}{I_{0}}+\alpha_{2}\frac{I_{s_{N+1}}(p+d/2L)}{I_{0}}+\cdots+\alpha_{2N-1}\frac{I_{s_{N+1}}(p-Nd/2L)}{I_{0}}+\alpha_{2N}\frac{I_{s_{N+1}}(p+Nd/2L)}{I_{0}}
\tag{式4.1-9}
\end{equation}

其中,$\alpha_{1}, \alpha_{2}, \ldots, \alpha_{2N}$ 是每个狭缝对最终光强分布的影响系数,确定改组系数是项复杂的任务,在此我们不便过多赘述。至此,我们给出了基于龚祖同院士光的波粒二象性猜想的碰撞模型的完整定性与定量表示、说明。

\subsection{模型二——电磁场偏转模型(基尔霍夫衍射积分)的分析与证明}

\subsubsection{模型假设}

1、假设光子是一个按光速直线前进并围绕前进方向旋转的电磁场;

2、假设光子的速度均为 $c$;

3、假设进行衍射、干涉等物理实验时的挡板不会产生透射、反射等会对原光产生干扰的现象。

\subsubsection{模型分析}

麦克斯韦早在 19 世纪 60 年代提出了光的电磁理论,从电磁波的角度解释了光的传播、干涉、衍射、散射、偏振等现象,这里我们假设光子本身并不存在,它是一种能量的形式,其表现方式就是电磁场。根据北京大学学者所提出的思想,这个电磁场按照光速直线前进,并按照由光的波长所决定的角速度围绕上述前进直线旋转,相位由电场在 $x$ 方向的投影或磁场在 $y$ 方向的投影所决定,其旋转方向也只有左旋、右旋两种,因此事实上单个光子只有左旋光子、右旋光子两种。但一个左旋光子与一个右旋光子可以组成一个线性偏振光子对。下面我们来推导在这种电磁场偏转模型下的衍射公式。

\subsubsection{公式推导}

首先,根据 Helmholtz 方程 [5](413-453),在无源空间中,电磁场满足 $\nabla^{2} \psi + k^{2} \psi = 0$,其中,$\nabla^{2}$ 是拉普拉斯算子,$\psi$ 是波函数,$k = \frac{\omega}{c} = \frac{2\pi}{\lambda}$。

接着,设一个满足 Helmholtz 方程对应的格林函数,取 $G(x, y, z) = \frac{e^{ikr}}{r}$,其中,$r = \sqrt{(x - x_{p})^{2} + (y - y_{p})^{2} + (z - z_{p})^{2}}$,指该点到面元中 P 点之间的距离。下面,我们要证明该函数 $G$ 满足电磁场条件:

\begin{align*}
\nabla^{2} G + k^{2} G &= \nabla^{2} \frac{e^{ikr}}{r} + k^{2} G \\
&= \frac{1}{r} \nabla^{2} e^{ikr} + 2 (\nabla \frac{1}{r}) (\nabla e^{ikr}) + e^{ikr} \nabla^{2} \frac{1}{r} + k^{2} G \\
&= \frac{1}{r^{3}} \frac{d}{dr} \left( r^{2} \frac{d}{dr} e^{ikr} \right) - 2 \left( \frac{\vec{r}}{r^{3}} \right) \left( ike^{ikr} \frac{\vec{r}}{r} \right) + e^{ikr} \nabla^{2} \frac{1}{r} + k^{2} G \\
&= -k^{2} \frac{e^{ikr}}{r} + e^{ikr} \nabla^{2} \frac{1}{r} + k^{2} G = e^{ikr} \nabla^{2} \frac{1}{r} \\
&= e^{ikr} \frac{1}{r^{2}} \frac{d}{dr} \left( r^{2} \frac{d}{dr} \frac{1}{r} \right) = 
\begin{cases}
0, & \text{当 } r \neq 0 \\
\infty, & \text{当 } r = 0
\end{cases}
\end{align*}

由于 $r = 0$ 是一个奇点,而 $G$ 必须保证是连续和可微的,所以要把这个奇点从积分区除掉。并且,

\[
\iiint_{V} \nabla^{2} \frac{1}{r} dV = \iiint_{V} \nabla \cdot \nabla \frac{1}{r} dV = \iint_{S} \nabla \frac{1}{r} \cdot d\vec{s} = -\iint_{S} \frac{\vec{r}}{r^{3}} \cdot d\vec{s} = -\iint_{S} d\Omega = -4\pi
\]

其中,$\nabla$ 是汉密尔顿算子。因此,可以得到:$\nabla^{2} G + k^{2} G = -4\pi \delta(\vec{x} - \vec{x}_{p})$。这充分说明了 $G$ 是与时间无关的波动方程。

设 \( \nu \) 是闭合面 \( S \) 所包围的体积,\( P \) 是 \( S \) 内任一点,根据上面对积分区的规定,则 \( \psi \) 和 \( G \) 在闭合面 \( S \) 内部和边界上具有连续的一阶和二阶偏微分,根据格林定理:
\[
\iiint\limits_{V} \left( \psi \nabla^2 G - G \nabla^2 \psi \right) d\nu = \iint\limits_{S} \left( G \nabla \psi - \psi \nabla G \right) d\vec{s},
\]
其中 \( d\vec{s} \) 是在区域 \( V \) 中沿 \( S \) 面指向内部的面元,设 \( \vec{n} \) 是沿 \( S \) 面指向内部的法线,则 \( d\vec{s} = \vec{n} ds \),\( ds \) 是 \( S \) 面上的面元的微分。这样便有如下的格林公式:
\[
\iiint\limits_{V} \left( \psi \nabla^2 G - G \nabla^2 \psi \right) d\nu = \iint\limits_{S} \left( G \nabla \psi - \psi \nabla G \right) \vec{n} ds.
\]
又根据 Helmholtz 方程,电磁场满足 \( \nabla^2 \psi + k^2 \psi = 0 \),\( \nabla^2 G + k^2 G = -4\pi \delta(\vec{x} - \vec{x}_p) \),则格林公式
\begin{align*}
\iiint\limits_{V} \left( \psi \nabla^2 G - G \nabla^2 \psi \right) d\nu &= \iiint\limits_{V} \left\{ \psi \left[ -k^2 G - 4\pi \delta \left( \vec{x} - \vec{x}_p \right) \right] - G \left[ -k^2 \psi \right] \right\} d\nu \\
&= \iiint\limits_{V} \left\{ -4\pi \psi \delta \left( \vec{x} - \vec{x}_p \right) \right\} d\nu = -4\pi \psi \left( \vec{x}_p \right)
\end{align*}
所以
\begin{align*}
\psi \left( \vec{x}_p \right) &= -\frac{1}{4\pi} \iint\limits_{S} \left( G \nabla \psi - \psi \nabla G \right) \vec{n} ds = -\frac{1}{4\pi} \iint\limits_{S} \left( \frac{e^{ikr}}{r} \nabla \psi - \psi \nabla \frac{e^{ikr}}{r} \right) \vec{n} ds \\
&= -\frac{1}{4\pi} \iint\limits_{S} \left[ \frac{e^{ikr}}{r} \nabla \psi - \psi \left( \frac{1}{r} \nabla e^{ikr} + e^{ikr} \nabla \frac{1}{r} \right) \right] \vec{n} ds \\
&= -\frac{1}{4\pi} \iint\limits_{S} \left[ \frac{e^{ikr}}{r} \nabla \psi - \psi \left( -ik \frac{e^{ikr} \vec{r}}{r} + \frac{e^{ikr} \vec{r}}{r^2} \right) \right] \vec{n} ds = -\frac{1}{4\pi} \iint\limits_{S} \frac{e^{ikr}}{r} \vec{n} \left[ \nabla \psi + \psi \frac{\vec{r}}{r} \left( ik - \frac{1}{r} \right) \right] ds
\end{align*}
该公式为基尔霍夫公式 \cite{ref}[134-145]。此处,我们有必要对公式作个解释:\( \psi \left( \vec{x}_p \right) \) 是指在由 \( S \) 面包围的区域 \( V \) 内任何一点的场,该点的场是由边界面 \( S \) 上的 \( \psi \) 和 \( \nabla \psi \) 表示。公式中的 \( \frac{e^{ikr}}{r} \) 是一个辅助变量,它的意义是表示 \( S \) 面上的点向其所包含的区域 \( V \) 内 \( x_p \) 点传播的波,波源的强度由 \( \psi \) 和 \( \nabla \psi \) 决定。因此,曲面上的每一点可以看成是次级光源发射的 \( z \) 波的叠加,示意图见图 4.2.1。

\begin{figure}[h]
    \centering
    \includegraphics[width=0.8\textwidth]{image.png}
    \caption{光的传播示意图}
    \label{fig:light_propagation}
\end{figure}

为简单起见,我们先假设缝隙是宽度极窄的矩形孔,设缝隙的边长分别为 \( 2a \) 和 \( 2b \),且除缝隙之外,其余挡板部分完全不透光。所以,在上面的基尔霍夫公式中,只要对矩形孔积分,就可以得到:
\[
\psi \left( \vec{x}_p \right) = -\frac{1}{4\pi} \iint\limits_{\text{缝隙}} \frac{e^{ikr}}{r} \vec{n} \left[ \nabla \psi + \psi \frac{\vec{r}}{r} \left( ik - \frac{1}{r} \right) \right] ds
\]

我们先假设光是垂直射向缝隙的。附件 2 中,北大学者给出了他提出的旋转光子所产生的波函数,其中左旋光子的波函数为:$\overline{E}\left(0,0,z,t\right)=E_{p}\left(\overline{e}_{x}+i\overline{e}_{y}\right)e^{-i\left(\omega t-kz\right)}$,则设入射波的波函数为:
\[
\psi\left(\overline{x}\right)=E_{p}\left(\overline{e}_{x}+i\overline{e}_{y}\right)e^{i\overline{k}_{1}\cdot\overline{x}}, \quad \nabla\psi\left(\overline{x}\right)=i\overline{k}_{1}E_{p}\left(\overline{e}_{x}+i\overline{e}_{y}\right)e^{i\overline{k}_{1}\cdot\overline{x}}=i\overline{k}_{1}\psi\left(\overline{x}\right) \tag{式 4.2-1}
\]
这时有
\[
\psi\left(\overline{x}_{p}\right)=-\frac{1}{4\pi}\iint_{\text{缝隙}}\frac{e^{ikr}}{r}\overrightarrow{n}\left[i\overline{k}_{1}\psi+\psi\frac{\overrightarrow{r}}{r}\left(ik-\frac{1}{r}\right)\right]ds=-\frac{1}{4\pi}\iint_{\text{缝隙}}\frac{e^{ikr}}{r}\overrightarrow{n}\psi\left[i\overline{k}_{1}+\frac{\overrightarrow{r}}{r}\left(ik-\frac{1}{r}\right)\right]ds
\]
由于 $\frac{1}{r}<<k$,略去 $\frac{1}{r}$ 的高阶量,上式化为:
\[
\psi\left(\overline{x}_{p}\right)=-\frac{1}{4\pi}i\iint_{\text{缝隙}}\frac{e^{ikr}}{r}\overrightarrow{n}\psi\left[\frac{\overrightarrow{k}_{1}+\overrightarrow{r}}{r}k\right]ds \tag{式 4.2-2}
\]

由于衍射不能改变波的频率,所以波矢量的大小应该相等,即 $\left|\overrightarrow{k}_{1}\right|=\left|\overrightarrow{k}_{2}\right|=k$,因此有如下等式成立:
\[
r=\sqrt{R^{2}+x^{2}-2\overrightarrow{R}\cdot\overrightarrow{x}}=R\left(1+\frac{x^{2}}{R^{2}}-\frac{2\left|\overrightarrow{R}\right|\cdot\overrightarrow{n}^{\prime}\cdot\overrightarrow{x}}{R^{2}}\right)^{1/2}
\]
\[
=R\left(1-\frac{1}{2}\frac{2\overrightarrow{n}^{\prime}\cdot\overrightarrow{x}}{R}+\cdots\right)=R-\overrightarrow{n}^{\prime}\cdot\overrightarrow{x}=R-\frac{\overrightarrow{k}_{2}}{k_{2}}\cdot\overrightarrow{x} \tag{式 4.2-3}
\]

代入式 4.2-2 得
\[
\psi\left(\overline{x}_{p}\right)=-\frac{1}{4\pi}i\iint_{\text{缝隙}}\frac{e^{ik\left(R-\frac{\overrightarrow{k}_{2}\cdot\overrightarrow{x}}{k_{2}}\right)}}{R-\frac{\overrightarrow{k}_{2}\cdot\overrightarrow{x}}{k_{2}}}\overrightarrow{n}\psi\left[\frac{\overrightarrow{k}_{1}+\frac{\overrightarrow{r}}{r}k}{k}\right]ds=-\frac{1}{4\pi}i\iint_{\text{缝隙}}\frac{e^{ik\left(R-\frac{\overrightarrow{k}_{2}\cdot\overrightarrow{x}}{k_{2}}\right)}}{R-\frac{\overrightarrow{k}_{2}\cdot\overrightarrow{x}}{k_{2}}}\overrightarrow{n}\psi\left[\frac{\overrightarrow{k}_{1}+\frac{\overrightarrow{k}_{2}}{k}}{k}\right]ds
\]
\[
=-\frac{1}{4\pi}ie^{ikR}\iint_{\text{缝隙}}\frac{e^{-i\overrightarrow{k}_{2}\cdot\overrightarrow{x}}}{R-\frac{\overrightarrow{k}_{2}\cdot\overrightarrow{x}}{k_{2}}}\overrightarrow{n}\psi\left[\overrightarrow{k}_{1}+\overrightarrow{k}_{2}\right]dxdy
\]

代入左旋光子的波函数又可得:
\[
\psi\left(\overline{x}_{p}\right)=-\frac{1}{4\pi}ie^{ikR}\iint_{\text{缝隙}}\frac{e^{-i\overrightarrow{k}_{2}\cdot\overrightarrow{x}}}{R-\frac{\overrightarrow{k}_{2}\cdot\overrightarrow{x}}{k_{2}}}\overrightarrow{n}E_{p}\left(\overline{e}_{x}+i\overline{e}_{y}\right)e^{i\overrightarrow{k}_{1}\cdot\overrightarrow{x}}\left[\overrightarrow{k}_{1}+\overrightarrow{k}_{2}\right]dxdy
\]
\[
=-\frac{1}{4\pi}ie^{ikR}E_{p}\left(\overrightarrow{k}_{1}+\overrightarrow{k}_{2}\right)\cdot\overrightarrow{n}\iint_{\text{缝隙}}e^{i\overrightarrow{x}\left(\overrightarrow{k}_{1}-\overrightarrow{k}_{2}\right)}\left(\overline{e}_{x}+i\overline{e}_{y}\right)\left[\overrightarrow{k}_{1}+\overrightarrow{k}_{2}\right]dxdy
\]
\[
=-\frac{ie^{ikR}E_{p}}{4\pi R}\left(\overrightarrow{k}_{1}+\overrightarrow{k}_{2}\right)\cdot\overrightarrow{n}\cdot\left(\overline{e}_{x}+i\overline{e}_{y}\right)\left(\int_{-a}^{a}e^{ix\left(k_{1x}-k_{2x}\right)}dx\int_{-b}^{b}e^{iy\left(k_{1y}-k_{2y}\right)}dy\right)
\]
\[
=-\frac{ie^{ikR}E_{p}}{4\pi R}\left(\overrightarrow{k}_{1}+\overrightarrow{k}_{2}\right)\cdot\overrightarrow{n}\cdot\left(\overline{e}_{x}+i\overline{e}_{y}\right)4\frac{\sin\left(k_{1x}-k_{2x}\right)a}{k_{1x}-k_{2x}}\frac{\sin\left(k_{1y}-k_{2y}\right)b}{k_{1y}-k_{2y}}
\]
\[
=-\frac{ie^{ikR}E_{p}}{\pi R}\left(\overrightarrow{k}_{1}+\overrightarrow{k}_{2}\right)\cdot\overrightarrow{n}\cdot\left(\overline{e}_{x}+i\overline{e}_{y}\right)\frac{\sin\left(k_{1x}-k_{2x}\right)a}{k_{1x}-k_{2x}}\frac{\sin\left(k_{1y}-k_{2y}\right)b}{k_{1y}-k_{2y}}
\]

因为光强 \( I \) 和振幅的模平方成正比,由此可知
\[
I = \frac{2E_{p}^{2}}{\pi^{2}R^{2}}(k_{1z} + k_{2z})^{2} \left[ \frac{\sin \left( k_{1x} - k_{2x} \right) a}{k_{1x} - k_{2x}} \frac{\sin \left( k_{1y} - k_{2y} \right) b}{k_{1y} - k_{2y}} \right]^{2}
\]
如果将 \( \overrightarrow{k_{2}} \) 与 \( x \)、\( y \)、\( z \) 轴的夹角分别设为 \( \left( \frac{\pi}{2} - \alpha \right) \)、\( \left( \frac{\pi}{2} - \beta \right) \)、\( \theta \) 表示。则
\[
k_{2x} = k \sin \alpha, k_{2y} = k \sin \beta, k_{2z} = k \cos \theta
\]
而入射光垂直于缝隙时,\( k_{1x} = k_{1y} = 0, k_{1z} = k \)。所以有
\[
I = \frac{2E_{p}^{2}}{\pi^{2}R^{2}} (k + k \cos \theta)^{2} \left[ \frac{\sin \left( ak \sin \alpha \right) \sin \left( bk \sin \beta \right)}{k \sin \alpha \cdot k \sin \beta} \right]^{2}
\]
定义 \( I_{0} = \frac{2E_{p}^{2}}{\pi^{2}R^{2}} (k + k \cos \theta)^{2} \),当衍射范围较小时,\( \alpha \)、\( \beta \) 角度很小,此时满足
\[
\sin \alpha \approx \alpha, \sin \beta \approx \beta, \cos \theta \approx 1 \text{,显然有 } I_{0} = \frac{8E_{p}^{2}k^{2}}{\pi^{2}R^{2}}, \, I = I_{0} \left[ \frac{\sin \left( ak \alpha \right) \sin \left( bk \beta \right)}{k \alpha \cdot k \beta} \right]^{2} \text{。可见,这种模型下的光强分布不仅与缝宽有关,还与缝长有关,而当缝长 } b \text{ 以及 } \beta \text{ 确定了之后,该模型与玻恩理论下的单缝衍射公式完全吻合,说明了模型的可靠性,也证明了北大学者提出的关于光子电磁场的诸多假设都是有意义的。}
\]

\subsection*{4.3 模型三——光子旋转模型的的分析、建立与求解}

\subsubsection{4.3.0 模型假设}
1. 光子运动中都是成对出现的,即左旋光子与右旋光子的光子对,光子对中的左旋光子与右旋光子都是瞬间隐藏状态,即左旋光子与右旋光子相位相反。
2. 我们基本上把光束看成是光子束,光子有大小,光子在前进的过程中它的相位也会发生变化。
3. 杨氏双缝干涉中,左旋光子和右旋光子到达双缝的相位相等。

\subsubsection{4.3.1 符号说明}
\begin{align*}
d: & \quad \text{双缝间距:} \\
L: & \quad \text{缝屏间距:} \\
r_{1}: & \quad \text{缝 1 到屏上 P 点的距离:} \\
r_{2}: & \quad \text{缝 2 到屏上 P 点的距离:} \\
w: & \quad \text{光子转动角速度:} \\
\vec{E}: & \quad \text{光子平面电场分量:} \\
\vec{H}: & \quad \text{光子平面磁场分量:}
\end{align*}

\subsubsection{4.3.2 模型分析}
依据北京大学学者的理论 \cite{ref7},光子是一个按光速直线前进的电磁场,光子的速度都相等。这个电磁场按照由光的波长所决定的角速度围绕上述前进直线旋转,相位由电场在 \( x \) 方向的投影或磁场在 \( y \) 方向的投影所决定,其旋转方向也只有左旋、右旋两种,因此事实上单个光子只有左旋光子、右旋光子两种。但一个左旋光子与一个右旋光子可以组成一个线性偏振光子对。当一个左旋光子与一个运动方向完全相同的右旋光子相遇时,如果相位完全相反,则发生光子湮灭现象。如果相位完全相同,则增强,由于能量

与电磁场的强度是平方关系,相当于有 4 个光子存在。基于此种分析,我们可以建立下述光子旋转模型。

\subsection{4.3.3 模型建立}

在空间中,光子像质点一样,凭借其惯性以光速 \( c \) 前进。光子的两个平面分量 \( \vec{E} \) 与 \( \vec{H} \) 分别表示为:

左旋光子的波函数:
\[
\vec{E}(0,0,z,t) = E_p \left( \vec{e}_x + i \vec{e}_y \right) e^{-i(\omega t - kz)} \tag{式 4.3-1 (a)}
\]
\[
\vec{H}(0,0,z,t) = H_p \left( \vec{e}_x + i \vec{e}_y \right) e^{-i(\omega t - kz)} \tag{式 4.3-1 (b)}
\]

右旋光子的波函数:
\[
\vec{E}(0,0,z,t) = E_p \left( \vec{e}_x - i \vec{e}_y \right) e^{-i(\omega t - kz)} \tag{式 4.3-2 (a)}
\]
\[
\vec{H}(0,0,z,t) = H_p \left( \vec{e}_x - i \vec{e}_y \right) e^{-i(\omega t - kz)} \tag{式 4.3-2 (b)}
\]

光子对的波函数:
\[
\vec{E}(0,0,z,t) = E_p \vec{e}_x e^{-i(\omega t - kz)} \tag{式 4.3-3 (a)}
\]
\[
\vec{H}(0,0,z,t) = H_p \vec{e}_y e^{-i(\omega t - kz)} \tag{式 4.3-3 (b)}
\]

其中,\( E_p = \left( \frac{3}{2\sqrt{10}} \right)^{\frac{3}{2}} \sqrt{\frac{h\mu_0}{\pi^3 c}} \omega^2 \),\( H_p = \left( \frac{3}{2\sqrt{10}} \right)^{\frac{3}{2}} \sqrt{\frac{h\varepsilon_0}{\pi^3 c}} \omega^2 \),\( \vec{e}_x \) 与 \( \vec{e}_y \) 分别为 \( x \) 轴和 \( y \) 轴方向的单位向量,\( z \) 为前进的距离,\( k = 2\pi / \lambda \)。

由分析已知,当一个左旋光子与一个运动方向完全相同的右旋光子相遇时,如果相位完全相反,则发生光子湮灭现象。如果相位完全相同,则增强。以单色光为例,假设两束光来自同一光源,则这两束光中光子的左旋右旋情况应是完全一致的,这两束光也称为完全相干光。由于两束光中包含很多光子,而各个光子到达双缝时的初相都是相等的,所以屏幕上 \( r_1 \) 和 \( r_2 \) 不等的地方相遇的光子一定是一前一后到达双缝的。某个时刻,每个缝隙的左旋光子和右旋光子同时运动,且两个缝隙的左旋光子和右旋光子的相位都相等。若某缝隙的左旋光子和另一个缝隙的右旋光子同时到达屏幕上某点,则它们相位相反时,发生光子湮没;相位相同,则能量增强。

\begin{figure}[h]
    \centering
    \includegraphics[width=0.8\textwidth]{interference_pattern.png}
    \caption{干涉图样}
    \label{fig:interference_pattern}
\end{figure}

设从 \( s_1 \),\( s_2 \) 射出的左旋光子 1 和右旋光子 2 同时到达 \( P \) 点,设所有电子对在 \( s_1 \),\( s_2 \) 射出时,其相位均是相一致的,这里我们所指的相位均为光子的电场分量所对应的相位。若将光子 1 的相位设为 0,则光子 2 的相位就为 \( \pi \)。由附件 2 可以知道,光子一边以光速 \( c \) 沿着前进方向运动,一边绕着前进的轴线进行自旋运动,其角速度为 \( w \),\( w = 2\pi / T = 2\pi \gamma \),其中 \( \gamma \) 为光子频率。不难得到,当光子 1 到达 \( P \) 点时相位改变量为 \( +r_1 \cdot w / c \),光子 2 到达 \( P \) 时相位为 \( -r_2 \cdot w / c \)。若两个光子到达 \( P \) 点时相位正好相同,

那么它们的能量将变为原来的 4 倍;若相位相反,则将变为 0。

情况 1:左旋光子与右旋光子相位相同,则所相遇的点为明条纹:
\[
0 + \frac{r_1}{c} \bullet w = \pi - \frac{r_2}{c} \bullet w + 2k\pi
\]
即
\[
\frac{r_1}{c} \bullet w + \frac{r_2}{c} \bullet w = \frac{r_1 + r_2}{c} \bullet w = \pi + 2k\pi
\]
将 \( w = 2\pi / T = 2\pi \gamma \) 代入得:
\[
r_1 + r_2 = \frac{(2k + 1)}{2} \lambda
\]

情况 2:左旋光子与右旋光子相位相反,则所相遇的点位暗条纹:
\[
0 + \frac{r_1}{c} \bullet w + \pi = \pi - \frac{r_2}{c} \bullet w + 2k\pi
\]
即
\[
0 + \frac{r_1}{c} \bullet w + \frac{r_2}{c} \bullet w + \pi = \pi + 2k\pi
\]
将 \( w = 2\pi / T = 2\pi \gamma \) 代入得:
\[
r_1 + r_2 = k\lambda
\]

这里我们不仅解释了干涉现象,而且明确给出了由干涉产生的明暗条纹的位置,只不过这个结论对于经典物理光学来说是个挑战。显然,我们的明暗条纹区与传统意义上的明暗条纹区不尽相同。例如,取波长为 \( \lambda = 5 \times 10^{-7} \, m \),双缝间距为 \( d = 2 \times 10^{-3} \, m \),缝屏间距 \( L = 1 \, m \),得到经典理论下的干涉条纹与我们模型下的干涉条纹对比情况如图 4.3.2 所示:

\begin{figure}[h]
    \centering
    \includegraphics[width=\textwidth]{image.png}
    \caption{模型所得杨氏双缝干涉条纹(左)与经典情况下(右)的对比}
    \label{fig:4.3.2}
\end{figure}

可见,若我们承认了经典物理光学的结论,那么这个基于北大学者猜想的模型就是有问题的,它在干涉条纹的明暗区域位置上产生了偏差 \cite{ref8},然而不可否认,它也定性定量地解释了双缝干涉条纹等间距的特性,仍有一定的研究价值。

\subsection{4.3.4 模型求解与分析}

下面我们将在此模型的基础上研究双缝干涉中两条缝之间的距离、缝的宽度与干涉光图样的光强空间分布形式的关系。

\textbf{情形 I:} 不考虑缝的宽度,即假设缝宽足够窄仅能容一个光子通过。

光的强度 I 是这样定义的,它等于单位时间内、在垂直于能流方向的单位面积上通过的能量的平均值。如图 4.3.1 的杨氏双缝干涉图所示,在 P 点相遇的可能是两个左旋光子,可能是两个右旋光子,也可能是一个左旋光子与一个右旋光子 \cite{ref9},这三种情形是等概率事件。若两个均为左旋光子,则
\begin{align*}
\vec{E}(0,0,r_1,t) &= E_p \left( \vec{e}_x + i \vec{e}_y \right) e^{-i(\omega t - kr_1)} \\
\vec{E}(0,0,r_2,t) &= E_p \left( \vec{e}_x + i \vec{e}_y \right) e^{-i(\omega t - kr_2)}
\end{align*}
叠加后得:
\begin{equation}
|E(P,1)|^2 = (E_p \sin kr_1 + E_p \sin kr_2)^2 + (E_p \cos kr_1 + E_p \cos kr_2)^2 \tag{式 4.3-4}
\end{equation}
若为两个右旋光子,叠加结果与上式类似,即
\begin{equation}
|E(P,2)|^2 = (E_p \sin kr_1 + E_p \sin kr_2)^2 + (E_p \cos kr_1 + E_p \cos kr_2)^2 \tag{式 4.3-5}
\end{equation}
若为一个左旋光子与一个右旋光子相遇,则
\begin{align*}
\vec{E}(0,0,r_1,t) &= E_p \left( \vec{e}_x + i \vec{e}_y \right) e^{-i(\omega t - kr_1)} \\
\vec{E}(0,0,r_2,t) &= E_p \left( \vec{e}_x - i \vec{e}_y \right) e^{-i(\omega t - kr_2)}
\end{align*}
\begin{equation}
|E(P,3)|^2 = (E_p \cos kr_1 - E_p \cos kr_2)^2 + (E_p \sin kr_1 + E_p \sin kr_2)^2 \tag{式 4.3-6}
\end{equation}

由物理知识可知,点 P 处的光强与电场分量振幅的平方成正比,而这三种情形又等概率发生,所以
\begin{align*}
I(P) &\propto \frac{1}{3} |E(P,1)|^2 + \frac{1}{3} |E(P,2)|^2 + \frac{1}{3} |E(P,3)|^2 = \frac{1}{3} \left( |E(P,1)|^2 + |E(P,2)|^2 + |E(P,3)|^2 \right) \\
&= \frac{2}{3} \left[ (E_p \sin kr_1 + E_p \sin kr_2)^2 + (E_p \cos kr_1 + E_p \cos kr_2)^2 \right] \\
&\quad + \frac{1}{3} \left[ (E_p \cos kr_1 - E_p \cos kr_2)^2 + (E_p \sin kr_1 + E_p \sin kr_2)^2 \right]
\end{align*}

考虑到粒子在屏幕上出现的概率与它到屏幕的距离有关,满足一定的概率分布,根据概率知识,粒子在空间的概率分布可近似看作满足正态分布。即:
\begin{align*}
I(P) &\propto \left\{ \frac{2}{3} \left[ (E_p \sin kr_1 + E_p \sin kr_2)^2 + (E_p \cos kr_1 + E_p \cos kr_2)^2 \right] \right. \\
&\quad \left. + \frac{1}{3} \left[ (E_p \cos kr_1 - E_p \cos kr_2)^2 + (E_p \sin kr_1 + E_p \sin kr_2)^2 \right] \right\} \cdot \frac{1}{2\pi} e^{-\frac{(r_1 - L)^2}{2}} \cdot \frac{1}{2\pi} e^{-\frac{(r_2 - L)^2}{2}}
\end{align*}

因为磁场能量与电场是对应的,所以只要将上式中的 \(E_p\) 换成 \(H_p\) 就可以得到光子在 P 点的磁场能量,两者相加即可得到光子在某点 P 处总的能量,那么只要考虑光子在 P 点相遇占据多大面积,以及光子到达 P 所需时间,就可以根据光强定义,求出 P 点的光强。根据附件 2,光子的半径可以近似为 \(r = \lambda / 4\),且 \((H_p^2 + E_p^2) 4\pi r^3 / 3 = mc^2 = h\gamma\),两

个光子在 P 点所占据的面积为 $S=2\pi r^{2}$,光子到达 P 点所用时间为 $t=\max(r_{1},r_{2})/c$。综上所述,我们可以得到 P 点处的光强公式为:
\[
I(P)=\frac{\frac{4}{3}[(\sin kr_{1}+\sin kr_{2})^{2}+(\cos kr_{1}+\cos kr_{2})^{2}]+\frac{2}{3}[(\cos kr_{1}-\cos kr_{2})^{2}+(\sin kr_{1}+\sin kr_{2})^{2}]}{(2\pi r^{2})\cdot(\frac{\max(r_{1},r_{2})}{c})}\cdot\frac{1}{2\pi}e^{-\frac{(r_{1}-L)^{2}}{2}}\cdot\frac{1}{2\pi}e^{-\frac{(r_{2}-L)^{2}}{2}}\cdot\frac{h\gamma}{\frac{4}{3}\pi r^{3}}
\]

取波长为 $\lambda=5\times10^{-7}m$,双缝间距为 $d=2\times10^{-3}m$,缝屏间距 $L=1m$,对该模型结果进行仿真,得到光强的大致分布如图 4.3.3 所示:

\begin{figure}[h]
    \centering
    \includegraphics[width=\textwidth]{image1.png}
    \caption{光子旋转模型所得光强分布图}
\end{figure}

由仿真结果可以看出,虽然由模型所得光强分布图在局部存在误差,但是它的大致走向与玻尔的结论是吻合的。从图上可以看出,在零点附近,很宽的区域光强很大,而光强的大致走向是 Sac 函数形状的,仿真图像与 Sac 函数已经很接近了。中间放大看时一些小起伏,其实就是玻尔函数里面描述的归一化干涉函数。

情形 II:考虑缝宽,并设缝宽为 $s$,示意图见图 4.3.4:

\begin{figure}[h]
    \centering
    \includegraphics[width=\textwidth]{image2.png}
    \caption{考虑缝宽时的双缝干涉示意图}
\end{figure}

已知波函数 \( E = E_{p}(\vec{e}_{x} + i\vec{e}_{y})e^{i(wt-kr)} \),进而可以写成
\[ E = E_{p}\cos(wt-kr) + iE_{p}\sin(wt-kr) \]。对 \( r \) 求导得
\[
\begin{aligned}
dE &= \frac{d[E_{p}\cos(wt-kr) + iE_{p}\sin(wt-kr)]}{dr}dr \\
&= [E_{p}k\sin(wt-kr) - iE_{p}k\cos(wt-kr)]dr \\
&= E_{p}k[\sin(wt-kr) - i\cos(wt-kr)]dr
\end{aligned}
\]
因为 \( r = \sqrt{L^{2} + (x-s)^{2}} \),所以
\[ dr = \frac{d\sqrt{L^{2} + (x-s)^{2}}}{ds}ds = [L^{2} + (x-s)^{2}]^{\frac{1}{2}}(s-x)ds \]
\[ dE = E_{p}k[\sin(wt-k\sqrt{L^{2} + (x-s)^{2}}) - i\cos(wt-k\sqrt{L^{2} + (x-s)^{2}})][L^{2} + (x-s)^{2}]^{\frac{1}{2}}(s-x)ds \]
最后,对 \( s \) 从 \( 0 \sim +\infty \) 积分就可以得到 \( E \) 关于 \( x \) 的函数,即可以得到各个点的场强公式,得到的 \( E \) 是个复数,它的模值即是场强大小。

下面我们就对单缝进行具体仿真研究,取波长为 \( \lambda = 5 \times 10^{-7}m \),双缝间距为 \( d = 2 \times 10^{-3}m \),缝屏间距 \( L = 1m \),缝宽 \( s = 2 \times 10^{-4} \)。仿真过程中,我们在某点 \( P \) 分别对 \( dE/ds \) 的实部和虚部积分,最后得到该点 \( P \) 处的场强平方,场强平方与光强成正比,进而就可得到各点的光强分布,仿真结果如下:

\begin{figure}[h]
\centering
\includegraphics[width=\textwidth]{image.png}
\caption{考虑缝宽时的单缝衍射光强分布图}
\end{figure}

图 4.3.5 是在考虑缝宽的情况下的单缝衍射光强分布图,与玻恩理论公式所得到的衍射 \( Sac \) 函数有很大的区别,我们大胆猜测,北京学者的猜想具有不太理想的适应性,有待于进一步的理论与实验完善。

\subsection{4.4 模型优化之光子蜂窝网络模型}
\subsubsection{4.4.0 模型假设}
1、我们认为光子是存在的,在我们的模型中,不考虑光子是否具有内在的结构,而是利用光子表现出来的特性,以及和其他光子的相互作用情况来解释题目中提到的三个问题。

\section*{2、光子之间存在着相互作用“力”,而且这些作用“力”具有方向性和大小。光子相互作用“力”大小与光的频率有关。}

3、光线光强大小、光子所产生的电场与磁场大小与方向,不仅与光子数目有关,还与光子之间作用力情况相关。

4、光子之间的作用“力”是相互的,也就是说不存在只受“力”不施加“力”的光子,也不存在只施加“力”而不受“力”的光子。当相互作用的两个光子之间的相互作用“力”同时存在,同时消失。

5、由于光子之间的相互作用,使得光子在与光线前进方向相垂直的平面上排列成蜂窝网络状结构,该结构保持稳定。在同一时刻处于同一平面(与光线前进方向相垂直的平面)上的光子主要受到周围6个光子的作用“力”影响,而外围光子的作用力影响可以忽略。

6、在同一时刻处于同一平面(与光线前进方向相垂直的平面)上的光子的相互作用力强,在衍射、干涉和偏振现象中,其主导作用;而同一时刻处于不同平面(与光线前进方向相垂直的平面)上的光子的相互作用力弱,在衍射、干涉和偏振现象中,作用可以忽略。

7、光子在外界作用下,会脱离蜂窝状网络。用时由于光子排列成的蜂窝网络状结构稳定,所以脱离网络的光子有重新组成和加入网络的趋势。

8、光子在光线前进方向上以螺旋线的方式前进,即各层蜂窝网络平面整体螺旋前进。

9、脱离网络的光子在与光线前进方向相垂直的平面空间中的运动带有随机性,在外界作用下,满足一定的概率分布。

\subsection*{4.4.1 模型的建立}

\textbf{A. 光子相互作用模型}

根据上述基本假设,建立光子模型。我们认为,光线光强大小、光子所产生的电场与磁场大小与方向,不仅与光子数目有关,还与光子之间作用“力”情况相关。所以在引入单个光子单元模型之前,有必要先讨论一下单个光子与周围光子相互作用的模型,如下图所示:

\begin{figure}[h]
\centering
\includegraphics[width=0.8\textwidth]{image.png}
\caption{单个光子与周围光子相互作用模型}
\label{fig:4.4.1}
\end{figure}

图正中间的光子是我们的考查对象,假设在同一时刻处于同一平面(与光线前进方向相垂直的平面)上的光子主要受到与它直接相邻的6个光子的作用“力”影响,而外围光子的作用“力”影响可以忽略。同时,我们再假设,两个光子的相互作用“力”只存在于有交接的边(或顶点)上,即对于中心光子周围的6个光子,每个光子只有2个作用“力”对中心光子有影响。因此,我们在图上将周围6个光子对中心光子的作用“力”用红色箭头表示。由假设4可知,中心光子对于周围6个光子也施加作用“力”,在图上用蓝色双箭头表示。

假设图中每一条标示箭头都代表相对作用“力”矢量,由于蜂窝网络的每一个单体都是正六边形,通过几何关系和矢量运算可知,中心光子受“力”平衡,处于稳定状态,这也是蜂窝网络整体稳定的微观力学解释。

\section*{B. 单个光子单元模型}

在引入了光子相互作用模型之后,我们可以进一步建立单个光子单元模型。由于我们的主要任务是尽量解释题目中提到的三个问题:第一个问题是光通常是按直线前进的,为什么光经过细缝后前进方向会发生改变?为什么必须是光通过很窄的细缝才会使前进方向发生改变?第二个问题是若干个光子同时作用的最终效果如何确定?是否仅取决于光子的个数?第三个问题是光通过偏振片的真实过程是什么?所以在我们建立的模型中,我们认为光子是存在的,但不考虑光子是否具有内在可分的结构。

根据上述基本假设,建立光子模型。由于光子之间存在着的相互作用,使得光子位于蜂窝网络的各个离散的中心位置。每个光子对于周围光子存在着相互之间呈60度的定向作用“力”,该“力”为指向图中正六边形各个顶点的背离本光子的斥“力”。

\begin{figure}[h]
\centering
\includegraphics[width=0.8\textwidth]{image.png}
\caption{单个光子单元模型}
\end{figure}

在相互排斥力作用下,每个光子形成自己光子单元的“领地”,即图中的正六边形区域,在此我们将其命名为“光子域”。在“光子域”的边界上,本光子和相邻光子的作用“力”达到制衡状态,形成类似化学键式的动态平衡结构,在此篇论文中,我们将其命名为“光子键”。从上述描述可以看出,当蜂窝状光子网结构形成时,相邻光子作用力达到平衡效果,我们称之为“光子键”的“生成”,而当在受到外部扰动时,相邻光子作用力的平衡效果被破坏,我们就称之为“光子键”的“断裂”。另外我们将相邻光子作用“力”,统一命名为“光子力”。

由经典的光学理论,光以及光的能量是电场和磁场的相互作用的结果。在我们的模型中,一个“光子域”中包含了6个“光子力”,分成两组,每组包含了互成120度的3个“光子力”。这两组“光子力”分别代表光子的电场作用和磁场作用,我们将其分别命名为:“光子电力”(图中红色单箭头所示)和“光子磁力”(图中蓝色双箭头所示)。

\section*{C. 光强模型}

我们认为,光强大小是由“光子键”所决定的,即光强不仅与蜂窝网络中的光子数目有关,还与光子之间作用力情况相关。

考虑单个的光子模型,当它和周围的光子均形成“光子键”时,我们认为该“光子域”所包含的光强为6个单位。而当光子周围发生光子缺失时,它们之间的“光子键”也被破坏,则该“光子域”中所包含的光强减少。设 \( n \) 为“光子域”周围缺失是的光子数,则该“光子域”所包含的光强为:

\[
I_{\text{photon}} = 6 - n \quad (n = 0, 1, 2, 3, 4, 5, 6)
\]

从上式我们可以看出,当 \( n = 6 \) 时,即 \( n \) 为“光子域”周围6个光子都缺失时,该光子域所包含的光强变为0。这时的光子仍然存在,但是由于它脱离了蜂窝网络,“光子键”均断裂,所以不包含光强。只有当它再次和网络中的其余光子形成新的“光子键”时,才会重新构成“光子域”,并再次包含光强。

\section*{D. 光子所产生的电场与磁场模型}

我们认为,光子所产生的电场与磁场大小与方向,是由各层的光子在光线前进方向上的螺旋运动,和蜂窝网络中的“光子电力”和“光子磁力”共同作用的结果。即一组旋转的“光子电力”形成电场,而一组旋转的“光子磁力”形成磁场。光子形成的电场和磁场相互正交成90度,并且和光线前进方向均正交。这和经典的电磁场理论相吻合,容易证明该模型中,光子所产生的电场与磁场满足麦克斯韦方程。

\section*{E. 光子的湮灭}

当两个光子在光线前进方向上重合时,如果两个光子的“光子电力”矢量相互重合或者相差 \( 2\pi/3 \) 的整数倍时,两个光子的“光子磁力”矢量也同时相互重合或者相差 \( 2\pi/3 \) 的整数倍时,此时产生的“合光强”为最大。

\begin{figure}[h]
    \centering
    \includegraphics[width=0.9\textwidth]{image.png}
    \caption{两个“光子电力”矢量相互重合(“光子磁力”矢量也重合)的光子叠加效果}
\end{figure}

当两个光子在光线前进方向上重合时,如果第一个光子的“光子电力”矢量与第二个光子的“光子磁力”矢量相互重合或者相差 \( 2\pi/3 \) 的整数倍时,第一个光子的“光子磁力”矢量与第二个光子的“光子电力”矢量也同时相互重合或者相差 \( 2\pi/3 \) 的整数倍时,此时产生的“合光强”为最小,此时光子湮灭(下图中将“光子域”用虚线表示,中心光子用空心圆圈表示,表明该光子湮灭)。

\begin{figure}[h]
    \centering
    \includegraphics[width=\textwidth]{image1.png}
    \caption{两个“光子电力”矢量与“光子磁力”矢量重合的光子叠加效果}
    \label{fig:4.4.4}
\end{figure}

\subsection{模型对于“三个问题”的解释}

\subsubsection{A. 对于衍射现象的解释}

当缝隙宽度远远大于单个“光子域”尺寸时,即同一层的光子蜂窝网络绝大部分可以轻易的通过缝隙时,由于“光子力”并没有被破坏,“光子键”未被打断,只有蜂窝网络边缘极少数的光子在狭缝的作用下脱离网络,所以光束通过很宽的缝隙时,前进方向不发生改变,并且由于只有光子蜂窝网络极其微量的光子受到的合“光子力”不平衡而发生偏转,所以在屏幕上几乎不出现衍射条纹。

\begin{figure}[h]
    \centering
    \includegraphics[width=\textwidth]{image2.png}
    \caption{光子蜂窝网络通过不同宽度狭缝示意图}
    \label{fig:4.4.5}
\end{figure}

当缝隙宽度逐渐减小到和单个“光子域”尺寸处于统一数量级时,即同一层的光子蜂窝网络绝大部分无法通过缝隙,此时由于大量的“光子力”被破坏,“光子键”被打断,极少数能够通过的狭缝的光子其周围光子可能“缺失”,造成光子受到的合“光子力”不平衡而发生偏转,所以光束通过很窄的缝隙时,前进方向发生改变。由于越靠近狭缝中央通过的光子,其周围光子缺失的可能性越小,受到的合“光子力”不平衡的概率越小,几乎不发生偏转,到达屏幕中央的光子的“光子键”最为完整,“光子域”包含的光强越强,所以屏幕中央光强最强。由于蜂窝网络中“光子域”的对称离散分布,造成在其周围光子“缺失”时,光子受力是对称且离散的,经过狭缝到屏幕的偏转运行,最终在屏幕上出现明暗相间,分布规律的衍射条纹。

\section*{B. 对于干涉现象的解释:}

当两列相干光束分别通过狭缝入射时,我们认为这两束光线包含的光子蜂窝网络中的光子的“光子电力”矢量方向一致或者相差 $2\pi/3$ 的整数倍时,则两个光子的“光子磁力”矢量也方向一致或者相差 $2\pi/3$ 的整数倍。由于两条狭缝是一样的,分别通过的光子蜂窝网络对称,所以两束光线通过狭缝的情况可以认为是一样的,即在狭缝作用下,光子缺失的概率,以及由于光子缺失造成剩余光子受到合“光子力”不平衡而发生偏转运动的情况都是一致的。因此从两条狭缝中通过的光子,经过螺旋线式的旋转运动到达光屏,如果两个相遇的光子的“光子电力”矢量方向一致或者相差 $2\pi/3$ 的整数倍,同时两个光子的“光子磁力”矢量也方向一致或者相差 $2\pi/3$ 的整数倍时,“合光强”达到最大值;反之如果第一个光子的“光子电力”矢量与第二个光子的“光子磁力”矢量相互重合或者相差 $2\pi/3$ 的整数倍时,第一个光子的“光子磁力”矢量与第二个光子的“光子电力”矢量也同时相互重合或者相差 $2\pi/3$ 的整数倍时,此时产生的“合光强”达到最小值,并且光子湮灭。由光子蜂窝网络对称性可以得到,最后在光屏上出现的是各级明纹中央等间隔的典型双缝干涉图样。

\section*{C. 对于多缝衍射现象的解释:}

因为多缝的衍射图样是干涉和衍射的共同作用结果,对于多缝的衍射情况(即光栅)由于各条狭缝尺寸完全一样,间距完全相等,再加上通过各条狭缝的光子网络均是对称相干的,因此可以看成是前述的推广形式,容易得到最后在光屏上出现的是典型的多缝衍射图样。

\section*{D. 对于偏振现象的解释:}

\begin{figure}[h]
    \centering
    \includegraphics[width=0.8\textwidth]{image.png}
    \caption{经过竖直偏正化方向的偏振片作用后,光子的四种可能状态}
    \label{fig:4.4.6}
\end{figure}

我们相信偏振化是这样一种作用,它倾向于使得每一个光子的三个“光子电力”矢量(上图中红色箭头)或三个“光子磁力”矢量(上图中蓝色双箭头)分别与偏振化方向成 45 度,165 度和 285 度。假设偏正化方向是竖直的,根据我们的模型假设,这样会存在着四种情况,如上图 4.4.6 所示。

我们认为,发生偏振时四种情况有且仅有两种会同时出现,即要么是 (a) (b) 同时出现,要么是 (c) (d) 同时出现。为了方便讨论,我们假设 (a) (b) 同时出现。从概率的角度,我们可以认为通过偏振片的作用之后,(a) (b) 两种被偏振的光子数量大约都占到 50%,当分别具有这样的两种光子的蜂窝网络在空间相遇时,由于此时它们的“光子电力”矢量夹角为 30 度,它们的“光子磁力”矢量夹角也为 30 度,此时“合光强” \( I \) 为入射光强 \( I_0 \) 的一半,即:

\[
I = \frac{1}{2} I_0
\tag{式4.4-2}
\]

假设另外有一个偏振片,其偏振化方向水平,根据我们的模型对于偏振化作用的解释,这样也会存在四种情况,如下图所示。发生偏振时四种情况有且仅有两种会同时出现,即要么是 (a) (b) 同时出现,要么是 (c) (d) 同时出现。为了方便讨论,我们假设 (c) (d) 同时出现。同理,此时“合光强” \( I \) 也为入射光强 \( I_0 \) 的一半。满足上式 (4.4-2)。

\begin{figure}[h]
    \centering
    \includegraphics[width=0.8\textwidth]{image1.png}
    \caption{(a) 和 (b) 的情况}
\end{figure}

\begin{figure}[h]
    \centering
    \includegraphics[width=0.8\textwidth]{image2.png}
    \caption{(c) 和 (d) 的情况}
\end{figure}

图 4.4.7 经过水平偏正化方向的偏振片作用后,光子的四种可能状态

现在将上述两个偏振化方向呈 90 度的偏振片“串联”放置,让光子网络依次通过,并且假设通过第一片偏振化方向是竖直的偏振片时,图 4.4.6 中的(a)(b)同时出现,通过第而片偏振化方向是水平的偏振片时,图 4.4.7 中的(c)(d)同时出现。图 4.4.6 中的(a)光子的“光子磁力”矢量和图 4.4.7 中的(d)光子的“光子电力”矢量相互重合;同时,图 4.4.6 中的(a)光子的“光子电力”矢量和图 4.4.7 中的(d)光子的“光子磁力”矢量相互重合,此时(a)(d)产生的“合光强”达到最小值,发生光子湮灭。同理(b)(c)光子也会发生湮灭,由于从概率学角度来说,(a)(b)(c)(d)四种光子产生的数目是相等的,所以通过两片偏振化方向相互垂直的串联偏振片后,“合光强”为 0。

若在一般情况下,光子蜂窝网络通过两片偏振化方向夹角为 $\theta$ 的串联偏振片时,此时,仍然会产生两对四种光子,他们的“光子电力”矢量和“光子磁力”矢量夹角为 $\theta$,叠加之后的“合光强”满足:

\[
I = I_0 \cos^2 \theta
\tag{式 4.4-3}
\]

可见我们建立的模型可以很好的解释“线偏振光起偏后,出射光强 $I$ 为入射光强 $I_0$ 的一半”以及“线偏振光以此通过两片偏振化方向相互垂直的串联偏振片后,出射光强 $I$ 为 0 的消光现象”,并且推广到一般情况下时,出射光强 $I$ 和入射光强 $I_0$ 之间满足经典的马吕斯定律。

\subsection*{4.4.3 模型定量分析}

A. 对于衍射现象的“光子力”分析:

\begin{figure}[h]
\centering
\includegraphics[width=0.8\textwidth]{image.png}
\caption{缺失一个周围光子时,中心光子受“光子力”的情况}
\end{figure}

我们考虑当缝隙宽度减小到和单个“光子域”尺寸处于统一数量级时,来考虑图中中心光子受“光子力”的情况。假设蓝色矩形为左侧狭缝,它首先打断了中心光子左侧的光子(横向阴影表示)的“光子键”,使得该光子从网络中缺失。

由于左侧缺失的光子对于中心光子的作用力如图中红色箭头所示,根据原先中心光子受力是平衡的,由作用力和反作用力的关系,那么现在剩余的 5 个周围光子对于中心光子的作用力为图中蓝色单箭头所示,我们可以求出它的合力,在图上用紫红色双向箭

头表示。我们假设每个“光子力”矢量的模值为 \( f \),那么根据矢量运算容易得到合力矢量的模值为 \( \sqrt{3} f \)。于是中心光子在该力的作用下向左侧偏转。

假设缝隙是有厚度的,那么随着中心光子的偏移,位于中心光子左上方和左下方的光子也会和缝隙发生相互作用,由于它们位置上的对称性,我们可以认为这两个光子同时缺失。

\begin{figure}[h]
    \centering
    \includegraphics[width=\textwidth]{image.png}
    \caption{缺失三个周围光子时,中心光子受“光子力”的情况}
    \label{fig:4.4.9}
\end{figure}

由于缺失的3个光子对于中心光子的作用力如图中红色箭头所示,根据原先中心光子受力是平衡的,由作用力和反作用力的关系,那么现在剩余的3个周围光子对于中心光子的作用力为图中4个蓝色单箭头所示,我们可以求出它的合力,在图上用2条紫红色双向箭头表示。我们假设每个“光子力”矢量的模值为 \( f \),那么根据矢量运算容易得到合力矢量的模值为 \( 2\sqrt{3} f \)。于是中心光子在该力的作用下进一步向左侧偏转。

\subsection{B. 衍射现象的光强公式和对于干涉现象的适用性}

根据上文对于光子力的分析,我们可以得出如下结论:1. 光子受到的不平衡“光子力”合力是离散的,且大小取值满足整数倍。2. 随着明纹级数的增加,到达该处的光子和光强大大减弱。3. 对于任意一级明纹,到达该明纹周围的光子数满足正态分布。因此我们给出单缝衍射的相对光强公式:

\begin{equation}
\frac{I}{I_0} = \frac{1}{\sqrt{2\pi\sigma}} \exp\left(-\frac{(p-u(0))^2}{2(\sigma/2)^2}\right) + \sum_{\substack{i=-\infty \\ i\neq 0}}^{+\infty} \frac{0.1}{2^{i-1}} \frac{1}{\sqrt{2\pi\sigma}} \exp\left(-\frac{(p-u(i))^2}{2(\sigma/2)^2}\right) \quad i \in \mathbb{Z} \tag{式 4.4-4}
\end{equation}

其中 \( u(0) = 0 \);\( u(i) = 2^i \cdot \frac{3}{5} \cdot \frac{\lambda}{s} \),\( i \in \mathbb{Z}^+ \);\( u(i) = -2^i \cdot \frac{3}{5} \cdot \frac{\lambda}{s} \),\( i \in \mathbb{Z}^- \)。

因为我们认为“光子电场”和“光子磁场”具有对称性,所以对于正态分布函数作出改进,将 \( e \) 的指数中分母 \( 2\sigma^2 \) 变为 \( 2(\sigma/2)^2 \)。

我们取波长 \( \lambda = 0.6 \times 10^{-7} \, \text{m} \),缝宽 \( s = 0.000001 \, \text{m} \),\( \sigma = 1/\sqrt{2\pi} \),分别利用我们给出的

\begin{figure}[h]
    \centering
    \includegraphics[width=\textwidth]{image1.png}
    \caption{相对光强曲线相对于 \( p \) 的分布}
    \label{fig:4.4.10}
\end{figure}

其中红色点线为波恩给出的相对光强公式,蓝色点线为我们给出的相对光强公式。

\begin{figure}[h]
    \centering
    \begin{minipage}{0.48\textwidth}
        \centering
        \includegraphics[width=\textwidth]{image2.png}
    \end{minipage}
    \hfill
    \begin{minipage}{0.48\textwidth}
        \centering
        \includegraphics[width=\textwidth]{image3.png}
    \end{minipage}
    \caption{左图为利用波恩给出的相对光强公式得出的衍射图样,右图为我们给出的相对光强公式得出的衍射图样。}
    \label{fig:4.4.11}
\end{figure}

\begin{figure}[h]
    \centering
    \begin{minipage}{0.48\textwidth}
        \centering
        \includegraphics[width=\textwidth]{image4.png}
    \end{minipage}
    \hfill
    \begin{minipage}{0.48\textwidth}
        \centering
        \includegraphics[width=\textwidth]{image5.png}
    \end{minipage}
    \caption{左图为利用波恩给出的相对光强公式得出的衍射图样的傅立叶变换频谱图\cite{ref10},右图}
    \label{fig:4.4.12}
\end{figure}

为我们给出的相对光强公式得出的衍射图样的傅立叶变换频谱图。

通过对比发现,我们的模型中央明纹宽度更窄,亮度更为集中一些,总体与波恩模型吻合较好。

最后,在我们引入的光子蜂窝网络模型中,我们认为双缝干涉现象是两个对称的单缝衍射满足一定电场和磁场矢量相位的叠加,所以在此不在给出具体的推导和证明。

\section*{五、模型的评价及推广}

本文紧密围绕题干提出的三个问题和两大关系,从现有的经典物理光学理论和专业实验结果出发,运用数学思维,综合光子理论,建立了基于光的波粒二象性一种猜想的数学模型。

针对龚祖同院士的光微子碰撞猜想 \({ }^{[11]}\),我们建立了基于光子碰撞后概率分布的模型,在最终模型的参数确定中,采用了最小二乘估计 RSS 准则,使得最终模型在满足与实验数据相吻合的前提下充分逼近玻恩理论公式。另外,模型的适应性很强,在模型提供的衍射公式的基础上稍作延伸便能得到与玻恩公式相匹配的干涉及多缝衍射公式,有力地证明了龚院士的猜想可靠性。

针对北大学者提出的光子作为电磁场自我旋转的猜想,我们采用双管齐下的办法,各自从专业证明和数学模型分析方面建立了电磁场偏转模型和光子旋转模型,一方面证明了北大学者提出的关于光子电磁场的诸多假设都是有意义的,另一方面也发现他提出的光子自旋猜想与玻恩公式有些许差别,有待于提升和完善。

最后,经过定性、定量分析两大猜想各自对于大量物理实验的优缺性,我们建立了自己的猜想模型——光子蜂窝网络模型。该模型引入了“光子域”、“光子电力”“光子磁力”、“光子键”等概念,从五个子模型出发,定性解释了四个光学现象,合理回答了题目提出的三大问题,并定量证明了衍射光强分布,使模型具有强大的说服力,可信度高。

当然,由于时间有限和专业知识的限制,我们的模型还有很多值得推敲和进一步改进的地方,但总体上已完成了题目本身的各项要求,并有所创新。

\section*{六、参考文献}

[1] 徐家恺,沈庆宏,阮雅端等,《通信原理教程》,北京:科学出版社,2007。

[2] Trevor Hastie, Robert Tibshirani, Jerome Friedman, 《The Elements of Statistical Learning》,北京:世界图书出版公司北京公司,2008。

[3] 陈宝林,《最优化理论与算法(第 2 版)》,北京:清华大学出版社,2005。

[4] 张德丰,《MATLAB 数值分析与应用(第 2 版)》,北京:国防工业出版社,2009。

[5] Max Born and Emil Wolf, 《Principles of Optics-7\(^{\text{th}}\) Edition》,北京:世界图书出版公司北京公司,2001。

[6] 张以漠,《应用光学》,北京:机械工业出版社,1987。

[7] D.Zu,《THE CLASSICAL STRUCTURE MODEL OF SINGLE PHOTON AND CLASSICAL POINT OF VIEW WITH REGARD TO WAVE-PARTICLE DUALITY OF PHOTON》,《Progress In Electromagnetics Research Letters》,Vol.1:109-118,2008。

[8] 倪苏敏,赵敏,《杨氏双缝干涉的误差研究》,《北京联合大学学报》,第 24 卷 1 期:73-76,2010。

[9] David Bohm, A Suggested Interpretation of the Quantum Theory in Terms of "Hidden" Variables, Physical Review,Volume 85, 166–179 (1952)

[10] 洪云,吴大健,王纪俊,《夫琅禾费衍射结果的计算机数值计算和显示》,《广西物理》,1003-7551(2003)02:13-16,2003。

[11] 龚祖同,《A THEORY OF THE PHOTON STRUCTURE》,《光子学报》,第 28 卷 1 期:1-17,1999。