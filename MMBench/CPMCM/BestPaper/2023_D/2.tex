\begin{center}
\includegraphics[width=0.2\textwidth]{image1.png} \quad
\includegraphics[width=0.2\textwidth]{image2.png} \quad
\includegraphics[width=0.2\textwidth]{image3.png} \quad
\includegraphics[width=0.2\textwidth]{image4.png} \quad
\includegraphics[width=0.2\textwidth]{image5.png}
\end{center}

\begin{center}
\textbf{中国研究生创新实践系列大赛} \\
\textbf{“华为杯”第二十届中国研究生} \\
\textbf{数学建模竞赛}
\end{center}

\begin{center}
\textbf{学 校} 东南大学
\end{center}

\begin{center}
\textbf{参赛队号} 23102860126
\end{center}

\begin{center}
\textbf{队员姓名} \\
1. 孙欣然 \\
2. 艾俊杰 \\
3. 石峰
\end{center}

\begin{center}
\textbf{中国研究生创新实践系列大赛}\\
\textbf{“华为杯”第二十届中国研究生}\\
\textbf{数学建模竞赛}
\end{center}

\bigskip

\textbf{题目:区域双碳目标与路径规划研究}

\bigskip

\section*{摘要:}

双碳政策的实施对我国的现代化建设具有深远的影响,能够能源效率提升,在减缓气候变化的同时为中国经济的可持续发展做出贡献。以我国东南沿海区域为例,结合区域经济、人口、能源消费量等相关因素,进行双碳目标的路径规划研究,有助于破解发展与碳减排之间的矛盾,具有较好的现实意义。

针对问题一,基于改进型 Kaya 模型的指标体系,构建描述指标内部及关于区域碳排放量、经济、人口、能耗等特征的平衡方程组,通过该指标体系进行十二五和十三五碳排放状况分析。通过皮尔逊相关性分析构造影响该区域碳排放量的特征集,再利用随机森林算法获得主要特征重要性,分析各部门对区域碳排放的贡献程度后,确定实现双碳目标的关键在于加强能源利用效率和提高非化石能源消费比例。计算相关指标的同比变化并比较各部门区域碳排放量以及经济、人口、能源消费量变化的折柱图。为构建指标关联模型,综合 CV、MIC、CatBoost 回归特征重要性分析得到各主要指标的综合性得分,获得信息量和贡献度最大的指标共计 11 项,再通过熵权法建立基于双碳政策及技术进步的碳排放影响因子,使用岭回归模型建立了区域经济、人口、能源消费量与碳排放量的关联,解决了特征指标间多重共线性的问题。

针对问题二,为构建与人口和经济相关联的能源消费量预测模型,查阅文献完善特征数据集,通过搭建 ARIMA-MLP 组合预测模型提取区域人口总量与 GDP 变化的趋势序列与残差序列,获得两者的预测值,再建立人口总量与 GDP 对于能源消费量的预测模型。其中,MLP 神经网络对非线性残差序列的处理能有效地提高能源消费量的预测精度。对于能源消费量的预测问题,通过分析二元多项式回归模型与 PCA 处理后的一元回归模型的预测结果,得到后者预测效果更好的结论。最后通过单位 GDP 能耗与单位 GDP 碳排放量的 OLS 回归算法预测区域碳排放量。基于问题一的指标体系,结合 SPSS 软件进行碳排放量与各部门能耗分布、非化石能耗比重之间的可视化分析。

针对问题三,首先通过查阅有关文献资料,提出能效提升、非化石能源消费比重等指标的合理假设,并结合双碳时间点,设计了自然情景、基准情景、雄心情景三种情景。接着基于上述各种情景及有关假设,建立了各情景下多指标的数学模型,再依据问题二中的预测模型及改进的 Kaya 模型,得到了多场景下碳排放量核算方法。最后选择在基准情景

下,通过情景设计时的有关假设及基准情景下的碳核算方法,确定了有关指标在各个双碳时间节点的目标值。又建立了基于四大重点工程的 STIRPAT 模型,实现了对四大重点工程的指标具体化,再通过 GA-MPSO 最优化算法对模型进行求解,从而对四大重点工程进行了定性和定量的分析,确定了双碳目标的最优路径。

本文对于所构建模型均进行了检验与评价,结合模型运行结果及算法的复杂程度,分析模型的优缺点,提出相对应的模型推广可行性及进一步改进的建议。

关键词:改进型 Kaya 模型;皮尔逊相关性分析;随机森林算法;CatBoost 回归;熵权法;ARIMA-MLP 组合预测模型;STIRPAT 模型;GA-MPSO 最优化算法

\section*{目录}

\begin{itemize}
    \item[1.] 问题重述 \dotfill 4
    \begin{itemize}
        \item[1.1] 问题背景 \dotfill 4
        \item[1.2] 问题提出 \dotfill 4
    \end{itemize}
    \item[2.] 问题分析 \dotfill 5
    \item[3.] 模型假设 \dotfill 7
    \item[4.] 符号说明 \dotfill 7
    \item[5.] 问题一 基于指标体系的模型建立与求解 \dotfill 9
    \begin{itemize}
        \item[5.1] 问题分析 \dotfill 9
        \item[5.2] 建立指标与指标体系 \dotfill 10
        \item[5.3] 区域碳排放量与经济、人口、能源消费量的现状分析 \dotfill 13
        \item[5.4] 指标的同比变化规律 \dotfill 19
        \item[5.5] 碳排放预测模型 \dotfill 27
    \end{itemize}
    \item[6.] 问题二 区域碳排放量以及经济、人口、能源消费量的预测 \dotfill 31
    \begin{itemize}
        \item[6.1] 问题分析 \dotfill 31
        \item[6.2] 指标分析 \dotfill 31
        \item[6.3] 基于 ARIMA-MLP 组合模型的人口和经济预测 \dotfill 33
        \item[6.4] 基于回归模型能源消费量预测 \dotfill 40
        \item[6.5] 碳排放量预测模型 \dotfill 44
        \item[6.6] 碳排放量的关联性分析 \dotfill 46
    \end{itemize}
    \item[7.] 问题三 区域双碳目标与路径规划 \dotfill 49
    \begin{itemize}
        \item[7.1] 问题分析 \dotfill 49
        \item[7.2] 基于双碳政策的情景设计 \dotfill 50
        \item[7.3] 多情景下碳排放量及相关指标的核算方法 \dotfill 51
        \item[7.4] 基于双碳情景下的各指标目标 \dotfill 54
        \item[7.5] 基于四大重点工程改进的 STIRPAT 模型 \dotfill 55
        \item[7.6] 基于遗传算法与分子运动论改进的粒子群的路径优化 \dotfill 56
    \end{itemize}
    \item[8.] 模型评价与改进 \dotfill 61
    \begin{itemize}
        \item[8.1] 模型的优点 \dotfill 61
        \item[8.2] 模型的缺点 \dotfill 61
        \item[8.3] 模型的改进与推广 \dotfill 61
    \end{itemize}
    \item[] 参考文献 \dotfill 62
    \item[] 附录 \dotfill 63
\end{itemize}

\section{问题重述}

\subsection{问题背景}

中国是世界上最大的温室气体排放国之一,其经济发展与双碳政策之间存在密切关系。经济快速增长伴随着对化石燃料的高度依赖,这导致了大量的二氧化碳排放。目前,气候变化已经引发了一系列的极端天气事件和环境问题,包括洪水、干旱、海平面上升以及生态系统的崩溃。在这一背景下,双碳政策的目标变得至关重要。

双碳政策是应对气候变化的重要举措,这一政策的背景源于全球气候变化对人类社会和生态系统构成的巨大威胁,以及国际社会在减排方面达成的共识,旨在减少温室气体排放,实现碳达标的碳中和的目标,以保护全球气候和促进可持续发展。

中国是全球最大的发展中国家,现已规划了在 2035 年基本实现现代化,并在 2050 年实现中国式现代化的经济社会发展目标,这也是中华民族伟大复兴的一部分。因此,实现 2060 年碳中和的目标需要解决发展和减排之间的矛盾。

碳中和目标是针对可持续性发展的核心政策之一,意味着未来的碳排放总量将不会超过自然界能够吸收的总量。这一目标可以通过减少碳排放和增加碳吸收的措施来实现,从而减缓全球气温上升的速度。此外,为了实现双碳政策的目标,中国正在积极推动能源结构的转型。这包括增加清洁能源(如风能、太阳能和核能)的比重,减少对煤炭等高碳能源的依赖。这将促使中国加快可再生能源和新能源技术的发展,同时减少温室气体排放。这有助于创造就业机会,提高竞争力,同时减少对有限资源的依赖。

为提高能源利用效率和非化石能源消费比重,我国鼓励绿色技术和创新,推动清洁能源、智能交通、节能建筑等领域的发展,同时为经济带来新的增长机会。深刻影响下一步我国产业链的重构、重组,同时对全球形成新的国际标准也有重要意义。通过探索碳排放量的变动与产业结构变化之间的互动影响机制,在此基础上模拟不同情境下实现“碳达峰”时产业结构特征,结合碳中和的目标为中长期产业结构的及时优化调整提供政策建议,实现经济低碳发展目标。此外,积极履行减排承诺和实施双碳政策有助于提升国家、地区或组织的国际声誉。这也有助于吸引国际投资和合作伙伴,推动可持续发展。

总之,双碳政策是应对气候变化的国际共识,在全球范围内具有重要的意义,对于中国的现代化建设具有深远的影响,能够推动能源结构转型、能源效率提升、碳市场建设和创新发展,在减缓气候变化的同时为中国经济的可持续发展做出贡献。

\subsection{问题提出}

要求利用区域数据结合数学建模方法解决如下问题:

\textbf{问题一:区域碳排放量以及经济、人口、能源消费量的现状分析}

建立指标与指标体系,要求指标能够描述某区域经济、人口、能源消费量和碳排放量的状况以及各部门(能源供应部门、工业消费部门、建筑消费部门、交通消费部门、居民生活消费、农林消费部门)的碳排放状况。指标体系需能够描述各主要指标之间的相互关系,部分指标的变化(同比或环比)可以成为碳排放量预测的基础。分析区域碳排放量以及经济、人口、能源消费量的现状,以 2010 年为基期,分析某区域的碳排放量状况,分析该区域实现双碳政策需要面对的主要问题,作为该区域双碳路径规划中差异化的路径的选择依据。建立区域碳排放量以及经济、人口、能源消费量各指标及其关联模型,建立各项指标间的关联,结合双碳政策与技术进步确定碳排放预测模型参数值。

\textbf{问题二:区域碳排放量以及经济、人口、能源消费量的预测模型}

基于人口和经济变化的能源消费量预测模型,以 2020 年为基期,结合中国式现代化的两个时间节点,预测某区域十四五至二十一五期间人口、经济和能源消费量变化。建立区域碳排放量预测模型,建立碳排放量与人口、GDP 和能源消费量的联系,要求碳排放量与各能源消费部门以及能源供应部门的能源消费量相关联,建立碳排放量与各能源消费部门的能源消费品种以及能源供应部门的能源消费品种相关联。

\textbf{问题三:区域双碳(碳达峰与碳中和)目标与路径规划方法}

设计不少于三种情景,要求与碳达峰和碳中和的时间节点、能效提升和非化石能源消费比重提升相关联。进行多情景下碳排放量的核算,基于基本假设使得区域碳排放与多情景假设、各部门碳排放量的总和相一致,碳排放量核算模型与问题二中预测模型相一致。最终确定双碳(碳达峰与碳中和)目标与路径,即确定 2025 年、2030 年、2035 年、2050 年和 2060 年 GDP、人口和能源消费量的目标值和能源利用效率、非化石能源消费比重的目标值,完成能效提升、产业(产品)升级、能源脱碳和能源消费电气化分析。

\section*{2. 问题分析}

对于问题一,要进行区域碳排放量以及经济、人口、能源消费量的现状分析,并要求建立指标与指标体系,且要求建立区域碳排放量以及经济、人口、能源消费量各指标及其关联模型。首先,根据扩展阅读给出的 Kaya 公式进行拓展来建立指标体系,根据题目给出的特征数据进行随机森林特征筛选与对数变化率分析。其次,对建立的指标进行同比的分析并结合 CV 评价、CatBoost 评分、MIC 评分方式建立指标关联模型,同时通过 Pearson 相关性分析。之后运用熵权法结合指标关联模型得到双碳政策的影响因子。最后通过双碳政策的影响因子得到基于改进型 OLS 的回归预测模型,建立了区域碳排放量以及经济、人口、能源消费量的关联模型,并进行了性能分析。

对于问题二,根据附件的人口、生产总值、能源消费量、碳排放量、碳排放因子等数据,对基于人口和经济变化的能源消费量与基于人口、GDP 和能源消费量的区域碳排放量进行建模。首先对附件中的数据进行筛选和整合,再搜集新的数据,组成新的数据集。其次使用新的数据集里的数据,利用 ARIMA-MLP 组合预测模型对人口、GDP 进行预测。之后,将新数据集里的人口、GDP 和能源消费量进行 PCA-线性回归与二元多项式回归,根据预测的人口、GDP 来预测能源消费量并进行算法的对比与分析。最后,将单位 GDP 的碳排放量与单位 GDP 的能源消费量进行线性回归,得到能源消费量与碳排放量的关系式来预测碳排放量。

对于问题三,要根据双碳节点和有关指标设计若干情景,并基于假设的情景,在给定的指标假设与要求下,确立多情景下的碳排放核算方法,最后确定双碳的目标与路径,完成四大重点工程的定性和定量分析。首先,面对不同的情景,需要通过对参数的合理假设,使其与双碳节点相关联。接着通过查阅文献资料,在建立不同情景下其他各个指标的数学模型后,基于改进后的 Kaya 模型得到多种情景下的碳排放模型。最后,基于上述所提出的各指标数学模型,得到双碳时间节点上各个指标的目标值,再通过 STIRPAT 模型将四大重点工程与实际的有关指标相对应,并通过 GA-MPSO 最优化算法进行模型求解,实现定性和定量分析,从而得到双碳路径。

全文技术线路如图 2.1 所示。

\begin{figure}[h]
    \centering
    \includegraphics[width=\textwidth]{image.png}
    \caption{全文技术路线图}
    \label{fig:technical-roadmap}
\end{figure}

\section{模型假设}

为了便于问题的研究,对题目中某些条件进行简化及合理的假设:
\begin{enumerate}
    \item 假设题目所给出的各变量数据全部真实可靠且有效;
    \item 假设各样本均存在偶然性,即不好的样本可以允许被剔除;
    \item 假设附件中的样本以及要预测的年份的样本测试条件与测试环境相同;
    \item 假设优化过程中各观测数据样本是独立的,不存在相互影响;
    \item 假设机器学习实验过程中计算机实验的环境不变;
    \item 假设碳排放和经济、人口相关指标的变化符合国家战略,能够实现十四五、二
    十五规划目标。
\end{enumerate}

\section{符号说明}

\begin{table}[h]
\centering
\caption{符号说明}
\begin{tabular}{l l l}
\hline
符号 & 意义 & 单位 \\
\hline
$P$ & 人口 & 万人 \\
$GDP$ & 区域生产总值 & 亿元 \\
$GDP_{i}$ & 第$i$产业生产总值 & 亿元 \\
$G$ & 人均区域生产总值 & 亿元/万人 \\
$E$ & 能源消费量 & 万tce \\
$E_{GDP}$ & 单位GDP能耗(能源消费强度) & 万tce/亿元 \\
$C$ & 碳排放量 & 万t$CO_{2}$ \\
$\eta_{C}$ & 碳效率 & 万t$CO_{2}$/亿元 \\
$B_{i}$ & 各部门与居民生活消费的能源消耗比重 & —— \\
$TH$ & 碳汇 & 万t$CO_{2}$ \\
$EF_{i}$ & 各部门与居民生活消费的碳排放因子 & t$CO_{2}$/tce \\
$CE$ & 单位能耗碳排放量 & t$CO_{2}$/tce \\
$E_{Non-fossil}$ & 非化石能源消费量 & 万tce \\
$E_{fossil}$ & 化石能源消费量 & 万tce \\
\end{tabular}
\end{table}

\begin{tabular}{l l l}
$Q$ & 非化石能源发电量 & 万tce \\
$W$ & 电力消费量 & 万tce \\
$NE$ & 新能源电力 & 万tce \\
$OE$ & 外来电力 & 万tce \\
$NH$ & 新能源热力 & 万tce \\
$YE_{i}$ & 第$i$个部门生产或消耗的电力 & 万tce \\
$YH_{i}$ & 第$i$个部门生产或消耗的热力 & 万tce \\
$R$ & 电力消费比重 & —— \\
$\eta_{Non-fossil}$ & 非化石能源消费比重 & —— \\
$\alpha_{policy}$ & 双碳政策影响因子 & —— \\
$\nu_{C}$ & 碳排放量增长率 & —— \\
$\rho_{e}$ & 非化石能源发电比重 & —— \\
$\eta_{E}$ & 能源效率(与$E_{GDP}$成反比) & —— \\
\end{tabular}

注:未列出符号及重复的符号以出现处为准。

\section{问题一 基于指标体系的模型建立与求解}

\subsection{问题分析}

结合双碳政策,为后续该区域未来能源消费量与碳排放量的预测提供依据,合理规划该区域双碳路径,本问题有两个任务主要需要解决,第一个任务是通过机理及现状分析,构建描述指标内部及涉及区域碳排放量、经济、人口、能耗等特征的指标体系。第二个任务是基于各特征的同比变化,进行可视化分析,定量描述与双碳政策及技术相关影响,构造指标建立评价体系,采用改进型的回归模型预测未来区域碳排放量。

先基于 Kaya 模型分析主要指标的变化趋势、构建描述区域经济、人口、能源消费量、碳排放量的指标体系,最终得到了改进的 Kaya 模型,构建描述指标内部及涉及区域碳排放量、经济、人口、能耗等特征的平衡方程。然后借用该模型进行十二五和十三五碳排放状况分析,分析模型误差。针对该区域对碳排放量产生影响的因素,先广泛收集特征数据,通过皮尔逊相关性分析得到该数据集中各因素之间的相关性以及与碳排放量的影响程度。最后考虑到因素数量多、高度非线性的特点,同时为描述各个部门对区域碳排放的贡献度,通过随机森林算法进行重要特征重要性排序,获得各特征因素对于区域碳排放量的贡献度,实现主要特征筛选,定量分析及比较各部门的碳排放的权重。

基于附件数据,可进行指标的同比变化分析,建立区域碳排放量以及经济、人口、能源消费量各指标的关联。关于碳排放指标,可以建立各产业及居民生活的碳排放总量与各产业碳排放量的可视图,分析碳效率与碳排放因子的同比变化;关于经济指标,进行各产业的 GDP 变化的可视化,分析人均 GDP 的同比变化;关于能源消费量指标,分别进行能源品种分布与各产业能耗的可视化分析,得到各产业及居民消费的能源消费量与能源效率的同比变化曲线,此外,考虑到能源的分类及运用占比对碳排放量会产生影响,需要进行非化石能源消费比重的可视化分析,得到非化石能源消费比重的同比变化。基于以上可视化分析与同比变化,定量分析各特征的变化相关性。

为建立具体的指标关联模型,需要先对问题进行简化。结合构造的新型指标,对原有指标和构造指标建立评价体系,得到各指标得分大小,得分越大的指标对于碳排放量的影响越大。再利用 Pearson 相关性分析进行指标间两两相关性检验,可得到指标的共线性程度。为充分可视化分析,选择构建各指标间的散点图矩阵,为是否降维处理提供判断依据。

双碳政策与技术进步势必对碳排放量产生影响,结合各主要指标建立双碳政策影响因子,该因子为二次指标,参与碳排放总量的拟合。由于时间序列较短,ARIMA 与 LSTM 模型的匹配性降低,若建立碳排放量的回归预测模型,对于共线性特征需要进行特殊处理,完成回归的可行性检验后,采用改进型的回归模型预测未来区域碳排放量。

\begin{figure}[h]
    \centering
    \includegraphics[width=\textwidth]{image.png}
    \caption{问题一流程}
    \label{fig:process}
\end{figure}

\subsection{建立指标与指标体系}

Kaya 模型是由 Yoichi Kaya 教授首先提出,常常用来分析区域碳排放量和该区域人口数量、生活水平、能源效率与碳排放因子的关系 \cite{ref1},其表达式为:

\begin{equation}
C = \frac{C}{E} \times \frac{E}{GDP} \times \frac{GDP}{P} \times P
\tag{5.1}
\end{equation}

\begin{equation}
CE = \frac{C}{E}, \ \eta_E = \frac{E}{GDP}, \ G = \frac{GDP}{P}
\tag{5.2}
\end{equation}

其中,$GDP$ 为区域生产总值,$P$ 为人口总数,$E$ 为能源消耗量,$C$ 为碳排放量,$CE$ 为单位能耗碳排放量,$\eta_E$ 为能源效率,$G$ 为人均区域生产总值。根据题目中所给出的数据,可以得到能耗总量、区域生产总值、人口、碳排放量在 2010~2020 年间的数据如图 5.2 所示。

\begin{figure}[h]
    \centering
    \includegraphics[width=\textwidth]{image.png}
    \caption{各指标变化柱状图}
    \label{fig:bar_chart}
\end{figure}

一般来说,区域社会发展水平的重要标志是人口规模 \( P \),人口数量较多的地区,其社会活力也会相应增强,社会发展水平也会相对较高。然而,随着人口增长,生活能耗也会呈现刚性增长的趋势;人均区域生产总值 \( G \) 是衡量区域经济发展水平的关键指标,当人均区域生产总值水平较高时,表明该地区的经济发展状况良好;能源效率 \( \eta_E \) 是判断区域能源消耗强度的重要标志。单位人均区域生产总值能耗较低的地区,其能源利用效率相应较高;二氧化碳排放量也与能源消费的碳排放因子 \( CE \) 密切相关。单位能耗二氧化碳排放量较低的地区,表明该地区能源消费中非化石能源消费比重较大,因此能源消费产生的温室气体排放量也较低。

因此我们选择人均区域生产总值 \( G \)、人口 \( P \)、区域生产总值 \( GDP \)、能源消费量 \( E \)、能源效率 \( \eta_E \),各部门与居民生活消费的碳排放因子 \( EF_i \),通过这些指标建立如图 \ref{fig:bar_chart} 所示的指标体系,部分指标的表达式如下。

\begin{equation}
\eta_E = \frac{E}{GDP} \tag{5.3}
\end{equation}

\begin{equation}
E = \sum_{i=1}^{6} E_i \tag{5.4}
\end{equation}

\begin{equation}
B_i = \frac{E_i}{E} \tag{5.5}
\end{equation}

\begin{equation}
C = \sum_{i=1}^{6} C_i \tag{5.6}
\end{equation}

\begin{equation}
EF_i = \frac{C_i}{E_i} \tag{5.7}
\end{equation}

\begin{equation}
GDP = \sum_{i=1}^{5} GDP_i \tag{5.8}
\end{equation}

其中,$B_{i}$ 为各部门与居民生活消费的能源消耗比重,$i$ 为消费部门、供应部门、居民生活的编号,对应方式见表 2。由题目可得能源供应部门产生的碳排放,将按照电力和热力的消费份额被折算至工业,建筑,交通,生活和农业等能源消费部门,即能源供应部门产生的碳排放量等效于能源消费部门的电力和热力乘以对应的电力碳排放因子与热力碳排放因子。因此,在本文中可以把能源供应部门产生的等效碳排放量 $C_{2}$ 的值视为 0。

\begin{table}[h]
\centering
\caption{项目序号对照表}
\begin{tabular}{c c c}
\hline
编号 & 产业化 & 项目名称 \\
\hline
1 & 第一产业 & 农林消费部门 \\
2 & 第二产业 & 能源供应部门 \\
3 & & 工业消费部门 \\
4 & 第三产业 & 交通消费部门 \\
5 & & 建筑消费部门 \\
6 & 居民生活 & 居民生活消费 \\
\hline
\end{tabular}
\end{table}

\begin{figure}[h]
\centering
\includegraphics[width=\textwidth]{image.png}
\caption{指标预测体系图}
\end{figure}

根据参考文献 [2],可以分别从生产与消费部门的碳排放量与居民生活消费的碳排放量两个角度对 Kaya 模型进行扩展,得到拓展的 Kaya 模型。依据生产与消费部门的碳排放量的情况,对原始的 Kaya 等式进行拓展,即将碳排放量依据能源种类、生产与消费部门进一步细分,考虑生产与消费部门碳排放量受到产业结构因素、能源结构因素和能源强度等因素的影响,并考虑居民生活消费的碳排放量变化的实际情况主要受到人均 GDP、消费支出费用的影响,对原始的 Kaya 等式进行变形,分析居民生活能源消费强度、居民生活消费占区域生产总值比重、经济发展水平和碳排放因子对居民生活消费的碳排放量的影响,得到拓展型 Kaya 等式如下所示:

\begin{equation}
D = D_{1} + D_{2}
\tag{5.9}
\end{equation}

其中,$D$ 表示整体的碳排放强度,$D_{1}$ 和 $D_{2}$ 分别表示各类部门(指生产与消费)和居民生活消费的碳排放强度。$D_{1}$ 与 $D_{2}$ 的表达式如下所示:

\begin{equation}
D_{1}=\frac{C^{p}}{GDP}=\sum_{i,j} \frac{C_{ij}^{p}}{GDP}=\sum_{i,j} \frac{C_{ij}^{p}}{E_{ij}^{p}} \times \frac{E_{ij}^{p}}{E_{i}^{p}} \times \frac{E_{i}^{p}}{GDP_{i}} \times \frac{GDP_{i}}{GDP}
\tag{5.10}
\end{equation}

\begin{equation}
D_{2}=\frac{C^{L}}{P}=\sum_{j} \frac{C_{j}^{L}}{P}=\sum_{j} \frac{C_{j}^{L}}{E_{j}^{L}} \times \frac{E_{j}^{L}}{E^{L}} \times \frac{E^{L}}{RC} \times \frac{RC}{GDP} \times \frac{GDP}{P}
\tag{5.11}
\end{equation}

其中,$C^{p}$ 和 $C^{L}$ 分别表示生产与消费部门和居民生活消费所产生的碳排放总量,$i(1,2,...,6)$ 为如表 2 所示编号,$j(j=1,2,3)$ 表示能源的类型(分别为煤碳、油品、天然气),$GDP$ 表示生产总值,$GDP_{i}$ 表示某个生产或消费部门 $i$ 的生产总值,$E_{i}^{p}$ 表示部门的能源消费量,$E_{ij}^{p}$ 表示生产或消费部门 $i$ 使用第 $j$ 类能源的消费量,$C_{ij}^{p}$ 表示生产或消费部门 $i$ 使用第 $j$ 类能源所产生的碳排放量,$C_{j}^{L}$ 表示居民生活消费第 $j$ 类能源所得到的碳排放量,$E^{L}$ 表示居民生活消费的能源总量,$E_{j}^{L}$ 表示居民生活消费第 $j$ 类能源的消费量,$RC$ 表示居民生活消费支出。在本文中我们只考虑碳排放量,可以得到如下式所示的改进型 Kaya 恒等式:

\begin{equation}
C=C^{p}+C^{L}=\sum_{i} B_{i} \times EF_{i} \times \eta_{E} \times G \times P
\tag{5.12}
\end{equation}

因此,结合上文建立的指标,可以得到基于 Kaya 模型的指标体系,定量描述各指标关联如下:

\begin{equation}
\left\{
\begin{aligned}
C &= C^{p}+C^{L}=\sum_{i} B_{i} \times EF_{i} \times \eta_{E} \times G \times P \\
C^{p} &= \sum_{i-1,j} C_{ij}^{p}=\sum_{i-1,j} \frac{C_{ij}^{p}}{E_{ij}^{p}} \times \frac{E_{ij}^{p}}{E_{i}^{p}} \times \frac{E_{i}^{p}}{GDP} \times GDP_{i} \\
C^{L} &= \sum_{j} C_{ij}^{L}=\sum_{j} \frac{C_{j}^{L}}{E_{j}^{L}} \times \frac{E_{j}^{L}}{E^{L}} \times \frac{E^{L}}{RC} \times \frac{RC}{GDP} \times GDP
\end{aligned}
\right.
\tag{5.13}
\end{equation}

\subsection{区域碳排放量与经济、人口、能源消费量的现状分析}

\subsubsection{十二五与十三五期间碳排放状况分析}

对上述所得到的改进的 Kaya 恒等式 (5.12) 左右两侧进行对数转换后再同时对时间 $t$ 求导,可以得到如下的等式:

\begin{equation}
\frac{\mathrm{d}(\ln C)}{\mathrm{d}t}=\frac{\mathrm{d}\left(\ln \left(\sum_{i+1} B_{i} \times EF_{i}\right)\right)}{\mathrm{d}t}+\frac{\mathrm{d}(\ln (\eta_{E}))}{\mathrm{d}t}+\frac{\mathrm{d}(\ln G)}{\mathrm{d}t}+\frac{\mathrm{d}(\ln P)}{\mathrm{d}t}
\tag{5.14}
\end{equation}

由上式可得,在一段的时间内,Kaya 恒等式右边的四个指标发生改变的百分比的总和应该与该时间段内的碳排放量改变的百分比近乎相等。右边四个指标改变的百分比表示的

\begin{table}
\centering
\begin{tabular}{c c c c c c c}
\hline
YEAR & $C\%$ & $CE\%$ & $\eta_{E}\%$ & $G\%$ & $P\%$ & Resid\% \\
\hline
2010 & & & & & & \\
\hline
\multicolumn{7}{c}{十二五(2011-2015年)} \\
\hline
2011 & 1.33\% & -5.2\% & -4.83\% & 5.14\% & 0.22\% & 6.00\% \\
2012 & 0.31\% & -6.53\% & 10.43\% & 4.90\% & 0.13\% & -8.62\% \\
2013 & -0.10\% & -13.72\% & 14.41\% & 4.58\% & 0.10\% & -5.46\% \\
2014 & -0.26\% & -15.71\% & 12.33\% & 3.75\% & 0.12\% & -0.74\% \\
2015 & 0.17\% & -18.34\% & 6.87\% & 3.95\% & 0.05\% & 7.64\% \\
\hline
\multicolumn{7}{c}{十三五(2016-2020年)} \\
\hline
2016 & 0.33\% & -13.68\% & 5.42\% & 3.25\% & 0.09\% & 5.25\% \\
2017 & 0.25\% & -3.66\% & 5.32\% & 3.03\% & 0.06\% & -4.50\% \\
2018 & 0.13\% & -14.74\% & 4.67\% & 2.83\% & 0.03\% & 7.35\% \\
2019 & 0.32\% & -2.21\% & 3.17\% & 2.40\% & 0.03\% & -3.07\% \\
2020 & -0.18\% & 0.04\% & 6.22\% & 1.51\% & 0.01\% & -7.96\% \\
\hline
\end{tabular}
\caption{各指标相对变化率统计表}
\end{table}

\begin{figure}[h]
\centering
\includegraphics[width=\textwidth]{image.png}
\caption{碳排放量变化量及各指标变化率}
\end{figure}

由表3数据以及图5.4曲线趋势分析可知:在十二五期间,碳排放量的相对变化率呈先正后负再正的态势,对应于碳排放总量即在这五年间碳排放总量上下波动,最终十二五开局之年2011年与十二五收官之年2015年时的碳排放总量变化不大,即节能减排政策在此时期发挥了一定的作用。在十三五期间,碳排放量的相对变化率呈先正后负的态势,对应于碳排放总量即在这五年间碳排放总量持续上涨,直到2020年开始回落,即我国在十三五期间的经济处于高速发展状态但未有相应完善的科学技术去改善碳排放量。

从表3及上图我们还可以发现在十二五与十三五的十年间:我国人口增长速度逐渐放缓,即我国的人口福利逐渐转化为人才福利。人均GDP增速从快速性增长转化为高质量增长态势。能源利用效率逐年稳重有升。单位能耗碳排放量逐年降低,向绿色发展方向靠拢。

\subsection*{5.3.2 特征选择及相关性分析}

首先合并所有可能影响到该区域碳排放量的特征参数数据集,包括自2010至2020年来该区域的人口数量、该区域的生产总值、人均生产总值、能源效率、该区域以及农林消费部门、工业消费部门、交通消费部门、建筑消费部门、能源供应部门各有关部门的碳排放量、能源消耗量、能源消耗比重、碳排放因子等数据。而由于特征参数的维数过高导致特征重要性分析较为困难,以及各个参数间可能存在某些相关性,故对特征参数进行了以下处理:

1) 由于各部门的能源消耗各部门与居民生活消费的碳排放因子 \( EF_i \)、各部门与居民生活消费的能源消耗比重 \( B_i \) 均可分别作为各部门及居民生活对碳排放产生影响的特征属性,故将各能源消费部门的这两个特征属性分别相乘合并为下式:
\begin{equation}
A_i = EF_i \cdot B_i
\tag{5.16}
\end{equation}

2) 由题意可知,能源供应部门在消耗能源的同时,其碳排放以按照电力与热力转换至各个能源消费部门及居民活动,故通过第二产业总能耗减去工业消费部门能耗得到能源消费部门的能耗 \( E \),再通过其余各能源消费部门以及居民活动所消耗的热力和电力,得到能源消费部门的等价碳排放 \( C \),最后以两者之比,即能源供应部门的碳排放因子 \( EF_2 \) 作为能源供应部门的特征属性。

3) 由Kaya模型的定义,本文以人口数量 \( P \) 作为区域人口的特征数据,以人均生产总值 \( G \) 为该区域的社会经济发展水平的特征数据,以能源消费强度 \( E_{GDP} \) 作为能源消费量的特征数据。

最终得到各因素的特征数据如表4所示:

\begin{table}[h]
\centering
\caption{各因素的特征数据}
\begin{tabular}{|c|c|c|c|c|}
\hline
农业部门特征 & 工业部门特征 & 交通部门特征 & 建筑部门特征 & 居民生活特征 \\
\hline
A1 & A3 & A4 & A5 & A6 \\
\hline
能源部门特征 & 人口特征 & 社会经济发展特征 & 能源消费量特征 & \\
\hline
EF2 & P & G & EGDP & \\
\hline
\end{tabular}
\end{table}

虽然此时已经筛选出特征,但某些重要特征之间可能存在较强的相关性,即可能存在冗余特征,此时便需要对特征间的相关性进行检测。

皮尔逊(Pearson)相关系数法是秩相关系数中的一种,是一种用于度量两个连续变量之间线性关系的统计方法。它的取值范围在-1到1之间,用来表示两个变量之间的关联强度和方向。“秩”即秩序,可以理解为一种顺序或排序[3],其具体定义如下:

\begin{equation}
r = \frac{\sum\limits_{i=1}^{n} \left( x_i - \overline{x} \right) \left( y_i - \overline{y} \right)}{\sqrt{\sum\limits_{i=1}^{n} \left( x_i - \overline{x} \right)^2 \sum\limits_{i=1}^{n} \left( y_i - \overline{y} \right)^2}}
\tag{5.17}
\end{equation}

其中 \(x_i\) 和 \(y_i\) 是两个变量在样本中的第 \(i\) 个观测值,\(\overline{x}\) 和 \(\overline{y}\) 分别是两个变量的样本均值,\(n\) 为样本数量。

当主要特征间的相关性较弱时,某种意义上任何主要变量均可独立地描述因变量地某方面性质,此时主要变量具有一定的独立性。对所选的几个与碳排放量的相关特征进行了相关性热力图可视化,如图 5.5 可见,变量间的独立性较好,满足特征独立性的要求。

\begin{figure}[h]
    \centering
    \includegraphics[width=\textwidth]{heatmap_image.png}
    \caption{各特征的相关性热力图}
    \label{fig:heatmap}
\end{figure}

\subsection{5.3.3 随机森林原理}

随机森林(Random Forest, RF) \cite{breiman2001random} 利用随机重采样技术 Bootstrap 以及节点随机分裂技术构造若干决策树(Decision Tree),通过投票得到最终分类结果,是近年来兴起的一种强大而又灵活的集成机器学习方法,目前已经在各种分类、预测、特征选择以及异常点检测问题中得到了广拓应用 \cite{ho1995random}. 它基于以下两种随机森林的随机化思想:

1) 特征子空间思想:在决策树的各个分支点产生分裂时,在所有特征中等概率地随机抽取一个子集,一般取特征总数即 \(\log_2(M) + 1\) 个属性,然后在该子集中令一个最优属性

使节点产生分支(本文设定为 \( M = 150 \))。

2) Bagging 思想:在原样本集 \( X \) 之中进行随机抽样,每次抽取 \( N \) 个与原样本容量相等的训练样本集 \(\{T_n, n = 1, 2, \ldots, N\}\),再利用各训练样本集 \( T_N \) 来构建相应的决策树(本文设定 \( N = 500 \))。

随机森林的训练本质即对每个决策树进行训练,而因为每个决策树的训练具有独立性,故随机森林的训练可通过并行计算从而极大地优化模型的生成效率。随机森林的训练流程如图 5.6 所示。

\begin{figure}[h]
\centering
\includegraphics[width=0.8\textwidth]{random_forest_training_flowchart.png}
\caption{随机森林训练流程图}
\end{figure}

而在本题之中,就特征参数数量众多、耦合较度高、高度非线性等特点,使用随机森林模型进行数学建模,以得到在所选择区域中对碳排放量产生影响的特征因素的贡献度。

\subsection*{5.3.4 基于随机森林的综合重要特征筛选}

随机森林算法不仅可以对样本集进行分类,还可以基于此进行回归分析。决策树的构建过程中,每次分裂都是基于某个特征和特征值的选择,在这个过程中,算法会根据不纯

度减少(例如,Gini 不纯度或信息增益)来选择最佳的特征进行分裂,与此同时,每个决策树都会记录特征的使用情况以及每个特征用于分裂的效果,这些信息在决策树的构建过程中积累,最终便可计算出特征的重要性,而特征值的大小也同对预测结果的影响成正比关系。在决策树中,计算某一个节点 \( k \) 的重要性:

\begin{equation}
n_k = w_k * G_k - w_{left} * G_{left} - w_{right} * G_{right}
\tag{5.18}
\end{equation}

其中,\( w_k \)、\( w_{left} \)、\( w_{right} \) 分别为分支点 \( k \) 与其左右分支点之中训练样本数目与总训练样本数目之比,\( G_k \)、\( G_{left} \)、\( G_{right} \) 分别为分支点 \( k \) 与其左右分支点的不纯度指标,包括基尼不纯度(Gini Impurity)和熵(Entropy)。在得到各节点的重要性之后,便可通过下式得出某一特征的重要性:

\begin{equation}
f = \frac{\sum\limits_{j \in node \ split \ on \ feature_i} n_j}{\sum\limits_{k \in all \ nodes} n_k}
\tag{5.19}
\end{equation}

通过上述随机森林模型对样本的特征进行特征重要性计算,便可得到对该区域碳排放量产生影响的各特征的贡献度如图 5.7 所示。

\begin{figure}[h]
\centering
\includegraphics[width=\textwidth]{image.png}
\caption{对碳排放量产生影响的各特征贡献度}
\end{figure}

由图可知,由模型所得的对碳排放量产生影响的特征重要性从大到小排序为:工业消费部门、能源供应部门、经济发展状况、建筑消费部门、能源消费强度、人口数量、居民生活消费、交通消费部门、农业消费部门。

将随机森林所得的特征重要性结果中贡献度大于 10\% 的特征因素,作为该地区实现双谈目标需要面对的主要挑战进行进一步分析:

- 工业消费部门与建筑消费部门作为两个主要的能源消费部门,其贡献度相加将近 \( 1/3 \),须提高能源消费电气化的水平,积极探索产业结构改革以提高电力消费的比重。能源供应部门作为各能源消费部门的电力热力消费来源,则要着眼于加大非化石能源发电占比,提高能源脱碳指标。人均 GDP 作为经济发展水平的度量之一,其与科学技术发展即产业(产品)升级密切相关,须积极探索人口福利到人才福利的转型方向。能源消费强度降低即能效提升,可以提升企业能源资源利用效率,降低能源成本。

\subsection{5.4 指标的同比变化规律}

环比和同比都是用于比较不同时间段内数据变化的方法,但它们关注的时间跨度不同。环比比较相邻时间段,而同比比较相同时间段的不同年份的数据\cite{ref6}。这两种比较方法可以帮助分析趋势、季节性变化和长期变化。

同比变化(YoY,Year-over-Year)是指将当前时间点的数据与去年同一时间点的数据进行比较,计算方式如下:
\begin{equation}
\text{同比变化} = \frac{\text{当前数据} - \text{去年同期数据}}{\text{去年同期数据}} \times 100\%
\tag{5.20}
\end{equation}

本节通过同比变化计算和 SPSS 绘图软件的可视化,分析与经济、人口、碳排放、能耗相关的各指标同比变化。

\subsubsection{5.4.1 碳排放量和碳效率的同比变化}

碳排放量是指伴随着能源消费而产生的二氧化碳排放量,主要与化石能源消费量相关,能源消费部门的碳排放既包含化石能源消费所产生的直接碳排放,也包含电力和热力消费所产生的间接碳排放。由图 5.8 可以看出,碳排放总量的同比变化较为平缓,总体上为负值,符合该区域碳排放量波动上升的趋势,且碳排放总量的同比变化与第二产业相似,这与折柱混合图的结果具有一致性。

\begin{figure}[h]
\centering
\includegraphics[width=\textwidth]{image.png}
\caption{碳排放总量与各产业碳排放变化折柱图}
\end{figure}

\begin{figure}[h]
    \centering
    \includegraphics[width=\textwidth]{image1.png}
    \caption{碳排放总量与各产业同比变化曲线}
    \label{fig:5.9}
\end{figure}

定义碳效率 $\eta_{C}$ 的计算方式:
\begin{equation}
\eta_{C} = C / GDP
\tag{5.21}
\end{equation}

与碳排放因子 $CE$ 的计算公式:
\begin{equation}
CE = C / E
\tag{5.22}
\end{equation}

得到了碳效率和碳排放因子的同比变化情况,二者在时间上呈现波动性。单位 GDP 的碳排放量近几年逐年下降,而单位能耗的碳排放呈现更小的波动趋势,二者的同比变化一定程度上说明随着科技发展,新型绿色节能产品的应用对双碳目标一定的促进作用。

\begin{figure}[h]
    \centering
    \begin{subfigure}[b]{0.45\textwidth}
        \includegraphics[width=\textwidth]{image2a.png}
        \caption{碳效率同比变化曲线}
        \label{fig:5.10a}
    \end{subfigure}
    \hfill
    \begin{subfigure}[b]{0.45\textwidth}
        \includegraphics[width=\textwidth]{image2b.png}
        \caption{碳排放因子同比变化曲线}
        \label{fig:5.10b}
    \end{subfigure}
    \caption{碳相关同比变化图}
    \label{fig:5.10}
\end{figure}

\subsection{5.4.2 经济同比变化}

如题干所述,人均 GDP 是区域经济发展水平的重要标志,人均 GDP 水平高,表明该

区域经济发展状况好。因此,除生产总值 GDP,本文还利用人均 GDP 来表征该区域的经济发展水平。

由折线图可知,生产总值和人均 GDP 均呈现同比变化值降低的特征,说明 GDP 正在保持增长的同时增长速度基本上在逐年放缓,说明今年来的经济发展趋于稳定,由高速度迈向高质量,这是一个经济转型的重要过程,视为该东南部地区经济发展的必然阶段。这一转型的目标是实现更为可持续、稳定和有利于全体公民的经济增长。

\begin{figure}[h]
    \centering
    \includegraphics[width=\textwidth]{image1.png}
    \caption{生产总值与各产业 GDP 同比变化曲线}
    \label{fig:5.11}
\end{figure}

\begin{figure}[h]
    \centering
    \begin{subfigure}[t]{0.45\textwidth}
        \centering
        \includegraphics[width=\textwidth]{image2a.png}
        \caption{人均 GDP 同比变化}
        \label{fig:5.12a}
    \end{subfigure}
    \hfill
    \begin{subfigure}[t]{0.45\textwidth}
        \centering
        \includegraphics[width=\textwidth]{image2b.png}
        \caption{人口同比变化}
        \label{fig:5.12b}
    \end{subfigure}
    \caption{人均 GDP 同比和人口同比变化曲线}
    \label{fig:5.12}
\end{figure}

可以认为,虽然人口增速减缓,但持续性的人口扩增使得该区域的人口规模不断扩大,更多的人口需要能源来满足他们的生活和经济活动需求,直接导致了能源的消费呈现直线上升,即生产总值平稳上升,但同时人均 GDP 增速下降,人口的增加随之碳排放量就越多。

\subsection{5.4.3 能源指标同比变化}

能源消费量即为能耗,其定义从品种分布看,包含化石能源消费(有碳排放)和非化石能源消费(无碳排放)。单位 GDP 能耗(又称为能源消费强度)是区域能源利用效率的重要标志,单位 GDP 能耗低,则能源利用效率高。单位能耗二氧化碳排放(又称为能源消费的碳排放因子)是区域能源消费低碳排放的重要标志,单位能耗二氧化碳排放量低,表示能源消费中非化石能源消费比重大,能源消费产生的温室气体(主要为二氧化碳)排放低。

\begin{figure}[h]
    \centering
    \includegraphics[width=\textwidth]{image1.png}
    \caption{能源消费量同比变化}
    \label{fig:energy_consumption}
\end{figure}

能源利用效率与单位 GDP 能耗成反比,因此,本文用单位 GDP 能耗代表能源利用效率 $\eta_{E}$。

\begin{figure}[h]
    \centering
    \includegraphics[width=\textwidth]{image2.png}
    \caption{能源效率 $\eta_{E}$ 同比变化曲线}
    \label{fig:energy_efficiency}
\end{figure}

非化石能源消费比重的计算式:

\begin{equation}
\eta_{Non-fossil} = E_{Non-fossil} / E
\tag{5.23}
\end{equation}

煤炭、油品、天然气、热力、电力、其他能源为不同的能源大类。其中煤炭、油品、其他能源分别包括了众多不同的能源小类,如煤炭包括原煤、洗精煤、焦炭等;油品包括原油、柴油、汽油等;其他能源包括生物质能、氢能等清洁能源,其碳排放因子视为 0。

根据能耗品种结构,非化石能源包含新能源热力、新能源电力、外地调入电、其他新能源四个方面,即:

\begin{equation}
E = E_{NET} + E_{NEE} + E_{EPI} + E_{ONE}
\tag{5.24}
\end{equation}

\begin{equation}
E_{fossil} = E_{coal} + E_{oil} + E_{gas}
\end{equation}

其中,$E_{NET}$、$E_{NEE}$、$E_{EPI}$、$E_{ONE}$ 分别表示新能源热力、新能源电力、外地调入电以

及其他新能源,$E_{coal}$、$E_{oil}$、$E_{gas}$分别表示煤、石油、天然气传统能源。

从图 5.15 中可以看出非化石能源消费比重的同比变化量平均为正数,这与碳排放总量同比变化的规律具有一致性,结合人均 GDP 均呈现同比变化值降低的特征,说明今年来的经济发展趋于稳定,由高速度迈向高质量,这是一个经济转型的重要过程,一方面,该区域经济得到了稳定的发展,同时由于技术发展,部分传统能源行业可能受到影响,因为减少对化石燃料的需求可能导致产业调整和失业问题。新兴能源的使用有助于能源效率的改进,减少温室气体排放、提高能源可持续性的同时降低能源供应风险。

\begin{figure}[h]
    \centering
    \includegraphics[width=\textwidth]{image.png}
    \caption{非化石能源消费比重 $\eta_{Non-fossil}$ 同比变化曲线}
    \label{fig:5.15}
\end{figure}

\subsection{5.4.4 指标关联模型}

为了充分考虑关联性强的指标,除题目给定指标外,需要结合指标的物理性质构造新的指标,通过综合性打分的方式,筛选出信息量和贡献度最大的指标。通过第一节的指标体系构建,得到了完善的体系内指标结果如表 5。

\begin{figure}[h]
    \centering
    \includegraphics[width=\textwidth]{image2.png}
    \caption{构建指标关联流程}
    \label{fig:5.16}
\end{figure}

由附件数据,获得各指标 2010-2020 年的统计量。其中,单位 GDP 能耗可用于表征能源利用效率(反比),同比变化即为指标增长率。关于产能耗量和居民生活耗量,做各部门和居民消费的非化石能耗比重及各能源消费品种的发热发电比重的简化分析。对各个指标的进行可视化验证其设计的合理性,结合题意具体需要结合自身发散程度及与预测碳排放量的相关性。可见得指标总量过高,我们简化整体指标体系,得到了表 7 所示的除碳排放量外的 10 项指标。

\begin{table}
\centering
\caption{给定指标与构造指标}
\begin{tabular}{l l l}
\hline
指标来源 & 指标名称 & 单位 \\
\hline
\multirow{6}{*}{原有值} & 经济总量 & 亿元 \\
& 能源消费量 & 万tce \\
& 碳排放量 & 万tCO2 \\
& 人口 & 万人 \\
& 各部门和居民生活能源消费量 & 万tce \\
& 各部门生产总值 & 亿元 \\
& 各部门和居民生活碳排放量 & 万tCO2 \\
\hline
\multirow{9}{*}{构造值} & 碳排放量的同比 & —— \\
& 单位GDP能耗 & 万tce/亿元 \\
& 单位GDP能耗同比 & —— \\
& 人均GDP & 亿元 \\
& 碳效率 & 万tCO2/亿元 \\
& 各部门和居民生活平均碳排放因子 & tCO2/tce \\
& 各部门碳效率 & —— \\
& 非化石能源消费量 & 万tce \\
& 非化石能源消费比重 & —— \\
& 各部门和居民消费的非化石能耗比重 & —— \\
\hline
\end{tabular}
\end{table}

过滤式指标评价是一种特征打分排名的常见方式,在相关性分析的过程中不涉及模型的学习。常见的方式包括皮尔逊相关系数 (PCC, Pearson Correlation Coefficient)、最大信息系数 (Maximal Information Coefficient, MIC)、T检验 (t-test) 等,结果质量基于统计学的计算分析,独立考察各个指标的关联度。

针对特征自身的发散程度的评价,我们采用了变异系数 (CV, Coefficient of Variation) 指标来完成。CV 的定义方式是数据 \(X\) 的标准差 \(\sigma_X\) 和均值 \(\mu_X\) 的商,是数据离散程度的一种非常有效的归一化量度方式 \({}^{[7]}\)。

最大信息系数 (MIC, Maximal Information Coefficient) 是一种用于度量两个变量之间关系强度的统计方法,尤其适用于探索非线性关系 \({}^{[8]}\)。MIC 评分是 MIC 方法生成的一个度量,表示两个变量之间的相关性。MIC 评分的计算过程是通过比较原始数据和随机重新排列数据(随机噪声)的情况来进行的,取值范围通常在 0 到 1 之间。

\begin{table}
\centering
\caption{指标评价模型组成}
\begin{tabular}{l l}
\hline
MIC & 最大信息系数 (Maximal Information Coefficient) \\
CV & 特征变异系数 (Coefficient of Variation) \\
CatBoost & 梯度提升算法 (Categorical Boosting) \\
PCC & 皮尔森相关系数 (Pearson Correlation Coefficient) \\
\hline
\end{tabular}
\end{table}

综合 CV 评价、碳排放量的 CatBoost 评分、MIC 评分方式获得各指标的得分结果如表 7,各评价得分的绝对值和缩放为 100。

\begin{table}
\centering
\caption{主要指标在各个评价体系下的得分大小}
\begin{tabular}{l c c c}
\hline
指标 & MIC评分 & CatBoost评分 & CV评分 \\
\hline
能耗总量 & 9.67 & 17.7 & 2.06 \\
人口 & 9.67 & 24.2 & 0.58 \\
生产总值GDP & 9.67 & 3.7 & 5.93 \\
碳排放因子 & 4.71 & 3.2 & 0.58 \\
碳排放量增长率 & 2.49 & 3.7 & 43.86 \\
人均GDP & 9.67 & 10.2 & 5.43 \\
碳效率 & 15.04 & 2.7 & 4.80 \\
单位GDP能耗 & 15.04 & 10.6 & 4.24 \\
能源利用效率提升率 & 4.71 & 5 & 14.68 \\
非化石能源消费量 & 9.67 & 9.7 & 9.76 \\
非化石能源消费比重 & 9.67 & 9.3 & 8.07 \\
\hline
\end{tabular}
\end{table}

关于指标的关联性,获得三类评分后,需要对各个打分机制进行加权评价,加权总和为1。考虑到CV评分针对指标内部的离散度,与其他指标的关联性较弱,因此对其定义了0.2的权重。由于数据量偏低,导致MIC评分结果离散性较差,对其定义0.3的权重,则CatBoost回归评分的权重最高。

\begin{equation}
S = \alpha X_{MIC} + \beta Y_{Cat} + \delta Z_{CV}
\tag{5.25}
\end{equation}

使用公式(5.24)对3种评价方式的得分加权计算综合关联度,并依据相关性高低给出的特征排名记录在表8中。

\begin{table}
\centering
\caption{不同指标在各个评价体系下的排名高低}
\begin{tabular}{l c c}
\hline
排名 & 指标 & 关联度 \\
\hline
1 & 单位GDP能耗 & 35.38 \\
2 & 碳效率 & 33.29 \\
3 & 碳排放量增长率 & 28.02 \\
4 & 非化石能源消费量 & 27.13 \\
5 & 人口总量 & 26.89 \\
6 & 非化石能源消费比重 & 26.165 \\
7 & 能耗总量 & 25.68 \\
8 & 人均GDP(GDP/P) & 25.115 \\
9 & 生产总值GDP & 23.415 \\
10 & 能源利用效率提升率 & 18.26 \\
11 & 碳排放因子 & 10.67 \\
\hline
\end{tabular}
\end{table}

分析指标各个评价方案得分以及关联度可以看到,各指标的CV得分相差较大,其中碳排放因子和人口的离散程度不足1%,碳排放因子的MIC打分和CatBoost打分结果同样较低,因此它的综合得分是最低的,在该指标评价体系中的影响最弱。排在最前面的是单位GDP能耗、碳效率、碳排放量增长率和非化石能源消费量,即相较于能耗总量,单位GDP的能耗影响更强,接下来人口总量和非化石能源消费比重的影响也很重要。

完成指标打分后,再利用皮尔逊相关性分析对指标间之间是否存在统计上的显著性关系进行检验,判断P值是否呈现显著性(P<0.05)。若呈现显著性,则说明两变量之间存在相关性,反之,则两变量之间不存在相关性。对运用Pearson系数分析相关系数的正负向以及相关性程度结果进行可视化分析,由图5.17可知各指标具有强相关性,进一步分析各

\begin{figure}[h]
    \centering
    \includegraphics[width=\textwidth]{heatmap_image.png}
    \caption{图 5.17 Pearson 相关性分析热力图}
\end{figure}

\begin{figure}[h]
    \centering
    \includegraphics[width=\textwidth]{scatter_matrix_image.png}
    \caption{图 5.18 指标散点图矩阵}
\end{figure}

通过分析,基本确定了指标的关联程度大小以及相互之间的相关性,通过 SPSS 软件的可视化处理确定了正向或者负向的关联性,体系中的大部分指标均能实现线性耦合。

\section{5.5 碳排放预测模型}

\subsection{5.5.1 基于双碳政策及技术进步的碳排放影响因子}

双碳政策是一种旨在减少碳排放的政策框架,通常包括两个主要目标:碳达峰和碳中和[9]。碳达峰目标即减少温室气体排放,特别是二氧化碳(CO2)的排放。碳中和目标也就是减少的碳排放可以被自然或技术手段完全吸收或抵消,这通常涉及到增加碳汇或采用碳捕获和储存技术。碳中和是指碳排放量与碳汇(生态碳汇+工程碳汇+碳交易)消纳量相平衡,碳中和的要求是 \(\delta \text{CO2}<<0\),在《意见》中要求非化石能源消费比重 \(>80\%\)。

双碳政策旨在通过限制化石燃料的使用、提高能源效率、促进清洁能源等措施,减少工业、能源生产、交通等领域的碳排放。这可以降低碳排放水平,有助于应对气候变化。该政策鼓励技术创新:为实现碳中和目标,双碳政策通常会鼓励技术创新和发展,以开发更清洁的能源技术和碳捕获技术。这有助于推动清洁能源产业的发展。

但与此同时,双碳政策可能对一些传统高碳产业带来挑战,但同时也为新兴的清洁技术和产业创造了机会。政府通常会采取政策来支持那些受到影响的行业,以平稳过渡。

为了探究双碳政策以及技术创新对该区域碳排放量的影响,定义双碳政策影响因子 \(\alpha_{policy}\),因此,需要利用 \(\alpha_{policy}\) 与各个指标的相关性调节回归模型中的部分自变量。

该因子对于多项指标具体的正负相关性如图 5.19 所示,由题干可知,单位 GDP 能耗与能源利用效率成强负相关,且技术创新能够促进非化石能源的消费比重,同时,我们认为,对于我国东南部地区,传统行业的 GDP 可代表该地区整体的 GDP 水平,因此,为确定双碳政策影响因子的取值,我们通过熵权法进行了关于双碳政策影响指标的权重分析,考虑了碳效率、非化石能源消费比重、非化石能源消费量、碳排放量增长率等指标。

\begin{figure}[h]
    \centering
    \includegraphics[width=0.8\textwidth]{image.png}
    \caption{双碳政策体系}
    \label{fig:5.19}
\end{figure}

因此,\(\alpha_{policy}\) 的线性表示形式如下:

\begin{equation}
\alpha_{policy} = \sum_{i=1}^{7} \xi_{i} x_{i}
\tag{5.26}
\end{equation}

其中,\(\xi_{i}\) 为第 \(i\) 项指标的权重,\(x_{i}\) 为第 \(i\) 项指标的数值。2022 年,中国的排放量相对平稳,仍然为 102 亿吨左右,下降了 2300 万吨,下降了 \(0.2\%\)。与双碳政策体系因子为同一数量级,表明了该因子的合理性。

\begin{table}
\centering
\caption{熵权法特征权重}
\begin{tabular}{c c c c c}
\hline
相关性 & 指标项 & 信息熵值 & 信息效用值 & 权重$\xi(\%)$ \\
\hline
\multirow{3}{*}{正相关} & 碳效率 & 0.856 & 0.144 & 17.456 \\
 & 非化石能源消费比重 & 0.869 & 0.131 & 15.84 \\
 & 非化石能源消费量 & 0.843 & 0.157 & 19.01 \\
\hline
\multirow{4}{*}{负相关} & 能耗总量 & 0.875 & 0.125 & 15.179 \\
 & 生产总值$GDP$ & 0.886 & 0.114 & 13.826 \\
 & 碳排放量增长率 & 0.955 & 0.045 & 5.43 \\
 & 单位$GDP$能耗 & 0.89 & 0.11 & 13.259 \\
\hline
\end{tabular}
\end{table}

\subsection{5.5.2 基于改进型 OLS 的回归预测模型}

LSTM 模型、ARIMA 模型、灰度预测模型均可以进行时间序列的预测,观察题中表格数据,可以发现数据量偏少,上述预测模型通常需要大量历史数据才能产生准确的预测,为了提高模型的泛化能力的同时防止过拟合,现结合前文碳排放相关指标和影响因子,运用多元线性回归模型进行未来碳排量的预测。

一般情况下,成功的线性回归模型需要保证以下几点:因变量为连续型变量、自变量和因变量在理论上有因果关系、残差要满足正态性、独立性、方差齐性,多个自变量之间不存在多重共线性。多元线性回归扩展了线性回归,允许考虑多个自变量对因变量的影响。

不失一般性,多元线性回归模型的方程:
\begin{equation}
y = \alpha_1 x_1 + \alpha_2 x_2 + \ldots + \alpha_3 x_3 + b
\tag{5.27}
\end{equation}
其中,$\alpha_i$ 表示特征权重,$x_i$ 表示自变量,$y$ 表示因变量,$b$ 为偏置。

\begin{figure}[h]
\centering
\includegraphics[width=0.8\textwidth]{image.png}
\caption{回归模型标准化残差的 P-P 图}
\end{figure}

由 P-P 图可知,原始数据与正态分布的不存在显著的差异,残差满足线性模型的前提要求。

由图 5.18 可知该回归问题的自变量之间具有强相关性,进一步分析各种指标,由图 5.20 可知这些自变量为高度共线性数据,在多重共线性情况下,普通最小二乘线性回归的系数估计可能会导致奇异矩阵(Singular Matrix)问题,因此运用改进型 OLS 多元线性回归模型建立区域碳排放量与经济、人口、能源相关联的回归模型,即岭回归模型。

岭回归是用于共线性数据分析的有偏估计回归方法,实质上是一种改良的最小二乘估计法,通过放弃最小二乘法的无偏性,以损失部分信息、降低精度为代价获得回归系数更为符合实际、更可靠的回归方法,对病态数据的拟合要强于最小二乘法,该模型鲁棒性强,泛化能力强。

具体上,岭回归引入了正则化参数 \(\lambda\) 用于控制正则化的强度,通常通过交叉验证来选择最合适的 \(\lambda\) 值,以确保模型的性能达到最佳。较大的 \(\lambda\) 值会导致更强的正则化,使得系数趋向于零,从而减小多重共线性的影响。

一般情况下,OLS 求解回归问题的最小化目标为:
\[
\hat{\varphi} = \arg \min_{\beta} \sum_{i=1}^{N} \left( y_i - \varphi_0 - \sum_{j=1}^{m} \varphi_m x_i \right)^2
\tag{5.28}
\]

岭回归即在目标函数中增加一个惩罚项:
\[
\overline{\varphi^{penalty}} = \arg \min_{\beta} \left\{ \sum_{i=1}^{N} \left( y_i - \varphi_0 - \sum_{j=1}^{m} \varphi_m x_i \right)^2 + \lambda \sum_{j=1}^{m} \varphi_j^2 \right\}
\tag{5.29}
\]

在回归分析的时候还需要检验模型的残差图,模型残差图是以指定变量为横坐标、残差为纵坐标的图形,回归分析后需要使用残差图检验模型,因为使用最小二乘法进行回归分析涉及回归模型的随机性和不可预测性。

\begin{figure}[h]
\centering
\includegraphics[width=\textwidth]{image.png}
\caption{回归残差图}
\end{figure}

如果某个特征的方差比其他特征大几个数量级,则它可能会支配目标函数,另外,梯度下降法对自变量的尺度敏感,标准化处理能够加速模型的收敛,提高模型的训练效率。因此需要进行指标的 \(z\)-score 标准化处理,减少不同自变量间的尺度差异,确保各个自变量对回归模型的影响在相同尺度下进行比较。通过 SPSS 软件进行 Ridge 回归拟合模型的拟合。

\begin{table}[h]
\centering
\begin{tabular}{|c|c|c|c|c|c|}
\hline
人口 & 生产总值 & 能耗总量 & 碳排放量增长率 & 双碳影响因子 & 单位GDP能耗 \\
\hline
1106.14 & 650.67 & 2921.11 & 525.72 & 505.20 & 633.36 \\
\hline
碳效率 & 人均GDP & 能源利用效率提升率 & 非化石能源消费量 & 非化石能源消费比重 & 偏置 \\
\hline
1037.72 & 640.06 & 561.96 & 669.52 & 174.60 & 67631.16 \\
\hline
\end{tabular}
\caption{特征系数打分表}
\end{table}

注:$R^2$为0.963

\begin{equation}
\begin{split}
y = 1106.14P + 650.67GDP + 2921.11E + 505.20\alpha_{policy} + 633.36E_{GDP} \\
+ 1037.72\eta_c + 640.06G + 561.96\frac{\partial\eta_E}{\partial t} + 669.52E_{Non-fossil} \\
+ 174.60\eta_{Non-fossil} + 67631.16
\end{split}
\tag{5.30}
\end{equation}

\begin{figure}[h]
\centering
\includegraphics[width=\textwidth]{image.png}
\caption{岭回归拟合图}
\end{figure}

\section{问题二 区域碳排放量以及经济、人口、能源消费量的预测}

\subsection{问题分析}

本问题需要完成基于人口和经济变化的区域能源消费量预测和与人口、经济、能源消费相关的区域碳排放量预测。

为构建与人口和经济相关联的能源消费量预测模型,需要先进行人口与 GDP 的预测。由于附件中各特征的序列较短,先通过查找资料完善数据集,再通过预测模型完成人口与 GDP 的预测。二者的预测是通过 ARIMA-MLP 组合模型得到的。已知这两种模型均可用于数据的预测,两者的结合能够完善各自的局限性。其中,使用 ARIMA 模型处理线性部分的预测,使用 MLP 神经网络模型处理非线性部分的残差预测,最后将两部分预测值相加以提高该区域未来几十年的 GDP 和人口数量预测效果,旨在提高预测精度并更有效地捕捉数据的特征。

能源消费量的预测通过回归模型实现。为了得到尽量更合适的回归预测模型,考虑到自变量的共线性问题,将二元多项式回归模型预测结果与降维后的一元回归模型预测结果相比较,对比二者未来同一年中对于非化石能源能耗的预测结果,还可实现两种回归预测模型输出单位 GDP 能源消费量与《意见》中的单位 GDP 能耗预测结果比较。

\begin{figure}[h]
\centering
\includegraphics[width=\textwidth]{image.png}
\caption{问题二流程图}
\end{figure}

\subsection{指标分析}

采用 2020 年年底中国分性别、分年龄人口数作为预测基年数据,通过 PADIS-INT 人口预测软件,采用联合国经济社会事务部开发的贝叶斯分层概率预测方法,依据 1949 年以来中国的总和生育率实际水平对总和生育率进行概率预测,可以得到全国人口预测结果 [10]。该预测考察了 80\% 和 95\% 两种不同预测区间内中国人口的发展变动趋势,共产生了 5 个预测方案,其中方案具有 80\% 预测区间下限和 80\% 预测区间上限。

根据此预测结果,可以看出在全国总人口数大约在 2021 年达到峰值,之后呈缓慢下降趋势。考虑到本赛题涉及区域位于中国东南沿海,具有地势平坦、水陆交通便利、经济发达、科教资源丰富等特征,因此具有吸收外来人口的优势,可以认为未来该区域会吸引

外来人口,因此假定区域的人口数呈现缓慢上升的趋势。

人口、生产总值、能源消费量之间互成正的强相关性,因此,可以实现人口和生产总值预测能源消费量的目标。本文采用一种组合式的预测手段,即基于 ARIMA-MLP 模型的人口和生产总值预测与能耗回归,回归模型考虑了二元多项式拟合与 PCA 降维后的一元线性拟合的比较。

\begin{figure}[h]
    \centering
    \includegraphics[width=0.8\textwidth]{image1.png}
    \caption{人口和生产总值的线性关系}
    \label{fig:6.2}
\end{figure}

\begin{figure}[h]
    \centering
    \includegraphics[width=0.8\textwidth]{image2.png}
    \caption{人口和能源消费量的线性关系}
    \label{fig:6.3}
\end{figure}

\begin{figure}[h]
    \centering
    \includegraphics[width=0.8\textwidth]{image3.png}
    \caption{生产总值和能源消费量的线性关系}
    \label{fig:6.4}
\end{figure}

另外,经过中国东南部的区域间数据的结合与比较,基本可以确认该区域为中国江苏省。因此,可以完成关于人口和经济总量的数据补充,补充后得到该区域 1991-2020 的总人口和 1981-2020 年的 GDP 数据,作为 ARIMA-MLP 模型的输入特征,得到未来的特征预测值。

\section{基于 ARIMA-MLP 组合模型的人口和经济预测}

ARIMA 模型和 MLP 神经网络模型均可用于数据的预测 \cite{ref11},二者效果有所不同。ARIMA 模型表现出较好的准确性,相对误差一般低于 1\%,而 MLP 神经网络模型的预测更加不稳定,相对误差高达 5\%,过分迭代容易造成过拟合。为了充分利用两种模型的优势,我们提出了一种新的组合模型,使用 ARIMA 模型处理线性部分的预测,使用 MLP 神经网络模型处理非线性部分的残差预测,最后将两部分预测值相加以提高该区域未来几十年的 GDP 和人口数量预测效果,旨在提高预测精度并更有效地捕捉数据的特征。

\subsection{ARIMA 模型介绍}

ARIMA(Autoregressive Integrated Moving Average),即自回归积分滑动平均模型,是一种常用的时间序列分析和预测方法。它结合了自回归(AR)、差分(I)和滑动平均(MA)三个模型而组成。ARIMA 模型的主要思想是通过对时间序列的自回归、差分和滑动平均进行建模,来提取时间序列的趋势序列以及捕捉序列的趋势性、季节性等性质。

自回归部分 AR,AR 模型利用过去时间点的观测值来预测当前观测值,它建立在时间序列当前值与过去值之间的线性关系上。AR(p)模型的定义式如式 (6.1)。

\begin{equation}
X_t = \varphi_1 X_{t-1} + \varphi_2 X_{t-2} + \ldots + \varphi_p X_{t-p} + \varepsilon_t
\tag{6.1}
\end{equation}

差分部分 I,该部分模型用于消除序列的非平稳性。验证序列是否平稳,常用到 ADF(Augmented Dickey-Fuller)平稳性检验,而 ADF 检验的基本思想是检查时间序列数据是否具有单位根,即是否存在单整(非平稳)过程。若不平稳,则对其逐次差分,使其转化为平稳序列。当 d=0 时,ARIMA 模型可视为 ARMA 模型。

滑动平均部分 MA,MA 模型利用过去时间点的预测误差(残差)来预测当前观测值,它建立在时间序列当前值与过去预测误差(残差)之间的线性关系上。MA 模型假设当前值与过去的预测误差之间存在相关性,通过加权求和这些预测误差来进行预测。MA 模型的阶数(q)表示模型中包含的预测误差的个数。例如,MA(1) 模型表示使用过去一个预测误差来进行预测,MA(2) 模型表示使用过去两个预测误差来进行预测,以此类推。MA(q) 模型的定义式如式 (6.2):

\begin{equation}
X_t = \varepsilon_t + \theta_1 \varepsilon_{t-1} + \theta_2 \varepsilon_{t-2} + \cdots + \theta_p \varepsilon_{t-p}
\tag{6.2}
\end{equation}

其中,$\varepsilon_{t-1}, \varepsilon_{t-2}, \ldots, \varepsilon_{t-p}$ 是过去的预测误差,$\theta_1, \theta_2, \ldots, \theta_p$ 是 MA 过程的滑动平均系数,表示当前观测值与过去 q 个预测误差之间的权重,$\varepsilon_t$ 是当前时间点的随机误差项,通常被假设为白噪声。

由上述 AR(p),I(d),MA(q) 模型可得 ARIMA(p, d, q) 模型的定义式:

\begin{equation}
X_t = c + \varphi_1 X_{t-1} + \varphi_2 X_{t-2} + \cdots + \varphi_p X_{t-p} + \theta_1 \varepsilon_{t-1} + \theta_2 \varepsilon_{t-2} + \cdots + \theta_p \varepsilon_{t-p} + \varepsilon_t
\tag{6.3}
\end{equation}

其中,$c$ 为常数,$X_t$ 为原时间序列经 d 阶差分后平稳的时间序列在时间点 t 的观测值,

$X_{t-1}, X_{t-2}, \ldots X_{t-p}$ 为过去的观测值,$\varepsilon_{t}$ 是当前时间点的随机误差项,通常被假设为白噪声,$\varepsilon_{t-1}, \varepsilon_{t-2}, \ldots, \varepsilon_{t-p}$ 是过去的预测误差,$\varphi_{1}, \varphi_{2}, \ldots, \varphi_{p}$ 是 AR 过程的自回归系数,$\theta_{1}, \theta_{2}, \ldots, \theta_{p}$ 是 MA 过程的滑动平均系数。

得到 ARIMA 模型的定义式后,利用 ARIMA 模型提取时间序列的趋势序列步骤如下:

1) 首先进行数据的预处理,以确保时间序列的平稳性。而在本文综合异常检测算法的先前步骤里的的平稳性检验与差分序列提取中,已确认了平稳差分序列的阶数 $d$。

2) 观察自相关函数 (ACF) 和偏自相关函数 (PACF),确定 ARIMA 模型阶数。如果 ACF 图在滞后期之后截尾,并且在截尾之前有明显的峰值,$p$ 值可以取峰值前的滞后期数量。如果 PACF 图在滞后期之后截尾,并且在截尾之前有明显的峰值,$q$ 值可以取峰值前的滞后期数量。在实际应用中,可能需要结合其他方法进行验证和调整,例如使用信息准则 (AIC、BIC) 进行模型选择或进行网格搜索确定阶数。

3) 拟合 ARIMA 模型,根据确定的阶数,使用 ARIMA 模型拟合时间序列数据便得到了趋势序列 $tX$。

在提取出时间序列的趋势序列后,可以将公式 (5-15) 改写为 (5-19) 式,从而便能得到时间序列的残差序列:

\begin{equation}
rX = X - tX
\tag{6.4}
\end{equation}

\begin{figure}[h]
\centering
\includegraphics[width=0.8\textwidth]{arima_flowchart.png}
\caption{ARIMA 预测模型流程图}
\end{figure}

\subsection{6.3.2 MLP 模型介绍}

MLP 神经网络,即反向传播神经网络 (Backpropagation Neural Network),是一种广泛应用于机器学习和人工智能领域的神经网络模型 \cite{mlp_ref}. 它模拟了人脑神经元之间的信息传递和学习过程,具有强大的自适应能力和适用性。MLP 的结构如下图所示,主要由三部分

组成,至少要有一个隐藏层。

输入节点(Input Nodes):输入节点从外部世界提供信息,总称为输入层。在输入节点中,不进行任何的计算,仅向隐藏节点传递信息。

隐藏节点(Hidden Nodes):隐藏节点和外部世界没有直接联系。这些节点进行计算,并将信息从输入节点传递到输出节点。隐藏节点总称为隐藏层。尽管一个前馈神经网络只有一个输入层和一个输出层,但网络里可以没有隐藏层,也可以有多个隐藏层。

输出节点(Output Nodes):输出节点总称为输出层,负责计算,并从网络向外部世界传递信息。

\begin{figure}[h]
\centering
\includegraphics[width=0.8\textwidth]{image.png}
\caption{图 6.6 MLP 神经网络结构示意图}
\end{figure}

从上图中可以看到 MLP 神经网络主要有三个基本元素:权重 W、偏置 b 和激活函数。神经元之间的连接强度由权重表示,权重大小表示可能性的大小,用于调节该输入对输出影响的大小。偏置的设置是为了正确分类样本,是模型中一个重要参数,即保证通过输入得到的输出值不能随便激活。激活函数起到非线性映射的作用,其可将神经元的输出幅度限制在一定范围内。

\subsection*{6.3.3 ARIMA 模型的建立}

由题中所给数据并结合网络查阅资料可知,本赛题所涉及的东南沿海区域为我国江苏省。通过国家统计局中国统计年鉴 [13],收集 1991 年第八个五年计划开局之年至 2020 年十三五收官之年 30 年间江苏省人口以及 GDP 数据,补充赛题所给数据,最终统计记于表 11。

\begin{table}[h]
\centering
\caption{表 11 1991-2020 年区域人口与 GDP 数据统计}
\begin{tabular}{c c c | c c c | c c c}
\hline
Year & 人口 & GDP & Year & 人口 & GDP & Year & 人口 & GDP \\
\hline
1991 & 6843.7 & 1557.3 & 2001 & 7358.5 & 9442.0 & 2011 & 8023.0 & 45952.7 \\
1992 & 6911.2 & 1955.8 & 2002 & 7405.5 & 10521.0 & 2012 & 8120.0 & 50660.2 \\
1993 & 6967.3 & 2342.9 & 2003 & 7457.7 & 11954.3 & 2013 & 8192.0 & 55580.1 \\
1994 & 7020.5 & 2728.7 & 2004 & 7523.0 & 13636.4 & 2014 & 8281.0 & 60359.4 \\
1995 & 7066.0 & 3149.0 & 2005 & 7588.2 & 18121.3 & 2015 & 8315.0 & 65552.0 \\
1996 & 7110.2 & 3534.0 & 2006 & 7655.7 & 20828.4 & 2016 & 8381.0 & 70665.7 \\
1997 & 7147.9 & 3956.5 & 2007 & 7723.1 & 23940.4 & 2017 & 8423.0 & 75752.2 \\
1998 & 7182.5 & 4392.4 & 2008 & 7762.5 & 26980.4 & 2018 & 8446.0 & 80827.7 \\
1999 & 7213.1 & 4835.7 & 2009 & 7810.3 & 30339.4 & 2019 & 8469.0 & 85556.1 \\
2000 & 7327.2 & 8553.7 & 2010 & 7869.3 & 41383.9 & 2020 & 8477.3 & 88683.2 \\
\hline
\end{tabular}
\end{table}

注:人口单位为万人,GDP 单位为万亿元。

以表 7 统计的 30 年人口及 GDP 的数据为基础,通过 ADF 平稳性检验并差分处理提取出平稳的有关时间序列,绘制平稳化后的时间序列的 ACF 图和 PACF 图。图 6.7 从左至右从上至下,依次为人口时间序列的 ACF 与 PACF 图,以及 GDP 时间序列的 ACF 和 PACF 图。

\begin{figure}[h]
    \centering
    \begin{subfigure}[t]{0.45\textwidth}
        \centering
        \includegraphics[width=\textwidth]{acf_population.png}
        \caption{人口序列的 ACF 相关性图}
    \end{subfigure}
    \hfill
    \begin{subfigure}[t]{0.45\textwidth}
        \centering
        \includegraphics[width=\textwidth]{pacf_population.png}
        \caption{人口序列的 PACF 相关性图}
    \end{subfigure}
    \hfill
    \begin{subfigure}[t]{0.45\textwidth}
        \centering
        \includegraphics[width=\textwidth]{acf_gdp.png}
        \caption{GDP 序列的 ACF 相关性图}
    \end{subfigure}
    \hfill
    \begin{subfigure}[t]{0.45\textwidth}
        \centering
        \includegraphics[width=\textwidth]{pacf_gdp.png}
        \caption{GDP 序列的 PACF 相关性图}
    \end{subfigure}
    \caption{ACF 与 PACF 相关性图}
    \label{fig:acf_pacf}
\end{figure}

由 ADF 平稳性检验提取差分序列以及观察 ACF 与 PACF 图,最终可确认 ARIMA 模型的参数。人口为 ARIMA(3,1,1) 模型,GDP 为 ARIMA(2,2,1) 模型。进行白噪声检验,发现人口和 GDP 的 P 值均小于 0.05,均为非白噪声序列,具有一定的自回归性,即可通过自回归模型进行处理。

\begin{equation}
\left(1-\sum_{i=1}^{3}\varphi_{P_{i}}L^{i}\right)(1-L)P_{t}=c_{P}+(1+\theta_{P}L)\varepsilon_{P_{t}}
\tag{6.5}
\end{equation}

\begin{equation}
\left(1-\sum_{i=1}^{2}\varphi_{GDP_{i}}L^{i}\right)(1-L)^{2}GDP_{t}=c_{GDP}+(1+\theta_{GDP}L)\varepsilon_{GDP_{t}}
\tag{6.6}
\end{equation}

其中,$P_{t}$、$GDP_{t}$ 是时间序列的当前观测值,$L$ 是滞后算子,$L^{d}$ 表示 $d$ 次差分,$\varphi$ 为自回归系数,$\theta$ 为滑动平均阶数,$c$ 是模型的截距项,$\varepsilon$ 是白噪声的误差项。

\subsection{6.3.4 建立 ARIMA-MLP 组合模型}

对时间序列进行相关性检测后,即进行时间序列的趋势序列与残差序列的提取。按照时间序列的生成原理,任何时间序列均可被分解为趋势序列 $tX$ 以及残差序列 $rX$。时间序

\section*{列的趋势序列是指时间序列数据中长期的、持续的变化模式或趋势,可以反映出序列的季节性、周期性等性质。残差序列可以认为是噪声序列,反映着外部环境对时序序列的干扰、冲击等异常影响。根据线性叠加定理,任意时间序列 \( X \) 可以表示为:}

\begin{equation}
X = tX + rX
\tag{6.7}
\end{equation}

ARIMA-MLP 组合模型依据趋势-残差原理,将 ARIMA 模型的预测值作为新的组合模型的趋势部分,同时通过该预测值计算人口和经济总量的逐年残差。再通过构建滑动窗口的 MLP 神经网络模型来识别残差序列,时间窗口一次向前滑动一个时间步长以生成一个训练样本,将生成训练集输入矩阵,训练完模型后将模型的输出作为该区域 GDP 和人口 P 时间序列的非线性部分的预测。最后将 ARIMA 模型的趋势和 MLP 模型的残差两部分所得的预测值相加,作为后续多元回归模型的输入,该模型流程图如下:

\begin{figure}[h]
\centering
\includegraphics[width=\textwidth]{arima_mlp_flowchart.png}
\caption{ARIMA-MLP 组合模型流程图}
\end{figure}

根据 1991~2020 年的人口数据建立 ARIMA 模型,得到的模型残差序列如表 12 所示;根据 1981~2020 年的 GDP 数据建立 ARIMA 模型,得到的模型残差序列如表 13 所示。

\textbf{表 12 ARIMA 人口模型残差序列(单位:万人)}

\begin{table}[h]
\centering
\begin{tabular}{|c|c|c|c|c|c|}
\hline
年份 & 残差 & 年份 & 残差 & 年份 & 残差 \\
\hline
1991 & 19.84 & 2001 & 54.42 & 2011 & 88.59 \\
\hline
1992 & 58.73 & 2002 & 45.13 & 2012 & 87.18 \\
\hline
1993 & 55.36 & 2003 & 49.74 & 2013 & 69.71 \\
\hline
1994 & 52.31 & 2004 & 55.5 & 2014 & 71.52 \\
\hline
1995 & 48.57 & 2005 & 58.21 & 2015 & 56.2 \\
\hline
1996 & 46.37 & 2006 & 59.37 & 2016 & 57.28 \\
\hline
1997 & 43.27 & 2007 & 60.15 & 2017 & 54.7 \\
\hline
1998 & 40.45 & 2008 & 50.58 & 2018 & 43.94 \\
\hline
1999 & 37.89 & 2009 & 48.42 & 2019 & 40.16 \\
\hline
2000 & 68.74 & 2010 & 53.71 & 2020 & 34.69 \\
\hline
\end{tabular}
\end{table}

\begin{table}
\centering
\caption{表13 ARIMA生产总值模型残差序列(单位:亿元)}
\begin{tabular}{|c|c|c|c|c|c|c|c|}
\hline
年份 & 残差 & 年份 & 残差 & 年份 & 残差 & 年份 & 残差 \\
\hline
1981 & 187.5 & 1991 & 181.84 & 2001 & 1299.87 & 2011 & 5359.16 \\
\hline
1982 & 110.69 & 1992 & 279.81 & 2002 & 1497.76 & 2012 & 5787.81 \\
\hline
1983 & 102.79 & 1993 & 301.13 & 2003 & 1374.03 & 2013 & 5305.47 \\
\hline
1984 & 80.61 & 1994 & 344.19 & 2004 & 1436.73 & 2014 & 5049.77 \\
\hline
1985 & 78.89 & 1995 & 374.9 & 2005 & 2357.04 & 2015 & 5056.61 \\
\hline
1986 & 74.02 & 1996 & 382.55 & 2006 & 2536.72 & 2016 & 5050.37 \\
\hline
1987 & 79.97 & 1997 & 399.06 & 2007 & 2952.58 & 2017 & 5082.74 \\
\hline
1988 & 99.29 & 1998 & 410.02 & 2008 & 2960.85 & 2018 & 5084.58 \\
\hline
1989 & 79.22 & 1999 & 423.28 & 2009 & 3102.67 & 2019 & 4978.67 \\
\hline
1990 & 209.01 & 2000 & 1401.92 & 2010 & 5465.82 & 2020 & 4423.52 \\
\hline
\end{tabular}
\end{table}

在本文中我们构建只有单一隐藏层的MLP网络模型,我们将输入层和输出层的神经元数目分别设置为3和1,隐藏层的神经元数目由枚举法在1到20的范围内搜索误差最小的数目,开始建立网络,得到模型。针对上表得到的残差数据,采用滑动窗口方法生成如下表所示的训练集样本与验证集样本。

\begin{table}
\centering
\caption{表14 人口残差序列样本(单位:万人)}
\begin{tabular}{|c|c|c|c|c|c|}
\hline
 & 样本序号 & $x_1$ & $x_2$ & $x_3$ & $y$ \\
\hline
\multirow{4}{*}{训练集} & 1 & 19.84 & 58.73 & 55.36 & 52.31 \\
\cline{2-6}
 & 2 & 58.73 & 55.36 & 52.31 & 48.57 \\
\cline{2-6}
 & $\cdots\cdots$ & $\cdots\cdots$ & $\cdots\cdots$ & $\cdots\cdots$ & $\cdots\cdots$ \\
\cline{2-6}
 & 20 & 53.71 & 88.59 & 87.18 & 69.71 \\
\hline
\multirow{3}{*}{验证集} & 21 & 88.59 & 87.18 & 69.71 & 71.52 \\
\cline{2-6}
 & $\cdots\cdots$ & $\cdots\cdots$ & $\cdots\cdots$ & $\cdots\cdots$ & $\cdots\cdots$ \\
\cline{2-6}
 & 27 & 54.7 & 43.94 & 40.16 & 34.69 \\
\hline
\end{tabular}
\end{table}

\begin{table}
\centering
\caption{表15 生产总值残差序列样本(单位:亿元)}
\begin{tabular}{|c|c|c|c|c|c|}
\hline
 & 样本序号 & $x_1$ & $x_2$ & $x_3$ & $y$ \\
\hline
\multirow{4}{*}{训练集} & 1 & 187.5 & 110.69 & 102.79 & 80.61 \\
\cline{2-6}
 & 2 & 78.89 & 74.02 & 79.97 & 99.29 \\
\cline{2-6}
 & $\cdots\cdots$ & $\cdots\cdots$ & $\cdots\cdots$ & $\cdots\cdots$ & $\cdots\cdots$ \\
\cline{2-6}
 & 30 & 5465.82 & 5359.16 & 5787.81 & 5305.47 \\
\hline
\multirow{3}{*}{验证集} & 31 & 5359.16 & 5787.81 & 5305.47 & 5049.77 \\
\cline{2-6}
 & $\cdots\cdots$ & $\cdots\cdots$ & $\cdots\cdots$ & $\cdots\cdots$ & $\cdots\cdots$ \\
\cline{2-6}
 & 37 & 5082.74 & 5084.58 & 4978.67 & 4423.52 \\
\hline
\end{tabular}
\end{table}

\section*{6.3.5 ARIMA-MLP 组合模型的预测和结果分析}

人口与生产总值的两个模型训练结果如下所示。从图 6.9 中可以得到第一个模型的平均预测误差在 12% 左右,第二个模型的误差在 5% 左右,说明网络训练结果很好,可以用作后续的计算。

\begin{figure}[h]
    \centering
    \includegraphics[width=\textwidth]{image.png}
    \caption{MLP 神经网络测试集的预测值与实际值对比图 (P) 和 (GDP)}
    \label{fig:mlp_results}
\end{figure}

图 6.9 MLP 模型训练结果

通过 MLP 神经网络模型获得序列的非线性部分的预测值,然后与 ARIMA 模型的预测值进行组合,即用 ARIMA 模型的预测值减去 MLP 预测的残差值得到 ARIMA-MLP 组合模型十四五(2021-2025 年)与二十一五(2056-2060 年)的最终预测值。结果如表 16 所示。

\begin{table}[h]
    \centering
    \caption{人口与经济指标预测结果}
    \label{tab:population_economic_forecast}
    \begin{tabular}{c c c c c c c c}
        \hline
        \multirow{2}{*}{时期} & \multirow{2}{*}{年份} & \multicolumn{2}{c}{ARIMA 预测值} & \multicolumn{2}{c}{MLP 预测值} & \multicolumn{2}{c}{组合模型预测值} \\
        \cline{3-8}
        & & $P$ & GDP & $P$ & GDP & $P$ & GDP \\
        \hline
        \multirow{5}{*}{十四五} & 2021 & 8552.27 & 97471.21 & 48.01 & 8504.26 & 8504.26 & 88966.95 \\
        & 2022 & 8598.62 & 101692.1 & 60.5 & 4535.39 & 8538.12 & 97156.69 \\
        & 2023 & 8647.72 & 105947.8 & 59.83 & 5075.11 & 8587.89 & 100872.7 \\
        & 2024 & 8698.66 & 110213.9 & 55.82 & 4683.86 & 8642.84 & 105530 \\
        & 2025 & 8750.59 & 114473.8 & 53.47 & 5133.48 & 8697.12 & 109340.4 \\
        \hline
        \multirow{2}{*}{\dots} & \dots & \multicolumn{2}{c}{\dots} & \multicolumn{2}{c}{\dots} & \multicolumn{2}{c}{\dots} \\
        \hline
        \multirow{5}{*}{二十一五} & 2056 & 10403.02 & 246547.9 & 71.09 & 5189.34 & 10331.93 & 241358.5 \\
        & 2057 & 10456.37 & 250808.3 & 79.43 & 5189.69 & 10376.94 & 245618.7 \\
        & 2058 & 10509.73 & 255068.8 & 80.7 & 5189.41 & 10429.03 & 249879.4 \\
        & 2059 & 10563.08 & 259329.3 & 65.42 & 5189.63 & 10497.66 & 254139.6 \\
        & 2060 & 10616.44 & 263589.7 & 59.47 & 5189.45 & 10556.97 & 258400.3 \\
        \hline
    \end{tabular}
\end{table}

注:人口总量单位为万人,GDP 单位为亿元。

将 ARIMA-MLP 模型与 ARIMA 模型对人口和 GDP 的预测效果进行对比如图 6.10 所示。可以看到 ARIMA 模型对未来的人口与 GDP 的预测均为简单的线性,而 ARIMA-MLP 模型则在含有线性趋势的同时,包含着非线性的波动,更趋近于真实情况。

\begin{figure}[h]
    \centering
    \includegraphics[width=\textwidth]{image.png}
    \caption{ARIMA-MLP 与 ARIMA 模型对比}
    \label{fig:6.10}
\end{figure}

\subsection{基于回归模型能源消费量预测}

\subsubsection{二元多项式回归模型的建立和预测}

考虑到量纲的差异,将人口 \(P\) 和生产总值 \(GDP\) 进行 min-max 标准化处理,指标缩放到 \([0,1]\) 之间后再进行二元多项式拟合,min-max 标准化方法用数学公式表示为:

\begin{equation}
X_{\text{标准化}} = \frac{X - \min(X)}{\max(X) - \min(X)}
\tag{6.8}
\end{equation}

\begin{figure}[h]
    \centering
    \includegraphics[width=\textwidth]{image2.png}
    \caption{二元多项式回归模型流程图}
    \label{fig:6.11}
\end{figure}

以标准化处理后的人口 \(P\) 和生产总值 \(GDP\) 作为自变量,考虑到单位 \(GDP\) 能耗与能耗 \(E\) 呈强的正相关性,因此将单位 \(GDP\) 能耗作为因变量,通过二元二次拟合得到人口 \(P\) 和生产总值 \(GDP\) 与单位 \(GDP\) 能耗的映射关系:

\begin{equation}
E_{GDP} = a_{00} + a_{10}P + a_{01}GDP + a_{20}P^2 + a_{11}P \cdot GDP + a_{02}GDP^2
\tag{6.9}
\end{equation}

因此,二元多项式回归的能耗预测值 \(E\) 为:

\begin{equation}
E = E_{GDP} \cdot GDP
\tag{6.10}
\end{equation}

运用 SPSS 软件进行回归分析,得到了拟合评价表(表 17)和二元回归拟合结果表(表 18)。拟合得到 \(R^2\) 为 0.963,RMSE 为 0.0082,拟合结果较好。

\begin{table}[h]
    \centering
    \caption{拟合评价表}
    \label{tab:17}
    \begin{tabular}{|c|c|}
    \hline
    \(R^2\) & RMSE \\
    \hline
    0.963 & 0.0082 \\
    \hline
    \end{tabular}
\end{table}

\begin{table}
\caption{二元回归拟合结果}
\begin{tabular}{c c c c}
\hline
参数 & 估计值 & 95\%置信上限 & 95\%置信下限 \\
\hline
$a_{00}$ & 0.468 & 0.577 & 0.564 \\
$a_{10}$ & 0.379 & 0.759 & 0.525 \\
$a_{01}$ & -0.476 & -0.907 & -1.388 \\
$a_{20}$ & 1.126 & -1.065 & -1.664 \\
$a_{11}$ & -2.264 & 3.146 & 1.976 \\
$a_{02}$ & 1.147 & -0.596 & -1.215 \\
\hline
\end{tabular}
\end{table}

\begin{figure}[h]
\centering
\includegraphics[width=0.8\textwidth]{image.png} % 替换为实际图像文件名
\caption{二元回归拟合效果图}
\end{figure}

结合二元多项式回归模型和GDP与人口总量P的组合模型预测值,可以得到如表19所示的总体能耗和单位GDP能耗的十四五时期(2021-2025年)与二十一五(2056-2060年)时期的预测值。2021-2060年的单位GDP能耗预测值如图6.13所示,随着年份的增加,单位GDP能耗呈现指数衰减的趋势,体现了技术进步和双碳政策下绿色发展的合理影响。

\begin{table}
\caption{二元多项式预测能耗值}
\begin{tabular}{c c c c}
\hline
\multirow{2}{*}{时期} & \multirow{2}{*}{年份} & \multicolumn{2}{c}{二元多项式回归的能耗预测值} \\
\cline{3-4}
& & 能源消费量 & 单位GDP能耗 \\
\hline
\multirow{5}{*}{十四五(2021-2025年)} & 2021 & 34426.86 & 0.37 \\
& 2022 & 34439.40 & 0.35 \\
& 2023 & 35238.98 & 0.35 \\
& 2024 & 35961.51 & 0.34 \\
& 2025 & 36819.22 & 0.34 \\
\hline
\end{tabular}
\end{table}

\begin{table}
\centering
\begin{tabular}{c c c c}
\hline
 & 2056 & 49613.76 & 0.21 \\
\hline
 & 2057 & 50349.32 & 0.20 \\
\hline
\multirow{3}{*}{\textbf{十二五(2056-2060年)}} & 2058 & 50947.96 & 0.20 \\
\hline
 & 2059 & 51062.59 & 0.20 \\
\hline
 & 2060 & 51532.01 & 0.20 \\
\hline
\end{tabular}
\caption{能源消费量单位为万tce,单位GDP能耗单位为万tce/亿元。}
\end{table}

\begin{figure}[h]
\centering
\includegraphics[width=\textwidth]{image.png}
\caption{单位GDP的能耗预测结果}
\end{figure}

\subsection{6.4.2 PCA降维后的线性回归模型}

为探索能源消费量回归分析的更好方法,这里采用降维后的线性回归方法进行效果比较,以确定最适合数据和研究目标的方法。对于线性回归,需要检查各自变量指标是否具有共线性问题。处理具有共线性的数据是回归分析中的一个重要问题,因为共线性会导致模型参数不稳定,降低预测性能和可解释性。可以使用相关性矩阵或方差膨胀因子(VIF)来检测共线性,VIF应小于10或者5,严格为5。两个或多个特征高度相关时VIF值无穷大,需要进行特征的选择或者降维,也可以使用岭回归和Lasso回归,岭回归通过L2惩罚项,而Lasso回归通过L1惩罚项来实现。此外,稳健回归方法对异常值和共线性有较强的鲁棒性,如Huber回归或Theil-Sen回归。

对于人口和经济总量的共线性问题,这里使用PCA技术进行降维,减少共线性的同时保留数据中的大部分信息,再在新的主成分特征上进行一元线性回归分析[14]。

\begin{figure}[h]
\centering
\includegraphics[width=\textwidth]{image2.png}
\caption{PCA-线性回归模型流程图}
\end{figure}

使用主成分分析法进行特征降维,通过KMO检验和Bartlett球形检验验证是否可以进行主成分分析。通过KMO检验(KMO>0.6),水平上呈现显著性,拒绝原假设,各变量间具有相关性,说明变量之间是存在相关性的,符合主成分分析要求。通过Bartlett检验:P<0.05,呈显著性。

\begin{table}
\centering
\caption{主成分权重结果}
\begin{tabular}{l l l}
\hline
名称 & 方差解释率(\%) & 累积方差解释率(\%) \\
\hline
主成分 & 1.981 & 99.062 \\
\hline
\end{tabular}
\end{table}

主成分$S$的累积方差解释率为99.062\%,特征组合方式为:
\begin{equation}
S = F(P, GDP) = 0.5\log P + 0.5\log GDP
\tag{6.11}
\end{equation}

利用主成分$S$进行总能耗的OLS线性回归分析,分析结果如表20。

\begin{table}
\centering
\caption{线性回归分析结果}
\begin{tabular}{c c c c c c c c}
\hline
 & \multicolumn{2}{c}{非标准化系数} & $t$ & $P$ & VIF & $R^2$ & $F$ \\
\cline{2-3}
 & $B$ & 标准误差 & & & & & \\
\hline
常数 & 0.289 & 0.445 & 0.651 & 0.531 & - & 0.907 & 88.059 \\
系数 & 6.547 & 0.698 & 9.384 & 0.000**** & 1 & & \\
\hline
\end{tabular}
\end{table}

注:***代表10\%的显著性水平

\begin{figure}[h]
\centering
\includegraphics[width=0.8\textwidth]{image.png}
\caption{6.15PCA线性回归拟合效果}
\end{figure}

最终得到回归表达式为:
\begin{equation}
\log(E) = 6.547S + 0.289
\tag{6.12}
\end{equation}

结合ARIMA-MLP模型得到的人口总量与GDP预测结果与降维后的OLS回归模型,即可进行能耗的预测,预测结果如下。

\begin{table}
\centering
\caption{基于人口和经济变化的能耗预测结果}
\begin{tabular}{l l l l}
\hline
\multirow{2}{*}{时期} & \multirow{2}{*}{年份} & \multicolumn{2}{c}{PCA降维后的回归能耗预测值} \\
\cline{3-4}
 & & 能源消费量 & 单位GDP能耗 \\
\hline
\multirow{5}{*}{十四五(2021-2025年)} & 2021 & 32580.14 & 0.37 \\
 & 2022 & 33093.96 & 0.35 \\
 & 2023 & 33519.70 & 0.35 \\
 & 2024 & 34029.79 & 0.34 \\
 & 2025 & 34450.36 & 0.34 \\
\hline
二十一五(2056-2060年) & 2056 & 45477.22 & 0.21 \\
\hline
\end{tabular}
\end{table}

\begin{table}
\centering
\begin{tabular}{c c c c}
\hline
 & 2057 & 45758.10 & 0.20 \\
\hline
 & 2058 & 46047.51 & 0.20 \\
\hline
 & 2059 & 46360.29 & 0.20 \\
\hline
 & 2060 & 46656.14 & 0.20 \\
\hline
\end{tabular}
\end{table}

\textbf{注:}能源消费量单位为万tce,单位GDP能耗单位为万tce/亿元。

《意见》中明确,到2025年,重点行业能源利用效率大幅提升,单位GDP能耗比2020年下降13.5%;到2030年,重点耗能行业能源利用效率达到国际先进水平;到2060年,能源利用效率达到国际先进水平。通过PCA降维后的线性回归预测,得到了2025年的单位GDP能耗比为31.5%,单位GDP能耗比2020年下降11.2%,较为符合要求,模型拟合情况和实际理想目标较为符合。

图6.16比较了两种回归效果,二元多项式回归和PCA降维后的线性回归趋势都较为平稳。其中PCA降维后的线性回归预测结果与《意见》数据吻合程度更高,整体能耗偏低,符合国家高质量、可持续发展的战略,因此合理性更高。

\begin{figure}[h]
\centering
\includegraphics[width=\textwidth]{image.png} % 替换为实际图像文件名
\caption{能源消费量回归预测效果图}
\end{figure}

关于非化石能源消费比重,《意见》中明确指出,到2025年非化石能源消费比重达到20%左右;到2030年非化石能源消费比重达到25%左右;到2060年非化石能源消费比重超过80%,整理得到非化石能源能耗变化如表23。

\textbf{表23 非化石能源能耗变化}

\begin{table}
\centering
\begin{tabular}{c c c c c}
\hline
\multirow{2}{*}{年份} & \multirow{2}{*}{非化石能源消费比重/\%} & \multicolumn{2}{c}{线性回归预测} & \multicolumn{2}{c}{二元多项式回归预测} \\
\cline{3-6}
 & & 总能耗/万tce & 非化石能源能耗/万tce & 总能耗/万tce & 非化石能源能耗/万tce \\
\hline
2025 & 20 & 34450.36 & 6890.07 & 36819.22 & 7363.84 \\
\hline
2030 & 25 & 36566.72 & 9141.68 & 39842.68 & 9960.67 \\
\hline
2060 & 80 & 46656.14 & 37324.91 & 51532.01 & 41225.61 \\
\hline
\end{tabular}
\end{table}

\section*{6.5 碳排放量预测模型}

在6.2中,我们完成了基于人口和经济变化的能源消费量预测,为加入碳排放量的关联项,本节利用上一节中的PCA线性相关模型,通过分析能源消费强度(即单位GDP能

44

耗)与碳效率的关系,完成碳排放量预测模型的建立与求解。

通过上一章多项指标的皮尔逊相关性分析,我们发现 2010-2020 年的单位 GDP 能耗与碳效率具有相对更强的线性关系,继续分析二者的拟合回归图像,基本可以得到:

\[
\eta_{C} = k_{GDP;C} E_{GDP} + b
\tag{6.13}
\]

\begin{figure}[h]
    \centering
    \includegraphics[width=\textwidth]{image.png}
    \caption{碳效率和单位 GDP 能耗的线性规律}
    \label{fig:6.17}
\end{figure}

利用 OLS 拟合分析,得到 $k_{GDP;C} = 2.645$,$b = -0.14$,与 PCA 线性相关模型联立关系式即可完成碳排放量的预测。

\begin{table}[h]
    \centering
    \caption{拟合评价表}
    \label{tab:24}
    \begin{tabular}{|c|c|}
        \hline
        $R^2$ & RMSE \\
        \hline
        0.996 & 0.0132 \\
        \hline
    \end{tabular}
\end{table}

对式 (6.13) 左右两侧同时乘上 GDP 总量可得描述碳排放量与能源消费量与 GDP 的关系式如下:

\[
C = 2.645E - 0.14GDP
\tag{6.14}
\]

上一节中已经得到了总能耗和单位 GDP 能耗的预测值,再通过单位 GDP 能耗与碳效率的映射,最终得到逐年的碳排放量预测值如图 6.18 所示,整体上具有对数上升的趋势。

\begin{figure}[h]
    \centering
    \includegraphics[width=\textwidth]{image2.png}
    \caption{碳效率 C/GDP 和碳排放量}
    \label{fig:6.18}
\end{figure}

图 6.18 碳效率和碳排放量的预测结果

最后将所得公式进行整合,得到碳排放量与人口、GDP、能源消费量间的关系式如下:

\begin{equation}
\begin{cases}
C = 2.645E - 0.14GDP \\
\log(E) = 3.274\left(\log(P) + \log(GDP)\right) + 0.289 \\
\left(1 - \sum_{i=1}^{3} \varphi_{Pi} L^i\right)(1 - L)P_t = c_P + (1 + \theta_P L)\varepsilon_{Pt} \\
\left(1 - \sum_{i=1}^{2} \varphi_{GDPi} L^i\right)(1 - L)^2 GDP_t = c_{GDP} + (1 + \theta_{GDP} L)\varepsilon_{GDPt}
\end{cases}
\tag{6.15}
\end{equation}

\section{碳排放量的关联性分析}

\subsection{碳排放量与能源消费量关联分析}

结合我们第五章建立的目标体系,可以得到碳排放量与各能源消费部门(工业消费部门、建筑消费部门、交通消费部门、居民生活消费、农林消费部门)以及能源供应部门的能源消费量之间的关系,关系式如下所示。

\begin{equation}
\begin{aligned}
C &= C^p + C^L = \sum_{i+1} B_i \times CE_i \times \eta_E \times G \times P \\
&= \sum_{i-1,j} \frac{C_{ij}^p}{E_{ij}} \times \frac{E_{i,j}^p}{E_i} \times \frac{E_i^p}{GDP} \times GDP_i \\
&\quad + \sum_j \frac{C_j^L}{E_j^L} \times \frac{E_j^L}{E^L} \times \frac{E^L}{RC} \times \frac{RC}{GDP} \times GDP
\end{aligned}
\tag{6.16}
\end{equation}

其中,$C^p$ 和 $C^L$ 分别表示生产与消费部门和居民生活消费所产生的碳排放总量,$i(1, 2, ..., 6)$ 为如表 2 所示编号,$j(j = 1, 2, 3)$ 表示能源的类型(分别为煤碳、油品、天然气),$GDP$ 表示生产总值,$GDP_i$ 表示某个生产或消费部门 $i$ 的生产总值,$E_i^p$ 表示部门的能源消费量,$E_{ij}^p$ 表示生产或消费部门(包含居民生活)$i$ 使用第 $j$ 类能源的消费量,$C_{ij}^p$ 表示生产或消费部门(包含居民生活)$i$ 使用第 $j$ 类能源所产生的碳排放量,$C_j^L$ 表示居民生活消费第 $j$ 类能源所得到的碳排放量,$E^L$ 表示居民生活消费的能源总量,$E_j^L$ 表示居民生活消费第 $j$ 类能源的消费量,$RC$ 表示居民生活消费支出。

对各个部门从 2010-2020 年间的能源消费量进行宏观统计,再将该年间的单位 GDP 能耗变化趋势可视化,如图 6.19 所示。可以看出,能源消费强度正在减小。工业消费部门的能源消费量在此期间内呈持续且缓慢增长态势,占比始终领先且逐年增长。而除此之外的其他能源消费、能源供应部门及居民生活消费在此期间内偶有起伏,但整体呈现稳定态势。单位 GDP 能耗在十二五与十三五期间呈下降趋势,即随着我国节能减排、绿色发展战略的推出和落实,能源的利用效率逐渐升高。

\begin{figure}[h]
    \centering
    \includegraphics[width=\textwidth]{image.png}
    \caption{各部门能耗与总的单位 GDP 能耗折柱图}
    \label{fig:6.19}
\end{figure}

\subsection{碳排放量与能源消费量种类关联分析}

由题目可得,非化石能源消费比重近似等于非化石能源发电比重,非化石能源发电比重等于非化石能源发电占比乘以电力消费比重。即有如下的表达式。

\begin{equation}
\eta_{Non-fossil} \approx \rho_{e}
\tag{6.17}
\end{equation}

\begin{equation}
\rho_{e} = (Q/W) \cdot (W/E) = T \cdot R
\tag{6.18}
\end{equation}

其中,$\rho_{e}$ 为非化石能源发电比重,$Q$ 为非化石能源发电量,$W$ 为电力消费量,$T = Q/W$ 为非化石能源发电占比,$R = W/E$ 为电力消费比重,$\eta_{Non-fossil}$ 为非化石能源消费比重,$E$ 为能源消费量。同时经过查阅大量资料,了解到新能源电力与外来电力的总和等于各能源消费或供应部门的消耗或生产电力的总和;新能源热力等于各能源消费或供应部门的消耗或生产热力的总和,即有如下的数学表达式:

\begin{equation}
NE + OE = \sum_{i} YE_{i}
\tag{6.19}
\end{equation}

\begin{equation}
NH = \sum_{i} YH_{i}
\tag{6.20}
\end{equation}

其中,$NE$ 为新能源电力,$OE$ 为外来电力,$YE_{i}$ 表示第 $i$ 个部门生产或消耗的电力,$NH$ 为新能源热力,$YH_{i}$ 表示第 $i$ 个部门生产或消耗的热力。

因此可以得到碳排放量与各能源消费部门的能源消费品种(一次能源中化石能源消费与非化石能源消费以及二次能源消费)以及能源供应部门的能源消费品种之间的关系,得到的关系式如下所示。

\begin{equation}
\left\{
\begin{aligned}
C &= C^p + C^L = \sum_{i+1} B_i \times CE_i \times \eta_E \times G \times P \\
&= \sum_{i-1,j} \frac{C_{ij}^p}{E_{ij}} \times \frac{E_{i,j}^p}{E_i} \times \frac{E_i^p}{GDP} \times GDP_i \\
&\quad + \sum_j \frac{C_j^L}{E_j^L} \times \frac{E_j^L}{E^L} \times \frac{E^L}{RC} \times \frac{RC}{GDP} \times GDP \\
NE + OE &= \sum_i YE_i \\
NH &= \sum_i YH_i \\
\eta_{Non-fossil} &\approx \rho_e = \frac{Q}{E} \\
E &= \sum_{i,j} E_{ij}^p + E_j^L
\end{aligned}
\right.
\tag{6.21}
\end{equation}

对各个部门从 2010-2020 年间的一次能源中的非化石能耗比重进行宏观统计,再将该年间的碳排放量变化趋势可视化,如图 6.20 所示。由图可知,各能源消费部门的非化石能耗比重均在逐年缓慢上升,而与此同时,碳排放量的增长速率逐年降低,甚至在 2019-2020 年间呈负增长,也印证了提高非化石能源消费比重可以实现能源消费量与碳排放量增长的负相关变化。

\begin{figure}[h]
\centering
\includegraphics[width=\textwidth]{image.png}
\caption{碳排放量和非化石能耗比重变化}
\label{fig:6.20}
\end{figure}

\section{问题三:区域双碳目标与路径规划}

\subsection{问题分析}

问题三要求设置不少于三种情景,每个情景均需与双碳时间点以及相关因素关联,同时在给定 GDP 与碳汇条件下,在保证满足有关约束的情况下,得到多情景下碳排放量核算方法,最后确定达成双碳需满足的目标以及其路径。

为了完成对情景的设计,在满足有关双碳定义与时间节点的基础上,通过合理假设,我们又预设了碳排放量的增速、单位能耗二氧化碳排放量的变化率、电力消费比重三个因素与时间节点间的关系,从而实现了情景设计。

为了在多情境下得到碳排放量的核算方法,在充分使用情节设计中以及赛题所给定的条件的同时,通过查阅文献资料我们建立了多种情景下的各类指标的数学模型,并最终基于改进的 Kaya 模型获得了多情景下的碳排放量核算方法。

为了确定双碳(碳达峰与碳中和)目标与路径,在所设基准情景下,我们通过先前所建立的碳排放模型求得了各个时间节点的目标,又再建立了基于四大重点工程的 STIRPAT 模型,将四个重点工程与实际的指标相对应,最终通过 GA-MPSO 算法最优化算法对模型进行求解,得到了双碳路径。

\begin{figure}[h]
    \centering
    \includegraphics[width=\textwidth]{image.png}
    \caption{问题三流程图}
    \label{fig:problem3_flowchart}
\end{figure}

\subsection{7.2 基于双碳政策的情景设计}

考虑到非化石能源消费比重与能源效率与改进的 Kaya 模型之间的关联,可将式 5.13 等价替换为式 7.1。
\begin{equation}
C = \sum_{i} \frac{C_{i}^{p}}{\mu E_{i}^{p}} \times \frac{\mu E_{i}^{p}}{GDP_{i}} \times \frac{GDP_{i}}{P} \times P + \frac{C^{L}}{\mu E^{L}} \times \frac{\mu E^{L}}{RC} \times \frac{RC}{GDP} \times \frac{GDP}{P}
\tag{7.1}
\end{equation}

其中,$E_{i}^{p}/GDP_{i}$ 为各部门的能源效率 $\eta_{E}$,$\mu$ 为化石能源消费比重,其与非化石能源消费比重关系式如式 7.2。
\begin{equation}
\mu = 1 - \eta_{Non-fossil}
\tag{7.2}
\end{equation}
而非化石能源消费比重如下式 7.3。
\begin{equation}
\eta_{Non-fossil} \approx \rho_{e} = Q/E
\tag{7.3}
\end{equation}

对式 7.1 两侧对求时间偏导得到如下表达式:
\begin{equation}
\frac{\partial C}{\partial t} = \frac{\partial}{\partial t} \left( \sum_{i} \frac{C_{i}^{p}}{\mu E_{i}^{p}} \times \frac{\mu E_{i}^{p}}{GDP_{i}} \times \frac{GDP_{i}}{P} \times P + \frac{C^{L}}{\mu E^{L}} \times \frac{\mu E^{L}}{RC} \times \frac{RC}{GDP} \times \frac{GDP}{P} \right)
\tag{7.4}
\end{equation}

根据碳达峰以及碳中和的定义,结合式 7.4,参考我国“两个一百年”、“三步走战略”等目标政策及有关报道 [15],本文设计了自然情景、基准情景、雄心情景共三种情景及其相应的有关数学表达如下表 25 所示。

\begin{table}[h]
\centering
\caption{基于双碳政策的未来情景构建}
\begin{tabular}{c c c c}
\hline
时间节点 & 自然情景 & 基准情景 & 雄心情景 \\
\hline
 & $\frac{\partial C}{\partial t} > 0$ & $\frac{\partial C}{\partial t} = 0$ & $\frac{\partial C}{\partial t} < 0$ \\
2030 年 & $\alpha_{2030} = 1.5$ & $\alpha_{2030} = 2.0$ & $\alpha_{2030} = 2.5$ \\
 & $\beta_{2030} = 5\%$ & $\beta_{2030} = 7.5\%$ & $\beta_{2030} = 10\%$ \\
 & $\gamma_{2030} = 35\%$ & $\gamma_{2030} = 40\%$ & $\gamma_{2030} = 45\%$ \\
\hline
 & $\frac{\partial C}{\partial t} > 0$ & $\frac{\partial C}{\partial t} \ll 0$ & $\frac{\partial C}{\partial t} \ll 0$ \\
2060 年 & $\alpha_{2060} = 2.5$ & $\alpha_{2060} = 4.0$ & $\alpha_{2060} = 4.5$ \\
 & $\beta_{2060} = 20\%$ & $\beta_{2060} = 25\%$ & $\beta_{2060} = 30\%$ \\
 & $\gamma_{2060} = 50\%$ & $\gamma_{2060} = 60\%$ & $\gamma_{2060} = 70\%$ \\
\hline
\end{tabular}
\end{table}

其中 $\partial C / \partial t$ 为碳排放量的增速,$\alpha_{2030}$ 和 $\alpha_{2060}$ 分别为 2030 年和 2060 年的人均 GDP 相
50

较 2020 年人均 GDP 的比值, $\beta_{2030}$ 和 $\beta_{2060}$ 分别为 2030 年和 2060 年相较 2020 年单位能耗二氧化碳排放量的变化率, $\gamma_{2030}$ 和 $\gamma_{2060}$ 分别 2030 年和 2060 年相较 2020 年电力消费比重。

在自然情景中, 保持现有政策不变且不再出台有关政策进行干预, 最终在 2030 年与 2060 年两个时间节点均未实现碳达峰与碳中和, 电力消费占比较低, 且人均 GDP 增长与单位能耗二氧化碳排放量降低较为缓慢。

在基准情景中, 制定并实行了有关政策, 最终于 2030 年按时实现碳达峰, 于 2060 年按时实现碳中和, 电力消费占比达到预期, 且人均 GDP 增长与单位能耗二氧化碳排放量降低达到预期水平。

在雄心情景中, 积极出台并落实了相关政策, 最终提前于预设的时间节点实现了碳达峰与碳中和的目标, 电力消费占比较高, 并使人均 GDP 超预期增长、单位能耗二氧化碳排放量超预期降低。

在所设计的各场景下, 以 2020 年未来碳排放量为基底, 未来数十年碳排放量以及碳排放变化速率示意简图如下图。

\begin{figure}[h]
    \centering
    \includegraphics[width=0.45\textwidth]{image1.png}
    \caption{Future Carbon emissions under three scenarios}
    \label{fig:carbon_emissions}
\end{figure}
\begin{figure}[h]
    \centering
    \includegraphics[width=0.45\textwidth]{image2.png}
    \caption{Future Carbon emissions change rate under three scenarios}
    \label{fig:carbon_emissions_change_rate}
\end{figure}

图7.1 不同情境下碳排放量与碳排放变化速率构建

\section{7.3 多情景下碳排放量及相关指标的核算方法}

由假设一可知, 在三个情境之下, 2035 年 GDP 总量均为 2020 年的两倍, 2060 年 GDP 总量为均 2020 年四倍。考虑到我国未来经济稳中向好, 故以年份间隔为时间尺度, 基于 OLS 进行线性回归拟合, 得到基于 2020 年的 GDP 预测表达式:

\begin{equation}
GDP(t) = \alpha t + \beta
\tag{7.5}
\end{equation}

代入 2020 年真实 GDP 总量数据以及 2035 年和 2060 年期望 GDP 总量数据, 可求得其中 $\alpha = 6696$, $\beta = 84160$。

又通过情景设计中所涉及的三个情景中 2060 与 2030 年相较于基准 2020 年的人均 GDP 比值, 同样结合数据以年份间隔为时间尺度, 对人均 GDP 进行基于 OLS 进行线性回归:

\begin{equation}
G(t) = kt + b
\tag{7.6}
\end{equation}

从上至下即依次为自然情景、基准情景、雄心情景下的人均 GDP 预测表达式。

\begin{equation}
\begin{cases}
G_1(t) = 0.6638t + 9.857 \\
G_2(t) = 0.7644t + 11.67 \\
G_3(t) = 0.8651t + 13.48
\end{cases}
\tag{7.7}
\end{equation}

又由于人均 GDP 为 GDP 总量与人口总量之比, 故可得人口数量表达式:

\begin{equation}
P(t) = \frac{GDP(t)}{G(t)}
\tag{7.8}
\end{equation}

故可分别得到三个场景之下未来人口总量的预测表达式:

\begin{equation}
\begin{cases}
P_1(t) = \frac{6696t + 84160}{0.6638t + 9.857} \\
P_2(t) = \frac{6696t + 84160}{0.7644t + 11.67} \\
P_3(t) = \frac{6696t + 84160}{0.8651t + 13.48}
\end{cases}
\tag{7.9}
\end{equation}

通过问题二,我们得到了能源消费量与人口及经济(GDP)之间的关系如下式:

\begin{equation}
\log(E(t)) = 3.27[\log(P(t)) + \log(GDP(t))] + 0.289
\tag{7.10}
\end{equation}

由中国环境网查阅资料$^{[16]}$可知2020年中国总的温室气体排放量已经达到140亿吨二氧化碳当量,而可靠碳汇为10亿吨左右,同理按比例可知2020年江苏省的可靠碳汇量。而由假设二与假设三可知,2060年江苏省的生态碳汇、工程碳汇或碳交易为2020年基期碳排放量的10%,又由碳汇定义可知,碳汇包含生态碳汇和工程碳汇两种类型。故最终可得到有关碳汇数据如下表所示。

\begin{table}[h]
\centering
\caption{碳汇统计表}
\begin{tabular}{|c|c|c|}
\hline
中国2020年碳汇总量 & 江苏省2020年碳汇总量 & 江苏省2060年碳汇总量 \\
\hline
100,000 & 5,188 & 14,527 \\
\hline
\end{tabular}
\end{table}

注:碳汇单位为万吨二氧化碳当量。

随着各类节能减排政策和人们环保意识的提高,以及科学技术的不断发展,生态和工程两个方面对二氧化碳的消纳能力即碳汇会逐渐提高。而又因非化石能源的消费比重在未来的不断增大,化石能源所产生的碳排放量也会逐渐减少,故碳汇的增长速率也会逐渐变小,最终在达到碳中和后趋于一个平稳值。通过上述分析,选用对数函数的形式对表26中所得数据进行拟合,得到未来碳汇的预测式如下:

\begin{equation}
TH(t) = 2515\log(t+1) + 5188
\tag{7.11}
\end{equation}

通过查找资料与阅读文献,可以知道随着能源利用效率的提高,碳汇量将随之增大而单位GDP能耗将减少。即能源利用率与碳汇量成正比关系,与单位GDP能耗成反比关系。且能源利用效率的取值在0到1之间,故不妨令能源利用效率与能源利用率、碳汇量间的关系式如下:

\begin{equation}
\eta_E(t) = \frac{2 \cdot e^{-k \cdot \frac{E_{GDP}(t)}{TH(t)}}}{1 + e^{-k \cdot \frac{E_{GDP}(t)}{TH(t)}}}
\tag{7.12}
\end{equation}

其中$k$为比例调和系数,$TH$为碳汇量,$E_{GDP}$为单位GDP能耗,其表达式如下:

\begin{equation}
E_{GDP}(t) = \frac{E(t)}{GDP(t)}
\tag{7.13}
\end{equation}

通过情景设计中所涉及的三个情景中2060与2030年相较于基准2020年的单位能耗

二氧化碳排放量的变化率,同样结合数据以年份间隔为时间尺度,对单位能耗二氧化碳排放量的变化率进行基于 OLS 进行线性回归。从上至下即依次为自然情景、基准情景、雄心情景下的单位能耗二氧化碳排放量预测表达式:
\begin{equation}
\begin{cases}
CE_{1}(t)=-0.011t+2.312 \\
CE_{2}(t)=-0.014t+2.298 \\
CE_{3}(t)=-0.017t+2.283
\end{cases}
\tag{7.14}
\end{equation}

而又由于非化石燃料消费比重的增大会导致单位能耗二氧化碳排放量的减小,即二者成反比关系,因此,不妨令非化石燃料消费比重与单位能耗二氧化碳排放量间的关系式如下:
\begin{equation}
\eta_{non-fossil}(t)=r\cdot\left(\frac{1}{CE(t)}\right)+b
\tag{7.15}
\end{equation}
其中,$\eta_{non-fossil}$ 为非化石燃料消费比重,$CE$ 为单位能耗二氧化碳排放量,$r$ 为修正因子,$b$ 为常数项。

通过查阅网络资料,得知 2020 年电力消费比重为 27%。再根据情景设计中所涉及的三个情景中 2060 与 2030 年相较于基准 2020 年的电力消费比重,同样结合数据以年份间隔为时间尺度,对电力消费比重进行基于 OLS 进行线性回归。从上至下即依次为自然情景、基准情景、雄心情景下的电力消费比重的预测表达式:
\begin{equation}
\begin{cases}
R(t)=0.58t+28.04 \\
R(t)=0.79t+29.19 \\
R(t)=1.02t+30.35
\end{cases}
\tag{7.16}
\end{equation}

最后对改进的 Kaya 公式 (7.4) 进一步简化为下式:
\begin{equation}
\frac{\partial C}{\partial t}=\frac{\partial\sum\limits_{i,j}\frac{C_{ij}}{\mu E_{ij}}\cdot\frac{GDP_{i}}{P}}{\partial t}
\tag{7.17}
\end{equation}

结合上述各式便可以得到以下各情景中碳排放量以及能源利用效率的核算公式:

1) 自然情景下碳排放量核算公式:
\begin{equation}
\begin{cases}
\frac{\partial C}{\partial t}=\frac{\partial\sum\limits_{i,j}\frac{C_{ij}}{\mu E_{ij}}\cdot\frac{GDP_{i}}{P}\cdot P}{\partial t}; \frac{\partial C}{\partial t}>0, t=2030; \frac{\partial C}{\partial t}>0, t=2060 \\
G(t)=GDP(t)/P(t)=0.6638t+9.857 \\
P(t)=\frac{6696t+84160}{0.6638t+9.857} \\
\log\big(E(t)\big)=3.27[\log\big(P(t)\big)+\log(GDP(t))]+0.289 \\
CE(t)=-0.011t+2.312 \\
\eta_{non-fossil}(t)=1-\mu=r\cdot\left(\frac{1}{CE(t)}\right)+b
\end{cases}
\tag{7.18}
\end{equation}

2) 基准情景下碳排放量核算公式:
\begin{equation}
\left\{
\begin{aligned}
\frac{\partial C}{\partial t} &= \frac{\partial \sum\limits_{i,j} \frac{C_{ij}}{\mu E_{ij}} \cdot \frac{\mu E_{ij}}{GDP_i} \cdot \frac{GDP_i}{P} \cdot P}{\partial t}; \frac{\partial C}{\partial t} = 0, t = 2030; \frac{\partial C}{\partial t} \ll 0, t = 2060 \\
G(t) &= GDP(t) / P(t) = 0.7644t + 11.67 \\
P(t) &= \frac{6696t + 84160}{0.7644t + 11.67} \\
\log\big(E(t)\big) &= 3.27[\log\big(P(t)\big) + \log(GDP(t))] + 0.289 \\
CE(t) &= -0.014t + 2.298 \\
\eta_{non-fossil}(t) &= 1 - \mu = r \cdot \left(\frac{1}{CE(t)}\right) + b
\end{aligned}
\right.
\tag{7.19}
\end{equation}

3) 雄心情景下碳排放量核算公式:
\begin{equation}
\left\{
\begin{aligned}
\frac{\partial C}{\partial t} &= \frac{\partial \sum\limits_{i,j} \frac{C_{ij}}{\mu E_{ij}} \cdot \frac{\mu E_{ij}}{GDP_i} \cdot \frac{GDP_i}{P} \cdot P}{\partial t}; \frac{\partial C}{\partial t} < 0, t = 2030; \frac{\partial C}{\partial t} \ll 0, t = 2060 \\
G(t) &= GDP(t) / P(t) = 0.8651t + 13.48 \\
P(t) &= \frac{6696t + 84160}{0.8651t + 13.48} \\
\log\big(E(t)\big) &= 3.27[\log\big(P(t)\big) + \log(GDP(t))] + 0.289 \\
CE(t) &= -0.017t + 2.283 \\
\eta_{non-fossil}(t) &= 1 - \mu = r \cdot \left(\frac{1}{CE(t)}\right) + b
\end{aligned}
\right.
\tag{7.20}
\end{equation}

4) 各场景下能源利用效率核算公式:
\begin{equation}
\left\{
\begin{aligned}
\eta_E(t) &= \frac{2 \cdot e^{-k \cdot \frac{E_{GDP}(t)}{TH(t)}}}{1 + e^{-k \cdot \frac{E_{GDP}(t)}{TH(t)}}} = 1 - \mu \\
E_{GDP}(t) &= \frac{E(t)}{GDP(t)} \\
TH(t) &= 2515 \log(t + 1) + 5188
\end{aligned}
\right.
\tag{7.21}
\end{equation}

7.4 基于双碳情景下的各指标目标

三种场景中的基准场景按时完成了双碳即碳达峰与碳中和的目标,故在此选用基准场景下的碳排放量及各有关变量的核算公式,以预测未来各时间节点的重要变量的目标值。通过代入各时间节点,最终得到目标值如下表27所示:

\begin{table}
\centering
\caption{实现双碳时各指标目标}
\begin{tabular}{c c c c c c}
\hline
时间 & $GDP$ & 人口 & 能源消费量 & 能源利用效率 & 非化石能源消费比重 \\
\hline
2025年 & 117640 & 7593.6 & 33417 & 34.2\% & 35.4\% \\
2030年 & 151120 & 7824.4 & 32255 & 40.2\% & 50.1\% \\
2035年 & 184600 & 7979.9 & 29976 & 51.5\% & 58.5\% \\
2050年 & 285040 & 8237.7 & 25632 & 65.3\% & 81.3\% \\
2060年 & 352000 & 8332.1 & 21008 & 74.0\% & 90.7\% \\
\hline
\end{tabular}
\end{table}

注:单位与前文一致。

\section{基于四大重点工程改进的 STIRPAT 模型}

目前,STIRPAT 模型是研究环境问题的主要工具,被广泛用于解决环境压力问题 \cite{ref17}。STIRPAT 模型是在 IPAT 模型的基础上发展而来,但 IPAT 模型过于简单,只考虑了人口、财富水平和技术水平等有限因素,没有引入指数概念,因此无法充分反映不同社会条件,从而对研究结果产生影响。随着社会的快速发展,IPAT 模型在当今社会已经不再适用。相比之下,STIRPAT 模型具有显著的优势。STIRPAT 模型的变量可以根据研究目的和需求进行灵活添加和修改,从而获得不同的模型估计结果。因此,学者们引入了 STIRPAT 模型,这一模型是 IPAT 模型的进化版,引入了指数概念,使其能够更好地适应不同的社会结构情境。STIRPAT 模型的一般表达式如下:

\begin{equation}
I_t = a P_t^{\alpha_1} A_t^{\alpha_2} T_t^{\alpha_3} e^{\varepsilon}
\tag{7.22}
\end{equation}

其中 $a$ 为常数项,$\alpha_1, \alpha_2, \alpha_3$ 分别表示人口总量、经济发展水平、技术水平的弹性系数,$\varepsilon$ 为噪声扰动系数。该模型在 IPAT 模型的基础上,又为三个特征因素赋予了不同的影响系数。特别地,当 $\alpha_1 = \alpha_2 = \alpha_3 = 1$ 时,STIPRAT 模型即为 IPAT 模型。

能效提升指标即能源利用效率提升。提升企业能源资源利用效率,不仅降低能源成本,还能够推动重点行业领域工艺流程、生产设备更新换代,提升行业绿色低碳发展水平。

产业(产品)的升级指标在本文即等效为人均 GDP。由于科学技术是第一生产力,其带动了产业及产品升级的同时,也势必提高人均 GDP。

又由题意分析可知,非化石能源发电站比或非化石能源消费比重可以表示能源脱碳指标,电力消费比重可以表示能源消费电气化指标。

而现如今 STIRPAT 模型被许多不同学者以及研究者运用不同影响因素或变量的根据具体内容进行研究,基于本文的研究目的是通过完成能效提升、产业升级、能源脱碳和能源消费电气化四个重大工程的路径,以完成双碳目标的实现,故改进后的 STIRPAT 模型如下:

\begin{equation}
C_k = a \left( \eta_E \right)_k^{\alpha_1} \left( GDP_P \right)_k^{\alpha_2} \left( \eta_{non-fossil} \right)_k^{\alpha_3} \left( R \right)_k^{\alpha_4} e^{\varepsilon}
\tag{7.23}
\end{equation}

为方便分析,将模型进一步对数化处理为:

\begin{equation}
\ln C_k = \ln a + \alpha_1 \ln \left( \eta_E \right)_k + \alpha_2 \ln \left( GDP_P \right)_k + \alpha_3 \ln \left( \eta_{non-fossil} \right)_k + \alpha_4 \ln \left( R \right)_k + \varepsilon \ln e
\tag{7.24}
\end{equation}

其中,$C$ 为二氧化碳排放量,$\eta_E$ 为能源利用效率即能效提升指标,$GDP_P$ 为人均 GDP

即产业升级指标,$\eta_{non-fossil}$ 为非化石能源消费占比即能源脱碳指标,$R$ 为电力消费占比即能源消费电气化指标,$e$ 为误差项,$\alpha_{1}, \alpha_{2}, \alpha_{3}, \alpha_{4}$ 分别表示各因子的弹性系数。

\section{基于遗传算法与分子运动论改进的粒子群的路径优化}

\subsection{碳达峰与碳中和目标与路径规划模型建立}

(1) 确定目标函数

针对该问题,确定规划模型的目标函数为最大化能源利用效率、产业(产品)升级程度、能源脱碳程度和能源消费电气化程度,其表达式如下所示:

\begin{equation}
Goal = Max \, f\left(\eta_{E}, G, \eta_{non-fossil}, R\right)
\tag{7.25}
\end{equation}

\begin{equation}
f = \ln a + \alpha_{1} \ln (\eta_{E})_{k} + \alpha_{2} \ln \left(GDP_{P}\right)_{k} + \alpha_{3} \ln (\eta_{non-fossil})_{k} + \alpha_{4} \ln (R)_{k} + \varepsilon \ln e
\tag{7.26}
\end{equation}

(2) 定义决策变量

该规划模型的决策变量为 4 个指标即能源利用效率、人均 GDP、非化石能源消费占比以及电力消费比重,分别代表着能效提升、产业(产品)升级、能源脱碳、能源消费电气化四大重点工程。将上述有关决策变量表达式整理如下:

\begin{equation}
\begin{cases}
\eta_{E}(t) = \frac{2 \cdot e^{\frac{-k \cdot E_{GDP}(t)}{TH(t)}}}{1 + e^{\frac{-k \cdot E_{GDP}(t)}{TH(t)}}} \\
G(t) = 0.7644t + 11.67 \\
\eta_{non-fossil}(t) = r \cdot \left(\frac{1}{CE(t)}\right) + b \\
R(t) = 0.79t + 29.19
\end{cases}
\tag{7.27}
\end{equation}

(3) 规范约束条件

根据题目要求,2035 年的 GDP 比基期(2020 年)翻一番;2060 年比基期翻两番;2060 年生态碳汇的碳消纳量为基期碳排放量的 10%;2060 年工程碳汇或碳交易的碳消纳量为基期碳排放量 10%。因此构建的约束条件如下所示:

\begin{equation}
\begin{cases}
GDP_{t=2035} = 2 \times GDP_{t=2020} \\
GDP_{t=2060} = 4 \times GDP_{t=2020} \\
C_{t=2060} \leqslant 0.2 \times C_{t=2020}
\end{cases}
\tag{7.28}
\end{equation}

根据题意,需要在给定的时间(2025 年、2030 年、2035 年、2050 年和 2060 年)情形下,分别求解目标函数的极值,所以构建的时间 $t$ 约束如下:

\begin{equation}
t = 2025, 2030, 2035, 2050, 2060
\end{equation}

(4) 建立综合模型

综上所述,针对如下的碳达峰与碳中和目标与路径规划模型:

\begin{equation}
\left\{
\begin{aligned}
Goal &= Max \, f\left(\eta_{E}, G, \eta_{non-fossil}, R\right) \\
\eta_{E}(t) &= \frac{2 \cdot e^{-k \cdot \frac{E_{GDP}(t)}{TH(t)}}}{1 + e^{-k \cdot \frac{E_{GDP}(t)}{TH(t)}}} \\
G(t) &= 0.7644t + 11.67 \\
\eta_{non-fossil}(t) &= r \cdot \left(\frac{1}{CE(t)}\right) + b \\
R(t) &= 0.79t + 29.19 \\
GDP_{t=2035} &= 2 \times GDP_{t=2020} \\
GDP_{t=2060} &= 4 \times GDP_{t=2020} \\
C_{t=2060} &\leq 0.2 \times C_{t=2020} \\
t &= 2020, 2025, 2030, 2050, 2060
\end{aligned}
\right.
\tag{7.29}
\end{equation}

\subsection{7.6.2 粒子群算法介绍}

粒子群算法(Particle Swarm Optimization, PSO)是一种进化计算技术,源于对鸟群觅食行为的模拟。它通过群体中个体之间的协作和信息共享来寻找最优解 \cite{18}。

在粒子群算法中,每个优化问题的候选解被视为一个粒子,每个粒子都有一个位置和一个速度。粒子仅具有两个属性:速度和位置,速度是矢量,包括速度大小和方向,位置表示粒子当前所在的坐标。

粒子群算法的基本思想是利用群体中的个体对信息的共享使整个群体的运动在问题求解空间中产生从无序到有序的演化过程,从而获得问题的可行解。

该算法通过初始化为一群随机的粒子(随机解),然后根据迭代找到最优解。每一次迭代中,粒子通过跟踪两个极值来更新自己,第1个是粒子本身所找到的最优解,这个称为个体极值;第2个是整个种群目前找到的最优解,这个称为全局极值。也可以不用整个种群,而是用其中的一部分作为粒子的邻居,称为局部极值。下式为粒子群算法更新位置的公式 \cite{19}。

\begin{equation}
V_{i} = \alpha V_{i} - c_{1}(P_{i} - X_{i}) + c_{2}(P_{g} - X_{i})
\tag{7.30}
\end{equation}

\begin{equation}
X_{i} = X_{i} + \beta V_{i}
\tag{7.31}
\end{equation}

其中,$i = 1, \dots, N$,N是种群规模;$V_{i}$表示当前$i$粒子的速度;$X_{i}$表示当前粒子$i$的位置;$P_{i}$表示粒子$i$迄今为止所搜索到的最优位置;$P_{g}$表示整个粒子群迄今为止所搜索的最优位置;$c_{1}$、$c_{2}$为学习因子;$\alpha$为惯性因子;$\beta$为速度控制约束因子。

下图给出了粒子群优化算法的流程图。

\begin{figure}[h]
    \centering
    \includegraphics[width=0.8\textwidth]{pso_flowchart.png}
    \caption{PSO算法流程图}
    \label{fig:pso_flowchart}
\end{figure}

\subsection{基于 GA-MPSO 算法的模型求解}

GA-MPSO 在综合了遗传算法(GA)和基于分子运动论的粒子群优化算法(MPSO)优点的同时,有克服遗传算法容易达到局部最优并利用分子运动论粒子群算法快速收敛的特性。该算法的目标即寻找恰当的结合点,并利用 GA 和 MPSO 的搜索能力,避免过早收敛。该算法的主要步骤如下所示。

\begin{table}[h]
    \centering
    \caption{基于遗传算法与分子运动论改进粒子群优化算法步骤}
    \begin{tabular}{l}
        \hline
        Step1:GA 和 MPSO 参数初始化(种群规模 N、变异概率 $P_{1}$、交叉概率 $P_{2}$、惯性因子 $\alpha$); \\
        Step2:根据适应度公式,计算每个粒子的适应值;并更新粒子群产生新的种群 $pop_{1}$; \\
        Step3:对种群 $pop_{1}$ 中的各个体根据适应度公式进行评估,并随机选出染色体对,以概率 $P_{2}$ 进行交叉操作,得到种群 $pop'$; \\
        Step4:对种群 $pop'$ 以概率 $P$ 进行方向变异操作,产生新的种群 $pop+$; \\
        Step5:从种群 $pop_{1}$ 和种群 $pop+$ 中挑选精英产生下一代种群 $pop''$; \\
        Step6:若超过迭代次数或当前的最优个体的误差满足要求,则输出结果,否则返回 Step2。 \\
        \hline
    \end{tabular}
    \label{tab:ga_mpso_steps}
\end{table}

\begin{table}[h]
\centering
\caption{基于遗传算法与分子运动论改进粒子群算法参数设置表}
\begin{tabular}{l l l}
\hline
参数名称 & 符号 & 值 \\
\hline
种群大小 & $N$ & 120 \\
维数 & $D$ & 4 \\
惯性因子 & $\alpha$ & 0.9 \\
学习因子 & $c_1$、$c_2$ & 1.2 \\
最大迭代步长 & $step$ & 200 \\
初始化各粒子的位置 & $X_0$ & 随机数 \\
初始化各粒子的速度 & $V_0$ & 随机数 \\
\hline
\end{tabular}
\end{table}

如下图7.3所示,在演化过程中,发现适应度曲线最终均达到收敛。说明,粒子群算法参数设置合理且求解性能较高。

\begin{figure}[h]
\centering
\includegraphics[width=\textwidth]{image.png}
\caption{适应性进化曲线图}
\end{figure}

在演化过程中,求解给定的时间(2025年、2030年、2035年、2050年和2060年)下,各样本收敛时的最优解见下表。

\begin{table}[h]
\centering
\caption{最优解求解值}
\begin{tabular}{l l l l l}
\hline
时间 & 能效 & 产业升级 & 能源脱碳 & 能源消费电气化 \\
\hline
2025年 & 38.2\% & 33.8\% & 37.5\% & 30.5\% \\
2030年 & 45.7\% & 40.5\% & 52.1\% & 37.1\% \\
2035年 & 57.3\% & 55.1\% & 60.3\% & 55.3\% \\
2050年 & 72.1\% & 69.3\% & 85.8\% & 77.8\% \\
2060年 & 81.5\% & 73.5\% & 92.6\% & 87.6\% \\
\hline
\end{tabular}
\end{table}

注:产业升级只考虑GDP能源强度下降幅度.

\subsection*{7.6.4 基于最优双碳路径下的四大工程的定性与定量分析}

通过对表30进行分析,我们可以就基准情景,即2030年恰好实现碳达峰、2060年恰好实现碳中和的情景,对能效提升、产业(产品升级)、能源脱碳、能源消费电气化四大工程进行定性及定量分析如下:

- 能效在此给定时间段须不断提升,提升速率逐渐增大后稳定,能源利用效率须于2025年达38.2\%、2030年达45.7\%、2035年达57.3\%、2050年达72.1\%、2060年达81.5\%。
- 产业在此给定时间段须不断升级,升级速率逐渐加快后减慢,人均GDP增幅须于2025年达33.8\%、2030年达40.5\%、2035年达55.1\%、2050年达69.3\%、2060年达73.5\%。
- 能源脱碳效应在此给定时间段须不断提升,提升速率逐渐减慢,非化石能源消费占比须于2025年达37.5\%、2030年达52.1\%、2035年达60.3\%、2050年达85.8\%、2060年达92.6\%。
- 能效在此给定时间段须不断提升,提升速率逐渐增大后稳定,电力消费比重须于2025年达30.5\%、2030年达37.1\%、2035年达55.3\%、2050年达77.8\%、2060年达87.6\%。

\section{模型评价与改进}

\subsection{模型的优点}

1) 对于所给附件与收集数据的处理较为完备,对于附件中指标选用皮尔逊相关性分析,避免了共线性问题。

2) 根据数据分布的特性,从附件中运用不同角度选用对应的数据集进行处理验证,充分考虑变量特性的同时,提高数据选用的精准性。

3) 问题一二的建模基于各类简单算法的组合,采用算法计算量相对较小,代码运行时间短。

4) 运用多个模型进行对照性分析,将模型的预测结果与基于双碳政策的预测值进行比较,增加了模型评判的角度。

\subsection{模型的缺点}

1) 鉴于附件时间序列较短,回归预测可能会存在着过拟合的情况。

2) 仅考虑了过滤式的指标相关性分析,没有充分考虑不同指标间的组合效果。

3) 由于时间的限制,数据资料获取及分析上存在的难度,对双碳政策的知识水平有限,因此对区域碳排放的影响因素的研究深度还有很大的挖掘空间。

4) 本研究在进行碳排放预测时,所采用的碳排放系数可能由于各类能源计量方式不同产生一定的误差,经过计算得到的碳排放数据不一定完全正确。

5) GA-WPSO 粒子群算法是一种优化算法,用于求解各种优化问题。尽管粒子群算法在某些问题上表现出色,但它也具有一些缺点和局限性。比方说收敛速度不稳定,即算法可能需要较长时间才能收敛或者可能根本无法收敛,以及在复杂的高维问题中容易陷入局部最优解。

\subsection{模型的改进与推广}

本文通过 MLP 神经网络模型获得时间序列的非线性部分的预测值,即预测残差。但采用滑动窗法的全连接网络对于序列有一定的长度要求,由于本文时间序列较短,训练精度和准确性会有所下降。可以考虑增加 MLP 神经网络的层数或者神经元的数量以增强其表示能力,通过采用更为复杂的结构,使其能够更好地学习复杂的模式。还可以考虑使用集成学习方法,如投票、堆叠或 Bagging,以结合多个简单模型的预测结果,从而提高模型残差预测的准确性。而对于 PSO 算法,本文没有进行调参后算法性能的改进性比较,可尝试增加粒子数、调整惯性权重等方式,为双碳政策规划问题提供更好的收敛效果。本论文针对生产部门碳排放量的影响因素只是从为 6 个部门,7 种能源消费品种结构进行分析,并继续深入细分。

\section*{参考文献}

[1] 任晓松, 赵涛. 中国碳排放强度及其影响因素间动态因果关系研究——以扩展型 KAYA 公式为视角 [J]. 干旱区资源与环境, 2014, 28(03): 6-10. DOI: 10.13448/j.cnki.jalre.2014.03.002.

[2] Johan A, Delphine F Koen S. A Shapley Decomposition of Carbon Emissions without Residuals [J]. Energy Policy 2002 30: 727-736.

[3] Burgund, D.; Nikolovski, S.; Galić, D.; Maravić, N. Pearson Correlation in Determination of Quality of Current Transformers. Sensors 2023, 23, 2704. https://doi.org/10.3390/s23052704

[4] Verikas A, Gelzinis A, Bacauskiene M. Mining data with random forests: a survey and results of new test [J]. Pattern Recognition, 2011, 44(2): 330-349.

[5] Breiman L. Random forests [J]. Machine Learning, 2001, 45(1): 5-32.

[6] 张谊生. "同比"与"环比"——统计术语的个案分析 [J]. 术语标准化与信息技术, 2006, 000(003): 37-41.

[7] 储莎, 陈来. 基于变异系数法的安徽省节能减排评价研究 [J]. 中国人口·资源与环境, 2011, 21(S1): 512-516.

[8] Chen W, Cai Y, Li A, Su Y, Jiang K. EEG feature selection method based on maximum information coefficient and quantum particle swarm. Sci Rep. 2023 Sep 4; 13(1): 14515. doi: 10.1038/s41598-023-41682-5. PMID: 37666919; PMCID: PMC10477332.

[9] 钟茂初. “双碳”目标有效路径及误区的理论分析 [J]. 中国地质大学学报 (社会科学版), 2022, 22(01): 10-21. DOI: 10.16493/j.cnki.42-1627/c.2022.01.012.

[10] https://china.unfpa.org/zh-Hans/publications/22070101

[11] 曹磊, 张义, 刘峰等. ARIMA-MLP 与 ARIMA-RBF 模型在流行性腮腺炎发病预测中的应用 [J]. 公共卫生与预防医学, 2016, 27(02): 26-30.

[12] 郭文伟, 陈泽鹏, 钟明. 基于 MLP 神经网络构建小企业信用风险预警模型 [J]. 财会月刊, 2013(06): 22-26. DOI: 10.19641/j.cnki.42-1290/f.2013.06.007.

[13] 中华人民共和国国家统计局. 中国统计年鉴 [M]. 北京: 中国统计出版社, 2021.

[14] Chen X, Hou Y, Xi P. Parameter estimation of the structured illumination pattern based on principal component analysis (PCA): PCA-SIM. Light Sci Appl. 2023 Feb 8; 12(1): 41. doi: 10.1038/s41377-022-01043-9. PMID: 36755013; PMCID: PMC9908970.

[15] 项目综合报告编写组. 《中国长期低碳发展战略与转型路径研究》综合报告 [J]. 中国人口·资源与环境, 2020, 30(11): 1-25.

[16] http://cenews.com.cn/index.html

[17] 唐诗航. “双碳”背景下我国产业结构优化调整的研究 [D]. 重庆工商大学, 2022. DOI: 10.27713/d.cnki.gcqgs.2022.000165.

[18] Johan Schubert. Managing Inconsistent Intelligence [C]. In: The third international conference of Information Fusion. Sunnyvale, CA, 2000

[19] P Smets, R kennes. The transferable belief model [J]. Artificial Intelligence, 1994