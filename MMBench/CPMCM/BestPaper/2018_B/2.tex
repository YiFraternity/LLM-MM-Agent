\begin{center}
\includegraphics[width=0.3\textwidth]{image1.png} \quad
\includegraphics[width=0.2\textwidth]{image2.png} \quad
\includegraphics[width=0.2\textwidth]{image3.png} \\
\includegraphics[width=0.2\textwidth]{image4.png} \quad
\includegraphics[width=0.3\textwidth]{image5.png}
\end{center}

\begin{center}
\textbf{“华为杯”第十五届中国研究生数学建模竞赛}
\end{center}

\begin{table}[h]
\centering
\begin{tabular}{l l}
学校 & 上海大学 \\
\hline
参赛队号 & 18102800030 \\
\hline
队员姓名 & 1. 张睿鑫 \\
 & 2. 夏雨蒙 \\
 & 3. 叶鹏 \\
\end{tabular}
\end{table}

\begin{center}
\textbf{“华为杯”第十五届中国研究生}
\end{center}

\begin{center}
\textbf{数学建模竞赛}
\end{center}

\begin{center}
\textbf{题目} \quad \textbf{光传送网建模与价值评估}
\end{center}

\begin{center}
\textbf{摘} \quad \textbf{要:}
\end{center}

随着光纤通信的急速发展,其在人们日常生活中扮演着越来越重要的角色,光传送网构建起的全球通信骨架,为人类提供了大容量、高可靠性和低能耗的信息传输管道。本文研究了光传送网的建模与网络价值评估问题,其中包括对光传送链路进行简单建模问题,制定光传送网规划,并探讨网络价值问题以及对传输信号星座图优化问题。

针对问题一:对光传送链路进行简单建模。分为两个子问题,子问题一中,本文首先进行公式方面的推导,得到纠前误码率与信噪比在公式上的关系,为了尽量接近实际光传送网,使用 MATLAB 对信号传输的过程直接进行建模,得到 BER-SNR 曲线,在 BER=0.02 时 QPSK,8QAM,16QAM 的 SNR 容限点分别为 7dB,10dB,13dB。子问题二,本文从 SNR 容限点入手,根据其相关定义和题设条件,对光信号在光传输链路中的传输过程进行建模分析,将每个跨段中的信号与噪声变化过程转化成数学表达式,继而得到光传输链路中噪声的递推模型,由此递推模型得到当输入光功率为 1mW 时,不同调制方式下 80KM 和 100KM 单跨光纤的最远传输距离。在此基础上进一步探究了不同大小的输入光功率对传输距离的影响,得到随着输入功率的逐渐增大,光信号的传输最远距离先增后减,并且在 1mW 附近时取得最大值。

针对问题二:制定光传送网规划,并探讨网络价值。分为三个子问题,子问题一,论文在保证每一个城市都被连接的前提下进行方案设计,首先构建关于网络价值的搜索树,找到连接 12 个区域的最大网络价值的 11 条连接,在此基础上增加连接数,最终得到连接数为 16 条时最大网络总价值为 5433,连接数为 33 条时最大网络总价值为 8906。然后通过穷举法对上述的规划模型进行了验证,证明了其合理性。子问题二,两个城市之间可以存在中间节点,并且所有中间节点传输容量可以任意分配,由传输链路存在中间传输节点的特点可知,增加一条链

路的容量至少需要降低额外的两条通道的容量,论文先分析了各相关城市的通道间容量增减的关系对于网络价值的影响,来确定是否需要增加中间节点,对于需要增加中间节点的情况,从资源,地域,包括人口的角度分析了容量如何分配。实验结果表明,只考虑中间节点时对最大网络价值影响不大,而人口数量的变化则直接对最后的连接结果以及网络总价值的产生很大的影响,也从侧面反映出当前目标函数的鲁棒性不足,易受人口变化的影响。子问题三,本文结合前两个子问题的分析和资料查阅,在目标函数中考虑到区域消费水平,人口结构,及产业结构三个因素的影响,提出两个新的目标函数,并对其进行求解及分析。

针对问题三:改进调制格式。基于调制格式信息熵需要保持 $3\text{bit}$ 的条件,文章使用 Lingo 工具得到不同数量(9-16 个)星座点情况下,概率组合均只有唯一解,为了便于实现,大多数通信系统均只采用 $N$ 为偶数的 $2^N$ 阶 QAM 调制方式,所以本文分别进行了大量 8 个点星座图以及 16 个点星座图的实验,最终在 8 个点的情况下得到优于 8QAM 的 SNR 容限点的调制方式,而 16 个点的情况下,无论星座图形状如何,同等形状下不等概率的星座图总比等概率出现的星座图性能更好。

\section*{目录}

\section*{一、问题重述}
\dotfill 4

\subsection*{1.1 研究背景}
\dotfill 4

\subsection*{1.2 研究问题}
\dotfill 4

\section*{二、问题假设}
\dotfill 4

\section*{三、符号说明与数据来源}
\dotfill 5

\subsection*{3.1 问题假设}
\dotfill 5

\subsection*{3.2 数据来源}
\dotfill 6

\section*{四、问题一:光传送链路建模}
\dotfill 6

\subsection*{4.1 纠前误码率与信噪比计算}
\dotfill 6

\subsubsection*{4.1.1 问题分析}
\dotfill 6

\subsubsection*{4.1.2 模型建立与求解}
\dotfill 6

\subsubsection*{4.1.3 实验结果分析}
\dotfill 7

\subsection*{4.2 光链路性能计算}
\dotfill 9

\subsubsection*{4.2.1 问题分析}
\dotfill 9

\subsubsection*{4.2.2 模型建立与求解}
\dotfill 9

\subsubsection*{4.2.3 实验结果分析}
\dotfill 11

\section*{五、问题二:光传送网规划}
\dotfill 12

\subsection*{5.1 子问题 1}
\dotfill 12

\subsubsection*{5.1.1 问题分析}
\dotfill 12

\subsubsection*{5.1.2 模型建立与求解}
\dotfill 13

\subsubsection*{5.1.3 模型验证}
\dotfill 17

\subsubsection*{5.1.4 实验结果及分析}
\dotfill 19

\subsection*{5.2 子问题 2}
\dotfill 19

\subsubsection*{5.2.1 问题分析}
\dotfill 19

\subsubsection*{5.2.2 模型建立与求解}
\dotfill 20

\subsubsection*{5.2.3 实验结果分析}
\dotfill 24

\subsection*{5.3 子问题 3}
\dotfill 24

\subsubsection*{5.3.1 问题分析}
\dotfill 24

\subsubsection*{5.3.2 模型建立与求解}
\dotfill 24

\subsubsection*{5.3.3 实验结果分析}
\dotfill 26

\section*{六、问题三:调制格式改进}
\dotfill 26

\subsection*{6.1 问题分析}
\dotfill 26

\subsection*{6.2 模型优化}
\dotfill 27

\subsubsection*{6.2.1 8 个点的星座图优化}
\dotfill 27

\subsubsection*{6.2.2 16 个点的星座图优化}
\dotfill 29

\section*{七、总结与展望}
\dotfill 31

\subsection*{7.1 模型评价}
\dotfill 31

\subsection*{7.2 模型改进}
\dotfill 31

\section*{参考文献}
\dotfill 31

\section*{光传送网建模与价值评估}

\section*{一、问题重述}

\subsection{1.1 研究背景}

光纤通信正是整个通信领域急速发展的前提,从诞生至今,50 多年里基于数字光纤通信技术的光传送网构建起了全球通信的骨架。从城市内的传输,到跨越大洋的传输,光传送网为人类提供了大容量、高可靠性和低能耗的信息传输管道,人类对通信容量的追求也成为光传送技术发展的源源不断的动力。

光传送网的规划与建设是运营商、设备商以及政府必须考虑的课题。光传送的基本规律是——在相同技术条件下传输的容量会随着传输距离增加而减小。网络规划者需要在有限资源的条件下,综合考虑传输距离、传输容量、网络拓扑等各种因素,以最大化网络的价值。

\subsection{1.2 研究问题}

问题一:光传送路建模问题,分为两个子问题。首先需要给出 QPSK、8QAM、16QAM 的纠前误码率与信噪比的关系;然后需要在纠前误码率一定的情况下,求三种调制方式分别在传输距离不同情况下的最远传输距离。

问题二:光传送网规划问题,分为三个子问题。首先需要在给定的全国 12 个区域中不考虑中间节点建立起一定数量连接数的网络规划,并给出最多的网络价值;然后是考虑中间节点以及容量分配重新对网络进行规划;最后是从不同角度对光传送网络的目标函数进行探究。

问题三:改善星座图问题,通过第一问中纠前误码率与信噪比的关系,探究任意改变 16QAM 星座图的位置、数量或每个点的概率,并保证调制格式信息熵保持在 3bit 的情况下,如何降低信噪比的容限点。

\section*{二、问题假设}

为了能够在建模过程中尽量接近实际光传送网,合理简化模型的分析过程,在建模时提出如下假设:

假设 1:从发射机发出的光信号为理想信号,不包含噪声;

假设 2:光传输链路中每一跨中的放大器噪声指数相同;

假设 3:光传输链路中每一跨中的放大器增益相同;

假设 4:光传输链路中每一跨中的光纤衰减值相等;

假设 5:光放大器的增益刚好可以补偿光信号的衰减;

假设 6:假设引入噪声和输入信号相互独立;

假设 7:问题 5.1 及问题 5.2 中每条连接权重均为 1。

\section*{三、符号说明与数据来源}

\subsection*{3.1 问题假设}

\begin{table}[h]
\centering
\begin{tabular}{|c|c|c|}
\hline
符号 & 含义说明 & 单位 \\
\hline
$S_{m}(t)$ & QAM信号的发送波形 & E \\
$A_{mc}$, $A_{ms}$ & 承载信息的幅度 & -- \\
$g(t)$ & 调制信号 & E \\
$f_{c}$ & 载波频率 & THz \\
$T_{s}$, $T_{c}$ & 周期 & S \\
$\mathcal{E}_{g}$ & 调制信号的能量 & mW \\
$S_{m}$ & 发送信号向量 & -- \\
$M$ & 发送信号个数 & -- \\
$\mathcal{E}_{av}$ & 平均能量 & mW \\
$P_{av}$ & 平均功率 & mW \\
$P_{M}$ & M元QAM信号的错误概率 & -- \\
$N_{0}$ & 噪声功率 & mW \\
$G_{j}$ & 第j跨放大器的增益 & -- \\
$L_{j}$ & 第j段光纤的衰减系数 & -- \\
$l_{j}$ & 第j跨中光纤的长度 & Km \\
$\Delta_{j}$ & 第j跨的增益 & -- \\
$G_{all}$ & 光纤链路的总增益 & -- \\
$P_{in}$ & 链路的输入功率 & mW \\
$P_{out}$ & 链路的输出功率 & mW \\
$P_{f}$ & 非线性噪声功率 & mW \\
$P_{n}$ & 自发辐射噪声功率 & mW \\
$h$ & 普朗克常数 & J.S \\
$f$ & 光波频率 & THz \\
$B$ & 带宽 & GHz \\
$NF$ & 噪声指数 & -- \\
$G$ & 放大器增益 & -- \\
$P_{e}$ & 噪声功率 & mW \\
$x$, $y$ & 城市A和B的人口 & 百万 \\
$R_{a}$, $R_{b}$, $R_{c}$, $R_{d}$ & 不同城市间传输通道的容量 & Tb/s \\
$V_{ABC}$, $V_{ABC}^{*}$ & A,B,C三座城市直接相连/增加节点后的总价值 & -- \\
$M_{c}$ & C市的总人口数 & 百万 \\
$\omega_{1c}$ & C市中15-64岁年龄段所占的人口比例 & -- \\
$\omega_{2c}$ & C市产业结构中第三产业在所有区域中所占比例 & -- \\
$\omega_{3c}$ & C市经济水平在所有区域中所占比例 & -- \\
$\alpha_{ic}$ & C市的第三产业比 & -- \\
$\beta_{ic}$ & C市的区域总GDP & 亿元 \\
$\sum\alpha_{i}$ & 所有区域第三产业比之和 & -- \\
$\sum\beta_{i}$ & 所有区域的GDP总和 & 亿元 \\
\hline
\end{tabular}
\end{table}

\subsection{数据来源}

人口总数:均来自百度百科,以 2017 年末统计数值为准;

城市距离:均来自百度地图,以两点间直线距离为准;

人口结构:均来自百度,以 2010 年第 6 次人口普查为准;

产业结构:均来自百度,以 2016 年产业结构统计数据为准;

区域总 GDP:均来自百度,以 2017 年末区域总 GDP 统计数值为准。

\section{问题一:光传送链路建模}

\subsection{纠前误码率与信噪比计算}

\subsubsection{问题分析}

误码率(BER)定义是错误的比特数占总传输比特数的比例,信噪比(SNR)的定义是信号和噪声功率的比值,两者均可用来衡量信号的有效性,因为二者均有明确的公式定义,在探索 BER 与 SNR 的关系时,可以根据相关定义的公式进行推导。为了尽量接近实际光传送网,可以直接对光信号传输的过程进行建模,通过科学计算工具来绘制 BER 与 SNR 的关系曲线,最终找到目标值。

\subsubsection{模型建立与求解}

正交振幅调制技术 QAM(Quadrature Amplitude Modulation)就是用调制信号对频率相同、相位正交的两个载波进行调幅,然后将已调信号加在一起进行传输或发射。一般形式定义为:
\begin{equation}
S_{m}(t)=A_{mc} g(t) \cos \left(2 \pi f_{c} t\right)-A_{ms} g(t) \sin \left(2 \pi f_{c} t\right) \quad 0 \leq t \leq T_{s}
\tag{4-1}
\end{equation}
其中,$A_{mc}, A_{ms}$ 承载信息。发送信号的向量表达式为:
\begin{equation}
S_{m}=\left[\begin{array}{ll}
A_{mc} \sqrt{\frac{\varepsilon_{s}}{2}} & A_{ms} \sqrt{\frac{\varepsilon_{s}}{2}}
\end{array}\right]
\tag{4-2}
\end{equation}
假设 M 个发送信号等概率出现,平均能量:
\begin{equation}
\varepsilon_{av}=\frac{1}{M} \sum_{m=1}^{M}\left(A_{mc}^{2} \frac{\varepsilon_{s}}{2}+A_{ms}^{2} \frac{\varepsilon_{s}}{2}\right)=\frac{\varepsilon_{s}}{2 M} \sum_{m=1}^{M}\left(A_{mc}^{2}+A_{ms}^{2}\right)
\tag{4-3}
\end{equation}
平均功率:
\begin{equation}
P_{av}=\frac{\varepsilon_{av}}{T_{c}}=\frac{\varepsilon_{s}}{2 M T_{c}} \sum_{m=1}^{M}\left(A_{mc}^{2}+A_{ms}^{2}\right)
\tag{4-4}
\end{equation}

对于 $M=2^{K}$ 且 K 为偶数的矩形星座 QAM 信号,可把它等价为两个具有正交载波的 PAM 信号,每一个含着 $\sqrt{M}=2^{k / 2}$ 个星座点,解调器可以把这两个正交载波 PAM 信号完全分开。因此,M 元 QAM 信号的错误概率可由上面分析得:

\begin{equation}
P_{M} = 1 - (1 - P_{\sqrt{M}})^2
\tag{4-5}
\end{equation}

通过查阅资料[2][7]得到:

\begin{equation}
P_{\sqrt{M}} = 2 \left( 1 - \frac{1}{\sqrt{M}} \right) Q \left( \sqrt{\frac{3}{M-1} \frac{\varepsilon_{av}}{N_0}} \right)
\tag{4-6}
\end{equation}

其中 $N_0$ 表示噪声的功率,$Q(x)$ 表示高斯 $Q$ 函数。

\begin{equation}
Q(x) = \frac{1}{\sqrt{2\pi}} \int_{x}^{\infty} e^{-t^2/2} dt, \quad x \geq 0
\tag{4-7}
\end{equation}

基于以上分析,对信号传输的过程进行建模,假设经过信道中的噪声为加性高斯白噪声(Additive White Gaussian Noise, AWGN),通过 MATLAB 对传输模型进行建模,如图 4.1 所示,$x$ 为发送序列,根据三种调制方式,每种调制方式都生成满足条件的 30000 个随机信号点。

\begin{figure}[h]
\centering
\includegraphics[width=0.8\textwidth]{image1.png}
\caption{传输模型}
\label{fig:4.1}
\end{figure}

通过 MATLAB 对传输模型进行建模过程如下:调用 MATLAB 中的 modem.genqammod 和 modem.genqamdmod 函数,通过描述图 4.2 给出的星座图中点的位置来构建调制器和解调器对信号进行调制解调,$n$ 为高斯白噪声,$z$ 为接收序列。由上面传输模型可以分别得到三种调制方式对应的解调信号,最后通过 biterr 函数求出误比特率和误码率,进而计算出 QPSK,8QAM 和 16QAM 三种情况下的误码率-信噪比(BER-SNR)曲线,最后求出 BER=0.02 时 SNR 的容限点。详细代码见附录。

\begin{figure}[h]
\centering
\includegraphics[width=\textwidth]{image2.png}
\caption{三种调制方式的星座图}
\label{fig:4.2}
\end{figure}

\subsection{4.1.3 实验结果分析}

图 4.3 为不同调制方式下的 BER-SNR 曲线,图 4.4 为不同调制方式下的发送信号的星座图,与题目的要求一致,图 4.5 为不同调制方式下的接收信号的星座图。从图 4.3 中可以得出,当 BER =0.02 时:

\begin{align*}
\text{SNR(QPSK)} &\approx 7(\text{dB}) \\
\text{SNR(8QAM)} &\approx 10(\text{dB}) \\
\text{SNR(16QAM)} &\approx 13(\text{dB})
\end{align*}

模型的仿真结果如下所示:

\begin{figure}[h]
    \centering
    \includegraphics[width=\textwidth]{image1.png}
    \caption{不同调制方式下的 BER-SNR 曲线}
    \label{fig:ber_snr}
\end{figure}

\begin{figure}[h]
    \centering
    \includegraphics[width=\textwidth]{image2.png}
    \caption{不同调制方式下的发送信号星座图}
    \label{fig:send_constellation}
\end{figure}

\begin{figure}[h]
    \centering
    \includegraphics[width=\textwidth]{image3.png}
    \caption{不同调制方式下的接收信号星座图}
    \label{fig:receive_constellation}
\end{figure}

\subsection{光链路性能计算}

\subsubsection{问题分析}

由 4.1 节已经可以得到 BER-SNR 的关系式以及关系曲线,在设定了纠前 BER 门限时我们也可以同时得到 SNR 容限点,对于此问可以从两个角度出发进行思考:思路一是可以直接从 BER 的角度,直接通过科学工具分别模拟在单跨传输距离为 80KM 和 100KM 时,不同调制方式下的信号,在不同数量跨段组成的光传输链路模型中的 BER,并将它们绘制出来与纠前 BER 门限值作对比,曲线的交点即为最远传输距离。思路二则是通过纠前 BER 计算得到 SNR 门限值,因为门限值有简单明确的定义,并且根据其相关定义和题设条件,每跨中间信噪比保持不变,每跨叠加的噪声间相互独立,经过分析此过程可以经过公式推导得到关系式,更易建模,且理论依据更易表述,综上所述,本文采用思路二,从 SNR 容限点入手,结合 4.1 节的结论,对本题进行建模及求解。

\subsubsection{模型建立与求解}

\textbf{模型建立:}

\begin{figure}[h]
    \centering
    \includegraphics[width=\textwidth]{image.png}
    \caption{光传输链路模型}
    \label{fig:4.6}
\end{figure}

参考文献[3][4]得到如下分析,假设现在有一个如图 4.6 所示的多跨级联光线路放大器系统,有 N 跨级联,输入端光功率为 $P_{in}$,每一跨中放大器的增益分别为 $G_1, G_2, \ldots, G_N$,光纤段衰减系数分别为 $L_1, L_2, \ldots, L_N$,将衰减系数表示为:

\begin{equation}
L_j = e^{-\alpha l_j} \quad (j = 1, 2, \ldots, N)
\tag{4-8}
\end{equation}

$\alpha$ 表示某一常数,$l_j$ 表示第 $j$ 跨中光纤的长度。用 $\Delta_j$ 表示第 $j$ 跨中光线衰减与第 $j$ 个放大器增益的乘积,即:

\begin{equation}
\Delta_j = L_j G_j \quad (j = 1, 2, \ldots, N)
\tag{4-9}
\end{equation}

光信号经过此传输链路传输以后总增益 $G_{all}$ 可以表示为:

\begin{equation}
G_{all} = \Delta_1 \Delta_2 \cdots \Delta_N
\tag{4-10}
\end{equation}

已假设信号与噪声相互独立,最后传输链路输出端的光信号功率 $P_{out}$ 表示为:

\begin{equation}
P_{out} = P_{in} G_{all}
\tag{4-11}
\end{equation}

在信号传输过程中每一跨中主要考虑两种噪声,一种是由于光纤非线性效应引入的非线性噪声功率 \( P_f \),另一种是由放大器自身引入的自发辐射噪声功率 \( P_n \)。已知光纤的非线性噪声与入纤功率 \( P_{in}^j \)(j=1,2,…,N)呈平方关系可得:

\[
P_f^j = A \left( P_{in}^j \right)^{\wedge 2} \quad (j=1,2,\ldots,N) \tag{4-12}
\]

\[
P_n = 2\pi hfB \left( NF - 1/G \right) \tag{4-13}
\]

为了不与变量的上标混淆,式 4-12 平方关系用 \((\cdot)^{\wedge 2}\) 的形式表示。式 4-13 中 \( h \) 是普朗克常数,\( f \) 是光波频率,\( B \) 表示带宽,\( NF \) 为噪声指数,\( G \) 为放大器增益。假设从发射机发出的信号只有光信号,由上面的推理可得每跨的输入与输出的关系式有如下关系:

\[
P_{out}^1 = \left( P_{in}^1 + P_f^1 \right) \Delta_1 + P_n \quad \text{且} \quad P_{in}^1 = P_{in} \tag{4-14}
\]

\[
P_{out}^2 = \left( P_{out}^1 + A \left( P_{out}^1 + P_f^2 \right)^{\wedge 2} \right) \Delta_2 + P_n \tag{4-15}
\]

\[
\ldots \ldots
\]

\[
P_{out}^N = \left( P_{out}^{N-1} + A \left( P_{out}^{N-1} + P_f^N \right)^{\wedge 2} \right) \Delta_N + P_n \tag{4-16}
\]

由假设条件可知,每一次光放大器刚好可以补偿光信号功率的衰减,并且光信号与噪声相互独立,可以得到表达式 4-9 为 \(\Delta = 1\),因此 \( G_{all} = 1 \),且可以将 4-14——4-16 分开写成光信号功率 \( P_s \) 和噪声功率 \( P_e \) 两部分和的形式,因为光功率经过放大器后衰减被刚好补偿,保持不变,其中噪声部分 \( P_e \) 存在下列关系式,即:

\[
P_e^1 = P_f^1 + P_n \tag{4-17}
\]

\[
P_e^2 = P_e^1 + P_f^2 + P_n \tag{4-18}
\]

\[
\ldots \ldots
\]

\[
P_e^N = P_e^{N-1} + P_f^N + P_n \tag{4-19}
\]

模型求解:

由题可知光信号每传输 15km,光功率衰减一半,代入式 4-8 可求解出衰减表达式中得常数 \(\alpha\):

\[
e^{-15\alpha} = 0.5 \rightarrow \alpha \approx 0.0462
\]

当传输中光纤长度为 80KM 时,衰减 \( L = e^{-0.0462 \times 80} \approx 0.0248 \),对应得放大器增益

\begin{equation}
G = \frac{1}{L} \approx 40.3225, \text{ 这里已知 } h = 6.62606896 \times 10^{-34} \, \text{J} \cdot \text{S}, \text{ 为了便于分析和计算,}
\end{equation}

设定光波频率 \( f = 193.1 \, \text{THz} \), 带宽 \( B = 50 \, \text{GHz} \), 噪声指数 \( NF = 4 \), 将上面数值代入式 4-13, 求得放大器自发辐射噪声:

\begin{equation}
P_n = 2\pi \times 6.62606896 \times 10^{-34} \times 193.1 \times 10^{12} \times 50 \times 10^9 \left( 4 - \frac{1}{40.3225} \right) \approx 1.5979 \times 10^{-7} \, \text{W}
\end{equation}

当输入光信号功率 \( P_{in} \) 为 \( 0.001 \, \text{W} \) 时可得:

\begin{equation}
A (0.001)^2 = \frac{2}{3} \times 1.5979 \times 10^{-7} \rightarrow A \approx 0.1065
\end{equation}

将以上所有推导所得值代入 4-17—4-19 的递推式可以得到最终的总噪声 \( P_e^N \), 输入光信号功率 \( P_{in} \) 设为 \( 1 \, \text{mW} \), 最终输出光信号功率经过放大器补偿仍为 \( P_{in} \), 由 SNR 的定义式最终可得不同调制方式下的 SNR 值:

\begin{equation}
\text{SNR} = \frac{P_{in}}{P_e^N} \tag{4-20}
\end{equation}

将由式 4-20 求解到的值与 4.1 节中得到的 SNR 门限值做一个比对即可得到不同调制方式在单跨传输距离为 \( 80 \, \text{KM} \) 时的最远传输距离。当单跨传输距离为 \( 100 \, \text{KM} \) 时分析及建模方式同上。

\subsection*{4.2.3 实验结果分析}

根据 4.2.2 节的建模及求解, 得到传输线路中关于总噪声的递推式 4-19, 对此过程通过 MATLAB 代码实现, 将已知数值均代入方程, 最终得到 \( 80 \, \text{KM} \) 和 \( 100 \, \text{KM} \) 单纤传输距离下的三种不同调制方式的传输距离与 SNR 之间的关系曲线, 如图 4.7 所示。

\begin{figure}[h]
    \centering
    \includegraphics[width=0.45\textwidth]{image1.png}
    \includegraphics[width=0.45\textwidth]{image2.png}
    \caption{不同传输距离下 SNR 与跨段数的关系曲线}
    \label{fig:4.7}
\end{figure}

由 4.1 节可得在 QPSK, 8QAM, 16QAM 调制方式下,当纠前 BER 为 0.02 时,对应得 SNR 容限点分别为 7dB, 10dB, 13dB,将这三种调制方式的容限点一并画在图 4.7 中,交点即为信号传输最远可取的段数,由上图可得:  
80KM 时不同传输格式传输得最远距离分别为:
\begin{align*}
\text{QPSK: } 692 \times 80 &= 55360 \, \text{KM} \\
\text{8QAM: } 360 \times 80 &= 28800 \, \text{KM} \\
\text{16QAM: } 184 \times 80 &= 14720 \, \text{KM}
\end{align*}
100KM 时不同传输格式传输得最远距离分别为:
\begin{align*}
\text{QPSK: } 689 \times 100 &= 68900 \, \text{KM} \\
\text{8QAM: } 359 \times 100 &= 35900 \, \text{KM} \\
\text{16QAM: } 183 \times 100 &= 18300 \, \text{KM}
\end{align*}

在上述求解过程中我们是在发射机输入光功率 \( P_{in} = 1 \, \text{mW} \) 的情况下,基于上述建模过程及分析,本文继续探究了不同的输入光功率对信号最大传输距离的影响,即通过改变输入功率的大小观察最大传输段数的变化,最终得到如图 4.8 的结果:

\begin{figure}[h]
    \centering
    \includegraphics[width=0.45\textwidth]{image1.png}
    \includegraphics[width=0.45\textwidth]{image2.png}
    \caption{跨段数与输入功率关系曲线}
    \label{fig:4.8}
\end{figure}

由上图可以看出,在本文的建模下不论哪种调制格式,最远传输段数并不是随着输入光功率的增加而线性增长的,三种调制格式的最远传输段数均随着输入光功率的增加有一个先增再减的过程,且三种变化方式几乎在同一输入功率下取得最大值。

\section*{五、问题二:光传送网规划}

\subsection{5.1 子问题 1}

\subsubsection{5.1.1 问题分析}

本问需要基于题目给出的 12 个区域需要分别规划 16 条及 33 条相关城市间的连接,并给出相关的最多的网络价值,不考虑中间节点。首先在本文中明确的一点是所有连接的情况得保证每一个城市都有被连接起来。虽然两点之间可以直接建立多条连接,但是在线路规划的同时不能一味的只考虑网络价值而忽略实用因素。由上面的思考出发,12 个城市都要被连接,最少需要 11 条连接线,解题

\begin{table}[h]
\centering
\begin{tabular}{|c|c|c|c|c|}
\hline
编号 & 单波传输容量 & 最大传输距离 & 总容量 \\ \hline
1 & $100Gb/s$ & $3000km$ & $8Tb/s$ \\ \hline
2 & $200Gb/s$ & $1200km$ & $16Tb/s$ \\ \hline
3 & $400Gb/s$ & $600km$ & $32Tb/s$ \\ \hline
\end{tabular}
\caption{不同传输格式的传输距离}
\end{table}

\begin{table}[h]
\centering
\begin{tabular}{|c|c|c|c|c|c|c|c|c|c|c|c|c|}
\hline
距离 & 乌鲁木齐 & 哈尔滨 & 北京\&天津 & 西安 & 郑州 & 上海 & 拉萨 & 成都 & 重庆 & 武汉 & 昆明 & 广州\&深圳 \\ \hline
乌鲁木齐 & 0 & 3020 & 2500 & 2120 & 2458 & 3259 & 1606 & 2065 & 2308 & 2760 & 2520 & 3290 \\ \hline
哈尔滨 &  & 0 & 1046 & 1958 & 1646 & 1678 & 3560 & 2581 & 2512 & 1999 & 3144 & 2789 \\ \hline
北京\&天津 &  &  & 0 & 900 & 565 & 940 & 2570 & 1523 & 1437 & 979 & 2076 & 1811 \\ \hline
西安 &  &  &  & 0 & 433 & 1219 & 1762 & 617 & 573 & 653 & 1195 & 1317 \\ \hline
郑州 &  &  &  &  & 0 & 831 & 2188 & 1009 & 886 & 472 & 1505 & 1294 \\ \hline
上海 &  &  &  &  &  & 0 & 2900 & 1661 & 1447 & 684 & 1958 & 1210 \\ \hline
拉萨 &  &  &  &  &  &  & 0 & 1251 & 1494 & 2232 & 1265 & 2320 \\ \hline
成都 &  &  &  &  &  &  &  & 0 & 263 & 981 & 642 & 1234 \\ \hline
重庆 &  &  &  &  &  &  &  &  & 0 & 771 & 617 & 1296 \\ \hline
武汉 &  &  &  &  &  &  &  &  &  & 0 & 1303 & 840 \\ \hline
昆明 &  &  &  &  &  &  &  &  &  &  & 0 & 1081 \\ \hline
广州\&深圳 &  &  &  &  &  &  &  &  &  &  &  & 0 \\ \hline
\end{tabular}
\caption{距离的邻接矩阵}
\end{table}

\begin{table}
\centering
\begin{tabular}{|c|c|c|c|c|c|c|c|c|c|c|c|c|}
\hline
方案 & 乌鲁木齐 & 哈尔滨 & 北京&天津 & 西安 & 郑州 & 上海 & 拉萨 & 成都 & 重庆 & 武汉 & 昆明 & 广州&深圳 \\
\hline
乌鲁木齐 & 0 & 0 & 1 & 1 & 1 & 0 & 1 & 1 & 1 & 1 & 1 & 1 & 0 \\
\hline
哈尔滨 & & 0 & 2 & 1 & 1 & 1 & 0 & 1 & 1 & 1 & 1 & 0 & 1 \\
\hline
北京&天津 & & & 0 & 2 & 3 & 2 & 1 & 1 & 1 & 2 & 1 & 1 \\
\hline
西安 & & & & 0 & 3 & 1 & 1 & 1 & 2 & 3 & 2 & 2 & 1 \\
\hline
郑州 & & & & & 0 & 2 & 2 & 1 & 2 & 2 & 3 & 1 & 1 \\
\hline
上海 & & & & & & 0 & 0 & 1 & 1 & 1 & 2 & 1 & 1 \\
\hline
拉萨 & & & & & & & & 0 & 1 & 1 & 1 & 1 & 1 \\
\hline
成都 & & & & & & & & & 0 & 3 & 2 & 2 & 1 \\
\hline
重庆 & & & & & & & & & & 0 & 2 & 2 & 1 \\
\hline
武汉 & & & & & & & & & & & 0 & 1 & 2 \\
\hline
昆明 & & & & & & & & & & & & 0 & 2 \\
\hline
广州&深圳 & & & & & & & & & & & & 0 \\
\hline
\end{tabular}
\caption{传输格式的邻接矩阵}
\end{table}

再由网上查阅相关城市的人口资料,得出图5.4中的各个区域的人口数(百万),数值精确到小数点后2位:

\begin{table}
\centering
\begin{tabular}{|c|c|c|c|c|c|c|c|c|c|c|c|c|}
\hline
人口 & 乌鲁木齐 & 哈尔滨 & 北京&天津 & 西安 & 郑州 & 上海 & 拉萨 & 成都 & 重庆 & 武汉 & 昆明 & 广州&深圳 \\
\hline
2.22 & 10.92 & 37.26 & 9.62 & 9.57 & 24.18 & 0.9 & 16.04 & 30.75 & 10.91 & 6.73 & 32.68 \\
\hline
\end{tabular}
\caption{不同区域人口数}
\end{table}

最终根据公式5-1可以得到最后的价值邻接矩阵,其中0代表两个城市间没有连接,如图5.5所示,数值精确到小数点后2位:

\begin{table}
\centering
\begin{tabular}{|c|c|c|c|c|c|c|c|c|c|c|c|c|c|}
\hline
价值 & 乌鲁木齐 & 哈尔滨 & 北京&天津 & 西安 & 郑州 & 上海 & 拉萨 & 成都 & 重庆 & 武汉 & 昆明 & 广州&深圳 \\
\hline
乌鲁木齐 & 0 & 0 & 72.76 & 36.97 & 36.87 & 0 & 11.31 & 47.74 & 66.1 & 39.37 & 30.92 & 0 \\
\hline
哈尔滨 & & 0 & 322.74 & 82 & 81.78 & 130 & 0 & 105.88 & 146.6 & 87.32 & 0 & 151.13 \\
\hline
北京&天津 & & & 0 & 302.92 & 604.26 & 480.25 & 46.33 & 195.58 & 270.79 & 322.59 & 126.68 & 279.16 \\
\hline
西安 & & & & 0 & 307.04 & 122.01 & 23.54 & 198.75 & 550.38 & 163.92 & 128.74 & 141.85 \\
\hline
郑州 & & & & & 0 & 243.39 & 23.48 & 198.23 & 274.47 & 326.98 & 64.2 & 141.48 \\
\hline
上海 & & & & & & 0 & 37.32 & 157.55 & 218.14 & 259.87 & 102.05 & 224.88 \\
\hline
拉萨 & & & & & & & & 0 & 30.4 & 42.09 & 25.07 & 19.69 & 43.39 \\
\hline
成都 & & & & & & & & & 0 & 710.68 & 211.66 & 166.24 & 183.16 \\
\hline
重庆 & & & & & & & & & & 0 & 293.06 & 230.17 & 253.6 \\
\hline
武汉 & & & & & & & & & & & 0 & 68.55 & 302.12 \\
\hline
昆明 & & & & & & & & & & & & 0 & 237.28 \\
\hline
广州&深圳 & & & & & & & & & & & & 0 \\
\hline
\end{tabular}
\caption{连接价值的邻接矩阵}
\end{table}

将12个区域假设为树的12个节点,根据树的性质,一颗N个结点的树有N-1条边,每个子点都是连通的。因此,选择11条链路构成树以满足12个区域相连的情况,再增添剩余链路中最大的n-11条就能构成网络。图5.6为网络规划模型的流程图:

\begin{figure}[h]
    \centering
    \includegraphics[width=0.8\textwidth]{network_planning_flowchart.png}
    \caption{网络规划模型流程图}
    \label{fig:network_planning_flowchart}
\end{figure}

首先将61条链路从大到小排序,每次取最大的一条链路连通,然后判断已经接通的链路是否闭合,树中的边不会形成闭合形状,如果新连通的链路形成了闭合情况,则把新连通的链路用剩余链路中价值最大的一条进行更换,直到找到11条不形成闭合形状的链路,最后将剩余链路中价值最大的n-11条链路再连通,就形成了网络结构。最终得到的价值连接邻接表,“\textbackslash”表示两地之间不连接,如图\ref{tab:connection_values}及图\ref{fig:network_structure}所示,数值精确到小数点后2位。

\begin{table}[h]
    \centering
    \begin{tabular}{|c|c|c|c|c|c|c|c|c|c|c|c|c|}
        \hline
        价值 & 乌鲁木齐 & 哈尔滨 & 北京\&天津 & 西安 & 郑州 & 上海 & 拉萨 & 成都 & 重庆 & 武汉 & 昆明 & 广州\&深圳 \\
        \hline
        乌鲁木齐 & \textbackslash & \textbackslash & 72.76 & \textbackslash & \textbackslash & \textbackslash & \textbackslash & \textbackslash & \textbackslash & \textbackslash & \textbackslash & \textbackslash \\
        \hline
        哈尔滨 & \textbackslash & \textbackslash & 322.74 & \textbackslash & \textbackslash & \textbackslash & \textbackslash & \textbackslash & \textbackslash & \textbackslash & \textbackslash & \textbackslash \\
        \hline
        北京\&天津 & \textbackslash & \textbackslash & \textbackslash & 302.92 & 604.26 & 480.25 & 46.33 & \textbackslash & \textbackslash & 322.59 & \textbackslash & 279.16 \\
        \hline
        西安 & \textbackslash & \textbackslash & \textbackslash & \textbackslash & 307.04 & \textbackslash & \textbackslash & \textbackslash & 550.38 & \textbackslash & \textbackslash & \textbackslash \\
        \hline
        郑州 & \textbackslash & \textbackslash & \textbackslash & \textbackslash & \textbackslash & \textbackslash & \textbackslash & \textbackslash & 274.47 & 326.98 & \textbackslash & \textbackslash \\
        \hline
        上海 & \textbackslash & \textbackslash & \textbackslash & \textbackslash & \textbackslash & \textbackslash & \textbackslash & \textbackslash & \textbackslash & \textbackslash & \textbackslash & \textbackslash \\
        \hline
        拉萨 & \textbackslash & \textbackslash & \textbackslash & \textbackslash & \textbackslash & \textbackslash & \textbackslash & \textbackslash & \textbackslash & \textbackslash & \textbackslash & \textbackslash \\
        \hline
        成都 & \textbackslash & \textbackslash & \textbackslash & \textbackslash & \textbackslash & \textbackslash & \textbackslash & \textbackslash & 710.68 & \textbackslash & \textbackslash & \textbackslash \\
        \hline
        重庆 & \textbackslash & \textbackslash & \textbackslash & \textbackslash & \textbackslash & \textbackslash & \textbackslash & \textbackslash & \textbackslash & 293.06 & \textbackslash & \textbackslash \\
        \hline
        武汉 & \textbackslash & \textbackslash & \textbackslash & \textbackslash & \textbackslash & \textbackslash & \textbackslash & \textbackslash & \textbackslash & \textbackslash & \textbackslash & 302.12 \\
        \hline
        昆明 & \textbackslash & \textbackslash & \textbackslash & \textbackslash & \textbackslash & \textbackslash & \textbackslash & \textbackslash & \textbackslash & \textbackslash & \textbackslash & 237.28 \\
        \hline
        广州\&深圳 & \textbackslash & \textbackslash & \textbackslash & \textbackslash & \textbackslash & \textbackslash & \textbackslash & \textbackslash & \textbackslash & \textbackslash & \textbackslash & \textbackslash \\
        \hline
    \end{tabular}
    \caption{16条连接线的价值邻接表}
    \label{tab:connection_values}
\end{table}

\begin{figure}[h]
    \centering
    \includegraphics[width=0.8\textwidth]{network_structure.png}
    \caption{网络结构图}
    \label{fig:network_structure}
\end{figure}

\begin{table}
\centering
\begin{tabular}{|c|c|c|c|c|c|c|c|c|c|c|c|c|}
\hline
价值 & 乌鲁木齐 & 哈尔滨 & 北京&天津 & 西安 & 郑州 & 上海 & 拉萨 & 成都 & 重庆 & 武汉 & 昆明 & 广州&深圳 \\ \hline
乌鲁木齐 & \ & \ & 72.76 & \ & \ & \ & \ & \ & \ & \ & \ & \ & \ \\ \hline
哈尔滨 & \ & \ & 322.74 & \ & \ & \ & \ & \ & \ & 146.6 & \ & \ & 151.13 \\ \hline
北京&天津 & \ & \ & 302.92 & 604.26 & 480.25 & 46.33 & 195.58 & 270.79 & 322.59 & \ & \ & 279.16 \\ \hline
西安 & \ & \ & \ & 307.04 & \ & \ & \ & 198.75 & 550.38 & 163.92 & \ & \ & \ \\ \hline
郑州 & \ & \ & \ & \ & \ & 243.39 & \ & 198.23 & 274.47 & 326.98 & \ & \ & \ \\ \hline
上海 & \ & \ & \ & \ & \ & \ & \ & 157.55 & 218.14 & 259.87 & \ & \ & 224.88 \\ \hline
拉萨 & \ & \ & \ & \ & \ & \ & \ & \ & \ & \ & \ & \ & \ \\ \hline
成都 & \ & \ & \ & \ & \ & \ & \ & \ & 710.68 & 211.66 & 166.24 & \ & 183.16 \\ \hline
重庆 & \ & \ & \ & \ & \ & \ & \ & \ & \ & 293.06 & 230.17 & \ & 253.6 \\ \hline
武汉 & \ & \ & \ & \ & \ & \ & \ & \ & \ & \ & \ & \ & 302.12 \\ \hline
昆明 & \ & \ & \ & \ & \ & \ & \ & \ & \ & \ & \ & \ & 237.28 \\ \hline
广州&深圳 & \ & \ & \ & \ & \ & \ & \ & \ & \ & \ & \ & \ \\ \hline
\end{tabular}
\caption{图5.8 33条连接线的价值邻接表}
\end{table}

\begin{figure}[h]
\centering
\includegraphics[width=\textwidth]{image.png}
\caption{图5.9 16条链路网络规划}
\end{figure}

基于上述算法,图5.9中黑色连接部分为求得的11条基本线路,然后红色的5条为在前面的基础上增加的连接数,最终这种连接方法得到最大的网络价值为5433。同理,当连接数为33时,在16条的基础上根据剩余连接可能的价值排序再添加额外排在前面的17条连接,如图5.10中绿色部分所示,最终得到33条连接方式时的规划图,此种规划下网络的最大价值为8906。

\begin{figure}[h]
    \centering
    \includegraphics[width=\textwidth]{image.png}
    \caption{33 条链路网络规划}
    \label{fig:network}
\end{figure}

\subsection{模型验证}

为了验证本文思路的正确性,本文采用了最朴素的算法,对所有的连接情况进行依次遍历,在保证所有区域都被连接的前提下,进行网络规划模型重新建模。在所有可以直接连接的 61 条链路中,将所有链路的价值进行排序,将价值最高的 \( n \) 条链路连通,并判断是否所有的区域都是连通。如果所有的区域没有连通,则将 \( n \) 条链路中价值最低的一条用剩下的价值最大的链路代替,如果剩下的所有链路遍历完不能找到连通结果,则将价值最低的两条用剩下的链路中最大的两条代替,以此类推,直到找到 \( n \) 条链路连通所有区域的结果。以 16 条链路的情况为例,模型流程图如图 \ref{fig:flowchart} 所示。

\begin{figure}[h]
    \centering
    \includegraphics[width=\textwidth]{flowchart.png}
    \caption{模型流程图}
    \label{fig:flowchart}
\end{figure}

\begin{figure}[h]
    \centering
    \includegraphics[width=0.8\textwidth]{network_flowchart.png}
    \caption{网络规划模型流程图}
    \label{fig:network_flowchart}
\end{figure}

最终可以得到连接数为 16 条以及 33 条的结果为:

\begin{figure}[h]
    \centering
    \includegraphics[width=\textwidth]{network_planning_16_links.png}
    \caption{16 条链路网络规划}
    \label{fig:network_planning_16_links}
\end{figure}

\begin{figure}[h]
    \centering
    \includegraphics[width=\textwidth]{image.png}
    \caption{33 条链路网络规划}
    \label{fig:network}
\end{figure}

\subsection{实验结果及分析}

本问题通过基于将 12 个区域都连通的前提下,进行了搜索树的构建,得到最终解,16 条连接线时网络的最大总价值为 5433,33 条则是基于 16 条连接线的基础上再添加 17 条,得到最大价值为 8906。并通过穷举法依次遍历所有可能分别得到最后不同连接数下的连接情况及最优解,得到与本文建立的模型一样的网络总价值结果,从而验证了建模方法的正确性。

\subsection{子问题 2}

\subsubsection{问题分析}

现在两个城市之间可以存在中间节点,并且所有中间节点传输容量可以任意分配,由传输链路存在中间传输节点的特点可知,增加一条链路的容量至少需要降低额外的两条通道的容量,因此可以先分析各相关城市的通道间容量的增减的关系对于网络价值的影响,来确定是否需要增加中间节点,如果需要增加中间节点,可以再从资源、地域,包括人口的角度来分析容量进行如何分配。

\subsection*{5.2.2 模型建立与求解}

\begin{figure}[h]
    \centering
    \includegraphics[width=0.8\textwidth]{image.png} % 替换为实际图片路径
    \caption{两点间有一个中间节点}
    \label{fig:5.14}
\end{figure}

如图 \ref{fig:5.14},假设有三个不同的城市,以其中的人口数(百万)来代表它们,三个城市分别表示为 $A$, $B$, $C$,连接三个城市传输通道的容量分别表示为 $R_{a}$, $R_{b}$, $R_{c}$,三个城市之间可能相互连接,也可能存在有两个城市没有相连的情况。现假设 $A$, $B$ 城市可能需要通过 $C$ 为中间节点来增加信息传输(如图 \ref{fig:5.14} 左)或者是通过 $C$ 为中间节点来进行通信(如图 \ref{fig:5.14} 右),且增加的信息传输容量为 $x$,以三个城市相互连接的情况为例进行建模。为了便于分析,设每条直接连接两个城市/区域的链路的权重为 1,根据网络价值的定义式可以得到三者直接连的总价值为:

\begin{equation}
V_{ABC} = \sqrt{AB}R_{c} + \sqrt{AC}R_{b} + \sqrt{BC}R_{a}
\tag{5-2}
\end{equation}

当选取 $C$ 为中间节点时,且增加的传输容量为 $x$ 时,如图 \ref{fig:5.14} 虚线所示,可以得到增加中间节点后的总价值为:

\begin{equation}
V_{ABC}^{*} = \sqrt{AB}(R_{c} + x) + \sqrt{AC}(R_{b} - x) + \sqrt{BC}(R_{a} - x)
\tag{5-3}
\end{equation}

由公式 5-2 与公式 5-3 可以得到增加节点之间的关系式与直接相连的关系式为:

\begin{equation}
V_{ABC}^{*} = V_{ABC} + (\sqrt{AB} - \sqrt{AC} - \sqrt{BC})x
\tag{5-4}
\end{equation}

此时问题就变成了一次函数的问题,已知 $x$ 肯定是大于 0 的,所以如果增加节点以后网络的总价值是降低还是增加了完全由三个城市的人口有关,当 $\sqrt{AB} - \sqrt{AC} - \sqrt{BC} > 0$ 时,增加中间节点 $C$ 可以增加网络总价值,反之会降低网络总价值。下面对可能存在两个中间节点的情况进行建模。

\begin{figure}[h]
    \centering
    \includegraphics[width=0.8\textwidth]{image2.png} % 替换为实际图片路径
    \caption{两点间有两个中间节点}
    \label{fig:5.15}
\end{figure}

如图 5.15 所示,现有四座城市,仍然用其人口数(百万)A,B,C,D 表示它们。相互之间的传输链路的容量分别表示为 $R_{\mathrm{a}}$,$R_{\mathrm{b}}$,$R_{\mathrm{c}}$,$R_{\mathrm{d}}$,由网络总价值定义式可得四个点相互直连时的总价值为:
\begin{equation}
V_{ABCD} = \sqrt{AB}R_{\mathrm{a}} + \sqrt{BC}R_{\mathrm{b}} + \sqrt{CD}R_{\mathrm{c}} + \sqrt{AD}R_{\mathrm{d}}
\tag{5-5}
\end{equation}

现以 B,C 作为 A,D 的中间节点,AD 间增加的传输容量为 $x$ 时,网络的总价值为:
\begin{equation}
V_{ABCD}^{*} = \sqrt{AB}(R_{\mathrm{a}} - x) + \sqrt{BC}(R_{\mathrm{b}} - x) + \sqrt{CD}(R_{\mathrm{c}} - x) + \sqrt{AD}(R_{\mathrm{d}} + x)
\tag{5-6}
\end{equation}

由公式 5-5 与公式 5-6 可以得到增加节点前后的网络价值的关系式为:
\begin{equation}
V_{ABCD}^{*} = V_{ABCD} + \left( \sqrt{AD} - \sqrt{AB} - \sqrt{BC} - \sqrt{CD} \right)x
\tag{5-7}
\end{equation}

由公式 5-7 可以得到,$x$ 为一个大于 0 的数,当:
\begin{equation}
\sqrt{AD} - \sqrt{AB} - \sqrt{BC} - \sqrt{CD} > 0
\tag{5-8}
\end{equation}

公式 5-8 满足时,网络总价值会增加,而当此条件满足时,$\sqrt{AD}$ 减去 $\sqrt{AB}$,$\sqrt{BC}$,$\sqrt{CD}$ 中任意两个均大于 0,且有两个节点时 $\sqrt{AD}$ 有 3 个减数,而有一个节点时 $\sqrt{AD}$ 只有 2 个减数,因此两个节点的增长量不如一个节点的增长量多。

为了在增加中间节点的情况下使网络的总价值最大化增加,可以将两个节点的问题转变为一个节点的问题,以此类推,即将多边问题转变成三边的求解问题。

不考虑实际直线距离能否直接相连的情况下,在题中给出的 12 个区域点中任意取出 3 个点均可以组成一个三角形,共 $C_{12}^{3} = 220$ 种取法,再考虑实际调制模式可以传输的最远距离,且满足其中类如 $\sqrt{AB} - \sqrt{AC} - \sqrt{BC} > 0$ 的条件,即增加中间节点可以增加网络总价值的情况的条件限制,经由编程求解得 220 种情况中共有 101 种情况符合上述条件,其中的三角形均是以拉萨(54 种),乌鲁木齐(41 种),昆明(6 种)三个城市为中间连接点(对应于图 5.12 中的 C 点),详情见表 5.1——表 5.3。

\begin{table}[h]
\centering
\caption{拉萨为中间节点的连接方式}
\begin{tabular}{|l|l|l|}
\hline
拉萨>>昆明 广\&深 & 拉萨>>西安 重庆 & 拉萨>>上海 成都 \\
拉萨>>重庆 昆明 & 拉萨>>哈尔滨 广\&深 & 拉萨>>郑州 广\&深 \\
拉萨>>重庆 武汉 & 拉萨>>哈尔滨 昆明 & 拉萨>>郑州 昆明 \\
拉萨>>成都 广\&深 & 拉萨>>哈尔滨 武汉 & 拉萨>>郑州 武汉 \\
拉萨>>成都 昆明 & 拉萨>>哈尔滨 重庆 & 拉萨>>西安 郑州 \\
拉萨>>武汉 广\&深 & 拉萨>>哈尔滨 成都 & 拉萨>>京\&津 广\&深 \\
拉萨>>武汉 昆明 & 拉萨>>哈尔滨 上海 & 拉萨>>京\&津 昆明 \\
拉萨>>重庆 广\&深 & 拉萨>>哈尔滨 郑州 & 拉萨>>京\&津 武汉 \\
拉萨>>成都 武汉 & 拉萨>>哈尔滨 西安 & 拉萨>>京\&津 重庆 \\
拉萨>>成都 重庆 & 拉萨>>哈尔滨 京\&津 & 拉萨>>京\&津 成都 \\
\hline
\end{tabular}
\end{table}

\begin{table}
\centering
\begin{tabular}{|p{0.3\textwidth}|p{0.3\textwidth}|p{0.3\textwidth}|}
\hline
拉萨>>上海 广&深 & 拉萨>>乌鲁木齐 广&深 & 拉萨>>京&津 上海 \\
拉萨>>上海 昆明 & 拉萨>>乌鲁木齐 武汉 & 拉萨>>京&津 郑州 \\
拉萨>>上海 武汉 & 拉萨>>乌鲁木齐 重庆 & 拉萨>>京&津 西安 \\
拉萨>>郑州 重庆 & 拉萨>>乌鲁木齐 成都 & 拉萨>>西安 上海 \\
拉萨>>郑州 成都 & 拉萨>>乌鲁木齐 上海 & 拉萨>>乌鲁木齐 哈尔滨 \\
拉萨>>郑州 上海 & 拉萨>>乌鲁木齐 郑州 & \\
拉萨>>西安 广&深 & 拉萨>>乌鲁木齐 西安 & 拉萨>>西安 成都 \\
拉萨>>西安 昆明 & 拉萨>>乌鲁木齐 京&津 & \\
拉萨>>西安 武汉 & 拉萨>>上海 重庆 & \\
\hline
\end{tabular}
\caption{以昆明为中间节点的连接方式}
\end{table}

\begin{table}
\centering
\begin{tabular}{|p{0.3\textwidth}|p{0.3\textwidth}|p{0.3\textwidth}|}
\hline
昆明>>重庆 广&深 & 昆明>>上海 重庆 & 昆明>>京&津 重庆 \\
昆明>>上海 广&深 & 昆明>>京&津 广&深 & 昆明>>京&津 上海 \\
\hline
\end{tabular}
\end{table}

\begin{table}
\centering
\begin{tabular}{|p{0.3\textwidth}|p{0.3\textwidth}|p{0.3\textwidth}|}
\hline
乌鲁木齐>>昆明 广&深 & 乌鲁木齐>>西安 & 乌鲁木齐>>哈尔滨 郑州 \\
乌鲁木齐>>武汉 广&深 & 乌鲁木齐>>西安 重庆 & 乌鲁木齐>>哈尔滨 西安 \\
乌鲁木齐>>重庆 广&深 & 乌鲁木齐>>西安 成都 & 乌鲁木齐>>京&津 重庆 \\
乌鲁木齐>>重庆 昆明 & 乌鲁木齐>>西安 上海 & 乌鲁木齐>>京&津 成都 \\
乌鲁木齐>>重庆 武汉 & 乌鲁木齐>>西安 郑州 & 乌鲁木齐>>京&津 上海 \\
乌鲁木齐>>成都 广&深 & 乌鲁木齐>>京&津 昆明 & 乌鲁木齐>>京&津 郑州 \\
乌鲁木齐>>成都 昆明 & 乌鲁木齐>>京&津 武汉 & 乌鲁木齐>>京&津 西安 \\
乌鲁木齐>>成都 武汉 & 乌鲁木齐>>郑州 武汉 & 乌鲁木齐>>哈尔滨 广&深 \\
乌鲁木齐>>上海 广&深 & 乌鲁木齐>>郑州 重庆 & 乌鲁木齐>>哈尔滨 京&津 \\
乌鲁木齐>>上海 昆明 & 乌鲁木齐>>郑州 成都 & 乌鲁木齐>>郑州 广&深 \\
乌鲁木齐>>上海 武汉 & 乌鲁木齐>>郑州 上海 & 乌鲁木齐>>京&津 广&深 \\
乌鲁木齐>>上海 重庆 & 乌鲁木齐>>哈尔滨 武汉 & \\
乌鲁木齐>>上海 成都 & 乌鲁木齐>>哈尔滨 重庆 & \\
乌鲁木齐>>西安 广&深 & 乌鲁木齐>>哈尔滨 成都 & \\
 & 乌鲁木齐>>哈尔滨 上海 & \\
\hline
\end{tabular}
\caption{以乌鲁木齐为中间节点的连接方式}
\end{table}

通过上面的建模结果与分析,结合 5.1 节中图 5.7 可知,在已有的 16 条线连接的情况下拉萨,昆明,乌鲁木齐三个城市均是三角形中形如图 5.14 中 A 点或者 B 点的位置,所有无需对连接策略进行调整,此时网络的最大价值仍为 5433。

结合 5.1 节中图 5.8 可得,在已有的 33 条连接线的情况下,拉萨和乌鲁木齐也均是三角形中形如图 5.14 中 A 点或者 B 点的位置,所有无需对连接策略进行调整,只有以昆明作为中心节点的连接需要进行容量重新分配,需要分配的连接线路为:

\begin{table}
\centering
\begin{tabular}{|c|}
\hline
昆明>>重庆 广&深 \\
\hline
\end{tabular}
\end{table}

\begin{table}
\centering
\begin{tabular}{|c|c|c|c|c|c|c|c|c|c|c|c|c|}
\hline
人口 & 新疆 & 黑龙江 & 北京&天津 & 陕西 & 河南 & 上海 & 西藏 & 四川 & 重庆 & 湖北 & 云南 & 广东 \\
\hline
 & 22.98 & 37.89 & 37.26 & 38.35 & 9.57 & 24.18 & 3.3 & 83.02 & 30.75 & 59.02 & 47.71 & 111.69 \\
\hline
\end{tabular}
\end{table}

图5.16 省区人口(百万)

得到新的人口分布以后,对网络进行了重新价值分析,并分别进行16条及33条连接线路不包含中间节点的连接,结果分别如图5.17及图5.18所示,此时16条总价值为:18654.95,而33条连接的总价值为:27491.58。

\begin{table}
\centering
\begin{tabular}{|c|c|c|c|c|c|c|c|c|c|c|c|c|}
\hline
价值 & 乌鲁木齐 & 哈尔滨 & 北京&天津 & 西安 & 郑州 & 上海 & 拉萨 & 成都 & 重庆 & 武汉 & 昆明 & 广州&深圳 \\
\hline
乌鲁木齐 & \ & \ & \ & \ & 374.94 & \ & \ & \ & \ & \ & \ & \ & \ \\
\hline
哈尔滨 & \ & \ & 601.17 & \ & \ & \ & \ & \ & \ & \ & \ & \ & \ \\
\hline
北京&天津 & \ & \ & \ & \ & 1909.75 & \ & \ & \ & \ & \ & \ & \ \\
\hline
西安 & \ & \ & \ & \ & 1937.48 & \ & \ & \ & 902.8 & 1098.89 & \ & \ & \ \\
\hline
郑州 & \ & \ & \ & \ & \ & 769.22 & \ & \ & 1425.33 & 867.45 & 2403.56 & \ & \ \\
\hline
上海 & \ & \ & \ & \ & \ & \ & \ & \ & \ & \ & \ & \ & \ \\
\hline
拉萨 & \ & \ & \ & \ & \ & \ & \ & \ & \ & \ & \ & \ & 153.58 \\
\hline
成都 & \ & \ & \ & \ & \ & \ & \ & \ & \ & 1616.82 & 1119.98 & 1006.96 & \ \\
\hline
重庆 & \ & \ & \ & \ & \ & \ & \ & \ & \ & \ & \ & \ & \ \\
\hline
武汉 & \ & \ & \ & \ & \ & \ & \ & \ & \ & \ & \ & \ & 1299.05 \\
\hline
昆明 & \ & \ & \ & \ & \ & \ & \ & \ & \ & \ & \ & \ & 1167.97 \\
\hline
广州&深圳 & \ & \ & \ & \ & \ & \ & \ & \ & \ & \ & \ & \ \\
\hline
\end{tabular}
\end{table}

图5.17 16条连接线的价值邻接表

\begin{table}
\centering
\begin{tabular}{|c|c|c|c|c|c|c|c|c|c|c|c|c|}
\hline
价值 & 乌鲁木齐 & 哈尔滨 & 北京&天津 & 西安 & 郑州 & 上海 & 拉萨 & 成都 & 重庆 & 武汉 & 昆明 & 广州&深圳 \\
\hline
乌鲁木齐 & \ & \ & \ & \ & 374.94 & \ & \ & \ & \ & \ & \ & \ & \ \\
\hline
哈尔滨 & \ & \ & 601.17 & \ & 481.45 & \ & \ & \ & 448.68 & \ & \ & \ & 520.42 \\
\hline
北京&天津 & \ & \ & \ & 604.81 & 1909.75 & 480.25 & \ & \ & \ & 750.31 & \ & 516.08 \\
\hline
西安 & \ & \ & \ & \ & 1937.48 & \ & \ & \ & 902.8 & 1098.89 & 761.2 & 684.39 & 523.57 \\
\hline
郑州 & \ & \ & \ & \ & \ & 769.22 & \ & \ & 1425.33 & 867.45 & 2403.56 & 540.25 & \ \\
\hline
上海 & \ & \ & \ & \ & \ & \ & \ & \ & \ & \ & 604.43 & \ & \ \\
\hline
拉萨 & \ & \ & \ & \ & \ & \ & \ & \ & \ & \ & \ & \ & 153.58 \\
\hline
成都 & \ & \ & \ & \ & \ & \ & \ & \ & \ & 1616.82 & 1119.98 & 1006.96 & 770.35 \\
\hline
重庆 & \ & \ & \ & \ & \ & \ & \ & \ & \ & \ & 681.61 & \ & 468.83 \\
\hline
武汉 & \ & \ & \ & \ & \ & \ & \ & \ & \ & \ & \ & \ & 1299.05 \\
\hline
昆明 & \ & \ & \ & \ & \ & \ & \ & \ & \ & \ & \ & \ & 1167.97 \\
\hline
广州&深圳 & \ & \ & \ & \ & \ & \ & \ & \ & \ & \ & \ & \ \\
\hline
\end{tabular}
\end{table}

图5.18 33条连接线的价值邻接表

随之由前面的分析得到带限制条件模型中,筛选出16条及33条包含中间节点且可以改动的情况,最终符合条件共58种情况,可以修改的区域中,均是以西藏(55种),新疆(2种),上海(1种)作为中间节点的分配方式,详见表5.4——表5.6。

表5.4 以上海为中间节点的连接方式

上海>>河南 广东

表5.5 以新疆为中间节点的连接方式

新疆>>四川 广东 新疆>>河南 广东

表5.6 以西藏为中间节点的连接方式

\begin{tabular}{|c|c|c|}
\hline
西藏>>云南 广东 & 西藏>>河南 四川 & 西藏>>黑龙江 湖北 \\
\hline
西藏>>湖北 广东 & 西藏>>河南 上海 & 西藏>>黑龙江 重庆 \\
\hline
西藏>>湖北 云南 & 西藏>>陕西 广东 & 西藏>>黑龙江 四川 \\
\hline
西藏>>重庆 广东 & 西藏>>陕西 云南 & 西藏>>黑龙江 上海 \\
\hline
西藏>>重庆 云南 & 西藏>>陕西 湖北 & 西藏>>黑龙江 河南 \\
\hline
西藏>>重庆 湖北 & 西藏>>陕西 重庆 & 西藏>>黑龙江 陕西 \\
\hline
\end{tabular}

\begin{table}
\centering
\begin{tabular}{|p{0.3\textwidth}|p{0.3\textwidth}|p{0.3\textwidth}|}
\hline
西藏>>四川 广东 & 西藏>>陕西 四川 & 西藏>>黑龙江 京&津 \\
西藏>>四川 云南 & 西藏>>陕西 上海 & 西藏>>新疆 广东 \\
西藏>>四川 湖北 & 西藏>>陕西 河南 & 西藏>>新疆 云南 \\
西藏>>四川 重庆 & 西藏>>京&津 广东 & 西藏>>新疆 湖北 \\
西藏>>上海 广东 & 西藏>>京&津 云南 & 西藏>>新疆 重庆 \\
西藏>>上海 云南 & 西藏>>京&津 湖北 & 西藏>>新疆 四川 \\
西藏>>上海 湖北 & 西藏>>京&津 重庆 & 西藏>>新疆 上海 \\
西藏>>上海 重庆 & 西藏>>京&津 四川 & 西藏>>新疆 河南 \\
西藏>>上海 四川 & 西藏>>京&津 上海 & 西藏>>新疆 陕西 \\
西藏>>河南 广东 & 西藏>>京&津 河南 & 西藏>>新疆 京&津 \\
西藏>>河南 云南 & 西藏>>京&津 陕西 & 西藏>>新疆 黑龙江 \\
西藏>>河南 湖北 & 西藏>>黑龙江 广东 &  \\
西藏>>河南 重庆 & 西藏>>黑龙江 云南 &  \\
\hline
\end{tabular}
\end{table}

\subsection{5.2.3 实验结果分析}

本节对增加中间节点的过程进行了建模,并归纳出增加中间节点可以增加网络总价值的条件,位于传输端点的两个城市人口计算表达式要大于它们与中间节点人口计算表达式的和,与 5.1 节中的结果进行对比发现,除了偏远地区需要中间节点调节外,基本上不同传输方式的传输距离可以满足实际需求的情况下,直连是网络价值最高的连接方式。当人口发生变化时,即人口由市变成省(区)时,题目中的网络价值邻接表发生了很大的变化,直接对最后的连接结果以及网络总价值的产生了很大的影响,也从侧面反映出当前目标函数的鲁棒性不足,并且极易受人口变化的影响。

\section{5.3 子问题 3}

\subsection{5.3.1 问题分析}

光传送网深远的影响着人们的生活,从不同的角度考虑,对于光传送网的认知不同,对光传送网的价值理解就不同。前面 5.1 节以及 5.2 节我们默认各条连接的权重为 1,在此题环境下就显得不再合适,所以可以尝试从企业发展的利益考虑充分利用经济发达地区的特点,以及政府角度想要保障发展的角度考虑,结合可以从网上查找到的统计数据,例如加入不同地区的人口结构,产业结构或者区域总 GDP 的因素,对网络规划的目标函数进行重新调整。

\subsection{5.3.2 模型建立与求解}

本文查阅了大量的网络资料首先考虑人均 GDP,虽然如图 5.19 所示,各区域的总体 GDP(亿元)差距较大,较能体现该区域经济水平,但是在查阅资料后发现例如新疆,西藏等地区因为所含人口较少,所以人均 GDP 依然很高,这种情况使我们仅从地区总 GDP 来考虑目标函数。为了更加全面合理的分配资源,本文还从人口结构和产业结构入手,查阅资料得到不同城市的人口结构,产业结构统计数据分别如图 5.20 和图 5.21 所示:

\begin{table}
\centering
\begin{tabular}{|c|c|c|c|c|c|c|c|c|c|c|c|c|}
\hline
经济水平 & 乌鲁木齐 & 哈尔滨 & 京&津 & 西安 & 郑州 & 上海 & 拉萨 & 成都 & 重庆 & 武汉 & 昆明 & 广&深 \\
\hline
区域总GDP & 2799 & 6609 & 23298 & 7206 & 9003 & 30133 & 422 & 13890 & 19530 & 13400 & 4856 & 21893 \\
\hline
\end{tabular}
\caption{不同地区经济水平(亿元)}
\end{table}

\begin{table}
\centering
\begin{tabular}{|c|c|c|c|c|c|c|c|c|c|c|c|c|}
\hline
产业结构 & 乌鲁木齐 & 哈尔滨 & 京&津 & 西安 & 郑州 & 上海 & 拉萨 & 成都 & 重庆 & 武汉 & 昆明 & 广&深 \\
\hline
第一产业 & 0.17 & 0.108 & 0.01 & 0.037 & 0.0206 & 0.002 & 0.037 & 0.039 & 0.049 & 0.03 & 0.046 & 0.0066 \\
\hline
第二产业 & 0.5 & 0.287 & 0.341 & 0.351 & 0.4956 & 0.2907 & 0.374 & 0.43 & 0.441 & 0.437 & 0.387 & 0.3497 \\
\hline
第三产业 & 0.33 & 0.605 & 0.649 & 0.612 & 0.4838 & 0.7073 & 0.589 & 0.531 & 0.49 & 0.533 & 0.567 & 0.6437 \\
\hline
\end{tabular}
\caption{不同地区产业结构}
\end{table}

\begin{table}
\centering
\begin{tabular}{|c|c|c|c|c|c|c|c|c|c|c|c|c|}
\hline
人口结构 & 乌鲁木齐 & 哈尔滨 & 京&津 & 西安 & 郑州 & 上海 & 拉萨 & 成都 & 重庆 & 武汉 & 昆明 & 广&深 \\
\hline
0-14 & 0.1375 & 0.11 & 0.101 & 0.1276 & 0.2141 & 0.086 & 0.2437 & 0.1094 & 0.2399 & 0.0998 & 0.1782 & 0.134 \\
\hline
15-64 & 0.7825 & 0.801 & 0.7739 & 0.7635 & 0.684 & 0.813 & 0.7054 & 0.7935 & 0.6663 & 0.8189 & 0.7362 & 0.8323 \\
\hline
65以上 & 0.08 & 0.089 & 0.1251 & 0.1089 & 0.1019 & 0.101 & 0.0509 & 0.0971 & 0.0938 & 0.0813 & 0.0856 & 0.0337 \\
\hline
\end{tabular}
\caption{不同地区人口结构}
\end{table}

考虑到年龄在15-64岁阶段的用户是对通信网需求最强的,所以在计算人口时,可以主要考虑此年龄段的用户。我国的产业可以划分为三类,第一产业主要包括种植业、林业、畜牧业以及水产养殖业。第二产业主要指加工制造业,第三产业包括交通运输业、通讯产业、金融业等。通信网效益与电信技术以及市场密切相关,因此区域的产业结构也影响着资源分配,信息行业即第三产业较为发达的城市应该被分配较大的权重。

设 $M_c$ 表示 $C$ 市的总人口数,$\omega_{1c}$ 表示 $C$ 市中15-64岁年龄段所占的人口比例,$\omega_{2c}$ 表示 $C$ 市产业结构中第三产业在所有区域中所占的比例,$\omega_{3c}$ 表示 $C$ 市的经济水平在所有区域中所占的比例:

\begin{equation}
\omega_{2c} = \frac{\alpha_{ic}}{\sum \alpha_i} \tag{5-9}
\end{equation}

\begin{equation}
\omega_{3c} = \frac{\beta_{ic}}{\sum \beta_i} \tag{5-10}
\end{equation}

其中 $\alpha_{ic}$ 表示 $C$ 市的第三产业比,$\sum \alpha_i$ 表示所有区域第三产业比之和,$\beta_{ic}$ 表示 $C$ 市的区域总GDP,$\sum \beta_i$ 表示所有区域的GDP总和。为了平衡两个不同区域的第三产业比重及区域总GDP的问题,直接将两个市的对应数值相加取平均,最终本文提出两种不同的价值目标函数:

\begin{equation}
\text{价值1} = \sum_{i,j} \frac{1}{2} \left( \omega_{2i} + \omega_{2j} \right) \sqrt{M_i \omega_{1i} M_j \omega_{2i}} R \tag{5-11}
\end{equation}

\begin{equation}
\text{价值2} = \sum_{i,j} \frac{1}{2} \left( \omega_{3i} + \omega_{3j} \right) \sqrt{M_i \omega_{1i} M_j \omega_{2i}} R \tag{5-12}
\end{equation}

其中 $R$ 表示城市间传输容量。模型建立过程同5.1节,只是通过新的目标函数计算网络之间的权值,已知不同地区人口数量,将价值1以及价值2的表达式代入模型进行重新建模求解,为了简化计算过程,本文仅对区域间不存在中间节点及只有16条连接线的情况进行实验和比较。最终得到新的网络价值邻接表。

\begin{table}
\centering
\begin{tabular}{|c|c|c|c|c|c|c|c|c|c|c|c|c|}
\hline
价值1 & 乌鲁木齐 & 哈尔滨 & 北京&天津 & 西安 & 郑州 & 上海 & 拉萨 & 成都 & 重庆 & 武汉 & 昆明 & 广州&深圳 \\
\hline
乌鲁木齐 & 0 & \ & 4.23 & 1.99 & 1.61 & \ & 0.59 & 2.45 & 2.87 & 2.05 & 1.53 & \ & \ \\
\hline
哈尔滨 & \ & 0 & 24.06 & 5.75 & 4.82 & 9.94 & \ & 7.15 & 8.58 & 6.01 & \ & 11.69 & \ \\
\hline
北京&天津 & \ & \ & 0 & 22.01 & 37.16 & 37.92 & 3.25 & 13.72 & 16.53 & 23.07 & 8.6 & 22.31 \\
\hline
西安 & \ & \ & \ & 0 & 17.65 & 9.09 & 1.55 & 13.09 & 31.4 & 10.99 & 8.2 & 10.7 & \ \\
\hline
郑州 & \ & \ & \ & \ & 0 & 15.35 & 1.3 & 10.89 & 12.96 & 18.31 & 3.41 & 9.03 & \ \\
\hline
上海 & \ & \ & \ & \ & \ & 0 & 2.68 & 11.34 & 13.65 & 19.06 & 7.11 & 18.4 & \ \\
\hline
拉萨 & \ & \ & \ & \ & \ & \ & 0 & 1.934 & 2.32 & 1.62 & 1.21 & 3.16 & \ \\
\hline
成都 & \ & \ & \ & \ & \ & \ & \ & 0 & 38.7 & 13.6 & 10.16 & 13.34 & \ \\
\hline
重庆 & \ & \ & \ & \ & \ & \ & \ & \ & 0 & 16.29 & 12.15 & 16.07 & \ \\
\hline
武汉 & \ & \ & \ & \ & \ & \ & \ & \ & \ & 0 & 4.27 & 22.43 & \ \\
\hline
昆明 & \ & \ & \ & \ & \ & \ & \ & \ & \ & \ & 0 & 16.7 & \ \\
\hline
广州&深圳 & \ & \ & \ & \ & \ & \ & \ & \ & \ & \ & \ & 0 \\
\hline
\end{tabular}
\caption{图5.22 价值1网络价值及连接情况}
\end{table}

\begin{table}
\centering
\begin{tabular}{|c|c|c|c|c|c|c|c|c|c|c|c|c|}
\hline
价值2 & 乌鲁木齐 & 哈尔滨 & 北京&天津 & 西安 & 郑州 & 上海 & 拉萨 & 成都 & 重庆 & 武汉 & 昆明 & 广州&深圳 \\
\hline
乌鲁木齐 & 0 & \ & 4.79 & 0.92 & 1.03 & \ & 0.08 & 2.04 & 3.48 & 1.66 & 0.58 & \ & \ \\
\hline
哈尔滨 & \ & 0 & 24.69 & 2.88 & 3.07 & 12.55 & \ & 5.63 & 9.17 & 4.63 & \ & 11.45 & \ \\
\hline
北京&天津 & \ & \ & 0 & 23.06 & 46.12 & 66.18 & 2.65 & 18.53 & 27.22 & 30.76 & 8.79 & 32.91 \\
\hline
西安 & \ & \ & \ & 0 & 11.7 & 11.68 & 0.43 & 10.63 & 34.38 & 8.74 & 3.81 & 10.7 & \ \\
\hline
郑州 & \ & \ & \ & \ & 0 & 23.12 & 0.5 & 10.89 & 117.32 & 17.94 & 2.07 & 10.73 & \ \\
\hline
上海 & \ & \ & \ & \ & \ & 0 & 2.83 & 18.14 & 26.11 & 30.18 & 9.04 & 31.34 & \ \\
\hline
拉萨 & \ & \ & \ & \ & \ & \ & 0 & 1.06 & 1.9 & 0.87 & 0.24 & 2.43 & \ \\
\hline
成都 & \ & \ & \ & \ & \ & \ & \ & 0 & 56.61 & 15.24 & 7.81 & 17.35 & \ \\
\hline
重庆 & \ & \ & \ & \ & \ & \ & \ & \ & 0 & 23.45 & 12.96 & 25.62 & \ \\
\hline
武汉 & \ & \ & \ & \ & \ & \ & \ & \ & \ & 0 & 3.2 & 28.78 & \ \\
\hline
昆明 & \ & \ & \ & \ & \ & \ & \ & \ & \ & \ & 0 & 16.27 & \ \\
\hline
广州&深圳 & \ & \ & \ & \ & \ & \ & \ & \ & \ & \ & \ & 0 \\
\hline
\end{tabular}
\caption{图5.23 价值2网络价值及连接情况}
\end{table}

\subsection*{5.3.3 实验结果分析}

基于5.1节与5.2节中的建模与分析,可知题目已给的目标函数存在可以改进的地方。本节通过查阅大量资料,提出对目标函数进行改进的看法,设计了新的目标函数,在有16条连接数且不包含中间节点连接的情况下,代入模型中最终求解出新的规划方案,与图5.7进行对比可以发现,在价值1表达式下的光传送网与原目标函数相似度较高,但是连接情况更加地均匀,比较符合通信的均衡发展,而价值2表达式下的连接方式更偏向于例如北京&天津,广州&深圳经济发达的地区。因此若为了通信均衡发展,可以更多将人口结构以及产业分布考虑在内,而想要获得更多的收入,除了人口和人口结构,需要更多地将区域GDP考虑在内。

\section*{六、问题三:调制格式改进}

\subsection*{6.1 问题分析}

已知某一概率系统的信息熵定义为:
\begin{equation}
\Omega = -\sum_{1}^{N} p_k \log_2 \left( p_k \right) \tag{6-1}
\end{equation}
其中$p_k$为每个符号状态出现的概率,$N$称为状态数。根据信息熵的基本性质,如果某一系统的概率分布是均匀的,即系统中每一状态产生的概率相等时,该系统的不确定性最大,或称信息熵$\Omega$最大[8]。在等概率情况下,8QAM的信息熵最大等于3bit,所以为了满足调制格式的信息熵保持为3bit,对16QAM星座图改善时,状态数(星座图点数量)$N$需要满足大于等于8。

为了便于实现,大多数通信系统均只采用 $N$ 为偶数的 $2^N$ 阶 QAM 作为数字调制方式 [9]。因此,只考虑实际应用时,对 16QAM 星座图进行改进有两种情况:第一种情况是,保持星座点的数量为 16,改变星座点的位置和每个点的概率;第二种情况是,将星座点的数量改为 8 个,则每个点的概率需保持相等,只能改变星座点的位置。

\subsection{6.2 模型优化}

为了保持调制格式的信息熵为 3bit,在对 16QAM 进行优化时需要选择星座点的数量和每个点的概率满足下面的公式:

\begin{equation}
-\sum_{1}^{N} p_k \log \left(p_k\right) = 3 \text{bit}
\tag{6-2}
\end{equation}

其中 $p_k$ 为每个符号状态出现的概率,需要满足下面的条件:

\begin{equation}
0 \leq p_k \leq 1, k = 1, 2, \ldots, N
\tag{6-3}
\end{equation}

\begin{equation}
p_1 + p_2 + \cdots + p_{N-1} + p_N = 1
\tag{6-4}
\end{equation}

根据公式 6-2、6-3 和 6-4 三个基本条件,通过 LINGO 求解相应的非线性优化问题,已知在等概率情况下,星座点的数量越多信息熵越大,而信息熵越大系统的不确定性越大,相应的误码率增高,抗干扰性能下降。所以本文只考虑星座点数量小于等于 16 的情况。当系统的星座点的数量从 8 到 16 时,通过计算可得相应的每个点的概率,如表 6.1 所示,表中出现的同概率的多个点均已加粗,并在后面乘上其个数。

\begin{table}[h]
\centering
\caption{星座点数量和星座点概率组合的对应关系}
\begin{tabular}{|c|c|}
\hline
星座点数量 & 星座点概率 $p_k$ 组合 \\
\hline
8 & $0.125 \times 8$ \\
\hline
9 & $0.7838 \mathrm{E}-01 \times 6, 0.1067, 0.1576, 0.2654$ \\
\hline
10 & $0, 0.7358 \mathrm{E}-01 \times 6, 0.1370, 0.1949, 0.2266$ \\
\hline
11 & $0, 0.5074 \mathrm{E}-01, 0.7507 \mathrm{E}-01, 0.7561 \mathrm{E}-01 \times 7, 0.3449$ \\
\hline
12 & $0.4269 \mathrm{E}-01 \times 9, 0.6197 \mathrm{E}-01, 0.2113, 0.3425$ \\
\hline
13 & $0.3749 \mathrm{E}-01, 0.3803 \mathrm{E}-01 \times 9, 0.7964 \mathrm{E}-01, 0.1345, 0.4061$ \\
\hline
14 & $0.3164 \mathrm{E}-01, 0.3319 \mathrm{E}-01 \times 10, 0.8196 \mathrm{E}-01, 0.1352, 0.4193$ \\
\hline
15 & $0.2828 \mathrm{E}-01, 0.3119 \mathrm{E}-01 \times 11, 0.5585 \mathrm{E}-01, 0.1315, 0.4413$ \\
\hline
16 & $0.2974 \mathrm{E}-01, 0.3034 \mathrm{E}-01 \times 12, 0.4313 \mathrm{E}-02, 0.1874, 0.4145$ \\
\hline
\end{tabular}
\end{table}

\subsubsection{6.2.1 8 个点的星座图优化}

若 16QAM 的星座点的数量改为 8 个,则每个点的概率需保持相等,那么只能改变星座点的位置。常见的 QAM 星座图的形式主要有圆形、三角形和矩形等三种,最基本的 8QAM 星座图如图 6.1(a)。

三角形星座图中相邻最近 3 点的连线构成一个正三角形,且一般不在原点处安排星座点,这样的设计使得相邻的星座点之间距离相等,从而可以提高系统性能。为了更好地在实际应用中利用发信功放的输出功率,应尽可能将各星座点按三角形排列原则布置在一个圆环内,因此本文将 8QAM 星座图旋转 $45^\circ$ 设计了

如图 6.1(d)的三角形星座图。

圆形星座图的基本特征是所有星座点都处在以原点为圆心的一个或多个圆周上。为了提高实际应用的系统性能,各圆周上的星座点数应与该圆半径成正比,且同一圆周上各星座点应保持均匀分布。本文分别将全部八个点放在圆上,将 7 个点放在圆上并且 1 个点放在圆心设计了图 6.1(b)和图 6.1(e)的两种圆形星座图。

矩形星座图由于易于实现、误码率低应用最广。它以正方形作为星座图的基本形状,本文考虑方向因素设计了如图 6.1(c)和图 6.1(f)的两种矩形星座图。

\begin{figure}[h]
    \centering
    \includegraphics[width=\textwidth]{image1.png}
    \caption{8QAM 时的发送星座图}
    \label{fig:6.1}
\end{figure}

将最基本 8QAM 星座图和图 6.1(b)—(f)中设计的 5 种星座图分别进行调制和解调的仿真,并绘出各自的 BER 和 SNR 的关系曲线,可以得到如图 6.2 和图 6.3 的实验结果。

\begin{figure}[h]
    \centering
    \includegraphics[width=\textwidth]{image2.png}
    \caption{8QAM 信号的 BER-SNR 曲线}
    \label{fig:6.2}
\end{figure}

\begin{figure}[h]
    \centering
    \includegraphics[width=\textwidth]{image1.png}
    \caption{8QAM 时的接收星座图}
    \label{fig:8QAM_constellation}
\end{figure}

从图 6.2 可以得出相关的 BER-SNR 关系曲线,可以看到虽然改变了位置的排列但是整体的曲线变化趋势仍然保持相同。图 6.1(e) 的星座图排列方式会在 BER 较小时明显优于基本的 8QAM。图 6.1(f) 的星座图排列方式会在 BER 较大时稍微优于基本的 8QAM。

\subsection{16 个点的星座图优化}

参考文献[5][6]得若 16QAM 星座点的数量保持不变,则可以改变每个点的概率和星座点的位置。由 6.2.1 部分对星座图概率分配的分析,对于 16QAM 星座图,由表 6.1 我们可以得到固定的 16 个概率,由于概率有大有小,本文考虑分配给 16 个星座点不同的概率组合。同时,基本的 16QAM 星座图如图 6.4(a),是一种矩形的星座图。考虑到不同形状的星座图对信号的传输可能有不同的影响,本文设计了如图 6.4(b) 所示的三角形的星座图,以及如图 6.4(c) 所示的圆形的星座图。三种不同形状的星座图又分为星座点等概率出现和不等概出现两种情况,不等概出现的情况分别为图 6.4(d)、图 6.4(e) 和图 6.4(f),后面以图中颜色的轻重来分辨两种情况,色彩重的图为等概率的情况,色彩浅的为不等概率出现的情况。

\begin{figure}[h]
    \centering
    \includegraphics[width=\textwidth]{image2.png}
    \caption{16QAM 时的发送星座图}
    \label{fig:16QAM_constellation}
\end{figure}

\begin{figure}[h]
    \centering
    \includegraphics[width=\textwidth]{image1.png}
    \caption{16QAM 时的 BER-SNR 曲线}
    \label{fig:ber-snr}
\end{figure}

\begin{figure}[h]
    \centering
    \includegraphics[width=\textwidth]{image2.png}
    \caption{16QAM 时接收信号图}
    \label{fig:signal-diagrams}
\end{figure}

在星座点数量固定为 16 个时,使用不同形状和不同概率的星座图,可得如图 \ref{fig:ber-snr} 及图 \ref{fig:signal-diagrams} 的实验结果。由大量的对比实验可以得出若想改善传输性能,可以将离原点越近的星座图点分配概率越大,离原点越远的星座图点分配概率越小,如图 \ref{fig:ber-snr} 所示,不等概率情况的性能优于等概率的情况,例如图 \ref{fig:signal-diagrams}(d) 优于图 \ref{fig:signal-diagrams}(a)。通过图 \ref{fig:ber-snr} 还可以看出所有等概率情况下(图 \ref{fig:signal-diagrams}(a) — (c)),图 \ref{fig:signal-diagrams}(a) 所表示的调制方式性能最佳。所有不等概率情况下(图 \ref{fig:signal-diagrams}(d) — (f)),图 \ref{fig:signal-diagrams}(d) 所表示的调制方式是最优的。当纠前 BER=0.02 时,图 \ref{fig:signal-diagrams}(d) 的情况可以取得 11dB,原 16QAM 的 SNR 门限值为 13dB,而 8QAM 的 SNR 门限值由 4.1 节可得为 10dB,因此对星座图的点进行概率分配可以在一定程度上改善传输性能,但仍不能产生优于 8QAM 的结果。

\section*{七、总结与展望}

\subsection{7.1 模型评价}

本文通过对光传送网建模与价值评估问题的研究,针对问题一中的 BER-SNR 关系曲线,本文建立了数学表达式推导,并且用 MATLAB 进行光信号传输建模,对每一跨光纤传输过程进行独立分析,建立了关于光链路中光信号及噪声的递推模型,结合理论与实际,得到的模型有很好的合理性。

针对问题二,文章进行了算法分析,进行规划网络的建模,并通过穷举法验证了规划网络模型的合理性,对网络存在节点的情况进行数学分析建模,计算过程十分快速,易于理解与实现。推理出存在中间节点可以增加网络总价值所需要满足的条件,基于这个条件进行问题求解。

针对问题三,本文从星座图中点的位置、大小,及出现概率入手进行探究,得出在调制格式的信息熵需保持在 3bit 的前提下,不同数量(9-16 个)星座点情况下,概率组合均只有唯一解,基于 8 个点的情况和 16 个点的情况进行了大量实验,实验充足,并由大量的对比实验可以得出若想改善传输性能,可以将离原点越近的星座图点分配概率越大,离原点越远的星座图点分配概率越小。

\subsection{7.2 模型改进}

在问题一中的建模过程假设条件仍然过多,后期需要加强模型的适用性。对问题二中模型的目标函数的合理性还有待商榷,且目标函数对最终结果影响差距很大,模型鲁棒性需要提高。对问题三中的模型分析多半靠实验得到,缺乏理论依据,可以尝试阅读更多的文献优化模型。

\section*{参考文献}

[1] 邓俊辉. 数据结构:C++语言版.第 2 版[M]. 清华大学出版社, 2012.

[2] Proakis J G, Manolakis D G. Digital Signal Processing: Principles, Algorithms, and Applications / J.G. Proakis, D.G. Manolakis[M]. Prentice-Hall, 1996.

[3] 莫徽忠. 级联掺铒光纤放大器噪声特性研究[J]. 电脑知识与技术, 2008, 4(33):1517-1518.

[4] 杨宁, 杨铸, 漆启年, 等. DWDM 系统中级联 EDFA 光信噪比计算[J]. 通信学报, 2003, 24(1):75-82.

[5] 庞琳娜, 张阳安. 信号星座的优化设计[C]// 中国高校通信类院系学术研讨会. 2009.

[6] 刘继红, 李佳泯, 梁猛. 16-QAM 相干光纤通信系统星座图的优化与选择[J]. 半导体光电, 2012, 33(1):110-112.

[7] 樊信昌. 通信原理[M]. 国防工业出版社, 2001.

[8] 田振清, 周越. 信息熵基本性质的研究[J]. 内蒙古师范大学学报(自然科学汉文版), 2002, 31(4):347-350.

[9] 吴毅凌, 李红滨, 赵玉萍. 一种新的 22n+1 阶 QAM 星座图设计[J]. 电子与信息学报, 2010, 32(6):1510-1514.