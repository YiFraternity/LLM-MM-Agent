\begin{center}
\textbf{“华为杯”第十五届中国研究生数学建模竞赛}
\end{center}

\textbf{题目} \hfill \textbf{关于跳台跳水体型系数设置的建模分析}

\begin{abstract}    

在跳水运动中,动作难度系数是影响运动员动作完成分的一个重要因素。难度系数与运动员的起跳方式及空中动作有关,难度系数的大小直接影响着运动员最终的得分。除去起跳和动作本身的因素,运动员本身的体型如身高和体重也影响着跳水动作的完成质量。为了使得运动员完成动作更为公平,本文通过建立人的跳水模型来分析难度系数和运动员体型的关系,并以此给出体型校正系数来校正各个动作的难度系数。

在问题一中,本文通过研究国际泳联(FINA)对于各个动作难度系数的规定,可以得到跳水动作的难度系数与翻腾的圈数正相关,同时与转体的圈数正相关。对于只有翻腾没有转体的跳水动作而言,可以通过两种方法来增加动作的难度系数,一是增加改变空中的姿态,而是增加翻腾的圈数。对于翻腾和转体都有的跳水动作,运动员可以通过改变跳水的动作(向前起跳或者向后起跳),增加翻腾的圈数,改变空中姿态。如果想要较好的完成某些动作,特别是高难度动作,就需要增加运动员腾空的时间和角动量,即起跳时的垂直速度和角速度的大小。

在问题二中,需要建立不同体型的跳水运动员完成各个动作的时间的模型。跳水运动员跳水的过程可以看作是一个多刚体的运动。利用多刚体运动的物理学知识可以得到对应的运动方程和方向方程。将需要完成 $m$ 个翻腾和 $n$ 个转体的跳水的过程可以划分为九个阶段。偶数阶段均为形变,奇数阶段主要用来实现翻腾和转体任务。各个奇数阶段的变换都是通过手臂的下垂和上举来完成转体和加速。对奇数的各个阶段所花费的时间求和可以得到跳水动作的时间为
\[ T = (129.4905m - 1.6310n - 3.2389) / l \]。
本文需要建立一个身高和体重对跳水动作完成时间的模型。考虑到运动员的体重和身高分别变为原来的 $k_1$ 倍和 $k_2$ 倍的时候,最终的完成 5133C 这套动作所花费的时间分别变为原来的 $1.489k_1 + 0.397$ 和 $1.489k_2^2 + 0.397$ 倍。

在问题三中,判断体型修正系数的必要时,实际上就是分析出两个体型不同的人完成同一套跳水动作的具体差异。利用问题二的模型可以得知,体型较大的人和体型相对较小的人完成同一套动作所需要的时间不同。并且体型较小的人对具体或是抱膝的动作能够更快的完成,所以体型修正系数的添加是有必要的。对 5133C 跳水动作,构建关于身高和体重的体型修正系数表达,利用身高体重的不同与垂直初速的关系,最终求到系数表示为 $Cor = 0.0043 \times \sqrt{mh} + 0.4182$。

在问题四中,利用问题二中的完成 $m$ 个翻腾和 $n$ 个转体所需要的时间 $T = (129.4905m - 1.6310n - 3.2389) / l$ 和问题三中求解得到体型修正系数 $Cor = 0.0043 \times \sqrt{mh} + 0.4182$ 可以计算出指定动作的对应的难度系数。从结果来看,与附件 1 中的难度系数存在一定差异,不同动作的难度系数有增加有减少,其主要原因是加入了体型修正系数。

总的来说,本文对运动员在跳台跳水时,合理的设置体型修正系数可以使得跳水的难度系数更加合理。

关键词:难度系数;角动量;多刚体运动;体型修正系数
\end{abstract}

\tableofcontents

\section{问题重述}

自人类掌握了游泳技能以后,人类就开始了简单的跳水活动。跳水运动被誉为奥林匹克运动中最优美的运动赛事,该项运动要求运动员在较短的时间内,从跳板或跳台蹦出,在空中做出相应的姿势,并以特定的姿势入水。跳水运动要求运动员在空中具备一定的协调性、柔韧性以及平衡感和时间感。

随着运动员身体素质的提升以及训练质量的提升,跳水运动的动作难度越难越高,作为国际泳联管理的竞技运动,该项运动的规则每四年更新一次,每次规则更新会对一些动作的难度系数 \footnote{做出更改,也可能会增加新的动作并赋予难度系数,运动员的最终得分则是由运动员所做出动作的难度系数决定的。}

不同的难度系数的规定主要是由跳水运动员的起跳方式和空中动作有关。裁判们再根据跳水过程中,选手的完成程度,变现的优劣和入水的效果给出一定客观的评分。然后将总的得分乘上选手完成的动作的难度系数就是选手这一跳的得分。因此,考虑到比赛的公平性,跳水动作的难度系数应该是能够充分反应动作的真实难度。但是有人提出异议,体型瘦小的运动员在空中进行翻腾和转体动作的时候存在着体型上的优势,所以应该设置体型修正系数来修正。

本文所需要解决的问题为:

(1)针对附录中给定的关于十米跳台的跳水动作的难度系数计算方式和各个跳水动作的总的难度系数。通过分析,整理出对后面的问题有所帮助的结论,例如翻腾的圈数与翻腾的难度系数的关系。

(2)利用物理学的知识,分析体型对跳水动作时间的影响,对不同体型(身高和体重)的运动员完成各个跳水动作的时间的关系进行建模。

(3)对于第二问中建立的不同体型的运动员完成各个动作时间的模型,分析在 10 米跳台的跳水比赛中是否有必要设置体型校正系数,如果有必要,说明体型校正系数该如何进行设定。

(4)基于问题二和问题三中建立的模型,建立一个根据跳水动作完成的难度系数评定模型。对给定的 26 组动作代码进行评定,与实际的难度系数进行比较,如果有区别,请对这种现象做出解释。

\section{问题分析}

在解决问题之前,所需要做的首要事情就是对跳水难度系数的组成及评定的规则的理解,包括了各个动作的含义等。

对于问题一,在附录三中是计算一套跳水动作难度系数的组成部分,在附录四中是每一套动作的难度系数。首要的任务是理解难度系数的计算方式,包括一套完整的跳水动作中的翻腾的圈数(A),空中姿态(B),转体的圈数(C),起跳方式(D)以及非自然入水(E)五个部分。在附录四中主要包含了各套动作

的难度系数。通过对附录三和附录四的理解,可以分为两个方面。一是对附录三中的每一项难度系数评分进行分析,得到影响这些项目得分的一些因素。另外一方面的是对附录四中的难度系数得分按照一些起跳的起跳方式和控制姿态分别分析每一类动作对难度系数的影响因素。

对于问题二,在分析完了附录三和附录四,首先了解了每一组动作代码的具体含义,其次分析运动员的体型对跳水动作完成时间的影响。首先利用自由落体的原理先计算的一个安全的跳水时间范围。由于运动员的跳水运动可以看作是一个多刚体\cite{ref2}\cite{ref3}的运动。利用多刚体运动学方面的之间建立对应的运动方程和定向方程。假设在跳水的过程中共需要完成 $m$ 个翻腾和 $n$ 个转体动作,那么可以将运动员跳水动作的过程分为九个部分,每偶数个部分看作是一次脉冲式的形变。在着九个阶段中,主要是利用手臂的上下形变控制身体开始或结束转体。由于偶数部分是脉冲式的,所以这些阶段的时间可以忽略不计,最终总的时间只需要考虑奇数部分即可。在得到跳水时间模型之后,分析身高体重\cite{ref3}对其的影响。为了能够比较直观的看出不同体型对跳水动作完成时间的影响,本文针对不同体型的人,利用控制变量法分析这些人完成 5133C 动作的时间与身高和体重的关系。

对于问题三,判断体型修正系数的必要性,实际就是研究不同体型的人完成一套跳水动作时的具体差异哪里,对于体型有差异的两个人,完成同一套动作的时间不同。完成动作的时间与翻腾转体时候的角动量\cite{ref4}\cite{ref5}有关。相对而言,体型较小的人有更小的回转半径,也就是说对应的转动惯量\cite{ref6}也变小,角速度较大,所以需要完成的时间就少。从分析来看,可以知道添加体型修正系数有必要。所以同样对 5133C 的动作进行分析,通过身高和体重的成倍变化,计算得到垂直的初速度与倍数之间的关系。从得到的关系可以知道,构造质量和身高的修正系数。由于身高和体重的变化对角速度的影响不同,最终可以求解得到体型修正系数。

对于问题四,利用问题二中求解到的不同体型的人完成各个跳水动作所花费的时间关系。由于体型的差别,所以导致不同体型的人对同一套动作的完成时间不同。为了弥补体型的区别,利用问题三中得到的体型的修正系数,最终可以得到一个能够计算各个跳水动作的难度系数的模型。

\section{模型假设}

\begin{enumerate}
    \item 假设所有跳水运动员起跳时没有助跑阶段,都是原地起跳。
    \item 假设运动员跳水过程中不受空气阻力影响。
    \item 假设运动员身体各部分密度均匀。
    \item 假设运动员刚准备起跳时,只有向前和向后的速度无以自身为轴的转体速度。也就是说运动员起跳时只会提供翻腾的速度,转体的角度需要手臂的动作来实现。
\end{enumerate}

\section{四、符号说明}

\begin{tabular}{c|l}
\hline 符号 & 符号说明 \\
\hline $H_{p}, M_{p}$ & 人体的身高和体重 \\
$B_{i}$ & 每一个刚体 \\
$\tilde{J}_{i}^{j}$ & 每一个刚体质心到相应的“铰”的向量表示 \\
$C_{i}$ & 每个刚体的质心向量 \\
$R_{i}$ & 不同坐标系之间的旋转矩阵 \\
$I=I(t)$ & 多刚体系统的惯性张量 \\
$A=A(t)$ & 由于形变产生的动量偏移 \\
$\Omega_{i}$ & $B_{i}$对应的角速度 \\
$U, D$ & 分别表示手臂向上和向下 \\
$(\alpha_{l}, \alpha_{r})$ & 两个手臂刚体与身体的夹角 \\
$L$ & 角动量 \\
$c, S$ & 相移和方向 \\
$K(k^{2})$ & 第一类完全椭圆积分 \\
$\Delta \phi$ & 翻腾角度 \\
$L_{+}, L_{-}$ & 形变签收的瞬间角动量 \\
$\tau_{i}$ & 各个阶段花费的时间 \\
$m, n$ & 翻腾圈数和转体圈数 \\
\hline
\end{tabular}

\section{五、模型求解}

\subsection{5.1 问题一的求解}

在附录三中是计算一套跳水动作难度系数的组成部分,在附录四中是每一套动作的难度系数。所以通过对附录三和附录四的理解,可以分为两个方面。一是对附录三中的每一项难度系数评分进行分析,得到影响这些项目得分的一些因素。另外一方面的是对附录四中的难度系数得分按照一些起跳的起跳方式和控制姿态分别分析每一类动作对难度系数的影响因素

\subsubsection{5.1.1 难度系数与整套动作构成的关系}

跳水运动一直是备受关注的一项体育运动。动作难度系数是表明运动员完成跳水动作的难易程度。对于十米跳台而言,根据不同形式的跳水动作,其难度系数的判定主要由五个部分组成。国际泳联在制定跳水竞赛规则时为每一个跳水动作制定了相应的难度系数,从所给的附录中可以看出,这五个部分分别是翻腾圈数(A)、空中姿态(B)、转体圈数(C)、起跳方式(D)、非自然入水(E)五个方面。具体可表示为:难度系数(DD)=翻腾圈数+空中姿态+转体圈数+起跳
方式+非自然入水。

\paragraph{1、难度系数与翻腾圈数的关系}

要想在有限的时间内在空中尽可能的完成较多的翻腾动作是存在一定难度的,因此难度系数与翻腾圈数存在一定的相关性。以附录中 10m 跳台数据,如表 1 所示,对应的关系如图 1 所示。

\begin{table}[h]
\centering
\caption{翻腾圈数与对应的难度系数}
\begin{tabular}{c c | c c}
\hline
圈数 & 难度系数 & 圈数 & 难度系数 \\
\hline
0 & 1 & 3 & 2.5 \\
0.5 & 1.3 & 3.5 & 2.7 \\
1 & 1.4 & 4 & 3.5 \\
1.5 & 1.5 & 4.5 & 3.5 \\
2 & 1.9 & 5.5 & 4.5 \\
2.5 & 2.1 & & \\
\hline
\end{tabular}
\end{table}

从图 1 中不难看出,跳水动作的难度系数与翻腾圈数有近似线性相关的性质,即翻腾圈数越多,难度系数越大,翻腾系数越少,难度系数越小。

\begin{figure}[h]
\centering
\includegraphics[width=\textwidth]{image.png}
\caption{翻腾圈数与难度系数的关系曲线图}
\end{figure}

\paragraph{2、难度系数与空中姿态的关系}

跳水运动员在下落过程中可以做不同的动作姿态,该姿态也是评分的重要依据,因此分析姿态也具有必要性。该项可以根据两个方面进行难度系数制定,第一方面根据运动员起跳前正面朝向以及翻腾方向进行评判,动作分别可表示为面朝水池、背向水池、向外翻腾、向内翻腾以及臂立。另一方面可以根据落水前所做动作进行难度系数判定,分别是直体(用“A”表示)、屈体(用“B”表示)、抱膝(用“C”表示)、翻腾兼转体任意姿态(用“D”表示)以及自由飞行用(E)表示。本文对附录三中 B 表数据进行分析,只取运动员做 0-1 圈翻腾动作的数据如表 2 所示,示意图如图 2 所示。

\begin{table}
\centering
\caption{翻腾 0 到 1 圈时不同空中姿态的难度系数}
\begin{tabular}{l c c c c c}
\hline
 & 向前 & 向后 & 倒立 & 向内 & 臂立 \\
\hline
C=Tuck & 0.1 & 0.1 & 0.1 & -0.3 & 0.1 \\
B=Pike & 0.2 & 0.2 & 0.2 & -0.2 & 0.3 \\
A=Strait & 0.3 & 0.3 & 0.3 & 0.1 & 0.4 \\
D=Free & 0.1 & 0.1 & 0.1 & -0.1 & 0 \\
E=Fly & 0.2 & 0.1 & 0.1 & 0.4 & \\
\hline
\end{tabular}
\end{table}

结合以上图表,难度系数与空中姿态有关,具体来说做除了向内翻腾跳水外,在其他起跳方式下采用直体(用“A”表示)的难度系数最高。另外,做特定动作会产生难度系数扣分的情况,如向内翻腾做屈体,抱膝以及自由组合动,所以难度系数与运动员跳水前正面朝向以及翻腾方向有关。

\begin{figure}[h]
\centering
\includegraphics[width=\textwidth]{image.png} % 替换为实际图片路径
\caption{翻腾 0 到 1 圈时不同空中姿态的难度系数}
\end{figure}

\paragraph{难度系数与转体圈数的关系}

运动员在落水前可以做转体动作,转体转的圈数的不同对运动员的成绩也有不同的影响,所以难度系数与转体圈数有关,本文分析附录三中翻转的部分数据,数据如表 3 所示,本文只分析向前部分的数据,由于研究的是难度系数与转体跳圈数的关系,本文只研究转体不同圈数与难度系数的关系。以 0.5 圈做一次分析,详细的难度系数与转体圈数的关系如图 3 所示。

\begin{table}
\centering
\caption{不同转体圈数对应的难度系数}
\begin{tabular}{c c | c c}
\hline
转体圈数 & 难度系数 & 转体圈数 & 难度系数 \\
\hline
0.5 & 0.4 & 3 & 1.5 \\
1 & 0.6 & 3.5 & 1.6 \\
1.5 & 0.8 & 4 & 1.9 \\
2 & 1.0 & 4.5 & 2 \\
2.5 & 1.2 & & \\
\hline
\end{tabular}
\end{table}

\begin{figure}[h]
    \centering
    \includegraphics[width=\textwidth]{image.png}
    \caption{难度系数与转体圈数的关系图}
    \label{fig:difficulty_rotation}
\end{figure}

通过以上图表可以看出,难度系数与转体动作圈数相关,总的来说,转体运动圈数越多,难度系数越大。另外,表中同一转体圈数产生了不同的难度系数,这是因为运动员做的是转体与翻转运动相结合的动作,做翻转动作翻转圈数不同难度系数也不同,但大体上是做翻转圈数越多,难度系数越大,这与本文分析难度系数与翻转圈数得出的结论是一致的。

\paragraph{难度系数与起跳方式的关系}

不同的起跳方式需要不同的动作执行,不同的动作对运动员的成绩具有不同的加成,因此跳水的难度系数与起跳方式也有关,本文只对 10m 跳台运动进行分析,本文共研究四种跳水方式,分别是向前起跳、向后起跳、翻身跳以及向内跳,不同跳水方式可以做出不同的翻转圈数,由于研究的是跳水方式与难度系数之间的关系,本文将同一跳水方式作为一组。不同起跳方式对应的难度系数如表 \ref{tab:difficulty_start} 所示,关系图如图 4 所示。

\begin{table}[h]
    \centering
    \caption{不同起跳方式对应的难度系数}
    \label{tab:difficulty_start}
    \begin{tabular}{|c|c|c|c|c|c|c|c|}
    \hline
    Fwd 0.5-3.5 & Fwd 4-5.5 & Back 0.5-3 & Back 3.5-4.5 & Rev 0.5-2 & Rev 2.5-3 & Rev 3.5-4.5 & Inw 0.5-1 & Inw 1.5-4.5 \\ \hline
    0 & 0.2 & 0.2 & 0.2 & 0.3 & 0.4 & 0.3 & 0.3 & 0.2 \\ \hline
    \end{tabular}
\end{table}

\begin{figure}[h]
    \centering
    \includegraphics[width=\textwidth]{image.png}
    \caption{不同起跳方式对应的难度系数关系图}
    \label{fig:difficulty_coefficient}
\end{figure}

根据以上图表可以看出,难度系数的设定与跳水前运动员的跳水方式有一定关系,如果是选择翻身跳,难度系数就会比较高,这也是取得最大难度系数的跳水方式,另外选择向内跳也有一定的难度系数。而向前起跳和向后起跳的难度系数最小,而且如果选择向前起跳的跳水方式,同时如果翻转圈数太小,若小于3.5圈的话,难度系数都为零,若翻转圈数大于该范围,那么会有一定的难度系数,但是难度系数仍比较小。这说明了难度系数与翻转周数之间有关,关系与第一组得出的结论是一致的,结合上述图表,本文可以发现,选择特定的跳水方向,就会产生不同的难度系数。

下面分析一下在臂立跳水(仅在跳台跳水中采用)不同跳水方式与难度系数之间的关系。在本组中本文只研究三种跳水方式,分别是向前起跳、向后起跳以及翻身跳,不同的跳水方式可以做出不同的翻转圈数,本文着重研究臂立跳水条件下不同跳水朝向与难度系数之间的关系,所以将同一跳水方式作为同一类进行研究。本文只研究10m跳台的数据具体数据如表\ref{tab:difficulty_coefficient}所示,示意图如图5所示。

\begin{table}[h]
    \centering
    \begin{tabular}{|c|c|c|c|c|c|}
        \hline
        臂立前翻 & 臂立前翻 & 臂立后翻 & 臂立后翻 & 臂立倒立 & 臂立倒立 \\
        0-2圈 & >2圈 & 0-0.5圈 & 1-4圈 & 0-0.5圈 & 1-4圈 \\
        \hline
        0.2 & 0.4 & 0.2 & 0.4 & 0.3 & 0.5 \\
        \hline
    \end{tabular}
    \caption{不同起跳方式组合跳水的难度系数}
    \label{tab:difficulty_coefficient}
\end{table}

\begin{figure}[h]
    \centering
    \includegraphics[width=\textwidth]{image.png}
    \caption{不同跳水方式的组合跳水的难度系数示意图}
    \label{fig:difficulty_coefficient}
\end{figure}

根据以上图表本文可以得出跳水朝向与难度系数之间有一定关系,具体表现为转身跳难度系数大于向前起跳和向后起跳的难度系数,但是向前起跳与向后起跳之间的难度系数差异不大。另外可以看出不同的翻转周数之间难度系数有差距,但是这是由于翻转圈数不同而产生的结果,这与第一组分析情况一致。结合上一组的数据分析以及这组的数据分析,本文可以得出,跳水运动的难度系数与跳水方向有一定的关联性,但是只体现在转身跳与其他跳水方向之间。

\paragraph{难度系数与非自然跳水方式的关系}

该组条件下又可将跳水方式分为四组进行难度系数分析,分别是向前起跳/向内跳、向后起跳/转身跳、臂立向后起跳/臂立转身跳和臂立向前起跳。从附录中可以得知,不同的跳水方式之间的难度系数差距不大,只是在不同的翻转圈数的情况下产生了差异,再一次验证了本文在第一组得出的结论。鉴于以上数据分析,本文可以得出在臂立跳水的情况下,跳水运动难度系数与跳水方式没有太大的关联性。

\subsubsection{难度系数与起跳方式的关系}

纵观所有的十米跳台跳水动作,可以按照不同的起跳方向和动作结构将跳水动作划分为六个部分。这六个部分分别为向前动作,向后动作,向外动作(反身跳),向内动作,转体动作和臂立动作。根据在空中呈现的状态,又可以把每一个部分细分为直体、屈体、抱膝和自由跳动作四个子部分。而第五组转体动作需要与其他几组动作融合进行难度系数判定,分别课分为转体动作-向前动作、转体动作-向后动作、转体动作-向外动作以及转体动作-向内动作。

由于研究的是难度系数与起跳方式的关系,而第五组动作与其他几种动作有交叉地方,本文只取其他的几组数据,基于控制变量法思想,本文控制翻转圈数一定,结合表中数据,本文取翻转一周的数据,本文取直体、屈体、抱膝动作三部分数据,如表6所示。

\begin{table}
\centering
\begin{tabular}{|c|c|c|c|}
\hline
组别 & 直体 & 屈体 & 抱膝 \\ \hline
向前动作 & 1.8 & 1.7 & 1.6 \\ \hline
向后动作 & 1.9 & 1.8 & 1.7 \\ \hline
向外动作 & 2 & 1.9 & 1.8 \\ \hline
向内动作 & 1.9 & 1.6 & 1.5 \\ \hline
臂立动作 & 2 & 1.9 & 1.7 \\ \hline
\end{tabular}
\caption{直体、屈体、抱膝动作三部分数据与难度系数}
\end{table}

若将表格纵向比较,可以得出难度系数与跳水方向的关系图,根据表 6 可以得知,本文可以看出,难度是与跳水方向有一定的关联性,具体表现为向外方向与臂立动作的困难系数比较大,向内方向与向前方向困难系数较小,所以苦难系数与跳水方向有一定的关联性。

若将表格横向比较,本文可以得出难度系数与运动员入水前的落水姿势图,本文绘制以下图像,如图 6 所示。

\begin{figure}[h]
\centering
\includegraphics[width=0.8\textwidth]{image.png}
\caption{运动员落手姿势与难度系数关系图}
\end{figure}

本文可以看出难度系数与跳水姿态有关,直体姿态难度系数较高,而抱膝难度系数较小而屈体的难度系数介于二者之间。

\subsubsection{5.1.3 难度系数规则得到的结论}

通过对附录三的每一项难度系数评分指标和附录四的每一类跳水方式的研究和分析,可以得到以下四条结论:

1. 跳水动作的难度系数与翻腾的圈数正相关,同时与转体的圈数正相关。
2. 对于只有翻腾没有转体的跳水动作而言,可以通过两种方法来增加动作的难度系数,一是增加改变空中的姿态,而是增加翻腾的圈数。
3. 对于翻腾和转体都有的跳水动作,运动员可以通过改变跳水的动作(向前起跳或者向后起跳),增加翻腾的圈数,改变空中姿态。
4. 如果想要较好的完成某些动作,特别是较难的动作,就需要增加运动员腾空的时间和角动量,即起跳时的垂直速度和角速度的大小。

\subsection{5.2 问题二的求解}

常看比赛的人都知道,大部分的跳水运动员都是中等偏低的身高和适中的体

重。在跳水比赛中,处于公平性的考虑,跳水的难度系数应该充分反映了该动作的真实难度。但是这就带来了一个问题,不同体型的人对于同一套动作的完成难度可能存在一定的差异,比如体型瘦小的运动员在做翻腾及转体动作的时候可能有体型的优势,所以应当设置一定的体型系数进行矫正。

\subsubsection{5.2.1 只考虑自由落体时的跳水时间}

为了能够确定完成各个动作的时间与运动员体型的关系,首先要确保完成动作的时间在一定的安全范围内,即如果运动员过快完成动作的话则说明选取的动作难度系数过小,如果运动员超过了最大时间没有完成指定的动作,最终运动员在落水前则很难完成该套动作。为了确定跳水时间的安全范围,本文将运动员在跳水的过程中看作为一个质点的运动。具体的跳水路线如图 7 所示。

\begin{figure}[h]
    \centering
    \includegraphics[width=0.45\textwidth]{image1.png}
    \caption{跳水运动员跳水的质心路线}
    \label{fig:7}
\end{figure}
\begin{figure}[h]
    \centering
    \includegraphics[width=0.45\textwidth]{image2.png}
    \caption{考虑运动员身高的跳水的质心路线}
    \label{fig:8}
\end{figure}

假设跳水运动员起跳的初始速度为 \( v_0 \),起跳的方向与跳台的夹角为 \( \theta \),跳台的高度为 \( H \)。则根据自由落体方面的知识可以得到,下落的总时间可以表示为

\begin{equation}
T_{safe} = T_{up} + T_{down}
\tag{1}
\end{equation}

其中,\( T_{up} \) 表示的是运动员在上升过程中的时间,同理 \( T_{down} \) 为下降时间。由于只需要考虑运动员落水的时间,所以对于水平方向的位移可以不作为考虑对象,只需要对垂直速度进行分析即可。根据自由落体方面的只是,可以得知,在最高点的时候,运动员的垂直速度为 0,并且运动员在跳水的过程中只收到了重力的影响。所以运动员上升过程中,可以看作是一个反自由落体的过程。由于只受到重力的影响,所以可以计算得到上升时间为

\begin{align}
\cos_{+}(\alpha_{l},\alpha_{r}) &= \cos\alpha_{l} + \cos\alpha_{r} \\
\sin_{+}(\alpha_{l},\alpha_{r}) &= \sin\alpha_{l} + \sin\alpha_{r} \\
\cos_{-}(\alpha_{l},\alpha_{r}) &= \cos\alpha_{l} - \cos\alpha_{r} \\
\sin_{-}(\alpha_{l},\alpha_{r}) &= \sin\alpha_{l} - \sin\alpha_{r}
\end{align}

同样的,下落的过程只受到重力的影响。但是由于落水的速度未知,所以可以通过计算高度的方法同样求得下落的时间。

\begin{align}
A_{l} &= a_{7} - a_{1}\cos\alpha_{l} + a_{2}\sin\alpha_{l} - a_{3}\cos(\alpha_{l}+\alpha_{r}) \\
A_{r} &= -a_{7} + a_{1}\cos\alpha_{r} - a_{2}\sin\alpha_{r} + a_{3}\cos(\alpha_{l}+\alpha_{r})
\end{align}

\begin{equation}
T_{down} = \sqrt{\frac{2gh}{g}}
\tag{4}
\end{equation}

所以最终的时间可以表示为

\begin{equation}
T_{dive} = T_{up} + T_{down} = \frac{v_0 \cos \Theta}{g} + \sqrt{\left( \frac{v_0 \cos \Theta}{g} \right)^2 + \frac{2H}{g}}
\tag{5}
\end{equation}

为了能够得到精确的时间范围,本文将选取两种极端的情况分别计算在空中的最小时间和最大时间。为了方便计算,本文中所用的重力加速度 $g = 10 \, m/s^2$。从(5)中可以看出,跳水的时间与起跳的速度 $v_0$ 有正相关性。所以两种极端的情况分别取最小垂直起跳速度和(即 $v_0 \cos \Theta = 0$)和最大的垂直起跳速度($v_{\text{max}} \cos \Theta$)。当 $v_0 = 0$,计算得到 $T_{dive} \approx 1.40 \, \text{s}$。但是由于对于起跳速度的最大值比较难确定,所以本文将采用跳高的世界记录来计算最大的跳水时间。本文再次选取的是索托马约尔在 1993 年创下的世界记录,最高的记录为 $H_{\text{max}} = 2.45 \, \text{m}$。因此可以计算得到最大的初始速度 $v_{\text{max}} \cos \Theta = 7.0 \, \text{m/s}$,计算得到 $T_{dive} \approx 2.27 \, \text{s}$。所以正常的跳水时间应该控制在的合理范围为 $1.4 \, \text{s} < T_{dive} < 2.27 \, \text{s}$。

由于人的质心起始时并不是在十米跳台的位置,所以如果将人的身高考虑进去,则跳水的时间将会有稍微不同。考虑到跳水运动员身高的跳水的质心路线图如图 8 所示。

假设人的身高表示为 $H_p$,根据人的质心位置会在人体高度的 $57.66\%$ 处。所以可以得到加入了身高因素后,安全时间的范围为

\begin{equation}
T_{dive} = T_{up} + T_{down} = \frac{v_0 \cos \Theta}{g} + \sqrt{\left( \frac{v_0 \cos \Theta}{g} \right)^2 + \frac{2(H + 1 - 0.5766 H_p)}{g}}
\tag{6}
\end{equation}

其中 $v_0 \cos \Theta$ 可以表示为

\begin{equation}
v_0 \cos \Theta = \sqrt{2g (H_{\text{max}} - 0.5766 H_p)}
\tag{7}
\end{equation}

世界记录的创造者索托马约尔拥有着 $1.95 \, \text{m}$ 的身高,通过计算可以得到他的质心距离跳台有 $0.5766 H_p = 1.12 \, \text{m}$。由于最小的安全跳水时间是在 $v_0 \cos \Theta = 0$ 的时候,所以最小的安全跳水时间为 $T_{dive} \approx 1.49 \, \text{s}$。对于最大的垂直起跳速度,同样选取世界记录 $H_{\text{max}} = 2.45 \, \text{m}$,所以对应的最大垂直起跳速度为 $v_{\text{max}} \cos \Theta = 5.14 \, \text{m/s}$,所以最终计算得到的最大时间为 $T_{dive} \approx 2.1 \, \text{s}$,最终得到的安全时间范围为 $1.49 \, \text{s} < T_{dive} < 2.1 \, \text{s}$。

\subsubsection{5.2.2 跳水时间与身高体重的关系建模}

1、影响跳水时间的因素分析 \cite{ref8}

首先需要解决的问题是起跳高度的跳水时间,身体反转速度之间的关系。在常人的眼中,都会这样认为:起跳高度越高,腾空的时间越多。所以为了验证这种说法,本文选取了不同的质心高度分别在垂直方向上上升一定高度之后的初速
度、跳水时间以及不同高度之间的跳水时间差。为了方便研究,所以本文每次升
高的高度选取5cm。详细的结果如表7 所示。

\begin{table}
\centering
\caption{起跳高度的跳水时间}
\begin{tabular}{c c | c c | c c | c c}
\hline
\multirow{2}{*}{起跳} & & \multicolumn{6}{c}{质心高度 (m)} \\
\cline{3-8}
& & 0.7 & & 0.8 & & 0.9 & \\
\hline
高度 & 速度 & 时间 & 增量 & 时间 & 增量 & 时间 & 增量 \\
\hline
0.10m & 1.4142 & 1.6111 & & 1.6179 & & 1.6247 & \\
0.15m & 1.7321 & 1.6463 & 0.0352 & 1.6531 & 0.0352 & 1.6598 & 0.0351 \\
0.20m & 2.0000 & 1.6765 & 0.0302 & 1.6832 & 0.0301 & 1.6900 & 0.0302 \\
0.25m & 2.2361 & 1.7035 & 0.0270 & 1.7102 & 0.0270 & 1.7169 & 0.0269 \\
0.30m & 2.4495 & 1.7282 & 0.0247 & 1.7349 & 0.0247 & 1.7416 & 0.0247 \\
0.35m & 2.6458 & 1.7512 & 0.0230 & 1.7579 & 0.0230 & 1.7646 & 0.0230 \\
0.40m & 2.8284 & 1.7728 & 0.0216 & 1.7795 & 0.0216 & 1.7862 & 0.0216 \\
0.45m & 3.0000 & 1.7933 & 0.0205 & 1.8000 & 0.0205 & 1.8067 & 0.0205 \\
\hline
\end{tabular}
\end{table}

表中的数据均为假定的十米跳台的情况。假设起跳时的总起跳能量不变,为
了能够更加清楚的分析能量的转换,本文对起跳是每升高0.01m 到0.05m,重力
势能的增加和其导致的转动能力的减小也进行了计算。为了方便计算,假设起跳
后的转动半径为 0.5m,平动能量不变,并且没有其他的能量损失。计算的结果
如表8 所示。

\begin{table}
\centering
\caption{起跳高度的转动能量}
\begin{tabular}{c | c c | c c}
\hline
\multirow{2}{*}{上升高度} & \multicolumn{2}{c|}{能量变化} & \multicolumn{2}{c}{能量变化 (转动半径 0.5)} \\
\cline{2-5}
& 所需的能量 & 增量 & 所需的转速 & 增量 \\
\hline
0.01m & 0.1m & & 0.8944 & \\
0.02m & 0.2m & 0.1m & 1.2649 & 0.3705 \\
0.03m & 0.3m & 0.1m & 1.5492 & 0.2843 \\
0.04m & 0.4m & 0.1m & 1.7889 & 0.2388 \\
0.05m & 0.5m & 0.1m & 2.0000 & 0.2111 \\
0.10m & 1.0m & 0.5m & 2.8284 & 0.8284 \\
0.15m & 1.5m & 0.5m & 3.4641 & 0.6357 \\
\hline
\end{tabular}
\end{table}
从表 7 中可以看出,人的质心升高的相对高度越大,  对跳水时间的延长影
响相对减小,并且与运动员的身高无关。人的质心高度分别为0.7m,0.8m,0.9m
的时候,腾空高度由0.25m 增加到0.35m 的时候,腾空的时间仅仅增加了0.0477s。
同样的可以得到,腾空的高度由 0.35m 增加到 0.45m 时,腾空的时间增加了
0.0421s,这样的结果比0.25m 到0.35m 的时间增加比重还要少。 
从表8 中可以看出在垂直方向上,每升高0.01m 所增加的重力势能,能够使
得转移到转动上的能力较少 0.1m 焦耳(m 为人的体重),最终使得转速下降
0.8944rad/s。腾空高度的增加,相对转速下降的也在慢慢减少。如果十米跳台的
跳水时间按照 1.65s 来计算,前者的转角减小约为 77 度,后者减小的转角约为

66 度,转角的减小对动作的完成质量会有很大的影响。值得注意的是具备一定腾空高度的抛物线的运动规律,能够使得运动员更方便的调整身体入水的姿态,更加容易垂直入水,并且压住水花,最终取得比较乐观的分数。

在跳水运动中,要增大人体在空中的翻腾速度,可以通过增加人体在起跳时候能够使人体翻腾的冲量矩,但是如何增大起跳阶段的冲量矩是一个比较复杂的问题。由于人体在起跳阶段蹬腿力度的大小,方向都在不断变化。身体的姿态也是不断变化的,导致了人体的质心位置也在不断的变化,踏跳力的转动力矩也会随之改变。再加上在起跳的时候会有摆臂动作,两臂的摆动形成的附加转的变化力矩对起跳同样有着一定的影响。所以为了方便分析,本文假设在起跳蹬腿的阶段,力的大小和方向是不变的,人体的姿态也以离开跳台的瞬间为定格,人体执行的位置也认为是不变的,然后进行局部的计算比较。

2、运动员跳水模型建立及跳水时间的计算

假设跳水动作中包含了 \( m \) 个翻腾和 \( n \) 个转体。为了方便分析,在建立模型的时候假设跳水运动员可以分为三个刚体,即左和右手臂各为一个刚体,其他身体部件可以看成是一个刚体。三刚体模型示意图如图 9 所示。

\begin{figure}[h]
    \centering
    \includegraphics[width=\textwidth]{image.png}
    \caption{三刚体人体结构图}
    \label{fig:9}
\end{figure}

从图 9 中可以看出,每一个刚体 \( B_i \) 都有一个对应的坐标系 \( F_i \)。每两个刚体之间存在一个连接点称为“铰”,每一个刚体的质心到对应“铰”的向量表示为 \( \tilde{J}_i^j \),其中 \( i, j \) 表示的是两个不同的且通过“铰”相连的刚体。每一个刚体的质心向量为 \( C_i \)。假设 \( V \) 表示的在 \( F_c \) 坐标系中的任意一个向量,其在 \( F_s \) 坐标系中的表示为 \( \tilde{V} \),那么可以得到 \( V = R \tilde{V} \),其中 \( R = R(t) \) 表示的是一个旋转矩阵。这种转换的形式可以总结为

\begin{equation}
V = C_i + R_i \tilde{V}_i
\tag{8}
\end{equation}

现在为了能够利用 \( \{ \tilde{J}_i^j \} \) 和 \( \{ R_i \} \) 来表示出 \( C_i \),可以构建 \( B_i \) 和 \( B_j \) 之间的关系表达式

\begin{equation}
C_{i}+R_{i}\tilde{J}_{i}^{j}=C_{j}+R_{j}\tilde{J}_{j}^{i}
\tag{9}
\end{equation}

通过构建 $\sum\limits_{i}m_{i}C_{i}=0$,可以计算得到参考坐标系的中心

\begin{equation}
C_{b}=\frac{1}{M_{p}}\sum_{i\in\{l,r\}}m_{i}(R_{i}\tilde{J}_{i}^{b}-\tilde{J}_{b}^{i})
\tag{10}
\end{equation}

带入(9)可以得到

\begin{equation}
C_{i}=C_{b}-\tilde{J}_{b}^{i}+R_{i}\tilde{J}_{i}^{b}
\end{equation}

值得注意的是,当 $i=b$ 时,$R_{b}$ 为常数。

由[9]可以得到一个多刚体的系统满足如下两个公式

\begin{equation}
I=\sum_{i}(R_{i}\tilde{I}_{i}R_{i}^{t}+m_{i}[|C_{i}|^{2}1(c)-C_{i}C_{i}^{t}])
\tag{11}
\end{equation}

\begin{equation}
A=\sum_{i}(m_{i}C_{i}\times\dot{C}_{i}+R_{i}\tilde{I}_{i}\Omega_{i})
\tag{12}
\end{equation}

其中,$I=I(t)$ 表示的是多刚体系统的惯性张量,$A=A(t)$ 表示的是由于形变产生的动量偏移。对于刚体 $B_{i}$ 的角速度 $\Omega_{i}$ 都和 $B_{b}$ 有着一定的关系,可以定为 $R_{i}^{t}\dot{R}_{i}V=\hat{\Omega}_{i}V$,其中 $V$ 为任意的一个向量。假设手臂的动作是在实现外展内收的时候手臂所在的平面,则运动员的形状可以用 $(\alpha_{l},\alpha_{r})\in[0,\pi]^{2}$。如果运动的形状为 $(\alpha_{l},\alpha_{r})=(0,0)$,则此时的动作代表着运动员处于最自然的状态,也就是双手下垂。同样的,如果 $(\alpha_{l},\alpha_{r})=(\pi,\pi)$,则表示运动员的双手垂直举高,处于起跳姿态。根据旋转矩阵可以得到

\begin{equation}
R_{l}(\alpha_{l})=R_{x}(\alpha_{l}), R_{r}(\alpha_{r})=R_{x}(-\alpha_{r})
\end{equation}

其中第二个式子的负号表示的是在同一个坐标系中,左手的方向和右手是相反的。为了方便后面的表示,本文将用 $U$ 和 $D$ 分别表示手臂的两种位置状态,用 $H$ 和 $L$ 分别手臂的运动方式。$U$ 表示手臂向上垂直举着,$D$ 表示手臂自然放下。$H$ 表示手臂由 $D$ 向 $U$ 的过程,$L$ 表示手臂由 $U$ 向 $D$ 的过程。用这些形变来限制简化惯性张量,可以表示为

\begin{equation}
I=\begin{pmatrix}
I_{xx} & 0 & 0 \\
0 & I_{yy} & I_{yz} \\
0 & I_{zy} & I_{zz}
\end{pmatrix}
\tag{13}
\end{equation}

同样的,动量偏移 $A$ 可以简化为

\begin{equation}
A=(A_{l}\dot{\alpha}_{l}+A_{r}\dot{\alpha}_{r},0,0)^{T}
\tag{14}
\end{equation}

其中,惯性张量的每一元素可以用 $(\alpha_{l},\alpha_{r})$ 分别表示为

\begin{align}
I_{xx} &= a_{0} - 2a_{1}\cos_{+}(\alpha_{l},\alpha_{r}) + 2a_{2}\sin_{+}(\alpha_{l},\alpha_{r}) - 2a_{3}\cos(\alpha_{l}+\alpha_{r}) \\
I_{yy} &= a_{5} - 2a_{1}\cos_{+}(\alpha_{l},\alpha_{r}) + a_{4}\cos_{+}(2\alpha_{l},2\alpha_{r}) - 2a_{3}\cos\alpha_{l}\cos\alpha_{r} \\
I_{zz} &= a_{6} + 2a_{2}\sin_{+}(\alpha_{l},\alpha_{r}) - a_{4}\cos_{+}(2\alpha_{l},2\alpha_{r}) - 2a_{3}\sin\alpha_{l}\sin\alpha_{r} \\
I_{yz} &= a_{2}\cos_{-}(\alpha_{l},\alpha_{r}) - a_{1}\sin_{-}(\alpha_{l},\alpha_{r}) + a_{4}\sin_{-}(2\alpha_{l},2\alpha_{r}) - a_{3}\sin(\alpha_{l}-\alpha_{r})
\end{align}

其中,
\begin{align}
\cos_{+}(\alpha_{l},\alpha_{r}) &= \cos\alpha_{l} + \cos\alpha_{r} \\
\sin_{+}(\alpha_{l},\alpha_{r}) &= \sin\alpha_{l} + \sin\alpha_{r} \\
\cos_{-}(\alpha_{l},\alpha_{r}) &= \cos\alpha_{l} - \cos\alpha_{r} \\
\sin_{-}(\alpha_{l},\alpha_{r}) &= \sin\alpha_{l} - \sin\alpha_{r}
\end{align}

同样的,(14) 动量偏移可以用 $(\alpha_{l},\alpha_{r})$ 分别表示为
\begin{align}
A_{l} &= a_{7} - a_{1}\cos\alpha_{l} + a_{2}\sin\alpha_{l} - a_{3}\cos(\alpha_{l}+\alpha_{r}) \\
A_{r} &= -a_{7} + a_{1}\cos\alpha_{r} - a_{2}\sin\alpha_{r} + a_{3}\cos(\alpha_{l}+\alpha_{r})
\end{align}

式(16)和(17)中的 $a_{0},a_{1},...,a_{7}$ 是可以分别由 $\{m_{i},\tilde{I}_{i},\tilde{J}_{i}^{j}\}$ 的值计算出来的常量。一般来说,惯性张量是一个非对角矩阵,但是总会存在一个坐标变换 $R_{p}$ 使得其变为一个对角矩阵
\begin{equation}
J = diag(J_{x},J_{y},J_{z}) = R_{p}^{-1}IR_{p}
\end{equation}

坐标变换矩阵 $R_{p}$ 在一些刚体的坐标系中可以对角化。所以可以得到
\begin{equation}
R_{p} = R_{x}(p)
\end{equation}

式中的 $p$ 可以表示为
\begin{equation}
p = \frac{1}{2}\arctan\left(\frac{2I_{yz}}{I_{yy}-I_{zz}}\right)
\end{equation}

假设 $I_{s} = diag(I_{s,x},I_{s,y},I_{s,z})$ 表示为起跳状态 $UU$ 的惯性张量,$J_{t} = diag(J_{t,x},J_{t,y},J_{t,z})$ 表示人体位置状态为 $DU$ 和 $UD$ 下的对角化惯性张量。在解决了的运动员的模型之后,多刚体的运动方程可以表示为
\begin{equation}
L = I\Omega + A
\end{equation}
其中,$L$ 表示为角动量,$\Omega$ 表示为角速度。式(21)中的变量均是在质心坐标系下的变量。在没有形变的情况下,$I$ 是一个常数并且 $A = 0$。所以 (21) 可以简化为 $L = I\Omega$,为了得到多刚体的运动方程,对 $l = RL$ 进行求导并且利用 $R^{\prime}\dot{R}V = \hat{\Omega}V = \Omega \times V$,最终可以得到
\begin{equation}
\dot{L} = L \times \Omega = L \times I^{-1}(L - A)
\end{equation}

当没有形变的影响的时候,刚体的运动由六个维度的稳态运动。对于不平衡

的形状的时候,即 $(\alpha_{l}, \alpha_{r})$ 而言,多刚体的平衡点可以利用 $R_{p}$ 进行旋转得到。对于本文的模型,运动员的对角化惯性张量可以表示为 $J = diag(J_{x}, J_{y}, J_{z})$,其中对于所有的姿态 $(\alpha_{l}, \alpha_{r}) \in [0, \pi]^{2}$ 都有 $J_{x} > J_{y} > J_{z}$。

本文为了方便分析,假设形变的过程都是瞬时的,运动方程可以简化为
\begin{equation}
\dot{L} = I^{-1} A \times L
\tag{23}
\end{equation}

如果不考虑外力和形变的影响。总的能量 $E$ 和空间角动量矢量 $l$ 可以表示为常量。所以可以得到
\begin{equation}
L = diag(L_{x}, L_{y}, L_{z})^{T}, \ I = diag(I_{x}, I_{y}, I_{z})
\tag{24}
\end{equation}

角动能可以表示为
\begin{equation}
E = \frac{1}{2} \Omega^{T} I \Omega
\tag{25}
\end{equation}

由于角动能定义了一个庞索椭球的表面,所以对应的能量惯性椭球可以表示为
\begin{equation}
E = \frac{L_{x}^{2}}{2I_{x}} + \frac{L_{y}^{2}}{2I_{y}} + \frac{L_{y}^{2}}{2I_{y}}
\tag{26}
\end{equation}

由于角动量 $l$ 在空间坐标系下是一个常量,并且在质心坐标系中长度为一个常量。所以 $L$ 必须在对应的球面(L-球)上,并且满足
\begin{equation}
l^{2} = L_{x}^{2} + L_{y}^{2} + L_{z}^{2}
\tag{27}
\end{equation}

因此 $L$ 的轨迹必须在能量惯性椭球和 L-球相交上,一般会形成两条闭合的曲线上。在这两条闭合曲线上,时间的演化可以用雅各比椭圆方程的形式进行表示。因此可以得到
\begin{equation}
L_{J}(t; E, I, c) = (L_{x}(t), L_{y}(t), L_{z}(t))^{T}
\tag{28}
\end{equation}

公式右边的变量都是受限制的。特别的,当 $l^{2}I_{y}^{-1} < 2E < l^{2}I_{z}^{-1}$ 时候,右边的变量可以表示为
\begin{equation}
\begin{aligned}
L_{x} &= s \sqrt{\frac{I_{x}(l^{2} - 2EI_{z})}{I_{x} - I_{z}}} cn(\tau, k^{2}) \\
L_{y} &= s \sqrt{\frac{I_{y}(l^{2} - 2EI_{z})}{I_{y} - I_{z}}} sn(\tau, k^{2}) \\
L_{z} &= -s \sqrt{\frac{I_{z}(2EI_{x} - l^{2})}{I_{x} - I_{z}}} dn(\tau, k^{2})
\end{aligned}
\tag{29}
\end{equation}

其中,

\begin{equation}
\tau = \sqrt{\frac{(I_y - I_z)(2EI_x - l^2)}{I_xI_yI_z}}(t + c)
\tag{30}
\end{equation}
\begin{equation}
k^2 = \frac{(I_x - I_y)(l^2 - 2EI_z)}{(I_y - I_z)(2EI_x - L^2)}
\end{equation}

并且 \( c \) 和 \( s \) 分别表示相移和方向。假设本文建立的模型只考虑逆时针的转体运动,所以 \( s \) 的取值为 1。由于雅各比椭球函数是关于 \( \tau \) 的阶段性的函数,可以表示为

\begin{equation}
4K(k^2) = 4\int_{0}^{0.5\pi} \frac{du}{\sqrt{1 - k^2\sin^2 u}}
\tag{31}
\end{equation}

该时间段的轨道函数可以表示为

\begin{equation}
T(E, I) = 4K(k^2)\sqrt{\frac{I_xI_yI_z}{(I_y - I_z)(2EI_x - l^2)}}
\tag{32}
\end{equation}

其中,\( K(k^2) \) 表示第一类完全椭圆积分。对于任意的能量 \( E \) 和惯性张量 \( I \),对应的阶段的时间都可以通过上式计算得到。在此阶段的一个循环中,完成一周的转体时,对应完成的翻腾角度可以表示为 \( \Delta\phi \)。从文献 [10] 中可以得到

\begin{equation}
\Delta\phi = \frac{2ET}{l} - S \mod 2\pi
\tag{33}
\end{equation}

式中前半部分表示的是方向上的变化,后半部分表示的是几何的变化。由于蒙特卡洛公式只能适用于刚体,所以文献 [11] 总结了结果表示为

\begin{equation}
\Delta\phi = -S + \frac{1}{l}\int_{0}^{T} I^{-1}(L - A)Ldt \mod 2\pi
\tag{34}
\end{equation}

其中,\( I \) 和 \( A \) 都是关于时间 \( t \) 的函数,\( S \) 表示的是由轨迹公式推导出来的立体角。为了保证 \( A \) 在定向轨道 \( L \) 的左半部分,可以定义 \( S = A/l^2 \)。通过公式的推导可以得到

\begin{equation}
S(E/l^2, I) = 4\sqrt{\frac{I_y}{I_xI_z(I_y - I_z)H_x}}\left((I_x - I_z)\Pi(\nu, k^2) - I_xH_zK(k^2)\right)
\tag{35}
\end{equation}

其中 \( k^2 \) 可以由 (30) 得到,并且

\begin{equation}
H_x = \frac{2EI_x}{l^2} - 1, \, H_z = 1 - \frac{2EI_z}{l^2}, \, \nu = -\frac{I_z(I_x - I_y)}{I_x(I_y - I_z)}
\tag{36}
\end{equation}

如果将发生形变的阶段看作是脉冲式的,即一触即发,中间几乎不需要消耗什么时间,对应的运动方程可以表示为

\begin{equation}
\dot{L} = I^{-1}A \times L
\tag{37}
\end{equation}

根据 (13) 和 (14) 可以得到

\begin{equation}
\dot{L} = I_{xx}^{-1}(A_{l}\dot{\alpha}_{l} + A_{r}\dot{\alpha}_{r})ML
\tag{38}
\end{equation}

其中 $M$ 是一个常数矩阵。所以式(38)可以看作是一个关于时间的线性微分方程,并且

\begin{equation}
L_{+} = \exp(\theta M)L_{-} = R_{x}(\theta)L_{-}
\tag{39}
\end{equation}

其中 $L_{+}$ 和 $L_{-}$ 都是形变前后的瞬间变量,偏移角 $\theta$ 可以通过

\begin{equation}
\theta = \lim_{\tau \to 0^{+}} \int_{0}^{\tau} I_{xx}^{-1}(A_{l}\dot{\alpha}_{l} + A_{r}\dot{\alpha}_{r})dt
\tag{40}
\end{equation}

计算得到。

由于是三刚体模型,所以整个跳水的九个过程为 $S_{i}, (i = 1, 2, 3, 4, 5, 67, 8, 9)$。详细的过程图如图10所示。

\begin{figure}[h]
\centering
\includegraphics[width=\textwidth]{image10.png}
\caption{三刚体模型转体时手臂的变换过程}
\end{figure}

由于在三刚体模型的跳水的过程中,有四个部分是存在形变的,因此针对这几种脉冲式形变的过程,运动和方向关系可以如图 11 所示。

\begin{figure}[h]
\centering
\includegraphics[width=\textwidth]{image11.png}
\caption{脉冲式形变阶段的运动和方向关系}
\end{figure}

为了方便分析,本文对形变的角度进行适当的参数化来评价偏移角。对于脉冲式的形变,可以定义图11中的 $\chi$ 和 $\gamma$ 分别为

\begin{equation}
\chi = \int_{0}^{\pi} I_{t,xx}^{-1}(\alpha) A_{x}(\alpha) d\alpha
\tag{41}
\end{equation}

为了能够使用 (33) 来计算做了 $n$ 次转体之后,一共经过的翻腾圈数。本文将九个阶段进行结合可以得到

\begin{equation}
\Delta \phi_{kick} = \sum_{i=1(odd)}^{9} \frac{2E_{i}\tau_{i}}{l} - S(E_{3}/l^{2}, J_{t}) - (n-1)S(E_{5}/l^{2}, J_{t})
\tag{42}
\end{equation}

其中,$S$ 可以由式 (35) 计算得到,$E_{i}$ 可以由式 (26) 计算得到,$\tau_{i}$ 可以由 (32) 计算得到。为了能够得到精确的 $m$,令 $\Delta \phi_{kick} = 2m\pi$,并且总的翻腾时间为 $\tau_{1} + \tau_{9}$。通过 $^{[12]}$ 的阅读,可以得到对于跳水动作完成 $m$ 个翻腾和 $n$ 个转体在各个阶段所需要的时间如表9所示。

\begin{table}[h]
\centering
\caption{运动员完成 $m$ 个翻腾和 $n$ 个转体所需要的时间}
\begin{tabular}{c c}
\hline \hline
阶段 ($S_{i}$) & 时间花费 ($\tau_{i}$) \\
\hline
$UU$ & $(64.7452m - 6.5082n - 12.9073)/l$ \\
$DU$ & $16.9805/l$ \\
$UD$ & $11.3584(n-1)/l$ \\
$DU$ & $16.9805/l$ \\
$UU$ & $(64.7452m - 6.5082n - 12.9073)/l$ \\
\hline \hline
\end{tabular}
\end{table}

从表9中可以看出,一个跳水运动员完成 $m$ 个翻腾和 $n$ 个转体所需要的总的时间为

\begin{equation}
T_{air} = (129.4905m - 1.6310n - 3.2389)/l
\tag{43}
\end{equation}

所以最终的跳水时间与角动量的大小有关,并且角动量的在跳水的过程中满足角动量守恒 $^{[13]}$ 定理。

由在整个的跳水过程中,由于不受到空气阻力的影响,所以符合角动量守恒定理。由于完成 $m$ 个翻腾和 $n$ 个转体所需要的跳水时间只与角动量有关,因此只需要计算起跳瞬间的角动量就可以得到跳水时间。

为了方便分析,本文将一套跳水动作分为五个部分,并且五个部分在跳水的过程中发生的形状变化都是瞬间完成的。为了方便分析,可以将一个人划分为14个部分 $^{[14]}$ (本文将身只用一个部分表示,不分为两个部分,因此只划分了14个部分),划分的标准如图12所示。

\begin{figure}[h]
    \centering
    \includegraphics[width=\textwidth]{image.png}
    \caption{人体按照部位划分为 14 个部分。(a)表示的各个部位的名称,(b)树状图的形式表示的各个部位的关系。}
    \label{fig:body_parts}
\end{figure}

图 \ref{fig:body_parts} 人体按照部位划分为 14 个部分。(a)表示的各个部位的名称,(b)树状图的形式表示的各个部位的关系。

同样的,由于需要考虑跳水时间与人体重的关系,所以关于人各个部分的重量分布情况应该有所了解。一个身高 1.82m,体重为 75.639kg 的男性模特各部分重量分布如表 \ref{tab:body_weights} 所示。

\begin{table}[h]
    \centering
    \caption{身高 1.82m,体重为 75.639kg 的男性模特的人体分割详细值}
    \label{tab:body_weights}
    \begin{tabular}{|c|c|c|c|}
        \hline
         & 分区 & 重量(kg) & 几何距离(cm) \\
        \hline
        \multirow{8}{*}{$B_{b}$} & 头 & 5.575 & 球面: $r=11$ \\
        & 躯干 & 32.400 & 长方体: $18 \times 30 \times 60$ \\
        & 左大腿 & 8.650 & 圆柱体: $r=8, h=43$ \\
        & 右大腿 & 8.650 & 圆柱体: $r=8, h=43$ \\
        & 左小腿 & 4.086 & 圆柱体: $r=5.5, h=43$ \\
        & 右小腿 & 4.086 & 圆柱体: $r=5.5, h=43$ \\
        & 左脚 & 1.436 & 球面: $r=7$ \\
        & 右脚 & 1.436 & 球面: $r=7$ \\
        \hline
        \multirow{3}{*}{$B_{l}$} & 左上臂 & 2.356 & 圆柱体: $r=5, h=30$ \\
        & 左前臂 & 1.781 & 圆柱体: $r=4.5, h=28$ \\
        & 左手 & 0.523 & 球面: $r=5$ \\
        \hline
        \multirow{3}{*}{$B_{l}$} & 右上臂 & 2.356 & 圆柱体: $r=5, h=30$ \\
        & 右前臂 & 1.781 & 圆柱体: $r=4.5, h=28$ \\
        & 右手 & 0.523 & 球面: $r=5$ \\
        \hline
    \end{tabular}
\end{table}

本文将人体的模型分为 14 个部分(详见 5.2.2 中)。考虑到完成每个跳水都做所需要的角动量与该套动作的翻腾圈数和转体圈数有关,所以为了分析一个人身

高和体重对跳水时间的影响,本文选取了一套动作进行分析。根据假设,由于起跳时并没有转体方向的角速度,所以起跳的角速度决定了其角动量。也就是说,转动惯量大小会影响到转动的角速度。

人在起跳之后,由于只受作用于质心的重力作用,所以对于通过质心的任一轴线而言,重力的力矩为0。再加上不受空气阻力的影响,角动量守恒。在分析起跳时候的角动量,将运动员看作是一个刚体,此时运动员的翻腾可以看成是以角速度为 \(\omega\) 绕定轴转动的,其上的任意一点都在各自的转动平面上做圆周运动。取其上的任意一点 \(P\),其质量为 \(\Delta m_i\),速度为 \(v_i\) 与转轴的垂直距离为 \(r_i\),则质点对 \(z\) 轴的角动量为

\[
L_i = \Delta m_i v_i r_i = \Delta m_i \omega r^2
\tag{44}
\]

由于刚体上任一质点对 \(z\) 轴的角动量都具有相同的方向,因此整个刚体对 \(z\) 轴的角动量应为所有质点对 \(z\) 轴角动量之和,即

\[
L = \sum L_i = \sum \Delta m_i r_i^2
\tag{45}
\]

又因为

\[
J = \sum \Delta m_i r_i^2 = \int r^2 dm
\tag{46}
\]

所以当人体的四肢伸开或者靠拢的时候,就会改变转动惯量 \(J\) 的大小,从而引起角速度 \(\omega\) 的改变。因此可以得到

\[
\omega = \frac{L}{J} = \frac{L}{\int r^2 dm}
\tag{47}
\]

所以跳水运动员在上升时,四肢是展开的,处于远离质心的位置,回转半径 \(r\) 增大,进而转动惯量 \(J\) 也跟着增大,此时的角速度 \(\omega\) 较小,翻腾不明显。在达到最高点时,运动员开始收拢四肢,回转半径减小,转动惯量减小,进而角速度增大。当完成翻腾的时候,再次展开四肢到远离质心的位置,以增大回转半径而增大转动惯量,使得角速度减小到停止翻腾。

由于跳水运动员的角动量与翻腾时的角速度有很大的关系,所以假设在没有转体的时候,分别考虑运动员的屈体和抱膝时的转动惯量随着运动员身高和体重的变化。

首先假设运动员的身高不变,增大运动员的体重变为原来的 \(k_1\) 倍。由于身高不变,所以运动员在屈体或是抱膝的时候,质心的位置不发生改变。根据 (46) 可以得到增加了 \(k_1\) 倍的质量后,转动惯量也随之增大了 \(k_1\) 倍。因此,此时的角速度也会变为原来的 \(1/k_1\) 倍。导致完成动作的时间增加了 \(k_1\) 倍。为了更好的跳水时间的变化与体重的变化的关系。本文假设运动员完成的动作为 5133B(或是 5133C),假设此时的转动惯量为 \(I = 100\),即正常的完成这套动作的时间为 1.88s。由于体重增加了 \(k_1\) 倍,所以在翻腾的时候,由于对应的角速度变为原来的 \(1/k_1\) 倍,

包含翻腾部分所需要将会增加 $k_{1}$ 倍,则最终的时间为 $1.489k_{1}+0.397$。所以动作的完成时间与体重增加的关系如图所示。

\begin{figure}[h]
    \centering
    \includegraphics[width=0.8\textwidth]{image1.png}
    \caption{完成动作的时间与体重变化的关系图}
    \label{fig:13}
\end{figure}

从图\ref{fig:13}中可以看出,随着体重变化为原来的倍数的增加,完成动作所需要的时间也会随之增加。所以为了能够尽可能的完成较高难度的跳水动作,就需要选择体重较小或者适中的跳水运动员。

在分析完了体重的变化对跳水时间的影响之后,本文还分析了身高的变化对跳水时间的影响。为了简化分析,本文只考虑5133C跳水时间受身高的影响。假设一名跳水运动员的体重不变,升高变为原来的 $k_{2}$ 倍。假设在抱膝的时候,人体可以看作是一个质量分布均匀的圆柱体。所以对于原身高为 $H$ 的运动员,其对应的回转半径为 $H/(2\pi)$。若身高增加为原来的 $k_{2}$ 倍,对应的回转半径也变为原来的 $k_{2}$ 倍。由于角动量保持不变,根据(43)可以计算得到角速度变为原来的 $1/k_{2}^{2}$ 倍。所有翻腾的时间会变为原来的 $k_{2}^{2}$ 倍,及最终的时间为 $1.489k_{2}^{2}+0.397$。动作的完成时间与身高的增加的关系如图\ref{fig:14}所示。

\begin{figure}[h]
    \centering
    \includegraphics[width=0.8\textwidth]{image2.png}
    \caption{动作完成时间与身高变化倍数之间的关系}
    \label{fig:14}
\end{figure}

从图13可以看出对于跳水运动员的合理身高应选取身高较小或者适中的,才能够较好的完成较高难度的动作。但是考虑到动作的美观程度\textsuperscript{[15]},身高较小的跳水运动员不适合参加跳水比较,即使他能够完成难度较高的动作。因此身高要选取适中的。

从以上的分析可以得到,跳水运动员完成跳水动作所需要的时间与其体型有较大的关系。所以为了能够完成难度较高的动作,并且有较高的完成质量来看,选取跳水运动员应该选取身高中等偏低的,体重适中的,身体上下臂协调\textsuperscript{[16]}。总的来说,运动员的体重如果变为 $k_1$ 倍,时间的变化为一次的,最终的时间为 $1.489k_1+0.397$。同样的,如果运动员身高变为 $k_2$ 倍,时间的变化为二次的,最终的时间为 $1.489k_2^2+0.397$。

\subsection{5.3 问题三求解}

跳水运动中,运动员的最终得分为完成分和动作难度系数的乘积,难度系数的设置至关重要。不同的运动员,在完成相同跳水动作的情况下,其体型参数对动作完成的质量具有一定的影响。有人说,瘦小体型的运动员在做翻腾及转体动作时有体型优势。但是,现今实行的难度系数规则并未将体型因素考虑进去,该说法合理性需要得到科学的验证。从问题二中可以看出,体重和身高的变化,会带来一套动作完成时间的变化,但是考虑到一套动作的难度系数由动作的各个组成部分确定得到,与实际完成这套动作的运动员而言,不能客观的反应出改运动员完成这套动作的难度。所以本文认为增加一个体型修正参数很有必要。

\begin{figure}[h]
    \centering
    \includegraphics[width=\textwidth]{image.png}
    \caption{四个跳水动作的质心变化。其中 A,B,C,D 分别表示向前翻腾三周半(107B),向内翻腾三周半抱膝(407C),向后翻腾一周半转体两周半(5136B)和向后翻腾一周半转体两周半(5235B)。}
    \label{fig:15}
\end{figure}

从问题二中可以得到,对于体型不同的两个人完成同一套跳水动作的时间不同,并且体型瘦小的人完成的时间较少。一个完整的跳水过程应该是运动员从起跳开始到动作完成刚好垂直入水,并且完美的压住水花。所以为了方便分析不同体型的人完成同一套跳水动作的难易程度,本文假设运动员在完成跳水动作之后,人刚好入水。由于运动员在起跳之后不受到外力的作用,只受到重力的影响。

从 \cite{ref7} 中可以得知,作者研究了四个动作的质心变化过程。四个动作分别为向前翻腾三周半(107B),向内翻腾三周半抱膝(407C),向后翻腾一周半转体两周半(5136B)和向后翻腾一周半转体两周半(5235B)。实验的结果均为视频采集分析得到。四种动作的质心变化如图 14 所示。

图 15 中,$Z$ 方向表示为质心垂直高度变化,$Y$ 方向表示运动员的左右方向,$X$ 方向表示为向前的方向。不难看出在 $Z$ 方向上的运动员呈现出的运动状态属于抛物线运动。

所以为了更加方便地引入体型修正参数,本文假设运动员在达到最高点之后才开始进行跳水动作,由于下落的时候质心呈抛物线运动,因此不同体型的人完成同一套动作所需要的时间的不同,间接决定了运动员需要起跳后能够达到的最高高度的不同。不同体型的人完成同一套动作时所跳的高度如图 16 所示。

\begin{figure}[h]
    \centering
    \includegraphics[width=0.6\textwidth]{image.png} % 替换为实际图片路径
    \caption{不同体型的人完成同一套动作时所跳的高度}
    \label{fig:16}
\end{figure}

由于运动员完成动作之后就会入水,所以体型较小的人只需要跳到 $H_l$ 的高度就能完成。相反的如果是体型较大的运动员,则需要跳到 $H_h$。

假设在运动员跳水的过程中没有能量的损耗,所以运动员起跳时候的动能大小决定了能够达到最高的高度,即

\begin{equation}
\frac{1}{2} m (v_0 \sin \theta)^2 = mgH
\tag{48}
\end{equation}

通过计算得到

\begin{equation}
H = \frac{v_0^2 \sin^2 \theta}{2g}
\tag{49}
\end{equation}

人体起跳的高度 $H$ 取决于起跳时的初速度 $v_0$ 和腾起角 $\theta$,为形成对人体质心的翻腾的力矩效应,起跳时必须使得腾起角偏离 90 度,也就是说人体的质心

偏离支撑点的垂线以此来产生翻腾力矩。为了更好地分析体型地大小与跳水难度的关系,以及体型修正系数的对难度系数的评估的影响,本文选取不同体型的人完成同一套动作具体进行分析。假设运动员完成的还是问题二中的跳水动作 5133C。假设此时的转动惯量为 \( l=100 \),即正常的完成这套动作的时间为 \( 1.88 \, \text{s} \)。则如果一个人的体重变为原来的 \( k_1 \) 倍,身高不变,完成这套动作所需要的时间为 \( T=1.489k_1+0.397 \)。也就是说,能够达到的最大高度为

\begin{equation}
H+H_t=\frac{g(T-T_t)^2}{2}=\frac{g(1.489k_1+0.397-T_t)^2}{2}
\tag{50}
\end{equation}

所以起跳的时候需要的垂直速度为

\begin{equation}
v_0\cos\theta=\frac{gT^2-2H}{2T}=7.445k_1+1.985-\frac{1}{0.1489k_1+0.0397}
\tag{51}
\end{equation}

从图 15 中可以看出,体重越小的人对起跳速度的要求越小,可以得出完成同一套动作,体重越重的人完成这一套动作需要的垂直初速度越大,也就是难度越大。

如果一个人的身高变为原来的 \( k_2 \) 倍,体重不变,完成这套动作所需要的时间为 \( T=1.489k_2^2+0.397 \)。也就是说,能够达到的最大高度为

\begin{equation}
H+H_t=\frac{g(T-T_t)^2}{2}
\end{equation}

计算可以得到

\begin{equation}
v_0\cos\theta=\frac{gT^2-2H}{2T}=7.445k_2^2+1.985-\frac{1}{0.1489k_2^2+0.0397}
\tag{52}
\end{equation}

通过计算可以分别得到垂直初速度与质量的倍数和身高的倍数之间的关系,如图 17 所示。

\begin{figure}[h]
    \centering
    \begin{subfigure}[t]{0.45\textwidth}
        \centering
        \includegraphics[width=\textwidth]{image_a.png}
        \caption{}
    \end{subfigure}
    \hfill
    \begin{subfigure}[t]{0.45\textwidth}
        \centering
        \includegraphics[width=\textwidth]{image_b.png}
        \caption{}
    \end{subfigure}
    \caption{垂直初始加速度与质量和身高的倍数之间的关系}
    \label{fig:17}
\end{figure}

通过图 17 可知,垂直初始角速度与身高和体重的增大倍数之间存在一定的线性关系,而初始垂直角速度与落水所做动作的难易程度之间存在很大关系,这也从另一方面说明了运动员的身高和体重与难度系数之间存在一定的联系。因为垂直初速度与体重和身高之间是线性相关的,所以本文可以粗略地得出难度系数与身高和体重之间存在一定。

通过以上分析本文可以得出,若是在身高一定的情况下,体型比较瘦的运动员完成一些跳水动作会比较容易,而体型比较胖的运动员完成一些跳水动作会比较困难。另一方面,当运动员的胖瘦程度一致时,身材矮小的运动员完成一些跳水动作会比较容易,而身材比较高大的运动员完成一些跳水动作会比较困难。鉴于以上阐述,本文将引入校正系数。引入的策略为:在完成相同的跳水动作情况下,同等身高条件下,将身材较胖的运动员的难度系数适当增加;同样,若是两个运动员的身材胖瘦程度一致,本文将身高较高的运动员的难度系数做出相应增加。本文用 \( DD_{o} \) 表示原来的难度系数,用 \( DD_{n} \) 表示加入校正系数的难度系数,用 \( Cor \) 表示校正系数。则可以得到:
\begin{equation}
DD_{n} = DD_{o} + Cor
\tag{53}
\end{equation}

由上面可知难度系数与身高以及体重之间存在联系,而原先的难度系数设定并未将身高和体重因素考虑进去,那么,修改后的难度系数则是由校正系数 \( Cor \) 体现出来的。本文在研究校正系数与体重和身高关系时,由于校正参数是直接增加的,本文应该对该增加量划定一个范围,根据比赛规则的合理性和赛事的公平性,结合 \({}^{[17]}\),本文在研究校正系数与体重和身高关系时,由于校正参数是直接增加的,本文应该对该增加量划定一个范围,根据比赛规则的合理性和赛事的公平性,结合 \({}^{[17]}\),本文给校正系数设定一个范围,本文将此范围控制在 \([-0.1, 0.1]\) 范围之内。本文假设跳水运动员身高范围在 \([140, 180]\) 之间,体重范围在 \([40, 80]\) 范围。

\begin{figure}[h]
\centering
\includegraphics[width=0.8\textwidth]{image.png}
\caption{校正后难度系数}
\end{figure}

因此,本文构造校正系数 \( Cor = k \times \sqrt{mh} + b \),通过设定的校正系数范围本文得出 \( k = 0.0043 \), \( b = 0.4182 \)。于是有 \( Cor = 0.0043 \times \sqrt{mh} + 0.4182 \),得到的校正后的难度系数如图 18 所示。校正后难度系数与体重的关系如图 19 所示。

\begin{figure}[h]
    \centering
    \includegraphics[width=\textwidth]{image1.png}
    \caption{校正后难度系数与体重的关系}
    \label{fig:19}
\end{figure}

由图所示,身高相同的情况下,校正后的难度系数随体重增加而增加

\begin{figure}[h]
    \centering
    \includegraphics[width=\textwidth]{image2.png}
    \caption{校正后难度系数与身高的关系}
    \label{fig:20}
\end{figure}

由图 20 所示,体重相同的情况下,校正后的难度系数随身高增加而增加

\subsection{5.4 问题四的求解}

本题要求本文利用问题三建立的模型求解 10 米跳台跳水动作的难度系数。并与官方的难度系数表进行对比,判断是否有区别并解释区别存在的原因。

此题要求本文利用问题三的结论,将体型校正系数的模型运用到具体的跳水动作中去。

由于问题三所得出的难度系数与体重与身高线性相关。在解决该问题前,本文假设运动员在体型有差异的情况下,运动员完成各个跳水动作的能力水平是一致的。

本文假设官方给定的难度系数表是体重为 \(60\,\text{kg}\)、身高为 \(170\,\text{cm}\) 的运动员做出标准身高体重体型,该种体型运动员做出的相应跳水动作所产生的难度系数与官方的难度系数一致。本文分别选取体重为 \(65\,\text{kg}\)、身高为 \(170\,\text{cm}\) 的运动员甲和体重为 \(60\,\text{kg}\),身高为 \(165\,\text{cm}\) 运动员乙作为研究对象,由问题三可知:

\begin{equation}
DD_{n} = DD_{o} + Cor
\tag{54}
\end{equation}

\begin{equation}
Cor = 0.0043 \times \sqrt{mh} + 0.4182
\tag{55}
\end{equation}

首先将本文假设的标准体型运动员的身高和体重代入上述模型,从而得出相应的值。然后,本文将这两位运动员的体重和身高数据代入至问题三得出的模型。具体如表11和表12所示。

\textbf{表11 十米跳台难度系数表对比表(运动员甲)}

\begin{table}[h]
\centering
\begin{tabular}{|c|c|c|c|c|c|c|c|c|c|}
\hline
\multirow{4}{*}{动作代码} & \multicolumn{2}{c|}{PIKE} & \multicolumn{2}{c|}{TUCK} & \multirow{4}{*}{动作代码} & \multicolumn{2}{c|}{PIKE} & \multicolumn{2}{c|}{TUCK} \\ \cline{2-5} \cline{7-10}
 & 原 & 新 & 原 & 新 &  & 原 & 新 & 原 & 新 \\ \cline{2-5} \cline{7-10}
 & B & B & C & C &  & B & B & C & C \\ \hline
105 & 2.3 & 2.318 & 2.1 & 2.118 & 5154 & 3.3 & 3.318 & 3.1 & 3.118 \\ \hline
107 & 3.0 & 3.318 & 2.7 & 2.718 & 5156 & 3.8 & 3.818 & 3.6 & 3.618 \\ \hline
109 & 4.1 & 4.118 & 3.7 & 3.718 & 5172 & 3.6 & 3.618 & 3.3 & 3.318 \\ \hline
1011 & -- & -- & 4.7 & 4.718 & 5255 & 3.6 & 3.618 & 3.4 & 3.418 \\ \hline
205 & 2.9 & 2.918 & 2.7 & 2.718 & 5257 & 4.1 & 4.1181 & 3.9 & 3.918 \\ \hline
207 & 3.6 & 3.618 & 3.3 & 3.318 & 5271 & 3.2 & 3.218 & 2.9 & 2.918 \\ \hline
209 & 4.5 & 4.518 & 4.2 & 4.218 & 5273 & 3.8 & 3.818 & 3.5 & 3.518 \\ \hline
305 & 3.0 & 3.018 & 2.8 & 2.818 & 5275 & 4.2 & 4.218 & 3.9 & 3.918 \\ \hline
307 & 3.7 & 3.718 & 3.4 & 3.418 & 5353 & 3.3 & 3.318 & 3.1 & 3.118 \\ \hline
309 & 4.8 & 4.818 & 4.5 & 4.518 & 5355 & 3.7 & 3.718 & 3.5 & 3.518 \\ \hline
405 & 2.8 & 2.818 & 2.5 & 2.518 & 5371 & 3.3 & 3.318 & 3.0 & 3.018 \\ \hline
407 & 3.5 & 3.518 & 3.2 & 3.218 & 5373 & -- & -- & 3.6 & 3.618 \\ \hline
409 & 4.4 & 4.418 & 4.1 & 4.118 & 5375 & -- & -- & 4.0 & 4.018 \\ \hline
\end{tabular}
\end{table}

\begin{table}
\centering
\caption{十米跳台难度系数表对比表(运动员乙)}
\begin{tabular}{|c|c|c|c|c|c|c|c|c|c|}
\hline
\multirow{3}{*}{动作代码} & \multicolumn{2}{c|}{PIKE} & \multicolumn{2}{c|}{TUCK} & \multirow{3}{*}{动作代码} & \multicolumn{2}{c|}{PIKE} & \multicolumn{2}{c|}{TUCK} \\
\cline{2-5} \cline{7-10}
 & 原 & 新 & 原 & 新 & & 原 & 新 & 原 & 新 \\
\cline{2-5} \cline{7-10}
 & B & B & C & C & & B & B & C & C \\
\hline
105 & 2.3 & 2.294 & 2.1 & 2.094 & 5154 & 3.3 & 3.294 & 3.1 & 3.094 \\
\hline
107 & 3.0 & 2.994 & 2.7 & 2.694 & 5156 & 3.8 & 3.794 & 3.6 & 3.594 \\
\hline
109 & 4.1 & 4.004 & 3.7 & 3.694 & 5172 & 3.6 & 3.594 & 3.3 & 3.294 \\
\hline
1011 & -- & -- & 4.7 & 4.694 & 5255 & 3.6 & 3.594 & 3.4 & 3.394 \\
\hline
205 & 2.9 & 2.894 & 2.7 & 2.694 & 5257 & 4.1 & 4.094 & 3.9 & 3.894 \\
\hline
207 & 3.6 & 3.594 & 3.3 & 3.294 & 5271 & 3.2 & 3.194 & 2.9 & 2.894 \\
\hline
209 & 4.5 & 4.494 & 4.2 & 4.194 & 5273 & 3.8 & 3.794 & 3.5 & 3.494 \\
\hline
305 & 3.0 & 2.994 & 2.8 & 2.794 & 5275 & 4.2 & 4.194 & 3.9 & 3.894 \\
\hline
307 & 3.7 & 3.694 & 3.4 & 3.394 & 5353 & 3.3 & 3.294 & 3.1 & 3.094 \\
\hline
309 & 4.8 & 4.794 & 4.5 & 4.494 & 5355 & 3.7 & 3.694 & 3.5 & 3.494 \\
\hline
405 & 2.8 & 2.794 & 2.5 & 2.494 & 5371 & 3.3 & 3.294 & 3.0 & 2.994 \\
\hline
407 & 3.5 & 3.494 & 3.2 & 3.194 & 5373 & -- & -- & 3.6 & 3.594 \\
\hline
409 & 4.4 & 4.394 & 4.1 & 4.094 & 5375 & -- & -- & 4.0 & 3.994 \\
\hline
\end{tabular}
\end{table}

根据上表可以看出,在体重或者身高因素不同的情况下,运动员完成跳水动作的难度系数是不同的,这是因为体重或者身高不同导致了难度系数产生变化,主要变化是在身高条件一致的情况下,与标准体重相比,体重较重的运动员产生的难度系数有了一定的增加。而在体重一定的情况下,与标准身高相比,身高较小的运动员难度系数有了一定的较小,但是前面两种情况产生的难度系数变化都比较小。表格中的结果也在一定程度上支持了问题三的结论。

\section{六、模型的评价}

对于问题二构建的模型,模型在求解的时候将人体的转动过程分为了九个阶段,在计算时间的时候,偶数阶段的手臂动作实际并不是瞬间完成的。在得到了时间和 \(m\) 和 \(n\) 的关系后,时间还跟角动量有关系。本文在距离计算的时候只是针对 5133C 这个跳水动作,是否试用其他的跳水动作还有待验证。

对于问题三构建的模型中,本文给出了难度校正系数与体型(身高、体重)的大致关系。考虑到体型对难度系数的影响不应超过官方规则中各个子动作的最小难度系数增量,因此本文将校正系数所带来的难度系数变化限制在 $|0.1|$ 之间。同时合理分布专业跳水运动员的身高和体重的范围,本文分别划分在 $[40,80]kg$,$[140,180]cm$ 范围之间。分析基于控制变量法的难度系数与身高体重的关系,本文给出了其在身高和体重有约束范围下的关系表达式,从图中可以看出,校正系数很好的反映出了与身高体重的关系。

\section{参考文献}

[1] 张涛. 跳水动作难度系数是怎样确定的[J]. 游泳, 1995(2):22-23.

[2] 游佩林. 刚体角动量定理的应用中一个容易被疏忽的问题[J]. 物理与工程, 1995(2):14-15.

[3] Yeadon M R. The biomechanics of twisting somersaults. Part I: Rigid body motions[J]. Journal of Sports Sciences, 1993, 11(3):187.

[4] 李连在,成尔恒. 对跳水运动员的形态学与平衡能力的初步研究[J]. 中国体育科技, 1984(15):65-68.

[5] 王大愚. 朱建华起跳和过杆身体转动惯量、角速度和角动量计算及技术分析[J].

[6] 石磊,陈亮. 跳远前旋角动量研究[J]. 南京师大学报(自然科学版), 2014, 37(2):140-144.

[7] 郝卫亚,王智,艾康伟. 运动员空中翻腾和转体姿态控制过程中转动惯量的变化[J]. 中国运动医学杂志, 2013, 32(11):966-973.

[8] 孙亦光,吴延禧. 我国跳台跳水优秀选手起跳技术的生物力学研究[J]. 体育科学, 1997(4):71-78.

[9] Dullin H R, Tong W. Twisting Somersault[J]. Physics, 2016.

[10] Montgomery R. How much does the rigid body rotate? A Berry’s phase from the 18th century[J]. American Journal of Physics, 1998, 59(5):394-398.

[11] Cabrera A. A generalized Montgomery phase formula for rotating self-deforming bodies[J]. Journal of Geometry \& Physics, 2007, 57(5):1405-1420.

[12] Tong W, Dullin H R. A New Twisting Somersault: 513XD[J]. Journal of Nonlinear Science, 2017, 27(6):2037-2061.

[13] 孙春峰,邓翠芳. 角动量守恒定律在体育运动中的应用研究[J]. 湖北工程学院学报, 2004, 24(3):70-73.

[14] R. K. Jensen, Model for body segment parameters, In Biomechanics V-B, ed. P.V. Komi, University

[15] 王荣辉,任弘,胡晓丽. 我国跳水运动员形态特征研究[J]. 中国体育科技, 2008, 44(4):91-94.

\begin{enumerate}
\setcounter{enumi}{15}
    \item 王荣辉. 我国跳水优秀运动员竞技能力结构和选材指标体系研究[C]// 中国体育博士高层论坛. 2008.
    \item Cabrera A. A generalized Montgomery phase formula for rotating self-deforming bodies[J]. Journal of Geometry \& Physics, 2007, 57(5):1405-1420.
\end{enumerate}