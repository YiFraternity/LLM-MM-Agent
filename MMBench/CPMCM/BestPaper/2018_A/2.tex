\begin{center}
\textbf{“华为杯”第十五届中国研究生数学建模竞赛}
\end{center}

\textbf{题目} 关于跳台跳水体型系数设置的建模分析

\begin{abstract}    

为分析不同体型对完成跳水动作的影响,更合理的设计难度系数,充分反应各动作的真实难度,综合考虑了运动员跳水动作中的各阶段动作,对运动员跳水的过程进行了建模。

针对问题 1,基于国际泳联 $10 \, \mathrm{m}$ 跳台跳水难度系数确定规则总结结论。对国际泳联跳水难度系数制定准则表进行定性分析,可得出部分结论如下:

(1) 多数情况下,难度系数的大小与翻腾周数成正比,也与转体周数成正比;

(2) 在不同的动作组别(即向前、向后、反身等不同起跳方式)中,一般而言,反身的难度系数最大;

(3) 对不同的动作姿势,直体姿态的难度系数最大,其次是屈体姿态,再次是抱膝姿态。

针对问题 2,将跳水过程分解为:初始状态-空中动作 1、空中动作 1-空中动作 2、空中动作 2-入水动作、入水,最多 6 个部分。其中空中动作 1 和空中动作 2 看作刚体的旋转,而伴随人体姿态变化的过程(初始状态-空中动作 1、空中动作 1-空中动作 2、空中动作 2-入水动作)看作非刚体运动。刚体运动结合运动员身体姿态和人体实测数据,依据刚体旋转的物理模型,对运动员空中旋转动作进行建模。非刚体运动中将人体视为绕支点旋转的两段刚体,计算姿态变化时间,同时计算此过程中伴随转过的角度。之后对整个过程中还剩余的旋转角度按照所选动作,在空中动作 1 和空中动作 2 中完成。最终得出跳水整套动作所需时间的计算公式,详见本文 5.5 节。

针对问题 3,分析了难度系数的影响因素,分析了跳水动作必要完成时间、随机因素带来的发挥、角动量获取、起跳高度和陀螺效应等因素。建立了难度系数和必要完成时间的关系。经过分析论证,采用分量级且进行参数校正的方法,将各量级内校正系数设置为:$\delta = \frac{T(m, l)}{T(m_c, l_c)}$,并对其他因素做了定性分析。

针对问题 4,依据之前提出的分量级加校正的方法,以身高 $1.8 \, \mathrm{m}$ 体重 $70 \, \mathrm{kg}$ 的运动员为例,对新的难度系数进行了计算,并对结果进行了分析。结果参见表 7.1、表 7.2。

关键词:跳台跳水,耦合刚体,转动惯量,矫正系数,分量级修正
\end{abstract}

\tableofcontents

\section{一、问题重述}

\subsection{1.1 问题背景}

国际泳联在跳水竞赛规则中规定了不同跳水动作的代码及其难度系数,它们与跳水运动员的起跳方式(起跳时运动员正面朝向、翻腾方向)及空中动作(翻腾及转体圈数、身体姿势)有关。裁判员们评分时,根据运动员完成动作的表现优劣及入水效果,各自给出从 10 到 0 的动作评分,然后按一定公式计算该运动员该动作的完成分,完成分乘以该动作的难度系数即为该运动员该动作的最终得分。出于公平性考虑,一个跳水动作的难度系数应充分反映该动作的真实难度,且有人说,瘦小体型的运动员在做翻腾及转体动作时有体型优势,因此,应当设置体型系数予以校正。

\subsection{1.2 问题提出}

第一,通过分析国际泳联 10m 跳台跳水难度系数的确定规则,得出结论。

第二,应用物理学方法,建立模型描述运动员完成各个跳水动作的时间与运动员体型(身高,体重)之间的关系。

第三,请以上根据模型,说明在 10m 跳台跳水比赛中设置体型校正系数有无必要。

第四,尝试基于以上的建立模型,给出部分 10m 跳台跳水动作的难度系数,并对其区别,并作出解释。

\section{二、模型假设}

(1)假设人体为多部分的耦合刚体,不考虑形变产生。

(2)假设跳水过程中忽略空气阻力及环境风速的影响。

(3)假设运动员为专业运动员,心理和技术水平达到专业运动员标准。

(4)假设跳水过程中忽略教练员和其他技战术等其他因素对运动员的影响。

\section{三、符号说明}

\begin{tabular}{|c|c|}
\hline 符号 & 说明 \\
\hline $h$ & 角动量 \\
\hline $I$ & 转动惯量 \\
\hline $A$ & 绕 Y 轴转动惯量 \\
\hline
\end{tabular}

\begin{tabular}{c l}
\hline
$B$ & 绕X轴转动惯量 \\
\hline
$C$ & 绕Z轴转动惯量 \\
\hline
$\omega$ & 角速度 \\
\hline
$\varphi$ & 绕Y轴欧拉角 \\
\hline
$\theta$ & 绕X轴欧拉角 \\
\hline
$\psi$ & 绕Z轴欧拉角 \\
\hline
$\beta$ & 剩余需转动角度 \\
\hline
$\delta$ & 校正系数 \\
\hline
$P$ & 腰腹力量系数 \\
\hline
$Q$ & 结构张力系数 \\
\hline
$t$ & 时间 \\
\hline
$m$ & 质量 \\
\hline
$l$ & 身高 \\
\hline
$d$ & 偏心距 \\
\hline
$R$ & 姿态变化矩阵 \\
\hline
$M$ & 各部位质量 \\
\hline
$x$ & X坐标 \\
\hline
$y$ & Y坐标 \\
\hline
$z$ & Z坐标 \\
\hline
$r$ & 距离积分变量 \\
\hline
$a$ & 长方形长 \\
\hline
$b$ & 长方形宽 \\
\hline
$N$ & 转体圈数 \\
\hline
$M$ & 翻腾圈数 \\
\hline
$L$ & 力径 \\
\hline
$F$ & 力 \\
\hline
\end{tabular}

\section{四、问题一的求解}

跳水完成动作过程包括助跑、起跳、空中技巧和入水四个阶段,难度系数的给定仅针对空中技巧进行考虑。一般而言,难度系数根据动作组别、竞赛项目(跳板、跳台)、高度、动作姿势和翻腾转体的周数等差异来确定其数值。本文的分析仅考虑了 $10 \mathrm{~m}$ 跳台跳水项目,因此,在该情况下难度系数由动作组别、动作姿势和翻腾转体周数来决定。

根据国际泳联给出的各种跳水技巧的难度系数标准可知,跳台跳水的难度系数 $\mathrm{DD}$ (Degree of Difficulty)由五个方面的难度系数累加得到,即 $\mathrm{DD}=A+B+C+D+E$。其中,$A$ 为翻腾周数项(Somersaults),$B$ 为动作姿势项(Flight Position),$C$ 为转体项(Twists),$D$ 为动作组别项,$E$ 为非自然入水项(Unnatural Entry)。

动作姿势根据运动员在空中的身体姿态可分为直体(Strait)、屈体(Pike)、抱膝(Tuck)、翻腾兼转体(Free)四种。

动作组别根据运动员进行跳水动作起跳前站立的方向和起跳后身体运动的方向,可分为以下六个动作组别:面对池向前跳水(1)、面对台向后跳水(2)、面对池反身跳水(3)、面对板向内跳水(4)、转体跳水(5)和臂立跳水(6),括号内表示各组别对应的代表数字,作为跳水动作代号的首位数字。

在转体项中,需要注意的是,为了保证运动员能够以头先入水的方式垂直进入水中,某些特定的翻腾转体周数只能用部分动作姿势来完成。

非自然入水项是考虑了非正常入水姿态的难度系数项,非正常姿态是指以脚先入水的方式落入水中,在入水前,运动员看不到水面。

首先,根据附件 1 中的 APPENDIX 3,本文对组成总难度系数 $\mathrm{DD}$ 的五个难度系数项分别进行了分析,确定规则如下:

A 项难度系数中,总体而言,随着翻腾周数的增加,难度系数增大,最大能完成的翻腾周数为 5 周半,且大部分从整数周到半数周(如 1 周到 1 周半)的变化要小于从半数周到下一个整数周(如 1 周半到 2 周)的变化。

B 项难度系数中,当选择直体姿态时无法完成翻腾周数 $\geq 3$ 的动作,总体而言,屈体姿态的难度系数大于抱膝姿态。

C 项难度系数中,均采取了转体姿态,翻腾周数一般是一个半周数(如一周半、二周半),而转体周数在向前/向内情况下为整周数,在向后/反身情况下为半周数。从整体趋势上看,转体周数越大,难度系数是增大的,然而对于相同的转体周数,翻腾周数并不是越多难度系数越大,对向后和反身两个组别来说,翻

腾周数变大反而出现了难度系数减小的现象。

D 项难度系数中,对不同的动作组别,相同的翻腾周数,其难度系数也有所差别,总体而言,反身的难度系数较大。

E 项难度系数中,主要考虑运动员非自然入水时无转体情况下的起跳方向及翻腾周数,总体而言,翻腾周数越多,难度系数越大。且对于向前和向内组别而言,只有翻腾周数为整数周时该项有效;对向后和反身组别而言,只有翻腾周数为半整数周时该项有效。

然后,根据附件 1 中的 APPENDIX 4,本文从动作姿势、动作组别、翻腾转体圈数几个维度来分析得到国际泳联 $10 \mathrm{~m}$ 跳台跳水难度系数的确定规则如下:

(1) 仅对比动作姿势

对所有的动作组别而言,当某运动员选择相同的组别,相同的翻腾周数时,无论他的跳水技巧中有无转体动作,其难度系数 $\mathrm{DD}_{\text {直体 }}> \mathrm{DD}_{\text {屈体 }}> \mathrm{DD}_{\text {抱膝}}$,且整体来看采用屈体比抱膝难 $0.1-0.4$。

(2) 仅对比翻腾转体圈数

对第 1-4 组动作组别而言,采取相同的组别,相同的动作姿势(如直体、屈体等),翻腾周数越多,难度系数越大,但除了动作姿势为 2、3 时翻腾周数从 0 到 1 圈的变化,以及动作姿势为 6 且向前跳水时翻腾周数从 $1 / 2$ 到 1 圈的变化。

对第 5 组动作组别而言,相同的翻腾转体方向(第 1-4 组),相同的翻腾周数,相同的动作姿势,转体周数越多,难度系数越大;相同的翻腾转体方向(第 1-4 组),相同的转体周数,相同的动作姿势,翻腾周数越多,难度系数越大,且采取屈体姿态比抱膝姿态难度增加幅度更大。

(3) 仅对比动作组别

对第 1-4 组动作组别而言,总体来看,相同的翻腾圈数,相同的动作姿势,方向决定难度,$\mathrm{DD}_{\text {反身 }} \geq \mathrm{DD}_{\text {向后 }} \geq \mathrm{DD}_{\text {向内 }} \geq \mathrm{DD}_{\text {向前 }}$,但其中也有例外。

综上,对 APPENDIX 3-4 的规律总结可知:

a. 在多数情况下,难度系数正比于翻腾周数,也正比于转体周数;

b. 在不同的动作组别中,反身的难度系数普遍最大;

c. 直体姿态的难度系数最大,其次是屈体姿态,再次是抱膝姿态。

\section{五、问题二模型的建立与求解}

\subsection{5.1 问题分析}

\subsubsection{5.1.1 跳水过程分析}

在跳水过程中,运动员不受外力矩的作用,人体的运动状态满足角动量守恒。而跳水与体操以及水上舞蹈、花样滑冰等项目相似,主要是以空中完成的动作以及美感对其进行评分,所以在此问题分析中首先假设各运动员跳水过程中获取的角动量一定,运动员通过控制身体姿态调整质心,从而更好地完成复杂的空中动作。而在跳水比赛中,一次跳水可以抽象成以下几个状态:

\begin{figure}[h]
\centering
\includegraphics[width=0.8\textwidth]{image.png}
\caption{跳水过程}
\end{figure}

对以上过程解释如下:

(1)起始状态包含:forward、back、inward、reverse、armstand back 以及 armstand forward;

(2)空中一般最多可分为部分动作:空中动作 1 和空中动作 2。两种动作均由可完成的高难度旋转动作组成,通常包括翻腾(somersault,绕质心横轴旋转)和转体(twist,绕质心纵轴旋转)两种基本形式,可以选用 1 种动作或 2 种动作的组合,而根据运动员身体姿态还可将翻腾进一步分为直体(strait)、屈体(pike)、抱膝(tuck)、翻转加转体(free)。并且目前存在的技术动作有如下限制:

a. 当进行单一的翻腾动作时,空中动作仅有动作 1,且此时 \(N=0\);

b. 当空中动作包含动作 1 和动作 2 两部分时,一定在两部分中的一个进行了转体,即两个空中动作中最多只能有一个 \(N\) 不等于 0;

b. 转体的过程中往往伴随着翻腾,且转体时,身体姿态只能为直体;

c. 空中动作 1 和空中动作 2 中只能有其中一个包含转体;

d. 当进行单一翻腾动作时,过程中将无法改变身体姿态;

e. 翻腾运动方向又分为顺时针以及逆时针转动,过程中无法改变旋转方向;

(3)过程 1/2/3 为动作变换的过程,即从上一个动作变化为下一个动作,此过程不是独立进行,在变换的过程中,身体同样会伴随翻腾或转体。

*注意:对于存在耦合运动的动作,可以分为两部分:一个为动作中必须完成的旋转角度或身体姿态的变化(①型动作);一个为完成必须完成的动作时耦合完成的动作(②型动作)。

\subsubsection{5.1.2 跳水时间分析}

本问题中,考虑 10 米跳台跳水,由于跳台不存在弹性,假设运动员质心位于身体中心处,跳水前与入水时运动员身体均为伸展状态,并垂直于水平面,所以运动员起始状态下质心高度差与身高无关,均为 10 米。假设运动员在跳台上起跳高度为 0.5 米,则跳水可分为两个部分:起跳上升阶段和自由落体阶段,可以计算两个阶段所用时间为:
\begin{align}
R_x(\phi) &=
\begin{bmatrix}
1 & 0 & 0 \\
0 & \cos\phi & \sin\phi \\
0 & -\sin\phi & \cos\phi
\end{bmatrix} \\
R_y(\theta) &=
\begin{bmatrix}
\cos\theta & 0 & -\sin\theta \\
0 & 1 & 0 \\
\sin\theta & 0 & \cos\theta
\end{bmatrix} \\
R_z(\psi) &=
\begin{bmatrix}
\cos\psi & \sin\psi & 0 \\
-\sin\psi & \cos\psi & 0 \\
0 & 0 & 1
\end{bmatrix}
\tag{5.7}
\end{align}
\begin{align}
\dot{\phi} &= h(A\sin^{2}\psi + B\cos^{2}\psi)/AB \tag{5.13} \\
\dot{\theta} &= -h(A-B)\cos\theta\sin\psi\cos\psi/AB \tag{5.14} \\
\dot{\psi} &= \left(h/C - \dot{\phi}\right)\sin\theta \tag{5.15}
\end{align}
总时间为:$T_{\text{总}} = t_{\text{上}} + t_{\text{下}}$,经计算为 $1.76 \, \text{s}$。

而在基本固定的下落时间内,运动员需要完成要求的动作,即满足:
\begin{equation}
T_{\text{总}} = T_1 + T_{\text{空中}_1} + T_2 + T_{\text{空中}_2} + T_3 + T_{\text{入水}}
\tag{5.3}
\end{equation}
其中根据跳水的过程,$T_2$ 与 $T_{\text{空中}_2}$ 可能不存在。

并且在计算时间时需要注意的是,由于在各阶段完成动作时会耦合其他动作,所以各部分时间无法单独计算。比如,初始状态变换成第一个空中动作的工程中,也许会伴随翻腾,转动一定角度,对空中动作中的翻腾时间的计算产生影响。

此外,进行转体动作的过程中,可以耦合进行翻腾动作,对于此类问题,依据跳水比赛中的动作特点,运动员将动作转换为两个动作(空中动作 1 和空中动作 2),在其中一个动作中进行转体加翻腾的耦合动作,而在另一个动作中补全剩余动作所需旋转圈数(如图 5.2)。

同样,如果运动员在之前的空中动作完成顺利,用时较短,则可以通过调整最后的伸展时间,在适当时间入水。

而入水时间根据经验一般需要 $0.2 \, \text{s}$ 左右。

\begin{figure}[h]
    \centering
    \includegraphics[width=\textwidth]{image.png}
    \caption{空中动作分两部分时的动作过程}
    \label{fig:5.2}
\end{figure}

\subsubsection{求解流程}

由于跳水运动员在跳水过程中不受外力矩,针对此问题我们采用固定角动量的值,分步进行求解,按动作流程逐步分析各个阶段的时间,并针对动作的选择进行分析。而对于空中动作 1 和 2 可以看作刚体的转动,而动作变化过程则需要根据情况单独计算。

为了便于理解表述,本文中首先分析刚体模型,之后分析动作变化时的物理模型,最后根据计算结果,对整个过程的时间进行分析。

\subsection{刚体模型建模及计算}

\subsubsection{问题理解与分析}

此问中需要考虑运动员身高及体重对完成动作时间的影响。当刚体进行转动时,满足如下方程:

\begin{equation}
    \mathbf{h} = \mathbf{I} \boldsymbol{\omega}
    \tag{5.4}
\end{equation}

式中 $\mathbf{h}$ 为角动量;$\mathbf{I}$ 为转动惯量;$\boldsymbol{\omega}$ 为角速度。此时采用这种姿势进行转动的角度 $\beta$ 为上一个过程结束后,在这一动作中还需完成的旋转角度:

\begin{equation}
    t = \beta / \omega
    \tag{5.5}
\end{equation}

所以,当 $\mathbf{h}$ 一定时,角速度的大小取决于转动惯量 $\mathbf{I}$ 的值。

此外,刚体转动满足平行轴定理,即假如刚体绕旋转轴(旋转轴过刚体质心)(如图 5.3)旋转时的转动惯量为 $I_{\mathrm{C}}$,现刚体绕一个平行于之前旋转轴的直线进行旋转,此时的转动惯量 $\mathbf{I}$ 为:

\begin{equation}
    I = I_{\mathrm{C}} + md^2
    \tag{5.6}
\end{equation}

由此可知,转动惯量 $\mathbf{I}$ 的值取决于刚体绕过质心轴旋转时的转动惯量 $I_{\mathrm{C}}$,以

及运动员旋转时实际旋转轴与质心的偏差距离 \( d \),以及运动员的质量 \( m \),如图所示。所以,运动员在空中不受外力矩影响,当作刚体运动时,运动员水平的高低取决于运动员控制旋转轴与质心之间的偏差的能力,所以运动员在完成动作的过程中,保持实际旋转轴与自身质心重合,可以使运动员在同等角动量和同等下落时间内,完成更多更复杂的动作。同时,转动惯量也与运动员姿态和身高体重有关,以下内容将主要分析转动惯量与身高和体重的关系。

而在这一部分中,我们假设屈体和抱膝时运动员可以通过手臂抓紧,保持身体姿态。

\begin{figure}[h]
    \centering
    \includegraphics[width=0.6\textwidth]{image.png}
    \caption{实际转动角度不位于质心时的转动惯量}
    \label{fig:5.3}
\end{figure}

\subsubsection{刚体旋转建模}

描述刚体的姿态可以采用欧拉角,即刚体自身坐标系与地球坐标系三个轴之间的夹角(图 5.4),分别为:\( \phi \)、\( \theta \)、\( \psi \),可用向量 \( (\phi, \theta, \psi)^{\mathrm{T}} \) 表示。同样,刚体旋转时的角速度矩阵可以表示为:\( \boldsymbol{\omega} = (\dot{\phi}, \dot{\theta}, \dot{\psi}) \) 而每一个绕旋转轴转动的运动,可以看作一次矩阵的变换 \cite{ref1}:

\begin{align}
R_x(\phi) &=
\begin{bmatrix}
1 & 0 & 0 \\
0 & \cos\phi & \sin\phi \\
0 & -\sin\phi & \cos\phi
\end{bmatrix} \\
R_y(\theta) &=
\begin{bmatrix}
\cos\theta & 0 & -\sin\theta \\
0 & 1 & 0 \\
\sin\theta & 0 & \cos\theta
\end{bmatrix} \\
R_z(\psi) &=
\begin{bmatrix}
\cos\psi & \sin\psi & 0 \\
-\sin\psi & \cos\psi & 0 \\
0 & 0 & 1
\end{bmatrix}
\tag{5.7}
\end{align}

而一个动作可以看作三个分动作的组合,所以,由一个状态经过一个动作后变换为另一个状态的变化矩阵可以写作:

\begin{equation}
R_{f} = R_{X} \cdot R_{Y} \cdot R_{Z}
\tag{5.8}
\end{equation}

\begin{figure}[h]
    \centering
    \includegraphics[width=0.8\textwidth]{image.png}
    \caption{运动员姿态角}
    \label{fig:athlete_pose}
\end{figure}

同时,绕质心旋转的刚体满足:
\begin{equation}
\mathbf{h} = \mathbf{I} \boldsymbol{\omega}
\tag{5.9}
\end{equation}
其中,$\mathbf{I}$ 为人体各个方向的转动惯量矩阵,由于人体的高度对称性。矩阵 $\mathbf{I}$ 可以看作对角阵。
\begin{equation}
\mathbf{I} =
\begin{pmatrix}
A & 0 & 0 \\
0 & B & 0 \\
0 & 0 & C
\end{pmatrix}
\tag{5.10}
\end{equation}
所以,每一次动作变化可以看作三个动作对应的变换矩阵相乘
\begin{equation}
\mathbf{h}_{f} = R_{X} \cdot R_{Y} \cdot R_{Z} \cdot \mathbf{h}
\tag{5.11}
\end{equation}
因此可以得到欧拉角速度的变换形式:
\begin{equation}
\boldsymbol{\omega}_{f} =
\begin{pmatrix}
\cos\theta\cos\psi & \sin\psi & 0 \\
-\cos\theta\sin\psi & \cos\psi & 0 \\
\sin\theta & 0 & 1
\end{pmatrix}
\cdot
\begin{pmatrix}
\dot{\phi} \\
\dot{\theta} \\
\dot{\psi}
\end{pmatrix}
\tag{5.12}
\end{equation}
可以得到关于各欧拉角的角速度表达式:
\begin{align}
\dot{\phi} &= h(A\sin^{2}\psi + B\cos^{2}\psi)/AB \tag{5.13} \\
\dot{\theta} &= -h(A-B)\cos\theta\sin\psi\cos\psi/AB \tag{5.14} \\
\dot{\psi} &= \left(h/C - \dot{\phi}\right)\sin\theta \tag{5.15}
\end{align}

同时考虑到在目前跳水动作中,只存在绕 X 轴旋转的翻腾动作,以及绕 Z 轴旋转的转体动作,所以可以暂时不考虑 $\theta$ 角的角速度变化 [2]。

可以看出,角速度的瞬时值大小与当前姿态角和各个方向的转动惯量有关,对于比较相同的动作,角速度的值就取决于绕 X 轴与绕 Z 轴转动的转动惯量的大小。

\subsubsection{5.2.3 人体建模}

可以对人体进行抽象建模,将人身体分为头颈、躯干(含手臂)、大腿、小腿、脚 5 个部分。并且根据上一节分析,由于跳水动作中不包含绕 Y 轴旋转的运动,所以有关身体宽度的数据可以忽略(如髋关节宽等),同时,根据人体几何外形,将各部分抽象为而当人身高为 \(l\),质量为 \(m\) 时,根据实测人体数据 \({ }^{[3]}\),将男性人体长度(如腿长)、厚度(如腿围)和宽度数据整理如表 5.1,对应部分的质量分布数据如表 5.2。

\paragraph{表 5.1 人体长度、厚度和宽度数据}

\begin{tabular}{l c c c}
部位 & 长度 & 厚度 & 宽度 \\
\hline 头颈 & \(0.112l\) & \(0.0838l\) & \(0.0838l\) \\
躯干(含手臂) & \(0.335l\) & \(0.101l\) & \(0.1678l\) \\
大腿(两条) & \(0.240l\) & \(0.045l\) & \(0.090l\) \\
小腿(两条) & \(0.240l\) & \(0.031l\) & \(0.062l\) \\
双脚 & \(0.078l\) & \(0.031l\) & \(0.062l\) \\
\end{tabular}

\paragraph{表 5.2 人体质量分布数据}

\begin{tabular}{l c}
部位 & 质量 \\
\hline 头颈 & \(0.077m\) \\
躯干(含手臂) & \(0.550m\) \\
大腿(两条) & \(0.233m\) \\
小腿(两条) & \(0.100m\) \\
双脚 & \(0.040m\) \\
\end{tabular}

\paragraph{表 5.3 人体密度分布数据}

\begin{tabular}{l c}
部位 & 密度 \\
\hline 头颈 & \(97.9m/l^{3}\) \\
躯干(含手臂) & \(97.7m/l^{3}\) \\
大腿(两条) & \(229.4m/l^{3}\) \\
小腿(两条) & \(216.8m/l^{3}\) \\
双脚 & \(266.8m/l^{3}\) \\
\end{tabular}

采用以上数据,可以推算出几组不同身高和体重的人身体各部位重量及尺寸(表 5.4)。

\begin{table}
\centering
\begin{tabular}{c c c c c}
\hline
身高 & 体重 & 部位 & 尺寸 & 质量 \\
\hline
 & & 头颈 & $19*13*13$ & 4.6 \\
 & & 躯干(含手臂) & $56*17*28$ & 33 \\
170 & 60 & 大腿(两条) & $40*7.5*7.5$ & 14 \\
 & & 小腿(两条) & $40*5*5$ & 6 \\
 & & 双脚 & $13*5*5$ & 2.4 \\
 & & 头颈 & $19*13*13$ & 5.4 \\
 & & 躯干(含手臂) & $56*17*28$ & 38.5 \\
170 & 70 & 大腿(两条) & $40*7.5*7.5$ & 16.3 \\
 & & 小腿(两条) & $40*5*5$ & 7 \\
 & & 双脚 & $13*5*5$ & 2.8 \\
 & & 头颈 & $20*14*14$ & 4.6 \\
 & & 躯干(含手臂) & $59*18*30$ & 33 \\
180 & 60 & 大腿(两条) & $42*8*8$ & 14 \\
 & & 小腿(两条) & $42*5.3*5.3$ & 6 \\
 & & 双脚 & $13.7*5.3*5.3$ & 2.4 \\
 & & 头颈 & $20*14*14$ & 5.4 \\
 & & 躯干(含手臂) & $59*18*30$ & 38.5 \\
180 & 70 & 大腿(两条) & $42*8*8$ & 16.3 \\
 & & 小腿(两条) & $42*5.3*5.3$ & 7 \\
 & & 双脚 & $13.7*5.3*5.3$ & 2.8 \\
\hline
\end{tabular}
\caption{几组不同身高和体重的人身体各部分重量及尺寸}
\end{table}

可以看出与参考文献3中给出的运动员数据相符,说明人体建模方法可以接受。

\subsubsection{5.2.4 转动惯量的计算}

\paragraph{5.2.4.1 理想动作中的转动惯量}

理想动作指的是运动员旋转时,能够做到质心与实际转动轴完全重合,此时根据物理学定理,求解理想动作的转动惯量。

\paragraph{1、翻腾动作}

翻腾动作是跳水运动中使用最广泛的动作,根据身体姿态不同可以分为,直体翻腾、屈体翻腾和抱膝翻腾。假设人体为密度均匀的刚体,则对三种身体姿态抽象如下:

\paragraph{(1) 直体翻腾}

直体翻腾动作中,运动员身体挺直,延Y轴方向看去的截面可以抽象为图5.5。

\begin{figure}[h]
    \centering
    \includegraphics[width=0.8\textwidth]{image.png}
    \caption{直体翻腾动作的截面图}
    \label{fig:5.5}
\end{figure}

首先需要根据人体模型,以及之前给出的各部位的密度值,计算出此时人体的质心,质心计算公式如下\cite{ref4}:

\begin{equation}
z_{c} = \frac{M_{x}}{m} \tag{5.16}
\end{equation}

\begin{equation}
x_{c} = \frac{M_{z}}{m} \tag{5.17}
\end{equation}

式中:

\begin{equation}
m = \sum_{i} m_{i} \tag{5.18}
\end{equation}

\begin{equation}
M_{z} = \sum_{i} m_{i} \cdot x_{i} \tag{5.19}
\end{equation}

\begin{equation}
M_{x} = \sum_{i} m_{i} \cdot z_{i} \tag{5.20}
\end{equation}

而此时 $y_{c}$ 为 0。带入数据可以得到此时运动员的质心坐标为:(0,0,0)。之后计算运动员的转动惯量,根据定义,转动惯量表达式为:

\begin{equation}
\mathbf{I} = \iiint r^{2} dm \tag{5.21}
\end{equation}

而由于人体整体不规则,且为偏心转动,可根据公式:

\begin{equation}
\mathbf{I} = I_{C} + md^{2} \tag{5.22}
\end{equation}

进行计算,其中 $d$ 为实际转动轴到质心的距离。而通过计算,长方体绕自身质心的转动惯量为:

\begin{equation}
I_{C} = \frac{1}{12} m (a^{2} + b^{2}) \tag{5.23}
\end{equation}

从而可以计算转动惯量为:

\begin{equation}
A = \sum_{\text{分块}}^{i} I \tag{5.24}
\end{equation}

式中分块是根据转动轴,对转动轴上下,对不同密度区域进行累加运算。运算结果见表 5.5-5.6。

\paragraph{(2) 屈体翻腾}

根据动作,将人体屈体翻腾时抽象为图 5.6。

\begin{figure}[h]
    \centering
    \includegraphics[width=0.8\textwidth]{image1.png}
    \caption{屈体翻腾动作的截面图}
    \label{fig:5.6}
\end{figure}

同理可求得此姿态的质心和转动惯量,运算结果见表 5.5-5.6。

\paragraph{抱膝翻腾}

采用同样方法,抽象图 5.7 为:

\begin{figure}[h]
    \centering
    \includegraphics[width=0.8\textwidth]{image2.png}
    \caption{抱膝翻腾动作的截面图}
    \label{fig:5.7}
\end{figure}

同理可求得此姿态的质心和转动惯量,运算结果见表 5.5-5.6。

\paragraph{转体}

在跳水动作中,同样存在一些只转体,不翻腾的动作,此时运动员做绕 $Z$ 轴旋转的动作,此时运动员姿态可抽象为图 5.8:

\begin{figure}[h]
    \centering
    \includegraphics[width=0.8\textwidth]{image3.png}
    \caption{转体动作的截面图}
    \label{fig:5.8}
\end{figure}

\begin{figure}[h]
    \centering
    \includegraphics[width=\textwidth]{image.png}
    \caption{转体动作的截面图}
    \label{fig:5.8}
\end{figure}

同理可求得此姿态的质心和转动惯量,此姿态的转动惯量与直体翻腾的值相等,运算结果见表 \ref{tab:5.5}-\ref{tab:5.6}。

\documentclass{article}
\usepackage{array}

\begin{document}

\begin{table}[htbp]
\centering
\caption{各种动作对应的质心坐标}
\begin{tabular}{|c|c|}
\hline
\textbf{动作} & \textbf{质心坐标} \\ \hline
直体 & (0.039/, 0) \\ \hline
屈体 & (-0.033/, 0.024/) \\ \hline
抱膝 & (0.141/, 0.030/) \\ \hline
\end{tabular}
\end{table}

\vspace{1cm}

\begin{table}[htbp]
\centering
\caption{各种动作对应的转动惯量}
\begin{tabular}{|c|c|c|}
\hline
\textbf{动作} & \textbf{分量名称} & \textbf{转动惯量} \\ \hline
\multirow{3}{*}{直体/转体} 
& A & 0.060$m^2$ \\ \cline{2-3} 
& B & 0.059$m^2$ \\ \cline{2-3} 
& C & 0.005$m^2$ \\ \hline
\multirow{3}{*}{屈体} 
& A & 0.020$m^2$ \\ \cline{2-3} 
& B & 0.021$m^2$ \\ \cline{2-3} 
& C & 0.014$m^2$ \\ \hline
\multirow{3}{*}{抱膝} 
& A & 0.014$m^2$ \\ \cline{2-3} 
& B & 0.015$m^2$ \\ \cline{2-3} 
& C & 0.014$m^2$ \\ \hline
\end{tabular}
\end{table}

\end{document}

\paragraph{动作存在误差时的转动惯量}

根据本章一开始的分析,运动员水平的高低取决于运动员控制身体使质心与实际转动轴的距离,由于同等水平不同身高的运动员控制的偏差和身高成正比。由公式

\begin{equation}
\text{I} = I_{C} + md^{2}
\tag{5.25}
\end{equation}

可知,由于误差带来的额外转动惯量为:

\begin{equation}
md^{2}
\tag{5.26}
\end{equation}

可以看出,误差带来的转动惯量跟运动员质量成正比,与运动员身高的平方成正比。

\subsubsection{5.2.5 角速度的计算}

(1) 翻腾动作角速度与完成时间

根据翻腾动作的定义,翻腾动作仅是进行绕 X 轴旋转,则根据上一节分析,刚体在运动时,绕 X 轴转动的角速度如下:

\begin{equation}
\dot{\varphi} = h(A\sin^{2}\psi + B\cos^{2}\psi)/AB
\tag{5.27}
\end{equation}

而翻腾动作时,绕 Z 轴转过的角度 \(\psi\) 始终保持不变,并且根据转动方向的不同,\(\psi\) 的取值可能为两种情况为 0(面向转动方向),以及 \(\pi\)(背对转动方向)。所以可以计算出几种翻腾动作的角速度。

同时翻腾一圈转动的角度为 \(2\pi\),所以可以计算出完成时间。经计算可得:

a. 直体

当 \(\psi\) 等于 0 或 \(\pi\) 时,角速度相等为:

\begin{equation}
\dot{\varphi} = \frac{h}{A_{\text{直体}}} = \frac{h}{0.084ml^{2}}
\tag{5.28}
\end{equation}

b. 屈体

当 \(\psi\) 等于 0 或 \(\pi\) 时,角速度相等为:

\begin{equation}
\dot{\varphi} = \frac{h}{A_{\text{屈体}}} = \frac{h}{0.028ml^{2}}
\tag{5.29}
\end{equation}

c. 抱膝

当 \(\psi\) 等于 0 或 \(\pi\) 时,角速度相等为:

\begin{equation}
\dot{\varphi} = \frac{h}{A_{\text{抱膝}}} = \frac{h}{0.020ml^{2}}
\tag{5.30}
\end{equation}

经过计算可以发现,从单个动作完成时间来看,直体翻腾时间最长,屈体第二,抱膝时间最短;而从 \(\psi\) 的取值来看,当 \(\psi\) 为 0 时动作完成时间相比 \(\psi\) 值为 \(\pi\) 时较短。结合第一问中结论,同样运动方式和圈数情况下,直体难度大于屈体大于抱膝;而同样姿态和动作,逆时针旋转的难度普遍大于顺时针旋转。验证了此分析方法符合常识。

(2) 转体动作角速度与完成时间

转体和翻腾耦合的动作中,转体属于①型动作,必须在此过程中达到旋转角度;而耦合动作翻腾属于②型动作,整个动作中所需完成角度一部分在转体中进行,剩下的在其他动作中补齐。

根据转体动作的定义,转体动作为绕 $Z$ 轴旋转,则根据上一节分析,刚体在运动时,绕 $Z$ 轴转动的角速度如下:
\begin{equation}
\dot{\psi} = \left( \frac{h}{c} - \phi \right) \sin \theta
\tag{5.31}
\end{equation}

可以看出,转体动作的角速度与 $\theta$ 取值有关,同时也与绕 $X$ 轴旋转的角速度有关,可以看出转体角速度与翻腾角速度相互耦合。而运动员为了快速转体,转体时都会采用直体状态,此时对于翻腾,直体的转动惯量较大,不利于旋转,所以运动员会尽快完成转体动作,之后翻腾(常用抱膝或屈体,便于旋转)完成动作 [5]。

在此过程中,转体的角度必须在同一动作中完成,设转体过程的平均角速度为 $\overline{\psi}$,此时角速度与时间有:
\begin{equation}
2 \pi N = \overline{\psi} \cdot t_{\text{转体}}
\tag{5.32}
\end{equation}
此时,翻腾角速度为平均 $\varphi$ 翻腾的角度为:
\begin{equation}
\varphi_{\text{转体}} = \overline{\varphi} \cdot t_{\text{转体}}
\tag{5.33}
\end{equation}

\subsection{5.3 非刚体建模与计算}

针对翻腾及转体的组合动作,本文引用了 Tong [1] 于 2016 年提出建立的模型。如图 3.1 所示,该模型将人体看作由相互连接的两部分组成的耦合刚体,除了左手臂外,将身体的其余部分与躯干视为一个整体,在运动的整个过程中,左手臂始终保持伸直状态。此外,限定手肘只能做单一的外展-内收的平面运动,也即只能用单一的角度 $\alpha$ 来描述运动员的身体姿态,且姿态变换的过程被认为是瞬时发生的 [6]。

\begin{figure}[h]
\centering
\includegraphics[width=0.8\textwidth]{image.png}
\caption{两部分刚体模型示意图,$B_b$ 和 $B_l$ 分别代表两个部分,各部分的重心为 $C_b$ 和 $C_l$,总体的中心为 $C$,连接点为 $J_{b,l}$,$\alpha$ 为形态角度}
\end{figure}

在该模型中,建立了两个坐标系,一是空间固定的坐标系 $F_B$(principal-axes-frame),二是以自身为框架的坐标系 $F_C$(central-body-frame),后者会随着身体

位置的变化而变化。假设运动员进行了 \( M \) 次向前翻腾和 \( N \) 次转体,整个运动可以分为五个阶段,用 \( S_k \)(\( k=1, 2, 3, 4, 5 \))来表示,并令 \( S_k \) 阶段花费的时间为 \( \tau_k \)。其中,奇数阶段代表不包含姿态变化的刚体动力过程,偶数阶段代表姿态变化的转换过程(即左手臂移动)。图 3.2 给出了该过程的几个阶段。

\begin{figure}[h]
\centering
\includegraphics[width=0.8\textwidth]{image.png}
\caption{简化为五个阶段的翻腾及转体运动示意图,其中 UU 表示直体状态,LU 表示左手臂放下过程,DU 表示转体状态,HU 表示左手臂抬起过程}
\end{figure}

S1 阶段(UU),运动员从直体姿势以 \( v_0 \) 的垂直速度和 \( \boldsymbol{\Omega}_1(0) \) 的角速度从跳台上起跳。其中 \( \boldsymbol{\Omega}_1 \) 是基于惯性轴 \( F_{\mathrm{B}} \) 的,它对应了有恒定角动量 \( L_1 \) 的稳定旋转,从而运动员做的是简单的没有转体的翻腾运动。

S2 阶段(LU),运动员转变为扭转姿态的姿势变换。在该阶段中,手臂的移动导致身体剩下部分发生与之方向相反的转动,导致身体倾斜,并开始进入跳水过程中的扭转翻腾阶段。

S3 阶段(DU),运动员以刚体运动完成扭转翻腾动作。运动员在该阶段的时间 \( \tau_3 \) 在整个跳水过程中十分重要,因为它决定了运动员回到纯粹的翻腾动作(UU)前能够完成的扭转数目(即 \( N \))。

S4 阶段(HU),运动员进行另一次姿势变换以回到直体姿态,需要注意的是,该阶段的进入时机恰当与否(即 \( \tau_3 \) 是否合适)影响到后一阶段运动员能否回到直体姿态。

S5 阶段(UU),运动员回到直体姿态,该阶段的时间 \( \tau_5 \) 需要考虑到能否保证运动员以头先入水的方式进入泳池。

跳水过程中在空中进行上述一系列动作的总时间为五个阶段的总和,即 \( T_{\mathrm{air}} = \tau_1 + \tau_2 + \tau_3 + \tau_4 + \tau_5 \)。假定 \( \tau_2 = \tau_4 \to 0^+ \),那么 \( T_{\mathrm{air}} = \tau_1 + \tau_3 + \tau_5 \),其中,\( \tau_1 + \tau_5 \) 为纯粹的翻腾所花费的总时间,\( \tau_3 \) 为花费在扭转翻腾上的时间。由于只需要保证 \( \tau_1 \) 和 \( \tau_5 \) 它们的总时间一定,因此此处假定 \( \tau_1 = \tau_5 \)。此外,给定该运动员在初始时刻直体姿态时(UU)的角动量为 \( l \),从而得到该运动员在空中运动的总时间为

\begin{equation}
T_{\mathrm{air}}(l) = \frac{2\pi}{l} \left( M^s I_y + N \frac{J_z^t (J_y^s - I_y^s)}{J_y^t - J_z^t} \csc(P + X) \right)
\tag{5.34}
\end{equation}

其中,\( M, N \) 分别代表翻腾和转体的圈数,上标 \( s \) 和 \( t \) 分别用来代表翻腾(somersault)

和扭转翻腾(twisting somersault)对应的过程。$^tJ$ 为在 $F_B$ 坐标系下的惯性张量,$^sI$ 为在 $F_C$ 坐标系下的惯性张量。当运动员的体型一定时,惯性张量 $I(\alpha)$ 为姿态角度 $\alpha$ 的函数 [7],可表示为

\begin{equation}
I(\alpha)=\begin{pmatrix}
I_{xx} & 0 & 0 \\
0 & I_{yy} & I_{yz} \\
0 & I_{yz} & I_{zz}
\end{pmatrix}
\tag{5.35}
\end{equation}

其中,各部分的表达式为

\begin{align}
I_{xx} &= -4m\varepsilon_3F\cos\alpha + 4m\varepsilon_2F\sin\alpha + \left[2m(\varepsilon_2^2 + \varepsilon_3^2 + F^2) + I_{b_{xx}} + J_{l_x}\right] \\
I_{yy} &= -4m\varepsilon_3F\cos\alpha + \left[mF^2 + J_-\right]\cos 2\alpha + \left[m(2\varepsilon_3^2 + F^2) + I_{b_{yy}} + J_+\right] \\
I_{zz} &= 4m\varepsilon_2F\sin\alpha - \left[mF^2 + J_-\right]\cos 2\alpha + \left[m(2\varepsilon_2^2 + F^2) + I_{b_{zz}} + J_+\right] \\
I_{yz} &= 2m\varepsilon_2F\cos\alpha - 2m\varepsilon_3F\sin\alpha + \left[mF^2 + J_-\right]\sin 2\alpha + \left[I_{b_{yz}} - 2m\varepsilon_2\varepsilon_3\right]
\tag{5.36}
\end{align}

其中,参数 $m$ 由该运动员的质量决定,$\varepsilon_2$、$\varepsilon_3$ 和 $F$ 由运动员的身高体型决定。在直体姿态时,$^sI = \text{diag}(^sI_x, ^sI_y, ^sI_z) = I(\pi)$。在扭转姿态时,$^tJ = \text{diag}(^sJ_x, ^sJ_y, ^sJ_z) = R_x^{-1}(P)I(0)R_x(P)$。$I_b$ 是一个恒定的张量,$I_l$ 是一个与身高体重有关的 $\alpha$ 的函数,$J_l = \text{diag}(^sJ_x, ^sJ_y, ^sJ_z)$,可通过与 $I_l$ 之间的关系式求得,即:

\begin{equation}
I_l = R_\alpha J_l R_\alpha^{-1}
\tag{5.37}
\end{equation}

\begin{equation}
J_+ = \frac{1}{2}(J_{l_y} + J_{l_z}) \quad J_- = \frac{1}{2}(J_{l_y} - J_{l_z})
\tag{5.38}
\end{equation}

由于对上述解析公式进行数值求解得到精确的数值公式需要以较多的假定条件为前提,因此本部分仅对式 (3.1) 进行定性化分析 [8]。将各参数的表达式代入最终的 $T_{\text{air}}$ 表达式进行分析可得,对某一特定运动员而言,$T_{\text{air}}$ 与初始角动量有关。当初始角动量 $l$ 给定时,在保证相同的翻腾圈数 $M$ 和转体圈数 $N$ 的前提下,空中运动总时间 $T_{\text{air}}$ 仅与运动员的质量和身高有关。当身高一定时,$T_{\text{air}}$ 正比于质量参数 $m$ ($T_{\text{air}} \propto m$),当质量一定时,$T_{\text{air}}$ 正比于运动员的身型参数。也即,对体型相似的两个人而言,质量越大,$T_{\text{air}}$ 越大;对质量一致的两个人而言,体型越大,$T_{\text{air}}$ 越大。

值得注意的是,$T_{\text{air}}$ 指运动员完成如图 3.2 中所示五个阶段的动作需要的总时间,对某特定运动员和特定组合动作(特定 $M/N$)而言,该时间与初始角动量 $l$ 有关。在上述的公式推导过程中 $l$ 是人为给定的,而实际上初始角动量 $l$ 由运动员的起跳状态决定。对某运动员而言,他能够通过调整起跳时的垂直速度来控制在空中的时间。假设在 10 m 跳台上进行跳水,通过给初始起跳速度 $v_0$ 一个大致的范围就可以得到总时间 $T_{\text{air}}$ 的范围。保守假设 $v_0 = 0$,得到 $T_{\text{min}} = 1.4$。对于最大的 $v_0$ 数没有定量标准,因此这里采用目前世界上的最高的跳高记录 2.45 m。

在模型中,运动员站立时的重心大约为身体高度的 $57.66\%$,假设该运动员身高 $1.95 \, \text{m}$,那么其重心高度为 $1.12 \, \text{m}$,计算可得 $v_0 = 5.1$,$T_{\text{max}} = 2$。因此 $T_{\text{air}}$ 的范围可认为大致为 $1.4 < T_{\text{air}} < 2$。若某 $T_{\text{air}}$ 超过了这一范畴,那么该 $T_{\text{air}}$ 对应的翻腾和转体圈数 $M/N$ 就不能被完整地完成,也就是说运动员不能以头先入水的方式垂直落入水中。考虑到这一限制,假定两个运动员以相同的角动量 $l$ 起跳,那么质量大、体型大的人比质量小、体型小的人完成该 $M/N$ 组合动作的难度更大。

\subsection{5.4 过程 1/2/3 的建模与计算}

过程 1/2/3 同样属于耦合动作,在此过程中,姿态的变化为①型动作,必须在这部分完成,而伴随的翻腾或转体属于②型动作,剩余的角度可以在之后动作中补齐 \cite{ref9}。

根据对跳水动作的总结和分析,运动员身体姿态发生的变化有以下几种可能:

(1)直体-直体。当第一个空中动作为直体翻腾或转体时,过程 1 运动员身体状态未发生明显改变,典型动作如图 \ref{fig:5.11}:

\begin{figure}[h]
    \centering
    \includegraphics[width=0.8\textwidth]{image1.png}
    \caption{直体-直体动作过程}
    \label{fig:5.11}
\end{figure}

(2)直体-屈体。当翻腾圈数较多时,运动员无法进行直体翻腾,所以屈体在跳水动作中更为常见,在这类动作中,运动员过程 1 身体由直体变化为屈体,典型动作如图 \ref{fig:5.12}:

\begin{figure}[h]
    \centering
    \includegraphics[width=0.8\textwidth]{image2.png}
    \caption{直体-屈体动作过程}
    \label{fig:5.12}
\end{figure}

\begin{figure}[h]
    \centering
    \includegraphics[width=0.8\textwidth]{image1.png}
    \caption{直体-屈体动作过程}
    \label{fig:5.12}
\end{figure}

(3)直体-抱膝。抱膝同样适合多圈数的翻滚,并且相比屈体翻腾,更为简单,此时过程 1 为直体变换为抱膝,代表动作如图 \ref{fig:5.13}:

\begin{figure}[h]
    \centering
    \includegraphics[width=0.8\textwidth]{image2.png}
    \caption{直体-抱膝动作过程}
    \label{fig:5.13}
\end{figure}

(4)屈体-直体。主要存在于过程 3 中,屈体展开至直体;

(5)抱膝-直体。主要存在于过程 3 中,抱膝展开至直体。

\subsubsection{问题分析与建模}

根据人体结构的研究,膝部相对而言更为独立,膝部从伸直至弯曲到极限的过程中,结构张力力矩也相对不大,动作也相对容易完成,所以此处对其忽略;而腰部复杂程度大,由于发力方式和着力点的不同,动作的变化的难度主要集中在腰腹关节角度的变化。

将人的上半身下半身和腰腹支点看作一个绕轴旋转的系统,收缩的过程受力如图 \ref{fig:5.14} 所示。

\begin{figure}[h]
    \centering
    \includegraphics[width=0.8\textwidth]{image3.png}
    \caption{收缩过程受力示意图}
    \label{fig:5.14}
\end{figure}

\begin{figure}[h]
    \centering
    \includegraphics[width=\textwidth]{image1.png}
    \caption{运动员身体姿态变化受力分析}
    \label{fig:5.14}
\end{figure}

而身体在展开的过程中,结构张力和腰腹作用力方向相同,受力如图 \ref{fig:5.15} 所示。

\begin{figure}[h]
    \centering
    \includegraphics[width=\textwidth]{image2.png}
    \caption{运动员身体姿态变化受力分析}
    \label{fig:5.15}
\end{figure}

其中,腰腹作用力的大小与运动员质量有关(体重越大,力量越强),腰腹作用力着力点与身高成比例关系,结构张力着力点位于下半身质心处,大小与运动员柔韧度和质量有关。

\subsubsection{计算与求解}

而在上述(1)(2)(3)过程中,膝部的动作容易实现,但腰部动作则需要运动员具有足够强大的腰腹力量,此处将分别将上下半身视作刚体,绕腰部关节点进行转动,而腰部肌肉的着力点可视为与身高比例有关,此处认为在距离腰部转动轴十分之一身高的位置处。

此时,下半身可以看作一个费力杠杆,当动作的任意时刻,需要克服结构张力力矩的作用达到平衡,根据受力可写出状态方程:
\begin{equation}
    F_{\text{腰腹}} \times L_{\text{腰腹}} - F_{\text{结构}} \times L_{\text{结构}} = I_{\text{下半身}} \cdot \omega_{\text{收缩}}
    \tag{5.39}
\end{equation}

其中,$L_{\text{结构}}$ 和 $I_{\text{下半身}}$ 与屈体时下半身质心有关,可以求得:

下半身质心经计算,屈体时位于距离腰腹支点 $0.15l$ 的大腿中心,而抱膝时位于距离支点 $0.11l$ 的大腿上,由此可知:
\begin{equation}
    L_{\text{结构}} =
    \begin{cases}
        0.15 \cdot l & \text{屈体} \\
        0.11 \cdot l & \text{抱膝}
    \end{cases}
    \tag{5.40}
\end{equation}

而$L_{\text{腰腹}}$为身体的函数,设置为:
\begin{equation}
L_{\text{腰腹}} = l/20
\tag{5.41}
\end{equation}
假设同等柔韧度运动员,结构张力与质量有关,为:
\begin{equation}
F_{\text{结构}} = m * Q
\tag{5.42}
\end{equation}
同时认为腰腹力量同样与质量有关,为:
\begin{equation}
F_{\text{腰腹}} = m * P
\tag{5.43}
\end{equation}

将非自然状态区间的运动看作先加速后减速到0的过程,则对于屈体动作,需要转动$2/3\pi$,可以得到:
\begin{equation}
\omega_{\text{平均收缩}} = \frac{1}{2} * \frac{1}{2} \omega_{\text{收缩}}^{\cdot} t
\tag{5.44}
\end{equation}
所用时间为:
\begin{equation}
t_{\text{收缩}} = \frac{2\pi}{\omega_{\text{收缩平均}}}
\tag{5.45}
\end{equation}

而对于身体展开的时间,包括过程(4)、(5),由于此时结构张力方向不变,但腰腹可以反方向发力,可以更快的完成,根据此时受力可得:
\begin{equation}
F_{\text{腰腹}} \times L_{\text{腰腹}} + F_{\text{结构}} \times L_{\text{结构}} = I_{\text{下半身}} \cdot \omega_{\text{伸展}}^{\cdot}
\tag{5.46}
\end{equation}
其中,$L_{\text{长}}$和$I_{\text{下半身}}$与下半身质心有关,可以求得:
\begin{equation}
L_{\text{结构}} =
\begin{cases}
0.15 \cdot l & \text{屈体} \\
0.11 \cdot l & \text{抱膝}
\end{cases}
\tag{5.47}
\end{equation}
同时下半身转动惯量可求得:
\begin{equation}
I_{\text{下半身}} =
\begin{cases}
0.028ml^2 & \text{屈体} \\
0.007ml^2 & \text{抱膝}
\end{cases}
\tag{5.48}
\end{equation}
而$L_{\text{短}}$为身体的函数,为:
\begin{equation}
L_{\text{腰腹}} = l/20
\tag{5.49}
\end{equation}

假设同等柔韧度运动员,结构张力与质量有关,为:
\begin{equation}
F_{\text{结构}} = m * Q
\end{equation}
将非自然状态区间的运动看作先加速后减速到0的过程,则对于屈体动作,可以得到:
\begin{equation}
\omega_{\text{平均伸展}} = \frac{1}{2} * \frac{1}{2} \omega_{\text{伸展}}^{\cdot} t
\tag{5.50}
\end{equation}
所用时间为:

\begin{equation}
t_{\text{伸展}} = \frac{2\pi}{\omega_{\text{伸展平均}}}
\tag{5.51}
\end{equation}

根据以上模型,可以求得各种情况的完成时间:

\begin{equation}
t_{\text{收缩屈体}} = \frac{28\pi l}{25P - 75Q}
\tag{5.52}
\end{equation}

\begin{equation}
t_{\text{收缩抱膝}} = \frac{14\pi l}{50P - 110Q}
\tag{5.53}
\end{equation}

\begin{equation}
t_{\text{收缩屈体}} = \frac{28\pi l}{25P + 75Q}
\tag{5.54}
\end{equation}

\begin{equation}
t_{\text{收缩抱膝}} = \frac{14\pi l}{50P + 110Q}
\tag{5.55}
\end{equation}

而在此过程中伴随的翻腾的角度为:

\begin{equation}
\varphi = \omega_{\text{翻腾}} \cdot t
\tag{5.56}
\end{equation}

姿态变换的过程转动惯量取值设置为初始动作和最终动作转动惯量的平均值。此外,此动作对运动员来说较为轻松,通常在完成旋转等动作后根据情况,调整身体展开时间。

\subsection{5.5 结论}

根据本章开始的分析,跳水的过程时间表达式为:

\begin{equation}
T_{\text{总}} = T_1 + T_{\text{空中}_1} + T_2 + T_{\text{空中}_2} + T_3 + T_{\text{入水}}
\tag{5.57}
\end{equation}

\subsubsection{5.5.1 当空中只存在一个动作时}

\begin{equation}
T_{\text{总}} = T_1 + T_{\text{空中}_1} + T_3 + T_{\text{入水}}
\tag{5.58}
\end{equation}

首先对 \(T_1\) 求解如下,此过程中有可能收缩,身体不会伸展。

\textbf{表5.7 过程1计算结果}

\begin{tabular}{l c}
部位 & 质量 \\
\hline 头颈 & \(0.077m\) \\
躯干(含手臂) & \(0.550m\) \\
大腿(两条) & \(0.233m\) \\
小腿(两条) & \(0.100m\) \\
双脚 & \(0.040m\) \\
\end{tabular}

由于动作最后的伸展动作属于①型动作,所以首先计算 \(T_3\),在此过程中只存在伸展的情况。

\begin{tabular}{l c}
部位 & 密度 \\
\hline 头颈 & \(97.9m/l^{3}\) \\
躯干(含手臂) & \(97.7m/l^{3}\) \\
大腿(两条) & \(229.4m/l^{3}\) \\
小腿(两条) & \(216.8m/l^{3}\) \\
双脚 & \(266.8m/l^{3}\) \\
\end{tabular}

这两个过程中已经转过的角度已知,剩余角度在空中翻腾动作中完成,如表5.9所示。

\begin{table}
\centering
\caption{空中动作计算结果}
\begin{tabular}{c c c c c}
\hline
姿态 & 过程中转动惯量 & 角速度$\omega$ & 需要转动角度$\varphi$ & 所用时间 \\
\hline
直体 & $0.06ml^2$ & $\frac{16.7h}{ml^2}$ & $2\pi M$ & $\frac{\varphi}{\omega}$ \\
屈体 & $0.020ml^2$ & $\frac{50h}{ml^2}$ & $2\pi M - \frac{28\pi h}{(P-3Q)ml} - \frac{28\pi h}{(P+3Q)ml}$ & $\frac{\varphi}{\omega}$ \\
抱膝 & $0.014ml^2$ & $\frac{71.4h}{ml^2}$ & $2\pi M - \frac{7.56\pi h}{(P-2.2Q)ml} - \frac{7.56\pi h}{(P+2.2Q)ml}$ & $\frac{\varphi}{\omega}$ \\
\hline
\end{tabular}
\end{table}

\subsubsection{当空中存在两个动作时}

\begin{table}
\centering
\caption{过程2的计算结果}
\begin{tabular}{c c c c c}
\hline
动作变化前后 & 所用时间 & 过程中平均转动惯量 & 伴随翻腾角速度 & 转过角度 \\
\hline
直体-直体 & $0$ & $0.06ml^2$ & $\frac{16.7h}{ml^2}$ & $0$ \\
直体-屈体 & $\frac{28\pi l}{25P-75Q}$ & $0.04ml^2$ & $\frac{25h}{ml^2}$ & $\frac{28\pi h}{(P-3Q)ml}$ \\
直体-抱膝 & $\frac{14\pi l}{50P-110Q}$ & $0.037ml^2$ & $\frac{27h}{ml^2}$ & $\frac{7.56\pi h}{(P-2.2Q)ml}$ \\
屈体-直体 & $\frac{28\pi l}{25P+75Q}$ & $0.04ml^2$ & $\frac{25h}{ml^2}$ & $\frac{28\pi h}{(P+3Q)ml}$ \\
抱膝-直体 & $\frac{14\pi l}{50P+110Q}$ & $0.037ml^2$ & $\frac{27h}{ml^2}$ & $\frac{7.56\pi h}{(P+2.2Q)ml}$ \\
\hline
\end{tabular}
\end{table}

首先计算过程 1/2/3。过程 1 和 3 与一种动作情况相同,分别如表 5.7-5.8 所示。过程 2 计算如表 5.10 所示。之后,转体动作不会在其他动作中完成,所以此时转体为①型,首先计算转体动作所用时间。结果如表 5.11 所示。

\begin{table}[h]
\centering
\caption{转体过程计算结果}
\begin{tabular}{c c c c}
\hline
过程中转动 & 角速度 & 需要转动角度 & 所用时间 \\
惯量 & $\omega$ & $\varphi$ & \\
\hline
$0.005ml^2$ & $\dfrac{200h}{ml^2}$ & $2\pi N$ & $\dfrac{\pi N \cdot ml^2}{100h}$ \\
\hline
\end{tabular}
\end{table}

最后计算翻腾动作所用时间,如表 5.12。

\begin{table}[h]
\centering
\caption{翻腾过程计算结果}
\begin{tabular}{c c c c c}
\hline
姿态 & 过程中转动惯量 & 角速度 $\omega$ & 需要转动角度 $\varphi$ & 所用时间 \\
\hline
直体 & $0.06ml^2$ & $\dfrac{16.7h}{ml^2}$ & $2\pi M$ & $\dfrac{\varphi}{\omega}$ \\
\hline
屈体 & $0.020ml^2$ & $\dfrac{50h}{ml^2}$ & $2\pi M - \dfrac{28\pi h}{(P-3Q)ml} - \dfrac{28\pi h}{(P+3Q)ml}$ & $\dfrac{\varphi}{\omega}$ \\
\hline
抱膝 & $0.014ml^2$ & $\dfrac{71.4h}{ml^2}$ & $2\pi M - \dfrac{7.56\pi h}{(P-2.2Q)ml} - \dfrac{7.56\pi h}{(P+2.2Q)ml}$ & $\dfrac{\varphi}{\omega}$ \\
\hline
\end{tabular}
\end{table}

而根据跳水的经验,身体展开后会有一个入水阶段,此阶段一般为 0.2s 左右,供运动员调整姿态,进行入水。

为了更直观的观看结果,我们将公式中的参数进行假设,并进行了求解,参考参数如表。

\begin{table}[h]
\centering
\caption{运动员 1 参考数据}
\begin{tabular}{c c}
\hline
$m$ & 70 \\
\hline
$l$ & 1.8 \\
\hline
$h$ & 100 \\
\hline
$P$ & 522 \\
\hline
$Q$ & 58 \\
\hline
\end{tabular}
\end{table}

\begin{table}
\centering
\begin{tabular}{c c}
\hline
$m$ & 65 \\
\hline
$l$ & 1.75 \\
\hline
$h$ & 100 \\
\hline
$P$ & 522 \\
\hline
$Q$ & 58 \\
\hline
\end{tabular}
\caption{运动员 2 参考数据}
\end{table}

此时,对于表 5.13 中运动员,以 5255B 动作为例,计算了 5255B 动作的整个时间:为 1.38s;

对于表 5.14 中身高体重的运动员,完成同样的 5255B 动作,仅需要:1.26s

而对于跳水运动的其他动作,可以将运动员身高和体重带入表-表的表达式中进行整套动作时间的计算。

*说明:由于以上完成时间时运动员完成动作时质心与实际转动中心重合时的动作完成时间,所以计算值明显小于人自由落体的下落时间,而在实际动作中,运动员由于无法毫无误差的进行每个动作,所以导致角速度减小,整个跳水动作所需时间增加。

\section{六、问题三的求解}

\subsection{问题的分析与理解}

\subsubsection{重要概念的理解}

1、对上一问中完成时间的理解

在上一问中,计算了身高及质量对动作完成时间的影响,而在实际跳水中,并不是动作完成的越快越好,运动员可以根据每一个动作情况,随时调整姿态,在恰当的时间,完成动作,完美入水。

所以,上一问中计算的完成时间,可以理解为理论上完成动作的必要时间,例如当运动员距离入水的时间小于剩余动作的必要时间,则运动员不可能在入水前完成此动作,也更不可能根据情况调整入水姿态;相反,如果运动员距离入睡时间大于剩余动作的必要时间,则运动员可以根据经验进行调整,更从容地完成整个跳水动作。

2、对难度的理解

难度是对动作而言,在可以使用 “容错度” 进行表达,当一套动作的总完成时间,十分接近动作的必要完成时间时,运动员可能在完成动作时没有充足的余量去进行调整,而整个动作过程中,任何一个动作发生偏差,用其余动作进行补救的空间就很小,“容错度” 降低,使得难度提升;换句话说,运动员必须每一

\paragraph{3、完成动作的关键因素}

根据以上分析,跳水运动完成动作的关键在于控制自己的身体,在有限的身体下落的时间内,分配好各运动所用时间。而依据竞技体育的基本规律,一定水平的运动员可以轻松地实现难度低于自己水平的动作,而很难完成难度高于自己水平的动作(例如,能用 10 秒跑完百米的运动员,可以较为轻松地跑出 11 秒的成绩,却很难跑进 9.8 秒),所以如果运动员能够将跳水中的每个分动作从容地完成,将很容易完成整个一套跳水动作。

\paragraph{4、跳水运动的意义}

在制定一项体育运动规则的时候,除了分析出的技术指标外,往往更重要的是这项运动的意义。在常见的体育项目中,举重主要侧重于选手技术的应用,展现人类技巧与力量的结合,而不是仅仅比拼力量,所以举重比赛采用了分体重等级的比赛方式;游泳侧重的是人类在水中活动的能力,如果仅仅比拼速度,自由泳的速度明显大于其他三种泳姿,所以游泳比赛分成了不同泳姿;短跑比赛则是人类爆发力的体现,纯粹追求速度,所以短跑的评判标准仅为时间。

而跳水运动侧重于人类对身体的控制能力,而由于体型的不同,不同人的身体控制难度不同,在这个角度上,应当可以考虑将体型因素加入评判。

\subsubsection{6.1.2 问题解题思路}

由于完成动作的必要时间直接影响到动作完成的难度,所以此题中将动作完成的必要时间作为分析难度的主要因素,而体型又对动作完成时间产生影响,同时起跳后的角动量也影响转动速度,减少动作完成时间。此外,下落时间的增加可以延长做动作的时间,主要考虑的因素包括起跳初速度和陀螺原理。影响难度系数的相关因素如图 6.1 所示

\begin{figure}[h]
\centering
\includegraphics[width=0.8\textwidth]{image.png}
\caption{影响难度系数相关因素}
\end{figure}

对于上述各类因素,体型是造成难度不公平的决定性原因,应为体型直接影响了动作的完成时间,而即使体型庞大的运动员拥有足够的身体控制能力,有可能由于物理原因难以完成一些动作。

实际比赛中运动员不可能完美地重现自身的实力水准,发挥等随机因素同样对比赛影响很大,而当跳水过程中一些动作没有很好完成时,对随机状态进行调整的能力也同样影响了动作完成时间,对难度带来影响。

而过程中的角动量决定因素很多,尽管跟体型有一定关系,但难以定量分析。起跳初速度反映了运动员的弹跳能力,以及尽力跳跃后的身体控制,应当属于技术能力的一部分,在此不再作分析;而同时,运动员正面向前起跳的高度应高于向背后起跳的高度,使得运动员做正向起跳的时间更长,所以也验证了第一问中正向起跳难度小于向后跳跃的难度。

而运动员运动中告诉旋转带来的陀螺效应,在之前的文献中也未曾考虑,在此做简要分析。

\subsection{6.2 由体型造成动作完成时间带来的难度偏差的计算}

根据上一问中总结出的各套动作与完成时间的关系,我们可以根据运动员的身高 \( l \) 和体重 \( m \) 计算出对于此体型运动员的整套动作的必要完成时间 \( T(m, l) \),同时选取一个运动员平均体型(身高、体重)下的必要完成时间 \( T(m_c, l_c) \) 作为参考值,则校正系数为:

\[
\delta = \frac{T(m, l)}{T(m_c, l_c)} \tag{6.1}
\]

难度系数经过校正,为:

\[
\text{DD}_{\text{校正}}(m, l) = \text{DD} \cdot \delta \tag{6.2}
\]

通过以上计算即可根据运动员体型对难度系数进行校正。在此校正方法中,以 5255B 动作为例,假如参考值 \( m_c, l_c \) 分别为 \( 65 \, \text{kg} \) 和 \( 1.75 \, \text{m} \),则对于身高体重分别为 \( 1.8 \, \text{m} \),\( 70 \, \text{kg} \) 的运动员,校正系数为:\( 1.38 / 1.26 = 1.1 \)

通过乘以校正系数的方法对不同体型的难度进行修正,可以将产生难度差距的主要因素-动作完成时间进行消除,但同时,由于人体结构复杂,物理过程十分复杂,采用理论上的计算矫正可能存在一定误差,当运动员体型差距不大时,这种矫正可能会有很好的效果,而当运动员体型相差较大的情况,可能会加大误差,影响打分的客观性。

\subsection{6.3 由于随机因素造成动作完成时间带来的难度偏差}

在上一问的分析中,由于平行轴定理,当实际转动轴偏离运动员质心的时候,

转动惯量会明显增大,造成旋转速度减慢,影响动作的完成。同样,角动量的获取在运动员起跳后将不再改变,当起跳那一瞬间,获得的角动量也会有所偏差。

\begin{equation}
I = I_{c} + md^{2}
\tag{6.3}
\end{equation}

\begin{equation}
\mathbf{h} = I\boldsymbol{\omega}
\tag{6.4}
\end{equation}

而根据跳水规则的打分,运动员一次跳水最终得分计算方法为:

\begin{equation}
\text{Score} = \text{除去最高分和最低分以后的 5 名裁判平均分} \times \text{难度系数}
\tag{6.5}
\end{equation}

可见最终得分除去难度以外,还与裁判打分有关。一名体型较大的运动员与一名体型较小的运动员相比,不仅同一个动作的难度较大,考虑到发挥问题,体型大的运动员很有可能动作完成的质量也会降低。对于一个运动员恰好能完全完成的动作,发生一定的失误的情况下,需要在其他动作中超常进行补救,常见的失误有丢失部分角动量(设发生几率为 \( p_{1} \))和旋转偏差增大(设发生几率为 \( p_{2} \))。

\subsubsection{6.3.1 丢失部分角动量的修正}

由于比赛场地,心理素质,以及正常的心理波动,运动员可能在起跳时不完全,使得跳水动作中的角动量减少为 \( r \cdot \mathbf{h} \),其中 \( r \) 属于 \([0, 1]\)。在这种情况下,正常情况下有:

\begin{equation}
\mathbf{h} = (I_{c} + md^{2})\boldsymbol{\omega}
\tag{6.6}
\end{equation}

失误后:

\begin{equation}
r \cdot \mathbf{h} = (I_{c} + mx^{2})\boldsymbol{\omega}
\tag{6.7}
\end{equation}

要想在失误的情况下保持角速度不变,则需要减少转动时的偏差,设置运动员实际转动中偏差量为身高的正比例函数,即 \(\frac{d}{l} = u\) 时,解得:

\begin{equation}
\frac{x}{l} = r \cdot u + \frac{I_{A}(1-r)}{l \cdot m^{2}} \cdot u \approx r \cdot u
\tag{6.8}
\end{equation}

根据所需要做的动作(直体/屈体/抱膝)选取式中 \( I_{A} \) 的值,发现,纠正错误所需要提高旋转轴距离质心的距离,当 \( l \cdot m^{2} \) 越大(体型越大)时,需要将误差与身高的比例控制的越小,但经过计算,由于后一项数值过小,可以不考虑在内。

\subsubsection{6.3.2 转动误差增大的修正}

与上一部分类似,经过计算,由于旋转中某一圈失误带来的影响,需要就行修正时,体型越大的人需要有更高的控制精度比例,但同样,差别过小,可以不做考虑。

\subsection{6.4 角动量带来的难度偏差的计算}

在之前的分析中,我们假设运动员在跳水过程中角动量为同一个值,但在实际比赛中,不同运动员获取的角动量会有所不同。而角动量的获取与体型具体的关系,目前很难找到比较准确的解释。

而比较合理的解决方案将在之后提出。

\subsection{6.5 关于难度和体型的一些规律}

从之前的计算中可以发现,难度与对应动作的转动惯量有关,由此可知:

(1) 动作的难度不仅跟身高和质量有关,也与运动员体态有关,具体为,运动员的身高、体宽、体厚、与身体比例;

(2) 同等质量的运动员,偏高瘦的更擅长转体,偏矮胖的更擅长翻腾;

(3) 腰腹力量强、柔韧性强的运动员,适合做空中有动作变换的动作;

(4) 身体控制能力强的运动员,适合做单一动作多圈数动作。

\subsection{6.6 将体型因素考虑在内的校正方法}

\subsubsection{6.6.1 分量级比赛}

拳击、举重等运动,由于身体结构复杂,体重增加对协调性和力量的影响难以描述,为了竞赛公平,选择根据体重进行量级划分,不同量级分开进行比赛。而在跳水运动中,经过理论分析,体型对动作的完成有比较大的影响,而直接用校正系数可能在提醒相差过大的情况下,造成较大误差,所以也应当采用分量级的比赛方式。

其中,身高、体宽、体厚等因素虽然影响难度,但不同的身形的特点可以让运动员根据对手和自己的特点,选择适合的动作,增加比赛中的博弈;而质量影响了运动员的转动惯量、力量、角动量等,具体影响模型复杂,所以可以按运动员不同质量进行量级划分。

\subsubsection{6.6.2 各量级内采用校正参数进行校正}

由于分量级,各量级运动员体型差别不会太大,所以可以在同级别内使用 6.2 部分的校正系数,其参考体重可以设置为此量级体重中位数,身高设置为此量级内运动员的平均身高。

\subsection{6.7 将陀螺效应考虑在内的分析}

在目前的研究中,尚未发现有关于人体高速旋转影响重心的研究,但在此问题中,人体做高速旋转(尤其是转体时),可以将人体看作陀螺,具有陀螺的一些性质。

进动性。陀螺具有进动性,当高速自转的陀螺收到与自转轴垂直方向的力矩时,陀螺不会进行力矩方向的运动,而是采用进动的方式,向法向方向进行公转。如图 6.2 所示。

\begin{figure}[h]
    \centering
    \includegraphics[width=0.8\textwidth]{陀螺原理图.png}
    \caption{陀螺原理图}
    \label{fig:陀螺原理图}
\end{figure}

同样,高速进行转体的人体收到竖直向下的重力,此时重力一部分分量会让人体重心下降,但由于高速旋转的人体具有陀螺进动性,人体会向法线方向进行进动公转,将一部分能量转换为了身体的水平方向速度,从而延缓了下降的时间。

由于这一部分目前缺少相应的研究与文献,在此仅做定性分析,即,当动作中有转体时,人体在空中的时间会有一定的延长,使得存在转体的动作难度有所降低。

\subsection{6.8 结论}

本章中分析了不同体型运动员完成跳水动作时的难度,由于体型原因将导致同样动作不同体型的运动员完成动作的最小必要时间不同,造成了难度系数设置的不公平,所以可以根据跳水运动的实际意义,选择对造成的难度系数偏差进行校正。

由于理论模型本身存在误差,跳水比赛可根据运动员体重进行量级划分,将不同体重量级的运动员分开进行比较。

而在同一量级中,将量级体重中位数和此量级运动员平均身高作为参考值,以必要完成时间为参照对难度系数进行校正。分别计算出参考体型运动员的必要完成时间 $T(m_c, l_c)$,以及不同运动员对某一动作的必要完成时间 $T(m, l)$,则校正系数设置为:
\begin{equation}
    \delta = \frac{T(m, l)}{T(m_c, l_c)} \tag{6.9}
\end{equation}
难度系数经过校正为:
\begin{equation}
    \text{DD}_{\text{校正}}(m, l) = \text{DD} \cdot \delta \tag{6.10}
\end{equation}

\section{七、问题四的求解}

由于本文采用的方法首先采用根据体重分量级的方法,在同一量级内在原先难度系数的基础上,加入了校正系数对运动员身高体重就行了校正。而采用这种方法,难度系数表针对不同体型运动员,会有不同的运算值。

* 说明1:表7.1与表7.2是针对60-70公斤级的难度系数计算,此量级内参考体型为身高1.75m,体重65kg,此表是针对身高1.8m,体重70kg的运动员的校正系数和难度系数表,表中P、Q、h参考值见表5.13。具体计算公式参考表5.7-5.12,以及论文6.2、6.8节。

* 说明2:难度系数总体依托原先难度系数,因为人体结构复杂,原先的难度系数是由大量运动员和教练员以及专家实践分析得来的,可以较好地反应动作在实际中的难度;而之后使用校正系数校正体型影响,是基于本文的理论分析,对体型因素进行校正。

\textbf{表7.1 60-70公斤级针对身高1.8m体重70kg的运动员的校正系数}

\begin{table}[h]
\centering
\begin{tabular}{|c|c|c|c|c|c|}
\hline
\multirow{2}{*}{动作代码} & \multicolumn{2}{c|}{校正系数} & \multirow{2}{*}{动作代码} & \multicolumn{2}{c|}{校正系数} \\
\cline{2-3} \cline{5-6}
 & PIKE & TUCK & & PIKE & TUCK \\
\hline
105 & 1.104 & 1.095 & 5154 & 1.109 & 1.103 \\
\hline
107 & 1.112 & 1.105 & 5156 & 1.111 & 1.105 \\
\hline
109 & 1.117 & 1.111 & 5172 & 1.114 & 1.107 \\
\hline
1011 & -- & 1.115 & 5255 & 1.110 & 1.104 \\
\hline
205 & 1.104 & 1.095 & 5257 & 1.111 & 1.106 \\
\hline
207 & 1.112 & 1.105 & 5271 & 1.114 & 1.106 \\
\hline
209 & 1.117 & 1.111 & 5273 & 1.115 & 1.109 \\
\hline
305 & 1.104 & 1.095 & 5275 & 1.116 & 1.110 \\
\hline
307 & 1.112 & 1.105 & 5353 & 1.107 & 1.101 \\
\hline
309 & 1.117 & 1.111 & 5355 & 1.110 & 1.104 \\
\hline
405 & 1.104 & 1.095 & 5371 & 1.114 & 1.106 \\
\hline
407 & 1.112 & 1.105 & 5373 & -- & 1.109 \\
\hline
409 & 1.117 & 1.111 & 5375 & -- & 1.110 \\
\hline
\end{tabular}
\end{table}

\begin{table}
\centering
\caption{60-70公斤级针对身高1.8m体重70kg的运动员的难度系数}
\begin{tabular}{|c|c|c|c|c|c|c|c|c|}
\hline
\multirow{4}{*}{动作代码} & \multicolumn{2}{c|}{PIKE} & \multicolumn{2}{c|}{TUCK} & \multirow{4}{*}{动作代码} & \multicolumn{2}{c|}{PIKE} & \multicolumn{2}{c|}{TUCK} \\
\cline{2-5} \cline{7-10}
& 原 & 新 & 原 & 新 & & 原 & 新 & 原 & 新 \\
\cline{2-5} \cline{7-10}
& B & B & C & C & & B & B & C & C \\
\hline
105 & 2.3 & 2.5 & 2.1 & 2.3 & 5154 & 3.3 & 3.7 & 3.1 & 3.4 \\
\hline
107 & 3.0 & 3.3 & 2.7 & 3.0 & 5156 & 3.8 & 4.2 & 3.6 & 4.0 \\
\hline
109 & 4.1 & 4.6 & 3.7 & 4.1 & 5172 & 3.6 & 4.0 & 3.3 & 3.7 \\
\hline
1011 & -- & -- & 4.7 & 5.2 & 5255 & 3.6 & 4.0 & 3.4 & 3.8 \\
\hline
205 & 2.9 & 3.2 & 2.7 & 3.0 & 5257 & 4.1 & 4.6 & 3.9 & 4.3 \\
\hline
207 & 3.6 & 4.0 & 3.3 & 3.6 & 5271 & 3.2 & 3.6 & 2.9 & 3.2 \\
\hline
209 & 4.5 & 5.0 & 4.2 & 4.7 & 5273 & 3.8 & 4.2 & 3.5 & 3.9 \\
\hline
305 & 3.0 & 3.3 & 2.8 & 3.1 & 5275 & 4.2 & 4.7 & 3.9 & 4.3 \\
\hline
307 & 3.7 & 4.1 & 3.4 & 3.8 & 5353 & 3.3 & 3.7 & 3.1 & 3.4 \\
\hline
309 & 4.8 & 5.4 & 4.5 & 5.0 & 5355 & 3.7 & 4.1 & 3.5 & 3.9 \\
\hline
405 & 2.8 & 3.1 & 2.5 & 2.7 & 5371 & 3.3 & 3.7 & 3.0 & 3.3 \\
\hline
407 & 3.5 & 3.9 & 3.2 & 3.5 & 5373 & -- & -- & 3.6 & 4.0 \\
\hline
409 & 4.4 & 4.9 & 4.1 & 4.6 & 5375 & -- & -- & 4.0 & 4.4 \\
\hline
\end{tabular}
\end{table}

可以看出,在60-70公斤级针对身高1.8m体重70kg的运动员的难度系数比原先难度系数都有了一定的提高,主要是因为此体型运动员身高体重均大于当前量级的参考体型,导致该运动员在完成各套动作时难度更大,对应的难度系数也有了一定提升。

而对于其他量级中不同体型的运动员,同样可以将数据带入,得到针对不同运动员的难度系数表。

\section{八、总结}

1、根据对国际泳联跳水难度系数准则表得出部分主要结论:
(1) 多数情况下,难度系数的大小与翻腾周数成正比,也与转体周数成正比;

(2) 在不同的动作组别(即向前、向后、反身等不同起跳方式)中,一般而言,反身的难度系数最大;

(3) 对不同的动作姿势,直体姿态的难度系数最大,其次是屈体姿态,再次是抱膝姿态。

2、根据对问题的建模与分析,跳水的过程时间表达式为:
\begin{equation}
T_{\text{总}} = T_{1} + T_{\text{空中}_1} + T_{2} + T_{\text{空中}_2} + T_{3} + T_{\text{入水}}
\tag{8.1}
\end{equation}

(1) 当空中只存在一个动作时
\begin{equation}
T_{\text{总}} = T_{1} + T_{\text{空中}_1} + T_{3} + T_{\text{入水}}
\tag{8.2}
\end{equation}

其中 $T_{1}$ 计算如表 8.1:

\begin{table}[h]
\centering
\caption{$T_{1}$ 结果}
\begin{tabular}{c c c c}
\hline
动作变化前后 & 所用时间 & 过程中平均转动惯量 & 伴随翻腾角速度 & 转过角度 \\
\hline
直体-直体 & $0$ & $0.06ml^2$ & $\frac{16.7h}{ml^2}$ & $0$ \\
直体-屈体 & $\frac{28\pi l}{25P-75Q}$ & $0.04ml^2$ & $\frac{25h}{ml^2}$ & $\frac{28\pi h}{(P-3Q)ml}$ \\
直体-抱膝 & $\frac{14\pi l}{50P-110Q}$ & $0.037ml^2$ & $\frac{27h}{ml^2}$ & $\frac{7.56\pi h}{(P-2.2Q)ml}$ \\
\hline
\end{tabular}
\end{table}

其中 $T_{3}$ 计算如表 8.2:

\begin{table}[h]
\centering
\caption{$T_{3}$ 结果}
\begin{tabular}{c c c c}
\hline
动作变化前后 & 所用时间 & 过程中平均转动惯量 & 伴随翻腾角速度 & 转过角度 \\
\hline
直体-直体 & $0$ & $0.06ml^2$ & $\frac{16.7h}{ml^2}$ & $0$ \\
屈体-直体 & $\frac{28\pi l}{25P+75Q}$ & $0.04ml^2$ & $\frac{25h}{ml^2}$ & $\frac{28\pi h}{(P+3Q)ml}$ \\
抱膝-直体 & $\frac{14\pi l}{50P+110Q}$ & $0.037ml^2$ & $\frac{27h}{ml^2}$ & $\frac{7.56\pi h}{(P+2.2Q)ml}$ \\
\hline
\end{tabular}
\end{table}

其中 $T_{\text{空中}_1}$ 计算如表 8.3:

\begin{table}
\centering
\caption{T_{\text{空中}1}结果}
\begin{tabular}{c c c c c}
\hline
姿态 & 过程中转动惯量 & 角速度$\omega$ & 需要转动角度$\varphi$ & 所用时间 \\
\hline
直体 & $0.06ml^2$ & $\dfrac{16.7h}{ml^2}$ & $2\pi M$ & $\dfrac{\varphi}{\omega}$ \\
\hline
屈体 & $0.020ml^2$ & $\dfrac{50h}{ml^2}$ & $2\pi M - \dfrac{28\pi h}{(P-3Q)ml} - \dfrac{28\pi h}{(P+3Q)ml}$ & $\dfrac{\varphi}{\omega}$ \\
\hline
抱膝 & $0.014ml^2$ & $\dfrac{71.4h}{ml^2}$ & $2\pi M - \dfrac{7.56\pi h}{(P-2.2Q)ml} - \dfrac{7.56\pi h}{(P+2.2Q)ml}$ & $\dfrac{\varphi}{\omega}$ \\
\hline
\end{tabular}
\end{table}

$T_{\lambda水}$等于$0.2s$

(2) 当空中存在两个动作时,$T_1$、$T_3$计算同上:

其中$T_2$计算如表8.4:

\begin{table}
\centering
\caption{T_{2}结果}
\begin{tabular}{c c c c c}
\hline
动作变化前后 & 所用时间 & 过程中平均转动惯量 & 伴随翻腾角速度 & 转过角度 \\
\hline
直体-直体 & $0$ & $0.06ml^2$ & $\dfrac{16.7h}{ml^2}$ & $0$ \\
\hline
直体-屈体 & $\dfrac{28\pi l}{25P-75Q}$ & $0.04ml^2$ & $\dfrac{25h}{ml^2}$ & $\dfrac{28\pi h}{(P-3Q)ml}$ \\
\hline
直体-抱膝 & $\dfrac{14\pi l}{50P-110Q}$ & $0.037ml^2$ & $\dfrac{27h}{ml^2}$ & $\dfrac{7.56\pi h}{(P-2.2Q)ml}$ \\
\hline
屈体-直体 & $\dfrac{28\pi l}{25P+75Q}$ & $0.04ml^2$ & $\dfrac{25h}{ml^2}$ & $\dfrac{28\pi h}{(P+3Q)ml}$ \\
\hline
抱膝-直体 & $\dfrac{14\pi l}{50P+110Q}$ & $0.037ml^2$ & $\dfrac{27h}{ml^2}$ & $\dfrac{7.56\pi h}{(P+2.2Q)ml}$ \\
\hline
\end{tabular}
\end{table}

其中$T_{\text{空中}1}$计算如表8.5:

\begin{table}
\centering
\caption{T_{\text{空中}1}结果}
\begin{tabular}{c c c c}
\hline
过程中转动惯量 & 角速度$\omega$ & 需要转动角度$\varphi$ & 所用时间 \\
\hline
$0.005ml^2$ & $\dfrac{200h}{ml^2}$ & $2\pi N$ & $\dfrac{\pi N \cdot ml^2}{100h}$ \\
\hline
\end{tabular}
\end{table}

其中$T_{\text{空中}2}$计算如表8.6:

\begin{table}
\centering
\caption{T_{\text{空中}2}结果}
\begin{tabular}{c c c c c}
\hline
姿态 & 过程中转动惯量 & 角速度$\omega$ & 需要转动角度$\varphi$ & 所用时间 \\
\hline
直体 & $0.06ml^2$ & $\dfrac{16.7h}{ml^2}$ & $2\pi\text{M}$ & $\dfrac{\varphi}{\omega}$ \\
\hline
屈体 & $0.020ml^2$ & $\dfrac{50h}{ml^2}$ & $2\pi\text{M}-\dfrac{28\pi h}{(P-3Q)ml}-\dfrac{28\pi h}{(P+3Q)ml}$ & $\dfrac{\varphi}{\omega}$ \\
\hline
抱膝 & $0.014ml^2$ & $\dfrac{71.4h}{ml^2}$ & $2\pi\text{M}-\dfrac{7.56\pi h}{(P-2.2Q)ml}-\dfrac{7.56\pi h}{(P+2.2Q)ml}$ & $\dfrac{\varphi}{\omega}$ \\
\hline
\end{tabular}
\end{table}

$T_{\lambda水}$等于$0.2s$

使用如下参数时,具体结果为:

\begin{table}
\centering
\caption{运动员1参考数据}
\begin{tabular}{c c}
\hline
$m$ & $70$ \\
\hline
$l$ & $1.8$ \\
\hline
$h$ & $100$ \\
\hline
P & $522$ \\
\hline
Q & $58$ \\
\hline
\end{tabular}
\end{table}

\begin{table}
\centering
\caption{运动员2参考数据}
\begin{tabular}{c c}
\hline
$m$ & $65$ \\
\hline
$l$ & $1.75$ \\
\hline
$h$ & $100$ \\
\hline
P & $522$ \\
\hline
Q & $58$ \\
\hline
\end{tabular}
\end{table}

此时,对于表8.7中对应的运动员,以5255B动作为例,计算了5255B动作的整个时间:为$1.38s$

而对于表8.8中对应的运动员,完成同样的5255B动作,仅需要:$1.26s$

3、为了消除体型对难度的影响,采用分量级比赛的形式。在同一量级中,将量级体重中位数和此量级运动员平均身高作为参考值,以必要完成时间为参照对难度系数进行校正。分别计算出参考体型运动员的必要完成时间$T(m_c,l_c)$,以及不同运动员对某一动作的必要完成时间$T(m,l)$,则校正系数设置为:
\[
\delta = \dfrac{T(m,l)}{T(m_c,l_c)}
\]
难度系数经过校正为:

$$\sum_{j=1}^{110}x_{ij}^{down}=1$$

(1-5)


$$\mathrm{DD}_{\textit{校正}}(m,l)=\mathrm{DD}\cdot\delta $$

而随机因素对不同体型运动员影响几乎可以忽略;同等质量下,瘦长体型适合转
体,矮胖体型适合翻腾;陀螺效应使转体难度降低。
4、根据题意以 60-70 公斤级针对身高 1.8m 体重 70kg 的运动员为例的难度系数为:


\begin{table}
\centering
\caption{60-70公斤级针对身高1.8m体重70kg的运动员的难度系数}
\begin{tabular}{|c|c|c|c|c|c|c|c|c|c|}
\hline
\multirow{3}{*}{动作代码} & \multicolumn{2}{c|}{PIKE} & \multicolumn{2}{c|}{TUCK} & \multirow{3}{*}{动作代码} & \multicolumn{2}{c|}{PIKE} & \multicolumn{2}{c|}{TUCK} \\
\cline{2-5} \cline{7-10}
 & 原 & 新 & 原 & 新 & & 原 & 新 & 原 & 新 \\
\cline{2-5} \cline{7-10}
 & B & B & C & C & & B & B & C & C \\
\hline
105 & 2.3 & 2.5 & 2.1 & 2.3 & 5154 & 3.3 & 3.7 & 3.1 & 3.4 \\
\hline
107 & 3.0 & 3.3 & 2.7 & 3.0 & 5156 & 3.8 & 4.2 & 3.6 & 4.0 \\
\hline
109 & 4.1 & 4.6 & 3.7 & 4.1 & 5172 & 3.6 & 4.0 & 3.3 & 3.7 \\
\hline
1011 & -- & -- & 4.7 & 5.2 & 5255 & 3.6 & 4.0 & 3.4 & 3.8 \\
\hline
205 & 2.9 & 3.2 & 2.7 & 3.0 & 5257 & 4.1 & 4.6 & 3.9 & 4.3 \\
\hline
207 & 3.6 & 4.0 & 3.3 & 3.6 & 5271 & 3.2 & 3.6 & 2.9 & 3.2 \\
\hline
209 & 4.5 & 5.0 & 4.2 & 4.7 & 5273 & 3.8 & 4.2 & 3.5 & 3.9 \\
\hline
305 & 3.0 & 3.3 & 2.8 & 3.1 & 5275 & 4.2 & 4.7 & 3.9 & 4.3 \\
\hline
307 & 3.7 & 4.1 & 3.4 & 3.8 & 5353 & 3.3 & 3.7 & 3.1 & 3.4 \\
\hline
309 & 4.8 & 5.4 & 4.5 & 5.0 & 5355 & 3.7 & 4.1 & 3.5 & 3.9 \\
\hline
405 & 2.8 & 3.1 & 2.5 & 2.7 & 5371 & 3.3 & 3.7 & 3.0 & 3.3 \\
\hline
407 & 3.5 & 3.9 & 3.2 & 3.5 & 5373 & -- & -- & 3.6 & 4.0 \\
\hline
409 & 4.4 & 4.9 & 4.1 & 4.6 & 5375 & -- & -- & 4.0 & 4.4 \\
\hline
\end{tabular}
\end{table}

可以看出,在60-70公斤级针对身高1.8m体重70kg的运动员的难度系数比原先难度系数都有了一定的提高,主要是因为此体型运动员身高体重均大于当前量级的参考体型,导致该运动员在完成各套动作时难度更大,对应的难度系数也有了一定提升。

而对于其他量级中不同体型的运动员,同样可以将数据带入,得到针对不同运动员的难度系数表。

\section{九、参考文献}

[1] Tong W. Coupled Rigid Body Dynamics with Application to Diving, University of Sydney, 83-122, 2016.

[2] Yeadon M R. The biomechanics of twisting somersaults. Part I: Rigid body motions[J]. Journal of Sports Sciences, 1993, 11(3):187.

[3] 郝卫亚, 王智, 艾康伟. 运动员空中翻腾和转体姿态控制过程中转动惯量的变化[J]. 中国运动医学杂志, 2013, 32(11):966-973.

[4] 任桂泉. 空翻中转体理论及其应用[J]. 中国体育科技, 1981(21):28-32.

[5] 袁庆成, 王庆祥. 关于人体转动惯量的初步计算[J]. 辽宁体育科技, 1981(5):1-5.

[6] Dullin H R. The Diver with a Rotor[J]. 2015.

[7] Dullin H R, Tong W. Twisting Somersault[J]. Physics, 2016.

[8] Tong W, Dullin H R. A New Twisting Somersault: 513XD[J]. Journal of Nonlinear Science, 2017, 27(6):2037-2061.

[9] 曹立业. 我国优秀男子跳台跳水运动员 626B 技术动作的生物力学分析[J]. 当代体育科技, 2016, 6(8):124-126.