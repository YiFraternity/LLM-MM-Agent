\documentclass{article}
\usepackage{amsmath}
\usepackage{amssymb}

\title{基于卫星无源探测的空间飞行器主动段轨道估计与误差分析}
\author{}
\date{}

\begin{document}

\maketitle

\begin{abstract}
本文主要研究了卫星无源探测空间飞行器主动段的轨道估计及误差分析问题,根据4个问题的不同要求,建立了相应的数学模型并求解,实现了卫星轨道估计、飞行器轨道估计、系统误差估计以及联合轨道估计和误差分析等。主要工作包括:

(I) 问题1:对描述观测卫星轨道的简化运动方程这个二阶微分方程进行了降阶,继而提出了一种自适应定点的四阶龙格库塔算法求解运动微分方程,估计了09号卫星轨迹,该星在50s、100s、150s、200s、250s时刻的位置分别为$(1.77381 \times 10^6, 8.16138 \times 10^6, 4.51669 \times 10^6)$、$(1.50163 \times 10^6, 8.12676 \times 10^6, 4.68468 \times 10^6)$、$(1.22770 \times 10^6, 8.08270 \times 10^6, 4.84722 \times 10^6)$、$(9.52349 \times 10^5, 8.02925 \times 10^6, 5.00413 \times 10^6)$、$(6.75894 \times 10^5, 7.96650 \times 10^6, 5.15524 \times 10^6)$。

(II) 问题2:针对卫星观测的数据时间异步问题,提出了一种基于样条插值拟合的数据同步算法。因卫星对飞行器的量测所在的观测坐标系与数据处理所在的基础坐标系不一致,推导了观测坐标系与基础坐标系的转换关系,将数据信息统一到基础坐标系下。根据空间飞行器在卫星观测坐标系下的约束关系,建立了双星无源逐点交汇定位的算法模型,并利用同步后的数据解算出双星观测条件下的飞行器空间轨迹,继而基于中心差分理论得到飞行器各时刻的速度、加速度。根据$\vec{v}_r(t)$和$m(t)$的物理意义和变化规律,对二者分别建立了含有未知参量的近似模型,完善了主动段空间飞行器的运动方程,直接将飞行器的轨迹、速度和加速度数据代入运动方程可实现模型中未知参数的估计,为简化参数估计,提出了一种基于级数理论的多项式模型来拟合$\vec{v}_r(t)\dot{m}(t)/m(t)$这一整体,给出了对多项式各阶次系数的估计方法,并对不同阶次多项式模型的性能进行了分析对比。空间飞行器的运动数据同(I)用四阶龙格库塔算法求解。

阶龙格库塔法解算运动方程得出。基于上述模型解算出了 0 号飞行器在各个采样点的位置和速度,各位置分量的残差分别为:$1.60681 \times 10^2$、$1.19288 \times 10^2$、$1.97124 \times 10^2$,各速度分量的残差分别为:6.78000、4.03933、8.33612。

(III)问题 3:根据坐标转换模型和观测坐标系下的量测方程,确定系统误差、飞行器空间位置和卫星量测数据的关系,建立方程组,通过对方程组未知量个数与自由度关系的分析,发现不能利用逐点交汇方法估计系统误差;进而提出了一种系统误差估计的滑窗四点交汇方法,该方法通过相邻四点的滑动,实现了系统误差动态估计。估计出的系统误差为 $d_{\alpha_{06}} = 1.35166 \times 10^{-3}$,$d_{\beta_{06}} = 1.30491 \times 10^{-3}$,$d_{\theta_{06}} = 8.96365 \times 10^{-4}$,$d_{\alpha_{09}} = 8.38309 \times 10^{-4}$,$d_{\beta_{09}} = 5.15881 \times 10^{-4}$,$d_{\theta_{09}} = 1.17204 \times 10^{-3}$。通过求解的系统误差对附件的观测数据进行修正,利用修正后的数据,采用(II)的方法对飞行器进行轨道估计和误差分析。

(IV)问题 4:针对单星单飞行器无源轨道估计问题,结合短时间内飞行器的运动特征,建立了联合序贯多点轨道估计模型、无约束最优化轨道估计模型和基于轨道平面约束的轨道估计模型。利用(III)求得的系统误差对卫星观测数据进行修正,继而基于联合序贯多点轨道估计模型解算出飞行器的轨道参数,实现了单卫星对 1 号飞行器的轨道估计和误差分析。根据坐标转换模型、观测坐标系下的量测方程,通过讨论关于轨道参数和系统误差值的方程组是否可解来确定多星多飞行器联合轨道与误差估计的可行性。通过比较方程组自由度和待估计的轨道参数和系统误差的个数,得出了可估计系统误差的条件:在观测卫星个数 $M$ 与空间飞行器数目 $P$ 满足 $2MP \geq 3P + 2M$ 时,系统误差可估计。

关键词:轨道估计,系统误差,四阶龙格库塔算法,滑窗四点交汇,基于级数理论的多项式模型,联合序贯多点
\end{abstract}

\section*{第九届“华为杯”全国研究生数学建模竞赛}

\begin{center}
\textbf{题目} \quad 基于卫星无源探测的空间飞行器主动段轨道估计与误差分析
\end{center}

\begin{table}[h]
\centering
\begin{tabular}{l l}
学校 & 空军工程大学 \\
\hline
参赛队号 & 90052016 \\
\hline
队员姓名 & 1. 李思佳 \\
 & 2. 鹿传国 \\
 & 3. 刘玉磊 \\
\hline
\end{tabular}
\end{table}

\section*{目录}
\begin{itemize}
    \item[] 一、问题重述 \dotfill 2
    \item[] 二、问题假设 \dotfill 3
    \item[] 三、符号说明 \dotfill 4
    \item[] 四、问题分析 \dotfill 5
    \item[] 五、模型的建立与求解 \dotfill 6
    \begin{itemize}
        \item[] 5.1 问题一的建模与求解 \dotfill 7
        \begin{itemize}
            \item[] 5.1.1 卫星轨道模型及算法 \dotfill 7
            \item[] 5.1.2 计算结果及分析 \dotfill 9
        \end{itemize}
        \item[] 5.2 问题二的建模与求解 \dotfill 10
        \begin{itemize}
            \item[] 5.2.1 卫星观测数据同步 \dotfill 11
            \item[] 5.2.2 双星逐点交汇定位的建模与求解 \dotfill 16
            \item[] 5.2.3 $\vec{v}_r(t)$ 和 $m(t)$ 的模型建立与参数估计 \dotfill 21
            \item[] 5.2.4 飞行器轨道估计及残差计算 \dotfill 26
        \end{itemize}
        \item[] 5.3 问题三的建模与求解 \dotfill 31
        \begin{itemize}
            \item[] 5.3.1 逐点交汇法系统误差估计可行性分析 \dotfill 31
            \item[] 5.3.2 系统误差求解 \dotfill 32
            \item[] 5.3.3 飞行器轨道估计与残差计算 \dotfill 33
        \end{itemize}
        \item[] 5.4 问题四的建模与求解 \dotfill 38
        \begin{itemize}
            \item[] 5.4.1 单星单飞行器无源观测 \dotfill 38
            \item[] 5.4.2 多卫星多目标联合系统误差估计 \dotfill 46
        \end{itemize}
    \end{itemize}
    \item[] 六、模型优缺点分析 \dotfill 48
\end{itemize}

\section{一、问题重述}

对他国发射具有敌意的空间飞行器实施监控并做出快速反应,对于维护国家安全具有重要的战略意义。发现发射和探测飞行器轨道参数是实现监控和做出反应的第一步,没有观测,后续的判断与反应都无从谈起。卫星居高临下,是当今探测空间飞行器发射与轨道参数的重要平台。

装置于卫星上的探测器包括有源和无源两类:有源探测器采用主动方式(如雷达,激光)搜寻目标,同时具备定向和测距两种能力;无源探测器则被动接收目标辐射。采用无源探测器的观测卫星常采用红外光学探测器,只接收目标的红外辐射信息,可定向但不能测距。对于火箭尾部喷焰的高度敏感性是红外技术的长处,但易受气候影响与云层干扰则是其缺点。探测的目的是为了推断空间飞行器的轨道参数,推断是基于观测数据并通过数学模型与计算方法做出的。当观测卫星飞行一段时间,探测器测得目标相对于运动卫星的观测数据,以观测卫星和空间飞行器的运动模型和观测模型为基础,对空间飞行器的轨道参数(包括轨道位置、速度初值和其他模型参数)进行数学推断,为飞行器类别、飞行意图的判断提供信息基础。空间飞行器轨道一般可分为三段,依次为:靠火箭推进的主动段、在地球外层空间的惯性飞行段和再入大气层后的攻击段。主动段通常由多级火箭相继推进,前一级火箭完成推进后脱落,由后一级火箭接力。惯性飞行段在空气阻力极小的大气层外,靠末级火箭关机前获得的速度在椭圆轨道上作无动力惯性飞行。攻击段则根据任务需求,受控制后再入大气层,飞向目标。对于卫星而言,在其寿命结束前一直绕地飞行,故无攻击段。

主动段又可细分为若干子段:垂直上升段,程序拐弯段和重力斜飞段。按最优轨道设计,为节约燃料,箭体应尽快穿过稠密大气层,故火箭一般先垂直发射。程序拐弯段连接垂直上升段与重力斜飞段,在外力矩控制下使箭体转过一定角度,该段完成后外加力矩撤销,进入斜飞状态。第一级火箭通常负担“垂直段+程序拐弯段(加外力矩)+重力斜飞段的前段”的推进(视发动机的特性),重力斜飞段的后程则靠第二、第三级火箭相继完成。由于斜飞状态下地球引力与推力不在同一直线,所以箭体质心的运动轨迹为带一定弧度的光滑曲线。

单个红外光学探测器不具备测距能力,但借助多颗(含两颗)观测卫星的同步观测能够进行逐点定位,再结合空间飞行器的运动模型,可以进行轨道参数估计。在单星观测条件下,利用空间飞行器轨道的特殊性,结合较强的模型约束也可得到一定精度轨道参数估计。由于受大气影响,垂直上升段的火箭尾焰不易观测,程序拐弯段的运动方程又较为复杂,所以本题重点关注重力斜飞段的后程段,本题的研究集中于此段。

观测数据不可避免地带有各种误差,观测误差包括随机误差和系统误差。假设随机误差为直接叠加在观测数据上的白噪声,可能产生于背景辐射干扰与信息处理等多个方面。经由适当的简化模型,各种系统误差最终可以折合为观测坐标系的原点位置误差和三轴指向误差。根据工程经验,原点位置误差影响较小,而三轴指向误差影响较大,对三轴指向误差进行估计对于提高估计精度很有帮助,本题只考虑三轴指向误差。三轴指向误差在二维观测数据平面上表现为两个平移误差和一个旋转误差。

现以中低轨近圆轨道卫星为观测星座对假想的空间飞行器进行仿真观测,生成仿真观测数据,要求利用仿真观测数据,对假想空间飞行器的轨道参数进行估计。在仅考虑随机误差的条件下,研究下列问题:

问题1:观测卫星在任意时刻的位置计算是估计的前提,请根据附件satinfo.txt和观测卫星的简化运动方程(2),计算09号观测卫星在50.0s、100.0s、150.0s、200.0s、250.0s五个时刻的三维位置。结果保留6位有效数字。

问题2:在本题给定的仿真数据下,06号和09号观测卫星对0号空间飞行器形成了立体交叠观测,请结合立体几何知识按照逐点交汇定位的思路,给出0号空间飞行器在公式(1)框架下的轨道估计,注意选取适当的 $\vec{v}_r(t)$ 和 $m(t)$ 的表示模型。按照从50.0s到170.0s间隔10.0s进行采样,计算并列表给出0号空间飞行器在各个采样点的位置和速度,并给出估计残差。结果保留6位有效数字。同时绘制0号空间飞行器的三个位置t-x、t-y、t-z和三个速度t-vx、t-vy、t-vz曲线示意图。

问题3:若06和09号两颗观测卫星均有可能带有一定的系统误差,对系统误差进行正确的估计能够有效提高精度。利用上述的逐点交汇方法能否同时对系统误差进行估计?若不能,是否还有其他的思路能够同时估计系统误差与轨道?给出你的解决方案与估计结果。在报告中除给出与第二问要求相同的结果外,还应分别给出两颗观测卫星的系统误差估计结果,共六个数值 $d_\alpha, d_\beta, d_\theta$,分别是两颗卫星的。

问题4:对只有09号观测卫星单星观测的01号空间飞行器进行轨道估计,结果形式要求同第三问,注意参考第三问的系统误差估计结果。并进一步考虑在同时有多颗观测卫星观测多个空间飞行器的情况下能否联合进行系统误差估计?

\section{二、问题假设}

根据题意,可以进行如下假设:

1. 地球公转周期远大于空间飞行器的观测弧段时长,故本题在短时间内认定基础坐标系为惯性坐标系,该基础坐标系不随地球旋转;
2. 目标运动具有连续性和渐进性;
3. 假设随机误差为直接叠加在观测数据上的白噪声,可能产生于背景辐射干扰与信息处理等多个方面;
4. 假设利用两个平移误差可以代表三轴误差;
5. $m(t)$ 一般而言应为严格单调递减的非负函数,$\vec{v}_r(t)$ 的方向一般应与飞行器的速度方向相同,其大小稳定;
6. 题目仅关注飞行器的重力斜飞段的后程段;
7. 不考虑空间飞行器各点的速度差异;
8. 不考虑空间飞行器因火箭燃料消耗而产生的位移;
9. 作用于飞行器上的所有外力都通过质心,即不计空间飞行器在飞行过程中绕质心的转动。

\section{三、符号说明}

论文中涉及到的符号说明如下:

\begin{table}[h]
\centering
\begin{tabular}{|c|p{10cm}|c|c|}
\hline
符号 & 含义 & 单位 & 备注 \\
\hline
$G_{m}$ & 地球引力常数 $G_{m}=3.986005*10^{14}$ & $m^{3}/s^{2}$ & \\
$\vec{r}_{c}(t)$ & 空间飞行器在基础坐标系下的位置矢量 & $m$ & \\
$\ddot{\vec{r}}_{c}(t)$ & 空间飞行器在基础坐标系下的加速度 & $m/s^{2}$ & \\
$m(t)$ & 瞬时质量 & $kg$ & \\
$\dot{m}(t)$ & 质量变化率 & $kg/s$ & \\
$\alpha_{mc}(t)$ & $t$ 时刻 m 号卫星对某一飞行器的估计比值 $\alpha=\frac{y}{x}$ & & \\
$\beta_{mc}(t)$ & $t$ 时刻 m 号卫星对某一飞行器的估计比值 $\beta=\frac{z}{x}$ & & \\
$\vec{\rho}_{m}(t)$ & $t$ 时刻 m 号卫星位置 $\vec{r}(t)$ 在 $x_{c}O_{c}y_{c}$ 面得投影向量 & & \\
$\ddot{\vec{r}}_{x}(t)$ & 空间飞行器在基础坐标系下位置矢量在 $O_{s}X_{s}$ 方 & $m$ & \\
$\ddot{\vec{r}}_{y}(t)$ & 空间飞行器在基础坐标系下位置矢量在 $O_{s}Y_{s}$ 方向 & $m$ & \\
$\ddot{\vec{r}}_{z}(t)$ & 空间飞行器在基础坐标系下位置矢量在 $O_{s}Z_{s}$ 方 & $m$ & \\
$\vec{R}_{i}(t)$ & i 号卫星在基础坐标系下的位置矢量 & $m$ & \\
$\vec{x}_{i}(t)$ & i 号卫星位置矢量 $\vec{R}_{i}(t)$ 在 $O_{s}X_{s}$ 方向上的分量 & $m$ & \\
$\vec{y}_{i}(t)$ & i 号卫星位置矢量 $\vec{R}_{i}(t)$ 在 $O_{s}Y_{s}$ 方向上的分量 & $m$ & \\
$\vec{z}_{i}(t)$ & i 号卫星位置矢量 $\vec{R}_{i}(t)$ 在 $O_{s}Z_{s}$ 方向上的分量 & $m$ & \\
$\ddot{\vec{r}}_{is}(t)$ & i 号卫星观测的飞行器加速度 & $m/s^{2}$ & \\
$\vec{r}_{is}(t)$ & i 号卫星观测的飞行器速度 & $m/s$ & \\
$\vec{e}_{x_{s}}$ & 观测坐标系的单位方向向量 $\overrightarrow{O_{c}X_{c}}$ & & \\
$\vec{e}_{x_{y}}$ & 观测坐标系的单位方向向量 $\overrightarrow{O_{c}Y_{c}}$ & & \\
$\vec{e}_{x_{y}}$ & 观测坐标系的单位方向向量 $\overrightarrow{O_{c}Z_{c}}$ & & \\
\hline
\end{tabular}
\end{table}

为方便理解,其余符号将在文中陆续引出。

\section{四、问题分析}

现阶段,太空战略已成为各个国家竞相发展的一个方面,而卫星无源探测系统是太空预警探测系统的重要组成部分,它不仅关系到导航、定位、气象、测绘等民用的各个方面,还关系到了国家领空安全,更是一个国家国防预警能力的体现。与有源探测相比,无源探测具有探测距离远、频率分配受限少、抗电子干扰和抗毁能力强等优势。在太空成为新战场的今天,卫星无源探测技术已成为取得太空优势的关键技术之一。

无源探测器就是不主动辐射电磁波,只是被动接收目标辐射。这使其可定向但不能测距,从而导致定位比较困难。目前卫星无源探测是卫星探测研究的热点,难点问题之一。因为无源探测常采用红外光学探测器,只接收目标的红外辐射信息,对于火箭尾部喷焰的高度敏感性是红外技术的长处,但易受气候影响与云层干扰则是其缺点,所以无源探测的主要任务之一就是对弹道导弹等空间飞行器的轨道进行估计,推断该类空间飞行器的轨道参数。当观测卫星飞行一段时间,探测器测得目标相对于运动卫星的观测数据,以观测卫星和空间飞行器的运动模型和观测模型为基础,对空间飞行器的轨道参数(包括轨道位置、速度初值和其他模型参数)进行数学推断,为飞行器类别、飞行意图的判断提供信息基础。

该类空间飞行器轨道一般包含三个阶段:火箭推进的主动段、地球外层空间的惯性飞行段和再入大气层后的攻击段。主动段又可细分为若干子段:垂直上升段,程序拐弯段和重力斜飞段。由于重力斜飞段的特殊性,因此对其研究成为了基于卫星无源探测的空间飞行器轨道估计的重点。

该类空间飞行器在基础坐标系下的主动段的简化运动方程如下:

\begin{equation}
\ddot{\vec{r}}(t) = \vec{F}_e + \vec{F}_T = -\frac{G_m}{|\vec{r}(t)|^3}\vec{r}(t) + \vec{v}_r(t)\frac{\dot{m}(t)}{m(t)}
\tag{4-1}
\end{equation}

观测卫星的简化运动方程为:

\begin{equation}
\ddot{\vec{r}}(t) = \vec{F}_e = -\frac{G_m}{|\vec{r}(t)|^3}\vec{r}(t)
\tag{4-2}
\end{equation}

通过两颗卫星进行无源探测的空间飞行器主动段轨道估计过程如图 4.1 所示。红色阴影(模块一)完成了卫星的轨迹估计,绿色阴影(模块二)表示了通过逐点交汇方法对空间飞行器定位,并对所建的 $\vec{v}_r(t)$ 和 $m(t)$ 模型进行参数估计,得出飞行器的运动方程进一步估计飞行器轨道,并估计残差。蓝色阴影(模块三)代表通过时间段交汇方法求解系统误差,然后在提高精度的前提下,对空间飞行器的轨道进行估计。

\begin{figure}[h]
    \centering
    \includegraphics[width=\textwidth]{image.png}
    \caption{基于卫星无源探测的空间飞行器轨道估计}
    \label{fig:4.1}
\end{figure}

问题一需要解决的问题是:根据题目给出的卫星观测简化运动方程,结合初始条件卫星的位置矢量和速度矢量,利用常微分方程组的数值解析方法计算出卫星的轨迹,同时确定几个整数时刻的位置矢量及速度矢量。需注意的一点是卫星的运动方程是二阶常微分方程,还需要进行降阶。

问题二需要解决的问题可以分为四个方面:一是卫星观测数据同步问题(对应图 4.1 中的双星同步模块),即对于不同卫星表示同一目标位置的时刻集合扩大为一致,使得所有观测轨迹的采样数据同步可利用样条插值的方法,将各卫星观测数据进行同步;二是通过坐标转换和双星无源逐点交汇方法,计算得到双星观测条件下的飞行器实际空间轨迹;三是通过对所给 $\bar{v}_{r}(t)$ 和 $m(t)$ 的建模,利用双星无源逐点交汇方法得到的轨迹对模型中的参数进行估计;四是结合所得到的估计参数,利用飞行器运动方程,通过微分方程组的解出估计出空间飞行器的轨迹,比较通过双星观测的飞行器空间轨迹和通过微分方程组解出空间飞行器的轨迹,求解残差,形成对比。

问题三需要解决的问题是:确定双星无源逐点交汇方法的系统误差估计的可行性,使用滑窗四点交汇法估计出系统误差,通过求出的系统误差对附件给出的 06 与 09 号卫星观测数据进行修正,利用修正后的数据,采用第二问的解法对飞行器进行逐点交汇定位和轨道估计,并计算出相应的残差。

问题四需要解决的问题可以分为两个方面:一是研究单星无源探测定位的特殊性,寻求建立综合多时刻观测集合的单星定位模型,通过利用第三问求出的系统误差对 09 号卫星的观测数据进行修正,利用修正后的观测数据对 01 号飞行器进行定位和轨道估计。二是对多卫星多飞行器无源跟踪问题进行研究,将系统误差估计转化到方程是否可解的问题上来。通过对方程个数与未知量的个数的分析比较,得出可以估计系统误差的条件。

\section{五、模型的建立与求解}

\subsection{5.1 问题一的建模与求解}

卫星轨道的精确估计是空间飞行器轨道估计的基础,该问题的目的便是对卫星的轨道参数进行估计。在已知卫星初始时刻的轨道参数及其运动方程的条件下,通过解常微分方程即可得到卫星轨道的参数。在对常微分方程进行降阶之后,采用了一种自适应定点的四阶龙格库塔算法,得到卫星轨道的数值解,为空间飞行器的轨道估计奠定基础。

\subsubsection{5.1.1 卫星轨道模型及算法}

根据题中所给的观测卫星的简化运动方程及其初始条件建立卫星轨迹的常微分方程组:
\begin{equation}
\left\{
\begin{aligned}
\frac{d^2 \vec{R}_{09}(t)}{dt^2} &= \vec{F}_e = -\frac{G_m}{|\vec{r}(t)|^3} \vec{R}_{09}(t) \\
\frac{d \vec{R}_{09}(t)}{dt} &= \vec{v}(t) \\
\vec{R}_{09}(0) &= \left( x_0, y_0, z_0 \right)^T \\
\vec{v}(0) &= \left( v_x(0), v_y(0), v_z(0) \right)
\end{aligned}
\right.
\tag{5.1-1}
\end{equation}

代入 $\vec{R}_{09}(t) = \left( x(t), y(t), z(t) \right)^T$ 和 $\vec{v}(t) = \left( v_x(t), v_y(t), v_z(t) \right)^T$ 于公式 (5.1-1) 得到模型:
\begin{equation}
\left\{
\begin{aligned}
\frac{d^2 x(t)}{dt^2} &= -\frac{G_m}{\left[ x^2(t) + y^2(t) + z^2(t) \right]^{3/2}} x(t) & \frac{dx(t)}{dt} &= v_x(t) \\
\frac{d^2 y(t)}{dt^2} &= -\frac{G_m}{\left[ x^2(t) + y^2(t) + z^2(t) \right]^{3/2}} y(t) & \frac{dy(t)}{dt} &= v_y(t) \\
\frac{d^2 z(t)}{dt^2} &= -\frac{G_m}{\left[ x^2(t) + y^2(t) + z^2(t) \right]^{3/2}} z(t) & \frac{dz(t)}{dt} &= v_z(t)
\end{aligned}
\right.
\tag{5.1-2}
\end{equation}

式 (5.1-2) 构成了基本的运动轨道模型,进一步联合初始时刻卫星的初值,采用自适应定点的四阶龙格库塔算法 \cite{ref1} (Adaptive Fixed Point-4-Optical Step Runge-Kutta, AFP-4-OSRK) 求解出卫星轨迹的数值解。

\begin{table}[h]
\centering
\caption{09 号卫星的初始时刻位置与速度}
\begin{tabular}{|c|c|c|c|}
\hline
位置/m & 数值 & 速度/m/s & 数值 \\
\hline
$x(t)$ & 2043922.166765 & $v_x(t)$ & -5379.544693 \\
\hline
$y(t)$ & 8186504.631471 & $v_y(t)$ & -407.095342 \\
\hline
$z(t)$ & 4343461.714791 & $v_z(t)$ & -3516.052656 \\
\hline
\end{tabular}
\end{table}

由于卫星轨道的确定是空间飞行器轨道估计的基础,因此很有必要在卫星轨道估计中保持较高的精度,同时确定采样点时刻卫星轨道估计值。为此在传统定步长的基础上提出了 AFP-4-OSRK 算法,该算法既保持了步长的自适应能力,同时提高了算法的精度。具体的 AFP-4-OSRK 算法如下:

Step1:根据动力学规律有 \( v_x(t) = \frac{dx(t)}{dt}, v_y(t) = \frac{dy(t)}{dt}, v_z(t) = \frac{dz(t)}{dt} \)

\[
\frac{d^2x(t)}{dt^2} = \frac{d(v_x(t))}{dt}, \frac{d^2y(t)}{dt^2} = \frac{d(v_y(t))}{dt}, \frac{d^2z(t)}{dt^2} = \frac{d(v_z(t))}{dt}
\]

根据动力学规律有 \( v(t) = \frac{dx}{dt} \),则 \( \frac{d^2x(t)}{dt^2} = \frac{dv_x(t)}{dt} \)。进而将模型(公式(5.1-2))降阶变换为如下式的常系数微分方程组:

\[
G(t) = \begin{pmatrix}
\frac{dv_x(t)}{dt} \\
\frac{dv_y(t)}{dt} \\
\frac{dv_z(t)}{dt} \\
\frac{dx(t)}{dt} \\
\frac{dy(t)}{dt} \\
\frac{dz(t)}{dt}
\end{pmatrix} =
\begin{pmatrix}
-\frac{G_m}{\left[x^2(t) + y^2(t) + z^2(t)\right]^{3/2}}x(t) \\
-\frac{G_m}{\left[x^2(t) + y^2(t) + z^2(t)\right]^{3/2}}y(t) \\
-\frac{G_m}{\left[x^2(t) + y^2(t) + z^2(t)\right]^{3/2}}z(t) \\
v_x(t) \\
v_y(t) \\
v_z(t)
\end{pmatrix}
\tag{5.1-3}
\]

Step2:在步长 \( h \) 条件下建立如下的模型公式:

\[
\begin{cases}
G(t+1) = G(t) + \frac{h}{6}(K_1 + 2K_2 + 2K_3 + K_4) \\
K_1 = f(G(t)) \\
K_2 = f(G(t) + \frac{h}{2}K_1) \\
K_3 = f(G(t) + \frac{h}{2}K_2) \\
K_4 = f(G(t) + hK_3)
\end{cases}
\tag{5.1-4}
\]

Step3:尝试以 \( h/2 \) 为步长积分一次,结果记为 \( \{Y_{i,h/2}^{t+1/2}\}_{i=1}^N \),再将步长连续计算三步得到同一点的值记为 \( \{Y_{i,h/6}^{t+1/2}\}_{i=1}^N \);

Step4:记 \( Er = \max \left| \frac{Y_{i,h/6}^{t+1/2} - Y_{i,h/2}^{t+1/2}}{\varepsilon \cdot Yal(i)} \right| \),利用 St.1 和 St.2 计算出 \( Er \),其中 \( \varepsilon \) 为误差限,

\begin{equation}
Yal(i) = \left| Y_{i}^{t} \right| + \left| h \cdot f_{i} \left( G_{k}, Y^{t} \right) \right|;
\end{equation}

Step5: 判断 $Er \leq 1$ 是否成立,若成立则表示满足精度要求,此步积分成功;否则缩小步长 $h = \frac{s \cdot h \cdot Er^{-0.25}}{2}$,重新带入进行 St.3 计算。

Step6: 在截断中考虑到四阶方法的截断误差,利用两次的计算结果做周期外插,得到 $Y^{t+1/2}$ 和 $Y^{t}$ 与 $Y^{t+1/2}$ 之间得 $Y_{i,h/3}^{t+1/2}$、$Y_{i,h/6}^{t+1/2}$ 值。$Y^{(t+1/2)} = \left\{ Y_{i,h/6}^{t+1/2} + \frac{1}{15} \left( Y_{i,h/3}^{t+1/2} - Y_{i,h/6}^{t+1/2} \right) \right\}_{i=1}^{N}$

Step7: 计算 $Y^{(t+1)} = Y^{(t+1/2)} + Y_{i,h/6}^{t+1/2} + \frac{1}{15} \left( Y_{i,h/3}^{t+1/2} - Y_{i,h/6}^{t+1/2} \right)$。这一结果能达到 5 阶的精度要求。为了选取下一步的尝试积分步长,算法中要选取误差界 $Er_{0}$。

Step8: 判断 $Er > Er_{0}$ 是否成立,成立则取 $\frac{s \cdot h \cdot Er^{-0.25}}{2}$,否则步长为 $3h/2$。

Step9: 通过表 5.1 的初值;依次通过 St.1-St.7 迭代解得微分方程的解。利用 AFP-4-OSRK 算法保持了采样时刻的取得更精确的值,从而通过卫星轨迹的更精准的估计为空间飞行器的无源探测做好较高精度的铺垫。

\subsubsection{5.1.2 计算结果及分析}

根据初始数据表 5.1 和步长 $h = 0.2$,$Er_{0} \leq 1 \times 10^{-6}$ 条件,$Er_{0}$ 的取值来源于题目所要最终结果保留 6 位有效数字。利用 Matlab 编程求得在 50s、100s、150s、200s、250s 的三维位置,具体结果如表 5.2 所示。

\begin{table}[h]
\centering
\caption{50s、100s、150s、200s、250s 的三维位置}
\begin{tabular}{|c|c|c|c|}
\hline
时刻/s & X 方向大小/m & Y 方向大小/m & Z 方向大小/m \\
\hline
50 & $1.77381 \times 10^{6}$ & $8.16138 \times 10^{6}$ & $4.51669 \times 10^{6}$ \\
\hline
100 & $1.50163 \times 10^{6}$ & $8.12676 \times 10^{6}$ & $4.68468 \times 10^{6}$ \\
\hline
150 & $1.22770 \times 10^{6}$ & $8.08270 \times 10^{6}$ & $4.84722 \times 10^{6}$ \\
\hline
200 & $9.52349 \times 10^{5}$ & $8.02925 \times 10^{6}$ & $5.00413 \times 10^{6}$ \\
\hline
250 & $6.75894 \times 10^{5}$ & $7.96650 \times 10^{6}$ & $5.15524 \times 10^{6}$ \\
\hline
\end{tabular}
\end{table}

利用 AFP-4-OSRK 算法求解卫星运动微分方程模型,得到 09 号卫星的轨迹(从 $t=0$s 至 $t=250$s)如图 5.1(a) 所示,其对应时刻的速度如图 5.1(b) 所示。从图 5.1(a) 可以看出 09 号卫星的空间轨迹近似一空间连续曲线。而从图 5.1(b) 和图 5.1(d) 可以看出 09 号卫星的速度随时间是连续变化的,速度沿 X 轴方向上变化最小,沿 Y 轴变化的幅度最大。图 5.1(c) 表明卫星在 Y 轴方向上的分量几乎不变,即卫星的运动轨道平面近乎平行于 XOZ

面。

\begin{figure}[h]
    \centering
    \includegraphics[width=\textwidth]{image.png}
    \caption{09号卫星从t=0s至t=250s间隔1s的空间轨迹和速度曲线}
    \label{fig:5.1}
\end{figure}

\subsection{5.2 问题二的建模与求解}

利用 5.1 节中的卫星轨迹估计方法求出 06 号卫星的轨迹,结果如图 5.2 所示。由于附件给出的不同卫星观测数据时间上不同步,如 meadata_06_00.txt 中观测的初始时刻为 50.175444807276s,而 meadata_06_00.txt 中观测的初始时刻为 50.181218909269s,所以对 09 与 06 号卫星的观测数据要进行数据处理,使得所有观测数据实现同步,从而确定飞行器同一时刻在不同卫星中的观测数据,完成双星逐点交汇定位。其次,根据 $\vec{v}_{r}(t)$ 和 $m(t)$ 的物理意义和变化规律,对二者分别建立了含有未知参量的近似模型,完善了主动段空间飞行器的运动方程,直接将飞行器的轨迹、速度和加速度数据代入运动方程可

实现模型中未知参数的估计,为简化参数估计,提出了一种基于级数理论的多项式模型来拟合 $\bar{\nu}_{r}(t) \dot{m}(t) / m(t)$ 这一整体,给出了对多项式各阶次系数的估计方法,并对不同阶次多项式模型的性能进行了分析对比。再次,通过双星逐点交汇定位得出的飞行器位置对建立的模型中各个参数进行估计,得到基础坐标系下的主动段的简化运动方程的明确形式,然后采用四阶龙格库塔来估计出飞行器主动段的轨道。利用估计出的飞行器轨迹与卫星观测到的轨迹进行比较,计算出残差。该问题可以分为以下步骤:卫星观测数据同步、双星逐点交汇定位、参量建模与估计、飞行器轨迹估计建模和残差计算。

\begin{figure}[h]
    \centering
    \includegraphics[width=\textwidth]{image.png}
    \caption{06号卫星从 $t=0\,\mathrm{s}$ 至 $t=250\,\mathrm{s}$ 间隔 $1\,\mathrm{s}$ 的空间轨迹和速度曲线}
    \label{fig:5.2}
\end{figure}

\subsubsection{5.2.1 卫星观测数据同步}

\subsubsection{5.2.1.1 样条插值}

对异步数据的同步算法主要有最小二乘拟合、多项式拟合、非线性拟合、拉格朗日插值、样条插值等,各种算法都有相应的优缺点。采用样条插值绘制的曲线不仅有很好的光滑度,而且当节点逐渐加密时,其函数值在整体上能很好地逼近被插函数,相应的导数值也收敛于被插函数的导数,不会发生龙格现象。因此采用样条插值法对卫星观测数据进行同步。

根据附件 meadata\_i\_j.txt 给出的 06 与 09 号卫星的不同时刻观测数据,分别画出 06 与 09 号卫星各观测时刻的观测量 $\alpha$、$\beta$ 的散点图,如图 \ref{fig:5.3} 所示。

\begin{figure}[h]
    \centering
    \includegraphics[width=\textwidth]{image.png}
    \caption{06 与 09 号卫星各观测时刻的观测量 $\alpha$、$\beta$ 的散点图}
    \label{fig:5.3}
\end{figure}

\begin{figure}[h]
    \centering
    \begin{subfigure}[b]{0.45\textwidth}
        \includegraphics[width=\textwidth]{fig_a.png}
        \caption{09 号卫星观测量 $\alpha$ 的散点图}
    \end{subfigure}
    \hfill
    \begin{subfigure}[b]{0.45\textwidth}
        \includegraphics[width=\textwidth]{fig_b.png}
        \caption{09 号卫星观测量 $\beta$ 的散点图}
    \end{subfigure}
    \hfill
    \begin{subfigure}[b]{0.45\textwidth}
        \includegraphics[width=\textwidth]{fig_c.png}
        \caption{06 号卫星观测量 $\alpha$ 的散点图}
    \end{subfigure}
    \hfill
    \begin{subfigure}[b]{0.45\textwidth}
        \includegraphics[width=\textwidth]{fig_d.png}
        \caption{06 号卫星观测量 $\beta$ 的散点图}
    \end{subfigure}
    \caption{06 与 09 号卫星观测量 $\alpha$、$\beta$ 的散点图}
    \label{fig:scatter_plots}
\end{figure}

采用样条插值法对卫星观测数据进行同步,这里以三次样条插值的为例,具体的步骤如下:

\textbf{Step1:} 给定区间 $[a, b]$ 的一个划分 $a = t_0 < t_1 < \cdots < t_{n-1} < t_n = b$,$\alpha(t)$ 是 $\alpha$ 的实际函数 $f_\alpha(t)$ 在该区间上关于该划分的一个三次样条函数。设 $\alpha(t)$ 在节点 $t_i$ 处的二阶导数为:
\[
\alpha''(t_i) = M_i \quad (i = 0, 1, \cdots, n),
\]
因为在子区间 $[t_{i-1}, t_i]$ 上 $\alpha(t) = \alpha_i(t)$ 是三次多项式,所以 $\alpha''(t)$ 在此小区间上是 $t$ 的线性函数,且因为用线性插值,可知其表达式为:
\[
\alpha''(t_{i-1}) = M_{i-1}, \quad \alpha''(t_i) = M_i \quad t \in [t_{i-1}, t_i]
\]
\[
\alpha''_i(t) = M_{i-1} \frac{t_i - t}{t_{i-1} - t_i} + M_i \frac{t - t_{i-1}}{t_i - t_{i-1}}
\]
记 $h_i = t_i - t_{i-1}$,则有 $\alpha''_i(t) = M_{i-1} \frac{t_i - t}{h_i} + M_i \frac{t - t_{i-1}}{h_i}$ 连续两次积分得:
\[
\alpha_i(t) = M_{i-1} \frac{(t_i - t)^3}{6h_i} + M_i \frac{(t - t_{i-1})^3}{6h_i} + A_i(t_i - t) + B_i(t - t_{i-1})
\]
\begin{equation}
\tag{5.2-1}
\end{equation}

其中,$A_{i}, B_{i}$ 为积分常数,可利用插值条件确定,即要求 $A_{i}, B_{i}$ 满足
\[
\alpha(t_{i-1}) = \frac{1}{6} M_{i-1} h_{i}^{2} + A_{i} h_{i} = f_{\alpha}(t_{i-1}), \quad \alpha(t_{i}) = \frac{1}{6} M_{i} h_{i}^{2} + B_{i} h_{i} = f_{\alpha}(t_{i})
\]
并记 $f_{\alpha}(t_{i-1}) = y_{i-1}, f_{\alpha}(t_{i}) = y_{i}$,则得
\[
A_{i} = \frac{1}{h_{i}} \left( y_{i-1} - \frac{1}{6} M_{i-1} h_{i}^{2} \right)
\]
\[
B_{i} = \frac{1}{h_{i}} \left( y_{i} - \frac{1}{6} M_{i} h_{i}^{2} \right)
\]
\[
\alpha_{i}(t) = M_{i-1} \frac{(t_{i} - t)^{3}}{6 h_{i}} + M_{i} \frac{(t - t_{i-1})^{3}}{6 h_{i}} + \left( y_{i-1} - \frac{M_{i-1}}{6} h_{i}^{2} \right) \frac{(t_{i} - t)}{h_{i}} + \left( y_{i} - \frac{M_{i}}{6} h_{i}^{2} \right) \frac{(t - t_{i-1})}{h_{i}}
\]
\[
(t \in \left[ t_{i-1}, t_{i} \right], i = 1, 2, \ldots, n)
\]

Step2:由上讨论可知,只要确定这 $n+1$ 个值,就可定出三样条插值函数 $\alpha(t)$。为了求出 $M_{i} (i = 0, 1, \ldots, n)$,利用一阶导数在子区间连接点上连续的条件 $\alpha'(t_{i} - 0) = \alpha'(t_{i} + 0)$,求导一次,得在区间 $\left[ t_{i-1}, t_{i} \right]$ 上的表达式为:
\[
\alpha'_{i}(t) = -M_{i-1} \frac{(t_{i} - t)^{2}}{2 h_{i}} + M_{i} \frac{(t - t_{i-1})^{2}}{2 h_{i}} + \frac{y_{i} - y_{i-1}}{h_{i}} - \frac{h_{i}}{6} (M_{i} - M_{i-1}) \tag{5.2-2}
\]
也就是在右端点 $t_{i}$ 上有:
\[
\alpha'_{i}(t_{i} - 0) = \frac{h_{i}}{2} M_{i} - \frac{h_{i}}{6} (M_{i} - M_{i-1}) + \frac{y_{i} - y_{i-1}}{h_{i}} = \frac{h_{i}}{6} M_{i-1} + \frac{h_{i}}{3} M_{i} + \frac{y_{i} - y_{i-1}}{h_{i}}
\]
在左端点 $t_{i+1}$ 上有:
\[
\alpha'_{i}(t_{i-1} + 0) = -\frac{h_{i}}{2} M_{i-1} - \frac{h_{i}}{6} (M_{i} - M_{i-1}) + \frac{y_{i} - y_{i-1}}{h_{i}} = -\frac{h_{i}}{3} M_{i-1} - \frac{h_{i}}{6} M_{i} + \frac{y_{i} - y_{i-1}}{h_{i}}
\]

Step3:将上式中的 $i-1$ 改为 $i$,即得在子区间 $\left[ t_{i-1}, t_{i} \right]$ 上的表达式 $\alpha'_{i+1}(t)$,并由此得
\[
\alpha'_{i+1}(t_{i} + 0) = -\frac{h_{i+1}}{3} M_{i} - \frac{h_{i+1}}{6} M_{i+1} + \frac{y_{i+1} - y_{i}}{h_{i+1}}
\]
利用 $\alpha'(t)$ 在内接点的连续性,即 $\alpha'_{i}(t_{i} - 0) = \alpha'_{i+1}(t_{i} + 0)$ 就可得到关于参数的 $M_{i-1}, M_{i}, M_{i+1}$ 一个方程:
\[
\frac{h_{i}}{6} M_{i-1} + \frac{h_{i} + h_{i+1}}{3} M_{i} + \frac{h_{i+1}}{6} M_{i+1} = \frac{y_{i+1} - y_{i}}{h_{i+1}} - \frac{y_{i} - y_{i-1}}{h_{i}}
\]
上式两边同乘以 $\frac{6}{h_{i}+h_{i+1}}$,即得方程

\[
\frac{h_{i}}{h_{i}+h_{i+1}} M_{i-1}+2 M_{i}+\frac{h_{i+1}}{h_{i}+h_{i+1}} M_{i+1}=\frac{6}{h_{i}+h_{i+1}}\left(\frac{y_{i+1}-y_{i}}{h_{i+1}}-\frac{y_{i}-y_{i-1}}{h_{i}}\right)(i=1,2, \cdots, n-1)
\]

若记 $\left\{\begin{array}{l}\mu_{i}=\frac{h_{i}}{h_{i}+h_{i+1}} \\\lambda_{i}=\frac{h_{i+1}}{h_{i}+h_{i+1}}=1-\mu_{i} \\g_{i}=\frac{6}{h_{i}+h_{i+1}}\left(f_{\alpha}\left[t_{i}, t_{i+1}\right]-f_{\alpha}\left[t_{i-1}, t_{i}\right]\right)\end{array}\right.$

则所得方程可简写成:

\[
\mu_{i} M_{i-1}+2 M_{i}+\lambda_{i} M_{i+1}=g_{i} \quad(i=1,2, \cdots, n-1)
\]

即:

\[
\left\{\begin{array}{l}\mu_{1} M_{0}+2 M_{1}+\lambda_{1} M_{2}=g_{1} \\\mu_{2} M_{1}+2 M_{2}+\lambda_{2} M_{3}=g_{2} \\\cdots \\\mu_{n-1} M_{n-2}+2 M_{n-1}+\lambda_{n-1} M_{n}=g_{n-1}\end{array}\right.
\tag{5.2-3}
\]

这是一个含有 $n+1$ 个未知数、$n-1$ 个方程的线性方程组。要完全确定 $M_{i}(i=0,1, \cdots, n)$ 的值还需要补充两个条件,这两个条件通常根据实际问题的需要,根据插值区间 $[a, b]$ 的两个端点处的边界条件来补充。即已知插值区间两端的一阶导数值 $\alpha^{\prime}\left(t_{0}\right)=f_{\alpha}^{\prime}\left(t_{0}\right), \alpha^{\prime}\left(t_{n}\right)=f_{\alpha}^{\prime}\left(t_{n}\right)$,则可得到包含 $\alpha_{i}$ 的两个线性方程,$\alpha(t)$ 在子区间 $\left[t_{0}, t_{1}\right]$ 上的导数为

\[
\alpha_{1}^{\prime}(\alpha)=-M_{0} \frac{\left(t_{1}-t\right)^{2}}{2 h_{1}}+M_{1} \frac{\left(t-t_{0}\right)^{2}}{2 h_{1}}+\frac{y_{1}-y_{0}}{h_{1}}-\frac{h_{1}}{6}\left(M_{1}-M_{0}\right)
\]

由条件 $\alpha^{\prime}\left(t_{0}\right)=f_{\alpha}^{\prime}\left(x_{0}\right)=y_{0}^{\prime}$ 得

\[
y_{0}^{\prime}=-M_{0} \frac{h_{1}}{2}+\frac{y_{1}-y_{0}}{h_{1}}-\frac{h_{1}}{6}\left(M_{1}-M_{0}\right)
\]

即 $2 M_{0}+M_{1}=\frac{6}{h_{1}}\left(\frac{y_{1}-y_{0}}{h_{1}}-y_{0}^{\prime}\right)$

Step4:同理由条件 $\alpha^{\prime}\left(t_{n}\right)=f_{\alpha}^{\prime}\left(t_{n}\right)=y_{n}^{\prime}$ 得

\[
M_{n-1}+2 M_{n}=\frac{6}{h_{n}}\left(y_{n}^{\prime}-\frac{y_{n}-y_{n-1}}{h_{n}}\right)
\]

即得确定 $M_{0}, M_{1}, \cdots, M_{n}$ 的线性方程组

\[
\left[\begin{array}{cccccc}2 & 1 & & & & \\ \mu_{1} & 2 & \lambda_{1} & & & \\ & \ddots & \ddots & \ddots & & \\ & & \mu_{n-1} & 2 & \lambda_{n-1} & \\ & & & 1 & 2 &\end{array}\right]\left[\begin{array}{c}M_{0} \\ M_{1} \\ \vdots \\ M_{n-1} \\ M_{n}\end{array}\right]=\left[\begin{array}{c}g_{0} \\ g_{1} \\ \vdots \\ g_{n-1} \\ g_{n}\end{array}\right]
\tag{5.2-4}
\]

其中,
\[
\begin{cases}
g_{0}=\frac{6}{h_{1}}\left(f_{\alpha}\left[t_{0}, t_{1}\right]-y_{0}^{\prime}\right) \\
g_{n}=\frac{6}{h_{n}}\left(y_{n}^{\prime}-f_{\alpha}\left[t_{n-1}, t_{n}\right]\right)
\end{cases}
\]

利用线性代数知识, 可以证明方程组的系数矩阵都是非奇异的, 因此有惟一解。求解确定 \(\alpha(t)\)。同理可求 \(\beta(t)\)。对 06 与 09 号卫星的观测数据分别进行样条插值, 完成数据同步。

\subsection{5.2.1.2 计算结果及分析}

对附件中的两个卫星观测数据进行同步处理时, 分别选取不同次数 \(n\) 的样条插值对其插值, 并对不同次数 \(n\) 样条插值得到的数值与真实数值的偏差即进行比较, 最后确认较优的 \(n\) 值。

\begin{table}[h]
\centering
\caption{09 号卫星观测数据不同次数插值结果分析}
\begin{tabular}{|c|c|c|c|c|}
\hline
\(\alpha\) & \(n=3\) & \(n=4\) & \(n=5\) & \(n=6\) \\
\hline
插值偏差和 & 0.0206241 & 0.00258375 & 0.00145688 & 0.00141936 \\
\hline
插值偏差均值 & 3.43735e-005 & 4.30624e-006 & 2.42813e-006 & 2.36561e-006 \\
\hline
插值偏差方差 & 4.41193e-010 & 9.51605e-012 & 3.47706e-012 & 3.30412e-012 \\
\hline
\end{tabular}
\end{table}

根据表 5.3 中的数据可以看出当次数取到 4 时插值偏差的均值的数量级就保持不变, 所以综合考虑精度与复杂度对观测数据进行 \(n=4\) 次的数据较优。

\begin{table}[h]
\centering
\caption{09 号卫星观测数据不同次数插值结果分析}
\begin{tabular}{|c|c|c|c|c|}
\hline
\(\beta\) & \(n=1\) & \(n=2\) & \(n=3\) & \(n=4\) \\
\hline
插值偏差和 & 0.921839 & 0.0625451 & 0.00165505 & 0.00160726 \\
\hline
插值偏差均值 & 0.001536399 & 1.04242e-004 & 2.75841e-006 & 2.678768e-006 \\
\hline
插值偏差方差 & 8.33924e-007 & 3.90181e-009 & 4.88237e-012 & 4.52322e-012 \\
\hline
\end{tabular}
\end{table}

根据表 5.4 中的数据可以看出当次数取到 3 时插值偏差的均值的数量级就保持不变, 所以综合考虑精度与复杂度对观测数据进行 \(n=3\) 次的数据较优。

\begin{table}[h]
\centering
\caption{06 号卫星观测数据不同次数插值结果分析}
\begin{tabular}{|c|c|c|c|c|}
\hline
\(\alpha\) & \(n=3\) & \(n=4\) & \(n=5\) & \(n=6\) \\
\hline
插值偏差和 & 0.00858123 & 0.00191533 & 0.00144179 & 0.00141628 \\
\hline
插值偏差均值 & 1.43021e-005 & 3.19222e-006 & 2.40299e-006 & 2.36047e-006 \\
\hline
插值偏差方差 & 7.86630e-011 & 6.35052e-012 & 2.98034e-012 & 3.11212e-012 \\
\hline
\end{tabular}
\end{table}

根据表 5.5 中的数据可以看出当次数取到 4 时插值偏差的均值的数量级就保持不变, 所以综合考虑精度与复杂度对观测数据进行 \(n=4\) 次的数据较优。

\begin{table}
\centering
\caption{06 号卫星观测数据不同次数插值结果分析}
\begin{tabular}{|c|c|c|c|c|}
\hline
$\beta$ & $n=3$ & $n=4$ & $n=5$ & $n=6$ \\
\hline
插值偏差和 & 0.0112285 & 0.00220244 & 0.00141590 & 0.00141628 \\
\hline
插值偏差均值 & 1.87142e-005 & 3.67073e-006 & 2.35983e-006 & 2.36047e-006 \\
\hline
插值偏差方差 & 1.30330e-010 & 6.93551e-012 & 3.13215e-012 & 3.11212e-012 \\
\hline
\end{tabular}
\end{table}

根据表 5.6 中的数据可以看出当次数取到 4 时插值偏差的均值的数量级就保持不变,所以综合考虑精度与复杂度对观测数据进行 $n=4$ 次的数据较优。

\subsection{5.2.2 双星逐点交汇定位的建模与求解}

在卫星无源探测系统中,目标量测所在的观测坐标系和数据处理所在的基础坐标系并不一致,为将数据处理信息进行统一,需推导出观测坐标系与基础坐标系的转换关系。

\subsubsection{5.2.2.1 坐标转换建模}

建立基础坐标系如图 5.4 所示。其中 $O_{c}-X_{c}Y_{c}Z_{c}$ 坐标系为基础坐标系。$O_{s}-X_{s}Y_{s}Z_{s}$ 为观察坐标系。

\begin{figure}[h]
\centering
\includegraphics[width=0.8\textwidth]{image.png}
\caption{观测坐标系示意图}
\end{figure}

对于在轨道上的某一卫星中心点 $O_{s}$,在基础坐标系下的坐标为 $(x_{0}, y_{0}, z_{0})$,$\overrightarrow{O_{c}O_{s}} = (x_{0}, y_{0}, z_{0})$。对于观测坐标系下的三个方向的单位向量为

\begin{equation}
\vec{e}_{x_{s}} =
\begin{pmatrix}
\frac{x_{0}}{r_{0}} \\
\frac{y_{0}}{r_{0}} \\
\frac{z_{0}}{r_{0}}
\end{pmatrix}, \quad
\vec{e}_{z_{s}} =
\begin{pmatrix}
\frac{-x_{0}z_{0}}{r_{0}\rho_{0}} \\
\frac{-y_{0}z_{0}}{r_{0}\rho_{0}} \\
\frac{\rho_{0}}{r_{0}}
\end{pmatrix}, \quad
\vec{e}_{x_{y}} = \vec{e}_{x_{z}} \times \vec{e}_{z_{s}} =
\begin{pmatrix}
-\frac{y_{0}}{\rho_{0}} \\
\frac{x_{0}}{\rho_{0}} \\
0
\end{pmatrix}
\tag{5.2-5}
\end{equation}

其中 $r_{0} = \sqrt{x_{0}^{2} + y_{0}^{2} + z_{0}^{2}}$,$\rho_{0} = \sqrt{x_{0}^{2} + y_{0}^{2}}$。

推导过程如下:
\[
\vec{e}_{x_{s}}=\frac{\overrightarrow{O_{c} O_{s}}}{\left|\overrightarrow{O_{c} O_{s}}\right|}, \quad \vec{e}_{z_{s}}=\frac{\left(0,0, z_{z_{s}}\right)-\left(x_{0}, y_{0}, z_{0}\right)}{\sqrt{\left(x_{0}-0\right)^{2}+\left(y_{0}-0\right)^{2}+\left(z_{z_{s}}-z_{0}\right)^{2}}}
\]

点 \((0,0,z_{z_{s}})\) 表示以 \(\overrightarrow{O_{c} O_{s}}\) 为法向量,过 \(O_{s}\left(x_{0}, y_{0}, z_{0}\right)\) 点的平面与 \(O_{c}-X_{c} Y_{c} Z_{c}\) 坐标系下的 \(\overline{O_{c} Z_{c}}\) 的交点,平面方程为
\[
x_{0}\left(x-x_{0}\right)+y_{0}\left(y-y_{0}\right)+z_{0}\left(z-z_{0}\right)=0
\tag{5.2-6}
\]

计算得到 \(z_{z_{s}}=\frac{r_{0}^{2}}{z_{0}}\),由于观测坐标系对于 \(\overline{O_{s} Z_{s}}\) 始终指向正北,因此,
\[
\vec{e}_{z_{s}}=\frac{\left(-x_{0},-y_{0}, \frac{r_{0}^{2}}{z_{0}}-z_{0}\right)}{\sqrt{\left(-x_{0}\right)^{2}+\left(-y_{0}\right)^{2}+\left(\frac{r_{0}^{2}}{z_{0}}-z_{0}\right)^{2}}}=\frac{\left(-z_{0} x_{0},-z_{0} y_{0}, \rho_{0}^{2}\right)}{\sqrt{\left(-z_{0} x_{0}\right)^{2}+\left(-z_{0} y_{0}\right)^{2}+\left(\rho_{0}^{2}\right)^{2}}}
\]
\[
=\frac{\left(-z_{0} x_{0},-z_{0} y_{0}, \rho_{0}^{2}\right)}{r_{0} \rho_{0}}
\tag{5.2-7}
\]

由上面坐标转换可以得到,对于某一卫星观测的空间飞行器坐标 \(M\left(x_{s}, y_{s}, z_{x}\right)\),它在观察坐标系下的向量为 \(\overrightarrow{O_{s} M}=\left(x_{s}, y_{s}, z_{s}\right)^{T}\),则在基础坐标系下的向量为 \(\overrightarrow{O_{c} M}\):
\[
\overrightarrow{O_{c} M}=\overrightarrow{O_{c} O_{s}}+\overrightarrow{O_{s} M}
\]
\[
\Rightarrow\left(\begin{array}{l}
x_{c} \\
y_{c} \\
z_{c}
\end{array}\right)=\left(\begin{array}{l}
x_{0} \\
y_{0} \\
z_{0}
\end{array}\right)+\left(\begin{array}{lll}
\vec{e}_{x_{s}} & \vec{e}_{x_{y}} & \vec{e}_{x_{y}}
\end{array}\right)\left(\begin{array}{l}
x_{s} \\
y_{s} \\
z_{s}
\end{array}\right)=\left(\begin{array}{l}
x_{0} \\
y_{0} \\
z_{0}
\end{array}\right)+x_{s} \boldsymbol{E}\left(\begin{array}{l}
1 \\
\alpha \\
\beta
\end{array}\right)
\tag{5.2-8}
\]
其中 \(\boldsymbol{E}=\left(\begin{array}{ccc}
\frac{x_{0}}{r_{0}} & \frac{-y_{0}}{\rho_{0}} & \frac{-x_{0} z_{0}}{r_{0} \rho_{0}} \\
\frac{y_{0}}{r_{0}} & \frac{x_{0}}{\rho_{0}} & \frac{-y_{0} z_{0}}{r_{0} \rho_{0}} \\
\frac{z_{0}}{r_{0}} & 0 & \frac{\rho_{0}}{r_{0}}
\end{array}\right)\),\(\alpha\)、\(\beta\) 是题目所给出的无量纲比值。因此在观测坐标系下的坐标点 \(\left(x_{s}, y_{s}, z_{s}\right)\) 转换到基础坐标系下的点 \(\left(x_{c}, y_{c}, z_{c}\right)\),转换公式为:
\[
\left(\begin{array}{l}
x_{c} \\
y_{c} \\
z_{c}
\end{array}\right)=\left(\begin{array}{l}
x_{0} \\
y_{0} \\
z_{0}
\end{array}\right)+\boldsymbol{E}\left(\begin{array}{l}
x_{s} \\
y_{s} \\
z_{s}
\end{array}\right)=\left(\begin{array}{l}
x_{0} \\
y_{0} \\
z_{0}
\end{array}\right)+x_{s} \boldsymbol{E}\left(\begin{array}{l}
x_{s} \\
y_{s} \\
z_{s}
\end{array}\right)
\tag{5.2-9}
\]

\subsubsection{5.2.2.2 双星逐点交汇定位的建模}

在 09 号和 06 号卫星无源定位时,通过基于样条差值拟合的数据同步算法实现了两卫星观测数据的同步。设某一时刻空间飞行器在基础坐标系下的位置为 $(x_{c}, y_{c}, z_{c})^{T}$,利用公式 (5.2-8),可以得到在 06 号和 09 号卫星下的坐标关系:

\begin{equation}
\left\{
\begin{aligned}
\begin{pmatrix}
x_{c} \\
y_{c} \\
z_{c}
\end{pmatrix}
& =
\begin{pmatrix}
x_{09} \\
y_{09} \\
z_{09}
\end{pmatrix}
+ x_{09s} \boldsymbol{E}
\begin{pmatrix}
1 \\
\alpha_{09} \\
\beta_{09}
\end{pmatrix}
\\
\begin{pmatrix}
x_{c} \\
y_{c} \\
z_{c}
\end{pmatrix}
& =
\begin{pmatrix}
x_{06} \\
y_{06} \\
z_{06}
\end{pmatrix}
+ x_{06s} \boldsymbol{E}
\begin{pmatrix}
1 \\
\alpha_{06} \\
\beta_{06}
\end{pmatrix}
\end{aligned}
\right.
\tag{5.2-10}
\end{equation}

将式 (5.2-10) 联立化简为 $A \boldsymbol{x} = \boldsymbol{b}$ 的形式有:

\begin{equation}
\begin{pmatrix}
1 & 0 & 0 & -\frac{x_{09}}{r_{09}} & \frac{y_{09}}{\rho_{09}} & \frac{x_{09} z_{09}}{r_{09} \rho_{09}} & 0 & 0 & 0 \\
0 & 1 & 0 & -\frac{y_{09}}{r_{09}} & -\frac{x_{09}}{\rho_{09}} & \frac{y_{09} z_{09}}{r_{09} \rho_{09}} & 0 & 0 & 0 \\
0 & 0 & 1 & -\frac{z_{09}}{r_{09}} & 0 & -\frac{\rho_{09}}{r_{09}} & 0 & 0 & 0 \\
1 & 0 & 0 & 0 & 0 & 0 & -\frac{x_{06}}{r_{06}} & \frac{y_{06}}{\rho_{06}} & \frac{x_{06} z_{06}}{r_{06} \rho_{06}} \\
0 & 1 & 0 & 0 & 0 & 0 & -\frac{y_{06}}{r_{06}} & -\frac{x_{06}}{\rho_{06}} & \frac{y_{06} z_{06}}{r_{06} \rho_{06}} \\
0 & 0 & 1 & 0 & 0 & 0 & -\frac{z_{06}}{r_{06}} & 0 & -\frac{\rho_{06}}{r_{06}} \\
0 & 0 & 0 & \alpha_{09} & -1 & 0 & 0 & 0 & 0 \\
0 & 0 & 0 & \beta_{09} & 0 & -1 & 0 & 0 & 0 \\
0 & 0 & 0 & 0 & 0 & 0 & \alpha_{06} & -1 & 0 \\
0 & 0 & 0 & 0 & 0 & 0 & \beta_{06} & 0 & -1
\end{pmatrix}
\begin{pmatrix}
x_{c} \\
y_{c} \\
z_{c} \\
x_{09s} \\
y_{09s} \\
z_{09s} \\
x_{06s} \\
y_{06s} \\
z_{06s}
\end{pmatrix}
=
\begin{pmatrix}
x_{09} \\
y_{09} \\
z_{09} \\
x_{06} \\
y_{06} \\
z_{06} \\
0 \\
0 \\
0
\end{pmatrix}
\tag{5.2-11}
\end{equation}

其中 $(x_{09s}, y_{09s}, z_{09s})^{T}$ 和 $(x_{06s}, y_{06s}, z_{06s})^{T}$ 分别表示该时刻空间飞行器通过 06 号和 09 号卫星在各自观测坐标系下估计出的位置。代入数值,可以得到通过双星观测条件下的空间飞行器基础坐标系的空间位置。

采用最小二乘法 [3] 解得:

\begin{equation}
\boldsymbol{x} = \left( A^{T} A \right)^{-1} A^{T} \boldsymbol{b}
\tag{5.2-12}
\end{equation}

但是由于卫星在观测过程中,随机扰动的影响是不可避免的。因此在实际求解过程中考虑观测的噪声干扰,并且假设噪声的各分量统计独立,为了保证结果的可靠性,这

里同时给出了总体最小二乘法$^{[4]}$的具体求解过程:

Step1: 对方程 $Ax=b$,构造增广矩阵 $\boldsymbol{C}$,$\boldsymbol{C}=[-b:A]$;

Step2: 对增广矩阵 $\boldsymbol{C}$ 进行奇异值分解得到:
\begin{equation}
\boldsymbol{C}=\boldsymbol{U} \boldsymbol{\Sigma} \boldsymbol{V}^{H}=\sum_{j=1}^{10} \sigma_{j} \boldsymbol{u}_{1 j} \boldsymbol{v}_{1 j}^{H}
\tag{5.2-13}
\end{equation}
其中,$\boldsymbol{u}_{1 j}$ 为矩阵 $\boldsymbol{U}$ 的第 $j$ 列,$\boldsymbol{v}_{1 j}$ 为矩阵 $\boldsymbol{V}$ 的第 $j$ 列,$\sigma_{1 j}$ 为所对应的奇异值,并且假设奇异值按照递减的顺序排列:$\sigma_{1} \geq \sigma_{2} \geq \sigma_{3} \geq \sigma_{4} \geq \sigma_{5} \geq \sigma_{6} \geq \sigma_{7} \geq \sigma_{8} \geq \sigma_{9} \geq \sigma_{10}$;

Step3: 当无测量误差时,则有 $rank\left(\boldsymbol{C}\right)=rank\left(\boldsymbol{H}\right)=9$,此时 $\sigma_{10}=0$,并且(5.2-11)式存在唯一解;当存在测量误差时,则有 $rank\left(C\right)=10$,此时 $\sigma_{10} \neq 0$,但当测量误差不是很大时,满足 $\sigma_{1} \geq \sigma_{2} \geq \sigma_{3} \geq \sigma_{4} \geq \sigma_{5} \geq \sigma_{6} \geq \sigma_{7} \geq \sigma_{8} \geq \sigma_{9} \geq \sigma_{10}$,此时(5.2-11)式存在唯一的总体最小二乘意义下的解为:
\begin{equation}
\vec{X}_{TLS}=\left[\boldsymbol{v}_{10}(2) / \boldsymbol{v}_{10}(1), \boldsymbol{v}_{10}(3) / \boldsymbol{v}_{10}(1), \cdots, \boldsymbol{v}_{10}(10) / \boldsymbol{v}_{10}(1)\right]
\tag{5.2-14}
\end{equation}
通过以上步骤即可得到方程组(5.2-11)式的解。在此基础上利用中心差分方法计算得到的速度公式,即:
\begin{equation}
v(t)=\frac{s(t+\Delta t)-s(t-\Delta t)}{2 \Delta t}
\tag{5.2.15}
\end{equation}
从而得到了 $t$ 时刻速度的观测估计值。

\subsubsection{5.2.2.3 计算结果及分析}

利用最小二乘法(Least Squares, LS)和总体最小二乘法(Total Least Squares, TLS)求解 5.2.2.2 的模型。初始时刻的 06 号卫星、09 号卫星的位置由题中提供的附录 satinfo 得到。双星在 $0 \mathrm{~s}$ 至 $170 \mathrm{~s}$ 的轨迹结果如图 5.5 所示,通过 LS 和 TLS 计算得到的 0 号空间飞行器的轨迹估计如图 5.6 所示。从图中的估计轨迹可看到这两个方法计算的结果十分接近。图 5.7 分别两种算法估计差值的绝对值,从图可以看出估计轨迹在 $Y$ 轴方向上的变化最小,说明随机扰动对 $Y$ 轴方向上的位置分量影响最小。

\begin{figure}[h]
    \centering
    \includegraphics[width=\textwidth]{image1.png}
    \caption{06号卫星、09号卫星在0s至170s的轨迹}
    \label{fig:5.5}
\end{figure}

\begin{figure}[h]
    \centering
    \includegraphics[width=\textwidth]{image2.png}
    \caption{LS与TLS估计的0号飞行器轨迹}
    \label{fig:5.6}
\end{figure}

\begin{figure}[h]
    \centering
    \includegraphics[width=\textwidth]{image3.png}
    \caption{LS与TLS的空间飞行器轨迹估计的残差}
    \label{fig:5.7}
\end{figure}

由上图可知LS与TLS结果相差不大,主要原因是系数矩阵A不是病态,其条件数较小,因此以TLS估计的结果作为0号空间飞行器的观测轨迹估计值。从50.0s到170.0s间隔10.0s进行采样,估计的位置结果见表\ref{tab:5.7}。在此基础上利用中心差分方法计算得到的速度见表\ref{tab:5.8}。

\begin{table}[h]
    \centering
    \caption{空间飞行器轨迹观测的位置估计}
    \label{tab:5.7}
    \begin{tabular}{|c|c|c|c|}
    \hline
    时刻/s & X分量/m & Y分量/m & Z分量/m \\ \hline
    50 & $-1.10882 \times 10^{6}$ & $6.20089 \times 10^{6}$ & $1.12993 \times 10^{6}$ \\ \hline
    60 & $-1.11733 \times 10^{6}$ & $6.20844 \times 10^{6}$ & $1.14007 \times 10^{6}$ \\ \hline
    70 & $-1.12748 \times 10^{6}$ & $6.21656 \times 10^{6}$ & $1.15219 \times 10^{6}$ \\ \hline
    80 & $-1.13936 \times 10^{6}$ & $6.22519 \times 10^{6}$ & $1.16641 \times 10^{6}$ \\ \hline
    90 & $-1.15304 \times 10^{6}$ & $6.23427 \times 10^{6}$ & $1.18283 \times 10^{6}$ \\ \hline
    100 & $-1.16867 \times 10^{6}$ & $6.24377 \times 10^{6}$ & $1.20161 \times 10^{6}$ \\ \hline
    110 & $-1.18638 \times 10^{6}$ & $6.25364 \times 10^{6}$ & $1.22291 \times 10^{6}$ \\ \hline
    \end{tabular}
\end{table}

\begin{table}
\centering
\begin{tabular}{|c|c|c|c|}
\hline
120 & $-1.20630 \times 10^{6}$ & $6.26388 \times 10^{6}$ & $1.24690 \times 10^{6}$ \\
\hline
130 & $-1.22860 \times 10^{6}$ & $6.27447 \times 10^{6}$ & $1.27378 \times 10^{6}$ \\
\hline
140 & $-1.25343 \times 10^{6}$ & $6.28539 \times 10^{6}$ & $1.30376 \times 10^{6}$ \\
\hline
150 & $-1.28099 \times 10^{6}$ & $6.29664 \times 10^{6}$ & $1.33704 \times 10^{6}$ \\
\hline
160 & $-1.31145 \times 10^{6}$ & $6.30823 \times 10^{6}$ & $1.37386 \times 10^{6}$ \\
\hline
170 & $-1.34500 \times 10^{6}$ & $6.32013 \times 10^{6}$ & $1.41445 \times 10^{6}$ \\
\hline
\end{tabular}
\end{table}

\begin{table}
\centering
\caption{空间飞行器轨迹的速度估计}
\begin{tabular}{|c|c|c|c|}
\hline
时刻/s & X分量(单位:m/s) & Y分量(单位:m/s) & Z分量(单位:m/s) \\
\hline
50 & $-7.71586 \times 10^{2}$ & $7.23253 \times 10^{2}$ & $9.18072 \times 10^{2}$ \\
\hline
60 & $-9.31376 \times 10^{2}$ & $7.84719 \times 10^{2}$ & $1.11156 \times 10^{3}$ \\
\hline
70 & $-1.09915 \times 10^{3}$ & $8.38579 \times 10^{2}$ & $1.31479 \times 10^{3}$ \\
\hline
80 & $-1.27633 \times 10^{3}$ & $8.86364 \times 10^{2}$ & $1.52958 \times 10^{3}$ \\
\hline
90 & $-1.46438 \times 10^{3}$ & $9.29435 \times 10^{2}$ & $1.75775 \times 10^{3}$ \\
\hline
100 & $-1.66476 \times 10^{3}$ & $9.68976 \times 10^{2}$ & $2.00113 \times 10^{3}$ \\
\hline
110 & $-1.87889 \times 10^{3}$ & $1.00599 \times 10^{3}$ & $2.26144 \times 10^{3}$ \\
\hline
120 & $-2.10811 \times 10^{3}$ & $1.04127 \times 10^{3}$ & $2.54031 \times 10^{3}$ \\
\hline
130 & $-2.35364 \times 10^{3}$ & $1.07545 \times 10^{3}$ & $2.83921 \times 10^{3}$ \\
\hline
140 & $-2.61653 \times 10^{3}$ & $1.10893 \times 10^{3}$ & $3.15934 \times 10^{3}$ \\
\hline
150 & $-2.89761 \times 10^{3}$ & $1.14196 \times 10^{3}$ & $3.50165 \times 10^{3}$ \\
\hline
160 & $-3.19744 \times 10^{3}$ & $1.17455 \times 10^{3}$ & $3.86669 \times 10^{3}$ \\
\hline
170 & $-3.51623 \times 10^{3}$ & $1.20656 \times 10^{3}$ & $4.25459 \times 10^{3}$ \\
\hline
\end{tabular}
\end{table}

\subsubsection{5.2.3 $\bar{v}_r(t)$ 和 $m(t)$ 的模型建立与参数估计}

\subsubsection{5.2.3.1 $\bar{v}_r(t)$ 和 $m(t)$ 的建模}

由于飞行器在主动段飞行过程中,燃料消耗及其他因素导致质量不断变化,题中认为飞行器质量 $m(t)$ 一般为严格单调递减的非负函数。$\bar{v}_r(t)$ 为燃料相对于火箭尾部喷口的喷射速度,题中认为 $\bar{v}_r(t)$ 的方向应与飞行器的速度方向接近反向共线,其大小一般较为稳定。

根据以上信息,对飞行器质量 $m(t)$ 做出如下建模

\begin{equation}
m(t) = a - bt \tag{5.2-16}
\end{equation}

其中 $a$ 为飞行器的初始质量,$b$ 为飞行器主动段的损耗率。$\bar{v}_r(t)$ 的假设为:

\begin{equation}
\vec{v}_{r}(t)=
\begin{pmatrix}
\lambda_{1}\frac{\dot{r}_{x}(t)}{|\dot{\vec{r}}(t)|} \\
\lambda_{2}\frac{\dot{r}_{y}(t)}{|\dot{\vec{r}}(t)|} \\
\lambda_{3}\frac{\dot{r}_{z}(t)}{|\dot{\vec{r}}(t)|}
\end{pmatrix}
\tag{5.2-17}
\end{equation}

其中 $\lambda_{1}, \lambda_{2}, \lambda_{3}$ 为飞行器在各个方向上速度分量的系数。则

\begin{equation}
\vec{v}_{r}(t)\frac{\dot{m}(t)}{m(t)}=
\begin{pmatrix}
-\frac{\lambda_{1}b}{a-bt}\frac{\dot{r}_{x}(t)}{|\dot{\vec{r}}(t)|} \\
-\frac{\lambda_{2}b}{a-bt}\frac{\dot{r}_{y}(t)}{|\dot{\vec{r}}(t)|} \\
-\frac{\lambda_{3}b}{a-bt}\frac{\dot{r}_{z}(t)}{|\dot{\vec{r}}(t)|}
\end{pmatrix}
\tag{5.2-18}
\end{equation}

\subsubsection{5.2.3.2 模型的参数估计}

欲得到待估参数 $\lambda_{1}, \lambda_{2}, \lambda_{3}, a, b$ 的估计,传统的可用于参数估计的方法有最小二乘、极大似然等。但由于式(5.2-18)中各分量用分式表示,利用传统的参数估计方法思想得到待估参数的估计是相当困难的,甚至是行不通的。为此我们针对本模型提出一种新的参数估计的方法——基于级数理论的多项式模型方法,该方法对分式 $-\frac{\lambda b}{a-bt}$ 利用级数展开 $^{[5]}$,取有限次多项式近似,如下:

\begin{equation}
\vec{v}_{r}(t)\frac{\dot{m}(t)}{m(t)}=
\begin{pmatrix}
-(a_{x}+b_{x}t+c_{x}t^{2}+\cdots+n_{x}t^{n})\frac{\dot{r}_{x}(t)}{|\dot{\vec{r}}(t)|} \\
-(a_{y}+b_{y}t+c_{y}t^{2}+\cdots+n_{y}t^{n})\frac{\dot{r}_{y}(t)}{|\dot{\vec{r}}(t)|} \\
-(a_{z}+b_{z}t+c_{z}t^{2}+\cdots+n_{z}t^{n})\frac{\dot{r}_{z}(t)}{|\dot{\vec{r}}(t)|}
\end{pmatrix}
\tag{5.2-19}
\end{equation}

至此将参数 $(\lambda_{1}, \lambda_{2}, \lambda_{3})$ 的估计问题转化为对 $(a_{x}, b_{x}, \cdots, n_{x}, a_{y}, b_{y}, \cdots, n_{y}, a_{z}, b_{z}, \cdots, n_{z})$ 的估计。

将上式(5.2-19)带入飞行器的运动方程(4-1)得

\begin{equation}
\ddot{\vec{r}}(t) = -\frac{G_{m}}{|\vec{r}(t)|^{3}}\vec{r}(t) + \begin{bmatrix}
-(a_{x} + b_{x}t + c_{x}t^{2} + \cdots + n_{x}t^{n})\frac{\dot{r}_{x}(t)}{|\dot{\vec{r}}(t)|} \\
-(a_{y} + b_{y}t + c_{y}t^{2} + \cdots + n_{y}t^{n})\frac{\dot{r}_{y}(t)}{|\dot{\vec{r}}(t)|} \\
-(a_{z} + b_{z}t + c_{z}t^{2} + \cdots + n_{z}t^{n})\frac{\dot{r}_{z}(t)}{|\dot{\vec{r}}(t)|}
\end{bmatrix}
\tag{5.2-20}
\end{equation}

根据式(5.2-20)来确定参数 $(a_{x}, b_{x}, \cdots, n_{x}, a_{y}, b_{y}, \cdots, n_{y}, a_{z}, b_{z}, \cdots, n_{z})$ 的一个基本前提是知道飞行器各时刻的速度和加速度。对此我们基于双星逐点交汇定位的飞行器各时刻的位置采用中心差分法、二阶中心差分法求出各时刻飞行器的速度和加速度。

通过第 5.2.3 节双星逐点交汇定位得出的飞行器各时刻的位置,对飞行器各时刻的速度和加速度采用一阶中心差分,二阶中心差分的滑窗法进行计算。

\begin{equation}
\dot{\vec{r}}(t) = \frac{\vec{r}(t+\Delta t) - \vec{r}(t-\Delta t)}{2\Delta t}
\end{equation}

\begin{figure}[h]
\centering
\begin{tikzpicture}[scale=0.8]
    \draw[->, thick] (0,0) -- (10,0) node[right] {滑窗的滑动};
    \draw[->, thick] (0,0) -- (0,3);
    \draw[->, thick] (0,0) -- (3,0);
    \draw[->, thick] (0,0) -- (0,-3);
    \draw[->, thick] (0,0) -- (-3,0);

    \draw[thick] (0,0) grid (10,2);
    \draw[thick] (0,0) grid (10,-2);

    \node at (0.5,1.5) {$\cdots$};
    \node at (2.5,1.5) {$r(t-\Delta t)$};
    \node at (5,1.5) {$r(t)$};
    \node at (7.5,1.5) {$r(t+\Delta t)$};
    \node at (9.5,1.5) {$\cdots$};
    \node at (9.5,-0.5) {$r(t+n\Delta t)$};
    \node at (9.5,-1.5) {$\cdots$};

    \draw[->, thick] (5,0) -- (5,-3);
    \draw[->, thick] (7.5,0) -- (7.5,-3);
    \draw[->, thick] (2.5,0) -- (2.5,-3);
\end{tikzpicture}
\caption{滑窗法原理}
\end{figure}

\begin{equation}
\ddot{\vec{r}}(t) = \frac{\vec{r}(t+\Delta t) - 2\vec{r}(t) + \vec{r}(t-\Delta t)}{\Delta t^{2}}
\end{equation}

序列 $(r(t), r(t+\Delta t), r(t+2\Delta t), \cdots, r(t+n\Delta t))$ 表示双星逐点交汇定位测定的飞行器的对应各离散时刻的位置,$\Delta t$ 为步长。以 $x$ 方向为例,对滑窗具体的差分运算如下,对滑窗内的三点作中心差分,则对 $t$ 时刻的速度和加速度分别为:

\begin{equation}
\begin{cases}
\ddot{r}_{x}(t) = \frac{\vec{r}_{x}(t+\Delta t) - 2\vec{r}_{x}(t) + \vec{r}_{x}(t-\Delta t)}{\Delta t^{2}} \\
\dot{r}_{x}(t) = \frac{\vec{r}_{x}(t+\Delta t) - \vec{r}_{x}(t-\Delta t)}{2\Delta t}
\end{cases}
\tag{5.2-21}
\end{equation}

同理可得 $\dot{r}_{y}(t)$、$\ddot{r}_{y}(t)$、$\dot{r}_{z}(t)$、$\ddot{r}_{z}(t)$ 根据滑窗的滑动,继而可得不同时刻的状态向量:令

\begin{equation}
\left\{
\begin{aligned}
k_{x}(t) &= \ddot{r}_{x}(t) + \frac{G_{m}}{|r(t)|^{3}} \vec{r}_{x}(t) \\
k_{y}(t) &= \ddot{r}_{y}(t) + \frac{G_{m}}{|r(t)|^{3}} \vec{r}_{y}(t) \\
k_{z}(t) &= \ddot{r}_{z}(t) + \frac{G_{m}}{|r(t)|^{3}} \vec{r}_{z}(t)
\end{aligned}
\right.
\tag{5.2-22}
\end{equation}

为求解待估参数 $a_{x}, b_{x}, c_{x}, \ldots, n_{x}, a_{y}, b_{y}, c_{y}, \ldots, a_{z}, b_{z}, c_{z}, \ldots$, 建立最优平方逼近函数如下

\begin{equation}
f = \min \sum_{i=1}^{m} \left\{ \left[ k_{x}(t_{i}) + (a_{x} + b_{x} t_{i} + c_{x} t_{i}^{2} + \cdots + n_{x} t_{i}^{n}) \frac{\dot{r}_{x}(t_{i})}{|\dot{r}(t)|} \right]^{2} \right.
\end{equation}
\begin{equation}
\left. + \left[ k_{y}(t_{i}) + (a_{y} + b_{y} t_{i} + c_{y} t_{i}^{2} + \cdots + n_{y} t_{i}^{n}) \frac{\dot{r}_{y}(t_{i})}{|\dot{r}(t)|} \right]^{2} \right.
\tag{5.2-23}
\end{equation}
\begin{equation}
\left. + \left[ k_{z}(t_{i}) + (a_{z} + b_{z} t_{i} + c_{z} t_{i}^{2} + \cdots + n_{z} t_{i}^{n}) \frac{\dot{r}_{z}(t_{i})}{|\dot{r}(t)|} \right]^{2} \right\}
\end{equation}

其中 $i$ 为时间索引, $t_{i}$ 表示时间。

为求解上述无约束优化问题只需令

\begin{equation}
\left\{
\begin{aligned}
\frac{\partial f}{\partial a_{x}} &= 0 \\
\frac{\partial f}{\partial b_{x}} &= 0 \\
\frac{\partial f}{\partial c_{x}} &= 0 \\
&\vdots \\
\frac{\partial f}{\partial n_{x}} &= 0
\end{aligned}
\right\}
\left\{
\begin{aligned}
\frac{\partial f}{\partial a_{y}} &= 0 \\
\frac{\partial f}{\partial b_{y}} &= 0 \\
\frac{\partial f}{\partial c_{y}} &= 0 \\
&\vdots \\
\frac{\partial f}{\partial n_{y}} &= 0
\end{aligned}
\right\}
\left\{
\begin{aligned}
\frac{\partial f}{\partial a_{z}} &= 0 \\
\frac{\partial f}{\partial b_{z}} &= 0 \\
\frac{\partial f}{\partial c_{z}} &= 0 \\
&\vdots \\
\frac{\partial f}{\partial n_{z}} &= 0
\end{aligned}
\right\}
\tag{5.2-24}
\end{equation}

以 $x$ 轴分量为例进行推导:

\begin{equation}
\frac{\partial f}{\partial a_{x}} = \sum_{i=1}^{m} 2 \left[ k_{x}(t_{i}) + (a_{x} + b_{x} t_{i} + c_{x} t_{i}^{2} + \cdots + n_{x} t_{i}^{n}) \frac{\dot{r}_{x}(t_{i})}{|\dot{r}(t)|} \right] \frac{\dot{r}_{x}(t_{i})}{|\dot{r}(t)|} = 0
\end{equation}
\begin{equation}
\frac{\partial f}{\partial b_{x}} = \sum_{i=1}^{m} 2 \left[ k_{x}(t_{i}) + (a_{x} + b_{x} t_{i} + c_{x} t_{i}^{2} + \cdots + n_{x} t_{i}^{n}) \frac{\dot{r}_{x}(t_{i})}{|\dot{r}(t)|} \right] \frac{\dot{r}_{x}(t_{i})}{|\dot{r}(t)|} t_{i} = 0
\tag{5.2-25}
\end{equation}
\begin{equation}
\vdots
\end{equation}
\begin{equation}
\frac{\partial f}{\partial n_{x}} = \sum_{i=1}^{m} 2 \left[ k_{x}(t_{i}) + (a_{x} + b_{x} t_{i} + c_{x} t_{i}^{2} + \cdots + n_{x} t_{i}^{n}) \frac{\dot{r}_{x}(t_{i})}{|\dot{r}(t)|} \right] \frac{\dot{r}_{x}(t_{i})}{|\dot{r}(t)|} t_{i}^{n} = 0
\end{equation}

将式(5.2-25)代入式(5.2-24)并整理得:

\begin{equation}
\sum_{i=1}^{m} k_{x}(t_{i}) \begin{pmatrix} 1 \\ 1 \\ \vdots \\ 1 \end{pmatrix} + \sum_{i=1}^{m} \frac{\dot{r}_{x}(t_{i})}{\dot{r}(t)} (1, t_{i}, t_{i}^{2}, \ldots, t_{i}^{n}) \begin{pmatrix} a_{x} \\ b_{x} \\ \vdots \\ n_{x} \end{pmatrix} = 0
\tag{5.2-26}
\end{equation}

同理可得到 $y$, $z$ 轴分量的方程。对以上得到的方程联立

令
\[
\mathbf{A}_{1} = \begin{pmatrix}
\sum\limits_{i=1}^{m} \frac{\dot{r}_{x}(t_{i})}{\dot{r}(t)} & \sum\limits_{i=1}^{m} t_{i} \frac{\dot{r}_{x}(t_{i})}{\dot{r}(t)} & \sum\limits_{i=1}^{m} t_{i}^{2} \frac{\dot{r}_{x}(t_{i})}{\dot{r}(t)} & \cdots & \sum\limits_{i=1}^{m} t_{i}^{n} \frac{\dot{r}_{x}(t_{i})}{\dot{r}(t)}
\end{pmatrix}^{T},
\]
\[
\mathbf{B}_{1} = \sum_{i=1}^{m} k_{x}(t_{i}) (1 \quad 1 \quad \cdots \quad 1)^{T}, \quad \mathbf{X}_{1} = \begin{pmatrix} a_{x} & b_{x} & c_{x} & \cdots & n_{x} \end{pmatrix}^{T}
\]
同理可得出 $a_{y}, b_{y}, c_{y}, \ldots, n_{y}$ 与 $a_{z}, b_{z}, c_{z}, \ldots, n_{z}$ 方程对应的 $\mathbf{A}_{2}, \mathbf{B}_{2}, \mathbf{X}_{2}$, $\mathbf{A}_{3}, \mathbf{B}_{3}, \mathbf{X}_{3}$。联立 $xyz$ 各方程组可得 $\mathbf{A} \mathbf{X} + \mathbf{B} = 0$,
\[
\text{其中 } \mathbf{A} = \begin{pmatrix} \mathbf{A}_{1} & 0 & 0 \\ 0 & \mathbf{A}_{2} & 0 \\ 0 & 0 & \mathbf{A}_{3} \end{pmatrix}, \mathbf{B} = \begin{pmatrix} \mathbf{B}_{1} \\ \mathbf{B}_{2} \\ \mathbf{B}_{3} \end{pmatrix}, \mathbf{X} = \begin{pmatrix} \mathbf{X}_{1} \\ \mathbf{X}_{2} \\ \mathbf{X}_{3} \end{pmatrix},
\]
解出 $n$ 次多项式下 $\vec{v}_{r}(t)$ 和 $m(t)$ 模型中各参数

针对本题我们初步选定次数分别为 1、2、3 的多项式进行逼近,得到的各参数如表 5.9 所示:

\begin{table}[h]
\centering
\caption{各次数多项式逼近的 $\vec{v}_{r}(t)$ 和 $m(t)$ 模型中各参数}
\begin{tabular}{|c|c|c|c|}
\hline
多项式次数 & $n=1$ & $n=2$ & $n=3$ \\
\hline
$a_{x}$ & 2.21949E-02 & 3.18840E-02 & 2.21003E-02 \\
\hline
$b_{x}$ & -7.81425E-05 & -2.60235E-04 & -7.66121E-05 \\
\hline
$c_{x}$ & 0 & 7.83864E-07 & -3.27608E-09 \\
\hline
$d_{x}$ & 0 & 0 & -1.80620E-09 \\
\hline
$a_{y}$ & 2.29120E-02 & 3.24635E-02 & 4.31794E-02 \\
\hline
$b_{y}$ & -8.42335E-05 & -2.72623E-04 & -6.02931E-04 \\
\hline
$c_{y}$ & 0 & 8.40750E-07 & 3.99306E-06 \\
\hline
$d_{y}$ & 0 & 0 & -9.44157E-09 \\
\hline
$a_{z}$ & 2.21711E-02 & 3.18569E-02 & 4.35150E-02 \\
\hline
$b_{z}$ & -7.80411E-05 & -2.59920E-04 & -6.08452E-04 \\
\hline
$c_{z}$ & 0 & 7.82491E-07 & 4.01811E-06 \\
\hline
$d_{z}$ & 0 & 0 & -9.46440E-09 \\
\hline
\end{tabular}
\end{table}

\subsubsection{5.2.4 飞行器轨道估计及残差计算}

\subsubsection{5.2.4.1 模型的建立}

对于飞行器轨迹的估计,利用题中所给出简化运动方程联合 $\bar{v}_{r}(t)$ 和 $m(t)$ 的确立微分方程组:
\begin{equation}
\left\{
\begin{aligned}
\ddot{\vec{r}}(t) &= -\frac{G_{m}}{|\vec{r}(t)|^{3}} \vec{r}(t) + \bar{v}_{r}(t) \frac{\dot{m}(t)}{m(t)} \\
\frac{d \vec{r}(t)}{dt} &= \vec{v}(t) \\
\vec{r}(0) &= \left(x_{0}, y_{0}, z_{0}\right)^{T} \\
\vec{v}(0) &= \left(v_{x}(0), v_{y}(0), v_{z}(0)\right)
\end{aligned}
\right.
\tag{5.2-27}
\end{equation}

将式(5.2-19)代入式(5.2-27)进一步化简得到:
\begin{equation}
\left\{
\begin{aligned}
\frac{d^{2} x(t)}{dt^{2}} &= -\frac{G_{m}}{\left[x^{2}(t) + y^{2}(t) + z^{2}(t)\right]^{3/2}} x(t) - \left(a_{x} + b_{x} t + c_{x} t^{2} + \cdots + n_{x} t^{n}\right) \ddot{r}_{x}(t) \\
\frac{d^{2} y(t)}{dt^{2}} &= -\frac{G_{m}}{\left[x^{2}(t) + y^{2}(t) + z^{2}(t)\right]^{3/2}} y(t) - \left(a_{y} + b_{y} t + c_{y} t^{2} + \cdots + n_{y} t^{n}\right) \ddot{r}_{y}(t) \\
\frac{d^{2} z(t)}{dt^{2}} &= -\frac{G_{m}}{\left[x^{2}(t) + y^{2}(t) + z^{2}(t)\right]^{3/2}} z(t) - \left(a_{z} + b_{z} t + c_{z} t^{2} + \cdots + n_{z} t^{n}\right) \ddot{r}_{z}(t) \\
v_{x}(t) &= \frac{dx(t)}{dt} = \dot{r}_{x}(t) \\
v_{y}(t) &= \frac{dy(t)}{dt} = \dot{r}_{y}(t) \\
v_{z}(t) &= \frac{dz(t)}{dt} = \dot{r}_{z}(t)
\end{aligned}
\right.
\tag{5.2-28}
\end{equation}

利用 5.1 节中的定步长四阶龙格库塔方法求解,结合初始时刻值完成模型的求解。从而确定了空间飞行器的轨迹位置的估计结果。通过公式(5.2.20)求解得到速度的估计值。

\subsubsection{5.2.4.2 计算结果及残差分析}

利用四阶龙格库塔方法求得的 0 号空间飞行器的轨道估计结果如图 5.9 所示。从图中可以看出 0 号空间飞行器的轨道估计与图 5.5 的结果基本相同。而速度的估计值如图 5.10 所示。具体的从 50.0s 到 170.0s 间隔 10.0s 进行采样,估计的位置结果见表 5.10。在此基础上利用中心差分方法计算得到的速度见表 5.11。通过表 5.10 与表 5.7 的比较,可以看出:在采样点时刻,观测的飞行器轨迹估计与通过运动方程求解的结果基本相同。由表 5.11 与表 5.8 的比较可知,0 号飞行器的速度估计近似一致。位置和速度的估计相

\begin{table}
\centering
\caption{龙格库塔法估计的空间飞行器位置}
\begin{tabular}{|c|c|c|c|}
\hline
时刻/s & X分量/m & Y分量/m & Z分量/m \\
\hline
50 & $-1.10807 \times 10^{6}$ & $6.20018 \times 10^{6}$ & $1.12904 \times 10^{6}$ \\
\hline
60 & $-1.11652 \times 10^{6}$ & $6.20768 \times 10^{6}$ & $1.13911 \times 10^{6}$ \\
\hline
70 & $-1.12656 \times 10^{6}$ & $6.21574 \times 10^{6}$ & $1.15109 \times 10^{6}$ \\
\hline
80 & $-1.13833 \times 10^{6}$ & $6.22431 \times 10^{6}$ & $1.16519 \times 10^{6}$ \\
\hline
90 & $-1.15197 \times 10^{6}$ & $6.23336 \times 10^{6}$ & $1.18154 \times 10^{6}$ \\
\hline
100 & $-1.16760 \times 10^{6}$ & $6.24286 \times 10^{6}$ & $1.20032 \times 10^{6}$ \\
\hline
110 & $-1.18535 \times 10^{6}$ & $6.25275 \times 10^{6}$ & $1.22167 \times 10^{6}$ \\
\hline
120 & $-1.20532 \times 10^{6}$ & $6.26301 \times 10^{6}$ & $1.24572 \times 10^{6}$ \\
\hline
130 & $-1.22764 \times 10^{6}$ & $6.27359 \times 10^{6}$ & $1.27263 \times 10^{6}$ \\
\hline
140 & $-1.25244 \times 10^{6}$ & $6.28448 \times 10^{6}$ & $1.30257 \times 10^{6}$ \\
\hline
150 & $-1.27990 \times 10^{6}$ & $6.29566 \times 10^{6}$ & $1.33573 \times 10^{6}$ \\
\hline
160 & $-1.31020 \times 10^{6}$ & $6.30714 \times 10^{6}$ & $1.37236 \times 10^{6}$ \\
\hline
170 & $-1.34360 \times 10^{6}$ & $6.31895 \times 10^{6}$ & $1.41277 \times 10^{6}$ \\
\hline
\end{tabular}
\end{table}

\begin{table}
\centering
\caption{空间飞行器轨迹的速度估计}
\begin{tabular}{|c|c|c|c|}
\hline
时刻/s & X分量(单位:m/s) & Y分量(单位:m/s) & Z分量(单位:m/s) \\
\hline
50 & $-7.70274 \times 10^{2}$ & $7.22320 \times 10^{2}$ & $9.16455 \times 10^{2}$ \\
\hline
60 & $-9.21680 \times 10^{2}$ & $7.78321 \times 10^{2}$ & $1.09962 \times 10^{3}$ \\
\hline
70 & $-1.08807 \times 10^{3}$ & $8.31854 \times 10^{2}$ & $1.30116 \times 10^{3}$ \\
\hline
80 & $-1.26841 \times 10^{3}$ & $8.82034 \times 10^{2}$ & $1.51984 \times 10^{3}$ \\
\hline
90 & $-1.46164 \times 10^{3}$ & $9.28257 \times 10^{2}$ & $1.75440 \times 10^{3}$ \\
\hline
100 & $-1.66692 \times 10^{3}$ & $9.70254 \times 10^{2}$ & $2.00381 \times 10^{3}$ \\
\hline
110 & $-1.88381 \times 10^{3}$ & $1.00812 \times 10^{3}$ & $2.26752 \times 10^{3}$ \\
\hline
\end{tabular}
\end{table}

\begin{table}
\centering
\begin{tabular}{|c|c|c|c|}
\hline
120 & $-2.11248 \times 10^{3}$ & $1.04231 \times 10^{3}$ & $2.54571 \times 10^{3}$ \\ \hline
130 & $-2.35388 \times 10^{3}$ & $1.07366 \times 10^{3}$ & $2.83952 \times 10^{3}$ \\ \hline
140 & $-2.60999 \times 10^{3}$ & $1.10332 \times 10^{3}$ & $3.15132 \times 10^{3}$ \\ \hline
150 & $-2.88399 \times 10^{3}$ & $1.13280 \times 10^{3}$ & $3.48492 \times 10^{3}$ \\ \hline
160 & $-3.18050 \times 10^{3}$ & $1.16394 \times 10^{3}$ & $3.84590 \times 10^{3}$ \\ \hline
170 & $-3.50592 \times 10^{3}$ & $1.19895 \times 10^{3}$ & $4.24196 \times 10^{3}$ \\ \hline
\end{tabular}
\end{table}

\begin{figure}[h]
    \centering
    \includegraphics[width=\textwidth]{image1.png}
    \caption{0号空间飞行器的t-x曲线}
\end{figure}

\begin{figure}[h]
    \centering
    \includegraphics[width=\textwidth]{image2.png}
    \caption{0号空间飞行器的t-y曲线}
\end{figure}

\begin{figure}[h]
    \centering
    \includegraphics[width=\textwidth]{image3.png}
    \caption{0号空间飞行器的t-z曲线}
\end{figure}

图5.11 0号空间飞行器t-x、t-y、t-z曲线

\begin{figure}[h]
    \centering
    \includegraphics[width=\textwidth]{image4.png}
    \caption{0号空间飞行器的t-vx曲线}
\end{figure}

\begin{figure}[h]
    \centering
    \includegraphics[width=\textwidth]{image5.png}
    \caption{0号空间飞行器的t-vy曲线}
\end{figure}

\begin{figure}[h]
    \centering
    \includegraphics[width=\textwidth]{image6.png}
    \caption{0号空间飞行器的t-vz曲线}
\end{figure}

图5.12 0号空间飞行器t-vx、t-vy、t-vz曲线

残差的计算

28

残差定义 \cite{ref6} 为:记连续 $m$ 个时刻的参数估计值集合为 $\mathbf{A} = \{a_i \mid i = 1, 2, \cdots m\}$,对应的真实值集合为 $\mathbf{B} = \{b_i \mid i = 1, 2, \cdots m\}$,则估计值与真实值之间的差值集合 $\mathbf{C} = \mathbf{A} - \mathbf{B}$,即 $\mathbf{C} = \{c_i \mid c_i = a_i - b_i, i = 1, 2, \cdots m\}$,则估计值集合 $\mathbf{A}$ 的残差定义为:

\[
\sigma_{\mathbf{A}} = \sqrt{\frac{\sum_{i=1}^{m} (c_i - \overline{c})^2}{m}}
\]

其中 $\overline{c} = \frac{1}{m} \sum_{i=1}^{m} c_i$。

在本文进行轨道估计和速度估计的残差计算时,如无特殊说明,用于与估计结果进行比较的真值均是通过求解运动方程得到的。

分别用不同次数多项式逼近的 $\vec{v}_r(t)$ 和 $m(t)$ 模型估计的轨道数据计算相应的残差。如下表 5.12 所示。

\begin{table}[h]
\centering
\caption{不同次数逼近模型的轨道估计残差}
\begin{tabular}{|c|c|c|c|}
\hline
 & $n=1$ & $n=2$ & $n=3$ \\
\hline
X 分量 / m & 1.54424E+03 & 1.60681E+02 & 1.55775E+03 \\
\hline
Y 分量 / m & 1.01906E+03 & 1.19288E+02 & 3.87881E+01 \\
\hline
Z 分量 / m & 1.85035E+03 & 1.97124E+02 & 4.71987E+01 \\
\hline
Vx 分量 / m/s & 1.78934E+01 & 6.78000 & 1.81037E+01 \\
\hline
Vy 分量 / m/s & 1.09165E+01 & 4.03933 & 9.05067E-01 \\
\hline
Vz 分量 / m/s & 2.14528E+01 & 8.33612 & 1.82491 \\
\hline
\end{tabular}
\end{table}

从表 5.12 中可以看出取 $n=1$ 时估计的残差较大,取 $n=3$ 时 $X$ 周分量上残差较大,比 $Y$、$Z$ 分量的残差高两个数量级,此时建立的模型是不稳定的。因此我们选用 $n=2$ 来作为 $\vec{v}_r(t)$ 和 $m(t)$ 模型中逼近多项式的次数。再接下来的几问中也同样采用此方法,对不同次数得出的估计的残差进行了比较,均得出 $n=2$ 时,残差较小且比较稳定,故在以后几问中都是用 $n=2$ 来建模,各问的详细对比过程不再赘述。

图 5.13 和图 5.14 是两种轨道估计方法对轨道位置和速度差值绝对值,从图中曲线可以看出所提出的轨道估计模型是有效的,并且估计的精度较高。

\begin{figure}[h]
    \centering
    \includegraphics[width=\textwidth]{image1.png}
    \caption{位置差值绝对值随时间的变化曲线}
    \label{fig:position_error}
\end{figure}

\begin{figure}[h]
    \centering
    \includegraphics[width=\textwidth]{image2.png}
    \caption{速度差值绝对值随时间的变化曲线}
    \label{fig:speed_error}
\end{figure}

通过进一步评估了所选取的 $\bar{v}_{r}(t)$ 和 $m(t)$ 模型的可靠性。这里定义了相对误差 $\overline{\delta}$,$\overline{\delta}$ 越小,模型越可靠。$\overline{\delta}$ 定义为:
\begin{equation}
    \overline{\delta} = \max \left\{ \frac{\delta_{y}(t)}{y(t)}, t=0,1,2\ldots \right\}
    \tag{5.2-29}
\end{equation}

根据公式(5.2-29)计算出的相对误差如表 5.13 所示。从表中可以看出相对误差最大。

在$10^{-3}$数量级,因此可以说明$\bar{v}_{r}(t)$和$m(t)$模型比较准确且合理。

\begin{table}[h]
\centering
\caption{各分量相对误差}
\begin{tabular}{|c|c|c|}
\hline
 & 分量 & $\overline{\delta}$ \\ \hline
\multirow{3}{*}{位置} & $x$ & $1.037 \times 10^{-3}$ \\ \cline{2-3}
 & $y$ & $1.872 \times 10^{-4}$ \\ \cline{2-3}
 & $z$ & $1.19 \times 10^{-3}$ \\ \hline
\end{tabular}
\end{table}

\subsection{5.3 问题三的建模与求解}

问题三的基本思路是根据坐标转换模型和观测坐标系下的量测方程,确定系统误差、飞行器空间位置和卫星量测数据的关系,建立方程组,通过对方程组未知量个数与自由度关系来判断是否能够进行系统误差估计。首先对逐点交汇系统误差估计的可行性进行了分析,进而提出了联合多点交汇法,可同步实现轨道参数及卫星系统误差的联合估计。

\subsubsection{5.3.1 逐点交汇法系统误差估计可行性分析}

利用双星进行空间飞行器的定位,其系统误差是不可避免的,为消除系统误差带来的估计偏差以提高系统的测量精度,就需要对系统误差进行估计。

假设$d_{\alpha}, d_{\beta}, d_{\theta}$分别是$\alpha, \beta, \alpha\beta$平面内旋转能量误差的误差界,如图5.15所示。不同时刻三轴指向误差是不同的。

\begin{figure}[h]
\centering
\includegraphics[width=0.6\textwidth]{image.png}
\caption{$t_{i}$时刻各系统误差的示意图}
\end{figure}

对任意离散时刻$t_{i}, i=0,1,2,\cdots$,设飞行器在基础坐标系下的坐标为:

\begin{equation}
\vec{r}_{c}(t_{i}) = (x_{c}(t_{i}), y_{c}(t_{i}), z_{c}(t_{i}))^{T}
\tag{5.3-1}
\end{equation}

卫星观测到的飞行器的坐标分别为:

\begin{equation}
\begin{cases}
\vec{r}_{06s}(t_{i}) = (x_{06s}(t_{i}), y_{06s}(t_{i}), z_{06s}(t_{i}))^{T} \\
\vec{r}_{09s}(t_{i}) = (x_{09s}(t_{i}), y_{09s}(t_{i}), z_{09s}(t_{i}))^{T}
\end{cases}
\tag{5.3-2}
\end{equation}

卫星在基础坐标系下的坐标为:
\begin{equation}
\begin{cases}
\vec{r}_{06}(t_i) = (x_{06}(t_i), y_{06}(t_i), z_{06}(t_i))^T \\
\vec{r}_{09}(t_i) = (x_{09}(t_i), y_{09}(t_i), z_{09}(t_i))^T
\end{cases}
\tag{5.3-2}
\end{equation}

在考虑两个卫星系统误差 $d_{\alpha_{06}}, d_{\beta_{06}}, d_{\theta_{06}}, d_{\alpha_{09}}, d_{\beta_{09}}, d_{\theta_{09}}$ 的情形下按照基础坐标系与观测坐标系的转换关系,任意时刻 $t_i$ 下的坐标满足:

\begin{equation}
\begin{pmatrix}
x_c(t_i) \\
y_c(t_i) \\
z_c(t_i)
\end{pmatrix}
=
\begin{pmatrix}
x_{06}(t_i) \\
y_{06}(t_i) \\
z_{06}(t_i)
\end{pmatrix}
+ x_{06s}(t_i) \mathbf{E}
\begin{pmatrix}
1 & \alpha_{06}(t_i) + d_{\alpha_{06}}(t_i) \\
0 & \beta_{06}(t_i) + d_{\beta_{06}}(t_i)
\end{pmatrix}
\tag{5.3-3}
\end{equation}

\begin{equation}
\begin{pmatrix}
x_c(t_i) \\
y_c(t_i) \\
z_c(t_i)
\end{pmatrix}
=
\begin{pmatrix}
x_{09}(t_i) \\
y_{09}(t_i) \\
z_{09}(t_i)
\end{pmatrix}
+ x_{09s}(t_i) \mathbf{E}
\begin{pmatrix}
1 & \alpha_{09}(t_i) + d_{\alpha_{09}}(t_i) \\
0 & \beta_{09}(t_i) + d_{\beta_{09}}(t_i)
\end{pmatrix}
\tag{5.3-4}
\end{equation}

\begin{equation}
\begin{cases}
\cos d_{\theta_{06}}(t_i) = \frac{(\alpha_{06}(t_i), \beta_{06}(t_i)) \cdot (\alpha_{06}(t_i) + d_{\alpha_{06}}(t_i), \beta_{06}(t_i) + d_{\beta_{06}}(t_i))}{\sqrt{\alpha_{06}^2(t_i) + \beta_{06}^2(t_i)} \cdot \sqrt{(\alpha_{06}(t_i) + d_{\alpha_{06}}(t_i))^2 + (\beta_{06}(t_i) + d_{\beta_{06}}(t_i))^2}} \\
\cos d_{\theta_{09}}(t_i) = \frac{(\alpha_{09}(t_i), \beta_{09}(t_i)) \cdot (\alpha_{09}(t_i) + d_{\alpha_{09}}(t_i), \beta_{09}(t_i) + d_{\beta_{09}}(t_i))}{\sqrt{\alpha_{09}^2(t_i) + \beta_{09}^2(t_i)} \cdot \sqrt{(\alpha_{09}(t_i) + d_{\alpha_{09}}(t_i))^2 + (\beta_{09}(t_i) + d_{\beta_{09}}(t_i))^2}}
\end{cases}
\tag{5.3-5}
\end{equation}

联立以上方程组,其中 $x_c(t_i), y_c(t_i), z_c(t_i), d_{\alpha_{06}}(t_i), d_{\beta_{06}}(t_i), d_{\theta_{06}}(t_i), d_{\alpha_{09}}(t_i), d_{\beta_{09}}(t_i), d_{\theta_{09}}(t_i)$ 为 $t_i$ 时刻的未知量,共 9 个,而方程组的自由度(即方程的个数)为 8。因为未知量的个数大于方程组的自由度,所以该方程组为欠定方程组 [7],求解不唯一,这与某一时刻的参数唯一确定的实际情况矛盾,所以不能利用逐点交汇的方法同时对系统误差进行估计。

\subsubsection{5.3.2 系统误差求解}

由于受到各种因素的影响导致系统误差在每次观测值中有所不同,但可以认为在短时间内观测值中的系统误差几乎不变。基于这一理论,对式 (5.3-3) 和式 (5.3-4) 进行分析得出:在短时间内,$t_i$ 时刻的两个定位方程组中含有 5 个时变未知量 $x_c(t_i), y_c(t_i), z_c(t_i), x_{06s}(t_i), x_{09s}(t_i)$,4 个非时变未知量 $d_{\alpha_6}(t_i), d_{\beta_6}(t_i), d_{\alpha_9}(t_i), d_{\beta_9}(t_i)$(短时间内认为其是不变的),并且每增加一个时刻的两个定位方程组,增加 5 个时变未知量。所以在短时间内,当取四个时刻的定位方程组,共 24 个定位方程,其中 20 个时变未知量,4 个非时变未知量,共 24 个未知量,此时可以联立这 24 个定位方程

\begin{equation}
\left\{
\begin{aligned}
\begin{pmatrix}
x_{c}(t_{1}) \\
y_{c}(t_{1}) \\
z_{c}(t_{1})
\end{pmatrix}
&=
\begin{pmatrix}
x_{06}(t_{1}) \\
y_{06}(t_{1}) \\
z_{06}(t_{1})
\end{pmatrix}
+ x_{06s}(t_{1}) \mathbf{E}
\begin{pmatrix}
1 \\
\alpha_{06}(t_{1}) + d_{\alpha_{06}} \\
\beta_{06}(t_{1}) + d_{\beta_{06}}
\end{pmatrix} \\
\begin{pmatrix}
x_{c}(t_{1}) \\
y_{c}(t_{1}) \\
z_{c}(t_{1})
\end{pmatrix}
&=
\begin{pmatrix}
x_{09}(t_{1}) \\
y_{09}(t_{1}) \\
z_{09}(t_{1})
\end{pmatrix}
+ x_{09s}(t_{1}) \mathbf{E}
\begin{pmatrix}
1 \\
\alpha_{09}(t_{1}) + d_{\alpha_{09}} \\
\beta_{09}(t_{1}) + d_{\beta_{09}}
\end{pmatrix} \\
&\vdots \\
\begin{pmatrix}
x_{c}(t_{4}) \\
y_{c}(t_{4}) \\
z_{c}(t_{4})
\end{pmatrix}
&=
\begin{pmatrix}
x_{06}(t_{4}) \\
y_{06}(t_{4}) \\
z_{06}(t_{4})
\end{pmatrix}
+ x_{06s}(t_{4}) \mathbf{E}
\begin{pmatrix}
1 \\
\alpha_{06}(t_{4}) + d_{\alpha_{06}} \\
\beta_{06}(t_{4}) + d_{\beta_{06}}
\end{pmatrix} \\
\begin{pmatrix}
x_{c}(t_{4}) \\
y_{c}(t_{4}) \\
z_{c}(t_{4})
\end{pmatrix}
&=
\begin{pmatrix}
x_{09}(t_{4}) \\
y_{09}(t_{4}) \\
z_{09}(t_{4})
\end{pmatrix}
+ x_{09s}(t_{4}) \mathbf{E}
\begin{pmatrix}
1 \\
\alpha_{09}(t_{4}) + d_{\alpha_{09}} \\
\beta_{09}(t_{4}) + d_{\beta_{09}}
\end{pmatrix}
\end{aligned}
\right.
\end{equation}

计算求出该时间段内的系统误差 $d_{\alpha_{06}}, d_{\beta_{06}}, d_{\alpha_{09}}, d_{\beta_{09}}$,并将得到的值代入式(5.3-5)求解出 $d_{\theta_{06}}, d_{\theta_{09}}$。采用 5.2.3.2 节中滑窗法的原理对各个短时间段进行系统误差的估计,最后求出它们的均值作为系统误差。

求解出系统误差为:
\begin{equation}
\left\{
\begin{aligned}
\overline{d}_{\alpha_{06}} &= 1.35166 \times 10^{-3} \\
\overline{d}_{\beta_{06}} &= 1.30491 \times 10^{-3} \\
\overline{d}_{\theta_{06}} &= 8.96365 \times 10^{-4} \\
\overline{d}_{\alpha_{09}} &= 8.38309 \times 10^{-4} \\
\overline{d}_{\beta_{09}} &= 5.15881 \times 10^{-4} \\
\overline{d}_{\theta_{09}} &= 1.17204 \times 10^{-3}
\end{aligned}
\right.
\end{equation}

\subsubsection{5.3.3 飞行器轨道估计与残差计算}

通过 5.3.2 节中求出的系统误差对附件给出的 06 与 09 号卫星观测数据进行修正,利用修正后的数据,采用第二间的解法对飞行器进行逐点交汇定位和轨道估计,结果如下所示。从 50.0s 到 170.0s 间隔 10.0s 进行采样,计算并列表给出的 0 号空间飞行器在各个采样点的位置和速度见表 5.14-表 5.18。图 5.18 给出的是修正后 0 号飞行器轨道位置随时间变化曲线。图 5.19 给出的是修正后 0 号飞行器轨道速度随时间变化曲线。由图 5.18 和图 5.19 可知本节所建立的模型方法十分可靠,且精确度高。

\begin{table}
\centering
\begin{tabular}{|c|c|c|c|}
\hline
时刻/s & X分量/m & Y分量/m & Z分量/m \\
\hline
50 & -1.11226E+06 & 6.20065E+06 & 1.13324E+06 \\
\hline
60 & -1.12075E+06 & 6.20816E+06 & 1.14338E+06 \\
\hline
70 & -1.13088E+06 & 6.21624E+06 & 1.15550E+06 \\
\hline
80 & -1.14274E+06 & 6.22483E+06 & 1.16970E+06 \\
\hline
90 & -1.15641E+06 & 6.23388E+06 & 1.18612E+06 \\
\hline
100 & -1.17203E+06 & 6.24334E+06 & 1.20490E+06 \\
\hline
110 & -1.18971E+06 & 6.25318E+06 & 1.22618E+06 \\
\hline
120 & -1.20961E+06 & 6.26338E+06 & 1.25016E+06 \\
\hline
130 & -1.23188E+06 & 6.27393E+06 & 1.27703E+06 \\
\hline
140 & -1.25669E+06 & 6.28482E+06 & 1.30699E+06 \\
\hline
150 & -1.28421E+06 & 6.29604E+06 & 1.34025E+06 \\
\hline
160 & -1.31464E+06 & 6.30760E+06 & 1.37705E+06 \\
\hline
170 & -1.34815E+06 & 6.31948E+06 & 1.41761E+06 \\
\hline
\end{tabular}
\caption{系统误差修正后的空间飞行器轨道观测位置}
\end{table}

\begin{table}
\centering
\begin{tabular}{|c|c|c|c|}
\hline
时刻/s & X分量/m/s & Y分量/m/s & Z分量/m/s \\
\hline
50 & -7.71586E+02 & 7.23253E+02 & 9.18072E+02 \\
\hline
60 & -9.31376E+02 & 7.84719E+02 & 1.11156E+03 \\
\hline
70 & -1.09915E+03 & 8.38579E+02 & 1.31479E+03 \\
\hline
80 & -1.27633E+03 & 8.86364E+02 & 1.52958E+03 \\
\hline
90 & -1.46438E+03 & 9.29435E+02 & 1.75775E+03 \\
\hline
100 & -1.66476E+03 & 9.68976E+02 & 2.00113E+03 \\
\hline
110 & -1.87889E+03 & 1.00599E+03 & 2.26144E+03 \\
\hline
120 & -2.10811E+03 & 1.04127E+03 & 2.54031E+03 \\
\hline
130 & -2.35364E+03 & 1.07545E+03 & 2.83921E+03 \\
\hline
140 & -2.61653E+03 & 1.10893E+03 & 3.15934E+03 \\
\hline
150 & -2.89761E+03 & 1.14196E+03 & 3.50165E+03 \\
\hline
160 & -3.19744E+03 & 1.17455E+03 & 3.86669E+03 \\
\hline
170 & -3.51623E+03 & 1.20656E+03 & 4.25459E+03 \\
\hline
\end{tabular}
\caption{系统误差修正后的空间飞行器轨道速度观测估计}
\end{table}

\begin{table}
\centering
\caption{系统误差修正后龙格库塔法计算的空间飞行器轨道位置估计}
\begin{tabular}{|c|c|c|c|}
\hline
时刻/s & X分量/m & Y分量/m & Z分量/m \\
\hline
50 & -1.10807E+06 & 6.20018E+06 & 1.12904E+06 \\
\hline
60 & -1.11652E+06 & 6.20768E+06 & 1.13911E+06 \\
\hline
70 & -1.12656E+06 & 6.21574E+06 & 1.15109E+06 \\
\hline
80 & -1.13833E+06 & 6.22431E+06 & 1.16519E+06 \\
\hline
90 & -1.15197E+06 & 6.23336E+06 & 1.18154E+06 \\
\hline
100 & -1.16760E+06 & 6.24286E+06 & 1.20032E+06 \\
\hline
110 & -1.18535E+06 & 6.25275E+06 & 1.22167E+06 \\
\hline
120 & -1.20532E+06 & 6.26301E+06 & 1.24572E+06 \\
\hline
130 & -1.22764E+06 & 6.27359E+06 & 1.27263E+06 \\
\hline
140 & -1.25244E+06 & 6.28448E+06 & 1.30257E+06 \\
\hline
150 & -1.27990E+06 & 6.29566E+06 & 1.33573E+06 \\
\hline
160 & -1.31020E+06 & 6.30714E+06 & 1.37236E+06 \\
\hline
170 & -1.34360E+06 & 6.31895E+06 & 1.41277E+06 \\
\hline
\end{tabular}
\end{table}

\begin{table}
\centering
\caption{系统误差修正后龙格库塔法计算的空间飞行器轨道的速度估计}
\begin{tabular}{|c|c|c|c|}
\hline
时刻/s & X分量/m/s & Y分量/m/s & Z分量/m/s \\
\hline
50 & -7.70274E+02 & 7.22320E+02 & 9.16455E+02 \\
\hline
60 & -9.21680E+02 & 7.78321E+02 & 1.09962E+03 \\
\hline
70 & -1.08807E+03 & 8.31854E+02 & 1.30116E+03 \\
\hline
80 & -1.26841E+03 & 8.82034E+02 & 1.51984E+03 \\
\hline
90 & -1.46164E+03 & 9.28257E+02 & 1.75440E+03 \\
\hline
100 & -1.66692E+03 & 9.70254E+02 & 2.00381E+03 \\
\hline
110 & -1.88381E+03 & 1.00812E+03 & 2.26752E+03 \\
\hline
120 & -2.11248E+03 & 1.04231E+03 & 2.54571E+03 \\
\hline
130 & -2.35388E+03 & 1.07366E+03 & 2.83952E+03 \\
\hline
140 & -2.60999E+03 & 1.10332E+03 & 3.15132E+03 \\
\hline
150 & -2.88399E+03 & 1.13280E+03 & 3.48492E+03 \\
\hline
160 & -3.18050E+03 & 1.16394E+03 & 3.84590E+03 \\
\hline
170 & -3.50592E+03 & 1.19895E+03 & 4.24196E+03 \\
\hline
\end{tabular}
\end{table}

\begin{table}
\centering
\caption{空间飞行器轨道估计的残差}
\begin{tabular}{|c|c|c|c|}
\hline
 & X分量/m & Y分量/m & Z分量/m \\
\hline
位置/m & 1.60681E+02 & 1.19288E+002 & 1.97124E+02 \\
\hline
速度/m/s & 6.78000 & 4.03933 & 8.33612 \\
\hline
\end{tabular}
\end{table}

\begin{figure}[h]
    \centering
    \includegraphics[width=\textwidth]{image1.png}
    \caption{修正后 0 号飞行器轨道位置随时间变化曲线}
    \label{fig:5.18}
\end{figure}

\begin{figure}[h]
    \centering
    \includegraphics[width=\textwidth]{image2.png}
    \caption{修正后 0 号飞行器轨道速度随时间变化曲线}
    \label{fig:5.19}
\end{figure}

\begin{figure}[h]
    \centering
    \includegraphics[width=\textwidth]{image1.png}
    \caption{修正后 0 号飞行器轨道位置残差随时间变化曲线}
    \label{fig:5.20}
\end{figure}

图 5.20 是修正后 0 号飞行器轨道位置残差随时间变化曲线,从三条曲线的变化情况来看,Y 轴方向上的位置残差变化与其它两条曲线不同,且其变化分为最小;和图 5.13 相比较可以明显看出:经过系统误差修正 Y 轴方向上的位置残差较小,而 X、Z 轴方向上的位置残差增大。说明系统误差修正对有 Y 轴方向上的位置残差更大的修正意义。

\begin{figure}[h]
    \centering
    \includegraphics[width=\textwidth]{image2.png}
    \caption{修正后 0 号飞行器轨道速度残差随时间变化曲线}
    \label{fig:5.21}
\end{figure}

图 5.21 是修正后 0 号飞行器轨道速度绝对误差随时间变化曲线,从三条曲线的变化情况来看,Y 轴方向上的速度绝对误差变化与其它两条曲线不同,随着时间的增大残差

差越显著;和图 5.13 相比较可以明显看出:经过系统误差修正后,X 轴方向上的速度绝对误差增大,而 X、Z 轴方向上的速度绝对误差变化很小。

\subsection{5.4 问题四的建模与求解}

\subsubsection{5.4.1 单星单飞行器无源观测}

由于可获取的信息量较少,不能采用较为准确的逐点交汇定位等轨道估计方法,因此利用一个卫星对空间飞行器进行无源探测的轨道估计必须做出适当的合理假设,从而建立出基于单星无源探测的单个空间飞行器轨道估计模型,估计单个空间飞行器的轨道。这里在适当假设的基础上提出了一种基于角度信息约束的单星单飞行器的轨道估计模型,通过模型求解确定了 01 号卫星的轨道。

\subsubsection{5.4.1.1 联合序贯多点轨道估计}

为了实现单星对单飞行器对应时刻的空间轨道估计,这里需要做出如下假设:

假设 1:在较短的时间周期内,空间飞行器的各方向速度分量保持不变。这是由于一方面飞行器的速度远小于其位置的变化,另一方面在较短的时间周期内,处于主动段末尾的空间飞行器在较短的时间内其速度不可能有较大的变化;

假设 2:在较短的时间内,卫星可连续对空间飞行器多次进行观测。这是因为题中附录给出了间隔 0.2s 的观测数据,且这一假设符合现阶段的实际技术能力。

利用 09 号卫星观测观测 01 号飞行器,其轨道估计方程为

\begin{equation}
\begin{pmatrix}
x_c(t) \\
y_c(t) \\
z_c(t)
\end{pmatrix}
=
\begin{pmatrix}
x_0(t) \\
y_0(t) \\
z_0(t)
\end{pmatrix}
+ x_s(t)E(t)
\begin{pmatrix}
1 \\
\alpha(t) \\
\beta(t)
\end{pmatrix}
\tag{5.4-1}
\end{equation}

对 \( t \) 时刻卫星定位方程 (5.4-1) 分析可知:定位方程组中含有 3 个方程,4 个未知量,此时无法求出方程组的解。因此对单颗卫星估计空间飞行器轨迹是十分困难的。现考虑题目中给出的飞行器的飞行特性,由于飞行器的重力斜飞段后半程是进入自由状态的最后阶段,空间曲线相对弯曲程度较小,故假设短时间内飞行器做匀速运动,即在时间区间 \([t, t+\Delta t]\) 内有:

\begin{equation}
\begin{cases}
\dot{\vec{r}}_x(t) = \dot{\vec{r}}_x(t+\Delta t) = \dot{\vec{r}}_x(t+2\Delta t) \\
\dot{\vec{r}}_y(t) = \dot{\vec{r}}_y(t+\Delta t) = \dot{\vec{r}}_y(t+2\Delta t) \\
\dot{\vec{r}}_z(t) = \dot{\vec{r}}_z(t+\Delta t) = \dot{\vec{r}}_z(t+2\Delta t)
\end{cases}
\tag{5.4-2}
\end{equation}

基于这一等式,对 \( t+\Delta t \) 时刻的轨道估计方程进行分析得出:当方程组增加一个时,方程个数增加 3 个,时变未知量增加 1 个 \( x_s(t+\Delta t) \),非时变未知量增加 3 个 \( v_x(t), v_y(t), v_z(t) \)。此刻方程组仍无法求解。对 \( t+2\Delta t \) 时刻的定位方方程进行分析得出:此刻增加一个方程组,方程个数增加 3 个,时变未知量增加 1 个 \( x_s(t+2\Delta t) \)。此时方程个数为 9,未知量 9 个,方程组存在唯一解。不失一般性当考虑总共有 \( N \) 个时刻,此时联立 \( N \) 个时刻的轨道估计方程组中方程的个数是 \( 3N \) 个,未知变量的个数是 \( N+6 \),所

以当方程个数大于未知数个数即 $3N \geq N+6 (N \geq 3)$ 时方程组有唯一解,可对飞行器进行定位。但是考虑到假设 1,当给定步长 $\Delta t$ 后,$N$ 较大对应的时间跨度较大,飞行器的运动形式不在满足假设 1 中的短时间周期内,空间飞行器的各方向速度分量保持不变,故算法求解过程中,应取满足条件的 $N$ 的最小值,即 $N=3$。下面给出联合序贯三点几何轨道估计方法的具体过程。

联立时间区间 $[t, t+2\Delta t]$ 内的三个时刻轨道估计方程组如式 (5.4-2),求解方程组,得到 $t$ 时刻飞行器的位置矢量,从而完成对飞行器的轨道估计。

\begin{equation}
\left\{
\begin{aligned}
\begin{pmatrix}
x_c(t) \\
y_c(t) \\
z_c(t)
\end{pmatrix}
&=
\begin{pmatrix}
x_0(t) \\
y_0(t) \\
z_0(t)
\end{pmatrix}
+ x_s(t) \boldsymbol{E}(t)
\begin{pmatrix}
1 \\
\alpha(t) \\
\beta(t)
\end{pmatrix} \\
\begin{pmatrix}
x_c(t) + \Delta t v_x(t) \\
y_c(t) + \Delta t v_y(t) \\
z_c(t) + \Delta t v_z(t)
\end{pmatrix}
&=
\begin{pmatrix}
x_0(t+\Delta t) \\
y_0(t+\Delta t) \\
z_0(t+\Delta t)
\end{pmatrix}
+ x_s(t+\Delta t) \boldsymbol{E}(t+\Delta t)
\begin{pmatrix}
1 \\
\alpha(t+\Delta t) \\
\beta(t+\Delta t)
\end{pmatrix} \\
\begin{pmatrix}
x_c(t) + 2\Delta t v_x(t) \\
y_c(t) + 2\Delta t v_y(t) \\
z_c(t) + 2\Delta t v_z(t)
\end{pmatrix}
&=
\begin{pmatrix}
x_0(t+2\Delta t) \\
y_0(t+2\Delta t) \\
z_0(t+2\Delta t)
\end{pmatrix}
+ x_s(t+2\Delta t) \boldsymbol{E}(t+2\Delta t)
\begin{pmatrix}
1 \\
\alpha(t+2\Delta t) \\
\beta(t+2\Delta t)
\end{pmatrix}
\end{aligned}
\right.
\tag{5.4-3}
\end{equation}

将式 (5.4-3) 所示的方程组化简成 $\mathbf{A}\mathbf{x} = \mathbf{b}$ 的形式。

对于此类方程的求解问题,传统的方法是采用最小二乘,然而在本模型中简单地利用最小二乘所得到的结果并不能满足精度需求。究其原因,主要在于本模型建立的基本假设,即飞行器在较短的时间范围内速度分量保持不变。对于时间间隔 $\Delta t$,如果 $\Delta t$ 较大,显然与速度分量恒定这一假设矛盾;反之,如果 $\Delta t$ 很小,相邻时刻的空间位置坐标变化微小,带来的直接影响就是 $\mathbf{A}$、$\mathbf{b}$ 中各行的差异性较小,而在本模型假设框架下,$\Delta t$ 只能取较小的值,这也就使得矩阵 $\mathbf{A}$ 可能是病态的,针对此类型的求解,采用总体最小二乘 TLS 所得到的解更加稳健。

为对矩阵 $\mathbf{A}$ 的病态特性进行分析,我们利用矩阵的条件数来进行判断:条件数越大,则矩阵越病态。

条件数的定义为 $\mathbf{A}$ 的范数与 $\mathbf{A}$ 的逆的范数的乘积,即

\begin{equation}
\text{cond}(\mathbf{A}) = \|\mathbf{A}\| \cdot \|\mathbf{A}^{-1}\|
\tag{5.4-4}
\end{equation}

条件数实质上反映了矩阵计算对于误差的敏感性。对于线性方程组 $\mathbf{A}\mathbf{x} = \mathbf{b}$,如果 $\mathbf{A}$ 的条件数大,$\mathbf{b}$ 的微小改变就能引起解 $\mathbf{x}$ 较大的改变,数值稳定性差。如果 $\mathbf{A}$ 的条件数小,$\mathbf{b}$ 有微小的改变,$\mathbf{x}$ 的改变也很微小,数值稳定性好。它也可以表示 $\mathbf{b}$ 不变,而 $\mathbf{A}$ 有微小改变时,$\mathbf{x}$ 的变化情况。

为了实现单颗卫星无源探测的空间飞行器轨道估计,我们借助多个时刻观测结果,通过连续多时刻的观测信息实现了单颗卫星无源探测的空间飞行器轨道估计。

$\Delta t = 0.2 \, \text{s}$ 时矩阵条件数随步长变化的曲线如图 5.22 所示。由图 5.22 可知:由于条件数较大,所以采用传统的最小二乘法求解误差较大,而采用 TLS 求解是最好的选择。进一步比较发现,时间间隔 $\Delta t$ 的选取十分关键。这里选取 $\Delta t$ 的原则是为了使公式 (5.4-2) 更接近于实际的状态,$\Delta t$ 应尽可能小;但 $\Delta t$ 越小,所联立的方程组相关性越高,因此结合题中所给测量数据。

\begin{figure}[h]
    \centering
    \includegraphics[width=0.8\textwidth]{image1.png}
    \caption{轨道估计矩阵条件数曲线}
    \label{fig:5.22}
\end{figure}

\begin{figure}[h]
    \centering
    \includegraphics[width=0.8\textwidth]{image2.png}
    \caption{单星观测 01 号飞行器的位置估计}
    \label{fig:5.23}
\end{figure}

\begin{figure}[h]
    \centering
    \includegraphics[width=0.8\textwidth]{image3.png}
    \caption{龙格库塔算法解算 01 号飞行器的位置估计}
    \label{fig:5.24}
\end{figure}

\begin{figure}[h]
    \centering
    \includegraphics[width=\textwidth]{image1.png}
    \caption{单星观测01号飞行器的速度估计}
    \label{fig:5.25}
\end{figure}

\begin{figure}[h]
    \centering
    \includegraphics[width=\textwidth]{image2.png}
    \caption{龙格库塔算法解算01号飞行器的位置估计}
    \label{fig:5.26}
\end{figure}

\begin{figure}[h]
    \centering
    \includegraphics[width=\textwidth]{image3.png}
    \caption{01号飞行器轨道估计的位置变化曲线}
    \label{fig:5.27}
\end{figure}

\begin{figure}[h]
    \centering
    \includegraphics[width=\textwidth]{image.png}
    \caption{01号飞行器轨道估计的速度变化曲线}
    \label{fig:velocity_curve}
\end{figure}

\begin{table}[h]
    \centering
    \caption{09号卫星观测01号飞行器的轨道位置估计}
    \label{tab:position_estimation}
    \begin{tabular}{|c|c|c|c|}
        \hline
        时刻/s & X分量/m & Y分量/m & Z分量/m \\
        \hline
        50 & 1.76833E+06 & 8.20766E+06 & 4.55839E+06 \\
        \hline
        60 & 1.71135E+06 & 8.21247E+06 & 4.60318E+06 \\
        \hline
        70 & 1.65372E+06 & 8.21664E+06 & 4.64783E+06 \\
        \hline
        80 & 1.59553E+06 & 8.21957E+06 & 4.69176E+06 \\
        \hline
        90 & 1.53695E+06 & 8.22069E+06 & 4.73438E+06 \\
        \hline
        100 & 1.47815E+06 & 8.21962E+06 & 4.77525E+06 \\
        \hline
        110 & 1.41932E+06 & 8.21622E+06 & 4.81415E+06 \\
        \hline
        120 & 1.36055E+06 & 8.21066E+06 & 4.85116E+06 \\
        \hline
        130 & 1.30188E+06 & 8.20327E+06 & 4.88655E+06 \\
        \hline
        140 & 1.24327E+06 & 8.19449E+06 & 4.92069E+06 \\
        \hline
        150 & 1.18463E+06 & 8.18475E+06 & 4.95400E+06 \\
        \hline
        160 & 1.12584E+06 & 8.17442E+06 & 4.98683E+06 \\
        \hline
        170 & 1.06677E+06 & 8.16379E+06 & 5.01949E+06 \\
        \hline
    \end{tabular}
\end{table}

\begin{table}[h]
    \centering
    \caption{09号卫星观测01号飞行器的轨道速度估计}
    \label{tab:velocity_estimation}
    \begin{tabular}{|c|c|c|c|}
        \hline
        时刻/s & X分量/m/s & Y分量/m/s & Z分量/m/s \\
        \hline
        50 & -5.66302E+03 & 4.94644E+02 & 4.46783E+03 \\
        \hline
        60 & -5.73152E+03 & 4.58978E+02 & 4.48106E+03 \\
        \hline
        70 & -5.79341E+03 & 3.65251E+02 & 4.43923E+03 \\
        \hline
        80 & -5.84177E+03 & 2.10753E+02 & 4.33639E+03 \\
        \hline
    \end{tabular}
\end{table}

\begin{table}
\centering
\begin{tabular}{|c|c|c|c|}
\hline
90 & -5.87199E+03 & 6.58860E+00 & 4.17981E+03 \\
\hline
100 & -5.88374E+03 & -2.23854E+02 & 3.98968E+03 \\
\hline
110 & -5.88127E+03 & -4.52255E+02 & 3.79284E+03 \\
\hline
120 & -5.87174E+03 & -6.54312E+02 & 3.61408E+03 \\
\hline
130 & -5.86276E+03 & -8.15683E+02 & 3.46972E+03 \\
\hline
140 & -5.86054E+03 & -9.32623E+02 & 3.36601E+03 \\
\hline
150 & -5.86919E+03 & -1.00898E+03 & 3.30143E+03 \\
\hline
160 & -5.89096E+03 & -1.05237E+03 & 3.27011E+03 \\
\hline
170 & -5.92695E+03 & -1.07096E+03 & 3.26505E+03 \\
\hline
\end{tabular}
\caption{龙格库塔法计算的01号空间飞行器轨道位置估计}
\end{table}

\begin{table}
\centering
\begin{tabular}{|c|c|c|c|}
\hline
时刻/s & X分量/m & Y分量/m & Z分量/m \\
\hline
50 & 1.76850E+06 & 8.20620E+06 & 4.55706E+06 \\
\hline
60 & 1.71317E+06 & 8.19843E+06 & 4.58844E+06 \\
\hline
70 & 1.65717E+06 & 8.19217E+06 & 4.61961E+06 \\
\hline
80 & 1.60071E+06 & 8.18579E+06 & 4.65011E+06 \\
\hline
90 & 1.54397E+06 & 8.17802E+06 & 4.67962E+06 \\
\hline
100 & 1.48706E+06 & 8.16735E+06 & 4.70793E+06 \\
\hline
110 & 1.43010E+06 & 8.15156E+06 & 4.73494E+06 \\
\hline
120 & 1.37314E+06 & 8.12751E+06 & 4.76068E+06 \\
\hline
130 & 1.31621E+06 & 8.09138E+06 & 4.78523E+06 \\
\hline
140 & 1.25929E+06 & 8.04010E+06 & 4.80876E+06 \\
\hline
150 & 1.20232E+06 & 7.97432E+06 & 4.83150E+06 \\
\hline
160 & 1.14523E+06 & 7.90132E+06 & 4.85373E+06 \\
\hline
170 & 1.08788E+06 & 7.83420E+06 & 4.87579E+06 \\
\hline
\end{tabular}
\caption{龙格库塔法计算的01号空间飞行器轨道速度估计}
\end{table}

\begin{table}
\centering
\caption{01号空间飞行器轨迹的残差}
\begin{tabular}{|c|c|c|c|}
\hline
 & X分量 & Y分量 & Z分量 \\
\hline
位置/m & 6.60540E+003 & 1.00580E+005 & 4.42672E+004 \\
\hline
速度/m/s & 8.13220E+000 & 2.00701E+003 & 1.22304E+002 \\
\hline
\end{tabular}
\end{table}

\subsubsection{5.4.1.2 无约束最优化轨道估计模型}

为了实现单星对单飞行器对应时刻的空间轨道估计,这里需要做出的假设同 5.4.1.1 节。在时间周期 $[t_0, t_0 + T]$ 内,有 $N+1$ 个采样时刻,记为 $t_i, (i = 0, 1 \cdots, N)$,对应 $t_i$ 时刻飞行器在基础坐标系下的坐标为 $\vec{r}_i = \left( r_{xi}, r_{yi}, r_{zi} \right)$,速度 $\vec{v}_i = \left[ v_{xi}, v_{yi}, v_{zi} \right]^T$ 在观测坐标系下的坐标 $\vec{r}_{si} = \left( r_{xsi}, r_{ysi}, r_{zsi} \right)^T$,卫星观测向量 $\vec{m}_i = \left[ \alpha_i, \beta_i \right]^T$。

根据观测卫星对空间飞行器的观测关系,显然有:
\begin{equation}
\left\{
\begin{aligned}
\alpha_i &= \frac{y_{si}}{x_{si}} \\
\beta_i &= \frac{z_{si}}{x_{si}}
\end{aligned}
\right.
\tag{5.4-5}
\end{equation}

根据假设 1 空间飞行器在改时间段内的速度保持不变,即
\begin{equation}
\vec{v}_i = \vec{v}_0 \qquad (i = 0, 1, \cdots, N)
\tag{5.4-6}
\end{equation}

根据运动学公式有
\begin{equation}
\vec{r}_i = \vec{R}_i + \left( t_i - t_0 \right) \vec{v}_0 \qquad (i = 0, 1, \cdots, N)
\tag{5.4-7}
\end{equation}

由公式 (5.2-9) 知
\begin{equation}
\vec{r}_i = \vec{R}^i + x_{si} \vec{E}_i \left[ 1, \vec{m}_i \right]^T \qquad (i = 0, 1, \cdots, N)
\tag{5.4-8}
\end{equation}

其中:$\vec{R}^i$ 为 $t_i$ 时刻卫星的空间位置,$\vec{E}_i$ 为 $t_i$ 时刻卫星的坐标转换矩阵。

将式 (5.4-8) 进一步化简为
\begin{equation}
\vec{H}_i = \left[ 1, \vec{m}_i \right]^T = \frac{\vec{E}_i^{-1}}{x_{si}} \left( \vec{r}_i - \vec{R}_0^i \right) \quad (i = 0, 1, \cdots, N)
\tag{5.4-9}
\end{equation}

式 (5.4-9) 给出了根据各个时刻空间飞行器空间位置反解出观测向量的关系式。

记增广观测向量 $\vec{H}_i = \left[ 1, \vec{m}_i^T \right]^T$,有 $\vec{H}_i = \left[ h_i^1, h_i^2, h_i^3 \right]^T$,则

\begin{equation}
\begin{bmatrix}
h_{i}^{1} \\
h_{i}^{2} \\
h_{i}^{3}
\end{bmatrix}
=
\begin{bmatrix}
f_{i}^{1}\left(\vec{r}_{i}, \vec{R}_{x}^{i}, t_{i}, x_{si}\right) \\
f_{i}^{2}\left(\vec{r}_{i}, \vec{R}_{y}^{i}, t_{i}, x_{si}\right) \\
f_{i}^{3}\left(\vec{r}_{i}, \vec{R}_{z}^{i}, t_{i}, x_{si}\right)
\end{bmatrix}
\tag{5.4-10}
\end{equation}

其中:$h_{i}^{j}$ 和 $f_{i}^{j}$ 中的下标 $i$ 为时间索引且 $i=0,1,\cdots,N$,其上标 $j$ 为方向索引,$j=1,2,3$ 分别表示为 $x,y,z$ 方向。

综合式(5.4-5)-(5.4-10)可建立待求变量(即 $\vec{r}_{0}, \vec{v}_{0}, x_{si}|_{i=0}^{N}$,共 $N+7$ 个待求变量)与函数关系:

\begin{equation}
\begin{bmatrix}
h_{i}^{1} \\
h_{i}^{2} \\
h_{i}^{3}
\end{bmatrix}
=
\begin{bmatrix}
g_{i}^{1}\left(r_{x0}, v_{x0}, R_{xi}, t_{i}, x_{si}\right) \\
g_{i}^{2}\left(r_{y0}, v_{y0}, R_{yi}, t_{i}, x_{si}\right) \\
g_{i}^{3}\left(r_{z0}, v_{z0}, R_{zi}, t_{i}, x_{si}\right)
\end{bmatrix}
= G(\vec{r}_{x0}, \vec{v}_{x0}, R_{xi}, t_{i}, x_{si})
\tag{5.4-11}
\end{equation}

因此为求解目的变量,建立了下述无约束最优化模型。

目标函数为:

\begin{equation}
\min \quad F = \sum_{i=0}^{N} \left[ G(\vec{r}_{x0}, \vec{v}_{x0}, R_{xi}, t_{i}, x_{si}) - \bar{H}_{i} \right]^{T} \left[ G(\vec{r}_{x0}, \vec{v}_{x0}, R_{xi}, t_{i}, x_{si}) - \bar{H}_{i} \right]
\tag{5.4-12}
\end{equation}

即

\begin{equation}
F = \sum_{i=0}^{N} \sum_{j=1}^{3} \left( g_{i}^{j}\left(r_{i}, v_{i}, R_{i}, t_{i}, x_{si}\right) - h_{i}^{j} \right)^{2}
\tag{5.4-13}
\end{equation}

令

\begin{equation}
\left\{
\begin{aligned}
\frac{\partial F}{\partial r_{x0}} &= 0 & \frac{\partial F}{\partial r_{y}} &= 0 & \frac{\partial F}{\partial r_{z0}} &= 0 \\
\frac{\partial F}{\partial v_{x0}} &= 0 & \frac{\partial F}{\partial v_{y0}} &= 0 & \frac{\partial F}{\partial v_{z0}} &= 0 \\
\frac{\partial F}{\partial x_{si}} &= 0 & (i=0,1,\cdots,N)
\end{aligned}
\right.
\tag{5.4-14}
\end{equation}

利用式(5.4-14)可得 $N+7$ 个关于待求目的变量 $\vec{r}_{0}, \vec{v}_{0}, x_{si}|_{i=0}^{N}$ 的方程,联立上述方程求解即可得出待求目的变量,继而达到轨道估计的目的。

需指出的一点是:该模型所建立的目标函数 $F$ 是二次的,则基于式(5.4-14)得到的方程组必然是非线性的,形式较为复杂,计算量较大。而且对非线性方程组求解时,现有的大部分算法往往需要设定初值,然而初值的选取比较困难。

\subsubsection{5.4.1.3 基于轨道平面约束的轨道估计模型}

使用 09 号卫星观测 01 号飞行器,要通过单颗卫星估计飞行器的轨道是十分困难的,但如果能利用飞行器在重力斜飞段后半程飞行轨道的特性,再加上某种强制性的约束,得到一定精度下轨道参数的估计仍然是可能的。由于重力斜飞段后半程是进入自由状态的最后阶段,空间曲线相对弯曲程度较小。

我们不妨假设在观测坐标系下,飞行器横坐标关于时间 \( t \) 的运动规律是二次曲线,即 \( x_s(t) = a_0 + a_1 t + a_2 t^2 \),其中 \( a_0 \), \( a_1 \), \( a_2 \) 是待估参数。而重力斜飞段后半程轨道上各时间节点对应的空间坐标基本处于同一平面上。因此,我们可利用空间曲线 \( x_s(t) \) 的二次特性及空间点处于同一平面的强性约束条件可得到轨道估计。由此可得

\[
\vec{r}_c(t_i) = \vec{r}_{09}(t_i) + (a_0 + a_1 t_i + a_2 t_i^2) \mathbf{E}(t_i) \begin{pmatrix} 1 \\ \alpha(t_i) \\ \beta(t_i) \end{pmatrix}
\tag{5.4-15}
\]

令 \( \mathbf{n}_0 = \begin{pmatrix} a_0 \\ a_1 \\ a_2 \end{pmatrix} \), \( \mathbf{T}(t_i) = \begin{pmatrix} 1 \\ t_i \\ t_i^2 \end{pmatrix} \), \( \vec{\varepsilon}(t_i) = \mathbf{E}(t_i) \begin{pmatrix} 1 \\ \alpha(t_i) \\ \beta(t_i) \end{pmatrix} \),则 (5.4-15) 式可化为

\[
\vec{r}_c(t_i) = \vec{r}_{09}(t_i) + \mathbf{n}_0^T \mathbf{T}(t_i) \vec{\varepsilon}(t_i)
\tag{5.4-16}
\]

设 \( \vec{r}_c(t_i) \) 落在同一平面 \( \mathbf{n}^T \begin{pmatrix} x \\ y \\ z \end{pmatrix} + 1 = 0 \) 上,其中 \( \mathbf{n} = \begin{pmatrix} A \\ B \\ C \end{pmatrix} \) 为该平面的法向量,因此有

\[
\mathbf{n}^T \vec{r}_{09}(t_i) + \mathbf{n}^T \mathbf{n}_0^T \mathbf{T}(t_i) \vec{\varepsilon}(t_i) + 1 = 0
\]

需待定参数 \( \mathbf{n} = \begin{pmatrix} A \\ B \\ C \end{pmatrix} \) 和 \( \mathbf{n}_0 = \begin{pmatrix} a_0 \\ a_1 \\ a_2 \end{pmatrix} \)。

构造模型参数估计的最优平方逼近模型:

\[
\min f(\mathbf{n}, \mathbf{n}_0) = \sum_{i=1}^m \left( \mathbf{n}^T \vec{r}_{09}(t_i) + \mathbf{n}^T (\mathbf{n}_0^T \mathbf{T}(t_i) \vec{\varepsilon}(t_i)) + 1 \right)^2
\tag{5.4-17}
\]

则梯度:

\[
\begin{cases}
\nabla f_{\mathbf{n}} = \sum_{i=1}^m 2 \left( \vec{r}_{09}(t_i) + \mathbf{n}_0^T \mathbf{T}(t_i) \vec{\varepsilon}(t_i) + 1 \right) \left( \vec{r}_{09}(t_i) + \mathbf{n}_0^T \mathbf{T}(t_i) \vec{\varepsilon}(t_i) \right) = 0 \\
\nabla f_{\mathbf{n}_0} = \sum_{i=1}^m 2 \left( \mathbf{n}^T \vec{r}_{09}(t_i) + \mathbf{n}^T (\mathbf{n}_0^T \mathbf{T}(t_i) \vec{\varepsilon}(t_i)) + 1 \right) \left( \mathbf{n}^T \mathbf{T}(t_i) \vec{\varepsilon}(t_i) \right) = 0
\end{cases}
\tag{5.4-18}
\]

求解梯度方程 (5.4-18) 得出参数 \( A, B, C, a_0, a_1, a_2 \),进而得到飞行器的定位方程 (5.4-15) 对飞行器进行轨道估计。

由于时间关系,本模型仅给出了理论推导,并未进行仿真实验,因此该模型建立的好坏有待下一步研究。

\subsubsection{5.4.2 多卫星多目标联合系统误差估计}

实现多卫星多空间飞行器的联合定位与系统误差估计,一个基本前提确定各卫星获得的量测集合与目标的映射关系。在下文的论述中,我们认为 “哪个量测来源于哪个目标” 是已知的,对此类数据关联问题不作具体分析。

实现多颗观测卫星多空间飞行器联合轨道与系统误差估计,需满足一定的条件。利用任意一个时刻的量测信息,都可以采用交汇定位方法建立观测卫星的系统误差和多个空间飞行器的轨道之间的数量关系。不失一般性,我们讨论有 $M$ 颗观测卫星,$P$ 个空间飞行器联合轨道和系统误差估计。

记任一时刻 $t$ 卫星在基础坐标系下的坐标为 $\vec{r}_{m}(t), (m=1,2,\cdots,M)$,设 $P$ 个空间飞行器在基础坐标系下的坐标为 $\vec{r}_{c_{p}}, (p=1,2,\cdots,P)$(待求坐标),空间飞行器 $p$ 在卫星 $m$ 的观测坐标系下的坐标为 $\vec{r}_{sm_{n}}(t), (m=1,2,\cdots,M, n=1,2,\cdots,N)$。卫星 $m$ 的系统误差记为 $d_{\alpha m}, d_{\beta m}, d_{\theta m}, (m=1,2,\cdots,M)$,其中 $d_{\theta m}$ 可由 $d_{\alpha m}$ 和 $d_{\beta m}$ 计算。因此,我们暂时不考虑 $d_{\theta m}$ 的估计。只考虑 $d_{\alpha m}, d_{\beta m}$ 的估计,由此可解系统估计误差如下:

\[
\vec{r}_{c_{p}}(t) = \vec{r}_{m}(t) + x_{sm_{p}}(t) E_{n}(t)
\begin{pmatrix}
1 \\
\alpha_{n} + d_{\alpha m} \\
\beta_{n} + d_{\beta m}
\end{pmatrix}
\quad p=1,2,\cdots,P \quad m=1,2,\cdots,M
\]

上述方程组中共有 $3MP$ 个方程,而待求的未知数及其数量分别为:空间飞行器在基础坐标系下的坐标变量为 $3P$ 个;空间飞行器在 $M$ 个卫星的观测坐标系下的横坐标变量共有 $MP$;待确定卫星的系统误差有 $2M$。

由上分析不难发现,待求的未知变量数目 $Q$ 为:

\[
Q = 3P + MP + 2M
\]

联立 $3MP$ 个方程求解 $Q$ 个未知数,有解的基本条件是:

\[
Q \leq 3MP
\]

化简可得:

\[
3P + 2M \leq 2MP
\]

下表 5.24 给出了不同 $M$、$P$ 取值时可以得到的方程数目与未知量个数,以及进行轨道和系统误差估计的可行性。

\begin{table}[h]
\centering
\caption{卫星数与飞行器数与能否确定系统误差的关系}
\begin{tabular}{|c|c|c|c|c|}
\hline
卫星数 $M$ & 飞行器数 $P$ & 方程个数 & 未知量个数 & 能否估计系统误差 \\
\hline
2 & 1 & 6 & 9 & 否 \\
\hline
2 & 2 & 12 & 14 & 否 \\
\hline
2 & 3 & 18 & 19 & 否 \\
\hline
3 & 3 & 27 & 24 & 能 \\
\hline
4 & 2 & 24 & 22 & 能 \\
\hline
4 & 3 & 36 & 29 & 能 \\
\hline
\end{tabular}
\end{table}

由上式可以看出,为实现多卫星多飞行器联合轨道和系统误差估计,需采用多个时刻的量测数据集合,卫星数目 $M$、飞行器数目 $P$ 以及观测时刻数目 $N$ 需满足 $3P + 2M \leq 2MP$。

\section{六、模型优缺点分析}

\subsection{模型的优点}

1、各模型的算法实现具有可行性,实时性和精度都能够满足实际轨道估计的应用需要;

2、每一问的模型都给出了精确的表达式和推导;

3、第四问的采用了两种方法仿真模型,对模型一进行了仿真分析,并用采用该方法对0号飞行器进行了定位,得出的结果与双星逐点交汇定位比较,验证了该模型的正确性,具有很高的现实性。

\subsection{模型的缺点}

对于飞行器运动模型的具体参数考虑不够,与实际情况存在一定偏差,虽然在本文数据及仿真环境下得到了一定验证,但仍需在实际中检验。

\section{参考文献:}

[1] 范新南. 现代数值计算[M]. 北京: 人民邮电出版社, 2009:225-240

[2] 谢文龙. 三次样条函数的构造方法[J]. 江南学院学报, 2000, 5(1).33-36

[3] 王正明, 易东云. 测量数据建模与参数估计[M]. 长沙: 国防大学出版社, 1996.

[4] 邓乃扬. 无约束最优化算法[M]. 北京: 科学出版社, 1982.

[5] 韩云瑞, 刘庆华. 高等数学[M]. 北京: 清华大学出版社, 2002.

[6] 张信守. 外弹道测量与卫星轨道测量基础[M]. 北京: 国防工业出版社. 1992.

[7] 孙仲康, 周一宇, 何黎星. 单多基地有源无源定位技术[J]. 北京: 国防出版社. 1996.

[8] 李盾, 周一宇, 苗雨. 空军预警探测子系统弹道导弹的监视与跟踪[J]. 系统工程与电子技术, 200, 24(3):248-251.

[9] 贾沛然, 陈克俊, 何力. 远程火箭弹道学[M]. 长沙: 国防科技大学出版社, 1993

[10] 余明生. 地球观测卫星的轨道捕获和轨道保持[J]. 中国空间科学技术, 1982, 8(3).

[11] 张铮, 杨文平, 石博强, 等. MATLAB 程序设计与实例应用[M]. 北京: 北京邮电出版社, 2003.

[12] 孙继广. 矩阵扰动分析[M]. 北京: 科学出版社, 1987.

\end{document}