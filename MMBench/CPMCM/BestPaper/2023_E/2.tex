\documentclass{article}
\usepackage{amsmath}
\usepackage{amssymb}
\usepackage{graphicx}

\title{出血性脑卒中临床智能诊疗建模分析}
\author{}
\date{}

\begin{document}

\maketitle

\begin{abstract}
出血性脑卒中是一种急性脑血管疾病,主要由创伤性和自发性两种因素导致的脑部局部血管破裂造成脑部机械性损伤,常伴随引发一系列复杂的生理病理反应。我国脑卒中疾病负担仍然沉重,其防治虽然取得了一定进展,但仍面临诸多挑战。本文通过整合真实临床患者的影像学特征、患者临床信息及临床诊疗方案等数据,针对出血性脑卒中进行了一系列的预测和统计分析。

本文针对患者的个人史、疾病史、治疗历史以及首次+随访的影像信息进行了合理的量化分析,进行了缺失值和异常值的处理。也通过给出的数据集对患者是否发生血肿扩张以及 90 天的 mRS 评分进行了预测,且分析了血肿体积、水肿体积和治疗方案三种之间的关系,并构建了相应的数学模型。本文综合利用了 CNN 卷积神经网络、灰色关联度分析、主成分分析、K-means 三维聚类算法、BP 神经网络、斯皮尔曼相关性分析等方法进行了研究和预测。

针对问题一(a): 判断 48 小时内是否发生血肿扩张事件,根据题目所给的判断血肿体积前后变化方法来对原有数据进行归一化、标准化处理,去除了一些异常值,最终通过题目中所给血肿扩张条件进行判断得到前 100 位患者是否发生血肿扩张。针对问题一(b),对特征数据进行分类,使用灰色关联分析来获得与是否发生血肿扩张最具关联性的十五个特征数据,使用 CNN 卷积神经网络根据阈值进行二分类预测,最终得到所有患者是否发生血肿扩张的概率。并利用 RMSE 和 MAE 进行模型评价指标,关于血肿扩张的发生以及血肿扩张概率的最终结果为答案表中的列 C、列 D 和列 E 中。

针对问题二(a),将数据预处理后的水肿体积和时间点进行了曲线拟合,拟合出了一条较为准确的曲线,用拟合曲线分别对前 100 个患者按照拟合优度计算残差评价值。针对问题二(b),将血压分为收缩压和舒张压,使用 PCA 主成分分析来对已有特征进行降维,最终选定按照年龄和收缩压、舒张压作为特征变量,使用 K-means 三维聚类将 160 个患者分为 5 类,并对前 100 个患者按照所述类别分别进行曲线拟合,最终得到 5 条拟合较好的曲线。针对问题二(c),按照单位时间水肿体积变化,使用多元线性回归分析不同治疗方法对水肿进展模式的影响,并进行了如 $R^2$ 检测、SSE 等多个评价指标对拟合模型进行评价,得到了较好的评价结果。针对问题二(d),将单位时间血肿体积变化、单位时间水肿体积变化与治疗方法按照斯皮尔曼相关系数分析,通过热力图对三者之间的联系做了较为清晰的比较。最终答案记录在答案表列 F、列 G 和列 H 中。

针对问题三(a),使用灰色关联分析来选取与前 100 个患者 90 天 mRS 评分最相关的十五个特征变量,使用 BP 神经网络,按照阈值范围进行分类,并预测前 160 个患者的 90 天 mRS 评分。针对问题三(b),将单位时间血肿总体积及各个区域体积变化、单位时间水肿总体积及各个区域体积变化作为首次+随访影像的综合特征变量,与患者其他特征变量进行整合,最终通过灰色关联分析选取前 20 个特征变量,使用 BP 神经网络对具有随访影像的患者 90 天 mRS 评分进行预测。针对问题三(c),对之前所有特征变量,通过灰色关联分析获得了其与出血性脑卒中患者的预后 90 天 mRS 的相关性,通过对问题和实际诊疗的分析,并根据灰色关联度的大小来对临床的决策提出相关建议。最后关于 90 天 mRS 评分的结果填入答案表 I 列和 J 列。

关键词:脑血肿与水肿分析预测、CNN 卷积神经网络;K-means 三维聚类;BP 神经网络;
\end{abstract}

\begin{center}
\includegraphics[width=0.9\textwidth]{image.png}
\end{center}

\begin{center}
\textbf{中国研究生创新实践系列大赛}
\end{center}

\begin{tabular}{c c c}
MajorAxisLength & Maximum2DDiameterColumn & Maximum2DDiameterSlice \\
MeshVolume & Sphericity & SurfaceArea \\
\hline
\end{tabular}

\section*{目录}

\begin{itemize}
    \item[1.] 问题重述 \dotfill 5
        \begin{itemize}
            \item[1.1.] 问题背景 \dotfill 5
            \item[1.2.] 问题解析 \dotfill 6
        \end{itemize}
    \item[2.] 模型假设与符号说明 \dotfill 8
        \begin{itemize}
            \item[2.1.] 模型假设 \dotfill 8
            \item[2.2.] 符号说明 \dotfill 8
        \end{itemize}
    \item[3.] 问题一建模的分析与求解 \dotfill 9
        \begin{itemize}
            \item[3.1.] 数据集概述 \dotfill 9
                \begin{itemize}
                    \item[3.1.1.] 数据缺失处理 \dotfill 9
                    \item[3.1.2.] 数据分析 \dotfill 9
                    \item[3.1.3.] 数据异常分析 \dotfill 10
                \end{itemize}
            \item[3.2.] 问题(a)分析与计算过程 \dotfill 12
                \begin{itemize}
                    \item[3.2.1.] 问题一(a)模型建立 \dotfill 12
                    \item[3.2.2.] 问题一(a)模型求解 \dotfill 13
                \end{itemize}
            \item[3.3.] 问题(b)分析与计算过程 \dotfill 14
                \begin{itemize}
                    \item[3.3.1.] 问题一(b)基于CNN卷积神经网络模型建立 \dotfill 14
                    \item[3.3.2.] 问题一(b)模型求解与评价 \dotfill 17
                \end{itemize}
        \end{itemize}
    \item[4.] 问题二建模的分析与求解 \dotfill 18
        \begin{itemize}
            \item[4.1.] 问题二(a)分析与计算过程 \dotfill 18
                \begin{itemize}
                    \item[4.1.1.] 问题二(a)基于多项式拟合模型建立 \dotfill 18
                    \item[4.1.2.] 问题二(a)模型求解 \dotfill 19
                \end{itemize}
            \item[4.2.] 问题二(b)分析与计算过程 \dotfill 20
                \begin{itemize}
                    \item[4.2.1.] 数据处理与基于K-means聚类模型建立 \dotfill 20
                    \item[4.2.2.] 问题二(b)模型的求解 \dotfill 23
                \end{itemize}
            \item[4.3.] 问题二(c)分析与计算过程 \dotfill 29
                \begin{itemize}
                    \item[4.3.1.] 问题(c)数据预处理 \dotfill 29
                    \item[4.3.2.] 问题(c)基于多元回归分析模型建立 \dotfill 29
                    \item[4.3.3.] 问题(c)模型求解 \dotfill 30
                \end{itemize}
            \item[4.4.] 问题二(d)分析与计算过程 \dotfill 31
                \begin{itemize}
                    \item[4.4.1.] 问题数据预处理 \dotfill 31
                    \item[4.4.2.] 问题二(d)基于斯皮尔曼相关性分析模型建立 \dotfill 31
                    \item[4.4.3.] 问题二(d)模型求解 \dotfill 32
                \end{itemize}
        \end{itemize}
    \item[5.] 问题三建模的分析与求解 \dotfill 35
        \begin{itemize}
            \item[5.1.] 问题三(a)建模的分析与求解 \dotfill 35
                \begin{itemize}
                    \item[5.1.1.] 数据处理与基于BP神经网络模型建立 \dotfill 35
                    \item[5.1.2.] 问题三(a)模型的求解 \dotfill 38
                \end{itemize}
            \item[5.2.] 问题三(b)建模的分析与求解 \dotfill 39
                \begin{itemize}
                    \item[5.2.1.] 数据的处理与整合 \dotfill 39
                    \item[5.2.2.] 问题三(b)模型的求解 \dotfill 40
                \end{itemize}
            \item[5.3.] 问题三(c)建模的分析与求解 \dotfill 41
                \begin{itemize}
                    \item[5.3.1.] 数据处理 \dotfill 41
                    \item[5.3.2.] 数据分析 \dotfill 43
                \end{itemize}
        \end{itemize}
    \item[6.] 问题解决的模型评价与改进 \dotfill 48
        \begin{itemize}
            \item[6.1.] 模型的优点 \dotfill 48
            \item[6.2.] 模型的缺点 \dotfill 48
        \end{itemize}
    \item[] 参考文献 \dotfill 49
    \item[] 附录一 \dotfill 50
    \item[] 附录二 \dotfill 58
\end{itemize}

\section{问题重述}

\subsection{问题背景}

出血性脑卒中是一种急性脑血管疾病,主要由创伤性和自发性两种因素导致的脑部局部血管破裂造成脑部机械性损伤,常伴随引发一系列复杂的生理病理反应。出血性脑卒中具有起病急、进展快,预后较差等特点,该病在急性期内病死率高达 45-50%,约 80% 的患者会遗留较严重的神经功能障碍。全球疾病负担研究数据显示,2012 年——2022 年 10 年来,我国出血性脑卒中患病率、发病率、死亡率虽然总体呈现相对稳定的变化趋势,但都持续高于全球平均水平和英美日等发达国家同期水平。我国出血性脑卒中疾病负担仍然沉重,其防治虽然取得了一定进展,但仍面临诸多挑战。

\begin{figure}[h]
    \centering
    \includegraphics[width=\textwidth]{image1.png}
    \caption{近十年出血性脑卒中患病率统计(1/10万)}
    \label{fig:1-1}
\end{figure}

\begin{figure}[h]
    \centering
    \includegraphics[width=\textwidth]{image2.png}
    \caption{近十年出血性脑卒中发病率统计(1/10万)}
    \label{fig:1-2}
\end{figure}

\begin{figure}[h]
    \centering
    \includegraphics[width=\textwidth]{image1.png}
    \caption{近十年出血性脑卒中死亡率统计}
    \label{fig:death_rate}
\end{figure}

根据出血性脑卒中临床表现,血肿范围扩大和血肿周围水肿是其预后不良的两个关键危险因素。如图 \ref{fig:ct_scans} 是同一出血性脑卒中患者不同时间点 CT 平扫,图像中白色部位为血肿,红色部位为血肿周围水肿。其中,血肿范围会因脑组织受损、炎症反应等因素在出血发生后的短时间内逐渐扩大,导致颅内压迅速增加,进一步损害神经功能甚至威胁患者生命,其发生率在 13\% 至 38\% 之间。其次,血肿周围水肿是脑出血后的次生损伤指标,由此引起的脑组织受压会影响神经元功能,进一步损害脑组织,加剧患者的神经功能损伤。为此,对脑卒中患者血肿扩张和血肿周围水肿的发生及演进进行早期识别和预测对于改善患者预后具有重要意义。

\begin{figure}[h]
    \centering
    \includegraphics[width=\textwidth]{image2.png}
    \caption{同一出血性脑卒中患者不同时间点 CT 平扫图像[13]}
    \label{fig:ct_scans}
\end{figure}

\subsection{问题解析}

从上述背景和数据中可以看出,脑血管疾病会严重影响人的生命健康安全。故通过对给出的真实临床数据分析,使用机器学习的方法建立相应的预测分析模型,研究出血性脑卒中患者血肿扩张风险并预测概率、血肿周围水肿发生的演进规律,最终结合临床和影像信息,可以很大程度上为临床预防和治疗出血性脑卒中,并对患者疾病进展、康复或生存情况进行预测/评估。根据给出的问题进行分析:

问题 1:以血肿扩张事件作为目标变量进行建模,探索血肿扩张风险相关因素。其中:
(a) 问要求判断患者样本发病后 48 小时内是否发生血肿扩张。首先查看表 1 和表 2 数据,

对其中的数据进行清洗操作,处理异常值和缺失值。而要求判断病人发病 48 小时之内是否发生血肿扩张现象。则通过比较发病 48 小时内病人首次做检查的血肿总体积和最后做的一次检查血肿总体积的绝对体积是否增加 $6 \, \text{mL}$ 或者相对体积增加 $33\%$,即可判断患者是否发生了血肿扩张。计算的时间不仅包括每次检查的时间间隔,还需要额外加上表 1 中患者发病到首次进行检查的时间间隔。(b)问要求构建模型预测所有患者发生血肿扩张的概率。考虑到有多个变量会对预测结果产生影响,为了更好分析影响因素,通过相关性分析进行降维处理。而观察数据特征可以发现,数据分为离散型数据和连续性数据。为增加模型的准确度,将特征分为离散型和连续性,其中离散型特征有 16 个,连续性数据 35 个,分别进行相关性分析。从两类中分别选取与血肿扩张相关性最强的 5 个离散型变量和 10 个连续性变量作为模型的输入,并考虑利用 CNN 卷积神经网络解决预测血肿扩张发生的概率。

问题 2:对血肿周围水肿的发生及进展进行建模,探索治疗干预和水肿进展的关联关系。其中:(a)问要求构建水肿体积随时间进展曲线和计算残差。首先对数据进行预处理,从表 2 中提取前 100 个患者的水肿体积和检查时间数据,然后建立患者水肿体积(因变量 $y$ 表示水肿体积 $\text{ED\_volume}$)与相对于首次检查时间(自变量 $x$ 表示时间)的关系,采用回归分析拟合数据并获得拟合函数。最后,使用实际数据与拟合曲线计算残差;(b)问要求构建不同亚组的水肿体积曲线以探索个体差异影响。首先使用 PCA 主成分分析进行数据降维和特征提取,选出贡献率最高的两个特征变量,其次对数据进行标准化处理。然后使用 K-means 聚类曲算法度对患者进行分组并对每个分组使用回归分析拟合数据并获得拟合函数。最后使用实际数据和拟合曲线计算残差;(c)问要求分析不同治疗方法对水肿体积进展模式的影响。首先对数据进行处理,从表 1 中提取治疗方法数据,并与表 2 中的水肿体积数据进行整合,然后通过构建多元线性回归模型分析拟合系数并解释统计结果,进而分析治疗方法对水肿体积的影响;(d)问在(c)问的基础上,增加血肿体积这一关联因素,要求分析血肿体积、水肿体积和治疗方法三者之间的关系。首先对表 1 和表 2 中的数据进行整合,然后使用相关性分析研究血肿体积和水肿体积之间的关系,最后根据关联性分析的结果进行统计分析,探究三者之间的关系。

问题 3:建立出血性脑卒中患者的预后预测模型,探索该病预后关键因素。其中:(a)问要求基于患者首次影像预测患者 90 天 mRS 评分,首先对数据进行预处理,从表 1 中提取患者的个人史、疾病史和发病相关数据,从表 2 和表 3 中提取首次影像数据,对这些数据进行标准化与归一化处理。然后通过灰色关联分析对数据进行权重分析以识别关键特征。最后使用 BP 神经网络进行训练和预测;(b)问要求基于患者所有临床、治疗、首次影像和随访影像预测患者 90 天 mRS 评分;通过整合前 100 个患者的所有已知临床、治疗结果,预测所有含随访影像检查的患者。则需要整合所有的表 1、表 2 和表 3 的特征变量,包括患者个人史、治疗方法、疾病史、首次和随访的所有影像信息,并且对血肿/水肿体积在不同位置进行权值计算,统一进行统计分析。最后通过 BP 神经网络进行训练和预测 130 位患者的 90 天 mRS 评分;(c)问要求根据前两问的结果分析出血性脑卒中预后与多个因素的关联关系。考虑对所有特征变量进行灰色关联分析,通过对问题和实际诊疗的分析,并根据灰色关联度的大小来对临床的决策提出相关建议。

\section{模型假设与符号说明}

\subsection{模型假设}

1. 假设患者无其他基础疾病影响;\\
2. 假设该临床数据集数据真实有效,无虚假数据,且符合数据分析的基本要求;\\
3. 假设在患者每次随访影像检查间隔时间内不受其他突发因素影响;

\subsection{符号说明}

\begin{tabular}{c c}
评价指标 & 数值 \\
\hline
根均方差(RMSE) & 0.43109 \\
平均相对误差(MAE) & 0.34281 \\
\hline
\end{tabular}

\section{问题一建模的分析与求解}

\subsection{数据集概述}

观察给出的数据,可以看出本题给出了多变量的数据集,所以本文首先根据题目所给条件对原始数据进行一定的处理,观察表 1 和表 3 给出的关于病人个人史、疾病史、发病相关,以及影响检查结果,要求对前 100 位患者进行分析并根据附表 1 给出的检查时间来判断 48 小时内,患者是否发生血肿扩张,并且以是否发生血肿扩张事件为目标变量,基于上述数据前 100 例患者首次影像信息数据,构建模型预测所有患者(sub001 至 sub160)发生血肿扩张的概率。

由于多个变量使得对发生血肿扩张概率的预测变得十分复杂,考虑使用相关性分析。按照一定的法则来对数据的属性进行取舍达到降维的目的。相关性分析主要考量两组数据之间的相关性,以一种指标来判定,观察数据中哪些属性与目标数据的相关性较强,从而做出保留,哪些较弱,进行剔除。同时为了将不同特征的数据缩放到相同的尺度或范围内,以确保不同特征之间的权重在建立机器学习模型时能够平等地影响模型的训练和性能。如图 3-1 为本文数据处理的主要流程图。

\begin{figure}[h]
    \centering
    \includegraphics[width=0.8\textwidth]{data_processing_flowchart.png}
    \caption{数据处理流程图}
    \label{fig:data_processing_flowchart}
\end{figure}

\subsubsection{数据缺失处理}

读取表格数据,统计每次检查的数据,针对问题一(a),整理所用数据可以发现并没有缺失值的出现。

\subsubsection{数据分析}

\paragraph{归一化处理}

针对表 1 中患者的年龄和表 3 中患者影像信息多种形状特征和灰度特征进行归一化处理,由于不同特征可能具有不同的数值范围和单位。例如,一个特征的取值范围在 0 到 1

之间,而另一个特征的取值范围在 0 到 1000 之间。如果不进行归一化处理,那么数值较大的特征可能会在模型训练中占据主导地位,导致模型忽略了数值较小但重要的特征。

同时某些机器学习算法,如梯度下降,对输入数据的尺度非常敏感。特征尺度差异大的数据可能需要更多的迭代才能收敛到最优解。考虑到需要通过机器学习预测患者血肿扩张的概率,通过归一化处理,可以加快模型的收敛速度,减少训练时间。常见的数据归一化方法包括最小-最大归一化和标准化。最小-最大归一化将数据缩放到指定的范围内,通常是 $[0, 1]$。标准化将数据转化为均值为 0,标准差为 1 的分布,使数据更符合正态分布。

故选择最小-最大归一化(Min-Max Scaling)

\begin{align}
\text{HM\_ACA\_R}_{i\_\text{volume}} &= \text{HM\_ACA\_R\_Radio} \times \text{HM\_volume}_i \tag{5-10} \\
\text{ED\_ACA\_R}_{i\_\text{volume}} &= \text{ED\_ACA\_R\_Radio} \times \text{ED\_volume}_i \tag{5-11}
\end{align}

线性函数将原始数据线性化的方法转换到 $[0, 1]$ 的范围,计算结果为归一化后的数据,$x$ 为原始数据,$\min(x)$ 和 $\max(x)$ 代表原始数据集中的最小和最大值。具体的归一化处理内容如下。其中将患者血压分为舒张压和收缩压,并将其和年龄和发病到首次检查时间间隔一并归一化处理,将表 1 患者首次检查的流水号以及表 3 共计 576 次检查的流水号进行匹配处理,筛选出 160 名患者首次检查影像信息,并对其进行归一化处理。归一化处理后的表格数据分别为 `Sheet_1`, `Sheet_3`,使得数据处于同一尺度。

\subsection{离散数据分析}

在机器学习中,在处理离散数据时,通常需要进行特征工程,将离散数据转换为适合机器学习算法的特征表示形式。这可能包括独热编码、嵌入表示、特征哈希等技术,以确保机器学习模型能够有效地处理这些数据。

而在针对于表 1 中患者个人史的性别数据,定义

\begin{equation}
\text{Sheet_1}_{\text{F 列}} =
\begin{cases}
1, & \text{性别男} \\
0, & \text{性别女}
\end{cases}
\tag{3-2}
\end{equation}

\subsubsection{数据异常分析}

异常的数据值往往会干扰我们对问题的分析过程,影响建立模型的准确度,从而导致模型求解出的结果偏离实际情况,所以本文对表 1 和表 3 给出的信息进行了一定的异常值判断和处理。通过比较高斯分布 $3\sigma$ 原则以及箱型图异常值处理,最终选择箱型图的处理结果。

\paragraph{高斯分布 $3\sigma$ 原则}

正态分布的概率密度函数图像是关于均值点处对称的,假设总体服从均值为 $\mu$,标准差为 $\sigma$ 的正态分布,那么从该总体中随机抽取一个样本点,该点落在区间 $[\mu - 3\sigma, \mu + 3\sigma]$ 上的概率约为 99.73%,而超出这个范围的可能性仅占不到 0.3%,是典型的小概率事件,所以这些超出该范围的数据可以认为是异常值。为对数据进行可视化分析,选取 160 名患者首次影像检查信息进行 $3\sigma$ 原则去除异常值处理,由于影像信息特征比较多,会占用大量篇幅。所以本文选择了形状特征中的 `LeastAxisLength` 和 `MajorAxisLength` 以及灰度特征中的 `Energy` 和 `90Percentil` 进行了展示。

对所有的特征进行了数据可视化后,由于影像特征可以发现通过 $3\sigma$ 原则处理后的部分数据变化并不大,故各个特征并不都满足正态分布,所以考虑箱型图分析,如图 3-2 至 3-5

为挑选的 4 个典型较为典型的特征变量的对比图。

\begin{figure}[h]
    \centering
    \includegraphics[width=\textwidth]{image1.png}
    \caption{形状特征 LeastAxisLength 分布}
    \label{fig:leastaxislength}
\end{figure}

\begin{figure}[h]
    \centering
    \includegraphics[width=\textwidth]{image2.png}
    \caption{形状特征 MajorAxisLength 分布}
    \label{fig:majoraxislength}
\end{figure}

\begin{figure}[h]
    \centering
    \includegraphics[width=\textwidth]{image3.png}
    \caption{灰度特征 Energy 分布}
    \label{fig:energy}
\end{figure}

\begin{figure}[h]
    \centering
    \includegraphics[width=\textwidth]{image1.png}
    \caption{灰度特征 90Percentile 分布}
    \label{fig:3-5}
\end{figure}

\paragraph{箱型图}

箱线图则五个重要统计量:最小值、上四分位数、中位数、下四分位数和最大值。本文对前 100 例患者首次影像检查信息进行了箱型图分析,将其中的异常值用均值代替,从而减少异常值对模型的影响。如图 \ref{fig:3-6} 为患者首次影像信息的箱型图。

\begin{figure}[h]
    \centering
    \includegraphics[width=\textwidth]{image2.png}
    \caption{患者首次影像检查信息箱型图}
    \label{fig:3-6}
\end{figure}

\subsection{问题(a)分析与计算过程}

\subsubsection{问题一(a)模型建立}

问题一(a)需要计算患者 48 小时之内的血肿扩张情况,首要目标是对患者数据进行整合,确保能够跟踪每次检查时的血肿总体积。为了便于我们针对每个患者和每个时间点比较血肿体积的变化,我们将所给数据表进行一定的处理。

\subsubsection{表格数据预处理}

\begin{equation}
S_{\text{merged}} = f(\text{Sheet_1: } T_u, \text{Sheet_2: } \text{HM_volume}, \text{Sheet_x: } T_x)
\tag{3-3}
\end{equation}

其中: $S_{\text{merged}}$ 表示对表格数据合并预处理的结果; $f$ 表示一个合并操作; Sheet_1、Sheet_2、Sheet_x 分别表示附件表 1、附件表 2 和附件附表 1 数据; $T_u$、HM_volume、$T_x$ 分别表示首次发病到首次影像检查时间间隔、血肿体积、随访时间点。

\paragraph{血肿扩张发生时间的判别}

题目说明需要判别患者在发病 48 小时内是否发生血肿扩张, 若在 48 小时内发现血肿扩张, 即可确定发生时间。根据合并表格 $S_{\text{merged}}$, 初步筛选出 48 小时内所作的检查的 HM_volume, 初筛条件为

\begin{equation}
T = T_u + (T_1 - T_0) + (T_2 - T_1) + \cdots + (T_n - T_{n-1}) \leq 48h
\tag{3-4}
\end{equation}

其中:

- $T_u$ 表示首次发病到首次影像检查时间间隔;
- $T_0$ 表示入院首次检查时间点;
- $T_i$ 表示第 $i$ 次随访时间点, $i=1, 2, \cdots, 13$。

\paragraph{血肿体积变化}

\begin{equation}
\Delta \text{HM}_{\text{volume}} = \text{HM}_{\text{volume}_{48h_{\text{max}}}} - \text{HM}_{\text{volume}_{\text{first}}}
\tag{3-5}
\end{equation}

其中:

- $\Delta \text{HM}_{\text{volume}}$ 表示血肿体积变化;
- $\text{HM}_{\text{volume}_{\text{first}}}$ 和 $\text{HM}_{\text{volume}_{48h_{\text{max}}}}$ 分别表示首次检查的血肿体积和 48 小时内血肿体积的最大值。

\paragraph{血肿扩张发生条件}

是否发生血肿扩张可根据血肿体积前后变化, 具体定义为: 后续检查比首次检查绝对体积增加 $\geq 6 \, \text{mL}$ 或相对体积增加 $\geq 33\%$

\begin{equation}
\text{血肿扩张} =
\begin{cases}
1, & \text{如果 } \frac{\Delta \text{HM}_{\text{volume}}}{\text{HM}_{\text{volume}_{\text{first}}}} \geq 33\% \text{ 或 } \Delta \text{HM}_{\text{volume}} \geq 6 \, \text{ml} \\
0, & \text{否则}
\end{cases}
\tag{3-6}
\end{equation}

\paragraph{计算血肿扩张发生时间}

\begin{equation}
T_e = T_{48h_{\text{max}}} - T_{\text{first}}
\tag{3-7}
\end{equation}

其中:

- $T_e$ 表示血肿扩张发生时间;
- $T_{\text{first}}$ 和 $T_{48h_{\text{max}}}$ 分别表示第一次检查和在 48h 内检查血肿总体积最大的时间。

\subsubsection{问题一 (a) 模型求解}

将建立模型求得的数据记录到答案表 4 中, (3-8) 即为所得结果的数学模型。

\begin{equation}
\text{Sheet_4}_{\text{C列}} =
\begin{cases}
1, & \text{如果发生血肿扩张} \\
0, & \text{否则}
\end{cases}
\quad
\text{, Sheet_4}_{\text{D列}} =
\begin{cases}
\text{T}_{\text{e}}, & \text{如果发生血肿扩张} \\
\text{N/A}, & \text{否则}
\end{cases}
\tag{3-8}
\end{equation}

\subsection{问题(b)分析与计算过程}

\subsubsection{问题一(b)基于 CNN 卷积神经网络模型建立}

根据题目所给的要求,需要根据前 100 例患者的个人史、疾病史、发病相关及首次检查的影像信息进行一系列分析,并以问题 a 求解出的结果作为目标变量构建出相应的模型,预测所有病人血肿扩张的概率。考虑到有多个变量会对预测结果产生影响,为了更好分析影响因素,通过相关性分析进行降维处理。选取与血肿扩张相关性最强的 15 个变量作为模型的输入。并考虑利用 CNN 卷积神经网络算法解决预测血肿扩张发生的概率。而观察数据特征可以发现,数据分为离散型数据和连续性数据。为增加模型的准确度,将特征分为离散型和连续性,其中离散型特征有 16 个,连续性数据 35 个,分别进行相关性分析。如图 3-7 为问题一完整的研究思路。

\begin{figure}[h]
\centering
\includegraphics[width=\textwidth]{image.png}
\caption{问题一研究思路图}
\end{figure}

\subsubsection{特征工程}

定义特征序列,其中序列中的值已经进行了归一化处理。其中离散型特征为 \(X_{\text{i}}\),连续性数据为 \(\vec{X}_{\text{i}}\)。

\begin{equation}
X_{\text{i}} = (X_{1}(k), X_{2}(k), \dots, X_{16}(k))
\tag{3-9}
\end{equation}

\begin{equation}
\vec{X}_{\text{i}} = (\vec{X}_{1}(k), \vec{X}_{2}(k), \dots, \vec{X}_{35}(k))
\tag{3-10}
\end{equation}

对表格数据进行分析处理

\begin{equation}
X_{\text{merged}} = f(\text{Sheet_1}: X_{\text{E-w}}, \text{Sheet_3}: X_{\text{C-w}})
\end{equation}

其中:\(X_{\text{merged}}\) 表示对表格特征合并的结果;\(f\) 表示一个合并操作;Sheet_1、Sheet_3

分别表示进行归一化处理后的附件表 1 和附件附表 3 数据;$X_{\mathrm{E-W}}$、$X_{\mathrm{C-W}}$ 分别表示表格 E 列到 W 列和 C 列到 W 列的特征数据。

\paragraph{灰色关联分析}

通过分别对 $X^{'i}$ 和 $X_i$ 进行灰色关联分析。灰色关联分析是一种多变量数据分析方法,用于研究不同变量之间的关联程度。根据最后的关联程度来选择相关性最强的几个数据序列进行训练,从而达到降维的效果。使得数据分析的结果更加可靠,并且改善数据处理和分析的效率,降低计算成本。

确定分析数列:

母序列:反应是否发生血肿扩张的数据序列,此处即为 $Y = [y_1, y_2, \ldots, y_m]^T$

子序列:影响是否发生血肿扩张的因素组成的数据序列,即为 $X_i$ 和 $X_i^{'}$

对母序列和子序列中的每个指标进行预处理,先求出每个指标的均值,再用该指标中的每个元素都除以其均值。

\begin{equation}
Z_i(k) = \frac{X_i(k)}{\frac{1}{n} \sum_{i=1}^n X_i}
\tag{3-11}
\end{equation}

得到标准化矩阵 $Z$:

\begin{equation}
Z =
\begin{bmatrix}
Z_{11} & Z_{12} & \cdots & Z_{1m} \\
Z_{21} & Z_{22} & \cdots & Z_{2m} \\
\vdots & \vdots & \ddots & \vdots \\
Z_{n1} & Z_{n2} & \cdots & Z_{nm}
\end{bmatrix}
\tag{3-12}
\end{equation}

计算灰色关联系数:

\begin{equation}
\varepsilon_i(k) = \frac{\min_i \min_k |x_0(k) - x_i(k)| + \rho \cdot \max_i \max_k |x_0(k) - x_i(k)|}{|x_0(k) - x_i(k)| + \rho \cdot \max_i \max_k |x_0(k) - x_i(k)|}
\tag{3-13}
\end{equation}

将 $i$ 看作固定值,此时公式如下

\begin{equation}
\varepsilon_i(k) = \frac{\min_k |x_0(k) - x_i(k)| + \rho \cdot \min_k |x_0(k) - x_i(k)|}{|x_0(k) - x_i(k)| + \min_k |x_0(k) - x_i(k)|}
\tag{3-14}
\end{equation}

计算灰色关联度

\begin{equation}
r_{0i} = \frac{1}{m} \sum_{k=1}^m \varepsilon_i(k)
\tag{3-15}
\end{equation}

对于某一个因素,其中的每个维度进行计算,得到一个新的序列,这个序列中的每个点就代表着该子序列与母序列对应维度上的关联性,所得数值越大,代表关联性越强。图为经过灰色关联分析的所有特征变量的灰色相关度,根据特征变量的数据类型分为了离散型图 3-8 和连续性图 3-9 两类。

\begin{figure}[h]
    \centering
    \includegraphics[width=\textwidth]{image1.png}
    \caption{连续特征值与是否发生血肿扩张之间关联度}
    \label{fig:3-8}
\end{figure}

\begin{figure}[h]
    \centering
    \includegraphics[width=\textwidth]{image2.png}
    \caption{离散特征值与是否发生血肿扩张之间关联度}
    \label{fig:3-9}
\end{figure}

从连续特征值中选择关联度最高的5个特征序列,从离散特征值中选择关联度最高的10个特征序列作为训练特征。如表\ref{tab:3-1}即为筛选后的结果。

\begin{tabular}{c c c}
\hline
p4 & 644.4 & (-395.5, 1684) \\
p5 & -4146 & (-1.186e+04, 3569) \\
p6 & 1.943e+04 & (4528, 3.434e+04) \\
\hline
\end{tabular}

\begin{tabular}{c c c}
MajorAxisLength & Maximum2DDiameterColumn & Maximum2DDiameterSlice \\
MeshVolume & Sphericity & SurfaceArea \\
\hline
\end{tabular}

\section{CNN神经网络预测}

将通过灰色关联度筛选出的特征序列作为变量,考虑到具有是时序数据,则在Matlab工具箱中利用CNN卷积神经网络进行预测。CNN由多个卷积层、池化层、全连接层和输出层组成。

首先,将数据集划分为训练集和测试集,采用前100例患者的数据,将归一化、标准化的数据集$S_{\text{merged}}$、$X_{\text{merged}}$、Sheet_1和Sheet_3进行整合后传入CNN,最后可以使用RMSE和MAE值指标来评估模型的性能。

\subsection{问题一(b)模型求解与评价}

通过CNN卷积神经网络进行预测后的关于160例患者是否会发生血肿扩张概率在图3-10中展示,同时表3-2是通过验证后的模型评价。

\textbf{表3-2 模型评价指标}

\begin{tabular}{c c}
评价指标 & 数值 \\
\hline
根均方差(RMSE) & 0.43109 \\
平均相对误差(MAE) & 0.34281 \\
\hline
\end{tabular}

\begin{figure}[h]
\centering
\includegraphics[width=\textwidth]{image.png}
\caption{CNN卷积神经网络预测发生血肿扩张概率图}
\end{figure}

\textbf{图3-10 模型预测概率的分布}

\section{问题二建模的分析与求解}

\subsection{问题二(a)分析与计算过程}

根据问题一描述,血肿周围的水肿作为脑出血后继性发损伤的标志,对其发生及发展的预测也有利于临床治疗。根据问题二中的要求,需要构建前 100 个患者的水肿体积随时间变化进展曲线,并计算真实数据与拟合值之间存在的残差。如图 4-1 即为问题二的整体研究思路。

\begin{figure}[h]
\centering
\includegraphics[width=0.8\textwidth]{image.png}
\caption{问题二研究思路图}
\end{figure}

\subsubsection{问题二(a)基于多项式拟合模型建立}

\paragraph{特征工程}

根据表 2 给出的数据,可以知道每名患者的 ED\_volume 数据量是有限的,为了更好的构建出数据拟合曲线,需要对数据进行一定的处理。

根据流水号对应的检查时间可以知道,一些患者在首次影像检查与后几次的复查时间存在巨大的时间差,该类型数据由于时间跨度太大,患者自身的身体情况不好判断,所以本文对水肿随时间变化的研究同问题一相同,假设在发病 48 小时之外的水肿体积变化不具有分析价值,故采取 48 小时内的水肿体积 ED\_volume 变化。

\paragraph{模型建立与拟合}

使用散点图绘制出每个患者 48 小时内水肿体积 ED\_volume。则散点图内包含一组数据,假设为 $m$ 个点,称之为样本点。其中 $x$ 轴表示时间,$y$ 轴表示水肿体积 ED\_volume。

\begin{equation}
\{(x_1, y_1)(x_2, y_2) \ldots (x_m, y_m)\}
\tag{4-1}
\end{equation}

样本点中的某个点可以表示为

\begin{equation}
(x_i, y_i), i = 1, 2, 3, \ldots, m
\tag{4-2}
\end{equation}

观察样本点,由于数据有限,且样本点较为离散。故采用多项式拟合。

\begin{equation}
\hat{y} = a_0 x^n + a_1 x^{n-1} + a_2 x^{n-2} + \ldots + a_{n-1} x + a_n
\tag{4-3}
\end{equation}

而拟合后的曲线需要通过一个标准来衡量拟合效果,本文使用 R-Squared 衡量回归方程整体的拟合度。首先计算样本的总平方和 TSS 和残差平方和 RSS,最后根据两者求 $R^{2}$

\begin{equation}
\text{TSS} = \sum_{i=1}^{m} (y_{i} - \bar{y})^{2}
\tag{4-4}
\end{equation}

\begin{equation}
\text{RSS} = \sum_{i=1}^{m} (\widehat{y}_{i} - y_{i})^{2}
\tag{4-5}
\end{equation}

\begin{equation}
R^{2} = 1 - \frac{\text{RSS}}{\text{TSS}} = 1 - \frac{\sum (Y_{\text{actual}} - Y_{\text{predict}})^{2}}{\sum (Y_{\text{actual}} - Y_{\text{mean}})^{2}}
\tag{4-6}
\end{equation}

$R^{2}$ 越大,拟合效果越好。$R^{2}$ 的最优值为 1,若模型预测为随机值,$R^{2}$ 有可能为负值。若预测值恒为样本期望,$R^{2}$ 为 0。理论上 $R^{2}$ 取值范围 $(-\infty, 1]$,正常取值范围为 $[0, 1]$。实际操作中通常会选择拟合较好的曲线计算 $R^{2}$,因此很少出现 $-\infty$。越接近 1,表明方程的变量对 $y$ 的解释能力越强,这个模型对数据拟合的也较好。越接近 0,表明模型拟合的越差。通过经验值若 $R^{2} > 0.4$,则判断拟合效果好。

\subsubsection{问题二(a)模型求解}

由于数据量较小,而数据偏差过大,所以采用了问题一的异常值处理方法,去除了部分异常值。通过拟合工具求得拟合曲线如图 4-2 所示:

\begin{figure}[h]
\centering
\includegraphics[width=\textwidth]{image.png}
\caption{问题二(a)模型拟合曲线}
\end{figure}

该模型使用了一个 5 次多项式来拟合数据,其公式为:

\begin{equation}
f(x) = p_{1} \cdot x^{5} + p_{2} \cdot x^{4} + p_{3} \cdot x^{3} + p_{4} \cdot x^{2} + p_{5} \cdot x + p_{6}
\tag{4-7}
\end{equation}

其中,如表 4-1 中 $p_{1}$ 至 $p_{6}$ 是拟合后多项式模型的系数,置信区间值为 95\%。

\begin{tabular}{c c}
\hline
拟合质量指标 & 拟合质量指标值 \\
\hline
SSE & 5.658e+08 \\
R-square & 0.4854 \\
Adjusted R-square & 0.33 \\
RMSE & 4960 \\
\hline
\end{tabular}

\begin{tabular}{c c c}
\hline
p4 & 644.4 & (-395.5, 1684) \\
p5 & -4146 & (-1.186e+04, 3569) \\
p6 & 1.943e+04 & (4528, 3.434e+04) \\
\hline
\end{tabular}

该模型拟合质量共有 4 项指标,误差平方和 (SSE) 表示模型拟合数据的误差总和、$R^2$ (R-square) 表示模型对总方差的解释程度、调整后的 $R^2$ (Adjusted R-square) 通常用于多项式拟合,它考虑了模型中的自由度。负值通常表示模型不适合拟合数据,均方根误差 (RMSE) 表示模型的预测误差的标准偏差,各项指标值如表 4-2 所示:

\textbf{表 4-2 问题二(a)模型拟合质量评价}

\begin{tabular}{c c}
\hline
拟合质量指标 & 拟合质量指标值 \\
\hline
SSE & 5.658e+08 \\
R-square & 0.4854 \\
Adjusted R-square & 0.33 \\
RMSE & 4960 \\
\hline
\end{tabular}

\subsection{问题二(b)分析与计算过程}

题目需要探究患者水肿体积随时间进展模式的个体差异,则考虑使用聚类算法 K-means 对患者进行分组,对不同组的患者再分别进行水肿体积变化随时间进展的拟合,查看表 1 可以知道,由于需要考虑的特征变量较多,利用 PCA 主成分分析选出贡献率最高的特征变量进行分析,从而达到降维的效果,提高计算效率。

\subsubsection{数据处理与基于 K-means 聚类模型建立}

\paragraph{PCA 主成分分析}

PCA 主成分分析方法是一种常用的数据降维和特征提取的统计方法。主要思想是通过线性变换,将原始数据映射到一个新的坐标系中,使得在新坐标系中的数据具有最大的方差。换句话说,PCA 的目标是找到数据中的主要方向(主成分),以便用较少的维度来表示数据,同时尽量保留数据的信息。

\paragraph{对数据进行相应的标准化处理}

如问题一处理后的数据。考虑到本题需要的指标变量与问题一的特征有重合关系,用 $x_1, x_2, \dots, x_m$ 表示选择的特征变量。用 $x_{ij}$ 表示选择第 $i$ 个特征变量的取值,将其转化为标准化特征变量 $\tilde{x}_{ij}$。

\begin{equation}
\tilde{x}_{ij} = \frac{x_{ij} - \bar{x}_j}{s_j}, \, (i = 1, 2, \cdots, n; j = 1, 2, \cdots, m)
\tag{4-8}
\end{equation}

利用$\bar{x}_{j}$表示第 $i$ 个特征变量的平均值

\begin{equation}
\bar{x}_{j}=\frac{1}{n} \sum_{i=1}^{n} x_{i j}
\tag{4-9}
\end{equation}

求其标准差为:

\begin{equation}
s_{j}=\sqrt{\frac{1}{n-1} \sum_{i=1}^{n}\left(x_{i j}-\bar{x}_{j}\right)^{2}},(j=1,2, \ldots, m)
\tag{4-10}
\end{equation}

计算相关系数矩阵与其特征值、特征变量

相关系数矩阵 $R_{e}=\left(r_{i j}\right)_{m \times m}$,计算 $R_{e}=\left(r_{i j}\right)_{m \times m}$ 的特征值及特征向量,并且将求得的特征值按大小进行排序。

\begin{equation}
r_{i j}=\frac{\sum_{k=1}^{n} \tilde{x}_{k i} \cdot \tilde{x}_{k j}}{n-1},(i, j=1,2, \cdots, m)
\tag{4-11}
\end{equation}

\begin{equation}
\begin{cases}
y_{1}=u_{11} \tilde{x}_{1}+u_{21} \tilde{x}_{2}+\cdots+u_{n 1} \tilde{x}_{n} \\
y_{2}=u_{12} \tilde{x}_{1}+u_{22} \tilde{x}_{2}+\cdots+u_{n 2} \tilde{x}_{n} \\
\cdots \cdots \cdots \cdots \cdots \cdots \cdots \cdots \cdots \cdots \cdots \cdots \cdots \cdots \cdots \cdots \cdots \cdots \cdots \cdots \cdots \cdots \cdots \cdots \cdots \cdots \cdots \cdots \cdots \cdots \cdots \cdots \cdots \cdots \cdots \cdots \cdots \cdots \cdots \cdots \cdots \cdots \cdots \cdots \cdots \cdots \cdots \cdots \cdots \cdots \cdots \cdots \cdots \cdots \cdots \cdots \cdots \cdots \cdots \cdots \cdots \cdots \cdots \cdots \cdots \cdots \cdots \cdots \cdots \cdots \cdots \cdots \cdots \cdots \cdots \cdots \cdots \cdots \cdots \cdots \cdots \cdots \cdots \cdots \cdots \cdots \cdots \cdots \cdots \cdots \cdots \cdots \cdots \cdots \cdots \cdots \cdots \cdots \cdots \cdots \cdots \cdots \cdots \cdots \cdots \cdots \cdots \cdots \cdots \cdots \cdots \cdots \cdots \cdots \cdots \cdots \cdots \cdots \cdots \cdots \cdots \cdots \cdots \cdots \cdots \cdots \cdots \cdots \cdots \cdots \cdots \cdots \cdots \cdots \cdots \cdots \cdots \cdots \cdots \cdots \cdots \cdots \cdots \cdots \cdots \cdots \cdots \cdots \cdots \cdots \cdots \cdots \cdots \cdots \cdots \cdots \cdots \cdots \cdots \cdots \cdots \cdots \cdots \cdots \cdots \cdots \cdots \cdots \cdots \cdots \cdots \cdots \cdots \cdots \cdots \cdots \cdots \cdots \cdots \cdots \cdots \cdots \cdots \cdots \cdots \cdots \cdots \cdots \cdots \cdots \cdots \cdots \cdots \cdots \cdots \cdots \cdots \cdots \cdots \cdots \cdots \cdots \cdots \cdots \cdots \cdots \cdots \cdots \cdots \cdots \cdots \cdots \cdots \cdots \cdots \cdots \cdots \cdots \cdots \cdots \cdots \cdots \cdots \cdots \cdots \cdots \cdots \cdots \cdots \cdots \cdots \cdots \cdots \cdots \cdots \cdots \cdots \cdots \cdots \cdots \cdots \cdots \cdots \cdots \cdots \cdots \cdots \cdots \cdots \cdots \cdots \cdots \cdots \cdots \cdots \cdots \cdots \cdots \cdots \cdots \cdots \cdots \cdots \cdots \cdots \cdots \cdots \cdots \cdots \cdots \cdots \cdots \cdots \cdots \cdots \cdots \cdots \cdots \cdots \cdots \cdots \cdots \cdots \cdots \cdots \cdots \cdots \cdots \cdots \cdots \cdots \cdots \cdots \cdots \cdots \cdots \cdots \cdots \cdots \cdots \cdots \cdots \cdots \cdots \cdots \cdots \cdots \cdots \cdots \cdots \cdots \cdots \cdots \cdots \cdots \cdots \cdots \cdots \cdots \cdots \cdots \cdots \cdots \cdots \cdots \cdots \cdots \cdots \cdots \cdots \cdots \cdots \cdots \cdots \cdots \cdots \cdots \cdots \cdots \cdots \cdots \cdots \cdots \cdots \cdots \cdots \cdots \cdots \cdots \cdots \cdots \cdots \cdots \cdots \cdots \cdots \cdots \cdots \cdots \cdots \cdots \cdots \cdots \cdots \cdots \cdots \cdots \cdots \cdots \cdots \cdots \cdots \cdots \cdots \cdots \cdots \cdots \cdots \cdots \cdots \cdots \cdots \cdots \cdots \cdots \cdots \cdots \cdots \cdots \cdots \cdots \cdots \cdots \cdots \cdots \cdots \cdots \cdots \cdots \cdots \cdots \cdots \cdots \cdots \cdots \cdots \cdots \cdots \cdots \cdots \cdots \cdots \cdots \cdots \cdots \cdots \cdots \cdots \cdots \cdots \cdots \cdots \cdots \cdots \cdots \cdots \cdots \cdots \cdots \cdots \cdots \cdots \cdots \cdots \cdots \cdots \cdots \cdots \cdots \cdots \cdots \cdots \cdots \cdots \cdots \cdots \cdots \cdots \cdots \cdots \cdots \cdots \cdots \cdots \cdots \cdots \cdots \cdots \cdots \cdots \cdots \cdots \cdots \cdots \cdots \cdots \cdots \cdots \cdots \cdots \cdots \cdots \cdots \cdots \cdots \cdots \cdots \cdots \cdots \cdots \cdots \cdots \cdots \cdots \cdots \cdots \cdots \cdots \cdots \cdots \cdots \cdots \cdots \cdots \cdots \cdots \cdots \cdots \cdots \cdots \cdots \cdots \cdots \cdots \cdots \cdots \cdots \cdots \cdots \cdots \cdots \cdots \cdots \cdots \cdots \cdots \cdots \cdots \cdots \cdots \cdots \cdots \cdots \cdots \cdots \cdots \cdots \cdots \cdots \cdots \cdots \cdots \cdots \cdots \cdots \cdots \cdots \cdots \cdots \cdots \cdots \cdots \cdots \cdots \cdots \cdots \cdots \cdots \cdots \cdots \cdots \cdots \cdots \cdots \cdots \cdots \cdots \cdots \cdots \cdots \cdots \cdots \cdots \cdots \cdots \cdots \cdots \cdots \cdots \cdots \cdots \cdots \cdots \cdots \cdots \cdots \cdots \cdots \cdots \cdots \cdots \cdots \cdots \cdots \cdots \cdots \cdots \cdots \cdots \cdots \cdots \cdots \cdots \cdots \cdots \cdots \cdots \cdots \cdots \cdots \cdots \cdots \cdots \cdots \cdots \cdots \cdots \cdots \cdots \cdots \cdots \cdots \cdots \cdots \cdots \cdots \cdots \cdots \cdots \cdots \cdots \cdots \cdots \cdots \cdots \cdots \cdots \cdots \cdots \cdots \cdots \cdots \cdots \cdots \cdots \cdots \cdots \cdots \cdots \cdots \cdots \cdots \cdots \cdots \cdots \cdots \cdots \cdots \cdots \cdots \cdots \cdots \cdots \cdots \cdots \cdots \cdots \cdots \cdots \cdots \cdots \cdots \cdots \cdots \cdots \cdots \cdots \cdots \cdots \cdots \cdots \cdots \cdots \cdots \cdots \cdots \cdots \cdots \cdots \cdots \cdots \cdots \cdots \cdots \cdots \cdots \cdots \cdots \cdots \cdots \cdots \cdots \cdots \cdots \cdots \cdots \cdots \cdots \cdots \cdots \cdots \cdots \cdots \cdots \cdots \cdots \cdots \cdots \cdots \cdots \cdots \cdots \cdots \cdots \cdots \cdots \cdots \cdots \cdots \cdots \cdots \cdots \cdots \cdots \cdots \cdots \cdots \cdots \cdots \cdots \cdots \cdots \cdots \cdots \cdots \cdots \cdots \cdots \cdots \cdots \cdots \cdots \cdots \cdots \cdots \cdots \cdots \cdots \cdots \cdots \cdots \cdots \cdots \cdots \cdots \cdots \cdots \cdots \cdots \cdots \cdots \cdots \cdots \cdots \cdots \cdots \cdots \cdots \cdots \cdots \cdots \cdots \cdots \cdots \cdots \cdots \cdots \cdots \cdots \cdots \cdots \cdots \cdots \cdots \cdots \cdots \cdots \cdots \cdots \cdots \cdots \cdots \cdots \cdots \cdots \cdots \cdots \cdots \cdots \cdots \cdots \cdots \cdots \cdots \cdots \cdots \cdots \cdots \cdots \cdots \cdots \cdots \cdots \cdots \cdots \cdots \cdots \cdots \cdots \cdots \cdots \cdots \cdots \cdots \cdots \cdots \cdots \cdots \cdots \cdots \cdots \cdots \cdots \cdots \cdots \cdots \cdots \cdots \cdots \cdots \cdots \cdots \cdots \cdots \cdots \cdots \cdots \cdots \cdots \cdots \cdots \cdots \cdots \cdots \cdots \cdots \cdots \cdots \cdots \cdots \cdots \cdots \cdots \cdots \cdots \cdots \cdots \cdots \cdots \cdots \cdots \cdots \cdots \cdots \cdots \cdots \cdots \cdots \cdots \cdots \cdots \cdots \cdots \cdots \cdots \cdots \cdots \cdots \cdots \cdots \cdots \cdots \cdots \cdots \cdots \cdots \cdots \cdots \cdots \cdots \cdots \cdots \cdots \cdots \cdots \cdots \cdots \cdots \cdots \cdots \cdots \cdots \cdots \cdots \cdots \cdots \cdots \cdots \cdots \cdots \cdots \cdots \cdots \cdots \cdots \cdots \cdots \cdots \cdots \cdots \cdots \cdots \cdots \cdots \cdots \cdots \cdots \cdots \cdots \cdots \cdots \cdots \cdots \cdots \cdots \cdots \cdots \cdots \cdots \cdots \cdots \cdots \cdots \cdots \cdots \cdots \cdots \cdots \cdots \cdots \cdots \cdots \cdots \cdots \cdots \cdots \cdots \cdots \cdots \cdots \cdots \cdots \cdots \cdots \cdots \cdots \cdots \cdots \cdots \cdots \cdots \cdots \cdots \cdots \cdots \cdots \cdots \cdots \cdots \cdots \cdots \cdots \cdots \cdots \cdots \cdots \cdots \cdots \cdots \cdots \cdots \cdots \cdots \cdots \cdots \cdots \cdots \cdots \cdots \cdots \cdots \cdots \cdots \cdots \cdots \cdots \cdots \cdots \cdots \cdots \cdots \cdots \cdots \cdots \cdots \cdots \cdots \cdots \cdots \cdots \cdots \cdots \cdots \cdots \cdots \cdots \cdots \cdots \cdots \cdots \cdots \cdots \cdots \cdots \cdots \cdots \cdots \cdots \cdots \cdots \cdots \cdots \cdots \cdots \cdots \cdots \cdots \cdots \cdots \cdots \cdots \cdots \cdots \cdots \cdots \cdots \cdots \cdots \cdots \cdots \cdots \cdots \cdots \cdots \cdots \cdots \cdots \cdots \cdots \cdots \cdots \cdots \cdots \cdots \cdots \cdots \cdots \cdots \cdots \cdots \cdots \cdots \cdots \cdots \cdots \cdots \cdots \cdots \cdots \cdots \cdots \cdots \cdots \cdots \cdots \cdots \cdots \cdots \cdots \cdots \cdots \cdots \cdots \cdots \cdots \cdots \cdots \cdots \cdots \cdots \cdots \cdots \cdots \cdots \cdots \cdots \cdots \cdots \cdots \cdots \cdots \cdots \cdots \cdots \cdots \cdots \cdots \cdots \cdots \cdots \cdots \cdots \cdots \cdots \cdots \cdots \cdots \cdots \cdots \cdots \cdots \cdots \cdots \cdots \cdots \cdots \cdots \cdots \cdots \cdots \cdots \cdots \cdots \cdots \cdots \cdots \cdots \cdots \cdots \cdots \cdots \cdots \cdots \cdots \cdots \cdots \cdots \cdots \cdots \cdots \cdots \cdots \cdots \cdots \cdots \cdots \cdots \cdots \cdots \cdots \cdots \cdots \cdots \cdots \cdots \cdots \cdots \cdots \cdots \cdots \cdots \cdots \cdots \cdots \cdots \cdots \cdots \cdots \cdots \cdots \cdots \cdots \cdots \cdots \cdots \cdots \cdots \cdots \cdots \cdots \cdots \cdots \cdots \cdots \cdots \cdots \cdots \cdots \cdots \cdots \cdots \cdots \cdots \cdots \cdots \cdots \cdots \cdots \cdots \cdots \cdots \cdots \cdots \cdots \cdots \cdots \cdots \cdots \cdots \cdots \cdots \cdots \cdots \cdots \cdots \cdots \cdots \cdots \cdots \cdots \cdots \cdots \cdots \cdots \cdots \cdots \cdots \cdots \cdots \cdots \cdots \cdots \cdots \cdots \cdots \cdots \cdots \cdots \cdots \cdots \cdots \cdots \cdots \cdots \cdots \cdots \cdots \cdots \cdots \cdots \cdots \cdots \cdots \cdots \cdots \cdots \cdots \cdots \cdots \cdots \cdots \cdots \cdots \cdots \cdots \cdots \cdots \cdots \cdots \cdots \cdots \cdots \cdots \cdots \cdots \cdots \cdots \cdots \cdots \cdots \cdots \cdots \cdots \cdots \cdots \cdots \cdots \cdots \cdots \cdots \cdots \cdots \cdots \cdots \cdots \cdots \cdots \cdots \cdots \cdots \cdots \cdots \cdots \cdots \cdots \cdots \cdots \cdots \cdots \cdots \cdots \cdots \cdots \cdots \cdots \cdots \cdots \cdots \cdots \cdots \cdots \cdots \cdots \cdots \cdots \cdots \cdots \cdots \cdots \cdots \cdots \cdots \cdots \cdots \cdots \cdots \cdots \cdots \cdots \cdots \cdots \cdots \cdots \cdots \cdots \cdots \cdots \cdots \cdots \cdots \cdots \cdots \cdots \cdots \cdots \cdots \cdots \cdots \cdots \cdots \cdots \cdots \cdots \cdots \cdots \cdots \cdots \cdots \cdots \cdots \cdots \cdots \cdots \cdots \cdots \cdots \cdots \cdots \cdots \cdots \cdots \cdots \cdots \cdots \cdots \cdots \cdots \cdots \cdots \cdots \cdots \cdots \cdots \cdots \cdots \cdots \cdots \cdots \cdots \cdots \cdots \cdots \cdots \cdots \cdots \cdots \cdots \cdots \cdots \cdots \cdots \cdots \cdots \cdots \cdots \cdots \cdots \cdots \cdots \cdots \cdots \cdots \cdots \cdots \cdots \cdots \cdots \cdots \cdots \cdots \cdots \cdots \cdots \cdots \cdots \cdots \cdots \cdots \cdots \cdots \cdots \cdots \cdots \cdots \cdots \cdots \cdots \cdots \cdots \cdots \cdots \cdots \cdots \cdots \cdots \cdots \cdots \cdots \cdots \cdots \cdots \cdots \cdots \cdots \cdots \cdots \cdots \cdots \cdots \cdots \cdots \cdots \cdots \cdots \cdots \cdots \cdots \cdots \cdots \cdots \cdots \cdots \cdots \cdots \cdots \cdots \cdots \cdots \cdots \cdots \cdots \cdots \cdots \cdots \cdots \cdots \cdots \cdots \cdots \cdots \cdots \cdots \cdots \cdots \cdots \cdots \cdots \cdots \cdots \cdots \cdots \cdots \cdots \cdots \cdots \cdots \cdots \cdots \cdots \cdots \cdots \cdots \cdots \cdots \cdots \cdots \cdots \cdots \cdots \cdots \cdots \cdots \cdots \cdots \cdots \cdots \cdots \cdots \cdots \cdots \cdots \cdots \cdots \cdots \cdots \cdots \cdots \cdots \cdots \cdots \cdots \cdots \cdots \cdots \cdots \cdots \cdots \cdots \cdots \cdots \cdots \cdots \cdots \cdots \cdots \cdots \cdots \cdots \cdots \cdots \cdots \cdots \cdots \cdots \cdots \cdots \cdots \cdots \cdots \cdots \cdots \cdots \cdots \cdots \cdots \cdots \cdots \cdots \cdots \cdots \cdots \cdots \cdots \cdots \cdots \cdots \cdots \cdots \cdots \cdots \cdots \cdots \cdots \cdots \cdots \cdots \cdots \cdots \cdots \cdots \cdots \cdots \cdots \cdots \cdots \cdots \cdots \cdots \cdots \cdots \cdots \cdots \cdots \cdots \cdots \cdots \cdots \cdots \cdots \cdots \cdots \cdots \cdots \cdots \cdots \cdots \cdots \cdots \cdots \cdots \cdots \cdots \cdots \cdots \cdots \cdots \cdots \cdots \cdots \cdots \cdots \cdots \cdots \cdots \cdots \cdots \cdots \cdots \cdots \cdots \cdots \cdots \cdots \cdots \cdots \cdots \cdots \cdots \cdots \cdots \cdots \cdots \cdots \cdots \cdots \cdots \cdots \cdots \cdots \cdots \cdots \cdots \cdots \cdots \cdots \cdots \cdots \cdots \cdots \cdots \cdots \cdots \cdots \cdots \cdots \cdots \cdots \cdots \cdots \cdots \cdots \cdots \cdots \cdots \cdots \cdots \cdots \cdots \cdots \cdots \cdots \cdots \cdots \cdots \cdots \cdots \cdots \cdots \cdots \cdots \cdots \cdots \cdots \cdots \cdots \cdots \cdots \cdots \cdots \cdots \cdots \cdots \cdots \cdots \cdots \cdots \cdots \cdots \cdots \cdots \cdots \cdots \cdots \cdots \cdots \cdots \cdots \cdots \cdots \cdots \cdots \cdots \cdots \cdots \cdots \cdots \cdots \cdots \cdots \cdots \cdots \cdots \cdots \cdots \cdots \cdots \cdots \cdots \cdots \cdots \cdots \cdots \cdots \cdots \cdots \cdots \cdots \cdots \cdots \cdots \cdots \cdots \cdots \cdots \cdots \cdots \cdots \cdots \cdots \cdots \cdots \cdots \cdots \cdots \cdots \cdots \cdots \cdots \cdots \cdots \cdots \cdots \cdots \cdots \cdots \cdots \cdots \cdots \cdots \cdots \cdots \cdots \cdots \cdots \cdots \cdots \cdots \cdots \cdots \cdots \cdots \cdots \cdots \cdots \cdots \cdots \cdots \cdots \cdots \cdots \cdots \cdots \cdots \cdots \cdots \cdots \cdots \cdots \cdots \cdots \cdots \cdots \cdots \cdots \cdots \cdots \cdots \cdots \cdots \cdots \cdots \cdots \cdots \cdots \cdots \cdots \cdots \cdots \cdots \cdots \cdots \cdots \cdots \cdots \cdots \cdots \cdots \cdots \cdots \cdots \cdots \cdots \cdots \cdots \cdots \cdots \cdots \cdots \cdots \cdots \cdots \cdots \cdots \cdots \cdots \cdots \cdots \cdots \cdots \cdots \cdots \cdots \cdots \cdots \cdots \cdots \cdots \cdots \cdots \cdots \cdots \cdots \cdots \cdots \cdots \cdots \cdots \cdots \cdots \cdots \cdots \cdots \cdots \cdots \cdots \cdots \cdots \cdots \cdots \cdots \cdots \cdots \cdots \cdots \cdots \cdots \cdots \cdots \cdots \cdots \cdots \cdots \cdots \cdots \cdots \cdots \cdots \cdots \cdots \cdots \cdots \cdots \cdots \cdots \cdots \cdots \cdots \cdots \cdots \cdots \cdots \cdots \cdots \cdots \cdots \cdots \cdots \cdots \cdots \cdots \cdots \cdots \cdots \cdots \cdots \cdots \cdots \cdots \cdots \cdots \cdots \cdots \cdots \cdots \cdots \cdots \cdots \cdots \cdots \cdots \cdots \cdots \cdots \cdots \cdots \cdots \cdots \cdots \cdots \cdots \cdots \cdots \cdots \cdots \cdots \cdots \cdots \cdots \cdots \cdots \cdots \cdots \cdots \cdots \cdots \cdots \cdots \cdots \cdots \cdots \cdots \cdots \cdots \cdots \cdots \cdots \cdots \cdots \cdots \cdots \cdots \cdots \cdots \cdots \cdots \cdots \cdots \cdots \cdots \cdots \cdots \cdots \cdots \cdots \cdots \cdots \cdots \cdots \cdots \cdots \cdots \cdots \cdots \cdots \cdots \cdots \cdots \cdots \cdots \cdots \cdots \cdots \cdots \cdots \cdots \cdots \cdots \cdots \cdots \cdots \cdots \cdots \cdots \cdots \cdots \cdots \cdots \cdots \cdots \cdots \cdots \cdots \cdots \cdots \cdots \cdots \cdots \cdots \cdots \cdots \cdots \cdots \cdots \cdots \cdots \cdots \cdots \cdots \cdots \cdots \cdots \cdots \cdots \cdots \cdots \cdots \cdots \cdots \cdots \cdots \cdots \cdots \cdots \cdots \cdots \cdots \cdots \cdots \cdots \cdots \cdots \cdots \cdots \cdots \cdots \cdots \cdots \cdots \cdots \cdots \cdots \cdots \cdots \cdots \cdots \cdots \cdots \cdots \cdots \cdots \cdots \cdots \cdots \cdots \cdots \cdots \cdots \cdots \cdots \cdots \cdots \cdots \cdots \cdots \cdots \cdots \cdots \cdots \cdots \cdots \cdots \cdots \cdots \cdots \cdots \cdots \cdots \cdots \cdots \cdots \cdots \cdots \cdots \cdots \cdots \cdots \cdots \cdots \cdots \cdots \cdots \cdots \cdots \cdots \cdots \cdots \cdots \cdots \cdots \cdots \cdots \cdots \cdots \cdots \cdots \cdots \cdots \cdots \cdots \cdots \cdots \cdots \cdots \cdots \cdots \cdots \cdots \cdots \cdots \cdots \cdots \cdots \cdots \cdots \cdots \cdots \cdots \cdots \cdots \cdots \cdots \cdots \cdots \cdots \cdots \cdots \cdots \cdots \cdots \cdots \cdots \cdots \cdots \cdots \cdots \cdots \cdots \cdots \cdots \cdots \cdots \cdots \cdots \cdots \cdots \cdots \cdots \cdots \cdots \cdots \cdots \cdots \cdots \cdots \cdots \cdots \cdots \cdots \cdots \cdots \cdots \cdots \cdots \cdots \cdots \cdots \cdots \cdots \cdots \cdots \cdots \cdots \cdots \cdots \cdots \cdots \cdots \cdots \cdots \cdots \cdots \cdots \cdots \cdots \cdots \cdots \cdots \cdots \cdots \cdots \cdots \cdots \cdots \cdots \cdots \cdots \cdots \cdots \cdots \cdots \cdots \cdots \cdots \cdots \cdots \cdots \cdots \cdots \cdots \cdots \cdots \cdots \cdots \cdots \cdots \cdots \cdots \cdots \cdots \cdots \cdots \cdots \cdots \cdots \cdots \cdots \cdots \cdots \cdots \cdots \cdots \cdots \cdots \cdots \cdots \cdots \cdots \cdots \cdots \cdots \cdots \cdots \cdots \cdots \cdots \cdots \cdots \cdots \cdots \cdots \cdots \cdots \cdots \cdots \cdots \cdots \cdots \cdots \cdots \cdots \cdots \cdots \cdots \cdots \cdots \cdots \cdots \cdots \cdots \cdots \cdots \cdots \cdots \cdots \cdots \cdots \cdots \cdots \cdots \cdots \cdots \cdots \cdots \cdots \cdots \cdots \cdots \cdots \cdots \cdots \cdots \cdots \cdots \cdots \cdots \cdots \cdots \cdots \cdots \cdots \cdots \cdots \cdots \cdots \cdots \cdots \cdots \cdots \cdots \cdots \cdots \cdots \cdots \cdots \cdots \cdots \cdots \cdots \cdots \cdots \cdots \cdots \cdots \cdots \cdots \cdots \cdots \cdots \cdots \cdots \cdots \cdots \cdots \cdots \cdots \cdots \cdots \cdots \cdots \cdots \cdots \cdots \cdots \cdots \cdots \cdots \cdots \cdots \cdots \cdots \cdots \cdots \cdots \cdots \cdots \cdots \cdots \cdots \cdots \cdots \cdots \cdots \cdots \cdots \cdots \cdots \cdots \cdots \cdots \cdots \cdots \cdots \cdots \cdots \cdots \cdots \cdots \cdots \cdots \cdots \cdots \cdots \cdots \cdots \cdots \cdots \cdots \cdots \cdots \cdots \cdots \cdots \cdots \cdots \cdots \cdots \cdots \cdots \cdots \cdots \cdots \cdots \cdots \cdots \cdots \cdots \cdots \cdots \cdots \cdots \cdots \cdots \cdots \cdots \cdots \cdots \cdots \cdots \cdots \cdots \cdots \cdots \cdots \cdots \cdots \cdots \cdots \cdots \cdots \cdots \cdots \cdots \cdots \cdots \cdots \cdots \cdots \cdots \cdots \cdots \cdots \cdots \cdots \cdots \cdots \cdots \cdots \cdots \cdots \cdots \cdots \cdots \cdots \cdots \cdots \cdots \cdots \cdots \cdots \cdots \cdots \cdots \cdots \cdots \cd

\begin{equation}
C_{i} = (c_{i1}, c_{i2}, c_{i3})
\tag{4-14}
\end{equation}

分配数据点到最近的聚类中心

对于每个数据点,计算它与每个聚类中心的距离,并将其分配给距离最近的聚类中心所属的簇。在 K-means 中,我们有 K 个聚类中心,每个聚类中心是一个三维向量,可以表示为其中 j 是聚类中心的索引。使用一个分配矩阵 A 来表示每个数据点分配给哪个簇。A 的大小是 \(N \times K\),其中 \(A_{ij}\) 表示数据点 \(X_{i}\) 是否属于簇 \(C_{j}\),通常为 0 或 1。

更新聚类中心

对于每个簇,计算该簇中所有数据点的平均值,将该平均值作为新的聚类中心。K-means 的目标是最小化数据点与其所属簇中心之间的距离的平方和,这通常被称为“簇内平方和”(inertia)。目标函数可以表示为

\begin{equation}
\text{Inertia} = \sum_{i=1}^{N} \sum_{j=1}^{K} A_{ij} \cdot \|X_{i} - C_{j}\|^{2}
\tag{4-15}
\end{equation}

迭代并输出结果

重复步骤 3 和步骤 4,直到满足停止条件,例如,簇中心不再发生显著变化或达到预定的迭代次数。一旦停止条件满足,算法将返回 K 个簇,每个簇包含一组数据点,表示数据的不同子群。

K-means 算法的目标是找到最优的分配簇 A 和聚类中心 C,以最小化目标函数。K-means 是一个迭代的算法,通过不断更新分配簇和聚类中心,寻找最优的簇划分。通常,K-means 的性能受到初始聚类中心的选择和迭代次数的影响,因此可以多次运行算法以获得更好的结果。通过算法将 160 个患者数据根据年龄和血压分为了 5 个聚类,如图 4-3 是通过 Matlab 进行 K-means 三维聚类后分出的三维图形,其中五种不同的颜色即代表了不同的聚类。图 4-4 和表 4-3 是对这 5 个聚类进行了统计分析,展示 5 聚类的基本特征。

\begin{figure}[h]
    \centering
    \includegraphics[width=\textwidth]{image.png}
    \caption{K-means 聚类最优簇划分}
    \label{fig:4-3}
\end{figure}

\begin{figure}[h]
    \centering
    \includegraphics[width=0.8\textwidth]{image.png}
    \caption{K-means 曲线簇划分占比饼图}
    \label{fig:pie_chart}
\end{figure}

其中五个聚类的特征指标分布关系如下:

\begin{tabular}{l l l}
\hline
系数 & 系数值 \\
\hline
\(a_0\) & 2.799e+04 & (2.003e+04, 3.595e+04) \\
\(a_1\) & 3218 & (-5618, 1.205e+04) \\
\(b_1\) & -1192 & (-1.411e+04, 1.173e+04) \\
\(a_2\) & -1.299e+04 & (-2.407e+04, -1920) \\
\(b_2\) & 1480 & (-1.134e+04, 1.43e+04) \\
\(a_3\) & -1729 & (-1.475e+04, 1.129e+04) \\
\(b_3\) & -9455 & (-2.029e+04, 1380) \\
\(w\) & 0.3951 & (0.3738, 0.4164) \\
\hline
\end{tabular}

\subsection{问题二(b)模型的求解}

对每个聚类似合完后的曲线进行特征分析。

\subsubsection{聚类1曲线拟合与评价}

\begin{figure}[h]
    \centering
    \includegraphics[width=0.8\textwidth]{image2.png}
    \caption{问题二(b) 聚类1模型拟合曲线}
    \label{fig:cluster1_fit}
\end{figure}

该模型使用了一个5次多项式来拟合数据,图\ref{fig:cluster1_fit}为其拟合曲线图。其公式为:
\begin{equation}
f(x) = p_1 \cdot x^5 + p_2 \cdot x^4 + p_3 \cdot x^3 + p_4 \cdot x^2 + p_5 \cdot x + p_6
\tag{4-16}
\end{equation}
其中,如表4-4中$p_1$至$p_6$是拟合后多项式模型的系数,置信区间值为95\%。

\begin{tabular}{l l}
\hline
拟合质量指标 & 拟合质量指标值 \\
\hline
SSE & 2.39e+09 \\
R-square & 0.48 \\
Adjusted R-square & 0.32 \\
RMSE & 1.307e+04 \\
\hline
\end{tabular}

该模型拟合质量共有4项指标,误差平方和 (SSE) 表示模型拟合数据的误差总和、$R^{2}$ (R-square) 表示模型对总方差的解释程度、调整后的$R^{2}$ (Adjusted R-square) 通常用于多项式拟合,它考虑了模型中的自由度。负值通常表示模型不适合拟合数据,均方根误差 (RMSE) 表示模型的预测误差的标准偏差,各项指标值如表4-5所示:

\begin{tabular}{c c}
\hline
$b_{5}$ & 199.8148 \\
$b_{6}$ & -290.2422 \\
$b_{7}$ & 147.5962 \\
\hline
\end{tabular}

\subsubsection{聚类2曲线拟合与评价}

\begin{figure}[h]
\centering
\includegraphics[width=\textwidth]{image.png}
\caption{问题二(b)聚类2模型拟合曲线}
\end{figure}

该模型使用了一个5次多项式来拟合数据,图4-6为其拟合曲线图,其公式为:

\begin{equation}
f(x) = p_1 \cdot x^5 + p_2 \cdot x^4 + p_3 \cdot x^3 + p_4 \cdot x^2 + p_5 \cdot x + p_6
\tag{4-17}
\end{equation}

其中,如表4-6中p1至p6是模型的系数,置信区间值为95%。

\begin{tabular}{l l l}
Pons_Medulla_L 区域血肿 & 体积变化 & \\
体积变化 & & \\
单位时间内 ACA_R 水肿体积变化 & 单位时间内 MCA_R 水肿体积变化 & 单位时间内 PCA_R 水肿体积变化 \\
单位时间内 Pons_Medulla_R 水肿体积变化 & 单位时间内 Cerebellum_R 水肿体积变化 & 单位时间内 ACA_L 水肿体积变化 \\
单位时间内 MCA_L 水肿体积变化 & 单位时间内 PCA_L 水肿体积变化 & 单位时间内 Pons_Medulla_L 水肿体积变化 \\
单位时间内 Cerebellum_L 水肿体积变化 & 年龄 & 发病到首次影像检查时间间隔 \\
收缩压 & 舒张压 & original_shape_Elongation \\
original_shape_Flatness & original_shape_LeastAxisLength & original_shape_MajorAxisLength \\
original_shape_Maximum2DDiameterColumn & original_shape_Maximum2DDiameterRow & original_shape_Maximum2DDiameterSlice \\
original_shape_Maximum3DDiameter & original_shape_MeshVolume & original_shape_MinorAxisLength \\
original_shape_Sphericity & original_shape_SurfaceArea & original_shape_SurfaceVolumeRatio \\
original_shape_VoxelVolume & NCCT_original_firstorder_10Percentile & NCCT_original_firstorder_90Percentile \\
\hline
NCCT_original_firstorder_Energy & NCCT_original_firstorder_Entropy & NCCT_original_firstorder_InterquartileRange \\
NCCT_original_firstorder_Kurtosis & NCCT_original_firstorder_Maximum & NCCT_original_firstorder_MeanAbsoluteDeviation \\
NCCT_original_firstorder_Mean & NCCT_original_firstorder_Median & NCCT_original_firstorder_Minimum \\
NCCT_original_firstorder_Range & NCCT_original_firstorder_RobustMeanAbsoluteDeviation & NCCT_original_firstorder_RootMeanSquared \\
NCCT_original_firstorder_Skewness & NCCT_original_firstorder_Uniformity & NCCT_original_firstorder_Variance \\
\hline
\end{tabular}

该模型拟合质量指标值如表4-7所示:

\begin{tabular}{l l}
\hline
\multicolumn{2}{c}{第三梯度(灰色关联度:0.60-0.65)影响因素} \\
\hline
糖尿病史 & 冠心病史 \\
脑室引流 & 单位时间内 HM\_Pons\_Medulla\_R\_Ratio 水肿体积变化 \\
单位时间内 HM\_PCA\_L\_Ratio 血肿体积变化 & 单位时间内 ED\_ACA\_R\_Ratio 水肿体积变化 \\
单位时间内 ED\_ACA\_L\_Ratio 水肿体积变化 & 发病到首次影像检查时间间隔 \\
original\_shape\_SurfaceVolumeRatio & \\
\hline
\end{tabular}

\subsubsection{聚类3曲线拟合与评价}

\begin{figure}[h]
\centering
\includegraphics[width=\textwidth]{image.png}
\caption{问题二(b)聚类3模型拟合曲线}
\end{figure}

该模型使用了一个5次多项式来拟合数据,图4-7为其拟合曲线图,其公式为:

\begin{equation}
f(x) = p1 \cdot x^5 + p2 \cdot x^4 + p3 \cdot x^3 + p4 \cdot x^2 + p5 \cdot x + p6
\tag{4-18}
\end{equation}

其中,如表4-8中p1至p6是模型的系数,置信区间值为95%。

\begin{tabular}{l l}
\hline
\multicolumn{2}{c}{第四梯度(灰色关联度:0.55-0.60)影响因素} \\
\hline
高血压病史 & 饮酒史 \\
止血治疗 & 降压治疗 \\
止吐护胃 & 营养神经 \\
单位时间内 HM\_Cerebellum\_R\_Ratio 血肿体积变化 & 单位时间内 HM\_Cerebellum\_L\_Ratio 血肿体积变化 \\
单位时间内 ED\_Cerebellum\_R\_Ratio 水肿体积变化 & 单位时间内 ED\_Cerebellum\_L\_Ratio 水肿体积变化 \\
\hline
\end{tabular}

该模型拟合质量指标值如表4-9所示:

\begin{table}[h]
\centering
\caption{表4-9 问题二(b)聚类3模型拟合质量评价}
\begin{tabular}{c c}
\hline
拟合质量指标 & 拟合质量指标值 \\
\hline
SSE & 5.408e+08 \\
R-square & 0.8286 \\
Adjusted R-square & 0.7507 \\
RMSE & 7012 \\
\hline
\end{tabular}
\end{table}

\subsubsection{聚类4曲线拟合与评价}

\begin{figure}[h]
\centering
\includegraphics[width=\textwidth]{image.png}
\caption{图4-8 问题二(b)聚类4模型拟合曲线}
\end{figure}

该模型使用了一个包含三个谐波项的傅里叶级数来拟合数据,图4-8为其拟合曲线图,其公式为:
\begin{equation}
f(x) = a_0 + a_1 \cdot \cos(x \cdot w) + b_1 \cdot \sin(x \cdot w) + a_2 \cdot \cos(2 \cdot x \cdot w) +
\end{equation}
\begin{equation}
b_2 \cdot \sin(2 \cdot x \cdot w) + a_3 \cdot \cos(3 \cdot x \cdot w) + b_3 \cdot \sin(3 \cdot x \cdot w)
\tag{4-19}
\end{equation}
其中,如表4-10中 \(a_0\) 至 \(b_3\) 和 \(w\) 是模型的系数,\(w\) 表示频率,置信区间值为95%。

\textbf{表 4-10 问题二(b)聚类4多项式模型系数}

\begin{tabular}{l l l}
\hline
系数 & 系数值 \\
\hline
\(a_0\) & 2.799e+04 & (2.003e+04, 3.595e+04) \\
\(a_1\) & 3218 & (-5618, 1.205e+04) \\
\(b_1\) & -1192 & (-1.411e+04, 1.173e+04) \\
\(a_2\) & -1.299e+04 & (-2.407e+04, -1920) \\
\(b_2\) & 1480 & (-1.134e+04, 1.43e+04) \\
\(a_3\) & -1729 & (-1.475e+04, 1.129e+04) \\
\(b_3\) & -9455 & (-2.029e+04, 1380) \\
\(w\) & 0.3951 & (0.3738, 0.4164) \\
\hline
\end{tabular}

该模型拟合质量指标值如表4-11所示:

\textbf{表 4-11 问题二(b)聚类4模型拟合质量评价}

\begin{tabular}{l l}
\hline
拟合质量指标 & 拟合质量指标值 \\
\hline
SSE & 2.39e+09 \\
R-square & 0.48 \\
Adjusted R-square & 0.32 \\
RMSE & 1.307e+04 \\
\hline
\end{tabular}

\subsubsection{聚类5曲线拟合与评价}

\begin{figure}[h]
    \centering
    \includegraphics[width=\textwidth]{image.png}
    \caption{图 4-9 问题二(b)聚类 5 模型拟合曲线}
\end{figure}

该模型使用了一个包含三个谐波项的傅里叶级数来拟合数据,图 4-9 为其拟合曲线图。其公式为:
\begin{equation}
f(x) = a0 + a1 \cdot \cos(x \cdot w) + b1 \cdot \sin(x \cdot w) + a2 \cdot \cos(2 \cdot x \cdot w) + b2 \cdot \sin(2 \cdot x \cdot w) + a3 \cdot \cos(3 \cdot x \cdot w) + b3 \cdot \sin(3 \cdot x \cdot w)
\tag{4-20}
\end{equation}
其中,如表 4-12 中 a0 至 b3 和 w 是模型的系数,w 表示频率,置信区间值为 95\%。

\begin{table}[h]
\centering
\caption{表 4-12 问题二(b)聚类 5 多项式模型系数}
\begin{tabular}{c c c}
\hline
系数 & 系数值 \\
\hline
$a_{0}$ & $1.096e+04$ & $(6920, 1.5e+04)$ \\
$a_{1}$ & $642.2$ & $(-4029, 5314)$ \\
$b_{1}$ & $-2246$ & $(-8417, 3926)$ \\
$a_{2}$ & $-4584$ & $(-1.064e+04, 1473)$ \\
$b_{2}$ & $-2235$ & $(-1.078e+04, 6308)$ \\
$a_{3}$ & $3405$ & $(-4713, 1.152e+04)$ \\
$b_{3}$ & $3373$ & $(-3683, 1.043e+04)$ \\
$w$ & $0.2073$ & $(0.176, 0.2387)$ \\
\hline
\end{tabular}
\end{table}

该模型拟合质量指标值如表 4-13 所示:

\begin{table}[h]
\centering
\caption{表 4-13 问题二(b)聚类 5 模型拟合质量评价}
\begin{tabular}{c c}
\hline
拟合质量指标 & 拟合质量指标值 \\
\hline
SSE & $3.751e+08$ \\
\hline
\end{tabular}
\end{table}

\begin{table}[h]
\centering
\begin{tabular}{l r}
R-square & 0.4493 \\
Adjusted R-square & 0.36381 \\
RMSE & 6125 \\
\hline
\end{tabular}
\end{table}

\subsection{问题二(c)分析与计算过程}

需要分析表 1 中不同治疗方法对水肿体积进展模式的影响,所以本文考虑搭建多元线性回归模型,利用多元回归模型分析模型的拟合系数,从而对不同治疗方法对水肿体积变化的影响做出判断。

\subsubsection{问题(c)数据预处理}

由于需要分析不同治疗方法对水肿体积进展模式。则需要分析不同患者在随时间变化的过程中,水肿体积的变化,由于每个患者的随访次数不同,水肿体积变化的数据量也不同。故通过计算相邻两次随访水肿体积变化与时间间隔的比值,反应水肿体积在这段时间的变化情况,如果有多次随访记录,则通过求每相邻两次随访水肿体积变化与时间间隔的比值的平均值,来获得患者在整个治疗期间水肿体积的变化情况。

\begin{equation}
\text{水肿体积变化 } \Psi = \frac{1}{n} \sum_{i=1}^{n} \frac{\text{ED}_{\text{volume}_i} - \text{ED}_{\text{volume}_{i-1}}}{T_i - T_{i-1}}
\tag{4-21}
\end{equation}

其中患者随访次数为 \( n = 0, 1, 2, \ldots \),并认为患者首次影像检查信息为第 0 次。

\subsubsection{问题(c)基于多元回归分析模型建立}

多元线性回归的目标是找到最佳的回归系数,以最好地拟合数据并解释因变量的变异性。这通常涉及到最小化残差平方和(SSE),即观测值与模型预测值之间的差异的平方和,来估计回归系数。

其一般性模型的表达式为

\begin{equation}
Y = b_0 + b_1 x_1 + b_2 x_2 + \cdots + b_p x_p + \varepsilon \sigma^2
\tag{4-22}
\end{equation}

其中:\( Y \) 是因变量,即要预测或解释的变量,\( x \) 为自变量,\( b \) 表示预测因子,是回归系数,表示自变量对因变量的影响程度,该系数的计算采用最小二乘估计法,\( \varepsilon \) 是服从正态分布 \( N(0, \sigma^2) \) 的随机变量,表示模型不能解释的随机误差。方便起见,引入矩阵记号:

\begin{equation}
Y = \begin{bmatrix} y_1 \\ y_2 \\ \vdots \\ y_n \end{bmatrix}, X = \begin{bmatrix} x_1(u_1) & x_2(u_1) & \cdots & x_m(u_m) \\ x_1(u_2) & x_2(u_2) & \cdots & x_m(u_m) \\ \vdots & \vdots & \ddots & \vdots \\ x_n(u_1) & x_n(u_2) & \cdots & x_n(u_m) \end{bmatrix}, \beta = \begin{bmatrix} \beta_1 \\ \beta_2 \\ \vdots \\ \beta_m \end{bmatrix}
\tag{4-23}
\end{equation}

选取 \( \beta \) 的一个估计值 \( \hat{\beta} \),使得随机变量 \( \varepsilon \) 的平方和达到最小,即

\begin{equation}
m \varepsilon^t \varepsilon = m (Y - X \beta)^T (Y - X \beta)
\tag{4-24}
\end{equation}

写成分量形式为:

\begin{equation}
Q(\beta_1, \beta_3, \ldots, \beta_m) = \sum_{i=1}^n \left[ y_i - \beta_1 x_1(u_i) - \beta_2 x_2(u_i) - \cdots - \beta_m x_m(u_i) \right]^2 \tag{4-25}
\end{equation}

\begin{equation}
Q(\hat{\beta}_1, \hat{\beta}_2, \ldots, \hat{\beta}_m) = \min Q(\beta_1, \beta_2, \ldots, \beta_m) \tag{4-26}
\end{equation}

由于 \( Q(\beta_1, \beta_3, \ldots, \beta_m) \) 是非负二次式,根据多元函数取极值的必要条件,可得

\begin{equation}
\frac{\partial Q}{\partial \beta_j} = 0 \quad (j = 1, 2, \ldots, m) \tag{4-27}
\end{equation}

\begin{equation}
\sum_{i=1}^n \left[ y_i - \hat{\beta}_1 x_1(u_i) - \hat{\beta}_2 x_2(u_i) - \cdots - \hat{\beta}_m x_m(u_i) \right] x_j(u_i) = 0 \quad (j = 1, 2, \ldots, m) \tag{4-28}
\end{equation}

\begin{equation}
\begin{cases}
\left[ \sum_{i=1}^n x_1^2(u_i) \right] \hat{\beta}_1 + \left[ \sum_{i=1}^n x_1(u_i) x_2(u_i) \right] \hat{\beta}_2 + \cdots + \left[ \sum_{i=1}^n x_1(u_i) x_m(u_i) \right] \hat{\beta}_m \\
= \sum_{i=1}^n x_1(u_i) y_i \\
\left[ \sum_{i=1}^n x_1(u_i) x_m(u_i) \right] \hat{\beta}_1 + \left[ \sum_{i=1}^n x_2(u_i) x_m(u_i) \right] \hat{\beta}_2 + \cdots + \left[ \sum_{i=1}^n x_m^2(u_i) \right] \hat{\beta}_m \\
= \sum_{i=1}^n x_m(u_i) y_i
\end{cases} \tag{4-29}
\end{equation}

将 \(\hat{\beta}_1\) 代入模型 (4-23) 中,得模型的估计:\(\hat{Y} = X^T \hat{\beta}\)。

综合利用模型拟合程度 \(R^2\) 和 F 检验可以更全面地评估线性回归模型的拟合程度和统计显著性,便于对结果进行分析验证。

线性回归模型的预测值与观测值的相关系数 \(R\) 表示模型对观测数据的解释程度。本文认为大于 0.4 认为拟合成功,其表达式如下:

\begin{equation}
R = \frac{\sum (y_i - \bar{Y})(Y_i - \bar{Y})}{\sqrt{\sum (y_i - \bar{Y})^2 (Y_i - \bar{Y})^2}} \tag{4-30}
\end{equation}

\subsubsection{问题(c)模型求解}

使用多元线性回归,以不同治疗方法作为多维度特征值,以单位时间内水肿总体积变化作为因变量,做出相应的拟合曲线如下:

\begin{equation}
Y = b_0 + b_1 \cdot x_1 + b_2 \cdot x_2 + b_3 \cdot x_3 + b_4 \cdot x_4 + b_5 \cdot x_5 + b_6 \cdot x_6 + b_7 \cdot x_7
\end{equation}

其中表 4-14 中 \(b_0\) 为常数系数,\(b_1\) 到 \(b_7\) 分别为治疗方法为特征数据的回归系数。

\begin{table}[h]
\centering
\caption{问题二(c)多元线性回归拟合曲线系数}
\begin{tabular}{c c}
\hline \hline
系数 & 系数值 \\
\hline
\(b_0\) & -139.2952 \\
\(b_1\) & 462.3629 \\
\(b_2\) & 102.7279 \\
\(b_3\) & 174.6037 \\
\(b_4\) & 91.1463 \\
\hline \hline
\end{tabular}
\end{table}

\begin{tabular}{c c}
\hline
$b_{5}$ & 199.8148 \\
$b_{6}$ & -290.2422 \\
$b_{7}$ & 147.5962 \\
\hline
\end{tabular}

通过拟合所得的曲线如图 4-10 所示,计算出的多项拟合经过 $R^{2}$ 检测得 $R^{2}=0.4316$,通过 $R^{2}$ 分析知拟合效果较好。

\begin{figure}[h]
    \centering
    \includegraphics[width=\textwidth]{image.png}
    \caption{多元线性回归拟合曲线图}
    \label{fig:4-10}
\end{figure}

\subsection{问题二(d)分析与计算过程}

\subsubsection{问题数据预处理}

本题要求分析血肿体积、水肿体积及治疗方案三者之间的联系,可以根据问题(c)同样的数据处理方法对血肿体积进行同样的处理。

\begin{equation}
\text{血肿体积变化} \eta = \frac{1}{n} \sum_{i=1}^{n} \frac{\text{HM}_{\text{volume}_i} - \text{HM}_{\text{volume}_{i-1}}}{T_i - T_{i-1}}
\tag{4-31}
\end{equation}

\subsubsection{问题二(d)基于斯皮尔曼相关性分析模型建立}

上述的问题对数据的分析可以知道,表中数据整体并不呈现正态分布。本文通过斯皮尔曼相关性分析,分析血肿体积、水肿体积及治疗方案三者的联系,并通过 SPSS 数据统计分析软件进行计算。通过上述的数据预处理,本文将问题转化为分析血肿体积变化 $\eta$、水肿体积变化 $\Psi$ 以及治疗方法三者之间的关系。

斯皮尔曼相关系数是一种用于衡量两个变量之间的非线性关联关系的统计量。与皮尔逊相关系数不同,斯皮尔曼相关系数不要求变量之间的关系是线性的,而且可以用于评估等级数据或有序数据之间的关联。且斯皮尔曼相关对于异常值并不敏感,实际数值之间的差异大小对于计算结果并没有太大的影响。

计算斯皮尔曼秩的相关系数计算公式:

\begin{equation}
\rho = \frac{\frac{1}{n} \sum_{i=1}^{n} (R(x_i) - \overline{R(x)}) \cdot (R(y_i) - \overline{R(y)})}{\sqrt{\left( \frac{1}{n} \sum_{i=1}^{n} (R(x_i) - \overline{R(x)})^2 \right) \cdot \left( \frac{1}{n} \sum_{i=1}^{n} (R(y_i) - \overline{R(y)})^2 \right)}}
\tag{4-32}
\end{equation}

\begin{table}
\centering
\caption{斯皮尔曼相关系数与显著性系数}
\begin{tabular}{l c c c c c c c c}
\hline
 & 脑室引流 & 止血治疗 & 降颅压治疗 & 降压治疗 & 镇静、镇痛治疗 & 止吐护理 & 营养神经 & 血肿体积变化 \\
\hline
相关系数 & 1.000 & -0.077 & -0.055 & 0.075 & -0.012 & 0.044 & 0.052 & -0.139 \\
脑室引流 & & & & & & & & 0.088 \\
显著性(双尾) & & & & & & & & \\
\hline
相关系数 & -0.077 & 1.000 & 0.458** & 0.300** & 0.058 & -0.091 & 0.020 & 0.088 \\
止血治疗 & & & & & & & & 0.197* \\
显著性(双尾) & 0.449 & 1.000 & 0.002 & 0.564 & 0.370 & 0.843 & 0.384 & 0.050 \\
\hline
相关系数 & -0.055 & 0.458** & 1.000 & 0.352** & -0.039 & 0.038 & 0.005 & 0.038 \\
降颅压治疗 & & & & & & & & 0.307** \\
显著性(双尾) & 0.585 & 1.000 & 1.000 & 0.698 & 0.704 & 0.962 & 0.706 & 0.002 \\
\hline
相关系数 & 0.075 & 0.300** & 0.352** & 1.000 & 0.083 & -0.052 & 0.128 & 0.042 \\
降压治疗 & & & & & & & & 0.236* \\
显著性(双尾) & 0.461 & 0.002 & 1.000 & 0.414 & 0.608 & 0.205 & 0.677 & 0.018 \\
\hline
相关系数 & -0.012 & 0.058 & -0.039 & 0.083 & 1.000 & 0.419** & 0.200* & 0.063 \\
镇静、 & & & & & & & & 0.117 \\
\hline
\end{tabular}
\end{table}

\begin{table}
\centering
\begin{tabular}{l l l l l l l l l}
 & 显著性(双尾) & 相关系数 & 显著性(双尾) & 相关系数 & 显著性(双尾) & 相关系数 & 显著性(双尾) & 相关系数 \\
镇痛治疗 & 0.907 & 0.564 & 0.698 & 0.414 & 1.000 & 0.046 & 0.536 & 0.247 \\
 & & & & & & & & \\
 & & -0.05 & 0.419* & 1.000 & 0.263 & 0.074 & -0.031 & \\
 & & 0.044 & -0.091 & 0.038 & 2 & * & & \\
止吐护理 & & & & & & & & \\
 & & & & & & & & \\
 & & .661 & .370 & .704 & .608 & .000 & .008 & .464 \\
 & & & & & & & & .756 \\
 & & & & & & & & \\
 & & 0.052 & 0.020 & 0.005 & 0.128 & 0.200* & 0.263** & 1.000 \\
 & & & & & & & 0.175 & 0.122 \\
营养神经 & & & & & & & & \\
 & & 0.610 & 0.843 & 0.962 & 0.205 & 0.046 & 0.008 & 0.082 \\
 & & & & & & & & 0.227 \\
 & & & & & & & & \\
 & & -0.139 & 0.088 & 0.038 & 0.042 & 0.063 & 0.074 & 0.175 \\
 & & & & & & & & .336** \\
血肿体积变化 & & & & & & & & \\
 & & 0.169 & 0.384 & 0.706 & 0.677 & 0.536 & 0.464 & 0.082 \\
 & & & & & & & & 0.001 \\
 & & & & & & & & \\
 & & 0.088 & 0.197* & 0.307** & 0.236 & 0.117 & -0.031 & 0.336 \\
 & & & & & * & & 0.122 & 1.000 \\
水肿体积变化 & & & & & & & & \\
 & & 0.387 & 0.050 & 0.002 & 0.018 & 0.247 & 0.756 & 0.227 \\
 & & & & & & & 0.001 & \\
 & & & & & & & & \\
\end{tabular}
\end{table}

\begin{figure}[h]
\centering
\includegraphics[width=\textwidth]{heatmap_image.png}
\caption{图 4-11 形状特征 Least Axis Length 分布}
\end{figure}

33

根据模型求解出的数据进行绘制出可视化热力图,通过图 4-11 可得:

(1) 对血肿体积的影响:水肿体积 > 营养神经 > 止吐护胃 > 镇静、镇痛治疗 > 降压治疗 > 降颅压治疗 > 止血治疗 > 脑室引流

(2) 对水肿体积的影响:血肿体积 > 降颅压治疗 > 降压治疗 > 止血治疗 > 营养神经 > 镇静、镇痛治疗 > 脑室引流 > 止吐护胃

根据以上分析可以直观分析出适合于患者的情况的治疗方法。

\section{问题三建模的分析与求解}

\subsection{问题三(a)建模的分析与求解}

题目中说明需要根据前 100 个患者的个人史、疾病史、发病相关及首次影像结果来构建预测模型来预测所有患者 90 天的 mRS 评分,该类问题类似与问题一(b)的预测问题。则考虑利用机器学习的方法,用所给数据集进行训练,故同问题一一样需要对数据集进行一系列的数据预处理操作,将原始数据转换为更适合机器学习模型的形式。通过之前整合的数据 $S_{\text{merged}}$、$X_{\text{merged}}$ 以及归一化处理后的 `Sheet_1`, `Sheet_3`。问题三的研究思路由图 \ref{fig:5-1} 所示。

\begin{figure}[h]
    \centering
    \includegraphics[width=0.8\textwidth]{image.png}
    \caption{问题三研究思路图}
    \label{fig:5-1}
\end{figure}

\subsubsection{数据处理与基于 BP 神经网络模型建立}

\paragraph{数据标准化与归一化}

通过利用问题一处理后的特征数据,其余缺失值和异常值的处理过程与问题一相同。

\paragraph{基于灰色关联度对变量处理}

灰色关联分析是一种用于研究变量之间关联程度的方法,尤其适用于在样本数据有限或者数据质量较差的情况下进行分析。本题需要预测所有患者 90 天的 mRS 评分,计算前 100 个患者的个人史、疾病史、发病相关及首次影像信息结果中的 53 个特征计算与 90 天的 mRS 评分的关联度。把所有的特征分为离散型和连续性,分别进行关联度的计算,根据前面的工作通过确定分析数量、对变量进行去量纲化,计算关联系数:

\begin{equation}
    \varepsilon_{i}(k) = \frac{\min_{i}\min_{k}|x_{0}(k)-x_{i}(k)| + \rho \cdot \max_{i}\max_{k}|x_{0}(k)-x_{i}(k)|}{|x_{0}(k)-x_{i}(k)| + \rho \cdot \max_{i}\max_{k}|x_{0}(k)-x_{i}(k)|}
    \tag{5-1}
\end{equation}

\begin{equation}
    \varepsilon_{i}(k) = \frac{\min_{i}\min_{k}\Delta_{i}(k) + \rho \max_{i}\max_{k}\Delta_{i}(k)}{\Delta_{i}(k) + \rho \max_{i}\max_{k}\Delta_{i}(k)}
    \tag{5-2}
\end{equation}

其中 $\rho \in (0, \infty)$ 为分辨系数,其值越小分辨相关性的能力越好,通常取 $\rho = 0.5$。

计算关联度,其公式为:$r_{0i} = \frac{1}{m} \sum_{k=1}^{m} \varepsilon_i(k)$。最后所得的关联度数值分布如如图 5-2 和图 5-3 所示,根据所有的特征相关度进行排序选择,在已知的 35 个连续性特征和 16 个离散型特征,分别选择 10 个连续性和 5 个离散性,筛选的结果如表 5-1 所示。

\begin{figure}[h]
    \centering
    \includegraphics[width=\textwidth]{image1.png}
    \caption{离散特征值与患者 90 天 mRS 评分之间关联度}
    \label{fig:5-2}
\end{figure}

\begin{figure}[h]
    \centering
    \includegraphics[width=\textwidth]{image2.png}
    \caption{连续特征值与患者 90 天 mRS 评分之间关联度}
    \label{fig:5-3}
\end{figure}

\begin{table}[h]
    \centering
    \caption{筛选所得 15 个特征变量}
    \label{tab:5-1}
    \begin{tabular}{c c c}
        \hline
        \textbf{特征变量} & \textbf{特征变量} & \textbf{特征变量} \\
        \hline
        脑出血前 mRS 评分 & 糖尿病史 & 房颤史 \\
        冠心病史 & 脑室引流 & MajorAxisLength \\
        Maximum2DDiameterColumn & Maximum2DDiameterRow & Maximum2DDiameterSlice \\
        Maximum3DDiameter & MinorAxisLength & MeanAbsoluteDeviation \\
        Range & Uniformity & Variance \\
        \hline
    \end{tabular}
\end{table}

\paragraph{BP 神经网络预测}

BP 神经网络,也称为反向传播神经网络(Backpropagation Neural Network),是一种

常见的人工神经网络类型,用于解决各种监督学习问题,特别是回归和分类问题。BP 神经网络通过学习从输入到输出之间的映射关系,它的名称源于它的训练算法中的反向传播过程。它的学习规则是使用最速下降法,通过反向传播来不断调整网络的权值和阈值,使网络的误差平方和最小。其最主要的特点就是:信号是正向传播的,而误差是反向传播的。

如图 5-4 是 BP 神经网络的结构图,BP 神经网络由多层神经元组成,通常包括输入层、隐藏层(可以有多个隐藏层)和输出层。每个神经元接受来自前一层神经元的输入,并产生输出,并且每个神经元接收来自其他神经元的是带有权重的输入信号,通过将它们相加,形成总输入值。然后,这个总输入值与神经元的阈值进行比较,类似于模拟阈值电位。接下来,通过一个激活函数对总输入值进行处理,模拟细胞的激活过程,最终得到输出。这个输出会传递给下一层神经元,依此类推,实现信息在神经网络中逐层传递。图 5-5 为其算法的具体实现过程。

\begin{figure}[h]
    \centering
    \includegraphics[width=0.8\textwidth]{bp_network_structure.png}
    \caption{5-4 BP 神经网络的结构图}
\end{figure}

该算法的运行步骤如下:

(a) 将训练数据传递给网络,计算预测结果。

其中输入的向量 \( X = (x_1, x_2, \ldots, x_i, \ldots, x_m) \),为通过灰色关联分析筛选的特征变量,\( Y = (y_1, y_2, \ldots, y_k, \ldots, y_n) \) 为输出向量。每个神经元正向传播过程中其计算过程为每一层传递的数值乘以对应的权重再加上相应的偏置变量,偏置变量就是选择的激活函数。激活函数的目的是在模型中引入非线性特征,因为无论神经网络有多少层,在没有激活函数的情况下,最终得到的都是一个线性映射,这样对结果的分析就不够准确。

从输入层到隐藏层:
\begin{equation}
\alpha_h = \sum_{i=1}^d v_{ih} x_i + \theta_h
\tag{5-3}
\end{equation}

从隐藏层的输出层:

\begin{equation}
\beta_{j}=\sum_{h=1}^{q} w_{h j} b_{h}+\theta_{j}
\tag{5-4}
\end{equation}

(b) 计算预测结果与实际目标值之间的误差。

由于计算参数是随机的,所以需要通过多次训练,并根据误差调整参数,从而达到最好的拟合效果。误差函数用来衡量输出结果与期望输出的差距,$d(i)$ 为对应 $x(i)$ 其具体函数为:

\begin{equation}
E=\frac{1}{p} \sum_{i=1}^{p} E(i)
\tag{5-5}
\end{equation}

其中 $E(i)$ 为单个特征变量样本的训练误差为:

\begin{equation}
E(i)=\frac{1}{2} \sum_{k=1}^{n}\left(d_{k}(i)-y_{k}(i)\right)^{2}
\tag{5-6}
\end{equation}

因此全局误差函数:

\begin{equation}
E=\frac{1}{2 p} \sum_{i=1}^{p} \sum_{k=1}^{n}\left(d_{k}(i)-y_{k}(i)\right)^{2}
\tag{5-7}
\end{equation}

(c) 对于每个权重按照梯度下降法使更新权重。

\begin{equation}
W_{i j}=W_{i j}-\alpha \frac{\partial E}{\partial W_{i j}}
\tag{5-8}
\end{equation}

其中 $\alpha$ 为学习率,通过学习率可以调整每次更新的步幅,合适的学习率可以让模型在合适的时间内收敛。如果学习率设置的太小,可能导致收敛的,计算效率十分低;如果学习率设置的太大,结果会在最优值附近一直反复更新徘徊,难以收敛,一般取值 0.01-0.8。

(d) 重复以上步骤直到达到停止条件(如达到最大迭代次数或误差足够小)。

\subsubsection{问题三(a)模型的求解}

在使用 BP 神经网络训练分析后,选择训练次数为 1000 次,学习率 $\alpha=0.01$,训练目标最小误差设置为 0.00001。通过多次训练后如图 5-5 为训练所得的曲线,在前 100 例患者数据中,选择了前 80 个进行训练,后 20 个进行测试。最后对所有的患者 90 天 mRS 评分进行预测。

\begin{figure}[h]
\centering
\includegraphics[width=\textwidth]{image.png}
\caption{BP 神经网络的预测值与实际值对比图}
\end{figure}

根据题中所给条件,发病后 90 天 mRS 评分是一个 0-6 的有序等级变量,用于评估卒中后患者功能状态的重要指标。由于预测的值是处于一个 [0,6] 的区间范围,所需要对结果

进行阈值分析,通过统计 160 名患者基于首次影像信息预测的 90 天 mRS 评分,并用 \( m \) 表示所得的预测值,填入答案表数据如下(5-9)。

\[
\text{Sheet}_{4_{\text{列}}} = 90 \text{ 天 mRS 评分} =
\begin{cases}
0 & , m < 0.5 \\
1, & 0.5 \leq m < 1.5 \\
2, & 1.5 \leq m < 2.5 \\
3, & 2.5 \leq m < 3.5 \\
4, & 3.5 \leq m < 4.5 \\
5, & 4.5 \leq m < 5.5 \\
6 & , m \geq 5.5
\end{cases}
\tag{5-9}
\]

\subsection{问题三(b)建模的分析与求解}

本题需要根据前 100 个患者的所有已知临床、治疗结果,预测所有含随访影像检查的患者。则需要整合所有的表 1、表 2 和表 3 的特征变量,包括患者个人史、治疗方法、疾病史、首次和随访的所有影像信息,统一进行统计分析。

\subsubsection{数据的处理与整合}

\paragraph{血肿/水肿位置数据处理}

需要对血肿/水肿位置数据进行一定的处理,才能使得血肿/水肿位置的关系能够应用于所有特征变量之间的统计分析。本文将血肿/水肿位置比例系数作为权重,与相应检查中的 HM\_volume 和 ED\_volume 相乘,从而得到脑部各个位置血肿/水肿的发生情况。进而可以分析血肿/水肿位置对患者 90 天 mRS 评分的影响。

脑部位置分别为 ACA\_R、MCA\_R、PCA\_R、Pons\_Medulla\_R、Cerebellum\_R、ACA\_L、MCA\_L、PCA\_L、Pons\_Medulla\_L、Cerebellum\_L 十个位置特征。以 ACA\_R 为例,各个脑部为的血肿和水肿体积可以表示为

\begin{align}
\text{HM\_ACA\_R}_{i\_\text{volume}} &= \text{HM\_ACA\_R\_Radio} \times \text{HM\_volume}_i \tag{5-10} \\
\text{ED\_ACA\_R}_{i\_\text{volume}} &= \text{ED\_ACA\_R\_Radio} \times \text{ED\_volume}_i \tag{5-11}
\end{align}

其中:

- \( i \) 表示第 \( i \) 次随访影像。通过计算以上所有脑部位置的血肿/水肿体积,再通过处理总的血肿体积变化 \( \eta \) 和水肿体积变化 \( \Psi \) 的相同方式处理脑部各个位置的变化。并与个人史、治疗方法、疾病史、首次影像信息和随访的所有信息进行整合,从而进行统计分析,预测 sub001-sub100 和 sub131-sub160 号病人的 90 天 mRS 评分。

\paragraph{灰色关联分析}

通过统计所有数据集,总共有 72 个特征变量,所以考虑通过灰色关联分析进行降维处理,其中 72 个特征数据统计表见表 5-3。72 个特征选择关联度最高的前 20 个变量作为模型训练的特征数据。

\begin{table}[h]
\centering
\caption{特征序列筛选结果}
\begin{tabular}{c c c}
\hline \hline
\multicolumn{3}{c}{ 特征 } \\
\hline
单位时间 PCA\_R 区域血肿 & 单位时间 Pons\_Medulla\_L & 单位时间内水肿体积变化 \\
\hline \hline
\end{tabular}
\end{table}

\begin{table}
\centering
\begin{tabular}{c c c}
\hline \hline
体积变化 & 区域血肿体积变化 & \\
单位时间 MCA\_R 区域水肿 & 单位时间 PCA\_R 区域水肿 & 单位时间 Pons\_Medulla\_R \\
体积变化 & 体积变化 & 区域水肿体积变化 \\
单位时间 MCA\_L 区域水肿 & 单位时间 PCA\_L 区域水肿 & 收缩压 \\
体积变化 & 体积变化 & \\
original\_shape\_LeastAxisLength & original\_shape\_MajorAxisLength & original\_shape\_Maximum2DDiameterColumn \\
original\_shape\_Maximum2DDiameterRow & original\_shape\_Maximum2DDiameterSlice & original\_shape\_Maximum3DDiameter \\
original\_shape\_MinorAxisLength & original\_shape\_SurfaceArea & NCCT\_original\_firstorder\_Maximum \\
\hline \hline
\end{tabular}
\end{table}

\begin{figure}[h]
    \centering
    \includegraphics[width=\textwidth]{image1.png}
    \caption{各特征值与患者90天mRS评分之间关联度}
    \label{fig:5-7}
\end{figure}

图 5-7 各个特征变量与 90 天 mRS 评分之间的影响

\subsubsection{问题三(b)模型的求解}

经过 BP 神经网络训练对患者 sub001 至 sub100,sub131 至 sub160 号病人进行预测 90 天的 mRS 评价,预测值用 m 进行表示,则填入答案表的结果如(5-12)。

\begin{figure}[h]
    \centering
    \includegraphics[width=\textwidth]{image2.png}
    \caption{BP神经网络验证集的预测值与实际值对比图}
    \label{fig:5-8}
\end{figure}

图 5-8 BP 神经网络测试 90 天 mRS 评分图

40

故最后的结果数据模型表示为:
\begin{equation}
\text{Sheet}_4_{\text{列}} = 90 \text{天 mRS 评分} =
\begin{cases}
0 & , m < 0.5 \\
1 & , 0.5 \leq m < 1.5 \\
2 & , 1.5 \leq m < 2.5 \\
3 & , 2.5 \leq m < 3.5 \\
4 & , 3.5 \leq m < 4.5 \\
5 & , 4.5 \leq m < 5.5 \\
6 & , m \geq 5.5
\end{cases}
\tag{5-12}
\end{equation}

\subsection{问题三(c)建模的分析与求解}

本题需要根据各影响因素与 mRS 评分的关联关系,为出血性脑卒中临床决策提供建议,这些关联关系的分析有助于改善出血性脑卒中患者的护理和治疗,提供更个性化的医疗建议,以提高患者的康复机会和生活质量。这也突显了数据分析和医学领域的交叉合作的重要性,以促进医疗研究和患者护理的不断改进。

\subsubsection{数据处理}

\paragraph{影响因素}

根据所提供的数据以及问题一、问题二和问题三的数据分析和建模求解工作,得到了一个庞大的数据集,其中包含了 100 多个潜在影响因素。为了更好地理解出血性脑卒中患者的预后情况,整合了数据集中所有的 72 个特征变量,作为影响血肿/水肿的因素。这些因素具有潜在的与 90 天 mRS 评分相关的特征,涵盖了患者的个人史、疾病史、治疗方法以及多个影像特征,包括血肿/水肿体积、血肿/水肿位置、信号强度特征以及形状特征等。

\begin{table}[h]
\centering
\caption{表 5-3 整合特征变量集合}
\begin{tabular}{c c c}
\hline \hline
\multicolumn{3}{c}{ 影响因素 } \\
\hline
脑出血前 mRS 评分 & 高血压病史 & 卒中病史 \\
糖尿病史 & 房颤史 & 冠心病史 \\
吸烟史 & 饮酒史 & 脑室引流 \\
止血治疗 & 降颅压治疗 & 降压治疗 \\
镇静、镇痛治疗 & 止吐护胃 & 营养神经 \\
单位时间内血肿体积变化 & 单位时间内 ACA\_R 区域血肿体积变化 & 单位时间内 MCA\_R 区域血肿体积变化 \\
单位时间内 PCA\_R 区域血肿体积变化 & 单位时间内 Pons\_Medulla\_R 区域血肿体积变化 & 单位时间内 Cerebellum\_R 区域血肿体积变化 \\
单位时间内 ACA\_L 区域血肿体积变化 & 单位时间内 MCA\_L 区域血肿体积变化 & 单位时间内 PCA\_L 区域血肿体积变化 \\
单位时间内 & 单位时间内 Cerebellum\_L 区域血肿 & 单位时间内水肿体积变化 \\
\hline \hline
\end{tabular}
\end{table}

\begin{tabular}{l l l}
Pons_Medulla_L 区域血肿 & 体积变化 & \\
体积变化 & & \\
单位时间内 ACA_R 水肿体积变化 & 单位时间内 MCA_R 水肿体积变化 & 单位时间内 PCA_R 水肿体积变化 \\
单位时间内 Pons_Medulla_R 水肿体积变化 & 单位时间内 Cerebellum_R 水肿体积变化 & 单位时间内 ACA_L 水肿体积变化 \\
单位时间内 MCA_L 水肿体积变化 & 单位时间内 PCA_L 水肿体积变化 & 单位时间内 Pons_Medulla_L 水肿体积变化 \\
单位时间内 Cerebellum_L 水肿体积变化 & 年龄 & 发病到首次影像检查时间间隔 \\
收缩压 & 舒张压 & original_shape_Elongation \\
original_shape_Flatness & original_shape_LeastAxisLength & original_shape_MajorAxisLength \\
original_shape_Maximum2DDiameterColumn & original_shape_Maximum2DDiameterRow & original_shape_Maximum2DDiameterSlice \\
original_shape_Maximum3DDiameter & original_shape_MeshVolume & original_shape_MinorAxisLength \\
original_shape_Sphericity & original_shape_SurfaceArea & original_shape_SurfaceVolumeRatio \\
original_shape_VoxelVolume & NCCT_original_firstorder_10Percentile & NCCT_original_firstorder_90Percentile \\
\hline
NCCT_original_firstorder_Energy & NCCT_original_firstorder_Entropy & NCCT_original_firstorder_InterquartileRange \\
NCCT_original_firstorder_Kurtosis & NCCT_original_firstorder_Maximum & NCCT_original_firstorder_MeanAbsoluteDeviation \\
NCCT_original_firstorder_Mean & NCCT_original_firstorder_Median & NCCT_original_firstorder_Minimum \\
NCCT_original_firstorder_Range & NCCT_original_firstorder_RobustMeanAbsoluteDeviation & NCCT_original_firstorder_RootMeanSquared \\
NCCT_original_firstorder_Skewness & NCCT_original_firstorder_Uniformity & NCCT_original_firstorder_Variance \\
\hline
\end{tabular}

\paragraph{灰色关联度分析}

灰色关联度分析是一种多变量分析方法,用于研究不同因素之间的关联关系。本文采用灰色关联度分析来理清这 72 项高质量影响因素对出血性脑卒中预后的影响,通过量化这些因素与患者 90 天 mRS 评分之间的关联程度,识别出对患者的预后产生重要因素,并根据此提出可能的干预措施。

\begin{figure}[h]
    \centering
    \includegraphics[width=\textwidth]{image.png}
    \caption{各特征值与患者 90 天 mRS 评分之间的关联度}
    \label{fig:5-9}
\end{figure}

\subsubsection{数据分析}

\paragraph{各因素影响力分析}

灰度关联度越高,表示这些因素之间可能存在着较强的因果或相关性关系,其中一个因素的变化可能会引起另一个因素的变化。根据图 5-6 各特征值与患者 90 天 mRS 评分之间的关联度对各因素影响力进行分析。从图中可以分析出,所有特征值的灰色关联度都大于 0.5,说明这 72 项高质量影响因素对出血性脑卒中预后的影响都较大,但是这些特征值之间也存在关联度的高度。可以按照 0.05 为一个区间,将特征值分为 5 个梯度。

第一梯度特征值的灰色关联度在 0.70 到 0.75 之间,介于该区间内的 16 个影响因素对值对出血性脑卒中的影响力最高,如单位时间内 HM\_PCA\_R\_Ratio 血肿体积变化、单位时间呢 iHM\_Pons\_Medulla\_L\_Ratio 血肿体积变化、单位时间内水肿体积变化、单位时间内 ED\_MCA\_R\_Ratio 水肿体积变化、单位时间内 ED\_PCA\_R\_Ratio 水肿体积变化等。这些因素是研究出血性脑卒中过程中的重点关注对象,对于治疗有着重要的帮助。

\begin{table}[h]
\centering
\caption{第一梯度(灰色关联度:0.70-0.75)影响因素}
\label{tab:5-4}
\begin{tabular}{l l}
\hline
\multicolumn{2}{c}{第一梯度(灰色关联度:0.70-0.75)影响因素} \\
\hline
单位时间内 HM\_PCA\_R\_Ratio & 单位时间内 HM\_Pons\_Medulla\_L\_Ratio 血肿体积变化 \\
单位时间内水肿体积变化 & 单位时间内 ED\_MCA\_R\_Ratio 水肿体积变化 \\
\hline
\end{tabular}
\end{table}

\begin{table}
\centering
\begin{tabular}{l l}
单位时间内 ED\_PCA\_R\_Ratio 水肿体积变化 & 单位时间内 ED\_MCA\_L\_Ratio 水肿体积变化 \\
单位时间内 ED\_PCA\_L\_Ratio 水肿体积变化 & original\_shape\_LeastAxisLength \\
original\_shape\_MajorAxisLength & original\_shape\_Maximum2DDiameterColumn \\
original\_shape\_Maximum2DDiameterRow & original\_shape\_Maximum2DDiameterSlice \\
original\_shape\_Maximum3DDiameter & original\_shape\_MinorAxisLength \\
NCCT\_original\_firstorder\_Maximum & NCCT\_original\_firstorder\_Variance \\
\end{tabular}
\end{table}

第二梯度特征值的灰色关联度在 0.65 到 0.70 之间,介于该区间内的 24 个影响因素对值对出血性脑卒中的影响力较高,如单位时间血肿体积变化、单位时间内 ED\_Pons\_Medulla\_R\_Ratio 水肿体积变化、年龄、收缩压、舒张压等。位于该梯度内的影响因素占比最高,这些因素也是出血性脑卒中预后预测应重点关注的对象。

\textbf{eDeviation}

\textbf{NCCT\_original\_firstorder\_Skewness} \hfill \textbf{NCCT\_original\_firstorder\_Uniformity}

\bigskip

第三梯度特征值的灰色关联度在 0.60 到 0.65 之间,介于该区间内的 10 个影响因素对值对出血性脑卒中的影响力中等,如糖尿病史、冠心病史、脑室引流、单位时间内 HM\_Pons\_Medulla\_R\_Ratio、HM\_PCA\_L\_Ratio 血肿体积变化等。这些因素可以在前两个梯度的高影响力影响因素的基础上起辅助预后预测的作用,值得关注。

\textbf{表 5-6 第三梯度(灰色关联度:0.60-0.65)影响因素}

\begin{tabular}{l l}
\hline
\multicolumn{2}{c}{第三梯度(灰色关联度:0.60-0.65)影响因素} \\
\hline
糖尿病史 & 冠心病史 \\
脑室引流 & 单位时间内 HM\_Pons\_Medulla\_R\_Ratio 水肿体积变化 \\
单位时间内 HM\_PCA\_L\_Ratio 血肿体积变化 & 单位时间内 ED\_ACA\_R\_Ratio 水肿体积变化 \\
单位时间内 ED\_ACA\_L\_Ratio 水肿体积变化 & 发病到首次影像检查时间间隔 \\
original\_shape\_SurfaceVolumeRatio & \\
\hline
\end{tabular}

\bigskip

第四梯度和第五梯度的灰色关联度分别在 0.55 到 0.60、0.50 到 0.55 之间,介于该区间内的影响因素对出血性脑卒中的影响力较低,如高血压病史、饮酒史、止血治疗、降压治疗等,这些因素在条件允许的情况下也可纳入考虑范围,但不是预后预测的重点关注对象。

\textbf{表 5-7 第四梯度(灰色关联度:0.55-0.60)影响因素}

\begin{tabular}{l l}
\hline
\multicolumn{2}{c}{第四梯度(灰色关联度:0.55-0.60)影响因素} \\
\hline
高血压病史 & 饮酒史 \\
止血治疗 & 降压治疗 \\
止吐护胃 & 营养神经 \\
单位时间内 HM\_Cerebellum\_R\_Ratio 血肿体积变化 & 单位时间内 HM\_Cerebellum\_L\_Ratio 血肿体积变化 \\
单位时间内 ED\_Cerebellum\_R\_Ratio 水肿体积变化 & 单位时间内 ED\_Cerebellum\_L\_Ratio 水肿体积变化 \\
\hline
\end{tabular}

\begin{table}
\centering
\caption{第五梯度(灰色关联度:0.50-0.55)影响因素}
\begin{tabular}{l l}
\hline
脑出血前 mRS 评分 & 卒中病史 \\
房颤史 & 吸烟史 \\
降颅压治疗 & 镇静、镇痛治疗 \\
单位时间内 HM\_ACA\_R\_Ratio 血肿体积变化 & 单位时间内 HM\_MCA\_R\_Ratio 血肿体积变化 \\
单位时间内 HM\_ACA\_L\_Ratio 血肿体积变化 & 单位时间内 HM\_MCA\_L\_Ratio 血肿体积变化 \\
\hline
\end{tabular}
\end{table}

\paragraph{治疗干预措施分析}

针对第一梯度内影响力最高的 16 个影响因素,提出如下几个方面的预后预防措施:

\begin{enumerate}
    \item 脑出血区域的控制:控制出血扩散的速度可能有助于减少脑组织受损的范围。针对高速扩散的区域,可以考虑局部止血治疗。
    \item 水肿管理:针对脑组织的水肿,及时的脱水治疗和神经保护措施可能有助于减轻脑损伤。
    \item 形状特征分析:形状特征的分析可以帮助医生更好地理解出血的类型和扩散情况,从而指导治疗策略。
    \item 高血压和血压管理:高血压是脑出血的一个重要危险因素,积极管理高血压可能降低出血风险。
    \item 个体化治疗:患者的情况因人而异,因此治疗策略应该根据患者的个体情况进行定制化,包括治疗方法的选择和时机。
    \item 密切监测和随访:对出血性脑卒中患者的预后情况需要进行定期的临床监测和随访,以及时调整治疗方案。
\end{enumerate}

针对第二梯度内的影响力较高的 24 个因素,提出如下几个方面的预后预防措施:

\begin{enumerate}
    \item 控制高血压:高血压是脑卒中的重要危险因素之一。患者应定期测量血压,并遵循医生的建议,采取药物治疗或生活方式改变来控制高血压。
    \item 年龄管理:年龄是脑卒中风险的非可逆因素。尽管无法改变年龄,但在老年患者中,应特别注意脑卒中的预防和管理。
    \item 药物管理:对于高风险患者,如有医生建议,可能需要采取抗凝血药物或抗血小板药物来降低血栓形成的风险。
\end{enumerate}

针对第三梯度内影响力中等的 10 个影响因素,提出如下几个方面的预后预防措施:

\begin{enumerate}
    \item 病史管理:对于患有各种疾病特别是具有糖尿病史、冠心病史的患者,应该控制相关病的风险因素管理,做好定期检查,以减少血管损伤和脑出血的风险。
    \item 脑室引流风险评估:对于需要脑室引流的患者,医生应仔细评估治疗的必要性,并确保引流过程安全无误。患者需要接受定期的医学监测和随访。
\end{enumerate}

\section{发病到首次影像检查时间间隔}

尽早进行首次影像检查可以更早地发现脑卒中病变,有助于早期干预和治疗。

针对第四梯度和第五梯度中影响力较低的影响因素,尽管它们在预后中的作用相对较小,但仍然具有一定的辅助作用,特别是在制定个性化的治疗方案时。这些因素虽然不是主要的决定性因素,但在考虑整体治疗策略时应被纳入考虑。

总之,在跨学科团队的协作和大规模数据分析的支持下,应深入分析患者综合临床数据、医学影像、实验室检查和患者历史等多方面信息,以提高出血性脑卒中患者的预后。同时,患者也应积极配合医疗建议,采取健康的生活方式,以降低脑卒中发生和预后恶化 的风险。

\section{问题解决的模型评价与改进}

\subsection{模型的优点}

(1) 本文针对于病人信息进行了合理而仿真的模型建立,将复杂的个人健康状况抽象为易于处理的直观数据,如病人的年龄、性别、过往病史及影像检查信息等。这样的模型能够节约大量的时间和人力成本,从而为后续的数据处理打下了良好的基础。

(2) 本文采用了离散化处理、归一化处理等多种处理方式,来进行数据分析和归纳。一方面能够从多种角度更加科学地分析数据,确保结果更贴近现实;另一方面也弥补了数据缺失等异常情况,确保分析结果的可靠性。

(3) 本文对于模型的建立上采取了规范化的数学分析,对于单一的患者,不仅从静态维度分析了患者的身体情况,而且从动态维度分析了患者在发病过程中的数据变化,还利用了深度学习的方法进行科学的预测。

(4) 本文采用搭建多元线性回归模型的方式,利用多元回归模型分析模型的拟合系数,从而对不同治疗方法对水肿体积变化的影响做出判断,以便于得出不同治疗方式在不同场景下的治疗效果。

(5) 本文通过合理的建模和数据处理,基于灰色关联分析的方法,经过数据标准化、建立灰色关联度矩阵、确定关联系数、排序和分析等步骤,筛选出关联度最高的特征变量。

\subsection{模型的缺点}

本文的一切模型建立和数据分析,都是建立题目所给的的数据集基础上。虽然本次研究的方法严格按照了规范,但由于数据对象样本较小,可能存在一定的偶然性和偏差的缘故,最后的模型与实际应用场景还存在一定的差距。

\section{参考文献}

[1] 鲁伟, 向建平. 人工智能在脑血管疾病诊疗中的相关应用 [J]. 人工智能, 2021(03):72-78. DOI:10.16453/j.cnki.ISSN2096-5036.2021.03.008.

[2] 陈真诚, 蒋勇, 胥明玉, 等. 人工智能技术及其在医学诊断中的应用及发展 [J]. 生物医学工程学杂志, 2002, 19(3): 505-509.

[3] 陈华, 张倩, 王照丽等. 基于主成分分析与聚类分析的水质综合评价研究 [J]. 科技创新与应用, 2023, 13(26):88-92. DOI:10.19981/j.CN23-1581/G3.2023.26.020.

[4] 李倩, 刘芸宏, 吴晓慧等. 基于决策树和 Logistic 回归预测出血性脑卒中手术后医院感染风险 [J]. 中华医院感染学杂志, 2021, 31(23):3556-3561.

[5] 韩冬, 李其花, 蔡巍, 等. 人工智能在医学影像中的研究与应用 [J]. 大数据, 2019, 5(1): 2019004.

[6] 刘思峰. 灰色关联分析模型研究进展. 系统工程理论与实践 33.8 (2013): 2041-2046

[7] 张思琪, 杨添淞, 马帅等. 深度学习在脑卒中诊断与防治中的研究进展 [J]. 磁共振成像, 2022, 13(11):125-128.

[8] 欧艳艳. 基于多元回归分析的公共图书馆婴幼儿阅读服务体验评估体系的建立与影响因素量化分析 [J]. 河南图书馆学刊, 2023, 43(09):35-39.

[9] 沈志慧, 杨廷琴, 吴江等. 出血性与缺血性脑卒中发病特征与危险因素的对比分析 [J]. 中国医药指南, 2023, 21(23):66-68. DOI:10.15912/j.cnki.gocm.2023.23.022.

[10] 张晓林. 基于机器学习的首发缺血性脑卒中患者 1 年复发危险因素及预测研究 [D]. 南昌大学, 2023. DOI:10.27232/d.cnki.gnchu.2022.000640.

[11] 李萍, 曾令可, 税安泽, 等. 基于 MATLAB 的 BP 神经网络预测系统的设计 [J]. 计算机应用与软件, 2008, 25(4): 149-150.

[12] Chang P D, Kuoy E, Grinband J, et al. Hybrid 3D/2D convolutional neural network for hemorrhage evaluation on head CT [J]. American Journal of Neuroradiology, 2018, 39(9): 1609-1616.

[13] Yihao C, Shengpan C, Jianbo C, et al. Perihematomal Edema After Intracerebral Hemorrhage: An Update on Pathogenesis, Risk Factors, and Therapeutic Advances [J]. Frontiers in Immunology, 2021, 12.

[14] Kurita, Takio. "Principal component analysis (PCA)." Computer Vision: A Reference Guide (2019): 1-4.

\end{document}

% Missing placeholders restored
\begin{figure}[h]
    \centering
    \includegraphics[width=\textwidth]{bp_network_algorithm_flow.png}
    \caption{5-5 BP 神经网络算法流程}
\end{figure}
\begin{table}
\centering
\caption{第二梯度(灰色关联度:0.65-0.70)影响因素}
\begin{tabular}{l l}
单位时间血肿体积变化 & 单位时间内 ED\_Pons\_Medulla\_R\_Ratio 水肿体积变化 \\
年龄 & 收缩压 \\
舒张压 & original\_shape\_Elongation \\
original\_shape\_Flatness & original\_shape\_MeshVolume \\
original\_shape\_Sphericity & original\_shape\_SurfaceArea \\
original\_shape\_VoxelVolume & NCCT\_original\_firstorder\_10Percentile \\
NCCT\_original\_firstorder\_90Percentile & NCCT\_original\_firstorder\_Energy \\
NCCT\_original\_firstorder\_Entropy & NCCT\_original\_firstorder\_InterquartileRange \\
NCCT\_original\_firstorder\_Kurtosis & NCCT\_original\_firstorder\_MeanAbsoluteDeviation \\
NCCT\_original\_firstorder\_Mean & NCCT\_original\_firstorder\_Median \\
NCCT\_original\_firstorder\_Minimum & NCCT\_original\_firstorder\_Range \\
NCCT\_original\_firstorder\_RobustMeanAbsolute & NCCT\_original\_firstorder\_RootMeanSquared \\
\end{tabular}
\end{table}