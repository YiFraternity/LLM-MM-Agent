\begin{center}
\textbf{“华为杯”第十四届中国研究生数学建模竞赛}
\end{center}

\begin{center}
\includegraphics[width=0.3\textwidth]{image1.png} \quad
\includegraphics[width=0.3\textwidth]{image2.png}
\end{center}

\begin{center}
\textbf{“华为杯”第十四届中国研究生数学建模竞赛}
\end{center}

\begin{tabular}{ll}
学校 & 河海大学 \\
\hline
参赛队号 & 10294008 \\
\hline
队员姓名 & 1. 宗刘俊 \\
 & 2. 陈洁 \\
 & 3. 张玉玲 \\
\end{tabular}

\begin{center}
\textbf{参赛密码} \underline{\hspace{5cm}} \\
(由组委会填写)
\end{center}

\begin{center}
\includegraphics[width=0.3\textwidth]{image1.png} \\
\includegraphics[width=0.3\textwidth]{image2.png} \quad
\includegraphics[width=0.3\textwidth]{image3.png}
\end{center}

\begin{center}
\textbf{“华为杯”第十四届中国研究生数学建模竞赛}
\end{center}

\begin{center}
\textbf{题目} \quad \textbf{多波次导弹发射中的规划问题}
\end{center}

\begin{center}
\textbf{摘 \qquad 要:}
\end{center}

本文针对多波次导弹发射中的规划问题,建立了单目标优化模型、评价模型和多目标优化模型,综合利用 Floyd 算法、Lingo 单目标优化算法对多波次导弹发射运输规划问题进行分析和求解,利用层次分析法对路网节点重要程度进行评价。

针对问题一,采用单目标优化模型,由于问题一约束条件复杂,很难在初始阶段直接建立车载发射装置、导弹发射点和转载地域三者之间的连接关系,求得全局最优解。为求得较优解,先将目标函数整体暴露时间分成 3 个阶段,第 1 阶段从车辆驶离待机地域时刻到第一波次导弹齐射时刻,第 2 阶段从第一波次导弹齐射时刻到进入装载地域装弹完毕时刻,第 3 阶段从车辆驶离转载地域时刻到第二波次导弹齐射时刻,下一个阶段在上一个阶段优化结果的基础上进行求解。结合 Floyd 最小路径算法和 Lingo 0-1 规划算法依次求解 3 个阶段使整体暴露时间最短的机动路线并整合,最终得到第一波次导弹齐次发射时刻为第 178.4min,第二波次导弹齐次发射时刻为第 451.7min,两波次发射任务的整体暴露时间为 8598.8min。

问题二在问题一的基础上选择节点位置临时增设转载地域,同样是单目标优化问题,可以采用问题一的模型进行求解。由于问题二要求增设 2 个转载地域,排列组合情况较多,为便于解决问题,先考虑仅增加单个转载地域的情形,求解出使整体暴露时间最短的机动路线并分析其改善效果,最终选择对整体暴露时间缩减效果显著、对问题一产生集群现象的两转载地域有较好分流效果且靠近主干道交通便利的 J26 和 J49 作为新增的 2 个转载地域,两波次发射任务的整体暴露时间缩减 10.5\%。

问题三要求选择隐蔽待机点安放新增的 3 台 C 类车载发射装置以替代原有 3 台 C 类装置,同样也是单目标优化问题。首先剔除第一题中第 3 阶段暴露时间最长的 3 台 C 装置,从隐蔽点出发的 3 台 C 装置有 15 个可用发射点,用 0-1 规划算法把 6 个隐蔽待机点分配到 15 个发射点位使整体暴露时间最短。根据 Lingo 优化结果,在道路节点 J14 隐藏 1 台 C 装置前往发射点位 F27,在道路节点 J13 隐藏 2 台 C 装置分别前往发射点位 F06 和 F04,该调整方案使两波次发射任务的整体暴露时间缩减 9.5%。

问题四要求结合路网的特点,建立合理的评价指标,量化分析该路网最可能受到敌方攻击破坏的 3 个道路节点。考虑攻防双方的对抗博弈,假设敌方未知我方各道路节点功能性质,采用层次分析法,选取道路节点强度、道路节点与主干道节点连接数量作为 2 个评价指标,计算得出路网中重要程度最大的三个道路节点:J32、J03 和 J21,即作为最可能受到敌方攻击破坏的节点。

问题五在问题一的基础上增加了 2 个目标:采用适当分散机动策略、缩短单台车载发射装置的最长暴露时间,因此是一个多目标优化问题。首先通过引入不同的权重系数将其转为单目标优化问题,把采用分散机动策略,即分散发射点位这一目标放在最优先的位置,用 k-means 聚类法以转载地域为中心聚类,然后按照每个转载地域分到的发射点位数量比例将 24 个导弹车载发射装置分配给各个转载地域,最后综合应用 Floyd 算法和 Lingo 优化算法对路线进行部署规划,发现采用分散机动策略必须以两波次发射任务的整体暴露时间和单台发射装置最长暴露时间的增大为代价。

关键词:多波次打击;Floyd 算法;0-1 优化算法;层次分析法;聚类分析

\section*{目录}
\begin{itemize}
    \item[] 1 问题重述 \dotfill 6
    \begin{itemize}
        \item[] 1.1 问题背景 \dotfill 6
        \item[] 1.2 需要解决的问题 \dotfill 6
    \end{itemize}
    \item[] 2 问题分析 \dotfill 7
    \begin{itemize}
        \item[] 2.1 问题1的分析 \dotfill 7
        \item[] 2.2 问题2的分析 \dotfill 8
        \item[] 2.3 问题3的分析 \dotfill 8
        \item[] 2.4 问题4的分析 \dotfill 9
        \item[] 2.5 问题5的分析 \dotfill 9
    \end{itemize}
    \item[] 3 模型假设 \dotfill 9
    \item[] 4 符号说明 \dotfill 9
    \item[] 5 问题1模型建立与求解 \dotfill 10
    \begin{itemize}
        \item[] 5.1 问题1模型的建立 \dotfill 10
        \begin{itemize}
            \item[] 5.1.1 目标函数 \dotfill 10
            \item[] 5.1.2 各阶段的约束条件 \dotfill 10
        \end{itemize}
        \item[] 5.2 求解方案 \dotfill 12
        \item[] 5.3 第1阶段最短整体暴露时间 \dotfill 12
        \begin{itemize}
            \item[] 5.3.1 基于Floyd算法求解最短路问题 \dotfill 12
            \item[] 5.3.2 基于指派问题的路线分配方案求解 \dotfill 14
            \item[] 5.3.3 同一时段内路径冲突的解决方案 \dotfill 16
        \end{itemize}
        \item[] 5.4 第2阶段最短整体暴露时间 \dotfill 16
        \begin{itemize}
            \item[] 5.4.1 第一波次发射点位到转载地域的最短路径 \dotfill 17
            \item[] 5.4.2 转载地域的重分配 \dotfill 18
            \item[] 5.4.3 同一时段内路径冲突的解决方案 \dotfill 18
        \end{itemize}
        \item[] 5.5 第3阶段最短整体暴露时间 \dotfill 20
        \begin{itemize}
            \item[] 5.5.1 基于Floyd算法和0-1规划指派问题的路径分配 \dotfill 20
            \item[] 5.5.2 转载后车辆延迟出发的现象 \dotfill 21
            \item[] 5.5.3 同一时段内路径冲突的解决方案 \dotfill 21
        \end{itemize}
        \item[] 5.6 整体暴露时间最短的方案 \dotfill 21
    \end{itemize}
    \item[] 6 问题2模型的建立与求解 \dotfill 22
    \begin{itemize}
        \item[] 6.1 问题2模型的建立 \dotfill 22
        \item[] 6.2 问题2的求解 \dotfill 23
        \begin{itemize}
            \item[] 6.2.1 增加单个转载地域的效果 \dotfill 23
            \item[] 6.2.2 增加两个转载地域的效果 \dotfill 24
        \end{itemize}
    \end{itemize}
    \item[] 7 问题3模型的建立与求解 \dotfill 25
    \begin{itemize}
        \item[] 7.1 问题3模型的建立 \dotfill 25
        \item[] 7.2 问题3的求解 \dotfill 25
    \end{itemize}
\end{itemize}

\begin{itemize}
    \item[8] 问题4模型建立与求解 \dotfill 26
        \begin{itemize}
            \item[8.1] 问题4模型的建立 \dotfill 26
                \begin{itemize}
                    \item[8.1.1] 层次分析前的准备:图论中的节点强度 \dotfill 26
                    \item[8.1.2] 层次分析法 \dotfill 26
                \end{itemize}
            \item[8.2] 问题4求解 \dotfill 27
        \end{itemize}
    \item[9] 问题5模型的建立与求解 \dotfill 29
        \begin{itemize}
            \item[9.1] 问题5模型的建立 \dotfill 29
                \begin{itemize}
                    \item[9.1.1] 目标函数 \dotfill 29
                    \item[9.1.2] 各阶段的约束条件 \dotfill 29
                \end{itemize}
            \item[9.2] 问题5求解 \dotfill 29
        \end{itemize}
    \item[10] 模型的讨论与评价 \dotfill 32
        \begin{itemize}
            \item[10.1] 模型的优点 \dotfill 32
            \item[10.2] 模型的缺点及改进方向 \dotfill 32
        \end{itemize}
    \item[11] 结论 \dotfill 32
    \item[12] 参考文献 \dotfill 33
    \item 附录MATLAB和Lingo源程序 \dotfill 34
\end{itemize}

\section{问题重述}

\subsection{问题背景}

随着导弹武器系统的不断发展,导弹在未来作战中将发挥越来越重要的作用,导弹作战将是未来战场的主要作战样式之一。

为了提高导弹部队的生存能力和机动能力,常规导弹大都使用车载发射装置,平时在待机地域隐蔽待机,在接受发射任务后,各车载发射装置从待机地域携带导弹沿道路机动到各自指定发射点位实施发射。每台发射装置只能载弹一枚,实施多波次发射时 [1-4],完成了上一波次发射任务的车载发射装置需要立即机动到转载地域(用于将导弹吊装到发射装置的专门区域)装弹,完成装弹的发射装置再机动至下一波次指定的发射点位实施发射。连续两波次发射时,每个发射点位使用不超过一次。

某部参与作战行动的车载发射装置共有 24 台,依据发射装置的不同大致分为 A、B、C 三类,其中 A、B、C 三类发射装置的数量分别为 6 台、6 台、12 台,执行任务前平均部署在 2 个待机地域(D1,D2)。所属作战区域内有 6 个转载地域(Z01~Z06)、60 个发射点位(F01~F60),每一发射点位只能容纳 1 台发射装置。各转载地域最多容纳 2 台发射装置,但不能同时作业,单台转载作业需时 10 分钟。各转载地域弹种类型和数量满足需求。相关道路情况如图 1 所示(道路节点 J01~J62),相关要素的坐标数据如附件 1 所示。图 1 中主干道路(图中红线)是双车道,可以双车通行;其他道路(图中蓝线)均是单车道,只能在各道路节点处会车。A、B、C 三类发射装置在主干道路上的平均行驶速度分别是 70 公里/小时、60 公里/小时、50 公里/小时,在其他道路上的平均行驶速度分别是 45 公里/小时、35 公里/小时、30 公里/小时。

部队接受发射任务后,需要为每台车载发射装置规划每个波次的发射点位及机动路线,要求整体暴露时间(所有发射装置的暴露时间之和)最短。本问题中的“暴露时间”是指各车载发射装置从待机地域出发时刻至第二波次发射时刻为止的时间,其中发射装置位于转载地域内的时间不计入暴露时间内。暂不考虑发射装置在发射点位必要的技术准备时间和发射后发射装置的撤收时间。

\subsection{需要解决的问题}

(1)该部接受到实施两个波次的齐射任务(齐射是指同一波次的导弹同一时刻发射),每个波次各发射 24 枚导弹。给出具体发射点位分配及机动路线方案,使得完成两个波次发射任务的整体暴露时间最短。统一以第一波次的发射时刻作为第二波次机动的起始时刻。

(2)转载地域的合理布设是问题的“瓶颈”之一。除已布设的 6 个转载地域外,可选择在道路节点 J25、J34、J36、J42、J49 附近临时增设 2 个转载地域(坐标就取相应节点的坐标)。应该如何布设临时转载地域,使得完成两个波次发射任务的整体暴露时间最短。

(3)新增 3 台 C 类发射装置用于第二波次发射。这 3 台发射装置可事先选择节点 J04、J06、J08、J13、J14、J15 附近隐蔽待机(坐标就取相应节点的坐标),即这 3 台发射装置装弹后从待机地域机动到隐蔽待机点的时间不计入暴露时间内。每一隐蔽待机点至多容纳 2 台发射装置。待第一波次导弹发射后,这 3 台发射装置机动至发射点位参与第二波次的齐射,同时被替代的 3 台 C 类发射装置完成第一波次齐射后择机返回待机地域(返回时间不计入暴露时间)。转载地域仍为事先布设的 6 个的前提下,应该如何选择隐蔽待机点,使得完成两个波次发射任务的整体暴露时间最短。

(4)道路节点受到攻击破坏会延迟甚至阻碍发射装置按时到达指定发射点位。请

结合图 1 路网特点,考虑攻防双方的对抗博弈,建立合理的评价指标,量化分析该路网最可能受到敌方攻击破坏的 3 个道路节点。

(5)在机动方案的拟制中,既要考虑整体暴露时间尽可能短,也要规避敌方的侦察和打击,采用适当分散机动的策略,同时还要缩短单台发射装置的最长暴露时间。综合考虑这些因素,重新讨论问题(1)。

\section{问题分析}

\subsection{问题 1 的分析}

问题 1 要求给出具体发射点位分配及机动路线方案,使得完成两个波次发射任务的整体暴露时间最短,即所有发射装置的暴露时间之和最短。显然这是一个最优化问题。但要解决好这个问题首先得从理解“发射装置的暴露时间”这个概念着手。

“发射装置的暴露时间”是指在不计发射装置位于转载地域内的时间的情况下,各车载发射装置从待机地域出发时刻至第二波次发射时刻为止的时间。对 24 台发射装置编号,假设第 $k$ 台发射装置的暴露时间为 $t_k (k=1,2,\cdots,24)$,则对于 $t_k$ 我们可以建立如下的关系式:

\begin{equation}
t_k = t_{k_1} + t_{k_2} + t_{k_3} + t_{k_4} + t_{k_5} + t_{k_6} + t_{k_7} + t_{k_8}
\tag{2-1}
\end{equation}

\begin{equation}
T = \sum_{k=1}^{24} t_k
\tag{2-2}
\end{equation}

式中:$t$——整体暴露时间;

$t_k$——第 $k$ 台车载发射装置的暴露时间;

$t_{k_1}$——第一波次运输导弹的时间,指导弹从待机地域运输至第一波次发射点位消耗的时间;

$t_{k_2}$——第一波次会车等待时间,指从待机地域运输导弹至发射点位中途由于会车等待消耗的时间;

$t_{k_3}$——第一波次发射等待时间,指已经就位的导弹发射装置等待其他导弹发射装置就位的时间;

$t_{k_4}$——转载时间,指第一波次发射完成后导弹发射装置赶往转载地域消耗的时间;

$t_{k_5}$——转载等待时间,指转载地域内已有 2 台发射装置时第 3 台发射装置在转载地域外等待的时间;

$t_{k_6}$——第二波次导弹运输时间,指导弹从转载地域运输至第二波次发射点位消耗的时间;

$t_{k_7}$——第二波次会车等待时间,指从转载地域运输导弹至第二波次发射点位中途由于会车等待消耗的时间;

$t_{k_8}$——第二波次发射等待时间,指已经就位的导弹发射装置等待其他导弹发射装置就位的时间。

公式 2-1 中的组成成分较多,为了便于考虑问题,我们把第 $k$ 台车载发射装置的暴露时间 $t_k$ 分为 3 个阶段,划分规则见公式 2-3:

\begin{equation}
t_k = \underbrace{t_k^{(1)}}_{t_{k_1}+t_{k_2}+t_{k_3}} + \underbrace{t_k^{(2)}}_{t_{k_4}+t_{k_5}} + \underbrace{t_k^{(3)}}_{t_{k_6}+t_{k_7}+t_{k_8}}
\tag{2-3}
\end{equation}

式中:$t_{k}^{(1)}$——第 1 阶段,第一波次发射阶段,从导弹运出待机区域开始至第一波次发射时刻为止;  
$t_{k}^{(2)}$——第 2 阶段,导弹装载阶段,从第一波次发射时刻开始至导弹发射装置进入转载地域为止;  
$t_{k}^{(3)}$——第 3 阶段,第二波次发射阶段,从导弹离开装载地域开始至第二波次发射时刻为止。

要让整体暴露时间最小,只要让 3 个阶段的和最小即可,即:
\begin{equation}
\min T = \min \sum_{k=1}^{24} t_k = \min \sum_{k=1}^{24} \sum_{i=1}^{3} t_k^{(i)}
\tag{2-4}
\end{equation}

由此可见,本题是一个规划导弹发射装置的机动路线使整体暴露时间最短的单目标优化问题。

\subsection{2.2 问题 2 的分析}

问题 2 要求在道路节点 J25、J34、J36、J42、J49 附近增设 2 个转载地域同时保证完成两个波次发射任务的整体暴露时间最短,在考虑问题 1 时,我们已经发现了本题中转载点的“瓶颈效应”,主要表现在第 2 阶段规划从第一波次发射点位到转载地域的路径时出现的集群效应,即部分发射点位集中在同一个转载地域周围,这样会增加导弹发射装置的排队时间或者转道其他转载地域的额外消耗时间,最终导致整体暴露时间增大,因此在求解问题 2 时要充分考虑在规划问题 1 中发现的问题,要尽可能减轻周围发射点位密布的转载地域的压力;同时还要尽量靠近主干道,这样能有效缓解路径冲突的问题;再就是可以将增加的转载地域选在可用发射点位较多的区域,这样方便第二波次导弹发射点位的选取并且起到了减轻交通压力的效果。

\subsection{2.3 问题 3 的分析}

问题 3 需要新增 3 台 C 类发射装置用于第二波次发射。这 3 台发射装置可事先选择节点 J04、J06、J08、J13、J14、J15 附近隐蔽待机,待第一波次导弹发射后,这 3 台发射装置机动至发射点位参与第二波次的齐射,同时第一波次齐射后的 3 台 C 类发射装置替换下来择机返回待机地域(返回时间不计入暴露时间),每一隐蔽待机点至多容纳 2 台发射装置。我们需要求解选择隐蔽待机点的方案是使得完成两个波次发射任务的整体暴露时间最短。

问题 3 理论上也可用问题 1 的模型求解,二者的区别在于在第二波次发射的时候用新增的 3 台 C 类发射装置替换原来的 3 台 C 类发射装置,但可供选择的道路节点有 6 个,每个道路节点内的发射装置数量都有 0、1、2 三种可能,若采用与问题 1 相同的模型,则计算工程量太大,这是不合适的。因为只有在第一波次发射后才会发生替换,也不用去转载地域,所以我们将最晚到达第二波次的 3 辆 C 车替换掉,也就是第 2、3 阶段总耗时最长的 3 辆 C 类发射装置,替换时只要保证他们前往三个发射点位的整体暴露时间最短即可。去掉第一波次发射已用的 24 个发射点位和第二波次定好的 21 个发射点位,从这 6 个隐蔽点出发的 3 辆 C 类导弹发射装置可前往的发射点位共有 $60 - 24 - 21 = 15$ 个。

我们要求解的问题是将 6 个隐蔽点分配到 15 个发射点位使整体暴露时间最短的问题,可以采用 0-1 模型。

\subsection{问题 4 的分析}

问题 4 要求我们结合图 1 的路网特点,考虑攻防双方的对抗博弈,在敌方不知道我方各节点功能特性的前提下建立合理的评价指标,量化分析该路网最可能受到敌方攻击破坏的 3 个道路节点,这是一个评价类问题。直观上来看,路网节点是否可能受到敌方攻击与各道路节点的两个属性有关,一是该道路节点与其周边节点的连接程度,二是该道路节点是否与主干道相连。因此这里可以采用层次分析法,以破坏道路节点为目标层,该道路节点与其周边节点的连接程度和与主干道连接程度这两个因素为准则层,62 个道路结点为方案层。对于道路节点与主干道的连接程度这一因素我们很容易进行量化,计算其连接的主干道节点数量即可;而对于道路节点与其周边节点的连接程度这一因素,虽然较难量化,但我们可以借助图论中节点强度的理论完成。

\subsection{问题 5 的分析}

问题 5 是典型的多目标优化问题,主要需要达到的优化目标有:

(1) 考虑整体暴露时间尽可能短;

(2) 缩短单台发射装置的最长暴露时间;

(3) 为规避敌方的侦察和打击,采用适当分散机动的策略,在这里,我们把这一目标简化为:每个转载点至少有两台发射装置前往装载导弹。

综合考虑这些因素,为各目标引入不同的权重系数,将多目标问题转化为单目标问题进行求解。

\section{模型假设}

为了便于考虑问题,我们在不影响模型准确性的前提下,建立了如下假设:

(1) 假设车与车之间不存在安全间距即两辆同类型的车可同时经过某一路段的起始点;

(2) 若无更多的车在转载地域外等待进入,则在转载地域内装弹完成的车可继续停留其中接受掩护;

(3) 导弹发射装置的时速固定,不考虑速度较快的车在速度较慢的车后面减速跟车的情形;

(4) 不考虑天气、爆胎等不可抗力对导弹发射装置行驶速度的影响。

\section{符号说明}

$S_{k}^{(i)}$, 第 $k$ 辆导弹发射装置第 $i$ 阶段的出发位置;

$R_{k}^{(i)}$, 第 $k$ 辆导弹发射装置第 $i$ 阶段的停止位置;

$\delta_{h_{k,l}^{(i)},h_{k,l+1}^{(i)}}$, 第 $k$ 辆导弹发射装置第 $i$ 阶段所经路径中第 $l$ 个节点和第 $l+1$ 个节点的连通情况,因为均连通,所以为 1,否则为 0;

$w_{k}^{(i)}$, 第 $k$ 辆导弹发射装置第 $i$ 阶段的出发时刻;

$P_{h_{k,l}^{(i)}}$, 第 $k$ 辆导弹发射装置第 $i$ 阶段到达其机动路径第 $l$ 个节点的时刻;

$Q_{h_{k_{1},l}^{(1)},h_{k_{1},l+1}^{(1)}}$, 第 $k$ 辆导弹发射装置第 $i$ 阶段所经路径中第 $l$ 个节点到第 $l+1$ 个节点的时间区间。

\section{问题 1 模型建立与求解}

\subsection{问题 1 模型的建立}

根据前面对于问题 1 的分析,问题 1 是一个规划导弹发射装置的机动路线使整体暴露时间最短的单目标优化问题,因此找到目标函数和约束条件是解决本题的关键。

\subsubsection{目标函数}

在前文公式 2-4 的基础上考虑导弹发射装置沿程经过的路径节点,我们可以得到如下的目标函数:
\begin{equation}
\min T = \sum_{k=1}^{24} \sum_{i=1}^{3} \sum_{l=1}^{n_k-1} t_{h_{k,l}^{(i)}, h_{k,l+1}^{(i)}}
\tag{5-1}
\end{equation}
式中:$h_{k,l}^{(i)}$——第 $k$ 辆导弹发射装置第 $i$ 阶段机动路径中的第 $l$ 个节点;

$t_{h_{k,l}^{(i)}, h_{k,l+1}^{(i)}}$——第 $k$ 辆导弹发射装置第 $i$ 阶段机动路径中从第 $l$ 个节点到第 $l+1$ 个节点需要的时间。

\subsubsection{各阶段的约束条件}

由于本题的约束条件较为复杂,因此本节按照前文划分的 3 个阶段整理约束条件,最后可以列出的各阶段约束条件如下 [5]:

第 1 阶段:
\begin{equation}
\left\{
\begin{aligned}
S_k^{(1)} &= h_{k,1}^{(1)} \tag{5-2} \\
R_k^{(1)} &= h_{k,n_k-1}^{(1)} \tag{5-3} \\
R_{k_1}^{(1)} &\neq R_{k_2}^{(1)} (k_1 \neq k_2) \tag{5-4} \\
\delta_{h_{k,l}^{(1)}, h_{k,l+1}^{(1)}} &= 1 \tag{5-5} \\
w_k^{(1)} &= \max_{k'} \sum_{l=1}^{n_k'-1} t_{h_{k',l}^{(1)}, h_{k',l+1}^{(1)}} - \sum_{l=1}^{n_k} t_{h_{k,l}^{(1)}, h_{k,l+1}^{(1)}} \tag{5-6} \\
P_{h_{k,l}^{(1)}} &= \sum_{m=1}^{l-1} t_{h_{k,m}^{(1)}, h_{k,m+1}^{(1)}} + n_k^{(i)} \tag{5-7} \\
Q_{h_{k,l}^{(1)}, h_{k,l+1}^{(1)}} &= \left[ P_{h_{k,l}^{(1)}}, P_{h_{k,l+1}^{(1)}} \right] \tag{5-8} \\
Q_{h_{k_1,l}^{(1)}, h_{k_1,l+1}^{(1)}} \cap Q_{h_{k_2,l}^{(1)}, h_{k_2,l+1}^{(1)}} &= \Phi \quad (k_1 \neq k_2, \text{仅考虑单向车道节点}) \tag{5-9}
\end{aligned}
\right.
\end{equation}
式中:$S_k^{(i)}$——第 $k$ 辆导弹发射装置第 $i$ 阶段的出发位置;

$R_k^{(i)}$——第 $k$ 辆导弹发射装置第 $i$ 阶段的停止位置;

$\delta_{h_{k,l}^{(i)}, h_{k,l+1}^{(i)}}$——第 $k$ 辆导弹发射装置第 $i$ 阶段所经路径中第 $l$ 个节点和第 $l+1$ 个节点的连通情况,因为均连通,所以为 1,否则为 0;

$w_k^{(i)}$——第 $k$ 辆导弹发射装置第 $i$ 阶段的出发时刻;

$P_{h_{k,l}^{(i)}}$——第 $k$ 辆导弹发射装置第 $i$ 阶段到达其机动路径第 $l$ 个节点的时刻;

$Q_{h_{k_1,l}^{(1)}, h_{k_1,l+1}^{(1)}}$——第 $k$ 辆导弹发射装置第 $i$ 阶段所经路径中第 $l$ 个节点到第 $l+1$ 个节点的时间区间。

第 1 阶段的各约束条件具体含义如下:

(5-2) 式表示第 $k$ 辆导弹发射装置第 1 阶段的出发位置为其机动路径中的第 1 个节点; (5-3) 式表示第 $k$ 辆导弹发射装置第 1 阶段的停止位置为其机动路径中的第 $n_k$ 个节点; (5-4) 式表示任意两辆导弹发射装置 $(k_1, k_2)$ 第 1 阶段的停止位置不相同, 即导弹发射位置最后到达的发射点位互不冲突; (5-8) 式表示第 $k$ 辆导弹发射装置第 1 阶段所经路径中第 $l$ 个节点到第 $l+1$ 个节点的时间区间; (5-9) 式表示第 1 阶段单向道路的正反向行驶时间区间交集为空集 $\Phi$, 避免在单向车道上会车的情况。

第 2 阶段:

\begin{equation}
\left\{
\begin{aligned}
S_k^{(2)} &= R_k^{(1)} \tag{5-10} \\
R_k^{(2)} &\in \{Z1, \cdots, Z6\} \tag{5-11} \\
\delta_{h_{k,l}^{(2)}, h_{k,l+1}^{(2)}} &= 1 \tag{5-12} \\
w_k^{(2)} &= \max_{k'} \sum_{l=1}^{n_k'} t_{h_{k',l}^{(2)}, h_{k',l+1}^{(2)}} - \sum_{l=1}^{n_k} t_{h_{k,l}^{(2)}, h_{k,l+1}^{(2)}} \neq 0 \tag{5-13} \\
\left\{
\begin{aligned}
R_{k_1^{(2)}} &\neq R_{k_2^{(2)}} \, (k_1 \neq k_2) \\
P_{h_{k_1,n_{k_1}}^{(2)}} - P_{h_{k_2,n_{k_2}}^{(2)}} &\geq 10
\end{aligned}
\right. \quad \text{(转载地域内已有 2 台发射装置时需满足)} \tag{5-14} \\
Q_{h_{k_1,l}^{(2)}, h_{k_1,l+1}^{(2)}} \cap Q_{h_{k_2,l}^{(2)}, h_{k_2,l+1}^{(2)}} &= \Phi \quad (k_1 \neq k_2, \text{仅考虑单向车道节点}) \tag{5-15}
\end{aligned}
\right.
\end{equation}

第 2 阶段的约束条件的具体含义如下:

(5-10) 式表示第 $k$ 辆导弹发射装置第 2 阶段的出发位置为其第 1 阶段的停止位置, 即导弹发射装置转载导弹的起始位置为第一波次的发射点位; (5-11) 式表示第 $k$ 辆导弹发射装置第 2 阶段的停止位置包含于转载地域组成的集合, 即导弹发射装置转载导弹的停止位置为转载地域; (5-12) 式表示第 $k$ 辆导弹发射装置第 2 阶段所经路径中第 $l$ 个节点和第 $l+1$ 个节点互相连通; (5-13) 式表示第 $k$ 辆导弹发射装置第 2 阶段的出发时刻非负, 由于存在第 1 阶段故有此约束; (5-14) 式表示转载地域内已有 2 台发射装置时, 到该转载地域的其余发射装置与已有 2 台发射装置到达该配送中心的时间差不小于 $10 \, \text{min}$; (5-15) 式表示第 2 阶段单向道路的正反向行驶时间区间交集为空集 $\Phi$, 避免在单向车道上会车的情况。

第 3 阶段:

\begin{equation}
\left\{
\begin{aligned}
S_k^{(3)} &= R_k^{(2)} \tag{5-16} \\
R_k^{(3)} &\in \{\text{F1}, \text{F2}, \cdots, \text{F60}\} - \bigcup R_k^{(1)} \tag{5-17} \\
\delta_{h_{k,l}^{(3)}, h_{k,l+1}^{(3)}} &= 1 \tag{5-18} \\
w_k^{(3)} &= \max_{k'} \sum_{l=1}^{n_k'} t_{h_{k',l}^{(3)}, h_{k',l+1}^{(3)}} - \sum_{l=1}^{n_k} t_{h_{k,l}^{(3)}, h_{k,l+1}^{(3)}} \neq 0 \tag{5-19} \\
Q_{h_{k_1,l}^{(3)}, h_{k_1,l+1}^{(3)}} \cap Q_{h_{k_2,l}^{(3)}, h_{k_2,l+1}^{(3)}} &= \Phi \quad (k_1 \neq k_2, \text{仅考虑单向车道节点}) \tag{5-20}
\end{aligned}
\right.
\end{equation}

第 3 阶段的约束条件的具体含义如下:

(5-16) 式表示第 $k$ 辆导弹发射装置第 3 阶段的出发位置为其第 2 阶段的停止位置, 即导弹发射装置第二波次发射的起始位置为转载地域; (5-17) 式表示除第一波次发射

使用的点位外,第 $k$ 辆导弹发射装置第 3 阶段的停止位置包含于其余发射点位组成的集合;(5-18) 式表示第 $k$ 辆导弹发射装置第 3 阶段所经路径中第 $l$ 个节点和第 $l+1$ 个节点互相连通;(5-19) 式表示第 $k$ 辆导弹发射装置第 2 阶段的出发时刻非负,由于存在第 1、2 阶段故有此约束;(5-20) 式表示第 3 阶段单向道路的正反向行驶时间区间交集为空集 $\Phi$,避免在单向车道上会车的情况。

\section{求解方案}

虽然我们成功建立了问题(1)的目标函数和约束条件,但由于约束条件众多且较为复杂,为了便于解决问题,我们从之前划分的 3 个阶段出发,递进式求解 3 个阶段的最短整体暴露时间,即先求解第 1 阶段的最短整体暴露时间,第 2 阶段的最短整体暴露时间建立在第 1 阶段机动方案已经确定的基础上,第 3 阶段的最短整体暴露时间建立在第 1、2 阶段机动方案已经确定的基础上,这样即可求出一个较优的方案(不一定是最优解)。数学形式如下:

\begin{equation}
\begin{aligned}
\min T &= \min \sum_{i=1}^{24} \sum_{i=1}^{3} t_k^{(i)} \\
&\approx \min \sum_{k=1}^{24} t_k^{(1)} + \min \sum_{k=1}^{24} t_k^{(2)} \bigg|_{\min \sum_{k=1}^{24} t_k^{(1)}, \min \sum_{k=1}^{24} t_k^{(2)}} + \min \sum_{k=1}^{24} t_k^{(3)}
\end{aligned}
\tag{5-21}
\end{equation}

基于这一思想,我们设计出如下的求解步骤:

步骤一:先求出第 1 阶段的最短整体暴露时间,并确定第一波次发射点位分配及机动路线;

步骤二:在确定好第一波次具体发射点位分配及机动路线的基础上,求出第 2 阶段的最短整体暴露时间,并确定从第一波次发射点位到转载地域的机动路线;

步骤三:在确定好从第一波次发射点位到转载地域的机动路线的基础上,进一步求出第 3 阶段的最短整体暴露时间,并确定第二波次发射点位分配及机动路线;

步骤四:最后将上述三个阶段最短整体暴露时间对应的方案整合即为最优方案。

\subsection{第 1 阶段最短整体暴露时间}

在第 1 阶段内,所有的导弹发射车都从待机地域出发前往发射点位,要让整体暴露时间最短,那么走的路径相对来说也要最短,求解步骤如下:

第 1 步:求解 2 个待机地域到 60 个发射点位的 24 条最短路径;

第 2 步:把 24 辆导弹发射车分配到这 24 条最短路径上来,分配的目标是提取每种可能的组合中最长的时间,其中最短的一组即为所求的方案;

第 3 步:考虑同一时段内可能产生的路径冲突,对于路径冲突我们要尽量避免,因为齐射时刻已经确定,如果一辆车在道路节点处产生了等待时间,那么在它之前到达发射点位的所有的车都会增加这样一段等待时间,这样就大幅增加整体暴露时间。我们的策略是将两条路径中较长路径的出发时刻提前直至交集为空。

\subsubsection{基于 Floyd 算法求解最短路问题}

求解第 1 阶段最短整体暴露时间的第 1 步是求解 2 个待机地域到 60 个发射点位的 24 条最短路径,这是典型的最短路问题,但值得注意的是导弹发射装置执行任务前平均部署在 2 个待机地域,每个待机地域都有 12 台,因此最后需从每个待机地域到发射点位的最短路径中各寻找 12 条。

(1) 最短路问题构造赋权图 $G = (V, E, W)$。其中,顶点集 $V = \{v_1, \ldots, v_n\}$,$E$ 为边的集合;邻接矩阵 $W = (w_{ij})_{n \times n}$,$w_{ij}$ 表示顶点 $v_i$ 和 $v_j$ 之间直通的距离,若顶点 $v_i$ 和 $v_j$ 之间无路,则 $w_{ij} = \infty$。问题就是求赋权图 $G$ 中指定的两个顶点 $u_0$,$v_0$ 间的具有最小权的路。这条路称为 $u_0$,$v_0$ 间的最短路,它的权称为 $u_0$,$v_0$ 间的距离,亦记为 $d(u_0, v_0)$,求解 $d(u_0, v_0)$ 的问题即为最短路问题。

对于最短路问题,目前已有比较成熟的算法,如 Dijkstra 算法和 Floyd 算法。Dijkstra 算法的基本思路是按距 $u_0$ 从近到远为顺序,依次求得 $u_0$ 到 $G$ 的各顶点的最短路和距离,直至 $v_0$(或直至 $G$ 的所有顶点),算法结束。Dijkstra 算法的流程如下:

第 1 步:令 $l(u_0) = 0$,对 $v \neq u_0$,令 $l(v) = \infty$,$S_0 = u_0$,$i = 0$;

第 2 步:对每个 $v \in \overline{S_i}$($\overline{S_i} = V \setminus S_i$),用 $\min\limits_{u \in S_i} \{l(v), l(u) + w(uv)\}$ 代替 $l(v)$,这里 $w_{uv}$ 表示顶点 $u$ 和 $v$ 之间边的权值。计算 $\min\limits_{u \in S_i} \{l(v)\}$,把达到这个最小值的一个顶点记为 $u_{i+1}$,令 $S_{i+1} = S_i \cup \{u_{i+1}\}$;

第 3 步:若 $i = |V| - 1$,则停止;若 $i < |V| - 1$,则用 $i + 1$ 代替 $i$,转步骤 1。

算法结束时,从 $u_0$ 到各顶点 $v$ 的距离由 $v$ 的最后一次标号 $l(v)$ 给出。在 $v$ 进入 $S_i$ 之前的标号 $l(v)$ 叫 T 标号,$v$ 进入 $S_i$ 时的标号 $l(v)$ 叫 P 标号。算法就是不断修改各顶点的 T 标号,直至获得 P 标号。

然而根据 Dijkstra 算法的流程我们发现 Dijkstra 算法只能求出赋权图顶点到其余顶点的最短路径,对于赋权图中间的两点间的最短路径则无法求解,显然对于问题 1 并不适用。解决这一问题的方法是由 R.W.Floyd 提出的算法,称为 Floyd 算法。Floyd 算法是对 Dijkstra 算法的改进,除可以解决赋权图中任意两点间的最短路径外,还大大减少了算法的复杂度。算法的基本思想是递推产生一个矩阵序列 $A_1, \ldots, A_k, \ldots, A_n$,其中矩阵 $A_k$ 的第 $i$ 行第 $j$ 列元素 $A_k(i, j)$ 表示从顶点 $v_i$ 到顶点 $v_j$ 的路径上所经过的顶点序号不大于 $k$ 的最短路径长度。计算时用迭代公式

\begin{equation}
A_k(i, j) = \min(A_{k-1}(i, j), A_{k-1}(i, k) + A_{k-1}(k, j)),
\tag{5-22}
\end{equation}

$k$ 是迭代次数,$i, j, k = 1, 2, \ldots, n$,当 $k = n$ 时,$A_n$ 即是各顶点之间的最短通路值。

(2) 双向道距离调整

车载发射装置类型分为 A、B、C 三种,每种车型在双向道和单向道上行驶的速度也有较大差别。由于 Floyd 算法只能求解最小路径,而本题的目标是求最短时间,所以不能直接根据原始道路的距离分配车型,此处做以下处理。单向道距离按照原始距离带入计算。由于同种车型在双向道上速度更快,同样的距离花费的时间更短,在不改变速度的假设下,只有将双向道的原始距离乘上折减系数才能近似反映真实的时间花费情况。由题意,A、B、C 三类发射装置在双向道上的平均行驶速度分别为 $70 \, \text{km/h}$、$60 \, \text{km/h}$ 和 $50 \, \text{km/h}$,在单向道路上的平均行驶速度分别为 $45 \, \text{km/h}$、$35 \, \text{km/h}$ 和 $30 \, \text{km/h}$。粗略规定双向道距离折减系数为 $k_A(45/70)$、$k_B(35/60)$、$k_C(30/50)$ 三者的平均值,即

\begin{equation}
k = \frac{k_A + k_B + k_C}{3} = \frac{45/70 + 35/60 + 30/50}{3} = \frac{767}{1260}.
\tag{5-23}
\end{equation}

调整后的距离按如下定义:
\[
a_{ij} =
\begin{cases}
kw_{ij}, & v_i \text{与} v_j \text{连通,主干道} \\
w_{ij}, & v_i \text{与} v_j \text{连通,其他道路} \\
\infty, & v_i \text{与} v_j \text{不连通}
\end{cases}
\]

假设待机地域、发射点位、道路节点共同构成集合 \( V = v_1, \cdots, v_n \),则邻接矩阵为
\[
A =
\begin{bmatrix}
a_{11} & a_{12} & \cdots & a_{1n} \\
a_{21} & a_{22} & \cdots & a_{2n} \\
\vdots & \vdots & \ddots & \vdots \\
a_{n1} & a_{22} & \cdots & a_{nn}
\end{bmatrix}
\]

(3)最短路径求解结果

对不同路径的权值进行调整后用 Floyd 算法计算得到的 24 条最短路径见表 5-1:

\textbf{表 5-1 从待机地域到发射点位的 24 条最短路径}

\begin{tabular}{l l r}
\hline
\textbf{待机地域} & \textbf{最短路径} & \textbf{路径距离 (km)} \\
\hline
D1 & D1 \(\rightarrow\) J11 \(\rightarrow\) J46 \(\rightarrow\) F43 & 52.58 \\
D1 & D1 \(\rightarrow\) Z03 \(\rightarrow\) J61 \(\rightarrow\) F58 & 73.50 \\
D1 & D1 \(\rightarrow\) Z03 \(\rightarrow\) J61 \(\rightarrow\) F57 & 75.82 \\
D1 & D1 \(\rightarrow\) J11 \(\rightarrow\) J46 \(\rightarrow\) J44 \(\rightarrow\) F42 & 77.68 \\
D1 & D1 \(\rightarrow\) J11 \(\rightarrow\) J46 \(\rightarrow\) J44 \(\rightarrow\) F41 & 81.28 \\
D1 & D1 \(\rightarrow\) J09 \(\rightarrow\) J08 \(\rightarrow\) J07 \(\rightarrow\) Z04 \(\rightarrow\) J38 \(\rightarrow\) F34 & 103.96 \\
D1 & D1 \(\rightarrow\) J09 \(\rightarrow\) J08 \(\rightarrow\) J07 \(\rightarrow\) Z04 \(\rightarrow\) J38 \(\rightarrow\) F35 & 105.21 \\
D1 & D1 \(\rightarrow\) J09 \(\rightarrow\) J08 \(\rightarrow\) J07 \(\rightarrow\) Z04 \(\rightarrow\) J37 \(\rightarrow\) F31 & 106.71 \\
D1 & D1 \(\rightarrow\) J09 \(\rightarrow\) J08 \(\rightarrow\) J07 \(\rightarrow\) Z04 \(\rightarrow\) J37 \(\rightarrow\) F32 & 108.44 \\
D1 & D1 \(\rightarrow\) J09 \(\rightarrow\) J08 \(\rightarrow\) J07 \(\rightarrow\) J06 \(\rightarrow\) J36 \(\rightarrow\) F29 & 120.69 \\
D1 & D1 \(\rightarrow\) J09 \(\rightarrow\) J08 \(\rightarrow\) J07 \(\rightarrow\) J06 \(\rightarrow\) J36 \(\rightarrow\) F30 & 120.69 \\
D1 & D1 \(\rightarrow\) J09 \(\rightarrow\) J08 \(\rightarrow\) J07 \(\rightarrow\) Z04 \(\rightarrow\) J37 \(\rightarrow\) F33 & 111.21 \\
D2 & D2 \(\rightarrow\) J32 \(\rightarrow\) F24 & 27.90 \\
D2 & D2 \(\rightarrow\) J32 \(\rightarrow\) J33 \(\rightarrow\) F25 & 45.88 \\
D2 & D2 \(\rightarrow\) J03 \(\rightarrow\) J48 \(\rightarrow\) F47 & 51.45 \\
D2 & D2 \(\rightarrow\) J03 \(\rightarrow\) J48 \(\rightarrow\) F46 & 53.59 \\
D2 & D2 \(\rightarrow\) J03 \(\rightarrow\) J02 \(\rightarrow\) J47 \(\rightarrow\) F44 & 70.25 \\
D2 & D2 \(\rightarrow\) J03 \(\rightarrow\) J02 \(\rightarrow\) J47 \(\rightarrow\) F45 & 71.96 \\
D2 & D2 \(\rightarrow\) J12 \(\rightarrow\) J13 \(\rightarrow\) J21 \(\rightarrow\) F03 & 76.26 \\
D2 & D2 \(\rightarrow\) J12 \(\rightarrow\) J13 \(\rightarrow\) J21 \(\rightarrow\) F02 & 76.32 \\
D2 & D2 \(\rightarrow\) J12 \(\rightarrow\) J13 \(\rightarrow\) J21 \(\rightarrow\) F01 & 76.38 \\
D2 & D2 \(\rightarrow\) J32 \(\rightarrow\) J33 \(\rightarrow\) J34 \(\rightarrow\) F26 & 68.04 \\
D2 & D2 \(\rightarrow\) J03 \(\rightarrow\) J04 \(\rightarrow\) J05 \(\rightarrow\) J49 \(\rightarrow\) F49 & 89.13 \\
D2 & D2 \(\rightarrow\) J03 \(\rightarrow\) J04 \(\rightarrow\) J05 \(\rightarrow\) J49 \(\rightarrow\) F48 & 89.48 \\
\hline
\end{tabular}

\subsection*{5.3.2 基于指派问题的路线分配方案求解}

在求解完 24 条最短路径之后我们需要分析从每个待机地域出发的 12 辆导弹发射装置分配到 12 条路线的方案,提取每一种方案需要的最长的运输时间,取其中最长的一条作为齐射时刻,相应的分配方案即为所求,用齐射时刻减去每辆导弹发射装

置在路上的时间得到每辆车的出发时间。

为了求解的简便,我们先求解从待机区域 D1 出发的 12 辆导弹发射装置分配到 12 条路线的方案,对于待机区域 D2 按相同的方法求解即可。因为待机区域 D1 内的发射装置分为 A、B、C 三类,数量分别为 3 台 A、3 台 B、6 台 C,其中同一类导弹发射装置地位平等,因此我们研究将 3 类导弹发射装置分配到 12 条路线上的分配方案。

很明显这是一个指派问题,可以用 0-1 整数规划求解。我们需要将 12 辆导弹发射装置分配到 12 条路线上去,使得每一种方案需要的最长运输时间最短。

引入 0-1 变量和指派问题的系数矩阵 \( T = (t_{ij})_{m \times n} \),其中 \( t_{ij} \) 为第 \( i \) 类车走第 \( j \) 条路的用时。

\[
x_{ij} =
\begin{cases}
1, \text{第 } i \text{ 类车走第 } j \text{ 条路} \\
0, \text{第 } i \text{ 类车不走第 } j \text{ 条路}
\end{cases}
\quad (i = 1, 2, \cdots, 12; j = 1, 2, \cdots, 12)
\tag{5-24}
\]

本题指派问题的数学模型为

\[
\min \left( \max \sum_{i=1}^{12} \sum_{j=1}^{12} t_{ij} x_{ij} \right);
\tag{5-25}
\]

\[
\text{s.t. } \left\{
\begin{aligned}
\sum_{i=1}^{12} x_{i,j} &= 1 \\
\sum_{j=1}^{12} x_{i,j} &= 1 \\
x_{ij} &= 0 \text{ 或 } 1, (i = 1, 2, \cdots, 12; j = 1, 2, \cdots, 12)
\end{aligned}
\right.
\tag{5-26}
\tag{5-27}
\tag{5-28}
\]

至此我们确定齐射时刻和出发时间的模型就已经建立完成了。

用 Lingo 软件分别对从待机区域 D1、D2 出发的 12 辆导弹发射装置分配到 12 条路线的方案求解,比较两次计算的结果,以其中时间较长者为齐射时间,该时间减去对应方案下各导弹发射装置的运输时间,初步得到各导弹发射装置待机地域、机动路线以及对应的的时刻,计算结果见表 5-2:

\begin{table}
\centering
\caption{24辆导弹发射装置第1阶段的机动路线}
\begin{tabular}{lllll}
\hline
待机地域 & 车号 & 路径 & 出发时刻 & 齐射时刻1 \\
\hline
D1 & A01 & D1 $\rightarrow$ J09 $\rightarrow$ J08 $\rightarrow$ J07 $\rightarrow$ J06 $\rightarrow$ J36 $\rightarrow$ F30 & 47.9 & 178.4 \\
D1 & A02 & D1 $\rightarrow$ J09 $\rightarrow$ J08 $\rightarrow$ J07 $\rightarrow$ Z04 $\rightarrow$ J37 $\rightarrow$ F32 & 51.4 & 178.4 \\
D1 & A03 & D1 $\rightarrow$ J09 $\rightarrow$ J08 $\rightarrow$ J07 $\rightarrow$ Z04 $\rightarrow$ J37 $\rightarrow$ F33 & 47.7 & 178.4 \\
D2 & \textcolor{red}{A04} & D2 $\rightarrow$ J03 $\rightarrow$ J04 $\rightarrow$ J05 $\rightarrow$ J49 $\rightarrow$ F49 & 81.4(71.9) & 178.4 \\
D2 & \textcolor{red}{A05} & D2 $\rightarrow$ J03 $\rightarrow$ J04 $\rightarrow$ J05 $\rightarrow$ J49 $\rightarrow$ F48 & 80.9(71.9) & 178.4 \\
D2 & A06 & D2 $\rightarrow$ J12 $\rightarrow$ J13 $\rightarrow$ J21 $\rightarrow$ F02 & 87.8 & 178.4 \\
D1 & B01 & D1 $\rightarrow$ J09 $\rightarrow$ J08 $\rightarrow$ J07 $\rightarrow$ J06 $\rightarrow$ J36 $\rightarrow$ F29 & 17.1 & 178.4 \\
D1 & B02 & D1 $\rightarrow$ J09 $\rightarrow$ J08 $\rightarrow$ J07 $\rightarrow$ Z04 $\rightarrow$ J38 $\rightarrow$ F35 & 24.4 & 178.4 \\
D1 & B03 & D1 $\rightarrow$ J09 $\rightarrow$ J08 $\rightarrow$ J07 $\rightarrow$ Z04 $\rightarrow$ J37 $\rightarrow$ F31 & 21.8 & 178.4 \\
D2 & B04 & D2 $\rightarrow$ J32 $\rightarrow$ J33 $\rightarrow$ J34 $\rightarrow$ F26 & 61.8 & 178.4 \\
D2 & B05 & D2 $\rightarrow$ J12 $\rightarrow$ J13 $\rightarrow$ J21 $\rightarrow$ F03 & 64.4 & 178.4 \\
D2 & B06 & D2 $\rightarrow$ J12 $\rightarrow$ J13 $\rightarrow$ J21 $\rightarrow$ F01 & 64.2 & 178.4 \\
D1 & C01 & D1 $\rightarrow$ Z03 $\rightarrow$ J61 $\rightarrow$ F58 & 31.4 & 178.4 \\
D1 & C02 & D1 $\rightarrow$ Z03 $\rightarrow$ J61 $\rightarrow$ F57 & 26.8 & 178.4 \\
D1 & C03 & D1 $\rightarrow$ J09 $\rightarrow$ J08 $\rightarrow$ J07 $\rightarrow$ Z04 $\rightarrow$ J38 $\rightarrow$ F34 & 0 & 178.4 \\
D1 & C04 & D1 $\rightarrow$ J11 $\rightarrow$ J46 $\rightarrow$ F43 & 73.3 & 178.4 \\
D1 & C05 & D1 $\rightarrow$ J11 $\rightarrow$ J46 $\rightarrow$ J44 $\rightarrow$ F42 & 23.1 & 178.4 \\
D1 & C06 & D1 $\rightarrow$ J11 $\rightarrow$ J46 $\rightarrow$ J44 $\rightarrow$ F41 & 15.9 & 178.4 \\
D2 & C07 & D2 $\rightarrow$ J32 $\rightarrow$ F24 & 122.6 & 178.4 \\
D2 & C08 & D2 $\rightarrow$ J32 $\rightarrow$ J33 $\rightarrow$ F25 & 86.7 & 178.4 \\
D2 & \textcolor{red}{C09} & D2 $\rightarrow$ J03 $\rightarrow$ J48 $\rightarrow$ F47 & 75.5(75.5) & 178.4 \\
D2 & \textcolor{red}{C10} & D2 $\rightarrow$ J03 $\rightarrow$ J48 $\rightarrow$ F46 & 71.3(71.3) & 178.4 \\
D2 & \textcolor{red}{C11} & D2 $\rightarrow$ J03 $\rightarrow$ J02 $\rightarrow$ J47 $\rightarrow$ F44 & 61.2(61.2) & 178.4 \\
D2 & C12 & D2 $\rightarrow$ J03 $\rightarrow$ J02 $\rightarrow$ J47 $\rightarrow$ F45 & 57.8 & 178.4 \\
\hline
\end{tabular}
\end{table}

注:单位 min。

从表 5-2 可以看出从待机地域 D2 内的导弹发射装置的出发时刻普遍较晚,结合地形图我们不难看出这是因为 D2 离其周围的发射点位距离较近的原因,所需路上运输时间较短,所以先在待机地域内等待以缩短整体暴露时间。

\subsection*{5.3.3 同一时段内路径冲突的解决方案}

根据我们得到的发射点位的分配和机动路线方案,我们可以求出任意导弹发射装置到达其路径上每一节点的时刻,对这些时刻进行分析,我们发现对于第 1 阶段而言,不同导弹发射车在同一时段内的路径会可能会冲突。冲突的车号已在表 5-2 上用红色标出。为了解决这一问题,我们采取的策略是尽量避免冲突,将两条路径中较长路径的出发时刻提前直至无冲突。表 5-2 中红色括号内的时刻即为协调好的出发时刻,第 1 阶段整体暴露时间最短的发射点位分配和机动路线就都已经确定下来。

\subsection*{5.4 第 2 阶段最短整体暴露时间}

在第 2 阶段内,所有的导弹从第一波次的发射地域出发前往转载地域,求解第 2 阶段的最短整体暴露时间的思路如下:

第 1 步:求解第一波次所选的 24 个发射点位到 6 个转载地域的最短路径并取每个发射点位到 6 个转载地域最短路径中最短的一条前往,这样就确定好了转载地域;

第 2 步:将第 1 阶段的齐射时刻作为现在的出发时刻,计算各路径中经过各节点的时刻和到达时刻;

\section*{第 3 步:由于有的转载地域周围的发射点位比较集中,所以可能出现在转载地域外排队的情况,这时我们需要比较继续等待以及赶往较近的转载地域的时间,取其中时间较短者进行下一步动作;}

第 4 步:尽量避免同一时间段内路径的冲突,但由于此时出发时刻为齐射时刻,无法提前,所以我们的策略是较短路径的出发时刻延后至没有冲突。

\subsection*{5.4.1 第一波次发射点位到转载地域的最短路径}

根据 5.3.1 中的 Floyd 算法很容易求得第一波次所选的 24 个发射点位到转载地域的最短路径,将每个发射点位分配到最近的转载地域使其前往,24 个发射点位与转载地域的对应关系见表 5-3,分配方案在图 5-1 中标示出。

\begin{table}[h]
\centering
\caption{发射点位与其前往的转载地域的对应关系}
\begin{tabular}{l l r}
\hline
\textbf{转载地域} & \textbf{前往该转载地域的发射车对应发射点位} & \textbf{发射装置总数} \\
\hline
Z01 & F24, F25, F47, F46, F44, F45, F26, F49, F48 & 9 \\
Z02 & F29, F30 & 2 \\
Z03 & F58, F57 & 2 \\
Z04 & F03, F02, F01, F34, F35, F31, F32, F33 & 8 \\
Z05 & F43, F42, F41 & 3 \\
Z06 & 无 & 0 \\
\hline
\end{tabular}
\end{table}

\begin{figure}[h]
\centering
\includegraphics[width=\textwidth]{image.png}
\caption{根据最短路径确定的机动路线}
\end{figure}

从图 5-1 中我们可以看出,部分转载地域(Z01、Z04)出现了集群现象,由于其周围发射点位较多,所以若多辆导弹发射装置大批集中到该转载地域,则将在转载地域外排队,将大大增加整体暴露时间,为了解决这一问题,这时就涉及到转载地域的重分配问题。

\subsection*{5.4.2 转载地域的重分配}

对于多辆导弹发射装置在转载地域外的集群现象,我们决定对每个转载地域进行一次重分配,具体的分配方法如下:

步骤 1:首先对每个转载地域的导弹发射装置按到达时刻从早到晚排列;

步骤 2:根据题目中给的“转载地域最多容纳 2 台发射装置,但不能同时作业,单台转载作业需时 10 分钟”的条件,结合到达时刻判断是否会出现导弹发射装置在某转载地域 $Z(i)$ 外排队的情况;

步骤 3:若某导弹发射装置 $F(k)$ 出现了在距其最近的转载地域 $Z(i)$ 外排队的情况,则进一步判断是否需要更换转载地域,判断时需要比较 $F(k)$ 前往 $Z(i)$ 并排队直至进入的时间与 $F(k)$ 前往路径第二短的转载地域 $Z(j)$ 并排队进入的时间;若前者所用时间较短,则前往 $Z(i)$ 并在转载区域外等待;若后者所用时间较短,则前往 $Z(j)$ 并在转载区域外等待;

步骤 4:更新转载地域的分配方案后考虑需要排队的导弹发射装置 $F(l)$ 重复步骤 3,直至遍历所有需要等待的导弹发射装置。

为了保证整体暴露时间最短,使用上述算法对转载地域进行重分配,用 MATLAB 软件编程计算得到重分配后的机动路线方案,计算结果如表 5-4 所示:

\begin{table}[h]
\centering
\caption{发射点位与其前往的转载地域的对应关系(重分配)}
\begin{tabular}{l l r}
\hline
转载地域 & 前往该转载地域的发射车对应发射点位 & 发射装置总数 \\
\hline
Z01 & F24, F25, F47, F46, F44, F45, F49, F48 & 8 \\
Z02 & F29, F30, F26 & 3 \\
Z03 & F58, F57 & 2 \\
Z04 & F34, F35, F31, F32, F33 & 5 \\
Z05 & F43, F42, F41 & 3 \\
Z06 & F03, F02, F01 & 3 \\
\hline
\end{tabular}
\end{table}

我们可以看到,根据上面的重分配算法,更改了 F01、F02、F03、F26 分配的转载地域以使整体暴露时间最短,重分配转载区域后的机动路线方案图 5-2 所示。

\subsection*{5.4.3 同一时段内路径冲突的解决方案}

若存在同一时段内的路径冲突,我们初步的解决方案是尽量避免同一时间段内路径的冲突,但由于此时出发时刻为齐射时刻,无法提前,所以我们将较短路径的出发时刻延后至没有冲突。

经过验算,表中的机动路径之间没有冲突,即按照我们的方案,导弹发射装置在赶往转载地域的过程中不会发生同一时段内路径冲突的情形,使第 2 阶段整体暴露时间最短的机动路线方案见表 5-5。

\begin{figure}[h]
    \centering
    \includegraphics[width=\textwidth]{image.png}
    \caption{重分配转载区域后的机动路线}
    \label{fig:5-2}
\end{figure}

\begin{table}[h]
    \centering
    \caption{24辆导弹发射装置第2阶段的机动路线}
    \label{tab:5-5}
    \begin{tabular}{l l l l l}
        待机地域 & 车号 & 路径 & 出发时刻 & 到达时刻 \\
        \hline
        D1 & A01 & F30 $\rightarrow$ J36 $\rightarrow$ J06 $\rightarrow$ J51 $\rightarrow$ Z02 & 178.4 & 237.6 \\
        D1 & A02 & F32 $\rightarrow$ J37 $\rightarrow$ Z04 & 178.4 & 209.2 \\
        D1 & A03 & F33 $\rightarrow$ J37 $\rightarrow$ Z04 & 178.4 & 212.9 \\
        D2 & A04 & F49 $\rightarrow$ J49 $\rightarrow$ J50 $\rightarrow$ Z01 & 178.4 & 252.1 \\
        D2 & A05 & F48 $\rightarrow$ J49 $\rightarrow$ J50 $\rightarrow$ Z01 & 178.4 & 252.6 \\
        D2 & A06 & F02 $\rightarrow$ J21 $\rightarrow$ J14 $\rightarrow$ J15 $\rightarrow$ J16 $\rightarrow$ Z06 & 178.4 & 302.8 \\
        D1 & B01 & F29 $\rightarrow$ J36 $\rightarrow$ J06 $\rightarrow$ J51 $\rightarrow$ Z02 & 178.4 & 254.5 \\
        D1 & B02 & F35 $\rightarrow$ J58 $\rightarrow$ Z04 & 178.4 & 212.5 \\
        D1 & B03 & F31 $\rightarrow$ J37 $\rightarrow$ Z04 & 178.4 & 218.3 \\
        D2 & B04 & F26 $\rightarrow$ J34 $\rightarrow$ J05 $\rightarrow$ J06 $\rightarrow$ J51 $\rightarrow$ Z02 & 178.4 & 291.3 \\
        D2 & B05 & F03 $\rightarrow$ J21 $\rightarrow$ J14 $\rightarrow$ J15 $\rightarrow$ J16 $\rightarrow$ Z06 & 178.4 & 331.8 \\
        D2 & B06 & F01 $\rightarrow$ J21 $\rightarrow$ J14 $\rightarrow$ J15 $\rightarrow$ J16 $\rightarrow$ Z06 & 178.4 & 332.0 \\
        D1 & C01 & F58 $\rightarrow$ J61 $\rightarrow$ Z03 & 178.4 & 251.0 \\
        D1 & C02 & F57 $\rightarrow$ J61 $\rightarrow$ Z03 & 178.4 & 255.7 \\
        D1 & C03 & F34 $\rightarrow$ J38 $\rightarrow$ Z04 & 178.4 & 215.6 \\
        D1 & C04 & F43 $\rightarrow$ J46 $\rightarrow$ J44 $\rightarrow$ Z05 & 178.4 & 273.4 \\
        D1 & C05 & F42 $\rightarrow$ J44 $\rightarrow$ Z05 & 178.4 & 225.2 \\
        D1 & C06 & F41 $\rightarrow$ J44 $\rightarrow$ Z05 & 178.4 & 232.4 \\
        D2 & C07 & F24 $\rightarrow$ J32 $\rightarrow$ J04 $\rightarrow$ Z01 & 178.4 & 308.2 \\
        D2 & C08 & F25 $\rightarrow$ J33 $\rightarrow$ J04 $\rightarrow$ Z01 & 178.4 & 276.1 \\
        D2 & C09 & F47 $\rightarrow$ J48 $\rightarrow$ Z01 & 178.4 & 249.3 \\
        D2 & C10 & F46 $\rightarrow$ J48 $\rightarrow$ Z01 & 178.4 & 253.6 \\
        D2 & C11 & F44 $\rightarrow$ J47 $\rightarrow$ J48 $\rightarrow$ Z01 & 178.4 & 279.2 \\
        D2 & C12 & F45 $\rightarrow$ J47 $\rightarrow$ J48 $\rightarrow$ Z01 & 178.4 & 282.6 \\
        \hline
    \end{tabular}
\end{table}

注:单位 min。

\section{第 3 阶段最短整体暴露时间}

在第 3 阶段内,所有的导弹都从转载地域出发前往第二波次发射点位,6 个转载地域分配的导弹发射装置数量 \( m(i) \) 已经确定,首先想到的思路是用 Floyd 算法求解各转载地域到剩下的 36 个发射点位的 \( m(i) \) 条最短路径,但是这种方法存在的明显问题是不同最短路径的终点有可能冲突,故其不可行,考虑到 1 个发射点位只可以使用 1 次,所以可以通过设置决策变量用 0-1 规划求解,系数矩阵通过 Floyd 算法计算得到。确定好路径后再考虑路径冲突问题,与前面的解答类似,对于路径的冲突我们要尽量避免,避免策略是将两条路径中较短路径的出发时刻延后直至交集为空。

\subsection{基于 Floyd 算法和 0-1 规划指派问题的路径分配}

我们要研究的问题是将 24 辆导弹发射装置分配到 36 个发射点位使第 3 阶段整体暴露时间最短的分配方案,由于前 2 个方案已经确定,假设所有的导弹发射装置在转载区域内装完导弹后即刻出发(即不考虑导弹发射装置躲在装载区域的情形,这一假设不影响本小节最优分配方案的求解,5.5.2 有详细说明),那么使第 3 阶段整体暴露时间最短,只要使最晚到达发射点位的导弹发射装置的时刻最小即可。由于导弹发射装置均位于转载地域,所以同导弹发射装置到剩下的 36 个发射点位的最短距离可以通过 Floyd 算法计算得到。

很明显这是一个指派问题,我们需要将 24 辆导弹发射装置分配到 36 个发射点位,使最晚到达发射点位的导弹发射装置的时刻最小。

将 24 辆导弹发射装置编号,引入 0-1 变量和指派问题的系数矩阵 \( T = (t_{ij})_{24 \times 36} \),其中 \( t_{ij} \) 为第 \( i \) 辆车到第 \( j \) 个发射点位的用时,记第 \( i \) 辆车从转载地域出发的时刻为 \( t_{i0} \)。

\[
x_{ij} =
\begin{cases}
1, \text{第 } i \text{ 辆车到第 } j \text{ 个发射点位} & (i = 1, 2, \cdots, 24; j = 1, 2, \cdots, 36) \\
0, \text{第 } i \text{ 辆车不到第 } j \text{ 个发射点位}
\end{cases}
\tag{5-29}
\]

本题指派问题的数学模型为

\[
\min \left\{ \max \left( \sum_{i=1}^{24} \sum_{j=1}^{36} t_{ij} x_{ij} + t_{i0} \right) \right\};
\tag{5-30}
\]

\[
\text{s.t.}
\begin{cases}
\sum_{i=1}^{24} x_{ij} \leq 1 \\
\sum_{j=1}^{36} x_{ij} = 1 \\
x_{ij} = 0 \text{ 或 } 1, (i = 1, 2, \cdots, 24; j = 1, 2, \cdots, 36)
\end{cases}
\tag{5-31}
\tag{5-32}
\tag{5-33}
\]

借助 Lingo 软件求解上述数学模型,最后确定的第 3 阶段的机动路线方案见表 5-6。

\begin{table}
\centering
\caption{24辆导弹发射装置第3阶段的机动路线}
\begin{tabular}{lllll}
\hline
待机地域 & 车号 & 路径 & 出发时刻 & 齐射时刻2 \\
\hline
D1 & A01 & $Z02 \rightarrow J52 \rightarrow J08 \rightarrow J42 \rightarrow J40 \rightarrow F37$ & 247.6 & 451.7 \\
D1 & A02 & $Z04 \rightarrow J37 \rightarrow J15 \rightarrow J25 \rightarrow J23 \rightarrow F07$ & 269.2 & 451.7 \\
D1 & A03 & $Z04 \rightarrow J38 \rightarrow J42 \rightarrow J40 \rightarrow F39$ & 229.2 & 451.7 \\
D2 & A04 & $Z01 \rightarrow J50 \rightarrow J53 \rightarrow J56 \rightarrow J60 \rightarrow F56$ & 262.1 & 451.7 \\
D2 & A05 & $Z01 \rightarrow J50 \rightarrow J53 \rightarrow J56 \rightarrow J60 \rightarrow F55$ & 272.1 & 451.7 \\
D2 & A06 & $Z06 \rightarrow J28 \rightarrow F19$ & 312.8 & 451.7 \\
D1 & B01 & $Z02 \rightarrow J52 \rightarrow J08 \rightarrow J42 \rightarrow J40 \rightarrow F38$ & 277.9 & 451.7 \\
D1 & B02 & $Z04 \rightarrow J37 \rightarrow J15 \rightarrow J16 \rightarrow J39 \rightarrow F36$ & 239.2 & 451.7 \\
D1 & B03 & $Z04 \rightarrow J37 \rightarrow J15 \rightarrow J25 \rightarrow F10$ & 335.5 & 451.7 \\
D2 & B04 & $Z02 \rightarrow J51 \rightarrow J06 \rightarrow J05 \rightarrow J34 \rightarrow J35 \rightarrow F28$ & 314.1 & 451.7 \\
D2 & B05 & $Z06 \rightarrow J26 \rightarrow F13$ & 392.0 & 451.7 \\
D2 & B06 & $Z06 \rightarrow J26 \rightarrow F12$ & 396.8 & 451.7 \\
D1 & C01 & $Z03 \rightarrow J57 \rightarrow J58 \rightarrow J59 \rightarrow J62 \rightarrow F59$ & 265.8 & 451.7 \\
D1 & C02 & $Z03 \rightarrow J57 \rightarrow J58 \rightarrow J59 \rightarrow J62 \rightarrow F60$ & 271.0 & 451.7 \\
D1 & C03 & $Z04 \rightarrow J37 \rightarrow J75 \rightarrow J25 \rightarrow F11$ & 320.1 & 451.7 \\
D1 & C04 & $Z05 \rightarrow J41 \rightarrow F40$ & 412.7 & 451.7 \\
D1 & C05 & $Z05 \rightarrow J41 \rightarrow J18 \rightarrow J19 \rightarrow J31 \rightarrow F23$ & 235.2 & 451.7 \\
D1 & C06 & $Z05 \rightarrow J41 \rightarrow J18 \rightarrow J29 \rightarrow F20$ & 353.1 & 451.7 \\
D2 & C07 & $Z01 \rightarrow J50 \rightarrow F50$ & 388.8 & 451.7 \\
D2 & C08 & $Z01 \rightarrow J04 \rightarrow J05 \rightarrow J34 \rightarrow J35 \rightarrow F27$ & 302.1 & 451.7 \\
D2 & C09 & $Z01 \rightarrow J50 \rightarrow J53 \rightarrow J56 \rightarrow F53$ & 282.1 & 451.7 \\
D2 & C10 & $Z01 \rightarrow J50 \rightarrow J53 \rightarrow J56 \rightarrow F52$ & 292.1 & 451.7 \\
D2 & C11 & $Z01 \rightarrow J50 \rightarrow J53 \rightarrow J59 \rightarrow F54$ & 312.1 & 451.7 \\
D2 & C12 & $Z01 \rightarrow J50 \rightarrow J53 \rightarrow F51$ & 356.3 & 451.7 \\
\hline
\end{tabular}
\end{table}

注:单位 min。

\subsection{5.5.2 转载后车辆延迟出发的现象}

上面在用 0-1 规划求解时假设所有的导弹发射装置在转载区域内装完导弹后即刻出发,没有考虑到转载地域的最后两辆车可以躲在转载地域的问题,但这不影响用 0-1 规划求出最后一辆导弹发射装置就位的时间,但是在计算整体暴露时间以及从转载区域出发的时刻时则需要充分考虑该因素。

\subsection{5.5.3 同一时段内路径冲突的解决方案}

类似于以上的解答,对于同一时段内的路径冲突,则我们预计的解决方案是尽量避免同一时间段内路径的冲突,但由于此时出发时刻已经固定,所以我们将较短路径的出发时刻延后至没有冲突。经过验算,表中的机动路径之间没有有冲突。

\subsection{5.6 整体暴露时间最短的方案}

将上述 3 个阶段最终得出的方案结合得到使整体暴露时间最短的每个波次打击的发射点位及机动路线,由于篇幅有限,表 5-7 给出的机动路径略去了道路节点,仅给出了每辆导弹发射装置转载地域和两个波次的导弹发射点位,每辆导弹发射装置的出发时刻及第二波次齐射时刻也在表 5-7 中给出,计算得两个波次的整体暴露时间为 8598.8 min,其余内容见 E10294008.Xls。

\begin{table}
\centering
\caption{24辆导弹发射装置全过程的机动路线}
\begin{tabular}{c c l c c}
\hline
待机地域 & 车号 & 路径(仅列出发射点位与转载地域) & 出发时刻 & 齐射时刻2 \\
\hline
D1 & A01 & D1 $\rightarrow$ F30 $\rightarrow$ Z02 $\rightarrow$ F37 & 47.9 & 451.7 \\
D1 & A02 & D1 $\rightarrow$ F32 $\rightarrow$ Z04 $\rightarrow$ F07 & 51.4 & 451.7 \\
D1 & A03 & D1 $\rightarrow$ F33 $\rightarrow$ Z04 $\rightarrow$ F39 & 47.7 & 451.7 \\
D2 & A04 & D2 $\rightarrow$ F49 $\rightarrow$ Z01 $\rightarrow$ F56 & 71.9 & 451.7 \\
D2 & A05 & D2 $\rightarrow$ F48 $\rightarrow$ Z01 $\rightarrow$ F55 & 71.9 & 451.7 \\
D2 & A06 & D2 $\rightarrow$ F02 $\rightarrow$ Z06 $\rightarrow$ F19 & 87.8 & 451.7 \\
D1 & B01 & D1 $\rightarrow$ F29 $\rightarrow$ Z02 $\rightarrow$ F38 & 17.1 & 451.7 \\
D1 & B02 & D1 $\rightarrow$ F35 $\rightarrow$ Z04 $\rightarrow$ F36 & 24.4 & 451.7 \\
D1 & B03 & D1 $\rightarrow$ F31 $\rightarrow$ Z04 $\rightarrow$ F10 & 21.8 & 451.7 \\
D2 & B04 & D2 $\rightarrow$ F26 $\rightarrow$ Z02 $\rightarrow$ F28 & 61.8 & 451.7 \\
D2 & B05 & D2 $\rightarrow$ F03 $\rightarrow$ Z06 $\rightarrow$ F13 & 64.4 & 451.7 \\
D2 & B06 & D2 $\rightarrow$ F01 $\rightarrow$ Z06 $\rightarrow$ F12 & 64.2 & 451.7 \\
D1 & C01 & D1 $\rightarrow$ F58 $\rightarrow$ Z03 $\rightarrow$ F59 & 31.4 & 451.7 \\
D1 & C02 & D1 $\rightarrow$ F57 $\rightarrow$ Z03 $\rightarrow$ F60 & 26.8 & 451.7 \\
D1 & C03 & D1 $\rightarrow$ F34 $\rightarrow$ Z04 $\rightarrow$ F11 & 0 & 451.7 \\
D1 & C04 & D1 $\rightarrow$ F43 $\rightarrow$ Z05 $\rightarrow$ F40 & 73.3 & 451.7 \\
D1 & C05 & D1 $\rightarrow$ F42 $\rightarrow$ Z05 $\rightarrow$ F23 & 23.1 & 451.7 \\
D1 & C06 & D1 $\rightarrow$ F41 $\rightarrow$ Z05 $\rightarrow$ F20 & 15.9 & 451.7 \\
D2 & C07 & D2 $\rightarrow$ F24 $\rightarrow$ Z01 $\rightarrow$ F50 & 122.6 & 451.7 \\
D2 & C08 & D2 $\rightarrow$ F25 $\rightarrow$ Z01 $\rightarrow$ F27 & 86.7 & 451.7 \\
D2 & C09 & D2 $\rightarrow$ F47 $\rightarrow$ Z01 $\rightarrow$ F53 & 75.5 & 451.7 \\
D2 & C10 & D2 $\rightarrow$ F46 $\rightarrow$ Z01 $\rightarrow$ F52 & 71.3 & 451.7 \\
D2 & C11 & D2 $\rightarrow$ F44 $\rightarrow$ Z01 $\rightarrow$ F54 & 61.2 & 451.7 \\
D2 & C12 & D2 $\rightarrow$ F45 $\rightarrow$ Z01 $\rightarrow$ F51 & 57.8 & 451.7 \\
\hline
\end{tabular}
\end{table}

注:单位 min。

\section{问题2模型的建立与求解}

\subsection{问题2模型的建立}

问题2与问题1的区别在于增加了2个转载节点,可供选择的道路节点有5个,所以理论上来讲有 $C_{5}^{2}=10$ 种方案可供选择,问题2即求解这10种方案中使得完成两个波次发射任务的整体暴露时间最短的方案,这与问题1的目标是一致的,因此问题2可以使用与问题1一样的模型,对10种方案按问题1的方法分别求解整体暴露时间并选出其中的最小值作为问题2的解答。

因为可使用与问题1一样的模型,所以问题2也是一个单目标优化模型并将两个彼此的发射任务划分成3个阶段,目标函数为

\begin{equation}
T=\sum_{k=1}^{24} \sum_{i=1}^{3} \sum_{l=1}^{n_{k}-1} t_{h_{k,l}^{(i)}, h_{k,l+1}^{(i)}}
\tag{6-1}
\end{equation}

各阶段的约束条件见公式(5-2)~(5-20)。

\subsection{问题2的求解}

问题2的求解理论上来讲需要求解10种方案中使得完成两个波次发射任务的整体暴露时间最短的方案,但是一方面求解一次问题1的工作量就十分巨大,另一方面方案2中的每种方案都增加了2个转载地域,这就大大提高了工作量,根据2.2对问题2的分析,我们在增加转载地域的时候可以综合考虑如下因素:

(1) 尽可能减轻问题1中发射点位密布的转载地域的压力;

(2) 尽量靠近主干道,这样能有效避免路径冲突的问题;

(3) 尽可能将增加的转载地域选在可用发射点位较多的区域,方便第二波次导弹发射点位的选取并减轻减轻交通压力。

为了简化对问题2的求解,我们没有直接求解增加2个转载地域的方案,而是先评价增加单个转载地域的效果,组合效果较为显著的2个转载地域,比较可能的方案后再得出最优方案并进行详细的计算论证。

\subsubsection{增加单个转载地域的效果}

先增加单个转载地域,从J25、J34、J36、J42、J49这5个备选项分别增加,利用问题1中的优化模型求解,求解的时候做了合理的简化(总体暴露时间只包括发射装置路上行驶时间以及在转载点外的排队等待时间,未考虑由于单行道拥堵及第二次齐射造成的等待等待暴露时间,实际上这两部分均可以通过错开各发射装置从转载点的出发时间而合理避免或减少)。

增加单个转载地域的计算结果如下表所示,由于第1阶段没有变化,所以表6-1仅计算增加转载地域后第2、3阶段的整体暴露时间。

\begin{table}[h]
\centering
\caption{增加单个转载地域在不同阶段的整体暴露时间}
\begin{tabular}{c c c c c c}
\hline
\diagbox{阶段}{转载地域} & J25 & J34 & J36 & J42 & J49 \\
\hline
第2阶段 & 39.1 & 36.6 & 38.5 & 41.6 & 36.2 \\
第3阶段 & 47.1 & 56.6 & 53.4 & 55.1 & 52.9 \\
第2、3阶段 & 86.2 & 93.2 & 91.9 & 96.7 & 89.1 \\
\hline
\end{tabular}
\end{table}

注:单位为h

从表6-1中可以看出,增加单个转载地域的效果顺序为J25<J49<J36<J34<J42,下面结合作战区域道路示意图分析每个转载地域的优劣势:

J25:虽然对于Z01、Z04的分流效果不明显,但周边有较多未使用发射点,有利于转载后快速到达第二波次发射点并起到一定分流作用;

J49:分走了Z01、Z04的大部分流量,但反而不利于平衡车流,且周边第二波次发射点距离较远;

J36:对分流影响不大,但在转载后选择第二波次发射点时,和Z01位置的发射装置的冲突较大;

J34:对分流、和扩大第二波次发射点选择范围都影响不大;

J42:分走了Z01部分流量,但考虑到可以把Z04的F01、F02、F03点绕路到Z06,大大改善了车流平衡度,但周边未使用发射点相对较少。

根据增设单个转载点第2、3阶段效果评价,为了给转载点Z01和Z04分流缓解压力、扩大第二次发射点的选择范围以及考虑到靠近主干道的便利交通,最终决定选择效果最好的J25和J49作为增设的临时转载地域,下面对这一方案进行详细的计算,

\begin{figure}[h]
    \centering
    \includegraphics[width=\textwidth]{image.png}
    \caption{作战区域道路示意图}
    \label{fig:6-1}
\end{figure}

研究其效果。

\subsection{增加两个转载地域的效果}

根据 6.2.1 的分析,最后我们确定了增加 J25 和 J49 作为临时转载地域的方案,为了定量分析这一方案对于减少整体暴露时间的效果,我们对增设转载地域前后的整体暴露时间做了对比,对比结果见表 \ref{tab:6-2}:

\begin{table}[h]
    \centering
    \caption{增加两个转载地域在不同阶段的整体暴露时间}
    \label{tab:6-2}
    \begin{tabular}{c|c c c c}
        \hline
        \diagbox{对比}{阶段} & 第 1 阶段 & 第 2 阶段 & 第 3 阶段 & 合计 \\
        \hline
        增设前 & 50.1 & 34.0(1.9) & 59.2 & 143.3 \\
        增设后 & 50.1 & 29.0(0.2) & 53.7 & 132.8 \\
        减少百分比 & 0 & 14.9\%(91.0\%) & 9.3\% & 10.5\% \\
        \hline
    \end{tabular}
\end{table}

注:单位为 h,括号内为第 2 阶段内在转载地域外等待的整体时间。

从表中可知增设的两临时装载点后起到了发射装置在装载点的分流作用,从而大大减少了发射装置在装载点外的等候暴露时间,最终减少整体暴露时间 10.5\%,更具体的改善案例如下:在求解的过程中我们发现有三台发射装置原先会在 J37-Z04 的单行道上造成拥堵冲突,需要 A 类发射装置在 J37 等待其他两类先行,造成暴露时间的延长,增设两转载点后,Z06 空闲,另两台装置改走 J37-Z06 单行道前往 Z06 转载点,缩短了该 A 类发射装置发射装置的等待暴露时间。

\section{问题 3 模型的建立与求解}

\subsection{问题 3 模型的建立}

问题 3 采用 0-1 规划模型求解,将 6 个隐蔽点分配到 15 个发射点位是整体暴露时间最短的问题,约束条件有如下 3 点,6 个隐蔽点分配到 15 个发射点位的最短距离根据 Floyd 算法计算得到。

(1) 每个隐蔽点分配的发射点位最多为 2 个;

(2) 每个可用发射点位分配的隐蔽点最多为 1 个;

(3) 从隐蔽点到发射点位的车为 3 辆。

将 6 个隐蔽点编号,引入 0-1 变量和指派问题的系数矩阵 $T = (t_{ij})_{6 \times 15}$,其中 $t_{ij}$ 为 C 类导弹发射装置从第 $i$ 个隐蔽点位到第 $j$ 个发射点位的用时。

\begin{equation}
x_{ij} =
\begin{cases}
1, \text{有导弹发射装置从第 } i \text{ 个隐蔽点位到第 } j \text{ 个发射点位} \\
0, \text{无导弹发射装置从第 } i \text{ 个隐蔽点位到第 } j \text{ 个发射点位}
\end{cases}
\tag{7-1}
\end{equation}

本题 0-1 规划的数学模型为

\begin{equation}
\min \sum_{i=1}^{6} \sum_{j=1}^{15} t_{ij} x_{ij};
\tag{7-2}
\end{equation}

\begin{equation}
\text{s.t.}
\begin{cases}
\sum_{i=1}^{6} x_{ij} \leq 1 \\
\sum_{j=1}^{15} x_{ij} \leq 2 \\
\sum_{i=1}^{6} \sum_{j=1}^{15} x_{ij} = 3 \\
x_{ij} = 0 \text{ 或 } 1, (i = 1, 2, \cdots, 6; j = 1, 2, \cdots, 15)
\end{cases}
\tag{7-3}
\tag{7-4}
\tag{7-5}
\tag{7-6}
\end{equation}

\subsection{问题 3 的求解}

(1) 求解借助 Lingo 软件求解上述数学模型,可以求解出 3 台隐蔽的 C 类车从隐蔽点到发射点位的路径,见表 7-1:

\begin{table}[h]
\centering
\caption{隐蔽点到发射点位的路径}
\begin{tabular}{c c c}
\hline
隐蔽点位 & 机动路径 & 发射点位 \\
\hline
J14 & J14 $\rightarrow$ J35 & F27 \\
J13 & J13 $\rightarrow$ J21 $\rightarrow$ J22 $\rightarrow$ F06 & F06 \\
J13 & J13 $\rightarrow$ J21 $\rightarrow$ J22 $\rightarrow$ F04 & F04 \\
\hline
\end{tabular}
\end{table}

(2) 同一时段内路径冲突的解决方案类似于以上的解答,对于同一时段内的路径冲突,则我们预计的解决方案是尽量避免同一时间段内路径的冲突,但由于此时出发时刻已经固定,所以我们将较短路径的出发时刻延后至没有冲突。

经过验算,表中的机动路径之间没有冲突,即按照我们的方案,3 辆新的 C 类导弹发射装置从隐蔽点赶往第二波次发射点位的过程中不会与问题 1 中的路线发生同

一时段内路径冲突,表 7-1 即为最终定下的方案。

(3)设置隐蔽点前后整体暴露时间对比对于给出的最优方案的结果,我们比较了设置隐蔽点前后的整体暴露时间对比,见表 7-2:

\begin{table}[h]
\centering
\caption{隐蔽点方案对整体暴露时间的影响}
\begin{tabular}{c|c c c c}
\hline \hline
\diagbox{对比}{阶段} & 第 1 阶段 & 第 2 阶段 & 第 3 阶段 & 合计 \\
\hline
改动前 & 50.1 & 34.0(4.9) & 59.2(7.5) & 143.3 \\
改动后 & 50.1 & 29.1(0) & 54.6(7.4) & 133.8 \\
减少百分比 & 0 & 14.5\%(100\%) & 7.8\%(1.1\%) & 9.5\% \\
\hline \hline
\end{tabular}
\end{table}

注:单位为 h,第 2 阶段的括号内为第 2 阶段内前往转载地域以及在转载地域外等待的整体暴露时间,第 3 阶段的括号内为已经就绪的导弹发射装置等待其余装置的整体暴露时间。

从表 7-2 可以看出事先布置隐蔽点后,由于替代的 3 台发射装置直接从隐蔽点出发到附近发射点,与原来的方案相比,减少了这 3 台装置从第一波次发射点位到转载地域的运输暴露时间以及在转载区域外的等待暴露时间;此外,由于去掉了第三阶段暴露时间最长的三个发射装置,使得第二次导弹齐射的时间提前,减少了其他较快发射装置在第二波次发射点位的等待暴露时间。

\section{问题 4 模型建立与求解}

\subsection{问题 4 模型的建立}

\subsubsection{层次分析前的准备:图论中的节点强度}

通过对问题 4 的分析,假设敌方不知道我放个节点功能特性的前提下,可使用层次分析法进行道路节点重要性程度的评价分析,该分析法的难点在于如何量化道路节点与其周边节点的连接程度这一因素,解决这一问题我们借助了图论中的节点强度概念。

一个无向图可以抽象为一个由点集 \( V \) 和边集 \( E \) 组成的图 \( G = (V, E) \)。E 中的每条边都有 \( V \) 中的一对点与之相对应。考虑一对节点 \( i \) 和 \( j \),如果 \( i \) 和 \( j \) 之间有边相连,则 \( i \) 叫作 \( j \) 的邻居或 \( j \) 叫作 \( i \) 的邻居。定义一个点的节点强度为:

\begin{equation}
w(z) = \sum_{k \in V} w_{zk} \tag{8-1}
\end{equation}

式中 \( w_{zk} \) 表示边 \( e_{zk} \) 的权重值,对于无向图而言,当 \( z \) 和 \( k \) 之间不连通时 \( w_{zk} = 0 \),当 \( z \) 和 \( k \) 之间连通时 \( w_{zk} = 1 \)。

由公式可知,当某一节点与更多节点直接连接时,其节点强度越大。对图中节点按节点强度的大小进行降序排列,初步筛除节点强度小于 3 的节点,对剩下的 58 个点进行进一步的分析。

\subsubsection{层次分析法}

层次分析法的基本步骤如下:

(1)建立层次分析结构模型。深入分析实际问题,将有关因素自上而下分层(目标-准则-方案),上层受下层影响,而层内个因素基本上相对独立;

(2) 构造成对比矩阵。用成对比较法和 \(1 \sim 9\) 尺度,构造各层对上一层每一因素的判断矩阵;

(3) 计算权响亮并作一致性检验。对每一成对比较阵,计算最大特征根和特征响亮作一致性检验,若通过,则特征向量为权向量;

(4) 计算组合权向量(作组合一致性检验),组合权向量可作为决策的定量依据。以破坏道路节点为目标层,该道路节点与其周边节点的连接程度和与主干道连接程度这两个因素为准则层,62 个道路结点为方案层即可对所有道路节点的重要程度进行评价。

\section*{8.2 问题 4 求解}

采用层次分析法,设置两个评价节点重要性的指标:节点强度和节点连接的主干道节点数量,两个评价指标在最终得分中所占比例均为 0.5。经过初步筛选后的 58 个点的节点强度在范围 \(\{6, 5, 4, 3\}\) 内,与各节点直接连接的双车道节点数量在范围 \(\{3, 2, 1, 0\}\) 内,利用 \(1 \sim 9\) 尺度构造判断矩阵,均为正互反阵,\(1 \sim 9\) 尺度的选择见表 8-1,具体系数见表 8-2 及表 8-3:

\begin{table}[h]
\centering
\caption{1-9 尺度选择表}
\begin{tabular}{c c l}
\hline
标度 & 定义 & 含义 \\
\hline
1 & 同样重要 & 两元素对某准则同样重要 \\
3 & 稍微重要 & 两元素对某准则,一元素比另一元素稍微重要 \\
5 & 明显重要 & 两元素对某准则,一元素比另一元素明显重要 \\
7 & 强烈重要 & 两元素对某准则,一元素比另一元素强烈重要 \\
9 & 极端重要 & 两元素对某准则,一元素比另一元素极端重要 \\
2, 4, 6, 8 & 相邻标度中值 & 表示相邻两标度之间折衷时的标度 \\
上下列表倒数 & 反比较 & 元素 \(i\) 对元素 \(j\) 的标度为 \(a_{ij}\),反之为 \(1/a_{ij}\) \\
\hline
\end{tabular}
\end{table}

\begin{table}[h]
\centering
\caption{节点强度判断矩阵}
\begin{tabular}{c c c c c}
\hline
重要性标度 & 节点强度为 6 & 节点强度为 5 & 节点强度为 4 & 节点强度为 3 \\
\hline
节点强度为 6 & 1 & 3 & 5 & 7 \\
节点强度为 6 & 1/3 & 1 & 3 & 5 \\
节点强度为 6 & 1/5 & 1/3 & 1 & 3 \\
节点强度为 6 & 1/7 & 1/5 & 1/3 & 1 \\
\hline
\end{tabular}
\end{table}

\begin{table}[h]
\centering
\caption{节点连接的主干道节点数量判断矩阵}
\begin{tabular}{c c c c c}
\hline
重要性标度 & 连接数量为 3 & 连接数量为 2 & 连接数量为 1 & 连接数量为 0 \\
\hline
连接数量为 3 & 1 & 3 & 5 & 7 \\
连接数量为 2 & 1/3 & 1 & 3 & 5 \\
连接数量为 1 & 1/5 & 1/3 & 1 & 3 \\
连接数量为 0 & 1/7 & 1/5 & 1/3 & 1 \\
\hline
\end{tabular}
\end{table}

在进行层次单排序和总排序时需要进行一致性检验,将准则层和方案层的一致性指标 CI 与平均随机一致性指标 RI 进行比较,经检验,最终得到的准则层和方案层的一致性比例 CR 均小于 0.1,说明判断矩阵选取合理。

对重要性程度的最终结果按顺序排列,利用 MATLAB 软件以直方图的形式给出,见图 8-1:从图 8-1 中可知最可能受到攻击的三个节点分别是 J32、J03 和 J21。

\begin{figure}[h]
    \centering
    \includegraphics[width=\textwidth]{image1.png}
    \caption{道路节点重要程度(赋权前与赋权后)}
    \label{fig:8-1}
\end{figure}

上述解答均建立在敌方军队事先不知该道路节点位于主干道还是其他道路的前提下,如果敌方军队事先已知道路节点位于主干道或其他道路,此时对于主干道和其他道路节点攻击的可能性必然不同,此时应该增加考虑一个权重系数,将权重系数与重要性程度相乘,重新排序得到最应该攻击的三个节点,我们的方案是主干道道路节点的权重系数为 1.2,其他道路为 1.0,重新计算得到的结果见图

从图中可以看出虽然主干道节点的重要性有所变化,但最可能受到攻击的三个节点仍为 JJ32、J03 和 J21。

尽管我们已经根据道路节点在网络中的重要性程度得到了最可能受到攻击的三个节点,但还需要结合实际情况验证这 3 个节点的重要性,图 8-2 统计了问题 1 中各道路节点出现的频数,这与之前的方案差异较大,这是因为在整个网络中,部分节点是有功能的(如待机地域、转载地域),在这些区域周围的道路节点重要性程度自然较大,而敌军由于不知道这些功能节点的位置,所以是无法通过道路节点在网络中的重要性程度分析出来的。

\begin{figure}[h]
    \centering
    \includegraphics[width=\textwidth]{image2.png}
    \caption{问题 1 中各道路节点出现的频数}
    \label{fig:8-2}
\end{figure}

\section{问题 5 模型的建立与求解}

\subsection{问题 5 模型的建立}

\subsubsection{目标函数}

根据前面对于问题 5 的分析,这是一个多目标优化问题。结合 3 个优化目标,我们可以把这个多目标优化问题转为单目标优化问题

\begin{equation}
\begin{cases}
\min Z = \lambda_1 X / \chi + \lambda_2 Y & (9-1) \\
T = \sum_{k=1}^{24} \sum_{i=1}^{3} \sum_{l=1}^{n_{k-1}} t_{h_k^{(i)}, h_{k,l+1}^{(i)}} & (9-2) \\
Y = \max \left\{ \sum_{i=1}^{3} \sum_{l=1}^{n_{k-1}} t_{h_k^{(i)}, h_{k,l+1}^{(i)}} \right\} & (9-3)
\end{cases}
\end{equation}

式中:$Z$ 为目标函数,$X$ 为整体暴露时间,$Y$ 为单辆导弹发射装置的最大暴露时间,由于 $X$ 和 $Y$ 的数量级不一致,所以引入相关系数 $1/\chi$,对 $X$ 和 $Y$ 的数量级进行同一,$\lambda_1$, $\lambda_2$ 为各个目标的权重,其他符号的含义同 5.1。

\subsubsection{各阶段的约束条件}

此目标函数下各阶段的约束条件与 5.1 相同,见公式 (5-2)~(5-20)。

\subsection{问题 5 求解}

对于问题 5,尽管目标函数和约束条件都与问题 1 类似,但我们不准备采取求解问题 1 的思路,而是打算借助 Lingo 软件直接对整个过程进行优化。但是我们最终发现以我们目前的手段是无法直接求出目标函数的最优解的,因此我们另外独立设计了一种算法求解问题的较优解(不一定最优),我们的算法如下:

(1)首先考虑到目标 1,我们要使发射点位尽可能分散,这里我们的策略是把 60 个发射点位分配给 6 个转载地域,保证每个转载地域都分配到发射点位,具体的实现方法是以 6 个转载地域为中心进行 k-means 聚类,这里 k-means 聚类以距离为判断因子;

(2)根据每个转载地域分配到的发射点位的数量比例将 24 个导弹发射装置也分配给转载地域,这样(1)(2)步就保证了发射点位尽量分散;

(3)接下来综合考虑使整体暴露时间和单车最大暴露时间最短的目标,假设转载地域 $Z(i)$ 分配有 $n$ 个导弹发射装置,这 $n$ 个导弹发射装置需要完成两个波次的导弹发射任务,即要前往 $2n$ 个发射点位,因为无论哪种路径转载地域是必须要经过的节点,所以在这种情况下要使整体暴露时间和单车最大暴露时间最短,则两个波次的导弹发射点位均需距转载地域较近,将分配到转载地域的 $m(m \neq 2n)$ 个发射点位按与转载地域间的距离从小到大排序,选取前 $2n$ 个发射点位作为该转载地域使用的发射点位;

(4)关于出发点的选取,根据所选发射点,选取离它最近的出发点使从待机地域 D1、D2 出发的路径数一致;

(5)根据每点得出的前 $2n$ 个发射点列出其距该转载点的距离以及距最近出发点的距离,利用 Lingo 软件进行优化计算,自由配合最终得到 $n$ 条由待机地域到第一波次发射点位再到转载点再到第二波次发射点位的导弹运输路径。

使整体暴露时间最短的机动路线见表 9-1,

\begin{table}
\centering
\caption{24 辆导弹发射装置全过程的机动路线}
\begin{tabular}{c}
\hline
\hline
机动路线 \\
\hline
D2 $\rightarrow$ J03 $\rightarrow$ J04 $\rightarrow$ J50 $\rightarrow$ F50 $\rightarrow$ J50 $\rightarrow$ Z01 $\rightarrow$ J04 $\rightarrow$ J05 $\rightarrow$ J49 $\rightarrow$ F49 \\
D2 $\rightarrow$ J03 $\rightarrow$ J48 $\rightarrow$ F47 $\rightarrow$ J48 $\rightarrow$ Z01 $\rightarrow$ J50 $\rightarrow$ J53 $\rightarrow$ J56 $\rightarrow$ F52 \\
D2 $\rightarrow$ J03 $\rightarrow$ J48 $\rightarrow$ F46 $\rightarrow$ J48 $\rightarrow$ Z01 $\rightarrow$ J50 $\rightarrow$ J53 $\rightarrow$ J56 $\rightarrow$ F53 \\
D2 $\rightarrow$ J32 $\rightarrow$ J33 $\rightarrow$ F25 $\rightarrow$ J33 $\rightarrow$ Z01 $\rightarrow$ J04 $\rightarrow$ J05 $\rightarrow$ J34 $\rightarrow$ J35 $\rightarrow$ F27 \\
D2 $\rightarrow$ J03 $\rightarrow$ J02 $\rightarrow$ J47 $\rightarrow$ F44 $\rightarrow$ J47 $\rightarrow$ J48 $\rightarrow$ Z01 $\rightarrow$ J04 $\rightarrow$ J05 $\rightarrow$ J34 $\rightarrow$ F26 \\
D2 $\rightarrow$ J03 $\rightarrow$ J02 $\rightarrow$ J47 $\rightarrow$ F45 $\rightarrow$ J47 $\rightarrow$ J48 $\rightarrow$ Z01 $\rightarrow$ J04 $\rightarrow$ J05 $\rightarrow$ J49 $\rightarrow$ F48 \\
D2 $\rightarrow$ J32 $\rightarrow$ F24 $\rightarrow$ J32 $\rightarrow$ J33 $\rightarrow$ J04 $\rightarrow$ Z01 $\rightarrow$ J04 $\rightarrow$ J05 $\rightarrow$ J34 $\rightarrow$ J35 $\rightarrow$ F28 \\
D1 $\rightarrow$ J09 $\rightarrow$ J08 $\rightarrow$ J07 $\rightarrow$ J06 $\rightarrow$ J36 $\rightarrow$ F29 $\rightarrow$ J36 $\rightarrow$ J51 $\rightarrow$ Z02 $\rightarrow$ J54 $\rightarrow$ J55 $\rightarrow$ \\
\multicolumn{1}{r}{J59 $\rightarrow$ J62 $\rightarrow$ F59} \\
D1 $\rightarrow$ J09 $\rightarrow$ J08 $\rightarrow$ J07 $\rightarrow$ J06 $\rightarrow$ J36 $\rightarrow$ F30 $\rightarrow$ J36 $\rightarrow$ J51 $\rightarrow$ Z02 $\rightarrow$ J54 $\rightarrow$ J55 $\rightarrow$ \\
\multicolumn{1}{r}{J59 $\rightarrow$ J62 $\rightarrow$ F60} \\
D1 $\rightarrow$ Z03 $\rightarrow$ J61 $\rightarrow$ F58 $\rightarrow$ J61 $\rightarrow$ Z03 $\rightarrow$ J57 $\rightarrow$ J54 $\rightarrow$ J55 $\rightarrow$ J53 $\rightarrow$ F51 \\
D1 $\rightarrow$ Z03 $\rightarrow$ J61 $\rightarrow$ F57 $\rightarrow$ J61 $\rightarrow$ Z03 $\rightarrow$ J57 $\rightarrow$ J58 $\rightarrow$ J59 $\rightarrow$ F54 \\
D1 $\rightarrow$ J09 $\rightarrow$ J08 $\rightarrow$ J07 $\rightarrow$ Z04 $\rightarrow$ J38 $\rightarrow$ F34 $\rightarrow$ J38 $\rightarrow$ Z04 $\rightarrow$ J37 $\rightarrow$ J15 $\rightarrow$ J14 $\rightarrow$ \\
\multicolumn{1}{r}{J21 $\rightarrow$ F03} \\
D1 $\rightarrow$ J09 $\rightarrow$ J08 $\rightarrow$ J07 $\rightarrow$ Z04 $\rightarrow$ J38 $\rightarrow$ F35 $\rightarrow$ J38 $\rightarrow$ Z04 $\rightarrow$ J37 \\
D1 $\rightarrow$ J09 $\rightarrow$ J08 $\rightarrow$ J07 $\rightarrow$ Z04 $\rightarrow$ J37 $\rightarrow$ F31 $\rightarrow$ J37 $\rightarrow$ Z04 $\rightarrow$ J37 $\rightarrow$ F33 \\
\multicolumn{1}{r}{D1 $\rightarrow$ J11 $\rightarrow$ J46 $\rightarrow$ J44 $\rightarrow$ F42 $\rightarrow$ J44 $\rightarrow$ Z05 $\rightarrow$ J41 $\rightarrow$ F40} \\
\multicolumn{1}{r}{D1 $\rightarrow$ J11 $\rightarrow$ J46 $\rightarrow$ F43 $\rightarrow$ J46 $\rightarrow$ J44 $\rightarrow$ Z05 $\rightarrow$ J44 $\rightarrow$ F41} \\
D1 $\rightarrow$ J11 $\rightarrow$ J46 $\rightarrow$ J44 $\rightarrow$ Z05 $\rightarrow$ J41 $\rightarrow$ J18 $\rightarrow$ J29 $\rightarrow$ J30 $\rightarrow$ F19 $\rightarrow$ J30 $\rightarrow$ J28 $\rightarrow$ \\
\multicolumn{1}{r}{Z06 $\rightarrow$ J26 $\rightarrow$ F13} \\
D1 $\rightarrow$ J11 $\rightarrow$ J46 $\rightarrow$ J44 $\rightarrow$ Z05 $\rightarrow$ J41 $\rightarrow$ J18 $\rightarrow$ J29 $\rightarrow$ J30 $\rightarrow$ F22 $\rightarrow$ J30 $\rightarrow$ J28 $\rightarrow$ \\
\multicolumn{1}{r}{Z06 $\rightarrow$ J26 $\rightarrow$ F12} \\
D1 $\rightarrow$ J10 $\rightarrow$ J45 $\rightarrow$ J09 $\rightarrow$ J08 $\rightarrow$ J42 $\rightarrow$ J40 $\rightarrow$ F38 $\rightarrow$ J40 $\rightarrow$ J39 $\rightarrow$ J16 $\rightarrow$ Z06 $\rightarrow$ \\
\multicolumn{1}{r}{J16 $\rightarrow$ J39 $\rightarrow$ F36} \\
D2 $\rightarrow$ J12 $\rightarrow$ J13 $\rightarrow$ J14 $\rightarrow$ J15 $\rightarrow$ J25 $\rightarrow$ F11 $\rightarrow$ J25 $\rightarrow$ Z06 $\rightarrow$ J28 $\rightarrow$ J27 $\rightarrow$ F16 \\
D2 $\rightarrow$ J12 $\rightarrow$ J13 $\rightarrow$ J14 $\rightarrow$ J15 $\rightarrow$ J25 $\rightarrow$ F10 $\rightarrow$ J25 $\rightarrow$ Z06 $\rightarrow$ J28 $\rightarrow$ J27 $\rightarrow$ F17 \\
D2 $\rightarrow$ J12 $\rightarrow$ J13 $\rightarrow$ J14 $\rightarrow$ J15 $\rightarrow$ J25 $\rightarrow$ J23 $\rightarrow$ J24 $\rightarrow$ F09 $\rightarrow$ J24 $\rightarrow$ J26 $\rightarrow$ Z06 $\rightarrow$ \\
\multicolumn{1}{r}{J28 $\rightarrow$ F18} \\
D2 $\rightarrow$ J12 $\rightarrow$ J13 $\rightarrow$ J14 $\rightarrow$ J15 $\rightarrow$ J25 $\rightarrow$ J23 $\rightarrow$ J24 $\rightarrow$ F08 $\rightarrow$ J24 $\rightarrow$ J26 $\rightarrow$ Z06 $\rightarrow$ \\
\multicolumn{1}{r}{J28 $\rightarrow$ F19} \\
D2 $\rightarrow$ 80 $\rightarrow$ 81 $\rightarrow$ J21 $\rightarrow$ J22 $\rightarrow$ J23 $\rightarrow$ F07 $\rightarrow$ J23 $\rightarrow$ J25 $\rightarrow$ Z06 $\rightarrow$ J28 $\rightarrow$ J27 $\rightarrow$ F15 \\
\hline
\hline
\end{tabular}
\end{table}

\begin{table}
\centering
\caption{问题 1 和问题 5 暴露时间对比}
\begin{tabular}{c c c c c}
\hline
\hline
 & 第一次齐射时 & 第二次齐射时 & 整体暴露时间 & 单台最长暴露 \\
 & 刻 & 刻 & & 时间 \\
\hline
问题 1 & 2.97 & 7.53 & 143.41 & 6.98 \\
问题 5 & 4.27 & 9.24 & 156.2 & 8.12 \\
增加百分比 & 43.62\% & 22.66\% & 8.91\% & 26.31\% \\
\hline
\hline
\end{tabular}
\end{table}

表 9-2 给出了问题 1 与问题 5 计算得到的最短整体暴露时间与单台最长暴露时间,图 9-1 与 9-2 是问题 1 与问题 5 中机动路线的覆盖范围。

虽然已知暴露时间的增加是实现分散机动策略必须付出的代价,但与第一题结果

\begin{figure}[h]
    \centering
    \includegraphics[width=\textwidth]{image1.png}
    \caption{问题 1 机动路线覆盖范围}
    \label{fig:9-1}
\end{figure}

\begin{figure}[h]
    \centering
    \includegraphics[width=\textwidth]{image2.png}
    \caption{问题 5 机动路线覆盖范围}
    \label{fig:9-2}
\end{figure}

相比,最终得到的总体暴露时间和单个最长暴露时间的增大百分比仍超出我们的预估。分析布置网络图可以发现,转载点 Z06 周边的发射点分布范围较多且相对分散,但普遍离该转载点以及一开始的待机出发点距离较远,若为了达到分散机动的效果将导弹发射装置派往这些发射点位,就将造成暴露时间的大大增长。由此可见,这张布设网络图本身便存在缺陷,在发射点分布如此多且广的这部分区域内,不应该仅仅布置 Z06 这一个转载点。在实际的军事布设网络图的规划中,为了能同时考虑分散机动和缩短导弹暴露时间的效果,对于转载点、发射点的数量及相对位置的布设需要统筹规划、慎重考虑。当然,从前两题的分析可知,增设临时转载地域以及事先布置发射装置隐蔽点也是在军事布设网络图不完善的条件下减少暴露时间的有效的补救措施。

\section{模型的讨论与评价}

\subsection{模型的优点}

(1) 为解决问题 1 和问题 5 所建立的理想化两波次导弹作战运输模型较为合理、完整,并可以向多波次导弹作战运输模型推广;

(2) 同时采用聚类和单目标优化算法对不同类型车辆进行路线、转载地域和两次导弹发射点位的分配;

(3) 为避免多辆导弹车在同一转载地域排队拥挤的现象,合理平衡不同转载地域的车流,在不推延第二波导弹齐次发射时间的前提下,缩减车辆整体暴露时间,得到更优结果;

(4) 正确处理多辆车在单行道上相遇的问题:通向超车和相向相遇。以减小整体暴露时间为目标,在不同阶段采取不同的处理方式;

(5) 在解决第 2 题增设转载地域时,根据增设单个转载点后两阶段(第一波次发射点位-转载地域、转载地域-第二波次发射点位)的效果进行综合评价,最后选择效果最优的两个节点;

(6) 将路网分布简化为双向图,选择两个针对节点重要性的评价指标,采用层次分析法对节点重要性进行排序,并与实际情况相比较分析。

\subsection{模型的缺点及改进方向}

(1) 在第问题 1 的建模过程中,将同时考虑三个阶段的优化问题简化为分别求取每个阶段暴露时间最小的路线设计问题,求得的是较优解,非绝对最优解;

(2) 在解决问题 1 和问题 5 过程中,发现若要实现全路网车辆的最优化分配,“执行任务前车辆平均部署在 2 个待机地域”这一先决条件必须打破,即若自由分配初始车辆,会得到更优的结果;

(3) 根据问题 5 的结果发现,在采用分散机动策略规避敌方侦查打击的同时,必定以增加暴露时间作为牺牲。在分散路线的前提下,根据转载地域附近导弹发射点数量分配车辆路线的方法并非最好的解决办法。

\section{结论}

问题 1 结论:为求的使整体暴露时间最短的导弹发射装置的机动路线,可采用单目标优化模型,问题一中约束条件复杂,很难直接优化,为求得较优解,先将目标函数整体暴露时间分成 3 个阶段,第 1 阶段为导弹从待机地域到第一波次齐射点位的时段,第 2 阶段为导弹从第一波次齐射点位进入装载地域的时段,第 3 阶段为导弹从装载地域到第二波次齐射点位的时段,后面的阶段以前面阶段的方案为基础求解。结合 Floyd 最小路径算法和 0-1 规划依次求解 3 个阶段使整体暴露时间最短的机动路线并整合,最终得到第一波导弹齐次发射时间为第 178.4min,第二波导弹齐次发射时间为第 451.7min,整体暴露时间为 8598.8min。

问题 2 结论:判断分别增加单个转载地域对缩减整体暴露时间的效果,同时考虑转载地域的增加对之前存在的 6 个转载地域产生的分流效果,最终确定增设作为新的转载地域,转载地域数量由 6 个增加到 8 个以后,整体暴露时间缩减了 10.5\%。

问题 3 结论:去掉第 1 题中第三阶段暴露时间最长的 3 台 C 车,根据 Lingo 优化的结果,在道路节点 J14 隐藏 1 台 C 车,最终到达发射点位 F27;在道路节点 J13 隐藏 2 台 C 车,最终分别到达发射点位 F06 和 F04,该方案使整体暴露时间缩减了 9.5\%。

问题 4 结论:在敌方对该军事领域各节点属性不了解的前提下,根据节点强度和与双向道节点连接的数量两个指标,应用层次分析法对 62 个道路节点的重要性进行量化分析,得出最可能受到敌方攻击破坏的 3 个道路节点是 J32、J03 和 J21。

问题 5 结论:为规避敌方的侦查和打击,应用 K-means 聚类法对所有导弹发射点位进行聚类,根据各个转载地域所分得的发射点位数确定该转载地域在第二阶段所装弹的车辆数,再应用 Floyd 最小路径算法和单目标优化的 Lingo 算法对路线进行部署。得到第一波导弹齐次发射时间为第 256.3min,第二波导弹齐次发射时间为第 554.1min,整体暴露时间为 9371.8min,单条暴露最长时间为 487.0min。得出采用分散机动策略必须以整体暴露时间和单条暴露最长时间的增加为代价的结论。

\section*{12 参考文献}

[1] 金宏, 余跃, 张如飞. 常规导弹联合火力打击统一分配模型 [J]. 火力与指挥控制, 2014(7): 27-30.

[2] 宋志华, 张晗, 惠晓滨, 等. 导弹作战行动网络流模型及动态规划算法 [J]. 解放军理工大学学报, 2017(4): 1-6.

[3] 季青梅, 辛文芳. 多波次导弹火力打击任务研究 [J]. 信息技术与信息化, 2017(z1):122-128.

[4] 刘俊玮, 王子豪. 基于模拟退火与 Dijkstra 算法的复杂网状结构供应链最优化研究 [J]. 浙江大学学报, 2014, 41(2):149-152.

[5] 王梓行, 姜大立, 杨李, 等. 战时导弹火力打击任务分配与运输决策模型 [J]. 后勤工程学院学报, 2017, 33(04):77-85.