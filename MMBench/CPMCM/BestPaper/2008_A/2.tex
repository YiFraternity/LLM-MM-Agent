\begin{center}
\textbf{\Large 全国第五届研究生数学建模竞赛}
\end{center}

\begin{center}
\includegraphics[width=0.5\textwidth]{logo.png}
\end{center}

\begin{flushleft}
题目 \hspace{1cm} 唐家山堰塞湖泄洪问题的研究
\end{flushleft}

\begin{center}
\textbf{摘 \hspace{2cm} 要}
\end{center}

本文研究的是唐家山地震次生灾害引发的堰塞湖问题,结合数字高程地图和新闻报道所提供的数据分别建立模型研究了唐家山堰塞湖形成之后湖水高程、蓄水量、溃坝情形、溃坝灾害等一系列问题。

针对第一问,首先对数字高程地图进行等高图像分析求解了堰塞湖不同高程水位对应的湖区面积,建立了蓄水量体积与堰塞湖水位高程的离散化模型,然后建立了神经网络模型和多元线性回归模型研究了北川天气预报 3 天降雨量与堰塞湖入库流量的关系,继而求解得到不同降雨量下每日堰塞湖水位高程。针对第二问,首先研究泄洪过程和溃坝过程的区别与联系,从而分别建立正交多项式逼近和仿真模型得到溃坝时的溃口流量随时间变化的关系,继而分析求解得到溃坝时溃口宽度、深度、水位高程和水流速度随时间变化的关系。针对第三问,综合数字高程地图和行政区域地图,在数字地图中查找地势相对较低区域,进而得到洪水下泄区域及被淹没区域。确定洪水的流速之后,利用数字地图计算了洪水到达各被淹没区域的时间,淹没范围,以及淹没之后的安全区域,并据此制订了初步的撤离方案。针对第四问,建立安全最优化模型定量评价了决策部门的堰塞体导流渠掘进方案并定性分析了决策部门制定的撤离方案。

\textbf{关键词:图像分析 神经网络 正交拟合 溃坝仿真}

\begin{flushleft}
参赛队号 \hspace{1cm} \underline{9004521}
\end{flushleft}

\begin{flushright}
\begin{tabular}{|c|}
\hline
参赛密码 \hspace{2cm} \underline{\hspace{3cm}} \\
\hline
(由组委会填写) \\
\hline
\end{tabular}
\end{flushright}

\section{问题重述}

1. 建立唐家山堰塞湖以水位高程为自变量的蓄水量的数学模型。并以该地区天气预报的降雨情况的 $50\%$, $80\%$, $100\%$, $150\%$ 为实际降雨量建立模型,预计自 5 月 25 日起至 6 月 12 日堰塞湖水位每日上升的高度(不计及泄洪)。

2. 利用这次唐家山堰塞湖泄洪时科技人员记录下了大量宝贵的数据,在合理的假设下建立堰塞湖蓄水漫顶后在水流作用下发生溃坝的数学模型,包含缺口宽度、深度、水流速度、水量、水位高程、时间等变量。

3. 根据数字地图给出坝体发生溃塌,造成堰塞湖内 $1/3$ 的蓄水突然下泻时的洪水水流速度及淹没区域(包括洪水到达各地的时间),并在此基础上考虑洪水淹没区域中人口密集区域的人员撤离方案。

4. 请根据建立的数学模型分析当时所采取对策的正确性和改进的可能性。讨论为应对地震后次生山地灾害(不限堰塞湖),科技工作中应该设法解决的关键问题,并提出有关建议。

\section{问题分析}

1. 研究不同高程水位对应的堰塞湖蓄水量,即研究不规则体的体积问题,可以考虑引入面积对高度的微元进行积分这一思想进行求解。新闻报道中的实际数值可以作为求解时的重要参考和校正点。在研究降雨量和堰塞湖水位的关系之时,认为将这一问题转化为首先研究降雨量和库容变化量之间的关系更为直观,考虑到两者关系的复杂性,并不一定需要建立明确的函数关系式。

2. 现实中唐家山并没有发生溃坝,无法获得溃坝的直接数据。这里也十分缺乏利用成型模型求解溃坝问题的大量参数。但是唐家山的泄洪的规模也是非常大的,可以借助研究唐家山的泄洪过程分析这一泄洪和溃坝的区别与联系,继而通过拟合、仿真等手段得出唐家山的溃坝过程。

3. 结合数字地图可以求得不同位置的海拔高度,根据水顺势而流的原理就可以推算洪水的流经区域。结合泻出的洪水量就可以结合问题 1 的思想计算求解淹没范围。

\section{重要符号说明}

\begin{itemize}
    \item $V$ -- 堰塞湖蓄水量;
    \item $S$ -- 堰塞湖实际剖面面积;
    \item $S_{k}$ -- 堰塞湖图像面积;
    \item $h$ -- 水位高程;
    \item $w$ -- 溃口宽度值;
    \item $g$ -- 溃口深度值;
    \item $q$ -- 水流量大小;
    \item $f$ -- 溃口中水深;
    \item $d$ -- 掘进导流槽安全工作高度;
    \item $K$ -- 面积系数即堰塞湖实际面积与图像面积的比例系数。
\end{itemize}

\section{问题重要假设}

(1) 溃坝过程与泄洪过程内在机理是相同的,泄洪相当于一次小型溃坝;

(2) 溃坝后水位高程基本稳定在 713 米,与泄洪后的水位高程是基本一致的;

(3) 溃坝水位高程 750 米对应的堰塞体构造与泄洪 743 米对应的堰塞体构造是一样的,因而溃口处沙石泥土等内部结构一样;

\begin{table}[h]
\centering
\caption{唐家山堰塞湖水位(米)与蓄水量(亿立方米)}
\begin{tabular}{|c|c|c|c|c|c|}
\hline
水位 & 蓄水量 & 水位 & 蓄水量 & 水位 & 蓄水量 \\
\hline
713.41 & 0.834 & 717.51 & 1 & 738.5 & 2.159 \\
\hline
713.42 & 0.835 & 718.06 & 1.02 & 738.71 & 2.175 \\
\hline
713.49 & 0.837 & 719.48 & 1.12 & 738.85 & 2.185 \\
\hline
713.5 & 0.838 & 720.25 & 1.139 & 738.93 & 2.19 \\
\hline
713.54 & 0.839 & 733 & 1.8 & 739.37 & 2.222 \\
\hline
713.57 & 0.84 & 733.42 & 1.834 & 739.59 & 2.237 \\
\hline
713.64 & 0.843 & 735.53 & 1.957 & 740.51 & 2.328 \\
\hline
713.68 & 0.844 & 736.63 & 2.008 & 740.55 & 2.307 \\
\hline
713.79 & 0.849 & 736.78 & 2.055 & 741.12 & 2.351 \\
\hline
714.01 & 0.857 & 737.53 & 2.094 & 741.66 & 2.389 \\
\hline
714.13 & 0.861 & 737.56 & 2.071 & 742.58 & 2.457 \\
\hline
714.51 & 0.876 & 737.78 & 2.116 & 742.8 & 2.469 \\
\hline
\end{tabular}
\end{table}

数据特点:
\begin{itemize}
    \item 数据分布不均匀,主要集中在(710~720)和(730~740)两个水位段;
    \item 数据量较少,整个水位高程对应的数据不完善;
    \item 随着水位高程增加堰塞湖蓄水量的变化率趋势总体上越来越大,但增长规律不够稳定,有些高程大的水位蓄水量增长反而慢,数据可能有误差。
\end{itemize}

\begin{figure}[h]
    \centering
    \includegraphics[width=\textwidth]{image1.png}
    \caption{水位高程为730米时堰塞湖图像面积}
    \label{fig:5.1}
\end{figure}

\begin{figure}[h]
    \centering
    \includegraphics[width=\textwidth]{image2.png}
    \caption{水位高程为750米时堰塞湖图像面积}
    \label{fig:5.2}
\end{figure}

从图\ref{fig:5.1}、图\ref{fig:5.2}可以看出唐家山堰塞湖不同水位高程横向剖面面积不同,水位高程大对应横向剖面面积也大。

\section*{堰塞湖横向剖面图像面积的处理方法:}

相对面积的获取可以从附件所给的唐家山数字地图上得到,具体步骤如下:

\textbf{Step 1:} 利用所给的3DEM软件将tjs.dem文件载入。

\textbf{Step 2:} 利用唐家山的经纬度与高度,从709m开始至750m,以1m为间隔设定某一高度值,例如720m,为高度分界值后,将原数字地图分为两种明显的不同颜色。

\textbf{Step 3:} 保存新得到的数字地图。

\textbf{Step 4:} 使用Matlab载入新得的数字地图,运用不同颜色的像素点的值不同的原理,提取出此时堰塞湖所在地的地图,并求得湖面的图像面积。

\textbf{Step 5:} 再次设定另一高度值,重复进行Step 2→Step 4,直到高度值到达750m。

其流程图如下图所示:

\begin{figure}[h]
    \centering
    \includegraphics[width=\textwidth]{image3.png}
    \caption{堰塞湖相对面积处理流程图}
    \label{fig:5.4}
\end{figure}

通过对数字高程地图不同高度进行分色处理结合唐家山的经纬度,利用Matlab获取图像就可获得不同高程下唐家山堰塞湖湖面的图像面积(1个像素为单位1)如表\ref{tab:5.2}所示:

\begin{table}[h]
    \centering
    \caption{唐家山不同水位高程(米)对应的横向剖面相对面积}
    \label{tab:5.2}
    \begin{tabular}{|c|c|c|c|c|c|c|}
    \hline
    水位高程 & 709 & 710 & 711 & 712 & 713 & 714 \\ \hline
    相对面积 & 375 & 400 & 417 & 430 & 451 & 465 \\ \hline
    水位高程 & 715 & 716 & 717 & 718 & 719 & 720 \\ \hline
    相对面积 & 483 & 499 & 514 & 534 & 552 & 567 \\ \hline
    水位高程 & 721 & 722 & 723 & 724 & 725 & 726 \\ \hline
    相对面积 & 588 & 613 & 637 & 662 & 681 & 699 \\ \hline
    水位高程 & 727 & 728 & 729 & 730 & 731 & 732 \\ \hline
    相对面积 & 723 & 754 & 772 & 795 & 819 & 844 \\ \hline
    水位高程 & 733 & 734 & 735 & 736 & 737 & 738 \\ \hline
    相对面积 & 870 & 899 & 923 & 947 & 964 & 981 \\ \hline
    水位高程 & 739 & 740 & 741 & 742 & 743 & 744 \\ \hline
    相对面积 & 1003 & 1038 & 1053 & 1074 & 1097 & 1118 \\ \hline
    \end{tabular}
\end{table}

\begin{table}
\centering
\begin{tabular}{|c|c|c|c|c|c|c|}
\hline
水位高程 & 745 & 746 & 747 & 748 & 749 & 750 \\ \hline
相对面积 & 1138 & 1160 & 1194 & 1222 & 1244 & 1279 \\ \hline
\end{tabular}
\end{table}

\begin{figure}[h]
    \centering
    \includegraphics[width=\textwidth]{image.png}
    \caption{唐家山堰塞湖不同水位高程横向剖面图}
    \label{fig:5.5}
\end{figure}

\subsection{蓄水量体积离散化模型的建立}

\subsubsection{蓄水量体积离散化思想}

\begin{figure}[h]
    \centering
    \includegraphics[width=\textwidth]{image1.png}
    \caption{堰塞湖蓄水量离散化示意图}
    \label{fig:5.6}
\end{figure}

\begin{enumerate}
    \item \textbf{基本体积单位}
    
    通过 5.1.1 分析,堰塞湖总蓄水量可以离散成若干个水面高程下的体积单位,为了研究方便以水位高程 709 米为基准起始点,将高程差为 1 米的蓄水量定义为一个基本体积单位,令 $i=1$,直至水位高程 750 米,$i=42$,基本体积单位蓄水量为
    \begin{equation}
        \Delta V_{i} = S_{i} \Delta h, \quad S_{i} = S_{iK} K, \Delta h_{i} = h_{i+1} - h_{i} = 1, \quad i=1,2,\cdots,42
    \end{equation}
    式中:
    \begin{itemize}
        \item $\Delta V_{i}$ -- 第 $i$ 个基本体积单位的蓄水量;
        \item $S_{i}$ -- 第 $i$ 个基本体积单位的实际剖面面积;
        \item $\Delta h_{i}$ -- 基本体积单位对应的高程差;
        \item $S_{iK}$ -- 第 $i$ 个基本体积单位剖面的图像面积;
        \item $K$ -- 面积系数即堰塞湖实际面积与图像面积的比例系数。
    \end{itemize}
    
    \item \textbf{一般体积单位}
    
    由于蓄水量数据有限且不连续,需要对两个不相邻或高程不为整数的水面高程对应的蓄水量作为一个一般体积单位进行研究,此时两个高程之间的蓄水量可能由不完整的基本体积单位和若干个基本体积单位构成。研究某一高程 $h_{c}$ 对应的蓄水量时,假定蓄水量上边界水面高程为 $h_{a}$、相应剖面的面积为 $S_{a}$,下边界水面高程为 $h_{b}$、对应剖面的面
\end{enumerate}

积为 $S_{b}$,为了后面处理方便进行如下处理:
\begin{align*}
H_{a} &= \lfloor h_{a} \rfloor \quad -- \text{上边界水面高程向下取整}; \\
H_{b} &= \lfloor h_{b} \rfloor \quad -- \text{下边界水面高程向下取整}; \\
&\text{令 } i = H_{a} - 709 + 1, \quad j = H_{b} - 709 + 1.
\end{align*}

水位高程 $h_{a}, h_{b}$ 及其对应的蓄水量 $v_{a}, v_{b}$ 是从已知唐家山堰塞湖的数据中提取的,设集合 $D$ 表示已知数据中唐家山堰塞湖不同高程对应的蓄水量集合:
\[
h_{b} = \max \{ h \mid h \in D, h \leq h_{c} \}, \quad h_{a} = \min \{ h \mid h \in D, h \geq h_{c} \}
\]
依据离散化思想,先将不在整数值上的高程剖面面积转化到整数值对应的剖面面积上,采用反距离加权法将两个边界高程对应的剖面面积转化为

\begin{figure}[h]
    \centering
    \includegraphics[width=\textwidth]{image.png}
    \caption{反距离加权法示意图}
\end{figure}

\begin{align*}
S_{a} &= S_{i} \left( h_{a} - h_{i} \right) / \left( h_{i+1} - h_{i} \right) + S_{(i+1)} \left( h_{i+1} - h_{a} \right) / \left( h_{i+1} - h_{i} \right) \\
S_{b} &= S_{j} \left( h_{b} - h_{j} \right) / \left( h_{j+1} - h_{j} \right) + S_{(j+1)} \left( h_{j+1} - h_{b} \right) / \left( h_{j+1} - h_{j} \right)
\end{align*}

一般体积单位包含 2 种情况:
\begin{enumerate}
    \item 如果 $i - j < 2$,则 $i, j$ 段的蓄水量不包括基本体积单位,蓄水量可以表示为
    \[
    \Delta V_{ij} = S_{ij} \Delta h_{ij}, \quad S_{ij} = \left( S_{i} + S_{j} \right) / 2, \quad S_{i} = S_{iK} K, \quad \Delta h_{ij} = h_{a} - h_{b}
    \]
    \item 如果 $i - j \geq 2$,则 $i, j$ 段的蓄水量包含若干个基本体积单位和两侧不完整的
    基本体积单元,蓄水量可以表示为几个部分累加:
    \[
    \Delta V_{ij} = \sum_{l=i}^{j} S_{l} \Delta h_{l}
    \]
\end{enumerate}

其中,完整基本体积单位:实际剖面面积与高程差如下
\[
S_{l} = S_{lK} K, \quad \left( l = i+1, \cdots, j-1 \right), \quad \left( \Delta h_{i+1} = \Delta h_{i+2} = \cdots = \Delta h_{j-1} = 1 \right)
\]
不完整基本体积单位:两侧对应的蓄水量,剖面面积和高程差如下
\begin{align*}
S_{a} &= S_{i} \left( h_{a} - h_{i} \right) / \left( h_{i+1} - h_{i} \right) + S_{(i+1)} \left( h_{i+1} - h_{a} \right) / \left( h_{i+1} - h_{i} \right) \\
S_{b} &= S_{j} \left( h_{b} - h_{j} \right) / \left( h_{j+1} - h_{j} \right) + S_{(j+1)} \left( h_{j+1} - h_{b} \right) / \left( h_{j+1} - h_{j} \right) \\
\left( \Delta h_{i} &= h_{a} - H_{a}, \Delta h_{j} = h_{b} - H_{b} \right)
\end{align*}

通过上述分析,研究蓄水量与水面高程的关系,首先要研究实际剖面面积与图像面积的比例系数 $K$。

\section*{[2] 面积比例系数 $K$ 的研究}

\subsection*{① 面积比例系数 $K$ 的求解}

研究表 1 中的数据,将两个相邻数据按 [1] 中处理方法向下取整,并采用反距离加权法得到两个高程值对应的面积,再根据两个不同高程的高程差对应的蓄水量求解面积比例系数:

\begin{equation}
\Delta V_{ij} = V_j - V_i, \quad \Delta V_{ij} = \sum_{l=i}^{j} S_{lK} K \Delta h_l, \quad K = \frac{\Delta V_{ij}}{\sum_{l=i}^{j} S_{lK} \Delta h_l}
\end{equation}

按上述步骤求解,得到表 1 中相邻数据之间的面积比例系数 \( K \) 值,其中当两个相邻数据向下取整结果相等时按上述解法得到 \( K \) 值为 0 不采用,向下取整不等的相邻数据按照一般体积单位求解方法得到面积比例系数 \( K \),结果见表 5.3:

\begin{table}[h]
\centering
\caption{面积比例系数 \( K \) 的求解结果}
\begin{tabular}{|c|c|c|c|c|c|}
\hline
3.15E-05 & 6.5E-05 & 0.000129 & 4.9E-05 & 4.9E-05 & 7.21E-05 \\
\hline
0.000104 & 9.59E-05 & 4.36E-05 & 7.95E-05 & 6.97E-05 & 9.56E-05 \\
\hline
7.84E-05 & 6.87E-05 & 0.000133 & 6.44E-05 & 6.07E-05 & 3.59E-05 \\
\hline
\end{tabular}
\end{table}

\textbf{②面积比例系数 \( K \) 的结果分析与方差检验}

从表 2 所得结果分析,通过不同相邻数据得到的面积比例系数 \( K \) 不完全相同,存在一定的差异,导致这一问题的原因:图像处理所得到的堰塞湖横向剖面面积不够精确以及求解两侧横向剖面面积所用的反距离加权法存在误差,同时以 \( 1 \, \text{m} \) 的高程为步进值计算堰塞湖对应的体积变化也会带来一定的误差。为此采用面积比例系数的均值 \( E(K) = 6.96 \text{E-05} \) 作为本问题的通用面积比例系数,此时方差为 \( D(K) = 7.93 \text{E-010} \),这个结果是可行的。

\section*{[3] 堰塞湖蓄水量离散化模型}

唐家山堰塞湖正常时水位高程为 709 米,正常蓄水量 0.7 亿立方米,发生地震后水位高程逐渐上涨,极限水位高程不超过 750 米。在表 1 数据基础上研究并建立唐家山堰塞湖蓄水量与水位高程的模型,建模过程如下:

\textbf{符号说明:}

\begin{itemize}
    \item \( h_c \) ——需要研究的水位高程值,求其对应的蓄水量;
    \item \( h_a \) ——数据表 1 中比水位高程 \( h_c \) 大且紧挨 \( h_c \) 的水位高程;
    \item \( h_b \) ——数据表 1 中比水位高程 \( h_c \) 小且紧挨 \( h_c \) 的水位高程;
    \item \( \Delta V_{ac} \) ——\( h_a \)、\( h_c \) 水位高程之间的蓄水量;\( \Delta V_{bc} \) ——\( h_b \)、\( h_c \) 水位高程之间的蓄水量;
\end{itemize}

\textbf{①水位高程为 \( h_c \)、\( h_a \)、\( h_b \) 对应的横向剖面面积 \( S_c \)、\( S_a \)、\( S_b \)}

水位高程 \( h_c \) 时,依据反距离加权法,此时水位高程上下两侧的整数高程值:水位高程为 \( \lfloor h_c \rfloor \) 的横向剖面面积为 \( S_{c1K} K \),水位高程为 \( \lfloor h_c \rfloor + 1 \) 的横向剖面面积为 \( S_{c2K} K \),则水位高程为 \( h_c \) 时的堰塞湖横向剖面面积为

\begin{equation}
S_c = \left( h_c - \lfloor h_c \rfloor \right) S_{c1K} K + \left( \lfloor h_c \rfloor + 1 - h_c \right) S_{c2K} K
\end{equation}

水位高程 \( h_a \) 时,依据反距离加权法,此时水位高程上下两侧的整数高程值:水位高程为 \( \lfloor h_a \rfloor \) 的横向剖面面积为 \( S_{a1K} K \),水位高程为 \( \lfloor h_a \rfloor + 1 \) 的横向剖面面积为 \( S_{a2K} K \),则水位高程为 \( h_a \) 时的堰塞湖横向剖面面积为

\begin{equation}
S_a = \left( h_a - \lfloor h_a \rfloor \right) S_{a1K} K + \left( \lfloor h_a \rfloor + 1 - h_a \right) S_{a2K} K
\end{equation}

水位高程 \( h_b \) 时,依据反距离加权法,此时水位高程上下两侧的整数高程值:水位高程为 \( \lfloor h_b \rfloor \) 的横向剖面面积为 \( S_{b1K} K \),水位高程为 \( \lfloor h_b \rfloor + 1 \) 的横向剖面面积为 \( S_{b2K} K \),则水位高程为 \( h_b \) 时的堰塞湖横向剖面面积为

\begin{equation}
S_b = \left( h_b - \lfloor h_b \rfloor \right) S_{b1K} K + \left( \lfloor h_b \rfloor + 1 - h_b \right) S_{b2K} K
\end{equation}

\textbf{②水位高程 \( h_a \)、\( h_b \) 与 \( h_c \) 之间的蓄水量 \( \Delta V_{ca} \)、\( \Delta V_{bc} \)}

依据一般体积单位的研究,水位高程 \( h_a \) 与 \( h_c \) 之间的蓄水量

\begin{equation}
\Delta V_{ca} = \sum_{i=c}^{a} S_{iK} K \Delta h_i
\end{equation}

等式右边连加项中两侧的数据项 \(\Delta h_a\)、\(\Delta h_c\) 为 \(\Delta h_a = h_a - \lfloor h_a \rfloor\),\(\Delta h_c = \lfloor h_c \rfloor + 1 - h_c\),对应横向剖面面积为①中求解的 \(S_a\)、\(S_c\),中间数据项剖面面积为 \(S_{iK} K\),高程差 \(\Delta h_i\) 为 1。同样依据一般体积单位的研究,水位高程 \(h_b\) 与 \(h_c\) 之间的蓄水量

\begin{equation}
\Delta V_{bc} = \sum_{i=b}^{c} S_{iK} K \Delta h_i
\end{equation}

等式右边连加项中两侧的数据项 \(\Delta h_b\)、\(\Delta h_c\) 为 \(\Delta h_b = \lfloor h_b \rfloor + 1 - h_b\),\(\Delta h_c = h_c - \lfloor h_c \rfloor\),对应横向剖面面积为①中求解的 \(S_b\)、\(S_c\),中间数据项剖面面积为 \(S_{iK} K\),高程差 \(\Delta h_i\) 为 1。

③水位高程为 \(h_c\) 时的蓄水量 \(V_c\) 模型

水位高程为 \(h_c\) 时唐家山堰塞湖的蓄水量 \(V_c\) 为可以通过水位高程为 \(h_a\)、\(h_b\) 已知蓄水量数据 \(V_a\)、\(V_b\) 求得

\begin{equation}
V_{1c} = V_a - \Delta V_{ac}, \quad V_{2c} = V_b + \Delta V_{bc}
\end{equation}

得到水位高程为 \(h_c\) 的两个蓄水量体积 \(V_{1c}\)、\(V_{2c}\),再利用反距离加权法得到 \(V_c\)

\begin{equation}
V_c = V_{1c} \frac{(h_c - h_b)}{(h_a - h_b)} + V_{2c} \frac{(h_a - h_c)}{(h_a - h_b)}
\end{equation}

模型中各变量含义如下:

\begin{equation}
\left\{
\begin{aligned}
V_{1c} &= V_a - \Delta V_{ac} & \text{(1)} \\
V_{2c} &= V_b + \Delta V_{bc} & \text{(2)} \\
\Delta V_{ca} &= \sum_{i=c}^{a} S_{iK} K \Delta h_i & \text{(3)} \\
\Delta V_{bc} &= \sum_{i=b}^{c} S_{iK} K \Delta h_i & \text{(4)} \\
\Delta h_a &= h_a - \lfloor h_a \rfloor & \text{(5)} \\
\Delta h_c &= \lfloor h_c \rfloor + 1 - h_c & \text{(6)} \\
\Delta h_b &= \lfloor h_b \rfloor + 1 - h_b & \text{(7)} \\
\Delta h_c &= h_c - \lfloor h_c \rfloor & \text{(8)} \\
S_a &= \big(h_a - \lfloor h_a \rfloor\big) S_{a1K} K + \big(\lfloor h_a \rfloor + 1 - h_a\big) S_{a2K} K & \text{(9)} \\
S_b &= \big(h_b - \lfloor h_b \rfloor\big) S_{b1K} K + \big(\lfloor h_b \rfloor + 1 - h_b\big) S_{b2K} K & \text{(10)} \\
S_c &= \big(h_c - \lfloor h_c \rfloor\big) S_{c1K} K + \big(\lfloor h_c \rfloor + 1 - h_c\big) S_{c2K} K & \text{(11)}
\end{aligned}
\right.
\end{equation}

模型说明:

(1)(2)——通过水位高程为 \(h_a\)、\(h_b\) 的已知蓄水量数据 \(V_a\)、\(V_b\) 得到 \(h_c\) 的蓄水量;

(3)(4)——水位高程 \(h_a\) 与 \(h_c\) 之间的蓄水量、水位高程 \(h_b\) 与 \(h_c\) 之间的蓄水量;

(5)(6)(7)(8)——高程差;(9)(10)(11)——反距离加权法求解横向剖面面积。

[4]堰塞湖蓄水量离散化模型求解

求解过程中遇到的问题:

根据堰塞湖的结构,水位高程和蓄水量应该存在下列关系

\begin{equation}
h_b < h_c < h_a \Rightarrow V_b < V_c < V_a
\end{equation}

当求解得到的 \( V_c \) 比 \( V_a \) 还大时,求解出现了问题,问题的原因是采用面积比例 \( K \) 的均值作为本问题的通用系数在某些高程上不适用,针对这个问题需要再次运用前文对面积比例系数 \( K \) 的求解方法在水位高程 \( h_a, h_b \) 之间重新测算 \( K \) 值

\begin{equation}
K = \Delta V_{ab} \Bigg/ \sum_{l=b}^{a} S_{lK} \Delta h_l
\end{equation}

得到面积比例系数之后,再按[3]中离散化模型方法求解水位高程为 \( h_c \) 时唐家山堰塞湖的蓄水量 \( V_c \)。

求解结果:

表5.4列出了水位高程709米到750米之间每隔5米的水位高程对应的蓄水量:

\textbf{表5.4 唐家山堰塞湖水位高程(米)与蓄水量(亿立方米)}

\begin{table}[h]
\centering
\begin{tabular}{|c|c|c|c|}
\hline
水位高程 & 蓄水量 & 水位高程 & 蓄水量 \\
\hline
709 & 0.7 & 734 & 1.92073 \\
\hline
714 & 0.856821 & 739 & 2.203332 \\
\hline
719 & 1.110504 & 744 & 2.561075 \\
\hline
724 & 1.492407 & 749 & 2.968761 \\
\hline
729 & 1.700368 & 750 & 3.056095 \\
\hline
\end{tabular}
\end{table}

通过表5.4结果可以看出:其中750m高程对应的蓄水量为3.05亿立方米,可以看出模型的求解精度还是比较高的。

在下文的实际求解中,可以事先计算出相隔0.5米高程对应的蓄水量,对于具体要求的高程对应的蓄水量插值求解运算即可。

\subsection*{5.2 降雨量预测水位增长模型}

唐家山堰塞湖蓄水量主要来源有两种:一是每日正常入库的水流量,二是降雨后雨水入库的水流量。研究降雨量对唐家山堰塞湖水位的影响主要体现在唐家山堰塞湖蓄水量的改变,但降雨对堰塞湖蓄水量的影响内在关系较为复杂,唐家山堰塞湖蓄水量的3500平方公里的影响区域能够影响堰塞湖蓄水量但降雨区域不够明确,且降雨对蓄水量的影响时间也不确定,存在时间延迟。

在研究降雨量与水库水位高程的变化之时,认为降雨量与堰塞湖蓄水量的变化的相关性更为紧密,因此我们将建立模型重点研究降雨量变化与水库蓄水量变化的关系,在得出这一关系之后,利用第一问的模型就可以求解得出降雨量引起的每日湖水水位的变化。同时,考虑到降雨引起的径流是十分复杂的,在研究某日水库蓄水量变化之时,会同时此日前三天的降雨量。

针对上述问题,下面建立神经网络模型和多元线性拟合模型对降雨量与唐家山蓄水量的关系进行求解。

\subsubsection{5.2.1 数据提取与分析}

以天气预报降雨量的不同百分比为基本数据研究5月25日至6月12日每日上升的水位,5月25日之前没有已知数据可以用作参考,为此将预测期间实际天气预报和堰塞湖水位的改变量作为历史数据进行研究。

将天气预报100\%雨量为真实数据对唐家山堰塞湖水位上升的影响作为真实数据,对天气预报降雨量不同百分比的情形对唐家山水位的变化进行预测。

提取的数据:唐家山在北川区域内,其他县市距离唐家山堰塞湖较远不属于唐家山雨量聚集区域3500平方公里的范围内,为此以北川天气预报数据作为研究根据。

\begin{table}
\centering
\caption{北川72小时内降雨量(毫米)与唐家山堰塞湖蓄水量(亿立方米)}
\begin{tabular}{|c|c|c|c|c|}
\hline
 & \multicolumn{3}{c|}{预报降雨量(毫米)} & \\ \hline
日期 & 前48$\rightarrow$72小时 & 前24$\rightarrow$48小时 & 近24小时以内 & 蓄水量变化 \\ \hline
5.28$\rightarrow$5.29 & 26.5 & 0 & 0 & 0.0515 \\ \hline
6.1$\rightarrow$6.2 & 0 & 0 & 0.5 & 0.0502 \\ \hline
6.2$\rightarrow$6.3 & 0 & 0 & 0.5 & 0.051 \\ \hline
6.3$\rightarrow$6.4 & 0.5 & 0 & 20.5 & 0.046 \\ \hline
6.4$\rightarrow$6.5 & 0.5 & 0 & 20.5 & 0.063 \\ \hline
6.6$\rightarrow$6.7 & 0 & 20.5 & 14.7 & 0.101 \\ \hline
6.8$\rightarrow$6.9 & 20.5 & 14.7 & 1 & 0.0447 \\ \hline
\end{tabular}
\end{table}

\subsection{5.2.2 神经网络模型}

神经网络模型适合研究内部规律相互影响不明确的问题,降雨量对唐家山堰塞湖水位的影响内部规律相当复杂,由于缺乏历史数据和现场数据的有力支持,不能确定区域能够影响唐家山水位的降雨区域以及一次降雨完成对唐家山水位的影响周期,所以对水位高程变化的预测采用神经网络模型是合理的。

降雨量对唐家山堰塞湖水位的影响具有一定的延迟性,因为降雨后雨水需要一个汇聚的过程,可能当天的降雨需要经过多天才能完成对唐家山堰塞湖水位的影响周期,考虑将前3天即前72小时的降雨量对当天水位的增长作为输入量。

降雨量对堰塞湖水位的影响直观上是对唐家山堰塞湖蓄水量的影响,不同水位高程上降雨量对堰塞湖水位的影响不同,但堰塞湖蓄水量的改变是一定的,所以将蓄水量的改变作为神经网络模型的输出量。

\subsubsection{[1]神经网络模型的建立}

以表5.5中北川天气预报前24小时的雨量、前24至48小时的雨量以及前48至72小时的雨量为输入量,以唐家山堰塞湖蓄水量变化量为输出量,建立神经网络模型如图:

\begin{figure}[h]
\centering
\includegraphics[width=0.8\textwidth]{neural_network_diagram.png}
\caption{神经网络模型图}
\end{figure}

本文选用BP神经网络中较为成熟的批处理最速下降法(traingd)进行预测,具体步骤如下所示:

\textbf{Step1:设定训练次数}

由于题中所给数据较少,本文设定学习次数为5000次。

\textbf{Step2:选取合适的学习速率}

学习速率越大,步进越大。如果学习速率太大算法就会变得不稳定。如果学习速率

\begin{table}
\centering
\begin{tabular}{|c|c|c|c|c|}
\hline
 & \multicolumn{4}{c|}{每日蓄水量(亿立方米)的变化} \\
\hline
日期 & 预报降雨的50\% & 预报降雨的80\% & 预报降雨的100\% & 预报降雨的150\% \\
\hline
5.25$\rightarrow$5.26 & 0.059162921 & 0.059545327 & 0.059800265 & 0.060437609 \\
\hline
5.26$\rightarrow$5.27 & 0.071812152 & 0.079784098 & 0.085098729 & 0.098385305 \\
\hline
5.27$\rightarrow$5.28 & 0.077759433 & 0.089299748 & 0.096993291 & 0.116227148 \\
\hline
5.28$\rightarrow$5.29 & 0.062563333 & 0.064985988 & 0.066601091 & 0.070638848 \\
\hline
5.29$\rightarrow$5.30 & 0.058634391 & 0.05869968 & 0.058743206 & 0.058852021 \\
\hline
5.30$\rightarrow$5.31 & 0.060653353 & 0.061930019 & 0.062781129 & 0.064908906 \\
\hline
5.31$\rightarrow$6.1 & 0.059058865 & 0.059378838 & 0.059592153 & 0.060125442 \\
\hline
6.1$\rightarrow$6.2 & 0.058541121 & 0.058550448 & 0.058556666 & 0.058572211 \\
\hline
6.2$\rightarrow$6.3 & 0.058541121 & 0.058550448 & 0.058556666 & 0.058572211 \\
\hline
6.3$\rightarrow$6.4 & 0.058845089 & 0.059036797 & 0.059164602 & 0.059484115 \\
\hline
6.4$\rightarrow$6.5 & 0.059239105 & 0.059667222 & 0.059952633 & 0.060666162 \\
\hline
6.5$\rightarrow$6.6 & 0.059239105 & 0.059667222 & 0.059952633 & 0.060666162 \\
\hline
6.6$\rightarrow$6.7 & 0.071445291 & 0.07919712 & 0.084365005 & 0.09728472 \\
\hline
6.7$\rightarrow$6.8 & 0.070616876 & 0.077871656 & 0.082708177 & 0.094799477 \\
\hline
6.8$\rightarrow$6.9 & 0.061373325 & 0.063081974 & 0.064221074 & 0.067068822 \\
\hline
6.9$\rightarrow$6.10 & 0.058677944 & 0.058769365 & 0.058830312 & 0.058982681 \\
\hline
6.10$\rightarrow$6.11 & 0.058525576 & 0.058525576 & 0.058525576 & 0.058525576 \\
\hline
6.11$\rightarrow$6.12 & 0.058525576 & 0.058525576 & 0.058525576 & 0.058525576 \\
\hline
\end{tabular}
\caption{表5.6拟合模型预测的每日蓄水量(亿立方米)的变化}
\end{table}

\begin{table}
\centering
\begin{tabular}{|c|c|c|c|c|}
\hline
日期 & 预报降雨的50\% & 预报降雨的80\% & 预报降雨的100\% & 预报降雨的150\% \\
\hline
5.25$\rightarrow$5.26 & 1.207134235 & 1.236099775 & 1.261618751 & 1.329131307 \\
\hline
5.26$\rightarrow$5.27 & 1.542409639 & 1.724014252 & 1.816727165 & 1.972922878 \\
\hline
5.27$\rightarrow$5.28 & 1.39559915 & 1.472982177 & 1.484348272 & 1.614763175 \\
\hline
5.28$\rightarrow$5.29 & 1.140672234 & 1.128888588 & 1.167029689 & 1.455812602 \\
\hline
5.29$\rightarrow$5.30 & 1.126419844 & 1.088371834 & 1.070008976 & 1.014136739 \\
\hline
5.30$\rightarrow$5.31 & 1.208019252 & 1.21810875 & 1.226101099 & 1.213193861 \\
\hline
5.31$\rightarrow$6.1 & 1.143046197 & 1.131638836 & 1.091081993 & 1.073734986 \\
\hline
6.1$\rightarrow$6.2 & 1.111303875 & 1.068914924 & 1.062946846 & 1.04885115 \\
\hline
6.2$\rightarrow$6.3 & 1.078237073 & 1.035681277 & 1.029898753 & 1.016241307 \\
\hline
6.3$\rightarrow$6.4 & 1.0461013 & 1.008267925 & 1.003775888 & 0.993993652 \\
\hline
6.4$\rightarrow$6.5 & 0.855105386 & 0.850114483 & 0.865898151 & 0.908421286 \\
\hline
6.5$\rightarrow$6.6 & 0.833156157 & 0.850114483 & 0.847320535 & 0.888931347 \\
\hline
6.6$\rightarrow$6.7 & 1.069627396 & 1.188923975 & 1.25283224 & 1.341278036 \\
\hline
6.7$\rightarrow$6.8 & 0.883962617 & 0.838361649 & 0.825268902 & 0.806856446 \\
\hline
6.8$\rightarrow$6.9 & 0.881528142 & 0.84723946 & 0.804260735 & 0.770893198 \\
\hline
6.9$\rightarrow$6.10 & 0.914923572 & 0.912255283 & 0.910484692 & 0.884422429 \\
\hline
6.10$\rightarrow$6.11 & 0.897417183 & 0.897417183 & 0.897417183 & 0.882228176 \\
\hline
\end{tabular}
\caption{神经网络模型预测的每日水位上升高度(米)}
\end{table}

从表5.7可以看出雨量大时堰塞湖水位日增长幅度更大,符合实际。

\begin{figure}[h]
\centering
\includegraphics[width=\textwidth]{image.png} % 替换为实际图像文件名
\caption{神经网络所得每日堰塞湖水位}
\end{figure}

【注】图中水位曲线从下往上依次是天气预报50\%、80\%、100\%、150\%的降雨量

从拟合效果图来看,不同程度降雨量对堰塞湖水位的影响程度不同,很明显降雨量多堰塞湖水位高,降雨量少堰塞湖水位低些,符合实际情况。

\subsection*{5.2.3 多元线性拟合模型}

以表5.5中北川天气预报的8组数据即前24小时的雨量、前24至前48小时的雨量以及前48至前72小时雨量对唐家山堰塞湖蓄水量的关系建立多元线性拟合模型

\[ y = \beta_0 + \beta_1 x_1 + \beta_2 x_2 + \beta_3 x_3 \]

\( x_1 \) -- 表5.5中北川天气预报前48小时至72小时的雨量;

\( x_2 \) -- 表5.5中北川天气预报前24小时至48小时的雨量;

\( x_3 \) -- 表5.5中北川天气预报近24小时内的降雨量;

\( y \) -- 当日唐家山堰塞湖蓄水量变化值。

\begin{table}
\centering
\begin{tabular}{|c|c|c|c|c|}
\hline
 & \multicolumn{4}{|c|}{每日蓄水量(亿立方米)的变化值} \\
\hline
日期 & 预报降雨的50\% & 预报降雨的80\% & 预报降雨的100\% & 预报降雨的150\% \\
\hline
5.25$\rightarrow$5.26 & 0.059162921 & 0.059545327 & 0.059800265 & 0.060437609 \\
\hline
5.26$\rightarrow$5.27 & 0.071812152 & 0.079784098 & 0.085098729 & 0.098385305 \\
\hline
5.27$\rightarrow$5.28 & 0.077759433 & 0.089299748 & 0.096993291 & 0.116227148 \\
\hline
5.28$\rightarrow$5.29 & 0.062563333 & 0.064985988 & 0.066601091 & 0.070638848 \\
\hline
5.29$\rightarrow$5.30 & 0.058634391 & 0.05869968 & 0.058743206 & 0.058852021 \\
\hline
5.30$\rightarrow$5.31 & 0.060653353 & 0.061930019 & 0.062781129 & 0.064908906 \\
\hline
5.31$\rightarrow$6.1 & 0.059058865 & 0.059378838 & 0.059592153 & 0.060125442 \\
\hline
6.1$\rightarrow$6.2 & 0.058541121 & 0.058550448 & 0.058556666 & 0.058572211 \\
\hline
6.2$\rightarrow$6.3 & 0.058541121 & 0.058550448 & 0.058556666 & 0.058572211 \\
\hline
6.3$\rightarrow$6.4 & 0.058845089 & 0.059036797 & 0.059164602 & 0.059484115 \\
\hline
6.4$\rightarrow$6.5 & 0.059239105 & 0.059667222 & 0.059952633 & 0.060666162 \\
\hline
6.5$\rightarrow$6.6 & 0.059239105 & 0.059667222 & 0.059952633 & 0.060666162 \\
\hline
6.6$\rightarrow$6.7 & 0.071445291 & 0.07919712 & 0.084365005 & 0.09728472 \\
\hline
6.7$\rightarrow$6.8 & 0.070616876 & 0.077871656 & 0.082708177 & 0.094799477 \\
\hline
6.8$\rightarrow$6.9 & 0.061373325 & 0.063081974 & 0.064221074 & 0.067068822 \\
\hline
6.9$\rightarrow$6.10 & 0.058677944 & 0.058769365 & 0.058830312 & 0.058982681 \\
\hline
6.10$\rightarrow$6.11 & 0.058525576 & 0.058525576 & 0.058525576 & 0.058525576 \\
\hline
6.11$\rightarrow$6.12 & 0.058525576 & 0.058525576 & 0.058525576 & 0.058525576 \\
\hline
\end{tabular}
\end{table}

\begin{table}
\centering
\begin{tabular}{|c|c|c|c|c|}
\hline
 & \multicolumn{4}{|c|}{每日上升的高度} \\
\hline
日期 & 预报降雨的50\% & 预报降雨的80\% & 预报降雨的100\% & 预报降雨的150\% \\
\hline
5.25$\rightarrow$5.26 & 1.308763606 & 1.317222957 & 1.322862525 & 1.336961445 \\
\hline
5.26$\rightarrow$5.27 & 1.536535815 & 1.707108341 & 1.820823359 & 2.105110903 \\
\hline
5.27$\rightarrow$5.28 & 1.619178608 & 1.804560274 & 1.960030612 & 2.348706456 \\
\hline
5.28$\rightarrow$5.29 & 1.217194947 & 1.264328668 & 1.25414446 & 1.330178215 \\
\hline
5.29$\rightarrow$5.30 & 1.104126011 & 1.076434213 & 1.077232392 & 1.04780051 \\
\hline
5.30$\rightarrow$5.31 & 1.112260648 & 1.102601141 & 1.117754303 & 1.121586562 \\
\hline
5.31$\rightarrow$6.1 & 1.051483157 & 1.026030336 & 1.029716298 & 1.008017848 \\
\hline
6.1$\rightarrow$6.2 & 1.011555095 & 0.981612684 & 0.98171693 & 0.95144687 \\
\hline
6.2$\rightarrow$6.3 & 0.981456314 & 0.951093352 & 0.951194357 & 0.923770314 \\
\hline
6.3$\rightarrow$6.4 & 0.955879501 & 0.931097522 & 0.933113199 & 0.914071476 \\
\hline
6.4$\rightarrow$6.5 & 0.934288212 & 0.916885208 & 0.921271025 & 0.9122347 \\
\hline
6.5$\rightarrow$6.6 & 0.910306482 & 0.897213678 & 0.901505399 & 0.9122347 \\
\hline
6.6$\rightarrow$6.7 & 1.07432004 & 1.190883986 & 1.24641736 & 1.437294568 \\
\hline
6.7$\rightarrow$6.8 & 1.043301071 & 1.127868587 & 1.197919223 & 1.334700095 \\
\hline
6.8$\rightarrow$6.9 & 0.88891194 & 0.888143266 & 0.904180867 & 0.921695366 \\
\hline
6.9$\rightarrow$6.10 & 0.826138081 & 0.807639819 & 0.808477389 & 0.796852194 \\
\hline
6.10$\rightarrow$6.11 & 0.80428954 & 0.80428954 & 0.790676741 & 0.77465197 \\
\hline
6.11$\rightarrow$6.12 & 0.790676741 & 0.790676741 & 0.77465197 & 0.75926385 \\
\hline
\end{tabular}
\end{table}

\begin{figure}[h]
    \centering
    \includegraphics[width=\textwidth]{image.png}
    \caption{多元线性拟合所得每日堰塞湖水位}
    \label{fig:5.10}
\end{figure}

【注】图中水位曲线从下往上依次是天气预报50\%、80\%、100\%、150\%的降雨量。从拟合效果图来看,不同程度降雨量对堰塞湖水位的影响程度不同,很明显降雨量多堰塞湖水位高,降雨量少堰塞湖水位低些,符合实际情况。

\subsection{5.2.4 两种模型的检验}

分别将两个模型求得的水位高度差转化为相应水位高程值,与由附件得到的真实数据相比较后,求解得到绝对误差与相对误差,并求解得到均值和方差,结果如表\ref{tab:5.10}所示。

\begin{table}[h]
    \centering
    \caption{神经网络模型与多元线性拟合模型预测的效果分析}
    \label{tab:5.10}
    \begin{tabular}{c c c c c}
        \hline
        & \multicolumn{2}{c}{BP神经网络} & \multicolumn{2}{c}{多元线性拟合} \\
        \hline
        \multirow{2}{*}{绝对误差} & 均值 & 方差 & 均值 & 方差 \\
        & -2.1308 & 0.8339 & -1.8484 & 0.4068 \\
        \hline
        \multirow{2}{*}{相对误差} & 均值 & 方差 & 均值 & 方差 \\
        & -0.0029 & 0.000002 & -0.0025 & 0.0000007 \\
        \hline
    \end{tabular}
\end{table}

从神经网络模型与多元线性拟合模型均值方差来看,多元线性拟合的预测结果相对好一些,说明神经网络对于数据量较少的情况“学习”效果不好,这主要因为所能获取的真实数据较少的原因造成的。预测效果不如多元线性拟合模型。

\section{6 问题二:溃坝模型}

唐家山堰塞湖水位高程突破750米漫顶时由于水流作用发生溃坝的可能性很大,鉴于唐家山堰塞湖的特殊性,堰塞体长达800多米,与人工坝构造相差极大。为此在泄洪问题的基础上研究溃坝问题是科学的也是必要的。

\subsection{6.1 泄洪过程的研究}

[1] 泄洪过程中水流量和水位高程的研究

从附件中收集唐家山堰塞湖泄洪时的有关数据:流量数据、水位高程数据、流速数据、溃口形状数据(溃口宽度、深度)和槽内水深数据,并对流量数据和水位高程数据进行插值,得到0.5小时为一个间隔的数据,得到泄洪过程,见下图6.1、图6.2。

\begin{figure}[h]
    \centering
    \includegraphics[width=\textwidth]{image.png}
    \caption{泄洪过程图6.1}
    \label{fig:6.1}
\end{figure}

\begin{figure}[h]
    \centering
    \includegraphics[width=\textwidth]{image.png}
    \caption{泄洪过程图6.2}
    \label{fig:6.2}
\end{figure}

\begin{figure}[h]
    \centering
    \includegraphics[width=\textwidth]{image1.png}
    \caption{泄洪过程中水流量变化规律}
    \label{fig:6.1}
\end{figure}

\begin{figure}[h]
    \centering
    \includegraphics[width=\textwidth]{image2.png}
    \caption{泄洪过程中水位高程变化规律}
    \label{fig:6.2}
\end{figure}

从上面两图分析可得,泄洪过程大体可以分为 3 个过程:泄洪前期、泄洪期以及泄洪末期。泄洪前期从唐家山堰塞湖水面高程 742 米达到过流值时开始泄流过程,此时流量较小,溃口也小,随着时间的推移,溃口突然扩大,水流量随即增长,此时进入了真正的泄洪期。水流量的增加进一步导致了溃口的扩大,水位高程迅速降低,流量逐渐扩大达到最大值,此时为泄洪洪峰值,随后流量逐渐降低,直至到达泄洪末期,流量稳定下来,水位高程降至 714 米附近。

下面对流量数据和水位高程数据的可信性进行验证,对流量积分得到泄洪过程中唐家山堰塞湖的泄洪量,对水位高程插值得到泄洪过程中的泄洪量,两者相互比较验证数据的可信性,所得结果如下图:

\begin{figure}[h]
    \centering
    \includegraphics[width=\textwidth]{image3.png}
    \caption{水流量积分与水位高程插值所得泄洪量规律图}
    \label{fig:6.3}
\end{figure}

【注】图中光滑曲线为水流量对时间的积分所得泄洪量规律,粗糙曲线为水位高程插值得到泄洪量规律。

图 \ref{fig:6.3} 中两条泄洪量总体变化趋势相似,体现了泄洪的 3 个过程,泄洪初期泄洪较缓慢,泄洪量较小,泄洪期泄洪量迅速增大,泄洪末期泄洪量较小。

从图 \ref{fig:6.3} 中水位高程插值结果来看,曲线较粗糙,末期泄洪量仍在迅速增大,而水流量积分所得泄洪量总体较光滑,但泄洪量较真实值要小,两者都说明所得数据存在一定的误差。

\subsection{泄洪过程溃口的研究}

泄洪过程中,溃口的形成由两种因素造成,一是水流冲刷的结果,二是溃口自身由于各种因素坍塌造成,这里先只研究水流量对溃口形状的影响,并假定溃口宽度和深度。

只与水流量大小有关,同时假定溃口为倒三角形。假设水流量最大时溃口达到最大值,之前溃口与水流量成正比例关系:
\[
\frac{w}{w_{\max}} = \left( \frac{q}{q_{\max}} \right)^x, \quad \frac{g}{g_{\max}} = \left( \frac{q}{q_{\max}} \right)^x
\]
\( w \)-- 溃口宽度值;\( g \)-- 溃口深度;  
\( q \)-- 水流量大小;\( x \)-- 比例系数取值(\( 1/3 \),\( 0.5 \),\( 1 \))。  
其中 \( w_{\max} \) 为溃口宽度最大值 132 米,\( g_{\max} \) 溃口深度最大值 29.5 米,\( q_{\max} \) 为泄洪流量最大值 \( 6680 \, m^3/s \)。

取不同比例系数得到的不同的溃口宽度和深度的变化曲线,通过与河南鸭口溃坝溃口模型对比检验,比例系数最贴近的值是 0.5,以 0.5 为比例系数确定溃口变化规律:
\[
w = 132 \left( \frac{q}{6680} \right)^{0.5}, \quad g = 29.5 \left( \frac{q}{6680} \right)^{0.5}
\]
得到溃口宽度、深度与时间的变化曲线规律如图 6.4、图 6.5:

\begin{figure}[h]
    \centering
    \includegraphics[width=0.45\textwidth]{image1.png}
    \caption{溃口宽度变化曲线}
    \label{fig:6.4}
\end{figure}
\begin{figure}[h]
    \centering
    \includegraphics[width=0.45\textwidth]{image2.png}
    \caption{溃口深度变化曲线}
    \label{fig:6.5}
\end{figure}

[3] 溃口中水深、流速的研究

溃口深度是从水位最高处开始测算的,因此溃口中水深 \( f \) 可以表示为
\[
f = h - (h_{\max} - g)
\]
\( f \)-- 溃口中水深;\( h \)-- 当前水位高程;\( h_{\max} \)-- 最大水位高程;\( g \)-- 溃口深度。  
溃口处流速可以根据流量和水面截面积的关系求得
\[
v = 2qg / f^2 w_{\max}, \quad q \)-- 堰塞湖泄洪过程流量变化规律
\]
得到的溃口水深、流速与时间变化规律如图 6.6、图 6.7

\begin{figure}[h]
    \centering
    \includegraphics[width=0.45\textwidth]{image3.png}
    \caption{溃口流速的变化规律}
    \label{fig:6.6}
\end{figure}
\begin{figure}[h]
    \centering
    \includegraphics[width=0.45\textwidth]{image4.png}
    \caption{溃口水深的变化规律}
    \label{fig:6.7}
\end{figure}

[4] 流量与时间关系的正交最小二乘模型

上文的研究较为真实的反映了泄洪过程中各变量与时间的变化关系,而且可以看

出,其他变量(速度,溃口宽度等)都可以通过流量与时间的变化关系表示出来,因此如果能够较为清晰的认识 到这一过程中流量与时间的变化关系就能极大地方便后续对这一问题的研究。

对于时间与流量的关系,多年来学者提出了多种数学模型来解释,但是这些模型往往十分复杂,需要提前测定很多参数,而且鲜有较高的可靠性,这里采用函数逼近的方法获取流量与时间的变化关系。

已知的流量值是一些函数关系未知的离散点,要想较为准确地逼近这些离散点,较为可行的逼近方法是正交多项式最小二乘拟合。

\textbf{①正交多项式最小二乘拟合及其逼近算法}

拟合多项式设为
\[
q(t)=b_{0} B_{0}(t)+b_{1} B_{1}(t)+\cdots+b_{m} B_{m}(t)
\]
其中, \( q \) 为洪水流量, \( t \) 为时间,将已知的 \( q-t \) 离散点值代入上式中求解,其中
\[
b_{j}=\frac{\sum_{i=0}^{n} y_{i} B_{j}\left(t_{i}\right)}{\sum_{i=0}^{n} B_{j}^{2}\left(t_{i}\right)}
\]
构造基函数如下:
\[
\begin{cases}
B_{0}(t)=1 \\
B_{1}(t)=t-\alpha_{0} \\
B_{i+1}(t)=\left(t-\alpha_{i}\right) B_{i}(t)
\end{cases}
\]
其中
\[
\alpha_{i}=\frac{\sum_{k=0}^{n} t_{k} B_{i}^{2}\left(t_{k}\right)}{\sum_{k=0}^{n} B_{i}^{2}\left(t_{k}\right)}, \quad \beta_{i}=\frac{\sum_{k=0}^{n} B_{i}^{2}\left(t_{k}\right)}{\sum_{k=0}^{n} B_{i-1}^{2}\left(t_{k}\right)}
\]
其逼近算法如下:

\textbf{Step1:} 确定需要逼近的正交函数的最高次项,令 \( B_{0}(t)=1 \),则有 \( a_{0}=b_{0} \);

\textbf{Step2:} 令 \( B_{1}(t)=c_{0}+c_{1} t \),由递推公式有 \( c_{0}=-\alpha_{0}, \quad c_{1}=1 \),利用 \( \alpha_{i}, \beta_{i} \) 的计算式推导 \( \alpha_{1}, \beta_{1} \), \( b_{1}=\left(\sum_{k=0}^{n} y_{k} B_{1}\left(t_{k}\right)\right) /\left(\sum_{k=0}^{n} B_{1}^{2}\left(t_{k}\right)\right) \),更新逼近多项式系数 \( a_{0}=a_{0}+b_{1} c_{0}, a_{1}=b_{1} c_{1} \)。

\textbf{Step3:} 对于 \( i=2,3, \cdots, m \),设 \( B_{t}(t)=r_{0}+r_{1} t+\cdots r_{u} t^{u}, B_{t-1}(t)=s_{0}+s_{1} t+\cdots s_{u-1} t^{u-1} \), \( B_{u-2}(t)=w_{0}+w_{1} t+\cdots w_{u-2} t^{u-2} \),由递推公式可以推出
\[
\Rightarrow\left\{\begin{array}{l}
r_{u}=s_{u-1} \\
r_{u-1}=-\alpha_{u-1} s_{u-1}+s_{u-2} \\
r_{u}=-\alpha_{u-1} s_{i}+s_{i-1}-\beta_{u-1} w_{i}(i=1,2, \cdots, t-2) \\
r_{0}=-\alpha_{u-1} s_{0}-\alpha_{u-1} w_{0}
\end{array}\right.
\]
因此可以更新逼近函数
\[
\left\{\begin{array}{l}
a_{k}=a_{k}+b_{u} r_{k}, 0 \leq k \leq u-1 \\
a_{u}=b_{u} r_{u}
\end{array}\right.
\]

\textbf{Step4:} 循环执行 Step3,完成函数逼近。

\textbf{②对流量与时间函数关系的逼近}

从 \( q-t \) 的关系图上可以看出,其整个走势变化还是比较复杂的,直接进行正交函数逼近无法得到较好的结果,因此采用分段函数正交最小二乘逼近,利用逼近算法经过反

复试验,得到如下的正交逼近函数,函数共分为三段,第一、第三段采用三次逼近,第二段采用七次逼近:
\[
\begin{cases}
q = B_{10} - B_{11}t + B_{12}t^2 - B_{13}t^3, 0 \leq t \leq 25 \\
q = B_{20} - B_{21}t + B_{22}t^2 - B_{23}t^3 + B_{24}t^4 - B_{25}t^5 + B_{26}t^6 - B_{27}t^7, 25.5 \leq t \leq 34.5 \\
q = B_{30} - B_{31}t + B_{32}t^2 - B_{33}t^3, 35 \leq t \leq 47
\end{cases}
\]
其中各系数的值见表6.1

\begin{table}[h]
\centering
\caption{泄洪模型各系数结果}
\begin{tabular}{|c|c|c|c|c|c|c|c|c|}
\hline
 & $B_{y0}$ & $B_{y1}$ & $B_{y2}$ & $B_{y3}$ & $B_{y4}$ & $B_{y5}$ & $B_{y6}$ & $B_{y7}$ \\ \hline
$B_{1x}$ & -81.02 & 30.3 & -1.633 & 0.0273 & —— & —— & —— & —— \\ \hline
$B_{2x}$ & 4.20E+09 & -4.90E+08 & 2.40E+07 & -6.70E+05 & 1.10E+04 & -1.10E+02 & 6.02E-01 & -1.40E-03 \\ \hline
$B_{3x}$ & 3.00E+04 & -967.313 & 10.3355 & -0.0366 & —— & —— & —— & —— \\ \hline
\end{tabular}
\end{table}

将函数逼近得到的 $q-t$ 图与原始数据的 $q-t$ 图进行对比如下图所示;

\begin{figure}[h]
\centering
\includegraphics[width=\textwidth]{image.png}
\caption{正交最小二乘拟合的 $q-t$ 曲线与原始数据 $q-t$ 曲线}
\end{figure}

【注】光滑曲线为正交最小二乘拟合 $q-t$ 曲线

可以看出函数逼近的效果较好,基本符合实测数据的走势,以 0.5 小时为步进测得平均相对误差为 10.4%,而对两个图像的积分值均为 1.29 亿立方米,可以看出采用正交多项式最小二乘拟合进行函数逼近是比较正确的。

\subsection{溃坝过程的研究}

\subsubsection{溃坝与泄洪的综合分析}

唐家山溃坝过程研究的是水位高程到达 750 米漫顶后形成的溃坝问题,在 6.1 研究泄洪过程基础之上,首先找出溃坝过程和泄洪过程的相同点和不同点。

\textbf{相同点:}

溃坝和导流槽泄洪的溃口都是一个逐渐形成的过程,水从溃口处逐渐外泄;

溃坝和泄洪都是针对唐家山堰塞湖的问题,内部有相通性;  
泄洪过程一旦开始实际上就是一个溃坝过程不受人为因素控制。  
不同点:  
- 溃坝是在水面高程 750 米时开始的,泄洪是在 743 米时开始的;  
- 溃坝是被动的没有人为因素参与,泄洪有人为因素的参与。  

通过相同点和不同点的分析,泄洪过程实际上是一个 743 米的溃坝过程,因此泄洪过程可以作为溃坝过程的借鉴参考。  

\subsection*{6.2.2 溃坝过程水流量研究}

唐家山堰塞湖溃坝过程是在水位高程到达 750 米时,水位接近堰塞体上边缘,开始时缓慢泄流,水流量较小,此时溃口很微小,随着水流对溃口的侵蚀,溃口变大,溃坝水流量加大,当水流量达到最大值时,溃口也基本稳定了,最后将水流量逐渐稳定下来。因而溃坝过程也可以分为 3 个过程:溃坝前期、溃坝期以及溃坝后期。

通过上述分析,唐家山堰塞湖溃坝过程和泄洪过程从机理上分析是类似的,都由 3 个过程组成。下面引入 2 个基本假设:  
(1) 溃坝后水位高程基本稳定在 713 米,与泄洪后的水位高程是基本一致的;  
(2) 溃坝水位高程 750 米对应的堰塞体构造与泄洪 743 米对应的堰塞体构造是一样的,因而溃口处沙石泥土等内部结构是一样的。

\textbf{① 溃坝前期分析}

溃坝过程与泄洪过程都由 3 个过程组成,其中溃坝前期和泄洪前期的溃口都较小且缓慢扩大,水流量很小,由假设(2)750 米与 743 米堰塞体构造相同的,溃口的形成过程也应该是类似的,那么可以认为溃坝前期和泄洪前期是基本相同的。

\textbf{② 溃坝期分析}

溃坝发生在 750 米的水位高程,此时堰塞湖蓄水量为 3.06 亿立方米,较 743 米水位高程时堰塞湖的蓄水量多 0.65 亿立方米,由假设(1)溃坝后的水位高程与泄洪后的水位高程相同,溃坝比泄洪多泄 0.65 亿立方米,因而溃坝的水流量要比泄洪大。

\textbf{③ 溃坝后期分析}

溃坝后期与泄洪后期的过程都是堰塞湖水位与外界水位逐渐持平的过程,湖内水位不足以进一步切割溃口,水流量基本稳定下来,这个阶段对于溃坝和泄洪来说也是基本类似的。

\textbf{④ 溃坝水流量仿真模型}

溃坝与泄洪 3 个过程中只有溃坝期和泄洪期存在较大差别,溃坝过程要比泄洪过程多泄 0.65 亿立方米的洪水,由前面对泄洪过程和溃坝过程的研究,导流槽泄洪过程可以看成是一个缩小版的溃坝过程,因此溃坝规律可以在泄洪规律基础上放大得到。

研究图 12 中泄洪水流量变化规律,从泄洪开始后第 24 小时至第 25 小时到达峰值的过程中水流量几乎成直线增长,为叙述方便记这一过程为洪峰过程,由于溃坝过程较泄洪过程规模要大,所以溃坝的洪峰过程时间更长,到达的水流量峰值更大。

为此给出两个假设条件:  
- 溃坝开始后第 24 小时洪峰过程中水流量变化规律趋近于泄洪模型;  
- 溃坝和泄洪过程水流量变化曲线的形状是相同的。

泄洪开始后第 24 小时的水流量为 $2190 \, m^3/s$,水流量增长率 $0.6160 \, m^3/s$,第 24 小时至第 25 小时之间泄洪量 0.32 亿立方米,占整个泄洪量的 20%。由两个水流量变化曲线的相似性,溃坝洪峰过程的泄洪量也占整个泄洪总量的 20%,即 0.45 亿立方米。

建立离散变化系统仿真模型:  
溃坝洪峰过程泄洪量比泄洪洪峰过程泄洪量多 0.13 亿立方米

\begin{align*}
\Delta V &= V_2 - V_1 = 0.13
\end{align*}

\(V_2\) -- 溃坝洪峰过程泄洪量,\(V_1\) -- 泄洪洪峰过程泄洪量。

将溃坝洪峰过程离散成 \(N\) 段间隔为 1 秒泄洪过程,水流量 \(q\) 呈直线增长 \(0.6160 \, m^3/s\)

\[
q_{i+1} = q_i + 0.6160, \, (1 \leq i \leq N-1)
\]

\(q_{i+1}\) -- 第 \(i+1\) 段泄洪对应的水流量,\(q_i\) -- 第 \(i\) 段泄洪对应的水流量。

溃坝洪峰过程中第 \(i\) 段剩余泄洪量 \(V'_i\) 为第 \(i+1\) 段剩余泄洪量 \(V'_{i+1}\) 与 \(i+1\) 段水流量 \(q_{i+1}\) 之和

\[
V'_i = V'_{i+1} + q_{i+1}
\]

总泄洪量达到溃坝洪峰过程的泄洪总量满足

\[
V'_2 > 0, \, V'_1 < 0
\]

通过仿真得到溃坝过程水流量的变化规律如图 6.10 所示:

\begin{figure}[h]
    \centering
    \includegraphics[width=\textwidth]{image.png}
    \caption{泄洪过程水流量变化模型}
    \label{fig:6.9}
\end{figure}

\begin{figure}[h]
    \centering
    \includegraphics[width=\textwidth]{image.png}
    \caption{溃坝过程水流量仿真模型}
    \label{fig:6.10}
\end{figure}

图 6.10 中溃坝过程水流量的变化曲线总体形状与泄洪模型类似,水流量峰值 \(8818 \, m^3/s\),峰值时间为溃坝开始后 26 小时。

\section*{⑤ 溃坝水流量正交最小二乘模型}

溃坝过程和之前分析的泄洪过程其走势是基本一致的,假设这两个过程的时间已知,因此建立可以通过积分,确定溃坝时的 \(q-t\) 关系:

\[
\Delta V = \int_{0}^{45} q(t) \, dt
\]

其中 \(\Delta V = 2.25\) 亿立方米,是发生溃坝时洪水的下泄量,得出下泄时的 \(q-t\) 函数关系式,其函数关系为:

\[
\begin{cases}
q = B_{10} - B_{11}t + B_{12}t^2 - B_{13}t^3, & 0 \leq t \leq 25 \\
q = B_{20} - B_{21}t + B_{22}t^2 - B_{23}t^3 + B_{24}t^4 - B_{25}t^5 + B_{26}t^6 - B_{27}t^7, & 25.5 \leq t \leq 34.5 \\
q = B_{30} - B_{31}t + B_{32}t^2 - B_{33}t^3, & 35 \leq t \leq 47
\end{cases}
\]

各项系数如下表所示:

\begin{table}[h]
\centering
\caption{溃坝模型各系数结果}
\label{tab:6.2}
\begin{tabular}{|c|c|c|c|c|c|c|c|c|}
\hline
 & \(B_{y0}\) & \(B_{y1}\) & \(B_{y2}\) & \(B_{y3}\) & \(B_{y4}\) & \(B_{y5}\) & \(B_{y6}\) & \(B_{y7}\) \\ \hline
\(B_{1x}\) & -141.314 & 52.84883 & -2.84826 & 0.047616 & \_ & \_ & \_ & \_ \\ \hline
\(B_{2x}\) & 4.18E+07 & -1.16E+06 & 1.91E+04 & -1.91E+02 & 4.18E+07 & -1.16E+06 & 1.91E+04 & -1.91 \\ \hline
\end{tabular}
\end{table}

\begin{table}
\centering
\begin{tabular}{|c|c|c|c|c|c|c|c|}
\hline
$B_{3x}$ & 52325.58 & -1687.174 & 18.02703 & -0.063837 & --- & --- & --- \\
\hline
\end{tabular}
\end{table}

得到 $q-t$ 关系图:

\begin{figure}[h]
\centering
\includegraphics[width=0.8\textwidth]{image.png}
\caption{正交最小二乘模型所得水流量图}
\end{figure}

从图 6.11 可以看出,正交最小二乘模型所得水流量的曲线总体和泄洪模型水流量曲线是相似的,最大流量为 $10363 \, m^3/s$。

\subsection{6.3 溃坝模型}

经过以上分析,溃坝蓄水量变化与水流量的成积分关系

\[
\Delta V = \int_{0}^{t} q \, dt
\]

其中,水流量 $q$ 与时间 $t$ 的关系可以通过正交最小二乘模型和仿真模型求解得到,确立 $q-t$ 关系之后,溃坝模型其他参数如溃口宽度、溃口深度、水速通过下列模型求解

\begin{equation}
\left\{
\begin{aligned}
V &= V_{\max} - \int_{0}^{t} q \, dt & \text{(1)} \\
h &= H(V) & \text{(2)} \\
w &= w_{\max} \left( \frac{q}{q_{\max}} \right)^{0.5} & \text{(3)} \\
g &= g_{\max} \left( \frac{q}{q_{\max}} \right)^{0.5} & \text{(4)} \\
f &= h - (h_{\max} - g) & \text{(5)} \\
v &= 2qg / f^2 w_{\max} & \text{(6)}
\end{aligned}
\right.
\end{equation}

\subsection{6.4 溃坝模型的求解结果}

\subsubsection{6.4.1 溃坝过程水位高程的变化规律}

溃坝时堰塞湖水位高程的变化受溃口水流量的影响,溃口水流量大则堰塞湖水位下降速率快,水流量小则堰塞湖水位下降速率慢,水位高程与堰塞湖蓄水量关系由问题一已经得到 $h = H(V)$,堰塞湖蓄水量由溃坝前最大蓄水量减去泄洪量得到,而泄洪量由水流量的积分 $\int_{0}^{t} q \, dt$ 得到,因此水位高程为

\begin{equation}
h = H(V), V = 3.05 - \int_{0}^{t} q dt
\end{equation}

溃坝过程水位高程与时间的变化规律如图 6.12,溃坝过程泄洪量与时间的关系如图 6.13。

\begin{figure}[h]
    \centering
    \includegraphics[width=0.48\textwidth]{image1.png}
    \caption{溃坝过程水位高程的规律}
    \label{fig:6.12}
\end{figure}
\begin{figure}[h]
    \centering
    \includegraphics[width=0.48\textwidth]{image2.png}
    \caption{溃坝过程中泄洪量的规律}
    \label{fig:6.13}
\end{figure}

【注】实曲线为正交最小二乘模型结果;虚曲线为仿真结果

\subsection{6.4.2 溃坝过程溃口变化模型}

溃坝过程中溃口的变化和泄洪过程溃口变化规律类似,溃口宽度和溃口深度与水流量的 0.5 次方成正比,溃口最大深度 $g_{\text{max}}$ 为最大水位高程 750 米与最终水位高程 713 米之差 37 米,假定溃坝溃口形状与泄洪溃口形状是相似的,得到溃坝溃口最大宽度 $w_{\text{max}}$ 为 156.5 米,建立溃口变化模型

\begin{equation}
w = 156.5 \left( \frac{q}{8818} \right)^{0.5}, \quad g = 37 \left( \frac{q}{6680} \right)^{0.5}
\end{equation}

得到溃口宽度、深度与时间的变化曲线规律如图 6.14、图 6.15:

\begin{figure}[h]
    \centering
    \includegraphics[width=0.48\textwidth]{image3.png}
    \caption{溃口宽度的变化规律}
    \label{fig:6.14}
\end{figure}
\begin{figure}[h]
    \centering
    \includegraphics[width=0.48\textwidth]{image4.png}
    \caption{溃口深度的变化规律}
    \label{fig:6.15}
\end{figure}

【注】实曲线为正交最小二乘模型结果;虚曲线为仿真结果

\subsection{6.4.3 溃坝过程中水位流速模型}

溃口深度从水位最高处开始测算,因此溃口中水深 $f$ 可以表示为

\begin{equation}
f = h - (750 - g)
\end{equation}

$f$ -- 溃口中水深;$h$ -- 当前水位高程;$g$ -- 溃口深度。

溃口处流速可以根据流量和水面截面积的关系求得

\begin{equation}
v = \frac{2qg}{f^2 w_{\text{max}}}, \quad q -- 堰塞湖溃坝过程流量变化规律
\end{equation}

得到的溃口水深、流速与时间变化规律如图 6.16、图 6.17

\begin{figure}[h]
    \centering
    \includegraphics[width=\textwidth]{image1.png}
    \caption{图6.16 溃口水深的变化规律 \hspace{1cm} 图6.17 溃口处流速的变化规律}
    \label{fig:6.16-6.17}
\end{figure}

\textbf{【注】} 实曲线为正交最小二乘模型结果;虚曲线为仿真结果

\subsection{6.4.4 溃坝过程中几个关键时间点结果}

\subsubsection{表6.3 仿真模型研究溃坝过程结果}

\begin{table}[h]
    \centering
    \begin{tabular}{|c|c|c|c|c|c|}
        \hline
        时刻点(时) & 3 & 24 & 26 & 36 & 45 \\
        \hline
        水流量 & 62.7 & 1480 & 6406 & 454 & 58.7 \\
        \hline
        水位高程 & 2.6 & 13.1 & 24.5 & 5.05 & 1.84 \\
        \hline
        溃口宽 & 12.8 & 62.1 & 130 & 130 & 130 \\
        \hline
        溃口深 & 2.9 & 14.3 & 29.6 & 29.6 & 29.6 \\
        \hline
        水流速度 & 2.9 & 3.9 & 5.3 & 3.7 & 7.81 \\
        \hline
    \end{tabular}
    \caption{表6.3 仿真模型研究溃坝过程结果}
    \label{tab:6.3}
\end{table}

\subsubsection{表6.4 正交最小二乘模型研究溃坝过程结果}

\begin{table}[h]
    \centering
    \begin{tabular}{|c|c|c|c|c|c|}
        \hline
        时刻点(时) & 3 & 21 & 25 & 32 & 40 \\
        \hline
        水流量 & 149.7 & 893.2 & 10363.2 & 224.0 & 73.6 \\
        \hline
        水位高程 & 4.6 & 9 & 22.9 & 6.21 & 2.9 \\
        \hline
        溃口宽 & 19.1 & 46.7 & 159 & 159 & 159 \\
        \hline
        溃口深 & 4.7 & 11.4 & 39 & 39 & 39 \\
        \hline
        水流速度 & 3.5 & 5.1 & 9.7 & 7.9 & 4.2 \\
        \hline
    \end{tabular}
    \caption{表6.4 正交最小二乘模型研究溃坝过程结果}
    \label{tab:6.4}
\end{table}

\section{7 问题三}

对于坝体发生溃塌,造成堰塞湖内 \(1/3\) 的蓄水突然下泻时,根据问题一,问题二中对蓄水量的研究,大约有将近 1 亿立方米的水量,在溃坝后较短时间内以大流量,高流速涌出堰塞湖,对沿河下游地势低洼处危害极大。此处以这 1 亿立方米的洪峰主体作为研究对象,对如何测算洪水速度,如何预估受灾区域、如何选定群众撤离方案等给出较为实际的计算模型,并根据测算量确定撤离方案。

根据题意给出如下假设:

1. 假设洪水主体水量在途经各地时的减小系数基本保持不变
2. 假设所用数据量是准确的(误差在可接受范围)

以上假设来源对实际问题的综合考虑,洪水通过各地区时,必然有一部分水量被土壤、植被、低洼地等吸收或囤积,所以随着洪水途经各地,其洪水水量也有不同程度的减少。这里定义减小系数 \(z\),即 \(z \in (0, 1)\),含义为单位距离内途经各地后减少的水量占主体水量的百分比。由于地形地表结构的复杂性,以及水流速的影响,洪水途经各地的减小系数 \(z\) 的大小不同,甚至还受时间等方面的影响,但是为了计算的可行性,将减小系数 \(z\) 系数近似的认为是途经各地的减小系数 \(z\) 是一致的,且实际中减小系数 \(z\) 差异较小,由此产生的误差影响是可以接受的。

\subsection{7.1 主体水量流经区域的推算}

根据资料显示,唐家山堰塞湖三分之一蓄水量突然下泄时,将形成洪峰并主要沿河流向地势低洼处涌进。那么,在 GIS 技术支持下,即通过数字地图,主要对沿河流域地势高度进行分析,相对高度较低的地区将有被淹没的可能。根据三维图象结合数字地图进行对比分析,可以得到河流位置大致如下:

\begin{figure}[h]
    \centering
    \includegraphics[width=\textwidth]{image1.png}
    \caption{三维地图中河流走向}
\end{figure}

\begin{figure}[h]
    \centering
    \includegraphics[width=\textwidth]{image2.png}
    \caption{数字地图中河流走向}
\end{figure}

从原题附件的三维地图中,可以清楚的确定唐家山到绵阳地区的河流分布状况图,进而与数字地图进行比较分析,以在数字地图中确定该地区河流的大致位置,并通过数字地图确定河流的大致经纬度。

\begin{table}
\centering
\caption{主要城市的经纬度范围}
\begin{tabular}{|c|c|c|c|c|c|c|c|c|}
\hline
北川 & 纬度 & 经度 & 通口 & 纬度 & 经度 & 香水 & 纬度 & 经度 \\
\hline
左上 & 31.8485 & 104.4376 & 左上 & 31.7259 & 104.6022 & 左上 & 31.6548 & 104.6422 \\
\hline
左下 & 31.8235 & 104.4376 & 左下 & 31.6981 & 104.6063 & 左下 & 31.6202 & 104.6537 \\
\hline
右上 & 31.8479 & 104.4707 & 右上 & 31.7286 & 104.6442 & 右上 & 31.6548 & 104.6909 \\
\hline
右下 & 31.8248 & 104.4674 & 右下 & 31.6975 & 104.6462 & 右下 & 31.6229 & 104.6943 \\
\hline
青莲 & 纬度 & 经度 & 绵阳 & 纬度 & 经度 & & 纬度 & 经度 \\
\hline
左上 & 31.6046 & 104.6584 & 左上 & 31.5355 & 104.7011 & 左上 & & \\
\hline
左下 & 31.5870 & 104.6665 & 左下 & 31.4359 & 104.6998 & 左下 & & \\
\hline
右上 & 31.6080 & 104.6808 & 右上 & 31.5315 & 104.8095 & 右上 & & \\
\hline
右下 & 31.5870 & 104.6801 & 右下 & 31.4319 & 104.8129 & 右下 & & \\
\hline
\end{tabular}
\end{table}

上表列出了该地区的城市的经纬度,并且每个城市是以规则图形范围形式给出了,通过数字地图中河流的位置,可以发现,以上城市均在该河流两岸位置,并且地势较低,洪水将依次经过北川,通口,香水,青莲,绵阳。

\begin{figure}[h]
\centering
\includegraphics[width=\textwidth]{image.png}
\caption{海拔 460 米区域图}
\end{figure}

通过数字地图,做出 460 米海拔高度以下的区域(图中蓝色区域),发现从 450 米至 460 米之间,蓝色面积增幅较大,并涵盖了绵阳大部分地区,而随着每 10 米海拔高度的增加,蓝色面积增幅趋于减小的趋势。这说明,随着海拔高度的增加,地形坡度有所增加。

显示绵阳处在一个较为低洼处,途经北川,通口,香水,清莲等县城,到达绵阳时地势较平,容易囤水。由电子地图测算唐家山堰塞湖与绵阳的海拔高度,即唐家山堰塞湖 \( h_1 = 709m \),绵阳 \( h_2 = 460m \),由此可以得到两地地势落差为大约 \( h_3 = 250m \)。而由唐家山堰塞湖至绵阳的河流长度,根据原题资料中数据可以确定约为 \( h_4 = 60km \)。

\section*{7.2 关于洪水速度的计算}

对洪水前进速度的预测,是做好洪涝灾害预防,群众撤离决策的重要依据,这里根据实际应用环境提出借鉴分析法和机理计算法两种方法进行洪水流速的预测。

\subsection{7.2.1 借鉴分析法}

所谓借鉴分析法,是基于实验分析法的一种衍生,通过经验、专家评定、特征量标定等手段估计当前分析对象的规模与程度,进而参考借鉴历史已有的类似规模与程度的水流速度,并作为对当前研究对象的标准量。

通过对唐家山堰塞湖三分之一蓄水量突然下泄时,所造成规模与程度较大(原题参考资料中总工程师刘宁分析对下游构成严重威胁),所以此处选择历史上一些相类似的大洪水、特大洪水案例进行研究分析。由分析借鉴得到预测的唐家山堰塞湖三分之一蓄水量突然下泄时的水流速度 \( v \)。

此处,根据 2006 年资料:“上午 10 时的测速显示,马口水文站流速为每秒 4.34 米,估计最大流速会达每小时 4.4 米,而 2005 年 6 月大洪水的最大流速也只不过是每秒 4.05 米。” 由以上资料可以确定大洪水的速度约为 4 米左右,这里出于对唐家山问题的慎重考虑,对唐家山三分之一蓄水量突然溃坝时水流速度的确定,应该使用模糊数学悲观决策方法。即站在造成问题最为严重的立场上考虑问题。因为水流速度越大,对于下游城市的危害也越大,故建立在悲观决策方法上的方式,选用每秒 4.4 米水流速度是合理的。即:
\[
v = 4.4 \, \text{m/s}
\]

\subsection{7.2.2 机理计算法}

作为原理分析,首先从事物发展的根本原理作为出发点,进行事物本质研究。该问题要求对水流速度做出预测计算,可以从原理入手进行分析比较。

研究对象是主体水量,主体水量的加速原因在于重力作用是其顺势而流,故地势高度差将直接影响最终水流速度的快慢,这里将其表现为重力加速度的分加速度 \( g' \)。除此之外,阻力也是会对研究的主体水体造成阻力加速度 \( a \),而这个阻力加速度 \( a \) 随着主体水量的水速增加而增加,可以得到阻力加速度 \( a \) 与水流速度 \( v \) 的关系式,即
\[
a = -k \cdot v^2
\]

综上数据可以进行计算水流主体重力加速度的分速度 \( g' \),水流主体的重力分速度 \( g' \) 的计算式为
\[
g' = g \cdot \sqrt{\frac{h_3}{h_4}}
\]

进一步可以明确对主体水量的综合加速度可以表示为
\[
a' = g' + a
\]

在确定了主体水量加速度的基础上,根据物理学加速原理建立主体水量速度 \( v \) 与其变化的加速度 \( a' \) 之间的积分关系表达式,如下:
\[
v = v_0 + \int_0^t a' \, dt
\]

综合上述分析可以得到如下水流速度计算模型:
\[
v = v_0 + \int_0^t \left( g \cdot \sqrt{\frac{h_3}{h_4}} - k v^2 \right) dt
\]

其中,\( v_0 \) 主体水流量的初始速度,\( g \) 重力加速度。

此处,限于环境难以完成实验,故暂选用借鉴分析法 \( v = 4.4 \, \text{m/s} \) 求解。

通过查询“中国电子地图”软件可以得到以上城市分别到唐家山的沿河距离,同时得到洪水到达时间见表 7.2

\begin{table}
\centering
\begin{tabular}{|c|c|c|c|c|c|}
\hline
城市 & 北川 & 通口 & 香水 & 青莲 & 绵阳 \\
\hline
里程 & 3千米 & 30千米 & 37千米 & 50千米 & 60千米 \\
\hline
到达时间 & 11.34分 & 1时54分 & 2时20分 & 3时10分 & 3时48分 \\
\hline
\end{tabular}
\caption{城市到唐家山堰塞湖的沿河距离,同时得到洪水到达时间}
\end{table}

\section{7.3 关于淹没区域及安区的计算}

\subsection{[1]模型分析及建立}

根据数字高程地图,可以确定在一定高程范围内的等高距离,使用 MATLAB 软件分析出在一定等高范围内的面积。此处,问题一 5.1.2 蓄水量体积离散化模型的建立的微分思想,是将湖水按深度离散成若干层,以此求解蓄水量与高度的关系,其中涉及了面积计算。对于问题三,在问题一的基础上,可以简单理解为当该地区容纳 $(1-c)Q$ 的洪水时,湖水表面面积有多大。如图 7.4 所示:

\begin{figure}[h]
\centering
\includegraphics[width=0.6\textwidth]{image.png}
\caption{蓄水量体积离散化示意图}
\end{figure}

江湖水体积通过微分离散成高度为一米的众多湖水层,由数字高程地图分别标示出该地区的最低海拔高度线,由前面分析可知湖水将在绵阳市附近囤积,造成对绵阳市的严重危害。绵阳市所在区域在数字高程地图中查询,最低点在 445 米左右,所以,以 445 米为最低点,海拔每上升一米,使用数字地图标示出此海拔高度的面积,使用 MATLAB 进行像素分析,得到其实际面积。

由以上逻辑关系,若有 $I$ 个湖水层,当总量为 $Q$ 的洪水突然下泄时,设到达被淹没地区前洪水的消耗为 $c$,被淹没地域的最低高程为 $l_0$,高程为 $l_i$ 时对应的面积为 $s_i$,则该地区容纳 $(1-c)Q$ 的洪水时,对应的高程为 $l_f$,则

\[
(1-z)Q = \sum_{0}^{I} s_i
\]

其中,$z$ 为洪水水量减小系数,根据此模型可以得到主要受灾城市绵阳的被淹面积大小。

\subsection{[2]模型求解及结果}

关于上述模型的计算方法是:

\textbf{Step1:} 将数字高程地图的图片载入 Matlab,读取水平方向与竖直方向的像素值,与该图片对应的实际区域的大小进行对比,计算一个像素点所代表的实际位置的大小。

\textbf{Step2:} 对需要研究的某一地区的数字高程地图进行分色处理,将要确定的高程之下的地域设定为某一特定颜色,载入 Matlab,读取该地域这一高程对应的像素数,将像素数转化为实际面积从而确定该地区这一高程 $l_i$ 对应的横向剖面面积 $S_i$。

\textbf{Step3:} 由于目前对洪水水量减小系数 $z$ 的取值尚不确定,此处对 $z$ 分别取 $10\%$, $30\%$, $50\%$,代入模型进行计算。

计算结果见图:

黑色区域为淹没区域,绿色区域为安全区域,其余为正常区域。

\begin{figure}[h]
    \centering
    \includegraphics[width=\textwidth]{image1.png}
    \caption{z 为 50\% 的洪水淹没绵阳的区域及安全区域示意图}
    \label{fig:50_flood}
\end{figure}

\textbf{【注】} 黑色区域为淹没区域,绿色区域为安全区域

图 \ref{fig:50_flood} 中 z 为 50\% 的洪水淹没绵阳的区域的淹没面积为 913.1689 万平方米,安全面积 2332.2 万平方米。

\begin{figure}[h]
    \centering
    \includegraphics[width=\textwidth]{image2.png}
    \caption{z 为 30\% 的洪水淹没绵阳的区域及安全区域示意图}
    \label{fig:30_flood}
\end{figure}

\textbf{【注】} 黑色区域为淹没区域,绿色区域为安全区域

图 \ref{fig:30_flood} 中 z 为 30\% 的洪水淹没绵阳的区域的淹没面积为 1327.4 万平方米,安全面积 2373.5 万平方米。

\begin{figure}[h]
    \centering
    \includegraphics[width=\textwidth]{image1.png}
    \caption{图 7.7 $z$ 为 10\% 的洪水淹没绵阳的区域及安全区域示意图}
    \label{fig:7.7}
\end{figure}

【注】黑色区域为淹没区域,绿色区域为安全区域

图 7.7 中 $z$ 为 10\% 的洪水淹没绵阳的区域的淹没面积为 1498.7 万平方米,安全面积 2659.7 万平方米。

对于北川,通口,香水,青莲的过境洪水淹没面积的处理方法类似,不同之处是这四个地区坡度较大,洪水难以囤积。但是洪水过境时,相对地势较低的位置遭受洪水入侵得过程与绵阳市是一样的过程,其计算结果为:

\begin{figure}[h]
    \centering
    \includegraphics[width=\textwidth]{image2.png}
    \caption{图 7.8 北川可能被淹没的区域及安全区}
    \label{fig:7.8}
\end{figure}

【注】黑色区域为淹没区域,绿色区域为安全区域

北川可能被淹没的区域面积是:430.2 万平方米,以及安全面积为:508.9 万平方米;

\begin{figure}[h]
    \centering
    \includegraphics[width=\textwidth]{image1.png}
    \caption{通口可能被淹没的区域及安全区域}
    \label{fig:7.9}
\end{figure}

\textbf{【注】} 黑色区域为淹没区域,绿色区域为安全区域

通口可能被淹没的区域面积是:101.1 万平方米,以及安全面积为:181.2 万平方米;

\begin{figure}[h]
    \centering
    \includegraphics[width=\textwidth]{image2.png}
    \caption{香水可能被淹没的区域及安全区域}
    \label{fig:7.10}
\end{figure}

\textbf{【注】} 黑色区域为淹没区域,绿色区域为安全区域

香水可能被淹没的区域面积是:177.8 万平方米,以及安全面积为:207.1 万平方米;

\begin{figure}[h]
    \centering
    \includegraphics[width=\textwidth]{image.png}
    \caption{青莲可能被淹没的区域及安全区域}
    \label{fig:7.11}
\end{figure}

【注】黑色区域为淹没区域,绿色区域为安全区域

青莲可能被淹没的区域面积是:87.6万平方米,以及安全面积为:135.4万平方米;

\section{撤离方案的确定}

以上黑色区域为被淹没区域的面积,其中图中还标识的了绿色区域。这部分绿色区域是指洪水来临时的安全区域,这是因为在利用数字高程地图圈定被淹没区时,基于同样的理论容易得到其周围海拔较高而不被淹没的地区范围,这些范围可以被认为是安全区域(安全区域在上图中使用绿色区域标示)。并且这些绿色区域顾及了两个因素:一是距离被淹没区较近,方便步行到达;二是坡度合适,便于人群聚集生活。

另外,对于各地撤离方案的确定,此处拟定两套撤离方案:

\begin{enumerate}
    \item 方案一:告警撤离方案;
    \item 方案二:紧急撤离方案。
\end{enumerate}

对于方案一是为唐家山堰塞湖出现险情可能性较大,提前预警的前提下使用的方案,该撤离方案在撤离时往往有较为充裕的撤离时间,可以进行相对充分的准备工作,各级领导只要根据图中绿色安全区域进行有序撤离即可,这里不作详细的研究。

对于方案二,在唐家山堰塞湖突然出现人为估计之外的险情,导致了未能预警而突然溃坝的特殊状况时启用的方案。唐家山堰塞湖一旦出现堰塞湖内1/3的蓄水突然下泻的严重溃坝事故,到达绵阳仅损失10\%,假设下游各个城市将在第一时间得知险情,不计反映时间,边通知边疏散,那么此处给出安全撤离的各项参数,如下表:

\begin{table}[h]
    \centering
    \caption{各城镇撤离方案}
    \label{tab:7.3}
    \begin{tabular}{|c|c|c|c|}
        \hline
         & 撤离时间 & 安区高度 & 撤离方向 \\
        \hline
        北川 & 11.34分 & 720米 & 西部或东南方向 \\
        \hline
        通口 & 1时54分 & 610米 & 西南方向 \\
        \hline
        香水 & 2时20分 & 600米 & 西北方向 \\
        \hline
        青莲 & 3时10分 & 540米 & 南方或北方 \\
        \hline
        绵阳 & 3时48分 & 500米 & 北方或东、西方 \\
        \hline
    \end{tabular}
\end{table}

\section{分析政策的正确性与改进的可能性}

\section*{8.1 模型的分析与数据的引入}

由第一、第二问的模型可以知道,从5月25日到6月11日,随着时间的推移唐家山堰塞湖的水位和蓄水量是不断上涨的。在制定安全策略时,需要考虑较大的降雨量引发的堰塞湖水位的变化,这里将150\%的降雨量引发的堰塞湖湖水水位的变化以及相应的蓄水量列举如下表

\textbf{表8.1 150\%的降雨量引发的堰塞湖湖水水位的变化以及相应的蓄水量}

\begin{table}[h]
\centering
\begin{tabular}{|c|c|c|c|c|c|c|c|c|c|}
\hline
日期 & 5.25 & 5.26 & 5.27 & 5.28 & 5.29 & 5.30 & 5.31 & 6.01 & 6.02 \\ \hline
高程 & 724.5316 & 727.4785 & 729.8893 & 730.9585 & 731.9702 & 732.8593 & 733.7866 & 734.6325 & 735.4539 \\ \hline
蓄水量 & 1.555 & 1.641 & 1.732 & 1.763 & 1.785 & 1.798 & 1.890 & 1.923 & 1.955 \\ \hline
日期 & 6.03 & 6.04 & 6.05 & 6.06 & 6.07 & 6.08 & 6.09 & 6.10 & 6.11 \\ \hline
高程 & 736.2564 & 737.2956 & 738.3167 & 740.3043 & 741.1418 & 741.9284 & 742.6332 & 743.299 & 743.9516 \\ \hline
蓄水量 & 1.998 & 2.053 & 2.156 & 2.321 & 2.353 & 2.434 & 2.460 & 2.507 & 2.557 \\ \hline
\end{tabular}
\end{table}

为了降低堰塞湖的风险,政府采用在堰塞湖上作业掘进导流渠的方法。前文中已经说明,该方法相当于变相降低阻塞体的高程,使湖水在较低高程时通过堰塞体从而降低洪水溢流的风险。

根据新闻信息可以得到在开始导流渠掘进作业之后,导流渠进口方向每日的掘进高度变化。

\textbf{表8.2 导流渠进口方向每日的掘进高度变化}

\begin{table}[h]
\centering
\begin{tabular}{|c|c|c|c|c|c|}
\hline
日期 & 5.26 & 5.27 & 5.28 & 5.29 & 5.30 & 5.31 \\ \hline
入口高程 & 748 & 746 & 745 & 743 & 741.6 & 740 \\ \hline
出口高程 &  & 745 & 744 & 743 & 740.5 & 739 \\ \hline
\end{tabular}
\end{table}

\section*{8.2 模型的建立}

\subsection*{8.2.1 约束与目标分析}

\textbf{掘进工作安全期}

在掘进作业到5月30日之后,由于湖水的水位不断上升,受作业面减小导致作业空间不足等因素的影响。每日的掘进高程会在1.5m左右,因此可以推算出每日的掘进高程 $d_{i}$($i$ 为达到掘进高程的日期),结合对每日坝前水位高程 $f_{j}$($j$ 为到达该坝前水位的日期)。可以推算出掘进工作的安全期 $e$:

\[
e = j - i (d_{i} = f_{j})
\]

所谓安全期,即是在导流渠的入口掘进高度到达 $d_{i}$ 之后,停止掘进作业。等待至坝前水位到达该高度 $f_{j} (d_{i} = f_{j})$ 还需要的时间,在坝前水位到达导流渠入口掘进高度之后,湖水就会漫过导流渠继而导致整个堰塞体的垮塌。由于掘进的导流渠较为复杂,导流渠中间段高程高于导流渠入口高程1.5m左右,因此将 $d_{i} = f_{j}$ 修正为 $d_{i} + 1.5 = f_{j}$。

由于存在人员撤离、监测、安全冗余等问题,因此需要保证相对较大的掘进工作安全期,人员撤离、监测设备的安放的时间会超过一天,考虑到一定的安全冗余,要求安全期要大于四天,即:

\[
e > 4
\]

\textbf{导流渠入口高程安全高度}

从第二问对泄洪过程的分析可以看出,即便是导流渠入口高程达到 $d_{i} = 740m$ 之后,堰塞湖蓄水也会受导流渠中段高程较高等因素的影响,在742m左右的高程开始过水。而在这一高度开始的过水,导流放水直至最终引起的堰塞体垮塌整个过程还是会对下游产生较大的冲击。为使泄洪过程不给下游造成过大的冲击,需要限制泄洪时的最大流量,最大流量过大会导致泄洪的过程更加接近于溃坝,无法起到安全泄洪的作用。

因此,根据第二问的模型推算,要求泄流时洪水流量比较安全,则泄洪时的最大流量应该控制在 $7000m^{3}/s$ 以内,即
\[
q_{\max } \leq 7000m^{3} / s
\]
进而利用第二问模型插值推算出泄洪时的水位的安全高程要在 $745m$ 以下,结合导流渠中段较高的因素可以推算出导流渠入口的安全高度要在 $742m$ 以内,即
\[
d_{i} \leq 742
\]

\section*{安全目标}

导流渠的入口高程越低,堰塞湖就会在越低的水位泄洪,这种情况下下泄的洪水就会较少,从而使堰塞湖对下游产生的破坏降到最低,因此要求,目标函数:
\[
\min d_{i}
\]

\subsection*{8.2.2 模型的建立与求解}

\textbf{目标函数}
\[
\min d_{i}
\]

\textbf{约束条件:}
\[
S.T
\left\{
\begin{aligned}
d_{i}+1.5m &= f_{j} \tag{1} \\
e &> 4, \ e = j-i \tag{2} \\
d_{i} &\leq 742 \tag{3}
\end{aligned}
\right.
\]

\textbf{约束说明:}

(1) 坝前水位高程 $f_{j}$ 在降低至导流渠中部高程 $d_{i}+1.5m$ 时,开始过水。导流渠掘进作业第 $i$ 天完成,坝前水位第 $j$ 天降低到该高度;

(2) 安全期要在四天以上;

(3) 导流渠的高度至少要降低至 $742m$ 。

\textbf{模型的求解:}

搜索每日坝前水位的高程可以求解该模型,求解结果:

掘进工作进行至 6 月 1 日,入口高程降至 $d_{i}=738.5m$,导流渠中部高程 $740m$,在五天的安全期之后,坝前水位上涨至 $f_{j}=740.3m$,开始过水行洪。行洪时的坝前水位达到 $f_{j}=740.3m$,此时的库容 $v=2.321m^{3}$。

\subsection*{8.3 对决策部门采取的导流渠掘进措施的评价}

在唐家山的抢险过程中,5 月 31 日停止了导流渠的掘进工作,导流渠的入口高程在 $740m$,中部高程在 $741.5m$ 以上,在 5 月 10 日坝前水位涨至 $f_{j}=741.92m$ 时开始有少量过水,最终在 6 月 10 日水流速度大大增加,过水冲垮了堰塞体完成了泄洪任务。

可以看出,政府采取导流渠掘进作业措施是非常接近与我们所求解的最优解的,同时政府的措施相对较多考虑了堰塞体上作业人员的安全问题,政府在天气、掘进的具体进度未知的情况下,做出这一决策是比较正确的。

由于政府采取导流渠掘进作业措施的正确性,以及其合理性,注定在同一时间使用同样的预测数据为决策依据,对下游群众的疏散决策也将是具有很高的正确性。

\textbf{认为可以改进的措施:}

增加在不利天气下作业的预案,可以看出这次导流明渠的掘进工作很大程度上依赖与直升机能否将重型机械调运至唐家山堰塞体上,决策部门虽然准备了若干直升机不能作业时的预案,但是这些预案的操作难度都很大,可行性也不高。

设法在洪水流经区域设立行洪区,通过对数字高程地图的分析可以看出,在洪水流经的沿途,有部分人烟较少且相对高度较低的地区,这些地区中有相当部分的面积受小丘陵的影响位于洪水流经部分直接联通。如果能够采取引流,爆破等措施让这些地方成为行洪区。使洪水通过这些地方时会发生分流,这样就会大大降低绵阳等人口密集地区的洪水压力。

从全文的分析可以看出,在应对地震依法的各类灾害时,事先建立完备的地理信息系统(GIS)是十分必要的,只有拥有完备的地理信息系统才能为地震发生之后的抗争救灾工作提供强有力的技术支持和保障。同时为了更好的监测地震发生的状况以及地震引发的各类灾害,需要在可能发生地震以及地震次生灾害的地区部署一定数量的传感器,以方便对地震及地震此生灾害的监测和治理。