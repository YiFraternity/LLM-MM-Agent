\begin{center}
\textbf{\Large 全国第七届研究生数学建模竞赛}
\end{center}

\begin{center}
\includegraphics[width=0.5\textwidth]{logo.png}
\end{center}

\begin{center}
\textbf{题目} 与封堵溃口有关的重物落水后运动过程的数学建模
\end{center}

\section*{摘要}

我国经常发生洪水,溃坝溃堤引发的泥石流灾害对国家和人民生命财产造成了严重损失。因此,研究与封堵溃口有关的重物落水问题具有很高的实际应用价值。

本文根据固体力学、流体力学及岩土力学相关理论,分析了影响重物落水运动过程的主要因素,根据各因素权重关系及相关物理理论建立了多因素融合的大实心方砖落水过程模型。在此基础上,提出了适用于描述不同情况的单一物体运动模型,定性描述了两连接物体落水后耦合运动过程,并与实测数据进行拟合,根据对比的结果,实现了误差分析,从误差原因出发优化了模型以提高其准确度和适用性。最后,依据相似准则将优化后的模型加以推广,实现了理论-实验-工程三者的互相验证,同时对未来需要进行的试验和研究工作提出了一些建议。

对问题 1,利用了附录 1 中各试件的实验结果,并运用控制变量法,每次只改变一个因素,在 Origin 中作实验数据对比图,定性分析了各因素对重物运动过程的影响,并得到 6 个主要影响因素;然后依此及重物落水过程的受力情况,分阶段建立了大实心方砖入水过程及完全入水后的运动模型,采用量纲分析及经验法两种方式考虑阻力,得到一主模型和一辅模型。结果表明,大实心方砖主模型,总体效果较好,当其在特殊工况拟合度较差时采用辅模型进行辅助分析。

对问题 2,受到 3D 动画制作的启发,采用了蒙皮效应来刻画重物形状,引入形状函数反映重物形状变化对水流阻力的影响,建立了能适应不同情况的普适性模型;并从单个物体模型过渡到两个重物连接后的组件在水中耦合作用过程,通过 Origin 作单个与两个物体的对比图,分析了压差阻力对两个物体耦合时的重要影响。

对问题 3,联系到实际工程中封堵溃口材料的具有多样化,从两个大类出发把材料归一化,分别把颗粒类和块状类的材料近似看成模型中的实心方砖与四面体,从而提取出这两种类型材料,重点进行误差分析,并据分析结果,提出了更为全面的正交试验方法,可使数据拟合精度不断提高。

对问题 4,基于流体的相似性原理,得出本题中小型模型试验满足的相似准则;根据流体相似即几何相似、运动相似和动力相似分析了小型模型试验中主要相似条件,依此对未来需要进行的研究侧重点作了部分建议;另外,利用相似性原理将实际问题缩尺,把缩尺后的工况应用于已建问题 1 模型中,确定模型参数,使实际问题得到解决。

\textbf{关键词:} 封堵溃口 机理分析 普适性 压差阻力 误差分析 相似准则

\begin{table}[h]
\centering
\begin{tabular}{l l}
参赛队号 & 10291003 \\
队员姓名 & 孙田 陈苏 金丹丹 \\
\end{tabular}
\quad
\begin{tabular}{|c|}
\hline
参赛密码 \\
(由组委会填写) \\
\hline
\end{tabular}
\end{table}

\begin{center}
中山大学承办
\end{center}

\section*{与封堵溃口有关的重物落水后运动过程的数学建模}

\section*{一、问题重述}

由于重物落水本身是一个复杂的数学物理问题,它涉及到空气、水、重物等多种不同介质的耦合作用,且水流运动非常复杂,至今仍有一些水流现象无法从小型试验得到,即小型试验的结论不能任意推广到实际真实情况。因此,本问题具有很强的研究背景和实际应用价值。

根据上述分析,本文拟解决如下问题:

(1)、根据已有的资料数据,利用所学知识及相关参考资料,分析影响重物在水中运动过程的因素;

(2)、建立具体单个重物在水中运动过程的数学模型:大实心方砖模型。并根据小型试验数据及相关数学、物理方法,验证模型准确度和适用性;

(3)、建立包含各种影响因素,从而能够适应不同情况的重物水中运动过程的单个物体水中运动过程数学模型;

(4)、通过机理分析,定性分析了两个重物连接后的组件在水中的运动过程,从固-固耦合,流-固耦合入手深入到流-固-固耦合关系;

(5)、对建立的模型进行误差分析,通过模型参数调节优化模型效果,并根据修改模型预测最有效触底位置。根据误差原因分析等对未来需要进行的试验和研究工作提出了一些建议;

(6)、根据相似准则设计相关实验,对已建立模型进行推广和扩充。以达到理论-实验-工程三者的互相验证。

\section{问题分析}

对上述问题重述中提出的各个问题,本文拟采用如下思路进行分析,从而解决问题:

\begin{figure}[h]
    \centering
    \includegraphics[width=0.8\textwidth]{image.png}
    \caption{问题分析流程图}
    \label{fig:problem_analysis}
\end{figure}

\section{基本假设}

(1) 重物入水后不考虑重物翻转的影响,即:重物入水后 \(x\)、\(y\) 方向迎水面积不发生变化;

(2) 重物在同一稳定介质中阻力影响系数保持不变;

(3) 认为水平方向一旦速度达到水流速度,就不再增大速度,即作匀速运动;

(4) 建立重物落水模型时不考虑摄像头在水槽壁上折射对读数影响;

(5) 只考虑迎水面积上的摩阻力作用,忽略侧面粘滞阻力。

\section*{四、符号约定}

\begin{itemize}
    \item $x, y$: 重物在落水过程中的横坐标、纵坐标,坐标系约定见 6.1 节说明;
    \item $\rho_{\text{物}}$: 重物的密度;
    \item $\rho$: 水的密度;
    \item $S$: 物体有效迎水面积;
    \item $S_x, S_y$: 分别为 $x$ 方向的有效迎水面积、$y$ 方向的有效迎水面积;
    \item $\vec{v}$: 物体在水中相对运动速度;
    \item $v_x, v_y$: 物体在 $x$ 轴方向、$y$ 轴方向的绝对速度;
    \item $u_0$: 水流速度;
    \item $v_0$: 重物刚接触水表面时,竖向的初速度;
    \item $H$: 水槽深度;
    \item $a, b, c$: 分别为当重物迎水面的长度、宽度、垂直迎水面的厚度;
    \item $V$: 重物的体积;
\end{itemize}

\section*{五、主要创新点}

对问题(1)

在建立大实心方砖模型时,采用两阶段考虑的方式,对重物在水中的运动分成刚接触水面至刚好完全浸入水中、重物刚好完全浸入水体至沉底两段进行分析。并鉴于对阻力形式考虑的不同,得出一主一辅模型,相互补充;

对问题(2)

1. 在问题(1)阻力描述的基础上,猜测阻力与速度并非简单的一次、两次方关系,提出 $F = kv^R$ 的阻力描述方式;

2. 联系实际分析,工程应用中投掷物形状多样,很难定量刻画,受到 3D 动画制作的启发,采用了蒙皮效应来刻画重物形状,引入形状函数 $\kappa(s)$ 来反映重物形状变化,建立了能够适应不同情况的普适性模型;

对问题(3)

进行误差分析,并据分析结果,提出了更为全面的正交试验方法,可使数据拟合精度不断提高。

对问题(4)

1. 根据流体相似即几何相似、运动相似和动力相似分析了小型模型试验中主要相似条件,依此对未来需要进行的研究侧重点作了部分建议;

2. 合理简化实际溃口问题,充分利用已有模型试验数据及简化模型,方便求解实际问题。

\section*{六、模型的建立与求解}

\subsection*{§6.1 问题 1}

\subsubsection{6.1.1 控制变量法分析:影响重物在水中运动过程的因素}

影响重物在水中运动过程的因素包含多方面,为了研究清楚其中主要的因素,本文利用附件中的实验数据结果作 Origin 对比图,采用控制变量法,每次只改变一个因素,控制其余因素不变,来分析该因素对物体运动规律的影响及其影响程度。

下面分别描述分析过程及结果,可见以下 6 个主要因素影响重物水中运动轨迹:水流速度的因素、重物质量、重物入水方式、有效迎水面积、入水速度、物体间相互作用对运动过程的影响(固-固耦合)等。

\subsubsection{因素(1):水流速度的影响}

重物落水后,在水流和重力的共同作用下向河床运动。保持其他因素相同,水流速度越大,对重物的搬运作用越明显,如图 2 所示,大实心方块被平放入水后,水流速度越大,大实心方块漂流得越远。

\begin{figure}[h]
    \centering
    \includegraphics[width=0.8\textwidth]{image.png}
    \caption{重物在不同速度的水流中的运动轨迹}
    \label{fig:2}
\end{figure}

\subsubsection{因素(2):重物入水的方式}

保持其他因素相同,投放重物的方式不同,重物漂流距离不同,如图 3 所示。

\begin{figure}[h]
    \centering
    \includegraphics[width=\textwidth]{image1.png}
    \caption{在相同水流中重物不同投放方式的运动轨迹}
    \label{fig:3}
\end{figure}

\textbf{因素(3):重物的质量}

重物质量影响其竖直方向上的运动,如图 \ref{fig:4} 所示实心大方块和空心大方块分别平放入速度为 0.34m/s 的水中的运动轨迹有较大差异。

\begin{figure}[h]
    \centering
    \includegraphics[width=\textwidth]{image2.png}
    \caption{不同密度的重物在相同水流中的运动轨迹}
    \label{fig:4}
\end{figure}

\textbf{因素(4):有效迎水面积}

本文约定有效迎水面积为重物落水过程中受阻的有效面积。水流作用的强弱与有效迎水面积有关,因此物体的有效迎水面积是影响重物运动的一个重要因素。见如图 \ref{fig:5} 所示。

\begin{figure}[h]
    \centering
    \includegraphics[width=\textwidth]{image3.png}
    \caption{}
    \label{fig:5}
\end{figure}

\begin{figure}[h]
    \centering
    \includegraphics[width=\textwidth]{image1.png}
    \caption{有效迎水面积不同的重物在相同水流中的运动轨迹}
    \label{fig:5}
\end{figure}

\textbf{因素(5):重物入水的速度}

重物以不同初速度入水,也即其被投放时初始高度不同,重物在水中所受阻力有所差异,而使其运动过程不同,见图 \ref{fig:6} 所示。

\begin{figure}[h]
    \centering
    \includegraphics[width=\textwidth]{image2.png}
    \caption{重物以不同入水速度进入相同水流中的运动轨迹}
    \label{fig:6}
\end{figure}

\textbf{因素(6):物体间相互作用对运动过程的影响}

在其他因素相同的情况下,当重物不是以单件形式入水,而是两个(或多个)物体连接在一起后投入水中,显然其运动轨迹会有所不同。单个重物投入水中与两个重物并行入水的运动轨迹差异,如图 \ref{fig:7} 所示。

\begin{figure}[h]
    \centering
    \includegraphics[width=\textwidth]{image3.png}
    \caption{图7}
    \label{fig:7}
\end{figure}

\begin{figure}[h]
    \centering
    \includegraphics[width=\textwidth]{image.png}
    \caption{一个重物和两个重物在相同水流中的运动轨迹}
    \label{fig:7}
\end{figure}

\subsection{影响因素小结}

上述 6.1.1 中根据已有的实验数据分析了影响重物在水中运动过程的因素如:水流速度、重物入水方式、重物密度、有效迎水面积、入水速度、物体间相互作用对运动过程的影响(固-固耦合)等。

对问题 1,本文只作定性分析,即上述因素中,水流速度越大,重物落水后其相对于流水的速度越大;重物的密度越大,重力作用越明显,具体表现为竖直向速度变化越快,即在竖直方向相对于流水的速度越大。因此,忽略重物自身翻转运动后,重物落水后的水流作用可转换为与水流速度和有效迎水面积有关的水摩阻力的作用。

\subsection{大实心方砖落水后运动过程的数学模型}

基于流体力学的理论,单个重物在流水中运动,其受到的作用力为以下两类:

I 类:与流水——重物的相对运动无关的力(即使相对运动的速度和加速度为零,此力也不消失)。此题中为重物的自身重力;

II 类:依赖于流水——重物间相对运动的力。此题中此类力有重物受到流水对其的推力(即水平方向拖曳力)、摩阻力(即惯性阻力)、粘阻力(即粘滞阻力)、浮力(即压差阻力)。

\textbf{注明:}本题中,粘滞阻力可以忽略不计,因为此处流体为水体,其粘度非常小,相对其他外力来说,粘滞阻力对重物运动影响很小可不作考虑。

综上所述,重物在水中运动,其主要受力为:重力、推力、摩阻力、浮力。其中阻力及推力涉及到的影响因素较多,着重考虑,见如下具体分析。

影响水流阻力及推力的因素主要有:流水——重物的相对速度 $\bar{v}$,流水的密度 $\rho$,流体的动力粘滞系数 $\mu$,水深 $H$,重物在流水中的位置,用函数 $\varphi(y/H)$ 表示。即阻力或者推力均可表示成 $F(\bar{v}, \rho, \mu, H, \varphi(y/H))$。

对于问题 1,建立的模型为大实心方砖模型,问题变为具体化,上述因素中的形状 $\kappa(S, a, b, c)$、$H$ 可以认为已定,仅认为阻力或者推力为 $F(\bar{v}, \rho, \mu, \varphi)$。

\textbf{注明:}问题2中将建立适应各种情况的通用模型,到时将考虑 $F(\bar{v}, \rho, \mu, H, \varphi(y/H) \ldots)$。根据大实心方砖在水中的受力情况建立其模型

1) 先刻画大实心方砖所受的阻力 $F(\bar{v}, \rho, \mu, \varphi)$;

采用瑞利法对 $F$ 进行量纲分析(瑞利法定义见问题2中6.2.1),将阻力 $F$ 设为如下的函数关系:$F = KS^{\alpha_1} \rho^{\alpha_2} \mu^{\alpha_3} \bar{v}^{\alpha_4}$($F$ 中不含 $\varphi$,是因为 $\varphi(y/H)$ 本身量纲为 1,对量纲分析时不产生影响),根据量纲和谐原理,方程式 $F = KS^{\alpha_1} \rho^{\alpha_2} \mu^{\alpha_3} \bar{v}^{\alpha_4}$ 等号两边的变量积的基本量纲的指数必须相等。

用基本量纲 M、L、T 去替代各物理量,有

\begin{equation}
MLT^{-2} = L^{2\alpha_1} (ML^{-3})^{\alpha_2} (ML^{-1}T^{-1})^{\alpha_3} (LT^{-1})^{\alpha_4}
\tag{6-1}
\end{equation}

上式等号两边各量纲的指数对应相等,即

\begin{equation}
\text{M: } 1 = \alpha_2 + \alpha_3; \quad \text{L: } 1 = 2\alpha_1 - 3\alpha_2 - \alpha_3 - \alpha_4; \quad \text{T: } -2 = -\alpha_3 - \alpha_4
\end{equation}

这是 4 个未知数 3 个方程的方程组,以 $\alpha_3$ 作为待定指数,分别求出 $\alpha_2 = 1 - \alpha_3$,$\alpha_4 = 2 - \alpha_3$,$\alpha_1 = 3 - 2\alpha_3$,最终得到:

\begin{equation}
F = KS\rho\bar{v}^2 \left(\frac{\nu S^{1/2} \rho}{\mu}\right)^{-\alpha_3}.
\end{equation}

其中 $\frac{\nu S^{1/2} \rho}{\mu}$ 无量纲,记 $K \frac{\nu S^{1/2} \rho}{\mu} = C_D$,则 $F = C_D S \rho \bar{v}^2$;

同样,推力也可表示成 $F = C_T S \rho \bar{v}^2$,其中 $C_D$、$C_T$ 为阻力影响系数、推力影响系数。

2) 运动微分方程建立

坐标系的建立:取模型试验槽中水流方向为 $x$ 轴正方向,竖直向上为 $y$ 轴正方向,坐标原点取重物的入水点。

数学模型的建立:重物在水中的运动分两阶段:〈1〉接触水面至刚好完全浸入水中;〈2〉重物刚好完全浸入水体至沉底,见图8。在建立模型时,$x$ 方向在两阶段的情况可以近似认为完全相同,不需分阶段考虑,而 $y$ 方向的运动在两阶段有异,因此进行分阶段讨论。

\begin{figure}[h]
    \centering
    \includegraphics[width=\textwidth]{image.png}
    \caption{重物在水中运动的两个阶段}
    \label{fig:two_stages}
\end{figure}

\section*{分阶段讨论大方块在不同方向上的运动情况}

1) 在 \( x \) 轴方向不分阶段:

重物在 \( x \) 轴方向仅受水的推力作用,大小为 \( F_{\text{推}} = C_{T} S \rho \bar{v}^2 = C_{T} S \rho (u_0 - v_x)^2 \),其中 \( u_0 \) 为水流的速度,\( v_x \) 是重物在水中水平方向的绝对速度。本题中,在重物刚入水时刻,认为其水平方向的初速度等于 0,而水平向水流速度视为一个定值设为 \( u_0 \)(如后所示的 \( 0.34 \, m/s \)、\( 0.40 \, m/s \)、\( 0.47 \, m/s \) 等),重物在推力的作用下变加速运动,直至速度和水速相等时,认为重物开始匀速运动。

由牛顿第二定律得重物满足下面微分方程:
\begin{equation}
C_D \rho S (u_0 - v_x)^2 = m \frac{dv_x}{dt}
\tag{6-2}
\end{equation}
相应初始条件为:\( t = 0 \) 时,\( x = 0 \); \( t = 0 \) 时,\( \frac{dx}{dt} = 0 \),解方程 \( C_D \rho S (u_0 - v_x)^2 = m \frac{dv_x}{dt} \)

\[
\Rightarrow \frac{C_D \rho S}{m} dt = \frac{dv_x}{(u_0 - v_x)^2}, \text{两边同时对 } t \text{ 积分 } \Rightarrow \alpha_1 t = \frac{1}{u_0 - v_x} + A, \text{ 其中 } \alpha_1 = \frac{C_D \rho S_x}{m}, A \text{ 为积分常数;}
\]

代入初始条件 \( t = 0 \) 时,\( \frac{dx}{dt} = v_x = 0 \),解得 \( A = -\frac{1}{u_0} \);得到:\( \alpha_1 t = \frac{1}{u_0 - v_x} - \frac{1}{u_0} \);对其两边取自然对数,得到 \( v_x = u_0 - u_0 e^{-\alpha_1 t} \),两边对 \( t \) 进行积分最终得到:

\begin{equation}
x = u_0 t + \frac{u_0}{\alpha_1} e^{-\alpha_1 t} + B
\tag{6-3}
\end{equation}

代入初始条件 \( t = 0 \) 时,\( x = 0 \),解得 \( B = -\frac{u_0}{\alpha_1} \) 故:

\begin{equation}
x = u_0 t + \frac{u_0}{\alpha_1} e^{-\alpha_1 t} - \frac{u_0}{\alpha_1}
\tag{6-4}
\end{equation}

2) 在 \( y \) 轴方向分阶段考虑:

阶段〈1〉: 重物接触水面至刚好完全浸入水中

在此阶段中,在 \( y \) 轴方向物体受到浮力、水的阻力和重力共同作用,浮力 \( f_{\text{浮}} = \rho g S_y \left( \frac{c}{2} - y \right) \),重力 \( G = mg \),阻力 \( F = C_D S \rho \bar{v}^2 = C_D S \rho v_y^2 \),其中 \( v_y \) 是重物在水中水平方向的绝对速度。

由牛顿第二定律得重物在阶段〈1〉满足下面微分方程:

\begin{equation}
mg - \rho g S_y \left( \frac{c}{2} - y \right) - C_D S \rho v_y^2 = m \frac{dv_y}{dt} \quad \left( -\frac{c}{2} \leq y \leq \frac{c}{2} \right)
\tag{6-5}
\end{equation}

相应的初始条件如下:\( t = 0 \) 时,\( y = \frac{c}{2} \);\( t = 0 \) 时,\( \frac{dy}{dt} = v_0 \)。

解方程 \( mg - \rho g S_y \left( \frac{c}{2} - y \right) - C_D S_y \rho v_y^2 = m \frac{dv_y}{dt} \)。

此处解上述方程运用技巧处理:分三个作用力单独考虑后合成。原因是:重力、浮力及阻力对重物速度的影响互不影响。

(1) 自重对速度的影响:\( v_{y\text{重}} = v_0 + gt \);\( g \) 为重力加速度;(2) 浮力对速度的影响:

\[
\rho g S_y \left( \frac{c}{2} - y \right) = -m \frac{dv_y}{dt} \Rightarrow \rho g S_y v_y = m \frac{d^2 v_y}{dt^2}
\]

解得:

\begin{equation}
v_{y\text{浮}} = C e^{\alpha_2 t} + D
\tag{6-6}
\end{equation}

式中 \( \alpha_2 = \frac{\rho g S_y}{m} \)。(3) 阻力对速度的影响:根据牛顿第二定律可知,\( C_D \rho S_y v_{y\text{阻}}^2 = -m \frac{dv_{y\text{阻}}}{dt} \)。

分离变量得到 \( -\frac{C_D \rho S_y}{m} dt = \frac{dv_y}{v_{y\text{阻}}^2} \),两边对 \( t \) 积分 \( \Rightarrow -\alpha_2 t = -\frac{1}{v_{y\text{阻}}} + B \),两边自然对数处理得:\( v_{y\text{阻}} = E e^{\alpha_2 t} + F \),其中 \( \alpha_2 = \frac{C_D \rho S_y}{m} \),\( B \) 为积分常数。

综合考虑重力、浮力和水的阻尼的影响,三者之间相互不影响,通过叠加的方法可获得如下方程:\( v_y = v_{y\text{重}} - v_{y\text{浮}} - v_{y\text{阻}} \) 代入上面推导的结果有:

\begin{align}
v_y &= v_0 + gt - C e^{\alpha_2 t} - D - E e^{\alpha_2 t} - F \\
&= v_0 + gt - M e^{\alpha_2 t} - N
\tag{6-7}
\end{align}

根据速度初始条件(t=0,$v_{y}=v_{0}$)即 $\frac{dy}{dt}=v_{y}=v_{0}$ 代入(6—7)解得 $N=-M$,所以

\begin{equation}
v_{y}=v_{0}+gt-Me^{\alpha_{2}t}+M
\tag{6—8}
\end{equation}

对上式两边对 $t$ 积分,得到

\begin{equation}
y=v_{0}t+\frac{1}{2}gt^{2}-\frac{1}{\alpha_{2}}Me^{\alpha_{2}t}+Mt+T
\tag{6—9}
\end{equation}

位置初始条件 $t=0$ 时,$y=\frac{c}{2}$,$T=\frac{c}{2}+\frac{1}{\alpha_{2}}M$,得到:

\begin{equation}
y=v_{0}t+\frac{1}{2}gt^{2}-\frac{1}{\alpha_{2}}Me^{\alpha_{2}t}+Mt+\frac{c}{2}+\frac{1}{\alpha_{2}}M
\tag{6—10}
\end{equation}

其中,过程中的 $E$、$F$、$M$、$N$ 为积分常数。

阶段〈2〉:重物刚好完全浸入水体至沉底

在此阶段中,重物在 $y$ 轴方向受水的浮力、阻力和重力共同作用,重力 $G=mg$,浮力 $F_{\text{浮}}=\rho gV$(注:$V$ 为重物体积,$V=$ 迎水面积 $S \times$ 垂直迎水面的厚度 $c$),阻力 $F=C_{D}S_{y}\rho\overline{v}^{2}=C_{D}S_{y}\rho v_{y}^{2}$,其中 $v_{y}$ 是重物在水中水平方向的绝对速度。

由牛顿第二定律得重物在阶段〈2〉满足下面微分方程:

\begin{equation}
mg-\rho gV-C_{D}S_{y}\rho v_{y}^{2}=m\frac{dv_{y}}{dt}(y\leq-\frac{c}{2})
\tag{6—11}
\end{equation}

相应的初始条件如下:$t=0$ 时,$y=-\frac{c}{2}$;且上下阶段在分段处函数连续。所以 $t=0$ 时,$\frac{dv_{y}}{dt}$ 与上一阶段同位置处大小相等。解方程:与上一阶段对比,其中的浮力在该阶段为常数,因此,其结果为:

\begin{equation}
\begin{aligned}
v_{y} &= v_{y\text{重}}-v_{y\text{浮}}-v_{y\text{阻}} \\
&= v_{0}-B+(g-\alpha_{3})t-Ce^{\alpha_{2}t}
\end{aligned}
\tag{6—12}
\end{equation}

式中 $\alpha_{3}=\frac{\rho gV}{m}$。则:

\begin{equation}
y=(v_{0}-B)t+\frac{1}{2}(g-\alpha_{3})t^{2}-\frac{C}{\alpha_{2}}e^{\alpha_{2}t}+D
\tag{6—13}
\end{equation}

代入初始条件 \( t=0 \) 时,\( y=-\frac{c}{2} \),得到 \( D=-\frac{c}{2}+\frac{C}{\alpha_{2}} \)。

综上所述,整个过程物体运动的参数方程为:
\begin{equation}
\left\{
\begin{aligned}
x &= u_{0} t+\frac{u_{0}}{\alpha_{1}} e^{-\alpha_{1} t}-\frac{u_{0}}{\alpha_{1}} \\
y &= v_{0} t+\frac{1}{2} g t^{2}-\frac{1}{\alpha_{2}} M e^{\alpha_{2} t}+M t+\frac{c}{2}+\frac{1}{\alpha_{2}} M\left(-\frac{c}{2} \leq y \leq \frac{c}{2}\right) \\
y &= \left(v_{0}-B\right) t+\frac{1}{2}\left(g-\alpha_{3}\right) t^{2}-\frac{C}{\alpha_{2}} e^{\alpha_{2} t}-\frac{c}{2}+\frac{C}{\alpha_{2}}\left(y \leq-\frac{c}{2}\right)
\end{aligned}
\right.
\tag{6-14}
\end{equation}

其中 B、C、M 为积分常数;\( \alpha_{1}=\frac{C_{D} \rho S_{x}}{m} \),\( \alpha_{2}=\frac{C_{D} \rho S_{y}}{m} \),\( \alpha_{3}=\frac{\rho g V}{m} \)。

注意到,在上述大实心方砖模型的推导过程中,\( \alpha_{1}, \alpha_{2}, \alpha_{3} \) 是与重物自身特性有关的参数,即与重物的质量、密度、几何形状有关,相应的工况确定,参数 \( \alpha_{1}, \alpha_{2}, \alpha_{3} \) 也就唯一确定。而 B、C 是积分常数,利用附件中有关方砖的所有工况数据拟合,得到参数 B 为 2.415,C 为 0.0283。因此,式 (6-14) 为实心大方砖模型,其适用于方砖各种工况。现针对大实心方砖工况 1 可以定出其中的工况 1 对应的参数。

对于实际实验数据,我们只考虑其 \( y \) 向的第二阶段。水流速度 \( u_{0}=0.34 \, \mathrm{m/s} \),计算模型中参数得
\[
\alpha_{1}=\frac{C_{D} \rho S_{x}}{m}=\frac{1.05 \times 1000 \times 0.08 \times 0.04}{0.08 \times 0.08 \times 0.04 \times 2300}=5.71 \, ; \quad \frac{u_{0}}{\alpha_{1}}=\frac{0.34}{5.71}=0.06 \, ,
\]
将它们代入式 (6-14) 求得
\[
x=0.34 t+0.06 e^{-5.71 t}-0.06 \, .
\]
其中,\( C_{D} \) 为阻尼系数,是一个可以通过实验得到的常数,根据有关文献 Evett 等人 \({}^{[3]}\) 引用 Raymonde. Binder (1973) 的数据资料得出结果,大实心方砖 \( C_{D} \) 取值 1.05。

另外对于工况 1,计算相关参数如下,重物竖向初速度及其他参数为:
\[
v_{0}=0 \, , \, g=10 \, \mathrm{m/s^{2}} \, , \, \alpha_{2}=\frac{1.05 \times 1000 \times 0.08 \times 0.08}{0.08 \times 0.08 \times 0.04 \times 2300}=11.4 \, , \, \alpha_{3}=\frac{1000 \times 10}{2300}=4.35 \, , \, c=0.04 \, \mathrm{m} \, ,
\]
所以
\[
y=-2.145 t+2.83 t^{2}-0.0025 e^{11.4 t}-0.0175 \, .
\]

适应工况 1 模型:
\begin{equation}
\left\{
\begin{aligned}
x &= 0.34 t+0.06 e^{-5.71 t}-0.06 \\
y &= -2.145 t+2.83 t^{2}-0.0025 e^{11.4 t}-0.0175
\end{aligned}
\right.
\tag{6-15}
\end{equation}

对于工况 1 的数据拟合及残差图如下:

\begin{figure}[h]
    \centering
    \includegraphics[width=\textwidth]{image1.png}
    \caption{大实心方砖模型1对工况1的拟合图}
    \label{fig:9}
\end{figure}

\begin{figure}[h]
    \centering
    \includegraphics[width=\textwidth]{image2.png}
    \caption{残差分析图}
    \label{fig:10}
\end{figure}

由图\ref{fig:9}和图\ref{fig:10}可知,拟合曲线和实测曲线吻合度较好,残差分布属于正常范围。上述模型称为大实心方砖模型1,为主模型。

\subsection{大实心方砖主模型小结}

通过量纲分析得到 $F = C_{D} S \rho \bar{v}^{2}$,利用此关系建立模型后,无法得到 $x$、$y$ 的直接函数关系,只能建立
\begin{equation}
\begin{cases}
x = x(t) \\
y = y(t)
\end{cases}
\end{equation}
的参数方程,相比 $y = y(x)$,可能会产生误差放大效应。由上述拟合图及残差分析图可见,拟合程度一般。经过推测与分析,若 $F = k \bar{v}$ 时,对应的微分方程就可以直接求得 $y = y(x)$,可能会得到更好的拟合效果。同时,在大量查阅相关文献时,发现对低速物体在流体中的阻力作用,也常用 $F = k \bar{v}$ 表示,当然这里 $k(S, \rho, \mu, X)$ 是相关因素的函数。$\rho$ 是水的密度;$S$ 表示重物迎水面受阻有效面积;$\mu$

为反映流体物理性质的粘度;$\vec{v}$ 是物体在水中相对运动速度;然后,运用 $F = k\vec{v}$ 再次对大实心方砖建模,并和上述模型对比。

\subsection*{6.1.5 建立大实心方砖模型 2(作为辅模型)}

再次建立大实心方砖模型 2,坐标系定义、$y$ 方向的两阶段考虑均同上述模型,此处不再重复。

在 $x$ 轴方向不分阶段:

重物在 $x$ 轴方向仅受水的推力作用,大小为 $F = k(S, \rho, \mu,)(u_0 - v_x)$,其中 $u_0$ 为水流的速度,$v_x$ 是重物在水中水平方向的绝对速度。

由牛顿第二定律得重物运动满足下面微分方程:

\begin{equation}
k(u_0 - \frac{dx}{dt}) = m\frac{d^2x}{dt^2}
\tag{6-16}
\end{equation}

相应初始条件 $t = 0$ 时,$x = 0$;$t = 0$ 时,$\frac{dx}{dt} = 0$,其特征方程为 $mr^2 + kr = 0$,求得特征根 $r_1 = 0$;$r_2 = -\frac{k}{m}$。于是方程解的形式为:

\begin{equation}
x = C_1e^{r_2t} + C_2
\tag{6-17}
\end{equation}

在 $y$ 轴方向分阶段考虑

阶段 $<1>$:重物接触水面至刚好完全浸入水中

在此阶段中,在 $y$ 轴方向受水的浮力、水的阻力和重力共同作用,浮力 $f_{\text{浮}} = \rho_{\text{水}}gS_y(\frac{c}{2} - y)$,重力 $G = mg$,阻力 $F = k(S, \rho, \mu, )v_y$。

由牛顿第二定律得重物在阶段 1 满足下面微分方程:

\begin{equation}
mg - \rho gS_y(\frac{c}{2} - y) - k_1\frac{dy}{dt} = m\frac{d^2y}{dt^2} \quad (-\frac{c}{2} \leq y \leq \frac{c}{2})
\tag{6-18}
\end{equation}

其中,$k_1$ 为阶段 $<1>$ 阻力系数函数;该方程特征方程为 $mr^2 + k_2r - \rho gs = 0$。求得特征根

\begin{equation}
r_{3,4} = \frac{-k_2 \pm \sqrt{k_2^2 + 4m\rho gs}}{2m}
\end{equation}

因 $m > 0$;$k_2 > 0$,故 $k_2^2 + 4m\rho gs > 0$。于是方程解的形式为

\begin{equation}
y = C_3e^{r_3t} + C_4e^{r_4t} + D
\tag{6-19}
\end{equation}

所以阶段 $<1>$ 参数方程为

\begin{equation}
\begin{cases}
x = C_1e^{r_2t} + C_2 \\
y = C_3e^{r_3t} + C_4e^{r_4t} + D
\end{cases}
\end{equation}

消去参数 $t$ 后得到:

\begin{equation}
y = C_3\left(\frac{x - C_2}{C_1}\right)^{\frac{r_3}{r_1}} + C_4\left(\frac{x - C_2}{C_1}\right)^{\frac{r_4}{r_1}} + D
\tag{6-20}
\end{equation}

进一步简化
\[
\Rightarrow \quad y = A(x - C_2)^{K_1} + B(x - C_2)^{K_2} + D
\tag{6-21}
\]

其中 $C_1, C_2, C_3, C_4, A = \frac{C_3}{(C_1)^{r_3/r_1}}, B = \frac{C_4}{(C_1)^{r_4/r_1}}, D, K_1 = \frac{r_3}{r_1}, K_2 = \frac{r_4}{r_1}$ 均为常数。

阶段 $<2>$:重物刚好完全浸入水体至刚沉于水底接触面:

在重物完全浸水后,竖直向仍受:重力、水摩阻力、浮力作用,根据牛顿第二定律列出微分方程 $f_{\text{浮}} + k_2 \frac{dy}{dt} - mg = m \frac{d^2y}{dt^2} (c/2 < y < h)$。其中 $k_2$ 为阶段 $<2>$ 相应的阻力系数函数。

方程 2 的特征方程为 $mr^2 + k_4r = 0$,特征根 $r_3 = 0; r_4 = -\frac{k_4}{m}$。于是方程解的形式为:
\[
y = C_3 e^{r_4 t} + C_4
\tag{6-22}
\]

所以阶段 $<2>$ 参数方程为
\[
\begin{cases}
x = C_1 e^{r_2 t} + C_2 \\
y = C_3 e^{r_4 t} + C_4
\end{cases}
\]
消去参数 $t$ 后得到 $y$ 与 $x$ 的关系为:
\[
y = C_3 \left( \frac{x - C_2}{C_1} \right)^{\frac{r_2}{r_4}} + C_4
\tag{6-23}
\]

进一步简化
\[
\Rightarrow \quad y = A(x - C_2)^{K_1} + C_4
\tag{6-24}
\]

其中 $C_1, C_2, C_3, C_4, A = \frac{C_3}{(C_1)^{r_2/r_4}}, K_1 = \frac{r_2}{r_4}$ 均为常数。

综上所述,重物的运动轨迹方程如下:
\[
\begin{cases}
y = A_1 (x - C_2)^{K_1} + B(x - C_2)^{K_2} + D \dots \dots \dots (-c/2 \leq y \leq c/2) \\
y = A_2 (x - C_3)^{K_3} + C_4 \dots \dots \dots \dots \dots \dots \dots \dots \dots (c/2 \leq y < h)
\end{cases}
\tag{6-25}
\]

因此,实测数据采用多项式拟合方法,而非其他拟合方法。我们选取大实心方砖工况 1 拟合如下图所示:

\begin{table}
\centering
\begin{tabular}{|c|c|c|c|c|c|c|c|c|c|}
\hline
阶段 & \multicolumn{3}{|c|}{工况1} & \multicolumn{3}{|c|}{工况2} & \multicolumn{3}{|c|}{工况3} \\
\cline{2-10}
工况 & 平放 & 立放 & 竖放 & 平放 & 立放 & 竖放 & 平放 & 立放 & 竖放 \\
\hline
阶段1 & 0.99687 & 0.99509 & 0.97255 & 0.9723 & 0.99179 & 0.9757 & 0.97307 & 0.96671 & 0.95473 \\
\hline
阶段2 & 0.99209 & 0.9905 & 0.96508 & 0.98352 & 0.99773 & 0.98281 & 0.98105 & 0.99032 & 0.98358 \\
\hline
阶段 & \multicolumn{3}{|c|}{工况4} & \multicolumn{3}{|c|}{工况5} & \multicolumn{3}{|c|}{工况6} \\
\cline{2-10}
工况 & 平放 & 立放 & 竖放 & 平放 & 立放 & 竖放 & 平放 & 立放 & 竖放 \\
\hline
阶段1 & 0.98296 & 0.9884 & 0.98322 & 0.97933 & 0.97966 & 0.98576 & 0.96162 & 0.98966 & 0.98576 \\
\hline
阶段2 & 0.96452 & 0.98211 & 0.99113 & 0.97452 & 0.97113 & 0.98211 & 0.9708 & 0.98531 & 0.98745 \\
\hline
阶段 & \multicolumn{3}{|c|}{工况7} & \multicolumn{3}{|c|}{工况8} & \multicolumn{3}{|c|}{工况9} \\
\cline{2-10}
工况 & 平放 & 立放 & 竖放 & 平放 & 立放 & 竖放 & 平放 & 立放 & 竖放 \\
\hline
阶段1 & 0.99681 & 0.98099 & 0.98585 & 0.98112 & 0.98962 & 0.9967 & 0.96708 & 0.98799 & 0.99326 \\
\hline
阶段2 & 0.99967 & 0.99087 & 0.98257 & 0.97686 & 0.98301 & 0.99644 & 0.97515 & 0.97762 & 0.99121 \\
\hline
阶段 & \multicolumn{3}{|c|}{工况10} & \multicolumn{3}{|c|}{工况11} & \multicolumn{3}{|c|}{工况12} \\
\cline{2-10}
工况 & 平放 & 立放 & 竖放 & 平放 & 立放 & 竖放 & 平放 & 立放 & 竖放 \\
\hline
阶段1 & 0.98849 & 0.97915 & 0.97893 & 0.99397 & 0.97227 & 0.9877 & 0.97423 & 0.94927 & 0.94278 \\
\hline
阶段2 & 0.99382 & 0.98969 & 0.98459 & 0.99828 & 0.98861 & 0.98986 & 0.97111 & 0.96428 & 0.97122 \\
\hline
\end{tabular}
\end{table}

\subsection{6.1.6 问题 1 小结}

上述建立的两个大实心方砖的运动过程模型,第二个模型是在第一个模型基础上猜想其他形式分析的,其结果得其模型为一多项式形式,与实验数据拟合后,发现拟合效果非常理想。说明大实心方砖运动模型后者准确度可能更高。

进而猜想,物体在流体中运动过程中,阻力和相对速度的关系并非简单的一次、二次关系,对于普适性问题,本文推广到 $F = k\bar{v}^r$。

\subsection{6.2 问题 2}

\subsubsection{6.2.1 瑞利法量纲分析:单个物体运动过程}

对单个物体运动过程的量纲分析,本文采用瑞利法,即:如果根据理论分析和实验得知反映某一物理现象的各有关因素(变量)的数目 $(y, x_1, x_2, \ldots, x_n)$,并假定这一物理过程的方程式可以用变量的幂乘积形式来表示,也即:

\begin{equation}
y = K x_1^{\alpha_1} x_2^{\alpha_2} \ldots x_n^{\alpha_n}
\tag{6-26}
\end{equation}

则由量纲和谐原则得出结论:方程式等号两边的变量积的基本量纲的指数必须相等。其中 $K$ 为无量纲系数,$\alpha_1, \alpha_2, \ldots, \alpha_n$ 为待定的指数。

上述问题 1 中已对影响阻力 $F$ 的因素进行了合理分析,据此可以将阻力 $F$ 假设为如下的函数关系:$F = f(S, \rho, \mu, \bar{v})$ 已经分析得到:

\begin{equation}
F = \frac{1}{2} C_D \rho \bar{V}^2
\tag{6-27}
\end{equation}

但同时问题 1 的解题过程也发现,模型没有考虑实际应用过程中,物体本身形状的影响,并不能代表实际情况,不具有推广价值。本文为了解决上述问题,引入了:形状函数 $\kappa(s)$,其次,根据 6.1.1 分析可得影响物体在水中运动的因素主要有:流体的流速,常以垂线平均流速 $U$ 代替;水的密度 $\rho$,流体的动力粘滞系数 $\mu$;位置,形状等有关,影响因素本文推测各因素对阻力 $F$ 的影响,可以变现在速度 $\overline{V}^R$ 中 $R$ 的次幂上,因此,水流作用力可表达为:

\begin{equation}
F = f(U, \rho, \mu, \kappa(s) \cdots)
\tag{6-28}
\end{equation}

\begin{equation}
F = \frac{1}{2} C_D \rho \kappa(s) \overline{V}^R
\tag{6-29}
\end{equation}

分析得出:

其中,$U$ 为水平向重物与水流的相对速度,当重物水平向 $U \to u_0$,$F \to 0$,位置函数可以通过已有研究查表。

蒙皮效应:物体表面上的凸凹点和具有代表性的点作为蒙皮面的参考点,从而在

物体表面假想 “蒙皮”,物体的外部形状即为蒙皮所覆盖的区域。

本文根据上述蒙皮效应,引入形状函数 \(\kappa(s)\),进行定性分析,\(\kappa(s)=K\int_{\partial\Omega(t)}\vec{n}dS\),K 表示不同时刻有效面积与物体 \(S_{\text{迎水面积}_{\max}}\) 的比值(0<K<1),即:

\begin{equation}
\begin{aligned}
F_{\text{水平推力}} &= \frac{1}{2}C_{D}\rho K_{X}\int_{\partial\Omega(t)}\vec{n}d_{S}(u_{0}-V_{X})^{R} \\
F_{\text{竖向阻力}} &= \frac{1}{2}C_{D}\rho K_{Y}\int_{\partial\Omega(t)}\vec{n}d_{S}V_{Y}^{R}
\end{aligned}
\tag{6—30}
\end{equation}

则控制物体平动方程满足牛顿第二定律:(分为 \(X, Y\) 方向)

\begin{equation}
\begin{cases}
m\frac{dV_{X}}{dt} = F_{\text{水平推力}} = \frac{1}{2}C_{D}\rho K\int_{\partial\Omega(t)}\vec{n}d_{SX}(u_{0}-V_{X})^{R} \\
m\frac{dV_{Y}}{dt} = G - F_{\text{竖向阻力}} - F_{\text{浮}} = G - \frac{1}{2}C_{D}\rho K\int_{\partial\Omega(t)}\vec{n}d_{SY}V_{Y}^{R}
\end{cases}
\tag{6—31}
\end{equation}

求解上述过程

(1) 解 \(m\frac{dV_{X}}{dt} = \frac{1}{2}C_{D}\rho K\int_{\partial\Omega(t)}\vec{n}d_{SX}(u_{0}-V_{X})^{R}\),令 \(A = \frac{\frac{1}{2}C_{D}\rho K\int_{\partial\Omega(t)}\vec{n}d_{SX}}{m}\),则上式变为:

\begin{equation}
\frac{dV_{X}}{dt} = A(u_{0}-V_{X})^{R}
\tag{6—32}
\end{equation}

采用分离变量法:\(\frac{dV_{X}}{(u_{0}-V_{X})^{R}} = Adt\),分离变量积分得到 \(-\frac{1}{1-R}(u_{0}-V_{X})^{1-R} = At - \frac{u_{0}^{1-R}}{1-R}\),于是:

\begin{equation}
(u_{0}-V_{X}) = \left[A(R-1)t + u_{0}^{1-R}\right]^{\frac{1}{1-R}}
\end{equation}

\begin{equation}
\Rightarrow u_{0}t - x = \frac{1}{A(R-1)}\int\left[A(R-1)t + u_{0}^{1-R}\right]^{\frac{1}{1-R}}d\left[A(R-1)t + u_{0}^{1-R}\right]
\end{equation}

\begin{equation}
\Rightarrow x = u_{0}t - \frac{1}{A(R-1)}\left[A(R-1)t + u_{0}^{1-R}\right]^{\frac{2-R}{1-R}} + B
\end{equation}

其中 A 为常数,\(A = \frac{\frac{1}{2}C_{D}\rho K\int_{\partial\Omega(t)}\vec{n}d_{SX}}{m}\),B 为积分常数。

(2) 解 \(m\frac{dV_{Y}}{dt} = G - \frac{1}{2}C_{D}\rho K\int_{\partial\Omega(t)}\vec{n}d_{SY}V_{Y}^{R} - \rho gV_{\text{体积}}\),令常数 \(A = \frac{\frac{1}{2}C_{D}\rho K\int_{\partial\Omega(t)}\vec{n}d_{SX}}{m}\),求解方式同上,结果通解为:\(y = (V_{0}-C)t + \frac{1}{2}(g-D)t^{2} + \frac{1}{A(R-1)}\left[A(R-1)t + V_{0}^{1-R}\right]^{\frac{2-R}{1-R}} + B\)。

综上所述,普适性模型如下:
\begin{equation}
\begin{cases}
x = u_{0}t - \frac{1}{A(R-1)} \left[ A(R-1)t + u_{0}^{1-R} \right]^{\frac{2-R}{1-R}} + B \\
y = (V_{0} - C)t + \frac{1}{2}(g - D)t^{2} + \frac{1}{A(R-1)} \left[ A(R-1)t + V_{0}^{1-R} \right]^{\frac{2-R}{1-R}} + B
\end{cases}
\tag{6-33}
\end{equation}
式中 A、B、C、D、R 为常数,可根据具体工况计算得到。上述解即为适用于实际工程的理论解,具有普适性。

\subsection*{6.2.2 两物体耦合作用运动过程分析}

两重物连接入水后,体系受力分析过程发生变化,物体不仅仅受重力、阻力、浮力,还受到压差阻力。定性分析过程如下:

\textbf{压差阻力分析} 当物体在粘滞流体中运动时,前方流体受挤压,后方流体则松弛,因而使前方流体的压强增大,后方流体压强减小,从而造成压差。当物体运动速度较小时,在物体静止的参考系中看,流线分布与理想流体情况相仿,前后基本对称;随着速度的增大,流线分布不再对称,物体尾部将产生漩涡,并使前、后方压强差明显增大,此时,压差阻力将成为阻力的主要来源。

物体运动的过程可分析如下:

\begin{figure}[h]
\centering
\includegraphics[width=\textwidth]{image.png}
\caption{大空(实)心方块在 0.4m/s 水流中轨迹}
\end{figure}

\begin{figure}[h]
    \centering
    \includegraphics[width=\textwidth]{image1.png}
    \caption{大空(实)心蜂巢在0.4m/s水流中轨迹}
    \label{fig:13}
\end{figure}

\begin{figure}[h]
    \centering
    \includegraphics[width=\textwidth]{image2.png}
    \caption{小空(实)心方块在0.4m/s水流中轨迹}
    \label{fig:14}
\end{figure}

\begin{figure}[h]
    \centering
    \includegraphics[width=\textwidth]{image3.png}
    \caption{小空(实)心蜂巢在0.4m/s水流中轨迹}
    \label{fig:15}
\end{figure}

\begin{figure}[h]
    \centering
    \includegraphics[width=\textwidth]{image1.png}
    \caption{小空(实)心方块在$0.55\,m/s$水流中轨迹}
    \label{fig:16}
\end{figure}

\begin{figure}[h]
    \centering
    \includegraphics[width=\textwidth]{image2.png}
    \caption{小空(实)心蜂巢在$0.55\,m/s$水流中轨迹}
    \label{fig:17}
\end{figure}

\section*{现象分析:}

由模型试验中一个重物运动轨迹和两个重物运动轨迹的比较可知,当重物较小(体积和质量等)时,一个重物和两个重物运动轨迹很相近,见图 14$\sim$17。在模型试验的测试数据中,我们甚至发现有不止一组数据表现出两种情况下的轨迹完全重合,见图 15 和图 17;当重物较大(体积和质量等)时,两个重物一起运动的轨迹与单个重物运动的轨迹有较大的差别,见图 14 和图 16。

\section*{产生原因:}

对于上述两种试验现象,可以作如下解释:两个重物连接后,并行放置投入水中,两个重物重心的连线水平并垂直于水流方向,且保持不变。当重物较小,通过两重物之间的水流速度变化不大,即重物前后压差不大,重物之间的相互作用对它们的影响不大,它们将保持各自的运动状态,与单个重物运动相同。试验中观测到有两个重物连接后运动轨迹发生了改变,可能是由于刚刚放入水中时,两个重物在水流方向上一前一后,因在水流方向上速度不同一起相互牵拽,从而改变了各自的运动轨迹;而当重物较大时,

重物运动过程中产生的压差阻力大小不能忽略,由于重物之间流水的压强分布不均匀,在垂直于水流方向上两个物体由相互靠拢的趋势,这种趋势的强烈程度与重物之间所夹水的体积有关:重物之间相对面积越大,所夹水的体积越大,这种靠拢的趋势越大;由于重物的靠拢进一步增大了重物前后的压力差,重物受到的水的阻力就越大,单个重物的运动轨迹与两个重物的运动轨迹相差越大。

\section*{\S 6.3 问题 3 误差分析}

\subsection{6.3.1 对已建模型进行误差分析}

对于误差,首先分析:产生误差的原因主要有以下三种:系统误差、模型误差和随机误差,本文选取相对误差作为误差分析的参照标准,以工况1大方砖平放作为标准工况。对不同参数的变化下模型的适用性进行了分析。

本文对相对误差的定义为:残差/实测数据均值,则相对误差为无量纲的量,适合作为刻画误差的参数引入。

\[
\text { 相对误差 }=\frac{\sum_{i=1}^{n}\left|y_{\text {实测 }}-y_{\text {拟合 }}\right|}{y_{\text {实测 }}}
\]

式中:n为实测数据点个数。

根据上述分析,本文设计了如下几组对比性试验:

\begin{figure}[h]
\centering
\includegraphics[width=0.8\textwidth]{image.png}
\caption{模型对工况1的拟合图}
\end{figure}

(1)0.34m/s 水流作用下大实心方砖与 0.55m/s 大实心方砖,可以判断不同水流速度作用下模型的适用性。模型误差率为分别为:3.21\%,通过上述描述,可知:已建立的数学模型,在不同水流速度作用下具有良好的适用性。

\begin{figure}[h]
    \centering
    \includegraphics[width=\textwidth]{image1.png}
    \caption{模型对不同水流速度的适用性。}
    \label{fig:19}
\end{figure}

(2) 0.34m/s 水流作用下大实心方砖与大四面体,可以判断相同水流速度,不同形状作用下模型的适用性。模型误差率达到 33.6%,推测:本模型对不规则块体拟合效果较差,可以利用文本提出的带有形状函数的模型模拟。

\begin{figure}[h]
    \centering
    \includegraphics[width=\textwidth]{image2.png}
    \caption{模型对不同物体形状的适用性。}
    \label{fig:20}
\end{figure}

(3) 0.34m/s 水流作用下大实心方砖与大空心方砖,可以判断相同水流速度,不同有效面积作用下模型的适用性。模型误差率达到 6.54%,推测:本模型对有效迎水面积拟合一般,但可以利用本模型作为辅助模型,在实际工程中作为初次投放的试投函数。

\begin{figure}[h]
    \centering
    \includegraphics[width=\textwidth]{image1.png}
    \caption{模型对不同迎水面积的适用性。}
    \label{fig:21}
\end{figure}

(4) 0.34m/s 水流作用下单个大实心方砖与两个大实心方砖连接,可以判断相同水流速度,物体个数作用下模型的适用性。模型误差不断放大,因此推测得:本模型对多物体耦合作用拟合较差,对多物体的研究详见问题二。

\begin{figure}[h]
    \centering
    \includegraphics[width=\textwidth]{image2.png}
    \caption{模型对不同物体个数的适用性。}
    \label{fig:22}
\end{figure}

(5) 0.34m/s 水流作用下单个大实心方砖平放与单个大实心方砖竖放,可以判断相同水流速度,本模型适用于不同投放方式。

\begin{figure}[h]
    \centering
    \includegraphics[width=\textwidth]{image.png}
    \caption{模型对不同投放方式的适应性}
    \label{fig:23}
\end{figure}

根据上述误差分析理论和数据拟合处理,分析推测产生物体在水中运动轨迹误差的主要因素为以下几点:

(1) 考虑重物侧摩阻力作用;

(2) 重物轴断面与水流方向不平行即重物在水中的运动轨迹不在一个平面内;

(3) 试验水槽中水的流速不是均匀的;

(4) 玻璃折射的影响不考虑,空气折射率近似 $n_{\text{空}}=1$,$20^{\circ}\mathrm{C}$ 下水的折射率 $n_{\text{水}}=1.333$,玻璃折射率 $n_{\text{玻}}=1.5$。由折射定律可知,$n_{\text{空}}\sin\theta_{\text{空}}=n_{\text{玻}}\sin\theta_{\text{玻}}=n_{\text{水}}\sin\theta_{\text{水}}$,空气和水的折射率接近,且镜头离水槽距离较远,可以忽略由于折射产生的观察误差。

\subsection{对已建模型应用猜想及新型实验设计的一些建议}

根据上述分析,和已建模型中各参数物理意义表述,若想更好的获得触底位置,应该采用相关方法提高如下参数:

(1) 重物落水速度(例如溃口上沿可近距离投射,下沿投射位置则应相对较远);

(2) 控制迎水面积,对溃口上沿,应多采用整块材料,对下沿,应采用不规则体;

(3) 重物投放个数(溃口上沿采用多个重物结合投放,下沿则应采用单物体投放);

(4) 投放做平抛运动入水时的入射角。

根据上述对问题的分析:本文建议采用正交试验方法,选取主要参数,进行试验设计,设计正交试验表如下:

\begin{table}
\centering
\caption{正交试验方法设计}
\begin{tabular}{|p{6cm}|p{2cm}|p{2cm}|p{2cm}|p{2cm}|}
\hline
控制因素 入水方式 & 平放 & 竖放 & 立放 & 实验结果 \\
\hline
垂直入水速度(0..5m/s) & & & & \\
\hline
垂直入水速度(1.5m/s) & & & & \\
\hline
垂直入水速度(10m/s) & & & & \\
\hline
垂直入水速度(20m/s) & & & & \\
\hline
入射角(30°) & & & & \\
\hline
入射角(45°) & & & & \\
\hline
入射角(60°) & & & & \\
\hline
物体形状(流线型设计) & & & & \\
\hline
物体形状(扁平型设计) & & & & \\
\hline
物体形状(空心率较高设计) & & & & \\
\hline
物体形状(颗粒状) & & & & \\
\hline
重度 & & & & \\
\hline
初始高度(0cm) & & & & \\
\hline
初始高度(50cm) & & & & \\
\hline
初始高度(120cm) & & & & \\
\hline
物体个数(单个物体) & & & & \\
\hline
物体个数(多物体耦合) & & & & \\
\hline
\end{tabular}
\end{table}

\section*{6.4 问题4}

本题所提及的小型试验是模拟封堵溃口有关的重物落水后运动过程。为了将小型试验的结果运用到现实生活中重物封堵溃口问题,试验流场与现实溃口流场应该具有一定的相似性。

\subsection*{6.4.1 相似性}

流体力学中的相似主要是流动的力学相似。对于一般的流体运动,力学相似应包含以下几个方面。

(1)几何相似

几何相似又叫空间相似,即要求模型的边界原型的边界形状相似,且对应的线性尺寸成相同的比例。用小标“p”表示原型流动,小标“m”表示模型流动,几何相似常数为 \( C_l \),几何相似有:

\begin{align*}
\text{线性比例尺} & \quad \frac{l_p}{l_m} = C_l = const \\
\text{面积比例尺} & \quad \frac{A_p}{A_m} = C_l^2 = const
\end{align*}

\begin{align*}
\text{体积比例尺} \quad \frac{V_p}{V_m} = C_l^3 = const
\end{align*}

式中:$l$ 为原型与模型的特征尺寸。

(2) 运动形似

运动相似是指在几何相似的前提下,模型与原型内对点速度、加速度方向和大小互成一定的比例即运动相似常数 $C_l$,运动相似有:

\begin{align*}
\text{速度比例尺} \quad \frac{v_p}{v_m} &= C_v = const \\
\text{加速度比例尺} \quad \frac{a_p}{a_m} &= \frac{v_p / t_p}{v_m / t_m} = \frac{C_v}{C_t} = \frac{C_v^2}{C_l} = const \\
\text{流量比例尺} \quad \frac{Q_p}{Q_m} &= \frac{l_p^3 / t_p}{l_m^3 / t_m} = \frac{C_l^3}{C_t} = const
\end{align*}

(3) 动力相似

动力相似是在几何相似的前提下,模型与原型内对应点流体质点所受的同名力方向相同且大小互成一定的比例即动力相似常数 $C_F$。

\begin{align*}
\text{力的比例尺} \quad \frac{F_p}{F_m} &= \frac{m_p a_p}{m_m a_m} = \frac{\rho_p V_p^3 a_p}{\rho_m V_m^3 a_m} = C_\rho C_l^3 C_a = C_\rho C_l^2 C_v^2 = const, \text{ 或者 } \frac{C_F}{C_\rho C_l^2 C_v^2} = 1,
\end{align*}

其中 $C_\rho$ 为密度比例尺。

(4) 初始条件和边界条件的相似

初始条件和边界条件的相似是保证两个流动相似的必要条件。边界条件在一般情况下可以分为几何、运动和动力边界,如固体边界上的法线流速为零,自由表面上的压强为大气压强等等。初始条件和边界条件的相似是指原型及模型均满足上条件。在飞恒定流中,初始条件是必需的,在恒定流中初始条件失去实际意义。若把边界条件归类于几何相似,对于恒定流来说,又无需考虑初始条件相似问题。

\subsection*{6.4.2 相似准则}

一般说来,几何相似是运动相似和动力相似的前提和依据,动力相似是决定两个流动相似的主导因素,运动相似是几何相似和动力相似的表现。

两个流体要实现动力相似,作用在相应质点上各种作用力的比尺要满足约束关系,即相似准则。流体中常见的相似准则和相似准数如下:

(1) 弗劳德准则(重力准则)

若考虑促成流体运动的主要作用力为重力,则根据重力的物理性质,两个流体相似。

惯性力 $F_I = ma = \rho V a = \rho l^3 (l / t^2) = \rho l^2 v^2$,重力 $F_G = mg = \rho V g = \rho g l^3$,由 $(\frac{F_I}{F_G})_p = (\frac{F_I}{F_G})_m$ 得

到 $\frac{v_{p}}{\sqrt{l_{p} g_{p}}}=\frac{v_{m}}{\sqrt{l_{m} g_{m}}}$,等号两边均无量纲,称为弗劳德相似准数,简称弗劳德数,记为 $F_{r}$。

原型和模型流动惯性力和重力的相似关系 $(F_{r})_{p}=(F_{r})_{m}$ 或者 $\frac{C_{v}}{\sqrt{C_{l} C_{g}}}=1$ 即原型流动和模型流动弗劳德数相等,就是弗劳德准则。

(2) 雷诺准则

如果考虑促成流体运动的主要作用力为粘性力,惯性力 $F_{I}=\rho l^{2} v^{2}$ 和粘性力,由 $(\frac{F_{I}}{F_{\mu}})_{p}=(\frac{F_{I}}{F_{\mu}})_{m}$ 得到 $R_{e}=\frac{v l \rho}{\mu}=\frac{v l}{\nu}$ 为雷诺数。

原型流动和模型流动粘性力相似要求原型流动和模型流动的雷诺数相等。如长管或者长槽中的水流,液层之间的粘滞阻力对水流形态起显著作用,在设计模型是应首先考虑粘滞阻力的动力相似问题。

(3) 欧拉准则

压力 $F_{p}=(\Delta p) A=(\Delta p) l^{2}$,由 $(\frac{F_{I}}{F_{p}})_{p}=(\frac{F_{I}}{F_{p}})_{m}$ 得到 $(\frac{\Delta p}{\rho v^{2}})_{p}=(\frac{\Delta p}{\rho v^{2}})_{m}$,括号中的组合量也是无量纲数,称其为欧拉相似准则,简化欧拉数,即 $Eu=\frac{\Delta p}{\rho v^{2}}$,$\frac{C_{p}}{C_{p} C_{v}^{2}}=1$,原型流动与模型流动的欧拉数相等,就是欧拉准则。

6.4.3 小模型试验相似准则的分析

本题给出的流体为水,认为是不可压缩流体,密度不变,水受到作用力有重力、压力、粘性力和惯性力。运动相似要求速度和加速度均相似,而惯性力相似是其他各个作用力相似的结果,所以动力相似是运动相似的保证。要模型流体和原型流体相似,即要求雷诺数、弗劳德数和欧拉数分别相等。

影响重物落水后运动过程的主要因素是重力和水的阻力。在几何相似的条件下,两个流场均处在重力场的作用下,试验采用明槽流动模拟溃口河流(一种特殊的孔口出流),所以该小型试验应该满足重力相似准则(费劳德准则)。

对于任何系统的机械运动都必须服从牛顿第二定律 $\vec{F}=m \vec{a}$。对模型与原型流场中的流体微团应用牛顿第二定律,并按动力相似时各类力大小的比例相等得

$$
C_{F}=\frac{\rho^{\prime} \Delta l^{\prime 3} \frac{\Delta v^{\prime}}{\Delta t^{\prime}}}{\rho \Delta l^{3} \frac{\Delta v}{\Delta t}}=C_{\rho} C_{l}^{2} C_{v}^{2}
$$

即 $\frac{C_{F}}{C_{\rho} C_{l}^{2} C_{v}^{2}}=1$,处于重力场中的两个相似流场,重力必然相似。作用在流体微团上的重力之比可以表示为

$$
C_{F}=\frac{W^{\prime}}{W}=\frac{\rho^{\prime} V^{\prime} g^{\prime}}{\rho V g}=C_{\rho} C_{l}^{3} C_{g}
$$

式中 $C_{g}$ 为

重力加速度比例尺。将上式代入式 $\frac{C_{F}}{C_{\rho} C_{l}^{2} C_{v}^{2}}=1$,得 $\frac{C_{v}}{\left(C_{l} C_{g}\right)^{1 / 2}}=1$。在重力场中 $g^{\prime}=g, C_{g}=1$,则有 $C_{v}=\sqrt{C_{l}}$。

若该小型试验还满足粘性力相似准则(雷诺准则),两种流场的雷诺数必定相等粘性力作用下的两个相似流场,其粘性力必然相似。作用在二流场流体微团上的粘性力之比可表示为:

\[
C_{F}=\frac{F_{\mu}^{\prime}}{F_{\mu}}=\frac{\mu\left(\frac{d v_{x}^{\prime}}{d y^{\prime}}\right) A^{\prime}}{\mu\left(\frac{d v_{x}}{d y}\right) A}=C_{\mu} C_{v} C_{l}
\tag{6-34}
\]

代入式(4-15),得 $C_{\rho} C_{v} C_{l} / C_{\mu}=1, C_{v} C_{l} / C_{v}=1$,或 $\frac{\rho^{\prime} v^{\prime} l^{\prime}}{\mu^{\prime}}=\frac{\rho v l}{\mu}, \frac{v^{\prime} l^{\prime}}{\nu^{\prime}}=\frac{v l}{\nu}$。

令 $\frac{\rho v l}{\mu}=\frac{v l}{\nu}=\operatorname{Re}$,Re 称为雷诺(Reynolds)数,其物理意义为惯性力与粘性力的比值。两种流动的粘性力作用相似,它们的雷诺数必定相等,即 $\operatorname{Re}^{\prime}=\operatorname{Re}$;反之亦然。这就是粘性力相似准则,又称雷诺准则。当模型与原型用同一种流体时,$C_{\rho}=C_{\mu}=1$ 故有 $C_{v}=1 / C_{l}$。

由 $C_{v}=\sqrt{C_{l}}$ 和 $C_{v}=1 / C_{l}$ 得 $C_{l}=1$ 即要求模型试验与原场流具有相同的几何尺寸,不合理。所以,该小型试验只满足重力相似准则。

\section*{6.4.4 关于小型模型试验的思考}

实际中的溃口的纵、横断面千差万别,且不规则;溃口的底面也都不是水平或具有稳定斜率的平面,粗糙度各异;溃口各部分的流速分布也是不均匀的;并且溃口形状和大小一般是不断变化的,流速、流量也随着水位和溃口形状的变化而变化。

同时,对于水平长度较大的河流等等,若水平向和竖直向采用相同的比例尺,水深不大的河流的模型流场中水的表面张力的影响将很大。工程上模型试验可以因研究目的的不同,水平向和竖直向采用不同的比例尺。

由运动相似可知,为了保证流体流量比例尺的实现,在做小型试验时我们可以采用变截面槽来模拟溃口部分水流速度的变化;将重物表面打磨光滑或者涂抹润滑剂达到减小水流阻力的效果。

针对建立的数学模型,要适当调整与粘滞力相关的参数值才能更满足重力相似准则,改模型的推广意义更大。当原型流和模型流采用相同的流体的时候,可以通过正交设计,多考虑试验工况来达到试验目的。

\section*{\S 6.5 模型在实际工程中的应用}

假设 溃口几何形状及水流速度与小型试验相似

实际问题的简化:

1) 现场获取的重物形状千差万别,外形规则的重物简化为方块模型考虑,外形不规则的重物采用考虑形状的普适性模型考虑。具体的实际工程中,重物的外形不会太复杂,采用适合于大实心方块各个工况的数学模型即可满足工程精度要求;

2) 原型流动和模型流动具有相同的流体介质,按照模型试验几何比例尺缩小得到模型试验的试验工况;

3) 假定实物沉底后没有翻转和滑移;

4) 假定模型试验中水的高度为 0.275m,保持不变;

5) 假定重物的几何尺寸与大实心方块相似,且平放入水。

原流场属于明渠流,主要考虑重力相似准则,按照大实心方块模型处理该问题即重物在水中的轨迹满足下面方程:

\begin{equation}
\left\{
\begin{aligned}
x &= u_0 t + \frac{u_0}{\alpha_1} e^{-\alpha_1 t} - \frac{u_0}{\alpha_1} \\
y &= (v_0 - B)t + \frac{1}{2}(g - \alpha_3)t^2 - \frac{C}{\alpha_2} e^{\alpha_2 t} - \frac{c}{2} + \frac{C}{\alpha_2} \quad (y \leq -\frac{c}{2})
\end{aligned}
\right.
\tag{6-35}
\end{equation}

其中 B、C 为常数;$\alpha_1 = \frac{C_D \rho S_x}{m}$,$\alpha_2 = \frac{C_D \rho S_y}{m}$,$\alpha_3 = \frac{\rho g V}{m}$。

在大实心方砖落水运动过程模型推导过程中,$\alpha_1, \alpha_2, \alpha_3$ 是与重物自身特性有关的参数,即与重物的质量、密度、几何形状有关,一旦这些特性确定,参数 $\alpha_1, \alpha_2, \alpha_3$ 唯一确定。B、C 是一个积分常数,用已完成的大实心方块模型试验拟合得到,B 为 2.415,C 为 0.0283。

(1) 当溃口水深为 3m 时,几何比例尺为 $C_l = \frac{l_p}{l_m} = \frac{3}{0.275} = 10.91$,模型试验平均流速和流量因受到重力相似准则(弗劳德准则)控制。在相同的重力场作用下,$C_g = 1$,由 $C_v = \sqrt{C_l} = \sqrt{10.91} = 3.303$。

模型水流速度:$v_m = \frac{v_p}{C_v} = \frac{4}{3.303} = 1.211 \, (m/s)$;

模型中重物重心距水面的高度:$h_m = \frac{h_p}{C_l} = \frac{200}{10.91} = 18.33 \, (cm)$;

模型中重物的质量: \( m_{m} = \frac{m_{p}}{C_{l}^{3}} = \frac{1500}{10.91^{3}} = 1.1551 \, (kg) = 1155.1 \, (g) \);

取重物的密度和模型试验材料密度相同即 \( 2300 \, kg/m^{3} \),得到模型重物与大实心方块的线性比例尺为:
\[ C_{l} = \sqrt[3]{\frac{1155.1}{588.8}} = 1.252 \]

\[ \alpha_{1} = \frac{C_{D} \rho S_{x}}{m} = \frac{1.05 \times 1000 \times 0.08 \times 0.08 \times 1.252^{2}}{1.1551} = 9.12 \]

\[ \alpha_{2} = \frac{C_{D} \rho S_{y}}{m} = \frac{1.05 \times 1000 \times 0.08 \times 0.04 \times 1.252^{2}}{1.1551} = 4.56 \]

\[ \alpha_{3} = \frac{\rho g V}{m} = \frac{1 \times 10}{2.3} = 4.348 \]

重物接触水面时竖向初始速度为:
\[ v_{0} = \sqrt{2 g h_{m}} = \sqrt{2 \times 10 \times 0.1833} = 1.915 \, (m/s) \]

水流的速度为:
\[ u_{0} = v_{m} = 1.211 \, (m/s) \]
将得到的参数带入轨迹方程方程中有:
\begin{equation}
\left\{
\begin{aligned}
x &= 1.211 t + \frac{1.211}{9.12} e^{-9.12 t} - \frac{1.211}{9.12} = 1.211 t + 0.1328 e^{-9.12 t} - 0.1328 \\
y &= (1.915 - 2.415) t + \frac{1}{2} (10 - 4.348) t^{2} - \frac{0.0283}{4.56} e^{4.56 t} - 0.02 + \frac{0.0283}{4.56} \\
&\approx -0.5 t + 2.826 t^{2} - 0.025
\end{aligned}
\right.
\tag{6-36}
\end{equation}

计算得到 \( t = 0.4261 \, s \),\( x = 0.3859 \, m \),将计算的结果还原到实际问题中,重物在水中漂移的水平距离为 \( 0.3859 \times 3.303 = 1.27 \, (m) \)。

(2) 当溃口水深为 \( 4 \, m \) 时,几何比例尺为:
\[ C_{l} = \frac{l_{p}}{l_{m}} = \frac{4}{0.275} = 14.545 \]
考虑重力相似准则。模型试验平均流速和流量因受到重力相似准则(弗劳德准则)控制。在相同的重力场作用下,\( C_{g} = 1 \),由 \( C_{v} = \sqrt{C_{l}} = \sqrt{14.545} = 3.814 \),所以,模型流速为:
\[ v_{m} = \frac{v_{p}}{C_{v}} = \frac{5}{3.814} = 1.311 \, (m/s) \]

采用同 (1) 类似的方法,模型中重物重心距水面的高度:
\[ h_{m} = \frac{h_{p}}{C_{l}} = \frac{200}{14.545} = 13.75 \, (cm) \]

模型中重物的质量: \( m_{m} = \frac{m_{p}}{C_{l}^{3}} = \frac{1500}{14.545^{3}} = 0.4875 \, (kg) = 487.5 \, (g) \);

取重物的密度和模型试验材料密度相同即 \( 2300 \, kg/m^{3} \),得到模型重物与大实心方块的线性比例尺为:
\[ C_{l}^{'} = \sqrt[3]{\frac{487.5}{588.8}} = 0.939 \]

\[
\alpha_{1} = \frac{C_{D} \rho S_{x}}{m} = \frac{1.05 \times 1000 \times 0.08 \times 0.08 \times 0.939^{2}}{0.4875} = 12.15
\]

\[
\alpha_{2} = \frac{C_{D} \rho S_{y}}{m} = \frac{1.05 \times 1000 \times 0.08 \times 0.04 \times 0.939^{2}}{0.4875} = 6.08
\]

\[
\alpha_{3} = \frac{\rho g V}{m} = \frac{1 \times 10}{2.3} = 4.348
\]

重物接触水面时竖向初始速度为:
\[ v_{0} = \sqrt{2 g h_{m}} = \sqrt{2 \times 10 \times 0.1375} = 1.658 \, (m/s) \]

水流的速度为: \( u_{0} = v_{m} = 1.311 \, (m/s) \)。将得到的参数带入轨迹方程方程中有:
\begin{equation}
\left\{
\begin{aligned}
x &= 1.311 t + \frac{1.311}{12.15} e^{-12.15 t} - \frac{1.311}{12.15} = 1.311 t + 0.1079 e^{-12.15 t} - 0.1079 \\
y &= (1.658 - 2.415) t + \frac{1}{2} (10 - 4.348) t^{2} - \frac{0.0283}{6.08} e^{6.08 t} - 0.02 + \frac{0.0283}{6.08} \\
&\approx -0.757 t + 2.826 t^{2} - 0.04
\end{aligned}
\right.
\tag{6-37}
\end{equation}

计算得到 \( t = 0.4937 \, s \),\( x = 0.5396 \, m \),将计算的结果还原到实际问题中,重物在水中漂移的水平距离为 \( 0.5396 \times 3.814 = 2.06 \, (m) \)。

\section*{七、总结与展望}

\subsection*{7.1、模型优缺点总结}

力学规律是解决上述问题的关键。本文的模型服从基本力学概念,及运动规律,显然模型是符合实际问题的求解。

在问题1的求解过程中,我们根据固体力学规律推导了阻力影响因素,但解析解很难求得,因此,根据流体中低速运动原理分析,得到轨迹方程,较好的拟合了特定工况下的问题。

在问题2的求解中,采用蒙皮效应来刻画重物形状,引入形状函数来反映重物形状变化对水流阻力的影响,最终建立了能够适应不同情况的普适性模型;并从单个物体模型过渡到两个重物连接后的组件在水中耦合作用,定性分析了压差阻力对两个物体耦合时的重要影响。

在问题3的求解中,联系到实际工程中,把材料归一化,并选取实心方砖重点进行误差分析,据分析结果,提出了更为全面的正交试验方法,可以预测数据拟合精度将不断提高。

在问题4的求解中,基于流体的相似性原理,得出本题中小型模型试验满足的相似准则;根据流体相似即几何相似、运动相似和动力相似分析了小型模型试验中主要相似条件,依此对未来需要进行的研究侧重点作了部分建议;另外,利用相似性原理将实际问题缩尺,把缩尺后的工况应用于已建问题1模型中,确定模型参数,使实际问题得到解决。

\subsection*{7.2、模型的展望}

(1) 将普适性模型中的众多因素的融合,缩减模型参数。

(2) 分析引入的阻力和形状函数,将重物从下落瞬间至封堵溃口的全过程建立统一的控制方程。

(3) 将模型优化成仅和自然状态等可简单测得的经验公式,以便推广至多领域(例如:弹道,航空等)的广义模型。

\section*{参考文献}

[1] 韩世娜,空心块体水流拖拽力的试验研究[D],河海大学,2007.03.

[2] 李霞,陈徐均,孙芦忠,于伟,基于动网格的物体下沉运动的二维数值模拟[J],海工技术,2010年2月第47卷第1期. 4~7.

[3] Evett, J.B., and liu, C. (1987), Fundamental of Fluid Mechanics[M], New York: McGraw-Hill. 381-39.

[4] 毛伟,块体水平拖拽力相互影响试验研究[D],河海大学,2006.02.

[5] 杨火其,王文杰,钱塘江河口异型块体抗冲击稳定特性试验研究[J],长江科学院院报 2001年4月第28卷 第2期. 19~22.

[6] 金永强,水库大坝溃坝险情的分析方法研究[D],河海大学,2008.06.

[7] 庞启秀,水流作用下块体受力试验研究[D],河海大学,2005.03.

[8] CHEN Xujun, SUN Luzhong, SHEN Qing et al. 1998. Non-linearity process method for the anchor chain system of a restrained floating body in ocean[J]. Journal of Nanjing Engineering Institute. 13(1): 40~48. (in Chinese).

[9] 毛佩郁,河湾防冲护岸抛石试验研究[J],水利水运科学研究,1998. 1: 44~52.

[10] Steven R.A, Leech, J.R, Christopher I.T, and Lipscomb, C.M., (2001). Articulated concrete block stability testing[J]. American Water Resources Association. 37(1): 27~34.

[11] 钱宁,万兆惠,泥沙运动力学叫. 北京: 科学出版社,2003,114~122.