\documentclass{article}
\usepackage{ctex}
\usepackage{titling}

\title{空气质量预报二次建模}
\author{}
\date{}

\begin{document}

\begin{center}
\includegraphics[width=0.2\textwidth]{image1.png} \quad
\includegraphics[width=0.2\textwidth]{image2.png} \quad
\includegraphics[width=0.2\textwidth]{image3.png} \quad
\includegraphics[width=0.2\textwidth]{image4.png}
\end{center}

\begin{center}
\textbf{中国研究生创新实践系列大赛} \\
\textbf{“华为杯”第十八届中国研究生} \\
\textbf{数学建模竞赛}
\end{center}

\begin{table}[h]
\centering
\begin{tabular}{l l}
学校 & 华北电力大学 \\
参赛队号 & 21100790043 \\
\hline
队员姓名 & 1. 黄寅峰 \\
 & 2. 胡江谕 \\
 & 3. 贡振华 \\
\end{tabular}
\end{table}

\maketitle

\begin{abstract}
尽管目前已有 WRF-CMAQ 模拟体系对空气质量进行预报,但由于部分污染物生成机理不完全明晰以及排放清单不确定等因素,空气质量的预报结果并不理想。因此,在 WRF-CMAQ 模型一次预报的基础上进行更加准确的二次预报对提前获知大气污染并采取相应控制措施具有深远的意义。

本文针对各监测点空气质量预报数据进行量化分析,对各监测点 2021 年 7 月 13 日至 2021 年 7 月 15 日的污染物浓度值和温度等气象条件特征值进行预测,分析时间、气象条件特征值以及污染物浓度三者之间的关系,并构建相应的数学模型。文章综合采用了 K-means++聚类算法、BP 神经网络模型、随机森林算法、皮尔逊相关性分析等方法研究污染物浓度、气象条件特征值的分析预测问题。

针对问题一,首先提取监测点 A 从 2020 年 8 月 25 日至 8 月 28 日的逐日污染物浓度实测数据,根据附录中空气质量指数 (AQI) 的计算公式与首要污染物的选择标准,通过 MATLAB 编程得到 8 月 25 日至 8 月 28 日的 AQI 分别是 60、46、109、138,除 8 月 26 日无首要污染物外,其余三天的首要污染物均为 $O_{3}$。

针对问题二,首先定性分析气象条件与污染物浓度之间的关系,由于污染物浓度变化直接受限于当时的气象条件,而与监测时间并非直接相关,所以在数据预处理方面直接删除监测点 A 逐小时实测数据表中与异常数据同监测时间的所有数据,不考虑时间轴可能不完整的情况,仅分析气象条件与污染物浓度之间的相关性。此外,参考 AQI 计算公式计算各监测时间的实时 AQI,根据空气质量等级划分表得到各监测时间的实时空气质量等级,并赋予一定的分值得到实时空气质量等级分数,将实时 AQI 与实时空气质量等级分数纳入到气象条件与污染物浓度的相关性分析中,用以描述污染物扩散或沉降会导致 AQI 下降的现象,接着采用 z-score 标准化方法处理数据,并运用 K-means++聚类算法进行聚类分析,最终得到 5 类气象条件。其中,对污染物浓度影响最大的是湿度与风速,湿度、风速越大,空气质量越好,各污染物浓度越小。

针对问题三,首先对数据表中的异常数据进行近邻均值填补,因为污染物浓度与监测时间并非直接相关,而是通过气象条件进行联系,所以可以将原先的二次预测问题变成一个两阶段预测问题。第一阶段是根据预测气象条件的时序预测问题,采用各监测点逐小时气象实测数据与对应的时间数据构建数据集,建立 BP 神经网络模型进行训练,利用训练好的 BP 神经网络预测各监测点 2021 年 7 月 13 日至 2021 年 7 月 15 日的气象条件特征值。第二阶段先分析各监测点逐小时污染物浓度、气象条件实测与一次预报的差异性(一次预报值取当天预报值,数据从 2020 年 7 月 23 日开始),发现误差虽大,但总体走向大致相同。

接着,计算各监测点污染物浓度、气象条件的一次预报误差,构建随机森林预测模型,采用输入为气象条件一次预报误差、输出为污染物浓度一次预报误差的数据集进行训练,将第一阶段预测的气象条件视作实测值与相应时间的一次预报值作差后代入到训练好的随机森林预测模型中得到各监测点从7月13日到7月15日的污染物浓度修正预测值,再与相应时间的一次预报值累加得到相应的二次预报值。运用问题1中的程序计算得到监测点A三天的AQI分别是56、62、80,首要污染物均为$O_{3}$;监测点B三天的AQI分别是20、22、23,均没有首要污染物;监测点C三天的AQI分别是77、74、91,首要污染物均为$O_{3}$。

针对问题四,采用与问题三大致相同的预测流程,但在训练随机森林预测模型时进行改进,将监测点A、A1、A2、A3的污染物浓度、气象条件的一次误差合并为一个数据集进行训练,构建随机森林协同预测模型,并按照问题三中的计算流程得到监测点A三天的AQI分别是48、53、52,7月13日没有首要污染物,14日、15日首要污染物均为$O_{3}$;监测点A1三天的AQI分别是56、64、59,首要污染物均为$O_{3}$;监测点A2三天的AQI分别是49、53、49,7月14日首要污染物为$O_{3}$,13日、15日均没有首要污染物;监测点A3三天的AQI分别是46、49、52,7月15日首要污染物为$O_{3}$,13日、14日均没有首要污染物。最终比较问题三与问题四中监测点A的预测结果,比较了A、A1、A2、A3之间的距离大小与皮尔逊相关性大小,最终得出协同预报模型不能提升污染物浓度预测准确度的结论。

\textbf{关键词:}空气质量预报;K-means++聚类算法;BP神经网络;随机森林;修正预测值;协同预报
\end{abstract}

\section*{目录}

\section{一、问题重述}
\subsection{1.1 问题背景} \dotfill 4
\subsection{1.2 问题提出} \dotfill 4

\section{二、模型假设} \dotfill 5

\section{三、符号说明} \dotfill 5

\section{四、问题一的分析与求解} \dotfill 6

\section{五、问题二的分析与求解} \dotfill 7
\subsection{5.1 问题二的分析与思路} \dotfill 7
\subsection{5.2 数据处理} \dotfill 7
\subsubsection{5.2.1 预处理} \dotfill 7
\subsubsection{5.2.2 数据标准化} \dotfill 7
\subsection{5.3 模型建立与求解} \dotfill 8
\subsubsection{5.3.1 相关性分析} \dotfill 8
\subsubsection{5.3.2 空气质量指数(AQI)的处理} \dotfill 9
\subsubsection{5.3.3 求解} \dotfill 9

\section{六、问题三分析与求解} \dotfill 10
\subsection{6.1 问题分析} \dotfill 10
\subsection{6.2 研究方法} \dotfill 10
\subsection{6.3 利用BP神经网络预测气象条件} \dotfill 10
\subsubsection{6.3.1 BP神经网络原理介绍} \dotfill 10
\subsubsection{6.3.2 建模步骤} \dotfill 11
\subsubsection{6.3.3 数据预处理} \dotfill 12
\subsubsection{6.3.4 神经网络训练} \dotfill 12
\subsubsection{6.3.5 监测点A、B、C的气象预测结果} \dotfill 18
\subsection{6.4 随机森林算法} \dotfill 18
\subsubsection{6.4.1 随机森林算法原理介绍} \dotfill 18
\subsubsection{6.4.2 问题分析} \dotfill 19
\subsubsection{6.4.4 随机森林预测结果} \dotfill 19
\subsection{6.5 二次预报结果} \dotfill 19

\section{七、问题四的分析与求解} \dotfill 21
\subsection{7.1 问题分析} \dotfill 21
\subsection{7.2 研究方法} \dotfill 21
\subsection{7.3 神经网络训练} \dotfill 22
\subsection{7.4 预测结果} \dotfill 27
\subsection{7.5 讨论分析} \dotfill 28

\section{八、参考文献} \dotfill 29

\section{附录} \dotfill 30

\section{问题重述}

\subsection{问题背景}

随着经济生活生产和工业化的发展,空气污染问题越来越严重。空气污染是指空气中污染物浓度达到有害水平,以至于破坏生态系统和人类正常生存发展条件,对人和物造成危害的现象。空气污染物按其存在状态可分为气溶胶污染物和气体污染物。研究表明,长期暴露于高浓度的空气污染物之中会对人体免疫系统、神经系统和呼吸系统带来永久性的健康危害,甚至极有可能导致癌症。空气污染是全球公共卫生领域最大的环境风险,每年造成数百万人过早死亡。而开展环境空气质量预测工作是及时应对重污染天气的重要技术保障,对区域大气污染的联合减排具有指导意义。

空气质量指数(Air Quality Index, AQI)是空气污染物总效应的无量纲指数,它反映了空气的污染水平以及对人体健康的影响程度。在公共卫生领域及城市地区,主要有二氧化硫($\mathrm{SO}_{2}$)、二氧化氮($\mathrm{NO}_{2}$)、颗粒物($\mathrm{PM}_{10}$、$\mathrm{PM}_{2.5}$)、臭氧($\mathrm{O}_{3}$)和一氧化碳($\mathrm{CO}$)等空气污染成分。环境质量的评价方式一般是将 AQI 指数与污染物浓度分级标准进行比较,再对 AQI 指数进行等级划分,包括优、良、轻度污染、中度污染、重度污染、严重污染六个等级。

传统的空气质量预测主要分为数值预测和统计学预测,数值预测是指通过已有的空气质量数据来推导总结出一系列物理学和化学状态方程,它们通常是高阶微分方程,通过导入相应的参数即可得到未来的空气质量数值,但是这样的预测方式对计算力的要求较高,而且考虑的影响方面比较有限;统计学预测通过数学建模来分析已有数据,像非线性数值分析、灰色分析、切比雪夫展开等,但是统计学预测周期长且操作复杂,因此难以及时迅速准确地提供空气质量数据的相关信息。

目前,针对空气质量指数的预测问题的研究一般基于三类。第一类为统计模型,如主成分回归模型(PCR)、多元线性回归模型(MLR)等;第二类为机器学习模型,如 LSTM 长短期记忆网络、SVR 算法\cite{ref1}等;第三类为基于大气运动学的数学建模方法,以及在此理论的基础上衍生而来的其他各种组合预测方法。以上模型各自适合不同时间跨度或长度的序列数据,比如在预测 AQI 数据时,采用 ARIMA 模型做短期预测相比于大多数机器学习方法而言更容易取得优异的效果,但是在做长期预测时采用机器学习方法往往更占优势。空气质量数据主要来自地面监测、气象卫星等采集站点,是典型的时序数据,可以通过对空气质量数据进行数理分析来进行空气质量预测\cite{ref2}。

\subsection{问题提出}

由于模糊的气象场以及排放清单的不确定性,另外对包括臭氧在内的污染物生成机理的不完全明晰,导致一次预报(WRF-CMAQ)预报模型的结果并不理想,因此有了二次建模,我们依据实测数据以及一次建模预测数据,进行二次建模,具体需依次解决以下四个问题:

问题 1:附件 1 给出了检测点 A 的 6 类污染物的日监测浓度,我们需要按照附录所给的计算方法,计算监测点 A 从 2020 年 8 月 25 日到 8 月 28 日每天实测的 AQI 和首要污染物。

问题 2:不同的气象条件会对 AQI 值产生不同的影响,我们需要对所给的每天每小时的气象条件进行聚类处理,建立气象条件与 AQI 值变化的联系,并根据气象条件对 AQI 值的影响程度,对气象条件进行合理的分类,阐述清楚各类气象条件的具体特征。

问题 3:根据题中所给数据,我们需要建立一个同时适用于 A、B、C 三个监测点(监测点两两间直线距离 $>100\mathrm{km}$,忽略相互影响)的二次预报数学模型,用来预测未来三天 6

种常规污染物单日浓度值,题目要求二次预报模型预测结果中 AQI 预报值的最大相对误差应尽量小,且首要污染物预测准确度尽量高。之后我们需要利用所建立的二次预报模型预测监测点 A、B、C 在 2021 年 7 月 13 日至 7 月 15 日 6 种常规污染物的单日浓度值,计算相应的 AQI 和首要污染物。

问题 4: 考虑到相邻区域的污染物浓度往往具有一定的相关性,我们需要建立包含 A、A1、A2、A3 四个监测点的协同预报模型,题目要求二次模型预测结果中 AQI 预报值的最大相对误差应尽量小,且首要污染物预测准确度尽量高。之后我们需要利用所建立的二次协同预报模型预测监测点 A、A1、A2、A3 在 2021 年 7 月 13 日至 7 月 15 日 6 种常规污染物的单日浓度值,计算相应的 AQI 和首要污染物。并将此模型与前面所建立的二次预报模型作比较,判断其能否提高对监测点 A 的污染物浓度预报的准确度。

\section{二、模型假设}

本文模型基于以下合理假设:

假设一:本题中所提供的数据是基本真实可靠的,本题中坏数据与缺失的数据对预测结果的影响忽略不计;

假设二:A、B、C 三个监测点两两间直线距离 > 100km,忽略相互之间的影响。

\section{三、符号说明}

文中使用到的符号含义如下

\begin{tabular}{|c|c|}
\hline 符号 & 含义 \\
\hline IAQI\(_P\) & 污染物P的空气质量分指数,结果进位取整数 \\
\hline C\(_P\) & 污染物P的质量浓度值 \\
\hline BP\(_{Hi}\), BP\(_{Lo}\) & 与C\(_P\)相近的污染物浓度限值的高位值与低位值 \\
\hline IAQI\(_{Hi}\), IAQI\(_{Lo}\) & 与BP\(_{Hi}\), BP\(_{Lo}\)对应的空气质量分指数 \\
\hline x & 待处理原始数据 \\
\hline \(\mu\) & 原始数据均值 \\
\hline \(\delta\) & 原始数据标准差 \\
\hline x\(^*\) & z-score 标准化后的数据 \\
\hline x\(_{min}\) & 原始数据最小值 \\
\hline x\(_{max}\) & 原始数据最大值 \\
\hline x\(_{scaled}\) & 最小-最大标准化后的数据 \\
\hline
\end{tabular}

\section{四、问题一的分析与求解}

从监测点A中提取2020年8月25日到8月28日每天实测的6种污染物的监测浓度数据。

计算空气质量分指数 (IAQI)
\begin{equation}
\text{IAQI}_p = \frac{\text{IAQI}_{\text{Hi}} - \text{IAQI}_{\text{Lo}}}{\text{BP}_{\text{Hi}} - \text{BP}_{\text{Lo}}} \cdot (C_p - \text{BP}_{\text{Lo}}) + \text{IAQI}_{\text{Lo}}
\tag{4-1}
\end{equation}

计算空气质量指数 (AQI)
\begin{equation}
\text{AQI} = \max\{\text{IAQI}_1, \text{IAQI}_2, \text{IAQI}_3, \dots, \text{IAQI}_n\}
\tag{4-2}
\end{equation}

根据公式 (4-1)、(4-2),利用 MATLAB 编写相应的程序,再导入所提取的日监测浓度数据,得到监测点A从2020年8月25日到8月28日每天实测的AQI和首要污染物,程序见附录,计算结果表4-1所示:

\begin{table}[h]
\centering
\caption{实测的AQI和首要污染物}
\begin{tabular}{|c|c|c|c|}
\hline
\multirow{2}{*}{监测日期} & \multirow{2}{*}{地点} & \multicolumn{2}{c|}{AQI计算} \\
\cline{3-4}
 &  & AQI & 首要污染物 \\
\hline
2020/8/25 & 监测点A & 60 & $O_3$ \\
\hline
2020/8/26 & 监测点A & 46 & 无 \\
\hline
2020/8/27 & 监测点A & 109 & $O_3$ \\
\hline
2020/8/28 & 监测点A & 138 & $O_3$ \\
\hline
\end{tabular}
\end{table}

\section{五、问题二的分析与求解}

气象条件往往指不同天气现象下的水热情况,衡量特征主要包括:温度、相对湿度、气压、风速、风向、降水量、蒸发量、冻土深度、积雪深度、日照时长等。不同的气象条件会对污染物的排放情况有着不同的影响,当气象条件利于污染物扩散或沉降时,AQI 会下降,空气质量明显改善。基于这种特征,本章对附件 1 中监测点 A 逐小时污染物浓度与气象实测数据进行特性分析,并在此基础上对气象条件进行分类。

\subsection{5.1 问题二的分析与思路}

本题要求根据气象条件对污染物浓度的影响程度,合理划分气象条件,同时还需要阐述分类后各类气象条件的特征。换句话说,主要就是要探究不同的污染物浓度下气象条件各因素的分布情况。

本文所提及的污染物包括 $SO_2$、$NO_2$、$PM_{10}$、$PM_{2.5}$、$O_3$、$CO$。在不同的气象条件下,上述污染物会有着不同的扩散、聚集、沉降、上升等情况,所以它们的浓度实测值的变化其实是受制于气象条件的变化,也就是说气象条件可以看作成自变量,污染物浓度可以看作成因变量。整个问题就抽象成根据因变量划分自变量,将自变量定义域划分成若干个区间,同一区间的自变量、因变量关系相对一致,不同区间的自变量、因变量关系差异较大。

由于数据中所给的气象条件实测特征有 5 类,包括温度、相对湿度、气压、风速、风向,因此自变量可以看成 $1 \times 5$ 的向量;污染物有 6 类,所以因变量可以看成 $1 \times 6$ 的向量。但在实际生活中,我们的评判标准并不是污染物的浓度值,往往是根据空气质量或者天气情况等综合评价指标来反映气象条件的分布。因此,我们希望借助不同污染物浓度所对应的 AQI 值来对气象条件进行划分,得到与 AQI 相关的气象条件分类情况。

\subsection{5.2 数据处理}

\subsubsection{5.2.1 预处理}

在监测点 A 逐小时污染物浓度与气象实测数据表中,存在样本数据缺失、数据不合理、时间轴不完整的情况,而污染物的浓度变化与气象条件是直接相关的,但与时间轴的关系是借助气象条件这一中间因素进行挂钩,其与时间轴是一个间接关系。因此,在本题探究气象条件与污染物浓度的关系问题中,时间的相关性较低,可以删去数据表中所有存在异常数据的样本,仅保留数据合理且完整的样本数据用于分类。

\subsubsection{5.2.2 数据标准化}

由于 5 类气象条件特征值的量级不同,采用 $z$-score 标准化方法将 5 类气象条件特征值化为同一量级下,标准化公式如下:

\begin{equation}
x^* = \frac{x - \mu}{\delta}
\tag{5-1}
\end{equation}

其中:$\mu$ 为数据平均值;$\delta$ 为数据标准差;$x^* > 0$ 表示高于数据平均水平,反之表示低于数据平均水平。此外,对表中的污染物浓度数据进行整合,参照公式 (4-1)、(4-2) 的计算流程,得出各小时的空气质量指数 AQI,用于后续的分类计算,这里给出部分数据处理结果,如表 5-1、表 5-2:

\begin{table}[h]
\centering
\caption{五类特征平均水平}
\begin{tabular}{|c|c|}
\hline
特征 & 平均水平 \\
\hline
温度 & $25.130^{\circ}\mathrm{C}$ \\
\hline
湿度 & $68.193\%$ \\
\hline
气压 & $1010.607 \mathrm{MBar}$ \\
\hline
风速 & $1.408 \mathrm{~m} / \mathrm{s}$ \\
\hline
风向 & $155.645^{\circ}$ \\
\hline
\end{tabular}
\end{table}

\begin{table}[h]
\centering
\caption{数据处理结果}
\begin{tabular}{|c|c|c|c|c|c|c|}
\hline
地点 & 温度(pu) & 湿度(pu) & 气压(pu) & 风速(pu) & 风向(pu) & AQI \\
\hline
监测点A & -0.299 & 0.504 & 1.505 & -0.544 & 1.458 & 94 \\
\hline
监测点A & -0.624 & 0.813 & 1.076 & -0.874 & -0.871 & 80 \\
\hline
监测点A & -0.809 & 0.915 & 0.290 & -0.874 & -0.531 & 75 \\
\hline
监测点A & -0.902 & 0.915 & -0.354 & 0.445 & -0.443 & 49 \\
\hline
监测点A & -0.856 & 0.813 & -0.640 & 0.445 & -0.478 & 55 \\
\hline
监测点A & -0.809 & 0.813 & -0.640 & 1.765 & -0.456 & 55 \\
\hline
监测点A & -0.717 & 0.504 & -0.426 & -0.214 & -0.108 & 57 \\
\hline
监测点A & -0.717 & 0.607 & 0.433 & -0.544 & -0.929 & 67 \\
\hline
监测点A & -0.809 & 0.813 & 1.291 & 1.765 & 1.530 & 72 \\
\hline
监测点A & -0.948 & 0.813 & 1.577 & 1.435 & -0.853 & 73 \\
\hline
监测点A & -0.763 & 0.710 & 1.291 & 0.115 & -0.440 & 63 \\
\hline
监测点A & -0.392 & 0.401 & 1.076 & -0.874 & 0.693 & 67 \\
\hline
监测点A & -0.067 & 0.195 & 0.075 & -0.544 & 0.764 & 73 \\
\hline
监测点A & 0.489 & -0.525 & -0.068 & 0.115 & -0.880 & 64 \\
\hline
监测点A & 1.185 & -1.142 & -0.783 & -0.214 & -0.963 & 50 \\
\hline
监测点A & 1.788 & -1.656 & -1.212 & 0.445 & -0.957 & 74 \\
\hline
监测点A & 1.788 & -1.862 & -1.284 & 0.445 & -0.947 & 75 \\
\hline
监测点A & 1.556 & -1.553 & -1.141 & -0.214 & 1.469 & 95 \\
\hline
监测点A & 1.139 & -1.245 & -1.141 & 0.445 & 1.518 & 70 \\
\hline
监测点A & 0.768 & -0.833 & -0.926 & -2.524 & 1.423 & 78 \\
\hline
\end{tabular}
\end{table}

\textbf{注:表中的正、负值分别代表高于、低于平均水平}

\subsection{5.3 模型建立与求解}

\subsubsection{5.3.1 相关性分析}

针对5类标准化后的气象条件特征,进行相关性分析,计算其相关性矩阵,结果如表5-3。

\begin{table}[h]
\centering
\caption{五类特征相关性分析}
\begin{tabular}{|c|c|c|c|c|c|}
\hline
特征 & 温度 & 湿度 & 气压 & 风速 & 风向 \\
\hline
温度 & 1.000 & 0.141 & -0.828 & 0.076 & 0.146 \\
\hline
湿度 & 0.141 & 1.000 & -0.404 & -0.240 & 0.044 \\
\hline
气压 & -0.828 & -0.404 & 1.000 & -0.030 & -0.195 \\
\hline
风速 & 0.076 & -0.240 & -0.030 & 1.000 & -0.050 \\
\hline
风向 & 0.146 & 0.044 & -0.195 & -0.050 & 1.000 \\
\hline
\end{tabular}
\end{table}

从表中可以发现温度与气压有着极强的负相关,温度越高,气压越低;湿度与气压也有着较强的负相关,但相关性程度比不上温度与气压的关系程度。这两个结论对后续的分类结果有一定的参考意义。

\subsection{5.3.2 空气质量指数(AQI)的处理}

空气质量等级范围根据 AQI 进行划分,同时对进行打分,并将相关的空气质量分数纳入到气象条件的分类依据当中,相关数据见表 5-4。

\begin{table}[h]
\centering
\caption{空气质量等级及对应的分数值与 AQI 范围}
\begin{tabular}{|c|c|c|c|c|c|c|}
\hline
空气质量等级 & 优 & 良 & 轻度污染 & 中度污染 & 重度污染 & 严重污染 \\
\hline
分数 & 100 & 80 & 60 & 40 & 20 & 0 \\
\hline
AQI & $[0,50]$ & $[51,100]$ & $[101,150]$ & $[151,200]$ & $[201,300]$ & $[301,+\infty)$ \\
\hline
\end{tabular}
\end{table}

\subsection{5.3.3 求解}

样本的聚类特征值包括:空气质量分数、AQI、温度、湿度、气压、风速、风向,借助 matlab 采用 kmeans++ 算法进行聚类分析(聚类结果中的负数代表低于平均水平,正值表示高于平均水平),处理聚类结果并将其可视化,如图 5-1:

\begin{figure}[h]
\centering
\includegraphics[width=\textwidth]{image.png}
\caption{气象条件分类结果可视化}
\end{figure}

分析上图,得到以下气象条件分类特征:

- 类别 1:高温;湿度大;气压低;风速大;西北风、西南风、西风
- 类别 2:低温;湿度小;气压高;风速小;东北风、东南风、东风
- 类别 3:高温;湿度小;气压高;风速小;东北风、东南风、东风
- 类别 4:高温;湿度小;气压低;风速小;东北风、东南风、东风
- 类别 5:高温;湿度小;气压低;风速小;西北风、西南风、西风

在该分类结果中,温度、气压、风向与空气质量指数的相关性不强;而湿度、风速与空气质量指数呈现了较强的相关性。湿度、风速越大,空气质量越好,这可能是因为湿度大、风速大的气象条件有利于污染物扩散、沉淀,从而降低了大气中污染物浓度,改善了空气质量。但总体来看,该分类的结果不是很理想,未能找到温度、气压、风向对污染物浓度影响的明确关系。

\section{问题三分析与求解}

\subsection{问题分析}

本题根据附件 1 和附件 2 中的数据,建立一个同时适用于 A、B、C 三个监测点的二次预报模型,用来预测未来 3 天内 6 种常规污染物的单日浓度值。然而附件中的实测数据存在时间轴缺失以及数据错误的情况,为减少其影响,需对原始数据进行预处理,从而保证经过处理修正后污染物浓度数据能够具有良好的连续性,使得实测数据与一次预测数据时间上能够对应,方便后续的进一步处理。

\subsection{研究方法}

本节根据已有的 A、B、C 三个监测点污染物浓度与气象一次预报数据和实测数据,首先分析了气象条件随时间的变化规律,通过分析气象条件预报值与实测值的数据,我们发现气象条件随时间有一定的变化规律,而污染物浓度的数据随时间无明显变化规律,但却随气象条件的变化有一定的规律,相关的对比图见附录。因此,将原先需要解决的二次预报问题转化为两阶段预测问题,第一阶段是气象条件的时序预测问题,我们采用神经网络模型预测未来三天的气象条件数据;第二阶段是根据气象条件一次预报误差数据与污染物浓度一次预报误差进行数据进行训练,并根据第一阶段预测出的气象条件变化值采用随机森林预测模型预测出其对应的污染物浓度变化值,然后修正污染物二次预报数据得到二次预报结果具体的预测流程如图 \ref{fig:6-1}。

\begin{figure}[h]
    \centering
    \includegraphics[width=\textwidth]{image.png}
    \caption{空气质量二次预报流程}
    \label{fig:6-1}
\end{figure}

\subsection{利用 BP 神经网络预测气象条件}

\subsubsection{BP 神经网络原理介绍}

人工神经网络有很多强大的功能,它可以使用提供给它的相应数据来挖掘两者之间的未知规律,最后在新输入数据已知时使用挖掘出的规律作为工具来推断相应的结果。BP(Back Propagation)神经网络是按照误差逆向传播算法训练的多层前馈神经网络,是一种典型的深度学习模型,是近些年来应用最广泛的深度学习模型。其通过反向传播算法和最速下降法不断地调整网络中的参数,以使网络的实际输出值和期望输出值的误差达到最小,其突出优点是具有很强的非线性映射能力和柔性的网络结构。

BP 神经网络由一个输入层、一个或多个隐含层和一个输出层组成,每层含有若干个节点,各层节点通过加权路径来与相邻层的节点进行链接。当预测目标变量为分类变量时,输出层包含多个输出结点,但是当目标变量为区间型变量时,则只包含一个输出节点。

BP 神经网络的工作方法主要分为两个过程。首先是信号的前向传播,信号从输入层输入到隐含层,通过隐含层的计算输出新的权重,最后再到输出层;其次是误差的反向传播,BP 神经网络模型根据输出层的输出结果计算预测数据与实际数据的误差,该误差用来依次调节从隐含层到输出层的权重和偏差以及输入层到隐含层的权重和偏差。如此反复进行,直到输出结果与期望的输出结果均方误差达到一个合理的范围内则结束训练。

总而言之,BP 神经网络的核心思想就是根据得到的结果计算误差,并通过反馈误差,不断修改权重和阈值,从而使输出结果的误差最小。

\begin{figure}[h]
\centering
\includegraphics[width=\textwidth]{bp_neural_network_model.png}
\caption{BP 神经网络模型}
\end{figure}

BP 神经网络的步骤主要为

1) 通过分配各层中每一个神经元的随机权重和偏差值,对前向传播过程初始化。

2) 在得到每个神经元和偏差的和之后,在每个神经元中应用激活函数,将输出结果传递到下一层的神经元。

3) 若该层是输出层,则在此处应用激活函数,即可得到该模型的输出数据。

4) 基于训练集,通过采用实际输出结果减去激活函数的输出来识别每个输出神经元的误差,并计算总误差。

5) 对其计算偏导,将总的误差反向传递。

6) 根据偏导数对权重进行更新,对更新权重后的模型重复步骤 1) 到步骤 5)。

\subsection{6.3.2 建模步骤}

1) 模型设计

本文采用三层结构建立模型对天气状况进行预测,采用附件中的气象实测值作为输出、对应的时间信息作为输入进行训练,最终对 7 月 13 日至 15 日的气象条件进行预测。

2) 数据归一化处理

首先对数据采取归一化处理,以消除各因素因量级不同而对预测精度产生影响。此处采用最小-最大标准化方法,其公式为:
\begin{equation}
x_{scaled} = \frac{x - x_{min}}{x_{max} - x_{min}}
\tag{6-1}
\end{equation}
处理后所有指标的数值都在 0-1 之间。

3)模型训练

在对数据归一化处理后,在同一训练样本下逐个进行模型训练,得到模型在取不同隐含层节点数时的误差。

4)得出实验结果

### 6.3.3 数据预处理

分析附件中的数据,发现个别数据存在异常或丢失的情况,因此为了保证分析的结果合理性,需要对异常数据进行识别删除,而对缺失数据进行插值处理。实测数据中个别污染物浓度的实测值小于 0,考虑实际情况可以判断其为异常数据,因此需要对其进行剔除。对于缺失数据的处理,常见的处理方法有移动平均法、直接删除法和指数平滑法等。由于实测数据按时间排布,如果不考虑缺失数据,则会对后续的处理带来影响,本文的处理方法为取相邻时间的几组数据取平均值近似代替所缺数据。

### 6.3.4 神经网络训练

采用 MATLAB 神经网络工具箱进行气象条件的时序预测,分别对监测点 A、B、C 的温度、湿度、气压、风速以及风向的实测值进行了训练拟合,得到如下结果。

#### (1)以温度实测为输出的神经网络训练结果

\begin{figure}[h]
\centering
\includegraphics[width=\textwidth]{image.png}
\caption{监测点 A 温度训练情况}
\end{figure}

\begin{figure}[h]
    \centering
    \includegraphics[width=\textwidth]{image1.png}
    \caption{(b) 监测点B 温度训练情况}
\end{figure}

\begin{figure}[h]
    \centering
    \includegraphics[width=\textwidth]{image2.png}
    \caption{(c) 监测点C 温度训练情况}
\end{figure}

图 6-3 以温度为特征的神经网络训练情况

如图所示为监测点 A、B、C 的温度实测值随时间的变化情况,其中,外围波动较大的曲线为温度的实测值,而中央的红色曲线为根据实测值拟合出的等效值,其可作为今后预报时的参考,整体的训练效果比较良好。从图中可以发现,实测数据曲线波动较大,呈不规则的毛刺状。但通过分析曲线走势可以发现,温度的实测数据大体呈周期性变化,且波动周期在一年左右,数据起始时间为四月份,根据其走向可得,在八月份左右温度达到最大,而在十二月左右温度达到最低,这与实际情况相吻合。这表明,同一地点每年同一时期的温度大体相当,即使略有误差也不会有太大的变动。温度随时间规律性的变化可为之后污染物浓度二次预报提供一定的参考。

\subsection{以湿度实测为输出的神经网络训练结果}

\begin{figure}[h]
    \centering
    \includegraphics[width=\textwidth]{image1.png}
    \caption{(a) 监测点 A 湿度随时间的训练情况}
\end{figure}

\begin{figure}[h]
    \centering
    \includegraphics[width=\textwidth]{image2.png}
    \caption{(b) 监测点 B 湿度随时间的训练情况}
\end{figure}

图 6-4 以湿度为特征的神经网络训练情况

由于监测点 C 的实测数据缺失,所以此处不再分析监测点 C 的变化情况,仅对监测点 A、B 的实测数据进行分析。通过分析可得,相较于温度的预报数据来说,湿度的实测值波动程度较大,通过分析监测点 A 的湿度数据拟合曲线可以发现,监测点 A 处的实测数据波动比较大,但整体走势在考虑误差的情况下仍是周期性变化的,相较而言,监测点 B 的湿度变化情况则更有规律,明显呈周期性变化,因此结合两地实测数据,可以认为湿度随时间是周期性变化的,且变化周期也是一年。

(3) 以气压实测为输出的神经网络训练结果

\begin{figure}[h]
    \centering
    \includegraphics[width=\textwidth]{image1.png}
    \caption{(a) 监测点 A 气压随时间的训练情况}
\end{figure}

\begin{figure}[h]
    \centering
    \includegraphics[width=\textwidth]{image2.png}
    \caption{(b) 监测点 B 气压随时间的训练情况}
\end{figure}

\begin{figure}[h]
    \centering
    \includegraphics[width=\textwidth]{image3.png}
    \caption{(c) 监测点 C 气压随时间的训练情况}
\end{figure}

图 6-5 以气压为特征的神经网络训练情况

通过分析监测点 A、B、C 的气压实测值随时间的变化情况可以发现,A 点的拟合曲线变化比较规律,随时间呈明显的规则周期性变化,B 点和 C 点的拟合曲线部分区域波动较大,但整体上来看,其走势仍是有一定的规律,也是周期性变化的,说明拟合效果良好。可以认为监测点 A、B、C 处的气压变化情况是随时间周期性波动的,其变化周期也是一

年,其一年前同一时期的数据可为现在的预报提供参考。

(4) 以风速实测为输出的神经网络训练结果

\begin{figure}[h]
    \centering
    \includegraphics[width=\textwidth]{image1.png}
    \caption{(a) 监测点 A 风速随时间的训练情况}
\end{figure}

\begin{figure}[h]
    \centering
    \includegraphics[width=\textwidth]{image2.png}
    \caption{(b) 监测点 B 风速随时间的训练情况}
\end{figure}

\begin{figure}[h]
    \centering
    \includegraphics[width=\textwidth]{image3.png}
    \caption{(c) 监测点 C 风速随时间的训练情况}
\end{figure}

图 6-6 以风速为特征的神经网络训练情况

通过分析监测点 A、B、C 对风速的实测数据可以发现,相对于前面几组数据,风速的实测数据波动较大,但其拟合曲线的变化情况却较为平缓,即使偶尔有波动,波动的幅

\begin{figure}[h]
    \centering
    \includegraphics[width=\textwidth]{image1.png}
    \caption{(a) 监测点 A 风向随时间的训练情况}
\end{figure}

\begin{figure}[h]
    \centering
    \includegraphics[width=\textwidth]{image2.png}
    \caption{(b) 监测点 B 风向随时间的训练情况}
\end{figure}

\begin{figure}[h]
    \centering
    \includegraphics[width=\textwidth]{image3.png}
    \caption{(c) 监测点 C 风向随时间的训练情况}
\end{figure}

图 6-6 以风向为特征的神经网络训练情况

\newpage

17

通过分析监测点 A、B、C 对风向的实测数据可以发现,其数据波动较大,训练效果不理想,并不能明显发现有什么变化规律。

综上,三个监测点的温度、湿度、气压、风速随着时间会呈明显的周期性变化,而风向随时间无明显变化规律,影响风向的因素有水平气压梯度力、近地面摩擦力以及地转偏向力等,因此可能受到当地的地形、建筑、温度等多重因素的共同影响。

\subsection{6.3.5 监测点 A、B、C 的气象预测结果}

详见附录。

\subsection{6.4 随机森林算法}

\subsubsection{6.4.1 随机森林算法原理介绍}

由于题中所给数据中存在着一些异常值与失真值,这些值的存在可能会对最后预测结果产生影响,所以在利用五类气象数据进行污染物浓度预测时,采用随机森林预测算法可以保证对于噪声数据以及一些异常值有很高的容忍度,能够很大程度上避免图像出现过拟合的情况,并且处理过程容易实现,效率高,其有三大特点:特征的随机性;训练集的随机性;组合投票原则和取均值原则。

随机森林算法是一种区别于传统神经网络的回归分类算法,它的基分类器为决策树算法,由于本题所给数据为一系列离散型数据,所以此时的决策树可以视为一种分类树,并把基尼指数作为过程中使用的分裂规则。

可以用下面得到公式来计算决策树的基尼指数:

\begin{equation}
P_{i}=\frac{\left|C_{i, D}\right|}{|D|}
\tag{6-2}
\end{equation}

\begin{equation}
Gini(D)=1-\sum_{i=1}^{m} P_{i}^{2}
\tag{6-3}
\end{equation}

其中,D 是气象条件的原始数据集,P_{i} 是数据集 D 中记录属于 C_{i} 类的概率,|D| 是数据集 D 中记录的总条数,|C_{i, D}| 是数据集 D 中属于 C_{i} 类的记录的条数,m 是数据集中类的个数。

针对气象条件的原始数据集,我们假设里面含有 K 个特征,这 K 个特征相互之间并不产生影响。具体的随机森林算法有以下几个步骤:

(1)首先,采用 Bagging 的思想,从气象条件的原始数据集中进行有放回的随机挑选,进行 n 次过程,产生与原始数据集数量一致的 n 个气象条件训练子集 \(\{D_{1}, D_{2}, D_{3}, \ldots D_{n}\}\)。

(2)然后,开始构建决策树,从上一步中得到的 n 个气象条件训练子集中,选择一个作为决策树的训练子集,并选取 k 个特征(注意:k < K),利用这 k 个特征选取用于决策树分裂的一种最优分裂方式,并不断重复此过程,直到到达系统满足的条件才停止。

(3)通过以上过程会产生 n 棵完全生长的决策树,并将它们进行一定的组合产生我们所需要的随机森林模型。

(4)最后将所选取的测试样本输入到随机森林模型中,其产生的结果为简单多数投票或取平均值。算法框架图如图 6-7 所示:

\begin{figure}[h]
    \centering
    \includegraphics[width=\textwidth]{random_forest_diagram.png}
    \caption{随机森林算法框架图}
    \label{fig:random_forest_diagram}
\end{figure}

\subsection{6.4.2 问题分析}

根据上述步骤建立关于气象条件数据的随机森林模型,原始数据集为5类与污染物浓度相关的气象条件,分别是温度、湿度、大气压、风向以及风速。本阶段预测模型的输出变量即预测目标为监测点A、B、C三点在2021年7月13日至7月15日的6种常规污染物即二氧化硫(SO\(_2\))、二氧化氮(NO\(_2\))、颗粒物(PM\(_{10}\)、PM\(_{2.5}\))、臭氧(O\(_3\))和一氧化碳(CO)的逐小时浓度误差值。

值得注意的是,对于臭氧O\(_3\)进行预测的过程中,因其属于二次污染物,也并非来自污染源的直接排放,而是在大气中经过一系列化学及光化学反应生成的,所以其实际的污染物浓度变化不单单与气象条件有关,与部分一次污染物浓度变化也有一定的关系,例如NO\(_2\),所以在模型进行训练时需要考虑这方面因素。

\subsection{6.4.3 数据准备}

训练随机森林模型所需的参数主要包括2部分,分别是气象条件一次预报误差、污染物浓度一次预报误差。运用近邻均值填补算法处理数据。

\subsection{6.4.4 随机森林预测结果}

采用MATLAB随机森林工具包进行训练,相关的程序工具包见附件;相关的预测结果附录。

\subsection{6.5 二次预报结果}

根据第二阶段随机森林预测模型预测出的污染物浓度的逐小时浓度修正值与污染物浓度的一次预报值得到2021年7月13日至7月15日各监测点的污染物浓度二次预报值,带入到问题1中的程序得到监测点A、B、C的二次预报结果,见表6-1、6-2、6-3。

\begin{table}[h]
    \centering
    \caption{监测点A二次模型预测}
    \label{tab:monitoring_point_A}
    \begin{tabular}{|c|c|c|c|c|c|c|c|c|}
        \hline
        \multirow{2}{*}{预报日期} & \multirow{2}{*}{地点} & \multicolumn{7}{c|}{二次模型日值预测} \\
        \cline{3-9}
        & & SO\(_2\) & NO\(_2\) & PM\(_{10}\) & PM\(_{2.5}\) & O\(_3\)最大八小时滑动平均 & CO & AQI \\
        & & (\(\mu\)g/m\(^3\)) & (\(\mu\)g/m\(^3\)) & (\(\mu\)g/m\(^3\)) & (\(\mu\)g/m\(^3\)) & (\(\mu\)g/m\(^3\)) & (mg/m\(^3\)) & 首要污染物 \\
        \hline
        2021/7/13 & 监测点A & 7.20923 & 12.64933 & 18.57122 & 4.64744 & 106.68426 & 0.33421 & 56 & O\(_3\) \\
        \hline
        2021/7/14 & 监测点A & 6.67690 & 12.16623 & 20.23076 & 5.39122 & 113.47266 & 0.50235 & 62 & O\(_3\) \\
        \hline
        2021/7/15 & 监测点A & 6.83677 & 11.52670 & 21.94202 & 5.56853 & 135.23068 & 0.45164 & 80 & O\(_3\) \\
        \hline
    \end{tabular}
\end{table}

\begin{table}
\centering
\caption{监测点B二次模型预测}
\begin{tabular}{c c c c c c c c c}
\hline
\multirow{2}{*}{预报日期} & \multirow{2}{*}{地点} & \multirow{2}{*}{$SO_{2}$ ($\mu g/m^{3}$)} & \multirow{2}{*}{$NO_{2}$ ($\mu g/m^{3}$)} & \multirow{2}{*}{$PM_{10}$ ($\mu g/m^{3}$)} & \multirow{2}{*}{$PM_{2.5}$ ($\mu g/m^{3}$)} & \multirow{2}{*}{$O_{3}$最大八小时滑动平均 ($\mu g/m^{3}$)} & \multirow{2}{*}{CO ($mg/m^{3}$)} & \multirow{2}{*}{AQI} & \multirow{2}{*}{首要污染物} \\
 & & & & & & & & & \\
\hline
2021/7/13 & 监测点B & 5.72546 & 10.15425 & 15.35623 & 3.54625 & 39.51635 & 0.68623 & 20 & 无 \\
\hline
2021/7/14 & 监测点B & 6.054856 & 7.18564 & 17.15366 & 4.58613 & 42.54631 & 0.61556 & 22 & 无 \\
\hline
2021/7/15 & 监测点B & 5.98215 & 8.68653 & 16.25621 & 3.98462 & 44.51625 & 0.58521 & 23 & 无 \\
\hline
\end{tabular}
\end{table}

\begin{table}
\centering
\caption{监测点C二次模型预测}
\begin{tabular}{c c c c c c c c c}
\hline
\multirow{2}{*}{预报日期} & \multirow{2}{*}{地点} & \multirow{2}{*}{$SO_{2}$ ($\mu g/m^{3}$)} & \multirow{2}{*}{$NO_{2}$ ($\mu g/m^{3}$)} & \multirow{2}{*}{$PM_{10}$ ($\mu g/m^{3}$)} & \multirow{2}{*}{$PM_{2.5}$ ($\mu g/m^{3}$)} & \multirow{2}{*}{$O_{3}$最大八小时滑动平均 ($\mu g/m^{3}$)} & \multirow{2}{*}{CO ($mg/m^{3}$)} & \multirow{2}{*}{AQI} & \multirow{2}{*}{首要污染物} \\
 & & & & & & & & & \\
\hline
2021/7/13 & 监测点C & 7.36452 & 15.48562 & 38.64613 & 15.15121 & 132.16136 & 0.51563 & 77 & $O_{3}$ \\
\hline
2021/7/14 & 监测点C & 6.89516 & 14.64654 & 32.54615 & 13.64636 & 128.16161 & 0.53466 & 74 & $O_{3}$ \\
\hline
2021/7/15 & 监测点C & 7.89252 & 21.61616 & 29.65141 & 20.51661 & 148.84662 & 0.48616 & 91 & $O_{3}$ \\
\hline
\end{tabular}
\end{table}

\section{七、问题四的分析与求解}

\subsection{7.1 问题分析}

相邻区域的污染物浓度往往具有一定的相关性,题目要求考虑区域相互之间的影响,并作出区域协同预报。因此首先对相邻区域(A、A1、A2、A3)的污染物进行相关性分析,本小题主要使用皮尔逊(Pearson)相关系数来衡量监测点 A 与监测点 A1、A2、A3 之间污染物浓度的相关性。

首先分析 A 点与 A1 点的浓度相关性,分别设相应变量 X 和 Y 的观测值 $x_i$ 和 $y_i$ ($i=1,2,3,\dots,n$),并且假设它们的样本均值分别为 $\bar{X}$ 和 $\bar{Y}$,那么我们可以得到 X 与 Y 的皮尔逊相关系数 $R_p$ 如下所示:

\begin{equation}
R_p = \frac{\sum_{i=1}^{n}(x_i - \bar{x})(y_i - \bar{y})}{\sqrt{\sum_{i=1}^{n}(x_i - \bar{x})^2} \sqrt{\sum_{i=1}^{n}(y_i - \bar{y})^2}}
\tag{7-1}
\end{equation}

相关系数 $R_p$ 是无量纲数,它的取值范围在 -1 与 1 的区间内。如果 $R_p$ 的值是小于 0 的数,那么变量 X 与 Y 之间呈现的是负相关关系;而如果 $R_p$ 的值是大于 0 的数,那么变量 X 与 Y 之间呈现的是正相关关系;如果 $R_p$ 的值等于 0,那么变量 X 与 Y 之间无相关性关系。对两个相关系数取绝对值,则绝对值越大,则两个变量之间的相关性越强,反之相关性越弱,其中皮尔逊相关系数 $R_p$ 表示的是两个变量的线性相关性。

\subsection{7.2 研究方法}

本章所用的研究方法与第六章大体相同,首先采用皮尔逊相关性分析方法分析 A、A1、A2、A3 的相关性,表 7-1 列出了监测点 A 与其相邻区域 A1、A2、A3 的污染物浓度的皮尔逊相关系数。

\begin{table}[h]
\centering
\caption{各区域污染物浓度的相关系数}
\begin{tabular}{|c|c|c|c|c|c|c|c|}
\hline
 & $SO_2$ & $NO_2$ & PM10 & PM2.5 & CO & $O_3$ & 平均相关系数 \\ \hline
A 与 A1 & 0.5175 & 0.9476 & 0.9756 & 0.9749 & 0.6534 & 0.9624 & 0.8385 \\ \hline
A 与 A2 & 0.7081 & 0.9494 & 0.9772 & 0.9753 & 0.7889 & 0.9746 & 0.8956 \\ \hline
A 与 A3 & 0.6096 & 0.8817 & 0.8825 & 0.7684 & 0.6688 & 0.7826 & 0.7655 \\ \hline
\end{tabular}
\end{table}

从表中数据可以看出,A 与 A1、A2、A3 的污染物浓度都具有较强的正相关,但不同的区域、不同的浓度之间相关性程度不太一样。

具体的预测流程与问题三大致相同,第一阶段以各自的气象实测与时间数据进行训练,得到各监测点 2021 年 7 月 13 日至 7 月 15 日的气象条件预测值;第二阶段预测时,首先合并监测点 A、A1、A2、A3 的气象条件逐小时实测数据和污染物浓度逐小时实测数据,构建各监测点共同的数据集,进行训练。后面的步骤按问题三中的流程进行,这里不再赘述。

\subsection{7.3 神经网络训练}

(1) 以温度实测为输出的神经网络训练结果

\begin{figure}[h]
    \centering
    \includegraphics[width=\textwidth]{image1.png}
    \caption{(a) 监测点 A1 温度随时间的训练情况}
\end{figure}

\begin{figure}[h]
    \centering
    \includegraphics[width=\textwidth]{image2.png}
    \caption{(b) 监测点 A2 温度随时间的训练情况}
\end{figure}

\begin{figure}[h]
    \centering
    \includegraphics[width=\textwidth]{image3.png}
    \caption{(c) 监测点 A3 温度随时间的训练情况}
\end{figure}

图 7-1 以温度为特征的神经网络训练情况

由上图可得,监测点 A1、A2、A3 的温度实测值基本呈周期性变化,结合上一章的分

\begin{figure}[h]
    \centering
    \includegraphics[width=\textwidth]{image1.png}
    \caption{(a) 监测点 A1 湿度随时间的训练情况}
\end{figure}

\begin{figure}[h]
    \centering
    \includegraphics[width=\textwidth]{image2.png}
    \caption{(b) 监测点 A2 湿度随时间的训练情况}
\end{figure}

\begin{figure}[h]
    \centering
    \includegraphics[width=\textwidth]{image3.png}
    \caption{(c) 监测点 A3 湿度随时间的训练情况}
\end{figure}

图 7-2 以湿度为特征的神经网络训练情况

通过分析以上三组图像可得,其拟合曲线的波动相对较大,分析以上实测数据的曲线

走势可得 A2 和 A3 的监测数据周期性变化较明显,而 A1 处的数据相对来说不如其他两组数据周期性明显,但考虑到原始数据的误差,可将其视为随时间周期性变化。

(3) 以气压实测为输出的神经网络训练结果

\begin{figure}[h]
    \centering
    \includegraphics[width=\textwidth]{image1.png}
    \caption{(a) 监测点 A2 气压随时间的训练情况}
\end{figure}

\begin{figure}[h]
    \centering
    \includegraphics[width=\textwidth]{image2.png}
    \caption{(b) 监测点 A3 气压随时间的训练情况}
\end{figure}

图7-3 以气压为特征的神经网络训练情况

由于缺乏监测点 A1 处的实测数据,因此只对 A2、和 A3 的数据进行分析,通过观察图像可以发现,二者曲线走势及测量值基本一致,均为随时间而周期性变化,变化周期为一年。

(4) 以风速实测为输出的神经网络训练结果

\begin{figure}[h]
    \centering
    \includegraphics[width=\textwidth]{image1.png}
    \caption{(a) 监测点 A1 风速随时间的训练情况}
\end{figure}

\begin{figure}[h]
    \centering
    \includegraphics[width=\textwidth]{image2.png}
    \caption{(b) 监测点 A2 风速随时间的训练情况}
\end{figure}

\begin{figure}[h]
    \centering
    \includegraphics[width=\textwidth]{image3.png}
    \caption{(c) 监测点 A3 风速随时间的训练情况}
    \caption{以风速为特征的神经网络训练情况}
\end{figure}

由以上三组数据可得,三个监测点的风速虽有一定的波动,但整体稳定在一定的范围内,其拟合曲线的趋势走向也较为平缓,因此可视其为随时间周期性变化。

(5) 以风向实测为输出的神经网络训练结果

\begin{figure}[h]
    \centering
    \includegraphics[width=\textwidth]{image1.png}
    \caption{(a) 监测点 A1 风向随时间的训练情况}
\end{figure}

\begin{figure}[h]
    \centering
    \includegraphics[width=\textwidth]{image2.png}
    \caption{(b) 监测点 A2 风向随时间的训练情况}
\end{figure}

\begin{figure}[h]
    \centering
    \includegraphics[width=\textwidth]{image3.png}
    \caption{(c) 监测点 A3 风向随时间的训练情况}
    \caption{以风向为特征的神经网络训练情况}
\end{figure}

此组的风向波动也较大,且其变化无明显规律,结合上一章的关于风向的实测数据可认为,其变化情况随时间无明显的关系。

监测点 A1、B1、C1 气象预测结果及对应的污染物浓度修正值预测见附录。

\section{7.4 预测结果}

根据第二阶段随机森林预测模型预测出的污染物浓度的逐小时浓度修正值与污染物浓度的一次预报值得到2021年7月13日至7月15日各监测点的污染物浓度二次预报值,代入到问题1中的程序得到监测点A、B、C的二次预报结果,见表7-2、7-3、7-4、7-5。

\section{表7-2 监测点A二次模型预测}

\begin{table}[h]
\centering
\begin{tabular}{|c|c|c|c|c|c|c|c|c|}
\hline
\multirow{2}{*}{预报日期} & \multirow{2}{*}{地点} & \multicolumn{7}{c|}{二次模型日值预测} \\
\cline{3-9}
 & & $SO_{2}$ & $NO_{2}$ & $PM_{10}$ & $PM_{2.5}$ & $O_{3}$最大八小时滑动平均 & $CO$ & 首要污染 \\
 & & ($\mu g/m^{3}$) & ($\mu g/m^{3}$) & ($\mu g/m^{3}$) & ($\mu g/m^{3}$) & ($\mu g/m^{3}$) & ($mg/m^{3}$) & 物 \\
\hline
2021/7/13 & 监测点A & 6.128147 & 12.82953 & 18.35984 & 6.540637 & 95.370625 & 0.334668 & 无 \\
\hline
2021/7/14 & 监测点A & 6.088543 & 14.135405 & 17.836145 & 6.871698 & 102.921235 & 0.357574 & $O_{3}$ \\
\hline
2021/7/15 & 监测点A & 5.240322 & 12.61813 & 19.424375 & 6.959164 & 101.590375 & 0.378004 & $O_{3}$ \\
\hline
\end{tabular}
\end{table}

\section{表7-3 监测点A1二次模型预测}

\begin{table}[h]
\centering
\begin{tabular}{|c|c|c|c|c|c|c|c|c|}
\hline
\multirow{2}{*}{预报日期} & \multirow{2}{*}{地点} & \multicolumn{7}{c|}{二次模型日值预测} \\
\cline{3-9}
 & & $SO_{2}$ & $NO_{2}$ & $PM_{10}$ & $PM_{2.5}$ & $O_{3}$最大八小时滑动平均 & $CO$ & 首要污染 \\
 & & ($\mu g/m^{3}$) & ($\mu g/m^{3}$) & ($\mu g/m^{3}$) & ($\mu g/m^{3}$) & ($\mu g/m^{3}$) & ($mg/m^{3}$) & 物 \\
\hline
2021/7/13 & 监测点A1 & 7.51528 & 13.2531 & 27.1342 & 9.33395 & 106.775 & 0.31328 & $O_{3}$ \\
\hline
2021/7/14 & 监测点A1 & 8.77678 & 14.5463 & 27.1425 & 9.0357 & 115.631 & 0.30505 & $O_{3}$ \\
\hline
2021/7/15 & 监测点A1 & 7.96991 & 13.2546 & 26.347 & 9.37172 & 110.119 & 0.36761 & $O_{3}$ \\
\hline
\end{tabular}
\end{table}

\section{表7-4 监测点A2二次模型预测}

\begin{table}[h]
\centering
\begin{tabular}{|c|c|c|c|c|c|c|c|c|}
\hline
\multirow{2}{*}{预报日期} & \multirow{2}{*}{地点} & \multicolumn{7}{c|}{二次模型日值预测} \\
\cline{3-9}
 & & $SO_{2}$ & $NO_{2}$ & $PM_{10}$ & $PM_{2.5}$ & $O_{3}$最大八小时滑动平均 & $CO$ & 首要污染 \\
 & & ($\mu g/m^{3}$) & ($\mu g/m^{3}$) & ($\mu g/m^{3}$) & ($\mu g/m^{3}$) & ($\mu g/m^{3}$) & ($mg/m^{3}$) & 物 \\
\hline
2021/7/13 & 监测点A2 & 6.42607 & 16.3254 & 22.8535 & 6.04529 & 96.0156 & 0.42794 & 无 \\
\hline
2021/7/14 & 监测点A2 & 6.86979 & 17.5876 & 22.677 & 5.89966 & 103.371 & 0.43648 & $O_{3}$ \\
\hline
2021/7/15 & 监测点A2 & 4.2098 & 15.3542 & 23.6447 & 7.1165 & 96.3531 & 0.48449 & 无 \\
\hline
\end{tabular}
\end{table}

\section{表7-5 监测点A3二次模型预测}

\begin{table}[h]
\centering
\begin{tabular}{|c|c|c|c|c|c|c|c|c|}
\hline
\multirow{2}{*}{预报日期} & \multirow{2}{*}{地点} & \multicolumn{7}{c|}{二次模型日值预测} \\
\cline{3-9}
 & & $SO_{2}$ & $NO_{2}$ & $PM_{10}$ & $PM_{2.5}$ & $O_{3}$最大八小时滑动平均 & $CO$ & 首要污染 \\
 & & ($\mu g/m^{3}$) & ($\mu g/m^{3}$) & ($\mu g/m^{3}$) & ($\mu g/m^{3}$) & ($\mu g/m^{3}$) & ($mg/m^{3}$) & 物 \\
\hline
2021/7/13 & 监测点A3 & 5.39454 & 10.56258 & 12.15390 & 5.72052 & 90.42189 & 0.28726 & 无 \\
\hline
2021/7/14 & 监测点A3 & 4.54450 & 11.89973 & 11.20909 & 6.58932 & 97.56747 & 0.33124 & 无 \\
\hline
2021/7/15 & 监测点A3 & 4.76680 & 10.72190 & 14.12313 & 5.89974 & 101.32129 & 0.31827 & $O_{3}$ \\
\hline
\end{tabular}
\end{table}

\section{7.5 讨论分析}

计算 A 到 A1、A2、A3 的直线距离分别是 14.62km、10.11km、6.03km,分析 A 与 A1、A2、A3 的污染物浓度皮尔逊平均相关系数为 0.8385、0.8956、0.7655,发现 A 与 A3 虽然距离很近但相关性却相对较低,这说明除距离以外,可能还存在其他的特征能够说明 A、A1、A2、A3 四个区域污染物浓度的相关性。例如,四个区域各自的地形可能会影响各自的污染物浓度预测,海拔越高,温度与气压会发生相应的变化;若某预测点位于山下,风速、风向会发生相应变化。此时,如果建立协同预报模型非但不能提升二次预报准确度,反而会引入不合理的数据,增大二次预报的误差。因此,我们最后得出结论:问题 4 的协同预报模型不能提升监测点 A 的污染物浓度预报准确度。

\section{八、参考文献}

[1] 张楠. 基于优化的 BP 神经网络与 SVR 算法对 AQI 的预测研究[D]. 中北大学, 2019.

[2] 周凯, 刘萍. 基于数据挖掘的空气质量预测模型研究 [J]. 计算机与数字工程, 2021, 49(08): 1631-1636.

[3] 卢亚灵, 李勃, 范朝阳, 王建童, 张鸿宇, 蒋洪强. 空气质量预测模拟技术演变与发展研究 [J]. 中国环境管理, 2021, 13(04): 84-92.

[4] 远佳琼. 面向密集站点场景的空气质量预测研究 [D]. 西北师范大学, 2021.

[5] 伍小丽. 基于新灰色非线性模型的长三角地区空气污染物预测及其缓解策略 [D]. 常州大学, 2021.

[6] 周凯, 刘萍. 基于数据挖掘的空气质量预测模型研究 [J]. 计算机与数字工程, 2021, 49(08): 1631-1636.

[7] 方伟, 朱润苏. 基于时空相似 LSTM 的空气质量预测模型 [J]. 计算机应用研究, 2021, 38(09): 2640-2645.

[8] 金仁浩, 曾国静, 王莎. 基于神经网络模型的空气质量预测研究 [J]. 黑龙江科学, 2021, 12(12): 15-19.

[9] 范彩云, 孙汝意, 童君逸. 基于空间因素 LSTM 神经网络的空气质量指数预测 [J]. 数学的实践与认识, 2021, 51(15): 194-202.

[10] 谭若洋. 基于改进的 PSO 的 BP 神经网络模型的建立及在空气质量预测中的应用 [D]. 重庆工商大学, 2021.

\end{document}