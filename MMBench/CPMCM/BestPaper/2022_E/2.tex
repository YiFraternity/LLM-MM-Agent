\begin{center}
\includegraphics[width=0.2\textwidth]{image1.png} \quad
\includegraphics[width=0.2\textwidth]{image2.png} \quad
\includegraphics[width=0.2\textwidth]{image3.png} \quad
\includegraphics[width=0.2\textwidth]{image4.png}
\end{center}

\begin{center}
\textbf{中国研究生创新实践系列大赛} \\
\textbf{中国光谷·“华为杯”第十九届中国研究生} \\
\textbf{数学建模竞赛}
\end{center}

\begin{table}[h]
\centering
\begin{tabular}{l l}
学 校 & 南通大学 \\
\hline
参赛队号 & 22103040108 \\
\hline
队员姓名 & 1. 沈陈强 \\
 & 2. 张瑞年 \\
 & 3. 袁梦 \\
\end{tabular}
\end{table}

\begin{center}
\textbf{中国研究生创新实践系列大赛}\\
\textbf{中国光谷·“华为杯”第十九届中国研究生}\\
\textbf{数学建模竞赛}
\end{center}

\begin{center}
\textbf{题目}\\
\textbf{草原放牧策略研究}
\end{center}

\begin{center}
\textbf{摘 要:}
\end{center}

草原放牧策略的研究对于我国生态环境具有重要影响,不仅响应我国绿色可持续发展的号召,也是带动区域经济与保障民生的关键方法。以内蒙锡林郭勒草原为例,从放牧方式与放牧强度的角度入手,结合植物生长、土壤沙漠化、板结化等影响因素,优化放牧策略问题,具有较好的研究意义。

针对问题一,首先分析模型所需附件数据,从土壤物理性质与植被生长机理角度入手,结合扩展阅读模型,初步筛选附件 3~15 作为基础数据集;其次,根据数据分布特性,进行 Spearman 相关性分析,得出同土壤物理性质和植被生长机理相关数据,确定 3、4、5、6、8、15 数据集,并通过 nan 值、异常值、补充修正等对数据进行预处理;然后,依据植被机理性质,明确放牧方式与强度关系,结合放牧-植物生长关系与附件 15 轮牧情况的时序性特性,确定以差分自回归移动平均模型为主数学模型,融入各类植被生物影响交互因子,构建放牧-植被影响模型;最后,依据土壤物理性质,考虑题干湿度因素,结合附件所给土壤物理因素,以物理性的平衡方程作为主数学模型,构建放牧-土壤物理性质影响模型。

针对问题二,根据题干要求附件 3、4、8 等数据以“年月信息”整合与挖掘分析,建立不同深度土壤湿度预测模型。首先,所选附件数据规模存在部分差异,在合并处理前进行二次清洗与处理;其次,土壤湿度会受到蒸发量、降水量、气温等因素影响,进行信息增益降维;然后,根据附件数据的时序特性,构建 ARI 主模型,并融合各类影响因子,构建多维影响-时序性 ARI 预测模型;最后,为了验证多维影响-时序性 ARI 预测模型的准确性,选择具有较高时序特性的 LSTM 模型进行初步验证分析,为确保模型能够捕获全局特性,融入 Self-attention 机制,构建 LSTM-SA 模型进行二次验证。

针对问题三,为从机理分析角度建立放牧策略对土壤化学性质影响模型,结合附件 14 对有机碳、无机碳等数值进行预测分析。首先,对附件 14 数据分别进行斯皮尔曼相关性分析和化学性质机理性分析,得出土壤全碳(STC)等于土壤有机碳(SOC)与无机碳(SIC)之和,土壤碳氮比为土壤全碳(STC)与全氮(AN)的比值以及有机碳逐步转化为无机碳的化学机理性,故主要考虑有机碳、无机碳、全氮三个影响因素;然后,根据数据的时序特性,得出不同放牧强度对 SOC、SIC、AN 的影响;最后,建立放牧与 SOC、SIC、AN 的预测模型,将预测结果填充至问题三表单,并对预测结果可视化。

针对问题四,首先根据问题所给沙漠化指数预测模型,结合《栅格化预警模式研究》,对涉及到 9 个因子进行数据关联与收集,同时进行清洗与处理;其次,根据附件 14、15 中牧区与示范牧户进行监测点分类并计算沙漠化程度指数值;然后,根据本文所给板结化模型描述与指标关系,利用问一、问三模型,给出土壤板结化定量定义;最后,根据所明确

的板结化与沙漠指数化定义,在不同放牧强度约束的条件下,建立多目标最小优化模型。

针对问题五,为满足可持续发展要求,给定降水情形,分析放牧最大阈值。首先,根据题干要求附件 14、15 并不包含降水信息,即纳入附件 8 降水量信息,进行表单合并与二次处理;其次,为确定羊群的最优阈值,构建降水量-羊群阈值模型,同时为满足可持续发展要求,结合问四的土壤沙漠化指数进行共同量化;然后,使用粒子群优化算法,通过不断迭代的方法,从全局角度,获取最优解;最后,根据要求填充电量降水情况下,最大羊群阈值表。

针对问题六,在两种不同情况(附件 13 和问题 4 中得到的放牧方案)下,分别预测四个示范区的土地状态。首先,使用附件 13 的放牧压力推测出四个示范区的放牧强度,结合附件 14 分析出示范区所属放牧小区;其次,从物理、化学角度对土地状态进行机理性分析,确定土壤肥力主要影响因素为全氮和温度,土壤湿度与植被覆盖数据集沿用上文;最后,利用问题一、二、三的模型分别对四个示范区的土壤肥力(全氮和温度)、土壤湿度、植被覆盖进行预测,并用图示呈现预测结果。

关键词:放牧策略;交互性;时间序列平稳性分析;ARI 预测模型;差分自回归移动平均

\section*{目录}

\begin{itemize}
    \item[1] 问题重述 \dotfill 6
    \begin{itemize}
        \item[1.1] 问题背景 \dotfill 6
        \item[1.2] 数据介绍与建模目标 \dotfill 6
        \item[1.3] 问题重述 \dotfill 7
    \end{itemize}
    \item[2] 模型假设 \dotfill 8
    \item[3] 符号说明 \dotfill 8
    \item[4] 问题一:建立放牧-机理影响模型 \dotfill 9
    \begin{itemize}
        \item[4.1] 问题一分析 \dotfill 9
        \item[4.2] 数据预处理 \dotfill 9
        \begin{itemize}
            \item[4.2.1] 数据集的确定 \dotfill 9
            \item[4.2.2] nan值的处理 \dotfill 11
            \item[4.2.3] 异常值处理 \dotfill 12
        \end{itemize}
        \item[4.3] 建立放牧-植被影响模型 \dotfill 13
        \begin{itemize}
            \item[4.3.1] 影响因素的确定 \dotfill 13
            \item[4.3.2] 植被生长机理性研究 \dotfill 13
            \item[4.3.3] 建立放牧-植被影响模型 \dotfill 14
        \end{itemize}
        \item[4.4] 建立放牧-土壤物理性质影响模型 \dotfill 16
        \begin{itemize}
            \item[4.4.1] 影响物理性质因素的选取 \dotfill 16
            \item[4.4.2] 模型建立 \dotfill 17
        \end{itemize}
    \end{itemize}
    \item[5] 问题二:多维影响-时序性的ARI预测模型 \dotfill 19
    \begin{itemize}
        \item[5.1] 问题二分析 \dotfill 19
        \item[5.2] 建立多维影响-时序性的ARI预测模型 \dotfill 19
        \begin{itemize}
            \item[5.2.1] 数据合并与处理 \dotfill 19
            \item[5.2.2] 数据降维与筛选 \dotfill 20
            \item[5.2.3] 建立多维影响-时序性的ARI预测模型 \dotfill 20
        \end{itemize}
        \item[5.3] 验证模型-基于时序性的LSTM-SA模型 \dotfill 23
        \begin{itemize}
            \item[5.3.1] 验证模型的建立 \dotfill 23
            \item[5.3.2] 验证数据对比 \dotfill 25
        \end{itemize}
        \item[5.4] 问二表单填充 \dotfill 26
    \end{itemize}
    \item[6] 问题三:建立土壤-放牧化学模型 \dotfill 27
    \begin{itemize}
        \item[6.1] 问题三分析 \dotfill 27
        \item[6.2] 数据集的确定 \dotfill 27
        \item[6.3] 建立放牧-土壤化学性质影响预测模型 \dotfill 28
        \begin{itemize}
            \item[6.3.1] 影响因素的确定 \dotfill 28
            \item[6.3.2] 土壤化学性质机理性研究 \dotfill 28
            \item[6.3.3] 建立放牧-土壤化学性质影响预测模型 \dotfill 29
            \item[6.3.4] 放牧-化学性质影响预测模型求解 \dotfill 31
        \end{itemize}
        \item[6.4] 问三表单填充 \dotfill 31
    \end{itemize}
    \item[7] 问题四:建立土壤-放牧化学模型 \dotfill 33
    \begin{itemize}
        \item[7.1] 问题四分析 \dotfill 33
        \item[7.2] 附件数据处理 \dotfill 33
        \begin{itemize}
            \item[7.2.1] 数据收集 \dotfill 33
        \end{itemize}
    \end{itemize}
\end{itemize}

\begin{itemize}
    \item[7.2.2] 数据预处理 \dotfill 34
    \item[7.3] 确定监测点沙漠化指数 \dotfill 35
    \begin{itemize}
        \item[7.3.1] 确定沙漠化指数预测模型 \dotfill 35
        \item[7.3.2] 分析因子与放牧强度间相关性 \dotfill 35
        \item[7.3.3] 建立放牧-沙漠化指数模型 \dotfill 36
        \item[7.3.4] 指数确定 \dotfill 36
    \end{itemize}
    \item[7.4] 定量土壤板结化定义 \dotfill 37
    \item[7.5] 沙漠化程度与板结化程度最小优化问题 \dotfill 38
    \begin{itemize}
        \item[7.5.1] 建立多目标最优模型 \dotfill 38
        \item[7.5.2] 模型求解 \dotfill 38
    \end{itemize}
    \item[8] 问题五:最优阈值确定 \dotfill 39
    \begin{itemize}
        \item[8.1] 问题五分析 \dotfill 39
        \item[8.2] 附件数据处理 \dotfill 39
        \item[8.3] 确定最优阈值 \dotfill 40
        \begin{itemize}
            \item[8.3.1] 优化目标 \dotfill 40
            \item[8.3.2] 目标函数 \dotfill 40
            \item[8.3.3] 决策变量 \dotfill 40
            \item[8.3.4] 约束条件 \dotfill 40
            \item[8.3.5] 建立降水量-羊群阈值模型 \dotfill 41
            \item[8.3.5] 降水量-羊群阈值模型求解 \dotfill 42
            \item[8.3.6] 最优阈值确定 \dotfill 42
        \end{itemize}
    \end{itemize}
    \item[9] 问题六:图示演化四个示范区土地状态预测结果 \dotfill 43
    \begin{itemize}
        \item[9.1] 问题六分析 \dotfill 43
        \item[9.2] 数据分析与机理分析 \dotfill 43
        \item[9.3] 示范区土地状态预测结果 \dotfill 44
    \end{itemize}
    \item[10] 模型的评价和推广 \dotfill 46
    \begin{itemize}
        \item[10.1] 模型的评价 \dotfill 46
        \begin{itemize}
            \item[10.1.1] 模型的优点 \dotfill 46
            \item[10.1.2] 模型的缺点 \dotfill 46
        \end{itemize}
        \item[10.2] 模型的推广 \dotfill 46
        \begin{itemize}
            \item[10.2.1] 适用性 \dotfill 46
            \item[10.2.2] 鲁棒性 \dotfill 46
        \end{itemize}
    \end{itemize}
    \item[11] 参考文献 \dotfill 47
    \item 附录 \dotfill 48
\end{itemize}

\section{问题重述}

\subsection{问题背景}

草原作为世界分布最广、占地面积最大的陆地植被类型之一,对于生态环境具有重要作用。我国为维持和改善草原生态环境,颁布“退牧还草”政策,针对多草原实施五种不同的放牧方式,以合理性的放牧政策推动绿色可持续发展。

我国草原类型多样,以内蒙古锡林郭勒草原作为温带草原代表与典型示例,其地理坐标介于东经$[110^{\circ}50', 119^{\circ}58']$,北纬$[41^{\circ}30', 46^{\circ}45']$间,且全年降雨量均处于$340 \mathrm{~mm}$,具有良好的畜牧环境,也发挥着天然生态屏障作用。

考虑到放牧政策对于草原生态发展趋势的影响,主要从两个方面进行考虑,放牧方式与放牧强度,根据有关文献进行量化划分,同时也对畜牧种类进行统一的标准化,以羊作为主单位,具体内容见表1.1所示。

\begin{table}[h]
\centering
\caption{放牧方式与放牧强度量化划分表}
\begin{tabular}{l l}
\hline
名称 & 内容 \\
\hline
放牧方式 & 全年连续放牧、禁牧、选择划区轮牧、轻度放牧、生长季休牧共计5种 \\
放牧强度 & 对照、轻度放牧强度、中度放牧强度、重度放牧强度共计4种 \\
理想轮牧的 & 对照(NG: 0)、轻度放牧强(LGI: 2)、中度放牧强度(MGI: 4)和重度放 \\
放牧强度量化指标 & 牧强度(HGI: 8) (单位: 羊/天/公顷) \\
实际轮牧的 & 对照(NG: 0)、轻度放牧强(LGI: 1-2)、中度放牧强度(MGI: 3-4)和重 \\
放牧强度量化指标 & 度放牧强度(HGI: 5-8) (单位: 羊/天/公顷) \\
牲畜折算系数 & 标准单位: 羊; 大牲畜折算系数6.0 (牛、马、骆驼),大牲畜幼崽折算系数3.0 (包括羊羔) \\
\hline
\end{tabular}
\end{table}

同时,草原植物的生长也受到环境与放牧的影响,例如降水、土壤物化性质、畜牧吞食等,直接关乎草原整体的发展趋势。

鉴于草原放牧方式与放牧强度的两种因素,以内蒙古锡林郭勒草原为主要示例,采用可持续发展原则为基准,研究最佳草原放牧策略,寻求生态环境良性发展中经济与利益的最大化。

\subsection{数据介绍与建模目标}

本文主要包含十五个附件数据,分为两个大类:基本数据与监测点数据,其具体内容见表1.2所示。

\begin{table}[h]
\centering
\caption{数据文件介绍表}
\begin{tabular}{l l l}
\hline
序号 & 文件名 & 内容 \\
\hline
附件1 & 内蒙古锡林郭勒草原概况 & 草原介绍、地理位置、占地面积、气候类型、降水量 \\
 & & 变化、土壤性质、植被覆盖类型 \\
附件2 & 锡林郭勒统计年鉴(2016-2021) & 锡林郭勒盟统计局对该市的整体数据进行调查统计 \\
 & & 分析(具体针对第一产业、自然资源、公报等信息分析) \\
附件3 & 土壤湿度(2012-2022) & 经纬度、10cm湿度、40cm湿度、100cm湿度、200cm \\
 & & 湿度 \\
 & & 单位$(\mathrm{kg} / \mathrm{m}^{2})$ \\
附件4 & 土壤蒸发量(2012-2022) & 经纬度、土壤蒸发量$(\mathrm{W} / \mathrm{m}^{2})$、土壤蒸发量$(\mathrm{mm})$ \\
\hline
\end{tabular}
\end{table}

\begin{tabular}{l l l}
附件5 & 绿植覆盖率(2020-2022) & 经纬度、绿植覆盖率 \\
附件6 & 植被指数-NDVI(2012-2022) & 经纬度、植被指数(NDVI) ---- NDVI 含义: $[-1,1]$ \\
附件7 & 锡林郭勒土壤基本数据 & 土壤化学成分(浅层、深层含量) \\
附件8 & 锡林郭勒盟气候 (2012-2022) & 经纬度、气温、时间、降水天数、降雨量、积雪深度、能见度、海平面、风速 \\
附件9 & 径流量(2012-2022) & 经纬度、径流量($\mathrm{m}^3/\mathrm{s}$)、径流量($\mathrm{m}^3$) \\
附件10 & 叶面积指数(LAI) (2012-2022) & 经纬度、高层植被 (LAIH, $\mathrm{m}^2/\mathrm{m}^2$)、低层植被 (LAIH, $\mathrm{m}^2/\mathrm{m}^2$) \\
附件11 & 一些历史数据 & 观测场土壤养分 \\
附件12 & 不同畜牧业群落样方 & 生物量种类划分且干重占比 \\
附件13 & 不同示范牧户畜牲数量 & 人口、经济数据、牲畜数据、牲畜重量 \\
附件14 & 不同放牧强度土壤化学性质影响 & 放牧小区、放牧强度、SOC 土壤有机碳、SIC 土壤无机碳、STC 土壤全碳、全氮 N、土壤 C/N 比 \\
附件15 & 轮牧放牧样地群落结构监测数据 & 轮次、处理、植物种类、种群功能、放牧校区、重复率营养苗、生殖苗、株/丛数、从幅 12、鲜重、干重 \\
\end{tabular}

本文建模目标需根据所给出的十五个附件选取所需的数据集进行考虑,注意放牧的方式、土壤成分、植被生物量的生长等问题,同时注重区域性的因素,获取最佳草原放牧策略。

\subsection*{1.3 问题重述}

基于上述附件数据与草原知识背景,本文需要解决下列六个问题:

\textbf{问题一:建立放牧-机理影响模型}

从植被生长机理与土壤物理性质角度入手,通过所给附件数据,分析土壤物理性质与植被生物量受放牧策略的影响性,分别建立放牧-机理影响的数学模型。

\textbf{问题二:多维影响-时序性的 AR 预测模型}

结合附件 3 土壤湿度数据、附件 4 土壤蒸发数据和附件 8 降水量等数据分析关联性,在现行的放牧策略下,建立对 2022、2023 年不同深度土壤湿度的预测模型,且填充所给表格。

\textbf{问题三:建立放牧-土壤化学模型}

结合土壤化学性质的机理公式,充分考虑附件 14 与附件 7 的土壤数据,深入分析土壤浅层与深层的化学物质,同时也结合土壤本身含有的有机碳、无机碳等化学值,建立放牧-土壤化学模型,并填充预测锡林郭勒草原监测样地中不同放牧状态下的土壤化学值。

\textbf{问题四:定量板结化,给定放牧法}

根据所定义的沙漠化程度指数预测方式,且结合补充数据信息,以此明确放牧强度与沙漠化指数的关联性,定量土壤板结化。另外,在问题三的基础上,建立全新的放牧方法,将沙漠化程度指数与板结化程度最小。

\textbf{问题五:最佳阈值确定}

结合附件 8 锡林郭勒盟气候 2012-2022 年的变化,在给定四种降水量的情况下,且满足我国推行的放牧可持续化发展政策,对于实验草场内(附件 14、15)放牧羊的数量进行求解,以确定最佳的阈值。

\textbf{问题六:图示演化四个示范区土地状态预测结果}

基于附件 13 示范放牧策略不变的情况和问题 4 两种放牧方案下,建立示范区土地状态预测模型,同时以图示的方式演示 2023 年 9 月土地状态。

\section*{3 符号说明}

\begin{tabular}{c c l}
\hline
序号 & 符号 & 含义 \\
\hline
1 & $IC_{store}$ & 植被截流量 \\
2 & $c_{p}$ & 植被覆盖率 \\
3 & $k$ & 植被密度校正因子 \\
4 & $R_{cum}$ & 累积降雨量 \\
5 & $LAI$ & 时变参数 \\
6 & $\varphi_{i}$ & 干重值 \\
7 & $Me$ & 放牧强度 \\
8 & $\Delta V_{1}$ & 植物生长率 \\
9 & $S$ & 单位面积载畜率 \\
10 & $P$ & 降水率 \\
11 & $We_{i}$ & $i$cm土壤湿度指标 \\
12 & $\Delta V_{2}$ & 土壤物理性质变化速率 \\
13 & $\delta$ & 误差值 \\
14 & $Gg$ & 土壤物理性质 \\
15 & $\beta_{0}$ & 影响因子 \\
16 & $\rho$ & 比例系数 \\
17 & $SM$ & 沙漠化指数 \\
18 & $B$ & 土壤板结化 \\
19 & $C$ & 土壤容重 \\
20 & $O$ & 土壤有机物 \\
21 & $Gc$ & 植物群落功能 \\
22 & $Pl$ & 放牧政策变量 \\
\hline
\end{tabular}

\section{问题一:建立放牧-机理影响模型}

\subsection{问题一分析}

根据问题一的要求与实际采集数据的情况,所给的附件会存有异常情况,其异常情况可能包含人工采集的误差、天气环境影响和设备故障等因素,需要对所给附件数据进行预处理操作。

为明确问题一所需要用到的附件数据,从土壤物理性质与植被生物量机理角度入手,需要将关乎土壤物理性质与植被生物量的所有附件进行相关性分析,结合扩展阅读所给的部分模型,涉及土壤含水量-降水量-地标蒸发模型、放牧与植物生长关系模型与土壤-植被-大气系统的水平衡方程,暂定附件 3、4、5、6、7、8、10、15 作为问题一的主要数据,其具体的数据处理方法如下:

(1) 鉴于部分数据均是以时间序列为基线进行的采集,若部分节点存在着较多的 nan 值,难以补充,即剔除。

(2) 鉴于时间序列基线采集的数据,对于部分的空白点位,结合锡林郭勒草原地域气候的特性,部分土壤湿度、降雨量以及绿植覆盖率的变化趋势,对于 nan 的节点进行拉格朗日插值法处理。

(3) 鉴于采集数据必定包含异常的情况,数据偏离的问题不可避免,根据时间序列的特性进行趋势变化图分析,其波动若在限值范围内,即默认为合理情况,同时也通过拉依达准则对异常数据剔除与修正。

对于附件数据进行预处理后,结合相关性分析的结果,确定最终所需的数据。根据题目机理性要求,首先量化与明确放牧方式与强度之间的关系,然后,从植被生物量角度入手,结合放牧与植物生长关系模型,主以附件 15 的轮牧情况为例分析,以差分自回归移动平均模型作为主数学模型,融入各类植被生物量影响因素的交互因子,建立放牧-植被影响模型;最后,从土壤物理性质角度入手,主以土壤湿度作为关键影响因素,结合附件所给土壤物理因素,以物理性的平衡方程作为主数学模型,建立放牧-土壤物理性质影响模型,具体的解题思路见图 4.1 所示。

\begin{figure}[h]
\centering
\includegraphics[width=\textwidth]{image.png}
\caption{问题一思路流程图}
\label{fig:4.1}
\end{figure}

\subsection{数据预处理}

\subsubsection{数据集的确定}

基于本问所给的已知条件,首先对土壤物理性质进行分析,根据土壤湿度为主要因素的提示,选用附件3作为主要数据;其次,根据名词解释内容中土壤含水量的定义,选用附件8中的降水量数据;然后,根据扩展知识内容中的第二与第四模型,将附件4与7纳入考虑;最后,鉴于各类因素指标之间的关联性,其数据量不满足正态分布的特性,因此它们之间具有非线性相关,决定选用斯皮尔曼相关性分析,以此确定最终选用的数据集,具体分析见图4.2所示。

\begin{figure}[h]
\centering
\includegraphics[width=\textwidth]{heatmap_soil_properties.png}
\caption{土壤物理性质各指标相关性分析}
\label{fig:soil_properties}
\end{figure}

根据图4.2所做的相关性分析,结合土壤含水量名词解释内容,查询相关文献,如公式4.1所示。

\begin{equation}
\text{土壤湿度} = \frac{\text{含水量}}{\text{烘干土重}} \times 100\%
\tag{4.1}
\end{equation}

不难看出,将降水量等同于土壤含水量 $10\text{cm}$ 土壤湿度与土壤蒸发量存在强相关,同时与降水量存在弱相关。原因:表层土壤受到风速、温度等气象条件的影响,蒸发较快,故土壤湿度较低。其他土层的土壤湿度与降水量相关性较弱主要因为降雨只有达到一定量之后才会向深层次地表渗透。结合实际地理因素,内蒙古位于我国中西部,距离海洋较远,属温带季风气候,降雨量较少,深层土壤含量湿度几乎不受到降雨量与蒸发的影响,决定选用附件3的 $10\text{cm}$ 土壤湿度数据、附件4土壤蒸发量数据、附件8的降雨量。

基于本文所给的已知条件,首先对植被生物量进行分析,从植被生长机理角度,由Woodward\cite{woodward1987}建立的放牧与植被生长关系式,确定以附件15作为主数据;其次,根据名词解释内容中叶面积指数,将附件10纳入考虑;然后,根据“植被”自身特性而言,直接选用附件5、6,同时,根据植被生长机理,必定收到光照、雨水、海拔等影响,而附件8中包含这类数据,也选用这部分的数据;最后,鉴于各类因素指标之间的关联性,其数据量分布的问题,具有非线性相关的特性,决定选用斯皮尔曼相关性分析,确定最终选取的数据集,具体分析见图4.3所示。

\begin{figure}[h]
    \centering
    \includegraphics[width=\textwidth]{heatmap_image.png}
    \caption{植被生物量各指标相关性分析}
    \label{fig:heatmap}
\end{figure}

依据本问中拓展知识:绿植覆盖率与植被指数相关。由图\ref{fig:heatmap}可以得出,绿植覆盖率与植被指数、底层植被存在强相关关系,绿植覆盖率与不同深度的土壤湿度、平均气温、降水量均呈现强相关,主要因为温度、土壤湿度、降水均是植物生长所必须的物理条件。一般情况下,温度越高,植物生长越快;降水量越大,土壤湿度越大,植物生长越快。

根据上述的相关性分析,确定最终的问题一数据集,具体见如表\ref{tab:dataset}所示。

\begin{table}[h]
    \centering
    \caption{数据集的确定}
    \label{tab:dataset}
    \begin{tabular}{l l}
        \hline
        分类 & 数据集选用 \\
        \hline
        土壤物理性质 & 附件3:10cm土壤湿度数据、附件4:土壤蒸发量数据、 \\
        & 附件8:降水量 \\
        植被生物量 & 附件5:绿植覆盖率、附件6:植被指数、附件8:降水量 \\
        & /气温、附件10:叶面积指数、附件15:放牧处理/干重 \\
        \hline
    \end{tabular}
\end{table}

\subsection{nan值的处理}

遍历最终所确定的数据集,其中空白值的表现形式均是以nan的形式呈现,共23条nan的数据,将2020-2022年绿植覆盖率生成对应的折线图,具体见图\ref{fig:line_plot}所示,不难看出,其数据整体的量较少,且前后的绿植覆盖率变化较大,对于这部分的nan值直接删除处理。

\begin{figure}[h]
    \centering
    \includegraphics[width=\textwidth]{line_plot_image.png}
    \caption{2020年绿植覆盖率折线图}
    \label{fig:line_plot}
\end{figure}

\begin{figure}[h]
    \centering
    \includegraphics[width=\textwidth]{image1.png}
    \caption{2021年植被覆盖率变化趋势}
    \label{fig:2021_vegetation}
\end{figure}

\begin{figure}[h]
    \centering
    \includegraphics[width=\textwidth]{image2.png}
    \caption{2022年植被覆盖率变化趋势}
    \label{fig:2022_vegetation}
\end{figure}

\subsection{异常值处理}

鉴于数据时间序列的特性而言,分别对附件 3 的土壤湿度、附件 4 的土壤蒸发量、附件 8 的降雨量、附件 15 的干重进行异常值排查。决定使用拉以达准则对异常值进行分析,假设对选取的 4 部分数据含有随机误差,其次对于原始数据进行标准差处理,然后按照概率进行区间的确定,默认超过该区间的数值,满足 $|v_a| = |x_a - x| > 3\sigma$,即隶属于异常值,具体的处理方式为:通过检验后,以时间序列为准则,若超过 20 条,即进行全部剔除处理,少于即进行前后平均加权法处理。

以土壤湿度的趋势分析为例,具体见图 \ref{fig:soil_moisture} 所示,不难看出其 10cm、40cm、100cm 和 200cm 的深度变化趋势波动性不明显,无明显异常值,即不处理。

\begin{figure}[h]
    \centering
    \includegraphics[width=\textwidth]{image3.png}
    \caption{土壤湿度趋势分析}
    \label{fig:soil_moisture}
\end{figure}

\begin{figure}[h]
    \centering
    \includegraphics[width=\textwidth]{soil_moisture_trend.png}
    \caption{土壤湿度变化趋势图}
    \label{fig:soil_moisture_trend}
\end{figure}

重复上述步骤,对所选的数据均进行趋势分析,且使用拉以达准则进行异常处理,其具体的异常数据处理见表 \ref{tab:abnormal_data} 所示。

\begin{table}[h]
    \centering
    \caption{异常值处理表}
    \label{tab:abnormal_data}
    \begin{tabular}{l l l}
        \hline
        序号 & 表名 & 数据处理说明 \\
        \hline
        附件3 & 土壤湿度 & 数据正常,不做处理 \\
        附件4 & 土壤蒸发量 & 蒸发量趋势相同,数据正常,不做处理 \\
        附件8 & 锡林郭勒盟气候 & 异常值数据共1条,2018年12月份数据修正为17.6182 \\
        附件15 & 轮牧放牧样地群落结构监测数据 & 异常值数据共6条,采用前后加权平均法处理 \\
        \hline
    \end{tabular}
\end{table}

\subsection{建立放牧-植被影响模型}

\subsubsection{影响因素的确定}

鉴于 4.2 的相关性分析,其涉及的指标均为强相关,需要全部纳入考虑,以下是对各类附件指标的处理:

- 对于附件5绿植覆盖率对于植被的生长有较强的相关性,但因分析相关性前并未对数据进行处理,预处理操作结束后,附件5的绿植覆盖率数据规模并不能够满足后续的植被生长机理分析,即对该指标进行剔除;
- 对于附件6的植被指数而言,结合本问所给的植被指数 NDVI 含义,已做归一化处理,可直接将其量化成对应的权重指标;
- 对于附件8的降水量、附件10的叶面积指数而言,对于植被生长机理的影响十分重要,根据扩展知识第四模型中的植被截流量公式,将二者进行结合量化处理;
- 对于附件15的干重而言,依据所给附件的名词解释,干重即为植物地上生物量,即轮牧方式与干重必然存在着某种关系,以二者关系作为主数学模型进行展开融合。

\subsubsection{植被生长机理性研究}

根据机理性分析的要求,植被生长满足自身的生长规律,同时受到降水、温度、土壤湿度、土壤 PH、营养等均会影响到植被的生长情况与生长环境[2-3]。结合 4.2 的相关性分析,融合土壤-植被-大气系统的水平衡基本方程中的 $IC$ 值,且满足草地的植被生长与放牧强度成正相关行,即降水量、植被覆盖率、叶面积指数均满足植被的生长机理问题,具体公式如 4.2 所示。

\begin{equation}
\begin{cases}
IC_{store} = c_{p} \cdot IC_{\max} \left[ 1 - \exp \left( -k \cdot R_{cum} / IC_{\max} \right) \right] \\
IC_{\max} = 0.935 + 0.498 \cdot LAI - 0.005775 \cdot LAI^2
\end{cases}
\tag{4.2}
\end{equation}

其中,$IC_{store}$ 表示植被截流量(mm);$c_{p}$ 表示植被覆盖率;$IC_{\max}$ 表示特定植被的最大截流量(mm);$k$ 表示植被密度校正因子;$R_{cum}$ 表示累积降雨量(mm);$LAI$ 表示一个分

\section*{布式的时变参数。}

另外,锡林郭勒草原的地势平坦,水分循环以垂直形式交换,降水量与强度较低,根据植被自身光合作用的机理性,化学方程式如下:
\[
6CO_{2} + 12H_{2}O \rightarrow C_{6}H_{12}O_{6} + 6O_{2} + 6H_{2}O
\]
根据阳光、自身化合物等因素,通过附件 15 中对植物进行群落区域的划分进行结合分析,如表 4.3 所示,

\begin{table}[h]
\centering
\caption{不同群落功能植被的不同年份干重和}
\begin{tabular}{c c c c c c}
\hline
\multirow{2}{*}{\textbf{植物群落功能}} & \multicolumn{5}{c}{\textbf{年干重和(单位:g)}} \\
\cline{2-6}
 & 2016 & 2017 & 2018 & 2019 & 2020 \\
\hline
AB & 3.93 & 14.92 & 91.09 & 8.23 & 17.47 \\
PB & 5660.42 & 6247.22 & 5371.065 & 6192.32 & 10777.85 \\
PF & 2138.63 & 2110.78 & 2090.95 & 2083.34 & 2128.77 \\
PR & 5837.5 & 4457.41 & 3164.97 & 2992.5 & 3512.22 \\
\hline
\end{tabular}
\end{table}

根据上表所示,鉴于附件 15 所给轮牧周期缘故,植物群落自身具有的长势特性无法完整的在放牧强度中体现,且在五轮周期性的干重质量分析中,其质量并未出现较大的变化,因此,植被生长受到光照进行光合作响并不会导致枯萎率陡增问题,均符合自然生长规律,对于该条件暂不考虑。

\subsection*{4.3.3 建立放牧-植被影响模型}

根据锡林郭勒盟自身的地理特性缘故,结合所给附件 15 的数据,具体定位的草原植被影响为典型草原,如图 4.6 所示。

\begin{figure}[h]
\centering
\includegraphics[width=\textwidth]{image1.png}
\caption{数据草原类型定位图}
\end{figure}

根据上述分析的过程,以附件 15 的轮牧方式与干重的因素进行处理,鉴于放牧方式与放牧强度具有重复含义,且所提供数据的放牧方式固定,本问仅考虑放牧强度为 \(Me\),另外,根据放牧的处理方式,这里明确定义:一轮内每条放牧处理均是时间连续的,并非单独实验,具有一定的时序性。

\subsubsection{(1) 差分自回归移动平均模型}

根据该时序性的特性,对附件 15 数据仅选用轮牧方式、干重、时间序列这三列值进行差分自回归移动平均模型的处理。

首先进行平稳性的监测,通过时序图与自相关图进行检验,这里分别进行处理,将全部干重进行处理后,具体以无牧方式为例,见图 4.7 所示,其轮牧后的植被生长趋势均类似,并没有出现较大的波动,符合时序性。

\begin{figure}[h]
    \centering
    \includegraphics[width=\textwidth]{image1.png}
    \caption{时序性分析图}
    \label{fig:time_series}
\end{figure}

另外,采用自相关系数(ACF)和偏自相关系数(PACF)检验处理后的数据是否具有平稳性,如图\ref{fig:acf_pacf}所示。

\begin{figure}[h]
    \centering
    \includegraphics[width=\textwidth]{image2.png}
    \caption{ACF 与 PACF 相关性图}
    \label{fig:acf_pacf}
\end{figure}

根据生成的 ACF 与 PACF 图,ACF 与 PACF 后续的均显示为收敛状态,分别为拖尾与拖尾样式。

另外,根据本问所需的放牧强度与植被生长关系的影响因素,决定融入放牧强度 $Me$ 的交互因子,建立如下数学模型 4.3:

\begin{equation}
W_{t} = \mu + \sum_{i=1}^{p} \varphi_{i} W_{t-1} + \sum_{i=1}^{q} \rho_{i} \epsilon_{t-1} + \theta_{0} Me + \theta_{1} Me * W_{t-1} + \epsilon_{t}
\tag{4.3}
\end{equation}

其中 $C$ 表示常数项,$\epsilon_{t}$ 表示提前假设平均数为 0 的值,$p$ 表示自回归模型的拖尾度,$\varphi_{i}$ 表示干重值,$q$ 表示自回归模型的滑动平均项数,$\theta_{0}$ 表示放牧强度对植被生物量的影响因子,$\theta_{1}$ 表示放牧强度对植被生物量的交互因子;$Me$ 表示放牧强度。

(2) 明确各类影响因素

\section*{A. 量化放牧植物生长因素}

鉴于扩展阅读中的放牧与植物生长之间的关系,定义放牧与植物生长比率为 \(\Delta V_1\),即 4.3 所示。

\begin{equation}
\Delta V_1 = \frac{dw}{dt} = 0.049w\left(1 - \frac{w}{4000}\right) - 0.0047Sw
\tag{4.4}
\end{equation}

其中 \(w\) 为植被生物量,\(S\) 为单位面积的载畜率(即一定时间内,在一定草地面积上,草地实际放牧的家畜头数),鉴于 \(Me\) 的单位为羊/天/公顷,将其量化成对应关系式 4.5 为:

\begin{equation}
Me \Leftrightarrow S
\tag{4.5}
\end{equation}

\section*{B. 量化 IC 值}

鉴于扩展阅读中的第四个模型提示,植被截流量能够很好反映植被的生长能力,且根据 IC 的公式,其 \(IC\) 值与 \(LAI\) 直接相关,分析其附件 10 所给的叶面积指数趋势图,如图 4.9 所示,其叶面积指数的变化规律符合季节性,与放牧强度的影响因素并不大,没有明显的异常数据,这里暂不考虑 \(IC\) 值与 \(Me\) 间存有相关性。

\section*{(3) 建立放牧-植被影响因素模型}

根据步骤 1 与 2 所建立的差分自回归平均模型与明确各类影响因素问题,综合建立数学模型如 4.6 所示。

\begin{equation}
\begin{cases}
W_t = \mu + \sum_{i=1}^p \varphi_i W_{t-1} + \sum_{i=1}^q \rho_i \epsilon_{t-1} + \theta_0 Me + \theta_1 Me * W_{t-1} + \epsilon_t + \Delta V_1 \\
\Delta V_1 = \frac{dw}{dt} = 0.049w\left(1 - \frac{w}{4000}\right) - 0.0047Me * w
\end{cases}
\tag{4.6}
\end{equation}

其中 \(\mu\) 表示常数项,\(\varphi_i\) 表示植物影响因素,\(\epsilon_t\) 表示假设平均数为 0 值,\(\Delta V_1\) 表示放牧强度与牧草生长率,\(\theta_0\) 表示放牧强度对植被的影响因子;\(\theta_1\) 表示放牧强度对植被的干扰因子

\begin{figure}[h]
\centering
\includegraphics[width=0.8\textwidth]{image.png}
\caption{叶面积指数生长周期图}
\end{figure}

\section*{4.4 建立放牧-土壤物理性质影响模型}

\subsection*{4.4.1 影响物理性质因素的选取}

鉴于 4.2 的相关性分析,不难看出附件 4 的土壤蒸发量两个指标具有相同的内容,默认以单位为 \((mm)\) 的指标作为主指标,同时发现 \(10cm\) 的土壤湿度、土壤蒸发量和降雨量这三者具有强相关性,将这三个指标进行统一的量化,作为主要的影响因子。另外,鉴于 \(40cm\)、\(100cm\) 与 \(200cm\) 的土壤湿度具有弱相关,虽然对土壤物理性质影响较小,但为了避免深层湿度产生的潜在影响,这里将其进行统一的量化,融合成一个影响指标。

另外,根据本问所提供的扩展知识部分,以土壤含水量-降水量-地表蒸发模型,如公式

4.7 所示。

\begin{equation}
\frac{d\beta}{dt} = P - E(a)
\tag{4.7}
\end{equation}

其中 P 为降水率,E 为蒸发率,$\beta$ 为土壤含水量,$a$ 为土壤植被覆盖率。

根据所提供的蒸发模型,不难看出土壤蒸发量与 E($\alpha$) 相关,P 与降雨量相关,需要将这两类指标进行关联。

\subsection*{4.4.2 模型建立}

根据选取的影响物理性质指标,逐个进行分析融合,以建立放牧-土壤物理性质影响模型 [4]。

(1) 降水量指标

根据扩展公式,P 代表该牧区供水率,且主要为该地区的降水率,为了明确 P 值,结合附件 8 所给的降水量进行拟合,定义某一时段降水量为 $Ra$,即如公式 4.8:

\begin{equation}
P = k_{1} * Ra + k_{2} * Ra^{2} + k_{3} * Ra^{3} + \varepsilon
\tag{4.8}
\end{equation}

为了明确拟合的方式,通过枚举实验进行处理,如图 4.X 所示,发现经过三次拟合,降水量的数据拟合形式较好。

(2) 土壤蒸发量指标、10cm 土壤湿度指标

鉴于附件 4 的土壤蒸发量数据,定义土壤蒸发量为 $Ev$;鉴于附件 3 的土壤湿度数据,定义 10cm 土壤湿度指标为 $We_{10}$。

\begin{figure}[h]
\centering
\includegraphics[width=\textwidth]{image.png}
\caption{降雨量拟合曲线}
\end{figure}

(3) 统一指标

鉴于附件 3 的土壤湿度数据,加上 4.2 的相关性分析,这里对 40cm、100cm、200cm 的土地湿度指标进行融合量化,根据相关性,将权重进行重新赋值,具体如公式 4.9 所示。

\begin{equation}
We_{all} = 0.5We_{40} + 0.3We_{100} + 0.2We_{200}
\tag{4.9}
\end{equation}

(4) 放牧-土壤物理性质影响模型

为了明确土壤物理性质影响的因素,结合所给的公式进行整合,给出定性的土壤物理性质影响的整体公式,如公式 4.10 所示。

\begin{equation}
\Delta V_{2} = \frac{d\beta}{dt} = P - E(a) + \alpha_{1}Ev + \alpha_{2}We_{10} + \alpha_{3}We_{all} + \delta
\tag{4.10}
\end{equation}

其中 $\Delta V_{2}$ 表示土壤物理性质变化速率,$\delta$ 表示实际存在的误差值。

鉴于放牧强度与放牧方式之间存在着重复的问题,且附件中并未涉及到放牧方式的相关数据,这里仅从放牧强度进行入手,从实际问题与理论角度出发,不难发现放牧强度与土壤物理性质的相关性成反比,这里给出具体的放牧-土壤物理性质影响模型,如公式 4.11 所示。

\begin{equation}
\left\{
\begin{aligned}
Me &= \mu * \frac{1}{Gg} + \epsilon \\
Gg &= \Delta V_{2} * t \\
\Delta V_{2} &= \frac{d\beta}{dt} = P - E(a) + \alpha_{1}Ev + \alpha_{2}We_{10} + \alpha_{3}We_{all} + \delta \\
P &= k_{1} * Ra + k_{2} * Ra^{2} + k_{3} * Ra^{3} + \varepsilon \\
We_{all} &= 0.5We_{40} + 0.3We_{100} + 0.2We_{200}
\end{aligned}
\right.
\tag{4.11}
\end{equation}

其中 $\epsilon$、$\delta$、$\varepsilon$ 代表容错误差,Gg 表示土壤物理性质。

\section{问题二:多维影响-时序性的 ARI 预测模型}

\subsection{问题二分析}

根据问二要求,需对附件 3、4、8 等数据进行整合与挖掘分析,因此对三个附件进行合并与预处理,作为问题二的基础数据。某地的土壤湿度会受到蒸发数据、降水量、降水天数、气温等影响,对各类影响因素进行相关性分析,将其降维排序,以筛选出重要的影响指标,以建立综合性的土壤湿度预测模型。首先,所选附件数据规模存在着部分的差异,在合并处理前需要二次清洗与处理;其次,鉴于三个附件共 14 个影响因素,结合题干所给的“土壤湿度数据”、“蒸发数据”以及“降水”关键词,决定对 14 个影响因素降维并筛选,由于未明确变量间的相关特性,需进行独立性检验,以提出高相关变量;然后,根据附件数据的时序特性,选用 ARI 序列模型作为主模型,建立土壤湿度预测模型;最后,为了验证所建立的土壤湿度预测模型准确性,采用具有较高时序特性的深度学习模型(LSTM+Self-attention)进行二次验证,以多维度证实预测结果的精度,进而填充问二的预测表,具体思路见图 5.1 所示。

\begin{figure}[h]
\centering
\includegraphics[width=0.8\textwidth]{image.png}
\caption{问题二思路流程}
\label{fig:5.1}
\end{figure}

\subsection{建立多维影响-时序性的 ARI 预测模型}

\subsubsection{数据合并与处理}

根据问题二所给要求,需要将附件 3、4、8 三个附件数据合成同一表单,进行整体性的分析。根据附件所给内容,其经纬度位置固定,且三表单包含年份与月份的关键信息,满足数据表单的关联特性,进行整合,具体操作如下:

(1)对于附件 3 而言,包含年份属性,并没有月份信息,结合问二所给表单月份信息与表名信息,自行添加月份属性栏,以关联其他表格。

(2)对于附件 4 而言,包含年份、月份属性,可以直接关联,且根据问一 4.2 的相关性分析,其中土壤蒸发量 $(W/m^2)$ 与土壤蒸发量 $(mm)$ 的具有相同性质,为了避免后续高相

关性所带来的耦合程度,提出单位为$(W/m^2)$的数据,保留$(mm)$数据进行主要指标。

(3)对于附件 8 而言,包含 24 个指标,其中平均气温$\geq 18^\circ \mathrm{C}$、$35^\circ \mathrm{C}$与$\leq 0^\circ \mathrm{C}$的天数其 0 填充的值较多,且从常理角度,对土壤湿度影响不大进行剔除;其中关于气温的共有 5 种指标,而平均最高温、平均最低温、最高气温极值与最低气温极值均不能代表该月的整体性质,故保留平均气温作为指标;其中积雪深度 nan 值较多,例如 14 年全年缺失等,进行剔除;其中风速指标共有 5 中指标,同气温类似,均取平均风速为代表,作为该问指标;其中能见度指标共 3 种,同气温类似,均取平均能见度为代表,作为该问指标。

\subsection*{5.2.2 数据降维与筛选}

根据上述整合的表单信息,结合题干所给附件数据关键词,为了进一步探究数据之间的相关性,决定采用信息增益的方法,进行降维与筛选主要因素指标。

信息增益表示得知某种特征 $X$ 的信息在目标信息 $Y$ 的不确定性减少的程度,即目标信息 $Y$ 的熵与属性 $X$ 的条件信息熵的差值,其信息熵的计算公式如下:

\begin{equation}
Ent(Y) = -\sum_{i=1}^{N} p_i \log_2 p_i
\tag{5.1}
\end{equation}

其中条件信息熵的计算公式如 5.2 所示。

\begin{equation}
Ent(Y|X) = \sum_i p_i Ent(Y|X=X_i), p_i = p(X=X_i), i=1,2,3,\dots,n
\tag{5.2}
\end{equation}

因此,可以根据信息增益性质推论出公式 5.3:

\begin{equation}
Gain(Y,X) = Ent(Y) - Ent(Y|X)
\tag{5.3}
\end{equation}

我们根据自己处理合并的表单数据,通过信息增益计算,得到各类指标的信息增益图,如图 5.2 所示。

\begin{figure}[h]
\centering
\includegraphics[width=\textwidth]{image.png}
\caption{各类因素指标的信息增益图}
\end{figure}

根据因素指标的信息增益,可以明显的看出,土壤蒸发量、降水量、最大单日降水量与降水天数的信息增益最大,说明与不同深度土壤湿度的关系最为密切;其余的因素指标信息增益普遍偏小,说明与不同深度土壤湿度的关系不紧密,为方便后续时序模型的建立,忽略该部分的因素指标所产生的影响。

\subsection*{5.2.3 建立多维影响-时序性的 ARI 预测模型}

根据问题二所填表单信息,结合筛选的因素指标,均具有时序特性,根据土壤蒸发量、降水量、最大单日降水量与降水天数的变化趋势,建立对应的时序模型[5],其逻辑图 5.3 如下所示,另外,根据附件 4 的经纬度定位与附件 15 的经纬度定位相同规则,其放牧方式为轮牧的形式,即默认放牧的形式为轮牧且强度不变。

\begin{figure}[h]
    \centering
    \includegraphics[width=\textwidth]{image.png}
    \caption{预测模型建立逻辑图}
    \label{fig:5.3}
\end{figure}

\subsubsection{各类因素指标自身预测模型}

令不同因素指标在原始状态为 $a_{tj}$,其中 $t$ 表示第 $t$ 月份的采集,$j$ 表示第 $j$ 个因素指标,即最初始状态各类因素指标的公式如 (5.4) 所示。

\begin{equation}
A_{j} = \frac{1}{n_{1}} \sum_{t=1, j=1}^{n_{1}} a_{tj}
\tag{5.4}
\end{equation}

其中,$t$ 的取值为月份,$j$ 的取值为第 $j$ 个因素指标。

令不同因素指标在后续采集状态为 $b_{tj}$,其中 $t$ 表示第 $t$ 月份的采集,$j$ 表示第 $j$ 个因素指标,即经过时间变迁后各类因素指标的公式如 (5.5) 所示。

\begin{equation}
B_{j} = \frac{1}{n_{2}} \sum_{t=1, j=1}^{n_{2}} b_{tj}
\tag{5.5}
\end{equation}

其中,$t$ 的取值为月份,$j$ 的取值为第 $j$ 个因素指标。

构建整体的预测系数公式 (5.6) 所示。

\begin{equation}
C_{j} = \frac{(A_{j} - B_{j})}{B_{j}} \times 100\% \quad (j=1, 2, \dots, m)
\tag{5.6}
\end{equation}

即各类因素指标的整体预测模型如公式 (5.7) 所示。

\begin{equation}
A_{j} = B_{j}(1 + C_{j})
\tag{5.7}
\end{equation}

\subsubsection{建立多维影响-时序性的湿度预测模型}

\paragraph{A. 平稳性检验}

鉴于所给的附件 3 的土壤湿度数据,其采集具有时序特性,采用自相关图进行检验,同问一的自相关系数 (ACF) 与偏自相关系数 (PACF) 进行处理,如图 5.4 所示。

\begin{figure}[h]
    \centering
    \includegraphics[width=\textwidth]{acf_pacf.png}
    \caption{ACF 和 PACF 相关性图}
    \label{fig:5.4}
\end{figure}

根据上述生成的 ACF 与 PACF 图,ACF 为拖尾式,而 PACF 为截尾式,我们选用 AR 作为主要的自回归模型。

\paragraph{B. 白噪声检验}

为进一步验证土壤湿度自身具有回归性,本文选用 Ljung-Box 检验方法来进行处理,通过检验 $m$ 阶滞后范围内序列自相关的显著性,来获取 $p$ 值,具体流程如图 5.5 所示。

\begin{figure}[h]
    \centering
    \includegraphics[width=\textwidth]{image1.png}
    \caption{白噪声检验流程图}
    \label{fig:5.5}
\end{figure}

通过对四种不同深度的土壤湿度进行白噪声检验,发现P值均小于0.05,均为非白噪声序列,具有一定的自回归性,即可通过自回归模型进行处理。

\subsection{C. 差分检验}

鉴于附件3所给数据具有一定的时序性,以10cm的土壤湿度进行差分处理,具有很明确的数据波动与对称性,如图\ref{fig:5.6}所示。

\begin{figure}[h]
    \centering
    \includegraphics[width=\textwidth]{image2.png}
    \caption{10cm土壤湿度的差分检验图}
    \label{fig:5.6}
\end{figure}

\subsection{D. 多维影响-时序性的 ARI 预测模型}

根据步骤1所创建的各类因素指标自身预测模型,结合降维变量的筛选,我们可看出各类因素指标影响着不同的土壤湿度,因此,建立多维影响-时序性的AR模型,其定义如\ref{eq:5.8}所示。

\begin{equation}
    Wet_{t} = C + \sum_{i=1}^{p} \vartheta_{i,ca} Wet_{t-i} + \beta_{0} A_{j} + \varepsilon_{t}
    \tag{5.8}
    \label{eq:5.8}
\end{equation}

其中C为常数项;$\varepsilon_{t}$是提前假设平均数为0的值;p为拖尾度;$\vartheta_{i,ca}$表示不同深度的土壤自身湿度;$\beta_{0}$表示各类影响因素对不同深度的土壤湿度的影响因子;$A_{j}$表示各类因素指标。

根据ACF与PACF图观测,这里将p假设为一阶自回归,将合并表单进行预测,获取预测模型如图\ref{fig:5.7}所示,以10cm与40cm为例:

\begin{figure}[h]
    \centering
    \includegraphics[width=0.8\textwidth]{image3.png}
    \caption{预测模型图}
    \label{fig:5.7}
\end{figure}

\begin{figure}[h]
    \centering
    \includegraphics[width=\textwidth]{image.png}
    \caption{10cm 与 40cm 不同深度的土壤湿度预测图}
    \label{fig:soil_moisture_prediction}
\end{figure}

\subsection{验证模型-基于时序性的 LSTM-SA 模型}

\subsubsection{验证模型的建立}

鉴于合并的表单具有较强的时序性质,且上述建立的模型根据题干对部分的影响因素进行剔除,为了避免剔除部分的影响指标存在的潜在可能性,且各个影响因素间的相关性并不明确,决定选用长短时记忆网络进行数据的二次验证。

\subsubsection{LSTM-SA 模型的构建}

LSTM 模型是一种递归性的神经网络模型,是 RNN 的一种改进,能够对长期以来的某些信息进行学习。RNN 网络特点是将原始的历史数据作为输入且连接对当前信息的判断中,对于序列性的数据处理具有较好的表现效果,但是存有梯度消失的致命问题,因此 LSTM 的出现能够很好地避免这种问题,捕获序列信息中的长期特性。其机理特性如图 \ref{fig:lstm_mechanism} 所示。

\begin{figure}[h]
    \centering
    \includegraphics[width=\textwidth]{lstm_mechanism.png}
    \caption{LSTM 工作机理}
    \label{fig:lstm_mechanism}
\end{figure}

从土壤湿度概念入手,根据问一的机理性分析,土壤湿度是土壤物理性质的重要体现,可定义为一种全局性的特征,为了能够使得时序模型预测的更加准确,融入 Self-attention 机制,从全局的信息进行把握,符合湿度自身的全局特性,其融合后的机制流程如图 5.9 所示。

\begin{figure}[h]
    \centering
    \includegraphics[width=\textwidth]{lstm_sa_model.png}
    \caption{LSTM-SA 模型}
    \label{fig:lstm_sa_model}
\end{figure}

(2) LSTM-SA 模型的预测

鉴于 LSTM-SA 模型其原理仍是神经网络的原型,包含输入层、隐含层与输出层,为了贴合本文时序性的性质,需提前对于 LSTM-SA 模型进行参数设置,在隐含层获取临时输出,汇总到最后的输出模型,其参数设置具体如表 \ref{tab:param_settings} 所示。

\begin{table}[h]
    \centering
    \caption{参数设置表}
    \label{tab:param_settings}
    \begin{tabular}{l l}
        \hline
        参数设置 & 设置理由 \\
        \hline
        数据集划分 & 考虑到月份因素,时序性的前后逻辑,将数据集以月份的进行归类预测 \\
        激活函数 & Relu: $f(x) = max(0, w^T x + b)$ \quad 作用:避免梯度爆炸问题 \\
        隐含节点 & 100 层:$H = \sqrt{m + n} + \gamma$ \quad 作用:经验公式处理 \\
        学习率设置 & 0.001 \quad (学习率的上限与下限分别设置为 0.01 和 0.001) \\
        训练次数 & 50 次 \\
        \hline
    \end{tabular}
\end{table}

(3) 模型求解

通过 LSTM 分别对 10cm、40cm、100cm 和 200cm 的土壤湿度进行预测,具体以七月份的预测为样例进行,如图 \ref{fig:prediction_example} 所示。

\begin{figure}[h]
    \centering
    \includegraphics[width=\textwidth]{prediction_example.png}
    \caption{预测样例}
    \label{fig:prediction_example}
\end{figure}

\begin{figure}[h]
    \centering
    \includegraphics[width=\textwidth]{image.png}
    \caption{5.10 LSTM 七月预测数据}
    \label{fig:5.10}
\end{figure}

\subsection{5.3.2 验证数据对比}

由于 LSTM 与多维影响-时序性的 AR 预测模型所采用的预测方式不同,前者是以月的形式进行预测,后者是以年的形式进行预测,需要将两部分数据进行实验效果的对比,具体以 1 月的预测数据进行对比,具体如表 5.2 所示。

\begin{table}[h]
    \centering
    \caption{2022 年与 2023 年 7 月数据预测对比}
    \label{tab:5.2}
    \begin{tabular}{l c c c c c}
        \hline
        模型 & 年份月份 & 10cm 湿度 & 40cm 湿度 & 100cm 湿度 & 200cm 湿度 \\
        \hline
        多维影响-时序性的 ARI 预测模型 & 2022-7 & 17.49489938 & 48.37765523 & 71.61846008 & 165.6 \\
        & 2023-7 & 17.75938388 & 49.83688391 & 85.59494611 & 165.49 \\
        \hline
        LSTM-SA & 2022-7 & 20.46536647 & 49.31589562 & 39.65386331 & 166.82677 \\
        & 2023-7 & 20.42134568 & 49.65656955 & 35.02548622 & 166.51525 \\
        \hline
        误差范围 & & 15\% & 2\% & 82\% & 1\% \\
        \hline
    \end{tabular}
\end{table}

不难看出以月预测的 LSTM 与以年预测的多维影响-时序性的 ARI 预测模型对比其 100cm 湿度的数据差距较大,结合附件 3 所给的真实数据,对不同深度的土壤湿度进行趋势分析,如图 5.11 所示。

\begin{table}
\centering
\caption{表5.3 问题三表单}
\begin{tabular}{c c c c c c}
\hline
年份 & 月份 & 10cm湿度 & 40cm湿度(kg/m2) & 100cm湿度(kg/m2) & 200cm湿度(kg/m2) \\
 & & (kg/m2) & & & \\
\hline
 & 04 & 15.746325 & 51.53291 & 93.61446 & 164.436 \\
 & 05 & 16.790705 & 49.01263 & 93.2793 & 164.6394 \\
 & 06 & 18.445235 & 49.26309 & 91.8871 & 164.2728 \\
 & 07 & 20.310235 & 60.29955 & 95.59675 & 164.1797 \\
2022 & 08 & 19.676975 & 56.40006 & 99.08659 & 164.3842 \\
 & 09 & 18.50171 & 57.01694 & 101.0235 & 164.1487 \\
 & 10 & 16.620675 & 53.20206 & 103.3331 & 163.9864 \\
 & 11 & 14.908695 & 52.44073 & 104.4948 & 163.7461 \\
 & 12 & 14.529775 & 50.90798 & 104.6208 & 163.9484 \\
 & 01 & 13.43098 & 50.58822 & 104.4928 & 163.7547 \\
 & 02 & 12.98925 & 50.96888 & 104.4586 & 163.8993 \\
 & 03 & 13.7761 & 50.58425 & 105.0062 & 163.6225 \\
 & 04 & 15.27788 & 50.45489 & 104.9766 & 163.5991 \\
 & 05 & 16.460795 & 48.07582 & 103.4446 & 163.9104 \\
 & 06 & 18.23836 & 49.22661 & 102.323 & 163.5551 \\
2023 & 07 & 20.59874 & 60.95818 & 108.1063 & 163.721 \\
 & 08 & 19.33304 & 57.14289 & 111.2008 & 163.2961 \\
 & 09 & 18.627465 & 58.1428 & 112.4789 & 163.3703 \\
 & 10 & 16.882745 & 54.59592 & 116.6836 & 162.8119 \\
 & 11 & 16.18692 & 52.85633 & 117.274 & 162.1374 \\
 & 12 & 15.8363 & 53.27041 & 117.3573 & 162.2935 \\
\hline
\end{tabular}
\end{table}

\section{问题三:建立土壤-放牧化学模型}

\subsection{问题三分析}

根据问题三的要求与所给附件,考虑到实际实验过程中存在的测量等误差,需要对所给附件数据进行预处理操作。

为明确问题三所需要用到的附件数据,从土壤化学性质角度入手,需要对关乎土壤化学性质的附件数据进行相关性分析,结合附件 14 涉及到的有机碳(SOC)、土壤无机碳(SIC)、全氮(TN)等化学物质,考虑到自然界不同物质间的相互影响,例如:降水会对土壤有机碳和无机碳之间的耦合关系产生影响 \cite{ref6},暂定附件 3、4、7、8、14 作为问题三的主要数据,其中附件 3、4、7、8 已在问题一处理,遍历附件 14,无缺失值、无异常值,不需要处理。

对附件数据进行预处理后,结合相关性分析的结果和对题目的理解,确定最终数据集。根据题目要求,从机理性角度入手分析,首先沿用问题一放牧方式与放牧强度的关系,然后,从土壤化学性质入手 \cite{ref6},主以有机碳(SOC)、土壤无机碳(SIC)、全氮(TN)作为关键影响因素,结合附件所给其他化学因素,建立放牧-土壤化学性质影响预测模型,最后,将预测结果填充至问题三的预测表具体的解题思路见图 \ref{fig:flowchart} 所示。

\begin{figure}[h]
    \centering
    \includegraphics[width=0.6\textwidth]{flowchart.png}
    \caption{问题三思路流程图}
    \label{fig:flowchart}
\end{figure}

\subsection{数据集的确定}

基于本问所给的已知条件,首先,遍历预处理后的暂定附件数据,附件 3、4、8 主要涉及土壤湿度、地区气象条件等物理性质,且是在固定位置(经度 $115.375^\circ$,纬度 $44.125^\circ$)连续 10 年的测量值,附件 14 为 12-20 年偶数年 2 实验数据,无法以时间为基准整合。由于本文主要探讨放牧策略(放牧方式和放牧强度)与土壤化学性质的影响,因此选用附件 14 为主要数据;其次,根据名词解释中的土壤 PH 和土壤 PH 值的影响因素,将附件 7 纳入考虑范围;最后,绘制附件 14 中土壤有机碳和无机碳等化学因素数据直方图,结果显示数据量不满足正态分布的特性,故属于非线性相关,决定进行斯皮尔曼相关性分析,以确定最终选用的数据集,具体相关性分析见图 \ref{fig:correlation} 所示。

\begin{figure}[h]
    \centering
    \includegraphics[width=\textwidth]{heatmap.png}
    \caption{土壤化学性质各指标相关性分析}
    \label{fig:heatmap}
\end{figure}

根据图 \ref{fig:heatmap} 所做的相关性分析,不难看出土壤有机碳(SOC)和全氮(TN) \cite{ref7} 呈现正向强相关,与土壤无机碳(SIC)成中等负相关;土壤 C/N 比与土壤全碳(STC)比成中等正相关,与全氮(TN)成负向强相关。

根据上述分析,确定最终的问题三数据集,具体见表 \ref{tab:data} 所示。

\begin{table}[h]
    \centering
    \caption{问题三数据集的确定}
    \label{tab:data}
    \begin{tabular}{l l}
        \hline
        分类 & 数据集选用 \\
        \hline
        土壤化学性质 & 附件 14:SOC 土壤有机碳、SIC 土壤无机碳、全氮 TN \\
        & 附件 7:浅层酸碱度 pH、深层酸碱度 pH、 \\
        & 浅层碳酸钙含量 (\% weight)、深层碳酸钙含量 (\% weight) \\
        \hline
    \end{tabular}
\end{table}

\subsection{建立放牧-土壤化学性质影响预测模型}

\subsubsection{影响因素的确定}

考虑到自然界 C 循环时间周期不明显,有机碳与无机碳间的相互转化十分复杂,而本问涉及到依据历史数据对未来 2022 年土壤各化学物质的预测,且历史数据周期较短,时间划分明显,无法将无机碳与有机碳的转化机理融入时间因素,故本问不考虑无机碳与有机碳间的相互转化,即对深、浅层碳酸钙含量指标给予剔除。

附件 7 显示监测点(经度 115.375°,纬度 44.125°)土壤浅层 PH 值为 7.3、深层 PH 值为 7.9。结合问题中给出的名词解释,该地区土壤酸碱度为中性,作物在中性土壤生长最适宜。考虑到土壤 PH 值的稳定性,即就近区域 PH 值相同,因此认为该地区土壤 PH 值为定值,且对土壤有机碳含量起正向作用。

\subsubsection{土壤化学性质机理性研究}

根据机理性分析要求,观察附件 14 中的数据,结合对土壤化学性质的已有研究,发现土壤有机碳(SOC)、土壤无机碳(SIC)、土壤全碳(STC)以及土壤全碳(STC)、全氮(TN)、土壤 C/N 比间的数学关系,如公式 (6.1)、(6.2) 所示。

\begin{equation}
    STC = SIC + SOC
    \tag{6.1}
\end{equation}

\begin{equation}
    C/N = STC / TN
    \tag{6.2}
\end{equation}

依据守恒定律,土壤中 C 元素守恒,即 C 元素总量等于无机碳与有机碳的加和,故公式 (6.1) 一定成立;研究表明,土壤中有机碳和土壤全氮成正比关系,因此土壤有机碳(SOC)和全氮(TN)呈现正向强相关合理,量化后如公式 (6.3) 所示。

\begin{equation}
    TN = \rho \times SOC
    \tag{6.3}
\end{equation}

张力研究表明,在干旱、半干旱地区,土壤中的有机碳借助二氧化碳转化为无机碳,其转移机理如公式 6.4 所示。

\begin{equation}
\text{CaC}^0\text{O}_3 + \text{C}^n\text{O}_2 + \text{H}_2\text{O} = \text{CaC}^n\text{O}_3 + \text{C}^0\text{O}_2 + \text{H}_2\text{O}
\tag{6.4}
\end{equation}

其中,$\text{CaC}^0\text{O}_3$ 中 $\text{C}^0$ 是土壤木质的古碳,而 $\text{C}^n$ 是来自植物呼吸和植物残体、土壤有机碳的分解。公式 6.4 表示原生碳酸盐向次生碳酸盐的转化,即新碳逐步取代古碳,即有机碳逐步转化为土壤无机碳,简化为 $\text{SOC} - \text{CO}_2 - \text{STC}(\text{CaCO}_3)$。

\subsection*{6.3.3 建立放牧-土壤化学性质影响预测模型}

本问探索不同放牧策略对土壤化学性质的影响,因此对确定好的数据集按照放牧方式进行划分整理,问题三还要求附件 14 介绍不同放牧强度与各化学物质的含量变化,鉴于放牧方式与放牧强度具有重复含义,且所提供数据的放牧方式固定,本问仅考虑放牧强度为 $Me$。

(1)明确不同放牧强度对化学物质的影响

由上述土壤化学物质的机理分析,对本问提到的土壤全碳 (STC)、土壤 C/N 比采用公式 6.1、6.2 进行换算。为了明确不同放牧强度对 SOC、SIC、TN 的大致影响趋势,对附件 14 采取如下处理:

\begin{enumerate}
    \item 首先按照不同放牧强度对数据集进行划分;
    \item 计算不同放牧强度下明年 SOC、SIC、TN 平均值;
    \item 绘制同一种化学物质不同年份、不同放牧强度折线图,如图 6.3 所示。
\end{enumerate}

\begin{figure}[h]
    \centering
    \includegraphics[width=\textwidth]{soc_line_plot.png}
    \caption{a: 不同放牧强度下不同年份 SOC 平均值折线图}
\end{figure}

\begin{figure}[h]
    \centering
    \includegraphics[width=\textwidth]{sic_line_plot.png}
    \caption{b: 不同放牧强度下不同年份 SIC 平均值折线图}
\end{figure}

\begin{figure}[h]
    \centering
    \includegraphics[width=\textwidth]{image.png}
    \caption{不同放牧强度下不同年份 TN 平均值折线图}
    \label{fig:6.3}
\end{figure}

图 \ref{fig:6.3}:同一种化学物质不同年份、不同放牧强度折线图

由图 \ref{fig:6.3} 可以看出,随着时间的变化,SOC 成递增趋势,SIC 成递减趋势,TN 在 2014 年略显下降,总体成递增趋势。同一种放牧强度下,SOC、SIC、TN 均与时间序列有关,即

\begin{equation}
\begin{cases}
SOC_{t} = \varphi_{1i} * SOC_{t-1} \\
SIC_{t} = \varphi_{2i} * SIC_{t-1} \\
TN_{t} = \varphi_{3i} * TN_{t-1}
\end{cases}
\tag{6.5}
\end{equation}

(2) 建立放牧-化学性质影响预测模型

① 放牧-SOC 影响预测模型

由于该地土壤 PH 值对植物生长起促进作用,本文考虑到 PH 值为定值,其促进效果也为定值,植物生长主要影响土壤有机碳,因此,

\begin{equation}
SOC_{t} = \varphi_{1i} * SOC_{t-1} + A
\tag{6.6}
\end{equation}

其中,$A$ 为土壤 PH 值对 SOC 的促进效果。

观察图 6.3-a,放牧强度对 SOC 产生一定的阻碍作用,即成反比关系,且放牧强度对 SOC 的影响并不强烈,因此使用指数函数放大并取倒数以保证产生不强烈的阻碍作用,见公式 6.7 所示。

\begin{equation}
Me = \frac{\eta_{1}}{e^{soc_{t} + \delta_{1}}}
\tag{6.7}
\end{equation}

② 放牧-SIC 影响预测模型

由于不考虑 SOC 与 SIC 间的转化作用,故土壤 SIC 的含量与放牧方式相关性较弱,但也会因草原植被与动物的发生变化,故借助立方根表示,如式 6.8:

\begin{equation}
Me = \eta_{2} \sqrt[3]{SIC_{t}}
\tag{6.8}
\end{equation}

③ 放牧-AN 影响预测模型

由图 6.2 相关性分析得,AN 与 SOC 成正向强相关,对比观察图 6.3-a、c,放牧强度对 AN 产生的阻碍作用大于 SOC,因此选用对数函数减缓数值变化并取倒数以保证阻碍作用大于 SOC,见公式 6.9 所示。

\begin{equation}
Me = \frac{\eta_{3}}{lg AN + \delta_{2}}
\tag{6.9}
\end{equation}

综合①②③模型,给出具体的放牧-土壤化学性质影响预测模型,见公式 6.10 所示。

\begin{equation}
\left\{
\begin{aligned}
Me &= \frac{\eta_1}{e^{s_0ct + \delta_1}} \\
Me &= \eta_2 \sqrt[3]{SIC_t} \\
Me &= \frac{\eta_3}{\lg AN + \delta_2} \\
SOC_t &= \varphi_{1i} * SOC_{t-1} + A \\
SIC_t &= \varphi_{2i} * SIC_{t-1} \\
TN_t &= \varphi_{3i} * TN_t \\
STC &= SIC + SOC \\
\frac{C}{N} &= \frac{STC}{TN}
\end{aligned}
\right.
\tag{6.10}
\end{equation}

\subsection{6.3.4 放牧-化学性质影响预测模型求解}

利用 Python 对模型进行求解,分别对不同放牧强度下 2022 年 SOC、SIC、TN 进行预测,结合公式 6.1、6.2 计算 STC、土壤 C/N 比,预测结果具体以放牧强度为 MGI 时,G8 小区为样例,如图 6.4 所示。

\begin{figure}[h]
    \centering
    \includegraphics[width=\textwidth]{image.png}
    \caption{放牧强度为 MGI 时 G8 小区各指标预测结果}
    \label{fig:6.4}
\end{figure}

\subsection{6.4 问三表单填充}

\begin{table}[h]
\centering
\caption{问题三表单}
\begin{tabular}{c c c c c c}
\hline
放牧 & Plot & SOC & SIC & STC & 全 N & 土壤 C/N 比 \\
强度 & 放牧 & 土壤有机碳 & 土壤无机碳 & 土壤全碳 & & \\
 & 小区 & & & & & \\
\hline
 & G17 & 16.8583 & 6.08675 & 22.94505 & 2.1152 & 10.8477 \\
NG & G19 & 17.2985 & 4.30145 & 21.59995 & 2.19805 & 9.826869 \\
 & G21 & 20.1335 & 4.0649 & 24.1984 & 2.43115 & 9.953479 \\
 & G6 & 14.78965 & 3.0012 & 17.79085 & 2.0165 & 8.822638 \\
LGI & G12 & 16.1858 & 3.8029 & 19.9887 & 1.9448 & 10.27802 \\
\hline
\end{tabular}
\end{table}

\begin{tabular}{l l l l l l}
 & G18 & 19.306 & 7.2073 & 26.5133 & 2.25715 & 11.74636 \\
 & G8 & 14.4421 & 1.8855 & 16.3276 & 1.94105 & 8.411736 \\
MGI & G11 & 14.74435 & 3.3536 & 18.09795 & 1.96145 & 9.226822 \\
 & G16 & 14.777 & 9.81945 & 24.59645 & 1.71225 & 14.36499 \\
 & G9 & 17.2 & 3.12215 & 20.32215 & 2.23695 & 9.084758 \\
HGI & G13 & 16.6919 & 3.5647 & 20.2566 & 2.10285 & 9.632927 \\
 & G20 & 15.9696 & 4.65705 & 20.62665 & 2.0546 & 10.03925 \\
\end{tabular}

\section{问题四:建立土壤-放牧化学模型}

\subsection{问题四分析}

本问核心在于利用沙漠化程度指数预测模型确定附件 15 中不同放牧强度下监测点中的沙漠化程度指数值;同时明确土壤板结化的定义下,给出最优的沙漠化程度指数与板结化程度值。首先,根据问题所给的沙漠化指数预测模型,结合文献《栅格化预警模式研究》[2],需要对涉及的 9 个因子进行数据关联与收集,同时进行数据清洗与处理;其次,结合附件 14、15 中的牧区划分与强度确定监测点的沙漠化程度指数值;然后,根据本问所给出的土壤板结化模型描述与涉及指标的关系规律,结合问题一与问题三模型,将模型描述具象化[8];最后,根据所明确的板结化与沙漠化指数进行结合,根据不同的放牧策略,建立最优的沙漠化程度指数与板结化程度最小的模型,具体流程如图 7.1 所示。

\begin{figure}[h]
\centering
\includegraphics[width=0.8\textwidth]{flowchart.png}
\caption{问题四思路流程图}
\label{fig:7.1}
\end{figure}

\subsection{附件数据处理}

\subsubsection{数据收集}

根据本问所提供的补充数据来源中“国家冰川冻土沙漠科学数据中心”词条,将沙漠(沙地)分布数据集作为本问的荒漠数据,如图 7.2 所示。

\begin{figure}[h]
    \centering
    \includegraphics[width=\textwidth]{image.png}
    \caption{中国1:10万沙漠(沙地)分布数据图}
    \label{fig:desert_distribution}
\end{figure}

同时,根据沙漠化程度指数预测模型所涉及的《基于栅格化预警模式研究》文献内提及的,涉及到九个因子,其中气象因素、地表因素均由对应的数据信息,而人文因素涉及到人口密度、牲畜密度、人均家庭经营纯收入这三个指标,根据所提供的附件2锡林郭勒统计年鉴中人文因素数据进行统计与关联,具体数据如表\ref{tab:data_supplement}所示。

\begin{table}[h]
    \centering
    \caption{数据补充表}
    \label{tab:data_supplement}
    \begin{tabular}{l l}
        \hline
        数据名 & 补充说明 \\
        \hline
        中国1:10万沙漠(沙地)分布数据集 & 荒漠化指数:(来源) \\
        & http://www.ncdc.ac.cn/portal/metadata/fb82482c-d594-46ea-bd60-92272487cb5e \\
        锡林郭勒统计年鉴(2016-2021) & 人口密度:6条、牲畜密度:6条、人均家庭收入:6条 \\
        \hline
    \end{tabular}
\end{table}

另外,题干也给出了沙漠化指数的区间范围,如表\ref{tab:desertification_index}所示。

\begin{table}[h]
    \centering
    \caption{沙漠化程度指标划分标准}
    \label{tab:desertification_index}
    \begin{tabular}{l l l l l}
        \hline
        划分内容 & 非沙漠化 & 轻度沙漠化 & 中度沙漠化 & 重度沙漠化 & 极重度沙漠化 \\
        \hline
        沙漠化程度指数 & [0,0.20] & (0.20,0.40] & (0.40,0.60] & (0.60,0.80] & (0.80,1.00] \\
        \hline
    \end{tabular}
\end{table}

\subsection{数据预处理}

根据附件14、15的数据,鉴于问题二已经进行处理,这里不在进行处理,经过收集的荒漠化信息与锡林郭勒统计年鉴,均是以年单位进行处理,这里将其量化成普通常量,关联到附件14、15,具体量化数值如下所示。

\begin{table}[h]
    \centering
    \caption{量化常量数据}
    \label{tab:quantified_data}
    \begin{tabular}{l l l l}
        \hline
        年份 & 人口密度(人/平方公里) & 牲畜密度(羊/平方公里) & 人均家庭收入(年/万) \\
        \hline
        2016 & 5.08 & 93 & 2.8017 \\
        2017 & 5.1 & 24.5 & 2.3095 \\
        2018 & 5.2 & 24.5 & 3.3138 \\
        2019 & 5.21 & 24.5 & 3.8723 \\
        2020 & 5.29 & 24.5 & 4.1936 \\
        2021 & 5.55 & 24.5 & 4.7277 \\
        \hline
    \end{tabular}
\end{table}

\section{7.3 确定监测点沙漠化指数}

\subsection{7.3.1 确定沙漠化指数预测模型}

首先根据文献所提供的预警指标体系模型,构建本间的沙漠化程度指数预测模型[9~10],鉴于提供模型涉及的因子包含:平均风速、降水量、平均气温、植被盖度、地表水量、地下水埋深、人口密度、牲畜密度和人均纯收入,直接化用文献所包含的 7.4 权重系数表,建立本模型。

\begin{table}[h]
\centering
\caption{沙漠化指数权重系数表[2]}
\begin{tabular}{c c c c c}
\hline
\multirow{2}{*}{指标体系} & 权重系数 & 影响因素 & 权重系数 & 指标因子 \\
 & $(W_{A})$ & $(B_{i})$ & $(W_{B_{i}})$ & $(C_{i})$ \\
\hline
\multirow{9}{*}{沙漠化预警指标体系A} & 1 & 气象因素$B_{1}$ & 0.3275 & 平均风速$C_{1}$ \\
 & & & & 降水量$C_{2}$ \\
 & & & & 平均气温$C_{3}$ \\
 & & 地表因素$B_{2}$ & 0.4126 & 植被盖度$C_{4}$ \\
 & & & & 地表水资源量$C_{5}$ \\
 & & & & 地下水埋深$C_{6}$ \\
 & & 人文因素$B_{3}$ & 0.2599 & 人口密度$C_{7}$ \\
 & & & & 牲畜密度$C_{8}$ \\
 & & & & 人均家庭经营纯收入$C_{9}$ \\
\hline
\end{tabular}
\end{table}

即沙漠化指数的模型如公式 7.1 所示。

\begin{equation}
\begin{cases}
SM = 0.3275B_{1} + 0.4126B_{2} + 0.2599B_{3} \\
B_{1} = 0.1802C_{1} + 0.0787C_{2} + 0.0685C_{3} \\
B_{2} = 0.2036C_{4} + 0.0808C_{5} + 0.1282C_{6} \\
B_{3} = 0.0509C_{7} + 0.1282C_{8} + 0.0808C_{9}
\end{cases}
\tag{7.1}
\end{equation}

\subsection{7.3.2 分析因子与放牧强度间相关性}

(1) 气象因素

查阅相关文献,平均风速$C_{1}$与平均气温$C_{3}$隶属温带的锡林郭勒草原,分析其平均风速与平均气温的变化趋势,其限值在 $7^{\circ}\mathrm{C} \sim 8^{\circ}\mathrm{C}$,符合自然温带的特性,这里暂不考虑与放牧强度间关系的影响。鉴于第一问所分析降雨量对地表湿度的影响因素,会间歇性产生对放牧强度的影响,同问一中放牧-土壤物理性质影响模型相关,在后续进行分析。

(2) 地表因素

在地表因素影响中,地表水资源量$C_{5}$与植被盖度$C_{4}$均在问一中已经分析,直接结合问一中的放牧-土壤物理性质影响模型进行关联处理;另外,地下水埋深含义为地下水水面至地表的距离,这里为了简化地下水埋深的含义,默认等同于不同深度的土壤湿度进行处理。如公式 7.2 所示。

\begin{equation}
Me = \mu * \frac{1}{k * f(C_{2}) - E(C_{4}) + \alpha_{1}Ev + \alpha_{2} * g(C_{5})} + \epsilon
\tag{7.2}
\end{equation}

(3) 人文因素

对于收集的人口密度、牲畜密度以及人均家庭年收入进行趋势化分析,如图 7.3 所示。

\begin{figure}[h]
    \centering
    \includegraphics[width=\textwidth]{image1.png}
    \caption{趋势分析图}
    \label{fig:trend}
\end{figure}

人口密度与人均家庭年收入均呈现稳固上涨,可默认人口密度与人均家庭年收入部分关联放牧强度,即收入增长带动人口增长,通过放牧强度增长,带动整体经济的发展,默认二者与放牧强度呈现正相关性;另外,牲畜密度自 2016 年突然骤减,经查阅相关资料,发现 2016 年实行禁牧策略,但后续实时轮牧后,其牲畜密度恢复平稳状态,因此,也可默认该数据自 2017 年后实现平稳性增长,且根据牲畜密度与强度直接相关,即可发现其 $Me$ 与 $C_{7}$、$C_{8}$、$C_{9}$ 存在的关系式 (7.3) 为:

\begin{equation}
Me = k_{1}e^{C_{8}} + k_{2}C_{7} + k_{3}\ln C_{9}
\tag{7.3}
\end{equation}

其中,$k_{i}$ 表示不同的影响度。

\subsection{7.3.3 建立放牧-沙漠化指数模型}

根据上述沙漠化指数分析与分析因子与放牧强度相关性的公式,融合成单一的模型 [11],即公式 (7.4):

\begin{equation}
\begin{cases}
SM = 0.3275B_{1} + 0.4126B_{2} + 0.2599B_{3} \\
Me = \mu * \frac{1}{k*f(C_{2}) - E(C_{4}) + \alpha_{1}Ev + \alpha_{2}*g(C_{5})} + \epsilon \\
Me = k_{1}e^{C_{8}} + k_{2}C_{7} + k_{3}\ln C_{9}
\end{cases}
\tag{7.4}
\end{equation}

\subsection{7.3.4 指数确定}

根据附件 15 的监测点分区图,同时结合附件 12,以四户典型的牧户定位,如图 \ref{fig:monitoring} 和图 \ref{fig:location} 所示。

\begin{figure}[h]
    \centering
    \includegraphics[width=\textwidth]{image2.png}
    \caption{监测点分区图}
    \label{fig:monitoring}
\end{figure}

\begin{figure}[h]
    \centering
    \includegraphics[width=\textwidth]{image3.png}
    \caption{四户示范农户地理位置}
    \label{fig:location}
\end{figure}

根据附件 12 所给数据,四户示范农户的经纬度位置较为相近,均在 1km 左右,其土壤性质的差异不会存有较大差异,主要以附件 15 给出的监测点划分为主要依据进行处理。

\begin{table}[h]
\centering
\caption{监测点轮牧方式}
\begin{tabular}{c c c c}
\hline
放牧小区 & 放牧强度 & 放牧小区 & 放牧强度 \\
\hline
G6 & 轻度 & G14 & 轻度->重度->中度->禁牧 \\
G7 & 中度->轻度->禁牧->重牧 & G15 & 轻度->重度->中度->禁牧 \\
G8 & 中度 & G16 & 中度 \\
G9 & 重度 & G17 & 禁牧 \\
G10 & 轻度->重度->中度->禁牧 & G18 & 轻度 \\
G11 & 中度 & G19 & 禁牧 \\
G12 & 轻度 & G20 & 重度 \\
G13 & 重度 & G21 & 禁牧 \\
\hline
\end{tabular}
\end{table}

根据分析的监测点数据,填充不同放牧策略下监测点沙漠化程度指数值,如表7.6所示。

\begin{table}[h]
\centering
\caption{不同放牧策略下监测点沙漠化程度指数值}
\begin{tabular}{c c c c}
\hline
放牧小区 & 沙漠化指数值 & 放牧小区 & 沙漠化指数值 \\
\hline
G6 & 0.56723 & G14 & 0.48692 \\
G7 & 0.61254 & G15 & 0.50125 \\
G8 & 0.50345 & G16 & 0.47428 \\
G9 & 0.60513 & G17 & 0.47864 \\
G10 & 0.53644 & G18 & 0.44536 \\
G11 & 0.49698 & G19 & 0.50863 \\
G12 & 0.50261 & G20 & 0.55311 \\
G13 & 0.55106 & G21 & 0.50127 \\
\hline
\end{tabular}
\end{table}

\subsection{7.4 定量土壤板结化定义}

根据题干所给的土壤板结化模型描述[12~13],如公式7.5所示。

\begin{equation}
B = f(W, C, O) \tag{7.5}
\end{equation}

其中,土壤板结化同土壤有机物O、土壤湿度W和土壤的容重C相关联,且符合公式规律如下:W越小,C越大,有机物含量O越低,导致B越严重。

即可获取相关关系式如下所示。

\begin{equation}
B \sim \frac{C}{W, O} \tag{7.6}
\end{equation}

通过问一与问二的分析,已经建立土壤有机物O与土壤湿度W同放牧强度Me的关系式,具体如下所示。

\begin{equation}
Wet_{t} = C + \sum_{i=1}^{p} \vartheta_{i, ca} Wet_{t-i} + \beta_{0} A_{j} + \varepsilon_{t} \tag{4.11}
\end{equation}

\begin{equation}
\begin{cases}
SOC_{t} = \varphi_{1i} * SOC_{t-1} + A \\
SIC_{t} = \varphi_{2i} * SIC_{t-1}
\end{cases} \tag{6.10}
\end{equation}

根据名词扩展信息,土壤容重是指一定容积土壤(包含土粒同粒间的空隙)烘干后质量与烘干前体积的比值,具体公式如7.7所示。

\begin{equation}
C = \frac{m_{quality_{f}}}{v_{volume_{b}}} \tag{7.7}
\end{equation}

根据所获取的相关关系式与公式规律,需要将 \( W, O \) 变化趋势与 \( C \) 的变化趋势统一,统一维度,即开方处理,具体定义如 (7.8) 所示。

\[
\mathbf{B} = \rho * \frac{C}{\sqrt{W * O}} = \rho * \frac{m_{quality_f}}{v_{volume_p} \sqrt{(C + \sum_{i=1}^{p} \vartheta_{i,ca} Wet_{t-i} + \beta_0 A_j + \varepsilon_t) * (\varphi_{2i} \times SIC_{t-1} + \varphi_{1i} \times SOC_{t-1} + A)}}
\]

其中 \(\rho\) 作为调节 \( W, O \) 同 \( C \) 之间趋势关系,且板结化系数均具有时序性的状态。

### 7.5 沙漠化程度与板结化程度最小优化问题

#### 7.5.1 建立多目标最优模型

优化目标:1、寻求最小的沙漠化指数;2、寻求最小的板结化程度。

目标函数:建立最优沙漠化-板结化程度的双目标规划模型,目标函数如下所示。

\[
\begin{cases}
\min Z_1 = SM \\
\min Z_2 = B
\end{cases}
\tag{7.9}
\]

其中 \( SM \) 为沙漠化程度;\( B \) 为板结化程度。

决策变量:本文建立的模型中,根据题干所给提示,且结合上述问题分析,其决策变量共 4 个,记录为:

\[
X = \{Wet, Me, O, C\}
\tag{7.10}
\]

约束条件:根据问题四要求,在优化 \( SM \) 的过程中,需要保证放牧方式 \( Me \) 对于 \( SM \) 的影响,即约束 1:

\[
0 \leq Me \leq 8
\tag{7.11}
\]

在本问中,湿度 \( Wet \)、有机物 \( O \) 和容重 \( C \) 的约束均为等式处理,即略写为公式 7.12 所示约束条件 2 为:

\[
B = f(W, C, O) \text{ and } 0 \leq B \leq 1
\tag{7.12}
\]

且在本问中,\( SM \) 约束为公式 7.14 所示。

\[
SM = 0.3275 B_1 + 0.4126 B_2 + 0.2599 B_3 \text{ and } 0 \leq SM \leq 1
\tag{7.13}
\]

即该优化模型具体表示为 7.15 所示。

\[
\begin{cases}
\min Z_1 = SM \\
\min Z_2 = B
\end{cases}
\tag{7.14}
\]

s.t.
\[
\begin{cases}
0 \leq Me \leq 12 \\
0 \leq B \leq 1 \\
0 \leq SM \leq 1 \\
B = f(W, C, O) \\
SM = 0.3275 B_1 + 0.4126 B_2 + 0.2599 B_3
\end{cases}
\]

#### 7.5.2 模型求解

根据附件 7 所提示的数据,默认该问的容重 \( C \) 为 1.37,取 2021 年数据进行整体处理,进行拟合,在给定放牧强度下,以监测点 \( 1.44 \, \text{ha} \) 为例,且默认当前单位土地内土壤状态一致,获取最优化的土地沙漠化与板结化指数,如表 7.7:

\begin{table}[h]
\centering
\caption{土地沙漠化与板结化指数表}
\begin{tabular}{c c c}
\hline
放牧强度 & 沙漠化指数值 & 板结化程度 \\
\hline
禁牧 & 0.47864 & 0.37661 \\
轻牧 & 0.50261 & 0.38224 \\
中牧 & 0.51345 & 0.40121 \\
重牧 & 0.60513 & 0.41856 \\
\hline
\end{tabular}
\end{table}

\section*{8 问题五:最优阈值确定}

\subsection*{8.1 问题五分析}

本问的核心在符合可持续发展情况下,给定降水量情形,分析实验草场内放牧羊的最大阈值。首先,根据题干附件 14、15 所给指标,并未包含降水量信息,为明确降水量与放牧羊数量的影响关系,纳入附件 8 的降水量信息,进行表单合并与数据二次处理;其次,为了确定羊群的最优阈值,建立降水量-羊群阈值模型,同时需要将放牧政策的可持续发展进行量化,以设置优化目标、决策变量与约束条件等;然后确定优化的求解模型-粒子群优化算法,通过不断迭代粒子群位置,跳出局部最优视域,从全局最优视域入手,获取最优解;最后,根据求解的最优问题填充问五在不同降雨量下,且满足可持续发展的情况下的羊群阈值最优解。

\begin{figure}[h]
\centering
\includegraphics[width=0.6\textwidth]{image.png}
\caption{思路流程图}
\label{fig:flowchart}
\end{figure}

\subsection*{8.2 附件数据处理}

对于选取的附件 8、14、15 的数据均在前问中处理,仅从题干所给的降水量信息入手,且考虑 14、15 中的指标因素不包含降水信息,需结合附件 8 中锡林郭勒盟气候的降水量。为了将该部分数据互相关联,从附件 15 中的轮牧处理方式与日期入手,根据问题假设一,以 2019 年放牧小区 G6 群落 PB 关联降雨量为例,鉴于 2019.5.30 开始采用轻牧强度,以时间序列为例,通过 10 轮处理,历经 31 天,其对应 2019 年 5 月与 6 月平均降水量分别为 30.23 与 64.77 进行关联,如表 8.1 所示。

\begin{table}[h]
\centering
\caption{附件 15 与附件 8 并表}
\label{tab:table1}
\begin{tabular}{c c c c c c c c}
\hline
年份 & 轮次 & 处理 & 日期 & 植物种名 & 植物群落能群 & 放牧小区 & 降水量 (mm) \\
\hline
2019 & 牧前 & 轻牧 (3 天) & 2019.5.10 & 大针茅 & PB & G6 & 30.23 \\
2019 & 牧前 & 轻牧 (3 天) & 2019.5.10 & 糙隐子草 & PB & G6 & 30.23 \\
\hline
\end{tabular}
\end{table}

\begin{tabular}{llllllll}
\hline
2019 & 牧前 & 轻牧(3天) & 2019.5.10 & 大针茅 & PB & G6 & 6.13 & 30.23 \\
2019 & 牧前 & 轻牧(3天) & 2019.5.10 & 糙隐子草 & PB & G6 & 4.87 & 30.23 \\
2019 & 牧前 & 轻牧(3天) & 2019.5.10 & 大针茅 & PB & G6 & 6.89 & 30.23 \\
2019 & 牧前 & 轻牧(3天) & 2019.5.10 & 糙隐子草 & PB & G6 & 0.09 & 30.23 \\
2019 & 牧前 & 轻牧(3天) & 2019.5.10 & 大针茅 & PB & G6 & 15.71 & 30.23 \\
2019 & 牧前 & 轻牧(3天) & 2019.5.10 & 糙隐子草 & PB & G6 & 1.73 & 64.77 \\
2019 & 牧前 & 轻牧(3天) & 2019.5.10 & 大针茅 & PB & G6 & 8.16 & 64.77 \\
2019 & 牧前 & 轻牧(3天) & 2019.5.10 & 糙隐子草 & PB & G6 & 0.19 & 64.77 \\
\hline
\end{tabular}

\section{8.3 确定最优阈值}

\subsection{8.3.1 优化目标}

根据题目要求需确定实验草场内放牧羊的最大数量,且需要满足给定的不同降雨量与可持续发展政策,即共有两个目标:

1. 要求不同降雨量(300mm、600mm、900mm 与 1200mm)下,实验草场放牧羊最大值。

2. 尽可能满足可持续化发展需求。

\subsection{8.3.2 目标函数}

建立降雨量-羊群阈值优化模型,即目标函数如下所示。

\begin{equation}
\max (Sheep) = f(Ra, \tau_i)
\tag{8.1}
\end{equation}

其中 $sheep$ 为最大羊群阈值,$Ra$ 为降水量,$\tau_i$ 表示各类变量。

\subsection{8.3.3 决策变量}

本问所建立的优化模型中影响羊群阈值的主要是降雨量 $Ra$ 和其他各类变量 $\tau_i$,其具体变量如下所示。

(1) 降雨量 $Ra$ 变量:鉴于降雨量所给固定值,其 $Ra = \{300, 600, 900, 1200\}$

(2) 其他各类变量 $\tau_i$ 定义:

\begin{table}[h]
\centering
\caption{变量定义说明与选用说明}
\begin{tabular}{ll}
\hline
其他各类变量 $\tau_i$ & 变量选取理由与合理性 \\
\hline
放牧政策变量 $Pl$ & “退牧还林”政策会间歇性影响羊群在单位面积内的阈值 \\
实验放牧场 $Sq$ & 实验放牧场地大小直接决定羊群阈值 \\
植物群落功能群 $Gc$ & 植被的种类影响到草自身的长势问题 \\
放牧-植被影响模型 $W_t$ & 降水量同植被生长有重要影响,直接影响羊群阈值 \\
叶面积指数 $LAI$ & 根据扩展知识:LAI 关乎植被生长,直接影响羊群 \\
\hline
\end{tabular}
\end{table}

\subsection{8.3.4 约束条件}

根据问题五的要求,在寻求最优羊群阈值的情况下,拥有下述约束条件:

(1) 放牧政策条件

根据百度百科词条《退牧还草政策》中将 30\% 重度退化草原建设围栏实时禁牧操作,结合问题 4 所给的土壤板结化与沙漠化定义,即放牧政策间接性的影响羊群活动的区域 [13],即 $pl$ 表示单位面积内重度退化草场的还草率,具体定义如公式 8.2 所示。

\begin{equation}
pl = 0.3 * \sum SM
\tag{8.2}
\end{equation}

其中 $\sum SM$ 表示该地的沙漠化指数与板结程度。

即约束条件 1 为:

\begin{equation}
0 \leq pl \leq 0.3
\tag{8.3}
\end{equation}

(2) 实验放牧场条件

根据附件 14 所给的实验草场缩略图而言,实验草场共划分为 20 块实验草地,且每块实验场地均为 1.44ha,即在 1.44ha 中寻求羊群的最大阈值,而不是在单位面积内寻求,如 40

\begin{figure}[h]
    \centering
    \includegraphics[width=\textwidth]{image.png}
    \caption{实验草场 block 划分示意图}
    \label{fig:8.2}
\end{figure}

图 8.2 所示。

即可知 \( Me \) 与 Sheep 间的关系式如下 8.4 所示。

\begin{equation}
Me = sheep / 1.44
\tag{8.4}
\end{equation}

为了方便后续的阈值求解,这里假定以试验场地的 block 作为主要的阈值求解区。

(3) 植物群落功能群

鉴于附件 15 所给出的植物群落功能群划分,同问一中所述的放牧与群落种类影响无法完全看出,但从实际角度入手 [14~16],植物群落功能群具有一定的生长特性,为了明确这部分的变化,根据这三类群落的描述,重新赋值给出对应的生长率,如公式 8.4 所示。

\begin{equation}
Gc_{i} = dw_{i}/dt =
\begin{cases}
0.1 & i = 1 \\
0.2 & i = 2 \\
0.3 & i = 3 \\
0.4 & i = 4
\end{cases}
\tag{8.5}
\end{equation}

其中 \( i=1 \) 表示 PR 是多年生根茎型禾草 (PR)、\( i=2 \) 表示多年生杂草 (PF)、\( i=3 \) 表示多年生丛草型禾草 (PB)、\( i=4 \) 表示一二年生草本 (AB)。

即约束条件 2 为:

\begin{equation}
0.1 \leq Gc_{i} \leq 0.4
\tag{8.6}
\end{equation}

(4) 放牧-植被影响模型

鉴于问题一已对植被影响模型进行机理性分析后,无需在进行处理,直接沿用问一的模型为基础进行展开,即约束条件 3:

\begin{equation}
\begin{cases}
0 \leq Me \leq 8 \\
Wt \sim Me
\end{cases}
\tag{8.7}
\end{equation}

(5) 叶面积指数 LAI

根据本问所给名词定义,叶面积指数满足叶片总面积 / 土地面积,且通过对附件 10 的遍历,获取约束条件 4:

\begin{equation}
0.5 \leq LAI \leq 1
\tag{8.8}
\end{equation}

\subsection*{8.3.5 建立降水量-羊群阈值模型}

根据上述的约束条件与目标函数,建立对应模型 8.9:

\begin{equation}
\text{Max}(Sheep) = f(Ra, \tau_{i})
\end{equation}

\begin{equation}
\text{s.t.} \begin{cases}
0 \leq pl \leq 0.3 \\
0.1 \leq Gc_i \leq 0.4 \\
0 \leq Me \leq 8 \\
0.5 \leq LAI \leq 1
\end{cases}
\tag{8.9}
\end{equation}

\subsection*{8.3.5 降水量-羊群阈值模型求解}

根据问题所给的降水量定义,为年降水量固定,对附件15的数据进行划分,明确所选用的数据集,具体如表8.3所示。

\begin{table}[h]
\centering
\caption{附件数据划分}
\begin{tabular}{l l}
\hline
降水量 & 数据选用 \\
\hline
300mm & 轮牧为2020年数据 \\
600mm & 轮牧为2019年数据 \\
900mm & 轮牧为2016年数据 \\
1200mm & 轮牧为2017与2020年数据 \\
\hline
\end{tabular}
\end{table}

分别选用四轮实际数据进行处理与迭代,带入具体的羊群优化模型,利用基于粒子群算法进行求解流程:

\begin{table}[h]
\centering
\begin{tabular}{l l}
\hline
Step 1 & 初始化粒子群(规模为50),其中初始位置采用固定降雨量信息,其余种群的初始位置随机选择,所有种群的初始速度随机化 \\
\hline
Step 2 & 根据适应度公式,计算每个粒子的适应值 \\
\hline
Step 3 & 对每个粒子而言,将当前适应值同个体历史最佳位置所对应的适应值进行比对,若当前适应值更高,则判断粒子当前是否满足所提的约束条件,若满足,即用当前位置更替为最新位置,否则不更新 \\
\hline
Step 4 & 对每个例子而言,将其当前适应值与全局最佳位置应对适应值做对比,若当前位置更高,即更新交替 \\
\hline
Step 5 & 根据公式更新每个粒子的速度与位置 \\
\hline
Step 6 & 判断是否满足约束,若没有满足,即返回step 2。 \\
\hline
\end{tabular}
\end{table}

\subsection*{8.3.6 最优阈值确定}

\begin{table}[h]
\centering
\caption{羊群最大阈值表}
\begin{tabular}{l l}
\hline
降水量 & 羊群最大阈值(以1.44ha为例) \\
\hline
300mm & 3.0723 \\
600mm & 4.5881 \\
900mm & 5.8613 \\
1200mm & 7.7496 \\
\hline
\end{tabular}
\end{table}

\section{问题六:图示演化四个示范区土地状态预测结果}

\subsection{问题六分析}

根据问题六要求,需要对附件 2、7、13、14、15 的数据进行挖掘分析,因此对这四个附件进行预处理,作为问题六的基础数据。针对问题六提到的两种情况(附件 13 放牧策略和问题 4 中得到的放牧方案)分别处理附件数据预测土地状态。首先,分析附件 13,得出四个示范区的放牧策略;其次,对土地状态(土壤肥力变化、土壤湿度、植被覆盖等)进行化学、物理的机理性分析,确定影响土地状态的因素,结合前面问题已分析的土壤湿度、植被覆盖的影响因素,所以此处主要考虑土壤肥力的影响因素;最后,结合问题一、二、三所建模型,在上述两种情况下分别预测示范区的土地状态,并用图示呈现预测结果,具体思路见图 9.1 所示。

\begin{figure}[h]
\centering
\includegraphics[width=0.8\textwidth]{image.png}
\caption{问题六思路流程图}
\label{fig:9.1}
\end{figure}

\subsection{数据分析与机理分析}

根据问题六要求,首先需要对附件 13 示范区的放牧强度进行分析,遍历附件 13 所有数据,探索得出放牧压力与放牧强度间的关系,即放牧压力等价于放牧强度,结合附件 14,分析示范牧区所处的放牧小区。

其次,从物理、化学角度对土地状态进行机理分析 \cite{ref17},结合问题一、二,在土壤湿度影响因素已处理的基础上,同问题一植被生物量与植被覆盖情况清晰的情况下,本问主要分析土壤肥力。土壤肥力是反映土壤肥沃的重要性指标,其成分主要受到养分、物理、化学因素的影响,具体分析如下:

- 从养分因素入手,考虑到植物生长所必备的 N、P、K 三大元素,结合本文给出的有机物含量,选用相同元素 N 作为养分的主影响因素;
- 从物理因素入手,将土壤的质地、水分、温度等状况划分为影响土壤肥力的物理因素;同时结合附件 8 所给气候因素指标和前文分析,认为水分与土壤湿度相关,故按作重复处理,因此仅选用温度作为物理影响因素;
- 从化学因素入手,因土壤的酸碱度、含盐量、土壤中还原性物质等因素会直接影响植物的生长,是土壤肥力的主要化学因素,同时结合附件 7 提供的锡林郭勒土壤基本数据,

选定土壤深、浅层 PH 平均值作为影响土壤肥力的化学影响因素 \cite{ref18},考虑到土壤 PH 值短期内不会发生改变,因此仅选用全氮 N 和温度分别表示土地肥力的强弱。

最后,沿用问题 4 中得到的放牧方案,借助问题一、二、三所建模型,在上述两种情况下分别预测示范区的土地状态,并用图示呈现预测结果。

\subsection{9.3 示范区土地状态预测结果}

通过问题一、二、三所建模型,在附件 13 和问题 4 两种放牧策略(a、b 策略)下,分别对四个示范区的土壤肥力(温度和全氮 N)、土壤湿度、植被覆盖进行预测,预测结果如图 9.2、9.3、9.4、9.5 所示。

\begin{figure}[h]
    \centering
    \includegraphics[width=\textwidth]{image1.png}
    \caption{示范区 1 两种放牧策略土地状态预测结果}
    \label{fig:9.2}
\end{figure}

\begin{figure}[h]
    \centering
    \includegraphics[width=\textwidth]{image2.png}
    \caption{示范区 2 两种放牧策略土地状态预测结果}
    \label{fig:9.3}
\end{figure}

\begin{figure}[h]
    \centering
    \includegraphics[width=\textwidth]{image3.png}
    \caption{示范区 3 两种放牧策略土地状态预测结果}
    \label{fig:9.4}
\end{figure}

\begin{figure}[h]
    \centering
    \includegraphics[width=\textwidth]{image.png}
    \caption{示范区4两种放牧策略土地状态预测结果}
    \label{fig:9.5}
\end{figure}

\begin{figure}[h]
    \centering
    \includegraphics[width=\textwidth]{image.png}
    \caption{示范区4两种放牧策略土地状态预测结果}
    \label{fig:9.5}
\end{figure}

\section*{10 模型的评价和推广}

\subsection{10.1 模型的评价}

\subsubsection{10.1.1 模型的优点}

1) 对于所给附件与收集数据的处理较为完备,对于附件间表单的合并处理操作,均以时序特性进行关联,防止因数据规模问题影响到回归模型的预测性。

2) 根据数据分布的特性,选用斯皮尔曼相关性分析,从十五个附件中选用对应的数据集进行处理验证,充分考虑变量特性的同时,提高数据选用的精准性。

3) 分别从物理、化学、养分角度对土壤肥力进行机理性分析,充分考虑影响土壤肥力的因素,使得预测结果更科学。

4) 基于变量时序特性,差分自回归平均移动模型与具有较高时序性的 LSTM-SA 深度学习模型分别以年与月的方式分开预测,从多方角度验证预测数据,使得模型具有更高的精度与鲁棒性。

\subsubsection{10.1.2 模型的缺点}

1) 鉴于附件三土壤湿度数据较少,使用深度学习模型,可能会存在着部分过拟合的情况。

2) 鉴于所给的附件数据较多,仅从机理性角度入手,其附件数据并未完全利用,可能会忽略一些变量所带来的放牧影响。

\subsection{10.2 模型的推广}

\subsubsection{10.2.1 适用性}

该问从机理性的角度入手,建立放牧策略与各类影响因素的模型,根据这种机理的特性,可以延伸至海洋渔与各类因素的影响模型,具有一定的适用性。

\subsubsection{10.2.2 鲁棒性}

问题二的 LSTM-SA 模型与多维影响-时序性 ARI 模型,这两个模型具有较高的时序特性,对于时序性数据处理具有较高的预测效果且较为稳定,其模型思想也可运用于环境空气的预测等,具有较为强健的鲁棒性。

\section*{11 参考文献}

[1] 王悦骅. 模拟降水对不同载畜率放牧荒漠草原植物多样性的影响[D]. 内蒙古农业大学, 2019.

[2] 张娜. 不同放牧强度对典型草原植被群落特征及土壤理化性状的影响[D]. 中国农业科学院, 2020.

[3] 杨晨晨. 不同放牧强度对锡林郭勒草甸草原群落及植物功能性状的影响[D]. 内蒙古大学, 2021.

[4] 侯琼, 王英舜, 杨泽龙等. 基于水分平衡原理的内蒙古典型草原土壤水动态模型研究[J]. 干旱地区农业研究, 2011, 29(05): 197-203.

[5] Woodward S J R. Wake G C. McCall D G. Optimal grazing of a multi-paddock system using a discrete time model[J]. Agri-cultural Systems, 1995, 48: 119-139.

[6] 张力. 土壤有机碳和无机碳耦合关系研究进展[J]. 安徽农业科学, 2017, 45(32): 121-123.

[7] 敖小蔓. 氮、磷添加对呼伦贝尔草甸草原碳循环关键过程的影响[D]. 内蒙古大学, 2021.

[8] 宫海静, 王德利. 草地放牧系统优化模型的研究进展[J]. 草业学报, 2006(06): 1-8.

[9] 马清霞, 王星晨, 高志国. 锡林郭勒草原荒漠化气候因素分析[J]. 北方环境, 2011, 23(12): 31-34.

[10] 梁晓谦, 李建平, 张翼等. 荒漠草原植被及土壤生态化学计量对降水的响应[J]. 草业科学, 2022, 39(05): 864-875.

[11] 杨雅楠, 杨振奇, 郭建英. 放牧强度对荒漠草原植被、土壤及其侵蚀特征的影响[J]. 水土保持通报, 2022, 42(04): 66-73.

[12] 刘敦利. 基于栅格尺度的土地沙漠化预警模式研究[D]. 新疆大学, 2010.

[13] 许宏斌, 辛晓平, 宝音陶格涛等. 放牧对呼伦贝尔羊草草甸草原生物量分布的影响[J]. 草地学报, 2020, 28(03): 768-774.

[14] 李颖, 龚吉蕊, 刘敏等. 不同放牧强度下内蒙古温带典型草原优势种植物防御策略[J]. 植物生态学报, 2020, 44(06): 642-653.

[15] 赵康. 季节性放牧对典型草原群落生产力的影响[D]. 内蒙古大学, 2014.

[16] 李想. 锡林郭勒草原群落生物量及物种多样性研究[D]. 内蒙古师范大学, 2022.

[17] Simon J. R. Woodward, Graeme C. Wake, et al. A Simple Model for Optimizing Rotational Grazing[J]. Agricultural Systems, 1993, 41: 123-155.

[18] 王莉. 土地沙漠化原因及林业防沙治沙措施[J]. 新农业, 2022(06): 20.