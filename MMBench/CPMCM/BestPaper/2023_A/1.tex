\begin{center}
\includegraphics[width=0.9\textwidth]{image1.png} \\
\textbf{中国研究生创新实践系列大赛} \\
\textbf{“华为杯”第二十届中国研究生} \\
\textbf{数学建模竞赛}
\end{center}

\title{WLAN 网络信道接入机制建模}
\maketitle

\begin{abstract}
无线局域网(WLAN)所提供的通信服务质量,可以参考吞吐这一指标,WLAN 较高的吞吐量可以在某种程度上反应该网络的通信较为便捷。随着移动互联网等通信技术应用场景的增多,如手机、物联设备等站点(STA)和无线路由器等无线接入点(AP)这两类节点的部署密度也日益增加,使 STA 关联于 AP 所得基本服务集(BSS)相互干扰的概率增大,导致网络吞吐量降低。

对于多节点共享信道进行同频数据传输的场景,分布式协调功能(DCF)机制提供了一种各节点避免冲突依次接入的方法,其模式有载波侦听多址接入/退避(CSMA/CA)机制。Bianchi 模型使用 Markov chain 对单 BSS 的 DCF 机制进行建模,可以求出其所有传输状态的稳态概率,从而进一步计算网络的吞吐量,该方法具有较高的精确度。本文对于 CSMA/CA 机制在不同类型的应用场景,首先使用 Bianchi 模型对吞吐量进行计算,后通过离散事件仿真方法,对该场景进行仿真模拟,从而对 Bianchi 模型的精确度进行验证。

对于问题一:两 BSS 互听系统存在同频干扰情形,标准 Bianchi 模型数值分析方法求解结果为 \( S = 6.72 \times 10^7 \, \text{bps} \),改进的 Bianchi 模型结果为 \( S = 6.60 \times 10^7 \, \text{bps} \),仿真器模拟结果均值为 \( \bar{S} = 6.35 \times 10^7 \, \text{bps} \),改进的 Bianchi 模型结果与仿真更为接近,仅相差 \( 3.8\% \)。

对于问题二:两 BSS 互听系统存在终端接收数据 SIR 较高,两个 AP 数据传输都能成功的情形,此时不存在重传。构建 Markov chain 利用数值分析方法求解结果为 \( S = 7.06 \times 10^7 \, \text{bps} \),仿真器求解结果为 \( \bar{S} = 6.72 \times 10^7 \, \text{bps} \),相差 \( 4.8\% \)。

对于问题三:两 BSS 不互听系统存在 AP 发包在时间上有交叠时 SIR 较小,导致两 AP 发包均失败的情形。此问题需要分类讨论交叠时的状态,包括处于成功传输的状态和传输失败的状态。传统的 Bianchi 模型简化了这两个状态,不适用于本问题,因此在传统 Bianchi 模型中引入了成功传输状态和传输失败状态,更贴近真实情况。数值分析方法求解结果为 \( S = 5.45 \times 10^7 \, \text{bps} \),仿真器求解结果为 \( \bar{S} = 5.55 \times 10^7 \, \text{bps} \),仅相差 \( 1.8\% \)。

对于问题四:三 BSS 系统存在 AP1 与 AP3 不互听,AP2 与两者都互听,SIR 较大使得 AP1 和 AP3 发包时间交叠也都发送成功的情形。本问题结合了问题一和问题三两种情景,其中不互听的两 AP 具有对称性,可以同时建立 Markov chain 模型进行分析,而另一个 AP 的性质与另外两个 AP 不同,因此需要单独建立 Markov chain 模型进行分析。最后将两个模型的参数联立求解系统内各节点的稳态概率。数值分析方法求解结果为 \( S = 9.76 \times 10^7 \, \text{bps} \),仿真器求解结果为 \( \bar{S} = 1.10 \times 10^8 \, \text{bps} \),相差 \( 12.27\% \)。

\textbf{关键词:DCF;CSMA/CA;Bianchi 模型;离散事件仿真}
\end{abstract}

\tableofcontents

\section{问题重述}

\subsection{问题背景}

无线局域网(WLAN, wireless local area network)技术在现代社会有着越来越广泛的应用,其具有通信范围广、速度快、成本低、组建方便等优势。WLAN 技术已经在笔记本电脑、手机、平板电脑等移动端广泛应用。目前 WLAN 技术正在推动物联网应用的发展,如智能家居、智慧医疗、工业制造等场景的实现,具有非常高的应用价值与广阔的应用前景。提高 WLAN 技术的通信速率和通信质量,是 WLAN 技术发展的重要一环 \cite{ref1}。

WLAN 基本组成部分是基本服务集(BSS, basic service set)。在一个特定的覆盖区域内,包括站点(STA, station)和一个专门管理 BSS 的无线接入点(AP, access point)。STA 可以是手机、笔记本电脑、物联网设备等,而 AP 通常是无线路由器、Wi-Fi 热点等设备。在这个系统中,AP 向 STA 发送数据称为下行方向,STA 向 AP 发送数据称为上行方向。本文将 AP 和 STA 统称为节点,并且每个节点的发送和接收是互斥的。多个节点共享一个信道,通过载波侦听多址接入/退避(CSMA/CA, carrier sense multiple access with collision avoidance)机制来避免冲突,这被称为分布式协调功能(DCF, distributed coordination function)。每个节点发出的信号都会形成一个通信区域,只有位于通信区域内的节点才能互相听到彼此的信号。

CSMA/CA 机制可以将节点在信道中进行接入传输数据的过程分为三个阶段,分别为信道可用评估(CCA, clear channel assessment)阶段、随机回退阶段和数据传输阶段。吞吐量可以在某种程度上反应 CSMA/CA 机制进行数据传输的效率高低,其计算方式为信道的利用率和物理层速率的乘积。随着 WLAN 的广泛应用,在许多场景中 AP 的部署密度会增加。考虑所有 AP 使用同一种信道的情况,当同一信道上的多个 AP 的通信区域发生重叠时,就会发生同频干扰问题。同频干扰是 WLAN 组网最值得注意的干扰问题之一,会导致网络的性能和吞吐量有一定程度的降低。

Bianchi 在 1998 年对于单 BSS,将 DCF 机制使用 Markov chain 进行模型建构,该模型假设信道理想,不会因信道质量对数据传输产生干扰。该模型的 Markov chain 的所有状态具有稳态解,根据该性质可以推导出信道在几种虚拟时隙的稳态概率,进一步计算 BSS 的吞吐量。Bianchi 模型对于其对应的传输场景具有很高的精确度,后续也有许多工作在该模型的基础上进行扩展,从而可以对不同的应用场景的吞吐量进行计算。

此外,对于真实场景中多 BSS 的同频数据传输,可以看作一种离散系统,总是可以找到某一时间点来标记系统的变化,信道接入状态是在一个个离散的时间点上发生变化。因此,通过寻找节点接入信道在离散时间点上的变化规律,通过计算机程序编程对该过程进行离散事件仿真(DES, Discrete Event Simulation)模拟是可行的。

\subsection{问题重述}

根据上文所述问题背景,题目围绕多同频 BSS 以 CSMA/CA 机制进行信道接入数据传输问题,设定是否互听,是否存在隐藏节点等不同的应用场景,对不同系统进行建模以评估其系统的吞吐。本文将解决下列四个问题:

(1) 两个 AP 分别向各自关联的 STA 下行数据,组成两个 BSS,AP 之间存在互听,并发传输时两个 STA 都接收失败,对该场景进行建模以评估系统的吞吐。

(2) 两个 AP 分别向各自关联的 STA 下行数据,组成两个 BSS,AP 之间存在互听,并发传输时两个 STA 都接收成功,对该场景进行建模以评估系统的吞吐。

(3) 两个 AP 分别向各自关联的 STA 下行数据,组成两个 BSS,AP 之间不互听(互为隐藏节点),其 STA 在两个 AP 发包存在交叠时会接收失败,且因信道质量不理想会存在一定的丢包率,对该场景进行建模以评估系统的吞吐。

(4) 三个 AP 分别向各自关联的 STA 下行数据,组成三个 BSS,其中 AP2 会与 AP1 和 AP3 互听,但 AP1 与 AP3 不互听(互为隐藏节点),AP1 与 AP2 或 AP3 与 AP2 并发传输时对应 STA 都接收失败,AP1 和 AP3 发包存在交叠时不影响对应 STA 的接收。对该场景进行建模以评估系统的吞吐。

四个问题所对应的场景可概括为图 1.1,不同的 BSS 有各自的范围圈,AP 进入另一个圈范围内则会与该圈 AP 互听,STA 在另一圈范围内则可能受该圈 AP 信号干扰而接收失败。

\begin{figure}[h]
    \centering
    \begin{subfigure}[t]{0.45\textwidth}
        \centering
        \includegraphics[width=\textwidth]{image_a.png}
        \caption{}
        \label{fig:1.1a}
    \end{subfigure}
    \hfill
    \begin{subfigure}[t]{0.45\textwidth}
        \centering
        \includegraphics[width=\textwidth]{image_b.png}
        \caption{}
        \label{fig:1.1b}
    \end{subfigure}
    \vskip\baselineskip
    \begin{subfigure}[t]{0.45\textwidth}
        \centering
        \includegraphics[width=\textwidth]{image_c.png}
        \caption{}
        \label{fig:1.1c}
    \end{subfigure}
    \hfill
    \begin{subfigure}[t]{0.45\textwidth}
        \centering
        \includegraphics[width=\textwidth]{image_d.png}
        \caption{}
        \label{fig:1.1d}
    \end{subfigure}
    \caption{对应场景:(a) 问题一 (b) 问题二 (c) 问题三 (d) 问题四}
    \label{fig:1.1}
\end{figure}

\subsection{解题思路}

根据上述问题,本文解题思路如图 1.2 所示。对第一问和第二问,利用 Bianchi 模型求解各状态的稳态概率,根据系统中时间的构成计算信道吞吐量。对问题三和问题四,利用改进的 Bianchi 模型求解计算系统吞吐量。为了验证模型的有效性,四个问题均采用离散事件仿真进行验证。

\begin{figure}[h]
    \centering
    \includegraphics[width=\textwidth]{image.png}
    \caption{解题思路}
    \label{fig:1.2}
\end{figure}

\section{模型假设与符号说明}

\subsection{模型假设}

(1) 所有节点具有相同长度的数据帧。

(2) 该场景中只有问题所需要的节点,没有其他额外节点的频率干扰。

(3) 所有节点在同一频率进行数据传输。

(4) 所有节点拥有同样的发送速率。

\subsection{符号说明}

\begin{table}[htbp]
\centering
\caption{符号说明}
\begin{tabular}{|c|l|}
\hline
\textbf{符号} & \textbf{定义} \\ \hline
$P_{act}$ & 实际传输失败条件概率 \\
$\tau$ & 节点在某时隙传输数据的概率 \\
$T_s$ & 成功传输时长 \\
$T_e$ & 单个回退时隙时长 \\
$P_{overlap}$ & 发包交叠条件概率 \\
$T_c$ & 传输失败时长 \\
\hline
\end{tabular}
\end{table}

\section{基本模型}

\subsection{Bianchi 模型}

Bianchi 在 1998 年根据二维 Markov chain 对于理想同一信道 [2-5],多 STA 给单 AP 上行数据场景所建立的模型,以阶数和退避次数这两个维度建立状态转移矩阵,该 Markov chain 具有常返性,能够求出其中所有状态的稳态解,进而可以计算数据发送成功,碰撞以及信道空闲的时间期望值,以简单易用的方法求出该场景下的信道吞吐量。基本的 Bianchi 模型如图 3.1 所示。本文从基本的 Bianchi 模型出发,通过增加状态对其进行了相应改进,后又通过仿真模拟对改进模型的效果进行测试验证。

\begin{figure}[h]
    \centering
    \includegraphics[width=\textwidth]{image.png}
    \caption{Bianchi 模型}
    \label{fig:bianchi_model}
\end{figure}

\subsection{离散事件仿真}

离散事件仿真是一种抽象模型,将某一系统的变化看成一个事件,系统会保持原状态直到相关事件的出现,在两个事件之间,系统是一个稳定状态。

具体来说,离散与连续对应,若一个系统为离散的,则系统的变化最终能够找到一个时间点来标注,而这些标注的时间点就对应着各类事件的发生,事件出现的规律为系统变化预测的依据[6-9]。

对本文来说,在本文给定条件下的 CSMA/CA 机制信道接入问题能够抽象为一系列的离散事件,后文将使用计算机编写仿真器,将不同场景抽象为离散事件进行模拟求解。

\section{问题一、二的分析与求解}

\subsection{问题分析}

问题一为两 BSS 互听的情形,当一个 AP 传输数据时,另一 AP 通过 DIFS 侦测到 RSSI 的提高,判断信道为忙从而进入 hold time。当两 AP 同时回退到 0 时,会产生同频干扰,SIR 较低,两 AP 数据传输均失败。此问题可以通过 Bianchi 模型中 Markov chain 公式推导,求得在一个时隙发送数据的概率 $\tau$ 和并发传输的条件概率 $p$,最后分析系统中各部分时间的构成,计算吞吐量。

问题二为两 BSS 互听的另一种情形,与问题一区别在于当两 AP 同时回退到 0 时,并发时两个终端接收到数据的 SIR 较高,两 AP 的数据可以并发传输成功,因此没有重传。此问题需通过 Bianchi 模型推导,并分析系统中各部分时间占比并计算吞吐量。

\subsection{问题一求解}

\subsubsection{数值求解}

根据题干计算系统中的基本参数,如式(4.1)至式(4.4)所示。

\begin{align}
1 &= \sum_{i=0}^{r} \sum_{k=0}^{W_{i}-1} b_{i,k} + \sum_{i=0}^{r} p(Failed_{i}) + p(Trans) \\
&= \sum_{i=0}^{r} \sum_{k=0}^{W_{i}-1} \frac{W_{i}-k}{W_{i}} b_{i,0} + \sum_{i=0}^{r} b_{i,0} \\
&= \sum_{i=0}^{r} b_{i,0} \cdot \frac{W_{i}+3}{2}
\tag{4.15}
\end{align}

\begin{align}
1 &= \sum_{i=0}^r \sum_{k=0}^{W_i-1} b_{i,k} + \sum_{i=0}^r p(Failed_i) + p(Trans) \\
&= \sum_{i=0}^r \sum_{k=0}^{W_i-1} \frac{W_i - k}{W_i} b_{i,0} + \sum_{i=0}^r b_{i,0} \\
&= \sum_{i=0}^r b_{i,0} \cdot \frac{W_i + 3}{2}
\tag{5.12}
\end{align}

\begin{equation}
T_s = H + E[P] + SIFS + ACK + DIFS = 14.13 + 26.33 + 16 + 32 + 43 = 131.46\mu s
\tag{4.3}
\end{equation}

\begin{equation}
T_c = H + E^*[P] + DIFS + ACKTimeout = 14.13 + 26.33 + 43 + 65 = 148.46\mu s
\tag{4.4}
\end{equation}

由 Bianchi 模型中 Markov chain 推导可得两 AP 在某时隙发送数据概率 \(\tau\) 如式(4.5)所示,两 AP 冲突条件概率 \(p\) 如式(4.6)所示。

\begin{equation}
P(AP_1) = P(AP_2) = \tau = b_{0,0} \times \frac{1-p^{r+1}}{1-p}
\tag{4.5}
\end{equation}

\begin{equation}
P(AP_2 | AP_1) = p = 1 - (1-\tau)^{N-1}
\tag{4.6}
\end{equation}

Markov chain 中各状态稳态概率和为 1,即式(4.7)。

\begin{equation}
1 = \sum_{i=0}^{r} \sum_{k=0}^{W_i-1} b_{i,k} = \sum_{i=0}^{r} \frac{W_i - k}{W_i} \cdot p^i \cdot b_{0,0}
\tag{4.7}
\end{equation}

联立式(4.5)、式(4.6)和式(4.7),当 \(N = 2\),\(m = 6\),\(r = 32\),\(W_0 = 16\) 时可得求得节点某时隙发送数据概率 \(\tau = 10.46\%\),冲突条件概率 \(p = 10.46\%\)。

对两 BSS 系统的时间占比进行分析,时间占比概率图如图 4.1 所示。系统时间包括成功传输时间、传输失败时间和回退空闲时间。图中空白部分为回退空闲时间,淡阴影部分为成功传输时间,深色阴影为并发传输失败的时间。

\begin{figure}[h]
    \centering
    \includegraphics[width=0.8\textwidth]{image.png} % 替换为实际图片路径
    \caption{问题一时间分析}
    \label{fig:time_analysis}
\end{figure}

根据图 4.1 可计算信道吞吐量 \(S\) 如式(4.8)所示。

\begin{equation}
S = \frac{\left[P(\overline{AP_1}AP_2) + P(\overline{AP_2}AP_1)\right] \cdot E[P]}{P(\overline{AP_2}AP_1)T_s + P(\overline{AP_1}AP_2)T_s + P(AP_1AP_2)T_c + (1-P(AP_1)-P(AP_2)+P(AP_1AP_2))T_e} \cdot rate
\tag{4.8}
\end{equation}

\begin{equation}
= \frac{2\tau(1-p)E(P)}{2\tau(1-p)T_s + p\tau T_c + (1-2\tau+\tau p)T_e} \cdot rate
\end{equation}

根据式 (4.8) 代入参数计算得吞吐量 $S = 6.72 \times 10^7 \, \text{bps}$。

为更准确地描述系统所有可能的状态,对 Bianchi 模型进行改进,加入成功传输状态 Trans 和失败传输状态 Failed,如图 4.2 所示。

\begin{figure}[h]
\centering
\includegraphics[width=\textwidth]{image.png}
\caption{改进的 Bianchi 模型}
\end{figure}

改进的 Bianchi 模型状态转移概率如式 (4.9) 至式 (4.13) 所示。

\begin{equation}
P\{i,k \mid i,k+1\} = 1, k \in [0, W_i-2], i \in [0, r]
\tag{4.9}
\end{equation}

\begin{equation}
P\{i,k \mid Failed_{i-1}\} = 1/W_i, k \in [0, W_i-1], i \in [0, r]
\tag{4.10}
\end{equation}

\begin{equation}
P\{0,k \mid Trans\} = 1/W_0, k \in [0, W_0-1], i \in [0, r]
\tag{4.11}
\end{equation}

\begin{equation}
P\{Trans \mid i,0\} = (1-p), i \in [0, r]
\tag{4.12}
\end{equation}

\begin{equation}
P\{Failed_{i} \mid i, 0\} = p, i \in [0, r]
\tag{4.13}
\end{equation}

Trans 状态和 Failed 状态稳态概率之和如式 (4.14) 所示。
\begin{equation}
\sum_{i=0}^{r} p(Failed_{i}) + p(Trans) = p \sum_{i=0}^{r} b_{i,0} + (1-p) \sum_{i=0}^{r} b_{i,0} = \sum_{i=0}^{r} b_{i,0}
\tag{4.14}
\end{equation}

Markov chain 中状态稳态概率之和为 1,即:
\begin{align}
1 &= \sum_{i=0}^{r} \sum_{k=0}^{W_{i}-1} b_{i,k} + \sum_{i=0}^{r} p(Failed_{i}) + p(Trans) \\
&= \sum_{i=0}^{r} \sum_{k=0}^{W_{i}-1} \frac{W_{i}-k}{W_{i}} b_{i,0} + \sum_{i=0}^{r} b_{i,0} \\
&= \sum_{i=0}^{r} b_{i,0} \cdot \frac{W_{i}+3}{2}
\tag{4.15}
\end{align}

联立式 (4.5)、式 (4.6) 和式 (4.15) 可求得节点在某时隙发送数据概率 \(\tau = 9.57\%\),冲突条件概率 \(p = 9.57\%\)。代入式 (4.8) 可求得吞吐量 \(S = 6.60 \times 10^{7} \, \text{bps}\)。

\subsubsection{仿真检验}

离散事件仿真由状态驱动,需要分析不同场景下状态的变化。对于问题一,以图 4.3 为例,说明该场景下 CSMA/CA 机制的运行规律。

(a) 当某一 AP 数据发送次数不大于最大退避阶数时:若两个 AP 不同时回退到 0,先回退至 0 的 AP 由对应的 STA 成功接收数据,并重置其数据发送次数;后回退至 0 的 AP 则暂停回退,等待下一轮数据发送。若同时回退到 0,则发生冲突,该 AP 对应的 STA 接收失败,该 AP 数据发送次数加 1,退避窗口翻倍。

(b) 当某一 AP 数据发送次数大于最大退避阶数并小于最大重传次数时:若两个 AP 不同时回退到 0,规律与情形 (a) 相同。若同时回退到 0,则发生冲突,该 AP 对应的 STA 接收失败,该 AP 数据发送次数加 1,但退避窗口不翻倍。

(c) 当数据发送次数等于最大重传次数时:若两 AP 不同时回退到 0,规律与情形 (a) 相同。若同时回退到 0,则发生冲突,该 AP 对应的 STA 接收失败,数据发送次数清 0,退避窗口重置为最小值。

\begin{figure}[h]
\centering
\includegraphics[width=\textwidth]{image.png}
\caption{(a)}
\end{figure}

\begin{figure}[h]
    \centering
    \includegraphics[width=\textwidth]{image.png}
    \caption{问题一的三种情形:(a) $i \leq m$ (b) $m < i < r$ (c) $i = r$}
    \label{fig:4.3}
\end{figure}

根据该场景的CSMA/CA机制的数据传输规律,可以将其进行离散事件仿真建模,伪代码如算法1所示。第1行至4行为参数初始化,其中仿真总时长$t$进入循环前已经经过一个DIFS时长。第5行开始进入离散事件循环,第6至11行为AP数据传输次数对退避阶数的影响,通过判断数据发送次数对退避翻倍阶数进行更新,第12至16行对应两AP不同时回退至0的情况。第17至21行对应两AP同时回退至0的情况。第23行在多轮仿真循环后进行吞吐量计算。

\begin{algorithm}[H]
    \caption{问题一离散事件仿真伪代码}
    \begin{algorithmic}[1]
        \REQUIRE 最大退避阶数$m$;最大重传次数$r$;单个空闲时隙长度$T_{e}$;成功传输时长$T_{s}$;碰撞传输时长$T_{c}$;DCF帧间距$DIFS$;仿真循环轮数$P$;有效载荷传输时长$EP$;物理层速率$rate$
        \ENSURE 离散事件仿真总时长$t$;离散事件仿真阶段数$p \in P$;$AP_{a}$数据的发送次数$i_{a}$ ($a \in \{1, 2\}$);$AP_{a}$退避翻倍阶数$j_{a}$ ($a \in \{1, 2\}$);$AP_{a}$回退次数$CW_{a}$ ($a \in \{1, 2\}$);两AP总体冲突次数$c$
        \ENSURE 信道吞吐量$S$
        \STATE \textbf{Begin:}
        \STATE $t = DIFS$
        \STATE $i_{a} = 0, \, a \in \{1, 2\} \, \forall a \in \{1, 2\}$
        \STATE $c = 0$
        \FOR{$p = 1$ \textbf{To} $P$}
            \IF{$i_{a} \leq m$}
                \STATE $j_{a} = i_{a}, \, \forall a \in \{1, 2\}$
            \ENDIF
            \STATE \dots
        \ENDFOR
    \end{algorithmic}
\end{algorithm}

\begin{tabular}{r l}
\hline
\hline
07 & \texttt{Elif } $m<i_{a}<r:j_{a}=m,\forall a\in\{1,2\}$ \quad //超过最大退避阶数 \\
08 & \texttt{Elif } $i_{a}=r:$ \quad //达到最大重传次数 \\
09 & \quad $j_{a}=0,\forall a\in\{1,2\}$ \\
10 & \quad $i_{a}=0,\forall a\in\{1,2\}$ \\
11 & \texttt{Else: Pass} \\
\\
12 & \texttt{If } $CW_{a}<CW_{b}:\forall(a,b)\in\{(1,2),(2,1)\}$ \quad //AP$_{a}$先回退至0,成功传输 \\
13 & \quad $t=t+(CW_{a}+1)\times T_{e}+T_{s}$ \\
14 & \quad $CW_{a}=random[0,2^{j_{a}}\times CW_{\min}-1]$ \quad //更新$CW_{a}$ \\
15 & \quad $CW_{b}=CW_{b}-CW_{a}-1$ //AP$_{b}$剩余回退次数 \\
16 & \quad $i_{a}=0$ //重置AP$_{a}$数据发送次数 \\
17 & \texttt{Else: } //两AP同时回退至0,冲突致传输失败 \\
18 & \quad $t=t+(CW_{a}+1)\times T_{e}+T_{c}$ \\
19 & \quad $CW_{a}=random[0,2^{j_{a}}\times CW_{\min}-1],\forall a\in\{1,2\}$ \quad //更新$CW_{a}$ \\
20 & \quad $i_{a}=i_{a}+1,\forall a\in\{1,2\}$ \\
21 & \quad $c=c+1$ \\
22 & \texttt{End For} \\
\\
23 & \texttt{Return } $S=(P-c)\times EP\times rate/t$ \quad //输出该场景吞吐量 \\
24 & \texttt{End Begin} \\
\hline
\hline
\end{tabular}

为了验证模型精确度,本文基于 Python3.9 编程语言编写仿真器实现,使用 Minitab 软件进行数值实验,每组参数下执行 40 次实验,研究在不同参数取值条件下,实验结果是否符合正态分布(显著性水平 $\alpha=0.05$),并计算仿真器与 Bianchi 模型实验误差。其中,为检验仿真结果的稳定性,选择 Anderson-Darling Test(简称为 A-D 检验)进行正态性检验,该检验主要通过计算数据的累积分布曲线与理想正态分布的累积分布曲线之间的差异来进行检验,与 K-S 检验不同,该方法考虑了两条累积分布曲线之间的所有差异,因此它比 K-S 检验效果更好。问题一和问题二的实验结果和分析在正文中呈现,问题三和问题四中题干给出参数实验结果和分析在正文中呈现,其余在附录中呈现。

问题一实验结果如表 4.1 所示,统计计算结果如图 4.4 所示,均值为 $\overline{S}=6.35\times10^{7}\text{bps}$,标准差为 $\sigma=3.3\times10^{4}$。均值 $\overline{S}$ 与改进的 Bianchi 模型求解结果相差 3.8%,与标准 Bianchi 模型求解结果相差 5.2%,证明了 Bianchi 模型的准确性和改进的有效性。

令零假设 $H_{0}$:服从正态分布。备择假设 $H_{1}$:不服从正态分布。如果 $P$ 值小于 0.05,就拒绝 $H_{0}$,如果 $P$ 值大于 0.05,则无法拒绝原假设 $H_{0}$。概率图如图 4.5 所示,使用 Minitab 计算出 $P$ 值为 0.908 大于 0.05,实验结果服从正态分布,仿真结果具有稳定性。

\begin{table}[htbp]
\centering
\caption{实验结果}
\begin{tabular}{|c|c|c|c|c|c|c|c|c|}
\hline
\textbf{序号} & \textbf{实验结果} & \textbf{序号} & \textbf{实验结果} & \textbf{序号} & \textbf{实验结果} & \textbf{序号} & \textbf{实验结果} \\ \hline
1 & 63444845 & 11 & 6345818 & 21 & 6341038 & 31 & 63426032 \\
2 & 6310458 & 12 & 63461696 & 22 & 63457 & 32 \\
3 & 6348186 & 13 & 6341720 & 23 & 6348 & 33 \\
\hline
4 & 6346513 & 14 & 63486721 & 24 & 63536 & 34 \\
5 & 63503238 & 15 & 6345919 & 25 & 63440 & 35 \\
6 & 635222 & 16 & 6347151 & 26 & 6351 & 36 \\
7 & 6348 & 17 & 63511 & 27 & 6342 & 37 \\
8 & 6348 & 18 & 6347 & 28 & 6345 & 38 \\
9 & 6349 & 19 & 63480052 & 29 & 6341 & 39 \\
10 & 6344 & 20 & 6350 & 30 & 635 & 40 \\
\hline
\end{tabular}
\end{table}

\begin{figure}[h]
\centering
\includegraphics[width=0.8\textwidth]{image1.png}
\caption{Anderson-Darling 正态性检验}
[TABLEENV:2]
\end{figure}

\begin{figure}[h]
\centering
\includegraphics[width=0.8\textwidth]{image2.png}
\caption{95\%置信区间}
\end{figure}

\begin{figure}[h]
\centering
\includegraphics[width=0.8\textwidth]{image3.png}
\caption{图4.4 问题一仿真结果正态性检验报告}
[TABLEENV:3]
\end{figure}

\begin{figure}[h]
\centering
\includegraphics[width=0.8\textwidth]{image4.png}
\caption{信道吞吐量}
\end{figure}

\begin{figure}[h]
\centering
\includegraphics[width=0.8\textwidth]{image5.png}
\caption{图4.5 问题一仿真结果概率图}
\end{figure}

\subsection{问题二求解}

\subsubsection{数值求解}

根据题干计算系统的基本参数,如式(4.16)至式(4.19)所示。
\begin{equation}
H = MAC + PHY = 30Bytes / 275.3Mbps + 13.6\mu s = 14.47\mu s
\tag{4.16}
\end{equation}
\begin{equation}
E^*[P] = E[P] = L_{payload} / rate = 1500Bytes / 455.8Mbps = 43.59\mu s
\tag{4.17}
\end{equation}
\begin{equation}
T_s = H + E[P] + SIFS + ACK + DIFS = 58.06 + 16 + 32 + 43 = 149.06\mu s
\tag{4.18}
\end{equation}
\begin{equation}
T_c = H + E^*[P] + DIFS + ACKTimeout = 14.13 + 26.33 + 43 + 65 = 166.06\mu s
\tag{4.19}
\end{equation}

问题二为互听的两BSS系统的另一种场景,由于并发时SIR较大,传输总是成功,因此不存在重传,Markov chain如图4.6所示。

\begin{figure}[h]
\centering
\includegraphics[width=0.8\textwidth]{markov_chain_diagram.png}
\caption{问题二 Markov chain}
\end{figure}

该Markov chain状态转移概率如式(4.20)和式(4.21)所示。
\begin{equation}
P\{0, k | 0, 0\} = 1 / W_0, k \in [0, W_0 - 1]
\tag{4.20}
\end{equation}
\begin{equation}
P\{0, k | 0, k + 1\} = 1, k \in [0, W_0 - 1]
\tag{4.21}
\end{equation}

Markov chain中各状态之和为1,如式(4.22)所示。
\begin{equation}
1 = \sum_{k=0}^{W_0-1} b_{0,k} = \sum_{k=0}^{W_0-1} b_{0,0} \frac{W_0 - k}{W_0} = b_{0,0} \frac{W_0 + 1}{2}, 0 \leq k \leq W_0 - 1
\tag{4.22}
\end{equation}

由式(4.22)可得AP在任意时隙发送数据的概率$\tau$如式(4.23)所示。
\begin{equation}
P(AP_1) = P(AP_2) = \tau = b_{0,0} = 2 / (W_0 + 1)
\tag{4.23}
\end{equation}

并发传输条件概率$p$如式(4.24)所示。
\begin{equation}
P(AP_2 | AP_1) = p = 1 - (1 - \tau)^{N-1}
\tag{4.24}
\end{equation}

当$N = 2$,$W_0 = 16$时,解得$\tau = p = 2 / 17$。

对当前系统的时间组成进行分析,如图4.7所示。图中阴影部分均为成功传输数据的时间。

\begin{figure}[h]
    \centering
    \includegraphics[width=0.8\textwidth]{image.png}
    \caption{问题二时间分析}
    \label{fig:4.7}
\end{figure}

由图 \ref{fig:4.7} 可计算信道吞吐量 $S$ 如式 (4.25) 所示。
\begin{equation}
S = \frac{[P(AP_1) + P(AP_2)] \cdot E[P]}{[P(\overline{AP_2}AP_1) + P(\overline{AP_1}AP_2) + P(AP_1AP_2)]T_s + (1 - P(AP_1) - P(AP_2) + P(AP_1AP_2))T_c} \cdot rate
\tag{4.25}
\end{equation}
\begin{equation}
= \frac{2\tau E[P]}{2\tau(1-p)T_s + \tau pT_s + (1-2\tau + \tau p)T_c} \cdot rate
\end{equation}
根据式 (4.25) 代入参数可计算得 $S = 7.06 \times 10^7 \, \text{bps}$。

\subsubsection{仿真检验}

对于问题二场景下 CSMA/CA 机制的运行规律如图 \ref{fig:4.8} 所示,在该场景下,若两 AP 不同时回退至 0,则先至 0 的 AP 发送数据,另一 AP 暂停回退,等待下一轮数据发送。若两个 AP 同时回退至 0,虽然并发但 STA 都能够接收成功。所以该过程中两 AP 的回退窗口一直保持最小值,数据发送次数也在每轮置 0。

\begin{figure}[h]
    \centering
    \includegraphics[width=\textwidth]{image2.png}
    \caption{问题二的情形}
    \label{fig:4.8}
\end{figure}

根据该场景的 CSMA/CA 机制的数据传输规律,可以将其进行离散事件仿真建模,伪代码如算法 2 所示。第 1 至 3 行为参数初始化,第 4 行开始进入事件仿真循环,第 5 至 9 行为两 AP 不同时回退到 0 的情形,在该情形下回退到 0 的 AP 成功传输次数增加一次,第 10 至第 13 行为两 AP 同时回退至 0 并传输成功的情况,此时两 AP 成功传输次数均增加一次,第 15 行根据相关参数计算信道吞吐量。

\textbf{算法2:问题二离散事件仿真伪代码}

\begin{tabular}{l l}
\hline
\textbf{输入:} & 单个空闲时隙长度 $T_{e}$;成功传输时长 $T_{s}$;DCF 帧间距 $DIFS$;仿真循环轮数 $P$; \\
 & 有效载荷传输时长 $EP$;物理层速率 $rate$; \\
\hline
\textbf{中间变量:} & 离散事件仿真总时长 $t$;离散事件仿真阶段数 $p \in P$;AP${}_{a}$ 回退次数 $CW_{a}$ \\
 & ($a \in \{1, 2\}$);AP${}_{a}$ 成功传输次数 $c_{a}$ ($a \in \{1, 2\}$) \\
\hline
\textbf{输出:} & 吞吐量 $S$ \\
\hline
\end{tabular}

\begin{enumerate}
\item \textbf{Begin:}
\item $t = DIFS$
\item $c_{a} = 0, \forall a \in \{1, 2\}$
\item \textbf{For} $p = 1$ \textbf{To} $P$:
\item \quad \textbf{If} $CW_{a} < CW_{b} \because, \forall (a, b) \in \{(1, 2), (2, 1)\}$ \quad // AP${}_{a}$ 先回退至 0
\item \quad \quad $t = t + (CW_{a} + 1) \times T_{e} + T_{s}$
\item \quad \quad $CW_{a} = random[0, 2^{0} \times CW_{\text{min}} - 1]$ \quad // 更新 $CW_{a}$
\item \quad \quad $CW_{b} = CW_{b} - CW_{a} - 1$ \quad // AP${}_{b}$ 剩余回退次数
\item \quad \quad $c_{a} = c_{a} + 1$
\item \quad \textbf{Else}: \quad // 两 AP 同时回退至 0,但不冲突
\item \quad \quad $t = t + (CW_{a} + 1) \times T_{e} + T_{c}$
\item \quad \quad $CW_{a} = random[0, 2^{0} \times CW_{\text{min}} - 1], \forall a \in \{1, 2\}$ \quad // 更新 $CW_{a}$
\item \quad \quad $c_{a} = c_{a} + 1$
\item \textbf{End For}
\item \textbf{Return} $S = (c_{a} + c_{b}) \times EP \times rate / t$
\item \textbf{End Begin}
\end{enumerate}

\textbf{问题二实验结果如表 4.2 所示,统计计算结果如图 4.9 所示,均值为 $\overline{S} = 6.72 \times 10^{7} \text{bps}$,标准差为 $\sigma = 3.8 \times 10^{4}$。均值 $\overline{S}$ 与 Bianchi 模型求解结果相差 4.8\%,证明了 Bianchi 模型的准确性。仿真结果概率图如图 4.10 所示,使用 Minitab 计算出 $P$ 值为 0.567 大于 0.05,实验结果服从正态分布,仿真结果具有稳定性。}

\textbf{表 4.2 问题二仿真器实验结果}

\begin{tabular}{c c c c c c c}
\hline
\textbf{序号} & \textbf{实验结果} & \textbf{序号} & \textbf{实验结果} & \textbf{序号} & \textbf{实验结果} & \textbf{序号} & \textbf{实验结果} \\
\hline
1 & 67234409 & 11 & 67194772 & 21 & 67209577 & 31 & 67221801 \\
2 & 67182053 & 12 & 67231289 & 22 & 67227662 & 32 & 67258448 \\
3 & 67187967 & 13 & 67223897 & 23 & 67244503 & 33 & 67271876 \\
4 & 67286002 & 14 & 67254716 & 24 & 67232115 & 34 & 67240098 \\
5 & 67259242 & 15 & 67253674 & 25 & 67211106 & 35 & 67159159 \\
6 & 67219516 & 16 & 67196454 & 26 & 67158727 & 36 & 67219064 \\
7 & 67222347 & 17 & 67164451 & 27 & 67111946 & 37 & 67259271 \\
8 & 67199432 & 18 & 67218354 & 28 & 67260686 & 38 & 67206443 \\
9 & 67247405 & 19 & 67272018 & 29 & 67274035 & 39 & 67237442 \\
10 & 67193636 & 20 & 67213725 & 30 & 67172417 & 40 & 67164319 \\
\hline
\end{tabular}

\textbf{单位:bps}

\begin{figure}[h]
    \centering
    \includegraphics[width=\textwidth]{image1.png}
    \caption{95\%置信区间}
    \label{fig:1}
\end{figure}

\begin{tabular}{c c c c c c c}
\hline
\textbf{序号} & \textbf{实验结果} & \textbf{序号} & \textbf{实验结果} & \textbf{序号} & \textbf{实验结果} & \textbf{序号} & \textbf{实验结果} \\
\hline
1 & 67234409 & 11 & 67194772 & 21 & 67209577 & 31 & 67221801 \\
2 & 67182053 & 12 & 67231289 & 22 & 67227662 & 32 & 67258448 \\
3 & 67187967 & 13 & 67223897 & 23 & 67244503 & 33 & 67271876 \\
4 & 67286002 & 14 & 67254716 & 24 & 67232115 & 34 & 67240098 \\
5 & 67259242 & 15 & 67253674 & 25 & 67211106 & 35 & 67159159 \\
6 & 67219516 & 16 & 67196454 & 26 & 67158727 & 36 & 67219064 \\
7 & 67222347 & 17 & 67164451 & 27 & 67111946 & 37 & 67259271 \\
8 & 67199432 & 18 & 67218354 & 28 & 67260686 & 38 & 67206443 \\
9 & 67247405 & 19 & 67272018 & 29 & 67274035 & 39 & 67237442 \\
10 & 67193636 & 20 & 67213725 & 30 & 67172417 & 40 & 67164319 \\
\hline
\end{tabular}

\begin{figure}[h]
    \centering
    \includegraphics[width=\textwidth]{image2.png}
    \caption{问题二仿真结果概率图}
    \label{fig:2}
\end{figure}

\begin{tabular}{l l}
\hline
\multicolumn{2}{l}{最大退避阶数 $m$;最大重传次数 $r$;单个空闲时隙长度 $T_{e}$;(PHY hdr+MAC} \\
\multicolumn{2}{l}{hdr+Payload)传输时长 $PMP$;DCF 帧间距 $DIFS$;短帧间距 $SIFS$;仿真循环} \\
\multicolumn{2}{l}{轮数 $P$;确认帧间距 $ACK$;等待超时时间 $ACKTimeout$;物理层速率 $rate$} \\
\hline
输入: & 回退开始时间点 $oldt_{a}$($a \in \{1, 2\}$);回退结束时间点 $tempt_{a}$($a \in \{1, 2\}$);传输 \\
 & 成功或冲突失败后时间点 $newt_{a}$($a \in \{1, 2\}$);离散事件仿真阶段数 $p \in P$;$AP_{a}$ \\
中间变量: & 数据的发送次数 $i_{a}$($a \in \{1, 2\}$);$AP_{a}$ 退避翻倍阶数 $j_{a}$($a \in \{1, 2\}$);$AP_{a}$ 回退 \\
 & 次数 $CW_{a}$($a \in \{1, 2\}$);$AP_{a}$ 成功传输次数 $s_{a}$($a \in \{1, 2\}$) \\
\hline
输出: & 信道吞吐量 $S$ \\
\hline
\end{tabular}

\section{问题三的分析与求解}

\subsection{问题分析}

问题三考虑隐藏节点问题,由于两AP间RSSI为$-90$dBm低于CCA门限$-82$dBm,不互听。当AP发包时间$H+E[P]$存在交叠时,SIR较小,发包均失败。问题三与问题一、二的区别在于两AP不互听,此时DIFS判断信道空闲,不存在hold time,当一个AP数据

传输时另一 AP 可以回退,使发包时间并非总是同时开始,因此会产生时间交叠。同时本问题还考虑了丢包的概率。

发包时间的交叠关系需要在连续时间轴上分析交叠发生时的各种情况。在本问题中连续的时间轴被分为 3 种状态,即回退时间 $T_{e}$,成功传输时间 $T_{s}$ 和失败传输时间 $T_{c}$,当分析当一个 AP 在 $t$ 时刻发包时,另一 AP 在 $t$ 时刻处于何种状态会产生交叠时,需要对这 3 种状态中各种可能的情况进行分类讨论。由于传统 Bianchi 模型的 Markov chain 仅保留了回退时间 $T_{e}$ 的状态,对成功传输时间 $T_{s}$ 和失败传输时间 $T_{c}$ 进行了简化,无法得到这两种状态的稳态概率,因此需要对 Bianchi 模型进行改进,加入成功传输状态和失败传输状态。在 Markov chain 求解获得各状态稳态概率,并根据交叠时的状态分析求解交叠条件概率 $p$ 和时隙传输数据帧概率 $\tau$ 后,需要对该系统的时间组成进行分析,并最终得到吞吐量模型。

\subsection{问题求解}

\subsubsection{数值求解}

问题三考虑隐藏节点发包交叠问题,由于信号包的交叠发生在信号传输过程中,因此需重点研究两节点信号在传输状态的先后关系,但传统的 Bianchi 模型中简化了节点传输信号的状态,不适用于本问题。因此本问题的求解在传统 Bianchi 模型的基础上,在 Markov chain 中加入了信号包成功传输状态 Trans 和信号包失败传输状态 Failed,改进的 Bianchi 模型如图 5.1 所示。

\begin{figure}[h]
    \centering
    \includegraphics[width=\textwidth]{image.png}
    \caption{改进的 Bianchi 模型}
    \label{fig:improved_bianchi_model}
\end{figure}

由题定义可知:

\begin{equation}
W_{i} =
\begin{cases}
2^{i}W_{0}, & 0 \leq i \leq m \\
2^{m}W_{0}, & m \leq i \leq r
\end{cases}
\tag{5.1}
\end{equation}

图 \ref{fig:improved_bianchi_model} 所示 Markov chain 状态转移概率如式 (5.2) 至式 (5.6) 所示

\begin{equation}
P\{i,k|i,k+1\} = 1, k \in [0, W_{i}-2], i \in [0, r]
\tag{5.2}
\end{equation}

\begin{equation}
P\{i,k|Failed_{i-1}\} = 1/W_{i}, k \in [0, W_{i}-1], i \in [0, r]
\tag{5.3}
\end{equation}

\begin{equation}
P\{0,k|Trans\} = 1/W_{0}, k \in [0, W_{0}-1], i \in [0, r]
\tag{5.4}
\end{equation}

\begin{equation}
P\{Trans|i,0\} = (1-p)(1-p_{e}), i \in [0, r]
\tag{5.5}
\end{equation}

\begin{equation}
P\{Failed_{i}|i,0\} = p + (1-p)p_{e}, i \in [0, r]
\tag{5.6}
\end{equation}

令 $b_{i,k} = \lim_{t \to \infty} P\{s(t) = i, b(t) = k\}, i \in (0, m), k \in (0, W_{i}-1)$ 为各状态稳态概率,$P(Trans)$ 为传

输成功的稳态概率, $P\{Failed\}$ 为传输失败的稳态概率,为方便求解,令 $p_{act}$ 表示考虑交叠和丢包后的实际传输失败率,如式(5.7)所示:

\begin{equation}
p_{act} = p + (1-p)p_e
\tag{5.7}
\end{equation}

根据 Markov chain 状态转移概率,可得:

\begin{equation}
b_{i,0} = b_{i-1,0}[W_i \cdot \frac{p_{act}}{W_i}] = b_{i-1,0}p_{act} = p_{act}^i b_{0,0}, i \in [0, r]
\tag{5.8}
\end{equation}

\begin{equation}
b_{i,k} =
\begin{cases}
b_{i-1,0} \cdot p_{act} \cdot \frac{W_i - k}{W_i}, & 0 < i < r \\
\frac{W_i - k}{W_i} \cdot \left[(1 - p_{act}) \cdot \sum_{j=0}^{r-1} b_{j,0} + b_{r,0}\right], & i = 0
\end{cases}
\tag{5.9}
\end{equation}

由式(5.8)和式(5.9)可得 $b_{i,k}$ 与 $b_{i,0}$ 的传递关系,如式(5.10)所示。

\begin{equation}
b_{i,k} = \frac{W_i - k}{W_i} b_{i,0}, i \in [0, r], k \in [0, W_i - 1]
\tag{5.10}
\end{equation}

计算传输成功和传输失败状态出现的概率之和如式(5.11)所示。

\begin{equation}
\sum_{i=0}^r p(Failed_i) + p(Trans) = p_{act} \sum_{i=0}^r b_{i,0} + (1 - p_{act}) \sum_{i=0}^r b_{i,0} = \sum_{i=0}^r b_{i,0}
\tag{5.11}
\end{equation}

由 Markov chain 的性质,所有状态的稳态概率和为 1,即:

\begin{align}
1 &= \sum_{i=0}^r \sum_{k=0}^{W_i-1} b_{i,k} + \sum_{i=0}^r p(Failed_i) + p(Trans) \\
&= \sum_{i=0}^r \sum_{k=0}^{W_i-1} \frac{W_i - k}{W_i} b_{i,0} + \sum_{i=0}^r b_{i,0} \\
&= \sum_{i=0}^r b_{i,0} \cdot \frac{W_i + 3}{2}
\tag{5.12}
\end{align}

节点在一个时隙发送数据帧的概率为:

\begin{equation}
\tau = \sum_{i=0}^r b_{i,0} = b_{0,0} \cdot \frac{1 - p_{act}^{r+1}}{1 - p_{act}}
\tag{5.13}
\end{equation}

求解发送失败的条件概率 $P(AP_2 | AP_1)$,可以首先计算两包不交叠的条件概率 $P(\overline{AP_2} | AP_1)$,以下分类讨论两包不交叠的各种状态:

状态 1:当 AP1 发包时,AP2 正处于传输状态且传输成功。状态 1 中各情况的交叠关系示意图如图 5.2 所示。

\begin{figure}[h]
    \centering
    \includegraphics[width=\textwidth]{image.png}
    \caption{AP1 发包时 AP2 处于成功传输状态各情形示意图}
    \label{fig:5.2}
\end{figure}

AP1 在 $t$ 时刻开始信号传输,情况 1 表示此时 AP2 恰好完成信号的有效载荷发送进入 SIFS 阶段,此时为 AP2 与 AP1 不发生交叠的极限情况。情况 2 为当 AP2 完成整个信号传输后随机到 0 个 TimeSlot,立即开始下一次信号传输,且 AP2 下一次信号传输开始的时刻恰好为 AP1 信号有效载荷传输完成的时刻,为不发生交叠的另一种极限情况。考虑两极限情况可知当 AP1 在 $t$ 时刻开始信号传输时,AP2 在 $t$ 时刻的状态位于数据成功传输时长 $T_s$ 的前 $H+E[P]$ 和后 $H+E[P]$ 之间时,即图 5.2 中斜阴影部分,传输一定不会交叠。在本问题中 $T_s=131.46\mu s$,$H+E[P]=40.46\mu s$,当 AP1 传输开始时,AP2 正处于 $T_s$ 的时间区间 $[40.46\mu s,91\mu s]$ 时,不会发生交叠。记该情形概率为 $p_1$,$p_1$ 定义如式 (5.14)。

\begin{equation}
p_1 = p\{Trans\} \cdot \frac{T_s - 2(H+E(p))}{T_s} = (1-p_{act}) \sum_{i=0}^{r} b_{i,0} \cdot \frac{T_s - 2(H+E(p))}{T_s}
\tag{5.14}
\end{equation}

\begin{equation}
= (1-p_{act}) \cdot \sum_{i=0}^{r} p_{act}^i b_{0,0} \cdot \frac{T_s - 2(H+E(p))}{T_s}
\end{equation}

对于 $t$ 时刻 AP2 正处于数据成功传输时长 $T_s$ 的后 $H+E[P]$ 的情况,是否交叠取决于 AP2 下一次发包随机得到 TimeSlot 的数量。例如情况 3 中,当第二次发包需要 3 个 TimeSlot 时不会交叠,若将其减少到 2 个,则会发生交叠。记 $H+E[P]$ 中可容纳 TimeSlot 的个数为 $n$($n$ 为连续实数),则:

\begin{equation}
n = \frac{H+E[P]}{T_e}
\tag{5.15}
\end{equation}

信息成功传输后,随机得到 TimeSlot 个数的概率均为 $1/W_0$,因此当 $t$ 处于数据成功传输时长 $T_s$ 的后 $H+E[P]$ 时,AP2 在 $t$ 时刻所指的位置越靠近 $T_s$ 末尾,则越可能发生交叠,因为在下一次传输前需要更多的 TimeSlot 填补 AP1 剩余有效载荷的传输时间,因此在后 $H+E[P]$ 段不发生交叠的概率密度函数线性下降,由 1 下降至 $(W_0-n)/W_0$,如图 5.3 所示。

\begin{figure}[h]
    \centering
    \includegraphics[width=\textwidth]{image.png}
    \caption{AP2 在 $t$ 时刻处于成功传输状态时各处的概率密度函数}
    \label{fig:5.3}
\end{figure}

设上述情况下不发生交叠的条件概率为 $p_2$,$p_2$ 定义如式 (5.16)。
\begin{equation}
p_2 = p\{Trans\} \cdot \frac{H + E[P]}{T_s} \cdot \frac{2W_0 - n}{2W_0} = (1 - p_{act}) \sum_{i=0}^{r} p_{act}^i b_{0,0} \cdot \frac{H + E[P]}{T_s} \cdot \frac{2W_0 - n}{2W_0}
\tag{5.16}
\end{equation}

状态 2:当 AP1 发包时,AP2 正处于传输状态且传输失败。状态 2 与状态 1 类似,区别在于此时 AP2 传输失败,消耗时长为 $T_c$,如图 5.4 所示。

\begin{figure}[h]
    \centering
    \includegraphics[width=\textwidth]{image2.png}
    \caption{AP1 发包时 AP2 处于失败传输状态各情形示意图}
    \label{fig:5.4}
\end{figure}

当 AP2 在 $t$ 时刻的状态位于数据失败传输时长 $T_c$ 的前 $H + E[P]$ 和后 $H + E[P]$ 之间时,传输一定不会交叠,设此时概率为 $p_3$,$p_3$ 定义如式 (5.17) 所示。
\begin{equation}
p_3 = p\{Failed\} \cdot \frac{T_c - 2(H + E(p))}{T_c} = p_{act} \sum_{i=0}^{r} b_{i,0} \cdot \frac{T_c - 2(H + E(p))}{T_c}
\tag{5.17}
\end{equation}
\begin{equation}
= p_{act} \cdot \sum_{i=0}^{r} p_{act}^i b_{0,0} \cdot \frac{T_c - 2(H + E(p))}{T_c}
\end{equation}

对于 $t$ 时刻 AP2 正处于数据失败传输时长 $T_c$ 的后 $H + E[P]$ 的情况,不发生交叠的条件概率密度函数为线性递减,设此时的概率为 $p_4$,$p_4$ 定义如式 (5.18) 所示。
\begin{equation}
p_4 = p\{Failed\} \cdot \frac{H + E[P]}{T_c} \cdot \frac{2W_i - n}{2W_i} = p_{act} \sum_{i=0}^{r} p_{act}^i b_{0,0} \cdot \frac{H + E[P]}{T_c} \cdot \frac{2W_i - n}{2W_i}
\tag{5.18}
\end{equation}

状态 3:当 AP1 发包时,AP2 正处于 TimeSlot 回退状态。状态 3 各情况如图 5.5 所示。

\begin{figure}[h]
    \centering
    \includegraphics[width=\textwidth]{image.png}
    \caption{AP1 发包时 AP2 处于 TimeSlot 状态各情形示意图}
    \label{fig:5.5}
\end{figure}

当 AP1 在 \( t \) 发包时,若 AP2 处于 TimeSlot 状态,由于 TimeSlot 逐个回退,因此当 \( t \) 时刻 AP2 的 TimeSlot 序号大于 \( H + E[P] \) 可容纳的 TimeSlot 数 \( n \) 时,一定不会发生交叠。如图 \ref{fig:5.5} 情况 1 所示。若 AP1 完成有效载荷传输后,AP2 恰好开始传输,此时 AP2 在 \( t \) 时刻所处的 TimeSlot 为极限情况,如情况 2 所示。记此状态的不交叠条件概率为 \( p_5 \),则有:
\begin{equation}
p_5 = \sum_{i=0}^{r} \sum_{k=0}^{W_i-1} \gamma_k \cdot b_{i,k} = \sum_{i=0}^{r} \sum_{k=0}^{W_i-1} \gamma_k \cdot \frac{W_i - k}{W_i} \cdot b_{i,0} = \sum_{i=0}^{r} \sum_{k=0}^{W_i-1} \gamma_k \cdot \frac{W_i - k}{W_i} \cdot p_{act}^i \cdot b_{0,0}
\tag{5.19}
\end{equation}

式中 \( \gamma_k \) 为 \( t \) 时刻 AP2 指向 \( b_{i,k} \) 时不会发生交叠的概率,\( \gamma_k \) 定义如式 (5.20) 所示,式中 \( \{ \} \) 表示取小数部分,\( \lfloor \rfloor \) 表示向下取整。
\begin{equation}
\gamma_k =
\begin{cases}
1 & , n < k \leq W_i - 1 \\
1 - \{ n \} & , k = \lfloor n \rfloor \\
0 & , 0 \leq k < n
\end{cases}
\tag{5.20}
\end{equation}

当 \( n < k \leq W_i \) 时,表示 \( t \) 时刻剩余的 Timeslot 足够填满 AP1 中 \( H + E[P] \) 所需的时间,一定不会交叠,故 \( \gamma_k = 1 \)。当 \( 0 < k < n \) 时,剩余 TimeSlot 不足以填满 AP1 中 \( H + E[P] \) 所需的时间,一定交叠,故 \( \gamma_k = 0 \),当 \( k = \lfloor n \rfloor \) 时,有 \( \gamma_k = 1 - \{ n \} \) 的概率不会交叠,如图 \ref{fig:5.5} 中情况 2 编号为 4 的 Timeslot 所示。

综上所述,可得两 AP 发包交叠的条件概率 \( p \) 为:
\begin{equation}
P(AP_1 | AP_2) = P(AP_2 | AP_1) = p = 1 - p_1 - p_2 - p_3 - p_4 - p_5
\tag{5.21}
\end{equation}

联立式 (5.7)、式 (5.12)、式 (5.13) 和式 (5.21) 可求得实际传输失败概率 \( p_{act} \)、一个节点在某时隙发送数据帧的概率 \( \tau \) 和两节点发包交叠条件概率 \( p \)。

考虑隐藏节点问题时,两 AP 之间不存在 hold time,当 AP2 传输时,AP1 可进行指数回退,此时两 AP 系统中时间概率占比示意图如图 5.6 所示。图分为上下两层,上层为 AP1 视角的时间轴,下层为 AP2 视角的时间轴,两层时间在同一垂直方向上的点为某一时刻 AP1 和 AP2 所处的状态。

\begin{figure}[h]
    \centering
    \includegraphics[width=0.8\textwidth]{image.png}
    \caption{问题三各部分时间占比分析}
    \label{fig:5.6}
\end{figure}

根据图 \ref{fig:5.6},在 AP1 所在时间轴上计算系统总时间,可得信道吞吐量 $S$ 定义如式 (5.22) 所示。

\begin{equation}
S = \frac{\left[ P(AP_1) - P(AP_1 \overline{AP_2}) - P(AP_1)p_e \right] + \left[ P(AP_2) - P(AP_2 \overline{AP_1}) - P(AP_2)p_e \right] \cdot E[p]}{(1 - P(AP_1))T_e + P(AP_1)(1 - P(AP_2|AP_1) - p_e)T_s + P(AP_1)(p_e + P(AP_1|AP_2))T_c} \cdot rate
\tag{5.22}
\end{equation}

\begin{equation}
= \frac{2\tau(1 - p - p_e + pp_e)E[P]}{(1 - \tau)T_e + \tau(1 - p - p_e + pp_e)T_s + \tau(p_e + p - pp_e)T_c} \cdot rate
\end{equation}

将参数代入式 (5.22),在此条件下,利用 Markov chain 求得发包交叠条件概率 $p = 26.19\%$,任意时隙发送数据帧概率 $\tau = 5.91\%$,信道吞吐量 $S = 5.45 \times 10^7 \, \text{bps}$。

\subsubsection{仿真检验}

对于问题三,根据 $AP_a$ 和 $AP_b$,$\forall (a, b) \in \{(1, 2), (2, 1)\}$ 是否同时回退至 0 将离散事件仿真模型划分为两种可能出现的情形,再根据 $AP_a$ 和 $AP_b$ 回退次数差异大小将情形 1 划分为情形 1-1 和情形 1-2。如图 5.7 所示,

在情形 1 中,令事件阶段为 $p$,$AP_a$ 优先回退至 0,由于两 BSS 不互听,$AP_b$ 继续回退。情形 1 可细分为情形 1-1 和情形 1-2 两种。

(a) 在情形 1-1 中,由于 $AP_a$ 传输完数据后 $AP_b$ 仍在回退过程中,$AP_a$ 和 $AP_b$ 不发生交叠,因此 $AP_a$ 成功传输,$AP_b$ 可能与下一事件阶段 $p+1$ 的 $AP_a$ 发生交叠冲突,也可能成功传输,但均属于下一阶段需要考虑的问题,所以令 $newt_b = oldt_b$,即重置 $AP_b$ 开始时间点,将下一事件阶段的 $AP_a$ 与其进行比较,并重复此过程,直至 $AP_b$ 进入下一事件阶段。

(b) 在情形 1-2 中,$AP_a$ 数据未传输完成,$AP_b$ 回退至 0,两者均同属一个事件阶段 $p$,由于 SIR 比较小,$AP_a$ 和 $AP_b$ 发生交叠冲突,导致两者均传输失败。

(c) 在情形 2 中,$AP_a$ 和 $AP_b$ 同时回退至 0,两者均传输失败。

\begin{figure}[h]
    \centering
    \includegraphics[width=\textwidth]{image_a.png}
    \caption{(a) 情形 1-1}
\end{figure}

\begin{figure}[h]
    \centering
    \includegraphics[width=\textwidth]{image_b.png}
    \caption{(b) 情形 1-2}
\end{figure}

\begin{figure}[h]
    \centering
    \includegraphics[width=\textwidth]{image_c.png}
    \caption{(c) 情形 2}
\end{figure}

图 5.7 问题三情形示意图 (a) 情形 1-1 (b) 情形 1-2 (c) 情形 2

问题三伪代码如算法 3 所示。第 5 行开始进入离散事件循环,其第 14 行至第 22 行对应上文情形 1-1,第 23 行至第 26 行对应情形 1-2,第 27 行至第 30 行对应情形 2,第 31 行对信道吞吐量计算结果进行输出。

\textbf{算法 3:问题三离散事件仿真伪代码}

\begin{tabular}{l l}
\hline
\multicolumn{2}{l}{最大退避阶数 $m$;最大重传次数 $r$;单个空闲时隙长度 $T_{e}$;(PHY hdr+MAC} \\
\multicolumn{2}{l}{hdr+Payload)传输时长 $PMP$;DCF 帧间距 $DIFS$;短帧间距 $SIFS$;仿真循环} \\
\multicolumn{2}{l}{轮数 $P$;确认帧间距 $ACK$;等待超时时间 $ACKTimeout$;物理层速率 $rate$} \\
\hline
输入: & 回退开始时间点 $oldt_{a}$($a \in \{1, 2\}$);回退结束时间点 $tempt_{a}$($a \in \{1, 2\}$);传输 \\
 & 成功或冲突失败后时间点 $newt_{a}$($a \in \{1, 2\}$);离散事件仿真阶段数 $p \in P$;$AP_{a}$ \\
中间变量: & 数据的发送次数 $i_{a}$($a \in \{1, 2\}$);$AP_{a}$ 退避翻倍阶数 $j_{a}$($a \in \{1, 2\}$);$AP_{a}$ 回退 \\
 & 次数 $CW_{a}$($a \in \{1, 2\}$);$AP_{a}$ 成功传输次数 $s_{a}$($a \in \{1, 2\}$) \\
\hline
输出: & 信道吞吐量 $S$ \\
\hline
\end{tabular}

\begin{algorithmic}
\STATE Begin:
\STATE $oldt_{a} = DIFS, \forall a \in \{1, 2\}$
\end{algorithmic}

\begin{verbatim}
03 $i_a = 0, \ \forall a \in \{1, 2\}$
04 $s_a = 0, \ \forall a \in \{1, 2\}$

For p = 1 To P:
06 If $i_a \leq m: j_a = i_a, \forall a \in \{1, 2\}$
07 Elif $m < i_a < r: j_a = m, \forall a \in \{1, 2\}$ //超过最大退避阶数
08 Elif $i_a = r$: //达到最大重传次数
09 $j_a = 0, \forall a \in \{1, 2\}$
10 $i_a = 0, \forall a \in \{1, 2\}$
11 Else: Pass

12 $tempt_a = oldt_a + CW_a \times T_e, \forall a \in \{1, 2\}$

13 If $tempt_a < tempt_b: \forall (a, b) \in \{(1, 2), (2, 1)\}$ //AP_a 先回退至 0
14 If $newt_a < tempt_b$: //AP_b 回退次数过多,在当前事件阶段不发生交叠
15 $newt_b = oldt_b$ //重置为下一个事件回退开始时间点
16 If 丢包: $newt_a = tempt_a + PMP + ACKTimeout + DIFS$ //丢包率 10%
17 Else:
18 $newt_a = tempt_a + PMP + SIFS + ACK + DIFS$
19 $s_a = s_a + 1$
20 $CW_a = random[0, 2^{j_a} \times CW_{\text{min}} - 1]$ //AP_a 成功传输,更新 $CW_a$
21 $CW_b = CW_b - CW_a - 1$ //AP_b 剩余回退次数
22 $i_a = 0$ //重置 AP_a 数据发送次数
23 Else: //两 AP 交叠冲突
24 $newt_a = tempt_a + PMP + ACKTimeout + DIFS, \ \forall a \in \{1, 2\}$
25 $CW_a = random[0, 2^{j_a} \times CW_{\text{min}} - 1], \forall a \in \{1, 2\}$ //更新两 AP 的 $CW_a$
26 $i_a = i_a + 1, \forall a \in \{1, 2\}$ //更新两 AP 数据发送次数

27 Else: //两 AP 同时回退至 0,冲突
28 $newt_a = tempt_a + PMP + ACKTimeout + DIFS, \forall a \in \{1, 2\}$
29 $CW_a = random[0, 2^{j_a} \times CW_{\text{min}} - 1], \forall a \in \{1, 2\}$ //更新两 AP 的 $CW_a$
30 $i_a = i_a + 1, \forall a \in \{1, 2\}$ //更新两 AP 数据发送次数

31 Return $S = 2 \times s_a \times EP \times rate / newt_a$ //输出该场景信道吞吐量
32 End For
33 End Begin
\end{verbatim}

问题三实验结果如表 5.1 所示,统计计算结果如图 5.8 所示,均值为 $\overline{S} = 5.55 \times 10^7 \text{dps}$,标准差为 $\sigma = 9.9 \times 10^4$。与改进的 Bianchi 模型求解结果相差 1.8%,证明了改进的 Bianchi 模型的准确性。仿真结果概率图如图 5.9 所示,使用 Minitab 计算出 P 值为 0.108 大于 0.05,实验结果服从正态分布,仿真结果具有稳定性。

\begin{table}
\centering
\begin{tabular}{c c c c c c c}
序号 & 实验结果 & 序号 & 实验结果 & 序号 & 实验结果 & 序号 \\
\hline
1 & 55480590 & 11 & 55413434 & 21 & 55653072 & 31 \\
2 & 55409212 & 12 & 55517155 & 22 & 55744441 & 32 \\
3 & 55513396 & 13 & 55546498 & 23 & 55577655 & 33 \\
4 & 55582593 & 14 & 55658627 & 24 & 55455481 & 34 \\
5 & 55404268 & 15 & 55751205 & 25 & 55489330 & 35 \\
6 & 55679467 & 16 & 55431322 & 26 & 55544545 & 36 \\
7 & 55474384 & 17 & 55528244 & 27 & 55421708 & 37 \\
8 & 55478699 & 18 & 55391090 & 28 & 55624511 & 38 \\
9 & 55547182 & 19 & 55490358 & 29 & 55525543 & 39 \\
10 & 55441709 & 20 & 55467536 & 30 & 55423052 & 40 \\
\end{tabular}
\caption{问题三仿真器实验结果}
\label{tab:5.1}
\end{table}

\begin{figure}[h]
\centering
\includegraphics[width=\textwidth]{image1.png}
\caption{问题三仿真结果正态性检验报告}
\label{fig:5.8}
\end{figure}

\begin{figure}[h]
\centering
\includegraphics[width=\textwidth]{image2.png}
\caption{问题三仿真结果概率图}
\label{fig:5.9}
\end{figure}

当系统取其他参数时,Bianchi 模型和仿真实验计算结果如表5.2 所示。当物理层速率
和最大重传次数较大时,两种方法误差在 3%以内,当物理层速率增加且最大重传次数降
低时,两方法的误差略有增加,但均在30%以内。系统取其他参数时的仿真结果正态性检
验概率图见附录A。

\begin{table}
\centering
\caption{其他参数下两方法结果对比}
\begin{tabular}{c c c c c c c}
\hline
序号 & $CW_{\text{min}}$ & $CW_{\text{max}}$ & $r$ & 物理层速率 & 数值计算 & 仿真实验 & Gap \\
\hline
1 & 16 & 1024 & 32 & 455.8Mbps & 44760000 & 43723195 & 2.31\% \\
2 & 16 & 1024 & 6 & 286.8Mbps & 41012000 & 32318324 & 21.19\% \\
3 & 32 & 1024 & 5 & 286.8Mbps & 44762000 & 43741428 & 2.28\% \\
4 & 16 & 1024 & 32 & 286.8Mbps & 32515000 & 28382805 & 12.7\% \\
5 & 16 & 1024 & 6 & 158.4Mbps & 30787000 & 21562020 & 29.96\% \\
6 & 32 & 1024 & 5 & 158.4Mbps & 32678000 & 28361848 & 13.20\% \\
\hline
\end{tabular}
\end{table}

\section{问题四的分析与求解}

\subsection{问题分析}

本问题为问题一和问题三的结合,AP2与另外两个AP互听,AP1和AP3不互听,AP2的传输机会可能被AP1和AP3抢占,而AP1和AP3会先后或同时发送数据,造成数据交叠,但交叠时信息能成功传输。类似于问题三,当考虑交叠问题时需要在Bianchi模型中加入Trans状态和Failed状态。

问题一、二和三所讨论的两BSS问题中,两AP具有对称性,但在本问题中,由于AP2与另外两个AP的状态不同,因此需要单独对AP2进行Markov Chain分析,AP1和AP3具有对称性,可以同时进行Markov chain分析。最后将两个Markov chain得到的关系联立计算系统中的三个AP的稳态概率。

本问题同样需要对系统总时间的构成进行分析,并根据时间占比计算信道吞吐量。

\subsection{问题求解}

\subsubsection{数值求解}

根据问题描述,绘制改进Bianchi模型的Markov chain如图6.1所示。

\begin{figure}[h]
    \centering
    \includegraphics[width=\textwidth]{markov_chain_diagram.png}
    \caption{问题四 Markov chain}
    \label{fig:markov_chain}
\end{figure}

图 \ref{fig:markov_chain} 所示 Markov chain 状态转移概率通式如式 (6.1) 至式 (6.5) 所示。其中 $p_{act}$ 表示节点的实际传输失败率。

\begin{equation}
P\{i,k|i,k+1\}=1, k\in[0,W_{i}-2], i\in[0,r] \tag{6.1}
\end{equation}

\begin{equation}
P\{i,k|Failed_{i-1}\}=1/W_{i}, k\in[0,W_{i}-1], i\in[0,r] \tag{6.2}
\end{equation}

\begin{equation}
P\{0,k|Trans\}=1/W_{0}, k\in[0,W_{0}-1], i\in[0,r] \tag{6.3}
\end{equation}

\begin{equation}
P\{Trans|i,0\}=1-p_{act}, i\in[0,r] \tag{6.4}
\end{equation}

\begin{equation}
P\{Failed_{i}|i,0\}=p_{act}, i\in[0,r] \tag{6.5}
\end{equation}

记 $b_{i,k,1}$ 为 AP1 和 AP3 的 Markov chain 中的状态稳态概率,$b_{i,k,2}$ 为 AP2 的 Markov chain 中的状态稳态概率。Markov chain 各状态概率之和为 1,如式 (6.6) 所示。

\begin{equation}
\begin{aligned}
1 &= \sum_{i=0}^{r} \sum_{k=0}^{W_i-1} b_{i,k,m} + \sum_{i=0}^{r} p(Failed_i) + p(Trans) \\
&= \sum_{i=0}^{r} \sum_{k=0}^{W_i-1} \frac{W_i-k}{W_i} b_{i,0,m} + \sum_{i=0}^{r} b_{i,0,m} \\
&= \sum_{i=0}^{r} b_{i,0,m} \cdot \frac{W_i+3}{2}, \quad m \in \{1,2\}
\end{aligned}
\tag{6.6}
\end{equation}

记 $p_{act1}$ 表示 AP1 和 AP3 的实际失败传输条件概率,$p_{act2}$ 表示 AP2 实际失败传输条件概率。对于不互听的 AP1 和 AP3,由问题三可知,其交叠条件概率 $p_{overlap}$ 可以由式(6.7)至式(6.12)得到。

\begin{equation}
\begin{aligned}
p_1 &= p\{Trans\} \cdot \frac{T_s - 2(H + E(p))}{T_s} = (1 - p_{act1}) \sum_{i=0}^{r} b_{i,0} \cdot \frac{T_s - 2(H + E(p))}{T_s} \\
&= (1 - p_{act1}) \cdot \sum_{i=0}^{r} p_{act1}^i b_{0,0} \cdot \frac{T_s - 2(H + E(p))}{T_s}
\end{aligned}
\tag{6.7}
\end{equation}

\begin{equation}
\begin{aligned}
p_2 &= p\{Trans\} \cdot \frac{H + E[P]}{T_s} \cdot \frac{2W_0 - n}{2W_0} = (1 - p_{act1}) \sum_{i=0}^{r} p_{act1}^i b_{0,0} \cdot \frac{H + E[P]}{T_s} \cdot \frac{2W_0 - n}{2W_0}
\end{aligned}
\tag{6.8}
\end{equation}

\begin{equation}
\begin{aligned}
p_3 &= p\{Failed\} \cdot \frac{T_c - 2(H + E(p))}{T_c} = p_{act1} \sum_{i=0}^{r} b_{i,0} \cdot \frac{T_c - 2(H + E(p))}{T_c} \\
&= p_{act1} \cdot \sum_{i=0}^{r} p_{act1}^i b_{0,0} \cdot \frac{T_c - 2(H + E(p))}{T_c}
\end{aligned}
\tag{6.9}
\end{equation}

\begin{equation}
\begin{aligned}
p_4 &= p\{Failed\} \cdot \frac{H + E[P]}{T_c} \cdot \frac{2W_i - n}{2W_i} = p_{act1} \sum_{i=0}^{r} p_{act1}^i b_{0,0} \cdot \frac{H + E[P]}{T_c} \cdot \frac{2W_i - n}{2W_i}
\end{aligned}
\tag{6.10}
\end{equation}

\begin{equation}
\begin{aligned}
p_5 &= \sum_{i=0}^{r} \sum_{k=0}^{W_i-1} \gamma_k \cdot b_{i,k} = \sum_{i=0}^{r} \sum_{k=0}^{W_i-1} \gamma_k \cdot \frac{W_i-k}{W_i} \cdot b_{i,0} = \sum_{i=0}^{r} \sum_{k=0}^{W_i-1} \gamma_k \cdot \frac{W_i-k}{W_i} \cdot p_{act1}^i \cdot b_{0,0}
\end{aligned}
\tag{6.11}
\end{equation}

\begin{equation}
p_{overlap} = 1 - p_1 - p_2 - p_3 - p_4 - p_5
\tag{6.12}
\end{equation}

节点在一个时隙发送数据帧的概率为:

\begin{equation}
P(AP_1) = P(AP_3) = \tau_1 = \sum_{i=0}^{r} b_{i,0,1} = b_{0,0,1} \cdot \frac{1 - p_{act1}^{r+1}}{1 - p_{act1}}
\tag{6.13}
\end{equation}

\begin{equation}
P(AP_2) = \tau_2 = \sum_{i=0}^{r} b_{i,0,2} = b_{0,0,2} \cdot \frac{1 - p_{act2}^{r+1}}{1 - p_{act2}}
\tag{6.14}
\end{equation}

AP1 和 AP3 实际发送失败条件概率 $p_{act1}$ 如式(6.15)所示。

\begin{equation}
p_{act1} = 1 - (1 - \tau_2)
\tag{6.15}
\end{equation}

对 AP2,当 AP1、AP3 单独传输数据和交叠传输数据时,都会进入 hold time,因此其实际发送失败条件概率 $p_{act2}$ 如式(6.16)所示。

\begin{equation}
\begin{aligned}
P_{act2} &= P(AP_1 + AP_2) = P(AP_1) + P(AP_2) - P(AP_1 AP_2) \\
&= 2\tau_1 - \tau_1 p_{overlap}
\end{aligned}
\tag{6.16}
\end{equation}

联立式(6.6)、式(6.12)、式(6.13)、式(6.14)、式(6.15)和式(6.16)共 7 个方程,可分别解得 $b_{0,0,1}$、$b_{0,0,2}$、$\tau_1$、$\tau_2$、$p_{act1}$、$p_{act2}$ 和 $p_{overlap}$。

该 3BSS 系统中时间构成如图所示。由于 AP1 和 AP3 不互听,因此这两个 AP 可以在另一 AP 传输数据时正常回退。AP2 则必须在另外两 AP 传输数据时进入 hold time。

\begin{figure}[h]
\centering
\includegraphics[width=0.8\textwidth]{image.png}
\caption{3BSS 系统时间构成}
\end{figure}

根据图 6.2 中 AP1 和 AP2 所在时间轴计算吞吐量。其中冲突导致的传输失败仅发生在 AP2 与另外两 AP 并行传输过程中,因此发生冲突的期望为:

\begin{equation}
E(Tc) = \tau_2 p_{act2} \times T_c
\tag{6.17}
\end{equation}

成功传输的期望和等待期望如式(6.18)和式(6.19)所示。

\begin{equation}
E(T_s) = [\tau_1(1 - p_{act1}) + \tau_2(1 - p_{act2})] \cdot T_s
\tag{6.18}
\end{equation}

\begin{equation}
E(T_e) = [1 - \tau_2 - \tau_1 \cdot (1 - p_{act1})] T_e
\tag{6.19}
\end{equation}

信道吞吐量计算如式(6.20)所示。

\begin{equation}
S = \frac{[2\tau_1(1 - p_{act1}) + \tau_2(1 - p_{act2})] \cdot E[P]}{E(T_s) + E(T_c) + E(T_e)} \cdot rate
\tag{6.20}
\end{equation}

本题计算得 $\tau_1 = 9.65\%$,$\tau_2 = 8.89\%$,$p_{overlap} = 42.13\%$,$p_{act1} = 8.89\%$,$p_{act2} = 15.23\%$,代入式(6.20)可得吞吐量 $S = 9.76 \times 10^7 \, \text{dps}$。

\subsubsection{仿真检验}

对于问题四的场景,根据 AP$_a$、AP$_b$ 和 AP$_c$,$\forall (a, b) \in \{(1, 3), (3, 1)\}$ 谁先回退至 0 将离散事

件仿真模型划分为 5 种可能出现的情形,再根据 $A P_{a}$ 和 $A P_{b}$ 回退次数差异大小将情形 1 划分为情形 1-1 和情形 1-2,如图 6.3 所示。

(a) 在情形 1 中,令事件阶段为 $p$,$A P_{a}$ 优先回退至 0,由于 $A P_{a}$ 与 $A P_{b}$ 两 BSS 不互听,$A P_{b}$ 继续回退。在情形 1-1 中,由于 $A P_{a}$ 传输完数据后 $A P_{b}$ 仍在回退过程中,$A P_{a}$ 和 $A P_{b}$ 不发生交叠,$A P_{b}$ 可能与下一事件阶段 $p+1$ 的 $A P_{a}$ 发生交叠冲突,也可能成功传输,但均属于下一阶段需要考虑的问题,所以令 $n e w t_{b}=o l d t_{b}$,即重置 $A P_{b}$ 开始时间点,将下一事件阶段的 $A P_{a}$ 与其进行比较,并重复此过程,直至 $A P_{b}$ 进入下一事件阶段;由于 $A P_{a}$ 和 $A P_{b}$ 两者的 BSS 互听,当 $A P_{a}$ 回退至 0 时,$A P_{b}$ 尚未回退至 0,两者均同属一个事件阶段 $p$,因此 $A P_{a}$ 成功传输。当 $A P_{b}$ 回退完最后一个间隙 $T_{c}$ 后,进入等待阶段直至 $A P_{a}$ 传输完成。

(b) 在情形 1-2 中,$A P_{a}$ 数据未传输完成,$A P_{b}$ 回退至 0,两者均同属一个事件阶段 $p$,由于 SIR 较大,$A P_{a}$ 和 $A P_{b}$ 均传输成功;$A P_{a}$ 与 $A P_{b}$ 不互听,$A P_{b}$ 与两者都互听,当 $A P_{a}$ 开始传输时,$A P_{b}$ 尚未回退至 0,当 $A P_{b}$ 等待时,$A P_{b}$ 开始传输,$A P_{b}$ 持续等待直至 $A P_{a}$ 和 $A P_{b}$ 均传输完成。

(c) 在情形 2 中,$A P_{b}$ 优先回退至 0,由于 $A P_{a}$ 与 $A P_{b}$ 不互听,$A P_{b}$ 与两者都互听,因此 $A P_{b}$ 成功传输。当 $A P_{b}$ 开始传输时,$A P_{a}$ 与 $A P_{b}$ 各自回退完一个间隙 $T_{c}$ 后,均进入等待阶段直至 $A P_{b}$ 传输完成。

(d) 在情形 3 中,$A P_{a}$ 和 $A P_{b}$ 同时回退至 0,由于 $A P_{a}$ 和 $A P_{b}$ 两 BSS 互听,两者均传输失败;由于 $A P_{b}$ 和 $A P_{b}$ 互听,$A P_{b}$ 回退完最后一个间隙 $T_{c}$ 后,进入等待阶段直至冲突结束。

(e) 在情形 4 中,$A P_{a}$ 和 $A P_{b}$ 同时回退至 0,由于 $A P_{a}$ 与 $A P_{b}$ 两 BSS 不互听,两者均能成功传输;由于 $A P_{b}$ 与两者均互听,$A P_{b}$ 回退完最后一个间隙 $T_{c}$ 后,进入等待阶段直至传输完成。

(f) 在情形 5 中,$A P_{a}$、$A P_{b}$ 和 $A P_{b}$ 同时回退至 0,由于 $A P_{a}$ 与 $A P_{b}$ 不互听,$A P_{b}$ 与两者都互听,发生冲突,三者均传输失败。

\begin{figure}[h]
\centering
\includegraphics[width=\textwidth]{image.png}
\caption{(a)}
\end{figure}

\begin{figure}[h]
    \centering
    \includegraphics[width=\textwidth]{image.png}
    \caption{}
    \label{fig:example}
\end{figure}

\begin{figure}[h]
    \centering
    \includegraphics[width=\textwidth]{image.png}
    \caption{问题四情形示意图 (a) 情形 1-1 (b) 情形 1-2 (c) 情形 2 (d) 情形 3 (e) 情形 4 (f) 情形 5}
    \label{fig:6.3}
\end{figure}

问题四实验结果如表 6.1 所示,统计计算结果如图 6.4 所示,均值为 $\overline{S}=1.1 \times 10^{8} \mathrm{dps}$,标准差为 $\sigma=3.5 \times 10^{4}$。仿真结果与改进 Bianchi 模型计算得到的结果相差 $12.27\%$,证明了 Bianchi 模型的准确性。仿真结果概率图如图 6.5 所示,使用 Minitab 计算出 $P$ 值为 0.767 大于 0.05,实验结果服从正态分布。

\begin{table}[h]
    \centering
    \caption{问题四仿真器实验结果}
    \label{tab:6.1}
    \begin{tabular}{c c c c c c c c}
        \hline
        序号 & 实验结果 & 序号 & 实验结果 & 序号 & 实验结果 & 序号 & 实验结果 \\
        \hline
        1 & 110189047 & 11 & 110161106 & 21 & 110145573 & 31 & 110200219 \\
        2 & 110218369 & 12 & 110162998 & 22 & 110151403 & 32 & 110126427 \\
        \hline
    \end{tabular}
    \textbf{单位: dps}
\end{table}

\begin{table}
\centering
\begin{tabular}{c c c c c c c}
序号 & 实验结果 & 序号 & 实验结果 & 序号 & 实验结果 & 实验结果 \\
\hline
3 & 110163517 & 13 & 110161312 & 23 & 110127523 & 110147504 \\
4 & 110153522 & 14 & 110109577 & 24 & 110187915 & 110234720 \\
5 & 110183703 & 15 & 110140386 & 25 & 110130020 & 110166142 \\
6 & 110143319 & 16 & 110264378 & 26 & 110159257 & 110171757 \\
7 & 110191848 & 17 & 110194872 & 27 & 110079578 & 110174729 \\
8 & 110184929 & 18 & 110222367 & 28 & 110171122 & 110188984 \\
9 & 110146376 & 19 & 110200330 & 29 & 110189165 & 110205864 \\
10 & 110154648 & 20 & 110176632 & 30 & 110196391 & 110195468 \\
\end{tabular}
\caption*{续表6.1}
\end{table}

\begin{figure}[h]
\centering
\includegraphics[width=0.8\textwidth]{image1.png}
\caption{问题四仿真结果正态性检验报告}
\end{figure}

\begin{figure}[h]
\centering
\includegraphics[width=0.8\textwidth]{image2.png}
\caption{问题四仿真结果概率图}
\end{figure}

\begin{table}[h]
\centering
\begin{tabular}{l l}
Anderson-Darling 正态性检验 & \\
A平方 & 0.24 \\
P值 & 0.767 \\
\hline
均值 & 110171825 \\
标准差 & 34754 \\
方差 & 1207819281 \\
偏度 & 0.013171 \\
峰度 & 0.984064 \\
N & 40 \\
\hline
最小值 & 110079578 \\
第一四分位数 & 110148478 \\
中位数 & 110171440 \\
第三四分位数 & 110194116 \\
最大值 & 110264378 \\
\hline
95\%均值置信区间 & \\
110160710 & 110182940 \\
95\%中位数置信区间 & \\
110160017 & 110188545 \\
95\%标准差置信区间 & \\
28469 & 44625 \\
\end{tabular}
\end{table}

\begin{table}[h]
\centering
\begin{tabular}{l l}
均值 & 110171825 \\
标准差 & 34754 \\
N & 40 \\
AD & 0.238 \\
P值 & 0.767 \\
\end{tabular}
\end{table}

当系统取其他参数时,Bianchi 模型和仿真实验计算结果如表6.2 所示,数值计算和仿
真实验的误差控制在5%-15%之间,证明了改进Bianchi 模型的准确性。系统取其他参数时
的仿真结果正态性检验概率图见附录B。

\begin{table}
\centering
\caption{其他参数下两方法结果对比}
\begin{tabular}{c c c c c c c}
\hline
序号 & $CW_{\text{min}}$ & $CW_{\text{max}}$ & $r$ & 物理层速率 & 数值计算 & 仿真实验 \\
\hline
1 & 16 & 1024 & 32 & 455.8Mbps & 89729360 & 101909404 \\
2 & 16 & 1024 & 6 & 286.8Mbps & 77076783 & 81048402 \\
3 & 32 & 1024 & 5 & 286.8Mbps & 89729857 & 101916034 \\
4 & 16 & 1024 & 32 & 286.8Mbps & 76234106 & 87737732 \\
5 & 16 & 1024 & 6 & 158.4Mbps & 67038178 & 71161702 \\
6 & 32 & 1024 & 5 & 158.4Mbps & 76234207 & 87736668 \\
\hline
\end{tabular}
\end{table}

\section{模型评价}
\subsection{优点}
(1)  根据离散事件仿真思想对 DSMA/CA 机制在不同场景进行建模仿真模拟,依靠计
算机能够获得较为贴近实际的仿真过程,并得到较为可靠的信道吞吐量评估结果。 
(2)  对传统的Bianchi 模型进行了创新,增加了信号包成功传输状态Trans 和信号包传
输失败状态 Failed,该改进使得传统模型更加贴近于真实场景的机制,并与离散事件仿真
结果进行对比,在某些场景下能够比传统Bianchi 模型有更高的精确度。 
\subsection{7.2 不足 }
本文的数值分析方法局限于对 Bianchi 模型的研究和改进,该传统模型在理想的假设
下有简洁易用的优势,但当模型越发复杂时,建模思路变得复杂和困难,不同的参数设置
也会使其估算结果与仿真结果不存在稳定跟随的关系。采用除 Bianchi 模型以外的其他模
型框架进行数值分析,可能会得到更精确的解。 
\subsection{7.3 展望} 
本文对设定了一些理想情况下的CSMA/CA 机制的信道接入场景进行了建模和分析,
当理想假设减少后,建立的模型将会更加贴近于真实的网络通讯情景。对更加贴近现实的
模型进行建立,有助于加深对现实通讯技术的认识并提出对现有通讯质量技术进一步发展
的可行性方案,未来可研究包括多BSS 物理速率不同等网络通讯情景,并挖掘其内部机理。
WLAN 技术的进步会推动万物互联,虚拟现实等益于提高人类生活质量技术的进步。
\section{参考文献}

\begin{enumerate}
    \item Hanzo L, Münster M, Choi B J. OFDM and MC-CDMA for Broadband Multi-User Communications, WLANs and Broadcasting[J]. John Wiley \& Sons, Inc. 2003.
    \item Bianchi G. IEEE 802.11-Saturation Throughput Analysis [J]. IEEE Communications Letters, 1998, 2(12): 318-320.
    \item 杜进. 面向工业场景的 Wi-Fi 网络 AP 优化部署与信道优化分配研究[D]. 东南大学, 2022.
    \item 郁帅鑫. 智能家居场景下 WLAN 随机接入优化研究[D]. 华中科技大学, 2022.
    \item 彭林哲. 无线局域网 DCF 性能分析与公平性改进研究[D]. 湖南大学, 2017.
    \item Kleijnen J P C, Bettonvil B, Persson F. Finding the Important Factors in Large Discrete-Event Simulation: Sequential Bifurcation and its Applications[J]. Social Science Electronic Publishing, 2009, 2003-104(2003-104): 287-307.
    \item Jafer S, Liu Q, Wainer G. Synchronization methods in parallel and distributed discrete-event simulation[J]. Simulation Modelling Practice and Theory, 2013, 30: 54-73.
    \item 崔晓峰. 面向对象的离散事件仿真建模与实现[J]. 计算机工程与应用, 1998(09): 40-41+47.
    \item 刘勇, 王德才, 冯正超. 离散事件系统仿真建模与仿真策略[J]. 西南师范大学学报(自然科学版), 2005(06): 1019-1025.
\end{enumerate}

\section{附录}
# 附录A 问题三其他参数下仿真结果正态性检验概率图
实验组号
min
# CW

max
# CW

r
物理层速率
1
16
1024
32
1. 8Mbps
2
16
1024
6
1. 8Mbps
3
32
1024
5
1. 8Mbps
4
16
1024
32
1. 8Mbps
5
16
1024
6
1. 4Mbps
6
32
1024
5
1. 4Mbps



38
>

39

## 附录B 问题四其他参数下仿真结果正态性检验概率图
>

40
>

41

# 附录C 代码
问题一数值求解Matlab 代码:
标准Bianchi:
function F = main1(x)
w0 = 16;
wmax = 1024;
r = 0;
rate = 455.8;
m = log(wmax/w0)/log(2);
w = zeros(1,r+1);
EP = 1500/rate/125000*1000000;
for i = 0:r
if (i<=m)
w(i+1) = 2^i*w0;
elseif (i<=r)
w(i+1) = 2^m*w0;
end
end
F(1) = x(1) * (1-x(2)^(r+1))/(1-x(2)) - x(2);
F(2) = - x(1) + 2 * (1-x(2)) * (1-2*x(2)) / ( w0*(1-(2*x(2))^(m+1))*(1-x(2)) +
(1-2*x(2))*(1-x(2)^(r+1)) + w0*2^m*x(2)^(m+1)*(1-x(2)^(r-m))*(1-2*x(2)));
F(3) = x(3) - x(2);

改进Bianchi main10.mat
function F = main10(x)
w0 = 16;
wmax = 1024;
r = 32;
rate = 455.8;
m = log(wmax/w0)/log(2);
w = zeros(1,r+1);
EP = 1500/rate/125000*1000000;
d = 30/rate/125000*1000000 + 13.6 + EP;
for i = 0:r
if (i<=m)
w(i+1) = 2^i*w0;
elseif (i<=r)
w(i+1) = 2^m*w0;
end
end



42

sum = 0;
for i = 0:r
sum = sum + x(2)^i * (w(i+1) + 3) / 2;
end
dd = 0;
for i = 0:r
dd = dd + x(2)^i * x(1);
end
F(1) = dd - x(3);
F(2) = x(1) * sum - 1;
F(3) = x(2)-x(3);

求解吞吐量 solve1.m
fun = @main10;
% fun = @main1
x0 = [0,0,0];
x = fsolve(fun,x0)
t = x(3);
p = x(2);
w0 = 16;
wmax = 1024;
r = 32;
rate = 455.8;
m = log(wmax/w0)/log(2);
w = zeros(1,r+1);
EP = 1500/rate/125000*1000000;
d = 30/rate/125000*1000000 + 13.6 + EP;
Ts = 30/rate/125000*1000000 + 13.6 + EP + 16 + 32 + 43;
Tc = 30/rate/125000*1000000 + 13.6 + EP + 43 + 65;
e = 9;
S = 2*t*(1-p)*EP*rate*1000000/(2*t*(1-p)*Ts+t*p*Tc+(1-2*t+t*p)*e)

问题一仿真求解python 代码:
import random
import xlwt
from math import log
# CWmin = 16
# CWmax = 1024
r = 32 #最大重传次数
rate = 455.8
# Te = 9
## EP = 1500/rate/125000*1000000



43

Ts = 30/rate/125000*1000000 + 13.6 + EP + 16 + 32 + 43
## Tc = 30/rate/125000*1000000 + 13.6 + EP + 43 + 65
PMP = 30/rate/125000*1000000 + 13.6 + EP  #PHY hdr + MAC hdr + Payload
# SIFS = 16
# DIFS = 43
# ACK = 32
# ACktimeout = 65
m = log(CWmax/CWmin,2)
#计算数据分布
data_list = []
for i in range(40):
time = DIFS #记录时间
cw1icounter, cw2icounter = 0, 0 #AP 数据的发送次数
cw1jcounter, cw2jcounter = 0, 0 #AP 回退翻倍次数
collisioncounter = 0 #崩溃次数
cw1rollbackflag, cw2rollbackflag = 1, 1 #判断是否需要产生新的回退
#离散事件仿真
for j in range(10**6):
if cw1icounter <= m:
cw1jcounter = cw1icounter
elif m < cw1icounter < r: #超过最大退避阶数
cw1jcounter = m
elif cw1icounter >= r: #超过最大重传次数
cw1icounter, cw1jcounter = 0, 0
else:
pass
if cw2icounter <= m:
cw2jcounter = cw2icounter
elif m < cw2icounter < r:
cw2jcounter = m
elif cw2icounter >= r:
cw2icounter, cw2jcounter = 0, 0
else:
pass
if cw1rollbackflag == 1:
CW1 = random.randint(0,(CWmin*(2**cw1jcounter)-1))
else:
pass
if cw2rollbackflag == 1:
CW2 = random.randint(0,(CWmin*(2**cw2jcounter)-1))
else:
pass
if CW1 < CW2:



44

cw1rollbackflag, cw2rollbackflag = 1, 0
time = time + (CW1+1)*Te + Ts
# CW2 = CW2 - CW1 - 1
cw1icounter = 0 #成功上传重置CW
elif CW2 < CW1:
cw1rollbackflag, cw2rollbackflag = 0, 1
time = time + (CW2+1)*Te + Ts
# CW1 = CW1 - CW2 - 1
cw2icounter = 0
else: #同时崩溃
cw1rollbackflag, cw2rollbackflag = 1, 1 #冲突产生新的CW
cw1icounter = cw1icounter + 1 #冲突更新数据发送次数
cw2icounter = cw2icounter + 1
collisioncounter = collisioncounter + 1
time = time + (CW1+1)*Te + Tc
data = (10**6 - collisioncounter)*EP*rate*(10**6)/time
data_list.append(data)
#excel
work_book = xlwt.Workbook(encoding = 'utf-8')
work_sheet = work_book.add_sheet('sheet')
for i in range(len(data_list)):
work_sheet.write(i, 0, data_list[i])
work_book.save('experiment_q1.xls')

问题二仿真求解python 代码:
import random
import xlwt
from math import log
# CWmin = 16
# CWmax = 1024
r = 32 #最大重传次数
rate = 275.3
# Te = 9
## EP = 1500/rate/125000*1000000
Ts = 30/rate/125000*1000000 + 13.6 + EP + 16 + 32 + 43
## Tc = 30/rate/125000*1000000 + 13.6 + EP + 43 + 65
PMP = 30/rate/125000*1000000 + 13.6 + EP  #PHY hdr + MAC hdr + Payload
# SIFS = 16
# DIFS = 43
# ACK = 32
# ACktimeout = 65
m = log(CWmax/CWmin,2)



45

#计算数据分布
data_list = []
for i in range(40):
time = DIFS #记录时间
cw1counter, cw2counter = 0, 0 #记录成功传输次数
cw1rollbackflag, cw2rollbackflag = 1, 1 #判断是否需要产生新的回退
#离散事件仿真
for j in range(10**6):
if cw1rollbackflag == 1:
CW1 = random.randint(0,CWmin-1)
else:
pass
if cw2rollbackflag == 1:
CW2 = random.randint(0,CWmin-1)
else:
pass
if CW1 < CW2:
cw1rollbackflag, cw2rollbackflag = 1, 0
time = time + (CW1+1)*Te + Ts
# CW2 = CW2 - CW1 - 1
cw1counter = cw1counter + 1
elif CW2 < CW1:
cw1rollbackflag, cw2rollbackflag = 0, 1
time = time + (CW2+1)*Te + Ts
# CW1 = CW1 - CW2 - 1
cw2counter = cw2counter + 1
else: #同时上传
cw1rollbackflag, cw2rollbackflag = 1, 1
time = time + (CW1+1)*Te + Ts
cw1counter = cw1counter + 1
cw2counter = cw2counter + 1
data = 2*cw1counter*EP*rate*(10**6)/time
data_list.append(data)
#excel
work_book = xlwt.Workbook(encoding = 'utf-8')
work_sheet = work_book.add_sheet('sheet')
for i in range(len(data_list)):
work_sheet.write(i, 0, data_list[i])
work_book.save('experiment_q2.xls')

问题三数值求解Matlab 代码:
# Bianchi 模型 main3.m



46

function F = main3(x)
w0 = 16;
wmax = 1024;
r = 32;
rate = 455.8;
m = log(wmax/w0)/log(2);
w = zeros(1,r+1);
EP = 1500/rate/125000*1000000;
d = 30/rate/125000*1000000 + 13.6 + EP;
Ts = 30/rate/125000*1000000 + 13.6 + EP + 16 + 32 + 43;
Tc = 30/rate/125000*1000000 + 13.6 + EP + 43 + 65;
e = 9;
kk = d/e;
for i = 0:r
if (i<=m)
w(i+1) = 2^i*w0;
elseif (i<=r)
w(i+1) = 2^m*w0;
end
end
sum = 0;
for i = 0:r
sum = sum + x(2)^i * (w(i+1) + 3) / 2;
end
s = 0;
for i = 0:r
s = s + (1-x(2)) * x(2)^i * x(1) * (Ts-2*d)/Ts;
s = s + x(2) * x(2)^i * x(1) * (Tc-2*d)/Tc;
s = s + (w0-kk+w0)/w0/2 * (1-x(2)) * x(2)^i * x(1) * (d)/Ts;
s = s + (w(i+1)-kk+w(i+1))/w(i+1)/2 * x(2) * x(2)^i * x(1) * (d)/Tc;
k = ceil(kk);
s = s + (w(i+1)-k+1)*(w(i+1)-k)/2/w(i+1) * x(2)^i * x(1);
l = floor(kk);
a = 1 - (kk * 100- l*100)/100;
s = s + a * (w(i+1)-l)/w(i+1) * x(2)^i * x(1);
end
dd = 0;
for i = 0:r
dd = dd + x(2)^i * x(1);
end
F(1) = dd - x(3);
% F(1) = x(1) * (1-x(2)^(r+1))/(1-x(2)) - x(3);
F(2) = x(1) * sum - 1;



47

## F(3) =  1 - s - x(4);%%%%%%%2
% F(3) = x(2)-x(3); %%%%%%%%1
% F(4) = x(4) + (1-x(4)) * 0.1 - x(2);%%%%%%%2
F(4) = x(3) - x(2);

求解 solve3.m
w0 = 16;
wmax = 1024;
r = 32;
rate = 455.8;
m = log(wmax/w0)/log(2);
w = zeros(1,r+1);
EP = 1500/rate/125000*1000000;
d = 30/rate/125000*1000000 + 13.6 + EP;
Ts = 30/rate/125000*1000000 + 13.6 + EP + 16 + 32 + 43;
Tc = 30/rate/125000*1000000 + 13.6 + EP + 43 + 65;
e = 9;
kk = d/e;
fun = @main3;
x0 = [0.5 ,0.5 ,0.5,0];
x = fsolve(fun,x0)
t = x(3);
% p = x(2);   %p13
p = x(4);     %p3
pe = 0.1;
% S = 2*t*(1-p)*EP*rate*1000000/(2*t*(1-p)*Ts+t*p*Tc+(1-2*t+t*p)*e)
% S = 2*t*EP*rate*1000000/(2*t*(1-p)*Ts+t*p*Ts+(1-2*t+t*p)*e)
% S = (2*t*(1-p-pe)*EP)/((1-t)*e+t*(1-p-pe)*Ts+t*(pe+p)*Tc)*rate*1000000

问题三仿真求解python 代码:
import random
import xlwt
from math import log
# CWmin = 16
# CWmax = 1024
r = 32 #最大重传次数
rate = 158.4
# Te = 9
## EP = 1500/rate/125000*1000000
Ts = 30/rate/125000*1000000 + 13.6 + EP + 16 + 32 + 43
## Tc = 30/rate/125000*1000000 + 13.6 + EP + 43 + 65
PMP = 30/rate/125000*1000000 + 13.6 + EP  #PHY hdr + MAC hdr + Payload



48

# SIFS = 16
# DIFS = 43
# ACK = 32
# ACktimeout = 65
m = log(CWmax/CWmin,2)
#计算数据分布
data_list = []
for i in range(40):
time1_old, time2_old = DIFS, DIFS #记录时间
cw1icounter, cw2icounter = 0, 0 #AP 数据的发送次数
cw1jcounter, cw2jcounter = 0, 0 #AP 回退翻倍次数
cw1rollbackflag, cw2rollbackflag = 1, 1 #判断是否需要产生新的回退
## AP1successcounter = 0 #AP1 成功传输次数计数器
## AP1flag = 0 #判别AP1 是否成功
#离散事件仿真
for j in range(10**6):
if cw1icounter <= m:
cw1jcounter = cw1icounter
elif m < cw1icounter < r: #超过最大退避阶数
cw1jcounter = m
elif cw1icounter >= r: #超过最大重传次数
cw1icounter, cw1jcounter = 0, 0
else:
pass
if cw2icounter <= m:
cw2jcounter = cw2icounter
elif m < cw2icounter < r:
cw2jcounter = m
elif cw2icounter >= r:
cw2icounter, cw2jcounter = 0, 0
else:
pass
if cw1rollbackflag == 1:
CW1 = random.randint(0,(CWmin*(2**cw1jcounter)-1))
else:
pass
if cw2rollbackflag == 1:
CW2 = random.randint(0,(CWmin*(2**cw2jcounter)-1))
else:
pass
time1_temp = time1_old + CW1*Te
time2_temp = time2_old + CW2*Te
if time1_temp > time2_temp: #站点2 优先回退至0



49

if time2_temp + PMP <= time1_temp: #站点2 信号在站点1 回退至0 前不发生重叠
if random.random() <= 0.9: #不丢包
time1_new = time1_temp + PMP + SIFS + ACK + DIFS
## AP1successcounter = AP1successcounter + 1
# AP1flag = 1
cw1icounter = 0 #成功上传重置CW
else: #丢包概率10%
time1_new = time1_temp + PMP + ACktimeout + DIFS
# AP1flag = 0
cw1icounter = cw1icounter + 1
if random.random() <= 0.9: #不丢包
time2_new = time2_temp + PMP + SIFS + ACK + DIFS
cw2icounter = 0 #成功上传重置CW
else: #丢包概率10%
time2_new = time2_temp + PMP + ACktimeout + DIFS
cw2icounter = cw2icounter + 1
else: #站点2 信号在站点1 回退至0 前发生重叠
time1_new = time1_temp + PMP + ACktimeout + DIFS
# AP1flag = 0
time2_new = time2_temp + PMP + ACktimeout + DIFS
cw1icounter = cw1icounter + 1
cw2icounter = cw2icounter + 1
elif time1_temp < time2_temp: #站点1 优先回退至0
if time1_temp + PMP <= time2_temp: #站点1 信号在站点2 回退至0 前不发生重叠
if random.random() <= 0.9: #不丢包
time1_new = time1_temp + PMP + SIFS + ACK + DIFS
## AP1successcounter = AP1successcounter + 1
# AP1flag = 1
cw1icounter = 0 #成功上传重置CW
else: #丢包概率10%
time1_new = time1_temp + PMP + ACktimeout + DIFS
# AP1flag = 0
cw1icounter = cw1icounter + 1
if random.random() <= 0.9: #不丢包
time2_new = time2_temp + PMP + SIFS + ACK + DIFS
cw2icounter = 0 #成功上传重置CW
else: #丢包概率10%
time2_new = time2_temp + PMP + ACktimeout + DIFS
cw2icounter = cw2icounter + 1
else: #站点1 信号在站点2 回退至0 前发生重叠
time1_new = time1_temp + PMP + ACktimeout + DIFS
# AP1flag = 0
time2_new = time2_temp + PMP + ACktimeout + DIFS



50

cw1icounter = cw1icounter + 1
cw2icounter = cw2icounter + 1
else: #同时回退至0 肯定发生重叠
time1_new = time1_temp + PMP + ACktimeout + DIFS
# AP1flag = 0
time2_new = time2_temp + PMP + ACktimeout + DIFS
cw1icounter = cw1icounter + 1
cw2icounter = cw2icounter + 1
if time1_new <= time2_temp:
cw1rollbackflag, cw2rollbackflag = 1, 0
time2_new = time2_old
else:
cw1rollbackflag, cw2rollbackflag = 1, 1
if time2_new <= time1_temp:
cw1rollbackflag, cw2rollbackflag = 0, 1
time1_new = time1_old
if AP1flag == 1:
## AP1successcounter = AP1successcounter - 1
else:
pass
else:
cw1rollbackflag, cw2rollbackflag = 1, 1
#进入下一事件阶段
time1_old = time1_new
time2_old = time2_new
data = 2*AP1successcounter*EP*rate*(10**6)/time1_new
data_list.append(data)
#excel
work_book = xlwt.Workbook(encoding = 'utf-8')
work_sheet = work_book.add_sheet('sheet')
for i in range(len(data_list)):
work_sheet.write(i, 0, data_list[i])
work_book.save('experiment_q3.xls')

问题四数值求解Matlab 代码:
改进Bianchi 模型: main4.m
function F = main4(x)
w0 = 16;
wmax = 1024;
r = 6;
rate = 158.4;
m = log(wmax/w0)/log(2);



51

w = zeros(1,r+1);
EP = 1500/rate/125000*1000000;
d = 30/rate/125000*1000000 + 13.6 + EP;
Ts = 30/rate/125000*1000000 + 13.6 + EP + 16 + 32 + 43;
Tc = 30/rate/125000*1000000 + 13.6 + EP + 43 + 65;
e = 9;
kk = d/e;
for i = 0:r
if (i<=m)
w(i+1) = 2^i*w0;
elseif (i<=r)
w(i+1) = 2^m*w0;
end
end
sum1 = 0;
for i = 0:r
sum1 = sum1 + x(5)^i * (w(i+1) + 3) / 2;
end
sum2 = 0;
for i = 0:r
sum2 = sum2 + x(6)^i * (w(i+1) + 3) / 2;
end
s = 0;
for i = 0:r
s = s + (1-x(5)) * x(5)^i * x(1) * (Ts-2*d)/Ts;
%     s = s + x(5) * x(5)^i * x(1) * (Tc-2*d)/Tc;
s = s + (w0-kk+w0)/w0/2 * (1-x(5)) * x(5)^i * x(1) * (d)/Ts;
%     s = s + (w(i+1)-kk+w(i+1))/w(i+1)/2 * x(5) * x(5)^i * x(1) * (d)/Tc;
k = ceil(kk);
s = s + (w(i+1)-k+1)*(w(i+1)-k)/2/w(i+1) * x(5)^i * x(1);
l = floor(kk);
a = 1 - (kk * 100- l*100)/100;
s = s + a * (w(i+1)-l)/w(i+1) * x(5)^i * x(1);
end
F(1) = x(1) * sum1 - 1;
F(2) = x(2) * sum2 - 1;
F(3) = 1 - s - x(7);
F(4) = x(1) * (1-x(5)^(r+1))/(1-x(5)) - x(3);
F(5) = x(2) * (1-x(6)^(r+1))/(1-x(6)) - x(4);
F(6) = x(4) - x(5);
F(7) = (2*x(3) - x(3)*x(7))-x(6);

模型求解:solve4.m



52

w0 = 16;
wmax = 1024;
r = 6;
rate = 158.4;
m = log(wmax/w0)/log(2);
w = zeros(1,r+1);
EP = 1500/rate/125000*1000000;
d = 30/rate/125000*1000000 + 13.6 + EP;
Ts = 30/rate/125000*1000000 + 13.6 + EP + 16 + 32 + 43;
Tc = 30/rate/125000*1000000 + 13.6 + EP + 43 + 65;
e = 9;
kk = d/e;
fun = @main4;
x0 = [0,0,0,0,0,0,0];
x = fsolve(fun,x0)
b001 = x(1);
b002 = x(2);
t1 = x(3);
t2 = x(4);
pact1 = x(5);
pact2 = x(6);
pov = x(7);
S = (t2*(1-pact2) + (2*t1*(1-pact1)))*EP;
t11 = (t2*(1-pact2)+t1*(1-pact1)) * Ts;
t22 = t2 * pact2 * Tc;
t33 = (1 - t2 - t1*(1-pact1)) * e;
## S = S/(t11+t22+t33)*rate*1000000

问题四仿真求解python 代码:
import random
import xlwt
from math import log
# CWmin = 16
# CWmax = 1024
r = 32 #最大重传次数
rate = 158.4
# Te = 9
## EP = 1500/rate/125000*1000000
Ts = 30/rate/125000*1000000 + 13.6 + EP + 16 + 32 + 43
## Tc = 30/rate/125000*1000000 + 13.6 + EP + 43 + 65
PMP = 30/rate/125000*1000000 + 13.6 + EP  #PHY hdr + MAC hdr + Payload
# SIFS = 16
# DIFS = 43



53

# ACK = 32
# ACktimeout = 65
m = log(CWmax/CWmin,2)
#计算数据分布
data_list = []
for i in range(40):
time1_old, time2_old, time3_old = DIFS, DIFS, DIFS #记录时间
cw1icounter, cw2icounter, cw3icounter = 0, 0, 0 #AP 数据的发送次数
cw1jcounter, cw2jcounter, cw3jcounter = 0, 0, 0 #AP 回退翻倍次数
cw1rollbackflag, cw2rollbackflag, cw3rollbackflag = 1, 1, 1 #判断是否需要产生新
的回退
AP1successcounter, AP2successcounter, AP3successcounter = 0, 0, 0 #AP 传输成功
次数计数器
#离散事件仿真
for j in range(10**6):
if cw1icounter <= m:
cw1jcounter = cw1icounter
elif m < cw1icounter < r: #超过最大退避阶数
cw1jcounter = m
elif cw1icounter >= r: #超过最大重传次数
cw1icounter, cw1jcounter = 0, 0
else:
pass
if cw2icounter <= m:
cw2jcounter = cw2icounter
elif m < cw2icounter < r:
cw2jcounter = m
elif cw2icounter >= r:
cw2icounter, cw2jcounter = 0, 0
else:
pass
if cw3icounter <= m:
cw3jcounter = cw3icounter
elif m < cw3icounter < r:
cw3jcounter = m
elif cw3icounter >= r:
cw3icounter, cw3jcounter = 0, 0
else:
pass
if cw1rollbackflag == 1:
CW1 = random.randint(0,(CWmin*(2**cw1jcounter)-1))
else:
pass



54

if cw2rollbackflag == 1:
CW2 = random.randint(0,(CWmin*(2**cw2jcounter)-1))
else:
pass
if cw3rollbackflag == 1:
CW3 = random.randint(0,(CWmin*(2**cw3jcounter)-1))
else:
pass
time1_temp = time1_old + CW1*Te
time2_temp = time2_old + CW2*Te
time3_temp = time3_old + CW3*Te
if time1_temp < min(time2_temp,time3_temp): #AP1 最先回退至0
## AP1successcounter = AP1successcounter + 1
time1_new = time1_temp + Ts
time3_new = time3_temp + Ts
if time1_new < time3_temp:
time3_new = time3_old #new 回退至old
time2_new = time1_new #hold 至AP1 发送结束
cw1rollbackflag, cw2rollbackflag, cw3rollbackflag = 1, 0, 0 #AP1 产
# 生新的回退 AP3 不改变
cw1icounter = 0 #AP1 重置 AP3 不改变
else:
## AP3successcounter = AP3successcounter + 1
time2_new = time3_new #hold 至AP3 发送结束
cw1rollbackflag, cw2rollbackflag, cw3rollbackflag = 1, 0, 1 #AP1 和
# AP3 产生新的回退
cw1icounter, cw3icounter = 0, 0 #AP1 和AP3 成功上传重置CW
time2_test = time2_old
while 1: #计算AP2 剩余回退间隙
# CW2 = CW2 - 1
time2_test = time2_test + Te
if time2_test > time1_temp:
break
else:
pass
elif time3_temp < min(time1_temp,time2_temp): #AP3 最先回退至0
## AP3successcounter = AP3successcounter + 1
time1_new = time1_temp + Ts
time3_new = time3_temp + Ts
if time3_new < time1_temp:
time1_new = time1_old
time2_new = time3_new
cw1rollbackflag, cw2rollbackflag, cw3rollbackflag = 0, 0, 1



55

cw3icounter = 0
else:
## AP1successcounter = AP1successcounter + 1
time2_new = time1_new
cw1rollbackflag, cw2rollbackflag, cw3rollbackflag = 1, 0, 1
cw1icounter, cw3icounter = 0, 0
time2_test = time2_old
while 1:
# CW2 = CW2 - 1
time2_test = time2_test + Te
if time2_test > time3_temp:
break
else:
pass
elif time2_temp < min(time1_temp,time3_temp): #AP2 最先回退至0
## AP2successcounter = AP2successcounter + 1
time2_new = time2_temp + Ts
time1_new = time2_new
time3_new = time2_new
cw1rollbackflag, cw2rollbackflag, cw3rollbackflag = 0, 1, 0
cw2icounter = 0
time1_test = time1_old
while 1:
# CW1 = CW1 - 1
time1_test = time1_test + Te
if time1_test > time2_temp:
break
else:
pass
time3_test = time3_old
while 1:
# CW3 = CW3 - 1
time3_test = time3_test + Te
if time3_test > time2_temp:
break
else:
pass
elif time1_temp == time2_temp < time3_temp: #AP1 和AP2 同时回退至0 冲突
time1_new = time1_temp + Tc
time2_new = time2_temp + Tc
time3_new = time1_new
cw1rollbackflag, cw2rollbackflag, cw3rollbackflag = 1, 1, 0 #冲突AP 重
新回退



56

cw1icounter = cw1icounter + 1 #冲突计数器加一
cw2icounter = cw2icounter + 1
time3_test = time3_old
while 1:
# CW3 = CW3 - 1
time3_test = time3_test + Te
if time3_test > time1_temp:
break
else:
pass
elif time3_temp == time2_temp < time1_temp: #AP3 和AP2 同时回退至0 冲突
time3_new = time3_temp + Tc
time2_new = time2_temp + Tc
time1_new = time3_new
cw1rollbackflag, cw2rollbackflag, cw3rollbackflag = 0, 1, 1
cw3icounter = cw3icounter + 1
cw2icounter = cw2icounter + 1
time1_test = time1_old
while 1:
# CW1 = CW1 - 1
time1_test = time1_test + Te
if time1_test > time3_temp:
break
else:
pass
elif time1_temp == time3_temp < time2_temp: #AP1 和AP3 同时回退至0 均能成功
上传
## AP1successcounter = AP1successcounter + 1
## AP3successcounter = AP3successcounter + 1
time1_new = time1_temp + Ts
time3_new = time3_temp + Ts
time2_new = time1_new
cw1rollbackflag, cw2rollbackflag, cw3rollbackflag = 1, 0, 1
cw1icounter, cw3icounter = 0, 0 #AP1 和AP3 均成功上传
time2_test = time2_old
while 1:
# CW2 = CW2 - 1
time2_test = time2_test + Te
if time2_test > time1_temp:
break
else:
pass
elif time1_temp == time2_temp == time3_temp: #AP1 和AP2 和AP3 同时回退至0 冲



57

突
time1_new = time1_temp + Tc
time2_new = time2_temp + Tc
time3_new = time3_temp + Tc
cw1rollbackflag, cw2rollbackflag, cw3rollbackflag = 1, 1, 1
cw1icounter = cw1icounter + 1
cw2icounter = cw2icounter + 1
cw3icounter = cw3icounter + 1
else:
pass
#进入下一事件阶段
time1_old = time1_new
time2_old = time2_new
time3_old = time3_new
data
=
(AP1successcounter+AP2successcounter+AP3successcounter)*EP*rate*(10**6)/time1_new
data_list.append(data)
#excel
work_book = xlwt.Workbook(encoding = 'utf-8')
work_sheet = work_book.add_sheet('sheet')
for i in range(len(data_list)):
work_sheet.write(i, 0, data_list[i])
work_book.save('experiment_q4.xls')

