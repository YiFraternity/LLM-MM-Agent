\documentclass[12pt]{article}
\usepackage{ctex}
\usepackage{amsmath,amssymb}
\usepackage{graphicx}
\usepackage{enumitem}
\usepackage{geometry}
\geometry{left=3cm,right=3cm,top=2.5cm,bottom=2.5cm}

\title{机械臂运动路径设计问题}
\author{}
\date{}

\begin{document}

\maketitle

\section{1. 问题的简述}

机器人最主要的操作机构是各种形态的机械臂,机械臂的动作形态是是由三种不同的单位动作——旋转、回转、伸缩组合而成的。旋转或回转是指机械臂产生相对转动,两者的不同仅在于转动部件的轴线与转动轴线是否同轴\footnote{1}。旋转如笔帽的转动,回转如摇柄的转动,伸缩指机械臂产生直线运动,引进伸缩动作,运动范围可以得到扩大。机器人机械手具有的独立的单位动作组合数称之为自由度。一个物体在三维空间中的位置和姿态分别需要三个独立的坐标才能得到确定。所以,能使手爪达到任意空间位置并具有任意姿态的机械手最少需要六个自由度。自由度是表达机器人通用性、灵活性的主要指标,自由度越多,机械臂越灵活,可达的空间更广泛。当然自由度越多,机械结构越复杂,从而控制也越困难。

本文所研究的六自由度的机器臂的简化图如图(1):

\begin{figure}[h]
    \centering
    \includegraphics[width=0.8\textwidth]{image.png}
    \caption{}
\end{figure}

自由度一、四、六,对应于图中的 $F, G, H$ 点为旋转关节,自由度二、三、五,对应的图中的 $B, C, D$ 点为回转关节。每一个关节对应一个角度 $\theta_{i}$,这个角度表示前一个连杆方向到后一个连杆方向转角(对于 $B, C, D$),连杆方向为 $AB, BC, CD, DE$,或者相对于初始位置的转角(对于 $F, G, H$),假设机器人的初始位置是在一个平面上的($y$-$z$ 平面)。为了使机器人运动得更加灵活,允许关节的转角超过 $360^\circ$。

机器人关于六个自由度的每一个组合 $\Theta = (\theta_1, \theta_2, \theta_3, \theta_4, \theta_5, \theta_6)$,表示机械臂的一个姿态,显然每个姿态确定顶端指尖的空间位置 $X$: $f(\Theta) \to X$。假定机器人控制系统只能够接收改变各个关节的姿态的关于连杆角度的增量指令(机器指令)$P = \Delta\Theta = (\Delta\theta_1, \Delta\theta_2, \Delta\theta_3, \Delta\theta_4, \Delta\theta_5, \Delta\theta_6)$,使得指尖(图 2 中的 E 点)移动到空间点 $X'$,通过一系列的指令序列 $P_1, P_2, P_3, \dots, P_n$ 可以将指尖依

次到达位置 $X_{0}, X_{1}, \cdots, X_{n}$,则称 $X_{0}, X_{1}, \cdots, X_{n}$ 为从指尖初始位置 $X_{0}$ 到达目标位置 $X_{n}$ 的一条路径(运动轨迹)。

根据具体的目标和约束条件计算出合理、便捷、有效的指令序列是机器人控制中的一个重要问题。本文就是对这个问题进行讨论。其中 $AB=140 \mathrm{~mm}, BC=255 \mathrm{~mm}, CD=255 \mathrm{~mm}, DE=65 \mathrm{~mm}$,各个增量 $\Delta \theta_{i}$ 只能取到 $-2,-1.9,-1.8, \cdots, 1.8, 1.9, 2$ 这 41 个离散值(即精度为 $0.1^{\circ}$,绝对值不超过 $2^{\circ}$)。讨论问题如下:

1. 已知初始姿态 $\Phi_{0}$ 和一个可达目标点的空间位置 $(O_{x}, O_{y}, O_{z})$,计算指尖到达目标点的指令序列。

2. 要求指尖沿着预先指定的一条空间曲线 $x=x(s), y=y(s), z=z(s), a \leqslant s \leqslant b$ 移动,计算满足要求的指令序列。

3. 在第①个问题中,假设在初始位置与目标位置之间的区域中有若干个已知大小、形状、方向和位置的障碍物,要求机械臂在运动中始终不能与障碍物相碰,否则会损坏机器。这个问题称机械臂避碰问题,要求机械臂末端在误差范围内到达目标点并且整个机械臂不碰到障碍物。

应用所讨论的算法就下面具体的数据给出计算结果:假设在机械臂的旁边有一个待加工的中空圆台形工件,上部开口。工件高 $180 \mathrm{~mm}$,下底外半径 $168 \mathrm{~mm}$,上底外半径 $96 \mathrm{~mm}$,壁厚 $8 \mathrm{~mm}$。竖立地固定在 $xy$-平面的操作台上,底部的中心在 $(210,0,0)$。

1. 要求机械臂(指尖)从初始位置移动到工具箱所在位置的 $(20,-200,120)$ 处,以夹取要用的工具。

2. 如果圆台形工件外表面与平面 $x=2z$ 的交线是一条裂纹需要焊接,请你给出机械臂指尖绕这条曲线一周的指令序列。

3. 有一项任务是在工件内壁点焊四个小零件,它们在内表面上的位置到 $xy$ 平面的投影为 $(320,-104)、(120,106)、(190,-125)$ 和 $(255,88)$。要求机械臂从圆台的上部开口处伸进去到达这些点进行加工,为简捷起见,不妨不计焊条等的长度,只考虑指尖的轨迹。

\section{2. 问题的分析}

机械臂的运动属于运动学的范畴。运动学研究旨在解决机器人的手臂转向何方。机械臂的运动学要研究两个问题:运动学正问题——已知各关节位移变量的值,要求机械臂指尖在空间的位置和姿态,这实际上是建立运动学方程的过程;运动学逆问题——指定指尖的空间位置和姿态,要求出各关节位移变量的相应值,这实际上是求解运动学的过程。

本文中的问题属于运动学逆问题,即已知初始状态求机械臂指尖到达目标点或者沿一定的轨迹曲线到达目标点的指令序列。运动学逆问题直接求解较困难。本文设计几何算法和牛顿迭代法求解自由空间中机械臂的运动逆问题。

\section{符号约定与模型假设}

为简化问题,作如下假设:机械臂连杆的粗细忽略不计。

全文符号约定如下:

\begin{align*}
f(\Theta) &= f(\theta_1, \theta_2, \theta_3, \theta_4, \theta_5, \theta_6): & \text{指尖任意姿态:} \\
f^0(\Theta) &= f\left(\theta_1^0, \theta_2^0, \theta_3^0, \theta_4^0, \theta_5^0, \theta_6^0\right): & \text{初始姿态:} \\
f(\tilde{\Theta}) &= f\left(\tilde{\theta}_1, \tilde{\theta}_2, \tilde{\theta}_3, \tilde{\theta}_4, \tilde{\theta}_5, \tilde{\theta}_6\right): & \text{最终达到的姿态:} \\
d: & & \text{机械臂指尖最终达到的位置与目标点之间的距离}
\end{align*}

\section{自由空间中机械臂运动逆问题的求解模型}

\subsection{机械臂指尖坐标公式求解}

对机械臂在空间运动的描述离不开各关节的坐标,因此须首先求出各关节的坐标。

首先,建立直角坐标系。原点设在图(1)中的A点。$z$ 轴与 $AB$ 重合,正向与 $AB$ 方向一致。$x$ 轴垂直纸面,正向向外。$y$ 轴在基座所在的水平面上,正向向右。如图(2)所示:

\begin{figure}[h]
\centering
\includegraphics[width=0.8\textwidth]{image.png}
\caption{}
\end{figure}

对角度方向作如下约定:机械臂的关节旋转时,约定顺时针方向为负。坐标系变换时,约定顺时针方向为正。假设 $\overline{FB}$ 相对于 $\overline{AF}$ 的旋转角为 $\theta_1$,$\overline{DH}$ 到 $\overline{BC}$ 的回转角为 $\theta_2$,$\overline{BC}$ 到 $\overline{CD}$ 的回转角为 $\theta_{3}$,$\overline{GD}$ 相对于 $\overline{CG}$ 的旋转角为 $\theta_{4}$,$\overline{CD}$ 到 $\overline{DE}$ 的回转角为 $\theta_{5}$,$\overline{HE}$ 相对于 $\overline{DH}$ 的旋转角为 $\theta_{6}$,由于 $\theta_{6}$ 控制手指的转动,用于操纵工具,它不影响机械臂指尖的定位问题,因此,本文不考虑 $\theta_{6}$ 对问题的影响。

\subsubsection{4.1.1 A、B、C、D 坐标的确定}

由于 A、B、C、D 点在同一平面上,易知 A、B、C、D 点坐标为:

A: $x_{A}=0 \quad y_{A}=0 \quad z_{A}=0$

B: $x_{B}=0 \quad y_{B}=0 \quad z_{B}=140$

C: $x_{C}=BC \sin (-\theta_{2}) \sin \theta_{1}, \quad y_{C}=BC \sin (-\theta_{2}) \cos \theta_{1} \quad z_{C}=BC \cos (-\theta_{2})+AB$

D: $x_{D}=\left[CD \sin (\pi+\theta_{2}+\theta_{3})+BC \sin (-\theta_{2})\right] \sin \theta_{1}$

$y_{D}=\left[CD \sin (\pi+\theta_{2}+\theta_{3})+BC \sin (-\theta_{2})\right] \cos \theta_{1}$

$z_{D}=\left[BC \cos (-\theta_{2})-CD \cos (\pi+\theta_{2}+\theta_{3})\right]+140$

\subsubsection{4.1.2 E 点坐标}

由于 E 点与上述 A、B、C、D 点异面,所以直接在此坐标系下求 E 的坐标比较困难。因此,我们在 D 点建立新坐标系,求得 E 点在新坐标系下的坐标后,再变换到原坐标系,得到 E 点在原坐标系下的坐标公式。

建立新坐标系 $Dx'y'z'$:以 D 为原点,以 $\overline{CD}$ 为 $y'$ 轴,正向与 $\overline{CD}$ 方向一致。$x'$ 轴垂直于平面 yoz,正向向外。垂直于 $\overline{CD}$ 的方向定义为 $z'$ 轴,轴 $x'$、$y'$、$z'$ 的方向满足右手准则。

将 Axyz 坐标系变换到 $Dx'y'z'$ 坐标系,可分为三步:(1)将原点平移到 D;(2)将 yAz 平面绕 x 轴旋转 $\left(-\frac{\pi}{2}-\theta_{2}-\theta_{3}\right)$,得到 $Dx''y''z''$ 坐标系;(3)将 $Dx''y''z''$ 坐标系统 z 轴旋转 $\theta_{1}$,得到新坐标系 $Dx'y'z'$。于是得到坐标变换公式:

\[
\left[\begin{array}{lll}
x_{E} & y_{E} & z_{E}
\end{array}\right]=\left[\begin{array}{lll}
x_{E}^{\prime} & y_{E}^{\prime} & z_{E}^{\prime}
\end{array}\right] T_{1} T_{2}+\left[\begin{array}{lll}
x_{D} & y_{D} & z_{D}
\end{array}\right]
\]

其中:

\begin{equation}
T_{2}=\left[\begin{array}{ccc}
\cos 0 & \cos \frac{\pi}{2} & \cos \frac{\pi}{2} \\
\cos \frac{\pi}{2} & \cos \left(\frac{\pi}{2}+\theta_{2}+\theta_{3}\right) & \cos \left(\theta_{2}+\theta_{3}\right) \\
\cos \frac{\pi}{2} & \cos \left(\pi+\theta_{2}+\theta_{3}\right) & \cos \left(\frac{\pi}{2}+\theta_{2}+\theta_{3}\right)
\end{array}\right]
\end{equation}

\begin{equation}
T_{1}=\left[\begin{array}{ccc}
\cos \theta_{1} & \cos \left(\frac{\pi}{2}+\theta_{1}\right) & \cos \frac{\pi}{2} \\
\cos \left(\frac{\pi}{2}-\theta_{1}\right) & \cos \left(\theta_{1}\right) & \cos \frac{\pi}{2} \\
\cos \frac{\pi}{2} & \cos \left(\frac{\pi}{2}\right) & \cos 0
\end{array}\right]
\end{equation}

机器臂指尖E点在新坐标系下的坐标为:
\begin{align*}
x_{E}^{\prime} &= 65 \sin \theta_{4} \sin \theta_{5} \\
y_{E}^{\prime} &= 65 \cos \theta_{5} \\
z_{E}^{\prime} &= 65 \sin \theta_{5} \cos \theta_{4}
\end{align*}

由式(1)可得E点在原坐标系下的坐标为:
\begin{align*}
x_{E} &= \sin \theta_{1} \left[-255 \sin \theta_{2} - 255 \sin \left(\theta_{2} + \theta_{3}\right)\right] + 65 \cos \theta_{1} \sin \theta_{4} \sin \theta_{5} \\
&\quad + \sin \theta_{1} \left[-65 \cos \theta_{5} \sin \left(\theta_{2} + \theta_{3}\right) - 65 \cos \left(\theta_{2} + \theta_{3}\right) \cos \theta_{4} \sin \theta_{5}\right] \\
y_{E} &= \cos \theta_{1} \left[-255 \sin \theta_{2} - 255 \sin \left(\theta_{2} + \theta_{3}\right)\right] - 65 \sin \theta_{1} \sin \theta_{4} \sin \theta_{5} \\
&\quad + \cos \theta_{1} \left[-65 \cos \theta_{5} \sin \left(\theta_{2} + \theta_{3}\right) - 65 \cos \left(\theta_{2} + \theta_{3}\right) \cos \theta_{4} \sin \theta_{5}\right] \\
z_{E} &= 140 + 255 \cos \theta_{2} + 255 \cos \left(\theta_{2} + \theta_{3}\right) + 65 \cos \left(\theta_{2} + \theta_{3}\right) \cos \theta_{5} \\
&\quad - 65 \cos \theta_{4} \sin \left(\theta_{2} + \theta_{3}\right) \sin \theta_{5}
\end{align*}

\subsection{4.2 几何算法}

从初始状态 \( f(\Theta_{0}) \) 出发,为了到达自由空间中任一目标点 \( Q(x_{Q}, y_{Q}, z_{Q}) \),机械臂先快速的接近该点,再通过对机械臂微调更精确接近目标点,微调的准则为:使机械臂指尖最终达到位置与目标点之间的距离 \( d \) 最小。

在自由空间中,可以令 \( \theta_{4} = 0 \)。机械臂精确定位算法如下:

(1)求机械臂指尖E最接近目标Q的位置

机械臂的初始状态如图(3):

\begin{figure}[h]
    \centering
    \includegraphics[width=0.8\textwidth]{image.png}
    \caption{}
    \label{fig:3}
\end{figure}

1) 对任意目标点 $\mathbf{Q}\left(x_{Q}, y_{Q}, z_{Q}\right)$,计算 $\mathbf{A}(0,0,0)$ 与点 $\mathbf{H}\left(x_{Q}, y_{Q}, 0\right)$ 之间的距离 $AH$;

2) 围绕 $z$ 轴旋转机械臂平面 $ABCD$(即 $yoz$)到 $ABH$ 平面,
\begin{equation}
\text{旋转角 } \theta_{1} =
\begin{cases}
arctg\left|\dfrac{x_{Q}}{y_{Q}}\right|, & x_{Q} \geq 0, y_{Q} > 0 \\
\pi - arctg\left|\dfrac{x_{Q}}{y_{Q}}\right|, & x_{Q} \geq 0, y_{Q} < 0 \\
-arctg\left|\dfrac{x_{Q}}{y_{Q}}\right|, & x_{Q} \leq 0, y_{Q} > 0 \\
arctg\left|\dfrac{x_{Q}}{y_{Q}}\right| - \pi, & x_{Q} \leq 0, y_{Q} < 0
\end{cases}
\tag{确定 $\theta_{1}$}
\end{equation}

2) 建立新坐标系,把 $ABH$ 平面作为新的 $yoz$ 坐标面,新 $y$ 轴与 $AH$ 重合,正向为 $\overline{AH}$ 方向,$z$ 轴不变。根据图(2),$\overline{AB}$ 到 $\overline{BC}$ 的转角为 $\theta_{2}$,$\overline{BC}$ 到 $\overline{CD}$ 的转角为 $\theta_{3}$,$\overline{CD}$ 到 $\overline{DE}$ 的转角为 $\theta_{5}$,则 $ABE$ 平面上机械臂指尖位置为:
\begin{align*}
y'_{E} &= BC\cos(\theta_{2}) + CD\cos(\theta_{2} + \theta_{3}) + DE\cos(\theta_{2} + \theta_{3} + \theta_{5}) \\
z'_{E} &= BC\sin(\theta_{2}) + CD\sin(\theta_{2} + \theta_{3}) + DE\sin(\theta_{2} + \theta_{3} + \theta_{5})
\end{align*}

3) 在原来的坐标系中,机械臂指尖的位置为
\begin{align*}
x_{E} &= y'_{E}\sin(\theta_{1}) \\
y_{E} &= y'_{E}\cos(\theta_{1}) = \left(BC\cos(\theta_{2}) + CD\cos(\theta_{2} + \theta_{3}) + DE\cos(\theta_{2} + \theta_{3} + \theta_{5})\right)\cos(\theta_{1}) \\
z_{E} &= z'_{E} = BC\sin(\theta_{2}) + CD\sin(\theta_{2} + \theta_{3}) + DE\sin(\theta_{2} + \theta_{3} + \theta_{5})
\end{align*}

按步进精度 $0.1^{\circ}$,搜索 $\theta_{2}, \theta_{3}, \theta_{5}$,使误差 $\Delta d = \sqrt{\left(x_{Q} - x_{E}\right)^{2} + \left(y_{Q} - y_{E}\right)^{2} + \left(z_{Q} - z_{E}\right)^{2}}$ 最小,最后给出精度为 $0.1^{\circ}$ 的机械臂指尖的最终状态 $\left(\widetilde{\theta}_{1}, \widetilde{\theta}_{2}, \widetilde{\theta}_{3}, 0, \widetilde{\theta}_{5}\right)$。

令 $\Phi = \{2.0, 1.9, \ldots, 0.1, -0.1, -0.2, \ldots, -1.9, -2.0\}$,

如果 $\widetilde{\theta}_{i} \geq 2.0$,则 $\widetilde{\theta}_{i} = 2.0 \left\lfloor \frac{\widetilde{\theta}_{i}}{2.0} \right\rfloor + d^{i}$,$d^{i} \in \Phi$,如果 $2.0 > \widetilde{\theta}_{i} \geq 0$,则 $\widetilde{\theta}_{i} = d^{i}$,$d^{i} \in \Phi$。

如果 $\widetilde{\theta}_{i} \leq -2.0$,则 $\widetilde{\theta}_{i} = -2.0 \left\lfloor \frac{\widetilde{\theta}_{i}}{-2.0} \right\rfloor + d^{i}$,$d^{i} \in \Phi$,如果 $-2.0 < \widetilde{\theta}_{i} < 0$,则 $\widetilde{\theta}_{i} = d^{i}$,$d^{i} \in \Phi$。即存在 $d^{i} \in \Phi$,整数 $a^{i}$,$b^{i}$,$a^{i} \geq 0$,$b^{i} = 0, 1$,使得 $\widetilde{\theta}_{i} = a^{i} \times 2.0 + b^{i} \times d^{i}$,或者 $\widetilde{\theta}_{i} = a^{i} \times (-2.0) + b^{i} \times d^{i}$。

(4) 给出步进序列:

设初始状态为 $\left( \theta_{1}^{1}, \theta_{2}^{1}, \theta_{3}^{1}, \theta_{4}^{1}, \theta_{5}^{1} \right)$,则 $\widetilde{\theta}_{i}$ 的步进过程:$\theta_{i}^{1}, \theta_{i}^{2}, \ldots, \theta_{i}^{a^{i}}, \theta_{i}^{a^{i} + b^{i}}$,其中:
\[
\theta_{i}^{s_{i} - 1} = \theta_{i}^{s_{i}} + 2.0 \quad s_{i} = 1, \ldots, a^{i}
\]
\[
\theta_{i}^{a^{i} + b^{i}} = \theta_{i}^{a^{i}} + b^{i} \times d^{i}
\]

将 $\widetilde{\theta}_{i}$ 的步进过程记为 $\widetilde{\theta}_{i}^{1}, \widetilde{\theta}_{i}^{2}, \ldots, \widetilde{\theta}_{i}^{s_{i}}$,则从初始状态 $\left( \theta_{1}^{0}, \theta_{2}^{0}, \theta_{3}^{0}, \theta_{4}^{0}, \theta_{5}^{0} \right)$ 到达终态 $\left( \theta_{1}, \theta_{2}, \theta_{3}, \theta_{4}, \theta_{5} \right) = \left( \widetilde{\theta}_{1}, \widetilde{\theta}_{2}, \widetilde{\theta}_{3}, 0, \widetilde{\theta}_{5} \right)$ 的步进序列为:
\begin{enumerate}
    \item $\left( \widetilde{\theta}_{1}^{1}, \widetilde{\theta}_{2}, \widetilde{\theta}_{3}, 0, \widetilde{\theta}_{5} \right), \left( \widetilde{\theta}_{1}^{2}, \widetilde{\theta}_{2}, \widetilde{\theta}_{3}, 0, \widetilde{\theta}_{5} \right), \ldots, \left( \widetilde{\theta}_{1}^{s_{1}}, \widetilde{\theta}_{2}, \widetilde{\theta}_{3}, 0, \widetilde{\theta}_{5} \right)$,
    \item $\left( \widetilde{\theta}_{1}^{s_{1}}, \widetilde{\theta}_{2}^{1}, \widetilde{\theta}_{3}, 0, \widetilde{\theta}_{5} \right), \left( \widetilde{\theta}_{1}^{s_{1}}, \widetilde{\theta}_{2}^{2}, \widetilde{\theta}_{3}, 0, \widetilde{\theta}_{5} \right), \ldots, \left( \widetilde{\theta}_{1}^{s_{1}}, \widetilde{\theta}_{2}^{s_{2}}, \widetilde{\theta}_{3}, 0, \widetilde{\theta}_{5} \right)$,
    \item $\left( \widetilde{\theta}_{1}^{s_{1}}, \widetilde{\theta}_{2}^{s_{2}}, \widetilde{\theta}_{3}^{1}, 0, \widetilde{\theta}_{5} \right), \left( \widetilde{\theta}_{1}^{s_{1}}, \widetilde{\theta}_{2}^{s_{2}}, \widetilde{\theta}_{3}^{2}, 0, \widetilde{\theta}_{5} \right), \ldots, \left( \widetilde{\theta}_{1}^{s_{1}}, \widetilde{\theta}_{2}^{s_{2}}, \widetilde{\theta}_{3}^{s_{3}}, 0, \widetilde{\theta}_{5} \right)$,
    \item $\left( \widetilde{\theta}_{1}^{s_{1}}, \widetilde{\theta}_{2}^{s_{2}}, \widetilde{\theta}_{3}^{s_{3}}, 0, \widetilde{\theta}_{5}^{1} \right), \left( \widetilde{\theta}_{1}^{2}, \widetilde{\theta}_{2}^{s_{2}}, \widetilde{\theta}_{3}^{s_{3}}, 0, \widetilde{\theta}_{5}^{2} \right), \ldots, \left( \widetilde{\theta}_{1}^{s_{1}}, \widetilde{\theta}_{2}^{s_{2}}, \widetilde{\theta}_{3}^{s_{3}}, 0, \widetilde{\theta}_{5}^{s_{5}} \right)$
\end{enumerate}

共需 $s_{1} \times s_{2} \times s_{3}$ 步。

按照 1)、2)、3)、4) 的顺序步进,可以较好地避障碍物。因为这个时候,机械臂 $BC$,$CD$ 在一条直线上,$DE$ 保持与 $CD$ 垂直,进过 2) 以后,$BC$,$CE$,$DE$ 达到了较好的高度,对于 $BCE$ 直角臂下的障碍物可以避碰。如图 (4) 所示

\begin{tikzpicture}
    \draw (0,0) -- (6,0) -- (6,4) -- (3,6) -- cycle;
    \draw (0,0) -- (3,4) -- (6,4);
    \node at (0,0) [below left] {A};
    \node at (3,4) [above] {C};
    \node at (6,0) [below right] {B};
    \node at (6,4) [right] {E};
    \node at (3,6) [above] {D};
\end{tikzpicture}

\begin{figure}[h]
    \centering
    \includegraphics[width=0.8\textwidth]{image.png}
    \caption{图 (4)}
    \label{fig:4}
\end{figure}

如果不考虑避碰问题,则 1)、2)、3)、4) 可以同时进行。

\subsection{4.3 牛顿迭代法}

由于机械臂指尖轨迹是离散的,而要求指尖沿着预先指定的一条空间曲线 $x = x(s), y = y(s), z = z(s), a \leqslant s \leqslant b$ 移动,需先将曲线抽样成离散点,再运用牛顿迭代法 \cite{ref2} 求可行解,使得指尖逼近抽样点。

假设机械臂指尖姿态为 $f(\Theta)$,$\Theta = (\theta_1, \theta_2, \theta_3, \theta_4, \theta_5)$。对于空间任一目标点 $Q(x_Q, y_Q, z_Q)$,建立运动学方程组:

\[
\begin{cases}
X(\Theta) = x_Q \\
Y(\Theta) = y_Q \\
Z(\Theta) = z_Q
\end{cases}
\]

利用牛顿迭代法求解此方程,迭代初始点为 $\Theta^0 = (\theta_1^0, \theta_2^0, \theta_3^0, \theta_4^0, \theta_5^0)$。具体步骤如下:

从 $T = 0$ 开始,

1) 对每一个方程的左边进行一阶泰勒展开,展开点为 $\Theta^T = (\theta_1^T, \theta_2^T, \theta_3^T, \theta_4^T, \theta_5^T)$,

例如:对 $X(\Theta)$ 进行一阶泰勒展开:

\[
X(\Theta) = X(\Theta^T) + \left( \theta_1 - \theta_1^T \right) \frac{\partial X(\Theta)}{\partial \theta_1} \bigg|_{\Theta = \Theta^T} +
\]

\[
\left( \theta_2 - \theta_2^T \right) \frac{\partial X(\Theta)}{\partial \theta_2} \bigg|_{\Theta = \Theta^T} \cdots + \left( \theta_5 - \theta_5^T \right) \frac{\partial X(\Theta)}{\partial \theta_5} \bigg|_{\Theta = \Theta^T}
\]

2) 对 1) 中的泰勒展开式,以 $\theta_i (i = 1, 2, 3, 4, 5)$ 为变量,建立方程组:

\[
\mathbf{A} \boldsymbol{\theta} = \mathbf{B}
\tag{2}
\]

对方程组 (2) 中的系数矩阵 $\mathbf{A} = \left[ \mathbf{A}_1 \quad \mathbf{A}_2 \quad \mathbf{A}_3 \quad \mathbf{A}_4 \quad \mathbf{A}_5 \right]$ 进行遍历查找,查找所有 $3 \times 3$ 大小的且行列式不为 0 的矩阵集 $\Gamma$($\Lambda$ 的列向量为 $\mathbf{A}$ 的列向量的一个排列),找出 $\Gamma$ 中行列式最大的矩阵,记为 $\Lambda$。假设 $\Lambda = \left[ \mathbf{A}_i \quad \mathbf{A}_j \quad \mathbf{A}_k \right], i, j, k = 1, 2, 3, 4, 5$,且 $i, j, k$ 互不相等,则集合 $\{1, 2, 3, 4, 5\}$ 中未被 $i, j, k$ 取中的值记为 $p, q$。

3) 将 $\Theta = (\theta_1, \theta_2, \theta_3, \theta_4, \theta_5)$ 中的 $\theta_i, \theta_j, \theta_k$ 作为变量,对 $\theta_p, \theta_q$ 赋值,$\theta_p = \theta_p^T, \theta_q = \theta_q^T$,建立自由度为 3 的方程组:

\[
\Lambda \left[ \begin{array}{ccc}
\theta_i & \theta_j & \theta_k
\end{array} \right] = \mathbf{B}'
\tag{3}
\]

由方程组 (3) 解出 $\theta_{i}, \theta_{j}, \theta_{k}$,其解记为 $\theta_{i}^{*}, \theta_{j}^{*}, \theta_{k}^{*}$,集合

\[
\boldsymbol{\Theta}^{\mathrm{T}+1}=\left\{\theta_{i}^{\mathrm{T}+1}=\theta_{i}^{*}, \theta_{j}^{\mathrm{T}+1}=\theta_{j}^{*}, \theta_{k}^{\mathrm{T}+1}=\theta_{k}^{*}, \theta_{p}^{\mathrm{T}+1}=\theta_{p}^{\mathrm{T}}, \theta_{q}^{\mathrm{T}+1}=\theta_{q}^{\mathrm{T}}\right\} .
\]

4) 判断 $\boldsymbol{\Theta}^{\mathrm{T}+1}$ 和 $\boldsymbol{\Theta}^{\mathrm{T}}$ 误差是否在误差容限内,若是,则停止迭代,如果 $\boldsymbol{\Theta}^{\mathrm{T}+1}$ 的参数 $\theta_{i}^{\mathrm{T}+1}$ 超出其限制范围,则重新选择迭代初始点 $\boldsymbol{\Theta}^{0}$,令 $\mathrm{T}=0$,转向步骤 1),若 $\theta_{i}^{\mathrm{T}+1}$ 在其范围内,输出最终的 $\boldsymbol{\Theta}^{\mathrm{T}+1}$,停止算法。若 $\boldsymbol{\Theta}^{\mathrm{T}+1}$ 和 $\boldsymbol{\Theta}^{\mathrm{T}}$ 不相等,$\mathrm{T}=\mathrm{T}+1$,转向步骤 1)

根据迭代停止的 $\boldsymbol{\Theta}^{\mathrm{T}+1}$,计算出此时机械手的位置 $\left(x_{\mathrm{E}}^{\Theta}, y_{\mathrm{E}}^{\Theta}, z_{\mathrm{E}}^{\Theta}\right)$ 按步进精度 $0.1^{\circ}$,调整 $\left(x_{\mathrm{E}}^{\Theta}, y_{\mathrm{E}}^{\Theta}, z_{\mathrm{E}}^{\Theta}\right)$。机械手最终到达位置与目标点之间的距离为:

\[
d=\sqrt{\left(x_{Q}-x_{E}^{\Theta}\right)^{2}+\left(y_{Q}-y_{E}^{\Theta}\right)^{2}+\left(z_{Q}-z_{E}^{\Theta}\right)^{2}}
\]

\section{5. 机械臂避碰问题}

在实际情况中,机械臂的运动空间可能存在障碍物,所以在从初始位置到达目标位置的过程中,需考虑避开障碍物的问题,不仅要考虑机械臂指尖避碰,也要考虑机械臂连杆避碰。并且,障碍物的外表形状对于机械臂避碰运动的影响很大,未知的或者不规则的外表形状增加了问题的复杂度,通过对障碍物的封装,可以降低问题求解的难度:

对于不规则外表的障碍物,沿与高垂直的方向做切平面,作一个使得不规则的切平面全部位于圆内的半径最小的圆,如果每个切平面上的圆的圆心都在一条垂直于底面的直线上,那么这些位于不同高度切平面上的圆就形成了一个外表规则的圆台,这样考虑简化了问题,并且由于所作的圆台是包含障碍物的最小圆台,所以在大多数情况下,由此规划的路径误差还是在一个可以接受的范围内。

从前述运动学逆问题模型中可知,可以将运动轨迹的分解为一系列的点,进行求解,所以在机械臂避碰问题中还是先考虑点精确定位过程中的避碰问题。

自由空间机械臂指尖定位问题的目标函数为:

\[
\min d^{2}=\left(x_{Q}-x_{E}^{\Theta}\right)^{2}+\left(y_{Q}-y_{E}^{\Theta}\right)^{2}+\left(z_{Q}-z_{E}^{\Theta}\right)^{2}
\]

对于有障碍的问题需要再增加约束条件。

设机械臂在 $\boldsymbol{\Theta}=\left(\theta_{1}, \theta_{2}, \theta_{3}, \theta_{4}, \theta_{5}\right)$ 姿态时对应的空间折线段为 $l_{\Theta_{o}}$,第 $i$ 个障碍物外表面为 $S_{i}$,若有 $n$ 个障碍物,则约束条件可以表示为:

\[
s.t. \quad l_{\Theta_{o}} \cap S_{i}(i=1, \ldots, n)=\varnothing
\]

对于运动轨迹避碰问题,一般首先构造一条连接初始点与目标点,与障碍物不相交的折线,然后在每个分段点上进行带约束条件的求解,理论上说是可以求解避碰的运动轨迹问题。

定义:如果对点 \( P \) 存在一个姿态,机械臂与障碍物无碰撞,则称 \( P \) 有一个良好的姿态。假设初始点为 \( \text{Src} \),目的点 \( \text{Dst} \)。

(1) 构造一条连接 \( \text{Src} \) 和 \( \text{Dst} \) 的折线 \( L(P_0P_1\ldots P_n) \), \( P_0 = \text{Src} \), \( P_n = \text{Dst} \),折线与障碍物不交,且 \( P_0 \), \( P_1 \), \(\ldots\), \( P_n \),都由一个良好的状态。

\( L(P_0P_1\ldots P_n) \) 构造方法

a. 连接 \( \text{Src} \) 与 \( \text{Dst} \),如果连线与障碍物相交,则求出这两点垂直平分面上的一点 \( P_1 \), \( P_1 \) 不在障碍物,且具有一个良好的状态, \( P_1 \) 与 \( \text{Src} \) 和 \( \text{Dst} \) 之间的距离和最小。如果连线与障碍物不相交,则中间不加入任何途经点。

b. 对 \( \text{Src} \) 与 \( P_1 \), \( P_1 \) 与 \( \text{Dst} \) 采取步骤 a 同样的方法可以得到 \( P_2 \), \( P_3 \) 。继续下去,最终得到 \( P_0 \), \( P_1 \), \(\ldots\), \( P_n \),按顺序重新编序后,不妨仍然记作 \( P_0 \), \( P_1 \), \(\ldots\), \( P_n \),则得到所求的折线。

(2) 无碰撞轨迹求解过程

顺序的求解 \( P_iP_{i+1} \) 的无碰撞运动轨迹。如果对 \( i = 0 \), \(\ldots\), \( n-1 \) 都能完成,则联合起来从 \( \text{Src} \) 到 \( \text{Dst} \) 无碰撞运动轨迹。如果 \( P_iP_{i+1} \) 对某 \( i \) 不能得到无碰撞运动轨迹,则就以当前点 \( P_i \) 代替 \( \text{Src} \),重复前面的工作。

根据实际情况,一般情况下,可以断言可行解是存在的。

(3) 伪算法

设求从 \( X \) 到 \( Y \) 之间的无碰撞轨迹问题算法为 \( \text{Orb}(X, Y) \),求 \( X \), \( Y \) 垂直平分面上的一点 \( P \), \( P \) 不在障碍物,且具有一个良好的状态, \( P \) 与 \( X \) 和 \( Y \) 之间的距离和最小的算法为 \( M(X, Y) \)。

\textbf{Orb}(X, Y)
\{
\quad \text{If}(\text{Orb}(X, Y) \text{不是空集})
\quad \quad \text{输出从 } X \text{ 到 } Y \text{ 的无碰撞轨迹,返回。}
\quad \text{Else}
\quad \quad P = M(X, Y);
\quad \quad \text{If}(|PX| + |PY| < \text{制定精度})
\quad \quad \quad \text{无解返回;}
\quad \quad \text{else}
\quad \quad \quad \text{Orb}(X, P) \cup \text{Orb}(P, Y)
\quad \quad \text{End if;}
\quad \text{End if;}
\}

\textbf{Main}()
\{
\quad \text{Orb}(\text{Src}, \text{Dst});
\}

\section{6. 具体问题的求解}

针对题目给出的具体数据,应用上述模型进行计算。

圆台外、内表面壁的曲面方程为:
\begin{align*}
\text{外表面: } & 2z = 168 \times 5 - 5 \times \sqrt{(x - 210)^2 + y^2} \\
\text{内表面: } & 2z = 160 \times 5 - 5 \times \sqrt{(x - 210)^2 + y^2}
\end{align*}

空间示意图如图(5):

\begin{figure}[h]
    \centering
    \includegraphics[width=0.8\textwidth]{image.png}
    \caption{}
    \label{fig:5}
\end{figure}

A、B、C、D、E 表示机械臂关节,机械臂初始位置在 $yoz$ 平面上。图中,圆台上的小正方形代表需要焊接的内表面点,曲线代表需要焊接的外裂纹,*代表裂纹的最低点 $L1(52.5, 0, 26.25)$ 与最高点 $L2(315, 0, 157.5)$。

圆台的裂纹方程为:

\begin{align*}
x &= \sqrt{\frac{826875}{48}} * \cos(t) + \frac{735}{4} \\
y &= \sqrt{\frac{826875}{50}} * \sin(t) \\
z &= \frac{1}{2} \left( \sqrt{\frac{826875}{48}} * \cos(t) + \frac{735}{4} \right)
\end{align*}

需要焊接的四个点的空间坐标为:$(320, -104, 21.5492)$, $(120, 106, 52.3651)$, $(190, -125, 83.5253)$, $(255, 88, 152.9044)$。

\subsection{6.1 问题①的求解}

应用几何解法求机械手从初始位置 $(0, 510, 75)$ 移动到工具位置 $(20, -200, 120)$ 处的指令序列。示意图如图(6)下:

\begin{figure}[h]
    \centering
    \includegraphics[width=0.8\textwidth]{image1.png}
    \caption{}
    \label{fig:6}
\end{figure}

将三维空间投影到 $yoz$ 平面上,如图(7):

\begin{figure}[h]
    \centering
    \includegraphics[width=0.8\textwidth]{image2.png}
    \caption{}
    \label{fig:7}
\end{figure}

应用几何算法计算指令序列,步骤如下:

1) 计算机械手最接近目标的位置

\[
AH = \sqrt{200^2 + 20^2},
\]

定位后 D 的坐标为: \( y'_D = AH \), \( z'_D = 185 \)

\[
BD = \sqrt{AH^2 + DG^2} = \sqrt{200^2 + 20^2 + 45^2} = 5\sqrt{1697}
\]

\[
\angle DBG = arctg \frac{DG}{AH} = arctg \frac{45}{5\sqrt{1697}} = arctg \frac{9}{\sqrt{1697}} \approx 12.6195^\circ
\]

\[
\angle CBD = ar\cos \frac{BD/2}{255} = ar\cos \frac{BD}{510} = ar\cos \frac{5\sqrt{1697}}{510} = ar\cos \frac{\sqrt{1697}}{102} \approx 66.1797^\circ
\]

定位后 C 的坐标为:

\[
y'_C = 255 \cos (\angle DBG + \angle CBD) \approx 255 \cos 78.7992^\circ \approx 49.5333
\]

\[
z'_C = 140 + 255 \sin (\angle DBG + \angle CBD) \approx 255 \sin 78.7992^\circ \approx 390.143
\]

此时:

\[
\theta_2 = 90^\circ - (\angle DBG + \angle CBD) \approx 11.2008^\circ
\]

\[
\theta_3 = 2 \angle CBD \approx 132.359^\circ
\]

\[
\theta_5 = 180^\circ - \angle CBD - (90^\circ - \angle DBG) \approx 36.4398^\circ
\]

\[
\theta_1 = arctg \frac{20}{200} + 90^\circ = 174.289^\circ
\]

按步进精度 \( 0.1^\circ \) 为得到实际的最终状态位置为:

\[
\theta_1 = 174.3^\circ, \, \theta_2 = 11.2^\circ, \theta_3 = 132.4^\circ, \theta_4 = 0^\circ, \, \theta_5 = 36.4^\circ
\]

对于此状态,机械手位置为:

\[
x_E = AH * \sin (\theta_1) \approx 19.985;
\]

\[
y_E = AH * \cos (\theta_1) \approx -199.855;
\]

\[
z_E = 119.986;
\]

工作误差为:

\[
d = \sqrt{(20 - 19.985)^2 + (-200 + 199.855)^2 + (120 - 119.986)^2} = 0.179408 \, \text{mm}
\]

2) 工作过程:

(1) $\theta_{1}, \theta_{3}, \theta_{5}$ 不变, 从 $90^{\circ}$ 开始按精度 $2^{\circ}, 1.9^{\circ}, \ldots$, 步进减小 $\theta_{2}$, 达到 $11.2^{\circ}$;

(2) $\theta_{2}, \theta_{3}, \theta_{5}$ 不变, 从 $0^{\circ}$ 开始按精度 $2^{\circ}, 1.9^{\circ}, \ldots$, 步进增加 $\theta_{1}$, 达到 $174.3^{\circ}$;

(3) $\theta_{1}, \theta_{2}, \theta_{5}$ 不变, 从 $0^{\circ}$ 开始按精度 $2^{\circ}, 1.9^{\circ}, \ldots$, 步进增加 $\theta_{3}$, 达到 $132.4^{\circ}$;

(4) $\theta_{1}, \theta_{2}, \theta_{3}$ 不变, 从 $90^{\circ}$ 开始按精度 $2^{\circ}, 1.9^{\circ}, \ldots$, 步进减小 $\theta_{5}$, 达到 $36.4^{\circ}$;

按照 (1), (2), (3), (4) 的顺序步进, 可以很好地避开障碍物。如果不步考虑避碰问题, 则 (1), (2), (3), (4) 可以同时进行, 步进序列见 answer1.xls, 共需要 89 步指令。

\subsection{6.2 问题②的求解}

\subsubsection{6.2.1 不考虑障碍物以及机械臂转角限制时, 用牛顿迭代法求解}

当不考虑障碍物时, 应用机械臂运动轨迹控制模型, 采用牛顿迭代法求指尖焊接裂纹的指令序列。首先将机械臂指尖从初始状态 $f(\Theta^{0})$ 移到裂纹的最低点 L1 $(52.5, 0, 26.25)$, 再从最低点开始沿着裂纹焊接。分为两个步骤:

(1). 从初始状态 $f(\Theta^{0})$ 移到裂纹的最低点 L1 $(52.5, 0, 26.25)$

找出一条可行的曲线从初始状态到 L1 点。本文中取从初始状态到 L1 点的直线作为第一步的运动轨迹, 将此轨迹记为 S1。对 S1 进行离散化, 均匀抽样 50 个点 $\left\{\mathbf{p}_{1}, \mathbf{p}_{2}, \cdots \mathbf{p}_{50}\right\}$, 其中 $\mathbf{p}_{1}$ 为初始状态 $f\left(\Theta^{0}\right)$, $\mathbf{p}_{50}$ 为 L1 点。从 $\mathbf{p}_{1}$ 点开始, 应用牛顿迭代法求解机械臂指尖从 $\mathbf{p}_{i}$ 点到 $\mathbf{p}_{i+1}$ 点的指令序列, 迭代初始点为 $\mathbf{p}_{i}$ 点。

(2). 焊接裂纹

对裂纹进行离散化: 裂纹在 x 轴上的投影范围为 $26.25 \leq x \leq 157.5$, 对 x 轴上的投影进行均匀抽样, 取 201 个离散点 $\left\{\mathbf{q}_{1}, \mathbf{q}_{2}, \cdots \mathbf{q}_{201}\right\}$ 去逼近裂纹, 其中 $\mathbf{q}_{1}$ 为 L1 点, $\mathbf{p}_{101}$ 为 L2 点。从 L1 点开始, 应用牛顿迭代法求解机械臂指尖从 $\mathbf{q}_{j}$ 点到 $\mathbf{q}_{j+1}$ 点的指令序列, 迭代初始点为 $\mathbf{q}_{j}$ 点。

迭代完成后, 将搜索出的步进序列保存在 answer2(1).xls 中。

计算出机器臂实际焊接点, 如图 (8) 所示 (其中的+号代表焊接点):

\begin{figure}[h]
    \centering
    \includegraphics[width=\textwidth]{image1.png}
    \caption{图(8)}
\end{figure}

上图的局部放大如图(9)下:

\begin{figure}[h]
    \centering
    \includegraphics[width=\textwidth]{image2.png}
    \caption{图(9)}
\end{figure}

由图可以看出,焊接点很好的逼近了裂缝,经过误差分析,发现焊接点的平均误差为 0.188mm。

\subsubsection{6.2.2 考虑障碍物时,应用牛顿迭代算法求解}

当考虑平台对机械臂的阻碍以及机械臂各关节弯角限制时,直接利用牛顿迭代法是不行的,因为其在迭代运算过程中未考虑是否发生碰撞,也未考虑机械臂各关节弯角的限制,在这里给出一种改进的利用牛顿迭代法的轨迹控制算法。算法的基本思想是,在固定其中两个关节弯角时,通过调整其他三个关节弯角可以到达的目标点可能是多个,这样可以连续找到多个目标点,在最后到达的目标点调整那两个固定的关节弯角,再利用这一对固定值,搜索此时通过调整其他三个关节弯角可以达到的目标点,如此重复执行,由于在判断目标点是否可达时加入了弯角和碰撞判断因素,因此可以起到考虑机械臂的阻碍以及机械臂各关节弯角限制时的空间曲线轨迹控制。

假设机械臂达到姿态 \((90^\circ, \theta_2, \theta_3, \theta_4, \theta_5)\),可以调整 \((90^\circ, \theta_2, \theta_3, \theta_4, \theta_5)\),使其处于一种既不发生碰撞,又满足机械臂各关节弯角限制要求,在具体实现时,可以微量调整,记此状态为 \(\Theta_0 = (\theta_1^0, \theta_2^0, \theta_3^0, \theta_4^0, \theta_5^0)\),其中 \(\theta_1^0\) 也可以不为 \(90^\circ\),但对于本题中这一特定位置的平台,立 \(\theta_1^0 = 90^\circ\) 便于计算讨论。同时,仍然对裂纹进行离散化,取若干个离散点去逼近裂纹。下面描述算法步骤:

(1) 从初始姿态出发,因为针对本题中特定的圆台 \(\theta_1^0\),\(\theta_3^0\) 对问题的求解最为重要,影响最大,所以这里固定 \(\theta_1^0\),\(\theta_3^0\),对关于另外三个参数的 \(\theta_2, \theta_4, \theta_5\) 位置非线性方程组采用牛顿迭代法求出焊接第一个目标点的机械臂姿态参数 \(\Theta_1\)。牛顿迭代法中的迭代初始值为 \((\theta_2^0, \theta_4^0, \theta_5^0)\),并判断此时没有发生碰壁,\(\Theta_1\) 的各关节弯角是在限制范围内,如果两种条件中有一个不满足,则需要调整 \(\Theta_0\),否则继续。

(2) 在第 \(i\) 个目标点,仍然在 \(\theta_1^{i-1} = \theta_1^{i-2}\),\(\theta_3^{i-1} = \theta_3^{i-2}\) 的前提下,对关于参数 \(\theta_2, \theta_4, \theta_5\) 的位置非线性方程组继续用牛顿迭代法确定下一个目标点 \(\Theta_i\) 的姿态,并判断此时没有发生碰壁,\(\Theta_2\) 的各关节弯角是在限制范围内,直到在确定目标点 \(\Theta_m\) 的姿态时,上述的两个条件有一个不满足,则转向步骤 (3)。

(3) 在目标点调整 \(\theta_2^{m-1}\),\(\theta_4^{m-1}\),\(\theta_5^{m-1}\) 分别等于 \(\theta_2^0\),\(\theta_4^0\),\(\theta_5^0\),同时调整 \(\theta_1^{m-1}\),\(\theta_3^{m-1}\) 的大小,可以分别在 \(\theta_1^{m-1}\),\(\theta_3^{m-1}\) 附近遍历搜索,搜索目标是机械臂在此新的姿态下可以向前连续到达的目标点数最多,如果遍历搜索得到的新姿态全部无法到达下一个目标点,则转向 (4),反之,回到步骤 (2),直到到达最终目的点,或者发现下一个无法到达的目标点 \(\Theta_j\),回到步骤 (3)。

(4) 报告发现不可达点,等待下一步的指令。

在利用上述算法解决问题 2 时,采用的裂纹离散化方法是:

令 \(t = \frac{\pi}{360} * i\),\(i\) 依次从 1 到 720 递增,这样得到 720 个离散的目标焊点,经过计算验证,取

\(\Theta_{0}=\{85^{\circ},-45^{\circ},-110^{\circ},0,-18^{\circ}\}\),下表是 \(i\) 取 1 到 360 时部分中间结果,如表(1):

\begin{table}[h]
\centering
\caption{表(1)}
\begin{tabular}{|c|c|c|}
\hline
第 \(i\) 次调整 & 调整后的固定角 \((\theta_{1}^{i}, \theta_{3}^{i})\)(单位:度) & 此时可以连续到达的目标点数(单位:个) \\
\hline
1 & \((85, -110)\) & 88 \\
\hline
2 & \((75, -120)\) & 62 \\
\hline
3 & \((65, -130)\) & 88 \\
\hline
4 & \((55, -135)\) & 89 \\
\hline
5 & \((50, -138)\) & 23 \\
\hline
6 & \((59, -138)\) & 8 \\
\hline
\end{tabular}
\end{table}

在计算曲线最低点 \((52.5, 0, 26.25)\) 时发生了找不到满足条件的解的情况,此时需要人工调整继续计算的初始值,本文在计算中使用的方法是原路返回,沿曲线的另一个方向运动,这样当从另外一个方向到达最低点时,表示裂缝修补完成。具体数据见 answer2(2).xls。

\subsubsection{6.2.3 对最低焊点的讨论}

\begin{figure}[h]
\centering
\includegraphics[width=0.8\textwidth]{image.png}
\caption{图(10)}
\end{figure}

得出最低焊点的坐标,由于机械臂的参数角有一定范围控制,因此讨论最低焊点是否为盲点问题,如(10)图所示,ABDE 分别为机械臂的关节点,假设 E 点已经处在焊点位置上,这样 E 为定点,连接 BD 做辅助线,可以得出 D 点在以 B 为球心,BD 长为半径的球,与 E 点为球心,ED 长为半径的球的交线上,其坐标方程满足为:

\[
\left\{
\begin{aligned}
(x_{D} - 52.5)^{2} + y_{D}^{2} + (z_{D} - 26.25)^{2} &= ED^{2} \\
x_{D}^{2} + y_{D}^{2} + (z_{D} - 140)^{2} &= BD^{2}
\end{aligned}
\right.
\]

并且假设 O 为 DE 与圆台的交点,则肯定在 DE 上,因此 O 点坐标满足如下方程:

\begin{align*}
y'_{E} &= BC\cos(\theta_{2}) + CD\cos(\theta_{2} + \theta_{3}) + DE\cos(\theta_{2} + \theta_{3} + \theta_{5}) \\
z'_{E} &= BC\sin(\theta_{2}) + CD\sin(\theta_{2} + \theta_{3}) + DE\sin(\theta_{2} + \theta_{3} + \theta_{5})
\end{align*}

其中若 $0 < \lambda \leq 1$,则 O 点在 DE 线段上。

有上述方程给定 BD 和 $z_{D}$ 的值可以求得 $\lambda$ 以及交点的坐标。

判定 E 点可以在不碰到外壁的情况下接触到焊点的条件是:

(1) $\lambda$ 大于 1 或小于 0

(2) D 点必须在圆台的外部

(3) $z_{D}$ 必须大于 0

因此,由于 BCD 的夹角最小为 42 度,可以求的 BD 的最小值为 182.768 毫米,利用计算机仿真,不停控制 BD 和 $z_{D}$ 的迭代值,在所有的解空间中找出满足条件时的 BD 值最大为 156.8 毫米,这样可以得证,在焊接过程中外壁碰撞问题时,最低的焊点为盲点,即机械臂指尖不可到达点。

\subsection{6.3 问题③的求解}

\subsubsection{6.3.1 对内表面四个焊接点的求解}

基于碰撞问题的情况的不确定性和复杂性的缘故,直接运用上述模型进行求解不仅耗时,而且可能陷入算法无法在多项式时间内运算结束的困境。因此采用计算机随机选取的蒙特卡罗方法进行解空间的判定和搜索。在文章第 6 部分中,已经叙述了将不规则的障碍物规划成规则的圆台问题。因此我们就从圆台的角度进行考虑。

该目标函数是具有六个参数的,因此如果让计算机漫无目的的进行查找,显然很难在一定很短的时间内得到一组满足不碰撞这个约束条件的可行解,确定一个搜寻方向至关重要。以本题为例,目的是在圆台内部进行焊接工作,并且是从小口进入,机械臂的中间臂长都比圆台的上下半径大很多。这样 BC, CD 杆在到达目的点的过程中,很大可能会造成碰壁。由于 AB 的高度比圆台低,直观可以看出如果进行内壁焊接的话,有两个步骤需要进行:

步骤 1:在碰壁条件约束的情况下,将机器工作端 E 先进入圆台内部。

步骤 2:进行内部点的焊接操作。

须注意,步骤 1 中的碰撞考虑,主要是针对圆台外壁。步骤 2,针对的是圆台内壁。经过分析,先得出以下结论,用于缩小计算机随机搜索的方向。

(1) 机械臂 C 关节处于圆台上方,使得 E 点进入圆台内部的可行参数选择范围更大。

圆台在 yoz 平面内的投影,如图 (11) 所示

\begin{figure}[h]
\centering
\includegraphics[width=0.8\textwidth]{image11.png}
\caption{圆台在 yoz 平面内的投影}
\end{figure}

\begin{figure}[h]
    \centering
    \includegraphics[width=0.8\textwidth]{image1.png}
    \caption{图 (11)}
\end{figure}

(2) 将 $DE$ 的臂蜷缩为最大限制角度时,易于放入。如图 (12) 所示:

\begin{figure}[h]
    \centering
    \includegraphics[width=0.8\textwidth]{image2.png}
    \caption{图 (12)}
\end{figure}

(3) 考虑 $ABCD$ 平面内 $E$ 点到达二维指定点的情况,应当首先将机械臂 $ABCD$ 平面旋转到与 $y=0$ 平面重合,此时 $ABCD$ 平面经过圆台的轴心,此时进入圆台内部的 $E$ 点在空间活动范围比较大。

由上述的分析,归纳以下随机搜索算法:

步骤 1:将机械臂 $BC$ 抬至圆台上方附近。

步骤 2:旋转 $\theta_{1}$ 角 90 度,使得 $ABCD$ 平面经过圆台轴心线。

步骤 3:保持 $\theta_{4}$ 为初始值不变,调 $\theta_{5}$ 至最大限制角,以上步骤如图 (13) 所示,以此姿态作为一个参数初始值。

\begin{figure}[h]
    \centering
    \includegraphics[width=0.8\textwidth]{image3.png}
    \caption{图 (13)}
\end{figure}

步骤 4:在焊点附近,找出相应的邻近点,邻近点选在 \(ABCD\) 平面上。

步骤 5:找出台口附近的点作为目标点 \(O\),利用牛顿迭代法,求出目的点参数值。

步骤 6:再以 \(O\) 点为起始值向邻近点靠近求出邻近点参数值。

步骤 7:接着使用牛顿迭代法对 \(E\) 点到焊点进行角度微调,注意由于是微调,因此碰壁的可能性很小。

步骤 8:焊完一个点后,按原路返回原来的邻近点,在 \(ABCD\) 平面内移至下一个邻近点,若还有焊点未焊,返回步骤 7,否则转至步骤 9

步骤 9:以计算机仿真为基础,调整参数值使得每次步进值以最大能够移动的角度为准,每次调整一次,进行碰撞检测,若到达目的点时,无发生碰撞,则指令序列便可求出,若不行,在碰撞点处倒退一步以较小步进值继续进行检测,碰撞退回次数可以自行设定。直到可以达到目的参数值或者达到收敛次数。

步骤 10:若由收敛结束的程序,继续调整初始步进值进行求解,直到求出可行解。

\begin{figure}[h]
    \centering
    \includegraphics[width=0.8\textwidth]{image.png}
    \caption{图 (13)}
\end{figure}

按照以上算法求出机械臂指尖的最终姿态和误差如表 (2):

\begin{table}[h]
    \centering
    \caption{表 (2)}
    \begin{tabular}{|c|c|c|c|c|}
        \hline
        & 需焊接位置 & 选择的临近点 & 实际焊接位置 & 误差(单位:mm) \\
        \hline
        A & (255,88,152.9044) & (210,0,175) & (254.84,88.25,152.91) & 0.2969 \\
        \hline
        B & (190,-125,83.52257) & (210,0,81) & (190.08,-124.86,83.82) & 0.3383 \\
        \hline
        C & (120,106,52.3651) & (210,0,55) & (120.16,106.31,52.40) & 0.3504 \\
        \hline
        D & (320,-104,21.5492) & (210,0,35) & (320.25,-104.06,21.67) & 0.2841 \\
        \hline
    \end{tabular}
\end{table}

\subsubsection{6.3.2 圆台内表面盲区的分析}

由于机械臂的连杆长度固定,每个关节的旋转范围受限制,圆台内表面必有机械臂指尖不能达到的地方,即为机械臂工作的盲区。在这些因素中,对盲区影响比较大的是连杆的长度与关节 \( C \) 的转角 \( \theta_5 \) 的活动范围。而连杆的长度是固定的,所以盲区主要受 \( \theta_5 \) 的活动范围限制。 \( \theta_5 \) 的活动范围为 \( -138^\circ \sim 138^\circ \),则盲区点 \( (x_n, y_n, z_n) \) 满足 \( 42 \leq x_n \leq 210 \),关键是求 \( z_n \) 的范围。其算法如下:

以平面族 \( y = t \) (\( -168 \leq t \leq 168 \))去切圆台,对于圆台的每一个剖切面,作如下分析:

将机械臂平面 \( ABCD \) 移到剖切面上,调整 \( \theta_2 \) 使连杆 \( BC \) 恰与圆台相交,再调整 \( \theta_3 \) 到最大限制范围,以关节点 \( D \) 为圆心,连杆 \( DE \) 长度为半径画圆,求出与剖切面内表面线段的上下交点 \( S \)、\( R \),若下交点 \( R \) 存在,则此时 \( R \) 为机械臂指尖在此平面上所能达到的最低点,\( R \) 点以下的部分为不可达点,若下交点 \( R \) 不存在,则此内表面线段上的点都可达,如图(13)。(上交点 \( S \) 以上的区域可以通过调整连杆 \( BC \) 达到)。

\begin{figure}[h]
    \centering
    \includegraphics[width=0.8\textwidth]{image.png}
    \caption{图(14)}
\end{figure}

以 \( y = 0 \) 平面切圆台为例计算该剖切面上的不可达点:

将连杆 \( BC \) 调整到与圆台表面相切,此时 \( \theta_2 = 70^\circ \)。调整 \( \theta = -138^\circ \),此时 \( D \) 的位置为 \( (118, 0, 2) \)。以 \( D \) 为圆心,连杆 \( DE \) 的长度为半径画圆,与内表面线段的交点为 \( R(0.61, 0, 0.26) \) 和 \( S(0.68, 0, 0.44) \),则此剖切面上 \( R \) 点以下的内表面线段为不可达点。

通过求出平面族 \( y = t \) (\( -168 \leq t \leq 168 \))的剖切面上所有不可达点即可得到机械臂指尖的盲区。

\section{7. 模型的分析}

几何算法简单、直观,精确度较高,经过对多个空间点的定位,最后计算误差,发现平均误差为 \( 0.11 \) mm。几何算法不仅适用于自由空间,也适用于有障碍物的空间。

牛顿迭代法中初始展开点的选择对于结果的影响很大,初始迭代值选择得不好会导致参数值超出限制范围。因此,当参数超出范围时需重新选择初始迭代点。在自由空间中,牛顿迭代法的精确性较高,最终的定位点与目标点之间的误差范围很小,在 $0.1\mathrm{mm}$ 数量级范围内。

对于问题三算法的分析,首先对杆的初始状态进行了分析,找出最大可行角度作为初始位置,接着利用了三次点的转移,由初始点到洞口点,由洞口点到焊点邻近点,再由焊点邻近点微调至焊点,对运动路径作了一定的限制,使得在每次步进过程中碰撞的可能性尽可能降到最小,再根据移动指令越少越好的原则选择步进值,进行计算机仿真判断,基本上可以得出一个满足条件限制的可行解。

但是这种算法由于有很大的随机性有可能造成解不精确,因为做了三次点的转移,每次的目的点的参数值,由初始参数影响较大。算法在进行过程中还需知道障碍物的具体位置和几何表达式,适用范围不是很广,不能在所有问题上较快得出解,特别在几何位置上比较奇异的位置,本身可行解就很少,在这种算法下有可能很难得出解。

由于理解误差,把壁厚 $8\mathrm{mm}$ 理解成内外半径之差,可能对我们的解有影响。但是,只要相应修正算法中的参数,仍然可以达到良好的效果。

\section{对机械臂制造厂家的建议}

制造厂家希望通过修改各条连杆的相对长度以及各关节最大旋转角度等设计参数提高机械臂的灵活性和适用范围。

机械臂的动作形态是是由三种不同的单位动作——旋转、回转、伸缩组合而成的。伸缩使改变机械臂的连杆长度,使机械臂产生直线运动。引进伸缩动作,可以扩大指尖的运动范围。

机械臂具有的独立的单位动作组合数称之为自由度。一个物体在三维空间中的位置和姿态分别需要三个独立的坐标才能得到确定。所以,能使手爪达到任意空间位置并具有任意姿态的机械手最少需要六个自由度。自由度是表达机器人通用性、灵活性的主要指标,自由度越多,机械臂越灵活,可达的空间更广泛。当然自由度越多,机械结构越复杂,从而控制也越困难。

机械臂各个连杆长度越小,自由度越多,机械臂越灵活,指尖可达的范围越大,不过此时机械结构复杂,操作指令更复杂,生产成本增加。如果机械臂厂家生产的机械臂主要用于对精度要求较高的场合,则可以考虑增加自由度,引进伸缩关节。而且,为了随时精确地定位机械臂的位置,可以考虑在机械臂上安装测距仪,随时更新当前与目标点的相对位置,调整步进指令,使指尖更快更精确地到达目的点。

对于本文中的机械臂的连杆 $BC$ 上加一个旋转轴,文章中讨论的盲点将可能变成可达点。

\section{参考文献}

[1] 周远清等,智能机器人系统,北京:清华大学出版社,1989 年第一版

[2] J.M. 奥特加等,多元非线性方程组得迭代解法,北京:科学出版社 1983 年第一版