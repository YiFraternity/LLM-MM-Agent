\documentclass[12pt]{article}
\usepackage{ctex}
\usepackage{amsmath,amssymb}
\usepackage{graphicx}
\usepackage{enumitem}
\usepackage{geometry}
\geometry{left=3cm,right=3cm,top=2.5cm,bottom=2.5cm}

\title{机械臂运动路径设计问题}
\author{}
\date{}

\begin{document}

\maketitle

\begin{abstract}
本题为串联六自由度机械臂的运动设计问题。通过各关节角的变化进行机械臂的运动分析与轨迹规划,实现由初始位置到目标位置的逆运动学求解与最佳轨迹设计。对于已知初始姿态和目标点空间位置的逆运动学求解问题,本文提出两种算法:坐标旋转法与混合解析法。将固定目标点对应的指令序列存放于\texttt{answer1.xls}文件中,并对两种算法的计算效率进行了比较。结果表明,混合解析法能够避免机械臂不必要的运动,且运行效率优于坐标旋转法。此外,由于各关节转角范围不同,机械臂的适用范围存在限制(盲区);针对这一点,文中通过计算与绘图给出了该六自由度机械臂的工作空间。
\end{abstract}

\noindent\textbf{关键词:} 旋转变换;演化算法;运动轨迹

\section{问题重述}

自 1959 年美国的英格伯格和德沃尔制造出世界上第一台工业机器人“尤尼梅特”开始,近半个世纪以来,机器人的研制和应用以惊人的速度发展并取得长足的进步。当今世界,机器人的应用领域已十分广泛,包括工业生产、海空探索、医疗康复和军事活动等,此外,机器人已逐渐在医院、家庭和一些服务行业获得应用。从生产车间中的焊接机械手,到水下自治式机器人,从娱乐性的拳击机器人,到伊拉克战场上的无人驾驶机,机器人已经与我们的日常生活息息相关。

机器人通常分为关节式机器人(或称机械臂、机械手、机器人操作臂、工业机器人等)和移动式机器人。一般来说,前者具有更多的自由度,而后者的作业范围则更大一些。

以某型号机器人为例,其示意图见图 1:

\begin{figure}[h]
    \centering
    \includegraphics[width=0.8\textwidth]{robot_structure.png}
    \caption{机器人结构图}
\end{figure}

这种机器人一共有 6 个自由度,分别由六个旋转轴(关节)实现,使机器人的末

\begin{figure}[h]
    \centering
    \includegraphics[width=0.8\textwidth]{robot_diagram.png}
    \caption{机器人的尺寸图}
    \label{fig:robot_diagram}
\end{figure}

机器人关于六个自由度的每一个组合 $\Theta = (\theta_1, \theta_2, \theta_3, \theta_4, \theta_5, \theta_6)$,表示机械臂的一个姿态,显然每个姿态确定顶端指尖的空间位置 $X$: $f(\Theta) \to X$。假定机器人控制系统只能够接收改变各个关节的姿态的关于连杆角度的增量指令(机器指令)$P = \Delta\Theta = (\Delta\theta_1, \Delta\theta_2, \Delta\theta_3, \Delta\theta_4, \Delta\theta_5, \Delta\theta_6)$,使得指尖(指尖——图 2 中的 E 点,具有夹工具、焊接、拧螺丝等多种功能,不过在这里不要求考虑这方面的控制细节)移动到空间点 $X'$,其中各个增量 $\Delta\theta_i$ 只能取到 $-2, -1.9, -1.8, \dots, 1.8, 1.9, 2$ 这 41 个离散值(即精度为 $0.1^\circ$,绝对值不超过 $2^\circ$)。通过一系列的指令序列 $P_1, P_2, P_3, \dots, P_n$ 可以将指尖依次到达位置 $X_0, X_1, \dots, X_n$,则称 $X_0, X_1, \dots, X_n$ 为从指尖初始位置 $X_0$ 到达目标位置 $X_n$ 的一条路径(运动轨迹)。根据具体的目标和约束条件计算出合理、便捷、有效的指令序列是机器人控制中的一个重要问题。我们约定直角坐标系的原点设在图 2 的 A 点,$z$ 轴取为 $AB$ 方向,$x$ 轴垂直纸面而 $y$ 轴则在基座所固定的水平台面上。

\begin{enumerate}
    \item 根据市场需求,机械臂制造厂打算为他们的产品研发一个软件系统,能够直接将用户的运动命令自动转换成机器指令序列。即为这类机器人设计一个通用的算法,用来计算执行下面指定动作所要求的指令序列,并要求对你们算法的适用范围、计算效率以及你们的近似算法所造成的误差和增量 $\Delta \theta_{i}$ 离散取值所造成的误差大小进行讨论(不考虑其他原因造成的误差):
    \begin{enumerate}
        \item 已知初始姿态 $\Phi_{0}$ 和一个可达目标点的空间位置 $(O_{x}, O_{y}, O_{z})$,计算指尖到达目标点的指令序列。
        \item 要求指尖沿着预先指定的一条空间曲线 $x = x(s)$,$y = y(s)$,$z = z(s)$,$a \leqslant s \leqslant b$ 移动,计算满足要求的指令序列。
        \item 在第①个问题中,假设在初始位置与目标位置之间的区域中有若干个已知大小、形状、方向和位置的障碍物,要求机械臂在运动中始终不能与障碍物相碰,否则会损坏机器。这个问题称机械臂避碰问题,要求机械臂末端在误差范围内到达目标点并且整个机械臂不碰到障碍物(机械臂连杆的粗细自己设定)。
    \end{enumerate}
    \item 应用你的算法就下面具体的数据给出计算结果,并将计算结果以三组六维的指令序列(每行 6 个数据)形式存放在 Excel 文件里,文件名定为 answer1.xls,answer2.xls 和 answer3.xls。

    假设在机械臂的旁边有一个待加工的中空圆台形工件,上部开口。工件高 180mm,下底外半径 168mm,上底外半径 96mm,壁厚 8mm。竖立地固定在 $xy$-平面的操作台上,底部的中心在 $(210, 0, 0)$。

    \begin{enumerate}
        \item 要求机械臂(指尖)从初始位置移动到工具箱所在位置的 $(20, -200, 120)$ 处,以夹取要用的工具。
        \item 如果圆台形工件外表面与平面 $x = 2z$ 的交线是一条裂纹需要焊接,请你给出机械臂指尖绕这条曲线一周的指令序列。
        \item 有一项任务是在工件内壁点焊四个小零件,它们在内表面上的位置到 $xy$ 平面的投影为 $(320, -104)$、$(120, 106)$、$(190, -125)$ 和 $(255, 88)$。要求机械臂从圆台的上部开口处伸进去到达这些点进行加工,为简捷起见,不妨不计焊条等的长度,只考虑指尖的轨迹。
    \end{enumerate}
    \item 制造厂家希望通过修改各条连杆的相对长度以及各关节最大旋转角度等设计参数提高机械臂的灵活性和适用范围。请根据你们的计算模型给他们提供合理的建议。
\end{enumerate}

\begin{table}
    \centering
    \begin{tabular}{|c|c|c|c|}
    \hline
    关节 & 变量符号 $\theta_{i}$ & 初始位置 $\alpha_{i}$ & 变量范围 \\
    \hline
    1 & $\theta_{1}$ & 0 & $\pm 180$ \\
    \hline
    2 & $\theta_{2}$ & -90 & $\pm 125$ \\
    \hline
    3 & $\theta_{3}$ & 0 & $\pm 138$ \\
    \hline
    4 & $\theta_{4}$ & 0 & $\pm 270$ \\
    \hline
    5 & $\theta_{5}$ & -90 & $\begin{aligned} -120 \\ +133.5 \end{aligned}$ \\
    \hline
    6 & $\theta_{6}$ & 0 & $\pm 270$ \\
    \hline
    \end{tabular}
    \caption{初始姿态 $\varphi_{0}$ 和动作范围}
\end{table}

\section{问题背景的理论}

机器人的操作可用一个开环关节链来建模,开链的一端固接在基座上,另一端是自由的,安装着末端执行器,以操作物体。工业机器人运动学涉及到机器人手臂相对于固定参考坐标系原点几何关系的分析研究,特别是机器人手臂末端执行器位置和姿态与关节空间变量之间的关系。机器人运动学的两个具有理论和实际意义的基本问题分别是:

\begin{enumerate}
    \item 对一给定的工业机器人运动模型,已知杆件几何参数和关节角矢量 $\theta_{1}, \theta_{2}, \ldots, \theta_{n}$,其中 $n$ 表示自由度数,求机械手末端执行器相对于参考坐标系的位置和姿态。
    \item 已知机器人杆件的几何参数,给定机械手末端执行器相对于参考坐标系的期望位置和姿态(位姿),机械手能否使其末端执行器达到这个预期的位姿?如能达到,机械手有几种不同的状态可满足条件?
\end{enumerate}

第一个问题常称为运动学正问题(直接问题),第二个问题常称为运动学逆问题(解臂形问题)。运动学正问题的解是唯一的,即各个关节变量给定之后,操作臂末端抓手或工具的位姿可以唯一确定。然而运动学逆问题往往具有多重解,甚至也可能不存在解。本题属于运动学的逆问题求解。

由于机器人各杆件可相对于参考坐标系转动和平移,末端执行器的空间总位移是由杆件的角转动和直线平移形成的。因此,Denavit 和 Hartenberg 于 1955 年提出了一个通用的方法,以矩阵代数来描述和表达机器手各杆件相对于固定参考系的空间几何关系,称为 D-H 方法,连杆坐标系称为 D-H 坐标系。此方法用 $4 \times 4$ 齐次变换矩阵描述相邻两刚性杆件的空间关系,把运动学正问题简化为寻求把“手部坐标系”与“参考坐标系”联系起来的 $4 \times 4$ 等价齐次变换矩阵。

一般来说,机器人逆问题可用几种方法求解,最常用的方法是矩阵代数法、迭代法和几何法。

\subsection{关节坐标系的建立与运动学参数描述}

\subsubsection{关节坐标系的建立}

图 2 所示是一个自由度为 6 的机器人的一种图示。它是一个由一系列刚体组成的多关节机器人。对机器人的第 $i$ 个杆件($i=1, 2, \ldots, n-1$)按图 3 来建立第 $i$ 个坐标系。

建立第 $i$ 个坐标系的规则如下:

\begin{enumerate}
    \item $Z_{i}$ 轴与第 $i$ 个关节的转动轴为同一轴,对转动自由度来说 $Z_{i}$ 就是转动轴,对平动自由度,$Z_{i}$ 轴正方向即为关节运动方向;
    \item $X_{i}$ 轴正交于 $Z_{i-1}$ 轴,即 $X_{i} = Z_{i-1} \times Z_{i}$ 服从右手准则。如果 $Z_{i-1}$ 与 $Z_{i}$ 平行,就使 $X_{i}$ 沿它们的公垂线;
    \item $Y_{i} = Z_{i} \times X_{i}$ 构成坐标系,方向服从右手准则。
\end{enumerate}

第 0 个坐标系的位置选在支撑底座上使 $Z_{0}$ 轴与第 1 个关节运动的轴为同轴,$X_{0}, Y_{0}$ 与 $Z_{0}$ 能构成笛卡尔坐标系即可,具体方向没有要求。最后一个坐标系可在开链的末端使得 $X_{n}$ 轴是 $Z_{n-1}$ 轴的正交方向的任何位置。中间的坐标系的确定则按上述规则依次进行。

\begin{figure}[h]
    \centering
    \includegraphics[width=0.8\textwidth]{image.png}
    \caption{运动学参数图示}
    \label{fig:3}
\end{figure}

\subsubsection{多关节机器人的运动学参数描述}

为便于应用 D-H 变换,需要定义与每个关节有关的运动学参数。每个关节由四个参数,即可完全描述任何转动(或平动)关节。如图 3 所示,这四个参数定义如下:

\begin{itemize}
    \item $\theta_i$: 以 $Z_{i-1}$ 轴为转动轴,从 $X_{i-1}$ 轴到 $X_i$ 轴的转角;
    \item $d_i$: 由第 $i-1$ 坐标系的原点到 $X_i$ 轴和 $Z_{i-1}$ 轴的交点处的距离(沿 $Z_{i-1}$ 轴为正向);
    \item $a_i$: 从 $Z_{i-1}$ 轴与 $X_i$ 轴的交点沿 $X_i$ 轴到第 $i$ 个坐标系的原点的偏离距离,即 $Z_{i-1}$ 轴与 $Z_i$ 轴的最短距离;
    \item $\alpha_i$: 以 $X_i$ 轴为转动轴,从 $Z_{i-1}$ 轴到 $Z_i$ 轴的偏差角。
\end{itemize}

对多关节机器人来说,若关节 $i$ 是转动关节,则 $\theta_i$ 是关节变量,而 $d_i$、$a_i$、$\alpha_i$ 是常数;若关节 $i$ 是平动关节,则 $d_i$ 是关节变量,而 $\theta_i$、$a_i$、$\alpha_i$ 是常数。本题给出的机器人关节均为转动关节,$\theta_i$ 为关节变量。

因此,如图 3 所示,两个相邻的坐标系 $i-1$ 和 $i$ 之间有如下关系:将坐标系 $i-1$

\begin{enumerate}
    \item 绕 $Z_{i-1}$ 轴旋转 $\theta_i$ 角,使 $X_{i-1}$ 轴与 $X_i$ 轴同方向(或 $X_{i-1}$ 轴平行于 $X_i$ 轴);
    \item 沿 $Z_{i-1}$ 轴平移 $d_i$ 距离使 $X_{i-1}$ 轴与 $X_i$ 轴一致;
    \item 沿经过上述(1)(2)变换后得到的 $X_i$ 轴平移 $a_i$ 距离,使两原点相同;
    \item 最后绕 $X_i$ 轴旋转 $\alpha_i$ 角,使两坐标系相同。
\end{enumerate}

经过以上变换,就可由坐标系 $i-1$ 得到坐标系 $i$,用齐次变换矩阵(D-H 矩阵)表示为:
\begin{align*}
A_i &= Rot(Z_{i-1}, \theta_i) Trans(0, 0, d_i) Trans(a_i, 0, 0) Rot(X_i, \alpha_i) \\
&= \begin{bmatrix}
\cos\theta_i & -\sin\theta_i & 0 & 0 \\
\sin\theta_i & \cos\theta_i & 0 & 0 \\
0 & 0 & 1 & 0 \\
0 & 0 & 0 & 1
\end{bmatrix}
\begin{bmatrix}
1 & 0 & 0 & 0 \\
0 & 1 & 0 & 0 \\
0 & 0 & 1 & d_i \\
0 & 0 & 0 & 1
\end{bmatrix}
\begin{bmatrix}
1 & 0 & 0 & a_i \\
0 & 1 & 0 & 0 \\
0 & 0 & 1 & 0 \\
0 & 0 & 0 & 1
\end{bmatrix}
\begin{bmatrix}
1 & 0 & 0 & 0 \\
0 & \cos\alpha_i & -\sin\alpha_i & 0 \\
0 & \sin\alpha_i & \cos\alpha_i & 0 \\
0 & 0 & 0 & 1
\end{bmatrix} \\
&= \begin{bmatrix}
c\theta_i & -c\alpha_i s\theta_i & s\alpha_i s\theta_i & a_i c\theta_i \\
s\theta_i & c\alpha_i c\theta_i & -s\alpha_i c\theta_i & a_i s\theta_i \\
0 & s\alpha_i & c\alpha_i & d_i \\
0 & 0 & 0 & 1
\end{bmatrix}
\end{align*}

其中,$cX_i = \cos X_i$, $sX_i = \sin X_i$

上式也可由 4 个 $3 \times 1$ 的向量描述为:
\[
A_i = \begin{bmatrix}
n_i & o_i & a_i & p_i \\
0 & 0 & 0 & 1
\end{bmatrix}
\]

借助于齐次变换矩阵 $A_i$,n 个自由度的多关节机器人的末端可表示为:
\[
T_N = A_1 \times A_2 \times \cdots \times A_N = \begin{bmatrix}
n & o & a & p \\
0 & 0 & 0 & 1
\end{bmatrix}
\]

则末端相对于 $i-1$ 关节坐标系的变换为:
\[
U_i = A_i \times A_{i+1} \times \cdots \times A_N
\]

由上述定义易得:
\[
U_1 = T_N, U_{N+1} = I \text{(单位矩阵)}
\]

矩阵 $U_i$ 也可以表示为:
\[
U_i = \begin{bmatrix}
n_i^u & o_i^u & a_i^u & p_i^u \\
0 & 0 & 0 & 1
\end{bmatrix},
\]
其中 $n_i^u$、$o_i^u$、$a_i^u$、$p_i^u$ 是四个 $3 \times 1$ 的向量。

由此,得到了多关节机器人的运动学参数描述。

\subsection{运动学正问题和逆问题的经典求解思路}

对于运动学正问题,只要根据运动学参数求出对应的齐次变换矩阵,通过公式
\[
T_N = A_1 \times A_2 \times \cdots \times A_N = \begin{bmatrix}
n & o & a & p \\
0 & 0 & 0 & 1
\end{bmatrix}
\]
得到 $T_N$ 中的每一个元素,即可得到唯一的明确的解答。对于运动学逆问题,情况则要复杂得多。

一般来说,运动学逆问题可用多种方法求解,例如反变换法、旋量代数、对偶矩阵、对偶四元数、迭代法和几何法等等。

下面先介绍反变换法。求解运动方程时,我们从 $T_N$ 开始求解关节位置。先使 $T_N$ 的符号表达式的各元素等于 $T_N$ 的一般形式,并据此确定 $\theta_1$,其他 $N-1$ 个关节参数不可能从 $T_N$ 求得,因为所求得的运动方程过于复杂而无法求解它们。我们可以由其他矩阵来求解它们。一旦求得 $\theta_1$ 之后,可由 $A_1^{-1}$ 左乘 $T_N$ 的一般形式,得:
\[
A_1^{-1} T_N = {^1T_N}
\]

上式中,左边为 $\theta_1$ 和 $T_N$ 各元的函数,可用来求解其他各关节变量,如 $\theta_2, \theta_3$ 等。不断地用逆矩阵左乘,可得到下列 $N-2$ 个矩阵方程式:

\begin{align*}
A_2^{-1}A_1^{-1}T_N &= {}^2T_N \\
A_3^{-1}A_2^{-1}A_1^{-1}T_N &= {}^3T_N \\
A_4^{-1}A_3^{-1}A_2^{-1}A_1^{-1}T_N &= {}^4T_N \\
\vdots \\
A_{N-1}^{-1}A_{N-2}^{-1}\cdots A_2^{-1}A_1^{-1}T_N &= {}^{N-1}T_N
\end{align*}

上列各方程的左式为 $T_N$ 和前 $i-1$ 各变量的函数,可用来确定各关节的位置。求解此运动方程组,即得机械手各关节关机变量。

在求机械手关节变量时,逆变换方法是一种通用的方法,但它并没有明确说明如何从几组可能的解中,为某一特定的手臂形态选出一组合适的解,这就要依靠使用者的直觉。在实际中,一种导出相容关节变量解的几何方法更为有用,它为使用者提供了一种为特定的手臂形态选择唯一解的方法。

\begin{figure}[h]
    \centering
    \includegraphics[width=\textwidth]{puma_robot_diagram.png}
    \caption{PUMA 机器人的图示}
    \label{fig:puma_robot}
\end{figure}

下面以一个六自由度的 PUMA 操作机(如图 \ref{fig:puma_robot} 所示)为例说明几何法。运用杆件坐标系和臂几何结构,借助于三个形态标志(臂、肘、腕)可确定 PUMA 式机器人的各种手臂形态。前两个标志与头 3 个关节的解有关,而另一个与后 3 个关节的解有关。对于 6 轴 PUMA 式机器人,前 3 个关节有四种可能的解,而对这四种解中的任何一种,后 3 个关节均有两种可能的解。头两个形态标志使我们能在前三个关节的四种可能解中选定一种解。类似地,第三个形态标志使我们能从后 3 个关节的两种解中选出一种。使用者在解逆问题时,事先选定一组臂形标志。本解法分两步进行。首先要导出由肩到腕的位置矢量。用它来导出前 3 个关节每个关节 $i$($i=1,2,3$)的解,求解时要观察位置矢量在 $x_{i-1}y_{i-1}$ 平面上的投影。求解后 3 个关节,要用已解出的前 3 个关节、$T_{i}$ 和 $A_{i}$(i=4,5,6)的姿态子矩阵、杆件坐标系在 $x_{i-1}y_{i-1}$ 面上的投影来计算关节变量。从几何形态出发,人们不难求得协调相容的手臂解。由相应的判别方程(它们是关节变量的函数)可决定臂形标志,这可作为对关节解的检验。进行必要的修改和调整后,几何方法可被推广用于求解大多数现代工业机器人(转动关节型)的运动学逆问题。

\subsection{机械臂路径轨迹规划}

在机器人运动之前,了解其在路径上有无障碍以及它是否必须沿特定路径运动,很有意义。轨迹规划方法一般是在机械臂初始位置和目标位置之间用多项式函数来“内插”或“逼近”给定的路径,并沿时间轴产生一系列“控制设定点”,供机械臂控制之用。路径端点既可以用关节坐标给定,也可用笛卡尔坐标给定。不过在一般情况下,多数在笛卡尔坐标中给出。

轨迹规划的两种主要方法是关节插值法和笛卡尔空间法。关节插值法设计一系列多项式,以得到平滑的关节轨迹。为提高计算速度和减少额外运动,最好采用低次多项式序列。关节轨迹被分成若干段,每一段轨迹用一个低次多项式拟合。而在笛卡尔空间规划中,最普通的方法是用低次多项式逼近路径来规划关节变量空间的直线路径。

本题属于在路径为曲线情况下的轨迹规划问题,路径上可能有障碍物。

\subsection{Matlab Robotics 工具箱的使用}

Matlab 的 Robotics 工具箱提供了对机器人运动学、动力学和轨迹规划非常有用的函数,有利于对机器人进行运动仿真和结果分析。

该工具箱基于表示串联杆件运动学和动力学模型的通用方法,比如针对 Puma 560 机器人和 Standford 机器人,也可自己定义机器人参数。它的优点是代码非常成熟,为类似算法的实现提供了一个参考。该工具箱提供了函数用于操作向量、齐次变换、四元法这些表示三维空间位置和方向的数据类型。它也能图形化展示机器人的姿态,给出 D-H 参数。

我们可以利用 Matlab 的 Robotics 工具箱确定机械臂的 D-H 参数,进而求解题目要求的运动学逆问题和轨迹规划问题,并进行动态仿真。如图 5、6、7 所示。

\begin{figure}[h]
    \centering
    \includegraphics[width=0.8\textwidth]{image1.png}
    \caption{机器臂的运动仿真图}
\end{figure}

\begin{figure}[h]
    \centering
    \includegraphics[width=0.8\textwidth]{image2.png}
    \caption{机器臂运动到点的三维图}
\end{figure}

\begin{figure}[h]
    \centering
    \includegraphics[width=\textwidth]{image.png}
    \caption{关节 2、3 的运动轨迹曲线}
    \label{fig:joint_curves}
\end{figure}

\section{问题假设}

\begin{enumerate}
    \item 只考虑理想情况下的运动学问题解答,即认为加速度等动力学因素、电机伺服等机械电子因素等处于理想状态,不会影响本题的运动学问题求解。
    \item 由于受机械本身的物理限制,指尖点与目标点间的距离在一定的范围内(如间距在不足一毫米时)时可以认为指尖与目标点是重合的。
\end{enumerate}

\section{问题的分析及求解}

\subsection{问题 1 的分析及求解}

\subsubsection{问题①的计算模型算法思想}

\paragraph{1. 算法一——坐标旋转法}

在对机器人某个关节进行旋转时,我们可以通过坐标变换矩阵以及各个关节点在旋转前的坐标得出关节点在旋转后新坐标系下的坐标,同时得出目标点在旋转后新坐标系下的坐标。由于旋转前和旋转后的某些关节点的转轴也发生了变化,我们需要保存坐标变换后各个关节点的转轴方向。当某个关节沿着转轴进行旋转时,我们只用对需要变换的关节点的坐标乘以相应的旋转矩阵。算法步骤如下:

\begin{enumerate}
    \item 坐标转换

    为了方便计算,每次转动某一个角度后我们把关节点和目标点以及各个关节点的转轴都换算到新的坐标系下进行计算。比如,若对机器人的第三个关节转动,则 A 点和 B 点以及目标点在新的坐标系下需要进行旋转和平移,对 A 点和 B 点的转轴只进行旋转即可,而 C 点、D 点以及 E 点都只需要进行平移,且 C 点、D 点及 E 点的转轴不需要进行变化。这样,我们就可以得到所有关节点、目标点以及各个关节点的转轴在新的坐标系下的坐标。这样我们就不需要记录各个坐标系的变化情况,只需在下一次旋转时,让各个关节点围绕各自的转轴进行变换即可。

    \item 转角选择

    当我们需要把机器人的指尖 E 转到目标点 P 时,如果经过若干次旋转后,E 点和 P 点重合,则完成要求。故 A、B、E、P 这四点最后要处在同一平面内,因此,在选取哪个关节点进行转动以及转动多少时,我们基于遗传算法的思想采用轮盘赌法进行搜索。如果 A、B、E、P 点不在同一平面内,则我们需要转动第一个或第四个关节点,使得旋转第一个关节点和第四个关节点的概率较大并且选取一个较大的旋转角度,比如旋转角度取值在 $-2^\circ$ 到 $-1^\circ$ 或在 $1^\circ$ 到 $2^\circ$。若 A、B、E、P 点在同一平面内,则采用同样的法则使得旋转第二、三、五个关节点的概率较大以及选取一个较大的旋转角度。

    \item 算法终止准则

    我们采用点 E 和目标点 P 之间的距离作为终止准则,如果 E 和 P 的距离小于给定参数 $\alpha$ 则算法停止。我们这里取 $\alpha=0.5$,即 E 和 P 的距离小于 $0.5\,mm$ 时,我们即认为找到了到达目标点 P 的一条路径。

    \item 评判准则

    经过试验验证,我们知道机器人指尖可以通过很多路径到达目标点。因此,我们采用每条路径的各个点的空间位移之和作为评判准则,空间位移等于旋转半径和旋转角度之积。这样,我们便可以在多条路径中选出空间位移较小的路径。
\end{enumerate}

在经典 D-H 坐标变换方法中,我们知道对每个目标点需要相乘多个矩阵,得出转动后的目标点在参考坐标系中的坐标。而在上述所运用的坐标变换方法中,我们把每个目标点乘以一个旋转矩阵得到各个目标点和各个关节点在新的坐标系中的坐标。

\paragraph{2. 算法二——混合解析法}

通过大量的数据测试,我们发现转动 6 个关节点的先后次序对结果不产生影响。由于进行坐标旋转较为复杂,容易出错,因此,本算法不进行坐标旋转,求解过程都是在参考坐标系 $Oxyz$ 中进行。算法步骤如下:

\begin{enumerate}
    \item 求关节点位置

    我们先求解达到目标点 P 时关节点 A、B、C、D、E 的位置,由于坐标系不进行旋转,故点 A 及点 B 的坐标不变,并且欲使机器人指尖达到目标点,则 P 点坐标即为 E 点坐标。因此,我们只需要求出 C、D 点的坐标。由于 A、B、C、D 点始终在同一平面内,且 $BC=255\,mm$,$CD=255\,mm$,$DE=65\,mm$,我们可得到如下方程组:

    \[
    \begin{cases}
    x_C y_D - x_D y_C = 0 \\
    (x_C - x_B)^2 + (y_C - y_B)^2 + (z_C - z_B)^2 = 255^2 \\
    (x_D - x_C)^2 + (y_D - y_C)^2 + (z_D - z_C)^2 = 255^2 \\
    (x_E - x_D)^2 + (y_E - y_D)^2 + (z_E - z_D)^2 = 65^2
    \end{cases}
    \]

    在解此非线性方程组时,我们用演化算法得出一个在各个关节点变化范围内的 C、D 点的初始位置,然后再用牛顿迭代法得到 C、D 点的精确位置,精度可以达到 $10^{-5}\,mm$。

    \item 求旋转角

    根据上一步得出的 A、B、C、D、E 的位置,我们再采用解析几何的方法求得各个关节从初始位置转到各自目标点所要转动的角度 $\theta_1, \theta_2, \ldots, \theta_5, \theta_6$。因此,我们可以把 $\theta_i (i=1,2,\ldots,6)$ 看作各个关节点转到各自目标点所转动的角度的代数和。

    \item 求旋转指令

    由于各个关节点旋转次序对结果不产生影响,我们可以任意生成旋转指令。为了使得旋转指令最少,我们让最初指令每个需要转动的旋转角取最大转角,即取 $-2^\circ$ 或 $2^\circ$。当某个旋转角的所有指令累加和等于我们需要旋转的角度时,让以下指令中此旋转角所对应的旋转角度取 $0^\circ$,直到所有指令的各个旋转角度累加均等于 $\theta_i (i=1,2,\ldots,6)$ 为止。
\end{enumerate}

混合解析法的优势是可以避免某些关节点的无效转动,减少各个关节点的空间位移,并且不需要对坐标系进行旋转。经过数据试验可知,坐标变换法求解机器人指尖到达预定目标点耗时 $1\,\mathrm{s}$ 左右,而混合解析法只需要 $0.1\,\mathrm{s}$ 左右,运行效率大大提高。另外,用坐标变换法求解此类问题需要 $80$ 步左右,而用混合解析法只需要 $60$ 步左右。

\subsubsection{问题②的计算模型算法思想}

由于给定了运行轨迹,而且 $\theta_i (i=1,2,\ldots,6)$ 只能取离散值,因此将轨迹离散化,可以得到 $N$ 个点构成了轨迹的离散形式。将笔尖运动到轨迹的某一点,不妨设为点 $1$,出则开始以第 $n-1$ 点为 $n$ 点的初始状态运行轨迹 $(n \leq N)$,反复迭代就可以在第一问的基础上逐步迭代求出其指令。当然这样的指令不是唯一的,而且初始值点不同及其离散化取值点集不同都对所取指令集合也有很大的影响。由于还要考虑到 $\theta_i (i=1,2,\ldots,6)$ 的最小取值的问题,因此离散化取点不能太过于密集,但是也不能取点数目也不宜过少,过少将加大所求的指令集合的运行轨迹与要求的轨迹的误差范围。

\subsubsection{问题③的计算模型算法思想}

考虑到机器人控制系统只能接收改变各个关节的姿态关于连杆角度的增量指令 $P = \Delta \Theta = (\Delta \theta_1, \Delta \theta_2, \Delta \theta_3, \Delta \theta_4, \Delta \theta_5, \Delta \theta_6)$,其中各个增量 $\Delta \theta_i$ 只能离散取值,精度为 $0.1^\circ$,绝对值不超过 $2^\circ$,我们的算法也只是近似的接近目标点或者曲线轨迹,误差是不可避免的。通过运行算法的程序,可知上述算法计算效率很高。

\subsection{问题 2 的分析及求解}

问题 1 中我们给出了两种通用算法——坐标旋转法和解析法,用于解决到达指定目标点的指令序列求取问题。这里,我们对目标点 $(20, -200, 120)$ 用两种算法分别进行计算可得到指令集合,详细结果见 answer1.xls。

\subsection{问题 3 的分析及求解}

\begin{figure}[h]
    \centering
    \includegraphics[width=0.8\textwidth]{image.png}
    \caption{机器臂的适用范围}
    \label{fig:robot_arm_range}
\end{figure}

根据题中已经给定的条件,如图 \ref{fig:robot_arm_range} 所示,考虑到机械转角限制的影响和杆件的长度,机械臂作业点 $E$ 点只能到达空间一部分区域。下面均以各机械臂转角的极限情况讨论,得其在 $y-z$ 平面的投影区域如图 \ref{fig:robot_arm_range} 阴影部分。首先,以 $B$ 为圆心,以 $BC$、$CD$、$DE$ 长度和 $575\,\text{mm}$ 为半径作圆 $o$,圆 $o$ 所覆盖的区域以外为 $E$ 点不能到达的区域。根据 $\theta_2$ 的初始角度和取值范围,$B$、$C$、$D$、$E$ 保持在一条直线上同时使 $\theta_2$ 取到最值,对应图中的位置分别为 $B$、$C$、$D$、$E$ 及 $B$、$C1$、$D2$、$E2$,且 $AB$ 与 $BCDE$ 所在直线、$AB$ 与 $BC1D2E2$ 所在直线所成夹角均为 $55^\circ$,其位置关于 $Z$ 轴对称。然后让 $CD$ 以 $C$ 为圆心、$CD$ 长度 $255\,\text{mm}$ 为半径旋转至 $CD1$ 处,由于 $\theta_3$ 的限制,可知 $BC$、$CD1$ 的最小夹角为 $42^\circ$,然后 $E1$ 绕 $D1$ 转动可得出 $B$、$E1$、$D1$ 共线情况下的 $BE1$ 长度,从而得到 $E$ 点不能到达的最小范围为此时的以B为圆心、以BE1为半径的圆,计算可得其半径为117.7677mm。再以C点为圆心、以CE长度320mm为半径旋转得$\overline{EE1}$,再以C1为圆心、以C1E2长度320mm为半径旋转得$\overline{E2E3}$。其余区域可以以B点为圆心进行旋转即可覆盖到。因此,根据机械臂转角的限制和杆件的长度,E点所能到达的区域(也就是机械臂的适用范围)在$y-z$平面的投影区域即为图8中的阴影部分。E点所能到达的整个空间区域是图8所示的阴影区域绕Z周旋转所得的空间。

在上述的机械臂的工作范围可知,通过修改各条连杆的相对长度以及各关节最大旋转角度等设计参数提高机械臂的灵活性和适用范围是可以的,但是机械手臂达到一定的范围的旋转角度即可,不必太大,因为这样也扩大不了其适用范围;如图8所示,关节五处的垂直连杆的范围略微缩小还是不会影响其适用范围的;当然过小肯定影响其适用范围的。通过修改关节长度也是能帮助提高机械臂的灵活性和适用范围的。

在模型计算中,有障碍的机械臂运动要受很多的制约,机械手臂能更灵活则可以更精确的执行指令,使得误差更小,也可以使机械手臂能完成更精细的工作。

\section{结果分析及模型评价}

对于已知初始姿态和目标点空间位置(20 -200 120)求指令序列的机器人逆运动学问题,我们给出了两种效率较高的方法——坐标旋转法和解析法,前者需要进行坐标变换,后者利用空间解析向量的关系,经实际运行得出算法结果如下:

\begin{table}[h]
\centering
\begin{tabular}{|c|c|c|c|c|}
\hline
 & 指令序列步数 & 运行时间 & 算法误差 & 空间位移 \\
\hline
坐标旋转法 & 131 & 1.3245s & 0.2789mm & 4689mm \\
\hline
混合解析法 & 65 & 0.0783s & 0.2664mm & 3999mm \\
\hline
\end{tabular}
\caption{两种算法数据分析}
\end{table}

\section{参考文献:}

[1] [美] R.C. 冈萨雷斯,C.S.G. 李 著,杨静昌 等译校. 机器人学 控制·传感技术·视觉·智能 [M]. 北京: 中国科学技术出版社, 1989

[2] 郭明. 多关节机器人误差分析与精度的评价方法. 兵工自动化 [J], 1988, 01:43-52

[3] Peter I. Corke. Robotics TOOLBOX for MATLAB (Release 7.1). http://www.cat.csiro.au/cmst/staff/pic/robot. April, 2002

\section{附件:}

包括各问题的源程序,answer1.xls,随文章附上。

\end{document}