\begin{center}
\textbf{全国第四届研究生数学建模竞赛}
\end{center}

\begin{abstract}
本文对小规模 NP 类邮路规划与邮车调度问题,建立了可精确求解方案的 0-1规划模型,并在满足邮政运输需求的前提下给出了最佳方案。问题一首先以县支局、县局为顶点构建无向赋权佟,建立最短路模型,求解各局问的最短距离阵 D, 其中顶点到自己路程为$\mathbf{0}$;然后在 D 的基础上以$F_{\ddot{y}k}($第$i$条邮路第$j$次是否收发第$\kappa$支局邮件)为决策变量,以邮车工作时间、车辆运载能力为主要约束,建立以总空载损失费用最小为目标的 0-1 非线性规划模型,运用规划软件 Lingo 得到最优或次优解 59.8154 元,具体邮路规划与邮车调度见 5.2.3。问题二考虑到市邮路成本,我们采用分层规划策略,首先以市支局、县局为顶点构建无向赋权图,求解出最短路矩阵 D20,以$F_{ijk}$为决策变量,邮车工作时间为主要约束,建立以邮路运行成本最小为目标的 0-1 非线性规划模型,求解得到最优或次优解 2×2742 元,邮路规划与邮年调度见 6.13;然后,建立各县区的最短路矩阵 D21~D25,同样建立规划模型$\Pi ($i)可求解得到最优或次优解,其中时间约束较复杂见 6.2.1:最后,全市总运行费用为 9549 元,全市邮路规划与邮车调度见 6.2.3。问题三由于县局地理位置不变,对区邮路无影响,故我们以全市各县支局为中心采用逐步最优方法对所有县区支局重新划分,得到的县分区方案见 7.2;然后构建所有新县区无向赋权图, 得到最短路知阵 D31~D35,米用第 2 问规划模型 II(in),得到全市总运行费用为9267 元,邮路规划与邮车调度见 7.3.3,结论是降低成本并不非常明显。所以在第四问中考虑县局迁移,建立了全局选址规划模型,但模型规模较大,我们建立近似的启发式算法完成县局选址(见 8.3),随后以D21~D2 为最短路矩阵,运用模型 II (ii) 求解得到目标值 8997 元,邮路规划与邮车调度见 8.3 表 4.2;在此基础上我们还考思县局拉送邻县较近站点,同样使用模型 II(ii)对新县区划分的邮路规划求解,得到目标值 8865 元,具体方案见表 4.4; 最后还对邮路规划进一步松弛推广与研究。关键字:无向赋权图 O-1 非线性规划  
\end{abstract}

\tableofcontents

\section{问题重述}

邮政运输网络采用邮区中心局体制, 即以邮区中心局作为基本封发单元和网路组织的基本节点, 在此基础上组织分层次的邮政网。邮路是邮政运输网络的基本组成单元, 本题即为对目标区域的邮路规划与邮车调度问题。

下面为本文所要研究的地区的邮政运输流程及时限规定:

\begin{itemize}
    \item Step1: 区级第一班次邮车从地市局D出发将邮件运送到各县局$X_{i}$和沿途支局, 并将各县局$X_{i}$和沿途支局收寄的邮件运送回地市局D; 区级第一班次邮车出发时间必须在06:00之后, 返回地市局D时间必须在11:00之前。
    \item Step2: 县局$X_{i}$将当天区级第一班次邮车及前一天的区级第二班次邮车所送达的本县邮件进行集中处理, 按寄达支局装上相应的县级邮车; 县局$X_{i}$对邮件的集中处理时间为1小时(包括邮件的卸装、分拣封发等处理时间)。
    \item Step3: 各县级邮车将邮件运送到其负责的支局并将这些支局收寄的邮件运送回县局$X_{i}$;
    \item Step4: 区级第二班次邮车从地市局D出发将邮件运送到各县局$X_{i}$和沿途支局, 并将各县局$X_{i}$收寄的邮件(包括当日各县级邮车运回县局$X_{i}$的邮件)和沿途支局收寄的邮件运送回地市局D; 请注意区级第二班次邮车在县局$X_{i}$卸装完邮件后的出发时间必须在县局$X_{i}$的全部县级邮车返回县局并集中处理1小时以后, 最终返回地市局D的时间必须在18:00之前。
\end{itemize}

在上述规定下, 假设区级两个班次邮车的行驶路线相同, 要求区级邮政运输网必须至少覆盖该地市附近的16个支局$Z_{58}, Z_{59}, \ldots, Z_{73}$和5个县局$X_{1}, \ldots, X_{5}$。各县级邮政运输网必须覆盖本县内区级邮车不到达的支局。该地区邮局间公路网分布见表1, 并且县级邮车平均时速为30km/h, 区级邮车的平均时速为65km/h, 邮车在各支局卸装邮件耗时5分钟, 在各县局卸装邮件耗时10分钟。

问题1: 以县局$X_{1}$及其所辖的16个支局为研究对象, 假设区级第一班次邮车08:00到达县局$X_{1}$, 区级第二班次邮车16:00从县局$X_{1}$再出发返回地市局D, 若每辆县级邮车最多容纳65袋邮件, 试问最少需要多少辆邮车才能满足该县的邮件运输需求? 同时, 为提高邮政运输效益, 应如何规划邮路和如何安排邮车的运行? (空车率$=$ (邮车最大承运的邮件量(袋)$-$邮车运载的邮件量(袋)) / 邮车最大承运的邮件量(袋), 单车由于空车率而减少的收入为 (空车率$\times 2$元/公里))

问题2: 采用尽可能少、尽可能短的邮路可以减少邮政部门车辆和人员等的投入, 从而降低全区邮政运输网的总运行成本。考虑投入较好的邮车, 每条邮路只需要一辆邮车即能满足运载能力要求, 试问应如何构建该地区的邮政运输网络(县的划分不变更),请给出邮路规划和邮车调度方案。邮车的调度必须满足上文中有关该地区的邮政运输流程及时限规定。(每条邮路的运行成本为3元/公里)

问题3: 考虑到部分县与县交界地带的支局, 其邮件由邻县县局负责运送可能会降低全区的运行成本, 带来可观的经济效益。若允许在一定程度上打破行政区域的限制, 你能否给出更好的邮路规划和邮车调度方案? (在此同样不必考虑邮车的运载能力的限制, 每条邮路的运行成本为3元/公里)

问题4: 县局选址的合理与否对构建经济、快速的邮政运输网络起到决定性的作用。假设图2中县局$X_{1}, \ldots, X_{5}$均允许迁址到本县内任一支局处, 同时原来的县局弱化为普通支局。设想你是该地区网运部门负责人, 请你重新为各个县局选址, 陈述你的迁址理由并以书面材料形式提交省局网运处。

\section{问题分析}

本题为邮政运输网络中的邮路规划和邮车调度问题,在构建邮运网络时,首要考虑的问题应是怎样在满足该市邮政运输流程及时限要求的前提下,用尽可能少的邮路覆盖该市内所有邮局,同时,需在运输成本尽量低的目标下对邮路进行具体规划及对邮车进行合理调度。

规划的终极目的为使邮政运输效益最大。则首先应使所用邮车数量尽量少,在此基础上,对第一问而言应使由空载率而损失的费用尽量小,其余各问皆为使邮路运行成本尽量低。在网络构建时主要考虑以下几方面约束因素:

(1) 邮车运输时限约束:所有邮车需在运输流程中指定的时间内完成运输任务;

(2) 运输任务约束:邮车须将寄达、寄出各支局的邮件全部运送完;

(3) 邮车运输能力限制:邮车上所载邮件数目不能超过上限(65袋);

(4) 区及县网的构建必须覆盖其行政区域内的所有邮局。

在允许一定程度上打破现有县的划分时,应首先根据就近原则对各县网中支局重新进行归属划分,然后在新支局归属划分下重新对县级邮路网络进行规划并给出具体邮车调度方案。

在第四问中涉及到了县局选址问题,在假设当前县局均允许迁址到本县内任一支局处,同时原来的县局弱化为普通支局,此时应综合考虑区级与县级两级网络,对全市邮运网络进行全局整体构建。

\section{模型假设}

[1] 假设一辆邮车仅负责一条邮路;

[2] 假设一个邮局的邮件仅由一辆邮车运送;

[3] 假设区、级邮车行驶中皆以其平均时速 $65, 30 \mathrm{~km} / \mathrm{h}$ 运行;

[4] 假设区级两个班次邮车的行驶路线相同,但方向可以不相同;

[5] 对于区车覆盖的各县中的支局,两班车允许仅有一班进行邮件装卸。

\section{符号说明}

\begin{itemize}
    \item $d_{kj}$ —— 图中节点 $k$ 到 $j$ 的最短路距离(由最短路模型求得)
    \item $F_{ijk}$ —— $0-1$ 变量,表示第 $i$ 辆车第 $j$ 次装卸邮件是否在第 $k$ 个节点
    \item $T$ —— 表示县级车运输的最大时间范围($T$ 的确定详见 6.2.1/(1))
    \item $G_{k}^{Q}$ —— 第 $k$ 个支局寄出的邮件总量
    \item $G_{k}^{H}$ —— 寄达第 $k$ 个支局的邮件总量
\end{itemize}

\section{X1 所在县邮路规划及邮车调度模型(问题一)}

本问以县局 X1 及其所辖的 16 个支局 $Z1, Z2, \ldots, Z16$ 为研究对象,研究满足该县的邮件运输需求的最少邮车数量,同时,在运输效益最大的目标下,对邮路进行规划并对邮车进行具体安排。

\subsection{模型准备}

1) 问题一模型假设

[1] 假设一辆邮车仅负责一条邮路;

2) 邮路网络的图论模型抽象

引用图论相关知识,将题中所提供的邮路网络抽象成一个连通网络图 $\overline{G}=(V,E,W)$,$\overline{G}$ 中的每个顶点为各级邮政局,如果从 $\overline{G}$ 中的顶点 $V_i$ 到 $V_j$ 有直达路线,那么这两点之间就用有边相连,记做 $(i,j)\in E$,相应的有一个数 $w(v_i,v_j)$ 称为该边的权,其权值赋为两顶点间最短距离。

本问所建立的邮路网络中有多个圈,其中每个圈为一条邮路,各圈务必在 $X_1$ 处相交。建图时将本县内 $Z1, Z2, \ldots, Z16$ 各支局编号为 $1 \sim 16$,$X_1$ 编号为 $17$,即本县邮运网络图中共有 $17$ 个顶点,在此抽象基础上进行分析建模。

\subsection{5.1 邮局间最短路矩阵的建立}

由于题中表 1 仅提供了可直达的两支局间距离,而在邮路规划时,当两支局间无直达线路时需要路过其它支局才能到达。由此,需首先计算任两节点间的最短路径,并进一步求出任两支局间最短距离。在邮路规划及邮车运行安排时任两节点间的路径选择应为最短路。

根据表 1 中提供的图中任两节点间可直达的线路,可得该图的邻接矩阵 $A=(a_{ij})_{18\times 18}$,以 $w_{ij}$ 表示可直达的两点 $i$ 与 $j$ 之间的距离,现需求从顶点 $s \to e$ 的最短路径。设 $0-1$ 决策变量为 $x_{ij}$,当 $x_{ij}=1$ 时,说明弧 $(i,j)$ 位于顶点 $s$ 至顶点 $e$ 的路上;否则,$x_{ij}=0$,则可得求任两顶点 $s \to e$ 最短路的模型如下:

\begin{align*}
H_{ijk} & \quad \text{——第 } i \text{ 辆车第 } j \text{ 次装卸邮件时, 收取第 } k \text{ 个点寄出的邮件数;} \\
Q_{ijk} & \quad \text{——第 } i \text{ 辆车第 } j \text{ 次装卸邮件时, 在第 } k \text{ 个顶点卸下的邮件数;} \\
G_{k}^{Q} & \quad \text{——表示第 } k \text{ 个支局寄出的邮件总量;} \\
G_{k}^{H} & \quad \text{——表示寄达第 } k \text{ 个支局的邮件总量;} \\
F_{ijk} & \quad \text{——0-1变量, 表示第 } i \text{ 辆车第 } j \text{ 次装卸邮件是否在第 } k \text{ 个节点;} \\
d_{kj} & \quad \text{——图中节点 } k \text{ 到 } j \text{ 的最短路距离 (由 5.1 求得的最短路矩阵确定);}
\end{align*}

根据此模型可求解出由 $X_1$ 所在县构成的图中任两顶点间的最短路线矩阵,矩阵中各元素表示任两节点间的最短路线,并可根据路线中各弧的长度及邮车平均时速 $30\,\text{km/h}$ 求得邮车在任两节点间运输的最短距离矩阵 $(d_{ij})_{18\times 18}$,见附录 1.1。

\subsection{5.2 $X_1$ 所在县邮路规划及邮车调度模型}

\subsubsection{5.2.1 模型 I 分析}

本模型建立首先要求确定最少邮车数目,并在邮车数最少的基础上,以运输效益最大为目标(使由于空车率而减少的收入尽量少),对该县邮路进行规划并给出邮车调度方案。

在求解时,需要满足以下邮运约束:

- 邮车运输时限约束:所有邮车需在指定时间内完成运输任务;
- 运输任务约束:邮车须将寄达、寄出各支局的邮件全部运送完;
- 邮车运输能力限制:邮车上所载邮件数目不能超过上限(65 袋);
- 该县级邮政运输网必须覆盖本县内所有支局;
- 各邮车每次仅到一个支局装卸;
- 本县邮车始、终点为 \( X_1 \)。

下面分别对以上约束及目标进行分析,并建立最优化模型求解。

### 约束分析

#### (1) 邮车运输时限约束

由题中关于该区邮政运输流程的规定,可知各县级邮车必须在第一班区级邮车到来后,并对邮件进行集中处理一小时后才能出发,且必须在第二班区级邮车出发前一小时到来(提前一小时到是为了对邮件进行集中处理一小时)。

根据问题一假设,区级第一班次邮车 08:00 到达县局 X1,区级第二班次邮车 16:00 从县局 X1 再出发返回地市局 D;除去前后共 2 小时的邮件处理时间,该县的邮车一天最多可运送 6 小时。

引入 0-1 变量 \( F_{ijk} \) 表示第 \( i \) 辆车第 \( j \) 次装卸邮件是否在第 \( k \) 个支局,即:
\[
F_{ijk} =
\begin{cases}
1 & \text{第 } i \text{ 辆车第 } j \text{ 次装卸邮件在第 } k \text{ 个支局} \\
0 & \text{否则}
\end{cases}
\]

设第 \( i \) 辆车最多经过 \( m \) 个节点,则第 \( i \) 辆车共在 \( \sum_{j=1}^{m} \sum_{k=1}^{16} F_{ijk} \) 个支局进行了装卸邮件工作,题中给出邮车在各支局装卸邮件耗时 5 分钟,则第 \( i \) 辆车在运输过程中耗费在各支局装卸邮件的时间为:
\[
T_1 = \frac{5}{60} \sum_{j=1}^{m} \sum_{k=1}^{16} F_{ijk} \quad \text{(h)}
\]

结合 \( F_{ijk} \) 定义可知,当且仅当 \( F_{ijk} \) 与 \( F_{i,j+1,q} \) 同时为 1 时表示第 \( i \) 辆车由支局 \( k \) 到支局 \( q \),即经过弧 \( (k,q) \),其余情况为不经过。则可用 \( F_{ijk} \cdot F_{i,j+1,q} \) 来表示第 \( i \) 辆是否经过弧 \( (k,q) \),即:
\[
F_{ijk} \cdot F_{i,j+1,q} =
\begin{cases}
1 & \text{第 } i \text{ 辆车经过弧 } (k,q) \\
0 & \text{第 } i \text{ 辆车不经过弧 } (k,q)
\end{cases}
\]

已知各县级车平均行驶速度为 30km/h,以 \( d_{kq} \) 表示第 \( k \) 个支局到第 \( j \) 个支局的最短路距离(由 5.1 求得的最短路矩阵确定),即 \( d_{kq} \) 为弧 \( (k,q) \) 的长度,则第 \( i \) 辆车运输途中耗时为:
\[
T_2 = \sum_{j=1}^{m-1} \sum_{k=1}^{17} \sum_{q=1}^{17} \frac{d_{kq}}{30} F_{ijk} F_{i,j+1,q} \quad \text{(h)}
\]

邮车在运输过程中耗时由在各支局装卸邮件耗时及运输途中耗时两部分构成。

成,即第 \( i \) 辆车邮运过程总耗时不超过运输时限(6 小时)约束可表示为:

\begin{equation}
T_1 + T_2 = \frac{5}{60} \sum_{j=1}^{m} \sum_{k=1}^{16} F_{ijk} + \sum_{j=1}^{m-1} \sum_{k=1}^{17} \sum_{q=1}^{17} \frac{d_{kq}}{30} F_{ijk} F_{i,j+1,q} \leq 6
\tag{1.1}
\end{equation}

\[
(i = 1, \dots, n)
\]

(2) 邮车运载能力限制(隐含运输任务约束)

根据问题一要求,每辆县级邮车最多容纳 65 袋邮件,这里实际上限制了两个方面:其一,每辆车在出发时装载所要运送的所有邮件量,而这些邮件需要在沿途支局全部卸下,所以各车卸下的邮件总量不能超过 65 袋;其二,邮车在经过支局时卸下一部分邮件,同时收取一部分邮件,则在每个分局装卸完成后,各车上所载邮件数不超过 65 袋。

以 \( G_k^H \) 表示寄达第 \( k \) 个支局的邮件总量,第 \( i \) 辆车最多经过节点数为 \( m \),\( F_{ijk} \) 表示第 \( i \) 辆车第 \( j \) 次装卸邮件是否在第 \( k \) 个支局,则假设一个支局的邮件仅由一辆邮车全部运送完,则第 \( i \) 辆车在出发时装载的所要运送的所有邮件量为:

\[
\sum_{j=1}^{m} \sum_{k=1}^{16} F_{ijk} G_k^H
\]

《注》:这里隐含约束了邮车将寄达各支局的邮件全部运送完,即运输任务约束。

邮车在运输途中不断装卸邮件,以 \( G_k^Q \) 表示第 \( k \) 个支局寄出的邮件总量,则经过第 \( k \) 个支局时邮车上邮件数量变化量为 \( G_k^Q - G_k^H \),则第 \( i \) 辆车在出发时及运输途中邮件总量始终不超过邮车运载能力(65 袋)约束可表示为:

\begin{equation}
\sum_{j=1}^{m} \sum_{k=1}^{16} F_{ijk} G_k^H + \sum_{j=1}^{q} \sum_{k=1}^{16} F_{ijk} \left( G_k^Q - G_k^H \right) \leq 65
\tag{1.2}
\end{equation}

\[
(q = 1, 2, \dots, m)
\]

《注》:这里隐含约束了邮车将各支局寄出的邮件全部运送完,即运输任务约束。

(3) 该县级邮政运输网必须覆盖本县内所有支局

由于无法知道区级邮车覆盖的支局,所以在构建该县邮政网络时必须使该县邮车经过该县内所有支局,且一次将该支局的邮件装卸完全。

设最少需要 \( n \) 量邮车,第 \( i \) 辆车最多经过节点数为 \( m \),\( F_{ijk} \) 表示第 \( i \) 辆车第 \( j \) 次装卸邮件是否在第 \( k \) 个支局,则该县级邮政运输网必须覆盖本县内所有支局且仅覆盖一次可表示为:

\begin{equation}
\sum_{i=1}^{n} \sum_{j=1}^{m} F_{ijk} = 1
\tag{1.3}
\end{equation}

\[
(k = 1, 2, \dots, 16)
\]

(4) 各邮车每次仅到一个支局装卸

由于 \( F_{ijk} \) 表示第 \( i \) 辆车第 \( j \) 次装卸邮件是否在第 \( k \) 个支局,则只要经过第 \( k \) 个顶点则 \( F_{ijk} = 1 \)(注意,此处经过某一邮局不一定有邮件的装卸工作,可能仅仅是

路过,则装卸量为零)。则对于 \( X_1 \) 县邮路图中的 17 个顶点而言,第 \( i \) 辆车第 \( j \) 次仅到一个邮局可表示为:
\begin{equation}
\sum_{k=1}^{17} F_{ijk} = 1
\tag{1.4}
\end{equation}
\[
(i=1,\dots,n; j=1,\dots,m)
\]

(5) 本县邮车始、终点为 \( X_1 \)

由题中关于该区邮政运输流程的规定,县局 \( X_1 \) 将邮件处理后通过县级邮车将邮件送往其县内各支局,同时将各支局需要寄送的邮件运回 \( X_1 \) 进行统一处理后配送,则本县级邮车行程的始终点皆为 \( X_1 \)。

在模型准备中将 \( X_1 \) 看作图中编号为 17 的顶点,设第 \( i \) 辆车最多经过节点数为 \( m \),则各邮车由 \( X_1 \) 出发并最终回到 \( X_1 \) 可表示为:
\begin{equation}
F_{i,1,17} = 1 \quad , \quad F_{i,m,17} = 1
\tag{1.5}
\end{equation}
\[
(i=1,\dots,n)
\]

目标分析

第一目标:所需邮车数目最少

根据问题一要求,需求在满足邮运约束下,最少需要的邮车数量。所以应以邮车数量最少为第一目标,再在最少车辆数的基础上,以由空载率损失的费用最小为第二目标对邮路进行规划及邮车运输安排。

由于双目标问题实际中不可解,而本问中研究点数较少,所以车辆数目不会太多,则这里采取将目标一转化为约束的方法使模型可通过编程求解。

具体方法为:以目标二为目标进行建模编程,在模型中以 \( n \) 表示所需最少车辆数,再依次对 \( n \) 赋值 \( 1,2,3,\dots \),则令模型可解的最小 \( n \) 值即为所求最小车辆数。

第二目标:运输效益最大(由空载率损失的费用最小)

在满足邮车数量最少的前提下,应尽量提高运输效益。提高运输效益的方法为尽量降低由于空车率而减少的收入,根据题目信息,空车率的计算表达式为:
\[
\text{空车率} = \frac{\text{邮车最大承运的邮件量(袋)} - \text{邮车运载的邮件量(袋)}}{\text{邮车最大承运的邮件量(袋)}}
\]

由式 (1.2) 的分析及构建过程可知,邮车在运输过程中所运载的邮件量随经过支局的过程而不断变化,可表示为:
\[
\sum_{j=1}^{m} \sum_{k=1}^{16} F_{ijk} G_k^H + \sum_{j=1}^{p} \sum_{k=1}^{16} F_{ijk} \left( G_k^Q - G_k^H \right)
\]

根据问题一要求,每辆县级邮车最多容纳 65 袋邮件,即空载率可表示为:
\[
\frac{65 - \left( \sum_{j=1}^{m} \sum_{k=1}^{16} F_{ijk} G_k^H + \sum_{j=1}^{p} \sum_{k=1}^{16} F_{ijk} \left( G_k^Q - G_k^H \right) \right)}{65}
\]

以 \( F_{ijk} \) 表示第 \( i \) 辆车第 \( j \) 次装卸邮件是否在第 \( k \) 个支局,则 \( F_{ijk} \cdot F_{i,j+1,q} \) 可表示

第 $i$ 辆是否经过弧 $(k,q)$,以 $d_{kq}$ 表示第 $k$ 到第 $j$ 个支局的最短路长度(由 5.1 求得的最短路矩阵确定),则第 $i$ 条线路上的第 $k$ 到第 $q$ 个支局的最短距离可表示为:
\[
\sum_{k=1}^{17} \sum_{\substack{q=1 \\ q \neq k}}^{17} d_{kq} F_{ipk} F_{i,p+1,q}
\]

根据题中关于由空载率而损失的费用的计算式,可得使由空车率而减少的收入最少的表达式为:
\[
Min \quad \sum_{i=1}^{n} \sum_{p=1}^{m-1} \left( 65 - \left( \sum_{j=1}^{m} \sum_{k=1}^{16} F_{ijk} G_{k}^{H} + \sum_{j=1}^{p} \sum_{k=1}^{16} F_{ijk} \left( G_{k}^{Q} - G_{k}^{H} \right) \right) \right) / \left( 65 \times 2 \times \sum_{k=1}^{17} \sum_{\substack{q=1 \\ q \neq k}}^{17} d_{kq} F_{ipk} F_{i,p+1,q} \right)
\]

\subsubsection{5.2.2 模型 I 的建立}

基于模型分析,以使由空车率而减少的收入最少为目标,式 (1.1)~(1.5) 为约束,建立 $X_{1}$ 所在县邮路规划及邮车调度模型如下:

\[
Min \quad \sum_{i=1}^{n} \sum_{p=1}^{m-1} \left( \sum_{k=1}^{17} \sum_{\substack{q=1 \\ q \neq k}}^{17} d_{kq} F_{ipk} F_{i,p+1,q} \times 2 \left( 65 - \left( \sum_{j=1}^{m} \sum_{k=1}^{16} F_{ijk} G_{k}^{H} + \sum_{j=1}^{p} \sum_{k=1}^{16} F_{ijk} \left( G_{k}^{Q} - G_{k}^{H} \right) \right) \right) / 65 \right)
\]

\[
S.T.
\begin{cases}
\frac{5}{60} \sum_{j=1}^{m} \sum_{k=1}^{16} F_{ijk} + \sum_{j=1}^{m-1} \sum_{k=1}^{17} \sum_{q=1}^{17} \frac{d_{kq}}{30} F_{ijk} F_{i,j+1,q} \leq 6 & (i=1,\dots,n) \tag{1.1} \\
\sum_{j=1}^{m} \sum_{k=1}^{16} F_{ijk} G_{k}^{H} + \sum_{j=1}^{q} \sum_{k=1}^{16} F_{ijk} \left( G_{k}^{Q} - G_{k}^{H} \right) \leq 65 & (q=1,\dots,m) \tag{1.2} \\
\sum_{i=1}^{n} \sum_{j=1}^{m} F_{ijk} = 1 & (k=1,\dots,16) \tag{1.3} \\
\sum_{k=1}^{17} F_{ijk} = 1 & \begin{cases} i=1,\dots,n \\ j=1,\dots,m \end{cases} \tag{1.4} \\
F_{i,1,17} = 1 \ ; \ F_{i,m,17} = 1 & (i=1,\dots,n) \tag{1.5} \\
F_{ijk} \in \{0,1\}
\end{cases}
\]

《注》:$n$ 为所需的最少车辆数,根据目标分析中对双目标转化成单目标的方法,在模型求解时依次取 $n=1,2,3\ldots$,使模型有解的最小 $n$ 值即为所需最小车辆数。

### 模型说明:
(1.1) 邮车运输时限约束:所有邮车须在指定时间内完成运输任务;
(1.2) 邮车运输能力限制:邮车上所载邮件数目不能超过上限。同时,邮车须将寄达、寄出各支局的邮件全部运送完约束;
(1.3) 该县级邮政运输网必须覆盖一次本县内所有支局;
(1.4) 各邮车每次仅到一个支局装卸;
(1.5) 邮车由 $X_{1}$ 发出并最终返回 $X_{1}$。

$F_{ijk}$ —— 0-1 变量,表示第 $i$ 辆车第 $j$ 次装卸邮件是否在第 $k$ 个节点;

\subsubsection{5.2.3 模型Ⅰ的求解}

首先, 通过将模型中 \( n \) 从 1 开始由小到大赋值时得出 \( n \) 最小为 3 时模型可解, 即最少需要 3 辆邮车才能满足该县的邮件运输需求。

然后, 将 \( n = 3 \) 带入 5.2.2 所建模型中, 根据模型 Lingo 求解出 \( X_1 \) 所在县邮路规划及邮车调度方案及各线路中对应各点装卸货量如下表, 目标值 (由于空载损失的总费用) 59.8154 元: (箭头上方数字为运载量, 下方为最短距离)

\textbf{表 1.1} \( X_1 \) 所在县邮路规划及邮车调度方案

\begin{tabular}{|c|c|c|c|}
\hline & 邮路 & 总时间 (h) & 费用 (元) \\
\hline 1 & \begin{tabular}{l}
\( X1 \xrightarrow{56} Z4 \xrightarrow{56} Z5 \xrightarrow{52} Z6 \xrightarrow{56} Z7 \xrightarrow{58} Z8 \) \\
\( \xrightarrow{63} Z9 \xrightarrow{63} Z15 \xrightarrow{60} Z10 \xrightarrow{60} Z4 \xrightarrow{61} X1 \) \\
\( \xrightarrow{11} \xrightarrow{13} \xrightarrow{9} \xrightarrow{13} \xrightarrow{19} \xrightarrow{11} \xrightarrow{14} \xrightarrow{9} \xrightarrow{15} \xrightarrow{11} \)
\end{tabular} & 4.667 & 24.5231 \\
\hline 2 & \begin{tabular}{l}
\( X1 \xrightarrow{57} Z12 \xrightarrow{62} Z11 \xrightarrow{57} Z9 \xrightarrow{59} Z16 \) \\
\( \xrightarrow{61} Z15 \xrightarrow{61} Z10 \xrightarrow{53} X1 \) \\
\( \xrightarrow{21} \xrightarrow{23} \xrightarrow{20} \xrightarrow{13} \xrightarrow{9} \xrightarrow{9} \xrightarrow{20} \)
\end{tabular} & 4.25 & 24.2154 \\
\hline 3 & \begin{tabular}{l}
\( X1 \xrightarrow{63} Z13 \xrightarrow{65} Z1 \xrightarrow{64} Z2 \xrightarrow{63} Z3 \) \\
\( \xrightarrow{62} Z4 \xrightarrow{62} Z14 \xrightarrow{56} X1 \) \\
\( \xrightarrow{21} \xrightarrow{21} \xrightarrow{31} \xrightarrow{19} \xrightarrow{14} \xrightarrow{15} \xrightarrow{18} \)
\end{tabular} & 5.05 & 11.0769 \\
\hline
\end{tabular}

\textbf{表 1.2} 各线路中对应各点装卸邮件数量表 (为 0 表示仅经过)

\begin{tabular}{|c|c|c|c|c|c|c|c|c|c|c|}
\hline & 支局 & Z4 & Z5 & Z6 & Z7 & Z8 & Z9 & Z15 & Z10 & Z4 \\
\hline 1 & 卸 & 0 & 13 & 6 & 11 & 4 & 0 & 13 & 0 & 9 \\
\hline & 装 & 0 & 9 & 10 & 13 & 9 & 0 & 10 & 0 & 10 \\
\hline 2 & 支局 & Z12 & Z11 & Z9 & Z16 & Z15 & Z10 & & & \\
\hline & 卸 & 2 & 11 & 13 & 14 & 0 & 17 & & & \\
\hline & 装 & 7 & 6 & 15 & 16 & 0 & 9 & & & \\
\hline 3 & 支局 & Z13 & Z1 & Z2 & Z3 & Z4 & Z14 & & & \\
\hline & 卸 & 11 & 10 & 15 & 6 & 0 & 21 & & & \\
\hline & 装 & 13 & 9 & 14 & 5 & 0 & 15 & & & \\
\hline
\end{tabular}

\begin{figure}[h]
\centering
\includegraphics[width=0.8\textwidth]{image.png}
\caption{\( X_1 \) 县 3 条邮路示意图}
\end{figure}

\section{严格区域限制下邮运网络的分层规划(问题二)}

本节主要研究在现有县的划分下,通过对邮路的规划及邮车的安排使得总运行成本最低(邮车数目最少、总费用最少),同时满足时限要求及运输任务要求。

由于区级车一天有 2 班而县级车一天仅 1 班,所以区级车费用远远大于县级,同时,结合现实中邮政网的分层构建及管理,应在优先满足区级车数尽量少、费用尽量低的前提下,分别构建区、县两级邮运网络。

\subsection{问题二模型假设}

[1] 假设县的划分不变,各县级邮车仅在各县内运输邮件;

[2] 假设区级两个班次邮车的行驶路线相同,但方向可以不同;

[3] 假设邮件由区车运到各县后立即进行处理,并紧接着运往各支局;

[4] 不考虑邮车运输能力,每条邮路只需要一辆邮车即能满足运载能力要求。

\subsection{区级车的邮运网络构建模型}

\subsubsection{区级邮路图中邮局间最短路径矩阵的建立}

基于 5.0 对 $X_{i}$ 所在县图论抽象思想,根据该地区的邮政运输流程,要求区级邮政运输网必须至少覆盖该地市附近的 16 个支局 $Z58, Z59, \cdots, Z73$ 和 5 个县局 $X1, \cdots, X5$。为了减少投入首先要区级车尽量少,则在建立该市区级网络图时仅考虑这 $16+5+1=22$ 个节点即可。这里为方便描述将 16 个支局编号 $1 \sim 16$;5 个县局编号 $17 \sim 21$;并将市局抽象为两个点,分别是(始点 22、终点 23)。

根据表一中提供的该地市附近的 16 个支局 $Z58, Z59, \cdots, Z73$ 和 5 个县局 $X1, \cdots, X5$ 间的直达线路里程,带入 5.1 所建立的最短路模型中可建立区级网络图中所有邮局间最短路径矩阵,同时进一步求解出区级网络中任两节点间最短距离矩阵 $(d_{ij})_{23 \times 23}$(见附录 1.2.0)。

\subsubsection{模型 II(i)分析}

本模型建立目的为建立区级邮车的邮运网络,基于总运行成本最小的目标,应首先求解最少区级邮车数目,并在区级邮车数最少的基础上,以运输总费用最小为目标,对该区的区级邮车邮路进行规划并给出区级邮车调度方案。

针对区级邮车的运输要求,需要满足以下邮运约束:

- 区级邮车运输时限约束:所有区级邮车须在指定时间内完成运输任务;
- 该区级邮政运输网必须覆盖该地市附近的 16 个支局和 5 个县局;
- 各区级邮车每次仅到其邮路上的一个邮局;
- 区级邮车始、终点为市局 $D$。

《注》:区级网络中节点间弧的权值为 6.1.0 所求得的最短路距离矩阵中的值,其两顶点最短路径中可能包含其它县的支局,这些支局即为区级运输网络直接覆盖的支局,在建立县级网络时可不必覆盖这些支局。

\subsubsection{约束分析}

(1) 区级邮车运输时限约束

根据该地区的邮政运输规定,一天中区级车分两班,第一班车出发时间必须在 06:00 之后,返回地市局 $D$ 时间必须在 11:00 之前,则最多可工作 5 小时。由

于假设区级两个班次邮车的行驶路线相同,那么第二班车最多也可工作 5 小时。

注意:当区车仅经过某支局一次时,均在经过时装卸邮件。当区车经过某支局两次(往返经过)时,为了延长县级车的运输时间,第一班车在去时不进行装卸工作而在回时装卸邮件;第二班车在去时装卸邮件而在回时仅仅路过。

引入 0-1 变量 \( F_{ijk} \) 表示第 \( i \) 辆区车第 \( j \) 次是否经过第 \( k \) 个节点,即:
\[
F_{ijk} =
\begin{cases}
1 & \text{第 } i \text{ 辆区车第 } j \text{ 次经过第 } k \text{ 个节点} \\
0 & \text{否则}
\end{cases}
\]

基于 6.1.0 中对区级邮路图节点编号的约定,当 \( k=1,2,\ldots,16 \) 时为在支局装卸邮件,设第 \( i \) 区辆车最多经过 \( m \) 个节点,则第 \( i \) 辆区车共在 \( \sum_{j=1}^{m} \sum_{k=1}^{16} F_{ijk} \) 个支局进行了装卸邮件工作,已知邮车在各支局装卸邮件耗时 5 分钟,则第 \( i \) 辆区车在运输过程中耗费在各支局装卸邮件的时间为:
\[
T_{1}^{D} = \frac{5}{60} \sum_{j=1}^{m} \sum_{k=1}^{16} F_{ijk} \quad \text{(h)}
\]
\[
(i=1,\ldots,n)
\]

当 \( k=17,\ldots,21 \) 时为在各县局装卸邮件,则第 \( i \) 辆区车共在 \( \sum_{j=1}^{m} \sum_{k=17}^{21} F_{ijk} \) 个县局进行了装卸邮件工作,已知邮车在各县局装卸邮件耗时 10 分钟,则第 \( i \) 辆区车在运输过程中耗费在各县局装卸邮件的时间为:
\[
T_{2}^{D} = \frac{10}{60} \sum_{j=1}^{m} \sum_{k=17}^{21} F_{ijk} \quad \text{(h)}
\]
\[
(i=1,\ldots,n)
\]

已知各区级车平均行驶速度为 \( 65 \) km/h,以 \( d_{kq} \) 表示区级邮路图的第 \( k \) 个节点到第 \( j \) 个节点的最短路距离(由 6.1.0 求得的最短路矩阵确定),且当且仅当 \( F_{ijk} \) 与 \( F_{i,j+1,q} \) 同时为 1 时表示第 \( i \) 辆区车经过弧 \( (k,q) \),则第 \( i \) 辆区车运输途中耗时为:
\[
T_{3}^{D} = \sum_{j=1}^{m-1} \sum_{k=1}^{23} \sum_{q=1}^{23} \left( \frac{d_{kq}}{65} \cdot F_{ijk} \cdot F_{i,j+1,q} \right) \quad \text{(h)}
\]
\[
(i=1,\ldots,n)
\]

区级邮车在运输过程中耗时由在各支局、县局装卸邮件耗时及运输途中耗时三部分构成,即第 \( i \) 辆区车邮运过程总耗时不超过区级车运输时限(5 小时)约束可表示为:
\[
T_{1}^{D} + T_{2}^{D} + T_{3}^{D} \leq 5
\tag{2.1}
\]

(2) 区级运输网覆盖面约束

根据该地区的邮政运输规定,区级邮政运输网必须覆盖该地市附近的 16 个支局和 5 个县局,即对于区级邮路网络中的任一节点而言,至少有一辆邮车经过。设最少需要 \( n \) 辆区邮车,第 \( i \) 辆区车最多经过节点数为 \( m \),\( F_{ijk} \) 表示第 \( i \) 辆区车

车第 $j$ 次是否经过第 $k$ 个节点,则该区级邮政运输网必须覆盖该地市附近的 16 个支局和 5 个县局可表示为:
\begin{equation}
\sum_{i=1}^{n} \sum_{j=1}^{m} F_{ijk} \geq 1
\tag{2.2}
\end{equation}
\[
(k=1,2,...,23)
\]

(3) 各区级邮车每次仅到一个邮局

由于在本问中不考虑邮件装卸问题,所以只要经过第 $k$ 个节点则 $F_{ijk}=1$(注意,此处经过某一邮局不一定有邮件的装卸工作,可能仅仅是路过)。则对于区级邮路图中的 23 个节点而言,第 $i$ 辆车第 $j$ 次仅到一个邮局可表示为:
\begin{equation}
\sum_{k=1}^{23} F_{ijk} = 1
\tag{2.3}
\end{equation}
\[
(i=1,2,...,n \, ; \, j=1,2,...,m)
\]

(4) 区级邮车始、终点为市局 $D$

由题中关于该区邮政运输流程的规定,区级邮车从地市局 $D$ 出发将邮件运送到各县局 $X_{i}$ 和沿途支局,并将各县局 $X_{i}$ 和沿途支局收寄的邮件运送回地市局 $D$;则区级邮车行程的始终点皆为 $D$。

由于在区级邮路图中将 $D$ 抽象为始点 22 及终点 23,则各区级邮车由 $D$ 出发并最终回到 $D$ 可表示为:
\begin{equation}
F_{i,1,22} = 1, \, F_{i,m,23} = 1
\tag{2.4}
\end{equation}
\[
(i=1,2,...,n)
\]

### 目标分析

本问要求采用尽可能少、尽可能短的邮路以减少邮政部门车辆和人员等的投入,则目标仍有两个:一、区级邮车数量最少;二、区网总运输成本最低。

由于第二班车最终返回地市局 $D$ 的时间必须在 18:00 之前,则为了使县级车运输时间尽量长,第二班车最早应于 13:00 出发,此时第一班车早已返回,所以可用相同的车跑第一、二班,即第一班车所需车辆数即为该市所需区级车数。

基于 5.2.1 中对模型 I 双目标的分析处理,这里仍采用将邮车数量最少转化成约束的方法进行计算机编程求解。所以在区级车规划时仅将运输成本最低做为模型目标。

以 0-1 变量 $F_{ipk}$ 表示第 $i$ 辆区车第 $p$ 次是否经过第 $k$ 个节点,以 $d_{kq}$ 表示区级邮路图的第 $k$ 个节点到第 $q$ 个节点的最短路距离(由 6.1.0 求得的最短路矩阵确定),则可得第 $i$ 辆区级车所行驶里程数为:
\[
\sum_{p=1}^{m-1} \sum_{k=1}^{23} \sum_{\substack{q=1 \\ q \neq k}}^{23} d_{kq} F_{ipk} F_{i,p+1,q} \quad (i=1,2,...,n) \, .
\]

由于每条邮路的运行成本为 3 元/公里,而区级车每天有 2 班,则所有区级邮路总运输成本最低为:
\[
\text{Min} \quad 3 \times 2 \sum_{i=1}^{n} \sum_{p=1}^{m-1} \sum_{k=1}^{23} \sum_{\substack{q=1 \\ q \neq k}}^{23} d_{kq} F_{ipk} F_{i,p+1,q}
\]

\subsection{6.1.2 模型II(i)的建立}

基于6.1.1对构建区级邮运网络模型的分析,以所有区级邮车总运输成本最低为目标,(2.1~2.4)为约束,建立区级邮路规划和邮车调度最优化模型如下:

\begin{equation}
\begin{aligned}
& \text{Min} \quad 3 \times 2 \sum_{i=1}^{n} \sum_{p=1}^{m-1} \sum_{k=1}^{23} \sum_{\substack{q=1 \\ q \neq k}}^{23} d_{kq} F_{ipk} F_{i,p+1,q} \\
& \text{S.T.} \left\{
\begin{aligned}
T_{1}^{D} + T_{2}^{D} + T_{3}^{D} & \leq 5 \tag{2.1} \\
\sum_{i=1}^{n} \sum_{j=1}^{m} F_{ijk} & \geq 1 \quad (k=1, \ldots, 23) \tag{2.2} \\
\sum_{k=1}^{23} F_{ijk} & = 1 \quad
\begin{cases}
i=1, \ldots, n \\
j=1, \ldots, m
\end{cases} \tag{2.3} \\
F_{i,1,22} = 1, \; F_{i,m,23} = 1 & \quad (i=1, \ldots, n) \tag{2.4} \\
F_{ijk} & \in \{0, 1\}
\end{aligned}
\right.
\end{aligned}
\end{equation}

《注》:$n$ 为区所需的最少车辆数,根据目标分析中对双目标转化成单目标的方法,在模型求解时依次取 $n=1, 2, 3 \ldots$,使模型有解的最小 $n$ 值即为所需最小区车辆数。

模型说明

(2.1) 区级邮车运输时限约束:所有区级邮车须在5小时内完成运输任务;

(2.2) 该县级邮政运输网必须覆盖本县内所有支局;

(2.3) 各区级邮车每次仅到一个邮局;

(2.4) 区级邮车必须由市局 $D$ 发出并最终返回 $D$。

$F_{ijk}$ —— 0-1变量,表示第 $i$ 辆区车第 $j$ 次是否经过第 $k$ 个节点;

$d_{kq}$ —— 图中节点 $k$ 到 $j$ 的最短路距离(由6.1.0求得的最短路矩阵确定);

其中,$T_{1}^{D}, T_{2}^{D}, T_{3}^{D}$ 分别表示第 $i$ 辆区车“在沿途各支局装卸邮件的总时间”、“在县局装卸邮件的总时间”、“在运输途中总时间”。

\subsubsection{6.1.3 模型II(i)的求解}

首先,通过将模型中 $n$ 从1开始由小到大赋值时得出 $n$ 最小为4时模型可解,即最少需要4辆区级邮车才能满足该市的邮件运输需求。

然后,将 $n=4$ 带入6.1.2所建模型中,根据模型Lingo求解出该市区级邮路网络的邮路规划和邮车调度方案如下表(各邮路可双向行驶,时间为两班车运输最短总时间),目标值(区级4条邮路2班次总费用)$2 \times 2742$元:

\begin{table}[htbp]
\centering
\caption{区级邮路规划及邮车调度方案}
\begin{tabular}{|l|c|c|c|c|}
\hline
\textbf{邮路} & \textbf{时间} & \textbf{费用} \\ \hline
D → Z62 → Z9 → Z10 → X1 → Z15 → Z16 → Z63 → Z6 → D & 7.1628628 & 1218 \\
D → Z67 → Z4 → X2 → Z27 → Z28 → Z65 Z68 → D & 9.3513 & 1596 \\
D → Z69 → Z70 → Z71 → X4 → Z73 Z72 → D & 6.4333 & 1092 \\
D → Z61 → Z52 → X5 Z9 Z6 → D & 8.8423 & 1578 \\
\hline
\end{tabular}
\end{table}

\section{绘出区级邮路网络中 4 条邮路示意图如下:}

\begin{figure}[h]
    \centering
    \includegraphics[width=\textwidth]{image.png}
    \caption{区级邮路网络中 4 条邮路示意图}
    \label{fig:post_network}
\end{figure}

\subsection{县级邮车的邮运网络构建模型}

\subsubsection{各县网中邮局间最短路径矩阵的建立}

基于 5.0 图论抽象思想,针对各县不同支局情况分别建立各县的邮路网络图,并根据表一中提供的直达线路里程数,带入 5.1 所建立的最短路模型中可分别建立各县的邮路网络图中所有邮局间直达路径矩阵,同时进一步求解出各县级网络中任两节点间最短距离矩阵 $(d_{ij})_{N \times N}$(见附录 1.2.1~1.2.5)。

注意:由于本问中假设县的划分不能变更,则各县级邮车仅能收寄本县内支局的邮件。同时,由于区级车在运输过程中已覆盖县的部分支局(由模型 II (i) 可求得),则应在构建县级网络图时首先将中被区车覆盖到的支局剔除。

\subsubsection{模型 II (ii) 分析}

通过模型 II (i) 的建立与求解构建出该市区级邮政网络,在此基础上本模型将进一步构建该市各县的县级邮政网络。

基于总运行成本最小的目标,应首先求解各县所需的最少邮车数目,并在各县级邮车数最少的基础上,以运输总费用最小为目标,对该区各县的邮车邮路进行规划并给出各县级邮车调度方案。

针对县级邮车的运输要求,各县皆需要满足以下邮运约束:

\begin{itemize}
    \item 各县的邮车运输时限约束:各县的邮车须在指定时间内完成运输任务;
    \item 各县的邮车必须覆盖各县中未被区级车所覆盖的支局;
    \item 各县的各辆邮车每次仅到其邮路上的一个邮局;
    \item 各县的邮车始、终点为所在县的县局 $X_{i}$。
\end{itemize}

由于各县邮运约束的建立思想相同,则下面取任一县为研究对象进行分析。

约束分析

(1) 各县级邮车运输时限约束

基于 6.2.0 中对各县邮路网络的图论抽象,可得到不同县的 4 个(除 $X_{4}$)网络抽象图,设图中顶点数为 $N$(不同县 $N$ 的取值不同)。

仍引用 0-1 变量 $F_{ijk}$ 表示县级第 $i$ 辆车第 $j$ 次是否经过该县图上的第 $k$ 个节点,基于 5.2.1 中对 $X_{1}$ 县的运输时限约束的思想,可知第 $i$ 辆县车在运输过程中耗费在所需覆盖的各支局装卸邮件的时间为:

\begin{equation}
T_{1} = \frac{5}{60} \sum_{j=1}^{m} \sum_{k=1}^{N} F_{ijk} \tag{h}
\end{equation}

\begin{equation}
(i = 1, \dots, n) \quad \text{已剔除被区级网覆盖的支局}
\end{equation}

已知各县级车平均行驶速度为 $30 \, \text{km/h}$,以 $d_{kq}$ 表示县级邮路图的第 $k$ 个节点到第 $j$ 个节点的最短路距离(由 6.2.0 求得的最短路矩阵确定),则可得第 $i$ 辆县车运输途中耗时为:

\begin{equation}
T_{2} = \sum_{j=1}^{m-1} \sum_{k=1}^{N} \sum_{q=1}^{N} \frac{d_{kq}}{30} F_{ijk} F_{i,j+1,q} \tag{h}
\end{equation}

\begin{equation}
(i = 1, \dots, n)
\end{equation}

以 $T$ 表示县级车运输的最大时间范围,则:

\begin{equation}
T_{1} + T_{2} = \frac{5}{60} \sum_{j=1}^{m} \sum_{k=1}^{N-1} F_{ijk} + \sum_{j=1}^{m-1} \sum_{k=1}^{N} \sum_{q=1}^{N} \frac{d_{kq}}{30} F_{ijk} F_{i,j+1,q} \leq T \tag{2.5}
\end{equation}

\begin{equation}
(i = 1, \dots, n)
\end{equation}
县级车运输时限 $T$ 的确定

由于题中仅说明第一、二班区级车的路线相同,则可能两班车行驶方向相同、不相同两种情况。下面分别分析这两种情况下的 $T$。为进一步增大县局的运输时间范围,两班区级车通过市内支局时,在两个班次中只停留一次。

情况 1:区级两班车行走路线方向一致

在极限情况下,为使县级车的运输总时间最大,令第一班车 6:00 离开市局到达县局运输时间为 $t_{1}$,第二班车最晚 18:00 到达市局,令第二班车由该县局到达市局的运输时间为 $t_{2}$。由于第一、二班车沿同一路线行走,另外加上邮车在县局停留的 10 分钟,得到每一班车的总时间 $t$,即:

\begin{equation}
t = t_{1} + t_{2} + \frac{10}{60}
\end{equation}

由于县级车出发时在第一班车到达 1 小时以后,并且在第二班车离开县局时的 1 小时之前返回。故县级车的最大运输时间范围满足:
\[
T \leq 12 - (t_1 + t_2 + 2)
\]
经化简得:
\[
T \leq 10 + \frac{10}{60} - t
\]

情况 2:区级两班车行走路线方向相反

为使县级车的运输总时间更大,第一班车走最短路到达县局,第二班车沿反向走远路到达县局。此时,第一班车由市局到达县局的时间和第二班车离开县局返回市局的时间都为 \( t_1 \)。故此时县级车的运输时间满足:
\[
T \leq 12 - (2t_1 + 2)
\]
经化简得:
\[
T \leq 10 - 2t_1
\]

(2) 各县的邮车覆盖面约束

各县的邮车必须覆盖各县中未被区级车所覆盖的支局,由于在构建各县网络图时已将这些支局剔除,所以各县的邮车需覆盖其县图中的所有顶点。则基于 6.1.1 中区级邮车覆盖面约束的建立思想,以 \( N \) 表示县图中顶点数(不同县 \( N \) 的取值不同),则可得各县的邮车覆盖面约束:
\begin{equation}
\sum_{i=1}^{n} \sum_{j=1}^{m} F_{ijk} \geq 1 \tag{2.6}
\end{equation}
\[
(k = 1, 2, \ldots, N)
\]

(3) 各县的各辆邮车每次仅到其邮路上的一个邮局

由于在构建各县网络图时已将区级车所覆盖的支局剔除,这样此约束的建立思想与 6.1.1 中区级车的同意义约束的构建思想相同,这里不再复述。以 \( N \) 表示县图中顶点数,则可得各县的各辆邮车每次仅到其邮路上的一个邮局约束:
\begin{equation}
\sum_{k=1}^{N} F_{ijk} = 1 \tag{2.7}
\end{equation}
\[
(i = 1, 2, \ldots, n \, ; \, j = 1, 2, \ldots, m)
\]

(4) 各县的邮车始、终点为所在县的县局 \( X_i \)

由题中关于该区邮政运输流程的规定,各县级邮车从其所在县局 \( X_i \) 出发将邮件运送到其所负责的支局,并将各支局收寄的邮件运回 \( X_i \);则县级邮车行程的始终点皆为 \( X_i \)。需注意,本模型中对于由于 \( X_i \) 的处理与前面有所不同,此处将 \( X_i \) 处理为图中 1 个顶点,则各县的邮车始、终点为所在县的县局为 \( X_i \) 为:
\begin{equation}
F_{i,1,N} = 1, \, F_{i,m,N} = 1 \tag{2.8}
\end{equation}
\[
(i = 1, 2, \ldots, n)
\]

目标分析

在构建各县邮运网络时,需针对每个县分别进行构建,但对各县其目标相同,皆为:一、县级邮车数量最少;二、各县的总运输成本最低。

基于 6.1.1 对目标的分析处理,本模型中仍采用将邮车数量最少转化成约束的方法计算机编程求解。所以在县网规划时仅将运输成本最低做为模型目标。

以 0-1 变量 $F_{ipk}$ 表示县级第 $i$ 辆车第 $p$ 次是否经过该县图上的第 $k$ 个节点,以 $d_{kq}$ 表示相应的县级邮路图的第 $k$ 个节点到第 $q$ 个节点的最短路距离(由 6.2.0 求得的各县邮路网络图最短路矩阵确定),以 $N$ 表示县图中顶点数(不同县 $N$ 的取值不同),则可得所求县的所有县级邮车总运输成本最低为:

\[
Min \quad 3\sum_{i=1}^{n}\sum_{p=1}^{m-1}\sum_{k=1}^{N}\sum_{\substack{q=1 \\ q\neq k}}^{N} d_{kq}F_{ipk}F_{i,p+1,q}
\]

\subsubsection{6.2.2 模型 II (ii) 的建立}

基于 6.2.1 对构建县级邮运网络模型的分析,以县级邮车运输成本最低为目标,(2.5~2.8) 为约束,建立各县级邮路规划和邮车调度最优化模型如下:

\[
\begin{aligned}
& Min \quad 3\sum_{i=1}^{n}\sum_{p=1}^{m-1}\sum_{k=1}^{N}\sum_{\substack{q=1 \\ q\neq k}}^{N} d_{kq}F_{ipk}F_{i,p+1,q} \\
& S.T. \left\{
\begin{aligned}
& \frac{5}{60}\sum_{j=1}^{m}\sum_{k=1}^{16}F_{ijk} + \sum_{j=1}^{m-1}\sum_{k=1}^{N}\sum_{q=1}^{N}\frac{d_{kq}}{30}F_{ijk}F_{i,j+1,q} \leq T \quad (i=1,\dots,n) \tag{2.5} \\
& \sum_{i=1}^{n}\sum_{j=1}^{m}F_{ijk} \geq 1 \quad (k=1,\dots,N) \tag{2.6} \\
& \sum_{k=1}^{N}F_{ijk} = 1 \quad
\begin{cases}
i=1,\dots,n \\
j=1,\dots,m
\end{cases} \tag{2.7} \\
& F_{i,1,N} = 1, \quad F_{i,m,N} = 1 \quad (i=1,\dots,n) \tag{2.8} \\
& F_{ijk} \in \{0,1\}
\end{aligned}
\right.
\end{aligned}
\]

《注》:$n$ 县局需要的最少车辆数,在模型求解时依次取 $n=1,2,3...$,使模型有解的最小 $n$ 值即为所需最小县车辆数。根据 6.2.0 中不同县的抽象图对各县进行分别求解。

#### 模型说明:
(2.5) 各县的邮车须在指定时间内完成运输任务;
(2.6) 各县的邮车必须覆盖各县中未被区级车所覆盖的支局;
(2.7) 各县的各辆邮车每次仅到其邮路上的一个邮局;
(2.8) 各县的邮车始、终点为所在县的县局。

- $F_{ijk}$ —— 0-1 变量,县级第 $i$ 辆车第 $p$ 次是否经过该县图上的第 $k$ 个节点;
- $d_{kq}$ —— 图中节点 $k$ 到 $j$ 的最短路距离(由 6.2.0 求得的最短路矩阵确定);
- $N$ —— 表示县图中顶点数(不同县 $N$ 的取值不同);
- $T$ —— 表示县级车运输的最大时间范围($T$ 的确定详见 6.2.1/(1));

\subsubsection{6.2.3 模型 II (ii) 的求解}

首先,通过将模型中 $n$ 从 1 开始由小到大赋值时得出 $X1 \sim X5$ 各县分别需要 2、2、1、2、3 辆邮车才能满足各县邮运要求。
然后,分别将以上结果带入6.1.2所建模型中,根据模型Lingo求解出该市
县级邮路网络的邮路规划和邮车调度方案如下表(各邮路可双向行驶),得到总
区、县运行费用9549元:
\begin{tabular}{|c|c|c|c|}
\hline & 邮路 & 总时间 (h) & 费用 (元) \\
\hline 1 & \begin{tabular}{l}
\( X1 \xrightarrow{56} Z4 \xrightarrow{56} Z5 \xrightarrow{52} Z6 \xrightarrow{56} Z7 \xrightarrow{58} Z8 \) \\
\( \xrightarrow{63} Z9 \xrightarrow{63} Z15 \xrightarrow{60} Z10 \xrightarrow{60} Z4 \xrightarrow{61} X1 \) \\
\( \xrightarrow{11} \xrightarrow{13} \xrightarrow{9} \xrightarrow{13} \xrightarrow{19} \xrightarrow{11} \xrightarrow{14} \xrightarrow{9} \xrightarrow{15} \xrightarrow{11} \)
\end{tabular}

\section{松弛区域限制时邮运网络的构建模型(问题三)}

本问在第二问的基础上允许一定程度上打破现有县的划分,则应首先根据就近原则对各县网中支局重新进行归属划分,然后在新支局归属划分下重新对县级邮路网络进行规划并给出具体邮车调度方案。

\subsection{问题三模型假设}

[1] 假设区级邮路网络不变;

[2] 县与县交界地带的支局,其邮件允许由邻县县局负责运送;

[3] 不考虑邮车运输能力,通常每条邮路只需要一辆邮车即能满足运载能力要求。

\subsection{各邮局间最短路径矩阵的建立}

基于5.0对$X_{1}$所在县图论抽象思想,将该市支局$Z_{1}, Z_{2}, \ldots, Z_{73}$定义为编号$1 \sim 73$的顶点,5个县局$X_{1}, \ldots, X_{5}$定义为编号$74 \sim 78$的顶点,市局$D$定义为编号79的顶点。

根据表一中提供的各邮局间直达线路里程,带入5.1所建立的最短路模型中可建立该市邮路网络图中所有邮局间直达路径矩阵(见光盘),同时进一步求解出网络图中任两节点间最短距离矩阵$(d_{ij})_{79 \times 79}$(见光盘),以及路径。

\subsection{县网中支局归属划分的算法与步骤}

在本问中考虑到部分县与县交界地带的支局,其邮件由邻县县局负责运送可能会降低全区的运行成本,则需要重新将县网中各支局(县边界处支局重点考虑)所归属的县局进行重新分配。此处采取就近原则,算法如下:

\textbf{Step1:} 初始化所有县支局,不属于县局 X1~X5;

\textbf{Step2:} 选取一个支局 $Z_{i}$ 判断与 X1~X5 距离;

\textbf{Step3:} 若 $Z_{i}$ 离 $X_{j}$ 县局距离小于其他县局,则 $Z_{i} \in X_{j}$;

\textbf{Step4:} 若所有支局 $Z_{i}$ 已属于 X1~X5 结束程序;Else 转 Step2。

基于此算法将各县所辖支局重新规划后结果如下:

\begin{tabular}{l}
\( X1 \xrightarrow{57} Z12 \xrightarrow{62} Z11 \xrightarrow{57} Z9 \xrightarrow{59} Z16 \) \\
\( \xrightarrow{61} Z15 \xrightarrow{61} Z10 \xrightarrow{53} X1 \) \\
\( \xrightarrow{21} \xrightarrow{23} \xrightarrow{20} \xrightarrow{13} \xrightarrow{9} \xrightarrow{9} \xrightarrow{20} \)
\end{tabular}

\subsection{7.3 新区域划分下县级邮运网络构建模型}

\subsubsection{7.3.1 模型 III 分析}

基于 7.2 对县网中支局归属的重新划分,对各县局所负责支局集进行了重新定义,要求在此基础上重新构建该市的县级邮政网络。至此,问题转化为与第二问县级网路构建相同的问题,可仍以总运行成本最小为目标,首先求解各县所需的最少邮车数目,并在各县级邮车数最少的基础上,以运输费用最小为目标,对该各县的邮车邮路进行规划并给出各县级邮车在新行政域划分下的调度方案。

由于本问在对各县局的行政区域进行重新划分后相当于重新定义了问题二中对于各县网络图的图论抽象图,设新图中顶点数为 $M$(不同县 $M$ 取值不同)。基于 6.2.1 模型 II (ii) 分析中对县级网络邮运约束的分析,可得新行政域划分下的各县皆需要满足的邮运约束如下:

\textbf{县级网络邮运约束:}

(1) 新行政区域划分下,各县的邮车须在指定时间内完成运输任务
\begin{equation}
\frac{5}{60} \sum_{j=1}^{m} \sum_{k=1}^{16} F_{ijk} + \sum_{j=1}^{m-1} \sum_{k=1}^{M} \sum_{q=1}^{M} \frac{d'_{kq}}{30} F_{ijk} F_{i,j+1,q} \leq T \tag{3.1}
\end{equation}
(详细分析见 6.2.1/(1))

(2) 各县局的邮车必须覆盖其新行政区域中未被区级车所覆盖的支局
\begin{equation}
\sum_{i=1}^{n} \sum_{j=1}^{m} F_{ijk} \geq 1 \tag{3.2}
\end{equation}
(详细分析见 6.2.1/(2))

(3) 新行政区域划分下,各县的各辆邮车每次仅到其邮路上的一个邮局。

\begin{equation}
\sum_{k=1}^{M} F_{ijk} = 1
\tag{3.3}
\end{equation}

(详细分析见 6.2.1/(3))

(4) 新行政区域划分下,各县的邮车始、终点为所在县的县局 $X_{i}$

\begin{equation}
F_{i,1,M} = 1, \ F_{i,m,M} = 1
\tag{3.4}
\end{equation}

(详细分析见 6.2.1/(4))

目标:邮车运输成本最低

在构建新县邮运网络时,仍需针对每个县局分别进行构建,但对各县其目标相同,仍为:一、县级邮车总数量最少;二、各县的总运输成本最低。

本模型中仍采用将邮车数量最少转化成约束的方法计算机编程求解。以 0-1 变量 $F_{ipk}$ 表示县级第 $i$ 辆车第 $p$ 次是否经过该县图上的第 $k$ 个节点,以 $d'_{kq}$ 表示新图中相应的县级邮路图的第 $k$ 个节点到第 $q$ 个节点的最短路距离,以 $M$ 表示新的县图中顶点数(不同县 $M$ 的取值不同),则可得新行政区域划分下,所求县的所有县级邮车总运输成本最低为:

\begin{equation}
\begin{aligned}
Min \quad & 3 \sum_{i=1}^{n} \sum_{p=1}^{m-1} \sum_{k=1}^{M} \sum_{\substack{q=1 \\ q \neq k}}^{M} d'_{kq} F_{ipk} F_{i,p+1,q}
\end{aligned}
\end{equation}

\subsubsection{7.3.2 模型 III 建立}

基于模型分析,以新行政域划分下各县级邮车运输成本最低为目标,(3.1~3.4) 为约束,建立新行政域划分下,各县级邮路规划和邮车调度模型如下:

\begin{equation}
\begin{aligned}
Min \quad & 3 \sum_{i=1}^{n} \sum_{p=1}^{m-1} \sum_{k=1}^{M} \sum_{\substack{q=1 \\ q \neq k}}^{M} d'_{kq} F_{ipk} F_{i,p+1,q} \\
S.T. \quad & \left\{
\begin{aligned}
& \frac{5}{60} \sum_{j=1}^{m} \sum_{k=1}^{16} F_{ijk} + \sum_{j=1}^{m-1} \sum_{k=1}^{M} \sum_{q=1}^{M} \frac{d'_{kq}}{30} F_{ijk} F_{i,j+1,q} \leq T \quad (i=1,\dots,n) \tag{3.1} \\
& \sum_{i=1}^{n} \sum_{j=1}^{m} F_{ijk} \geq 1 \quad (k=1,\dots,M) \tag{3.2} \\
& \sum_{k=1}^{M} F_{ijk} = 1 \quad \begin{cases} i=1,\dots,n \\ j=1,\dots,m \end{cases} \tag{3.3} \\
& F_{i,1,M} = 1, \ F_{i,m,M} = 1 \quad (i=1,\dots,n) \tag{3.4} \\
& F_{ijk} \in \{0,1\}
\end{aligned}
\right.
\end{aligned}
\end{equation}

《注》:$n$ 县局需要的最少县车辆数,在模型求解时依次取 $n=1,2,3\dots$,使模型有解的最小 $n$ 值即为所需最小县车辆数。

### 模型说明:

(3.1) 新行政区域划分下,各县的邮车须在指定时间内完成运输任务;

(3.2) 各县局的邮车必须覆盖其新行政区域中未被区级车所覆盖的支局;
(3.3) 各县的各辆邮车每次仅到其邮路上的一个邮局;
(3.4) 各县的邮车始、终点为所在县的县局 $X_{i}$;

$F_{ijk}$ —— 0-1变量,县级第 $i$ 辆车第 $p$ 次是否经过该县图上的第 $k$ 个节点;
$d_{kq}^{\prime}$ —— 表示新图中相应的县级邮路图的第 $k$ 到第 $q$ 个节点的最短里程;
$M$ —— 表示新县图中各县图顶点数(不同县 $M$ 的取值不同);
$T$ —— 表示县级车运输的最大时间范围($T$ 的确定详见 6.2.1/(1));

\subsubsection{7.3.3 模型 III 的求解}

首先,通过将模型中 $n$ 从 1 开始由小到大赋值时得出新行政区域划分下 $X1 \sim X5$ 各县分别需要 2、2、1、2、2 辆邮车才能满足各县邮运要求。

然后,分别将以上结果带入 7.3.2 所建模型中,根据模型 Lingo 求解出该市新行政区域划分下县级邮路网络的邮路规划和邮车调度方案如下:(各邮路可双向行驶)

\textbf{表3.2 新行政区域划分下,县级邮路规划及邮车调度方案}

\begin{tabular}{l}
\( X1 \xrightarrow{63} Z13 \xrightarrow{65} Z1 \xrightarrow{64} Z2 \xrightarrow{63} Z3 \) \\
\( \xrightarrow{62} Z4 \xrightarrow{62} Z14 \xrightarrow{56} X1 \) \\
\( \xrightarrow{21} \xrightarrow{21} \xrightarrow{31} \xrightarrow{19} \xrightarrow{14} \xrightarrow{15} \xrightarrow{18} \)
\end{tabular}

\section{8 县局最优选址模型(问题四)}

本节主要研究当县局均允许迁址到本县内任一支局处,同时原来的县局弱化为普通支局时,对全市邮运网络进行全局整体构建。

\subsection{8.1 模型 IV 分析}

本模型建立目的为在允许将县局迁址到本县内任一支局处的假设下,对该市区、县两级网络进行整体规划及对邮车进行调度。由此,本模型应针对全市建立邮路整体规划模型。

- 目标:全市区、县两级车总运行成本最少
- 约束:区、县级邮运网络的邮运约束

由于本问需要对全市邮运网络进行整体规划,所以需要基于问题三中建立全市邮局网最短路矩阵及最短里程矩阵 $(d_{ij})_{79 \times 79}$ 进行分析。

约束分析

一、邮运网络时限约束

基于问题二模型中关于区、县级邮运网络时限的相关约束分析,可建立邮运网络的时限约束如下:

(1) 区级邮车运输时限约束

以 0-1 变量 $F_{ijk}$ 表示第 $i$ 辆区车第 $j$ 次是否经过第 $k$ 个节点,则:

第 $i$ 辆区车耗费在各支局装卸邮件的时间为:
\[
T_{1}^{D} = \frac{5}{60} \sum_{j=1}^{m} \sum_{k=1}^{16} F_{ijk}
\]

第 $i$ 辆区车耗费在各县局装卸邮件的时间为:
\[
T_{2}^{D} = \frac{10}{60} \sum_{j=1}^{m} \sum_{k=17}^{21} F_{ijk}
\]

第 $i$ 辆区车运输途中耗时为:
\[
T_{3}^{D} = \sum_{j=1}^{m-1} \sum_{k=1}^{D} \sum_{q=1}^{D} \left( \frac{d_{kq}}{65} \cdot F_{ijk} \cdot F_{i,j+1,q} \right)
\]

第 $i$ 辆区车邮运过程总耗时不超过区级车运输时限(5 小时)约束可表示为:
\[
T_{1}^{D} + T_{2}^{D} + T_{3}^{D} \leq 5
\]

(详细分析见 6.1.1/(1))

(2) 县级邮车运输时限约束

以 0-1 变量 $W_{ijk}^{X}$ 表示第 $X$ 个县第 $i$ 辆县车第 $j$ 次是否经过第 $k$ 个节点,则各县的邮车须在指定时间内完成运输任务可表示为:
\[
\frac{5}{60} \sum_{j=1}^{m} \sum_{k=1}^{16} W_{ijks} + \sum_{j=1}^{m-1} \sum_{k=1}^{N} \sum_{q=1}^{N} \left( \frac{d'_{kq}}{30} \cdot W_{ijks} \cdot W_{i,j+1,q,s} \right) \leq T
\]

(详细分析见 6.2.1/(1))

二、运输网覆盖面约束

基于问题二模型中关于区、县级邮运网络覆盖面的相关约束分析,可建立邮运网络的覆盖面约束如下:

(1) 区级网络覆盖面约束

以 $E$ 表示区级内邮局点的集合,各区级邮车必须覆盖区级邮政区域中所有邮局可表示为:
\[
\sum_{i=1}^{n} \sum_{j=1}^{m} F_{ijk} \geq 1 \quad (k \in E)
\]

(详细分析见 6.1.1/(2))
(2) 县级网络覆盖面约束

以 $V_{s}$ 表示第 $s$ 个县区内邮局点的集合,各县级邮车必须覆盖其各自行政区域

中所有支局可表示为:
\begin{equation}
\sum_{i=1}^{n} \sum_{j=1}^{m} W_{ijks} \geq 1
\end{equation}
\begin{equation}
(k \in V_s) \quad \text{(详细分析见 6.2.1/(2))}
\end{equation}

三、邮车每次仅到一个邮局约束

基于问题二模型中关于区、县级邮运网络邮车每次仅到一个邮局相关约束分析,可建立邮车每次仅到一个邮局约束如下:

(1) 各区级邮车每次仅到一个邮局

以 $E$ 表示区级内邮局点的集合,$F_{ijk}$ 表示第 $i$ 辆区车第 $j$ 次是否经过第 $k$ 个节点,第 $i$ 辆区车第 $j$ 次仅到一个邮局可表示为:
\begin{equation}
\sum_{k \in E} F_{ijk} = 1
\end{equation}
\begin{equation}
(i = 1, \dots, n \, ; \, j = 1, \dots, m) \quad \text{(详细分析见 6.1.1/(3))}
\end{equation}

(2) 各县级邮车每次仅到一个邮局

以 0-1 变量 $W_{ijks}$ 表示第 $s$ 个县第 $i$ 辆县车第 $j$ 次是否经过第 $k$ 个节点,则各县级邮车每次仅到一个邮局可表示为:
\begin{equation}
\sum_{k \in E} W_{ijks} = 1
\end{equation}
\begin{equation}
(i = 1, \dots, n \, ; \, j = 1, \dots, m) \quad \text{(详细分析见 6.2.1/(3))}
\end{equation}

四、邮车运行始终点约束

基于问题二模型中关于区、县级邮运网络运行始终点约束的分析与建立,可建立邮车运行始终点约束如下:

(1) 区级邮车始、终点为市局 $D$

由于区局在本模型中位置不变,则各区级邮车由 $D$ 出发并最终回到 $D$ 可表示为:
\begin{equation}
F_{i,1,D} = 1, \, F_{i,m,D} = 1
\end{equation}
\begin{equation}
(i = 1, \dots, n) \quad \text{(详细分析见 6.1.1/(4))}
\end{equation}

(2) 各县的邮车始、终点为所在县的新县局

设第 $s$ 个县局选在编号为 $X_s$ 的顶点处,则各县的邮车始、终点为重新选址后的其所在县的新县局可表示为:
\begin{equation}
W_{i,1,X_s,s} = 1, \, W_{i,m,X_s,s} = 1
\end{equation}
\begin{equation}
(i = 1, \dots, n) \quad \text{(详细分析见 6.2.1/(4))}
\end{equation}

\section{目标分析}

县局重新选址目的为使全市区、县两级车总运行成本最少,由于区级车一天有两班,则费用需乘 2。知:

区级邮车行驶总费用:

\begin{equation}
2 \times 3 \sum_{i=1}^{n} \sum_{p=1}^{m-1} \sum_{\substack{k \in V_s \\ q \in V_s \\ q \neq k}} d_{kq} F_{ipk} F_{i,p+1,q}
\end{equation}
(详细分析见 6.1.1 目标分析)

县级邮车行驶总费用:
\begin{equation}
3 \sum_{X=1}^{5} \sum_{i=1}^{n} \sum_{p=1}^{m-1} \sum_{\substack{k \in V_s \\ q \in V_s \\ q \neq k}} d_{kq} \cdot W_{ipks} \cdot W_{i,p+1,q,s}
\end{equation}
(详细分析见 6.2.1 目标分析)

则全市区、县两级车总运行成本最少可表示为:
\begin{equation}
Min \quad 2 \times 3 \sum_{i=1}^{n} \sum_{p=1}^{m-1} \sum_{\substack{k \in E \\ q \in E \\ q \neq k}} d_{kq} F_{ipk} F_{i,p+1,q} + 3 \sum_{X=1}^{5} \sum_{i=1}^{n} \sum_{p=1}^{m-1} \sum_{\substack{k \in V_s \\ q \in V_s \\ q \neq k}} d_{kq} \cdot W_{ipks} \cdot W_{i,p+1,q,s}
\end{equation}

\subsection{8.2 模型 IV 建立}

以全市区县两级车总运行成本最少为目标,建立县局选址最优化模型如下:

\begin{equation}
\begin{aligned}
Min \quad & 2 \times 3 \sum_{i=1}^{n} \sum_{p=1}^{m-1} \sum_{\substack{k \in E \\ q \in E \\ q \neq k}} d_{kq} F_{ipk} F_{i,p+1,q} + 3 \sum_{X=1}^{5} \sum_{i=1}^{n} \sum_{p=1}^{m-1} \sum_{\substack{k \in V_s \\ q \in V_s \\ q \neq k}} d_{kq} \cdot W_{ipks} \cdot W_{i,p+1,q,s} \\
\text{s.t.} \quad & \begin{cases}
T_1^D + T_2^D + T_3^D \leq 5 \\
\frac{5}{60} \sum_{j=1}^{m} \sum_{k \in V_s} W_{ijs} + \sum_{j=1}^{m-1} \sum_{\substack{k \in V_p \\ q \in V_s}} \left( \frac{d'_{kq}}{30} \cdot W_{ijs} \cdot W_{i,j+1,q,s} \right) \leq T \\
\sum_{i=1}^{n} \sum_{j=1}^{m} F_{ijk} \geq 1 \\
\sum_{i=1}^{n} \sum_{j=1}^{m} W_{ijs} \geq 1 \quad (k \in V_p) \\
\begin{cases}
\sum_{k \in E} F_{ijk} = 1 \\
\sum_{k \in V_s} W_{ijs} = 1 \quad (i=1, \dots, n; \, j=1, \dots, m)
\end{cases} \\
\begin{cases}
F_{i,1,D} = 1, \, F_{i,m,D} = 1 \\
W_{i,1,X_s,s} = 1, \, W_{i,m,X_s,s} = 1
\end{cases} \\
F_{ijk}, W_{ijs} \in \{0, 1\}
\end{cases}
\end{aligned}
\tag{1}
\end{equation}

模型说明:
\begin{enumerate}
    \item 邮运网络时限约束
    \item 运输网覆盖面约束
    \item 邮车每次仅到一个邮局约束
    \item 邮车运行始终点约束
\end{enumerate}

\begin{itemize}
    \item $F_{ijk}$ —— 0-1 变量,县级第 $i$ 辆车第 $p$ 次是否经过该县图上的第 $k$ 个节点;
    \item $d'_{kq}$ —— 表示新图中相应的县级邮路图的第 $k$ 到第 $q$ 个节点的最短里程;
    \item $M$ —— 表示新县图中各县图顶点数(不同县 $M$ 的取值不同);
\end{itemize}

\begin{itemize}
    \item $T$ —— 表示县级车运输的最大时间范围($T$ 的确定详见 6.2.1/(1));
    \item $E$ —— 表示区级内邮局点的集合;
    \item $V_{s}$ —— 表示第 $s$ 个县区内邮局点的集合;
    \item $X_{s}$ —— 表示第 $X_{s}$ 个县区内作为县局的点。
\end{itemize}

\subsection{8.3 模型 IV 求解方法及结果}

由于上述模型中变量太多,限制条件复杂,计算机难以直接求解。故考虑运用启发式算法进行近似求解。

分析思路:极限分析法

假设整个邮政网络全部由县级车将各县局和支局的邮件运送到市局,令此时整个网络的最短路线长度为 $L$;由于县级车每天只运输一次,此时对应的全部邮路长度为 $L$。假设整个邮政网络的全部由区级车完成运输任务,则区级车每一班次行走的路线与上一个假设中县级车的行走路线相同,最短路线长度也为 $L$。由于区级车每天有两班车,故邮路总长度为 $2L$。根据极限法分析,为减少邮路总长度,在确定各个县的县局选址时,应尽量减少市局到县局的距离。

求解方法:启发式算法

根据极限法分析可知各县的县局要尽量靠近市局。由于区级邮政网络至少要覆盖市局附近的 16 个支局,且行每一班车行走线路相同。为尽量减少区级车的行走路线长度,县局的选址距离 16 个支局中最近的支局距离最短。

在保证市局距离县局尽量近的情况下,为保证各县内的支局到达县局的总线路尽量短,县局选址尽量在县区中心,即县局距离最远支局距离最短。

根据上述两种选址原则得到启发:县局选址既要靠近市区内任意一点的距离尽量短,又要尽量安排在县局中心。为解决这种矛盾,由于区级车行走线路对应成本权值是县级车行走线路对应权值的 2 倍,因此将县局选址问题抽象成如下问题:

将县区和市区内所有邮局抽象为点,令某县区内邮局点数量为 $m$,市区内的邮局点数量为 $n$,即 17。令县局内第 $i$ 个点到县区内第 $j$ 个点的最短距离为 $d_{ij} (i \neq j)$,县局内第 $i$ 个点到市区内第 $k$ 个点的最短距离为 $D_{ij}$。县局内第 $i$ 个点到县区内所有点中最长的距离为 $s_{i}$,县局内第 $i$ 个点到市区内所有点中最短的距离为 $S_{i}$。

为尽量缩短县区内区级车的路线长度,$s_{i}$ 尽量取最小值;为尽量缩短县局到市局的距离,$S_{i}$ 尽量取小值。为确定县局具体位置,根据县级邮路和市级邮路对应的成本权值,以 $s_{i} + 2S_{i}$ 最小为确定县局位置的标准。即:

\[
Min = s_{i} + 2S_{i}
\]

运用搜索求解的方法求解出各县的县局选址列表如下:

\begin{table}[htbp]
\centering
\caption{空间选址表}
\begin{tabular}{|c|c|c|c|c|c|c|c|c|c|}
\hline
\textbf{原县局编号} & \textbf{$X1$} & \textbf{$X2$} & \textbf{$X3$} & \textbf{$X4$} & \textbf{$X5$} \\ \hline
\textbf{原县局编号} & Z16 & X2 & Z29 & Z36 & Z52 \\ \hline
\end{tabular}
\end{table}

\begin{figure}[h]
    \centering
    \includegraphics[width=\textwidth]{image.png}
    \caption{图例}
    \label{fig:map}
\end{figure}

基于新县局下该市邮运网络规划

将县局选址后的结果带入问题二所建模型中,根据模型Lingo求解出该市县级邮路网络的邮路规划和邮车调度方案如下:(各邮路可双向行驶,区级邮路时间为两班车运输最短总时间)

\begin{table}[h]
    \centering
    \caption{县局选址后,该市邮路规划及邮车调度方案}
    \label{tab:post_route}
    \begin{tabular}{|c|p{10cm}|c|c|}
        \hline
        县 & 邮路(不包含仅路过的支局)区、新县邮路总费用为:8997元 & 总时间 & 费用 \\
        \hline
        \multirow{2}{*}{X1} & $Z16 \rightarrow Z9 \rightarrow Z8 \rightarrow Z7 \rightarrow Z6 \rightarrow Z5 \rightarrow Z2 \rightarrow Z3 \rightarrow Z4 \rightarrow Z10 \rightarrow Z16$ & 6.0167 & \multirow{2}{*}{951} \\
        \cline{2-3}
        & $Z16 \rightarrow Z11 \rightarrow Z12 \rightarrow Z13 \rightarrow Z1 \rightarrow X1 \rightarrow Z14 \rightarrow Z15 \rightarrow Z16$ & 4.9000 & \\
        \hline
        \multirow{2}{*}{X2} & $X2 \rightarrow Z22 \rightarrow Z23 \rightarrow Z24 \rightarrow Z25 \rightarrow Z26 \rightarrow X2$ & 5.3500 & \multirow{2}{*}{780} \\
        \cline{2-3}
        & $X2 \rightarrow Z17 \rightarrow Z19 \rightarrow Z20 \rightarrow Z21 \rightarrow X2$ & 4.0667 & \\
        \hline
        \multirow{2}{*}{X3} & $Z29 \rightarrow Z33 \rightarrow X3 \rightarrow Z27 \rightarrow Z28 \rightarrow Z29$ & 4.8667 & \multirow{2}{*}{606} \\
        \cline{2-3}
        & $Z29 \rightarrow Z31 \rightarrow Z32 \rightarrow Z30 \rightarrow Z29$ & 2.4500 & \\
        \hline
        \multirow{2}{*}{X4} & $Z36 \rightarrow Z37 \rightarrow Z38 \rightarrow Z39 \rightarrow Z40 \rightarrow Z43 \rightarrow X4 \rightarrow Z36$ & 5.7333 & \multirow{2}{*}{1038} \\
        \cline{2-3}
        & $Z36 \rightarrow Z35 \rightarrow Z34 \rightarrow Z41 \rightarrow Z42 \rightarrow Z36$ & 6.6333 & \\
        \hline
        \multirow{2}{*}{X5} & $Z52 \rightarrow Z51 \rightarrow Z46 \rightarrow Z44 \rightarrow Z45 \rightarrow Z47 \rightarrow Z48 \rightarrow Z49 \rightarrow Z50 \rightarrow Z52$ & 6.0000 & \multirow{2}{*}{972} \\
        \cline{2-3}
        & $Z52 \rightarrow Z53 \rightarrow X5 \rightarrow Z54 \rightarrow Z55 \rightarrow Z56 \rightarrow Z57 \rightarrow Z52$ & 5.9667 & \\
        \hline
        \multirow{4}{*}{区级} & $D \rightarrow Z62 \rightarrow Z16 \rightarrow Z63 \rightarrow Z65 \rightarrow Z29 \rightarrow Z68 \rightarrow D$ & 9.6013 & \multirow{4}{*}{4650} \\
        \cline{2-3}
        & $D \rightarrow Z73 \rightarrow Z72 \rightarrow D \rightarrow Z66 \rightarrow Z63 \rightarrow Z18 \rightarrow X2 \rightarrow Z64 \rightarrow Z67 \rightarrow D$ & 8.6308 & \\
        \cline{2-3}
        & $D \rightarrow Z71 \rightarrow Z36 \rightarrow Z70 \rightarrow Z69 \rightarrow D$ & 4.4295 & \\
        \cline{2-3}
        & $D \rightarrow Z61 \rightarrow Z58 \rightarrow Z52 \rightarrow Z59 \rightarrow Z60 \rightarrow D$ & 6.8513 & \\
        \hline
    \end{tabular}
\end{table}
将现有线路按第三问分区算法,重新对县分区: 
表4.3  各县局行政域规划表(Matlab程序见光盘M文件夹) 

\begin{table}
\centering
\caption{各县局行政域规划表 (Matlab程序见光盘M文件夹)}
\begin{tabular}{c c c c c}
\hline
X1 & X2 & X3 & X4 & X5 \\
\hline
Z1 & Z17 & Z27 & Z34 & Z44 \\
Z2 & Z19 & Z28 & Z35 & Z45 \\
Z3 & Z20 & Z29(原X3) & Z36(原X4) & Z46 \\
Z4 & Z21 & Z30 & Z37 & Z47 \\
Z5 & Z22 & Z31 & Z38 & Z48 \\
Z6 & Z23 & Z32 & Z39 & Z49 \\
Z7 & Z24 & Z33 & Z40 & Z50 \\
Z8 & Z25 & & Z41 & Z51 \\
Z9 & Z26 & & Z42 & Z52(原X5) \\
Z10 & & & Z43 & Z53 \\
Z11 & & & & Z54 \\
Z12 & & & & Z55 \\
Z13 & & & & Z56 \\
Z14 & & & & Z57 \\
Z15 & & & & \\
Z16(原X1) & & 注:原Z18由区级邮车负责 & & \\
\hline
\end{tabular}
\end{table}

运用第三问规划模型求解得目标值 (X1~X5 费用) 为:927、888、420、810、972元,分别使用2、2、1、2、2辆县级车,4辆区级车,全区、县总费用8667元,可以看出结合县局迁址与邻县互拉会更可观的节省费用(求解程序见光盘中Lingo目录)。

\begin{table}
\centering
\caption{县局选址、邻县收发边界,该市邮路规划及邮车调度方案}
\begin{tabular}{c p{10cm} c c}
\hline
县 & 邮路(不包含仅路过的支局)区、新县(允许拉临县站点)邮路总费用为:8667元 & 总时间 & 费用 \\
\hline
X1 & \begin{tabular}{l}
Z16 $\rightarrow$ Z10 $\rightarrow$ Z4 $\rightarrow$ Z3 $\rightarrow$ Z2 $\rightarrow$ Z5 $\rightarrow$ \\
Z6 $\rightarrow$ Z7 $\rightarrow$ Z8 $\rightarrow$ Z9 $\rightarrow$ Z16
\end{tabular} & 6.0167 & 927 \\
\cline{2-4}
 & Z16 $\rightarrow$ Z15 $\rightarrow$ Z11 $\rightarrow$ X1 $\rightarrow$ Z13 $\rightarrow$ Z1 $\rightarrow$ Z14 $\rightarrow$ Z16 $\rightarrow$ Z16 & 5.5333 & \\
\hline
X2 & X2 $\rightarrow$ Z27 $\rightarrow$ Z22 $\rightarrow$ Z23 $\rightarrow$ Z24 $\rightarrow$ Z25 $\rightarrow$ Z26 $\rightarrow$ Z21 $\rightarrow$ X2 & 6.0500 & 888 \\
\cline{2-4}
 & X2 $\rightarrow$ Z12 $\rightarrow$ Z17 $\rightarrow$ Z19 $\rightarrow$ Z20 $\rightarrow$ Z75 $\rightarrow$ X2 & 4.7333 & \\
\hline
X3 & \begin{tabular}{l}
Z29 $\rightarrow$ Z30 $\rightarrow$ Z34 $\rightarrow$ Z32 $\rightarrow$ Z33 $\rightarrow$ \\
X3 $\rightarrow$ Z28 $\rightarrow$ Z31 $\rightarrow$ Z29
\end{tabular} & 5.2500 & 420 \\
\hline
X4 & \begin{tabular}{l}
Z36 $\rightarrow$ Z38 $\rightarrow$ Z39 $\rightarrow$ Z40 $\rightarrow$ Z43 $\rightarrow$ \\
Z42 $\rightarrow$ Z41 $\rightarrow$ ZX4 $\rightarrow$ Z36
\end{tabular} & 6.8833 & 810 \\
\cline{2-4}
 & Z36 $\rightarrow$ Z37 $\rightarrow$ Z35 $\rightarrow$ Z36 & 2.8667 & \\
\hline
X5 & \begin{tabular}{l}
Z52 $\rightarrow$ Z51 $\rightarrow$ Z46 $\rightarrow$ Z44 $\rightarrow$ Z45 \\
$\rightarrow$ Z47 $\rightarrow$ Z48 $\rightarrow$ Z49 $\rightarrow$ Z50 $\rightarrow$ Z52
\end{tabular} & 6.0000 & 972 \\
\cline{2-4}
 & Z52 $\rightarrow$ Z53 $\rightarrow$ X5 $\rightarrow$ Z54 $\rightarrow$ Z55 $\rightarrow$ Z56 $\rightarrow$ Z57 $\rightarrow$ Z52 & 5.9667 & \\
\hline
区级 & 同表4.2 & & 4650 \\
\hline
\end{tabular}
\end{table}

关于县局选址的规划建议

县局选址的合理与否对构建经济、快速的邮政运输网络起到决定性的作用。我们通过对本市现有邮政运输网络的研究,认为目前县局的选址并不完全合理,为了提高邮政运输效益,我们建议将当前部分县局所在位置进行迁移(见表 4.1)具体情况分析如下。

我们做出如上建议是基于对目前县局设置及若迁址后邮路相关费用的计算得出的。通过建模求解可得到如下精确数据:

\begin{table}[h]
\centering
\begin{tabular}{c|c|c|c|c}
\hline
 & \multicolumn{2}{c|}{严格区域划分} & \multicolumn{2}{c}{松弛区域划分} \\
\hline
 & 总费用(元) & 总邮车数 & 总费用(元) & 总邮车数 \\
\hline
目前县局设置 & 9549 & 14 & 9267 & 13 \\
\hline
新县局设置 & 8997 & 14 & 8865 & 13 \\
\hline
\end{tabular}
\end{table}

由上表可见目前县局设置并不十分合理,下面主要阐述我们对各县邮局的迁移原则。由于邮政运输网络是分级网络。地市区内分区级网络和县级网络,由于区级网络每天分两个班次完成运输任务,而县级网络每天只有一班车。在进行县局选址规划时应尽量缩短区级车的行走路线。故县局应该尽量接近市局中心。又由于市区车要完成本区内所有支局的邮政运输任务。因此,可以使县局尽量接近市区内的某一支局。因此在只考虑缩短区级邮路的情况下,以县区与市区最接近的两点中的县级支局为县局,令此距离为 \(S\)。

另外,考虑到县区内所有支局的邮件运送到县局。考虑到所有支局到县局的总邮政线路近量短。根据各支局与县局间距离分布尽量均衡的思想,县局应尽量分布在县区中心。因此在只考虑缩短县级运输路线的情况下,以县区内到达任意支局中最长距离最短短的点作为县区的县局,令此距离为 \(L\)。

根据以上两种选址原则,二者互相矛盾。因此,在进行县局选址时要找到既靠近市局又靠近县区中心的平衡点。根据区级邮政线路与县级邮政线路的成本,可知区级成本权值是县级成本权值的 2 倍。为尽量减少邮路总成本,考虑采用对上面两种距离加权求和取最小值的办法确定县区内的最短距离。根据我们的求解结果可以看出此种方法简单易行且准确合理。

\section{9 模型的评价与扩展}

本文中模型的建立充分考虑了该区邮政运输流程及实限规定,在合理假设下完美的解决了该市邮政运输网络中的邮路规划和邮车调度问题,所建模型具有很强的通用性及实用性。
关于第一问模型的扩展

——松弛运输任务约束

在建立第一问模型时,根据题中关于需完成邮局运输任务的要求,严格限制了邮车将寄达各支局的邮件及由各支局寄出的邮件全部运完。实际中由于邮件量本身具有不确定性的特点,所以为达到效益最大的目标可松弛对运输任务的约束,即模型建立时可并不严格限制将所有邮件全部运输完,而是通过在目标中使空车率最低达到将邮件尽量运完的目的。

基于此,可针对问题一建立松弛运输任务约束后的理论模型。

【声明】:由于松弛运输任务约束后可能会有部分支局无法全部完成邮寄任务,不符合本文所要研究问题的基本假设,所以建立下面模型仅用于对问题深入思考后的学术性扩展研究讨论,模型本身可能不具可解性,并不作为本文中解决问题的数学模型。

\subsection{9.1 松弛运输任务约束后模型}

\subsubsection{9.1.1 扩展模型分析}

由于松弛约束后仅对问题一中含装卸货问题的约束 (1.2) 及目标有影响,并在运输任务的处理上与第一问不同,则下面仅对此进行分析。

约束松弛扩展分析
(1) 县局运输任务约束扩展
问题一中需要完成的运输任务为将县局中需寄送的邮件运送到各支局,同时将各支局中寄出的邮件运回县局,则对于运输任务分为县局与支局两部分考虑。

县局运输任务约束

在此讨论时需对邮路的进行重新抽象处理,将支局编号 \(1 \sim 16\),县局 \(X_{1}\) 看为图中 2 点,即始发点 17 及终止点 18。则对于从 17 发出的车 \(i\) 而言,其所载的邮件为需要分送到沿途各支局的邮件,即第 \(i\) 辆车在始点 17 装上的邮件数为沿途其余节点卸下的邮件数量之和。

设第 \(i\) 辆车最多经过 \(m\) 个顶点,以 \(H_{ijk}\) 表示第 \(i\) 辆车第 \(j\) 次装卸邮件时,收取第 \(k\) 个点寄出的邮件数;以 \(Q_{ijk}\) 表示第 \(i\) 辆车第 \(j\) 次装卸邮件时,在第 \(k\) 个顶点卸下的邮件数,则顶点 17 寄出的邮件数为 \(1 \sim 16\) 顶点卸下的邮件数之和可表示为:
\[
H_{i,1,17} = \sum_{j=1}^{m} \sum_{k=1}^{16} Q_{ijk}
\]

同理,在终点 18 处卸下的邮件为沿途其余节点装上的邮件数量之和,则节点 18 收寄的邮件总数为 \(1 \sim 16\) 节点装上的邮件数之和可表示为:
\[
\sum_{j=1}^{m} Q_{i,j,18} = \sum_{j=1}^{m} \sum_{k=1}^{16} H_{ijk}
\]

另外,邮车在始点不卸邮件,在终点不装邮件。即节点 17 处卸邮件量为 0,节点 18 处装邮件量为 0:
\[
Q_{i,1,17} = 0, \quad \sum_{j=1}^{m} H_{i,j,18} = 0
\]

支局运输任务松弛约束

表 2 中提供了该区内寄达局为 \(Z_{i}\) 的邮件量及由 \(Z_{i}\) 寄出的邮件量,为了满足该县运输需求,应使经过 \(Z_{i}\) 的各邮车卸下的邮件总数等于寄达 \(Z_{i}\) 的邮件量;同时,使收取的邮件总数等于由 \(Z_{i}\) 寄出的邮件量。

基于上面变量假设,由第 \(k\) 个节点寄出的邮件总量可表示为: \(\sum_{i=1}^{n} \sum_{j=1}^{m} H_{ijk}\)

寄达第 \(k\) 个节点的邮件总量可表示为: \(\sum_{i=1}^{n} \sum_{j=1}^{m} Q_{ijk}\)

以 \(G_{k}^{Q}\) 表示第 \(k\) 个节点寄出的邮件总量,\(G_{k}^{H}\) 表示寄达第 \(k\) 个节点的邮件总量,则在支局运输任务等式限制在本节讨论时可松弛为不等式限制:
\[
\sum_{i=1}^{n} \sum_{j=1}^{m} H_{ijk} \leq G_{k}^{Q}, \quad \sum_{i=1}^{n} \sum_{j=1}^{m} Q_{ijk} \leq G_{k}^{H}
\]

(2) 邮车运载能力约束扩展

由于邮车最多可载 65 袋邮件,则在松弛运输任务约束后由于允许有少量邮件剩余,则只要实际载邮件量在此约束内即可,由式 (1.2) 的分析可知,邮车在运输过程中所运载的邮件量随经过支局而不断变化,则邮车运载能力约束为:

\[
\sum_{j=1}^{m} \sum_{k=1}^{16} F_{ijk} G_{k}^{H} + \sum_{j=1}^{p} \sum_{k=1}^{16} F_{ijk} \left( Q_{ijk} - H_{ijk} \right) \leq 65 \quad (q = 1, 2, \ldots, m)
\]

目标:邮车运输成本最低的松弛扩展

基于问题一对目标的分析构建,可得松弛运输能力约束后的运输成本最低表达式为:

\[
Min \quad \sum_{i=1}^{n} \sum_{p=1}^{m-1} \left( \frac{2}{65} \left( 65 - \left( \sum_{j=1}^{m} \sum_{k=1}^{16} F_{ijk} G_{k}^{H} + \sum_{j=1}^{p} \sum_{k=1}^{16} F_{ijk} \left( Q_{ijk} - H_{ijk} \right) \right) \right) \sum_{k=1}^{17} \sum_{\substack{q=1 \\ q \neq k}}^{17} d_{kq} F_{ipk} F_{i,p+1,q} \right)
\]

\subsubsection{9.1.2 扩展模型建立}

以松弛运输能力约束(允许少量邮件不邮寄)后的运输成本最低为目标,建立对问题一的扩展模型如下:(其中与第一相同的约束这里直接引用)

\[
Min \quad \sum_{i=1}^{n} \sum_{p=1}^{m-1} \left( \frac{2}{65} \left( 65 - \left( \sum_{j=1}^{m} \sum_{k=1}^{16} F_{ijk} G_{k}^{H} + \sum_{j=1}^{p} \sum_{k=1}^{16} F_{ijk} \left( Q_{ijk} - H_{ijk} \right) \right) \right) \sum_{k=1}^{17} \sum_{\substack{q=1 \\ q \neq k}}^{17} d_{kq} F_{ipk} F_{i,p+1,q} \right)
\]

\[
\begin{aligned}
& \left\{
\begin{aligned}
H_{i,1,17} & = \sum_{j=1}^{m} \sum_{k=1}^{16} Q_{ijk} \quad ; \quad \sum_{j=1}^{m} H_{i,j,18} = 0 \\
\sum_{j=1}^{m} Q_{i,j,18} & = \sum_{j=1}^{m} \sum_{k=1}^{16} H_{ijk} \quad ; \quad Q_{i,1,17} = 0 \\
\sum_{i=1}^{n} \sum_{j=1}^{m} H_{ijk} & \leq G_{k}^{Q} \\
\sum_{i=1}^{n} \sum_{j=1}^{m} Q_{ijk} & \leq G_{k}^{H}
\end{aligned}
\right. \quad (i = 1, \ldots, n) \tag{9.1} \\
& S.T. \left\{
\begin{aligned}
\sum_{j=1}^{m} \sum_{k=1}^{16} F_{ijk} G_{k}^{H} + \sum_{j=1}^{q} \sum_{k=1}^{16} F_{ijk} \left( Q_{ijk} - H_{ijk} \right) & \leq 65 \quad (q = 1, \ldots, m) \tag{9.2} \\
\frac{5}{60} \sum_{j=1}^{m} \sum_{k=1}^{16} F_{ijk} + \sum_{j=1}^{m-1} \sum_{k=1}^{17} \sum_{q=1}^{17} \frac{d_{kq}}{30} F_{ijk} F_{i,j+1,q} & \leq 6 \quad (i = 1, \ldots, n) \tag{1.1} \\
\sum_{i=1}^{n} \sum_{j=1}^{m} F_{ijk} & = 1 \quad (k = 1, \ldots, 16) \tag{1.3} \\
\sum_{k=1}^{17} F_{ijk} & = 1 \\
F_{i,1,17} & = 1 \quad ; \quad F_{i,m,17} = 1 \quad (i = 1, \ldots, n) \tag{1.5} \\
F_{ijk} & \in \{0, 1\}
\end{aligned}
\right.
\]

《注》: $n$ 为所需的最少车辆数, 根据目标分析中对双目标转化成单目标的方法, 在模型求解时依次取 $n=1,2,3 \ldots$, 使模型有解的最小 $n$ 值即为所需最小车辆数。

模型说明:

(9.1) 松弛邮运任务约束后, 县局运输任务约束扩展;

(9.2) 松弛邮运任务约束后, 邮车运载能力约束扩展;

(1.*) 具体意义见模型一。

\begin{align*}
H_{ijk} & \quad \text{——第 } i \text{ 辆车第 } j \text{ 次装卸邮件时, 收取第 } k \text{ 个点寄出的邮件数;} \\
Q_{ijk} & \quad \text{——第 } i \text{ 辆车第 } j \text{ 次装卸邮件时, 在第 } k \text{ 个顶点卸下的邮件数;} \\
G_{k}^{Q} & \quad \text{——表示第 } k \text{ 个支局寄出的邮件总量;} \\
G_{k}^{H} & \quad \text{——表示寄达第 } k \text{ 个支局的邮件总量;} \\
F_{ijk} & \quad \text{——0-1变量, 表示第 } i \text{ 辆车第 } j \text{ 次装卸邮件是否在第 } k \text{ 个节点;} \\
d_{kj} & \quad \text{——图中节点 } k \text{ 到 } j \text{ 的最短路距离 (由 5.1 求得的最短路矩阵确定);}
\end{align*}

本模型通用性比较好, 但对于具体不同条件情况下可以简化, 在规模较大的时候应该考虑采用启发式算法求解。

\section{参考文献}

[1] 谢金星、薛毅, 优化建模与 LINDO/LINGO 软件, 北京: 清华大学出版社, 2005 年 7 月第一版

[2] Duane Hanselman、Bruce Littlefield 著, 朱仁峰译, Matlab 7, 北京: 清华大学出版社, 2006 年 5 月第一版

\section{附录}

\subsection{1.1 县X1最短路矩阵D,}

\begin{table}[h]
\centering
\begin{tabular}{|c|c|c|c|c|c|c|c|c|c|c|c|c|c|c|c|c|c|c|c|}
\hline
0 & 31 & 27 & 38 & 51 & 58 & 71 & 67 & 57 & 47 & 52 & 48 & 21 & 41 & 52 & 61 & 27 \\
\hline
31 & 0 & 19 & 33 & 27 & 32 & 45 & 64 & 53 & 47 & 61 & 57 & 52 & 48 & 56 & 63 & 36 \\
\hline
27 & 19 & 0 & 14 & 27 & 34 & 47 & 49 & 39 & 29 & 42 & 38 & 38 & 29 & 38 & 44 & 17 \\
\hline
38 & 33 & 14 & 0 & 13 & 20 & 33 & 35 & 25 & 15 & 33 & 32 & 32 & 15 & 24 & 30 & 11 \\
\hline
51 & 27 & 27 & 13 & 0 & 9 & 21 & 37 & 26 & 26 & 43 & 45 & 45 & 28 & 29 & 38 & 24 \\
\hline
58 & 32 & 34 & 20 & 9 & 0 & 13 & 32 & 32 & 35 & 47 & 52 & 52 & 35 & 33 & 42 & 31 \\
\hline
71 & 45 & 47 & 33 & 21 & 13 & 0 & 19 & 30 & 39 & 50 & 65 & 65 & 48 & 44 & 40 & 44 \\
\hline
67 & 64 & 49 & 35 & 37 & 32 & 19 & 0 & 11 & 20 & 31 & 54 & 54 & 34 & 25 & 21 & 40 \\
\hline
57 & 53 & 39 & 25 & 26 & 32 & 30 & 11 & 0 & 10 & 20 & 43 & 43 & 24 & 14 & 13 & 30 \\
\hline
47 & 47 & 29 & 15 & 26 & 35 & 39 & 20 & 10 & 0 & 18 & 36 & 36 & 41 & 14 & 9 & 18 & 20 \\
\hline
52 & 61 & 42 & 33 & 43 & 47 & 50 & 31 & 20 & 18 & 0 & 23 & 23 & 46 & 25 & 14 & 23 & 25 \\
\hline
48 & 57 & 38 & 32 & 45 & 52 & 65 & 54 & 43 & 36 & 23 & 0 & 27 & 22 & 33 & 42 & 21 \\
\hline
21 & 52 & 38 & 32 & 45 & 52 & 65 & 61 & 51 & 41 & 46 & 46 & 27 & 0 & 39 & 48 & 57 & 21 \\
\hline
41 & 48 & 29 & 15 & 28 & 35 & 48 & 34 & 24 & 14 & 25 & 22 & 39 & 0 & 0 & 11 & 20 & 18 \\
\hline
52 & 56 & 38 & 24 & 29 & 33 & 44 & 25 & 14 & 9 & 14 & 33 & 48 & 11 & 0 & 9 & 27 \\
\hline
61 & 63 & 44 & 30 & 38 & 42 & 40 & 21 & 13 & 18 & 23 & 42 & 57 & 20 & 9 & 0 & 36 \\
\hline
27 & 36 & 17 & 11 & 24 & 31 & 44 & 40 & 30 & 20 & 25 & 21 & 21 & 18 & 27 & 36 & 0 \\
\hline
\end{tabular}
\end{table}

\subsection{1.2.0 区最短路矩阵D0,}

\begin{table}[h]
\centering
\begin{tabular}{|c|c|c|c|c|c|c|c|c|c|c|c|c|c|c|c|c|c|c|c|}
\hline
0 & 21 & 37 & 41 & 45 & 86 & 101 & 90 & 67 & 76 & 97 & 88 & 99 & 84 & 76 & 71 & 97 & 134 & 143 & 116 & 61 & 63 \\
\hline
21 & 0 & 17 & 23 & 29 & 68 & 83 & 72 & 49 & 58 & 79 & 70 & 81 & 66 & 56 & 51 & 81 & 116 & 125 & 96 & 79 & 45 \\
\hline
37 & 17 & 0 & 40 & 33 & 74 & 96 & 85 & 62 & 61 & 77 & 68 & 75 & 58 & 39 & 34 & 85 & 122 & 138 & 79 & 96 & 43 \\
\hline
41 & 23 & 40 & 0 & 19 & 45 & 60 & 49 & 26 & 35 & 56 & 47 & 58 & 43 & 41 & 50 & 71 & 93 & 102 & 88 & 102 & 22 \\
\hline
45 & 29 & 33 & 19 & 0 & 41 & 68 & 63 & 42 & 51 & 73 & 65 & 76 & 61 & 59 & 67 & 52 & 89 & 116 & 106 & 106 & 40 \\
\hline
86 & 68 & 74 & 45 & 41 & 0 & 27 & 22 & 19 & 32 & 39 & 49 & 60 & 61 & 60 & 69 & 70 & 48 & 75 & 107 & 147 & 41 \\
\hline
101 & 83 & 96 & 60 & 68 & 27 & 0 & 18 & 34 & 37 & 44 & 55 & 66 & 73 & 74 & 83 & 85 & 44 & 58 & 121 & 162 & 55 \\
\hline
90 & 72 & 85 & 49 & 63 & 22 & 18 & 0 & 23 & 30 & 26 & 37 & 48 & 57 & 64 & 73 & 92 & 59 & 53 & 109 & 151 & 45 \\
\hline
67 & 49 & 62 & 26 & 42 & 19 & 34 & 23 & 0 & 13 & 34 & 30 & 41 & 42 & 41 & 50 & 89 & 67 & 76 & 88 & 128 & 22 \\
\hline
76 & 58 & 61 & 35 & 51 & 32 & 37 & 30 & 13 & 0 & 22 & 22 & 31 & 36 & 37 & 46 & 102 & 80 & 83 & 84 & 137 & 18 \\
\hline
97 & 79 & 77 & 56 & 73 & 39 & 44 & 26 & 34 & 22 & 0 & 16 & 24 & 36 & 46 & 58 & 109 & 85 & 66 & 88 & 158 & 34 \\
\hline
88 & 70 & 68 & 47 & 65 & 49 & 55 & 37 & 30 & 22 & 16 & 0 & 11 & 20 & 30 & 42 & 117 & 96 & 82 & 72 & 149 & 25 \\
\hline
99 & 81 & 75 & 58 & 76 & 60 & 66 & 48 & 41 & 31 & 24 & 11 & 0 & 17 & 36 & 48 & 128 & 107 & 79 & 65 & 160 & 36 \\
\hline
84 & 66 & 58 & 43 & 61 & 61 & 73 & 57 & 42 & 36 & 36 & 20 & 17 & 0 & 19 & 31 & 113 & 109 & 96 & 52 & 145 & 21 \\
\hline
76 & 56 & 39 & 41 & 59 & 60 & 74 & 64 & 41 & 37 & 46 & 30 & 36 & 19 & 0 & 13 & 111 & 108 & 112 & 47 & 135 & 19 \\
\hline
71 & 51 & 34 & 50 & 67 & 69 & 83 & 73 & 50 & 46 & 58 & 42 & 48 & 31 & 13 & 0 & 119 & 117 & 124 & 45 & 130 & 28 \\
\hline
97 & 81 & 85 & 71 & 52 & 70 & 85 & 92 & 89 & 102 & 109 & 117 & 128 & 113 & 111 & 119 & 0 & 62 & 116 & 158 & 107 & 92 \\
\hline
134 & 116 & 122 & 93 & 89 & 48 & 44 & 59 & 67 & 80 & 85 & 96 & 107 & 109 & 108 & 117 & 62 & 0 & 54 & 155 & 169 & 89 \\
\hline
143 & 125 & 138 & 102 & 116 & 75 & 58 & 53 & 76 & 83 & 66 & 82 & 79 & 96 & 112 & 124 & 116 & 54 & 0 & 133 & 204 & 98 \\
\hline
116 & 96 & 79 & 88 & 106 & 107 & 121 & 109 & 88 & 84 & 88 & 72 & 65 & 52 & 47 & 45 & 158 & 155 & 133 & 0 & 119 & 66 \\
\hline
61 & 79 & 96 & 102 & 106 & 147 & 162 & 151 & 128 & 137 & 158 & 149 & 160 & 145 & 135 & 130 & 107 & 169 & 204 & 119 & 0 & 124 \\
\hline
63 & 45 & 43 & 22 & 40 & 41 & 55 & 45 & 22 & 18 & 34 & 25 & 36 & 21 & 19 & 28 & 92 & 89 & 98 & 66 & 124 & 0 \\
\hline
\end{tabular}
\end{table}

\subsection{1.2.1 县X1最短路矩阵D1}

\begin{table}[h]
\centering
\begin{tabular}{|c|c|c|c|c|c|c|c|c|c|c|c|c|c|c|c|c|}
\hline
0 & 31 & 27 & 38 & 51 & 58 & 71 & 67 & 52 & 48 & 21 & 41 & 52 & 61 & 27 \\
\hline
31 & 0 & 19 & 33 & 27 & 32 & 45 & 64 & 61 & 57 & 52 & 48 & 36 \\
\hline
27 & 19 & 0 & 14 & 27 & 34 & 47 & 49 & 42 & 38 & 38 & 29 & 17 \\
\hline
\end{tabular}
\end{table}

\begin{table}
\centering
\begin{tabular}{|c|c|c|c|c|c|c|c|c|c|c|c|c|}
\hline
38 & 33 & 14 & 0 & 13 & 20 & 33 & 35 & 33 & 32 & 32 & 15 & 11 \\
\hline
51 & 27 & 27 & 13 & 0 & 9 & 21 & 37 & 43 & 45 & 45 & 28 & 24 \\
\hline
58 & 32 & 34 & 20 & 9 & 0 & 13 & 32 & 47 & 52 & 52 & 35 & 31 \\
\hline
71 & 45 & 47 & 33 & 21 & 13 & 0 & 19 & 50 & 65 & 65 & 48 & 44 \\
\hline
67 & 64 & 49 & 35 & 37 & 32 & 19 & 0 & 31 & 54 & 61 & 34 & 40 \\
\hline
52 & 61 & 42 & 33 & 43 & 47 & 50 & 31 & 0 & 23 & 46 & 25 & 25 \\
\hline
48 & 57 & 38 & 32 & 45 & 52 & 65 & 54 & 23 & 0 & 27 & 22 & 21 \\
\hline
21 & 52 & 38 & 32 & 45 & 52 & 65 & 61 & 46 & 27 & 0 & 39 & 21 \\
\hline
41 & 48 & 29 & 15 & 28 & 35 & 48 & 34 & 25 & 22 & 39 & 0 & 18 \\
\hline
27 & 36 & 17 & 11 & 24 & 31 & 44 & 40 & 25 & 21 & 21 & 18 & 0 \\
\hline
\end{tabular}
\caption{1.2.2 县X2最短路矩阵D2}
\end{table}

\begin{table}
\centering
\begin{tabular}{|c|c|c|c|c|c|c|c|c|c|c|c|}
\hline
0 & 38 & 30 & 42 & 42 & 51 & 63 & 81 & 61 & 69 & 22 \\
\hline
38 & 0 & 39 & 36 & 36 & 45 & 57 & 75 & 55 & 63 & 16 \\
\hline
30 & 39 & 0 & 15 & 40 & 49 & 58 & 63 & 43 & 42 & 24 \\
\hline
42 & 36 & 15 & 0 & 25 & 34 & 43 & 48 & 28 & 27 & 20 \\
\hline
42 & 36 & 40 & 25 & 0 & 9 & 21 & 39 & 19 & 32 & 20 \\
\hline
51 & 45 & 49 & 34 & 9 & 0 & 19 & 39 & 27 & 40 & 29 \\
\hline
63 & 57 & 58 & 43 & 21 & 19 & 0 & 20 & 22 & 35 & 41 \\
\hline
81 & 75 & 63 & 48 & 39 & 39 & 20 & 0 & 20 & 33 & 59 \\
\hline
61 & 55 & 43 & 28 & 19 & 27 & 22 & 20 & 0 & 13 & 39 \\
\hline
69 & 63 & 42 & 27 & 32 & 40 & 35 & 33 & 13 & 0 & 47 \\
\hline
22 & 16 & 24 & 20 & 20 & 29 & 41 & 59 & 39 & 47 & 0 \\
\hline
\end{tabular}
\caption{1.2.3 县X3最短路矩阵D3}
\end{table}

\begin{table}
\centering
\begin{tabular}{|c|c|c|c|c|}
\hline
0 & 17 & 9 & 28 & 30 \\
\hline
17 & 0 & 18 & 11 & 39 \\
\hline
9 & 18 & 0 & 29 & 21 \\
\hline
28 & 11 & 29 & 0 & 42 \\
\hline
30 & 39 & 21 & 42 & 0 \\
\hline
43 & 52 & 34 & 62 & 20 \\
\hline
\end{tabular}
\caption{1.2.4 县X4最短路矩阵D4}
\end{table}

\begin{table}
\centering
\begin{tabular}{|c|c|c|c|c|c|c|c|c|c|}
\hline
0 & 15 & 44 & 47 & 64 & 83 & 86 & 93 & 98 & 71 \\
\hline
15 & 0 & 29 & 32 & 49 & 68 & 71 & 78 & 83 & 56 \\
\hline
44 & 29 & 0 & 20 & 37 & 53 & 42 & 49 & 54 & 27 \\
\hline
47 & 32 & 20 & 0 & 17 & 36 & 42 & 60 & 57 & 30 \\
\hline
64 & 49 & 37 & 17 & 0 & 19 & 37 & 58 & 55 & 28 \\
\hline
83 & 68 & 53 & 36 & 19 & 0 & 18 & 56 & 47 & 26 \\
\hline
86 & 71 & 42 & 42 & 37 & 18 & 0 & 38 & 29 & 15 \\
\hline
93 & 78 & 49 & 60 & 58 & 56 & 38 & 0 & 29 & 30 \\
\hline
98 & 83 & 54 & 57 & 55 & 47 & 29 & 29 & 0 & 27 \\
\hline
71 & 56 & 27 & 30 & 28 & 26 & 15 & 30 & 27 & 0 \\
\hline
\end{tabular}
\caption{1.2.5 县X5最短路矩阵D5}
\end{table}

\begin{table}
\centering
\begin{tabular}{|c|c|c|c|c|c|c|c|c|c|c|c|c|}
\hline
0 & 20 & 22 & 31 & 40 & 50 & 58 & 41 & 69 & 91 & 93 & 121 & 108 & 72 \\
\hline
20 & 0 & 15 & 11 & 20 & 34 & 49 & 30 & 58 & 80 & 84 & 112 & 100 & 61 \\
\hline
\end{tabular}
\end{table}

\begin{table}
\centering
\begin{tabular}{|c|c|c|c|c|c|c|c|c|c|c|c|c|c|}
\hline
22 & 15 & 0 & 18 & 25 & 28 & 36 & 19 & 47 & 69 & 71 & 99 & 86 & 50 \\
\hline
31 & 11 & 18 & 0 & 9 & 23 & 38 & 19 & 47 & 69 & 73 & 101 & 89 & 50 \\
\hline
40 & 20 & 25 & 9 & 0 & 17 & 32 & 20 & 48 & 70 & 74 & 102 & 90 & 51 \\
\hline
50 & 34 & 28 & 23 & 17 & 0 & 15 & 15 & 37 & 57 & 63 & 91 & 81 & 35 \\
\hline
58 & 49 & 36 & 38 & 32 & 15 & 0 & 22 & 22 & 42 & 48 & 76 & 66 & 20 \\
\hline
41 & 30 & 19 & 19 & 20 & 15 & 22 & 0 & 28 & 50 & 54 & 82 & 70 & 31 \\
\hline
69 & 58 & 47 & 47 & 48 & 37 & 22 & 28 & 0 & 40 & 26 & 54 & 44 & 19 \\
\hline
91 & 80 & 69 & 69 & 70 & 57 & 42 & 50 & 40 & 0 & 14 & 42 & 33 & 22 \\
\hline
93 & 84 & 71 & 73 & 74 & 63 & 48 & 54 & 26 & 14 & 0 & 28 & 41 & 36 \\
\hline
121 & 112 & 99 & 101 & 102 & 91 & 76 & 82 & 54 & 42 & 28 & 0 & 13 & 64 \\
\hline
108 & 100 & 86 & 89 & 90 & 81 & 66 & 70 & 44 & 33 & 41 & 13 & 0 & 55 \\
\hline
72 & 61 & 50 & 50 & 51 & 35 & 20 & 31 & 19 & 22 & 36 & 64 & 55 & 0 \\
\hline
\end{tabular}
\end{table}

\subsection{2.0.1 前期数据处理 matlab 关键程序}

(全部程序见光盘中 M 文件夹, Matlab 版本 2007b)

\begin{verbatim}
%data
a=[1,17,27,34,44,58];
b=[16,26,33,43,57,73];
%直达距离
%第1~3问文件,题中提供源文件
[L,S]=xlsread('2007年D题 附件 邮局间直达公路里程.xls',1,'A2:C341');
GQ=[9,14,5,10,9,10,13,9,15,9,6,7,13,15,10,16];%寄达
GH=[10,15,6,9,13,6,11,4,13,17,11,2,11,21,13,14];%送出
d=10000*ones(18,18);
for i=1:size(S,1)
    if strcmp(S{i,1},'X1')
        if strcmp(S{i,2}(1),'Z')
            t1=str2double(S{i,2}(2:end));
            if t1<=16
                d(17,t1)=L(i);
                d(t1,17)=L(i);
                d(18,t1)=L(i);
                d(t1,18)=L(i);
            end
        end
    end
    if strcmp(S{i,1}(1),'Z') && strcmp(S{i,2}(1),'Z')
        t1=str2double(S{i,1}(2:end));
        t2=str2double(S{i,2}(2:end));
        if t1<=16 && t2<=16
            d(t1,t2)=L(i);
            d(t2,t1)=L(i);
        end
    end
end
%% 第一问
D=zeros(79,79);
for i=1:size(S,1)
    if strcmp(S{i,1}(1),'X') && strcmp(S{i,2}(1),'Z')
        t1=str2double(S{i,1}(2:end))+73;
        t2=str2double(S{i,2}(2:end));
        D(t1,t2)=L(i);
        D(t2,t1)=L(i);
    end
    if strcmp(S{i,1}(1),'Z') && strcmp(S{i,2}(1),'Z')
\end{verbatim}

\begin{lstlisting}
t1=str2double(S{i,1}(2:end));
t2=str2double(S{i,2}(2:end));
D(t1,t2)=L(i);
D(t2,t1)=L(i);
end
if strcmp(S{i,1}(1),'D') && strcmp(S{i,2}(1),'Z')
t1=79;
t2=str2double(S{i,2}(2:end));
D(t1,t2)=L(i);
D(t2,t1)=L(i);
end
end
%% 区级
D22=zeros(22,22);
for i=58:79
    for j=58:79
        if i==j
            continue
        end
        if i==79
            D22(i-57,j-57)=K2(i,j);
            D22(i-56,j-57)=K2(i,j);
            continue
        end
        if j==79
            D22(i-57,j-57)=K2(i,j);
            D22(i-57,j-56)=K2(i,j);
            continue
        end
        D22(i-57,j-57)=K2(i,j);
    end
end
%% 1 县
X21_T=[9,10,15,16];
t1=a(1);
t2=b(1);
t=t2-t1+2;
X21=zeros(t+1,t+1);
for i=t1:t2
    if IST(i,X21_T)
        continue
    end
    X21(i-t1+1,t)=K2(i,74);
    X21(t,i-t1+1)=K2(74,i);
    for j=t1:t2
        if IST(j,X21_T)
            continue
        end
        X21(i-t1+1,j-t1+1)=K2(i,j);
    end
end
X21(t+1,:)= [t1:t2 74 0];
X21(:,t+1)=X21(t+1,:)';
i=1;
while i<=t+1-length(X21_T)
    if sum(X21(i,1:end-1))==0
        X21(i,:)= [];
        X21(:,i)= [];
        continue
    end
    i=i+1;
\end{lstlisting}

\begin{verbatim}
end
%% 2 县
t1=a(2);
t2=b(2);
t=t2-t1+2;
X22=zeros(t+1,t+1);
for i=t1:t2
    X22(i-t1+1,t)=K2(i,75);
    X22(t,i-t1+1)=K2(75,i);
    for j=t1:t2
        X22(i-t1+1,j-t1+1)=K2(i,j);
    end
end
X22(t+1,:)= [t1:t2 75 0];
X22(:,t+1)=X22(t+1,:);

%% 3 县
X23_T=[27,28];
t1=a(3);
t2=b(3);
t=t2-t1+2;
X23=zeros(t+1,t+1);
for i=t1:t2
    if IST(i,X23_T)
        continue
    end
    X23(i-t1+1,t)=K2(i,76);
    X23(t,i-t1+1)=K2(76,i);
    for j=t1:t2
        if IST(j,X23_T)
            continue
        end
        X23(i-t1+1,j-t1+1)=K2(i,j);
    end
end
X23(t+1,:)= [t1:t2 76 0];
X23(:,t+1)=X23(t+1,:);
i=1;
while i<=t+1-length(X23_T)
    if sum(X23(i,1:end-1))==0
        X23(i,:)= [];
        X23(:,i)= [];
        continue
    end
    i=i+1;
end
%% 4 县
X24_T=[41];
t1=a(4);
t2=b(4);
t=t2-t1+2;
X24=zeros(t+1,t+1);
for i=t1:t2
    if IST(i,X24_T)
        continue
    end
    X24(i-t1+1,t)=K2(i,77);
    X24(t,i-t1+1)=K2(77,i);
    for j=t1:t2
        if IST(j,X24_T)

\begin{lstlisting}
continue
end
    X24(i-t1+1,j-t1+1)=K2(i,j);
end
end
X24(t+1,:)=[t1:t2 77 0];
X24(:,t+1)=X24(t+1,:)';
i=1;
while i<=t+1-length(X24_T)
    if sum(X24(i,1:end-1))==0
        X24(i,:)=[];
        X24(:,i)=[];
        continue
    end
    i=i+1;
end
%% 5 县
X25_T=[52];
t1=a(5);
t2=b(5);
t=t2-t1+2;
X25=zeros(t+1,t+1);
for i=t1:t2
    if IST(i,X25_T)
        continue
    end
    X25(i-t1+1,t)=K2(i,78);
    X25(t,i-t1+1)=K2(78,i);
    for j=t1:t2
        if IST(j,X25_T)
            continue
        end
        X25(i-t1+1,j-t1+1)=K2(i,j);
    end
end
X25(t+1,:)=[t1:t2 78 0];
X25(:,t+1)=X25(t+1,:)';
i=1;
while i<=t+1-length(X25_T)
    if sum(X25(i,1:end-1))==0
        X25(i,:)=[];
        X25(:,i)=[];
        continue
    end
    i=i+1;
end
\end{lstlisting}

\subsection{2.0.1 最短路 Dijkstra 算法,matlab 程序}

\begin{lstlisting}
TT=D;
tem=1:79;
K=zeros(79,79);
FANG(1:79,1:79)={' '};
for c=1:79
    t1=TT(c,:);
    for i=1:79
        if t1(i)==0
            t1(i)=inf;
        end
    end
    t1(2,:)=tem;
    key=inf*ones(1,79);
\end{lstlisting}

\begin{lstlisting}
key(c)=0;
key(TT(c,:)==0)=TT(c,TT(c,:)==0);
Dij_S(1:79)={' '};
for i=1:79
    if TT(c,i)==0
        Dij_S{i}=strcat(num2str(c),'->',num2str(i));
    end
end
%核心算法
t1(:,c)=[];
t1=T_SORT(t1,1,1);%自编排序
while ~isempty(t1)
    t2=tem(TT(t1(2,1),:)==0);
    for j=1:length(t2)
        if t1(2,1)==t2(j)
            t=key(t1(2,1))+TT(t1(2,1),t2(j));%行程时间
            if key(t2(j))>t
                key(t2(j))=t;
                Dij_S{t2(j)}=strcat(Dij_S{t1(2,1)},'->',num2str(t2(j)));
            end
        end
    end
    t1(:,1)=[];
    for i=1:size(t1,2)
        if t1(1,i)>key(t1(2,i))
            t1(1,i)=key(t1(2,i));
        end
    end
    t1=T_SORT(t1,1,1);
end
K(c,:)=key;
FANG(c,:)=Dij_S;
end
\end{lstlisting}

\subsection{2.1 第一问县X1最小空载费用}

\begin{lstlisting}
(model 1, Lingo程序)
model:
sets:
C1/1..3/:;
C2/1..8/:;
C3/1..7/:;
S1/1..17/:;
S2/1..16/:GH,GQ;
L1(S1,S1):d;
L2(C1,C2,S1):F;
endsets
data:
d=@file('d2.txt');
GH=10,15,6,9,13,6,11,4,13,17,11,2,11,21,13,14;
GQ=9,14,5,10,9,10,13,9,15,9,6,7,13,15,10,16;
enddata
min=2/65*@sum(C1(i):@sum(C3(p):
    @sum(L1(k,q)|k#ne#q:d(k,q)*F(i,p,k)*F(i,p+1,q))*
    (65-
    @sum(C2(j):@sum(S2(k):F(i,j,k)*GH(k)))
    -@sum(C2(j)|j#le#p:@sum(S2(k):F(i,j,k)*(GQ(k)-GH(k)))))
));
\end{lstlisting}

\begin{lstlisting}
));
! 总工作时间;
@for (C1(i):
5/60*@sum(C2(j):@sum(S2(k):F(i,j,k))) +
1/30*@sum(C3(j):@sum(L1(k,q)|k#ne#q:F(i,j,k)*F(i,j+1,q)*d(k,q))) <=6);
! 车容量;
@for (C1(i):
@for (C2(q):
@sum(C2(j):@sum(S2(k):F(i,j,k)*GH(k))) +
@sum(C2(j)|j#le#q:@sum(S2(k):F(i,j,k)*(GQ(k)-GH(k))))<=65
);
);
@for (C1(i):
@for (C2(j):
@sum(S1(k):F(i,j,k))=1;
F(i,1,@size(S1))=1;
F(i,@size(C2),@size(S1))=1;
);
);

! 每点都到;
@for (S2(k):@sum(C1(i):@sum(C2(j)|j#ne#1#and#j#ne#@size(C2):F(i,j,k))) >=1);
@for (L2:@bin(F));
end
\end{lstlisting}

\subsection{2.2.1 第二问市区邮路规划}

\subsubsection{(模型 II-i, Lingo 程序)}

\begin{lstlisting}
model:
sets:
C1/1..4/:;
C2/1..8/:;
C3/1..7/:;
S1/1..23/:;
S2/1..21/:;
S3/1..16/:;
L1(S1,S1):d;
L2(C1,C2,S1):F;
endsets
data:
d=@file('d22.txt');
enddata
min=3*@sum(C1(i):@sum(C3(p):
@sum(L1(k,q)|k#ne#q:d(k,q)*F(i,p,k)*F(i,p+1,q))
));
! 总工作时间;
@for (C1(i):
5/60*@sum(C2(j):@sum(S3(k):F(i,j,k))) +
1/6*@sum(C2(j):@sum(S2(k)|k#gt#16:F(i,j,k))) +
1/65*@sum(C3(j):@sum(L1(k,q)|k#ne#q:F(i,j,k)*F(i,j+1,q)*d(k,q))) <=5);
@for (S2(k):
@sum(C1(i):@sum(C2(j)|j#ne#1#and#j#ne#@size(C2):F(i,j,k)))>=1
);
@for (C1(i):
@for (C2(j):@sum(S1(k):F(i,j,k))=1);
\end{lstlisting}

\begin{lstlisting}
F(i,1,@size(S1))=1;
F(i,@size(c2),@size(S1))=1;
);
@for(L2:@bin(F));
end
\end{lstlisting}

\subsection{第二问县区X1邮路规划}
(模型II-ii, Lingo程序)

\begin{lstlisting}
model:
sets:
C1/1..2/:;
C2/1..11/:;
C3/1..10/:;
S1/1..16/:;
S2/1..15/:;
L1(S1,S1):d;
L2(C1,C2,S1):F;
endsets
data:
d=@file('D41.txt');
enddata
min=3*@sum(C1(i):@sum(C3(p):
@sum(L1(k,q)|k#ne#q:d(k,q)*F(i,p,k)*F(i,p+1,q))
));
!总工作时间;
@for(C1(i):
5/60*@sum(C2(j):@sum(S2(k):F(i,j,k)))+
1/30*@sum(C3(j):@sum(L1(k,q)|k#ne#q:F(i,j,k)*F(i,j+1,q)*d(k,q)))
<=6.7411);

@for(S2(k):
@sum(C1(i):@sum(C2(j)|j#ne#1#and#j#ne#@size(C2):F(i,j,k)))=1
);
@for(C1(i):
@for(C2(j):@sum(S1(k):F(i,j,k))=1);

F(i,1,@size(S1))=1;
F(i,@size(c2),@size(S1))=1;
);
@for(L2:@bin(F));
end
\end{lstlisting}

第二、三、四问县区X1~X5邮路规划(见光盘中Lingo目录)