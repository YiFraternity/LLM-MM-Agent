\title{实用下料的数学模型}

\author{陈璐,黄伟健,冯真 \\ 指导老师:数模指导组 \\ (信息工程大学,郑州 450004)}

\maketitle

\begin{abstract}
摘要:考虑到整数规划模型的下料方式数量难以穷尽的问题,本文以原材料最少为目标,采用启发式多级序列线性优化的方法建立一维下料模型。对于二维下料问题,采用降维启发式的方法即通过形成“板条”把二维下料问题化为一维下料问题。
\end{abstract}

\section{问题分析}

一维下料问题是组合优化中的一个经典问题,如果要得到理论上的严格全局最优下料方案,就要求所有可行下料方式都进入线性规划模型的系数矩阵,从计算的复杂性理论上看,这属于 NPC(NP 完全)难问题,所以,我们放弃常规的整数规划解法,而在以优化选取下料方式的前提下,寻找建立下料方案的模型。

本题要求一个好的下料方案在生产能力允许的条件下要满足三个要求:首先,应该使原材料的利用率最大,即用最少数量的原材料;其次要求所采用的不同下料方式尽可能少;还要满足每种零件各自的交货时间。这样,我们可以考虑分层建模:先仅考虑所用原材料最少的模型,再考虑下料方式增加时的方案,最后在解决具体问题时,再进一步考虑对零件加工的时间限制,改进方案以满足要求。

对于二维下料问题,我们受一维下料问题处理方式的启发,采用降维启发式方法。即通过形成“板条”而把二维下料问题降为一维下料问题。

\section{模型建立与求解}

\subsection{一维下料模型}

(1)一维下料模型——启发式多级序列线性优化方法

该模型的基本思想是在每级求解时,尽可能多的重复使用最优的一种方法进行下料,直到所涉及到的某种零件需求加工完;然后对剩余的零件重复上步的操作,直到所有剩余的零件数目均减小至零为止。原问题的最优解就是各个序列优化问题所求得的最优下料方式的总和。

给定 $m$ 种长度的零件 $l_{1}, l_{2}, \cdots, l_{m}$,所需的数量分别为 $b_{1}, b_{2}, \cdots, b_{m}$,已知原材料长度为 $L$。设在最优一种下料方式中,第 $i$ 件零件的加工数量为 $a_{i}$,由此建立如下模型:

\begin{equation}
\begin{aligned}
& \max S = \sum_{i=1}^{m} a_{i} l_{i} \\
& \text{st} \left\{
\begin{aligned}
& \sum_{i=1}^{m} a_{i} l_{i} \leq L \\
& 0 \leq a_{i} \leq b_{i} \quad i = 1, 2, \cdots, m
\end{aligned}
\right.
\end{aligned}
\end{equation}

优化参数变量:$a_{1}, a_{2}, \cdots, a_{m}$ 均为非负整数,且不同长度的零件种类有限(即该问题中要求的变量个数有限),可用分枝定界法来求解。

(2)启发式多级序列线性优化计算方法

将上述当前最优下料方式计算求解作为多级序列线性优化计算的子程序,在每级求解中重复调用。完整的求解过程如下:

\textbf{步骤 1} 调用当前最优下料计算子程序,求解得到优化值 $a_{i} l_{i}$ 组成的 $\sum_{i=1}^{m} a_{i} l_{i}$ 作为第一级下料方式。

\textbf{步骤 2} 计算此种下料方式的重复次数,即此种下料方式所需原材料 $L$ 的根数 $d$。其中

\begin{equation}
d = \min \left\{ \left\lceil \frac{b_{1}}{a_{1}} \right\rceil, \left\lceil \frac{b_{2}}{a_{2}} \right\rceil, \cdots, \left\lceil \frac{b_{m}}{a_{m}} \right\rceil \right\}.
\end{equation}

\textbf{步骤 3} 计算去掉 $d$ 根后,余下的每种待切割的零件个数为:$b_{i}=b_{i}-da_{i}$。

\textbf{步骤 4} 将 $b_{i}$ 作为新一级优化计算的给定值,如果所有的 $b_{i}$ 都已减小至零,则优化计算结束;否则转至步骤 1,重新用当前最优下料方式计算子程序,求得新一级的下料方式和重复次数。

\textbf{步骤 5} 各级最优下料方式及其重复次数的集合即为多级序列线性优化的最终结果。

(3)求解一维下料模型及结果分析

按照上述步骤进行编程求解,用 VC++6.0 外部反复调用 LINGO 求解当前最优下料方式,将处理结果在程序中记录并比较,最终得出一种最优方案。

对于 53 种规格的零件来说,能使原材料达到 100\% 利用率的下料方式可达百万种之多,通过分析该程序算法中求取当前最优下料方式步骤中,是按不同规格零件的输入顺序从前向后进行搜索计算,这样得到的那些 100\% 利用率的下料方式总是由编号在前的零件组成的。由此得出结论:程序计算出的方案与零件的输入顺序有关。

还应综合考虑影响解的各种因素,其中包括零件的加工个数及零件的长度,并且经过经验证明,在设计方案时应先大后小。由此启发我们综合考虑这两种因素优化零件的输入顺序,并提出两种零件输入原则:

1) 将加工个数相等或相近的零件放在一起,并把加工个数相等或相近的零件组按递增顺序排列;
2) 将这种零件组内部的零件按长度递减排列。

依据两种原则我们可以定出这 53 种零件的输入顺序:12、16、1、5、7、13、23、24、27、28、38、46、9、19、8、11、14、15、17、22、36、41、42、43、40、26、53、21、30、31、37、25、4、20、40、39、52、6、44、47、29、3、33、45、35、32、2、48、50、51、18、34、49。

按此序号顺序将零件输入程序,解得:下料方式($n$):56 种;所需原材料($C$):798 块;废料总长度:2304mm;原材料利用率:99.9\%。由符合这样两个原则的零件输入序列所得的方案应是兼顾原材料最少和下料方式最少的最优方案。

(4)按时间限制优选方案

判断某种方案是否可以满足时间限制要求可以采取如下方法:

设 4 天内完成的零件标号为集合 $P^{(4)}=\left\{p_{h}^{(4)}\right\}$($h=1,2,\cdots,11$),6 天内完成的零件标号为另一个集合 $Q^{(6)}=\left\{q_{k}^{(6)}\right\}$($k=1,2,\cdots,9$);求出该方案中包含 $P^{(4)}$ 集合元素的所有下料方式所切的原材料块数相加之和 $C_{P}$,比较 $C_{P}$ 与 400 的大小:

若 $C_{P} \leq 400$ 块,则说明在生产能力容许的范围内;
若 $C_{P} > 400$,则超出了企业最大的生产能力,即无法满足这些零件的时间要求。

对零件有 6 天要求的同理,只是这时在求该方案中包含 $Q^{(6)}$ 集合元素的所有下料方式时要减去即包含 $P^{(4)}$ 集合又包含 $Q^{(6)}$ 集合元素的下料方式,剩下的是只包含 $Q^{(6)}$ 集合元素的下料方式,算出这些下料方式所切的原材料的块数 $C_{Q}$,比较 $(C_{Q}+C_{P})$ 与 600 的大小即可。

若 $(C_{Q}+C_{P}) \leq 600$ 块,则在生产能力容许的范围内,道理同上。
若 $(C_{Q}+C_{P}) > 600$ 块,则超出最大生产能力,该方案无法满足这些零件的时间要求。

(5)最优解的选择

如何在原材料最少和下料方式最少这两个目标中取舍呢?由于每增加一种下料方式相当于使原材料总损耗增加 0.08\%,设有方案 $i-\left(c_{i}, n_{i}\right)$,$c_{i}$ 为方案 $i$ 的原材料数量,$n_{i}$ 为方案 $i$ 的下料方式数和方案 $j-\left(c_{j}, n_{j}\right)$,这时会出现四种情况:

(1) $c_{i} \geq c_{j}, n_{i} \geq n_{j}$;
(2) $c_{i} \geq c_{j}, n_{i} \leq n_{j}$;
(3) $c_{i} \leq c_{j}, n_{i} \geq n_{j}$;
(4) $c_{i} \leq c_{j}, n_{i} \leq n_{j}$;

对于情况 (1) 和情况 (4) 都好理解,对于情况 (2) 和情况 (3),就需要用题中所给的“增加一种下料方式大致相当于使原材料总损耗增加 0.08\%”这个原则进行比较判断。如对于情况 (2):

计算
$$
H=c_{j} *\left(n_{j}-n_{i}\right) * 0.08\%- \left(c_{i}-c_{j}\right),
$$

若 $H>0$,则方案 $j$ 优于方案 $i$;若 $H<0$,则方案 $j$ 劣于方案 $i$。

如果 $0.08\%$ 这个比例增大,则说明下料方式少的权重增加,这时在使用材料最少和下料方案最少这两个条件之间进行取舍的复杂性会增大。而本题将这个比率设定得如此之小以至对最终方案选择的影响微乎其微,实际上降低了问题的难度。

\subsection{二维下料模型}

对于二维下料问题,下料方式要满足零件长、宽方向上的套裁,所以远比一维下料复杂且数量大得多。因此,我们希望通过降维启发式方法即通过形成“板条”而把二维下料问题降为一维下料的方法来解决。在此称一维下料的原材料为“条材”,而二维下料的原材料为“板材”。板材与条材的区别在于:条材加工时只考虑长度而板材要同时考虑长、宽。

如果能把零件成组看待,板条就是这样一种零件组:其在一个方向上的长度等于或近似于原材料的长(或宽)方向的长度;然后在另一个方向即原材料的宽(或长)方向进行裁剪。这样,二维下料问题因“板条”的引入便降为一维下料问题。

零件的宽度决定板条的种类,当种类确定后,则要在不超过宽(或长)度的前提下在原材料上进行板条的布局(有若干种)。而布局方式一旦确定,就只剩下每个板条内部零件组的组合问题,这时就可利用一维中求解下料问题的启发式多级序列线性优化的方法来解决。其建模的具体过程类似于一维的分层建模思想,即先满足原材料最少要求,再从中选出其中满足时间要求的二维下料最优方案。

(1)\textbf{候选板条的形成过程}

将单一零件的长、宽 $(l_{i}, w_{i})$ 分别在原材料上排列。设板条的长为 $L_{s_{i}}$,宽为 $W_{s_{i}}$,且令 $m_{i}=\min \left\{n_{i},\left[L / l_{i}\right]\right\}$,$m_{i}^{\prime}=\min \left\{n_{i},\left[L / w_{i}\right]\right\}$,则有:
\[
L_{s_{i}}=\max \left\{l_{i}, m_{i}, w_{i} * m_{i}^{\prime}\right\}, \quad W_{s_{i}}=\begin{cases}l_{i}, & \text { 当 } L_{s_{i}}=w_{i} * m_{i}^{\prime} \\ w_{i}, & \text { 当 } L_{s_{i}}=l_{i} * m_{i}\end{cases}
\]
如图 1(a), (b) 所示:

\begin{figure}[h]
    \centering
    \includegraphics[width=\textwidth]{image.png}
    \caption{板条生成示意图}
    \label{fig:1}
\end{figure}

这样,板条的形成即将二维下料问题降为一维下料问题,所建立的整数规划模型同上述一维模型,用原材料总消耗数量最少为目标函数:
\[
\begin{aligned}
& \min \sum_{j=1}^{n} x_{j} \\
& \text { s.t. } \begin{cases}\sum_{j=1}^{n} a_{i j} x_{i} \leq b_{i}, i=1,2, \cdots, m \\ x_{j} \geq 0 \quad j=1,2, \cdots, n\end{cases}
\end{aligned}
\]

(2)\textbf{初始下料方式的优化选取}

可以通过背包问题来解决一维下料模型中下料方式数量巨大的问题:

\[
\begin{aligned}
& \max \sum_{i=1}^{n} S_{i} a_{i} \\
& \text{st} \left\{
\begin{aligned}
& \sum_{i=1}^{n} W_{s_{i}} a_{i} \leq W \\
& a_{i} \geq 0, \text{ 且为整数}, i=1,2,\ldots,n
\end{aligned}
\right.
\end{aligned}
\]

即可得到满足目标函数改进的下料方式(其中,$S_{i}$ 为一般整数规划模型在单纯形算法中基于某个基的影子价格系数 \cite{ref3},$W_{s_{i}}$ 为所产生的板条宽度,$a_{i}$ 为一种切割方式中所切出的第 $i$ 种规格件的数量)。这样,下料问题中用背包问题避免了对所有各种可能的初始切割方式的列举。将背包问题的解作为一个列向量通过转轴运算进入基解矩阵,以获得改进的目标函数。

(3)\textbf{剩余面积的处理}

由于“板条”的长度 $L_{s_{i}}$ 不一定等于原材料的长或宽,如图 2(a)、(b),就会在 L 方向(或 W 方向)产生余量,如果所有余量集中到一起,形成一个区域,假如该区域满足一阈值,则可把这个子区域作为一个子问题来处理。如果子区域不满足,则可动态地调整剩余面积,便能得到使余料更小的下料方式。

\begin{figure}[h]
\centering
\includegraphics[width=0.8\textwidth]{image.png}
\caption{板条动态调整示意图}
\end{figure}

图 2(a)为所选出的一种下料方式,阴影部分为剩余面积,在其固定时便无法排板条 4,但如果动态地调整板条 3,使其减少一块变成板条 $3'$,则可排下板条 4,剩余面积:$\delta(b) = \delta(a) + w_{3} * l_{3} - \delta_{4}$。余料 $\delta$ 减少。

(4)\textbf{具体问题的二维下料方案}

通过分析题中 43 种零件的长、宽,我们发现:

1) 所有零件的长度都大于原材料的宽度,即 $\min \{l_{1}, l_{2}, \ldots, l_{n}\} > W$。其中,$l_{i}$ 为零件长度,$W$ 为原材料的宽度,所以零件只能以顺着原材料长的方向切割。

2) 宽度只有 20、30、35、50mm 四种。且原材料宽度固定为 100mm,手工即可算出排满原材料宽度的 5 种组合方式:(50+50)、(35+35+30)、(50+20+30)、(20+30+20+30)、(20+20+20+20)。

在实施过程中,如果无法排满就考虑最接近的情况,这种情况下可能的组合为:(35+35+20)、(30+30+30)、(50+20+20)、(20+20+20+30)四种。

3) 将零件按宽度的大小分成 4 组,统计每组中的零件种类数目和所需零件总个数见下表

\begin{table}[h]
\centering
\begin{tabular}{|c|c|c|c|c|c|}
\hline
宽度 & 零件种类数 & 待加工个数总数 & 宽度 & 零件种类数 & 待加工个数总数 \\
\hline
50 & 6 & 270 & 30 & 20 & 3484 \\
\hline
35 & 2 & 624 & 20 & 15 & 7112 \\
\hline
\end{tabular}
\end{table}

因为每组的种类数、总个数是影响该组和其它规格零件组合优劣的两个重要因素,所以我们将种类数与总个数的乘积作为该组灵活性的衡量标准,即调整时的选择标准。

灵活性从低到高依次为 35、50、30、20。在方案的制定中,我们将从灵活性较低的一组开始。经求解,我们得到结果如下:下料方式($n$):49 种;所需原材料($C$):453 块。

\section{三、模型评价}

对于一维下料问题,当零件规格很多时,下料方式数量是巨大的,因此常规的整数规划求解方法不可行,我们采用启发式多级序列线性优化方法,其特点是运算速度快,空间小,不足的是可能无法得出严格全局最优解。

在处理加工零件的时间限制问题时,本文采用的是在由仅考虑原材料最少这种目标求出的若干种方案中先进行计算,选出即满足原材料消耗少又满足 4、6 天时间要求的方案,再进入下一层关于下料方式少这一目标的建模。这样就避免了在找到兼顾原材料和下料方式两者利益的最优解后,却由于无法达到 4、6 天的时间要求而不得不加以改进,其实改进也可以说是一定程度上的放弃,所以,在这点上还可以多加考虑。

我们在求解满足一维下料问题的三个要求(原材料最少,满足时间要求,下料方式最少)的最优方案时,采用的是分层建模,即逐层解决一个要求。其中,在第三层模型中,对下料方式少的要求转化成“每增加一种下料方式相当于原材料总损耗增加 0.08\%”这样一个判断最优的原则。0.08\% 是很小的一个数值,当我们把它作为一个变量 $\lambda$ 来考虑这层模型时,复杂性就会大大加大,最后得到的最优解也会随 $\lambda$ 的取值大小而改变。当 $\lambda$ 较大时,下料方式的多少对其影响会很大,而随着 $\lambda$ 的减小,下料方式对最终解的影响会越来越小于原材料的利用率对其的影响。

\section{参考文献}

(1) 闵仲求。合理下料的实用数学模型及其计算机软件的研究。系统工程理论与实践,1982,4。

(2) 刘勇彪。等界面长条类材料下料方案的最优化设计 [J]。机械设计与制造,1994,5。

(3) 运筹学。北京:清华大学出版社,1990,61。

\section{the mathematical models on the Practical cutting stock problem}

\textbf{CHEN Lu, HUANG Wei-jian, FENG Zhen}

(The Information Engineering University, Zhengzhou 450004)

\textbf{Abstract:} In consideration of the unlimited ways to cut the stock by the general integer programming model, In this paper, with the goal of minimizing the original material, we build up the 1-dimensional cutting-stock model by inspiring multilevel-sequences-linear approach. As to the 2-dimensional cutting-stock problem, we adopt the algorithm of inspireing decline of the dimensions to “batten”, by which we can descend the 2-dimensional problem to 1-dimensional’s.

有不妥之处敬请朱老师指正。

\section{联系人:}

指导教师:吴仕文,电话 13838099341 E-mail: wsw-lyz@163.com

参赛学生:陈璐,电话 13783687676 E-mail: babyjuejue@sohu.com