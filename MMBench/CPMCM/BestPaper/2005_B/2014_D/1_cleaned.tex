\documentclass{article}
\usepackage{amsmath}
\usepackage{amssymb}

\title{人体营养健康角度的中国果蔬发展战略研究}
\author{}
\date{}

\begin{document}

\maketitle

\begin{center}
\textbf{第十一届华为杯全国研究生数学建模竞赛}
\end{center}

\begin{tabular}{c c c c}
\hline
符号 & 说明 & 符号 & 说明 \\
\hline
\(Y\) & 总产量 & PE & 全国总人口 \\
\(YY\) & 总消费量 & \(d\) & 城乡人口比例 \\
\(X_{i}\) & 第\(i\)种水果的产量 & \(\bar{X}\) & 水果人均产量 \\
\(XX_{i}\) & 第\(i\)种水果的消费量 & \(\bar{XX}\) & 水果人均消费量 \\
\(Z_{i}\) & 第\(i\)种蔬菜的产量 & \(\bar{Z}\) & 蔬菜人均产量 \\
\(ZZ_{i}\) & 第\(i\)种蔬菜的消费量 & \(\bar{ZZ}\) & 蔬菜人均消费量 \\
\(V_{i}\) & 第\(i\)种营养素含量 & XP & 每公顷土地水果的产量 \\
\(\rho\) & 损耗率 & ZP & 每公顷土地蔬菜的产量 \\
\(\rho_{1}\) & 田间地头到大市场损耗率 & & \\
\(\rho_{2}\) & 大市场到零售市场损耗率 & & \\
\(\rho_{3}\) & 零售市场到餐桌损耗率 & & \\
\(P_{i}\) & 第\(i\)种价格 & & \\
\(RC\) & 居民购买成本 & & \\
\(PC\) & 种植者收入 & & \\
\hline
\end{tabular}

\begin{flushleft}
\textbf{题目} \quad 人体营养健康角度的中国果蔬发展战略研究
\end{flushleft}

\begin{abstract}
本文针对我国果蔬发展战略问题,从人体营养健康角度,建立我国居民果蔬人均消费量的预测模型,对居民果蔬按年度人均消费量的定量研究,并预测 2014-2020 年我国居民果蔬的人均消费趋势,并通过《中国居民膳食营养素参考日摄入量(2004)》及“中国居民平衡膳食宝塔(2007)”测算出营养素成分和含量的参考范围,来评价我国居民未来的人体营养健康状况。

针对问题一,本文基于系统聚类方法,针对果蔬营养素成分及含量,筛选出常见果蔬品种共计 20 种。利用了果蔬消费的损耗率来推测其按年度的居民人均消费量,对于数据缺失值,根据其特点选择三次样条插值法来进行估计。然后,建立了主成分回归模型以及线性回归模型,预测 2014-2020 年果蔬的总人均消费量以及常见种类果蔬的人均消费量,其中主成分回归模型平均误差为 2.84\%,线性回归模型平均误差仅为 1.52\%,两个模型误差均在合理范围之内。最后,将这两种方法的预测结果进行比较,用两种预测的平均值作为最终的果蔬人均消费量的预测值(结果见表 4,表 5)。

针对问题二,本文采用的方法是因子分析法,基于《中国居民膳食营养参考摄入量(2004)》和“中国居民平衡膳食宝塔(2007)”,推算出一般成人营养素的成分和含量参考范围,通过构建我国居民人体营养健康评价指标体系,结合问题一中预测的 2014-2020 年果蔬的人均消费量,可以得到 2014-2020 年我国居民果蔬营养素年摄入总量综合因子得分,说明这段期间我国居民人体营养状况总体来说略有好转(结果见表 18,图 12)。

针对问题三,本文建立线性规划模型,基于 2013 年月度数据中某些水果随季节变动存在较大波动的情况,利用 Lingo11.0 软件分别建立了关于蔬菜、水果夏季和水果其他季节消费量的线性规划模型。以居民购买成本最低为目标函数,以维生素 A、B1、B2、C、E,矿物质钙、钠、铁、锌、硒以及膳食纤维的果蔬人均日摄入范围为约束条件,得到2013年水果夏季人均日消费量0.693千克,其他季节人均日消费量1.154千克,蔬菜人均日消费量0.891千克,并用灵敏度分析得出费用系数在一定范围变化时,最优基保持不变(结果见表20,表21)。

针对问题四,本文建立多目标线性规划模型,以居民购买成本最低和种植者收入最高为双目标,并将其转化为单目标问题。本题相对于问题三的模型还增加了土地面积和进出口两个约束,更好的解决了人均果蔬日消费量的问题。基于2011年的数据得出人均日摄入蔬菜:0.086千克的胡萝卜、0.029千克的大白菜和0.585千克的菠菜,人均日摄入水果:0.031千克的橘子、0.039千克的梨和0.623千克的西瓜。(结果见表22,表23)并用该模型预测其他的年份,得出2020年我国应该增加蔬菜的人均日摄入量和小幅度减少水果的人均日摄入量,并且还应该丰富果蔬的摄入种类,使得营养更加均衡化的结论。

针对问题五,结合前面的研究结论,提出建议如下:(1)加强宣传力度,调整现存膳食习惯;(2)改善果蔬种植条件,调整果蔬种植规模;(3)改善果蔬储藏、加工、运输条件,改进果蔬交易市场;(4)调整居民收入水平,完善社会保障体系,增强消费信心;(5)积极采取措施,促进果蔬出口。

关键词:营养健康;果蔬人均消费量;主成分回归模型;因子分析法;多目标规划模型
\end{abstract}

\section*{目录}
\begin{itemize}
    \item[1] 问题重述 \dotfill -5-
    \begin{itemize}
        \item[1.1] 背景分析 \dotfill -5-
        \item[1.2] 问题提出 \dotfill -5-
    \end{itemize}
    \item[2] 问题分析 \dotfill -6-
    \item[3] 模型假设 \dotfill -7-
    \item[4] 模型符号与说明 \dotfill -7-
    \item[5] 模型的建立与求解 \dotfill -7-
    \begin{itemize}
        \item[5.1] 问题一的解答 \dotfill -8-
        \begin{itemize}
            \item[5.1.1] 问题一的分析 \dotfill -8-
            \item[5.1.2] 建模与求解 \dotfill -12-
        \end{itemize}
        \item[5.2] 问题二的解答 \dotfill -22-
        \begin{itemize}
            \item[5.2.1] 问题二的分析 \dotfill -22-
            \item[5.2.2] 模型的建立与求解 \dotfill -28-
        \end{itemize}
        \item[5.3] 问题三的解答 \dotfill -31-
        \begin{itemize}
            \item[5.3.1] 问题三的分析 \dotfill -31-
            \item[5.3.2] 模型的建立与求解 \dotfill -32-
        \end{itemize}
        \item[5.4] 问题四的解答 \dotfill -34-
        \begin{itemize}
            \item[5.4.1] 问题四的分析 \dotfill -34-
            \item[5.4.2] 模型的建立与求解 \dotfill -36-
        \end{itemize}
        \item[5.5] 问题五的解答 \dotfill -38-
    \end{itemize}
    \item[] 参考文献 \dotfill -40-
    \item[] 附录 \dotfill -41-
\end{itemize}

\section{问题重述}

\subsection{背景分析}

人体需要的营养素主要有蛋白质、脂肪、维生素、矿物质、糖和水。其中维生素对于维持人体新陈代谢的生理功能是不可或缺的,多达30余种,分为脂溶性维生素(如维生素A、D、E、K等)和水溶性维生素(如维生素B1、B2、B6、B12、C等)。矿物质无机盐等亦是构成人体的重要成分,约占人体体重的5\%,主要有钙、钾、硫等以及微量元素铁、锌等。另外适量地补充膳食纤维对促进良好的消化和排泄固体废物有着举足轻重的作用。

为了帮助人们合理的摄入各种营养素,从20世纪早期营养学家就开始建议营养素的参考摄入量,从40年代到80年代,许多国家都制定了各自的推荐的营养素供给量。我国自1955年开始制定“每日膳食中营养素供给量(RDA)”作为设计和评价膳食的质量标准,并作为制订食物发展计划和指导食品加工的参考依据。随着科学研究和社会实践的发展,特别是强化食品及营养补充剂的发展,国际上自20世纪90年代初期就逐渐开展了关于“RDA”的性质和适用范围的讨论。欧美各国先后提出了一些新的概念或术语,逐步形成了比较系统的新概念——“膳食营养素参考摄入量”。

国内方面,中国营养学会及时研究了这一领域的进展,决定革新传统的“RDA”概念,经过两年多的努力,于2000年10月出版了《中国居民膳食营养素参考摄入量》。膳食营养素参考摄入量是在“RDA”基础上发展起来的一组每日平均膳食营养素摄入量的参考值,包括4项内容:平均需要量(EAR)、推荐摄入量(RNI)、适宜摄入量(AI)和可耐受最高摄入量(UL)。《中国居民膳食营养素参考摄入量》是一部营养学科的专著,分别对各种营养素的理化性质、生理功能、营养评价及主要食物来源等方面,都进行了系统的论述,尤其对于各营养素的参考值都提供了丰富的科学研究依据,是营养学研究、教学和专业提高的重要参考书。十几年间,《中国居民膳食营养素参考摄入量》在我国的膳食指导、食物生产、食品加工等领域已经得到了越来越广泛的应用。

\subsection{问题提出}

水果和蔬菜作为重要的农产品,可以提供矿物质、维生素、膳食纤维等人体所需的营养元素。近年来,中国果蔬种植面积和产量迅速增长,果蔬品种也日益丰富,食物供需基本平衡。另一方面,中国居民生活水平不断提高,对人体营养均衡的意识也有所增强。然而,多数中国居民喜食、饱食、偏食、忽视人体健康所需的营养均衡的传统饮食习惯尚未根本扭转,正如近日公布的《中国食物与营养发展纲要(2014-2020年)》提出的那样,“食物生产还不能适应营养需求,居民营养不足与过剩并存,营养与健康知识缺乏”。这就使得我国的果蔬消费(品种和数量)在满足居民身体健康所需均衡营养的意义下,近乎盲目无序,进而影响到果蔬生产。

因此,预测我国果蔬的消费与生产趋势,科学地规划与调整我国果蔬的中长期的种植模式,具有重要的战略意义。为此,需解决以下问题。

\section{问题分析}

根据问题重述,本文从人体营养健康角度,探究中国果蔬发展战略。该问题关键在于为此从四个方面出发,分别建立模型。

对于问题一,首先需要比较准确地掌握我国果蔬各品种的产量情况,需要基于果蔬的营养素成分和含量,利用恰当的方法是筛选主要的果蔬品种进行研究。接着,采用主成分分析和线性回归两种方法建立预测模型,预测果蔬消费总量以及各主要品种果蔬的消费量。

对于问题二,根据《中国居民膳食营养参考摄入量(2004)》和“中国居民平衡膳食宝塔(2007)”,确定一般成人营养素的成分和含量参考范围,进而评价近年来我国居民目前营养素的日摄入水平是否合理。接着,通过构建我国居民人体营养健康评价指标体系来进行因子分析,来评价到2020年我国居民果蔬营养素年摄入总量的状况。

对于问题三,考虑果蔬之间营养素成分和含量的替代补充性以及价格的差异性,在保证营养均衡满足健康需要条件下,使得我国居民对果蔬的消费达到最优化,即居民果蔬购买成本最低。同时考虑果蔬产品不同季节价格的差异,建立优化模型,计算我国居民购买果蔬产品合理的人均消费量。

对于问题四,基于居民人体营养摄入量的合理范围,满足居民的购买成本最小化以及种植者收益最大化,同时考虑进出口、土地面积等影响因素,建立多目标规划模型。

\section{模型假设}

本文从人体营养健康角度出发,以《中国居民膳食营养素参考摄入量》为基础,结合我国果蔬主要品种产量、城乡居民果蔬人均消费量以及果蔬购买成本等客观情况,构建数学模型,模拟并预测主要品种果蔬各自的人均消费量,提出假设前提如下。

\begin{itemize}
    \item 假设1:附件中提供的所有数据和通过网络查询到的相关数据真实有效,并能反映指标所代表的真实含义;
    \item 假设2:由于数据收集困难,对于果蔬产量部品种存在缺失,假设其产量按照原有的趋势合理变动;
    \item 假设3:在预测果蔬人均消费量未来趋势时,假设在预测的时间区间内,无政策影响或其他突发事件发生,并忽略人为浪费等不确定性因素;
    \item 假设4:在目标期间内,假定各品种果蔬的价格按照原有趋势合理变动,且都为市场均衡时的价格。
\end{itemize}

\section{模型符号与说明}

\begin{tabular}{c c c c}
\hline
符号 & 说明 & 符号 & 说明 \\
\hline
\(Y\) & 总产量 & PE & 全国总人口 \\
\(YY\) & 总消费量 & \(d\) & 城乡人口比例 \\
\(X_{i}\) & 第\(i\)种水果的产量 & \(\bar{X}\) & 水果人均产量 \\
\(XX_{i}\) & 第\(i\)种水果的消费量 & \(\bar{XX}\) & 水果人均消费量 \\
\(Z_{i}\) & 第\(i\)种蔬菜的产量 & \(\bar{Z}\) & 蔬菜人均产量 \\
\(ZZ_{i}\) & 第\(i\)种蔬菜的消费量 & \(\bar{ZZ}\) & 蔬菜人均消费量 \\
\(V_{i}\) & 第\(i\)种营养素含量 & XP & 每公顷土地水果的产量 \\
\(\rho\) & 损耗率 & ZP & 每公顷土地蔬菜的产量 \\
\(\rho_{1}\) & 田间地头到大市场损耗率 & & \\
\(\rho_{2}\) & 大市场到零售市场损耗率 & & \\
\(\rho_{3}\) & 零售市场到餐桌损耗率 & & \\
\(P_{i}\) & 第\(i\)种价格 & & \\
\(RC\) & 居民购买成本 & & \\
\(PC\) & 种植者收入 & & \\
\hline
\end{tabular}

\section{模型的建立与求解}

\subsection{问题一的解答}

\subsubsection{问题一的分析}

2014年初,国务院办公厅正式颁布了《中国食物与营养发展纲要(2014-2020年)》,这是在系统总结《九十年代中国食物结构改革与发展纲要》、《中国食物与营养发展纲要(2001-2010年)》实施情况基础上的深入与发展。新《纲要》是遵照中央更加关注民生与建设和谐社会的新要求、我国国民经济发展进入新一轮增长期的新需要,以及在信息化、工业化、城镇化深入发展中同步推进农业现代化的新形势为依据制定的,对于如期完成小康社会建设任务具有十分重要的指导意义。

在问题一中,首先,我们需要比较准确地掌握我国果蔬各品种的产量情况。我国果蔬品种繁多,无论是中国官方公布的数据还是世界粮农组织(FAO)、美国农业部(USDA)等发布的数据均不完整,缺失较为普遍,而且品种、口径不一。然而,对这样的宏观问题,恰当的方法是筛选主要的果蔬品种进行研究。根据附件1中常见水果和蔬菜营养成分表中的水果和蔬菜项目来进行筛选,从相应的产量上考虑,剔除掉一些具有地区性和季节性较为明显并且产量较小的水果和蔬菜。其中,对于缺失的果蔬总产量以及主要品种的产量值,基于已有的数据通过三次样条插值法来进行估计。同时,选出的果蔬品种还要考虑所蕴含营养素无论在成分上还是在含量上都应满足研究的需要,基于初步筛选出来的果蔬的营养素成分和含量,通过聚类分析方法,将营养素成分和含量相似的果蔬品种进行进一步的筛选。

其次,问题一要求尝试用多种方法建立数学模型对其消费量进行估计,研究发展趋势。所以不仅要从果蔬总体的消费量来研究其各自的发展趋势,还需要考虑到主要果蔬品种的消费情况,研究到2020年时果蔬的消费量以及各主要果蔬的品种变化趋势。本文第一种方法采用主成分分析法来进行果蔬的人均消费量的预测,选取我国居民人均水果和蔬菜消费的影响因素,提取出主成分量,再利用OLS回归分析,从而建立出主成分回归预测模型,得到果蔬的总人均消费量预测值以及各品种的果蔬人均消费量预测值。第二种方法是采用线性回归法,利用2003-2011年我国居民各品种的果蔬消费量资料建立线性回归模型,从而进行预测。最后,通过两种预测方法预测出来的我国居民果蔬消费总量进行比较,用两种预测结果的平均值作为最终的果蔬消费总量以及各主要品种果蔬的消费量预测值。

\subsubsection{我国果蔬生产总量及城乡居民人均消费的描述统计分析}

本文所有数据主要来源于《中国统计年鉴》和《中国农业统计年鉴》,其中包括水果和蔬菜生产总量及主要品种生产总量、城乡居民水果和蔬菜人均消费量等,时间范围为1996年至2013年。

首先,我们应该对我国果蔬生产总量整体情况有一定的了解。十几年间,我国水果和蔬菜生产规模扩张迅速,产量增长势头迅猛,从1996年到2013年水果年产量从8120.81万吨上升到25093.04万吨,增长为3.09倍;同期蔬菜年产量从30123.09万吨上升到73511.99万吨,增长为2.44倍。相比而言,水果年产量增长比蔬菜年产量增加更快。

\begin{figure}[h]
    \centering
    \includegraphics[width=\textwidth]{image1.png}
    \caption{我国水果和蔬菜生产总量趋势图(单位:万吨)}
    \label{fig:1}
\end{figure}

其次,我国城镇居民和农村居民对于水果和蔬菜的人均消费状况,从1996年到2013年我国城镇居民水果人均消费量从46.2万吨上升到56.1万吨,增长为1.21倍;相对的,从1999年到2013年我国农村居民水果人均消费量从18.3万吨上升到22.8万吨,增长为1.25倍。另一方面,从1996年到2013年我国城镇居民蔬菜人均消费量从118.5万吨下降到112.3万吨,减小为0.95;我国农村居民蔬菜人均消费量从84.7万吨下降到106.3万吨,减小为0.80。相比而已,我国城镇居民和农村居民对于水果和蔬菜的人均消费状况变化基本一致。

\begin{figure}[h]
    \centering
    \includegraphics[width=\textwidth]{image2.png}
    \caption{我国城乡蔬菜人均消费量趋势图(单位:万吨)}
    \label{fig:2}
\end{figure}

\begin{figure}[h]
    \centering
    \includegraphics[width=\textwidth]{image1.png}
    \caption{我国城乡蔬菜人均消费量趋势图(单位:万吨)}
    \label{fig:3}
\end{figure}

\subsubsection{基于聚类分析法进行水果种类的筛选}

由于水果品种繁多,为了进行代表性研究,可以从附件1中常见水果营养成分表中的水果项目来进行筛选。首先,我们剔除掉一些具有地区性和季节性较为明显并且产量较小的水果,例如猕猴桃、柠檬、桑葚、荔枝和芒果等。

其次,因为有些水果之间所含有的营养素成分及含量比较相近,选取脂溶性维生素(维生素A、D、E等)、水溶性维生素(维生素B1、B2、B6、B12、C等)、矿物质(钠、铁、锌、硒等)、膳食纤维等21种营养素,运用SPSS 20软件对余下的21种水果在进行聚类分析,分析结果如下图所示:

\begin{figure}[h]
    \centering
    \includegraphics[width=\textwidth]{image2.png}
    \caption{基于营养素的主要种类水果聚类分析结果}
    \label{fig:4}
\end{figure}

从上图4可以看出,橙子、柿子、菠萝、桃子、葡萄、李子和柚子分为第一类,梨、枇杷、西瓜、杏子、龙眼、哈密瓜分为第二类,香蕉和椰子和山楂为第三类,草莓、木瓜、橘子、大枣和苹果各自为一类。对于分到一类的水果,由于考虑其所含营养素成分及其含量具有一定的相似性,具有一定的可替代性。

结合各类水果年产量占水果总产量的程度,选取具有代表性且高产量的水果。从第一类中,我们选取柿子、菠萝和葡萄;从第二类中,我们选取梨和西瓜;从第三类中,我们选取香蕉;对于草莓,木瓜、橘子、大枣和苹果各成一类的,由于木瓜产量少,我们只选取了草莓、橘子、大枣和苹果。最后,根据我们的筛选结果,我们筛选出10种无论是在营养素成分上还是含量上都满足研究需求的水果:苹果、香蕉、橘子、梨、葡萄、菠萝、大枣、柿子、西瓜和草莓。

\subsubsection{基于聚类分析法进行蔬菜种类的筛选}

(1) 基于三次样条插值法解决蔬菜年产量数据的缺失问题

由于从统计年鉴中得到的各类蔬菜的产量存在缺失,只包含蔬菜生产总量和部分蔬菜从2003年到2011年的数据,并且缺失2007年的数据。介于这种情况,我们利用三次样条插值方法先补充缺失值,再进行聚类分析。

三次样条函数,即给定一批数据点,需要确定满足特定要求的曲面或曲线,如果要求所求曲面或曲线通过所有的点,这就是插值问题。

在数学上已知某未知函数 \( n = f(m) \) 的一组观测(或试验)数据 \((m_i, n_i) (i = 1, 2, \ldots, n)\),要寻找一个函数 \(\vartheta(m)\),使 \(\vartheta(m_i) = n_i = f(m_i)\),则称此类问题为插值问题。并称 \(\vartheta(m)\) 为 \(f(m)\) 的插值函数,并称 \(m_1, m_2, \ldots, m_n\) 为样本点;称 \(\vartheta(m_i) = n_i\) 为插值条件,可得:\(\vartheta(m) \approx f(m)\)。

一般来说,插值一般所用的方法有拉格朗日插值、分段线性插值和三次样条插值。而本文中所用的补全数据的方法是三次样条插值。三次样条线性插值是存在较低次的分段多项式达到较高阶光滑性的一种方法。

利用 MATLAB 程序,结合所给出的 2003 年 2011 年的数据,补全 2007 年的数据。根据 2003 年到 2011 年的数据补全了 2007 年的蔬菜产量数据。从 2003 年至 2011 年整个期间中,绝大多数蔬菜产量都是递增的,从截取的 2006 年至 2008 年的数据中也可以看出,蔬菜产量仍然是逐年增长的,因此说明三次样条插值法补充的数据效果较好。

(2) 基于聚类分析法进行蔬菜种类的筛选

为了进行代表性研究,同样可以从附件 1 中常见蔬菜营养成分表中的蔬菜项目来进行筛选。剔除掉一些具有地区性和季节性较为明显并且产量较小的蔬菜,同时考虑蔬菜之间所含有的营养素成分及含量的相似性,选取脂溶性维生素(维生素 A、D、E 等)、水溶性维生素(维生素 B1、B2、B6、B12、C 等)、矿物质(钠、铁、锌、硒等)、膳食纤维等 21 种营养素,运用 SPSS 20 软件对 14 种蔬菜在进行聚类分析,分析结果如下图所示:

\begin{figure}[h]
    \centering
    \includegraphics[width=\textwidth]{image.png}
    \caption{基于营养素的主要种类蔬菜聚类分析结果}
    \label{fig:5}
\end{figure}

从上图 \ref{fig:5} 可以看出,主要种类蔬菜一共可以聚为三类。其中,菜豆、豇豆、大葱、黄瓜、蒜苗、土豆、莲藕、西红柿、萝卜、芹菜和大白菜分为第一类,胡萝卜和菠菜分为第二类,茄子为第三类。对于分到一类的蔬菜,由于考虑其所含营养素成分及其含量具有一定的相似性,具有一定的可替代性。

结合各类蔬菜年产量占蔬菜总产量的程度,选取具有代表性且高产量的蔬菜。从第一类中,我们选取黄瓜、蒜苗、土豆、西红柿、萝卜、芹菜和大白菜;从第二类中,我们选取胡萝卜和菠菜;从第三类中,我们选取茄子。最后,根据我们的筛选结果,我们筛选出10种无论是在营养素成分上还是含量上都满足研究需求的水果:萝卜、胡萝卜、土豆、大蒜、茄子、黄瓜、大白菜、芹菜、菠菜和西红柿。

\subsubsection{建模与求解}

\paragraph{水果人均消费量预测模型的建立}

(1)基于主成分分析法的水果人均消费量预测

对于我国居民人均水果消费总量 $\overline{\mathrm{XX}}$ 的数据,可以通过年度水果消费总量 $\mathrm{XX}$ 与全国总人口 $\mathrm{PE}$,进一步的可以通过我国城镇人均消费水果量 $\overline{\mathrm{XX}}_{\text{城镇}}$、农村人均消费水果量 $\overline{\mathrm{XX}}_{\text{农村}}$ 以及城乡人口比例 $d$ 来得到,具体关系如下所示:

\begin{align}
D(G_{i}) &= T_{i}'\Sigma T_{i}, \\
\mathrm{cov}(G_{i}, G_{j}) &= T_{i}'\Sigma T_{j}, \qquad i, j = 1, 2, \cdots m
\tag{5.1.5}
\end{align}

根据我国居民人均水果消费的影响因素,选取7个指标的数据进行人均水果消费的预测,主要包括年末总人口、果类居民消费价格指数、人均国内生产总值、居民消费水平、水果面积、水果产量、国民总收入。

设 $\mathbf{M} = (M_1, \cdots, M_k)'$ 为一个 $k$ 维随机向量,并假定存在二阶矩,其均值向量与协方差阵分别记为
\begin{equation}
\mu = \mathrm{E}(X), \quad \Sigma = \mathrm{D}(X)
\tag{5.1.2}
\end{equation}
考虑如下的线性变换
\begin{equation}
\begin{cases}
G_{1} = t_{11}M_{1} + t_{12}M_{2} + \cdots t_{1k}M_{k} = T_{1}'M \\
G_{2} = t_{21}M_{1} + t_{22}M_{2} + \cdots t_{2k}M_{k} = T_{2}'M \\
\vdots \\
G_{k} = t_{k1}M_{1} + t_{k2}M_{2} + \cdots t_{kk}M_{k} = T_{k}'M
\end{cases}
\tag{5.1.3}
\end{equation}
用矩阵表示为
\begin{equation}
G = T'M
\tag{5.1.4}
\end{equation}
其中,$G = (G_{1}, G_{2}, \cdots, G_{k})'$;$T = (T_{1}, T_{2}, \cdots, T_{k})$

我们希望寻找一组新的变量 $G_{1}, \cdots, G_{n}$ ($n \leq p$),这组新的变量要求充分地反应原变量的信息,而且相互独立。

这里对于 $G_{1}, \cdots, G_{n}$ 有
\begin{align}
D(G_{i}) &= T_{i}'\Sigma T_{i}, \\
\mathrm{cov}(G_{i}, G_{j}) &= T_{i}'\Sigma T_{j}, \qquad i, j = 1, 2, \cdots m
\tag{5.1.5}
\end{align}

我们希望用 $G_{1}$ 代替原来的 $k$ 个变量 $M_{1}, \cdots M_{k}$,这就要求 $G_{1}$ 尽可能地反应原 $k$ 个变量的信息,这里的 “信息” 用什么表达?一般我们用 $G_{1}$ 的方差来表达。$\mathrm{Var}(G_{1})$ 越大,表示 $G_{1}$ 所包含的信息越多。由 (2.5) 可以看出,需对 $T_{1}$ 做某些限制,否则可使 $\mathrm{Var}(G_{1}) \to \infty$,常用的限制是
\begin{equation}
T_{i}'T_{i} = 1 \text{ 或者 } |T| = 1
\tag{5.1.6}
\end{equation}

因此,我们希望在约束 (2.6) 下找 $T_{1}$,使得 $\mathrm{Var}(G_{1})$ 达到极大,$G_{1}$ 就称为主第一主成分。

要建立弹性不变的消费函数模型,需要对所有原始数据进行对数化后再进行标准化处理。运用 SPSS 20 软件进行主成分分析,得到变量相关系数矩阵的特征值及其贡献率。根据主成分个数提取原则,主成分因子过程提取了 1 个主成分量 $F_{1}$,它对信息的反应量是 89.279\%。

对提取的主成分变量建立初始因子载荷,根据初始因子载荷矩阵,看出主成分 $F_{1}$ 基本上反映所有因素的信息。主成分表达式为:
\begin{equation}
F_{1} = 0.157 \ln W_{1} + 0.115 \ln W_{2} + 0.158 \ln W_{3} + 0.158 \ln W_{4} + 0.158 \ln W_{5} + 0.149 \ln W_{6} + 0.159 \ln W_{7}
\tag{5.1.7}
\end{equation}
根据公式求出 $F_{1}$ 的主成分得分,具体数据见下表:

随后,采用 Eviews 7 软件对上表中数据进行 OLS 回归分析,从而建立主成分回归模型,方程具体形式如下:
\begin{equation}
\ln \overline{Y}_{\text{水果}} = 3.560 + 0.061 F_{1}
\tag{5.1.8}
\end{equation}
\begin{equation}
(203.319) \quad (2.896)
\end{equation}

将 $F_{1}$ 的主成分得分带入代入主成分回归方程中,得到 1996-2011 年的预测值,且预测值与实际值的平均误差仅为 2.84\%,回归模型的预测精度相当高。

采用三年移动平均法预测 2012-2020 年的主成分 $F_{1}$ 得分,将其代入主成分回归预测模型,得到 2012-2020 年的人均水果年消费量,如下表所示:

\begin{table}
\centering
\begin{tabular}{|c|c|c|}
\hline
年份 & 主成分F1预测值 & 居民人均水果年消费预测值(千克) \\
\hline
2012 & 1.25095 & 37.95145906 \\
\hline
2013 & 1.346663 & 38.17368727 \\
\hline
2014 & 1.368191 & 38.22384967 \\
\hline
2015 & 1.321935 & 38.11614797 \\
\hline
2016 & 1.345596 & 38.17120294 \\
\hline
2017 & 1.345241 & 38.17037487 \\
\hline
2018 & 1.337591 & 38.15256656 \\
\hline
2019 & 1.342809 & 38.16471382 \\
\hline
2020 & 1.34188 & 38.16255103 \\
\hline
\end{tabular}
\caption{2012-2020年居民人均水果年消费量主成分预测值}
\end{table}

\begin{figure}[h]
\centering
\includegraphics[width=0.8\textwidth]{image.png}
\caption{2012-2020年居民人均水果年消费量预测图}
\end{figure}

由于假设在预测的时间区间内,无政策影响或其他突发事件发生,并忽略人为浪费等不确定性因素,但是仍然需要考虑水果消费的损耗。因此,我国居民各种类水果的人均消费量主要通过单类水果国内供应量、水果消费损耗率来估算,具体关系如下所示:

\begin{equation}
\overline{x} \approx \text{单类水果国内人均供应量} \times (1 - \text{水果消费损耗率}) \tag{5.1.9}
\end{equation}

其中,水果国内供应量可以通过单类水果的人均产量、产量供应比例来得到;单类水果损耗率通过田间地头到大市场损耗率、大市场到零售市场损耗率、零售市场到餐桌损耗率来得到。具体关系如下所示:

\begin{equation}
\text{单类水果国内供应量} = \text{单类水果人均产量} \times \text{产量供应比例}
\end{equation}

\begin{equation}
\approx \text{单类水果人均产量} \times \frac{\text{总水果人均占有量}}{\text{总水果人均产量}} \tag{5.1.10}
\end{equation}

\begin{equation}
\text{水果消费损耗率} \approx 1 - (1 - p_1)(1 - p_2)(1 - p_3) \tag{5.1.11}
\end{equation}

得到我国居民各类水果的人均消费量的数据。

\begin{table}
\caption{2012-2020年各种类水果居民人均消费量主成分预测值(单位:千克)}
\begin{tabular}{|c|c|c|c|c|c|}
\hline
年份 & 香蕉 & 苹果 & 橘子 & 梨 & 葡萄 \\
\hline
2012 & 1.043545 & 5.636942 & 4.042844 & 2.303071 & 0.926302 \\
\hline
2013 & 1.056611 & 5.689303 & 4.14535 & 2.318975 & 0.937549 \\
\hline
2014 & 1.059572 & 5.701146 & 4.168761 & 2.322567 & 0.940098 \\
\hline
2015 & 1.053219 & 5.675728 & 4.11862 & 2.314855 & 0.93463 \\
\hline
2016 & 1.056464 & 5.688716 & 4.144193 & 2.318797 & 0.937423 \\
\hline
2017 & 1.056415 & 5.688521 & 4.143807 & 2.318737 & 0.937381 \\
\hline
2018 & 1.055365 & 5.684319 & 4.135523 & 2.317462 & 0.936477 \\
\hline
2019 & 1.056081 & 5.687185 & 4.141172 & 2.318332 & 0.937094 \\
\hline
2020 & 1.055954 & 5.686675 & 4.140166 & 2.318177 & 0.936984 \\
\hline
年份 & 菠萝 & 大枣 & 柿子 & 西瓜 & 草莓 \\
\hline
2012 & 0.172035 & 0.758995 & 0.412067 & 9.308364 & 0.249323 \\
\hline
2013 & 0.172327 & 0.791346 & 0.41648 & 9.227464 & 0.24919 \\
\hline
2014 & 0.172392 & 0.79881 & 0.417479 & 9.209366 & 0.24916 \\
\hline
2015 & 0.172251 & 0.782858 & 0.415335 & 9.248298 & 0.249224 \\
\hline
2016 & 0.172323 & 0.790978 & 0.416431 & 9.228362 & 0.249191 \\
\hline
2017 & 0.172322 & 0.790855 & 0.416414 & 9.228662 & 0.249192 \\
\hline
2018 & 0.172299 & 0.788221 & 0.41606 & 9.235103 & 0.249202 \\
\hline
2019 & 0.172315 & 0.790017 & 0.416302 & 9.230708 & 0.249195 \\
\hline
2020 & 0.172312 & 0.789697 & 0.416259 & 9.231491 & 0.249196 \\
\hline
\end{tabular}
\end{table}

\begin{figure}[h]
\centering
\includegraphics[width=\textwidth]{image.png}
\caption{2003-2011年我国居民各类水果人均消费量趋势图}
\end{figure}

\begin{table}[h]
\centering
\caption{2012-2020年各种类水果居民人均消费量线性回归预测值(单位:千克)}
\begin{tabular}{|c|c|c|c|c|c|c|}
\hline
年份 & 水果总消费 & 香蕉 & 苹果 & 橘子 & 梨 & 葡萄 \\
\hline
2003 & 38.9247 & 1.1144 & 5.857 & 4.427 & 2.3733 & 0.9738 \\
\hline
2004 & 39.4006 & 1.1431 & 5.9661 & 4.6197 & 2.4072 & 0.9971 \\
\hline
2005 & 39.8765 & 1.1718 & 6.0752 & 4.8124 & 2.4411 & 1.0204 \\
\hline
2006 & 40.3524 & 1.2005 & 6.1843 & 5.0051 & 2.475 & 1.0437 \\
\hline
2007 & 40.8283 & 1.2292 & 6.2934 & 5.1978 & 2.5089 & 1.067 \\
\hline
2008 & 41.3042 & 1.2579 & 6.4025 & 5.3905 & 2.5428 & 1.0903 \\
\hline
2009 & 41.7801 & 1.2866 & 6.5116 & 5.5832 & 2.5767 & 1.1136 \\
\hline
2010 & 42.256 & 1.3153 & 6.6207 & 5.7759 & 2.6106 & 1.1369 \\
\hline
2011 & 42.7319 & 1.344 & 6.7298 & 5.9686 & 2.6445 & 1.1602 \\
\hline
年份 & 菠萝 & 大枣 & 柿子 & 西瓜 & 草莓 & \\
\hline
2003 & 0.1732 & 0.8583 & 0.4305 & 8.8923 & 0.2488 & \\
\hline
2004 & 0.1738 & 0.9113 & 0.4396 & 8.6941 & 0.2485 & \\
\hline
2005 & 0.1744 & 0.9643 & 0.4487 & 8.4959 & 0.2482 & \\
\hline
2006 & 0.175 & 1.0173 & 0.4578 & 8.2977 & 0.2479 & \\
\hline
2007 & 0.1756 & 1.0703 & 0.4669 & 8.0995 & 0.2476 & \\
\hline
2008 & 0.1762 & 1.1233 & 0.476 & 7.9013 & 0.2473 & \\
\hline
\end{tabular}
\end{table}

\begin{equation}
\begin{aligned}
\begin{bmatrix}
x_{1} \\
x_{2} \\
x_{3} \\
x_{4} \\
x_{5} \\
x_{6} \\
x_{7} \\
x_{8} \\
x_{9} \\
x_{10}
\end{bmatrix}
=
\begin{bmatrix}
0.8274 \\
4.7660 \\
2.5003 \\
2.0343 \\
0.7408 \\
0.1672 \\
0.3283 \\
0.3395 \\
10.8743 \\
0.2518
\end{bmatrix}
+
\begin{bmatrix}
0.0287 \\
0.1091 \\
0.1927 \\
0.0339 \\
0.0233 \\
0.0006 \\
0.0530 \\
0.0091 \\
-0.1982 \\
-0.0003
\end{bmatrix}
*T
\end{aligned}
\tag{5.1.12}
\end{equation}

\begin{equation}
Y_{\text{水果}} = 113.77416175 - 0.996016196667 * T
\tag{5.1.13}
\end{equation}

\begin{table}
\centering
\begin{tabular}{c c c c c c c}
\hline
2009 & 0.1768 & 1.1763 & 0.4851 & 7.7031 & 0.247 & \\
\hline
2010 & 0.1774 & 1.2293 & 0.4942 & 7.5049 & 0.2467 & \\
\hline
2011 & 0.178 & 1.2823 & 0.5033 & 7.3067 & 0.2464 & \\
\hline
\end{tabular}
\end{table}

(3) 两种预测值的比较

表4 我国居民总的人均消费量两种预测值及平均值 (单位:千克)

\begin{table}
\centering
\begin{tabular}{c c c c}
\hline
年份 & 主成分回归预测值 & 线性回归预测值 & 两种预测平均值 \\
\hline
2012 & 37.95145906 & 38.9247 & 38.43808 \\
\hline
2013 & 38.17368727 & 39.4006 & 38.78714 \\
\hline
2014 & 38.22384967 & 39.8765 & 39.05017 \\
\hline
2015 & 38.11614797 & 40.3524 & 39.23427 \\
\hline
2016 & 38.17120294 & 40.8283 & 39.49975 \\
\hline
2017 & 38.17037487 & 41.3042 & 39.73729 \\
\hline
2018 & 38.15256656 & 41.7801 & 39.96633 \\
\hline
2019 & 38.16471382 & 42.256 & 40.21036 \\
\hline
2020 & 38.16255103 & 42.7319 & 40.44723 \\
\hline
\end{tabular}
\end{table}

通过比较分析,得知两种预测方法预测出来的居民总的人均消费量以及各类水果的人均消费量比较接近。一般来讲,一般线性回归预测值较大一些,实际值落在一般线性回归预测值与主成分预测值之间的可能性较大。为了增加预测的精度,可以取每年两种预测平均值作为当年的最终预测值。

表5 居民各类水果的人均消费量两种预测值平均值 (单位:千克)

\begin{table}
\centering
\begin{tabular}{c c c c c c}
\hline
年份 & 香蕉 & 苹果 & 橘子 & 梨 & 葡萄 \\
\hline
2012 & 1.078973 & 5.746971 & 4.234922 & 2.338186 & 0.950051 \\
\hline
2013 & 1.099856 & 5.827702 & 4.382525 & 2.363088 & 0.967325 \\
\hline
2014 & 1.115686 & 5.888173 & 4.490581 & 2.381834 & 0.980249 \\
\hline
2015 & 1.12686 & 5.930014 & 4.56186 & 2.394928 & 0.989165 \\
\hline
2016 & 1.142832 & 5.991058 & 4.670997 & 2.413849 & 1.002212 \\
\hline
2017 & 1.157158 & 6.045511 & 4.767154 & 2.430769 & 1.013841 \\
\hline
2018 & 1.170983 & 6.09796 & 4.859362 & 2.447081 & 1.025039 \\
\hline
2019 & 1.185691 & 6.153943 & 4.958536 & 2.464466 & 1.036997 \\
\hline
2020 & 1.199977 & 6.208238 & 5.054383 & 2.481339 & 1.048592 \\
\hline
年份 & 菠萝 & 大枣 & 柿子 & 西瓜 & 草莓 \\
\hline
2012 & 0.172618 & 0.808648 & 0.421284 & 9.100332 & 0.249062 \\
\hline
2013 & 0.173064 & 0.851323 & 0.42804 & 8.960782 & 0.248845 \\
\hline
2014 & 0.173396 & 0.881555 & 0.43309 & 8.852633 & 0.24868 \\
\hline
2015 & 0.173626 & 0.900079 & 0.436568 & 8.772999 & 0.248562 \\
\hline
2016 & 0.173962 & 0.930639 & 0.441666 & 8.663931 & 0.248396 \\
\hline
2017 & 0.174261 & 0.957078 & 0.446207 & 8.564981 & 0.248246 \\
\hline
2018 & 0.17455 & 0.982261 & 0.45058 & 8.469102 & 0.248101 \\
\hline
2019 & 0.174858 & 1.009659 & 0.455251 & 8.367804 & 0.247948 \\
\hline
\end{tabular}
\end{table}

\begin{table}
\centering
\begin{tabular}{c c c c c}
2020 & 0.175156 & 1.035999 & 0.45978 & 8.269096 & 0.247798 \\
\end{tabular}
\end{table}

\paragraph{蔬菜人均消费量预测模型的建立}

(1) 基于主成分分析法的蔬菜人均消费量预测

对于我国居民人均蔬菜消费总量 $\overline{ZZ}$ 的数据,可以通过年度蔬菜消费总量 $ZZ$ 与全国总人口 $PE$,进一步的可以通过我国城镇人均消费蔬菜量 $ZZ_{\text{城镇}}$、农村人均消费蔬菜量 $ZZ_{\text{农村}}$ 以及城乡人口比例 $d$ 来得到,具体关系如下所示:

\begin{equation}
\overline{ZZ} = \frac{ZZ}{PE} = \frac{ZZ_{\text{城镇}} \cdot d + ZZ_{\text{农村}}}{1 + d}
\tag{5.1.14}
\end{equation}

根据我国居民人均蔬菜消费的影响因素,选取 7 个指标的数据进行人均蔬菜消费的预测,主要包括年末总人口、蔬菜居民消费价格指数、人均国内生产总值、居民消费水平、蔬菜面积、蔬菜产量、国民总收入。

要建立弹性不变的消费函数模型,需要对所有原始数据进行对数化后再进行标准化处理。运用 SPSS 20 软件进行主成分分析,得到变量相关系数矩阵的特征值及其贡献率。根据主成分个数提取原则,主成分因子过程提取了 1 个主成分量 $F_2$,它对信息的反应量是 80.823\%,具体数据见下表。

对提取的主成分变量建立初始因子载荷,根据初始因子载荷矩阵,看出主成分 $F_2$ 基本上反映所有因素的信息。主成分表达式为:

\begin{equation}
F_2 = 0.172 \ln V_1 + 0.065 \ln V_8 + 0.171 \ln V_3 + 0.171 \ln V_4 + 0.159 \ln V_9 + 0.171 \ln V_{10} + 0.171 \ln V_7
\tag{5.1.15}
\end{equation}

根据公式求出 $F_2$ 的主成分得分,具体数据见下表:

随后,采用 Eviews 7 软件对上表中数据进行 OLS 回归分析,从而建立主成分回归模型,方程具体形式如下:

\begin{equation}
\ln Y_{\text{蔬菜}} = 4.711 - 0.043 F_2
\tag{5.1.16}
\end{equation}

\begin{equation}
(187.319) \quad (2.754)
\end{equation}

将 $F_1$ 的主成分得分带入代入主成分回归方程中,得到 1996-2011 年的预测值,且预测值与实际值的平均误差仅为 1.44\%,回归模型的预测精度相当高。

采用三年移动平均法预测 2012-2020 年的主成分 $F_2$ 得分,将其代入主成分回归预测模型,得到 2012-2020 年的人均水果年消费量,如下表所示:

\begin{table}
\centering
\caption{2012-2020 年居民人均水果年消费量主成分预测值}
\begin{tabular}{c c c}
\hline
年份 & 主成分 $F_2$ 预测值 & 居民人均水果年消费预测值(千克) \\
\hline
2012 & 1.220851 & 105.4781 \\
2013 & 1.324112 & 105.0108 \\
2014 & 1.344058 & 104.9208 \\
2015 & 1.29634 & 105.1363 \\
2016 & 1.321503 & 105.0226 \\
2017 & 1.320634 & 105.0265 \\
2018 & 1.312826 & 105.0618 \\
2019 & 1.318321 & 105.037 \\
2020 & 1.31726 & 105.0417 \\
\hline
\end{tabular}
\end{table}

\begin{figure}[h]
    \centering
    \includegraphics[width=0.8\textwidth]{image.png} % 替换为实际图像文件名
    \caption{2012-2020年居民人均蔬菜年消费量预测图}
    \label{fig:vegetable_consumption}
\end{figure}

由于假设在预测的时间区间内,无政策影响或其他突发事件发生,并忽略人为浪费等不确定性因素,但是仍然需要考虑蔬菜消费的损耗。因此,我国居民各种类蔬菜的人均消费量主要通过单类蔬菜国内供应量、蔬菜消费损耗率来估算,具体关系如下所示:

\begin{equation}
zz \approx \text{单类蔬菜国内人均供应量} \times (1 - \text{蔬菜消费损耗率}) \tag{5.1.17}
\end{equation}

其中,蔬菜国内供应量可以通过单类蔬菜的人均产量、产量供应比例来得到;单类蔬菜损耗率通过田间地头到大市场损耗率、大市场到零售市场损耗率、零售市场到餐桌损耗率来得到。具体关系如下所示:

\begin{equation}
\text{单类蔬菜国内供应量} = \text{单类蔬菜人均产量} \times \text{产量供应比例}
\end{equation}

\begin{equation}
\approx \text{单类蔬菜人均产量} \times \frac{\text{总蔬菜人均占有量}}{\text{总蔬菜人均产量}} \tag{5.1.18}
\end{equation}

\begin{equation}
\text{单类蔬菜损耗} \approx 1 - (1 - p_1)(1 - p_2)(1 - p_3) \tag{5.1.19}
\end{equation}

得到我国居民各类蔬菜的人均消费量的数据。同样基于主成分分析方法,将各种类蔬菜的人均消费量带入,与主成分 \(F_2\) 建立相应的各类蔬菜主成分回归模型,从而得到2012-2020年的各种类蔬菜的人均消费量,如下表所示:

\begin{table}[h]
    \centering
    \caption{2012-2020年各种类蔬菜居民人均消费量主成分预测值(单位:千克)}
    \label{tab:vegetable_consumption}
    \begin{tabular}{|c|c|c|c|c|c|}
    \hline
    年份 & 萝卜 & 胡萝卜 & 土豆 & 大蒜 & 茄子 \\ \hline
    2012 & 6.756489 & 2.319881 & 2.340825 & 2.892221 & 3.704288 \\ \hline
    2013 & 6.726555 & 2.307457 & 2.330454 & 2.88685 & 3.691305 \\ \hline
    2014 & 6.720789 & 2.305065 & 2.328456 & 2.885814 & 3.688803 \\ \hline
    2015 & 6.734593 & 2.310792 & 2.333239 & 2.888294 & 3.694793 \\ \hline
    2016 & 6.72731 & 2.30777 & 2.330716 & 2.886986 & 3.691633 \\ \hline
    2017 & 6.727562 & 2.307875 & 2.330803 & 2.887031 & 3.691742 \\ \hline
    2018 & 6.729821 & 2.308812 & 2.331585 & 2.887437 & 3.692722 \\ \hline
    2019 & 6.728231 & 2.308152 & 2.331035 & 2.887151 & 3.692032 \\ \hline
    \end{tabular}
\end{table}

\begin{table}
\centering
\begin{tabular}{c c c c c c}
\hline
2020 & 6.728537 & 2.30828 & 2.331141 & 2.887206 & 3.692166 \\
\hline
年份 & 黄瓜 & 大白菜 & 芹菜 & 菠菜 & 西红柿 \\
\hline
2012 & 6.975237 & 11.83153 & 2.817474 & 2.025599 & 6.242222 \\
\hline
2013 & 6.975453 & 11.64367 & 2.804123 & 2.012257 & 6.346211 \\
\hline
2014 & 6.975494 & 11.60773 & 2.801551 & 2.00969 & 6.366496 \\
\hline
2015 & 6.975395 & 11.6939 & 2.807708 & 2.015836 & 6.318074 \\
\hline
2016 & 6.975447 & 11.64838 & 2.80446 & 2.012593 & 6.343563 \\
\hline
2017 & 6.975445 & 11.64995 & 2.804572 & 2.012705 & 6.34268 \\
\hline
2018 & 6.975429 & 11.66406 & 2.805579 & 2.013711 & 6.334761 \\
\hline
2019 & 6.975441 & 11.65413 & 2.80487 & 2.013003 & 6.340333 \\
\hline
2020 & 6.975438 & 11.65604 & 2.805007 & 2.013139 & 6.339257 \\
\hline
\end{tabular}
\end{table}

(2) 基于线性回归法的蔬菜人均消费量预测

根据上述2.4.1中得到我国居民各类蔬菜的人均消费量,由下面消费情况趋势图可以看出,大白菜的含量是几种主要蔬菜人均消费量中最多的,菠菜是几种主要蔬菜人均消费量中最少的,并且每种这九年的变化趋势比较符合一元线性函数,是一个大致减少的趋势。于是我们考虑运用线性回归模型对于人均消费量进行预测。

\begin{figure}[h]
\centering
\includegraphics[width=\textwidth]{image.png}
\caption{我国居民各类蔬菜的人均消费情况趋势图}
\end{figure}

利用2003-2011年我国居民各类蔬菜的人均消费量以及蔬菜总的人均消费量数据,建立线性回归模型,形式如下:

\begin{table}[h]
\centering
\caption{2012-2020年各种类蔬菜居民人均消费量线性回归预测值(单位:千克)}
\begin{tabular}{|c|c|c|c|c|c|c|}
\hline
年份 & 蔬菜总消费 & 萝卜 & 胡萝卜 & 土豆 & 大蒜 & 茄子 \\
\hline
2003 & 103.8141 & 5.6852 & 2.2999 & 2.2747 & 2.9313 & 3.6829 \\
\hline
2004 & 102.8181 & 5.4956 & 2.2781 & 2.2464 & 2.9303 & 3.6606 \\
\hline
2005 & 101.8221 & 5.306 & 2.2563 & 2.2181 & 2.9293 & 3.6383 \\
\hline
2006 & 100.8261 & 5.1164 & 2.2345 & 2.1898 & 2.9283 & 3.616 \\
\hline
2007 & 99.8301 & 4.9268 & 2.2127 & 2.1615 & 2.9273 & 3.5937 \\
\hline
2008 & 98.8341 & 4.7372 & 2.1909 & 2.1332 & 2.9263 & 3.5714 \\
\hline
2009 & 97.8381 & 4.5476 & 2.1691 & 2.1049 & 2.9253 & 3.5491 \\
\hline
2010 & 96.8421 & 4.358 & 2.1473 & 2.0766 & 2.9243 & 3.5268 \\
\hline
2011 & 95.8461 & 4.1684 & 2.1255 & 2.0483 & 2.9233 & 3.5045 \\
\hline
\end{tabular}
\end{table}

\begin{table}[h]
\centering
\caption{2012-2020年各种类蔬菜居民人均消费量线性回归预测值(单位:千克)}
\begin{tabular}{|c|c|c|c|c|c|c|}
\hline
年份 & 黄瓜 & 大白菜 & 芹菜 & 菠菜 & 西红柿 & \\
\hline
2003 & 11.1687 & 2.7962 & 1.9798 & 6.6341 & 103.8141 & \\
\hline
2004 & 10.7574 & 2.7732 & 1.9518 & 6.838 & 102.8181 & \\
\hline
2005 & 10.3461 & 2.7502 & 1.9238 & 7.0419 & 101.8221 & \\
\hline
2006 & 9.9348 & 2.7272 & 1.8958 & 7.2458 & 100.8261 & \\
\hline
2007 & 9.5235 & 2.7042 & 1.8678 & 7.4497 & 99.8301 & \\
\hline
2008 & 9.1122 & 2.6812 & 1.8398 & 7.6536 & 98.8341 & \\
\hline
2009 & 8.7009 & 2.6582 & 1.8118 & 7.8575 & 97.8381 & \\
\hline
2010 & 8.2896 & 2.6352 & 1.7838 & 8.0614 & 96.8421 & \\
\hline
2011 & 7.8783 & 2.6122 & 1.7558 & 8.2653 & 95.8461 & \\
\hline
\end{tabular}
\end{table}

\begin{table}[h]
\centering
\caption{我国居民总的人均消费量两种预测值及平均值(单位:千克)}
\begin{tabular}{|c|c|c|c|}
\hline
年份 & 主成分回归预测值 & 线性回归预测值 & 两种预测平均值 \\
\hline
 &  &  & \\
\hline
\end{tabular}
\end{table}

\begin{table}
\centering
\begin{tabular}{|c|c|c|c|}
\hline
2012 & 105.4781 & 103.8141 & 104.6461 \\
\hline
2013 & 105.0108 & 102.8181 & 103.9145 \\
\hline
2014 & 104.9208 & 101.8221 & 103.3715 \\
\hline
2015 & 105.1363 & 100.8261 & 102.9812 \\
\hline
2016 & 105.0226 & 99.8301 & 102.4264 \\
\hline
2017 & 105.0265 & 98.8341 & 101.9303 \\
\hline
2018 & 105.0618 & 97.8381 & 101.45 \\
\hline
2019 & 105.037 & 96.8421 & 100.9396 \\
\hline
2020 & 105.0417 & 95.8461 & 100.4439 \\
\hline
\end{tabular}
\end{table}

\begin{table}
\centering
\caption{居民各类蔬菜的人均消费量两种预测值平均值(单位:千克)}
\begin{tabular}{|c|c|c|c|c|c|}
\hline
年份 & 萝卜 & 胡萝卜 & 土豆 & 大蒜 & 茄子 \\
\hline
2012 & 6.220845 & 2.309891 & 2.307763 & 2.911761 & 3.693594 \\
\hline
2013 & 6.111078 & 2.292779 & 2.288427 & 2.908575 & 3.675953 \\
\hline
2014 & 6.013395 & 2.280683 & 2.273278 & 2.907557 & 3.663552 \\
\hline
2015 & 5.925497 & 2.272646 & 2.26152 & 2.908297 & 3.655397 \\
\hline
2016 & 5.827055 & 2.260235 & 2.246108 & 2.907143 & 3.642667 \\
\hline
2017 & 5.732381 & 2.249388 & 2.232002 & 2.906666 & 3.631571 \\
\hline
2018 & 5.638711 & 2.238956 & 2.218243 & 2.906369 & 3.620911 \\
\hline
2019 & 5.543116 & 2.227726 & 2.203818 & 2.905726 & 3.609416 \\
\hline
2020 & 5.448469 & 2.21689 & 2.189721 & 2.905253 & 3.598333 \\
\hline
年份 & 黄瓜 & 大白菜 & 芹菜 & 菠菜 & 西红柿 \\
\hline
2012 & 7.015269 & 11.50012 & 2.806837 & 2.0027 & 6.438161 \\
\hline
2013 & 7.023427 & 11.20054 & 2.788662 & 1.982029 & 6.592106 \\
\hline
2014 & 7.031497 & 10.97692 & 2.775876 & 1.966745 & 6.704198 \\
\hline
2015 & 7.039498 & 10.81435 & 2.767454 & 1.955818 & 6.781937 \\
\hline
2016 & 7.047574 & 10.58594 & 2.75433 & 1.940197 & 6.896632 \\
\hline
2017 & 7.055623 & 10.38108 & 2.742886 & 1.926253 & 6.99814 \\
\hline
2018 & 7.063665 & 10.18248 & 2.73189 & 1.912756 & 7.096131 \\
\hline
2019 & 7.071721 & 9.971865 & 2.720035 & 1.898402 & 7.200867 \\
\hline
2020 & 7.079769 & 9.76717 & 2.708604 & 1.88447 & 7.302279 \\
\hline
\end{tabular}
\end{table}

许多国家都制定了膳食指南来指导本国居民的合理膳食。我国于1997年发布了《中国居民膳食指南》和“中国居民平衡膳食宝塔”。随着我国社会和经济的发展,我国居民膳食消费和营养状况都发生了较大改变,为了指导广大居民更好地平衡膳食,获得合理营养,提高国民健康水平,中国营养学会发布了新版的《中国居民膳食指南(2007)》和“中国居民平衡膳食宝塔(2007)”。

筛选《中国食物成分表(2004)》所列出的所有食物,剔除婴幼儿食物、饮料、蜜饯果脯、调味品、熟食、罕有食物。根据膳食宝塔中建议的食物种类,将其分为谷类薯类及杂豆、蔬菜类、水果类、畜禽肉类、鱼虾类、蛋类、奶类及奶制品、大豆类及坚果、油等9类,加上《中国食物成分表(2004)》未列出的盐,共计10类食物。根据《中国食物成分表(2004)》所提供的各种食物的能量和营养素的含量,可以求出各类食物能量和营养素的平均值,作为该类食物能量和营养的含量。参照膳食宝塔建议的各类食物的摄入量,计算膳食宝塔所能提供的能量和各营养素的量。水果和蔬菜是重要的农产品,主要为人体提供矿物质、维生素、膳食纤维。

矿物质是构成人体组织和维持正常生理功能必需的各种元素的总称,是人体必需的营养素之一。人体中含有的各种元素,除了碳、氧、氢、氮等主要以有机物的形式存在以外,其余的60多种元素统称为矿物质(也叫无机盐),其中钙、镁、钾、钠、铁、锌、硒元素含量较多。

维生素是维持人体生命活动必须的一类有机物质,也是保持人体健康的重要活性物质。维生素在体内的含量很少,但不可或缺。人体一共需要13种维生素,也就是通常所说的13种必要维生素:维生素A,维生素B,维生素C,维生素D,维生素H,维生素P,维生素PP,维生素M,维生素T,维生素U,水溶性维生素。

膳食纤维是一种不能被人体消化的碳水化合物,分为非水溶性和水溶性纤维两大类。纤维素、半纤维素和木质素是3种常见的非水溶性纤维,存在于植物细胞壁中;而果胶和树胶等属于水溶性纤维,则存在于自然界的非纤维性物质中。

问题二中,首先,为了摸清我国居民矿物质、维生素、膳食纤维等营养素摄入现状,需要比较准确地了解人体健康所需要的各类营养成分的范围。营养成分种类较多,根据《中国居民膳食营养素参考日摄入量》,同时分析膳食宝塔对一般成人营养素的满足程度,筛选出具有代表性的营养素成分和含量范围。根据问题一中得出的居民常见果蔬人均消费量计算得出实际的营养元素的人均日摄入量,将之与营养素参考范围进行比较,进而评价近年来我国居民目前矿物质、维生素、膳食纤维等营养素的日摄入水平是否合理。

其次,基于我国居民果蔬近期的消费趋势,预测到2020年我国居民的人体营养健康状况变化程度。同样根据问题一中预测出的2000-2020年水果和蔬菜人均消费总量,以及通过《中国食品成分表(2000)》,选取主要种类果蔬的矿物质、维生素、膳食纤维等营养素的成分与含量,包括膳食纤维、维生素A、维生素B1、维生素B2、维生素C、维生素E、钙、钠、铁、锌和硒。综合以上各方面的指标,构建出我国居民人体营养健康评价指标体系来进行因子分析,来进行居民水果和蔬菜营养素年摄入总量评价。

\subsubsection{居民常见品种的果蔬营养素日摄入量评价}

上面得到居民果蔬的人均消费量以及常见主要品种的人均消费量,其中水果10种,包括香蕉$x_{1}$、苹果$x_{2}$、橘子$x_{3}$、梨$x_{4}$、葡萄$x_{5}$、菠萝$x_{6}$、大枣$x_{7}$、柿子$x_{8}$、西瓜$x_{9}$、草莓$x_{10}$;蔬菜10种,包括萝卜$z_{1}$、胡萝卜$z_{2}$、土豆$z_{3}$、大蒜$z_{4}$、茄子$z_{5}$、黄瓜$z_{6}$、大白菜$z_{7}$、芹菜$z_{8}$、菠菜$z_{9}$、西红柿$z_{10}$。通过《中国食品成分表(2000)》,选取主要种类果蔬的矿物质、维生素、膳食纤维等营养素的成分与含量,包括膳食纤维、维生素A、维生素B1、维生素B2、维生素C、维生素E、钙、钠、铁、锌、硒。

通过2011-2012年居民果蔬的人均消费量以及常见品种的人均消费量,测算水果和蔬菜各类营养素的日摄入量平均值(以每100g可食部计)。

\begin{equation}
V_{\text{水果}} = \frac{\sum V_i X X_i}{\sum X X_i} \times \frac{1}{365} \tag{5.2.1}
\end{equation}

\begin{equation}
V_{\text{蔬菜}} = \frac{\sum V_i Z Z_i}{\sum Z Z_i} \times \frac{1}{365} \tag{5.2.2}
\end{equation}

\begin{table}[h]
\centering
\caption{2011-2012年常见10种水果营养素的日摄入量平均值}
\begin{tabular}{|c|c|c|c|}
\hline
营养成分 & 水果标准平均值 & 2011年 & 2012年 \\
\hline
膳食纤维(g) & 1.60 & 0.249603 (-) & 0.253029 (-) \\
\hline
维生素A(μg) & 46.11 & 43.32882 (-) & 43.09923 (-) \\
\hline
维生素B1(mg) & 0.03 & 0.021714 (-) & 0.021492 (-) \\
\hline
维生素B2(mg) & 0.02 & 0.028040 (+) & 0.027727 (+) \\
\hline
维生素C(mg) & 36.99 & 12.53486 (-) & 12.37243 (-) \\
\hline
维生素E(mg) & 0.10 & 0.449879 (+) & 0.445469 (+) \\
\hline
钙(mg) & 10.30 & 20.81445 (+) & 20.50297 (+) \\
\hline
钠(mg) & 7.35 & 22.38081 (+) & 21.97678 (+) \\
\hline
铁(mg) & 0.26 & 0.347438 (+) & 0.344847 (+) \\
\hline
锌(mg) & 0.12 & 0.129787 (+) & 0.128715 (+) \\
\hline
硒(μg) & 0.13 & 0.252611 (+) & 0.250363 (+) \\
\hline
\end{tabular}
\end{table}

(注:其中“+”表示常见品种水果营养素日摄入量超过平均摄入量,“-”表示常见品种水果营养素日摄入量超过平均摄入量)

\begin{figure}[h]
\centering
\includegraphics[width=\textwidth]{image.png}
\caption{2011-2012年主要10种水果营养素的日摄入量柱状图}
\end{figure}

\begin{table}
\centering
\caption{2011-2012年常见10种蔬菜营养素的日摄入量平均值}
\begin{tabular}{|c|c|c|c|}
\hline
营养成分 & 蔬菜标准平均值 & 2011年 & 2012年 \\
\hline
膳食纤维(g) & 3.10 & 2.224262(-) & 2.21629(-) \\
\hline
维生素A(μg) & 219.53 & 211.6632(-) & 214.065(-) \\
\hline
维生素B1(mg) & 0.04 & 0.043446(+) & 0.044114(+) \\
\hline
维生素B2(mg) & 0.05 & 0.051894(+) & 0.052499(+) \\
\hline
维生素C(mg) & 19.22 & 31.95703(+) & 33.06782(+) \\
\hline
维生素E(mg) & 0.44 & 0.925024(+) & 0.942022(+) \\
\hline
钙(mg) & 49.37 & 23.17955(-) & 23.6076(-) \\
\hline
钠(mg) & 59.39 & 2.396619(-) & 2.421867(+) \\
\hline
铁(mg) & 0.77 & 0.388012(-) & 0.393499(-) \\
\hline
锌(mg) & 0.27 & 0.385498(+) & 0.399941(+) \\
\hline
硒(μg) & 0.40 & 0.756483(+) & 0.772696(+) \\
\hline
\end{tabular}
\end{table}

(注:其中“+”表示常见品种蔬菜营养素日摄入量超过平均摄入量,“-”表示常见品种水果营养素日摄入量超过平均摄入量)

\begin{figure}[h]
    \centering
    \includegraphics[width=\textwidth]{image.png}
    \caption{2011-2012年常见10种蔬菜营养素的日摄入量柱状图}
    \label{fig:11}
\end{figure}

从上表和图中可以看出,常见品种蔬菜的11种营养素的日摄入量,2012年和2011年相比较结果正好与水果结果相反,除了膳食纤维摄入量减小0.36\%以外,其他营养素摄入量都有所增加,增长幅度均在4\%以内。其中,增长幅度最大的为锌,为37.47\%;增幅最小的钠,为1.05\%。

只考虑2012年的常见品种蔬菜营养素的日摄入量,与蔬菜营养素日摄入量标准平均值进行比较。可以得到,膳食纤维、维生素A、钙、钠及铁均不能满足,钠仅为标准平均值的4.08\%,摄入量严重不足;钙、铁和膳食纤维为标准平均值的47.82\%、51.10\%和71.49\%,同样不能满足摄入需要;维生素A占标准平均值的97.47\%,不满足标准平均值,但在合理的增减范围内。另一方面,维生素B1、维生素B2、维生素C、维生素E、锌、硒营养素日摄入量均超过标准平均值。其中,维生素E、维生素C、硒和锌日摄入量分别为标准平均值的2.14、1.72、1.93和1.48倍,属于摄入过量;而维生素B1和维生素B2日摄入量超过量占标准平均值的10.29\%和5.00\%,在合理增长范围内。

因此,我国居民仅仅从常见的10种蔬菜中摄入矿物质、维生素、膳食纤维等营养素的水平不能满足人们日常需要。

\subsubsection{居民水果和蔬菜营养素年摄入总量评价}

根据《中国居民膳食营养参考摄入量(2004)》、《中国居民膳食指南(2007)》以及“中国居民平衡膳食宝塔(2007)”,可以得到膳食宝塔对我国居民的各类食物摄入量及营养素。通过2011-2012年居民果蔬的人均消费量,测算水果和蔬菜各类营养素的年摄入总量,详见下表。

\begin{table}[h]
    \centering
    \caption{2011-2012年居民水果营养素的年摄入总量}
    \label{tab:13}
    \begin{tabular}{|c|c|c|c|c|}
    \hline
    营养成分 & 摄入量下限 & 摄入量上限 & 2011年 & 2012年 \\ \hline
    摄入量(kg) & 36.5 & 73 & 37.43902 & 38.43808 \\ \hline
    膳食纤维(g) & 584 & 2336 & 628.9755 & 645.7597 \\ \hline
    维生素A(μg) & 16790 & 67160 & 17263.13 & 17723.8 \\ \hline
    \end{tabular}
\end{table}

\begin{table}
\centering
\begin{tabular}{|c|c|c|c|c|}
\hline
维生素B1(mg) & 9.125 & 36.5 & 11.2317 & 11.53142 \\
\hline
维生素B2(mg) & 9.125 & 36.5 & 7.487803** & 7.687616** \\
\hline
维生素C(mg) & 13505 & 53983.5 & 13848.69 & 14218.25 \\
\hline
维生素E(mg) & 36.5 & 146 & 37.43902 & 38.43808 \\
\hline
钙(mg) & 3759.5 & 15038 & 3856.219 & 3959.122 \\
\hline
钠(mg) & 2737.5 & 10585 & 2751.768 & 2825.199 \\
\hline
铁(mg) & 98.55 & 379.6 & 97.34144** & 99.93901 \\
\hline
锌(mg) & 43.8 & 175.2 & 44.92682 & 46.1257 \\
\hline
硒(μg) & 45.625 & 182.5 & 48.67072 & 49.9695 \\
\hline
\end{tabular}
\end{table}

\textbf{注:其中“**”表示居民水果营养素的年摄入总量小于摄入量范围}

\textbf{表14 2011-2012年居民蔬菜营养素的年摄入总量}

\begin{table}
\centering
\begin{tabular}{|c|c|c|c|c|}
\hline
营养成分 & 摄入量下限 & 摄入量上限 & 2011年 & 2012年 \\
\hline
摄入量(kg) & 109.5 & 182.5 & 102.6477** & 99.52938** \\
\hline
膳食纤维(g) & 10183.5 & 28287.5 & 3182.078** & 3085.411** \\
\hline
维生素A(μg) & 721605 & 2002025 & 22534.24** & 21849.69** \\
\hline
维生素B1(mg) & 120.45 & 328.5 & 41.05906** & 39.81175** \\
\hline
维生素B2(mg) & 175.2 & 492.75 & 51.32383** & 49.76469** \\
\hline
维生素C(mg) & 63181.5 & 175382.5 & 19728.88** & 19129.55** \\
\hline
维生素E(mg) & 1423.5 & 4015 & 451.6497** & 437.9293** \\
\hline
钙(mg) & 162169.5 & 450592.5 & 50677.15** & 49137.66** \\
\hline
钠(mg) & 193815 & 538375 & 60962.45** & 59110.5** \\
\hline
铁(mg) & 2540.4 & 7062.75 & 790.387** & 766.3762** \\
\hline
锌(mg) & 886.95 & 2463.75 & 277.1487** & 268.7293** \\
\hline
硒(μg) & 1314 & 3650 & 410.5906** & 398.1175** \\
\hline
\end{tabular}
\end{table}

\textbf{注:其中“**”表示居民水果营养素的年摄入总量小于摄入量范围}

从上表可以看出,居民水果营养素的年摄入总量,2011-2012年大体符合建议的营养素摄入范围之内,但摄入量整体偏低,接近营养范围的下限。其中,维生素B2的年摄入量两年均低于营养范围之外;2011年铁的年摄入量也低于营养范围外,但2012年的年摄入量回到营养范围之内。因此,综上所述,我国目前从水果中摄入矿物质、维生素、膳食纤维等营养素的水平大致符合营养范围,但整体偏低。另一方面,居民蔬菜营养素的年摄入总量,2011-2012年均低于建议的营养素摄入范围。

\subsubsection{模型的建立与求解}

根据问题一中预测出的2000-2020年水果和蔬菜人均消费总量,以及通过《中国食品成分表(2000)》,选取主要种类果蔬的矿物质、维生素、膳食纤维等营养素的成分与含量,包括膳食纤维、维生素A、维生素B1、维生素B2、维生素C、维生素E、钙、钠、铁、锌和硒。综合以上各方面的指标,可以构建出我国居民人体营养健康评价指标体系,主要有矿物质、维生素、膳食纤维三个方面组成,详见下表所示。

\begin{table}[h]
\centering
\caption{我国居民人体营养健康评价指标体系}
\begin{tabular}{|c|c|c|c|}
\hline
一级指标 & 二级指标 & 符号代码 & 单位 \\
\hline
膳食纤维 & 膳食纤维 & W1 & g \\
\hline
\multirow{5}{*}{维生素} & 维生素A & W2 & $\mu$g \\
\cline{2-4}
 & 维生素B1 & W3 & mg \\
\cline{2-4}
 & 维生素B2 & W4 & mg \\
\cline{2-4}
 & 维生素C & W5 & mg \\
\cline{2-4}
 & 维生素E & W6 & mg \\
\hline
\multirow{5}{*}{矿物质} & 钙 & W7 & mg \\
\cline{2-4}
 & 钠 & W8 & mg \\
\cline{2-4}
 & 铁 & W9 & mg \\
\cline{2-4}
 & 锌 & W10 & mg \\
\cline{2-4}
 & 硒 & W11 & $\mu$g \\
\hline
\end{tabular}
\end{table}

\paragraph{因子分析的数学模型}

因子分析的任务一是构造一个因子模型,确定模型中的参数,然后根据分析结果进行因子分解;二是对公共因子进行估计,并作进一步分析。

二级指标 $W1, W2, \cdots, W11$,且每个变量(或经标准化处理后)的均值为0,标准差均为1。现将每个原有变量用 $n$($n<11$)个公共因子 $F_1, F_2, \ldots, F_n$ 的线性组来表示,即因子分析的一般模型为:
\begin{equation}
\left\{
\begin{aligned}
W1 &= a_{1,1}F_1 + a_{1,2}F_2 + \cdots + a_{1,n}F_n + \varepsilon_1 \\
W2 &= a_{2,1}F_1 + a_{2,2}F_2 + \cdots + a_{2,n}F_n + \varepsilon_2 \\
&\vdots \\
W11 &= a_{11,1}F_1 + a_{11,2}F_2 + \cdots + a_{11,n}F_n + \varepsilon_n
\end{aligned}
\right.
\tag{5.2.3}
\end{equation}
式中,$a_{i,j}$($i=1,2,\ldots,11$;$j=1,2,\ldots,n$)为因子载荷;$\varepsilon_i$($i=1,2,\ldots,11$)为特殊因子。

因子载荷。可以证明,在因子不相关的前提下,因子载荷 $a_{i,j}$ 是变量 $W_i$ 与因子 $F_j$ 的相关系数,反映了变量 $W_i$ 与因子 $F_j$ 的相关程度。因子载荷 $a_{i,j}$ 也反映了因子 $F_j$ 对解释变量的重要作用和程度。

变量共同度。因子载荷矩阵中各行数值的平方和,称为变量对应的共同度。变量共同度也即变量的方差,变量 $W_i$ 的共同度 $h_i^2$ 的数学定义为:

\begin{equation}
h_{i}^{2} = \sum_{j=1}^{n} a_{ij}^{2}
\tag{5.2.4}
\end{equation}

可以证明变量 \( Wi \) 的共同度刻画了因子全体对变量 \( Wi \) 信息解释的程度,是评价变量 \( Wi \) 信息丢失程度的重要指标。如果大多数原有变量的变量共同度均较高(如高于 0.8),则说明提取的因子能够反映原有变量的大部分(80\% 以上)信息,仅有较少的信息丢失,因子分析的效果较好。因此,变量共同度是衡量因子分析效果的重要依据。

公共因子是在各个变量中共同出现的因子,在高维空间中,它们是相互垂直的坐标轴。特殊因子实际上就是实测变量与估计值之间的残差值。如果特殊因子为零,则成为主成分分析。

\paragraph{相关系数矩阵}

本文通过 SPSS 20 统计软件对 2000-2020 年果蔬各营养素摄入量数据进行标准数据处理后,得到原有变量的相关系数矩阵。从而可以看出,大部分的相关系数都交高(如果相关系数矩阵中的大部分相关数值均小于 0.3,不适合因子分析),各变量呈现较强的线性相关,能够从中提取公共因子,适合进行因子分析。同时,也为下一步分析打下基础。

\paragraph{总方差分析}

本文提取因子的方法为主成份,通过 SPSS 20 统计软件,将多个线性相关的变量组合成独立的、少数的几个能反映总体信息的综合因子,进行进一步分析。表 25 是总方差分析表,是因子分析的初始解,显示了所有变量的共同度。

\begin{table}[h]
\centering
\caption{总方差分析表}
\begin{tabular}{c|c c c|c c c|c c c}
\hline
\multirow{2}{*}{ 成份 } & \multicolumn{3}{c|}{ 初始特征值 } & \multicolumn{3}{c|}{ 提取平方和载入 } & \multicolumn{3}{c}{ 旋转平方和载入 } \\
\cline{2-10}
& 合计 & 方差\% & 累积\% & 合计 & 方差\% & 累积\% & 合计 & 方差\% & 累积\% \\
\hline
1 & 9.244 & 84.04 & 84.04 & 9.244 & 84.04 & 84.04 & 8.662 & 78.741 & 78.741 \\
2 & 1.756 & 15.96 & 100 & 1.756 & 15.96 & 100 & 2.338 & 21.259 & 100 \\
3 & 5.78E-16 & 5.25E-15 & 100 & & & & & & \\
4 & 3.37E-16 & 3.06E-15 & 100 & & & & & & \\
5 & 1.23E-16 & 1.12E-15 & 100 & & & & & & \\
6 & 1.00E-17 & 9.12E-17 & 100 & & & & & & \\
7 & -7.43E-17 & -6.75E-16 & 100 & & & & & & \\
8 & -1.24E-16 & -1.13E-15 & 100 & & & & & & \\
9 & -1.84E-16 & -1.68E-15 & 100 & & & & & & \\
10 & -4.46E-16 & -4.06E-15 & 100 & & & & & & \\
11 & -8.91E-16 & -8.10E-15 & 100 & & & & & & \\
\hline
\end{tabular}
\end{table}

上表说明,对原有 11 个变量采用主成份分析方法提取所有特征根,那么原有变量的所有放出都可以被解释,第二列是在制定提取条件提取特征根时的共同度。可以看出,所有变量的绝大部分信息可被因子解释,变量信息丢失较少。因此,本次因子提取的总体效果比较理想。

此外,因子旋转后,累计方差比没有改变,也就是没有影响原有变量的共同度,但却重新分配了各个因子解释变量的方差,改变了各因子的方差贡献,使因子更容易解释。其中,第一个主因子旋转前提取的信息为 84.040\%,旋转后提取的信息为 78.741\%;第二个主因子旋转前提取的信息为 15.960\%,旋转后提取的信息为 21.259\%。最后,共提取了包含 100\% 信息的前 2 个综合因子 F1、F2。

\begin{table}[h]
\centering
\caption{旋转因子载荷负载表}
\begin{tabular}{c c c}
\hline
 & \multicolumn{2}{c}{成份} \\
\cline{2-3}
 & $F_{1}$ & $F_{2}$ \\
\hline
W8 & 0.994 & 0.113 \\
W7 & 0.990 & 0.139 \\
W6 & 0.989 & 0.145 \\
W11 & 0.984 & 0.178 \\
W9 & 0.983 & 0.183 \\
W4 & 0.979 & 0.205 \\
W10 & 0.975 & 0.222 \\
W1 & 0.965 & 0.262 \\
W3 & 0.934 & 0.357 \\
W2 & 0.116 & 0.993 \\
W5 & 0.264 & 0.965 \\
\hline
\end{tabular}
\end{table}

根据旋转因子载荷负载表中显示的各因子得分情况,可将11个因子分为2类。因子$F_{1}$支配的指标包括:膳食纤维、维生素B1、维生素B2、维生素E、钙、钠、铁、锌、硒,这些指标包括了全部膳食纤维、全部矿物质和部分维生素的二级指标,其方差贡献率为78.74\%。因子$F_{2}$支配的指标包括:维生素A、维生素C,这些指标包括了两个维生素的二级指标,其方差贡献率为21.26%。

\paragraph{居民人体营养健康状况各因子得分及排名}

根据因子得分系数矩阵,可以写出这两个主因子得分的计算公式,形式如下:
\begin{equation}
\begin{cases}
F_{1} = 0.105W1 + 0.084W3 + 0.117W4 + 0.128W6 + 0.129W7 \\
\quad + 0.134W8 + 0.121W9 + 0.113W10 + 0.122W11 \\
F_{2} = 0.516W2 + 0.491W5
\end{cases}
\tag{5.2.5}
\end{equation}

另外,用每个因子的公差贡献率作为相应因子的权数可以计算出居民人体营养健康状况综合得分,公式如下:
\begin{equation}
F = 0.7874F_{1} + 0.2126F_{2}
\tag{5.2.6}
\end{equation}

可以得到2014-2020年的居民人体营养健康状况各因子得分、综合得分及排名如下表所示。

\begin{table}[h]
\centering
\caption{2014-2020年的居民人体营养健康状况因子得分及排名}
\begin{tabular}{c c c c c c c}
\hline
年份 & 因子$F_{1}$ & 排名 & 因子$F_{2}$ & 排名 & 综合得分$F$ & 排名 \\
\hline
2014 & -0.82550 & 7 & 0.12900 & 1 & -0.62258 & 7 \\
2015 & -0.75889 & 1 & 0.11702 & 7 & -0.57268 & 1 \\
2016 & -0.79398 & 6 & 0.12179 & 3 & -0.59929 & 6 \\
2017 & -0.79281 & 5 & 0.12255 & 2 & -0.59821 & 5 \\
2018 & -0.78189 & 2 & 0.12045 & 6 & -0.59007 & 2 \\
2019 & -0.78955 & 4 & 0.12161 & 4 & -0.59585 & 4 \\
2020 & -0.7881 & 3 & 0.12151 & 5 & -0.59473 & 3 \\
\hline
\end{tabular}
\end{table}

\begin{figure}[h]
    \centering
    \includegraphics[width=\textwidth]{image.png}
    \caption{2014-2020年的居民人体营养健康状况预测因子趋势图}
    \label{fig:12}
\end{figure}

由上表和图可以看出,2014-2020年我国居民人体营养健康状况因子的综合得分结果略有上升,但波动不大。综合得分排名前三依次为2015年、2018年和2020年,且2015-2020年六年的综合得分均比2014年高,说明我国居民人体营养状况总体来说略有好转。

其中,作为包括了全部膳食纤维、全部矿物质和部分维生素的二级指标的因子$F_{1}$,其方差贡献率为78.74\%,得分排名前三依次为2015年、2018年和2020年,同样2015-2020年六年的综合得分均比2014年高,说明到2020年我国居民从果蔬中摄取膳食纤维、矿物质及部分维生素的年摄入水平略有好转,但波动不大。另外,作为包含维生素A和维生素C的因子$F_{2}$,其方差贡献率为21.26\%,其中得分排名第一的为2014年,说明从2014年开始,到2020年我国居民从果蔬中摄取维生素A和维生素C的年摄入水平有所下降。

\subsection{问题三的解答}

\subsubsection{问题三的分析}

不同的蔬菜和水果之间虽然营养素含量各不相同,但是营养素的种类大致相近,存在着食用功能的相似性。所以,水果与水果之间、蔬菜与蔬菜之间、水果与蔬菜之间从营养学角度在一定程度上可以相互替代、相互补充。所以本文选取了主要的水果和蔬菜进行研究,在算出主要蔬菜的基础上根据营养学角度的替代功能可以得出人体对于蔬菜和水果的合理饮食。

基于前文选取的10种水果和10种蔬菜,通过中国统计年鉴中得到了7种水果和10种蔬菜的2013年每个月的价格进行研究。本题要求从提供的主要水果和蔬菜产品中按照季节区别和合理的人均消费量选取果蔬,使得人们以较低的购买成本满足自身的营养需要。首先,观察10种蔬菜和7种水果的2013年的月度数据,发现10种蔬菜随季节变动价格变化相对平稳,而7种水果中,5种水果相对于季节变动价格变化平稳,西瓜和葡萄的数据在8-10月份与其它月份价格差别较大。所以,在分析时,将8-10月作为夏季的数据作为一个整体进行分析,将剩余的月份作为一个整体作为其它季节进行分析。所以,在建立优化模型时,需要分别建立数学模型。使用Lingo软件建立优化模型,对蔬菜全年价格,水果夏季价格和其它季节价格分别进行合理的线性规划。

\subsubsection{模型的建立与求解}

1. 线性规划模型

线性规划是运筹学中研究较早、发展较快、应用广泛、方法较成熟的一个重要分支,它是辅助人们进行科学管理的一种数学方法。

从实际问题中建立数学模型一般有以下三个步骤:根据影响所达到目的的因素找到决策变量;由决策变量和所在达到目的之间的函数关系确定目标函数;由决策变量所受的限制条件确定决策变量多要满足的约束条件。

一般函数表达式为:

\begin{equation}
\text{目标函数: } \min (\max ) \sum a_{i} m_{i}
\tag{5.3.1}
\end{equation}

\begin{equation}
\text{约束条件: } \sum_{i=1}^{m} \sum_{j=1}^{n} b_{ij} m_{i} \leq n
\tag{5.3.2}
\end{equation}

所建立的数学模型具有以下的特点:每个模型都有若干个决策变量 $(a_{1}, a_{2}, a_{3}, \ldots, a_{n})$,其中 $n$ 为决策变量个数。决策变量的一组值表示一种方案,同时决策变量一般是非负的;)墓边函数时决策变量的线性函数,根据具体问题可以是最大化或是最小化,二者统称为最优化;约束条件也是决策变量的线性函数。

当得到的数学模型的目标函数为线性函数,约束条件为线性等式或不等式时称此数学模型为线性规划模型。

水果中选取了前面分析水果中常见的7种水果:香蕉、苹果、橘子、梨、葡萄、菠萝和西瓜。蔬菜选取了10种:萝卜、胡萝卜、土豆、蒜苗、茄子、黄瓜、大白菜、芹菜、菠菜和西红柿。

(1)对于水果进行分季节讨论。

首先,我们选取了7种水果,基于2013年8-10月份各种水果的平均价格,对其建立优化模型,程序见附件,其中 $x e(1)-x e(7)$ 分别表示香蕉、苹果、橘子、梨、葡萄、菠萝和西瓜的食用量。分析结果见下表:

\begin{table}[h]
\centering
\caption{2013年8-10月水果线性规划结果表}
\begin{tabular}{|c|c|c|}
\hline
Variable & Value & Reduced Cost \\
\hline
$x e(1)$ & 0 & 2.873 \\
\hline
$x e(2)$ & 0 & 4.153 \\
\hline
$x e(3)$ & 0.031 & 0 \\
\hline
$x e(4)$ & 0.039 & 0 \\
\hline
$x e(5)$ & 0 & 7.887 \\
\hline
$x e(6)$ & 0 & 1.150 \\
\hline
$x e(7)$ & 0.623 & 0 \\
\hline
\end{tabular}
\end{table}

此时的8-10月份最低人均水果的日消费价格为2.835元。

接着我们再对其进行灵敏度分析,及在最优基保持不变的情况下我们研究系数的变化范围,我们可以看出:

$c 1=1.57$,则 $c 1$ 在 $(1.282,+\infty)$ 内原最优解不变,但最优值发生变化;

\begin{itemize}
    \item c2=2.35,则c2在(2.062,2.451)内原最优解不变,但最优值发生变化;
    \item c3=2.51,则c3在(1.495,$+\infty$)内原最优解不变,但最优值发生变化;
    \item c4=4.78,则c1在(3.845,$+\infty$)内原最优解不变,但最优值发生变化;
    \item c5=3.67,则c1在(2.777,$+\infty$)内原最优解不变,但最优值发生变化;
    \item c6=4.03,则c1在(1.068,$+\infty$)内原最优解不变,但最优值发生变化;
    \item c7=1.61,则c1在(1.282,$+\infty$)内原最优解不变,但最优值发生变化;
    \item c8=3.12,则c1在(2.991,$+\infty$)内原最优解不变,但最优值发生变化;
    \item c9=4.07,则c1在(3.632,$+\infty$)内原最优解不变,但最优值发生变化;
    \item c10=3.62,则c1在(1.068,$+\infty$)内原最优解不变,但最优值发生变化。
\end{itemize}

从表和灵敏度分析中可以看出:最优的结果是每天花费2.835元吃0.031千克的橘子、0.039千克的梨和0.623千克的西瓜,便可以满足人体8-10月份所需要的水果所包含的维生素、矿物质和膳食纤维的含量。如果不需要满足最低购买成本这个要求,我们可以对照Reduced Cost来观察结果,对于xe(1)来说,在我们满足需求的前提下,如果增加1千克的香蕉的消费量,我们需要多付出2.835元的价格。所以给定我们每天可以消费在水果上的价钱范围,我们可以根据上述表格进行合理的选择水果量。

接着,我们考虑2013年1-7月及11-12月。在其他月份中,上述的7种水果中有两种夏季的令水果:葡萄和西瓜。因为在其他月份,我们对于西瓜和葡萄的消费量较低,所以我们在研究其他月份的水果时,暂时不考虑西瓜和葡萄。所以我们选取了5种常见水果,基于2013年除了8、9和10月份以外的价格数据,建立线性规划模型,程序见附件。其中xe(1)-xe(5)分别表示香蕉、苹果、橘子、梨和菠萝。分析结果见下表:

\begin{table}[h]
    \centering
    \caption{2013年1-7月及11-12月水果线性规划表}
    \begin{tabular}{|c|c|c|}
        \hline
        Variable & Value & Reduced Cost \\
        \hline
        xe(1) & 0 & 3.361 \\
        \hline
        xe(2) & 0 & 4.068 \\
        \hline
        xe(3) & 1.154 & 0 \\
        \hline
        xe(4) & 0 & 1.488 \\
        \hline
        xe(5) & 0 & 2.013 \\
        \hline
    \end{tabular}
\end{table}

此时的其他月份最低人均水果的日消费价格为4.419元。

接着我们再对其进行灵敏度分析,及在最优基保持不变的情况下我们研究系数的变化范围,我们可以看出:
\begin{itemize}
    \item c1=4.57,则c1在(1.697,$+\infty$)内原最优解不变,但最优值发生变化;
    \item c2=6.83,则c2在(2.677,$+\infty$)内原最优解不变,但最优值发生变化;
    \item c3=3.82,则c3在(2.522,5.378)内原最优解不变,但最优值发生变化;
    \item c4=3.71,则c1在(6.111,7.085)内原最优解不变,但最优值发生变化;
    \item c5=9.14,则c1在(1.253,$+\infty$)内原最优解不变,但最优值发生变化;
    \item c6=4.52,则c1在(3.370,$+\infty$)内原最优解不变,但最优值发生变化;
    \item c7=4.13,则c1在(1.348,6.525)内原最优解不变,但最优值发生变化。
\end{itemize}

从表和灵敏度分析中可以看出:最优的结果是每天花费4.419元吃1.154千克的橘子,便可以满足人体8-10月份所需要的水果所包含的维生素、矿物质和膳食纤维的含量。虽然这里可能每天消耗的水果量可能比较大,但是也是在比较合理的范围内满足营养含量的最低购买成本。如果不需要满足最低购买成本这个要求,可以对照Reduced Cost来观察结果,对于xe(1)来说,在满足需求的前提下,如果增加1千克的香蕉的消

\begin{table}
\centering
\caption{蔬菜的线性规划结果表}
\begin{tabular}{|c|c|c|}
\hline
Variable & Value & Reduced Cost \\
\hline
ze(1) & 0 & 0.288 \\
\hline
ze(2) & 0.791 & 0 \\
\hline
ze(3) & 0 & 1.015 \\
\hline
ze(4) & 0 & 0.935 \\
\hline
ze(5) & 0 & 0.892 \\
\hline
ze(6) & 0 & 2.962 \\
\hline
ze(7) & 0.1 & 0 \\
\hline
ze(8) & 0 & 0.129 \\
\hline
ze(9) & 0 & 0.438 \\
\hline
ze(10) & 0 & 2.552 \\
\hline
\end{tabular}
\end{table}

此时的最低人均蔬菜的日消费价格为 2.019 元。

接着我们再对其进行灵敏度分析,及在最优基保持不变的情况下我们研究系数的变化范围,我们可以看出:
\begin{itemize}
    \item c1=4.54,则 c1 在 (1.178, +\infty) 内原最优解不变,但最优值发生变化;
    \item c2=6.72,则 c2 在 (2.652, +\infty) 内原最优解不变,但最优值发生变化;
    \item c3=3.83,则 c3 在 (6.593, 0) 内原最优解不变,但最优值发生变化;
    \item c4=3.55,则 c1 在 (2.062, +\infty) 内原最优解不变,但最优值发生变化;
    \item c5=4.37,则 c1 在 (2.357, +\infty) 内原最优解不变,但最优值发生变化。
\end{itemize}

从表和灵敏度分析中可以看出:最优的结果是每天花费 2.019 元吃 0.791 千克的胡萝卜和 0.1 千克的白菜,便可以满足人体每天需要的蔬菜所包含的维生素、矿物质和膳食纤维的含量。

这种结果可能比较单一,但是要求是在满足营养需求的基础上选取最低的购买成本。如果不需要满足最低购买成本这个要求,可以对照 Reduced Cost 来观察结果,对于 ze(1) 来说,在满足需求的前提下,如果增加 1 千克的萝卜的消费量,需要多付出 0.288 元的价格。所以,给定每天可消费在蔬菜上的价钱范围,可以根据上述表格进行合理的选择蔬菜量。

\subsection{问题四的解答}

\subsubsection{问题四的分析}

为实现人体营养均衡满足健康需要,国家可能需要对水果和蔬菜各品种的生产规模做出战略性调整。这里需要考虑的因素很多:第一是居民人体的营养均衡;其次是尽量使得购买成本最低,同时使得种植者收入最高;最后还要满足国家的政策以及进出口。这几种因素之间的关系如下图所示:

\begin{figure}[h]
\centering
\includegraphics[width=0.8\textwidth]{image.png}
\caption{影响果蔬人均消费量的各因素关系图}
\end{figure}

本文采用多目标的线性规划问题来解决中国居民主要水果和蔬菜产品按年度合理人均消费量的问题,主要研究 7 种水果和 10 种蔬菜,基本数据取 2011 年的为主。

首先,需要考虑居民从果蔬中摄取的营养素满足人体的营养均衡,并将其作为约束条件来建立优化模型。

其次,考虑居民购买成本最低,同时使得种植者收入最高,将这两个因素作为目标函数。其中,居民购买成本(RC)为价格与消费量的乘积:

\begin{equation}
\text{RC} = \sum_{i=1} p_i x e_i
\tag{5.4.1}
\end{equation}

种植者收入(PC)为卖出果蔬的总价钱与利润率的乘积,这里一般取水果的利润率为 0.2,蔬菜的利润率为 0.5:

\begin{equation}
\text{PC} = 0.2 \sum_{i=1} p_i x e_i + 0.5 \sum_{j=1} p_j z e_j
\tag{5.4.2}
\end{equation}

最后,还要考虑国家的政策以及进出口贸易额,其中国家政策简化为土地面积的约束,根据每公顷土地果蔬的产量与土地面积的乘积等于果蔬的产量可以得到以下的约束条件:

\begin{equation}
\sum_{i=1} \frac{x_i}{x p_i} + \sum_{j=1} \frac{z_j}{z p_j} \leq S
\tag{5.4.3}
\end{equation}

此外,对于进出口因素的影响,其进口数量和出口数量与产量、消费量的具体关系如下:

\begin{equation}
\text{产量} \approx \text{消费量} + \text{进口数量} - \text{出口数量}
\tag{5.4.4}
\end{equation}

由于此处消费量只是常见果蔬的消费量,所以产量和进出口数量都必须乘以一定的比例来计算。根据中国统计年鉴中的数据计算得到,2011 年所研究的 10 种蔬菜占总蔬菜量的 66.7\%,7 种水果占总水果量的 75\%。

\subsubsection{模型的建立与求解}

\paragraph{多目标规划模型}

多目标决策由于考虑的目标多,有些目标之间又彼此有矛盾,这就使多目标问题成为一个复杂而困难的问题。但由于客观实际的需要,多目标决策问题越来越受到重视,因而出现了许多解决此决策问题的方法。一般来说,其基本途径是,把求解多目标问题转化为求解单目标问题。其主要步骤是,先转化为单目标问题,然后利用单目标模型的方法,求出单目标模型的最优解,以此作为多目标问题的解。而本题就是采用这种方法来求解。

多目标规划问题的结构模型一般如下所示:
\begin{equation}
\begin{aligned}
\text{opt } F(M) &= (f_1(M), f_2(M), \ldots, f_p(M))^T \\
\text{s.t. } g_i(M) &\geq 0 \\
h_j(M) &= 0
\end{aligned}
\tag{5.4.5}
\end{equation}

如对于求极大(max)型,其各种解定义如下:
\begin{itemize}
    \item 绝对最优解:若对于任意的 $M$,都有 $F(M^*) \geq F(M)$
    \item 有效解:若不存在 $M$,使得 $F(M^*) \leq F(M)$
    \item 弱有效解:若不存在 $M$,使得 $F(M^*) < F(M)$。
\end{itemize}

基于多目标规划我们设置两个目标函数,一个是居民购买成本最小化,一个是种植者收入最大化。即:
\begin{equation}
\text{Min } RC = \sum_{i=1} p_i \cdot x e_i
\tag{5.4.6}
\end{equation}
\begin{equation}
\text{Max } PC = r^* \sum_{i=1} p_i x e_i
\tag{5.4.7}
\end{equation}
此时为了将其转化为单目标问题,将目标函数变为:
\begin{equation}
\text{Min } RC - PC
\tag{5.4.8}
\end{equation}
约束条件有:营养条件约束,土地约束,进出口约束。

\paragraph{水果的多目标规划模型}

本题选取了 7 种水果 2013 年的价格,运用 Lingo 11.0 软件,建立多目标线性规划模型,程序见附件,其中 $xx(1)$-$xx(7)$ 分别表示香蕉、苹果、橘子、梨、葡萄、菠萝和西瓜。分析结果见下表:

\begin{table}[h]
\centering
\caption{水果的多目标规划结果表}
\begin{tabular}{|c|c|c|}
\hline
Variable & Value & Reduced Cost \\
\hline
$xx(1)$ & 0 & 2.621 \\
\hline
$xx(2)$ & 0 & 2.566 \\
\hline
$xx(3)$ & 0.031 & 0 \\
\hline
$xx(4)$ & 0.039 & 0 \\
\hline
$xx(5)$ & 0 & 5.439 \\
\hline
$xx(6)$ & 0 & 1.552 \\
\hline
$xx(7)$ & 0.623 & 0 \\
\hline
\end{tabular}
\end{table}

此时的最目标函数最优值为 2.19 元。

接着再对其进行灵敏度分析,及在最优基保持不变的情况下我们研究系数的变化范围,可以看出:

\begin{itemize}
    \item c1=3.712,则 c1 在 $(1.089, +\infty)$ 内原最优解不变,但最优值发生变化;
    \item c2=4.16,则 c2 在 $(1.594, +\infty)$ 内原最优解不变,但最优值发生变化;
    \item c3=2.128,则 c3 在 $(1.334, 4.23)$ 内原最优解不变,但最优值发生变化;
    \item c4=2.472,则 c1 在 $(0.675, 7.025)$ 内原最优解不变,但最优值发生变化;
    \item c5=6.152,则 c1 在 $(0.713, +\infty)$ 内原最优解不变,但最优值发生变化;
    \item c6=3.6,则 c1 在 $(2.048, +\infty)$ 内原最优解不变,但最优值发生变化;
    \item c7=2.112,则 c1 在 $(0.775, 0.647)$ 内原最优解不变,但最优值发生变化。
\end{itemize}

从表和灵敏度分析中可以看出:从表中可以看出:最优的结果是每天吃 0.031 千克的橘子、0.039 千克的梨和 0.623 千克的西瓜,便可以满足人体需要的水果所包含的维生素、矿物质和膳食纤维的含量。如果不需要满足最低购买成本这个要求,可以对照 Reduced Cost 来观察结果,对于 xx(1) 来说,在满足需求的前提下,如果增加 1 千克的香蕉的消费量,需要多付出 2.621 元的价格。所以给定每天可以消费在水果上的价钱范围,可以根据上述表格进行合理的选择水果量。

\subsubsection{蔬菜的多目标规划模型}

本题选取了 10 种蔬菜 2013 年的价格,运用 Lingo11.0 软件,建立多目标线性规划模型。表中的 zz(1)-zz(10) 分别对应萝卜、胡萝卜、土豆、蒜苗、茄子、黄瓜、大白菜、芹菜、菠菜和西红柿。结果如下表所示:

\begin{table}[h]
\centering
\caption{蔬菜的多目标规划结果表}
\begin{tabular}{|c|c|c|}
\hline
Variable & Value & Reduced Cost \\
\hline
x(1) & 0 & 0.064 \\
\hline
x(2) & 0.086 & 0 \\
\hline
x(3) & 0 & 0.361 \\
\hline
x(4) & 0 & 0.685 \\
\hline
x(5) & 0 & 0.414 \\
\hline
x(6) & 0 & 0.904 \\
\hline
x(7) & 0.029 & 0 \\
\hline
x(8) & 0.585 & 0 \\
\hline
x(9) & 0 & 0.276 \\
\hline
x(10) & 0 & 0.984 \\
\hline
\end{tabular}
\end{table}

此时的最目标函数最优值为 0.673 元。

接着再对其进行灵敏度分析,及在最优基保持不变的情况下研究系数的变化范围,可以看出:

\begin{itemize}
    \item c1=0.52,则 c1 在 $(0.456, +\infty)$ 内原最优解不变,但最优值是要变的;
    \item c2=0.895,则 c2 在 $(0.821, 1.294)$ 内原最优解不变,但最优值是要变的;
    \item c3=0.945,则 c3 在 $(0.584, +\infty)$ 内原最优解不变,但最优值是要变的;
    \item c4=2.08,则 c1 在 $(1.395, +\infty)$ 内原最优解不变,但最优值是要变的;
    \item c5=1.35,则 c1 在 $(0.396, +\infty)$ 内原最优解不变,但最优值是要变的;
    \item c6=1.28,则 c1 在 $(0.376, +\infty)$ 内原最优解不变,但最优值是要变的;
    \item c7=0.495,则 c1 在 $(0.433, 0.623)$ 内原最优解不变,但最优值是要变的;
    \item c8=0.995,则 c1 在(0.085,1.132)内原最优解不变,但最优值是要变的;
    \item c9=1.57,则 c1 在(1.294,+∞)内原最优解不变,但最优值是要变的;
    \item c10=1.38,则 c1 在(0.396,+∞)内原最优解不变,但最优值是要变的。
\end{itemize}

从表和灵敏度分析中可以看出:最优的结果是每天吃 0.086 千克的胡萝卜、0.029 千克的大白菜和 0.585 千克的菠菜变以便可以满足人体每天需要的蔬菜所包含的维生素、矿物质和膳食纤维的含量。这种结果可能比较单一,但是要求是在满足营养需求的基础上选取最低的购买成本。如果不需要满足最低购买成本的要求,可以对照 Reduced Cost 来观察结果,对于 zz(1) 来说,在满足需求的前提下,如果增加 1 千克的萝卜的消费量,需要多付出 0.064 元的价格。所以给定每天可消费在蔬菜上的价钱范围,可以根据上述表格进行合理的选择蔬菜量。

\section{战略调整}

从上述的分析可以看出:在 2011 年,人均每日消费 0.7 千克蔬菜,0.693 千克的水果。将 2013 年的价格数据带入到本题中的程序中,可以得到人均日消费 0.845 千克蔬菜和 0.685 千克的水果。通过 2011 年和 2013 年的数据可以看出,人均日消费的水果缓慢减少,而蔬菜的日均消费量相比较于水果而言属于比较快的增长。

所以,到 2020 年,应该增加蔬菜的摄入量,而稍微减少水果的摄入量,并且可以增加水果和蔬菜的摄入种类,以保证我们的营养摄入更加合理。

\subsection{问题五的解答}

通过前面的分析可以看出,年末总人口、人均收入水平、物价水平、蔬菜水果产量、种植面积等各类因素均对居民水果和蔬菜的消费量有所影响,进而影响居民对各类矿物质、维生素、膳食纤维等营养素的摄入。因此,进一步的开发中国水果产品的消费需求,就必须要综合考虑以上几种因素对水果产品消费的影响。从本文的研究结论可以提出以下建议:

\begin{enumerate}
    \item 加强宣传力度,调整现存膳食习惯

近年来人均蔬菜水果消费量所带来的营养素的摄入并没有达到建议的营养素摄入量,可能是由于水果蔬菜的味道不如肉类、蛋类、奶类食品带给人们的食欲,或者蔬菜水果季节性较强,还有对蔬菜水果的营养价值没有正确的认识等原因造成。然而现存的这种以高蛋白、高脂肪为饮食主体的膳食习惯容易带来相关的慢性非传染性疾病如肥胖、糖尿病、心脑血管系统疾病等。政府、营养工作者需要进一步努力,大力宣传水果蔬菜的营养价值,制订科学的膳食模式和合理的营养结构,使人们意识到这类食物对人体的好处,促进人们更多的自觉自愿地消费蔬菜、水果,增加营养素的摄入量,使居民消费向营养保健型发展。

    \item 改善果蔬种植条件,调整果蔬种植规模

将水果蔬菜种植进行基地化、规模化和耕作机械化建设,大大减轻水果蔬菜生产劳动强度,提高了劳动生产率。狠抓良种普及、保护地栽培、无公害生产等技术,使果蔬产量、质量得到提高,市场供应充足,种类繁多。加强优质高档蔬菜的培育,追求有机、绿色、无公害品种多样、营养丰富、保健卫生、食用方便较高层次的生产目标。

    \item 改善果蔬储藏、加工、运输条件,改进果蔬交易市场

进一步完善果蔬交易市场体系,各地政府多渠道筹措资金,改建、扩建果蔬交易市场,形成功能齐全、辐射力强、网络本地、联通全国的果蔬营销市场。积极开辟“绿色通道”,支持、引导、发展民间中介组织和农民经纪人,形成以果蔬流通协会为龙头,销售信息服务中心为平台,中介组织为支撑,农民经纪人为网点的果蔬流通网络。

    \item 调整居民收入水平,完善社会保障体系,增强消费信心

收入与消费支出的关系大体上成正相关关系,即收入增加自然会带动消费支出的增加。特别是在当前我国大部分居民对于水果消费仍然处在一种享受性消费状态中,因此,只有大力提高居民可支配收入水平,水果消费才能从享受性消费变为常态消费,水果消费量才可以不断增加,消费市场才可以不断拓展。

    \item 积极采取措施,促进果蔬出口

果蔬的产出水平大大超过了国内消费能力。而且,随着近年来中国市场经济体制的不断完善与果蔬储运水平的提高,果蔬损耗率有下降的趋势。这样使得果蔬供给显得更加过剩。解决供给过剩最有效的方法就是加大果蔬出口力度,以缓解国内果蔬消费能力不足的局面。应当提高果蔬质量标准,打造国际品牌。防范技术性的非关税壁垒,增加科技投入,调整优化果蔬出口结构。
\end{enumerate}

\section{参考文献}

[1] 杨月欣,王光亚,潘兴昌,中国食物成分表 2002[M],北京:北京大学医学出版社,2002

[2] 李哲敏,中国城乡居民食物消费与营养发展的趋势预测分析,农业技术经济[J],第 6 期:57-62,2008

[3] 郭志薇,董加毅,童星,蒋霞,秦立强,中国居民平衡膳食宝塔(2007)的评价,现代预防医学[J],第 38 卷第 23 期:4835-4839,2011

[4] 中国营养学会,中国居民膳食指南(2007)[M],拉萨:西藏人民出版社,2008

[5] 曹志宏,陈志超,郝晋珉,中国城乡居民食物消费与营养发展的趋势预测分析,长江流域资源与环境[J],第 21 卷第 10 期:1171-1178,2012

[6] 张文彤,董伟,SPSS 统计分析高级教程[M],北京:高等教育出版社,2004

[7] 张峭,杨霞,中国水果消费现状分析及其预测,农业展望[J],第 8 期:30-33,2006

[8] 武拉平,张瑞娟,中国农村居民食品消费结构变化及趋势展望——基于 1950-2010 年统计数据的分析,农业展望[J],第 4 期:53-58,2011

[9] 崔朝辉,周琴,胡小琪等,中国居民蔬菜、水果消费现状分析,中国食物与营养[J],第 5 期:34-37,2008

[10] 杨晓光,孔灵芝,翟风英等,中国居民营养与健康状况调查的总体方案,中华流行病学杂志[J],第 26 卷第 7 期:77-83,2005

[11] 王伟新,于春燕,祁春节,水果和蔬菜消费上可以相互替代吗?——基于 VAR 模型的经济学检验,华中农业大学学报[J],第 113 期:60-67,2014

[12] 侯媛媛,王礼力,中国蔬菜家庭消费预测,统计与决策[J],第 323 期:91-93,2010

[13] 陶建平,熊刚初,徐晔,我国水果消费水平与城市化的相关性分析,中国农村经济[J],第 6 期:18-24,2004

[14] 刘汉成,中国水果供给结构性变化的实证研究,农业现代化研究[J],第 26 卷第 4 期:35-42,2005

[15] Michels KB,Giovannucci E,Chan AT et al,Fruit and vegetable consumption and colorectal adenomas in the Nurses’ Health Study,Cancer Res[J],2006;66(7):3942—3953

\begin{table}
\centering
\caption{水果总产量及各类水果年产量数据(单位:万吨)}
\begin{tabular}{|c|c|c|c|c|c|c|}
\hline
年份 & 水果总产量 & 香蕉 & 苹果 & 橘子 & 梨 & 葡萄 \\
\hline
2003 & 14,517.41 & 590.33 & 2,110.18 & 1,345.37 & 979.84 & 517.59 \\
\hline
2004 & 15,340.88 & 605.61 & 2,367.55 & 1,495.83 & 1,064.23 & 567.53 \\
\hline
2005 & 16,120.09 & 651.81 & 2,401.11 & 1,591.91 & 1,132.35 & 579.44 \\
\hline
2006 & 17,101.97 & 690.12 & 2,605.93 & 1,789.83 & 1,198.61 & 627.08 \\
\hline
2007 & 18,136.29 & 779.67 & 2,785.99 & 2,058.27 & 1,289.50 & 669.68 \\
\hline
2008 & 19,220.19 & 783.47 & 2,984.66 & 2,331.26 & 1,353.81 & 715.15 \\
\hline
2009 & 20,395.51 & 883.39 & 3,168.08 & 2,521.10 & 1,426.30 & 794.06 \\
\hline
2010 & 21,401.45 & 956.05 & 3,326.36 & 2,645.24 & 1,505.26 & 854.9 \\
\hline
2011 & 22,768.18 & 1,040.00 & 3,598.48 & 2,944.04 & 1,579.48 & 906.75 \\
\hline
年份 & 菠萝 & 大枣 & 柿子 & 西瓜 & 草莓 & \\
\hline
2003 & 82.19 & 171.87 & 179.51 & 5,800.24 & 169.79 & \\
\hline
2004 & 80.83 & 201.12 & 199.82 & 5,751.54 & 185.85 & \\
\hline
2005 & 84.89 & 248.85 & 218.5 & 5,989.34 & 195.71 & \\
\hline
2006 & 89.07 & 305.29 & 232.03 & 6,184.52 & 187.42 & \\
\hline
2007 & 90.51 & 303.06 & 257.41 & 6,203.62 & 187.18 & \\
\hline
2008 & 93.36 & 363.41 & 271.1 & 6,282.17 & 200.04 & \\
\hline
2009 & 104.26 & 424.78 & 283.42 & 6,478.47 & 220.6 & \\
\hline
2010 & 107.6 & 446.83 & 287.56 & 6,818.10 & 233 & \\
\hline
2011 & 119.11 & 542.68 & 318.72 & 6,889.35 & 249.08 & \\
\hline
\end{tabular}
\end{table}

\begin{table}
\centering
\caption{蔬菜总产量及各类蔬菜年产量数据(单位:万吨)}
\begin{tabular}{|c|c|c|c|c|c|c|}
\hline
年份 & 蔬菜总产量 & 萝卜 & 胡萝卜 & 土豆 & 大蒜 & 茄子 \\
\hline
2003 & 54032.3 & 3880.9 & 1312.4 & 1361.89 & 1555.6 & 2119.2 \\
\hline
2004 & 55064.7 & 3832.8 & 1329.9 & 1444.4 & 1566.6 & 2175.5 \\
\hline
2005 & 56451.49 & 3935.2 & 1331.4 & 1417.39 & 1925 & 2263.4 \\
\hline
2006 & 58325.54 & 4003 & 1443 & 1289.73 & 1832 & 2247 \\
\hline
2007 & 56452.04* & 3983.4* & 1459.9* & 1295.81* & 1882.9* & 2309.6* \\
\hline
2008 & 59240.3 & 3955 & 1476.7 & 1415.58 & 1933.7 & 2372.2 \\
\hline
2009 & 61823.8 & 4079.9 & 1505.7 & 1464.6 & 2075.4 & 2588.5 \\
\hline
2010 & 65099.4 & 4055 & 1585.6 & 1630.67 & 2067.8 & 2707.2 \\
\hline
2011 & 67929.7 & 4127.1 & 1640.6 & 1765.81 & 2094.2 & 2650.7 \\
\hline
年份 & 黄瓜 & 大白菜 & 芹菜 & 菠菜 & 西红柿 & \\
\hline
2003 & 3551.3 & 10197.4 & 1795.5 & 1573.9 & 2884.3 & \\
\hline
2004 & 3655.3 & 10346.3 & 1918.7 & 1567.4 & 3014.4 & \\
\hline
2005 & 3817.1 & 10308.3 & 1951.3 & 1617.9 & 3161.8 & \\
\hline
2006 & 4040 & 10506 & 2068 & 1627 & 3251.9 & \\
\hline
2007 & 4129.7* & 10525.8* & 2022* & 1644.9* & 3609.7* & \\
\hline
2008 & 4219.4 & 10545.5 & 1976 & 1662.8 & 3982.8 & \\
\hline
\end{tabular}
\end{table}

\begin{table}
\centering
\caption{我国居民人均水果消费量及其影响因素表}
\begin{tabular}{c c c c c c c c}
\hline
年份 & 年末总人口(万人)$V_{1}$ & 果类居民消费价格指数(上年=100)$V_{2}$ & 人均国内生产总值(元)$V_{3}$ & 居民消费水平(元)$V_{4}$ & 水果面积(千公顷)$V_{5}$ & 水果产量(万吨)$V_{6}$ & 国民总收入(亿万)$V_{7}$ \\
\hline
1996 & 122,389 & 104.5 & 5,845.89 & 2,789 & 9,756.00 & 8,120.81 & 70,142.49 \\
1997 & 123,626 & 94.4 & 6,420.18 & 3,002 & 9,951.20 & 8,982.12 & 78,060.85 \\
1998 & 124,761 & 96.2 & 6,796.03 & 3,159 & 10,142.11 & 10,565.44 & 83,024.28 \\
1999 & 125,786 & 97.8 & 7,158.50 & 3,346 & 10,431.39 & 11,700.85 & 88,479.15 \\
2000 & 126,743 & 96.6 & 7,857.68 & 3,632 & 10,975.29 & 6,225.15 & 98,000.45 \\
2001 & 127,627 & 99.9 & 8,621.71 & 3,887 & 11,310.25 & 13,501.62 & 108,068.22 \\
2002 & 128,453 & 103.1 & 9,398.05 & 4,144 & 11,452.75 & 14,374.58 & 119,095.69 \\
2003 & 129,227 & 103 & 10,541.97 & 4,475 & 11,790.49 & 14,517.41 & 134,976.97 \\
2004 & 129,988 & 104 & 12,335.58 & 5,032 & 11,914.94 & 15,340.88 & 159,453.60 \\
2005 & 130,756 & 102.2 & 14,185.36 & 5,596 & 12,242.52 & 16,120.09 & 183,617.37 \\
2006 & 131,448 & 117.9 & 16,499.70 & 6,299 & 12,368.51 & 17,101.97 & 215,904.41 \\
2007 & 132,129 & 102.2 & 20,169.46 & 7,310 & 12,722.63 & 18,136.29 & 266,422.00 \\
2008 & 132,802 & 110.8 & 23,707.71 & 8,430 & 12,990.81 & 19,220.19 & 316,030.34 \\
2009 & 133,450 & 107.1 & 25,607.53 & 9,283 & 13,472.67 & 20,395.51 & 340,319.95 \\
2010 & 134,091 & 114.6 & 30,015.05 & 10,522 & 13,933.51 & 21,401.45 & 399,759.54 \\
2011 & 134,735 & 115.9 & 35,197.79 & 12,570 & 14,219.80 & 22,768.18 & 468,562.38 \\
\hline
\end{tabular}
\end{table}

\begin{table}
\centering
\caption{居民人均水果年消费量主成分$F_{1}$得分表}
\begin{tabular}{c c}
\hline
年份 & 主成分$F_{1}$ \\
\hline
1996 & -1.3914 \\
1997 & -1.38072 \\
1998 & -1.17018 \\
1999 & -0.97942 \\
2000 & -1.072 \\
2001 & -0.57194 \\
2002 & -0.38051 \\
2003 & -0.22956 \\
2004 & -0.03399 \\
2005 & 0.119776 \\
2006 & 0.541565 \\
2007 & 0.52438 \\
2008 & 0.862793 \\
\hline
\end{tabular}
\end{table}

\begin{table}
\centering
\caption{我国居民各类水果的人均消费量(单位:千克)}
\begin{tabular}{c c c c c c}
\hline
年份 & 香蕉 & 苹果 & 橘子 & 梨 & 葡萄 \\
\hline
2003 & 1.079296411 & 5.611997313 & 4.094332521 & 2.242857412 & 0.930551775 \\
2004 & 1.052708072 & 5.504115826 & 3.903232664 & 2.26787445 & 0.930868316 \\
2005 & 1.053285002 & 5.676498518 & 4.028241551 & 2.326936027 & 0.936251066 \\
2006 & 0.935449109 & 5.355297813 & 3.730100367 & 2.211748046 & 0.844385023 \\
2007 & 1.037365631 & 5.570466317 & 3.669909772 & 2.347591445 & 0.881117739 \\
2008 & 0.966516769 & 5.484518338 & 3.359143631 & 2.296903854 & 0.868466986 \\
2009 & 0.885400402 & 4.90141776 & 2.897805404 & 2.10464824 & 0.778346098 \\
2010 & 0.846199477 & 4.971298512 & 2.800876534 & 2.034676421 & 0.78417704 \\
2011 & 0.880462063 & 4.729620167 & 2.688992212 & 1.999636699 & 0.763391386 \\
\hline
年份 & 菠萝 & 大枣 & 柿子 & 西瓜 & 草莓 \\
\hline
2003 & 0.176375901 & 0.812139415 & 0.40655734 & 8.879555777 & 0.252414837 \\
2004 & 0.169053141 & 0.709494526 & 0.389189324 & 9.32387218 & 0.250525476 \\
2005 & 0.177375964 & 0.730359799 & 0.415364151 & 9.593376094 & 0.256842896 \\
2006 & 0.159053146 & 0.62571137 & 0.397861951 & 9.315648137 & 0.233229386 \\
2007 & 0.171830884 & 0.581472297 & 0.420970383 & 10.25113178 & 0.243191898 \\
2008 & 0.177991744 & 0.616561975 & 0.399423871 & 10.75712897 & 0.256312237 \\
2009 & 0.164535212 & 0.487456336 & 0.364817139 & 10.10423087 & 0.259597402 \\
2010 & 0.161152082 & 0.40524191 & 0.343181419 & 9.980894308 & 0.253577613 \\
2011 & 0.174911515 & 0.369653863 & 0.329086099 & 10.74403354 & 0.247284292 \\
\hline
\end{tabular}
\end{table}

\begin{table}
\centering
\caption{我国居民人均蔬菜消费量及其影响因素表}
\begin{tabular}{c c c c c c c c}
\hline
年份 & 年末总人口(万人)$V_{1}$ & 蔬菜居民消费价格指数(上年=100)$V_{8}$ & 人均国内生产总值(元)$V_{3}$ & 居民消费水平(元)$V_{4}$ & 蔬菜面积(千公顷)$V_{9}$ & 蔬菜产量(万吨)$V_{10}$ & 国民总收入(亿万)$V_{7}$ \\
\hline
1996 & 122,389 & 119.1 & 5,845.89 & 2,789 & 10,490.70 & 30,123.09 & 70,142.49 \\
1997 & 123,626 & 100 & 6,420.18 & 3,002 & 11,288.10 & 35,962.39 & 78,060.85 \\
1998 & 124,761 & 99.6 & 6,796.03 & 3,159 & 12,292.76 & 38,491.93 & 83,024.28 \\
1999 & 125,786 & 101 & 7,158.50 & 3,346 & 13,346.85 & 40,513.52 & 88,479.15 \\
2000 & 126,743 & 104.7 & 7,857.68 & 3,632 & 15,237.27 & 44467.94 & 98,000.45 \\
2001 & 127,627 & 100.9 & 8,621.71 & 3,887 & 16,402.45 & 48,422.36 & 108,068.22 \\
2002 & 128,453 & 98.2 & 9,398.05 & 4,144 & 17,352.93 & 52,860.56 & 119,095.69 \\
2003 & 129,227 & 117.7 & 10,541.97 & 4,475 & 17,953.72 & 54,032.32 & 134,976.97 \\
2004 & 129,988 & 95.1 & 12,335.58 & 5,032 & 17,560.42 & 55,064.66 & 159,453.60 \\
\hline
\end{tabular}
\end{table}

\begin{table}
\centering
\begin{tabular}{c c c c c c c c}
\hline
2005 & 130,756 & 109.1 & 14,185.36 & 5,596 & 17,720.71 & 56,451.49 & 183,617.37 \\
\hline
2006 & 131,448 & 108.2 & 16,499.70 & 6,299 & 16,639.14 & 53,953.05 & 215,904.41 \\
\hline
2007 & 132,129 & 107.9 & 20,169.46 & 7,310 & 17,328.62 & 56,452.04 & 266,422.00 \\
\hline
2008 & 132,802 & 111 & 23,707.71 & 8,430 & 17,875.94 & 59,240.35 & 316,030.34 \\
\hline
2009 & 133,450 & 113.6 & 25,607.53 & 9,283 & 18,389.83 & 61,823.81 & 340,319.95 \\
\hline
2010 & 134,091 & 118.5 & 30,015.05 & 10,522 & 18,999.89 & 65,099.41 & 399,759.54 \\
\hline
2011 & 134,735 & 101.1 & 35,197.79 & 12,570 & 19,639.16 & 67,929.67 & 468,562.38 \\
\hline
\end{tabular}
\end{table}

\textbf{附表7 居民人均蔬菜年消费量主成分F₂得分表}

\begin{table}
\centering
\begin{tabular}{c c}
\hline
年份 & 主成分F₂ \\
\hline
1996 & -1.381068529 \\
\hline
1997 & -1.329135032 \\
\hline
1998 & -1.171088971 \\
\hline
1999 & -1.005624265 \\
\hline
2000 & -0.733854715 \\
\hline
2001 & -0.568203125 \\
\hline
2002 & -0.398889779 \\
\hline
2003 & -0.096950724 \\
\hline
2004 & -0.176177828 \\
\hline
2005 & 0.090622864 \\
\hline
2006 & 0.126017247 \\
\hline
2007 & 0.396221018 \\
\hline
2008 & 0.693435716 \\
\hline
2009 & 0.911067533 \\
\hline
2010 & 1.264276957 \\
\hline
2011 & 1.4872091 \\
\hline
\end{tabular}
\end{table}

\textbf{附表8 我国居民各类蔬菜的人均消费量 (单位:千克)}

\begin{table}
\centering
\begin{tabular}{c c c c c c}
\hline
年份 & 萝卜 & 胡萝卜 & 土豆 & 大蒜 & 茄子 \\
\hline
2003 & 7.202822 & 2.435775 & 2.539493 & 2.724278 & 3.786879 \\
\hline
2004 & 7.067834 & 2.452392 & 2.676039 & 2.72591 & 3.8625 \\
\hline
2005 & 6.839779 & 2.314113 & 2.475138 & 3.157106 & 3.787702 \\
\hline
2006 & 7.203385 & 2.596678 & 2.331764 & 3.110712 & 3.89308 \\
\hline
2007 & 6.818876 & 2.499095 & 2.228615 & 3.04137 & 3.806579 \\
\hline
2008 & 6.644911 & 2.481051 & 2.389527 & 3.065596 & 3.837366 \\
\hline
2009 & 6.473628 & 2.389117 & 2.334813 & 3.1073 & 3.954446 \\
\hline
2010 & 5.859616 & 2.291251 & 2.367441 & 2.819487 & 3.766499 \\
\hline
2011 & 5.587358 & 2.221084 & 2.401819 & 2.675241 & 3.455106 \\
\hline
年份 & 黄瓜 & 大白菜 & 芹菜 & 菠菜 & 西红柿 \\
\hline
2003 & 6.690318 & 14.34205 & 2.853716 & 2.211862 & 4.839499 \\
\hline
2004 & 6.841989 & 14.45795 & 3.029928 & 2.188572 & 5.025287 \\
\hline
2005 & 6.734385 & 13.57731 & 2.904386 & 2.129304 & 4.968204 \\
\hline
\end{tabular}
\end{table}

\begin{table}
\centering
\begin{tabular}{c c c c c c}
\hline
2006 & 7.379409 & 14.32649 & 3.186813 & 2.216917 & 5.290272 \\
\hline
2007 & 7.175737 & 13.65417 & 2.964115 & 2.132107 & 5.586242 \\
\hline
2008 & 7.195857 & 13.42645 & 2.843051 & 2.115404 & 6.04952 \\
\hline
2009 & 7.119491 & 12.73019 & 2.828191 & 2.107119 & 6.493219 \\
\hline
2010 & 6.875277 & 11.38644 & 2.740369 & 2.028611 & 6.257408 \\
\hline
2011 & 6.760114 & 11.12414 & 2.849927 & 1.94823 & 6.021662 \\
\hline
\end{tabular}
\end{table}

\textbf{程序1}

\textbf{问题三蔬菜:}

sets:

t/1..10/:x,p;

q/1..11/;

w(t,q):a;

endsets

data:

p=

1.57

2.35

2.51

4.78

3.67

4.03

1.61

3.12

4.07

3.62

;

a=

30 0.3 0.6 180 10 560 600 3 1.3 6 6

6880 0.4 0.3 130 4.1 320 714 10 2.3 6.3 11

50 0.8 0.4 270 3.4 80 27 8 3.7 7.8 7

470 1.1 0.8 350 8.1 290 51 14 4.6 12.4 18

80 0.2 0.4 50 10.13 240 54 5 2.3 4.8 13

150 0.2 0.3 90 4.6 240 49 5 1.8 3.8 5

420 0.6 0.7 470 9.2 690 893 5 2.1 3.3 6

100 0.1 0.8 120 22.1 480 738 8 4.6 5 14

4870 0.4 1.1 320 17.4 660 852 29 8.5 9.7 17

920 0.3 0.3 190 5.7 100 50 4 1.3 1.5 5

;

-45-

\begin{verbatim}
enddata
min=@sum(t:p*x);
@sum(t(i):a(i,1)*x)>=659;
@sum(t(i):a(i,2)*x)>=0.11;
@sum(t(i):a(i,3)*x)>=0.16;
@sum(t(i):a(i,4)*x)>=57.7;
@sum(t(i):a(i,5)*x)>=1.3;
@sum(t(i):a(i,6)*x)>=148.1;
@sum(t(i):a(i,7)*x)>=177;
@sum(t(i):a(i,8)*x)>=2.32;
@sum(t(i):a(i,9)*x)>=0.81;
@sum(t(i):a(i,10)*x)>=1.2;
@sum(t(i):a(i,11)*x)>=9.3;
\end{verbatim}

程序 2

问题三水果:

\begin{verbatim}
sets:
t/1..7/:x,p;
q/1..11/;
w(t,q):a;
endsets

data:
p=
4.57
6.83
3.82
3.71
9.14
4.52
4.13
;
a=
560  0.2  0.4  30  5  320  4  4  1.7  8.7  12
1000 0.1  0.3  80  14.6 110 9  1  0.1  10  12
2770 0.5  0.4  330 4.5  350 13 2  10   4.5  4
1000 0.3  0.3  40  14.6 30  7  7  1   9.8  31
50   0.5  0.3  40  3.4  110 5  2  0.2  5   4
330  0.8  0.2  240 2.3  180 8  5  1.4  2.4  13
1800 0.3  0.4  100 1   130 23 2  0.5  0.8  3
;
enddata
min=@sum(t:p*x);
\end{verbatim}

\begin{verbatim}
@sum(t(i):a(i,1)*x)>=92;
@sum(t(i):a(i,2)*x)>=0.05;
@sum(t(i):a(i,3)*x)>=0.05;
@sum(t(i):a(i,4)*x)>=74;
@sum(t(i):a(i,5)*x)>=0.2;
@sum(t(i):a(i,6)*x)>=20.6;
@sum(t(i):a(i,7)*x)>=15;
@sum(t(i):a(i,8)*x)>=0.54;
@sum(t(i):a(i,9)*x)>=0.24;
@sum(t(i):a(i,10)*x)>=0.25;
@sum(t(i):a(i,11)*x)>=3.2;

程序 3
第四问
蔬菜:
sets:
t/1..10/:x,p,b,c;
q/1..11/;
w(t,q):a;
endsets
data:
a=
30  0.3 0.6 180 10  560 600 3   1.3 6   6
6880 0.4 0.3 130 4.1 320 714 10  2.3 6.3 11
50  0.8 0.4 270 3.4 80  27  8   3.7 7.8 7
470 1.1 0.8 350 8.1 290 51  14  4.6 12.4 18
80  0.2 0.4 50  10.13 240 54  5   2.3 4.8 13
150 0.2 0.3 90  4.6 240 49  5   1.8 3.8 5
420 0.6 0.7 470 9.2 690 893 5   2.1 3.3 6
100 0.1 0.8 120 22.1 480 738 8   4.6 5   14
4870 0.4 1.1 320 17.4 660 852 29  8.5 9.7 17
920 0.3 0.3 190 5.7 100 50  4   1.3 1.5 5
;
p=
1.04
1.79
1.89
4.16
2.7
2.56
0.99
1.99
3.14
2.76
\end{verbatim}

\begin{verbatim}
b=
34249
36456
3132.85
23379
36008
44230
42390
38333
28167
51523
;
enddata
min=@sum(t:0.5*p*x);
@sum(t(i):a(i,1)*x)>=659;
@sum(t(i):a(i,2)*x)>=0.11;
@sum(t(i):a(i,3)*x)>=0.16;
@sum(t(i):a(i,4)*x)>=57.7;
@sum(t(i):a(i,5)*x)>=1.3;
@sum(t(i):a(i,6)*x)>=148.1;
@sum(t(i):a(i,7)*x)>=177;
@sum(t(i):a(i,8)*x)>=2.32;
@sum(t(i):a(i,9)*x)>=0.81;
@sum(t(i):a(i,10)*x)>=1.2;
@sum(t(i):a(i,11)*x)>=9.3;
@sum(t:x)<=18.73;
@sum(t:x/b)<=0.1964/365;
\end{verbatim}

程序 4

第四问:
水果:
sets:
t/1..7/:x,p,b;
q/1..11/;
w(t,q):a;
endsets
data:
a=
560 0.2 0.4 30 5 320 4 4 1.7 8.7 12
1000 0.1 0.3 80 14.6 110 9 1 0.1 10 12
2770 0.5 0.4 330 4.5 350 13 2 10 4.5 4
1000 0.3 0.3 40 14.6 30 7 7 1 9.8 31
50 0.5 0.3 40 3.4 110 5 2 0.2 5 4
\end{verbatim}

\begin{verbatim}
330 0.8 0.2 240 2.3 180 8 5 1.4 2.4 13
1800 0.3 0.4 100 1 130 23 2 0.5 0.8 3
;
p=
4.64
5.2
2.66
3.09
7.69
4.5
2.64
;
b=
26942.90674
16527.27323
12865.6011
14550.71488
15190.92645
23874.8792
38206.93
;
enddata
min=@sum(t:0.5*p*x);
@sum(t(i):a(i,1)*x)>=92;
@sum(t(i):a(i,2)*x)>=0.05;
@sum(t(i):a(i,3)*x)>=0.05;
@sum(t(i):a(i,4)*x)>=74;
@sum(t(i):a(i,5)*x)>=0.2;
@sum(t(i):a(i,6)*x)>=20.6;
@sum(t(i):a(i,7)*x)>=15;
@sum(t(i):a(i,8)*x)>=0.54;
@sum(t(i):a(i,9)*x)>=0.24;
@sum(t(i):a(i,10)*x)>=0.25;
@sum(t(i):a(i,11)*x)>=3.2;
@sum(t:x)<=20.03;
@sum(t:x/b)<=0.1422/365;
\end{verbatim}

\end{document}