\documentclass{article}
\usepackage{amsmath}
\usepackage{amssymb}
\usepackage{graphicx}
\usepackage{hyperref}

\title{唐家山堰塞湖泄洪问题的研究}
\author{}
\date{}

\begin{document}

\maketitle

\begin{abstract}
本文以唐家山堰塞湖为研究对象,参考权威报道的数据内容及来源于DEM模型的遥感影像,探讨了堰塞湖演化、溃坝、洪波传播等方面的问题。

针对第一个问题,先依据所采集的实际样本值回归得到库容回归模型,接着根据堰塞湖其形成机理建立库容一般模型,由于条件受限无法获取地质地形数据,因此修正了一般模型。最后依据库容的回归模型和修正模型分别建立以降雨量来预测堰塞湖水位日上升高度的模型,并以此预测5月25日至6月12日的水位日上升高度(见表3.1)。通过对比两个结果发现,库容修正模型较其回归模型的实用性和稳定性更强。

针对第二个问题,在第一问的库容模型的基础上,依据经典的溃坝洪水预报模型——DAMBRK模型思想分别建立了溃口形态模型和水体下泄模型,给出了溃口形态、溃口流量、堰塞湖水位等物理量的数学描述。同时根据实际情况,模拟了堰塞湖漫顶逐渐溃的过程,溃坝历时20584秒,并且给出了溃口形态、溃口流量、堰塞湖水位随时间的变化关系(见图4.3~4.6)。

针对第三个问题,在第二问的溃坝模型的基础上,通过假设漫溢性洪水在瞬时达到稳态,简化经典动力学洪水演进模型,建立结合动力学模型和基于DEM网格模型的洪水淹没模型,并采用种子蔓延算法计算出洪水淹没区域。运用该模型对唐家山堰塞湖出现三分之一蓄水下泄的情形进行了模拟,对洪水淹没区域的过程进行可视化演示(见图5.6),并且计算出洪水到达人口密集区域的时间(见表5.1),最后给出各区域的人员撤离方案。

最后,依据前三个问题所建立的模型,分析在抗震救灾中采取的应对策略,指出其适用性和局限性,并对如何应对地震后次生山地灾害提出一些建设性意见。

关键词:多项式回归 \quad DAMBRK \quad DEM网格模型
\end{abstract}

\begin{center}
\textbf{\Large 全国第五届研究生数学建模竞赛}
\end{center}

\begin{center}
\includegraphics[width=0.4\textwidth]{image.png}
\end{center}

\begin{flushleft}
题目 \hspace{1cm} 唐家山堰塞湖泄洪问题的研究
\end{flushleft}

\section{问题提出}

堰塞湖是一种自然的地貌现象,主要是在一定的地质和地貌条件下,由地震活动、冰碛物或火山熔岩流等原因引起崩塌、滑坡、泥石流等堵截山谷、河谷或河床后贮水而形成的湖泊。其中地震诱发的堰塞湖是地震诱发的次生灾害类型之一。由于地震诱发堰塞湖的坝体——堰塞体由松散土石料经天然快速堆积而成,其结构一般较为松散,基本没有胶结,在冲刷、侵蚀、溶解、崩塌等作用下通常很少能够保持长期稳定。据统计资料表明,堰塞湖形成后,10天内溃决者超过50\%,2个月内溃决者超过60\%,1年内溃决者超过90\%。在河道持续来流的作用下,堰塞湖的水位会迅速上升。一方面会淹没上游两岸致灾;另一方面,松散堰塞体在湖水漫溢冲刷或高水位渗透破坏作用下发生溃决,倾泻的洪水将迅速冲毁淹没下游两岸的房屋、田地、基础设施等,其破坏性不亚于地震,危险性极大。

2008年5月12日,四川汶川地区发生了8.0级强烈地震,给人民生命财产和国民经济造成了极大的损失。地震引发的次生灾害也是相当严重的,特别是地震的造地运动形成了34处主要堰塞湖,超过半数处于中危以上,对下游河道两侧数以百万计人民的生命及其财产和国家建设构成了巨大的威胁,其中以唐家山堰塞湖尤为重要。

该堰塞湖是由涪江支流湔江(又称通口河)右岸山体大规模滑坡堵江而成,距北川县城上游约6公里。坝体在平面上呈长条形,顺河长约803米,横河宽最大约611米,坝高82~124米,坝顶面宽约300米,体积约2037万立方米。坝体地形起伏较大,横河方向左侧高右侧低,左侧最高点高程793.9米,右侧最高点高程775.0米。唐家山堰塞湖库容约3亿立方米,日入湖水量约700万立方米,水位每天上涨1米左右。从最终实际情况看,从坝顶溢出而溃坝的可能性比其他原因溃坝的可能性大得多。因此,秉承加强对震后次生灾害规律的研究以提高国家抗震救灾能力的原则,我们以唐家山堰塞湖的资料为参考,着重研究堰塞湖及其泄洪规律,需要完成工作如下:

\begin{enumerate}
    \item 建立唐家山堰塞湖以水位高程为自变量的蓄水量的数学模型。并以该地区天气预报的降水情况的50\%、80\%、100\%、150\%为实际降雨量建立模型预计自5月25日起至6月12日堰塞湖水位每日上升的高度。
    \item 利用此次唐家山堰塞湖泄洪时科技人员记录下的大量宝贵数据,尝试在合理的假设下建立堰塞湖蓄水漫顶后再水流作用下发生溃坝额数学模型。
    \item 根据数字地图给出坝体发生溃塌,造成堰塞湖内1/3的蓄水突然下泻时的洪水水流速度及淹没区域(包括洪水到达各地的时间),并在此基础上考虑洪水淹没区域中人口密集的人员撤离方案。
    \item 根据所建的数学模型分析当时所采取对策的正确性和改进的可能性。并对科技工作中在应对地震次生山地灾害(不限堰塞湖)应该设法解决的关键问题提出有关建议。
\end{enumerate}

\section{模型假设和符号说明}

\subsection{模型假设}

\begin{enumerate}
    \item 在此期间内唐家山堰塞湖无余震影响;
    \item 该地区不同天气降雨量服从均匀分布;
    \item 本文所用的高程物理量为绝对高程,即以绝对基面——0海拔海平面为零点测算出来的,故在本文中高程就是海拔高度;
    \item 堰塞湖整个湖面在同一水平面上,没有落差;
    \item 在用降雨量预测水位上升模型中,泄洪的影响忽略不计;
    \item 天气预报给出分白天和黑夜的降雨量是以12小时计量的;
    \item 调整的降雨量全部汇集于堰塞湖;
    \item 河流上游流量稳定,没有突然暴涨现象;
    \item 溃坝的过程对河床没有冲刷影响;
    \item 水是均匀连续的从堰塞湖坝顶流出,不受上游径流量、降雨、风力等外界因素影响;
    \item 淹没区域中的水不会通过蒸发,渗入地下或是通过人为排水装置排出;
    \item 房屋、树木等的糙率可以综合等效成一个平均糙率;
    \item 假定洪水能在小段时间片内瞬间达到稳态。
\end{enumerate}

\subsection{符号说明}

\begin{tabular}{|c|c|c|c|c|}
\hline 地名 & 经度 & 纬度 & 距溃坝口 & 到达时间 \\
\hline 溃坝处 & $104.43^{\circ}$ & $31.85^{\circ}$ & $0 \mathrm{~m}$ & 0分钟 \\
\hline 北川 & $104.44^{\circ}$ & $31.89^{\circ}$ & $1648.4 \mathrm{~m}$ & 7分钟 \\
\hline 青莲场 & $104.68^{\circ}$ & $31.60^{\circ}$ & $34103 \mathrm{~m}$ & 146分钟 \\
\hline 江油 & $104.70^{\circ}$ & $31.80^{\circ}$ & $25394 \mathrm{~m}$ & 148分钟 \\
\hline 龙凤镇 & $104.70^{\circ}$ & $31.63^{\circ}$ & $25658 \mathrm{~m}$ & 157分钟 \\
\hline 安县 & $104.41^{\circ}$ & $31.64^{\circ}$ & $22829 \mathrm{~m}$ & 228分钟 \\
\hline 彰明镇 & $104.72^{\circ}$ & $31.72^{\circ}$ & $31831 \mathrm{~m}$ & 260分钟 \\
\hline 绵阳 & $104.73^{\circ}$ & $31.48^{\circ}$ & $48781 \mathrm{~m}$ & 301分钟 \\
\hline
\end{tabular}

注:其他未列出符号,在使用时将详述其代表含义。

\section{问题一}

\subsection{水位高程—蓄水量模型}

水位高程—蓄水量模型主要反映堰塞湖蓄水量 \( V \) 随水位高程 \( h \) 变化的数学模型,其主要目的是用于估算不同水位高程 \( h \) 所对应的蓄水量 \( V \)。

\subsubsection{回归模型}

从资料中搜集的59组关于唐家山堰塞湖水位高程 \( h \) 及其对应蓄水量 \( V \) 数据(见附表3.1)的散点图(见图3.1)可以看出,水位高程与蓄水量呈曲线关系,并且还知道水位高程依蓄水量的关系一般为单调递增的凸函数关系,可以考虑用水位高程的一元多项式来逼近蓄水量 \({}^{[1]}\)。

\begin{figure}[h]
    \centering
    \includegraphics[width=0.8\textwidth]{image.png}
    \caption{散点图及回归曲线}
    \label{fig:scatter_plot}
\end{figure}

通过MATLAB对59组样本点进行多次回归得到回归方程如下:
\begin{align*}
V &= 62215h^2 - 8.5068 \times 10^7h + 2.9107 \times 10^{10} \\
h &= -5.1662 \times 10^{-16}V^2 + 3.7352 \times 10^{-7}V + 683.16
\end{align*}
通过回归分析,调整的 \( R^2 = 0.99420 \),拟合优度相当高,即回归曲线拟合的比较好,见图3.1。

\subsubsection{一般模型}

多次回归得到的模型具有统计意义,但说服力不强。基于稳健地考虑,应该从堰塞湖形成机理及其结构入手对其建模。

对于一个既定的堰塞湖,如唐家山堰塞湖,它的地理结构和形状是固定的,那么湖内壁不规则曲面的形状也应该是既定不变。对于这个形状固定且类似于一个盛水容器的堰塞湖,它的蓄水量将可以由水位高程直接构造数学模型。

地球上的任意点都拥有其唯一地理身份证,即经纬度 $(\alpha, \beta)$ 和海拔 $(e)$。这样,可以在堰塞湖这个区域 $W$ 建立一个三维空间坐标系,将这个区域的每一点的地理身份证换算成已建立三维空间坐标系中的 $X$、$Y$、$Z$ 坐标,如下:

\begin{align*}
\tau &= 2712000(s); \\
b &= 462.3612(m);
\end{align*}

其中,为了方便处理,选取竖直方向为 $Z$ 方向,且以 0 海拔平面作为 $z = 0$ 平面。

由于既定的堰塞湖的内壁曲面是既定的,假设这个曲面为 $z = s(x, y)$,那么函数 $s(x, y)$ 应该是既定,虽然目前未知。堰塞湖类似一个盛水容器,要求出一个既定水位高程 $h$ 所对应的蓄水量 $V$,需求解出平面 $z = h$ 与内壁曲面 $z = s(x, y)$ 形成封闭空间的体积。定义区域 $D = \{(x, y) \mid s(x, y) \leq H\}$,且 $D \subset W$,那么可以通过如下求出给定堰塞湖既定水位高程 $h$ 所对应的蓄水量 $V$:

\begin{equation}
V = \iint_{(x, y) \in D} (h - s(x, y)) \, dx \, dy
\tag{3-3}
\end{equation}

综上所述,水位高程—蓄水量模型如下:

\begin{equation}
V = \iint_{(x, y) \in D} (h - s(x, y)) \, dx \, dy
\end{equation}

其中:
\begin{equation}
\begin{cases}
x = f(\alpha, \beta) \\
y = g(\alpha, \beta) \\
D = \{(x, y) \mid s(x, y) \leq h\} \\
D \subset W
\end{cases}
\end{equation}

对于既定堰塞湖,该模型求解的关键是要获取该堰塞湖的地理数据,即堰塞湖所在区域任意点的地理身份证:经度 $\alpha$、纬度 $\beta$ 以及海拔 $e$。但是,目前这些官方数据难以获得。

\subsection{修正模型}

\label{subsec:修正模型}

\begin{figure}[h]
\centering
\includegraphics[width=0.8\textwidth]{image.png}
\caption{内壁示意图}
\end{figure}

要精确确定堰塞湖蓄水量和水位高程—蓄水量模型,需要运用多时向(包括水位较低,特别是形成该堰塞湖前河道干涸时)遥感影像。换句话说,3DEM 不能采集到被水淹没地方的高程,除非给出河道干涸时的遥感影像。因此,在现有条件下,无法获得堰塞湖内壁曲面的资料,也就是说利用基于 DEM 的遥感影像

得到的 \( z = s(x, y) \) 并不是干涸时堰塞湖的内壁曲面方程。因此,有必要修正前面的水位高程—蓄水量模型。

由于给出的遥感影像无法判断堰塞湖当时已蓄水情况,前面的模型只能算出在此基础上水位高程上升所导致蓄水量的增加量。因此,只需将前面模型加上一个常数项即可得到较精确的水位高程—蓄水量模型。

假设给出的遥感影像中唐家山堰塞湖当时蓄水量为 \( V_0 \);并且通过3DEM只能获得如图3.2双线所示的内壁曲面资料,因此,得到的修正后的水位高程—蓄水量模型如下:
\[
V = V_0 + \iint_{(x, y) \in D} (h - s(x, y)) \, dx dy
\]
其中:
\[
\begin{cases}
x = f(\alpha, \beta) \\
y = g(\alpha, \beta) \\
D = \{(x, y) \mid s(x, y) \leq h\} \\
D \subset W
\end{cases}
\]
其中,\( V_0 \) 可以通过采集实际样本数据计算得出。

\section{降雨量—水位上升模型}

降雨量—水位上升模型主要反映堰塞湖水位随降雨量上升的高度,主要目的是根据当地降水量来预测堰塞湖水位每日上升的高度。

要弄清楚降水量与水位上升之间关系,首先必须要清楚蓄水量与水位之间的关系,也即要明确水位高程—蓄水量模型。前面已经给出了三个水位高程—蓄水量模型,其中有两个模型在现有条件下可以求解出来。因此,将以水位高程—蓄水量的一元多次回归模型和修正模型为基础,分别建立降水量—水位上升模型。

\subsection{基于水位高程—蓄水量回归模型建立的水位上升模型}

假设该地区天气预报的降雨量为 \( l \)(一般以毫米为单位,这里转化以米为单位);该地区的面积为 \( S \)(单位为平方米);调节因子为 \( \lambda \),以表示将天气预报的降雨量调节成汇集到堰塞湖的实际降雨量的乘数。为了简化问题,可以利用此调节因子,将上游的水折算成该地区的降雨量,所以本文不涉及到上游流下来水量的计算。

可以计算出实际降雨量,也即堰塞湖蓄水量的增量 \( \Delta V \):
\[
\Delta V = \lambda \cdot l S \tag{3-4}
\]
根据水位高程—蓄水量的一元多次回归模型,也就是方程(3-1),加入增量变化,即:
\[
V + \Delta V = 62215(h + \Delta h)^2 - 8.5068 \times 10^7(h + \Delta h) + 2.9107 \times 10^{10} \tag{3-5}
\]
将方程(3-5)—方程(3-1)化简,并代入方程(3-4)得:
\[
\lambda \cdot l S = 62215 \Delta h(2h + \Delta h) - 8.5068 \times 10^7 \Delta h \tag{3-6}
\]

根据方程(3-6),由于其他变量均为常量,在已知前一期水位高程 $H$ 及当期天气预报降雨量 $l$ 情况下,就可以求解关于 $\Delta V$ 的一元二次方程。由于依据方程(3-6),二根之积为负,即二根异号,结果取正根即可。

\subsection{基于水位高程—蓄水量修正模型建立的水位上升模型}

假设同上,根据水位高程—蓄水量的修正模型,加入增量变化,即:
\begin{equation}
V + \Delta V = V_0 + \iint_{(x,y) \in D_1} (h + \Delta h - s(x,y)) \, dx dy
\tag{3-7}
\end{equation}
其中,$D_1 = \{(x,y) \big| s(x,y) \leq h + \Delta h\}$,$\Delta h$ 是水位上升高度。

经过化简得:
\begin{equation}
\lambda \cdot l S = \iint_{(x,y) \in D_1} (h + \Delta h - s(x,y)) \, dx dy - \iint_{(x,y) \in D} (h - s(x,y)) \, dx dy
\tag{3-8}
\end{equation}
方程(3-8)比较复杂,手工无法求解出来,可以考虑利用计算机搜索方法。

\section{模型求解}

\subsection{修正模型的求解}

地理信息系统(Geographic Information System,简称 GIS)作为获取、处理、管理和分析地理空间数据的重要工具、技术和学科,近年来得到了广泛关注和迅猛发展 \cite{ref2}。其中,特别是一些技术的成熟,如免费开放的Google Earth 是一款Google公司开发的虚拟地球软件,它把卫星照片、航空照片和GIS布置在一个地球的三维模型上,可以从中获取任意地方的地形图,另外3DEM软件可以查看三维地形图的任意点的坐标数据。以上工具与技术为求解唐家山堰塞湖水位高程—蓄水量修正模型中变动部分 $V_1 = \iint_{(x,y) \in D} (h - s(x,y)) \, dx dy$ 提供了一定的硬件条件,但是无法从给出遥感影像中获知唐家山堰塞湖当时的已蓄水量 $V_0$。

“山穷水尽疑无路,柳暗花明又一村”,已搜集到的59组水位高程 $H$ 及其对应蓄水量 $V$ 样本数据(见附表3.1)可以结合变动部分计算出常数 $V_0$。为此,我们就模型求解分两步走:

(1)变动部分

首先,在Google Earth上确定唐家山堰塞湖的大致位置。如图3.3,我们将唐家山堰塞湖区域限定在N31°51′38.39″~N31°51′57.66″, E104°21′18.79″~N104°25′48.90″内。

\begin{figure}[h]
    \centering
    \includegraphics[width=\textwidth]{image1.png}
    \caption{Google Earth上唐家山堰塞湖地形图}
    \label{fig:google_earth}
\end{figure}

接着,通过3DEM软件查看选定区域三维地形图(如图\ref{fig:3dem})中任意点的坐标数据。为了尽量还原地理原貌,我们先将选定区域划分91×52个方格,再将每个小方格的长和宽分别以 $\frac{1}{20}$ 步长获取样本点。这样可以得到91×52×20×20=1892800个样本点。

\begin{figure}[h]
    \centering
    \includegraphics[width=\textwidth]{image2.png}
    \caption{3DEM上截取唐家山堰塞湖的地形图}
    \label{fig:3dem}
\end{figure}

最后,在MATLAB软件中对上面获取的样本点进行插值拟合盛水曲面,如图\ref{fig:matlab}所示。

\begin{figure}[h]
    \centering
    \includegraphics[width=\textwidth]{image3.png}
    \caption{MATLAB中插值拟合的盛水曲面}
    \label{fig:matlab}
\end{figure}

\begin{figure}[h]
    \centering
    \includegraphics[width=\textwidth]{image1.png}
    \caption{MATLAB上唐家山堰塞湖区域的拟合图}
    \label{fig:3.5}
\end{figure}

在708米至743米的高程范围内,以0.01米的步长,共3500个平行于 $z=0$ 平面的平面分别去截这个盛水曲面,同时计算出各自 $V_{1} = \iint_{(x,y)\in D} (h-s(x,y)) \, dx \, dy$,再对这些点进行拟合曲线,如图\ref{fig:3.6}所示。

\begin{figure}[h]
    \centering
    \includegraphics[width=\textwidth]{image2.png}
    \caption{变动部分 $V_{1}$ 的曲线与样本真实值}
    \label{fig:3.6}
\end{figure}

\subsubsection{固定部分}

根据上面方法求解出采集的59组样本点数据的配对变动部分 $V_{1}$,再结合 $V = V_{0} + V_{1}$ 可求出 $V_{0} = 0.4 \times 10^{8}$ 立方米。

综上所述,水位高程—蓄水量修正模型基本可以确定。

\subsection{降雨量—水位上升模型的求解}

唐家山堰塞湖位于距北川县城约6公里处,在北川县管辖范围内,所以计算唐家山堰塞湖地区降水量是以整个北川国土面积为计算面积。经资料查阅,北川县位于四川盆地西北部。地理坐标:北纬$31^{\circ}14' \sim 32^{\circ}14'$,东经$103^{\circ}44' \sim 104^{\circ}42'$。东接江油市,南邻安县,西靠茂县,北抵松潘、平武县,面积2867.83平方公里(国土详细面积2869.18平方公里)[4]。本文中$S$取$2867.83 \times 10^6$平方米。

通过官方网站搜集的天气预报资料见附表3.2。已假设降雨量服从均匀分布,那么下雨量$l$均取给定范围的均值,如图3.7。

\begin{figure}[h]
\centering
\includegraphics[width=\textwidth]{image.png}
\caption{5月25日—6月12日日降水量}
\end{figure}

要求出某一天因降雨量等原因所引起水位高程的增加,还必须知道前一天的水位高程,本文要求用降雨量预计出5月25日至6月12日堰塞湖每日上升的高度,那么就必须知道5月24日的水位高程,经查阅资料,取起始时间水位高程$h_0 = 717.92$米[5]。

另外,根据资料,调节因子$\lambda$分别取50\%、80\%、100\%和150\%。以上初始条件基本给出,再根据不同建模方法分别求解。

\subsubsection{基于一元多次回归模型的手动方法}

对于基于水位高程—蓄水量的一元多次回归模型得到的方程(3-6),在调节因子分别取50\%、80\%、100\%和150\%四种情况下,可以通过求根公式求出由降雨量引起每一天堰塞湖水位上升高度,见附表3.3。

但是,在这种方法求得的结果中,我们发现一些异常现象。按道理,对于同一天,调节因子取值越大,也相当于实际降雨量越大,从而$\Delta V$应该越大,那么相应的$\Delta H$也应该越大。当调节因子$\lambda$从50\%增至80\%,100\%,结果中的$\Delta h$基本呈增加趋势。但是调节因子$\lambda$从100\%增至150\%时,初始第一天(即5月25日)的

\begin{tabular}{|c|c|}
\hline 时间 & 政策 \\
\hline 5月25日前 & 挖明渠,降低水位,减小下游风险 \\
\hline 5月25日后 & 紧急调派,爆破挖渠相结合 \\
\hline 5月26日后 & 挖爆结合、先挖后爆、平挖深爆、以爆助挖 \\
\hline
\end{tabular}

\begin{figure}[h]
    \centering
    \includegraphics[width=\textwidth]{image1.png}
    \caption{不同调节因子情况下水位日上升高度}
    \label{fig:3.8}
\end{figure}

\begin{figure}[h]
    \centering
    \includegraphics[width=\textwidth]{image2.png}
    \caption{不同调节因子情况下水位上升高度}
    \label{fig:3.9}
\end{figure}

\section{模型评述}

水位高程—蓄水量的一元多次回归模型是通过采集的样本数据进行多次回归得到,虽然拟合优度比较高,操作比较简单,但是只是具备统计上的意义,实用性不强。在后面用来建立降雨量—水位上升模型求解唐家山堰塞湖水位高程日上升高度时,出现了与理论相悖的问题,这说明此模型实用性不强,不具备推广性。

水位高程—蓄水量的一般模型和修正模型是基于堰塞湖的机理而建的,修正模型只是在一般模型的基础上加上一个常数,可以说实用性较强,也很具推广性。基于唐家山堰塞湖这个例子,如图3.10所示,空心点代表的是在现有条件下利用一般模型(修正模型的变动部分)求解的结果所画的散点图,与所采集的实际样本数据(用小方块代表)的散点图的变化趋势基本相同,类似平移而成。这说明这两个模型比较可靠,基本符合事实。至于两个点集之间存在距离,可以有两点解释:第一,现有条件下(3DEM)不能采集到唐家山堰塞湖的湖内壁的高程,也即不能采集到水下的地质海拔,所以用MATLAB拟合的盛水曲面不是真实的堰塞湖内壁曲面,换句话说,采集数据的遥感影像图中的唐家山堰塞湖当时已有一定的蓄水量;第二,对于唐家山堰塞湖位置界定的不同,本文的界定估计与官方的界定有一定出入,这也造成模型计算数据与实际样本值呈整体偏移。为此,我们对模型进行了修正,建立了水位高程—蓄水量的修正模型。修正后模型拟合曲线与实际样本值有一点差别,但是基本在误差范围之内。总体上来讲,这两个模型实用性比较强,也比较稳定。

\begin{figure}[h]
    \centering
    \includegraphics[width=\textwidth]{image.png}
    \caption{模型计算数据与实际样本数据的对比}
    \label{fig:3.10}
\end{figure}

\section{问题二}

\subsection{模型建立}

堰塞体溃坝过程是水力学、泥沙动力学、土力学、和边坡稳定理论等的综合过程,是水、土二相介质相互作用的过程。堰塞湖溃坝过程是堰塞湖水体通过坝体溃口泄出的水流,将通过冲刷和坍塌导致溃口的产生和扩大,这一过程将持续到堰塞湖放空或者坝体能抵挡住水流的进一步冲刷为止 \cite{ref6}。

对堰塞湖溃坝发生的预测和模拟对抢险救灾具有重大的意义,但精确的模型往往需要精确的数据作支撑,否则反而效果很差。地震发生后,由于自然环境、时间限制等条件约束,要取得各项模型参数往往相对比较困难,因此所建立的预测模型所需参数越少、越粗略越好。美国国家气象局编制的溃坝洪水预报模型 DAMBRK 模型模拟方法所需参数较少,应用较为简单方便,同时精度也满足一般预测需求,在许多工程中得到了应用,并取得了良好的效果。本文以 DAMBRK 模型为基础,引入水流、水位对溃坝的反馈作用,建立基于自反馈的 DAMBRK 模型。该模型由两部分组成:

\begin{enumerate}
    \item 大坝溃口形态描述。用于确定大坝溃口形态随时间的变化,包括溃口底宽、溃口顶宽、溃口边坡及溃决历时。
    \item 水体下泄流量的计算 \cite{ref7}。
\end{enumerate}

\subsubsection{溃口形态模型}

溃口是坝体失事时形成的缺口。溃口的形态主要与坝型和坝筑材料有关。目前,对于实际溃坝机理仍不是很清楚,因此,溃口形态主要通过实际经验通过近似假设来确定。考虑到模型的直观性、通用性和适应性,一般假设溃口底宽从一点开始,在溃决历日内不断扩大,其中土石坝一般假定为类梯形溃口 \cite{ref8}。

在溃口从一点演化为类梯形缺口的过程中,其形态描述主要由四个参数确定:溃决历时 ($\tau$),溃口底部高程 ($h_{b}$),溃口边坡 ($z$) 和溃口宽度 ($b$)。溃口剖面图如图 4.1 所示。

最终溃口底宽 $b$ 与溃口平均宽度 $\overline{b}$ 的关系如图 4.1 所示:
\begin{equation}
b = \overline{b} - z h_{k}
\tag{4-1}
\end{equation}
式 (4-1) 中 $h_{k}$ 为坝高。形状参数 $z$ 用来定义溃口的边坡,为垂直方向: 水平方向 $= 1: z$,一般 $z$ 的大小与筑坝材料有关,取值范围为 $0 \leq z \leq 2$,$z = \tan(45^\circ + \varphi/2)$,式中 $\varphi$ 为筑坝材料内摩擦角。堰塞体结构为土石,根据资料,取 $\varphi = 18^\circ$。土石坝中,溃口平均宽度 $\overline{b}$ 一般为坝体长度 $B_{d}$ 的二分之一。

\begin{figure}[h]
    \centering
    \includegraphics[width=\textwidth]{image.png}
    \caption{溃口剖面示意图}
    \label{fig:4.1}
\end{figure}

DAMBRK模型中,失事历时 $\tau$ 估算公式为:
\begin{equation}
\tau = 0.8 \frac{\sqrt{V}}{h_k}
\tag{4-2}
\end{equation}
其中,$V$ 为初始蓄水量;$h_k$ 为坝高。

堰塞体和一般的大坝相比,坝体更厚,因而溃坝形式为没顶冲刷溃,而不是管流溃或者瞬时溃。因此失事历时更为长久,这里修正失事历时公式为:
\begin{equation}
\tau = 0.8 \sqrt{\frac{V}{h_k}}
\tag{4-3}
\end{equation}

在堰塞体的冲刷溃决中,溃口底宽从坝顶某一点开始,然后在溃决历时范围 $\tau$ 内,以线性或非线性增长。直到时刻 $\tau$ 为止,形成最终的溃口底宽 $b$ 和最终的溃口底高程 $h_{bm}$。

DAMBRK模型中,溃口底高程 $h_b$ 是溃决时间 $t_b$ 的函数,由式4-4表示为:
\begin{equation}
h_b = h_d - (h_d - h_{bm}) \left( \frac{t_b}{\tau} \right)^{\rho_0}
\tag{4-4}
\end{equation}
$h_{bm}$ 是最终的溃口底部高程,可根据具体情况而定,通常可取水库或泄水河道段底部高程;$t_b$ 为溃口形成开始经历的时间;$\tau$ 为溃坝的总历时;$\rho_0$ 是描述溃口非线性程度的参数,取值范围为 $1 \leq \rho_0 \leq 4$。若假设为线性速度变化,则有 $\rho_0 = 1$。

溃口瞬时底宽则由下式确定:
\begin{equation}
b_i = b \left( \frac{t_b}{\tau} \right)^{\rho_0} \quad 1 \leq t_b \leq \tau
\tag{4-5}
\end{equation}

溃口的底部高程,DAMBRK模型没有考虑到溃口水流和水位的影响作用。为了更好的应用于堰塞湖的溃决模拟,这里修正溃口底部高程公式为:
\begin{equation}
h_b = h_d - (h_d - h_{bm}) \left( 1 - \frac{h - h_b}{h_k} \right) \left( \frac{t_b}{\tau} \right)^{\rho_0}
\tag{4-6}
\end{equation}
根据类梯形溃口的几何关系,易得到溃口的上底,即水面上缺口高度为:

\begin{equation}
b' = \frac{b_i}{z} = b \left( \frac{t_b}{\tau} \right)^{\rho_0}
\end{equation}

缺口高度为:
\begin{equation}
h = \frac{\sqrt{z^2 - 1}}{z} b_i = \frac{\sqrt{z^2 - 1}}{z} b \left( \frac{t_b}{\tau} \right)^{\rho_0}
\end{equation}

\subsection{水体下泄模型}

堰塞湖下泄流速由三部分组成,一是通过溃口下泄流速 $q_b$,二是溢流坝过流量 $q_o$,三是上游水流流量 $q_i$。

\subsubsection{溃口下泄流速 $q_b$}

溃口流速应用宽顶堰流公式 (Fread, 1977) 计算:
\begin{equation}
q = c_v k_s \left[ 3.1 b_i (h - h_b)^{1.5} + 2.45 z (h - h_b)^{2.5} \right]
\tag{4-7}
\end{equation}

式 (4-7) 中,$c_v$ 是对行近流速的修正,$b_i$ 是溃口瞬时底宽(米);$h$ 是水库计算水位高程(米);$h_b$ 是溃口的底部高程(米);$z$ 为溃口边坡;$k_s$ 是考虑尾水影响出流的淹没修正系数 (Venard, 1959)。

行近流速修正系数 $c_v$,由式 4-8 计算:
\begin{equation}
c_v = 1.0 + 0.023 \frac{q_b^2}{B_d^2 (h - h_{bm}) (h - h_b)}
\tag{4-8}
\end{equation}
其中,$B_d$ 为堰塞体坝宽(米),$h_{bm}$ 为最终溃口底高程(米)。

尾水影响出流淹没修正系数 $k_s$,由式 4-9 计算:
\begin{equation}
k_s =
\begin{cases}
1.0 - 27.8 \left[ \frac{h_t - h_b}{h - h_b} - 0.67 \right]^3 & \frac{h_i - h_b}{h - h_b} \geq 0.67 \\
1.0 & \text{其他}
\end{cases}
\tag{4-9}
\end{equation}
其中,$h_t$ 为尾水位高程(米)。

\subsubsection{溢流坝过流流速 $q_o$}

堰塞湖溢流坝过流量 $q_o$ 可以由式 4-10 计算得到:
\begin{equation}
q_o = m B_d (h - h_d)^{1.5}
\tag{4-10}
\end{equation}
式中,$B_d$ 为堰塞体宽度(米);$m$ 为流量系数;$z$ 为堰塞湖水位高程(米);$h_b$ 为堰塞体坝顶高程(米)。

\subsubsection{上游水流流量 $q_i$}

根据新闻报道计算得到上游河水流速为 83.33(立方米每秒)。

\section{模型求解}

假定唐家山堰塞湖是由于降水和上游流水不断涌入湖内,导致水位不断上涨,直至漫顶而发生溃决的,并且溃决时间超过 1 分钟,所以定义溃坝类型为漫决。

顶逐渐溃。

另外从新闻资料中获知堰塞湖堰顶高程为752米,由前述蓄水量模型得到总蓄水量为$3.19 \times 10^{8}$立方米,这即为唐家山堰塞湖溃坝前状态。而且,在溃坝模拟仿真中,需要蓄水量$V$和水位高程$h$之间的关系,同样根据前述蓄水量模型,通过回归拟合,得到两者间关系的函数关系表达式如下:
\begin{align*}
V &= 62215h^2 - 8.5068 \times 10^7h + 2.9107 \times 10^{10} \\
h &= -5.1662 \times 10^{-16}V^2 + 3.7352 \times 10^{-7}V + 683.16
\end{align*}

基于唐家山堰塞湖以上情况,利用前面溃口状态模型和水体下泄模型对其溃坝过程进行模拟。溃坝过程的计算机模拟仿真算法流程图如图4.2所示。主要思路为按照修正后的DAMBRK模型计算相关参数,以一定时间步长进行模拟仿真,同时记录模拟仿真过程中相关物理量,便于对结果进行分析。

\subsection{溃决时间}

查阅附件材料,可知堰塞体坝高$h_k$为100米,则坝底高程$h_{bm} = h_d - h_k = 650$米。本文假定溃口宽度变化为非线性,设$\rho_0 = 2.2$。$\tau$和$b$根据模型公式,计算得:
\begin{align*}
\tau &= 2712000(s); \\
b &= 462.3612(m);
\end{align*}

对堰塞湖而言,逐渐溃的溃口始于堰塞体上一点,溃口流量增量起初较小,随着溃口的扩大,流量增量不断增大,随着时间的推移,堰塞湖水位高程不断下降,溃口流量逐渐减小,最终达到不再变化。此时所经历时间为溃决时间,此后仍然有水流出,但溃口不再变化,故溃决时间小于失事历时。

从溃口宽度、溃口底部高程、溃决流量、水位变化四个指标的变化来看,也可以得出堰塞体溃决所经历突变过程和平缓过程,即经历溃口形成的相变过程和渐变的缓和过程。本文以1秒为步长,对模型进行仿真实验,溃决过程一共历时20584秒,堰塞湖水位高程和溃口底部高程两者变化关系图如图4-2所示。

\begin{figure}[h]
\centering
\includegraphics[width=0.8\textwidth]{image.png}
\caption{堰塞湖水位和溃口底部高程两者变化关系图}
\end{figure}

\subsection{溃口宽度及底高程}

一旦发生溃坝,溃口的宽度主要是由水流的冲刷作用而形成,只要有水流过,溃决的宽度就会不断扩大,直至到达最终的溃口状态为止。因此,DAMBAR模型将溃口宽度设定为只以时间为自变量的函数。在对唐家山堰塞湖溃坝模拟预测中,本文并没有对其进行修改,应用计算机模拟得到溃口宽度变化如图4.4所示。

溃决发生过程中,溃口深度的变化一方面和溃坝历时有关,另一方面也和流速及水位高程相关。为保证模型的简易性和普适性,DAMBRK模型并没有将流速和水位高程考虑进来。本文对其进行了修改,考虑了水位对其的影响。由于溃口水体流速的变化会直接导致水位的变化,所以本文的溃口底部函数综合了时间、水位、水速的影响。用计算机进行模拟得到溃口底部高程变化如图4.5所示。

\begin{figure}[h]
    \centering
    \includegraphics[width=0.8\textwidth]{image.png} % 替换为实际图片路径
    \caption{溃口模型计算机模拟仿真算法流程图}
    \label{fig:4.3}
\end{figure}

\begin{figure}[h]
    \centering
    \includegraphics[width=\textwidth]{image1.png}
    \caption{溃口宽度变化曲线}
    \label{fig:4.4}
\end{figure}

\begin{figure}[h]
    \centering
    \includegraphics[width=\textwidth]{image2.png}
    \caption{溃口底部高程变化曲线}
    \label{fig:4.5}
\end{figure}

\subsection{溃决流量}

溃坝流量不仅受大坝溃口形态的影响,而且受坝址地形影响。即使是大坝全溃,其溃口尺寸也要结合坝址地形确定。本文对溃决流量的研究未考虑坝址地形的影响,仅由堰流流量计算公式计算得到,其变化如图\ref{fig:4.6}所示。

分析溃口处流量随时间变化过程可以看出,由于溃口不断扩大,出库流量急剧增加,同时下游水位不断升高,对出口流量形成顶托,抑制流量继续增加,当两种作用平衡时流量达到最大。此时溃口也达最大,而后流量由最大逐渐减小,由于下游水位顶托,流量衰减相应减慢,洪水波形成陡涨缓落型波形。

\begin{figure}[h]
    \centering
    \includegraphics[width=\textwidth]{image1.png}
    \caption{溃口水流速度变化曲线}
    \label{fig:4.6}
\end{figure}

\subsection{水位变化}

溃坝模型的研究时间从堰塞湖水位高程到达堰顶控制高程开始,即水位高程到达752m。随着溃口的形成,首先是一个短暂的准备器,在这小段时间内溃口流量并不大,水位变化不明显,甚至由于上游水量的加入,水位还会继续升高;当溃口形成一定规模,堰塞湖的水体将迅速涌出,造成堰塞湖水位的急速下降,直到到达堰塞湖水位和堰塞体水坝的平衡状态。需要注意的是此时由于上流来水,堰塞体已经冲开,因此河水和一直流出,但水位相当平缓。堰塞湖水位随时间的变化如图\ref{fig:4.7}所示。

\begin{figure}[h]
    \centering
    \includegraphics[width=\textwidth]{image2.png}
    \caption{堰塞湖溃决后水位变化曲线图}
    \label{fig:4.7}
\end{figure}

\subsection{模型优缺点}

\textbf{优点:}
\begin{enumerate}
    \item 溃口变化函数考虑了水流和水位的影响;
    \item 提出了堰塞湖红薯调度与溃口流量边界计算相关联的计算方法;
    \item 模型所需参数较少, 计算较为简洁, 适用于地震后条件参数不详尽的环境.
\end{enumerate}

\textbf{缺点:}
\begin{enumerate}
    \item 精度略差;
    \item 最终溃口的状况的估计存在一定的误差;
    \item 溃口形状的固定演化可能和实际情况有一定差距;
    \item 未考虑坝体土石间的相互结构;
    \item 未考虑溃口水流对河床冲刷的作用。
\end{enumerate}

\section{问题三}

洪水淹没是一个很复杂的过程, 受多种因素的影响, 其中洪水特性和受淹区的地形地貌是影响洪水淹没的主要因素。对于一个特定防洪区域而言, 洪水淹没可能有两种形式, 一种是漫堤式淹没, 即堤防并没有溃决, 而是由于河流中洪水水位过高, 超过堤防的高程, 洪水漫过堤顶进入淹没区; 另一种是决堤式淹没, 即堤防溃决, 洪水从堤防决口处流入淹没区。无论是漫堤式淹没还是决堤式淹没, 洪水的淹没都是一个动态的、变化的过程。

本文中讨论的是漫堤式淹没, 即洪水位高于堤防或大坝, 水流漫溢、淹没低平的三角洲平原或山前的一些冲积、洪积扇区的现象。洪水淹没最终的状态应该是水位达到平衡状态, 此时的淹没区就是最终的淹没区。本文建立模型的目的是在给定时间下确定淹没面积和水位高程。

问题三是假设堰塞湖内 $1 / 3$ 的蓄水突然下泻, 估计洪水到达各地的时间与淹水范围, 作为人员撤离方案的依据。洪水的下泻模型使用第二问的结论, 本章重点建立洪水淹没模型, 依据模型计算结果, 给出可行的人员撤离方案。

\subsection{模型建立}

在已有的基于水动力学洪水模型中, 较经典的是由二维非恒定流圣维南方程组成的洪水模拟模型 $^{[10]}$ 。它由描述水流运动的水流连续性方程和水流沿 $X$ 方向的动量方程及水流沿 $Y$ 方向的动量方程所组成。

\begin{equation}
\frac{\partial z}{\partial t}+\frac{\partial(u h)}{\partial x}+\frac{\partial(v h)}{\partial y}=0
\end{equation}

\begin{equation}
\frac{\partial u}{\partial t}+u \frac{\partial u}{\partial x}+v \frac{\partial u}{\partial y}+g \frac{\partial z}{\partial x}+g \frac{n^{2} u \sqrt{u^{2}+v^{2}}}{h^{4 / 3}}=0
\end{equation}

\begin{equation}
\frac{\partial v}{\partial t}+u \frac{\partial v}{\partial x}+v \frac{\partial v}{\partial y}+g \frac{\partial z}{\partial y}+g \frac{n^{2} v \sqrt{u^{2}+v^{2}}}{h^{4 / 3}}=0
\end{equation}

上式中 $t$ 表示时间, 单位为秒; $x$、$y$ 表示直角坐标系的横、纵坐标, 单位为米; $u$、$v$ 分别为 $x$、$y$ 方向的流速分量, 单位为米/秒; $z$、$h$ 分别为 $(x, y)$ 处的水位与水深, 单位为米; $g \frac{n^{2} u \sqrt{u^{2}+v^{2}}}{h^{4 / 3}}$, $g \frac{n^{2} v \sqrt{u^{2}+v^{2}}}{h^{4 / 3}}$ 分别为 $x$、$y$ 方向的水流运动阻力, 其中 $n$ 为曼宁糙率系数。

由于二维非恒定流圣维南方程计算较为复杂, 这里我们只对洪水淹没的面积随时间的变化规律感兴趣, 依据二维洪水模拟模型的思想, 我们通过一系列简化, 建立一维洪水水力学演算模型。它包括总水量平衡方程和淹没方程。

总水量平衡方程: $Q=W$, 其中 $Q$ 是从堰塞湖漫溢出的水量, $W$ 是淹没地区的水量。从而, 淹没方程:

\begin{equation}
W=\iint_{(x, y) \in D}(H-h(x, y)) d x d y-
\end{equation}

其中, $D$ 是淹没面积中点的集合, $H$ 为坝下水位高程。

由问题二溃坝模型可以得到堰塞湖漫溢流量与时间的函数 $q(t)$。但由于水流 出到淹没区域需要一定时间,洪水淹没地区时受到糙率的影响(例如树木、房屋),也减慢了淹没速度。

$q(t)$ 表示堰塞湖漫溢出速度,从第二问的溃坝模型得到。我们将等效糙率 \cite{ref11} 的效果反映到 $q(t)$ 上变成 $\tilde{q}(t)$

\begin{equation}
\tilde{q}(t) = q((1 - \gamma n S)t)
\tag{5-3}
\end{equation}

其中 $n$ 为平均曼宁糙率系数,$\gamma$ 为放缩因子,$S$ 为淹没区域中与堰塞坝最远的距离。由于随着洪水的泛滥,$S$ 会增大很多,所以 $\gamma$ 要足够小使得 $1 - \gamma n S$ 非负。从开始漫溢经历时间 $t$,根据水量平衡方程和淹没方程就得到稳态方程式:

\begin{equation}
\int_{0}^{t} q((1 - \gamma n S)t) dt = \iint_{(x, y) \in D} (H - h(x, y)) \, dx \, dy
\tag{5-4}
\end{equation}

从中可以推出淹没水位高程 $H$ 和淹没面积 $D$ 和时间 $t$ 的函数 $(H, D) = F(t)$

\subsection{模型求解及结果分析}

我们在数字高程模型 (DEM) 网格模型的基础上对洪水淹没模型进行求解。水文模型的求解往往采用有限差分、有限元等数值解法,即把研究区剖分成网格来进行求解。根据网格形状,可将计算网格分为规则网格和不规则网格。对于计算网格单元高程,主要是利用计算区域内的等高线数据,采用合适的内插值方法,计算各个计算网格单元水量的变化。这里采用规则的四方网格。

另外,用数字高程模型 (DEM) 求取给定水位条件下的淹没区,应当区分两种情形。第一种不考虑是否连通的情况,凡是高程值低于给定水位的点,皆计入淹没区;第二种是考虑连通的情况,即洪水只淹没它能流到的地方。这两种情形都具有实际意义。第一种相当于整个地区大面积均匀降水的情形,所有低洼处都可能积水成灾;第二种相当于高发洪水向邻域泛滥,例如洪水决堤,或局部暴雨引起的暴涨洪水向四周扩散。洪水首先是从洪水源处开始向外扩散淹没,只有水位高程达到一定程度之后,洪水才能越过某一地势较高的区域到达另一个洼地。因此,本文考虑连通情况。

洪水淹没的连通性算法可以采用种子蔓延算法,它是一种基于种子空间特征的扩散探测算法。其示意图见图 5.1,核心思想是将给定的点作为一个起始点,在某一平面区域上沿 4 个方向搜索出满足给定条件且具有连通关联分布的点。利用种子蔓延算法计算淹没区,就是按给定水位条件,求取与之连通的点的集合,由这些点确定的连续平面就是我们所要求算的淹没面积。

\begin{figure}[h]
\centering
\includegraphics[width=0.5\textwidth]{seed_spread_diagram.png}
\caption{种子蔓延算法示意图}
\end{figure}

我们依次用下列四步进行模型的求解并给出结果:

\begin{enumerate}
    \item 建立唐家山堰塞湖地区数字高程模型(DEM)

由附件3中提供的3DEM软件提取出唐家山堰塞湖地区数字高程模型 (DEM),用WSG84对其进行采样,采样比例是每88米取一个采样点。(见图5.2)

\begin{figure}[h]
    \centering
    \includegraphics[width=0.6\textwidth]{image1.png}
    \caption{唐家山堰塞湖地区数字高程模型}
    \label{fig:5.2}
\end{figure}

    \item 建立唐家山堰塞湖地区地形网格模型

将提取的采样集合导入Matlab中,建立基于DEM的规则四方网格模型。(见图5.3)

\begin{figure}[h]
    \centering
    \includegraphics[width=0.8\textwidth]{image2.png}
    \caption{唐家山堰塞湖地区地形网格模型}
    \label{fig:5.3}
\end{figure}

    \item 洪水淹没模型和数字高程模型的结合

问题三是假设堰塞湖内1/3的蓄水突然下泻,洪水的速度函数 $q(t)$ 用第二问结果拟合为:
\begin{equation}
q(t) = -0.1628 \times \left( \frac{t}{3600} \right)^3 + 2.5686 \times \left( \frac{t}{3600} \right)^2 + 182.8644 \times \left( \frac{t}{3600} \right) - 541.3947
\tag{5-5}
\end{equation}
利用公式(5-4),选取时间微元 $\Delta t$(实验中取一分钟)做数值模拟,边界条件为 $\int_{0}^{t} q((1-\gamma n S) t) d t \leq \frac{V}{3}$。计算出淹水区域面积,坝下水位高程随时间的变化(分别见图5.4和图5.5)。

从图中可知淹水区域面积,坝下水位高程都随时间增加呈递增趋势,淹水区域面积与地形关系密切。初始时,洪水在山地区域流动,河道迂回曲折,并可连通到可以蓄水的湖泊或河湾,所以变化比较平缓。当洪水流向平原时,由于平原地势低,没有阻碍物阻拦,淹水面积会迅速增大。因此,平原区的居民必须对洪灾进行有效防范,一旦洪灾到来,形势非常严峻。

由于采用的DEM模型的分辨率不够高,时间微元选取粒度稍大等问题,使得淹水区域面积与时间的变化规律呈现明显的折线效果。

\begin{figure}[h]
    \centering
    \includegraphics[width=0.45\textwidth]{image1.png}
    \caption{淹水区域面积与时间的关系}
    \label{fig:5.4}
\end{figure}
\begin{figure}[h]
    \centering
    \includegraphics[width=0.45\textwidth]{image2.png}
    \caption{坝下水位高程与时间的关系}
    \label{fig:5.5}
\end{figure}

    \item 洪水淹没区域的过程可视化演示

在实验中,我们对每一分钟都建立一个稳定态,记录它的淹水区域,这样就能动态描绘出洪水淹没区域的过程。图5.6显示了洪水在10分钟、20分钟、100分钟、150分钟、200分钟、301分钟时刻淹水区域的状态。

\begin{figure}[h]
    \centering
    \includegraphics[width=0.3\textwidth]{image3.png}
    \includegraphics[width=0.3\textwidth]{image4.png}
    \includegraphics[width=0.3\textwidth]{image5.png}
    \includegraphics[width=0.3\textwidth]{image6.png}
    \includegraphics[width=0.3\textwidth]{image7.png}
    \includegraphics[width=0.3\textwidth]{image8.png}
    \caption{洪水淹没过程(左上、中上、右上、左下、中下、右下分别表示 10 分钟、20 分钟、100 分钟、150 分钟、200 分钟、301 分钟洪水淹没区域)}
    \label{fig:5.6}
\end{figure}
\end{enumerate}

\section{表5.1 唐家山堰塞湖下游部分城镇经纬度}

\begin{tabular}{|c|c|c|c|c|}
\hline 地名 & 经度 & 纬度 & 距溃坝口 & 到达时间 \\
\hline 溃坝处 & $104.43^{\circ}$ & $31.85^{\circ}$ & $0 \mathrm{~m}$ & 0分钟 \\
\hline 北川 & $104.44^{\circ}$ & $31.89^{\circ}$ & $1648.4 \mathrm{~m}$ & 7分钟 \\
\hline 青莲场 & $104.68^{\circ}$ & $31.60^{\circ}$ & $34103 \mathrm{~m}$ & 146分钟 \\
\hline 江油 & $104.70^{\circ}$ & $31.80^{\circ}$ & $25394 \mathrm{~m}$ & 148分钟 \\
\hline 龙凤镇 & $104.70^{\circ}$ & $31.63^{\circ}$ & $25658 \mathrm{~m}$ & 157分钟 \\
\hline 安县 & $104.41^{\circ}$ & $31.64^{\circ}$ & $22829 \mathrm{~m}$ & 228分钟 \\
\hline 彰明镇 & $104.72^{\circ}$ & $31.72^{\circ}$ & $31831 \mathrm{~m}$ & 260分钟 \\
\hline 绵阳 & $104.73^{\circ}$ & $31.48^{\circ}$ & $48781 \mathrm{~m}$ & 301分钟 \\
\hline
\end{tabular}

\section{模型的评价}

模型的精度受划分时间片粒度大小的影响,越小的时间片将达到越好的效果,实验中采用分钟作为时间片,粒度稍大,但能减低计算复杂度。另外,计算淹没面积的准确性在很大程度上依赖DEM的分辨率,获取分辨率高的DEM模型也是提高洪水淹没模型精度的途径。

该模型是建立在洪水能在时间片内快速达到稳态这一假设之上,虽然洪水蕴含巨大的瞬间能量,很难达到假设要求。但对于漫溢型洪灾,洪水的瞬间动量不像溃决型洪灾那么大,更加符合假设条件。而该模型的研究对象正是漫溢型洪灾,所以该假设具有合理性。总体上,模型优缺点如下:

\textbf{优点}

\begin{enumerate}
    \item 此模型较之常见的基于流体动力学的洪水模型,计算更加简便。
    \item 模型中使用等效糙率的概念,将洪水在运动过程中受到的阻力而带来的延时转加到初始速度上,简化了计算,同时也可以根据实际情况调节参数,从而达到最佳效果。
    \item 模型将时间段划分为一个个时间片,假定洪水能在时间片内快速达到稳态,该模型能存储个时间片的稳态,可以对洪水演进的过程进行动态模拟以及可视化演示。
\end{enumerate}

\textbf{缺点}

\begin{enumerate}
    \item 该模型假设洪水能在小段时间片内瞬时达到稳态,而这是不现实的,必然影响模型的精度。
    \item 该模型不能给出洪峰的速度,以及洪水通过某一截面的流量。
\end{enumerate}

\section{问题四}

\subsection{政策评价}

堰塞湖的坝体——堰塞体由松散土石料经天然堆积而成,在冲刷、侵蚀、溶解、崩塌等作用下通常很少能够保持长期稳定。据统计资料表明,堰塞湖形成后,10天内溃决者超过50%,2个月内溃决者超过60%,1年内溃决者超过90%。溃决后,下游将发生严重的洪灾,其破坏性不亚于地震,危险性极大。2008年5月12日,四川汶川8.0级特大地震形成的34处主要堰塞湖,超过半数处于中危以上。在水位逐渐上涨和渗流加剧的情况下,堰塞湖随时都可能溃决,严重威胁下游河道两侧数以百万计的人民群众和救灾重建队伍的安全。此时,政策就显得尤为重要,轻率的政策可能导致巨大灾害的发生,而过于保守的政策则会造成救灾力量的浪费,引起群众恐慌。

根据新闻媒体报道,当时政府的应对政策变化为:

\begin{tabular}{|c|c|}
\hline 时间 & 政策 \\
\hline 5月25日前 & 挖明渠,降低水位,减小下游风险 \\
\hline 5月25日后 & 紧急调派,爆破挖渠相结合 \\
\hline 5月26日后 & 挖爆结合、先挖后爆、平挖深爆、以爆助挖 \\
\hline
\end{tabular}

下游群众撤离方案主要是:

\begin{itemize}
    \item 如果出现三分之一溃堤险情,绵阳将撤离常住人口十四点七万人,流动人口一点一万人,涉及三十三个乡镇,一百六十九个社区;
    \item 如果出现二分之一溃堤险情,绵阳将撤离常住人口九十一万余人,流动人口二十九万人;
    \item 如果出现全部溃堤险情,绵阳将撤离常住人口九十九万人,流动人口三十点九万人。
\end{itemize}

最终,抗震救灾总指挥部是按照最好的目标(仅三分之一溃坝)进行抢险努力,按照最坏的可能(全线溃坝)做撤离预案,以至于提前完成抢险工程,阻止了溃坝危险的发生。

在地震诱发堰塞湖的应急处理政策中,主要是要做到快速监测、准确模拟、提出方案,迅速执行。实际应用中政策很好地体现了这几个方面。首先,是政府配合专家对堰塞湖周边环境进行快速监测,提取尽可能多的预测情报。同时,绵阳也成立了专门的气象监测小组,详尽提供堰塞湖地区的气象变化情况。另外,在得到情报后,专家马上进行模拟预测,根据水文数据对堰塞湖的危险性进行评估,同时预测了可能威胁的地区和危险到达时间。其次,政府成立了堰塞湖排险领导小组,迅速组织解放军进行排险活动,数千官兵快速执行任务,提前排除了险情。

总体来说,基于当时的情况,作出这样的政策决定是符合当时的情况的,因为当时很多信息无法获取或者技术受限,以致无法做出全面而准确的决策,从而也导致一些徒劳。

为了评价政策的合理性,本文引入地区的危险性评价函数:

\[
d = \omega D(t)
\]

其中,$\omega$ 为距离修正因子,$D(t)$ 为堰塞湖危险风险函数。

定义 $\omega = \frac{d_i}{d_{\max}}$,其中,$d_{\max}$ 为唐家山堰塞湖所能影响的最远距离,$d_i$ 为该地到堰塞湖的距离。由于降雨量对地震引发的堰塞湖灾害影响最大,因此 $D(t)$ 是以降雨量为变量的危险风险函数。

综上所述,地区的危险性评价函数将是以该地区天气以及离堰塞湖的距离为变量的函数。以后只要知道该地区的天气预报降雨量以及离堰塞湖的距离,就可以评价该地的危险风险程度,从而可以制定该地区的抗险对策。本文问题一中已经建立了降雨量—水位上升模型,水位高程也可以评价堰塞湖的危险程度,已在问题一中详述。可见,政策的制定都是根据危险地区和危险时刻而采取的,这也说明了在抗震救灾中所采用政策的适用性。

\subsection{应对地震次生山地灾害的技术建议}

\subsubsection{震后主要次生灾害的发展趋势分析}

地震直接诱发的山地灾害主要是崩塌和滑坡,结合具体的地貌条件、水文条件和松散体规模,衍生出泥石流和堰塞湖灾害,从而形成地震山地灾害链,即:崩塌、滑坡$\rightarrow$(泥石流$\rightarrow$)堰塞湖$\rightarrow$溃决洪水或泥石流,如图5.1。

\begin{figure}[h]
\centering
\includegraphics[width=\textwidth]{image.png}
\caption{地震山地灾害链}
\end{figure}

在地震荷载作用下,斜坡的整体性遭到破坏,降低了斜坡岩土体的抗剪强度,使得大量斜坡处于欠稳定状态。在降雨条件下,雨水沿坡面裂隙入渗,进一步降低边坡的稳定性。因而,在近5年内,地震区还会发生大量的滑坡和崩塌;随着时间的推移,不稳定斜坡数量逐渐减少,滑坡和崩塌活动将呈现逐渐减弱的趋势,估计约持续10年。

地震诱发崩塌、滑坡产生了大量的松散固体物质,使得震区的泥石流活动主要受降雨条件的控制。地震后震区的泥石流活动的数量将增加,泥石流爆发需要的激发雨量条件将降低,泥石流爆发的频度会增加。随着雨季来临,在暴雨的激发下,泥石流将开始强烈活动。参照1973年四川炉霍地震(7.9级)产生泥石流呈带、呈群分布现象,以及1976年松潘-平武地震(7.2级)后,九寨沟进入10余年的泥石流活动活跃期,可以预测:泥石流活动将在未来5年内进入极度活跃期,随后逐渐衰减,持续时间约20年。

地震引发崩塌、滑坡及泥石流堵塞河道形成的堰塞湖中,处于高危险状态的

\subsubsection{应对地震次生山地灾害的建议}

根据前面对震后主要次生灾害发展趋势的分析,可以知道地震的次生灾害的演进主要与当地天气、地质地理等几何物理特性有关。因此,要应对地震后次生山地灾害,就要了解这些资料数据,以便提前做好防治工作及应对策略,防患于未然。那么,科技工作中应该设法解决这些问题,还要特别强调这些数据的真实和可靠性。

\paragraph{(1) 天气监测与降雨预报}

天气监测与降雨预报对泥石流和堰塞湖溃坝事件的防治非常重要。现在普遍采用的方法主要有:

\begin{enumerate}
    \item 地面雨量站。在地面设置雨量计连续监测所在位置的降雨量,观测人员每日定时讯取数值,也有连续自动记录的雨量计,有的雨量计可将降雨量用无线通讯装置自动发报到中心站。
    \item 雷达雨量站。根据空中雨滴、云团中水汽凝结物对雷达发射的电磁波的反射强度来测量降雨中心位置、强度、移动速度等。
    \item 卫星监测。在气象卫星云上装有可见光和红外及微波辐射仪器,从而可以得到卫星云图。根据云图的连续变化了解气旋、降雨和暴雨云团等天气系统的演变情况。
\end{enumerate}

总之,要确保天气监测和降雨预报的真实性和准确性。

\paragraph{(2) 地理监测}

地理监测包括了地形监测和地质监测。地形监测就是要动态实时地掌控地震后震区的几何物理特性的变化,通常通过遥感技术等手段,采用高分辨率的卫星遥感数据和地形图对震区进行检测。地质监测就是分析震区地质组成成分的情况,为研究泥石流和堰塞体提供数据支撑。

以上是要确保自然要素的真实性和可靠性,但是应对灾害,即监测、诊断和制定防灾治灾方案,还要考虑经济、社会和人文要素,并且这些要素要在动态中不断加以调整。正本清源,需要从根本上研究地球的形成、演变和发生发展过程,需要对地球系统进行全面系统的“体格检查”,作出“诊断”,科学地提出防治灾变的方案,进行规划、设计以及组织工程实施。

如果能够借助信息技术等高科技手段模拟地球系统的运动,那么在面对大范围灾害时,就能准确、快速、及时地获取数据和信息,做出快速有效的预测和预报,制订出比较全面的预警预案。

地球系统模拟网络平台就是这样一个平台。地球系统模拟网络是一个先进的

大型计算机网络系统,可以高效、快速地模拟地球上大气、海洋、地球物理、地球化学、生态、环境等诸多领域的动态变化过程,随时发现生态环境条件的异常和突变,监测某些地区的灾害发生的性质、范围和规模,定位定量地预测灾害的发生时段,事先制订出应急反应的预案,评估可能的损失以及恢复重建的规划,未雨绸缪,防患未然。

\section{参考文献}

[1] 谢欣荣,邹平桂,一元多次回归模型拟合水库静库容曲线,《湖南水利》,1999年第 3 期:P27-P28。

[2] 百度百科,地理信息系统,http://baike.baidu.com/view/5201.htm,2008-9-29。

[3] 故园旧梦,《Google Earth 看世界》系列:唐家山堰塞湖,http://blog.sina.com.cn/s/blog_499eff6801009h7e.html,2008-9-29。

[4] 百度百科,唐家山堰塞湖,http://baike.baidu.com/view/1608238.htm,2008-9-29。

[5] 东方热线论坛,艰难的抉择,http://forum.cnool.net/topic_show.jsp?id=4687720&oldpage=3&thesisid=501&flag=topic1,2008-9-29。

[6] 吴小川,溃坝过程及洪水波演进数值模拟研究[D],南京水利科学研究院,2004。

[7] 网易水利论坛,DAMBRK 程序用于水电站日调节不稳定流演算的一个实例,http://bbs.co.163.com/content/1793_637595_1.html,2008-9-30

[8] 谢家毕,孙东亚,水库大坝哭觉模拟方法研究进展[J],中国防汛抗旱,S1,2007。

[9] 陈杰,MATLAB 宝典[M],电子工业出版社,2007.

[10] 张新华等,二维浅水波模型在洪水淹没过程中的模拟研究[J],四川大学学报,2006 年 01 期。

[11] 姚志坚,陈文龙,溃坝洪水演进数学模型研究及其在大镜山水库的应用[J],人民珠江,2008 年第 3 期。

\begin{table}
\centering
\caption{附表3.1}
\begin{tabular}{c c c c c c}
\hline
序号 & 水位高程 & 蓄水量 & & & \\
 & (米) & (亿立方米) & & & \\
\hline
1 & 725.2 & 1.36 & 30 & 713.42 & 0.835 \\
2 & 731.92 & 1.712 & 31 & 713.44 & 0.835 \\
3 & 729.95 & 1.604 & 32 & 713.49 & 0.837 \\
4 & 736.86 & 2.0864 & 33 & 713.5 & 0.838 \\
5 & 736.63 & 2.008 & 34 & 713.54 & 0.839 \\
6 & 735.86 & 2.01 & 35 & 713.57 & 0.84 \\
7 & 739.59 & 2.237 & 36 & 713.64 & 0.843 \\
8 & 740.37 & 2.295 & 37 & 713.68 & 0.844 \\
9 & 742.11 & 2.42 & 38 & 713.79 & 0.849 \\
10 & 725.8 & 1.4 & 39 & 714.13 & 0.861 \\
11 & 727.94 & 1.45 & 40 & 714.51 & 0.876 \\
12 & 742.96 & 2.486 & 41 & 717.51 & 1 \\
13 & 742.82 & 2.475 & 42 & 718.06 & 1.02 \\
14 & 742.58 & 2.457 & 43 & 719.48 & 1.12 \\
15 & 716.01 & 0.977 & 44 & 720.25 & 1.139 \\
16 & 709 & 0.7 & 45 & 737.56 & 2.071 \\
17 & 711 & 0.725 & 46 & 743.13 & 2.469 \\
18 & 737.97 & 2.2 & 47 & 742.58 & 2.457 \\
19 & 741.82 & 2.4 & 48 & 741.66 & 2.389 \\
20 & 742.58 & 2.457 & 49 & 741.55 & 2.351 \\
21 & 743.13 & 2.469 & 50 & 740.51 & 2.328 \\
22 & 735.81 & 2.054 & 51 & 739.97 & 2.237 \\
23 & 742.96 & 2.486 & 52 & 739.75 & 2.222 \\
24 & 742.58 & 2.457 & 53 & 739.3 & 2.19 \\
25 & 742.13 & 2.424 & 54 & 739.23 & 2.185 \\
26 & 742.55 & 2.425 & 55 & 739.08 & 2.175 \\
27 & 725.8 & 1.4 & 56 & 738.85 & 2.159 \\
28 & 714.01 & 0.857 & 57 & 738.23 & 2.116 \\
29 & 713.41 & 0.834 & 58 & 737.91 & 2.094 \\
 & & & 59 & 737.33 & 2.055 \\
\hline
\end{tabular}
\end{table}

\begin{table}
\centering
\caption{附表3.2}
\begin{tabular}{c c c}
\hline
\multirow{2}{*}{日期} & \multicolumn{2}{c}{降雨量(毫米)} \\
\cline{2-3}
 & 白天 & 夜间 \\
\hline
5月25日 & 5-10 & 8-18 \\
5月26日 & 6-12 & 5-10 \\
5月27日 & 0 & 0 \\
5月28日 & 1-2 & 1-2 \\
\hline
\end{tabular}
\end{table}

\begin{table}
\centering
\begin{tabular}{|c|c|c|}
\hline
5月29日 & 2-5 & 2-5 \\
\hline
5月30日 & 0 & 0 \\
\hline
5月31日 & 0 & 0 \\
\hline
6月1日 & 0 & 0 \\
\hline
6月2日 & 0-1 & 0-1 \\
\hline
6月3日 & 0 & 0 \\
\hline
6月4日 & 0 & 0 \\
\hline
6月5日 & 0 & 1-5 \\
\hline
6月6日 & 5-8 & 18-23 \\
\hline
6月7日 & 0-2 & 0 \\
\hline
6月8日 & 0 & 0 \\
\hline
6月9日 & 0 & 0 \\
\hline
6月10日 & 0 & 2-6 \\
\hline
6月11日 & 1-5 & 0 \\
\hline
6月12日 & 0 & 0 \\
\hline
\end{tabular}
\caption{附表3.3(单位:米)}
\end{table}

\begin{table}
\centering
\begin{tabular}{|c|c|c|c|c|}
\hline
日期 & \multicolumn{4}{c|}{调整因子取值情况} \\
\hline
 & 50\% & 80\% & 100\% & 150\% \\
\hline
5月25日 & 6.02 & 10.19 & 12.78 & 18.55 \\
\hline
5月26日 & 5.27 & 7.43 & 8.46 & 9.56 \\
\hline
5月27日 & 0 & 0 & 0 & 0 \\
\hline
5月28日 & 0.60 & 0.80 & 0.87 & 0.80 \\
\hline
5月29日 & 1.46 & 1.91 & 2.02 & 1.68 \\
\hline
5月30日 & 0 & 0 & 0 & 0 \\
\hline
5月31日 & 0 & 0 & 0 & 0 \\
\hline
6月1日 & 0 & 0 & 0 & 0 \\
\hline
6月2日 & 0.29 & 0.36 & 0.38 & 0.28 \\
\hline
6月3日 & 0 & 0 & 0 & 0 \\
\hline
6月4日 & 0 & 0 & 0 & 0 \\
\hline
6月5日 & 0.85 & 1.06 & 1.09 & 0.72 \\
\hline
6月6日 & 6.75 & 7.36 & 6.33 & 4.27 \\
\hline
6月7日 & 0.44 & 0.39 & 0.22 & 0.11 \\
\hline
6月8日 & 0 & 0 & 0 & 0 \\
\hline
6月9日 & 0 & 0 & 0 & 0 \\
\hline
6月10日 & 0.85 & 0.71 & 0.34 & 0.13 \\
\hline
6月11日 & 0.62 & 0.47 & 0.17 & 0.04 \\
\hline
6月12日 & 0 & 0 & 0 & 0 \\
\hline
\end{tabular}
\end{table}

\end{document}