\documentclass{article}
\usepackage{amsmath}
\usepackage{amssymb}

\title{旅游路线规划问题}
\author{}
\date{}

\begin{document}

\begin{center}
\textbf{第十二届“中关村青联杯”全国研究生数学建模竞赛}
\end{center}

\begin{tabular}{l l}
学校 & 上海交通大学 \\
\hline
参赛队号 & 10248085 \\
\hline
队员姓名 & 1. 鲁斯嘉 \\
 & 2. 关士托 \\
 & 3. 王朝静 \\
\hline
\end{tabular}

\maketitle

\begin{abstract}
随着国民经济的快速发展,旅游活动成为现代生活的重要组成部分。本题要求为自驾游爱好者制定全国 201 个 5A 景点旅游计划。本文采用分层优化方法,建立两层带两个时间窗的旅行商问题模型、修正的带时间窗多车辆路径问题模型,采用改进的遗传算法并借助 MATLAB 编程求解,在第一问中得到以出游年数最短为目标的自驾出游方案。第二问中,在对两层模型目标和约束修正的基础上,为旅游者提供以体验最佳、费用最优的多交通方式出行的旅游规划。第三问中,本文将模型进行推广,为北京的自驾游游客制定了游览全国 5A 景点的十年旅游计划,并为旅游者和相关旅游部门提供可借鉴性建议。第四问中,基于第二问所建立的模型,增加游览时间比(游览时间占总出游时间的比例)最大的目标,将 4A 景区纳入出游考虑范围,制定了更为合理的十年旅游计划。

对于问题 1,本文采用分层优化的思想,下层为各省会城市辐射区内景点的路径规划,以出行时间最短为目标,在景点开放时间、车行区间等限制因素下,建立带两个时间窗的旅行商问题模型,采用双种群遗传算法并借助 MATLAB 编程软件进行求解,得到每个省会城市辐射区内景点的最短时间游览路线。在下层优化的基础上,上层模型以旅游年数最少为目标,将 201 个景点出游路径规划简化为 31 个省市游览路线优化,对带时间窗的多车辆路径问题(MVRPTW)模型进行修正,使同一城市游览时间超过 15 天限制时拆分为两次游览。采用遗传算

法并借助 MATLAB 编程求解, 得到 201 个 5A 景区出游总时间为 11 年零 10 天, 并给出每天出发地、行车时间里程、游览景区等信息。

对于问题 2, 本文首先比较“自驾”、“飞机+租车”和“高铁+租车”三种交通方式的费用-距离曲线, 得到高铁优先考虑、一天车程范围内采用自驾方式出游, 两天以上车程选用飞机出行的结论。在此基础上, 对第一问两层优化模型进行修正, 以总游览年数最少、自驾出行次数最少和费用最优为一、二级目标, 增加票价、车油费过路费、住宿费、租车费等费用约束。通过双种群遗传算法和 MATLAB 软件进行求解, 得到最优的十年出游计划总花费为 338447.96 元, 并给出每次出行方式、每天出发地、行车时间里程、游览景区等信息。

对于问题 3, 本文将前两问中的模型进行推广, 将出发地由西安改为北京, 运用问题 2 中的模型和算法进行求解, 得到最优的十年旅游计划, 并给出每次出行方式、出行行程、天数和游览景区信息, 有效验证了模型的可行性和适用性。最后根据不同优化目标的最优旅游计划为旅游者和旅游部门分别提出三条和四条建议。

对于问题 4, 基于问题 2 中模型, 在游览 5A 级景区的基础上, 将 4A 级景区纳入考虑范围, 并引入游览时间比 (游览时间/总出行时间) 为目标函数, 对问题 2 中得到的出行规划进行优化, 有效利用了“间隙”时间, 得到出行体验更佳、更加合理的出行规划, 并给出十年出行规划。

关键词: 旅行商问题(TSP) 多车辆路径问题(MVRP) 遗传算法(GA) 旅游计划
\end{abstract}

\tableofcontents

\section{问题重述}

一位自驾游爱好者拟制定旅游计划遍历全国 201 个 5A 级景区。该旅游爱好者自身的限制条件:

(1) 每年外出旅游时间不超过 30 天;

(2) 每年外出旅游的次数不超过 4 次;

(3) 每次旅游的时间不超过 15 天;

(4) 根据个人偏好,每个 5A 景区有最少游览时间;

时间窗限制条件为:

(5) 行车时间限定于每天 7:00 至 19:00 之间,每天开车时间不超过 8 小时;

(6) 景区开放时间统一为 8:00 至 18:00。

在每天的行程安排上,有以下规则:

(7) 若安排全天游览,开车时间控制在 3 小时内;

(8) 若安排半天景点游览,开车时间控制在 5 小时内;

(9) 在每个省会城市至少停留 24 小时感受风土人情(不安排景区浏览);

对于道路行车,速度限制如下:

(10) 高速公路上的行车平均速度为 90 公里/小时;

(11) 普通公路上的行车平均速度为 40 公里/小时。

需要考虑如下问题:

(1) 行车线路采用高速优先原则,即先通过高速公路到达与景区邻近的城市,再自驾到景区,以常住地在西安市为例,规划设计游遍 201 个 5A 级景区旅游线路,确定花费年数及每一次旅游的具体行程。

(2) 出行方式综合考虑全程自驾、先乘坐高铁或飞机到达省会城市后再租车自驾到景区等出行方式,租车费用 300 元/天,油费和高速过路费另计,租车和还车需在同一城市。住宿费简化为省会城市和旅游景区 200 元/人·天,地级市 150 元/人·天,县城 100 元/人·天。高速公路的油耗加过路费平均为 1.00 元/公里,普通公路上油耗平均为 0.60 元/公里。假设每一个景区最长逗留时间不超过最少时间的 2 倍,选择高铁出行要求当天乘坐高铁的时间不超过 6 个小时,且至多安排半天的游览。该旅游爱好者一家 3 人同行,要求设计一个十年游遍所有 201 个 5A 景区的费用最优、旅游体验最好的旅游线路,并给出每一次旅游的具体线路。

(3) 能否在第二问所建立的模型基础上加以推广,可以为全国的自驾游爱好者规划设计类似的旅游线路,进而给出常住地在北京市的自驾游爱好者的十年旅游计划;根据上述三问的结果给旅游爱好者和旅游有关部门提出建议。

(4) 自 2007 年 3 月 7 日至 2015 年 7 月 13 日,全国旅游景区质量等级评定委员会分 29 批共批准了 201 家景区为国家 5A 级旅游景区。结合国家 4A 级景区名单,请更为合理地规划该旅游爱好者的十年旅游计划。

\section{问题分析}

本题要求根据不同的条件约束不同的目标规划旅游路线,具体问题分析如下:

\subsection{问题一的分析}

本问中要求制定以西安为常驻地的最短年数 201 个 5A 景点出游计划,并给出每次具体行程。本文采用分层优化思想,下层以出行时间最短为目标,建立带两个时间窗的旅行商问题模型,求每个省会城市辐射区内景点的最短时间游览路线。上层模型以旅游年数最少为目标,将 201 个景点出游路径规划简化为 31 个省市游览路线优化,对带时间窗的多车辆路径问题模型进行修正,采用遗传算法并借助 MATLAB 编程求解,得到 201 个 5A 景区出游总年数最少的出游计划。

\subsection{问题二的分析}

比较“自驾”、“飞机+租车”和“高铁+租车”三种交通方式的费用-距离曲线,求得出行方式选择与距离关系。在此基础上,对第一问两层优化模型进行目标、约束的修正,以总游览年数最少、自驾出行次数最少和费用最优为一、二级目标,增加票价、车油费过路费、住宿费、租车费等费用约束,以求最优的十年出游计划。

\subsection{问题三的分析}

本题将前两问中的模型进行推广,将出发地由西安改为北京,运用问题 2 中的模型和算法进行求解,得到最优的十年旅游计划,以此验证模型的可行性和适用性,最后根据不同优化目标的最优旅游计划为旅游者和旅游部门提出可行性建议。

\subsection{问题四的分析}

本题基于问题 2 中模型,在游览 5A 级景区的基础上,将 4A 级景区纳入考虑范围,并引入游览时间比(游览时间/总出行时间)为目标函数,对问题 2 中得到的出行规划进行优化,有效利用了“间隙”时间,从而得到出行体验更佳、更加合理的十年出行规划。

\section{符号定义与说明}

\begin{tabular}{l l}
\textbf{符号} & \textbf{定义} \\
\hline
$F=\{1,2,\ldots,n\}$ & 表示省会城市 $m$ 辐射区内与景点邻近的城市点集,其中 $i=1$ 表示省会城市; \\
$T_{i}$ & 表示与 $i$ 城市邻近景点的游览时间(天); \\
$T_{mi}$ & 表示与 $i$ 城市邻近景点的最少游览时间(天); \\
$T_{ij}$ & 表示游客驾车从 $i$ 城市邻近景点到 $j$ 城市邻近景点的驾车行驶时间(天); \\
$PT_{ij}$ & 表示游客驾车从 $i$ 城市到达邻近景点的驾车行驶时间(天); \\
$S_{ij}$ & 表示游客驾车从 $i$ 景点邻近城市到 $j$ 景点邻近城市的驾车行驶距离(公里); \\
$X_{ij}$ & 表示 0-1 决策变量,游客从 $i$ 城市邻近景点到 $j$ 城市邻近景点参观时,$X_{ij}=1$ \\
$S$ & 表示集合 $S$ 中的元素个数; \\
$AT_{i}$ & 表示游客抵达 $i$ 城市邻近景点的时刻; \\
$e_{i}$ & 表示为允许最早游览时刻,由题意知景点开放时间为 8:00-18:00,取 $e_{i}=8:00$ \\
$l_{i}$ & 表示为允许最迟游览时刻,即景点开放结束时间与景点最少游览时间之差,有 $l_{i}=18-T_{mi}$; \\
$[e_{i},l_{i}]$ & 表示景点 $i$ 的时间窗; \\
$DT_{i}$ & 表示游客从 $i$ 邻近城市景点驾驶出发时刻; \\
$p_{i}$ & 表示为允许最早行车时刻,由题意知行车时间 7:00-19:00,取 $p_{i}=7:00$; \\
$q_{i}$ & 表示为允许最迟行车时刻,由题意知行车时间 7:00-19:00,取 $q_{i}=19:00$; \\
$[p_{i},q_{i}]$ & 表示景点 $i$ 的时间窗; \\
$F^{*}=\{1,2,\ldots,m\}$ & 表示省会城市点集,其中 $i=1$ 表示出发地西安,$m=31$; \\
$T_{i}^{*}$ & 表示 $i$ 省会城市所有游览次数总逗留时间,即下层优化求解结果(表 1)(天); \\
$T_{iyk}$ & 表示 $i$ 省会城市在第 $y$ 年的第 $k$ 次游览时的逗留时间(天); \\
$T_{ijyk}$ & 表示游客第 $y$ 年的第 $k$ 次从 $i$ 省会城市到 $j$ 省会城市的驾车行驶时间(天); \\
\end{tabular}

\begin{itemize}
    \item $S_{ijyk}$ 表示游客第 $y$ 年的第 $k$ 次从 $i$ 省会城市到 $j$ 省会城市的驾车行驶距离(公里);
    \item $X_{ijyk}$ 0-1 决策变量,表示游客第 $y$ 年的第 $k$ 次出游时,从 $i$ 省会城市到 $j$ 省会城市参观时,$=1$,否则,$=0$,$K=1,2,\ldots N$,$y=1,2,\ldots M$;
    \item $S^*$ 表示集合 $S$ 中的元素个数;
    \item $DT_i^*$ 表示游客从 $i$ 省会城市驾驶出发时刻;
    \item $p_i^*$ 表示为允许最早行车时刻,由题意知行车时间 7:00-19:00,取 $p_i^*=7:00$;
    \item $q_i^*$ 表示为允许最迟行车时刻,由题意知行车时间 7:00-19:00,取 $q_i^*=19:00$;
    \item $[p_i^*, q_i^*]$ 表示景点 $i$ 的时间窗;
    \item $PS_{ij}$ 表示游客驾车从 $i$ 城市到达邻近景点的驾车行驶距离(公里);
    \item $HC_{ij}$ 表示 $i$ 城市邻近景点到 $j$ 城市邻近景点游览的住宿费用(元);
    \item $CC_{ij}$ 表示从 $i$ 城市邻近景点到 $j$ 城市邻近景点游览的油费和过路费(元);
    \item $L_i$ 0-1 变量,表示与 $i$ 城市邻近的景区位于地级市时,$L_i=1$,位于县级市时 $L_i=0$;
    \item $C_{ijyk}$ 表示第 $y$ 年的第 $k$ 次从 $i$ 省会城市到 $j$ 省会城市游览的所有费用(元);
    \item $HC_{ijyk}$ 表示第 $y$ 年的第 $k$ 次从 $i$ 省会城市到 $j$ 省会城市游览的住宿费用(元);
    \item $CC_{ijyk}$ 表示第 $y$ 年的第 $k$ 次从 $i$ 省会城市到 $j$ 省会城市游览的油费和过路费(元);
    \item $HC_i$ 表示 $i$ 省会城市内部游览的住宿费用(元);
    \item $CC_i$ 表示 $i$ 省会城市内部游览的油费和过路费(元);
    \item $RC_{iyk}$ 表示第 $y$ 年的第 $k$ 次在 $i$ 省会城市游览的租车费用(元);
    \item $TC_{ijyk}$ 表示第 $y$ 年的第 $k$ 次从 $i$ 省会城市到 $j$ 省会城市游览的票价费用(元);
    \item $W_{ij}$ 0-1 变量,当从 $i$ 省会城市到 $j$ 省会城市自驾游览时,$W_{ij}=1$,当采用飞机、高铁交通方式时,$W_{ij}=0$;
\end{itemize}

\section{模型假设}

\begin{enumerate}
    \item 假设采用高速优先的策略,即先通过高速公路到达与景区邻近的城市,再自驾到景区。
    \item 假设在两个景点邻近城市之间的道路均为高速公路,在高速公路上的行车平均速度为 90 公里/小时;景点邻近城市到邻近景点的道路为普通公路,驾驶距离与时间参照附件 1。
    \item 据题意可假设“一天”等于“8 个小时”。
    \item 假设某省会城市辐射区景点是指与该省行政区域范围内城市邻近的所有景点(包括省内、省外景点)。
    \item 某一省市内景点邻近城市的选取根据附件 1 中景点与各城市距离时间数据确定。
    \item 游客若需要租车游览,均在省会城市租还车。
    \item 由于同一省内各城市间距离较近,且需要在省会城市租还车,因此为节省费用,省内出行均采用自驾或租车方式。
    \item 假设“体验最佳”是指游览 201 个景点总年数最少,并尽可能减少自驾出游次数,即对于远距离的景点城市尽量选用飞机、高铁交通方式。
    \item 假设“体验最佳”为第一级目标,“费用最优”为第二级目标,两者权重值 $P1:P2=3:2$。
    \item 由于门票、饮食等费用,不影响出游路线规划,为简化计算,假设出游费用不包括门票、饮食等费用。
    \item 若游客当天 19:00 结束驾驶时,位于两个城市的驾车途中,则假设当晚的住宿费用与地级市的住宿费用相同,为 150 元/天人。
\end{enumerate}

\section{模型的建立与求解}

\subsection{问题一的求解}

据题意可知,要求线路设计采用高速优先策略,先通过高速公路到达与景区邻近的城市,再自驾到景区,因此,与景区邻近的城市可作为路线规划中的必经节点。其二,要求旅游爱好者在每个省会城市除去游览景区外,至少停留 24 个小时感受风土人情,因此,每个省会城市也是路线选择的必经节点。在必须到达省会城市、景点邻近城市的要求基础上,本题采用分层优化思想,以简化求解。在下层路径优化中,以必经节点省会城市为起点,对省会城市辐射区内景点邻近城市进行路线规划,由此可以得到 31 个省会城市辐射区内景点最优旅游路线方案。在此基础上,每个省会城市辐射区内景点游览总时间等价于该省会城市所需观光时间,上层优化中原有的 201 个 5A 景区路线规划由此简化为 31 个省会城市路线优化。通过两层优化模型的求解,从而可以制定游览年数最少的 201 个 5A 景区出游计划。

\subsubsection{下层路径优化}

在下层优化中,由于游客由省会城市出发,到达各景区邻近地级城市,再前往景区游览。由于采用高速优先策略,辐射区内所有景区游览结束后需要回到省会城市,以便在上层中可以通过高速公路返回西安或者前往下一省会城市辐射区。由此,下层中某一省会城市辐射区内出游路线为一个闭合回路,每个节点均到达一次,离开一次,属于旅行商问题(TSP)[1]。由于行车区间和景点开放时间的限制,本层建立带两个时间窗的旅行商问题模型[2],以求得每个省会城市辐射区内景点最短时间出游计划。

\subsubsection{模型建立}

本层的带两个时间窗的旅行商问题模型以履行总时间最少为目标,以行车时间、景点开放时间等限制因素为约束,如下所示:

\textbf{目标函数:}
\[
Z_m = \text{Min} \sum_{i \neq j} (T_{ij} + T_i) X_{ij}
\]

\textbf{约束:}
\begin{align*}
A &= 5 \, 1 \, | \, 4 \, 3 \, 2 \\
B &= 3 \, 2 \, | \, 1 \, 4 \, 5
\end{align*}

\begin{align*}
A &= 5 \, 1 \, | \, 4 \, 3 \, 2 \\
B &= 3 \, 2 \, | \, 1 \, 4 \, 5
\end{align*}

\begin{align*}
A' &= 2 \, 3 \, | \, 1 \, 4 \, 5 \\
B' &= 1 \, 5 \, | \, 3 \, 4 \, 2
\end{align*}

\begin{align}
\sum_{j=1}^m \sum_{y=1}^M \sum_{k=1}^N X_{ijyk} &\geq 1 \tag{1} \\
\sum_{i=1}^m \sum_{y=1}^M \sum_{k=1}^N X_{ijyk} &\geq 1 \tag{2} \\
\sum_{j=1}^m \sum_{y=1}^M \sum_{k=1}^N X_{ijyk} &= \sum_{i=1}^m \sum_{y=1}^M \sum_{k=1}^N X_{ijyk} \tag{3} \\
\sum_{i=1}^m \sum_{j=1}^m (T_{ijyk} + T_{iyk}) X_{ijyk} &\leq 15 \tag{4} \\
\sum_{y=1}^M \sum_{i=1}^m \sum_{k=1}^N X_{ijyk} T_{jyk} &\geq T_j^* \tag{5} \\
\sum_{i,j \in F} X_{ijyk} &\leq |S^*| - 1, \, 2 \leq S^* \leq m - 2, \, S^* \in (1, 2, \dots, n) \tag{6} \\
p_i^* &\leq DT_i^* + 8T_{ijyk} \leq q_i^* \tag{7} \\
\sum_{i=1}^m \sum_{j=1}^m \sum_{k=1}^N X_{ijyk} (T_{ijyk} + T_{iyk}) &\leq 30 \tag{8} \\
1 \leq K &\leq 4 \tag{9} \\
T_{ijyk} &= 0 \, \text{ 若 } S_{ijyk}/90 < 3 \tag{10} \\
T_{ijyk} &= 0.5 \, \text{ 若 } 3 \leq S_{ijyk}/90 < 5 \tag{11} \\
T_{ijk} &= 1 \, \text{ 若 } 5 \leq S_{ijyk}/90 < 8 \tag{12} \\
T_{ijk} &= \left[ \left( \frac{S_{ijyk}}{90} \right) \backslash 8 \right] \quad \text{ 若 } 8 \leq \frac{S_{ijyk}}{90} \text{ 且 } \left( \frac{S_{ijyk}}{90} \right) Mod \, 8 < 3 \tag{13} \\
T_{ijk} &= \left[ \left( \frac{S_{ijyk}}{90} \right) \backslash 8 \right] + 0.5 \, \text{ 若 } 8 \leq \frac{S_{ijyk}}{90} \text{ 且 } 3 \leq \left( \frac{S_{ijyk}}{90} \right) Mod \, 8 < 5 \tag{14} \\
T_{ijk} &= \left[ \left( \frac{S_{ijyk}}{90} \right) \backslash 8 \right] + 1 \, \text{ 若 } 8 \leq \frac{S_{ijyk}}{90} \text{ 且 } 5 \leq \left( \frac{S_{ijyk}}{90} \right) Mod \, 8 \tag{15} \\
C_{ijyk} &= HC_{ijyk} + CC_{ijk} + (RC_{jyk} + TC_{ijyk}) * (1 - W_{ij}) \tag{16} \\
CC_{ijyk} &= \frac{S_{ijyk}}{90} + CC_j \tag{17} \\
&\quad \frac{36}{}
\end{align}

\begin{equation}
T_{ij} = \left[\left(\frac{S_{ij}}{90} + 8PT_{ij}\right) \backslash 8\right] + 1 \quad \text{若 } 8 \leq \frac{S_{ij}}{90} + 8PT_{ij} \text{ 且 } 5 \leq \left[\left(\frac{S_{ij}}{90} + 8PT_{ij}\right) \text{ Mod } 8\right]
\tag{15}
\end{equation}

\begin{equation}
X_{ij} \in (0,1), \quad i, j = 1, 2, \ldots, n
\tag{16}
\end{equation}

模型以省会城市辐射区内所有景点的游览总时间(包括景点游览时间、车行时间)最小为目标,优化该省市辐射区内景点的最优游览路径。其中,公式 1 表示游客从每一个景点走出一次,公式 2 表示游客到达每一个景点一次,公式 3 表示避免景点间小回路的发生,公式 4-5 表示景区开放时间限制,公式 6 表示前后景点的到达时间关系,公式 7-8 表示每天车行时间区间限制,公式 9 表示景点游览时间限制,公式 10-15 表示当行车时间控制在 3 小时内时,景点间车行时间可以忽略,全天游览;开车时间控制在 5 小时内,景点间车行时间记为半天,景点游览半天;当车行时间在 8 小时之内,视为行车时间为 1 天;超过 8 小时时,按照以上规则记为多天。公式 16 表示决策变量 \(X_{ij}\) 为 0-1 变量以及 \(i, j\) 的取值范围。

\subsubsection{模型求解}

本模型的输入数据为各省市内任意景点邻近城市之间的距离、邻近城市至景点的距离、每个景点的游览时间等。其中,景点邻近城市的选取根据题目所给景点至各城市的距离时间数据确定,则 31 个省市邻近城市选取及景点邻近城市之间的 OD 矩阵如附录 1 所示。本文通过双种群遗传算法 [3][4],对问题进行求解,具体步骤如下:

\subsubsection{编码}

这里我们采用简单易用的整数编码方法。对于有 \(n\) 个访问点的 TSP 问题,将景点所城市从 1 到 \(n\) 进行编号,每个从 1 到 \(n\) 的全排列都可以作为该问题的一个解,即染色体个体。例如有 5 个景点城市的 TSP 问题,\{3, 5, 4, 2, 1\}就是一个合法的染色体,表示车辆对 5 个省会城市的访问路线是 3-5-4-2-1。

\subsubsection{产生初始群体}

初始群体的产生是随机的,也可以应用 TSP 的构造启发式产生。遗传算法是通过选择、交叉和变异算子来完成相对有限种群的进化。当群体的规模太小时,遗传算法的性能一般不会太好,而采用较大规模的群体虽然可以减少遗传算法陷于局部最优的机会,但此时计算复杂度会较高。这里,我们借用正交设计来对问题的可行域空间进行正交采样,逐步产生优化群体。

\subsubsection{求取适应度函数}

利用遗传算法寻优,就是要找到适应度最大的染色体,而软时间窗约束的 TSP 的目标函数是计算复杂度会较高。由于适应度函数一般要求非负,因此将目标函数转化为适应度函数,这里我们通过变换 \(f_k = bz'/z\) 来达到目的。此处 \(f_k\) 为染色体 \(\nu_k\) 适应度,\(b\) 为一常数,\(z'\) 为初始种群最佳染色体的旅行成本,\(z_k\) 为染色体 \(\nu_k\) 所对应的旅行成本。

\section{遗传算子的设计}

(1) 交叉算子采用正交交叉方式。记当前的第 $K$ 代种群为先对种群进行两两配对,按照预先产生的正交阵对种群进行交叉,得到种群 $B_k$。

(2) 对交叉后得到的种群 $B_k$ 中的每个体 $x_i$ 以预先给定的变异概率进行变异。

(3) 对变异后的种群中最差的 $S$ 个体进行更新操作,并得的种群为 $D_k$。

\section{种群交叉}

将两个种群中最优解取出,再在每个种群中随机选取 num 个染色体,将这 num+1 个染色体互换,进入对方种群。从而形成最优的遗传信息。

\section{终止条件}

由于遗传算法具有较大的随机性,这里使用几条基于启发式规则的终止条件:

(1) 若算法迭代到 100 代,则终止算法。

(2) 若某代群体中染色体的平均适应度与当代最佳染色体适应度的比值大于 0.9,则终止算法。

(3) 若最佳染色体连续保持 10 代,则终止算法。

设计的双种群遗传算法程序如图所示。

\begin{figure}[h]
\centering
\includegraphics[width=0.8\textwidth]{image.png}
\caption{双种群遗传算法流程图}
\end{figure}

\begin{tabular}{l l}
学校 & 上海交通大学 \\
\hline
参赛队号 & 10248085 \\
\hline
队员姓名 & 1. 鲁斯嘉 \\
 & 2. 关士托 \\
 & 3. 王朝静 \\
\hline
\end{tabular}

\begin{figure}[h]
    \centering
    \includegraphics[width=0.8\textwidth]{image.png}
    \caption{多车辆路径问题(MVRP)示意图}
    \label{fig:mvrp}
\end{figure}

\subsection{模型的建立}

本题中要求每次出行时间不多于 15 天,因此当某一省会城市游览时间大于最大上限 15 时,需要两次游览。而多车辆路径问题模型中每个节点只能一次到达一次离开。因此,本文结合本题约束条件对 MVRPTW 问题模型进行修正如下:

\subsubsection{目标函数}

\begin{equation}
Z_m = Min \sum_{y=1}^{M} \left( 30 - \sum_{i=1}^{m} \sum_{j=1}^{m} \sum_{k=1}^{N} X_{ijky} \right)
\end{equation}

\subsubsection{约束}

\begin{equation}
\sum_{j=1}^{m} \sum_{y}^{M} \sum_{k=1}^{N} X_{ijyk} \geq 1 \tag{1}
\end{equation}

\begin{equation}
\sum_{i=1}^{m} \sum_{y}^{M} \sum_{k=1}^{N} X_{ijyk} \geq 1 \tag{2}
\end{equation}

\begin{equation}
\sum_{j=1}^{m} \sum_{y}^{M} \sum_{k=1}^{N} X_{ijyk} = \sum_{i=1}^{m} \sum_{y}^{M} \sum_{k=1}^{N} X_{ijyk} \tag{3}
\end{equation}

\begin{equation}
\sum_{i=1}^{m} \sum_{j=1}^{m} (T_{ijyk} + T_{iyk}) X_{ijyk} \leq 15 \tag{4}
\end{equation}

\begin{equation}
\sum_{y=1}^{M} \sum_{i=1}^{m} \sum_{k=1}^{N} X_{ijyk} T_{jyk} \geq T_j^* \tag{5}
\end{equation}

\begin{equation}
\sum_{i,j \in F} X_{ijyk} \leq |S^*| - 1, \ 2 \leq S^* \leq m - 2, \ S^* \in (1, 2, \dots, n) \tag{6}
\end{equation}

\begin{equation}
p_i^* \leq DT_i^* + 8T_{ijyk} \leq q_i^* \tag{7}
\end{equation}

\begin{equation}
\sum_{i=1}^{m} \sum_{j=1}^{m} \sum_{k=1}^{N} X_{ijyk} (T_{ijyk} + T_{iyk}) \leq 30 \tag{8}
\end{equation}

\begin{equation}
1 \leq K \leq 4 \tag{9}
\end{equation}

\begin{equation}
T_{ijyk} = 0 \ \text{若} \ S_{ijyk}/90 < 3 \tag{10}
\end{equation}

\begin{equation}
T_{ijyk} = 0.5 \ \text{若} \ 3 \leq S_{ijyk}/90 < 5 \tag{11}
\end{equation}

\begin{equation}
\text{T}_{ijk} = 1 \quad \text{若 } 5 \leq S_{ijyk}/90 < 8
\tag{12}
\end{equation}

\begin{equation}
T_{ijk} = \left[\left(\frac{S_{ijyk}}{90}\right) \backslash 8\right] \quad \text{若 } 8 \leq \frac{S_{ijyk}}{90} \text{ 且 } \left[\left(\frac{S_{ijyk}}{90}\right) Mod \ 8\right] < 3
\tag{13}
\end{equation}

\begin{equation}
T_{ijk} = \left[\left(\frac{S_{ijyk}}{90}\right) \backslash 8\right] + 0.5 \quad \text{若 } 8 \leq \frac{S_{ijyk}}{90} \text{ 且 } 3 \leq \left[\left(\frac{S_{ijyk}}{90}\right) Mod \ 8\right] < 5
\tag{14}
\end{equation}

\begin{equation}
T_{ijk} = \left[\left(\frac{S_{ijyk}}{90}\right) \backslash 8\right] + 1 \quad \text{若 } 8 \leq \frac{S_{ijyk}}{90} \text{ 且 } 5 \leq \left[\left(\frac{S_{ijyk}}{90}\right) Mod \ 8\right]
\tag{15}
\end{equation}

\begin{equation}
X_{ijyk} \in (0,1), \ W_{ij} \in (0,1). \ i,j = 1,2,\ldots,m; \ y = 1,2,\ldots,M; \ K = 1,2,\ldots,N
\tag{16}
\end{equation}

$(30 - \sum_{i=1}^{m} \sum_{j=1}^{m} \sum_{k=1}^{N} X_{ijky})$ 表示每一年出行天数与 30 天的差值,模型目标函数是指所有年份的的差值总和最小。即表示每年的出行天数尽可能达到 30 天最大上限,从而保证游览 201 个景区所花费的年数最少。公式 1-3 表示每个省会城市至少到达 1 次,且到达次数与离开次数相同。公式 4 表示每次出行不超过 15 天约束,公式 5 表示每个省会城市的所有次数的游览时间之和满足要求,公式 6 是每一次旅游的小回路消除条件,公式 7 是指每天车行时间区间限制,公式 8 表示每年出行的时间不超过 30 天,公式 9 表示每年的出行次数不超过 4 次,公式 10-15 表示当行车时间控制在 3 小时内时,景点间车行时间可以忽略,全天游览;开车时间控制在 5 小时内,景点间车行时间记为半天,景点游览半天;当车行时间在 8 小时之内,视为行车时间为 1 天;超过 8 小时时,按照以上规则记为多天。公式 16 表示决策变量 $X_{ijk}$ 为 0-1 变量以及 $i,j,k$ 的取值范围。

\subsection{模型的求解}

在车辆调度问题的研究中,常常将复杂的问题转化为相对简单的问题,然后进行求解,从而间接得到复杂问题的解。本题需要游客从西安多次出行遍历所有 31 个省会城市,可视为多个游客同时从西安出发,花费最短的时间遍历所有 31 个省会城市,因而可以转化为单车场多车辆调度规划问题。但是本题的单车场多车辆调度规划问题与传统的单车场多车辆调度规划问题有所不同,单个配送点的配送量可能大于配送车辆的载重。针对上述两点,我们做出如下改进:针对需求超过配送车辆载重的需求点,将该需求点拆分成多个可由一辆车辆完成分配的需求点。在此基础上,本文对传统的多车辆路径问题的遗传算法 [8] 进行改进,以求解本题模型。

\subsubsection{求解模型的基本遗传算法}

遗传算法是一种模仿生物进化过程的全局随机搜索方法。遗传算法的基本思想为:从优化问题的一个种群(一组可行解)开始,按照适者生存和优胜劣汰的原理,逐代演化产生出越来越好的一个种群(一组可行解)。

基本遗传算法的执行过程如下:

\begin{itemize}
    \item 第一步:初始化群体大小 $N$,交叉概率 $P_{e}$,变异概率 $P_{m}$ 等参数,随机生成初始种群:$X(0)$;
    \item 第二步:计算种群中个体的适应度;
    \item 第三步:按照遗传策略,对第 $t$ 代种群 $X(t)$ 进行选择操作、交叉操作和变异操作,形成下一代的种群 $X(t+1)$;
    \item 第四步:判断算法是否满足停止准则,如果不满足,则返回到第二步;如果满足,则输出种群中的最大适应度值的个体作为最优解 $X^*$,终止计算。
\end{itemize}

基本遗传算法使用三种遗传算子,即选择运算使用比例选择算子、交叉算子使用单点交叉算子、变异运算使用基本位变异算子或均匀变异算子。

编码方式:传统的二进制编码及其基于二进制编码的基本遗传操作并不适用于求解 VRP 问题,因此我们采用基于整数符号形式的编码方式,即个体染色体编码串中的基因值是一个无数值含义,只代表被车辆访问客户编号的符号集,如 $\{1, 2, 3, 4, \ldots\}$ 等。

\begin{figure}[h]
    \centering
    \includegraphics[width=0.8\textwidth]{genetic_algorithm_flowchart.png}
    \caption{遗传算法流程图}
    \label{fig:genetic_algorithm_flowchart}
\end{figure}

\section{算法的具体步骤}

1. 利用扫描算法将省会城市分组

其具体步骤如下:

\textbf{Step1:} 建立以出发点(西安)为原点的直角坐标系,旋转以出发点为中心、以纵坐标为边的射线对省会城市进行扫描,扫描到的第 $i$ 个省会省市作为 $v_{i}^{'},$ 记 $\Delta_{ip}^{'}=Q_{p}^{'}-v_{i}^{'}$ 。将 $\Delta_{ip}^{'}$ 中最小正值或 0 值对应的车辆分配给该省会城市。

\textbf{Step2:} 继续扫描省会城市,直到扫描到的省会城市的需求大于该类型的车辆载重,即 $\Delta_{ip}^{'}<0$ 为止;

\textbf{Step3:} 返回 Step1 进行下一轮的扫描。

这样就可将省会省市分成了几组,可以对每组省会城市进行路径的设计,将 MVRP 问题转化为几个相对简单的 VRP 问题。

2. 利用插入算法构造一个初步优化路径

设 $V=\left(v_{0}, v_{1}, v_{2}, \ldots, v_{n}\right)$ 是省会城市的集合,$S$ 为插入法已经得到的路径,其具体的操作为:

\textbf{Step1:} 选择出发市以外的任意一点作为起始路径,$S=\left\{v_{0}, v_{1}^{'}\right\}$;

\textbf{Step2:} 若 $S=V$ 则停止计算;否则确定 $v_{j}\left(v_{j} \in V \backslash S\right)$ 和点 $v_{i} 、 v_{k}\left(v_{i}, v_{k} \in S\right)$,使得 $d_{ij}+d_{jk}-d_{ik}$ 的值最小;

\textbf{Step3:} 将 $v_{i} 、 v_{k}$ 之间的路线断开,插入 $v_{j}$ 得到新的路径 $S=\left\{\cdots, v_{i}, v_{j}, v_{k}, \cdots\right\}$,返回 Step2。

3. 对每个省会城市组的 TSP 问题分别用遗传算法

(1) 编码

这里我们采用简单易用的整数编码方法。对于有 $n$ 个访问点的 TSP 问题,将省会城市从 1 到 $n$ 进行编号,每个从 1 到 $n$ 的全排列都可以作为该问题的一个解,即染色体个体。例如有 5 个省会城市的 TSP 问题,$\{3, 5, 4, 2, 1\}$ 就是一个合法的染色体,表示车辆对 5 个省会城市的访问路线是 3-5-4-2-1。

(2) 初始解的产生,即用上述插入法得到初始解。

(3) 适应度函数的设置

设 $\left\{k_{1}, k_{2}, \ldots, k_{n}\right\}$ 为一个采用整数编码的染色体,$d_{k_{i} k_{j}}$ 表示城市 $k_{i}$ 到城市 $k_{j}$ 的行程距离,则个体的适应度函数为:

\[
F=\frac{1}{\sum_{i=1}^{n-1} d_{k_{i} k_{j}}+d_{k_{n} k_{l}}}
\]

个体适应度越大被选择的概率越大,而 TSP 的目标函数是求总行程时间最短,所以,适应度函数取目标函数的导数。

(4) 选择

这里我们采用简单的轮盘赌方法

\[
p_{si}=\frac{f_{i}}{\sum_{j=1}^{M} f_{i}}
\]

(5) 交叉

在遗传算法中,对于整数编码的方法使用部分匹配、顺序交叉等,以解决交叉过程中,染色体的基因重复问题。还可以采用多种方法结合使用,以加强算子的作用。此处我们采用部分匹配法和顺序交叉相结合的方法。

\section{部分匹配法}

随机在两个父代染色体中各自选取两个交叉点,对两交叉点中间部分的染色体段内的元素定义一系列交换规则,并实施交换,得到两个新的染色体例如:有两条父代染色体

\begin{align*}
A &= 5 \, 1 \, | \, 4 \, 3 \, 2 \\
B &= 3 \, 2 \, | \, 1 \, 4 \, 5
\end{align*}

所选染色体片段内元素的交换规则为 $4 \leftrightarrow 1 \, | \, 3 \leftrightarrow 4 \, | \, 2 \leftrightarrow 5$。其中 4 是 1, 3 之间的连接元素,所以交换变 $1 \leftrightarrow 3 \, | \, 2 \leftrightarrow 5$,产生的子代为:

\begin{align*}
A' &= 2 \, 3 \, | \, 1 \, 4 \, 5 \\
B' &= 1 \, 5 \, | \, 3 \, 4 \, 2
\end{align*}

\section{顺序交叉法}

在染色体 $A'$ 和 $B'$ 中随机选择一个染色体片段,将选中的片段复制到子代中。将第二个母体中选中染色体片段后的码作为第一个母体染色体的初始码,将第一个母体中选中染色体片段后的码作为第二个母体染色体的初始码,分别按照原顺序重新排列。删除选中染色体片段中已经存在的码,得到两个染色体片段和。这两个片段中的元素从 $A'$ 和 $B'$ 的第二个交叉点后开始填充,得到新个体。将 $A''$ 和 $B''$ 作为最终的交叉自代染色体。

\section{变异}

采用位移变换的方法,如 $A'' = 5 \, 1 \, 4 \, 3 \, 2$ 被选中发生变异,其中 1, 4 被选中做出变换,其余位置上的元素保持不变,对选中的元素进行位移循环操作,可能产生的新个体 $A'' = 5 \, 4 \, 1 \, 3 \, 2$。

\section{逆转操作}

如对 $A'' = 5 \, 1 \, 4 \, 3 \, 2$ 做进化逆转操作,选中逆转区域为 $|1 \, 4 \, 3|$,将选中区域的元素反序插入到原来的位置得到最终个体 $A''' = 5 \, 3 \, 4 \, 1 \, 2$。

\section{加入时间因素、综合车型因素对路线做出调整}

\textbf{Step1:} 沿着 3 中计算的 VRP 路径,结合初始分派的车辆行驶速度,对省会城市组 $r$ 内的省会城市进行时间窗检验。若有车辆不满足时间窗的要求,则将此客户从该条路径中删除、暂时忽略,然后检验路径中下一项任务是否满足时间窗。当所有省会城市的时间窗检验完毕时,就在 3 中计算的 VRP 路径的基础上构造出一个可行的新子路径。

\textbf{Step2:} 找出省会城市组 $r$ 中被忽略的省会城市所需任务量最大的城市,记为 $g_{rmax}$,将剩余车辆中的载重与 $g_{rmax}$ 做差,将差值计做 $\Delta_r$,并对差值进行排序,选出正的最小差值对应的车辆分配给该省会城市;

\textbf{Step3:} 沿着 3 中计算的 VRP 路径,按照 Step1 中的方法,检验 Step2 中车辆能够服务的被忽略的省会城市;

\textbf{Step4:} 返回 Step2,直到该省会城市组内所有的省会城市都分配到车辆为止。

\begin{tabular}{l l}
\textbf{符号} & \textbf{定义} \\
\hline
$F=\{1,2,\ldots,n\}$ & 表示省会城市 $m$ 辐射区内与景点邻近的城市点集,其中 $i=1$ 表示省会城市; \\
$T_{i}$ & 表示与 $i$ 城市邻近景点的游览时间(天); \\
$T_{mi}$ & 表示与 $i$ 城市邻近景点的最少游览时间(天); \\
$T_{ij}$ & 表示游客驾车从 $i$ 城市邻近景点到 $j$ 城市邻近景点的驾车行驶时间(天); \\
$PT_{ij}$ & 表示游客驾车从 $i$ 城市到达邻近景点的驾车行驶时间(天); \\
$S_{ij}$ & 表示游客驾车从 $i$ 景点邻近城市到 $j$ 景点邻近城市的驾车行驶距离(公里); \\
$X_{ij}$ & 表示 0-1 决策变量,游客从 $i$ 城市邻近景点到 $j$ 城市邻近景点参观时,$X_{ij}=1$ \\
$S$ & 表示集合 $S$ 中的元素个数; \\
$AT_{i}$ & 表示游客抵达 $i$ 城市邻近景点的时刻; \\
$e_{i}$ & 表示为允许最早游览时刻,由题意知景点开放时间为 8:00-18:00,取 $e_{i}=8:00$ \\
$l_{i}$ & 表示为允许最迟游览时刻,即景点开放结束时间与景点最少游览时间之差,有 $l_{i}=18-T_{mi}$; \\
$[e_{i},l_{i}]$ & 表示景点 $i$ 的时间窗; \\
$DT_{i}$ & 表示游客从 $i$ 邻近城市景点驾驶出发时刻; \\
$p_{i}$ & 表示为允许最早行车时刻,由题意知行车时间 7:00-19:00,取 $p_{i}=7:00$; \\
$q_{i}$ & 表示为允许最迟行车时刻,由题意知行车时间 7:00-19:00,取 $q_{i}=19:00$; \\
$[p_{i},q_{i}]$ & 表示景点 $i$ 的时间窗; \\
$F^{*}=\{1,2,\ldots,m\}$ & 表示省会城市点集,其中 $i=1$ 表示出发地西安,$m=31$; \\
$T_{i}^{*}$ & 表示 $i$ 省会城市所有游览次数总逗留时间,即下层优化求解结果(表 1)(天); \\
$T_{iyk}$ & 表示 $i$ 省会城市在第 $y$ 年的第 $k$ 次游览时的逗留时间(天); \\
$T_{ijyk}$ & 表示游客第 $y$ 年的第 $k$ 次从 $i$ 省会城市到 $j$ 省会城市的驾车行驶时间(天); \\
\end{tabular}

\begin{tabular}{r l l r r}
\hline
15 & 北京-西安途中 & 无 & 4.0 & 360 \\
1 & 西安 & 无 & 8.0 & 720 \\
2 & 西安-乌鲁木齐途中 & 无 & 8.0 & 720 \\
3 & 西安-乌鲁木齐途中 & 无 & 8.0 & 720 \\
4 & 西安-乌鲁木齐途中 & 乌鲁木齐 & 4.2 & 378 \\
5 & 乌鲁木齐 & 乌鲁木齐 & 3.0 & 200 \\
6 & 乌鲁木齐 & 昌吉州阜康市天山天池风景名胜区 & 1.0 & 60 \\
7 & 乌鲁木齐 & 阿勒泰地区富蕴县可可托海景区 & 1.0 & 60 \\
8 & 乌鲁木齐 & 乌鲁木齐天山大峡谷 & 0.6 & 30 \\
9 & 乌鲁木齐 & 无 & 7.0 & 460 \\
10 & 乌鲁木齐 & 巴音郭楞蒙古自治州博湖县博斯腾湖景区 & 0.0 & 0 \\
11 & 乌鲁木齐 & 无 & 7.0 & 460 \\
12 & 乌鲁木齐 & 无 & 8.0 & 720 \\
13 & 乌鲁木齐-西安途中 & 无 & 8.0 & 720 \\
14 & 乌鲁木齐-西安途中 & 无 & 8.0 & 720 \\
15 & 乌鲁木齐-西安途中 & 无 & 4.2 & 378 \\
1 & 西安 & 无 & 8.0 & 720 \\
2 & 西安-乌鲁木齐途中 & 无 & 8.0 & 720 \\
3 & 西安-乌鲁木齐途中 & 无 & 8.0 & 720 \\
4 & 西安-乌鲁木齐途中 & 无 & 8.0 & 720 \\
5 & 乌鲁木齐-喀什途中 & 无 & 8.0 & 720 \\
6 & 乌鲁木齐-喀什途中 & 无 & 5.7 & 285 \\
7 & 喀什 & 喀什地区泽普县金胡杨景区 & 0.0 & 0 \\
8 & 喀什 & 喀什地区噶尔老城景区 & 3.0 & 210 \\
9 & 喀什 & 无 & 8.0 & 720 \\
10 & 喀什-阿勒泰途中 & 无 & 8.0 & 720 \\
11 & 喀什-阿勒泰途中 & 阿勒泰地区布尔津县喀纳斯景区 & 4.0 & 158 \\
12 & 阿勒泰 & 无 & 8.0 & 720 \\
13 & 阿勒泰-西安途中 & 无 & 8.0 & 720 \\
14 & 阿勒泰-西安途中 & 无 & 8.0 & 720 \\
14.5 & 阿勒泰-西安途中 & 无 & 3.5 & 315 \\
1 & 西安 & 无 & 8.0 & 720 \\
2 & 西安-乌鲁木齐途中 & 无 & 8.0 & 720 \\
3 & 西安-乌鲁木齐途中 & 无 & 8.0 & 720 \\
4 & 西安-乌鲁木齐途中 & 无 & 8.0 & 720 \\
5 & 乌鲁木齐 & 吐鲁番葡萄沟风景区 & 3.0 & 200 \\
6 & 乌鲁木齐-石河子途中 & 伊犁地区新源县那拉提旅游风景区 & 4.4 & 176 \\
\hline
\end{tabular}

\begin{tabular}{r l l r r}
\hline
7 & 石河子 & 伊犁地区新源县那拉提旅游 & 0.0 & 0 \\
 & & 风景区 & & \\
\hline
8 & 石河子 & 无 & 8.0 & 720 \\
\hline
9 & 石河子-西安途中 & 无 & 8.0 & 720 \\
\hline
10 & 石河子-西安途中 & 无 & 8.0 & 720 \\
\hline
11 & 石河子-西安途中 & 无 & 5.8 & 525.3 \\
\hline
1 & 西安 & 无 & 8.0 & 720 \\
\hline
2 & 西安-银川途中 & 银川 & 3.2 & 128 \\
\hline
3 & 银川 & 银川、石嘴山平罗县沙湖旅游区 & 1.0 & 60 \\
\hline
4 & 银川 & 中卫沙坡头旅游景区 & 3.0 & 230 \\
\hline
5 & 银川 & 银川镇北堡西部影视城、银川市灵武水洞沟旅游区 & 2.0 & 70 \\
\hline
6 & 银川 & 无 & 8.0 & 720 \\
\hline
6.5 & 银川-西安途中 & 无 & 3.2 & 128 \\
\hline
1 & 西安 & 无 & 5.3 & 480 \\
\hline
2 & 郑州 & 郑州 & 0.0 & 0 \\
\hline
3 & 郑州 & 郑州登封嵩山少林景区 & 0.8 & 70.8 \\
\hline
4 & 登封市 & 焦作(云台山-神农山-青天河)风景区 & 2.0 & 100 \\
\hline
5 & 登封市 & 焦作(云台山-神农山-青天河)风景区 & 0.0 & 0 \\
\hline
6 & 登封市 & 洛阳龙门石窟景区、洛阳嵩县白云山景区 & 3.8 & 243.1 \\
\hline
7 & 洛阳市 & 洛阳栾川县老君山-鸡冠洞旅游区 & 2.5 & 150 \\
\hline
8 & 洛阳市 & 洛阳新安县龙潭大峡谷景区 & 2.0 & 90 \\
\hline
9 & 洛阳市 & 平顶山鲁山县尧山-中原大佛景区 & 3.0 & 214 \\
\hline
10 & 洛阳市 & 南阳西峡伏牛山老界岭-恐龙遗址园旅游区 & 3.0 & 160 \\
\hline
11 & 平顶山 & 开封清明上河园景区、安阳殷墟景区 & 2.3 & 210 \\
\hline
12 & 安阳 & 无 & 7.3 & 661.1 \\
\hline
1 & 西安 & 无 & 8.0 & 720 \\
\hline
2 & 西安-哈尔滨途中 & 无 & 8.0 & 720 \\
\hline
3 & 西安-哈尔滨途中 & 无 & 8.0 & 720 \\
\hline
4 & 西安-哈尔滨途中 & 哈尔滨 & 1.6 & 150 \\
\hline
5 & 哈尔滨 & 哈尔滨太阳岛景区 & 4.5 & 370 \\
\hline
6 & 哈尔滨 & 黑河五大连池景区 & 5.0 & 410 \\
\hline
7 & 哈尔滨 & 牡丹江宁安市镜泊湖景区 & 5.0 & 390 \\
\hline
8 & 伊春市 & 伊春市汤旺河林海奇石景区 & 4.1 & 324 \\
\hline
9 & 五大连池市 & 无 & 8.0 & 320 \\
\hline
\end{tabular}

通过算法设计和借助 MATLAB 程序(主要程序详见附录 2),得到 31 个省会城市的出游年数最少的旅游方案,即全国 201 个 5A 景区的出游方案,出游总时间为 11 年零 10 天,如表 4 和图 4 所示。其中,图 4 中圆圈的大小表示每个省会城市的辐射内景点的游览总时间。每一天的出发地、行车时间、行车里程、游览景区等具体信息如表 5 所示。

\begin{figure}[h]
    \centering
    \includegraphics[width=\textwidth]{image.png}
    \caption{省会城市最短时间旅游计划(出发地:西安)}
    \label{fig:travel_plan}
\end{figure}

\begin{tabular}{r l l r r}
\hline
10 & 五大连池市 & 无 & 8.0 & 320 \\
\hline
11 & 哈尔滨 & 大兴安岭地区漠河北极村旅游区 & 3.0 & 310 \\
\hline
12 & 哈尔滨-天津途中 & 无 & 8.0 & 720 \\
\hline
13 & 哈尔滨-天津途中 & 天津 & 4.9 & 441 \\
\hline
14 & 天津 & 天津、天津蓟县盘山风景名胜区 & 4.2 & 380 \\
\hline
14.5 & 天津-西安途中 & 无 & 8.0 & 720 \\
\hline
1 & 西安 & 无 & 8 & 720 \\
\hline
2 & 西安-长春途中 & 无 & 8 & 720 \\
\hline
3 & 西安-长春途中 & 无 & 6.7 & 600 \\
\hline
4 & 长春 & 长春伪满皇宫博物馆, 长春净月潭景区 & 0 & 0 \\
\hline
5 & 长春 & 长春市长影世纪城景区 & 0 & 0 \\
\hline
6 & 长春 & 无 & 7 & 420 \\
\hline
7 & 长白山景区 & 长白山景区 & 0 & 0 \\
\hline
8 & 长白山景区-沈阳途中 & 无 & 9 & 800 \\
\hline
9 & 沈阳 & 沈阳 & 0 & 0 \\
\hline
10 & 沈阳 & 本溪市本溪水洞景区 & 2.5 & 100 \\
\hline
11 & 沈阳 & 沈阳植物园 & 4.3 & 387 \\
\hline
12 & 沈阳-大连途中 & 大连老虎滩海洋公园——老虎滩极地馆 & 0 & 0 \\
\hline
13 & 大连 & 大连金石滩景区(地质公园-发现王国-蜡像馆-文化博览广场) & 1 & 50 \\
\hline
14 & 大连-西安途中 & 无 & 0 & 0 \\
\hline
15 & 大连-西安途中 & 无 & 0 & 0 \\
\hline
1 & 西安 & 无 & 8 & 720 \\
\hline
2 & 西安-呼和浩特途中 & 无 & 2.8 & 250 \\
\hline
3 & 呼和浩特 & 呼和浩特 & 1 & 40 \\
\hline
4 & 鄂尔多斯 & 鄂尔多斯伊金霍洛旗成吉思汗陵旅游区 & 0 & 0 \\
\hline
5 & 鄂尔多斯 & 鄂尔多斯达拉特旗响沙湾旅游景区 & 1.5 & 90 \\
\hline
6 & 鄂尔多斯-大同途中 & 大同云冈石窟 & 4.3 & 390 \\
\hline
7 & 大同 & 忻州五台山风景名胜区 & 2.6 & 235 \\
\hline
8 & 五台县 & 晋中市乔家大院文化园区 & 2.5 & 228 \\
\hline
9 & 太原 & 太原 & 0 & 0 \\
\hline
10 & 晋中市平遥县途中 & 晋中市平遥县平遥古城景区, 晋中市介休市绵山风景名胜区 & 1 & 30 \\
\hline
11 & 介休 & 晋城阳城县皇城相府生态文化旅游区 & 3.4 & 305 \\
\hline
 & & 24 & & \\
\hline
\end{tabular}

\begin{tabular}{r l l r r}
\hline
1 & 西安 & 无 & 8 & 720 \\
\hline
2 & 西安-南京途中 & 南京钟山—中山陵风景名胜区(明孝陵-音乐台-灵谷寺-梅花山-紫金山天文台) & 4 & 360 \\
\hline
3 & 南京 & 南京夫子庙—秦淮河风光带(江南贡院-白鹭洲-中华门-瞻园-王谢故居) & 0 & 0 \\
\hline
4 & 南京 & 镇江三山风景名胜区(金山—北固山—焦山) & 1 & 86 \\
\hline
5 & 镇江 & 镇江句容茅山景区 & 2 & 146 \\
\hline
6 & 常州 & 常州溧阳市天目湖景区(天目湖-南山竹海-御水温泉) & 0 & 0 \\
\hline
7 & 常州 & 常州环球恐龙城景区(中华恐龙园-恐龙谷温泉-恐龙城大剧院) & 1 & 67 \\
\hline
8 & 无锡 & 中央电视台无锡影视基地三国水浒城景区 & 0 & 0 \\
\hline
9 & 无锡 & 无锡鼋头渚景区 & 5 & 400 \\
\hline
10 & 无锡-西安途中 & 无锡灵山大佛景区 & 8 & 720 \\
\hline
1 & 西安 & 无 & 8 & 720 \\
\hline
2 & 西安-苏州途中 & 无 & 6.5 & 580 \\
\hline
3 & 苏州 & 苏州园林(拙政园—留园—虎丘) & 1 & 45 \\
\hline
4 & 苏州 & 苏州昆山周庄古镇景区 & 0 & 0 \\
\hline
5 & 苏州 & 苏州吴江同里古镇景区 & 1 & 30 \\
\hline
6 & 苏州 & 苏州市金鸡湖国家商务旅游示范区 & 0 & 0 \\
\hline
7 & 苏州 & 苏州吴中太湖旅游区(旺山—穹窿山—东山) & 1 & 109 \\
\hline
8 & 南通 & 苏州常熟沙家浜—虞山尚湖旅游区 & 2.5 & 165 \\
\hline
9 & 泰州 & 泰州姜堰区溱湖国家湿地公园 & 1 & 69 \\
\hline
10 & 扬州 & 淮安市周恩来故里景区(周恩来纪念馆-周恩来故居-附马巷历史街区-河下古镇) & 2 & 180 \\
\hline
11 & 淮安 & 无 & 8 & 720 \\
\hline
12 & 淮安-西安 & 无 & 6.7 & 600 \\
\hline
1 & 西安 & 无 & 8 & 750 \\
\hline
2 & 成都 & 成都 & 0 & 0 \\
\hline
\end{tabular}

\begin{tabular}{r l l r r}
\hline
3 & 成都 & 成都青城山一都江堰旅游区 & 1.5 & 68 \\
\hline
4 & 成都 & 阿坝藏族羌族自治州汶川特别旅游区(震中映秀一水磨古镇一三江生态旅游区) & 2 & 110 \\
\hline
5 & 成都 & 乐山峨眉山景区 & 1.5 & 138.6 \\
\hline
6 & 乐山 & 乐山乐山大佛景区 & 0 & 0 \\
\hline
7 & 乐山 & 广安市邓小平故里旅游区 & 4.5 & 417.1 \\
\hline
8 & 广安 & 南充市阆中古城旅游景区 & 1 & 83.5 \\
\hline
9 & 南充 & 无 & 6.5 & 300 \\
\hline
10 & 广元 & 阿坝藏族羌族自治州九寨沟景区 & 0 & 0 \\
\hline
11 & 广元 & 无 & 6.5 & 300 \\
\hline
12 & 四川省广元市剑阁县 & 广元市剑门蜀道剑门关旅游景区 & 2 & 177 \\
\hline
13 & 绵阳 & 无 & 5 & 440 \\
\hline
14 & 阿坝藏族 & 阿坝藏族羌族自治州松潘县黄龙风景名胜区 & 3 & 130 \\
\hline
15 & 阿坝藏族-西安途中 & 无 & 8 & 750 \\
\hline
1 & 西安 & 无 & 7.8 & 700 \\
\hline
2 & 重庆 & 大足石刻景区 & 1.5 & 100 \\
\hline
3 & 重庆 & 武隆喀斯特旅游区(天生三硚、仙女山、芙蓉洞) & 3 & 190 \\
\hline
4 & 重庆市武隆县 & 武隆喀斯特旅游区(天生三硚、仙女山、芙蓉洞) & 0 & 0 \\
\hline
5 & 重庆市万盛区黑山镇 & 万盛黑山谷-龙鳞石海风景区 & 2 & 120 \\
\hline
6 & 重庆市南川区 & 南川金佛山一神龙峡风景区 & 2 & 110 \\
\hline
7 & 重庆市武隆县 & 酉阳桃花源旅游景区 & 0 & 0 \\
\hline
8 & 重庆-贵阳 & 无 & 4.2 & 380 \\
\hline
9 & 贵阳 & 贵阳 & 2.5 & 220 \\
\hline
10 & 毕节 & 毕节市百里杜鹃景区 & 3 & 250 \\
\hline
11 & 毕节-安顺途中 & 安顺镇宁县黄果树瀑布景区 & 1 & 90 \\
\hline
12 & 安顺 & 安顺龙宫景区 & 5 & 450 \\
\hline
13 & 河池 & 黔南布依族苗族自治州荔波樟江景区 & 2.5 & 190 \\
\hline
14 & 河池 & 无 & 8 & 720 \\
\hline
15 & 河池-西安途中 & 无 & 0 & 0 \\
\hline
1 & 西安 & 无 & 8 & 740 \\
\hline
2 & 武汉 & 武汉黄鹤楼公园武汉市东湖景区 & 0 & 0 \\
\hline
3 & 武汉 & 武汉市黄陂木兰文化生态旅游区 & 1.5 & 70 \\
\hline
\end{tabular}

\begin{tabular}{r l l r r}
\hline
15 & 北京-西安途中 & 无 & 4.0 & 360 \\
1 & 西安 & 无 & 8.0 & 720 \\
2 & 西安-乌鲁木齐途中 & 无 & 8.0 & 720 \\
3 & 西安-乌鲁木齐途中 & 无 & 8.0 & 720 \\
4 & 西安-乌鲁木齐途中 & 乌鲁木齐 & 4.2 & 378 \\
5 & 乌鲁木齐 & 乌鲁木齐 & 3.0 & 200 \\
6 & 乌鲁木齐 & 昌吉州阜康市天山天池风景名胜区 & 1.0 & 60 \\
7 & 乌鲁木齐 & 阿勒泰地区富蕴县可可托海景区 & 1.0 & 60 \\
8 & 乌鲁木齐 & 乌鲁木齐天山大峡谷 & 0.6 & 30 \\
9 & 乌鲁木齐 & 无 & 7.0 & 460 \\
10 & 乌鲁木齐 & 巴音郭楞蒙古自治州博湖县博斯腾湖景区 & 0.0 & 0 \\
11 & 乌鲁木齐 & 无 & 7.0 & 460 \\
12 & 乌鲁木齐 & 无 & 8.0 & 720 \\
13 & 乌鲁木齐-西安途中 & 无 & 8.0 & 720 \\
14 & 乌鲁木齐-西安途中 & 无 & 8.0 & 720 \\
15 & 乌鲁木齐-西安途中 & 无 & 4.2 & 378 \\
1 & 西安 & 无 & 8.0 & 720 \\
2 & 西安-乌鲁木齐途中 & 无 & 8.0 & 720 \\
3 & 西安-乌鲁木齐途中 & 无 & 8.0 & 720 \\
4 & 西安-乌鲁木齐途中 & 无 & 8.0 & 720 \\
5 & 乌鲁木齐-喀什途中 & 无 & 8.0 & 720 \\
6 & 乌鲁木齐-喀什途中 & 无 & 5.7 & 285 \\
7 & 喀什 & 喀什地区泽普县金胡杨景区 & 0.0 & 0 \\
8 & 喀什 & 喀什地区噶尔老城景区 & 3.0 & 210 \\
9 & 喀什 & 无 & 8.0 & 720 \\
10 & 喀什-阿勒泰途中 & 无 & 8.0 & 720 \\
11 & 喀什-阿勒泰途中 & 阿勒泰地区布尔津县喀纳斯景区 & 4.0 & 158 \\
12 & 阿勒泰 & 无 & 8.0 & 720 \\
13 & 阿勒泰-西安途中 & 无 & 8.0 & 720 \\
14 & 阿勒泰-西安途中 & 无 & 8.0 & 720 \\
14.5 & 阿勒泰-西安途中 & 无 & 3.5 & 315 \\
1 & 西安 & 无 & 8.0 & 720 \\
2 & 西安-乌鲁木齐途中 & 无 & 8.0 & 720 \\
3 & 西安-乌鲁木齐途中 & 无 & 8.0 & 720 \\
4 & 西安-乌鲁木齐途中 & 无 & 8.0 & 720 \\
5 & 乌鲁木齐 & 吐鲁番葡萄沟风景区 & 3.0 & 200 \\
6 & 乌鲁木齐-石河子途中 & 伊犁地区新源县那拉提旅游风景区 & 4.4 & 176 \\
\hline
\end{tabular}

\begin{tabular}{r l l r r}
\hline
7 & 石河子 & 伊犁地区新源县那拉提旅游 & 0.0 & 0 \\
 & & 风景区 & & \\
\hline
8 & 石河子 & 无 & 8.0 & 720 \\
\hline
9 & 石河子-西安途中 & 无 & 8.0 & 720 \\
\hline
10 & 石河子-西安途中 & 无 & 8.0 & 720 \\
\hline
11 & 石河子-西安途中 & 无 & 5.8 & 525.3 \\
\hline
1 & 西安 & 无 & 8.0 & 720 \\
\hline
2 & 西安-银川途中 & 银川 & 3.2 & 128 \\
\hline
3 & 银川 & 银川、石嘴山平罗县沙湖旅游区 & 1.0 & 60 \\
\hline
4 & 银川 & 中卫沙坡头旅游景区 & 3.0 & 230 \\
\hline
5 & 银川 & 银川镇北堡西部影视城、银川市灵武水洞沟旅游区 & 2.0 & 70 \\
\hline
6 & 银川 & 无 & 8.0 & 720 \\
\hline
6.5 & 银川-西安途中 & 无 & 3.2 & 128 \\
\hline
1 & 西安 & 无 & 5.3 & 480 \\
\hline
2 & 郑州 & 郑州 & 0.0 & 0 \\
\hline
3 & 郑州 & 郑州登封嵩山少林景区 & 0.8 & 70.8 \\
\hline
4 & 登封市 & 焦作(云台山-神农山-青天河)风景区 & 2.0 & 100 \\
\hline
5 & 登封市 & 焦作(云台山-神农山-青天河)风景区 & 0.0 & 0 \\
\hline
6 & 登封市 & 洛阳龙门石窟景区、洛阳嵩县白云山景区 & 3.8 & 243.1 \\
\hline
7 & 洛阳市 & 洛阳栾川县老君山-鸡冠洞旅游区 & 2.5 & 150 \\
\hline
8 & 洛阳市 & 洛阳新安县龙潭大峡谷景区 & 2.0 & 90 \\
\hline
9 & 洛阳市 & 平顶山鲁山县尧山-中原大佛景区 & 3.0 & 214 \\
\hline
10 & 洛阳市 & 南阳西峡伏牛山老界岭-恐龙遗址园旅游区 & 3.0 & 160 \\
\hline
11 & 平顶山 & 开封清明上河园景区、安阳殷墟景区 & 2.3 & 210 \\
\hline
12 & 安阳 & 无 & 7.3 & 661.1 \\
\hline
1 & 西安 & 无 & 8.0 & 720 \\
\hline
2 & 西安-哈尔滨途中 & 无 & 8.0 & 720 \\
\hline
3 & 西安-哈尔滨途中 & 无 & 8.0 & 720 \\
\hline
4 & 西安-哈尔滨途中 & 哈尔滨 & 1.6 & 150 \\
\hline
5 & 哈尔滨 & 哈尔滨太阳岛景区 & 4.5 & 370 \\
\hline
6 & 哈尔滨 & 黑河五大连池景区 & 5.0 & 410 \\
\hline
7 & 哈尔滨 & 牡丹江宁安市镜泊湖景区 & 5.0 & 390 \\
\hline
8 & 伊春市 & 伊春市汤旺河林海奇石景区 & 4.1 & 324 \\
\hline
9 & 五大连池市 & 无 & 8.0 & 320 \\
\hline
\end{tabular}

\begin{tabular}{r l l r r}
\hline
10 & 五大连池市 & 无 & 8.0 & 320 \\
\hline
11 & 哈尔滨 & 大兴安岭地区漠河北极村旅游区 & 3.0 & 310 \\
\hline
12 & 哈尔滨-天津途中 & 无 & 8.0 & 720 \\
\hline
13 & 哈尔滨-天津途中 & 天津 & 4.9 & 441 \\
\hline
14 & 天津 & 天津、天津蓟县盘山风景名胜区 & 4.2 & 380 \\
\hline
14.5 & 天津-西安途中 & 无 & 8.0 & 720 \\
\hline
1 & 西安 & 无 & 8 & 720 \\
\hline
2 & 西安-长春途中 & 无 & 8 & 720 \\
\hline
3 & 西安-长春途中 & 无 & 6.7 & 600 \\
\hline
4 & 长春 & 长春伪满皇宫博物馆, 长春净月潭景区 & 0 & 0 \\
\hline
5 & 长春 & 长春市长影世纪城景区 & 0 & 0 \\
\hline
6 & 长春 & 无 & 7 & 420 \\
\hline
7 & 长白山景区 & 长白山景区 & 0 & 0 \\
\hline
8 & 长白山景区-沈阳途中 & 无 & 9 & 800 \\
\hline
9 & 沈阳 & 沈阳 & 0 & 0 \\
\hline
10 & 沈阳 & 本溪市本溪水洞景区 & 2.5 & 100 \\
\hline
11 & 沈阳 & 沈阳植物园 & 4.3 & 387 \\
\hline
12 & 沈阳-大连途中 & 大连老虎滩海洋公园——老虎滩极地馆 & 0 & 0 \\
\hline
13 & 大连 & 大连金石滩景区(地质公园-发现王国-蜡像馆-文化博览广场) & 1 & 50 \\
\hline
14 & 大连-西安途中 & 无 & 0 & 0 \\
\hline
15 & 大连-西安途中 & 无 & 0 & 0 \\
\hline
1 & 西安 & 无 & 8 & 720 \\
\hline
2 & 西安-呼和浩特途中 & 无 & 2.8 & 250 \\
\hline
3 & 呼和浩特 & 呼和浩特 & 1 & 40 \\
\hline
4 & 鄂尔多斯 & 鄂尔多斯伊金霍洛旗成吉思汗陵旅游区 & 0 & 0 \\
\hline
5 & 鄂尔多斯 & 鄂尔多斯达拉特旗响沙湾旅游景区 & 1.5 & 90 \\
\hline
6 & 鄂尔多斯-大同途中 & 大同云冈石窟 & 4.3 & 390 \\
\hline
7 & 大同 & 忻州五台山风景名胜区 & 2.6 & 235 \\
\hline
8 & 五台县 & 晋中市乔家大院文化园区 & 2.5 & 228 \\
\hline
9 & 太原 & 太原 & 0 & 0 \\
\hline
10 & 晋中市平遥县途中 & 晋中市平遥县平遥古城景区, 晋中市介休市绵山风景名胜区 & 1 & 30 \\
\hline
11 & 介休 & 晋城阳城县皇城相府生态文化旅游区 & 3.4 & 305 \\
\hline
 & & 24 & & \\
\hline
\end{tabular}

\begin{tabular}{r l l r r}
\hline
1 & 西安 & 无 & 8 & 720 \\
\hline
2 & 西安-南京途中 & 南京钟山—中山陵风景名胜区(明孝陵-音乐台-灵谷寺-梅花山-紫金山天文台) & 4 & 360 \\
\hline
3 & 南京 & 南京夫子庙—秦淮河风光带(江南贡院-白鹭洲-中华门-瞻园-王谢故居) & 0 & 0 \\
\hline
4 & 南京 & 镇江三山风景名胜区(金山—北固山—焦山) & 1 & 86 \\
\hline
5 & 镇江 & 镇江句容茅山景区 & 2 & 146 \\
\hline
6 & 常州 & 常州溧阳市天目湖景区(天目湖-南山竹海-御水温泉) & 0 & 0 \\
\hline
7 & 常州 & 常州环球恐龙城景区(中华恐龙园-恐龙谷温泉-恐龙城大剧院) & 1 & 67 \\
\hline
8 & 无锡 & 中央电视台无锡影视基地三国水浒城景区 & 0 & 0 \\
\hline
9 & 无锡 & 无锡鼋头渚景区 & 5 & 400 \\
\hline
10 & 无锡-西安途中 & 无锡灵山大佛景区 & 8 & 720 \\
\hline
1 & 西安 & 无 & 8 & 720 \\
\hline
2 & 西安-苏州途中 & 无 & 6.5 & 580 \\
\hline
3 & 苏州 & 苏州园林(拙政园—留园—虎丘) & 1 & 45 \\
\hline
4 & 苏州 & 苏州昆山周庄古镇景区 & 0 & 0 \\
\hline
5 & 苏州 & 苏州吴江同里古镇景区 & 1 & 30 \\
\hline
6 & 苏州 & 苏州市金鸡湖国家商务旅游示范区 & 0 & 0 \\
\hline
7 & 苏州 & 苏州吴中太湖旅游区(旺山—穹窿山—东山) & 1 & 109 \\
\hline
8 & 南通 & 苏州常熟沙家浜—虞山尚湖旅游区 & 2.5 & 165 \\
\hline
9 & 泰州 & 泰州姜堰区溱湖国家湿地公园 & 1 & 69 \\
\hline
10 & 扬州 & 淮安市周恩来故里景区(周恩来纪念馆-周恩来故居-附马巷历史街区-河下古镇) & 2 & 180 \\
\hline
11 & 淮安 & 无 & 8 & 720 \\
\hline
12 & 淮安-西安 & 无 & 6.7 & 600 \\
\hline
1 & 西安 & 无 & 8 & 750 \\
\hline
2 & 成都 & 成都 & 0 & 0 \\
\hline
\end{tabular}

\begin{tabular}{r l l r r}
\hline
3 & 成都 & 成都青城山一都江堰旅游区 & 1.5 & 68 \\
\hline
4 & 成都 & 阿坝藏族羌族自治州汶川特别旅游区(震中映秀一水磨古镇一三江生态旅游区) & 2 & 110 \\
\hline
5 & 成都 & 乐山峨眉山景区 & 1.5 & 138.6 \\
\hline
6 & 乐山 & 乐山乐山大佛景区 & 0 & 0 \\
\hline
7 & 乐山 & 广安市邓小平故里旅游区 & 4.5 & 417.1 \\
\hline
8 & 广安 & 南充市阆中古城旅游景区 & 1 & 83.5 \\
\hline
9 & 南充 & 无 & 6.5 & 300 \\
\hline
10 & 广元 & 阿坝藏族羌族自治州九寨沟景区 & 0 & 0 \\
\hline
11 & 广元 & 无 & 6.5 & 300 \\
\hline
12 & 四川省广元市剑阁县 & 广元市剑门蜀道剑门关旅游景区 & 2 & 177 \\
\hline
13 & 绵阳 & 无 & 5 & 440 \\
\hline
14 & 阿坝藏族 & 阿坝藏族羌族自治州松潘县黄龙风景名胜区 & 3 & 130 \\
\hline
15 & 阿坝藏族-西安途中 & 无 & 8 & 750 \\
\hline
1 & 西安 & 无 & 7.8 & 700 \\
\hline
2 & 重庆 & 大足石刻景区 & 1.5 & 100 \\
\hline
3 & 重庆 & 武隆喀斯特旅游区(天生三硚、仙女山、芙蓉洞) & 3 & 190 \\
\hline
4 & 重庆市武隆县 & 武隆喀斯特旅游区(天生三硚、仙女山、芙蓉洞) & 0 & 0 \\
\hline
5 & 重庆市万盛区黑山镇 & 万盛黑山谷-龙鳞石海风景区 & 2 & 120 \\
\hline
6 & 重庆市南川区 & 南川金佛山一神龙峡风景区 & 2 & 110 \\
\hline
7 & 重庆市武隆县 & 酉阳桃花源旅游景区 & 0 & 0 \\
\hline
8 & 重庆-贵阳 & 无 & 4.2 & 380 \\
\hline
9 & 贵阳 & 贵阳 & 2.5 & 220 \\
\hline
10 & 毕节 & 毕节市百里杜鹃景区 & 3 & 250 \\
\hline
11 & 毕节-安顺途中 & 安顺镇宁县黄果树瀑布景区 & 1 & 90 \\
\hline
12 & 安顺 & 安顺龙宫景区 & 5 & 450 \\
\hline
13 & 河池 & 黔南布依族苗族自治州荔波樟江景区 & 2.5 & 190 \\
\hline
14 & 河池 & 无 & 8 & 720 \\
\hline
15 & 河池-西安途中 & 无 & 0 & 0 \\
\hline
1 & 西安 & 无 & 8 & 740 \\
\hline
2 & 武汉 & 武汉黄鹤楼公园武汉市东湖景区 & 0 & 0 \\
\hline
3 & 武汉 & 武汉市黄陂木兰文化生态旅游区 & 1.5 & 70 \\
\hline
\end{tabular}

\begin{tabular}{r l l r r}
\hline
4 & 武汉 & 无 & 3.6 & 323 \\
5 & 宜昌 & 宜昌三峡大坝旅游区 & 2.5 & 110 \\
 & & 宜昌长阳县清江画廊景区 & & \\
6 & 宜昌 & 宜昌三峡人家风景区 & 1 & 40 \\
7 & 宜昌 & 宜昌秭归县屈原故里文化旅游区 & 3.6 & 290 \\
8 & 恩施 & 恩施土家族苗族自治州恩施大峡谷景区 & 2 & 70 \\
9 & 恩施 & & 6.8 & 610 \\
10 & 十堰 & 十堰丹江口市武当山风景区 & 1 & 40 \\
11 & 十堰 & 神农架生态旅游区 & 4 & 190 \\
12 & 十堰 & 神农架生态旅游区 & 0 & 0 \\
13 & 十堰 & 神农架生态旅游区 & 4 & 380 \\
13.5 & 十堰-西安途中 & 无 & 4 & 360 \\
1 & 西安 & 无 & 8 & 720 \\
2 & 西安-杭州途中 & 杭州 & 6.7 & 610 \\
3 & 杭州 & 杭州西湖风景区 & 1.5 & 80 \\
 & & 嘉兴桐乡乌镇古镇 & & \\
4 & 杭州 & 杭州淳安千岛湖风景区 & 2.5 & 170 \\
5 & 杭州 & 金华东阳横店影视城景区 & 2 & 150 \\
 & & 杭州 & & \\
6 & 杭州 & 杭州西溪湿地旅游区 & 1.5 & 90 \\
 & & 湖州市南浔区南浔古镇景区 & & \\
7 & 嘉兴 & 嘉兴南湖旅游区 & 1.2 & 110 \\
8 & 绍兴 & 绍兴市鲁迅故里-沈园景区 & 1.3 & 117 \\
9 & 宁波 & 宁波奉化溪口-滕头旅游景区 & 1 & 90 \\
10 & 舟山 & 舟山普陀山风景区 & 0 & 0 \\
11 & 舟山 & 无 & 4 & 361 \\
12 & 温州 & 温州乐清市雁荡山风景区 & 1 & 45 \\
13 & 衢州 & 衢州市开化根宫佛国文化旅游区 & 5 & 450 \\
14 & 衢州-西安途中 & 无 & 8 & 720 \\
1 & 西安 & 无 & 8 & 720 \\
2 & 西安-济南途中 & 济南天下第一泉景区(趵突泉-大明湖-五龙潭-环城公园-黑虎泉) & 2 & 190 \\
3 & 济南 & 济南 & 0 & 0 \\
4 & 济南 & 泰安泰山景区 & 1.2 & 110 \\
5 & 泰安 & 济宁曲阜明故城三孔旅游区 & 2 & 188 \\
6 & 枣庄 & 枣庄台儿庄古城景区 & 5 & 450 \\
7 & 青岛 & 青岛崂山景区 & 2.9 & 262 \\
\hline
\end{tabular}

\begin{tabular}{r l l r r}
\hline
8 & 威海 & 威海刘公岛景区 & 1.7 & 175 \\
 & & 烟台龙口南山景区 & & \\
\hline
9 & 烟台 & 烟台蓬莱阁一三仙山一八仙 & 1 & 91 \\
 & & 过海旅游区 & & \\
\hline
10 & 蓬莱 & 无 & 2.3 & 210 \\
\hline
11 & 潍坊 & 山东沂蒙山旅游区(沂山景区一龟蒙景区一云蒙景区) & 2 & 190 \\
\hline
12 & 潍坊-西安 & 无 & 8 & 720 \\
\hline
1 & 西安 & 无 & 8 & 720 \\
\hline
2 & 合肥 & 合肥 & 3.5 & 321 \\
\hline
3 & 黄山 & 黄山市黄山风景区 & 1 & 60 \\
\hline
4 & 黄山 & 黄山市黟县皖南古村落一西递宏村 & 1.5 & 65 \\
\hline
5 & 黄山 & 宣城市绩溪县龙川景区 & 1.5 & 70 \\
\hline
 & & 黄山市古徽州文化旅游区 & & \\
18 & 6 & 黄山 & (徽州古城一牌坊群鲍家花园一唐模一港口民宅一呈 & 2.5 & 214 \\
 & & & 坎) & & \\
\hline
7 & 池州 & 池州青阳县九华山风景区 & 1 & 45 \\
\hline
8 & 安庆潜山 & 安庆潜山县天柱山风景区 & 2 & 170 \\
\hline
9 & 安庆潜山 & 六安市金寨县天堂寨旅游 & 2 & 11 \\
 & & 区 & & \\
\hline
10 & 麻城 & 阜阳市颍上县八里河风景区 & 3.5 & 313 \\
\hline
11 & 阜阳-西安途中 & 无 & 8 & 720 \\
\hline
1 & 西安 & 无 & 8 & 720 \\
\hline
2 & 西安-上海途中 & 无 & 7.4 & 666 \\
\hline
3 & 上海 & 上海野生动物园 & 0 & 0 \\
\hline
19 & 4 & 上海 & 东方明珠广播电视塔,上海科技馆 & 0 & 0 \\
\hline
5 & 上海 & 上海 & 0 & 0 \\
\hline
6 & 上海 & 无 & 8 & 720 \\
\hline
7 & 上海-西安途中 & 无 & 7.4 & 666 \\
\hline
1 & 西安 & 无 & 8 & 720 \\
\hline
2 & 西安-福州途中 & 无 & 8 & 720 \\
\hline
3 & 西安-福州途中 & 福州 & 2.3 & 210 \\
\hline
4 & 福州 & 福州市三坊七巷景区 & 3.8 & 346 \\
\hline
5 & 南平市武夷山市 & 南平武夷山风景名胜区 & 0 & 0 \\
\hline
6 & 南平市武夷山市 & 三明泰宁风景旅游区 & 2.5 & 100 \\
\hline
7 & 南平市武夷山市 & 无 & 8 & 720 \\
\hline
8 & 宁德市 & 宁德屏南(白水洋·鸳鸯溪)旅游景区 & 2.5 & 100 \\
\hline
9 & 泉州市 & 泉州市清源山风景名胜区 & 0 & 0 \\
\hline
10 & 厦门市 & 厦门鼓浪屿风景名胜区 & 0 & 0 \\
\hline
\end{tabular}

\begin{tabular}{r l l r r}
\hline
11 & 龙岩市 & 福建土楼(永定·南靖)旅游区 & 2 & 80 \\
\hline
12 & 福州 & 福州 & 6.3 & 567 \\
\hline
13 & 福州-西安途中 & 无 & 8 & 720 \\
\hline
14 & 福州-西安途中 & 无 & 4 & 360 \\
\hline
1 & 西安 & 无 & 8 & 720 \\
\hline
2 & 西安-海口途中 & 无 & 8 & 720 \\
\hline
3 & 西安-海口途中 & 无 & 7.3 & 660 \\
\hline
4 & 海口 & 海口 & 0 & 0 \\
\hline
5 & 海口 & 三亚南山文化旅游区 & 3 & 274 \\
\hline
6 & 三亚市 & 三亚南山大小洞天旅游区, 保亭县呀诺达雨林文化旅游区 & 2 & 80 \\
\hline
21 & 7 & 三亚市 & 陵水县分界洲岛旅游区, 保亭县海南槟榔谷黎苗文化旅游区 & 1.7 & 68 \\
\hline
8 & 三亚市 & 无 & 8 & 720 \\
\hline
9 & 南宁市 & 南宁市青秀山旅游区 & 1.8 & 90 \\
\hline
10 & 南宁市 & 南宁市 & 0 & 0 \\
\hline
11 & 南宁市 & 桂林独秀峰·靖江王城景区 & 4.3 & 388 \\
\hline
12 & 桂林市 & 桂林漓江风景区 & 1.5 & 60 \\
\hline
13 & 桂林市 & 无 & 4.3 & 388 \\
\hline
14 & 南宁市 & 无 & 8 & 720 \\
\hline
15 & 南宁市-西安途中 & 无 & 8 & 720 \\
\hline
1 & 西安 & 无 & 8 & 720 \\
\hline
2 & 西安-南昌途中 & 南昌 & 4.1 & 370 \\
\hline
3 & 南昌 & 南昌 & 1.5 & 135 \\
\hline
4 & 九江市 & 九江庐山风景名胜区 & 1 & 40 \\
\hline
5 & 景德镇市 & 景德镇古窑民俗博览区 & 1.6 & 146 \\
\hline
6 & 上饶市 & 上饶婺源县江湾景区 & 2.22 & 200 \\
\hline
7 & 上饶市 & 上饶三清山旅游景区 & 1 & 40 \\
\hline
22 & 8 & 鹰潭市贵溪市 & 鹰潭市贵溪龙虎山风景名胜区 赣州市瑞金市共和国摇篮景区 & 0.95 & 85 \\
\hline
9 & 赣州市瑞金市 & 赣州市瑞金市共和国摇篮景区 吉安井冈山风景旅游区 & 3.89 & 350 \\
\hline
10 & 赣州市瑞金市 & 无 & 8 & 680 \\
\hline
11 & 南昌-西安途中 & 无 & 8 & 720 \\
\hline
1 & 西安 & 无 & 8 & 720 \\
\hline
2 & 西安-昆明途中 & 无 & 8 & 720 \\
\hline
23 & 3 & 昆明 & 昆明 & 1.7 & 150 \\
\hline
4 & 昆明 & 无 & 5.8 & 523.5 \\
\hline
5 & 景洪市 & 中科院西双版纳热带植物园 & 1 & 40 \\
\hline
\end{tabular}

\begin{tabular}{r l l r r}
6 & 景洪市-大理市 & 无 & 8 & 720 \\
7 & 大理市 & 大理崇圣寺三塔文化旅游区 & 2 & 95.8 \\
8 & 香格里拉 & 迪庆藏族自治州香格里拉普达措国家公园 & 3.7 & 336 \\
9 & 香格里拉 & 丽江古城景区 & 1.93 & 174.1 \\
10 & 丽江市 & 丽江玉龙雪山景区 & 1 & 40 \\
11 & 丽江市 & 无 & 8 & 720 \\
12 & 昆明-西安 & 无 & 8 & 720 \\
1 & 西安 & 无 & 8 & 720 \\
2 & 西安-南宁途中 & 无 & 8 & 720 \\
3 & 南宁 & 南宁 & 2.22 & 200 \\
4 & 南宁 & 桂林兴安县乐满地度假世界 & 1.5 & 60 \\
5 & 南宁 & 南宁市青秀山旅游区 & 3.7 & 260 \\
6 & 南宁-西安途中 & 无 & 8 & 720 \\
7 & 南宁-西安途中 & 无 & 8 & 720 \\
1 & 西安 & 无 & 8 & 720 \\
2 & 西安-长沙途中 & 长沙 & 3 & 270 \\
3 & 长沙 & 湘潭韶山旅游区 & 1.5 & 60 \\
 & & 长沙岳麓山-橘子洲旅游 & & \\
4 & 长沙 & 区,长沙市宁乡县花明楼景区 & 0 & 0 \\
 & & 区 & & \\
5 & 长沙 & 张家界武陵源-天门山旅游区 & 3.6 & 322 \\
6 & 张家界市 & 张家界武陵源-天门山旅游区 & 0 & 0 \\
7 & 张家界市 & 岳阳岳阳楼-君山岛景区 & 6.4 & 364 \\
8 & 岳阳市 & 衡阳南岳衡山旅游区 & 3.6 & 320 \\
9 & 衡阳市 & 郴州市东江湖旅游区 & 2 & 178 \\
10 & 衡阳市 & 无 & 4.5 & 330 \\
11 & 长沙-西安途中 & 无 & 8 & 720 \\
1 & 西安 & 无 & 8 & 720 \\
2 & 西安-广州途中 & 无 & 8 & 720 \\
3 & 西安-广州途中 & 广州 & 2.3 & 210 \\
4 & 广州 & 广州长隆旅游度假区 & 1 & 40 \\
5 & 广州 & 广州白云山景区 & 1 & 35.1 \\
6 & 佛山市 & 佛山西樵山景区,佛山市顺德区长鹿旅游休博园 & 0 & 0 \\
7 & 佛山市 & 深圳华侨城旅游度假区深圳观澜湖休闲旅游区 & 1.8 & 161.7 \\
8 & 深圳市 & 惠州市罗浮山景区 & 1 & 92.1 \\
9 & 惠州市 & 梅州市梅县区雁南飞茶田景区 & 3.02 & 271.8 \\
10 & 梅州市 & 韶关仁化丹霞山景区 & 4.6 & 412.1 \\
 & & 30 & & \\
\end{tabular}

\begin{tabular}{r l l r r}
\hline
4 & 武汉 & 无 & 3.6 & 323 \\
5 & 宜昌 & 宜昌三峡大坝旅游区 & 2.5 & 110 \\
 & & 宜昌长阳县清江画廊景区 & & \\
6 & 宜昌 & 宜昌三峡人家风景区 & 1 & 40 \\
7 & 宜昌 & 宜昌秭归县屈原故里文化旅游区 & 3.6 & 290 \\
8 & 恩施 & 恩施土家族苗族自治州恩施大峡谷景区 & 2 & 70 \\
9 & 恩施 & & 6.8 & 610 \\
10 & 十堰 & 十堰丹江口市武当山风景区 & 1 & 40 \\
11 & 十堰 & 神农架生态旅游区 & 4 & 190 \\
12 & 十堰 & 神农架生态旅游区 & 0 & 0 \\
13 & 十堰 & 神农架生态旅游区 & 4 & 380 \\
13.5 & 十堰-西安途中 & 无 & 4 & 360 \\
1 & 西安 & 无 & 8 & 720 \\
2 & 西安-杭州途中 & 杭州 & 6.7 & 610 \\
3 & 杭州 & 杭州西湖风景区 & 1.5 & 80 \\
 & & 嘉兴桐乡乌镇古镇 & & \\
4 & 杭州 & 杭州淳安千岛湖风景区 & 2.5 & 170 \\
5 & 杭州 & 金华东阳横店影视城景区 & 2 & 150 \\
 & & 杭州 & & \\
6 & 杭州 & 杭州西溪湿地旅游区 & 1.5 & 90 \\
 & & 湖州市南浔区南浔古镇景区 & & \\
7 & 嘉兴 & 嘉兴南湖旅游区 & 1.2 & 110 \\
8 & 绍兴 & 绍兴市鲁迅故里-沈园景区 & 1.3 & 117 \\
9 & 宁波 & 宁波奉化溪口-滕头旅游景区 & 1 & 90 \\
10 & 舟山 & 舟山普陀山风景区 & 0 & 0 \\
11 & 舟山 & 无 & 4 & 361 \\
12 & 温州 & 温州乐清市雁荡山风景区 & 1 & 45 \\
13 & 衢州 & 衢州市开化根宫佛国文化旅游区 & 5 & 450 \\
14 & 衢州-西安途中 & 无 & 8 & 720 \\
1 & 西安 & 无 & 8 & 720 \\
2 & 西安-济南途中 & 济南天下第一泉景区(趵突泉-大明湖-五龙潭-环城公园-黑虎泉) & 2 & 190 \\
3 & 济南 & 济南 & 0 & 0 \\
4 & 济南 & 泰安泰山景区 & 1.2 & 110 \\
5 & 泰安 & 济宁曲阜明故城三孔旅游区 & 2 & 188 \\
6 & 枣庄 & 枣庄台儿庄古城景区 & 5 & 450 \\
7 & 青岛 & 青岛崂山景区 & 2.9 & 262 \\
\hline
\end{tabular}

\section{问题二的求解}

由题意知,第二问中要求制定费用最优、体验最佳的201个5A景点旅游计划。游客在每一次出行时有三种交通方式选择:自驾、机票+省内租车、高铁+省内租车。飞机、高铁的选择在节省西安至省会城市路途时间和成本的同时,也增加了每一省市内租车的额外成本。因此,只有当飞机、高铁的节约成本大于额外增加的租车费用时,才满足费用最优要求。

\subsection{交通方式选择}

为探究出行距离与交通方式选择之间的关系,将西安至各省会城市的自驾车、飞机、高铁的费用与距离数据统计如下表所示。

\begin{tabular}{r l l r r}
\hline
8 & 威海 & 威海刘公岛景区 & 1.7 & 175 \\
 & & 烟台龙口南山景区 & & \\
\hline
9 & 烟台 & 烟台蓬莱阁一三仙山一八仙 & 1 & 91 \\
 & & 过海旅游区 & & \\
\hline
10 & 蓬莱 & 无 & 2.3 & 210 \\
\hline
11 & 潍坊 & 山东沂蒙山旅游区(沂山景区一龟蒙景区一云蒙景区) & 2 & 190 \\
\hline
12 & 潍坊-西安 & 无 & 8 & 720 \\
\hline
1 & 西安 & 无 & 8 & 720 \\
\hline
2 & 合肥 & 合肥 & 3.5 & 321 \\
\hline
3 & 黄山 & 黄山市黄山风景区 & 1 & 60 \\
\hline
4 & 黄山 & 黄山市黟县皖南古村落一西递宏村 & 1.5 & 65 \\
\hline
5 & 黄山 & 宣城市绩溪县龙川景区 & 1.5 & 70 \\
\hline
 & & 黄山市古徽州文化旅游区 & & \\
18 & 6 & 黄山 & (徽州古城一牌坊群鲍家花园一唐模一港口民宅一呈 & 2.5 & 214 \\
 & & & 坎) & & \\
\hline
7 & 池州 & 池州青阳县九华山风景区 & 1 & 45 \\
\hline
8 & 安庆潜山 & 安庆潜山县天柱山风景区 & 2 & 170 \\
\hline
9 & 安庆潜山 & 六安市金寨县天堂寨旅游 & 2 & 11 \\
 & & 区 & & \\
\hline
10 & 麻城 & 阜阳市颍上县八里河风景区 & 3.5 & 313 \\
\hline
11 & 阜阳-西安途中 & 无 & 8 & 720 \\
\hline
1 & 西安 & 无 & 8 & 720 \\
\hline
2 & 西安-上海途中 & 无 & 7.4 & 666 \\
\hline
3 & 上海 & 上海野生动物园 & 0 & 0 \\
\hline
19 & 4 & 上海 & 东方明珠广播电视塔,上海科技馆 & 0 & 0 \\
\hline
5 & 上海 & 上海 & 0 & 0 \\
\hline
6 & 上海 & 无 & 8 & 720 \\
\hline
7 & 上海-西安途中 & 无 & 7.4 & 666 \\
\hline
1 & 西安 & 无 & 8 & 720 \\
\hline
2 & 西安-福州途中 & 无 & 8 & 720 \\
\hline
3 & 西安-福州途中 & 福州 & 2.3 & 210 \\
\hline
4 & 福州 & 福州市三坊七巷景区 & 3.8 & 346 \\
\hline
5 & 南平市武夷山市 & 南平武夷山风景名胜区 & 0 & 0 \\
\hline
6 & 南平市武夷山市 & 三明泰宁风景旅游区 & 2.5 & 100 \\
\hline
7 & 南平市武夷山市 & 无 & 8 & 720 \\
\hline
8 & 宁德市 & 宁德屏南(白水洋·鸳鸯溪)旅游景区 & 2.5 & 100 \\
\hline
9 & 泉州市 & 泉州市清源山风景名胜区 & 0 & 0 \\
\hline
10 & 厦门市 & 厦门鼓浪屿风景名胜区 & 0 & 0 \\
\hline
\end{tabular}

\begin{tabular}{l c c c c c c c c c}
 & 1080 & 1100 & 790 & 600 & 970 & 1750 & 2040 & 2310 & 1390 \\
 & & & & & & & & & 1090 \\
省会 & 杭州 & 合肥 & 福州 & 南昌 & 济南 & 长沙 & 武汉 & 广州 & 南宁 \\
城市 & & & & & & & & & 海口 \\
距离 & 1330 & 920 & 1650 & 1090 & 910 & 480 & 990 & 740 & 1650 \\
 & & & & & & & & & 1640 \\
飞机 & 1280 & 1060 & 1530 & 1010 & 960 & 980 & 1220 & 1540 & 1800 \\
票价 & & & & & & & & & 1830 \\
高铁 & & & & & & 587 & 454.5 & & \\
票价 & & & & & & & & & \\
自驾 & 1330 & 920 & 1650 & 1090 & 910 & 480 & 990 & 740 & 1650 \\
费用 & & & & & & & & & 1640 \\
省会 & 重庆 & 成都 & 贵阳 & 昆明 & 拉萨 & 兰州 & 西宁 & 银川 & 乌鲁木齐 \\
城市 & & & & & & & & & 郑州 \\
距离 & 700 & 750 & 1070 & 1590 & 2840 & 650 & 890 & 740 & 2540 \\
 & & & & & & & & & 480 \\
飞机 & 1200 & 740 & 1000 & 1540 & 1990 & 1000 & 710 & 850 & 2200 \\
票价 & & & & & & & & & \\
高铁 & & & & & & & & & 229 \\
票价 & & & & & & & & & \\
自驾 & 700 & 750 & 1070 & 1590 & 2840 & 650 & 890 & 740 & 2540 \\
费用 & & & & & & & & & 480 \\
\end{tabular}

\textbf{说明:}

1. 由题意知,高速公路的油耗加过路费平均为1.00元/公里,假定省会城市之间均走高速公路,则自驾费用与距离数值相等。

2. 西安至郑州无航空航班,西安至其他省会城市的高铁班次只包含行程时间在6小时之内的班次。

将每种交通方式的出行费用-距离数据进行拟合,得到费用-距离曲线如下图所示。

\begin{figure}[h]
\centering
\includegraphics[width=\textwidth]{image.png}
\caption{三种交通方式费用-距离曲线图}
\end{figure}

\textbf{图5 三种交通方式费用-距离曲线图}

就西安至各省市的往返程而言,由上图可知,高铁成本远低于自驾、航空两种交通方式,航空、自驾费用-距离曲线的交点 A 介于 \((720km, 1440km)\) 之间,即一天至两天的行程距离之间。当距离在 A 之前时,自驾游比飞机出行费用更优,加上后期城市至景区的租车费用,飞机更不具备优势;当省际城市距离超过 A 时,航空是费用最少的出行方式。因此,从西安至各省会城市的出游,如果有高铁班次,优先选择高铁交通方式,若没有,若行车时间在 1 天之内,选择自驾方式;当行车时间在 1-2 天之间,进一步比较整个出行旅程中飞机节省成本大与后期租车成本大小;当行车时间大于 2 天时,优先选择飞机交通方式,如下图所示。

\begin{figure}[h]
    \centering
    \includegraphics[width=\textwidth]{image.png}
    \caption{交通方式选择与车程关系图}
    \label{fig:transportation}
\end{figure}

\subsection{下层路径优化}

本题假设 “体验最佳” 包含两个含义:其一,游览 201 个景点总年数最小,其二,尽可能减少自驾出游次数,即对于远距离的景点城市尽量选用飞机、高铁交通方式。本题同样采用分层优化方法,目标函数为总游览时间最短、自驾次数最少、总游览费用最优。

\subsubsection{模型的建立}

下层优化是对某一省会城市辐射区内景点的旅游规划,由于同一省内各城市间距离较近,为节省费用,省内出行均采用自驾或租车方式。游客若需要租车游览,均在省会城市租还车。并且是否租车、自驾次数由上层交通方式选择决定,因此租车费用、票价因素和自驾次数最少目标均将在上层考虑。本层的优化目标为总游览年数最短、总费用最优,优化的费用仅考虑油费、过路费、住宿费。

\textbf{目标函数:}
\begin{equation}
Z_m = Min \sum_{i \neq j} \left[ 3(T_{ij} + T_i) + 2(HC_{ij} + CC_{ij}) \right] X_{ij}
\end{equation}

约束:

\begin{align*}
A' &= 2 \, 3 \, | \, 1 \, 4 \, 5 \\
B' &= 1 \, 5 \, | \, 3 \, 4 \, 2
\end{align*}

模型以省会城市辐射区内所有景点的游览总年数和总费用最小为目标,优化该省市辐射区内景点的最优游览路径。其中,公式 1 表示游客从每一个景点走出一次;公式 2 表示游客到达每一个景点一次;公式 3 表示避免景点间小回路的发生;公式 4-5 表示景区开放时间限制;公式 6 表示前后景点的到达时间关系,公式 7-8 表示每天车行时间区间限制;公式 9 表示景点游览时间限制;公式 10-15 表示当行车时间控制在 3 小时内时,景点间车行时间可以忽略,全天游览;开车时间控制在 5 小时内,景点间车行时间记为半天,景点游览半天;当车行时间在 8 小时之内,视为行车时间为 1 天;超过 8 小时时,按照以上规则记为多天。公式 16 表示住宿费用约束,$150 \, \text{int}[s_{ij}/90]$ 表示在前往 $j$ 城市邻近景区可能产生的中途住宿费,若景区位于省会城市,则住宿费为 200 元/天人,若位于地级市为 150 元/天人,位于县城为 100 元/天人。公式 17 表示车费油费约束,由公式 18 表示决策变量 $X_{ij}$ 为 0-1 变量以及 $i,j$ 的取值范围。

\begin{tabular}{r l l r r}
\hline
11 & 龙岩市 & 福建土楼(永定·南靖)旅游区 & 2 & 80 \\
\hline
12 & 福州 & 福州 & 6.3 & 567 \\
\hline
13 & 福州-西安途中 & 无 & 8 & 720 \\
\hline
14 & 福州-西安途中 & 无 & 4 & 360 \\
\hline
1 & 西安 & 无 & 8 & 720 \\
\hline
2 & 西安-海口途中 & 无 & 8 & 720 \\
\hline
3 & 西安-海口途中 & 无 & 7.3 & 660 \\
\hline
4 & 海口 & 海口 & 0 & 0 \\
\hline
5 & 海口 & 三亚南山文化旅游区 & 3 & 274 \\
\hline
6 & 三亚市 & 三亚南山大小洞天旅游区, 保亭县呀诺达雨林文化旅游区 & 2 & 80 \\
\hline
21 & 7 & 三亚市 & 陵水县分界洲岛旅游区, 保亭县海南槟榔谷黎苗文化旅游区 & 1.7 & 68 \\
\hline
8 & 三亚市 & 无 & 8 & 720 \\
\hline
9 & 南宁市 & 南宁市青秀山旅游区 & 1.8 & 90 \\
\hline
10 & 南宁市 & 南宁市 & 0 & 0 \\
\hline
11 & 南宁市 & 桂林独秀峰·靖江王城景区 & 4.3 & 388 \\
\hline
12 & 桂林市 & 桂林漓江风景区 & 1.5 & 60 \\
\hline
13 & 桂林市 & 无 & 4.3 & 388 \\
\hline
14 & 南宁市 & 无 & 8 & 720 \\
\hline
15 & 南宁市-西安途中 & 无 & 8 & 720 \\
\hline
1 & 西安 & 无 & 8 & 720 \\
\hline
2 & 西安-南昌途中 & 南昌 & 4.1 & 370 \\
\hline
3 & 南昌 & 南昌 & 1.5 & 135 \\
\hline
4 & 九江市 & 九江庐山风景名胜区 & 1 & 40 \\
\hline
5 & 景德镇市 & 景德镇古窑民俗博览区 & 1.6 & 146 \\
\hline
6 & 上饶市 & 上饶婺源县江湾景区 & 2.22 & 200 \\
\hline
7 & 上饶市 & 上饶三清山旅游景区 & 1 & 40 \\
\hline
22 & 8 & 鹰潭市贵溪市 & 鹰潭市贵溪龙虎山风景名胜区 赣州市瑞金市共和国摇篮景区 & 0.95 & 85 \\
\hline
9 & 赣州市瑞金市 & 赣州市瑞金市共和国摇篮景区 吉安井冈山风景旅游区 & 3.89 & 350 \\
\hline
10 & 赣州市瑞金市 & 无 & 8 & 680 \\
\hline
11 & 南昌-西安途中 & 无 & 8 & 720 \\
\hline
1 & 西安 & 无 & 8 & 720 \\
\hline
2 & 西安-昆明途中 & 无 & 8 & 720 \\
\hline
23 & 3 & 昆明 & 昆明 & 1.7 & 150 \\
\hline
4 & 昆明 & 无 & 5.8 & 523.5 \\
\hline
5 & 景洪市 & 中科院西双版纳热带植物园 & 1 & 40 \\
\hline
\end{tabular}

\subsection{上层路径优化}

\subsubsection{模型的建立}

通过下层优化求解,201 个景点最短时间的游览规划可以简化为 31 个省市的驾车游览路径优化。在该层优化中,需要考虑高铁、飞机等交通方式的选择对出游费用和时间的影响。同理,本层的“体验最佳”目标通过总游览时间最少、自驾出游次数最少来量化。总结来看,本层的出游费用包括票价、租车费用、车油费和过路费、住宿费用,优化目标是总游览年数最少、自驾次数最少、总游览费用最优。

\textbf{目标函数:}
\[
Z_m = Min \sum_{y=1}^M \left[ 90 + (3W_{ij} + 2C_{ijyk} - 3) \sum_{i=1}^m \sum_{j=1}^m \sum_{k=1}^N X_{ijyk} \right]
\]

\textbf{约束:}
\begin{align}
\sum_{j=1}^m \sum_{y=1}^M \sum_{k=1}^N X_{ijyk} &\geq 1 \tag{1} \\
\sum_{i=1}^m \sum_{y=1}^M \sum_{k=1}^N X_{ijyk} &\geq 1 \tag{2} \\
\sum_{j=1}^m \sum_{y=1}^M \sum_{k=1}^N X_{ijyk} &= \sum_{i=1}^m \sum_{y=1}^M \sum_{k=1}^N X_{ijyk} \tag{3} \\
\sum_{i=1}^m \sum_{j=1}^m (T_{ijyk} + T_{iyk}) X_{ijyk} &\leq 15 \tag{4} \\
\sum_{y=1}^M \sum_{i=1}^m \sum_{k=1}^N X_{ijyk} T_{jyk} &\geq T_j^* \tag{5} \\
\sum_{i,j \in F} X_{ijyk} &\leq |S^*| - 1, \, 2 \leq S^* \leq m - 2, \, S^* \in (1, 2, \dots, n) \tag{6} \\
p_i^* &\leq DT_i^* + 8T_{ijyk} \leq q_i^* \tag{7} \\
\sum_{i=1}^m \sum_{j=1}^m \sum_{k=1}^N X_{ijyk} (T_{ijyk} + T_{iyk}) &\leq 30 \tag{8} \\
1 \leq K &\leq 4 \tag{9} \\
T_{ijyk} &= 0 \, \text{ 若 } S_{ijyk}/90 < 3 \tag{10} \\
T_{ijyk} &= 0.5 \, \text{ 若 } 3 \leq S_{ijyk}/90 < 5 \tag{11} \\
T_{ijk} &= 1 \, \text{ 若 } 5 \leq S_{ijyk}/90 < 8 \tag{12} \\
T_{ijk} &= \left[ \left( \frac{S_{ijyk}}{90} \right) \backslash 8 \right] \quad \text{ 若 } 8 \leq \frac{S_{ijyk}}{90} \text{ 且 } \left( \frac{S_{ijyk}}{90} \right) Mod \, 8 < 3 \tag{13} \\
T_{ijk} &= \left[ \left( \frac{S_{ijyk}}{90} \right) \backslash 8 \right] + 0.5 \, \text{ 若 } 8 \leq \frac{S_{ijyk}}{90} \text{ 且 } 3 \leq \left( \frac{S_{ijyk}}{90} \right) Mod \, 8 < 5 \tag{14} \\
T_{ijk} &= \left[ \left( \frac{S_{ijyk}}{90} \right) \backslash 8 \right] + 1 \, \text{ 若 } 8 \leq \frac{S_{ijyk}}{90} \text{ 且 } 5 \leq \left( \frac{S_{ijyk}}{90} \right) Mod \, 8 \tag{15} \\
C_{ijyk} &= HC_{ijyk} + CC_{ijk} + (RC_{jyk} + TC_{ijyk}) * (1 - W_{ij}) \tag{16} \\
CC_{ijyk} &= \frac{S_{ijyk}}{90} + CC_j \tag{17} \\
&\quad \frac{36}{}
\end{align}

\begin{equation}
RC_{ik} = 300T_{iyk}
\tag{18}
\end{equation}

\begin{equation}
HC_{ijyk} =
\begin{cases}
150 \, int\left[\frac{S_{ijyk}}{90}\right] + HC_j + 200 \times 3 & j \neq 1 \\
150 \, int\left[\frac{S_{ijyk}}{90}\right] + HC_j & j = 1
\end{cases}
\tag{19}
\end{equation}

\begin{equation}
X_{ijyk} \in (0,1), \quad W_{ij} \in (0,1). \, i, j = 1, 2, \ldots, m. \, y = 1, 2, \ldots, M. \, K = 1, 2, \ldots, N
\tag{20}
\end{equation}

模型目标函数是指总费用最少与体验最佳(出行年数最少和自驾次数最少),两者权重比为 2:3。其中,$Min2 \sum_{y=1}^M \sum_{i=1}^m \sum_{j=1}^m \sum_{k=1}^N C_{ijyk} X_{ijyk}$ 表示总费用最小,$Min3 \sum_{y=1}^M \sum_{i=1}^m \sum_{j=1}^m \sum_{k=1}^N W_{ij} X_{ijyk}$ 表示自驾出行次数最少,$Min3 \sum_{y=1}^M \left(30 - \sum_{i=1}^m \sum_{j=1}^m \sum_{k=1}^N X_{ijky}\right)$ 表示出行年数最少。公式 1-3 表示每个省会城市至少到达 1 次,且到达次数与离开次数相同。公式 4 表示每次出行不超过 15 天约束,公式 5 表示每一次出行游览的始终点均是西安,公式 6 是每一次旅游的小回路消除条件;公式 7 是指每天车行时间区间限制;公式 8 表示每年出行的时间不超过 30 天,公式 9 表示每年的出行次数不超过 4 次;公式 10-15 表示当行车时间控制在 3 小时内时,景点间车行时间可以忽略,全天游览;开车时间控制在 5 小时内,景点间车行时间记为半天,景点游览半天;当车行时间在 8 小时之内,视为行车时间为 1 天;超过 8 小时时,按照以上规则记为多天。公式 16 表示第 $k$ 次从 $i$ 省会城市至 $j$ 省会城市游览的总费用包括飞机或高铁票价、租车费用、车油费和过路费、住宿费用。公式 17 表示油费和过路费组成;公式 18 表示租车费用等于租车单位价格与该省会城市游览时间的乘积;公式 19 表示每次从 $i$ 城市去 $j$ 城市游览的住宿费用,当从其他省会城市返程抵达西安时,西安当晚的住宿费不再考虑;公式 20 表示决策变量 $X_{ijk}$,$W_{ij}$ 为 0-1 变量以及 $i, j, k$ 的取值范围。

\subsubsection{模型的求解}

问题二的上层路径优化采用模型一的上层路径优化算法。其中编码方式,交叉概率,变异概率保持不变,适应度函数需要针对问题二的上层路径优化模型的目标函数,做出相应调整。由于求解的目标是费用最小,因此需要将求最小的目标根据适应度函数非负原则转化为求最大目标的形式。

\begin{equation}
f = \frac{bz'}{z}
\end{equation}

此处 $f_k$ 为染色体 $\nu_k$ 适应度,$b$ 为一常数,$z'$ 为初始种群最佳染色体的总旅行成本,$z_k$ 为染色体 $\nu_k$ 所对应的总旅行成本。同样借用 MATLAB,可以得到全国 201 个 5A 景区的 10 年出游方案计划,如表 8 和图 7 所示。其中,每一天的出发地、行车时间、行车里程、游览景区等具体信息如表 9 所示。

\begin{figure}[h]
    \centering
    \includegraphics[width=\textwidth]{image.png}
    \caption{图7 10年交通方式组合最优出行旅游规划}
\end{figure}

\begin{tabular}{r l l r r}
6 & 景洪市-大理市 & 无 & 8 & 720 \\
7 & 大理市 & 大理崇圣寺三塔文化旅游区 & 2 & 95.8 \\
8 & 香格里拉 & 迪庆藏族自治州香格里拉普达措国家公园 & 3.7 & 336 \\
9 & 香格里拉 & 丽江古城景区 & 1.93 & 174.1 \\
10 & 丽江市 & 丽江玉龙雪山景区 & 1 & 40 \\
11 & 丽江市 & 无 & 8 & 720 \\
12 & 昆明-西安 & 无 & 8 & 720 \\
1 & 西安 & 无 & 8 & 720 \\
2 & 西安-南宁途中 & 无 & 8 & 720 \\
3 & 南宁 & 南宁 & 2.22 & 200 \\
4 & 南宁 & 桂林兴安县乐满地度假世界 & 1.5 & 60 \\
5 & 南宁 & 南宁市青秀山旅游区 & 3.7 & 260 \\
6 & 南宁-西安途中 & 无 & 8 & 720 \\
7 & 南宁-西安途中 & 无 & 8 & 720 \\
1 & 西安 & 无 & 8 & 720 \\
2 & 西安-长沙途中 & 长沙 & 3 & 270 \\
3 & 长沙 & 湘潭韶山旅游区 & 1.5 & 60 \\
 & & 长沙岳麓山-橘子洲旅游 & & \\
4 & 长沙 & 区,长沙市宁乡县花明楼景区 & 0 & 0 \\
 & & 区 & & \\
5 & 长沙 & 张家界武陵源-天门山旅游区 & 3.6 & 322 \\
6 & 张家界市 & 张家界武陵源-天门山旅游区 & 0 & 0 \\
7 & 张家界市 & 岳阳岳阳楼-君山岛景区 & 6.4 & 364 \\
8 & 岳阳市 & 衡阳南岳衡山旅游区 & 3.6 & 320 \\
9 & 衡阳市 & 郴州市东江湖旅游区 & 2 & 178 \\
10 & 衡阳市 & 无 & 4.5 & 330 \\
11 & 长沙-西安途中 & 无 & 8 & 720 \\
1 & 西安 & 无 & 8 & 720 \\
2 & 西安-广州途中 & 无 & 8 & 720 \\
3 & 西安-广州途中 & 广州 & 2.3 & 210 \\
4 & 广州 & 广州长隆旅游度假区 & 1 & 40 \\
5 & 广州 & 广州白云山景区 & 1 & 35.1 \\
6 & 佛山市 & 佛山西樵山景区,佛山市顺德区长鹿旅游休博园 & 0 & 0 \\
7 & 佛山市 & 深圳华侨城旅游度假区深圳观澜湖休闲旅游区 & 1.8 & 161.7 \\
8 & 深圳市 & 惠州市罗浮山景区 & 1 & 92.1 \\
9 & 惠州市 & 梅州市梅县区雁南飞茶田景区 & 3.02 & 271.8 \\
10 & 梅州市 & 韶关仁化丹霞山景区 & 4.6 & 412.1 \\
 & & 30 & & \\
\end{tabular}

\begin{tabular}{l c c c c c c c c c}
 & 1080 & 1100 & 790 & 600 & 970 & 1750 & 2040 & 2310 & 1390 \\
 & & & & & & & & & 1090 \\
省会 & 杭州 & 合肥 & 福州 & 南昌 & 济南 & 长沙 & 武汉 & 广州 & 南宁 \\
城市 & & & & & & & & & 海口 \\
距离 & 1330 & 920 & 1650 & 1090 & 910 & 480 & 990 & 740 & 1650 \\
 & & & & & & & & & 1640 \\
飞机 & 1280 & 1060 & 1530 & 1010 & 960 & 980 & 1220 & 1540 & 1800 \\
票价 & & & & & & & & & 1830 \\
高铁 & & & & & & 587 & 454.5 & & \\
票价 & & & & & & & & & \\
自驾 & 1330 & 920 & 1650 & 1090 & 910 & 480 & 990 & 740 & 1650 \\
费用 & & & & & & & & & 1640 \\
省会 & 重庆 & 成都 & 贵阳 & 昆明 & 拉萨 & 兰州 & 西宁 & 银川 & 乌鲁木齐 \\
城市 & & & & & & & & & 郑州 \\
距离 & 700 & 750 & 1070 & 1590 & 2840 & 650 & 890 & 740 & 2540 \\
 & & & & & & & & & 480 \\
飞机 & 1200 & 740 & 1000 & 1540 & 1990 & 1000 & 710 & 850 & 2200 \\
票价 & & & & & & & & & \\
高铁 & & & & & & & & & 229 \\
票价 & & & & & & & & & \\
自驾 & 700 & 750 & 1070 & 1590 & 2840 & 650 & 890 & 740 & 2540 \\
费用 & & & & & & & & & 480 \\
\end{tabular}

\begin{tabular}{r l l r r r}
\hline
8 & 新疆博湖县 & 无 & 8.0 & 0 & 848.6 \\
9 & 乌鲁木齐-石河子途中 & 伊犁地区新源县那拉提旅游风景区 & 0.6 & 1 & 600 \\
10 & 石河子新源县 & 无 & 7.4 & 0 & 750 \\
11 & 阿勒泰 & 无 & 6 & 0 & 950 \\
12 & 阿勒泰布尔津县 & 阿勒泰地区布尔津县喀纳斯景区 & 0 & 1 & 700 \\
13 & 阿勒泰布尔津县 & 阿勒泰地区布尔津县喀纳斯景区 & 0 & 1 & 600 \\
14 & 阿勒泰布尔津县 & 无 & 7.93 & 0 & 1613.7 \\
15 & 乌鲁木齐 & 无 & 2.82 & 0 & 6600 \\
\hline
合计 & & & & & 23912.3 \\
\hline
1 & 西安 & 无 & 8 & 0 & 7453.8 \\
2 & 乌鲁木齐-喀什途中 & 无 & 8 & 0 & 957.2 \\
3 & 喀什 & 喀什地区泽普县金胡杨景区 & 3 & 1 & 863.8 \\
4 & 喀什-乌鲁木齐途中 & 无 & 8 & 0 & 1470 \\
5 & 喀什-乌鲁木齐途中 & 无 & 7.82 & 0 & 7410 \\
\hline
合计 & & & & & 18154.8 \\
\hline
1 & 西安 & 拉萨 & 3.2 & 0.5 & 6570 \\
2 & 拉萨 & 拉萨、拉萨布达拉宫景区 & 0 & 0.5+0.5 & 900 \\
3 & 拉萨 & 拉萨大昭寺景区 & 3.2 & 0.5 & 6120 \\
\hline
合计 & & & & & 13590 \\
\hline
1 & 西安 & 无 & 8.0 & 0 & 1170 \\
2 & 西安-银川途中 & 银川 & 3.2 & 0.5 & 878 \\
3 & 银川 & 银川、石嘴山平罗县沙湖旅游景区 & 1.0 & 0.5+0.5 & 810 \\
4 & 银川 & 中卫沙坡头旅游景区 & 3.0 & 1 & 980 \\
5 & 银川 & 银川镇北堡西部影视城、银川市灵武水洞沟旅游景区 & 2.0 & 0.5+0.5 & 820 \\
6 & 银川 & 无 & 8.0 & 0 & 1170 \\
6.5 & 银川-西安途中 & 无 & 3.2 & 0 & 128 \\
\hline
合计 & & & & & 5956 \\
\hline
6 & 1 & 西安 & 石家庄 & 3.45 & 0.5 & 3120 \\
\hline
\end{tabular}

\begin{tabular}{c c c c c}
\hline
2 & 石家庄 & 石家庄, 石家庄平山县西柏坡景区 & 2 & 0.5+0.5 \\
 & & & & 900 \\
\hline
3 & 平山县 & 石家庄平山县西柏坡景区 & 3.6 & 0.5 \\
 & & & & 900 \\
\hline
4 & 保定涞水县 & 保定涞水县野三坡景区(百里峡-白草畔-鱼谷洞-龙门天关) & 0 & 1 \\
 & & & & 600 \\
\hline
5 & 保定涞水县 & 保定安新白洋淀景区(文化苑-大观园-鸳鸯岛-元妃荷园-嘎子印象-渔人乐园) & 0.8 & 1 \\
 & & & & 600 \\
\hline
6 & 雄县 & 无 & 3.6 & 0 \\
 & & & & 1217.5 \\
\hline
7 & 北京 & 北京 & 0 & 1 \\
 & & & & 900 \\
\hline
高铁+ & 8 & 北京 & 故宫博物院、天坛公园 & 0 \\
租车 & & & & 0.5+0.5 \\
 & & & & 900 \\
\hline
9 & 北京 & 颐和园、八达岭—慕田峪长城旅游区 & 1.5 & 0.5+0.5 \\
 & & & & 670 \\
\hline
10 & 北京市延庆县 & 明十三陵景区(神路-定陵-长陵-昭陵) & 1 & 1 \\
 & & & & 960 \\
\hline
11 & 北京 & 恭王府景区 & 0 & 1 \\
 & & & & 900 \\
\hline
12 & 北京 & 北京奥林匹克公园(鸟巢-水立方-中国科技馆-国家奥林匹克森林公园) & 0 & 1 \\
 & & & & 900 \\
\hline
13 & 北京 & 天津古文化街旅游区(津门故里) & 2.5 & 1 \\
 & & & & 495 \\
\hline
13.5 & 北京 & 无 & 4.62 & 1546.5 \\
\hline
合计 & & & & 14609 \\
\hline
1 & 西安 & 无 & 8 & 0 \\
 & & & & 1170 \\
\hline
2 & 西安-呼和浩特途中 & 呼和浩特 & 2.8 & 0.5 \\
 & & & & 850 \\
\hline
3 & 呼和浩特 & 呼和浩特 & 1 & 1 \\
 & & & & 540 \\
\hline
3 & 7 & 自驾 & 鄂尔多斯伊金霍洛旗成吉思汗陵旅游区、鄂尔多斯达拉特旗响沙湾旅游景区 & 1.5 & 0.5+0.5 \\
 & & & & & 540 \\
\hline
41 & & & & & \\
\hline
\end{tabular}

\begin{tabular}{c c c c c}
\hline
5 & 鄂尔多斯 & 鄂尔多斯达拉特旗响沙湾旅游区 & 5.8 & 0.5 \\
 & & & & 840 \\
\hline
6 & 大同 & 大同云冈石窟、忻州五台山风景名胜区 & 2.6 & 0.5+0.5 \\
 & & & & 685 \\
\hline
7 & 大同 & 忻州五台山风景名胜区,晋中市乔家大院文化园区 & 2.225 & 0.5+0.5 \\
 & & & & 539 \\
\hline
8 & 晋中市 & 太原 & 2.5 & 1 \\
 & & & & 828 \\
\hline
9 & 太原 & 晋中市平遥县平遥古城景区、晋中市介休市绵山风景名胜区 & 1 & 0.5+0.5 \\
 & & & & 539 \\
\hline
10 & 晋城 & 晋城阳城县皇城相府生态文化旅游区 & 5 & 0.5 \\
 & & & & 900 \\
\hline
11 & 晋城-西安途中 & 无 & 7.9 & 0 \\
 & & & & 711 \\
\hline
合计 & & & & 8142 \\
\hline
1 & 西安 & 哈尔滨 & 2.6 & 0.5 \\
 & & & & 6420 \\
\hline
2 & 哈尔滨 & 哈尔滨、哈尔滨太阳岛景区 & 0 & 0.5+0.5 \\
 & & & & 900 \\
\hline
3 & 哈尔滨 & 黑河五大连池景区 & 4.5 & 0.5 \\
 & & & & 1120 \\
\hline
4 & 哈尔滨 & 牡丹江宁安市镜泊湖景区 & 5 & 0.5 \\
 & & & & 1160 \\
\hline
5 & 哈尔滨 & 伊春市汤旺河林海奇石景区 & 5 & 0.5 \\
 & & & & 1200 \\
\hline
6 & 伊春市 & 无 & 7.4 & 0 \\
 & & & & 1464 \\
\hline
7 & 五大连池 & 大兴安岭地区漠河北极村旅游区 & 4.9 & 3.4 \\
 & & & & 1605 \\
\hline
8 & 哈尔滨 & 天津 & 1.4 & 1 \\
 & & & & 4290 \\
\hline
9 & 天津 & 天津蓟县盘山风景名胜区 & 0 & 1 \\
 & & & & 900 \\
\hline
9.5 & 天津 & 无 & 1.2 & 0 \\
 & & & & 3240 \\
\hline
合计 & & & & 22299 \\
\hline
1.00 & 西安-郑州 & 郑州 & 2.3 & 0 \\
 & & & & 1587 \\
\hline
2.00 & 郑州 & 郑州 & 0.8 & 0 \\
 & & & & 821 \\
\hline
3.00 & 登封市 & 郑州登封嵩山少林景区 & 0.8 & 1 \\
 & & & & 821 \\
\hline
\end{tabular}

\begin{tabular}{c c l c c c}
\hline
4.00 & 登封市 & 焦作(云台山-神农山-青天河)风景区 & 1.0 & 1 & 800 \\
\hline
5.00 & 登封市 & 焦作(云台山-神农山-青天河)风景区 & 1 & 750 \\
\hline
6.00 & 登封市 & 洛阳龙门石窟景区、洛阳嵩县白云山景区 & 3.8 & 243.1 & 993 \\
\hline
7.00 & 洛阳市 & 洛阳栾川县老君山-鸡冠洞旅游区 & 2.5 & 150 & 900 \\
\hline
8.00 & 洛阳市 & 洛阳新安县龙潭大峡谷景区 & 2.0 & 90 & 840 \\
\hline
9.00 & 洛阳市 & 平顶山鲁山县尧山-中原大佛景区 & 3.0 & 214 & 964 \\
\hline
10.00 & 洛阳市 & 南阳西峡伏牛山老界岭·恐龙遗址园旅游区 & 3.0 & 160 & 910 \\
\hline
11.00 & 平顶山 & 开封清明上河园景区、安阳殷墟景区 & 2.3 & 各0.5 & 960 \\
\hline
12.00 & 安阳 & 无 & 7.3 & 661.1 & 1648 \\
\hline
合计 & & & & & 11994 \\
\hline
1 & 西安 & 无 & 8.0 & 0 & 1020 \\
\hline
2 & 西安-济南途中 & 济南天下第一泉景区(趵突泉-大明湖-五龙潭-环城公园-黑虎泉) & 2.0 & 1 & 780 \\
\hline
3 & 济南 & 济南 & 0.0 & 0 & 600 \\
\hline
4 & 济南 & 泰安泰山景区 & 1.2 & 1 & 710 \\
\hline
5 & 泰安 & 济宁曲阜明故城三孔旅游区 & 2.0 & 0.5 & 780 \\
\hline
6 & 枣庄 & 枣庄台儿庄古城景区 & 5.0 & 0.5 & 1050 \\
\hline
7 & 青岛 & 青岛崂山景区 & 2.9 & 0.5 & 860 \\
\hline
8 & 威海 & 威海刘公岛景区,烟台龙口南山景区 & 1.7 & 各0.5 & 453 \\
\hline
\end{tabular}

\begin{tabular}{c c l c c c}
\hline
9 & 烟台 & 烟台蓬莱阁-三仙山-八仙过海旅游区 & 3.3 & 0.5 & 750 \\
\hline
10 & 潍坊 & 山东沂蒙山旅游区(沂山景区-龟蒙景区-云蒙景区) & 2.0 & 1 & 710 \\
\hline
11 & 潍坊-西安 & 无 & 8.0 & 0 & 720 \\
\hline
合计 & & & & & 8433 \\
\hline
1 & 西安 & 无 & 0.0 & 0 & 0 \\
\hline
2 & 西安 & 西安秦始皇兵马俑博物馆西安华清池景区 & 2.0 & 各0.5 & 48 \\
\hline
3 & 西安 & 延安黄陵县黄帝陵景区 & 5.0 & 0.5 & 120 \\
\hline
4 & 西安 & 渭南华阴市华山风景区 & 2.0 & 1 & 672 \\
\hline
5 & 渭南华阴市华山风景区 & 西安大雁塔-大唐芙蓉园景区 & 4.0 & 0.5 & 642 \\
\hline
6 & 宝鸡 & 宝鸡扶风县法门寺佛文化景区 & 2.0 & 1 & 120 \\
\hline
合计 & & & & & 1602 \\
\hline
1 & 西安 & 上海 & 2.0 & 0 & 5070 \\
\hline
2 & 上海 & 上海野生动物园 & 0.0 & 0.5 & 900 \\
\hline
3 & 上海 & 东方明珠广播电视塔,上海科技馆 & 0.0 & 各0.5 & 900 \\
\hline
4 & 上海 & 杭州 & 1.0 & 0 & 1119 \\
\hline
5 & 杭州 & 杭州西湖风景区嘉兴桐乡乌镇古镇 & 2.5 & 0.5 & 1002 \\
\hline
6 & 杭州 & 金华东阳横店影视城景区杭州 & 4.0 & 0.5 & 1200 \\
\hline
7 & 杭州淳安千岛湖风景区 & 杭州淳安千岛湖风景区杭州西溪湿地旅游区,湖州市南浔区南浔古镇景区 & 2.5 & 1 & 1002 \\
\hline
8 & 杭州 & 嘉兴南湖旅游区绍兴市鲁迅故里-沈园景区 & 1.2 & 0.5 & 860 \\
\hline
9 & 嘉兴 & & 1.3 & 0.5 & 867 \\
\hline
10 & 绍兴 & & & & \\
\hline
\end{tabular}

\begin{tabular}{l l l l l}
\hline
11 & 宁波 & 宁波奉化溪口- & 1.0 & 0.5 \\
 & & 滕头旅游景区 & & 840 \\
\hline
12 & 舟山 & 舟山普陀山风景区 & 0.0 & 1 \\
 & & & & 900 \\
\hline
13 & 舟山-温州 & 温州乐清市雁荡 & 4.0 & 1 \\
 & & 山风景区 & & 1110 \\
\hline
14 & 温州 & 衢州市开化根宫 & 4.2 & 0.5 \\
 & & 佛国文化旅游区 & & 1100 \\
\hline
14.5 & 衢州 & 无 & 3.5 & 0 \\
 & & & & 4140 \\
\hline
合计 & & & & 21940 \\
\hline
1 & 西安 & 南京 & 1.2 & 0 \\
 & & & & 4440 \\
 & & 南京钟山—中山 \\
 & & 陵风景名胜区(明 \\
 & & 孝陵-音乐台-灵谷 \\
 & & 寺-梅花山-紫金山 \\
 & & 天文台), \\
2 & 南京 & 南京夫子庙—秦 & 0.0 & 各0.5 \\
 & & & & 900 \\
 & & 淮河风光带(江南 \\
 & & 贡院-白鹭洲-中华 \\
 & & 门-瞻园-王谢故居) \\
\hline
3 & 南京 & 镇江三山风景名 & 1.5 & 1 \\
 & & 胜区(金山—北 & & 836 \\
 & & 固山—焦山) \\
\hline
4 & 镇江 & 镇江句容茅山景区 & 2.0 & 1 \\
 & & & & 810 \\
\hline
13 & 飞机 & 常州溧阳市天目 & & \\
5 & 镇江 & 湖景区(天目湖-南 & 1.0 & 1 \\
 & & 山竹海-御水温泉) & & 835 \\
\hline
 & & 常州环球恐龙城 & & \\
 & & 景区(中华恐龙园- & & \\
6 & 常州 & 恐龙谷温泉-恐龙 & 2.0 & 各0.5 \\
 & & 城大剧院), & & 1007 \\
 & & 中央电视台无锡 & & \\
 & & 影视基地三国水 & & \\
 & & 浒城景区 & & \\
\hline
7 & 无锡 & 无锡鼋头渚景区 & & \\
 & & 苏州园林(拙政 & 1.0 & 各0.5 \\
 & & 园—留园—虎丘) & & 792 \\
\hline
8 & 苏州 & 苏州市金鸡湖国家商务旅游示范区, & 2.0 & 各0.5 \\
 & & & & 840 \\
\hline
 & 45 & & & \\
\hline
\end{tabular}

\begin{tabular}{c c l c c c}
 & & 苏州昆山周庄古镇景区 & & & \\
 & & 苏州吴中太湖旅游区(旺山一穹 & & & \\
9 & 苏州 & 隆山一东山) & 0.0 & 1 & 750 \\
 & & 苏州常熟沙家浜 & & & \\
10 & 苏州 & 一虞山尚湖旅游区 & 0.0 & 1 & 750 \\
 & & 苏州吴江同里古镇景区, & & & \\
11 & 苏州 & 南通市濠河风景区 & 2.6 & 各0.5 & 984 \\
 & & 泰州姜堰区溱湖 & & & \\
12 & 泰州 & 国家湿地公园, & 2.0 & 各0.5 & 859 \\
 & & 扬州瘦西湖风景区 & & & \\
 & & 淮安市周恩来故 & & & \\
 & & 里景区(周恩来 & & & \\
13 & 扬州 & 纪念馆-周恩来故居-附马巷历史街 & 2.0 & 0.5 & 930 \\
 & & 区-河下古镇) & & & \\
14 & 淮安 & 南京 & 3.4 & 0 & 4040 \\
合计 & & & & & 18773 \\
1 & 西安 & 成都 & 8.0 & 0 & 1050 \\
2 & 成都 & 成都青城山一都江堰旅游景区 & 1.5 & 1 & 668 \\
 & & 阿坝藏族羌族自治州汶川特别旅 & & & \\
3 & 成都 & 游区(震中映秀一水磨古镇一三 & 2.0 & 1 & 710 \\
 & & 江生态旅游区) & & & \\
4 & 成都 & 乐山峨眉山景区 & 1.5 & 1 & 682.8 \\
5 & 乐山 & 乐山乐山大佛景区 & 0.0 & 0.5 & 600 \\
6 & 乐山 & 广安市邓小平故里旅游区 & 4.5 & 0.5 & 700.2 \\
7 & 广安 & 南充市阆中古城旅游景区 & 1.0 & 0.5 & 383.5 \\
8 & 南充 & 无 & 6.5 & 0 & 750 \\
9 & 广元 & 阿坝藏族羌族自治州九寨沟景区 & 0.0 & 1 & 600 \\
10 & 广元 & 无 & 6.5 & 0 & 630 \\
11 & 四川省广元市剑阁县 & 广元市剑门蜀道剑门关旅游景区 & 2.0 & 1 & 556.2 \\
 & & 46 & & & \\
\end{tabular}

\begin{tabular}{l l l l l}
12 & 绵阳 & 无 & 5.0 & 0 \\
 & & 阿坝藏族羌族自 & & 714 \\
13 & 阿坝藏族 & 治州松潘县黄龙 & 3.0 & 0.5 \\
 & & 风景名胜区 & & 678 \\
14 & 阿坝藏族-西安 & 无 & 8.0 & 0 \\
 & 途中 & & & 732 \\
14.5 & 阿坝藏族-西安 & 无 & 2.0 & 0 \\
 & 途中 & & & 108 \\
合计 & & & & 9562.7 \\
1 & 西安 & 无 & 7.8 & 0 \\
 & & & & 1300 \\
2 & 重庆 & 大足石刻景区 & 1.5 & 0.5 \\
 & & & & 660 \\
 & & 武隆喀斯特旅游 & & \\
3 & 重庆 & 区(天生三硚、 & 3.0 & 1 \\
 & & 仙女山、芙蓉 & & 490 \\
 & & 洞) & & \\
 & & 武隆喀斯特旅游 & & \\
4 & 重庆市武隆县 & 区(天生三硚、 & 0.0 & 1 \\
 & & 仙女山、芙蓉 & & 300 \\
 & & 洞) & & \\
5 & 重庆市万盛区黑 & 万盛黑山谷-龙鳞 & 2.0 & 1 \\
 & 山镇 & 石海风景区 & & 570 \\
6 & 重庆市南川区 & 南川金佛山-神 & 2.0 & 1 \\
 & & 龙峡风景区 & & 560 \\
15 & 自驾 & & & \\
7 & 重庆市武隆县 & 酉阳桃花源旅游 & 0.0 & 1 \\
 & & 景区 & & 600 \\
8 & 重庆-贵阳 & 无 & 4.2 & 0 \\
 & & & & 980 \\
9 & 贵阳 & 贵阳 & 2.5 & 0 \\
 & & & & 670 \\
10 & 毕节 & 毕节市百里杜鹃 & 3.0 & 1 \\
 & & 景区 & & 550 \\
11 & 毕节-安顺途中 & 安顺镇宁县黄果 & 1.0 & 0.5 \\
 & & 树瀑布景区 & & 390 \\
12 & 安顺 & 安顺龙宫景区 & 5.0 & 0.5 \\
 & & & & 900 \\
13 & 河池 & 黔南布依族苗族 & & \\
 & & 自治州荔波樟江 & 2.5 & 1 \\
 & & 景区 & & 640 \\
14 & 河池 & 无 & 5.0 & 0 \\
 & & & & 1050 \\
15 & 贵阳-河池 & 无 & 7.0 & 0 \\
 & & & & 630 \\
合计 & & & & 10290 \\
1 & 西安-武汉 & 武汉 & 3.9 & 武汉0.5 \\
 & & & & 2463.5 \\
7 & 高铁+ & 武汉,武汉黄鹤 & 0.0 & 武汉 \\
16 & 租车 & 楼公园 & & 0.5,黄 \\
 & 2 & & & 900 \\
 & & & & 鹤楼公 \\
 & & & & 园0.5 \\
\end{tabular}

\begin{tabular}{c c l c c c}
\hline
3 & 武汉 & 武汉市东湖景区, 武汉市黄陂木兰文化生态旅游区 & 0.0 & 各0.5 & 900 \\
\hline
4 & 武汉-宜昌 & 宜昌三峡大坝旅游区 & 4.6 & 0.5 & 1097 \\
\hline
5 & 宜昌 & 宜昌三峡人家风景区 & 2.0 & 1 & 774 \\
\hline
6 & 宜昌 & 恩施土家族苗族自治州巴东神龙溪纤夫文化旅游区 & 6.0 & 0.5 & 894 \\
\hline
7 & 宜昌 & 宜昌长阳县清江画廊景区, 宜昌秭归县屈原故里文化旅游区 & 2.5 & 各0.5 & 810 \\
\hline
8 & 恩施宜昌-恩施 & 恩施土家族苗族自治州恩施大峡谷景区 & 2.7 & 1 & 990 \\
\hline
9 & 恩施-十堰 & 无 & 6.8 & 0 & 1360 \\
\hline
10 & 十堰 & 神农架生态旅游区 & 4.0 & 1 & 846 \\
\hline
11 & 十堰 & 神农架生态旅游区 & 0.0 & 1 & 750 \\
\hline
12 & 十堰 & 十堰丹江口市武当山风景区 & 1.0 & 1 & 774 \\
\hline
13 & 十堰-武汉-西安 & 无 & 8.8 & 0 & 2106.5 \\
\hline
合计 & & & & & 14665 \\
\hline
1 & 西安-长沙 & 长沙 & 5.3 & 0.5 & 2661 \\
\hline
2 & 长沙 & 长沙+湘潭韶山旅游区 & 0.0 & 各0.5 & 900 \\
\hline
3 & 长沙 & 长沙岳麓山一橘子洲旅游区+长沙市宁乡县花明楼景区 & 0.0 & 各0.5 & 900 \\
\hline
17 & 高铁+租车 & & & & \\
4 & 长沙-张家界 & 张家界武陵源一天门山旅游区 & 3.7 & 0.5 & 1072 \\
\hline
5 & 张家界 & 张家界武陵源一天门山旅游区 & 0.0 & 1 & 750 \\
\hline
6 & 张家界-岳阳 & 张家界武陵源一天门山旅游区 & 3.6 & 0.5 & 1074 \\
\hline
7 & 岳阳-衡阳 & 岳阳岳阳楼一君山岛景区 & 3.6 & 0.5 & 1070 \\
\hline
\end{tabular}

48

\begin{tabular}{r l l r r r}
\hline
8 & 衡阳-郴州 & 衡阳南岳衡山旅游区 & 2.0 & 0.5 & 928 \\
9 & 郴州-长沙 & 郴州市东江湖旅游区 & 3.4 & 0.5 & 1207 \\
10 & 长沙-西安 & 长沙 & 0.0 & 0 & 2061 \\
合计 & & & & & 12623 \\
\hline
1 & 西安-南昌 & 南昌 & 1.2 & 0.5 & 3930 \\
2 & 南昌-九江 & 南昌 & 1.5 & 1 & 885 \\
3 & 九江 & 九江庐山风景名胜区 & 1.0 & 1 & 774 \\
4 & 九江-景德镇-上饶 & 景德镇古窑民俗博览区 & 3.8 & 0.5 & 1096 \\
5 & 上饶 & 上饶三清山旅游景区 & 0.0 & 1 & 750 \\
6 & 上饶 & 上饶婺源县江湾景区 & 0.0 & 1 & 750 \\
7 & 上饶-鹰潭市贵溪市 & 鹰潭市贵溪龙虎山风景名胜区 & 0.9 & 1 & 835 \\
8 & 鹰潭市贵溪市-赣州市瑞金市 & 赣州市瑞金市共和国摇篮景区 & 3.9 & 0.5 & 1100 \\
9 & 赣州市瑞金市-吉安井冈山-南昌 & 吉安井冈山旅游区 & 3.1 & 0.5 & 1030 \\
10 & 南昌-西安 & 无 & 1.2 & 0 & 3659 \\
合计 & & & & & 14809 \\
\hline
1 & 西安-昆明 & 昆明 & 1.8 & 0.5 & 5520 \\
2 & 昆明 & 昆明+昆明石林风景区 & 0.0 & 各0.5 & 900 \\
3 & 昆明-景洪 & 无 & 5.8 & 0 & 1273 \\
4 & 景洪 & 中科院西双版纳热带植物园 & 1.0 & 1 & 774 \\
5 & 景洪-大理 & 无 & 7.0 & 0 & 1470 \\
6 & 大理-香格里拉 & 大理崇圣寺三塔文化旅游区 & 3.7 & 0.5 & 1086.7 \\
7 & 香格里拉 & 迪庆藏族自治州香格里拉普达措国家公园 & 1.0 & 1 & 790 \\
8 & 香格里拉-丽江 & 丽江玉龙雪山景区 & 1.9 & 1 & 924.1 \\
9 & 丽江-昆明 & 丽江古城景区 & 4.0 & 0.5 & 1170 \\
10 & 昆明-西安 & 无 & 3.3 & 0 & 5050 \\
合计 & & & & & 18957.8 \\
\hline
20 & 自驾 & 西安-合肥 & 无 & 8.0 & 0 & 1170 \\
\hline
\end{tabular}

\begin{tabular}{c c c c c}
\hline
2 & 合肥-黄山 & 合肥 & 4.9 & 1 \\
 & & 黄山市黄山风景区 & 1.0 & 1 \\
3 & 黄山 & & & 490 \\
 & & 黄山市黟县皖南 & & \\
4 & 黄山 & 古村落-西递宏村 & 1.0 & 1 \\
 & & & & 490 \\
5 & 黄山 & 宣城市绩溪县龙川景区 & 1.5 & 1 \\
 & & & & 490 \\
6 & 黄山-池州 & 黄山市古徽州文化旅游区 & 3.9 & 0.5 \\
 & & & & 724 \\
7 & 池州 & 池州青阳县九华山风景区 & 1.0 & 1 \\
 & & & & 490 \\
8 & 安庆潜山 & 安庆潜山县天柱山风景区 & 1.0 & 0 \\
 & & & & 450 \\
9 & 安庆潜山-麻城市 & 六安市金寨县天堂寨旅游景区 & 3.5 & 0.5 \\
 & & & & 765 \\
10 & 麻城-阜阳-合肥 & 阜阳市颍上县八里河风景区 & 5.1 & 0.5 \\
 & & & & 913 \\
11 & 合肥-西安 & 无 & 8.0 & 0 \\
 & & & & 720 \\
合计 & & & & 7743 \\
\hline
1 & 西安-福州 & 福州 & 1.8 & 0.5 \\
 & & 福州+福州市三坊七巷景区 & 0.0 & 0.5 \\
2 & 福州 & & & 900 \\
3 & 福州-宁德市 & 宁德市福鼎太姥山旅游区 & 1.1 & 1 \\
 & & & & 850 \\
4 & 宁德市 & 宁德屏南(白水洋·鸳鸯溪)旅游景区 & 2.5 & 1 \\
 & & & & 810 \\
5 & 宁德-福州-泉州 & 福州市三坊七巷景区 & 3.1 & 0.5 \\
 & & & & 1030 \\
6 & 泉州 & 泉州市清源山风景名胜区 & 0.0 & 1 \\
 & & & & 750 \\
7 & 泉州-厦门-龙岩市 & 厦门鼓浪屿风景名胜区 & 2.9 & 0.5 \\
 & & & & 1010 \\
8 & 龙岩市 & 福建土楼(永定·南靖)旅游景区 & 2.0 & 1 \\
 & & & & 798 \\
9 & 龙岩市-武夷山市 & 南平武夷山风景名胜区 & 1.9 & 1 \\
 & & & & 1222 \\
10 & 武夷山市 & 三明泰宁风景旅游区 & 0.0 & 1 \\
 & & & & 750 \\
11 & 武夷山市-福州-西安 & 无 & 5.7 & 0 \\
 & & & & 5237 \\
\hline
\end{tabular}

\begin{tabular}{l l l l l l}
 & & & & & 18847 \\
合计 & & & & & \\
1 & 西安-海口 & 海口 & 2.3 & 0.5 & 6390 \\
2 & 海口-三亚 & 海口 & 3.0 & 1 & 1024 \\
3 & 三亚 & 三亚南山文化旅游区+三亚南山大 & 2.0 & 各0.5 & 798 \\
 & & 小洞天旅游区 & & & \\
4 & 三亚 & 保亭县呀诺达雨林文化旅游区+陵水县分界洲岛旅游区 & 2.0 & 各0.5 & 798 \\
5 & 三亚-海口-南宁 & 保亭县海南槟榔谷黎苗文化旅游区 & 3.8 & 0.5 & 3874 \\
6 & 南宁 & 南宁 & 0.0 & 1 & 900 \\
7 & 南宁 & 南宁市青秀山旅游区 & 0.0 & 0.5 & 750 \\
8 & 南宁-桂林 & 无 & 4.5 & 0 & 1152 \\
9 & 桂林市 & 桂林漓江风景区 & 0.0 & 1 & 750 \\
10 & 桂林市 & 桂林兴安县乐满地度假世界 & 1.5 & 1 & 786 \\
11 & 桂林市 & 桂林独秀峰·靖江王城景区 & 3.5 & 1 & 1064 \\
12 & 桂林市-南宁 & 无 & 4.5 & 0 & 1302 \\
13 & 南宁-西安 & 南宁 & 2.3 & 0.5 & 5700 \\
合计 & & & & & 25288 \\
1 & 西安-广州 & 广州 & 1.8 & 0.5 & 5520 \\
2 & 广州 & 广州+广州白云山景区 & 0.0 & 1 & 900 \\
3 & 广州 & 广州长隆旅游度假区 & 0.0 & 1 & 900 \\
4 & 广州-佛山-深圳 & 佛山西樵山景区+深圳华侨城旅游度假区 & 2.2 & 各0.5 & 946 \\
5 & 深圳-惠州 & 深圳观澜湖休闲旅游区+惠州市罗浮山景区 & 1.0 & 各0.5 & 842.1 \\
6 & 惠州-梅州 & 惠州市罗浮山景区+梅州市梅县区雁南飞茶田景区 & 3.0 & 各0.5 & 1021.8 \\
7 & 梅州-韶关 & 韶关仁化丹霞山景区 & 4.6 & 0.5 & 1162.1 \\
8 & 韶关-郴州 & 清远连州地下河旅游景区 & 1.9 & 1 & 919.8 \\
 & & 51 & & & \\
\end{tabular}

\begin{tabular}{r l l r r r}
\hline
8 & 新疆博湖县 & 无 & 8.0 & 0 & 848.6 \\
9 & 乌鲁木齐-石河子途中 & 伊犁地区新源县那拉提旅游风景区 & 0.6 & 1 & 600 \\
10 & 石河子新源县 & 无 & 7.4 & 0 & 750 \\
11 & 阿勒泰 & 无 & 6 & 0 & 950 \\
12 & 阿勒泰布尔津县 & 阿勒泰地区布尔津县喀纳斯景区 & 0 & 1 & 700 \\
13 & 阿勒泰布尔津县 & 阿勒泰地区布尔津县喀纳斯景区 & 0 & 1 & 600 \\
14 & 阿勒泰布尔津县 & 无 & 7.93 & 0 & 1613.7 \\
15 & 乌鲁木齐 & 无 & 2.82 & 0 & 6600 \\
\hline
合计 & & & & & 23912.3 \\
\hline
1 & 西安 & 无 & 8 & 0 & 7453.8 \\
2 & 乌鲁木齐-喀什途中 & 无 & 8 & 0 & 957.2 \\
3 & 喀什 & 喀什地区泽普县金胡杨景区 & 3 & 1 & 863.8 \\
4 & 喀什-乌鲁木齐途中 & 无 & 8 & 0 & 1470 \\
5 & 喀什-乌鲁木齐途中 & 无 & 7.82 & 0 & 7410 \\
\hline
合计 & & & & & 18154.8 \\
\hline
1 & 西安 & 拉萨 & 3.2 & 0.5 & 6570 \\
2 & 拉萨 & 拉萨、拉萨布达拉宫景区 & 0 & 0.5+0.5 & 900 \\
3 & 拉萨 & 拉萨大昭寺景区 & 3.2 & 0.5 & 6120 \\
\hline
合计 & & & & & 13590 \\
\hline
1 & 西安 & 无 & 8.0 & 0 & 1170 \\
2 & 西安-银川途中 & 银川 & 3.2 & 0.5 & 878 \\
3 & 银川 & 银川、石嘴山平罗县沙湖旅游景区 & 1.0 & 0.5+0.5 & 810 \\
4 & 银川 & 中卫沙坡头旅游景区 & 3.0 & 1 & 980 \\
5 & 银川 & 银川镇北堡西部影视城、银川市灵武水洞沟旅游景区 & 2.0 & 0.5+0.5 & 820 \\
6 & 银川 & 无 & 8.0 & 0 & 1170 \\
6.5 & 银川-西安途中 & 无 & 3.2 & 0 & 128 \\
\hline
合计 & & & & & 5956 \\
\hline
6 & 1 & 西安 & 石家庄 & 3.45 & 0.5 & 3120 \\
\hline
\end{tabular}

\section{问题三的求解}

问题三是问题二的推广。对于全国的自驾游爱好者来说,问题二所建模型可以基于其出发地计算出10年旅游规划路线。基于问题二所建立的模型,在问题三中只需要修改出发地点和模型输入变量即可得到自北京出发的10年旅游规划。

\subsection{模型的求解}

基于问题二中所建立的模型,输入数据包括出发地点,出发地点到各省会城市的距离,高铁票价信息,飞机票价信息和自驾出游费用信息。从北京出发,距离及票价的具体信息如表10所示。

\begin{tabular}{c c c c c}
\hline
2 & 石家庄 & 石家庄, 石家庄平山县西柏坡景区 & 2 & 0.5+0.5 \\
 & & & & 900 \\
\hline
3 & 平山县 & 石家庄平山县西柏坡景区 & 3.6 & 0.5 \\
 & & & & 900 \\
\hline
4 & 保定涞水县 & 保定涞水县野三坡景区(百里峡-白草畔-鱼谷洞-龙门天关) & 0 & 1 \\
 & & & & 600 \\
\hline
5 & 保定涞水县 & 保定安新白洋淀景区(文化苑-大观园-鸳鸯岛-元妃荷园-嘎子印象-渔人乐园) & 0.8 & 1 \\
 & & & & 600 \\
\hline
6 & 雄县 & 无 & 3.6 & 0 \\
 & & & & 1217.5 \\
\hline
7 & 北京 & 北京 & 0 & 1 \\
 & & & & 900 \\
\hline
高铁+ & 8 & 北京 & 故宫博物院、天坛公园 & 0 \\
租车 & & & & 0.5+0.5 \\
 & & & & 900 \\
\hline
9 & 北京 & 颐和园、八达岭—慕田峪长城旅游区 & 1.5 & 0.5+0.5 \\
 & & & & 670 \\
\hline
10 & 北京市延庆县 & 明十三陵景区(神路-定陵-长陵-昭陵) & 1 & 1 \\
 & & & & 960 \\
\hline
11 & 北京 & 恭王府景区 & 0 & 1 \\
 & & & & 900 \\
\hline
12 & 北京 & 北京奥林匹克公园(鸟巢-水立方-中国科技馆-国家奥林匹克森林公园) & 0 & 1 \\
 & & & & 900 \\
\hline
13 & 北京 & 天津古文化街旅游区(津门故里) & 2.5 & 1 \\
 & & & & 495 \\
\hline
13.5 & 北京 & 无 & 4.62 & 1546.5 \\
\hline
合计 & & & & 14609 \\
\hline
1 & 西安 & 无 & 8 & 0 \\
 & & & & 1170 \\
\hline
2 & 西安-呼和浩特途中 & 呼和浩特 & 2.8 & 0.5 \\
 & & & & 850 \\
\hline
3 & 呼和浩特 & 呼和浩特 & 1 & 1 \\
 & & & & 540 \\
\hline
3 & 7 & 自驾 & 鄂尔多斯伊金霍洛旗成吉思汗陵旅游区、鄂尔多斯达拉特旗响沙湾旅游景区 & 1.5 & 0.5+0.5 \\
 & & & & & 540 \\
\hline
41 & & & & & \\
\hline
\end{tabular}

\begin{tabular}{l r r r r r r r r r}
\multicolumn{1}{c}{距离} & 140 & 300 & 510 & 490 & 700 & 980 & 1240 & 1060 & 1300 \\
\multicolumn{1}{c}{飞机票价} & - & - & 1070 & 550 & 1470 & 1050 & 1240 & 1780 & 1860 \\
\multicolumn{1}{c}{高铁票价} & 54.5 & 128.5 & 197 & - & 206 & 265.5 & - & 553 & 443.5 \\
\multicolumn{1}{c}{自驾费用} & 140 & 300 & 510 & 490 & 700 & 980 & 1240 & 1210 & 1060 \\
\multicolumn{1}{c}{省会城市} & 合肥 & 福州 & 南昌 & 济南 & 郑州 & 长沙 & 武汉 & 广州 & 南宁 \\
\multicolumn{1}{c}{距离} & 1040 & 1900 & 1430 & 440 & 700 & 1470 & 1200 & 2120 & 2310 \\
\multicolumn{1}{c}{飞机票价} & 1710 & 1680 & 1430 & 630 & 1160 & 1450 & 1850 & 1910 & 2150 \\
\multicolumn{1}{c}{高铁票价} & 427.5 & - & - & 184.5 & 315 & 649 & 519 & - & - \\
\multicolumn{1}{c}{自驾费用} & 1040 & 1900 & 1430 & 440 & 700 & 1470 & 1200 & 2120 & 2310 \\
\multicolumn{1}{c}{省会城市} & 重庆 & 成都 & 贵阳 & 昆明 & 拉萨 & 西安 & 兰州 & 西宁 & 银川 \\
\multicolumn{1}{c}{距离} & 1750 & 1800 & 2130 & 2670 & 3650 & 1080 & 1500 & 1700 & 1150 \\
\multicolumn{1}{c}{飞机票价} & 1640 & 1690 & 1980 & 2180 & 2930 & 1850 & 1390 & 1740 & 1180 \\
\multicolumn{1}{c}{高铁票价} & - & - & - & - & - & 515.5 & - & - & - \\
\multicolumn{1}{c}{自驾费用} & 1750 & 1800 & 2130 & 2670 & 3650 & 1080 & 1500 & 1700 & 1150 \\
\end{tabular}

同样运用MATLAB程序,计算得到的北京的10年规划路线如表11和图8所示:

\begin{figure}[h]
\centering
\includegraphics[width=0.8\textwidth]{image.png}
\caption{10年出行旅游规划}
\end{figure}

\begin{tabular}{c c c c c}
\hline
5 & 鄂尔多斯 & 鄂尔多斯达拉特旗响沙湾旅游区 & 5.8 & 0.5 \\
 & & & & 840 \\
\hline
6 & 大同 & 大同云冈石窟、忻州五台山风景名胜区 & 2.6 & 0.5+0.5 \\
 & & & & 685 \\
\hline
7 & 大同 & 忻州五台山风景名胜区,晋中市乔家大院文化园区 & 2.225 & 0.5+0.5 \\
 & & & & 539 \\
\hline
8 & 晋中市 & 太原 & 2.5 & 1 \\
 & & & & 828 \\
\hline
9 & 太原 & 晋中市平遥县平遥古城景区、晋中市介休市绵山风景名胜区 & 1 & 0.5+0.5 \\
 & & & & 539 \\
\hline
10 & 晋城 & 晋城阳城县皇城相府生态文化旅游区 & 5 & 0.5 \\
 & & & & 900 \\
\hline
11 & 晋城-西安途中 & 无 & 7.9 & 0 \\
 & & & & 711 \\
\hline
合计 & & & & 8142 \\
\hline
1 & 西安 & 哈尔滨 & 2.6 & 0.5 \\
 & & & & 6420 \\
\hline
2 & 哈尔滨 & 哈尔滨、哈尔滨太阳岛景区 & 0 & 0.5+0.5 \\
 & & & & 900 \\
\hline
3 & 哈尔滨 & 黑河五大连池景区 & 4.5 & 0.5 \\
 & & & & 1120 \\
\hline
4 & 哈尔滨 & 牡丹江宁安市镜泊湖景区 & 5 & 0.5 \\
 & & & & 1160 \\
\hline
5 & 哈尔滨 & 伊春市汤旺河林海奇石景区 & 5 & 0.5 \\
 & & & & 1200 \\
\hline
6 & 伊春市 & 无 & 7.4 & 0 \\
 & & & & 1464 \\
\hline
7 & 五大连池 & 大兴安岭地区漠河北极村旅游区 & 4.9 & 3.4 \\
 & & & & 1605 \\
\hline
8 & 哈尔滨 & 天津 & 1.4 & 1 \\
 & & & & 4290 \\
\hline
9 & 天津 & 天津蓟县盘山风景名胜区 & 0 & 1 \\
 & & & & 900 \\
\hline
9.5 & 天津 & 无 & 1.2 & 0 \\
 & & & & 3240 \\
\hline
合计 & & & & 22299 \\
\hline
1.00 & 西安-郑州 & 郑州 & 2.3 & 0 \\
 & & & & 1587 \\
\hline
2.00 & 郑州 & 郑州 & 0.8 & 0 \\
 & & & & 821 \\
\hline
3.00 & 登封市 & 郑州登封嵩山少林景区 & 0.8 & 1 \\
 & & & & 821 \\
\hline
\end{tabular}

\begin{tabular}{c c c c}
 & & & 宣城市绩溪县龙川景区;阜阳市颍上县八里河风景区;黄山市古徽州文化旅游区(徽州古城一牌坊群鲍家花园一唐模一潜口民宅一呈坎) \\
 & & & 苏州园林(拙政园一留园一虎丘) \\
 & & & 苏州昆山周庄古镇景区;南京钟山一中山陵风景名胜区(明孝陵-音乐台-灵谷寺-梅花山-紫金山天文台);中央电视台无锡影视基地三国水浒城景区;无锡灵山大佛景区;苏州吴江同里古镇景区;南京夫子庙一秦淮河风光带(江南贡院-白鹭洲-中华门-瞻园-王谢故居);常州环球恐龙城景区(中华恐龙园-恐龙谷温泉-恐龙城大剧院);扬州瘦西湖风景区;南通市濠河风景区;泰州姜堰区溱湖国家湿地公园;苏州市金鸡湖国家商务旅游示范区;镇江三山风景名胜区(金山一北固山一焦山);无锡鼋头渚景区;苏州吴中太湖旅游区(旺山一穹窿山一东山);苏州常熟沙家浜一虞山尚湖旅游区;常州溧阳市天目湖景区(天目湖-南山竹海-御水温泉);镇江句容茅山景区;淮安市周恩来故里景区(周恩来纪念馆-周恩来故居-附马巷历史街区-河下古镇) \\
4 & 高铁+ & 北京-南京 & 15 \\
 & 租车 & -北京 & \\
 & & & 杭州西湖风景区;温州乐清市雁荡山风景区;舟山普陀山风景区;杭州淳安千岛湖风景区;嘉兴桐乡乌镇古镇旅游区;宁波奉化溪口一滕头旅游景区;金华东阳横店影视城景区;嘉兴南湖旅游区;杭州西溪湿地旅游区;绍兴市鲁迅故里一沈园景区;衢州市开化根宫佛国文化旅游区;湖州市南浔区南浔古镇景区 \\
10 & 高铁+ & 北京-杭州 & 15 \\
 & 租车 & -上海-北京 & \\
 & & & 昌吉州阜康市天山天池风景名胜区;吐鲁番葡萄沟风景区;阿勒泰地区布尔津县喀纳斯景区;伊犁地区新源县那拉提旅游风景区;阿勒泰地区富蕴县可可托海景区;乌鲁木齐天山大峡谷;巴音郭楞蒙古自治州博湖县博斯腾湖景区 \\
5 & 飞机+ & 北京-乌鲁木齐 & 14 \\
 & 租车 & (13)-北京 & \\
 & & & 嘉峪关文物景区;平凉崆峒山风景名胜区;天水麦积山景区;酒泉市敦煌沙山月牙泉景区;青海湖风景区;西宁市湟中县塔尔寺景区;喀什地区泽普县金胡杨景区;喀什地区噶尔老城景区 \\
12 & 飞机+ & 北京-兰州 & 12.5 \\
 & 租车 & -西宁-乌鲁木齐 & \\
 & & (4)-北京 & \\
 & & & 拉萨布达拉宫景区;拉萨大昭寺景区;大足石刻景区;巫山小三峡一小小三峡旅游区;武隆喀斯特旅游区(天生三硚、仙女山、芙蓉洞);酉阳桃花源旅游景区;万盛黑山谷-龙鳞石海风景区;南川金佛山一神龙峡风景区 \\
6 & 13 & 飞机+ & 11.5 \\
 & & 租车 & 北京-拉萨 \\
 & & & -重庆-北京 \\
\end{tabular}

\begin{tabular}{c c c c}
 & 飞机+ & 北京-成都 & 13.5 \\
14 & 租车 & -北京 & \\
 & & & \\
 & & & 成都青城山-都江堰旅游景区;乐山峨眉山景区;阿坝藏族羌族自治州九寨沟景区;乐山乐山大佛景区;阿坝藏族羌族自治州松潘县黄龙风景名胜区;绵阳北川羌城旅游区(中国羌城-老县城地震遗址-“5·12”特大地震纪念馆-北川羌族民俗博物馆-北川新县城-吉娜羌寨);阿坝藏族羌族自治州汶川特别旅游区(震中映秀-水磨古镇-三江生态旅游区);南充市阆中古城旅游景区;广安市邓小平故里旅游区;广元市剑门蜀道剑门关旅游景区 \\
 & & & \\
 & 飞机+ & 北京-昆明 & 13 \\
15 & 租车 & -贵阳 & \\
 & & & 昆明石林风景区;丽江玉龙雪山景区;丽江古城景区;大理崇圣寺三塔文化旅游区;中科院西双版纳热带植物园;迪庆藏族自治州香格里拉普达措国家公园 \\
 & & & \\
7 & & & 桂林漓江风景区;桂林兴安县乐满地度假世界;桂林独秀峰·靖江王城景区;南宁市青秀山旅游区;三亚南山文化旅游区;三亚南山大小洞天旅游区;保亭县呀诺达雨林文化旅游区;陵水县分界洲岛旅游区;保亭县海南槟榔谷黎苗文化旅游区 \\
 & 飞机+ & 北京-南宁 & 13 \\
16 & 租车 & -海口-北京 & \\
 & & & \\
 & 高铁+ & 北京-长沙 & 10 \\
17 & 租车 & -北京 & 张家界武陵源-天门山旅游区;衡阳南岳衡山旅游区;湘潭韶山旅游区;岳阳岳阳楼-君山岛景区;长沙岳麓山-橘子洲旅游区;长沙市宁乡县花明楼景区;郴州市东江湖旅游区 \\
 & & & \\
8 & & & 武汉黄鹤楼公园;宜昌三峡大坝旅游区;宜昌三峡人家风景区;十堰丹江口市武当山风景区;恩施土家族苗族自治州巴东神龙溪纤夫文化旅游区;神农架生态旅游区;宜昌长阳县清江画廊景区武汉市东湖景区;宜昌秭归县屈原故里文化旅游区;武汉市黄陂木兰文化生态旅游区;恩施土家族苗族自治州恩施大峡谷景区 \\
 & 高铁+ & 北京-武汉 & 13.5 \\
18 & 租车 & -北京 & \\
 & & & \\
 & 飞机+ & 北京-南昌 & 8.5 \\
19 & 租车 & -北京 & 九江庐山风景名胜区;吉安井冈山风景旅游区;上饶三清山旅游景区;鹰潭市贵溪龙虎山风景名胜区;上饶婺源县江湾景区;景德镇古窑民俗博览区;赣州市瑞金市共和国摇篮景区 \\
 & & & \\
9 & & & 厦门鼓浪屿风景名胜区;南平武夷山风景名胜区;三明泰宁风景旅游区;福建土楼(永定·南靖)旅游景区;宁德屏南(白水洋·鸳鸯溪)旅游景区;泉州市清源山风景名胜区;宁德市福鼎太姥山旅游区;福州市三坊七巷景区 \\
 & 飞机+ & 北京-福州 & 10.5 \\
20 & 租车 & -北京 & \\
 & & & \\
 & 飞机+ & 北京-广州 & 10 \\
21 & 租车 & -北京 & 广州长隆旅游度假区;深圳华侨城旅游度假区;广州白云山景区;梅州市梅县区雁南飞茶田景区;深圳观澜湖休闲旅游区;清远连州地下河旅游景区;韶关仁化丹霞山景区;佛山西樵山景 \\
 & & & \\
10 & & & \\
\end{tabular}

\begin{tabular}{c c c c}
\hline
 & & & 区;惠州市罗浮山景区;佛山市德顺区长鹿旅游 \\
 & & & 休博园 \\
\hline
22 & 自驾 & 北京 & 7.5 故宫博物院;天坛公园;颐和园;八达岭—慕田 \\
 & & & 峪长城旅游区;明十三陵景区(神路-定陵-长陵- \\
 & & & 昭陵);恭王府景区;北京奥林匹克公园(鸟巢- \\
 & & & 水立方-中国科技馆-国家奥林匹克森林公园); \\
 & & & 天津古文化街旅游区(津门故里);天津蓟县盘 \\
 & & & 山风景名胜区;承德避暑山庄及周围寺庙景区 \\
 & & & (普陀宗乘-须弥福寺-普宁寺-普佑寺);秦皇岛 \\
 & & & 山海关景区(老龙头-山海关古城-天下第一关-孟 \\
 & & & 姜女庙) \\
\hline
\end{tabular}

\subsection{建议的提出}

通过对比时间最短的自驾出游方案和体验最佳、费用最优的组合方式出游方案的结果,本文对旅游者和相关旅游部门提出部分可借鉴性的建议,以期望推动中国旅游业稳步发展。

\subsubsection{对旅游者的建议}

对于旅游者而言,有选择出游交通方式的自主权,可供选择的方式有自驾游、“飞机+租车”、“高铁/铁路+租车”等。随着全国交通网络完善,以及游客对个性化出游的追求,自驾游已经越来越普遍。但自驾游在满足自由旅游的同时,其高强度的驾车时长要求会大大降低旅游体验效果。旅游者应当根据旅游景点的分布不同,适当选择出游交通方式。

\textbf{(1) 中短途自驾,提升旅游体验}

完全自驾出游只建议中短途距离的游览,当城市之间的车程超过两天时,长距离的驾车不仅不能达到休闲放松的目的,也存在较大的安全隐患。因此,为使旅游体验最大化,可在中短途(两天车程之内)的旅程中选择自驾出游方式。

\textbf{(2) 组合出游,高铁优先}

建议高铁+租车、飞机+租车的组合出游方式,既可以满足自驾出游的个性化、自由的要求,也能够避免长距离驾车路途的劳顿。由自驾、高铁和飞机的费用-距离曲线可以看出,同等距离条件下,高铁的费用远低于其他两种方式。与自驾相比,高铁不仅速度占优势,也可减少驾车路途劳累,因此当出行的省市与出发地有高铁班次时,高铁应作为首要选择。

\textbf{(3) 目标明确,合理规划}

确定每次出游的目的,合理规划线路,选择组合游览景点。由第一二问可知,以时间最短为目标的最佳出游路径与体验最佳、费用最优的出游方案完全不同。旅游者应当根据自身经济、时间等条件,制定合理的旅游路径,使旅游效用最大化。

\subsubsection{对旅游部门的建议}

\begin{enumerate}
    \item 构建自驾游服务体系

    自驾出游需要完备的自驾游服务作为保障,构建维修、服务、医疗等基本保障体系是自驾安全出游的前提。比如,设立汽车维修站提供维修服务,以防自驾汽车因自身或其他客观原因造成的意外故障;其次,在服务方面,加油站、停车场等相关服务站点的提供要为汽车的正常运行提供保障,医疗站、医疗组织等设施、设备供应齐全以便救援工作正常及时展开等。

    \item 完善景区交通网络

    由景区至各邻近城市的距离时间可以看出, 部分景区附近的交通网络十分不完善, 在一定程度上将降低对游客的吸引力。因此, 建议旅游局与相关部门提高景区之间的通达性, 形成景区间的带动作用。进一步完善景区内交通标识和旅游标识, 进一步完善公路网络与交通标识、提升公路等级要求, 加强道路绿化美化。

    \item 开展自驾游主题活动

    当地旅游部门应当扩大宣传, 举办旅游节, 开展自驾游的主题活动。将当地风土人情特色与景点旅游相结合, 利用其影响力, 提升本省市特种旅游的整体形象。如新疆千车万人穿越塔克拉玛干大沙漠自驾游活动, 千人千车环准噶尔盆地自驾游等活动的举办, 很大程度的提高了新疆旅游的影响力。

    \item 建立自驾车营地, 完善租赁市场

    在交通干线和风景优美之地或者旅游景区附近开设旅游营地, 为自驾车爱好者提供自助或半自助服务。同时, 成立自驾车旅游协会, 负责制定行业规范, 依靠组织力量监督行业内的不良行为, 解决租费费用、租车手续、保险等一系列纠纷问题, 保汽车租赁市场稳定有序发展。
\end{enumerate}

\subsection{模型四的求解}

\subsubsection{问题分析}

国家 5A 级旅游景区是中国旅游景区的最高等级, 代表着中国世界级精品的旅游风景区, 其对旅游体验要求之高、评审之严格为所有景区评选之最, 因此我国 5A 级旅游景区具有地理位置相对分散的特征, 除个别省市 (如北京, 江苏等) 因历史、文化、自然风光秀美的原因而具有较多的 5A 级景区之外, 多数 5A 景区彼此之间距离过远, 单纯的以游览 5A 景区为目指定出游计划, 会出现飞越数千公里、再驾车数天到达景点游览半天就返程的得不偿失的问题, 因此, 在出行距离较长、5A 景区相对分散的区域, 有必要将 4A 景区纳入出行旅游计划, 以此来提高游览时间在整个出行时间 (包括游览时间和在途时间) 的比重, 给出游者一个更好的旅游体验。

题中的出行时间可以看作是整个出行的时间成本, 游览时间可以看作是整个时间成本的产出, 即收益。

这里引入经济学中的一个概念: 边际成本 (Marginal cost)。在经济学和金融学中, 边际成本指的是每一单位新增生产的产品 (或者购买的产品) 带来的总成本的增量。这个概念表明每一单位的产品的成本与总产品量有关。比如, 仅生产一辆汽车的成本是极其巨大的, 而生产第 101 辆汽车的成本就低得多, 而生产第 10000 辆汽车的成本就更低了 (这是因为规模经济)。但是, 考虑到机会成本, 随着生产量的增加, 边际成本可能会增加。还是这个例子, 生产新的一辆车时, 所用的材料可能有更好的用处, 所以要尽量用最少的材料生产出最多的车, 这样才能提高边际收益。

边际成本 (Marginal cost) 计算方法: 增加一单位的产量 (Output) 随即而产生的成本增加量即称为边际成本。由定义得知边际成本等于总成本 (TC) 的变化量 ($\Delta \mathrm{TC}$) 除以对应的产量上的变化量 ($\Delta Q$): 总成本的变化量 (Changes in Total Cost) / 产量变化量 (Changes in Output) 即:

\[
M C(Q)=\Delta T C(Q) / \Delta Q
\]

or $MC(Q)=\lim =\Delta TC(Q)/\Delta Q=dTC/dQ$ (其中 $\Delta Q \to 0$)

回到题中的这个问题,在已经投入出行时间成本的基础上,游览时间便是其中的边际成本,游览时间的增加势必会造成总出行时间成本的增加,但是根据边际成本递减定律,一次出行的有效游览时间越长,即是说该次出行的效率越高,即是体验最佳。

之所以有增加有效旅游时间的可能,是因为在前三问中仅考虑游览 5A 景区,而在很多地方 5A 景区又是比较分散的,因此造成了在途中耗费了很多时间,却没有足够量的纯粹的游览时间。考虑 4A 景区之后,将可以有效的填补游览 5A 景区之余的时间空隙。

\subsubsection{问题假设}

假设该出行者依然将在十年内遍历所有 5A 景点、出行费用最小和旅行体验最佳为目标,4A 旅游景点的引入仅作为填充出行时间规划中未完全占满的“间隙”(例如:若一次出行为 13 天,另一次出行时间为 14 天,但每次出行的最大限制为 15 天,因此可以将该第一次出行时间和第二次出行时间均扩充为 15 天,多出来的 3 天时间用来旅游本没有在规划考虑中的 4A 景点,这样可以提高旅行车的出行体验);

基于第二问的出行计划,假设以 15 天为限,出游计划排满 15 天的出行路线不再安排 4A 景区的游览;对于单次出游未满 15 天、全年出行未满 30 天的出行线路进行优化,即安排 4A 景区的游览;

假设 4A 景区在距离省会城市、地级城市、或者 5A 景区的附近,不再额外产生交通时间成本,“间隙”时间全部用于游览,即“间隙”时间全部是游览的有效时间。

\subsubsection{模型建立}

本题的十年旅游规划将以第二题中十年旅游规划为基础,将国家 4A 旅游景区纳入考虑范围。

\begin{figure}[h]
\centering
\includegraphics[width=0.8\textwidth]{image.png}
\caption{模型建立思路}
\end{figure}

图 9 模型建立思路

59

基于第二题中的出行方式选择模型,引入变量游览时间比 \(\delta\),并以该变量最大为目标对模型进行优化,即:

\[
Max \delta = \frac{\sum_{i=1}^{m} T_i \cdot X_{ij}}{\sum_{i=1}^{m} \sum_{j=1}^{n} (T_i + T_{ij} + PT_{ij}) \cdot X_{ij}}
\]

约束同 5.2.2.1 模型建立中 (1)-(18) 约束条件。

此模型我们不再求解,下面给出基于第二问出行规划的优化后的十年出行规划。

第四次出行规划简表如下:

\begin{tabular}{c c l c c c}
\hline
4.00 & 登封市 & 焦作(云台山-神农山-青天河)风景区 & 1.0 & 1 & 800 \\
\hline
5.00 & 登封市 & 焦作(云台山-神农山-青天河)风景区 & 1 & 750 \\
\hline
6.00 & 登封市 & 洛阳龙门石窟景区、洛阳嵩县白云山景区 & 3.8 & 243.1 & 993 \\
\hline
7.00 & 洛阳市 & 洛阳栾川县老君山-鸡冠洞旅游区 & 2.5 & 150 & 900 \\
\hline
8.00 & 洛阳市 & 洛阳新安县龙潭大峡谷景区 & 2.0 & 90 & 840 \\
\hline
9.00 & 洛阳市 & 平顶山鲁山县尧山-中原大佛景区 & 3.0 & 214 & 964 \\
\hline
10.00 & 洛阳市 & 南阳西峡伏牛山老界岭·恐龙遗址园旅游区 & 3.0 & 160 & 910 \\
\hline
11.00 & 平顶山 & 开封清明上河园景区、安阳殷墟景区 & 2.3 & 各0.5 & 960 \\
\hline
12.00 & 安阳 & 无 & 7.3 & 661.1 & 1648 \\
\hline
合计 & & & & & 11994 \\
\hline
1 & 西安 & 无 & 8.0 & 0 & 1020 \\
\hline
2 & 西安-济南途中 & 济南天下第一泉景区(趵突泉-大明湖-五龙潭-环城公园-黑虎泉) & 2.0 & 1 & 780 \\
\hline
3 & 济南 & 济南 & 0.0 & 0 & 600 \\
\hline
4 & 济南 & 泰安泰山景区 & 1.2 & 1 & 710 \\
\hline
5 & 泰安 & 济宁曲阜明故城三孔旅游区 & 2.0 & 0.5 & 780 \\
\hline
6 & 枣庄 & 枣庄台儿庄古城景区 & 5.0 & 0.5 & 1050 \\
\hline
7 & 青岛 & 青岛崂山景区 & 2.9 & 0.5 & 860 \\
\hline
8 & 威海 & 威海刘公岛景区,烟台龙口南山景区 & 1.7 & 各0.5 & 453 \\
\hline
\end{tabular}

\begin{tabular}{r l l l l l}
 &  &  &  &  & \\
8 & 18 & 飞机+租车 & 西安-南昌-西安 & 9.5 & 0 & 9.5 \\
 & 19 & 飞机+租车 & 西安-昆明-西安 & 9.5 & 0 & 9.5 \\
9 & 20 & 自驾 & 西安-合肥-西安 & 11 & 0 & 11 \\
 & 21 & 飞机+租车 & 西安-福州-西安 & 10.5 & 4.5 & 15 \\
 & 22 & 飞机+租车 & 西安-海口-南宁-西安 & 13 & 2 & 15 \\
10 & 23 & 飞机+租车 & 西安-广州-西安 & 10 & 5 & 15 \\
 & 24 & 飞机+高铁+租车 & 西安-沈阳-长春-西安 & 12 & 3 & 15 \\
\end{tabular}

\section{模型总结与评价}

\begin{enumerate}
    \item 本文采用分层优化建模,成功将规划 201 个景点之间线路的问题简化为下层模型规划该省各景点之间线路和上层模型规划 31 个省会城市之间线路的问题。从算法角度来看,提高了计算效率,节省了运算时间;从实际角度来看,这样的分层建模更符合以省会城市为中心的实际情况。
    \item 在分层建模的基础上,本文将下层优化模型转化为旅行商问题(TSP),上层优化模型转化为多路径车辆问题(MVRPTW),从而是模型构建思路明确,结构清晰。
    \item 驾车时间窗(包括乘坐高铁和飞机)和旅游景点的时间窗难以用数学符号准确表达,建模过程中,本文以具体的到达时刻和结束时刻来匹配时间窗限制,能够很好地满足该约束。
    \item 题中限制条件较多,各种因素错综复杂,并且距离和时间跨度较大,所得结果仅是可行解,未必是最优结果。
\end{enumerate}

本文模型也需要进一步改进,例如未能充分考虑游览时间为半天和一天时间的景点和交通出行之间的组合,即一天之中半天的交通出行,应该对应需要游览半天的景点,而不是需要游览一天的景点拆分为两个半天进行游览。这是模型需要继续改进的地方。

\section{七、参考文献}

\begin{enumerate}
    \item 李敏, 吴浪, 张开碧. 求解旅行商问题的几种算法的比较研究[J]. 重庆邮电大学学报(自然科学版), 2008, 20(5): 624-626.
    \item 王大志, 汪定伟, 闫杨. 一类多旅行商问题的计算及仿真分析[J]. 系统仿真学报, 2009 (20): 6378-6381.
    \item 李军华, 黎明, 袁丽华. 一种改进的双种群遗传算法[J]. 小型微型计算机系统, 2008, 29(11): 2099-2102.
    \item 曾凡超, 朱征宇, 邓欣, 等. 车辆路径问题的改进的双种群遗传算法[J]. 计算机工程与设计, 2007, 28(20): 4998-5000.
    \item A. Beham. Parallel tabu search and the multiobjective vehicle routing problem with time windows. In 21th International Parallel and Distributed Processing Symposium, pages 1-8. IEEE Computer Society, 2007.
    \item 潘立军. 带时间窗车辆路径问题及其算法研究[D]. 中南大学, 2012.
    \item E. Alba and B. Dorronsoro. Computing nine new best-so-far solutions for capacitated VRP with a cellular genetic algorithm. Inform. Process. Lett., 98(6):225-230, 2006.
    \item 葛显龙, 许茂增, 王伟鑫. 多车型车辆路径问题的量子遗传算法研究[J]. 中国管理科学, 2013, 1: 125-133.
\end{enumerate}

\section{八、附录}

\textbf{附录一 31个省市景点邻近城市OD矩阵}

\textbf{表1:北京市}
\begin{tabular}{c c c}
北京 & 承德市 & 秦皇岛市 \\
\hline
北京 & 0 & \\
承德市 & 216 & 0 \\
秦皇岛市 & 295 & 230 & 0 \\
\end{tabular}

\textbf{表2:河北省}
\begin{tabular}{c c c}
雄县 & 保定市 & 石家庄市 \\
\hline
雄县 & 0 & \\
保定市 & 72 & 0 \\
石家庄市 & 208 & 148 & 0 \\
\end{tabular}

\textbf{表3:山西省}
\begin{tabular}{c c c c c c}
大同市 & 五台县 & 晋城市 & 介休市 & 太原市 & 平遥县 \\
\hline
大同市 & 0 & & & & \\
五台县 & 235 & 0 & & & \\
晋城市 & 581 & 440 & 0 & & \\
介休市 & 136 & 273 & 305 & 0 & \\
太原市 & 283 & 138 & 305 & 132 & 0 \\
平遥县 & 89 & 243 & 291 & 36 & 138 & 0 \\
\end{tabular}

\textbf{表4:内蒙古自治区}
\begin{tabular}{c c}
包头市 & 鄂尔多斯 \\
\hline
包头市 & 0 & \\
鄂尔多斯 & 142 & 0 \\
\end{tabular}

\textbf{表5:辽宁省}
\begin{tabular}{c c}
沈阳 & 大连 \\
\hline
沈阳 & 0 & \\
大连 & 387 & 0 \\
\end{tabular}

\textbf{表6:黑龙江省}
\begin{tabular}{c c c}
哈尔滨 & 伊春 & 五大连池 \\
\hline
哈尔滨 & 0 & & \\
伊春 & 330 & 0 & \\
五大连池 & 264 & 355 & 0 \\
\end{tabular}

64

\begin{table}
\caption{江苏省}
\begin{tabular}{c c c c c c c c c}
 & 苏州 & 南京 & 无锡 & 常州 & 扬州 & 南通 & 泰州 & 镇江 & 淮安 \\
苏州 & 0 & & & & & & & & \\
南京 & 215 & 0 & & & & & & & \\
无锡 & 42 & 188 & 0 & & & & & & \\
常州 & 96 & 130 & 67 & 0 & & & & & \\
扬州 & 203 & 103 & 161 & 120 & 0 & & & & \\
南通 & 109 & 254 & 128 & 163 & 153 & 0 & & & \\
泰州 & 168 & 160 & 122 & 106 & 69 & 125 & 0 & & \\
镇江 & 169 & 86 & 123 & 85 & 39 & 163 & 96 & 0 & \\
淮安 & 338 & 200 & 293 & 277 & 180 & 295 & 190 & 213 & 0 \\
\end{tabular}
\end{table}

\begin{table}
\caption{浙江省}
\begin{tabular}{c c c c c c c c}
 & 杭州 & 乐清 & 舟山 & 宁波 & 嘉兴 & 绍兴 & 衢州 & 湖州 \\
杭州 & 0 & & & & & & & \\
乐清 & 337 & 0 & & & & & & \\
舟山 & 240 & 332 & 0 & & & & & \\
宁波 & 159 & 240 & 76 & 0 & & & & \\
嘉兴 & 91 & 352 & 217 & 162 & 0 & & & \\
绍兴 & 70 & 283 & 181 & 117 & 111 & 0 & & \\
衢州 & 226 & 317 & 390 & 314 & 326 & 246 & 0 & \\
湖州 & 77 & 411 & 299 & 235 & 91 & 145 & 297 & 0 \\
\end{tabular}
\end{table}

\begin{table}
\caption{安徽省}
\begin{tabular}{c c c c c c}
 & 合肥市 & 黄山市 & 池州市 & 潜山县 & 麻城市 & 阜阳市 \\
合肥市 & 0 & & & & & \\
黄山市 & 321 & 0 & & & & \\
池州市 & 198 & 214.3 & 0 & & & \\
潜山县 & 170 & 300 & 140 & 0 & & \\
麻城市 & 271 & 550 & 446 & 316 & 0 & \\
阜阳市 & 219 & 534 & 410 & 302 & 313 & 0 \\
\end{tabular}
\end{table}

\begin{table}
\caption{福建省}
\begin{tabular}{c c c c c c c}
 & 厦门市 & 武夷山市 & 龙岩市 & 宁德市 & 泉州市 & 福州市 \\
厦门市 & 0 & & & & & \\
南平市 & 571 & 0 & & & & \\
武夷山 & & & & & & \\
龙岩市 & 160 & 450 & 0 & & & \\
宁德市 & 346 & 475 & 450 & 0 & & \\
泉州市 & 101 & 225 & 225 & 264 & 0 & \\
福州市 & 259 & 348 & 336.7 & 100 & 180 & 0 \\
\end{tabular}
\end{table}

\begin{table}
\caption{江西省}
\begin{tabular}{c c c c c c c}
 & 九江市 & 吉安市 & 鹰潭市 & 赣州市 & & 南昌市 \\
 & & 井冈山市 & 上饶市 & 贵溪市 & 景德镇市 & 瑞金市 \\
九江市 & 0 & & & & & \\
井冈山 & 431 & 0 & & & & \\
上饶市 & 336 & 497 & 0 & & & \\
贵溪市 & 300 & 429 & 86 & 0 & & \\
景德镇 & 145 & 525 & 200 & 194 & 0 & \\
瑞金市 & 498 & 282 & 414 & 346 & 457 & 0 \\
南昌市 & 130 & 306 & 260 & 193 & 226 & 389 \\
\end{tabular}
\end{table}

\begin{table}
\caption{山东省}
\begin{tabular}{c c c c c c c c}
 & 泰安市 & 烟台市 & 济宁市 & 青岛市 & 威海市 & 烟台市 & 枣庄市 \\
 & & 蓬莱市 & 曲阜市 & & & 济南市 & 潍坊市 \\
泰安市 & 0 & & & & & & \\
蓬莱市 & 464 & 0 & & & & & \\
曲阜市 & 83 & 540 & 0 & & & & \\
青岛市 & 338.8 & 232 & 387 & 0 & & & \\
威海市 & 568 & 152 & 633 & 262 & 0 & & \\
烟台市 & 506 & 91 & 572 & 219 & 65 & 0 & \\
枣庄市 & 174 & 594 & 105 & 415 & 623 & 600 & 0 \\
济南市 & 107.3 & 411 & 175 & 351 & 515 & 453 & 264 \\
潍坊市 & 263 & 210.8 & 338 & 162 & 314 & 253 & 420 \\
\end{tabular}
\end{table}

\begin{table}
\caption{河南省}
\begin{tabular}{c c c c c c}
 & 郑州市 & 郑州市登封市 & 洛阳市 & 安阳市 & 开封市 \\
 & & & & & 平顶山市 \\
郑州市 & 0 & & & & \\
登封市 & 71 & 0 & & & \\
洛阳市 & 141 & 73 & 0 & & \\
安阳市 & 198 & 266 & 296 & 0 & \\
开封市 & 85 & 142 & 194 & 214 & 0 \\
平顶山 & 143 & 137.6 & 134 & 330 & 200 \\
\end{tabular}
\end{table}

\begin{table}
\caption{湖北省}
\begin{tabular}{c c c c}
 & 武汉市 & 宜昌市 & 十堰市 \\
 & & & 恩施市 \\
武汉市 & 0 & & \\
宜昌市 & 324 & 0 & \\
十堰市 & 443 & 410 & 0 \\
恩施市 & 525 & 239 & 609 \\
\end{tabular}
\end{table}

\begin{table}
\caption{湖南省}
\begin{tabular}{c c c c c}
 & 张家界市 & 衡阳市 & 长沙市 & 岳阳市 \\
 & & & & 郴州市 \\
张家界市 & 0 & & 66 & \\
\end{tabular}
\end{table}

\begin{table}
\centering
\begin{tabular}{l l l l l l l l}
\multicolumn{8}{c}{表16:广东省} \\
 & 广州市 & 深圳市 & 梅州市 & 郴州市 & 韶关市 & 佛山市 & 惠州市 \\
广州市 & 0 & & & & & & \\
深圳市 & 141 & 0 & & & & & \\
梅州市 & 387 & 363 & 0 & & & & \\
郴州市 & 368 & 497 & 569 & 0 & & & \\
韶关市 & 221 & 346 & 412 & 412 & 0 & & \\
佛山市 & 35 & 163 & 413 & 412 & 245 & 0 & \\
惠州市 & 148 & 88 & 278 & 278 & 341 & 168 & 0 \\
\end{tabular}
\end{table}

\begin{table}
\centering
\begin{tabular}{l l l l}
\multicolumn{4}{c}{表17:广西省} \\
 & 桂林市 & 南宁市 & 河池市 \\
桂林市 & 0 & & \\
南宁市 & 388 & 0 & \\
河池市 & 315 & 246 & 0 \\
\end{tabular}
\end{table}

\begin{table}
\centering
\begin{tabular}{l l l l l l l l}
\multicolumn{8}{c}{表18:四川省} \\
 & 成都市 & 乐山市 & 广元市 & 阿坝藏族 & 绵阳市 & 南充市 & 广安市 \\
成都市 & 0 & & & & & & \\
乐山市 & 139 & 0 & & & & & \\
广元市 & 1648 & 437 & 0 & & & & \\
阿坝藏族 & 1080 & 471 & 608 & 0 & & & \\
绵阳市 & 432 & 273 & 177 & 440 & 0 & & \\
南充市 & 316 & 347 & 214 & 555 & 210 & 0 & \\
广安市 & 300 & 417 & 300 & 625 & 276 & 84 & 0 \\
\end{tabular}
\end{table}

\begin{table}
\centering
\begin{tabular}{l l l l}
\multicolumn{4}{c}{表19:贵州省} \\
 & 贵阳 & 安顺市 & 毕节市 \\
贵阳 & 0 & & \\
安顺市 & 89 & 0 & \\
毕节市 & 211 & 225.9 & 0 \\
\end{tabular}
\end{table}

\begin{table}
\centering
\begin{tabular}{l l l l l l}
\multicolumn{6}{c}{表20:云南省} \\
 & 昆明市 & 丽江市 & 大理市 & 景洪市 & 香格里拉 \\
昆明市 & 0 & & & & \\
丽江市 & 516 & 0 & & & \\
大理市 & 333 & 219 & 0 & & \\
景洪市 & 523 & 187 & 816 & 0 & \\
 & & & & 67 & \\
\end{tabular}
\end{table}

\begin{table}
\centering
\begin{tabular}{l l l l l l}
香格里拉 & 634 & 174 & 337 & 1118 & 0 \\
\hline
表21:陕西省 & & & & & \\
西安市 & 西安市 & 宝鸡市 & & & \\
宝鸡市 & 0 & 176 & 0 & & \\
\hline
表22:甘肃省 & & & & & \\
兰州 & 兰州 & 嘉峪关市 & 平凉市 & 天水市 & 酒泉市敦煌市 \\
嘉峪关市 & 0 & & & & \\
平凉市 & 727 & 0 & & & \\
天水市 & 318 & 1045.3 & 0 & & \\
敦煌市 & 305 & 1027 & 261 & 0 & \\
 & 1097 & 377 & 1417 & 1398 & 0 \\
\hline
表23:新疆自治区 & & & & & \\
乌鲁木齐 & 乌鲁木齐 & 阿勒泰市 & 石河子 & 喀什市 & \\
阿勒泰市 & 0 & & & & \\
石河子 & 837 & 0 & & & \\
喀什市 & 188 & 662 & 0 & & \\
 & 1309 & 1796 & 1600 & 0 & \\
\end{tabular}
\end{table}

\section{附录二 MATLAB 核心程序}

\begin{verbatim}
% 遗传算法 VRP 问题 Matlab实现

%%%%%%%%%%%%%%%%%%%%%%%%%%%%%%%%%%%%%%%%%%%%%%%%%%%%%%%%%%%%%%%
%tic%计时器
clear;
clc

%W=15; %每辆车的载重量
%Citynum=32; %客户数量(含1个拆分省会城市)
%Stornum=1; %仓库个数
%C %第二三列 省会城市坐标,第四列 省会城市辐射区旅游时长

G=100;%种群大小

[dislist, Clist] = vrp(C); %dislist为距离矩阵,Clist为点坐标矩阵及客户需求
L = []; %存每个种群的回路长度

for i=1:G
    Parent(i,:) = randperm(Citynum); %随机产生路径
    L(i,1) = curlist(Citynum, Clist(:,4), W, Parent(i,:), Stornum, dislist);
end

Pc = 0.8; %交叉比率
Pm = 0.3; %变异比率
species = Parent; %种群
children = []; %子代
%%%%%%%%%%%%%%%%%%%%%%%%%%%%%%%%%%%%%%%%%%%%%%%%%%%%%%%%%%%%%%%
disp('正在运行,时间比较长,请稍等........')
g = 10;
for generation=1:g
    %%%%%%%%%%%%%%%%%%%%%%%%%%%%%%%%%%%%%%%%%%%%%%%%%%%%%%%%%%%%%%%%
    tic
    fprintf('\n正在进行第%d次迭代,共%d次........', generation, g);
    Parent = species; %子代变成父代
    children = []; %子代
    Lp = L;
\end{verbatim}

\begin{verbatim}
% 选择交叉父代
[n m]=size(Parent);

% 交叉,代处理
for i=1:n
    for j=i:n
        if rand<Pc
            crossover
        end
    end
end

% % % % % % % % % % % % % % % % % % % % % % % % % % % % % % % % % % % % % % % % % % % % % % % % % % % % % % % % % % % % % % % % % % % % % % % % % % % % % % % % % % % % % % % % % % % % % % % % % % % % % % % % % % % % % % % % % % % % % % % % % % % % % % % % % % % % % % % % % % % % % % % % % % % % % % % % % % % % % % % % % % % % % % % % % % % % % % % % % % % % % % % % % % % % % % % % % % % % % % % % % % % % % % % % % % % % % % % % % % % % % % % % % % % % % % % % % % % % % % % % % % % % % % % % % % % % % % % % % % % % % % % % % % % % % % % % % % % % % % % % % % % % % % % % % % % % % % % % % % % % % % % % % % % % % % % % % % % % % % % % % % % % % % % % % % % % % % % % % % % % % % % % % % % % % % % % % % % % % % % % % % % % % % % % % % % % % % % % % % % % % % % % % % % % % % % % % % % % % % % % % % % % % % % % % % % % % % % % % % % % % % % % % % % % % % % % % % % % % % % % % % % % % % % % % % % % % % % % % % % % % % % % % % % % % % % % % % % % % % % % % % % % % % % % % % % % % % % % % % % % % % % % % % % % % % % % % % % % % % % % % % % % % % % % % % % % % % % % % % % % % % % % % % % % % % % % % % % % % % % % % % % % % % % % % % % % % % % % % % % % % % % % % % % % % % % % % % % % % % % % % % % % % % % % % % % % % % % % % % % % % % % % % % % % % % % % % % % % % % % % % % % % % % % % % % % % % % % % % % % % % % % % % % % % % % % % % % % % % % % % % % % % % % % % % % % % % % % % % % % % % % % % % % % % % % % % % % % % % % % % % % % % % % % % % % % % % % % % % % % % % % % % % % % % % % % % % % % % % % % % % % % % % % % % % % % % % % % % % % % % % % % % % % % % % % % % % % % % % % % % % % % % % % % % % % % % % % % % % % % % % % % % % % % % % % % % % % % % % % % % % % % % % % % % % % % % % % % % % % % % % % % % % % % % % % % % % % % % % % % % % % % % % % % % % % % % % % % % % % % % % % % % % % % % % % % % % % % % % % % % % % % % % % % % % % % % % % % % % % % % % % % % % % % % % % % % % % % % % % % % % % % % % % % % % % % % % % % % % % % % % % % % % % % % % % % % % % % % % % % % % % % % % % % % % % % % % % % % % % % % % % % % % % % % % % % % % % % % % % % % % % % % % % % % % % % % % % % % % % % % % % % % % % % % % % % % % % % % % % % % % % % % % % % % % % % % % % % % % % % % % % % % % % % % % % % % % % % % % % % % % % % % % % % % % % % % % % % % % % % % % % % % % % % % % % % % % % % % % % % % % % % % % % % % % % % % % % % % % % % % % % % % % % % % % % % % % % % % % % % % % % % % % % % % % % % % % % % % % % % % % % % % % % % % % % % % % % % % % % % % % % % % % % % % % % % % % % % % % % % % % % % % % % % % % % % % % % % % % % % % % % % % % % % % % % % % % % % % % % % % % % % % % % % % % % % % % % % % % % % % % % % % % % % % % % % % % % % % % % % % % % % % % % % % % % % % % % % % % % % % % % % % % % % % % % % % % % % % % % % % % % % % % % % % % % % % % % % % % % % % % % % % % % % % % % % % % % % % % % % % % % % % % % % % % % % % % % % % % % % % % % % % % % % % % % % % % % % % % % % % % % % % % % % % % % % % % % % % % % % % % % % % % % % % % % % % % % % % % % % % % % % % % % % % % % % % % % % % % % % % % % % % % % % % % % % % % % % % % % % % % % % % % % % % % % % % % % % % % % % % % % % % % % % % % % % % % % % % % % % % % % % % % % % % % % % % % % % % % % % % % % % % % % % % % % % % % % % % % % % % % % % % % % % % % % % % % % % % % % % % % % % % % % % % % % % % % % % % % % % % % % % % % % % % % % % % % % % % % % % % % % % % % % % % % % % % % % % % % % % % % % % % % % % % % % % % % % % % % % % % % % % % % % % % % % % % % % % % % % % % % % % % % % % % % % % % % % % % % % % % % % % % % % % % % % % % % % % % % % % % % % % % % % % % % % % % % % % % % % % % % % % % % % % % % % % % % % % % % % % % % % % % % % % % % % % % % % % % % % % % % % % % % % % % % % % % % % % % % % % % % % % % % % % % % % % % % % % % % % % % % % % % % % % % % % % % % % % % % % % % % % % % % % % % % % % % % % % % % % % % % % % % % % % % % % % % % % % % % % % % % % % % % % % % % % % % % % % % % % % % % % % % % % % % % % % % % % % % % % % % % % % % % % % % % % % % % % % % % % % % % % % % % % % % % % % % % % % % % % % % % % % % % % % % % % % % % % % % % % % % % % % % % % % % % % % % % % % % % % % % % % % % % % % % % % % % % % % % % % % % % % % % % % % % % % % % % % % % % % % % % % % % % % % % % % % % % % % % % % % % % % % % % % % % % % % % % % % % % % % % % % % % % % % % % % % % % % % % % % % % % % % % % % % % % % % % % % % % % % % % % % % % % % % % % % % % % % % % % % % % % % % % % % % % % % % % % % % % % % % % % % % % % % % % % % % % % % % % % % % % % % % % % % % % % % % % % % % % % % % % % % % % % % % % % % % % % % % % % % % % % % % % % % % % % % % % % % % % % % % % % % % % % % % % % % % % % % % % % % % % % % % % % % % % % % % % % % % % % % % % % % % % % % % % % % % % % % % % % % % % % % % % % % % % % % % % % % % % % % % % % % % % % % % % % % % % % % % % % % % % % % % % % % % % % % % % % % % % % % % % % % % % % % % % % % % % % % % % % % % % % % % % % % % % % % % % % % % % % % % % % % % % % % % % % % % % % % % % % % % % % % % % % % % % % % % % % % % % % % % % % % % % % % % % % % % % % % % % % % % % % % % % % % % % % % % % % % % % % % % % % % % % % % % % % % % % % % % % % % % % % % % % % % % % % % % % % % % % % % % % % % % % % % % % % % % % % % % % % % % % % % % % % % % % % % % % % % % % % % % % % % % % % % % % % % % % % % % % % % % % % % % % % % % % % % % % % % % % % % % % % % % % % % % % % % % % % % % % % % % % % % % % % % % % % % % % % % % % % % % % % % % % % % % % % % % % % % % % % % % % % % % % % % % % % % % % % % % % % % % % % % % % % % % % % % % % % % % % % % % % % % % % % % % % % % % % % % % % % % % % % % % % % % % % % % % % % % % % % % % % % % % % % % % % % % % % % % % % % % % % % % % % % % % % % % % % % % % % % % % % % % % % % % % % % % % % % % % % % % % % % % % % % % % % % % % % % % % % % % % % % % % % % % % % % % % % % % % % % % % % % % % % % % % % % % % % % % % % % % % % % % % % % % % % % % % % % % % % % % % % % % % % % % % % % % % % % % % % % % % % % % % % % % % % % % % % % % % % % % % % % % % % % % % % % % % % % % % % % % % % % % % % % % % % % % % % % % % % % % % % % % % % % % % % % % % % % % % % % % % % % % % % % % % % % % % % % % % % % % % % % % % % % % % % % % % % % % % % % % % % % % % % % % % % % % % % % % % % % % % % % % % % % % % % % % % % % % % % % % % % % % % % % % % % % % % % % % % % % % % % % % % % % % % % % % % % % % % % % % % % % % % % % % % % % % % % % % % % % % % % % % % % % % % % % % % % % % % % % % % % % % % % % % % % % % % % % % % % % % % % % % % % % % % % % % % % % % % % % % % % % % % % % % % % % % % % % % % % % % % % % % % % % % % % % % % % % % % % % % % % % % % % % % % % % % % % % % % % % % % % % % % % % % % % % % % % % % % % % % % % % % % % % % % % % % % % % % % % % % % % % % % % % % % % % % % % % % % % % % % % % % % % % % % % % % % % % % % % % % % % % % % % % % % % % % % % % % % % % % % % % % % % % % % % % % % % % % % % % % % % % % % % % % % % % % % % % % % % % % % % % % % % % % % % % % % % % % % % % % % % % % % % % % % % % % % % % % % % % % % % % % % % % % % % % % % % % % % % % % % % % % % % % % % % % % % % % % % % % % % % % % % % % % % % % % % % % % % % % % % % % % % % % % % % % % % % % % % % % % % % % % % % % % % % % % % % % % % % % % % % % % % % % % % % % % % % % % % % % % % % % % % % % % % % % % % % % % % % % % % % % % % % % % % % % % % % % % % % % % % % % % % % % % % % % % % % % % % % % % % % % % % % % % % % % % % % % % % % % % % % % % % % % % % % % % % % % % % % % % % % % % % % % % % % % % % % % % % % % % % % % % % % % % % % % % % % % % % % % % % % % % % % % % % % % % % % % % % % % % % % % % % % % % % % % % % % % % % % % % % % % % % % % % % % % % % % % % % % % % % % % % % % % % % % % % % % % % % % % % % % % % % % % % % % % % % % % % % % % % % % % % % % % % % % % % % % % % % % % % % % % % % % % % % % % % % % % % % % % % % % % % % % % % % % % % % % % % % % % % % % % % % % % % % % % % % % % % % % % % % % % % % % % % % % % % % % % % % % % % % % % % % % % % % % % % % % % % % % % % % % % % % % % % % % % % % % % % % % % % % % % % % % % % % % % % % % % % % % % % % % % % % % % % % % % % % % % % % % % % % % % % % % % % % % % % % % % % % % % % % % % % % % % % % % % % % % % % % % % % % % % % % % % % % % % % % % % % % % % % % % % % % % % % % % % % % % % % % % % % % % % % % % % % % % % % % % % % % % % % % % % % % % % % % % % % % % % % % % % % % % % % % % % % % % % % % % % % % % % % % % % % % % % % % % % % % % % % % % % % % % % % % % % % % % % % % % % % % % % % % % % % % % % % % % % % % % % % % % % % % % % % % % % % % % % % % % % % % % % % % % % % % % % % % % % % % % % % % % % % % % % % % % % % % % % % % % % % % % % % % % % % % % % % % % % % % % % % % % % % % % % % % % % % % % % % % % % % % % % % % % % % % % % % % % % % % % % % % % % % % % % % % % % % % % % % % % % % % % % % % % % % % % % % % % % % % % % % % % % % % % % % % % % % % % % % % % % % % % % % % % % % % % % % % % % % % % % % % % % % % % % % % % % % % % % % % % % % % % % % % % % % % % % % % % % % % % % % % % % % % % % % % % % % % % % % % % % % % % % % % % % % % % % % % % % % % % % % % % % % % % % % % % % % % % % % % % % % % % % % % % % % % % % % % % % % % % % % % % % % % % % % % % % % % % % % % % % % % % % % % % % % % % % % % % % % % % % % % % % % % % % % % % % % % % % % % % % % % % % % % % % % % % % % % % % % % % % % % % % % % % % % % % % % % % % % % % % % % % % % % % % % % % % % % % % % % % % % % % % % % % % % % % % % % % % % % % % % % % % % % % % % % % % % % % % % % % % % % % % % % % % % % % % % % % % % % % % % % % % % % % % % % % % % % % % % % % % % % % % % % % % % % % % % % % % % % % % % % % % % % % % % % % % % % % % % % % % % % % % % % % % % % % % % % % % % % % % % % % % % % % % % % % % % % % % % % % % % % % % % % % % % % % % % % % % % % % % % % % % % % % % % % % % % % % % % % % % % % % % % % % % % % % % % % % % % % % % % % % % % % % % % % % % % % % % % % % % % % % % % % % % % % % % % % % % % % % % % % % % % % % % % % % % % % % % % % % % % % % % % % % % % % % % % % % % % % % % % % % % % % % % % % % % % % % % % % % % % % % % % % % % % % % % % % % % % % % % % % % % % % % % % % % % % % % % % % % % % % % % % % % % % % % % % % % % % % % % % % % % % % % % % % % % % % % % % % % % % % % % % % % % % % % % % % % % % % % % % % % % % % % % % % % % % % % % % % % % % % % % % % % % % % % % % % % % % % % % % % % % % % % % % % % % % % % % % % % % % % % % % % % % % % % % % % % % % % % % % % % % % % % % % % % % % % % % % % % % % % % % % % % % % % % % % % % % % % % % % % % % % % % % % % % % % % % % % % % % % % % % % % % % % % % % % % % % % % % % % % % % % % % % % % % % % % % % % % % % % % % % % % % % % % % % % % % % % % % % % % % % % % % % % % % % % % % % % % % % % % % % % % % % % % % % % % % % % % % % % % % % % % % % % % % % % % % % % % % % % % % % % % % % % % % % % % % % % % % % % % % % % % % % % % % % % % % % % % % % % % % % % % % % % % % % % % % % % % % % % % % % % % % % % % % % % % % % % % % % % % % % % % % % % % % % % % % % % % % % % % % % % % % % % % % % % % % % % % % % % % % % % % % % % % % % % % % % % % % % % % % % % % % % % % % % % % % % % % % % % % % % % % % % % % % % % % % % % % % % % % % % % % % % % % % % % % % % % % % % % % % % % % % % % % % % % % % % % % % % % % % % % % % % % % % % % % % % % % % % % % % % % % % % % % % % % % % % % % % % % % % % % % % % % % % % % % % % % % % % % % % % % % % % % % % % % % % % % % % % % % % % % % % % % % % % % % % % % % % % % % % % % % % % % % % % % % % % % % % % % % % % % % % % % % % % % % % % % % % % % % % % % % % % % % % % % % % % % % % % % % % % % % % % % % % % % % % % % % % % % % % % % % % % % % % % % % % % % % % % % % % % % % % % % % % % % % % % % % % % % % % % % % % % % % % % % % % % % % % % % % % % % % % % % % % % % % % % % % % % % % % % % % % % % % % % % % % % % % % % % % % % % % % % % % % % % % % % % % % % % % % % % % % % % % % % % % % % % % % % % % % % % % % % % % % % % % % % % % % % % % % % % % % % % % % % % % % % % % % % % % % % % % % % % % % % % % % % % % % % % % % % % % % % % % % % % % % % % % % % % % % % % % % % % % % % % % % % % % % % % % % % % % % % % % % % % % % % % % % % % % % % % % % % % % % % % % % % % % % % % % % % % % % % % % % % % % % % % % % % % % % % % % % % % % % % % % % % % % % % % % % % % % % % % % % % % % % % % % % % % % % % % % % % % % % % % % % % % % % % % % % % % % % % % % % % % % % % % % % % % % % % % % % % % % % % % % % % % % % % % % % % % % % % % % % % % % % % % % % % % % % % % % % % % % % % % % % % % % % % % % % % % % % % % % % % % % % % % % % % % % % % % % % % % % % % % % % % % % % % % % % % % % % % % % % % % % % % % % % % % % % % % % % % % % % % % % % % % % % % % % % % % % % % % % % % % % % % % % % % % % % % % % % % % % % % % % % % % % % % % % % % % % % % % % % % % % % % % % % % % % % % % % % % % % % % % % % % % % % % % % % % % % % % % % % % % % % % % % % % % % % % % % % % % % % % % % % % % % % % % % % % % % % % % % % % % % % % % % % % % % % % % % % % % % % % % % % % % % % % % % % % % % % % % % % % % % % % % % % % % % % % % % % % % % % % % % % % % % % % % % % % % % % % % % % % % % % % % % % % % % % % % % % % % % % % % % % % % % % % % % % % % % % % % % % % % % % % % % % % % % % % % % % % % % % % % % % % % % % % % % % % % % % % % % % % % % % % % % % % % % % % % % % % % % % % % % % % % % % % % % % % % % % % % % % % % % % % % % % % % % % % % % % % % % % % % % % % % % % % % % % % % % % % % % % % % % % % % % % % % % % % % % % % % % % % % % % % % % % % % % % % % % % % % % % % % % % % % % % % % % % % % % % % % % % % % % % % % % % % % % % % % % % % % % % % % % % % % % % % % % % % % % % % % % % % % % % % % % % % % % % % % % % % % % % % % % % % % % % % % % % % % % % % % % % % % % % % % % % % % % % % % % % % % % % % % % % % % % % % % % % % % % % % % % % % % % % % % % % % % % % % % % % % % % % % % % % % % % % % % % % % % % % % % % % % % % % % % % % % % % % % % % % % % % % % % % % % % % % % % % % % % % % % % % % % % % % % % % % % % % % % % % % % % % % % % % % % % % % % % % % % % % % % % % % % % % % % % % % % % % % % % % % % % % % % % % % % % % % % % % % % % % % % % % % % % % % % % % % % % % % % % % % % % % % % % % % % % % % % % % % % % % % % % % % % % % % % % % % % % % % % % % % % % % % % % % % % % % % % % % % % % % % % % % % % % % % % % % % % % % % % % % % % % % % % % % % % % % % % % % % % % % % % % % % % % % % % % % % % % % % % % % % % % % % % % % % % % % % % % % % % % % % % % % % % % % % % % % % % % % % % % % % % % % % % % % % % % % % % % % % % % % % % % % % % % % % % % % % % % % % % % % % % % % % % % % % % % % % % % % % % % % % % % % % % % % % % % % % % % % % % % % % % % % % % % % % % % % % % % % % % % % % % % % % % % % % % % % % % % % % % % % % % % % % % % % % % % % % % % % % % % % % % % % % % % % % % % % % % % % % % % % % % % % % % % % % % % % % % % % % % % % % % % % % % % % % % % % % % % % % % % % % % % % % % % % % % % % % % % % % % % % % % % % % % % % % % % % % % % % % % % % % % % % % % % % % % % % % % % % % % % % % % % % % % % % % % % % % % % % % % % % % % % % % % % % % % % % % % % % % % % % % % % % % % % % % % % % % % % % % % % % % % % % % % % % % % % % % % % % % % % % % % % % % % % % % % % % % % % % % % % % % % % % % % % % % % % % % % % % % % % % % % % % % % % % % % % % % % % % % % % % % % % % % % % % % % % % % % % % % % % % % % % % % % % % % % % % % % % % % % % % % % % % % % % % % % % % % % % % % % % % % % % % % % % % % % % % % % % % % % % % % % % % % % % % % % % % % % % % % % % % % % % % % % % % % % % % % % % % % % % % % % % % % % % % % % % % % % % % % % % % % % % % % % % % % % % % % % % % % % % % % % % % % % % % % % % % % % % % % % % % % % % % % % % % % % % % % % % % % % % % % % % % % % % % % % % % % % % % % % % % % % % % % % % % % % % % % % % % % % % % % % % % % % % % % % % % % % % % % % % % % % % % % % % % % % % % % % % % % % % % % % % % % % % % % % % % % % % % % % % % % % % % % % % % % % % % % % % % % % % % % % % % % % % % % % % % % % % % % % % % % % % % % % % % % % % % % % % % % % % % % % % % % % % % % % % % % % % % % % % % % % % % % % % % % % % % % % % % % % % % % % % % % % % % % % % % % % % % % % % % % % % % % % % % % % % % % % % % % % % % % % % % % % % % % % % % % % % % % % % % % % % % % % % % % % % % % % % % % % % % % % % % % % % % % % % % % % % % % % % % % % % % % % % % % % % % % % % % % % % % % % % % % % % % % % % % % % % % % % % % % % % % % % % % % % % % % % % % % % % % % % % % % % % % % % % % % % % % % % % % % % % % % % % % % % % % % % % % % %

\begin{verbatim}
%%%%%淘汰子代 剩余前G个最优解
[m n]=size(children);
if (m>G)
    [m n]=sort(Lc);
    children=children(n(1:G),:);
    Lc=Lc(n(1:G));
end

%淘汰种群
species=[children;Parent];
L=[Lc;Lp];
[m n]=sort(L);

species=species(n(1:G),:);  %更新世代
L=L(n(1:G));

%%%%%%%%%%%%%%%%%%%%%%%%%%%%%%%%%%%%%%%%%%%%%%%
%%%%%%%%加入Opt优化

%分配仓库进行opt
temp=initialStor(Citynum,Clist(:,4),W,species(1,:),Stornum,dislist);%
存储分配仓库后的结果【52 14 5 52 53 6 9 8 53......】
Rbest=temp;
L_best=L(1);
[m n]=size(temp);

start=1;
car=[];%存放opt优化后的结果
i=2;
while (i<n+1)
    if (temp(i)>Citynum)
        cur=[];
        cur=Opt(i-start,[1:i-start,1:i-start],dislist,temp(start:i-1),Citynum);
        car=[car,[cur,cur(1)]];
        start=i+1;
        i=i+2;
    else
        i=i+1;
    end
end

L1=CalDist(dislist,car,Citynum);%计算进行优化后的回路长度
\end{verbatim}

\begin{verbatim}
if ( L1 < L(1) )
    fprintf('Opt优化有效! %f --> %f', L(1), L1);
    Rbest = car;
    car(find(car > Citynum)) = []; % 去掉编码中的仓库,再加入父代
    species = [species; car];
    L = [L; L1];
    L_best = L1;
end
    L_best
toc
%%%%%%%%%%%%%%%
%%%%%%%%%%%%end
%%

Rbest% 最优线路
L_best% 最优解

%% 画图
[m n] = size(Rbest);
start = 1;
temp = [];
i = 2;
while (i <= n)
    if (Rbest(i) > Citynum)
        temp = Rbest(start:i);
        plot(Clist(temp, 2), Clist(temp, 3), '-*')
        hold on;
        start = i + 1;
        i = i + 2;
    else
        i = i + 1;
    end
end
plot(Clist(Citynum+1:Citynum+Stornum, 2), Clist(Citynum+1:Citynum+Stornum, 3), 'or')
\end{verbatim}

\begin{verbatim}
% toc
\end{verbatim}

\end{document}