\begin{center}
\textbf{\Large 全国第二届部分高校研究生数模竞赛}
\end{center}

\begin{center}
\includegraphics[width=0.3\textwidth]{image.png} \\
\textbf{空中加油}
\end{center}

\begin{center}
\textbf{摘要:}
\end{center}

摘要:本文讨论了在辅机只能一次起飞和可以多次起飞情况下,辅机数量与主机最大作战半径的关系,给出了只有 1 到 8 架辅机时的最优作战半径和具体方案;并且本文提出了集群方式送主机的方案,分析了此方案下当辅机数量趋向无穷时,辅机数量与最优作战半径的关系,给出了近似关系式。当增加 2 个基地时,给出了有 1 架到 4 架辅机情况的最优作战半径,并利用前面的结论讨论了辅机数量趋向于无穷时辅机数量与最大作战半径的关系式。在此基础上本文分析了在给定基地位置时以最快和最省为目标函数时的最优方案。

\begin{flushleft}
参赛队号 \_\_\_\_\_\_\_\_\_\_\_\_\_\_\_\_\_\_\_\_\_\_\_\_\_\_\_\_\_\_\_\_\_\_\_\_\_\_\_\_\_\_\_\_\_\_\_\_\_\_\_\_\_\_\_\_\_\_\_\_\_\_\_\_\_\_\_\_\_\_\_\_\_\_\_\_\_\_\_\_\_\_\_\_\_\_\_\_\_\_\_\_\_\_\_\_\_\_\_\_\_\_\_\_\_\_\_\_\_\_\_\_\_\_\_\_\_\_\_\_\_\_\_\_\_\_\_\_\_\_\_\_\_\_\_\_\_\_\_\_\_\_\_\_\_\_\_\_\_\_\_\_\_\_\_\_\_\_\_\_\_\_\_\_\_\_\_\_\_\_\_\_\_\_\_\_\_\_\_\_\_\_\_\_\_\_\_\_\_\_\_\_\_\_\_\_\_\_\_\_\_\_\_\_\_\_\_\_\_\_\_\_\_\_\_\_\_\_\_\_\_\_\_\_\_\_\_\_\_\_\_\_\_\_\_\_\_\_\_\_\_\_\_\_\_\_\_\_\_\_\_\_\_\_\_\_\_\_\_\_\_\_\_\_\_\_\_\_\_\_\_\_\_\_\_\_\_\_\_\_\_\_\_\_\_\_\_\_\_\_\_\_\_\_\_\_\_\_\_\_\_\_\_\_\_\_\_\_\_\_\_\_\_\_\_\_\_\_\_\_\_\_\_\_\_\_\_\_\_\_\_\_\_\_\_\_\_\_\_\_\_\_\_\_\_\_\_\_\_\_\_\_\_\_\_\_\_\_\_\_\_\_\_\_\_\_\_\_\_\_\_\_\_\_\_\_\_\_\_\_\_\_\_\_\_\_\_\_\_\_\_\_\_\_\_\_\_\_\_\_\_\_\_\_\_\_\_\_\_\_\_\_\_\_\_\_\_\_\_\_\_\_\_\_\_\_\_\_\_\_\_\_\_\_\_\_\_\_\_\_\_\_\_\_\_\_\_\_\_\_\_\_\_\_\_\_\_\_\_\_\_\_\_\_\_\_\_\_\_\_\_\_\_\_\_\_\_\_\_\_\_\_\_\_\_\_\_\_\_\_\_\_\_\_\_\_\_\_\_\_\_\_\_\_\_\_\_\_\_\_\_\_\_\_\_\_\_\_\_\_\_\_\_\_\_\_\_\_\_\_\_\_\_\_\_\_\_\_\_\_\_\_\_\_\_\_\_\_\_\_\_\_\_\_\_\_\_\_\_\_\_\_\_\_\_\_\_\_\_\_\_\_\_\_\_\_\_\_\_\_\_\_\_\_\_\_\_\_\_\_\_\_\_\_\_\_\_\_\_\_\_\_\_\_\_\_\_\_\_\_\_\_\_\_\_\_\_\_\_\_\_\_\_\_\_\_\_\_\_\_\_\_\_\_\_\_\_\_\_\_\_\_\_\_\_\_\_\_\_\_\_\_\_\_\_\_\_\_\_\_\_\_\_\_\_\_\_\_\_\_\_\_\_\_\_\_\_\_\_\_\_\_\_\_\_\_\_\_\_\_\_\_\_\_\_\_\_\_\_\_\_\_\_\_\_\_\_\_\_\_\_\_\_\_\_\_\_\_\_\_\_\_\_\_\_\_\_\_\_\_\_\_\_\_\_\_\_\_\_\_\_\_\_\_\_\_\_\_\_\_\_\_\_\_\_\_\_\_\_\_\_\_\_\_\_\_\_\_\_\_\_\_\_\_\_\_\_\_\_\_\_\_\_\_\_\_\_\_\_\_\_\_\_\_\_\_\_\_\_\_\_\_\_\_\_\_\_\_\_\_\_\_\_\_\_\_\_\_\_\_\_\_\_\_\_\_\_\_\_\_\_\_\_\_\_\_\_\_\_\_\_\_\_\_\_\_\_\_\_\_\_\_\_\_\_\_\_\_\_\_\_\_\_\_\_\_\_\_\_\_\_\_\_\_\_\_\_\_\_\_\_\_\_\_\_\_\_\_\_\_\_\_\_\_\_\_\_\_\_\_\_\_\_\_\_\_\_\_\_\_\_\_\_\_\_\_\_\_\_\_\_\_\_\_\_\_\_\_\_\_\_\_\_\_\_\_\_\_\_\_\_\_\_\_\_\_\_\_\_\_\_\_\_\_\_\_\_\_\_\_\_\_\_\_\_\_\_\_\_\_\_\_\_\_\_\_\_\_\_\_\_\_\_\_\_\_\_\_\_\_\_\_\_\_\_\_\_\_\_\_\_\_\_\_\_\_\_\_\_\_\_\_\_\_\_\_\_\_\_\_\_\_\_\_\_\_\_\_\_\_\_\_\_\_\_\_\_\_\_\_\_\_\_\_\_\_\_\_\_\_\_\_\_\_\_\_\_\_\_\_\_\_\_\_\_\_\_\_\_\_\_\_\_\_\_\_\_\_\_\_\_\_\_\_\_\_\_\_\_\_\_\_\_\_\_\_\_\_\_\_\_\_\_\_\_\_\_\_\_\_\_\_\_\_\_\_\_\_\_\_\_\_\_\_\_\_\_\_\_\_\_\_\_\_\_\_\_\_\_\_\_\_\_\_\_\_\_\_\_\_\_\_\_\_\_\_\_\_\_\_\_\_\_\_\_\_\_\_\_\_\_\_\_\_\_\_\_\_\_\_\_\_\_\_\_\_\_\_\_\_\_\_\_\_\_\_\_\_\_\_\_\_\_\_\_\_\_\_\_\_\_\_\_\_\_\_\_\_\_\_\_\_\_\_\_\_\_\_\_\_\_\_\_\_\_\_\_\_\_\_\_\_\_\_\_\_\_\_\_\_\_\_\_\_\_\_\_\_\_\_\_\_\_\_\_\_\_\_\_\_\_\_\_\_\_\_\_\_\_\_\_\_\_\_\_\_\_\_\_\_\_\_\_\_\_\_\_\_\_\_\_\_\_\_\_\_\_\_\_\_\_\_\_\_\_\_\_\_\_\_\_\_\_\_\_\_\_\_\_\_\_\_\_\_\_\_\_\_\_\_\_\_\_\_\_\_\_\_\_\_\_\_\_\_\_\_\_\_\_\_\_\_\_\_\_\_\_\_\_\_\_\_\_\_\_\_\_\_\_\_\_\_\_\_\_\_\_\_\_\_\_\_\_\_\_\_\_\_\_\_\_\_\_\_\_\_\_\_\_\_\_\_\_\_\_\_\_\_\_\_\_\_\_\_\_\_\_\_\_\_\_\_\_\_\_\_\_\_\_\_\_\_\_\_\_\_\_\_\_\_\_\_\_\_\_\_\_\_\_\_\_\_\_\_\_\_\_\_\_\_\_\_\_\_\_\_\_\_\_\_\_\_\_\_\_\_\_\_\_\_\_\_\_\_\_\_\_\_\_\_\_\_\_\_\_\_\_\_\_\_\_\_\_\_\_\_\_\_\_\_\_\_\_\_\_\_\_\_\_\_\_\_\_\_\_\_\_\_\_\_\_\_\_\_\_\_\_\_\_\_\_\_\_\_\_\_\_\_\_\_\_\_\_\_\_\_\_\_\_\_\_\_\_\_\_\_\_\_\_\_\_\_\_\_\_\_\_\_\_\_\_\_\_\_\_\_\_\_\_\_\_\_\_\_\_\_\_\_\_\_\_\_\_\_\_\_\_\_\_\_\_\_\_\_\_\_\_\_\_\_\_\_\_\_\_\_\_\_\_\_\_\_\_\_\_\_\_\_\_\_\_\_\_\_\_\_\_\_\_\_\_\_\_\_\_\_\_\_\_\_\_\_\_\_\_\_\_\_\_\_\_\_\_\_\_\_\_\_\_\_\_\_\_\_\_\_\_\_\_\_\_\_\_\_\_\_\_\_\_\_\_\_\_\_\_\_\_\_\_\_\_\_\_\_\_\_\_\_\_\_\_\_\_\_\_\_\_\_\_\_\_\_\_\_\_\_\_\_\_\_\_\_\_\_\_\_\_\_\_\_\_\_\_\_\_\_\_\_\_\_\_\_\_\_\_\_\_\_\_\_\_\_\_\_\_\_\_\_\_\_\_\_\_\_\_\_\_\_\_\_\_\_\_\_\_\_\_\_\_\_\_\_\_\_\_\_\_\_\_\_\_\_\_\_\_\_\_\_\_\_\_\_\_\_\_\_\_\_\_\_\_\_\_\_\_\_\_\_\_\_\_\_\_\_\_\_\_\_\_\_\_\_\_\_\_\_\_\_\_\_\_\_\_\_\_\_\_\_\_\_\_\_\_\_\_\_\_\_\_\_\_\_\_\_\_\_\_\_\_\_\_\_\_\_\_\_\_\_\_\_\_\_\_\_\_\_\_\_\_\_\_\_\_\_\_\_\_\_\_\_\_\_\_\_\_\_\_\_\_\_\_\_\_\_\_\_\_\_\_\_\_\_\_\_\_\_\_\_\_\_\_\_\_\_\_\_\_\_\_\_\_\_\_\_\_\_\_\_\_\_\_\_\_\_\_\_\_\_\_\_\_\_\_\_\_\_\_\_\_\_\_\_\_\_\_\_\_\_\_\_\_\_\_\_\_\_\_\_\_\_\_\_\_\_\_\_\_\_\_\_\_\_\_\_\_\_\_\_\_\_\_\_\_\_\_\_\_\_\_\_\_\_\_\_\_\_\_\_\_\_\_\_\_\_\_\_\_\_\_\_\_\_\_\_\_\_\_\_\_\_\_\_\_\_\_\_\_\_\_\_\_\_\_\_\_\_\_\_\_\_\_\_\_\_\_\_\_\_\_\_\_\_\_\_\_\_\_\_\_\_\_\_\_\_\_\_\_\_\_\_\_\_\_\_\_\_\_\_\_\_\_\_\_\_\_\_\_\_\_\_\_\_\_\_\_\_\_\_\_\_\_\_\_\_\_\_\_\_\_\_\_\_\_\_\_\_\_\_\_\_\_\_\_\_\_\_\_\_\_\_\_\_\_\_\_\_\_\_\_\_\_\_\_\_\_\_\_\_\_\_\_\_\_\_\_\_\_\_\_\_\_\_\_\_\_\_\_\_\_\_\_\_\_\_\_\_\_\_\_\_\_\_\_\_\_\_\_\_\_\_\_\_\_\_\_\_\_\_\_\_\_\_\_\_\_\_\_\_\_\_\_\_\_\_\_\_\_\_\_\_\_\_\_\_\_\_\_\_\_\_\_\_\_\_\_\_\_\_\_\_\_\_\_\_\_\_\_\_\_\_\_\_\_\_\_\_\_\_\_\_\_\_\_\_\_\_\_\_\_\_\_\_\_\_\_\_\_\_\_\_\_\_\_\_\_\_\_\_\_\_\_\_\_\_\_\_\_\_\_\_\_\_\_\_\_\_\_\_\_\_\_\_\_\_\_\_\_\_\_\_\_\_\_\_\_\_\_\_\_\_\_\_\_\_\_\_\_\_\_\_\_\_\_\_\_\_\_\_\_\_\_\_\_\_\_\_\_\_\_\_\_\_\_\_\_\_\_\_\_\_\_\_\_\_\_\_\_\_\_\_\_\_\_\_\_\_\_\_\_\_\_\_\_\_\_\_\_\_\_\_\_\_\_\_\_\_\_\_\_\_\_\_\_\_\_\_\_\_\_\_\_\_\_\_\_\_\_\_\_\_\_\_\_\_\_\_\_\_\_\_\_\_\_\_\_\_\_\_\_\_\_\_\_\_\_\_\_\_\_\_\_\_\_\_\_\_\_\_\_\_\_\_\_\_\_\_\_\_\_\_\_\_\_\_\_\_\_\_\_\_\_\_\_\_\_\_\_\_\_\_\_\_\_\_\_\_\_\_\_\_\_\_\_\_\_\_\_\_\_\_\_\_\_\_\_\_\_\_\_\_\_\_\_\_\_\_\_\_\_\_\_\_\_\_\_\_\_\_\_\_\_\_\_\_\_\_\_\_\_\_\_\_\_\_\_\_\_\_\_\_\_\_\_\_\_\_\_\_\_\_\_\_\_\_\_\_\_\_\_\_\_\_\_\_\_\_\_\_\_\_\_\_\_\_\_\_\_\_\_\_\_\_\_\_\_\_\_\_\_\_\_\_\_\_\_\_\_\_\_\_\_\_\_\_\_\_\_\_\_\_\_\_\_\_\_\_\_\_\_\_\_\_\_\_\_\_\_\_\_\_\_\_\_\_\_\_\_\_\_\_\_\_\_\_\_\_\_\_\_\_\_\_\_\_\_\_\_\_\_\_\_\_\_\_\_\_\_\_\_\_\_\_\_\_\_\_\_\_\_\_\_\_\_\_\_\_\_\_\_\_\_\_\_\_\_\_\_\_\_\_\_\_\_\_\_\_\_\_\_\_\_\_\_\_\_\_\_\_\_\_\_\_\_\_\_\_\_\_\_\_\_\_\_\_\_\_\_\_\_\_\_\_\_\_\_\_\_\_\_\_\_\_\_\_\_\_\_\_\_\_\_\_\_\_\_\_\_\_\_\_\_\_\_\_\_\_\_\_\_\_\_\_\_\_\_\_\_\_\_\_\_\_\_\_\_\_\_\_\_\_\_\_\_\_\_\_\_\_\_\_\_\_\_\_\_\_\_\_\_\_\_\_\_\_\_\_\_\_\_\_\_\_\_\_\_\_\_\_\_\_\_\_\_\_\_\_\_\_\_\_\_\_\_\_\_\_\_\_\_\_\_\_\_\_\_\_\_\_\_\_\_\_\_\_\_\_\_\_\_\_\_\_\_\_\_\_\_\_\_\_\_\_\_\_\_\_\_\_\_\_\_\_\_\_\_\_\_\_\_\_\_\_\_\_\_\_\_\_\_\_\_\_\_\_\_\_\_\_\_\_\_\_\_\_\_\_\_\_\_\_\_\_\_\_\_\_\_\_\_\_\_\_\_\_\_\_\_\_\_\_\_\_\_\_\_\_\_\_\_\_\_\_\_\_\_\_\_\_\_\_\_\_\_\_\_\_\_\_\_\_\_\_\_\_\_\_\_\_\_\_\_\_\_\_\_\_\_\_\_\_\_\_\_\_\_\_\_\_\_\_\_\_\_\_\_\_\_\_\_\_\_\_\_\_\_\_\_\_\_\_\_\_\_\_\_\_\_\_\_\_\_\_\_\_\_\_\_\_\_\_\_\_\_\_\_\_\_\_\_\_\_\_\_\_\_\_\_\_\_\_\_\_\_\_\_\_\_\_\_\_\_\_\_\_\_\_\_\_\_\_\_\_\_\_\_\_\_\_\_\_\_\_\_\_\_\_\_\_\_\_\_\_\_\_\_\_\_\_\_\_\_\_\_\_\_\_\_\_\_\_\_\_\_\_\_\_\_\_\_\_\_\_\_\_\_\_\_\_\_\_\_\_\_\_\_\_\_\_\_\_\_\_\_\_\_\_\_\_\_\_\_\_\_\_\_\_\_\_\_\_\_\_\_\_\_\_\_\_\_\_\_\_\_\_\_\_\_\_\_\_\_\_\_\_\_\_\_\_\_\_\_\_\_\_\_\_\_\_\_\_\_\_\_\_\_\_\_\_\_\_\_\_\_\_\_\_\_\_\_\_\_\_\_\_\_\_\_\_\_\_\_\_\_\_\_\_\_\_\_\_\_\_\_\_\_\_\_\_\_\_\_\_\_\_\_\_\_\_\_\_\_\_\_\_\_\_\_\_\_\_\_\_\_\_\_\_\_\_\_\_\_\_\_\_\_\_\_\_\_\_\_\_\_\_\_\_\_\_\_\_\_\_\_\_\_\_\_\_\_\_\_\_\_\_\_\_\_\_\_\_\_\_\_\_\_\_\_\_\_\_\_\_\_\_\_\_\_\_\_\_\_\_\_\_\_\_\_\_\_\_\_\_\_\_\_\_\_\_\_\_\_\_\_\_\_\_\_\_\_\_\_\_\_\_\_\_\_\_\_\_\_\_\_\_\_\_\_\_\_\_\_\_\_\_\_\_\_\_\_\_\_\_\_\_\_\_\_\_\_\_\_\_\_\_\_\_\_\_\_\_\_\_\_\_\_\_\_\_\_\_\_\_\_\_\_\_\_\_\_\_\_\_\_\_\_\_\_\_\_\_\_\_\_\_\_\_\_\_\_\_\_\_\_\_\_\_\_\_\_\_\_\_\_\_\_\_\_\_\_\_\_\_\_\_\_\_\_\_\_\_\_\_\_\_\_\_\_\_\_\_\_\_\_\_\_\_\_\_\_\_\_\_\_\_\_\_\_\_\_\_\_\_\_\_\_\_\_\_\_\_\_\_\_\_\_\_\_\_\_\_\_\_\_\_\_\_\_\_\_\_\_\_\_\_\_\_\_\_\_\_\_\_\_\_\_\_\_\_\_\_\_\_\_\_\_\_\_\_\_\_\_\_\_\_\_\_\_\_\_\_\_\_\_\_\_\_\_\_\_\_\_\_\_\_\_\_\_\_\_\_\_\_\_\_\_\_\_\_\_\_\_\_\_\_\_\_\_\_\_\_\_\_\_\_\_\_\_\_\_\_\_\_\_\_\_\_\_\_\_\_\_\_\_\_\_\_\_\_\_\_\_\_\_\_\_\_\_\_\_\_\_\_\_\_\_\_\_\_\_\_\_\_\_\_\_\_\_\_\_\_\_\_\_\_\_\_\_\_\_\_\_\_\_\_\_\_\_\_\_\_\_\_\_\_\_\_\_\_\_\_\_\_\_\_\_\_\_\_\_\_\_\_\_\_\_\_\_\_\_\_\_\_\_\_\_\_\_\_\_\_\_\_\_\_\_\_\_\_\_\_\_\_\_\_\_\_\_\_\_\_\_\_\_\_\_\_\_\_\_\_\_\_\_\_\_\_\_\_\_\_\_\_\_\_\_\_\_\_\_\_\_\_\_\_\_\_\_\_\_\_\_\_\_\_\_\_\_\_\_\_\_\_\_\_\_\_\_\_\_\_\_\_\_\_\_\_\_\_\_\_\_\_\_\_\_\_\_\_\_\_\_\_\_\_\_\_\_\_\_\_\_\_\_\_\_\_\_\_\_\_\_\_\_\_\_\_\_\_\_\_\_\_\_\_\_\_\_\_\_\_\_\_\_\_\_\_\_\_\_\_\_\_\_\_\_\_\_\_\_\_\_\_\_\_\_\_\_\_\_\_\_\_\_\_\_\_\_\_\_\_\_\_\_\_\_\_\_\_\_\_\_\_\_\_\_\_\_\_\_\_\_\_\_\_\_\_\_\_\_\_\_\_\_\_\_\_\_\_\_\_\_\_\_\_\_\_\_\_\_\_\_\_\_\_\_\_\_\_\_\_\_\_\_\_\_\_\_\_\_\_\_\_\_\_\_\_\_\_\_\_\_\_\_\_\_\_\_\_\_\_\_\_\_\_\_\_\_\_\_\_\_\_\_\_\_\_\_\_\_\_\_\_\_\_\_\_\_\_\_\_\_\_\_\_\_\_\_\_\_\_\_\_\_\_\_\_\_\_\_\_\_\_\_\_\_\_\_\_\_\_\_\_\_\_\_\_\_\_\_\_\_\_\_\_\_\_\_\_\_\_\_\_\_\_\_\_\_\_\_\_\_\_\_\_\_\_\_\_\_\_\_\_\_\_\_\_\_\_\_\_\_\_\_\_\_\_\_\_\_\_\_\_\_\_\_\_\_\_\_\_\_\_\_\_\_\_\_\_\_\_\_\_\_\_\_\_\_\_\_\_\_\_\_\_\_\_\_\_\_\_\_\_\_\_\_\_\_\_\_\_\_\_\_\_\_\_\_\_\_\_\_\_\_\_\_\_\_\_\_\_\_\_\_\_\_\_\_\_\_\_\_\_\_\_\_\_\_\_\_\_\_\_\_\_\_\_\_\_\_\_\_\_\_\_\_\_\_\_\_\_\_\_\_\_\_\_\_\_\_\_\_\_\_\_\_\_\_\_\_\_\_\_\_\_\_\_\_\_\_\_\_\_\_\_\_\_\_\_\_\_\_\_\_\_\_\_\_\_\_\_\_\_\_\_\_\_\_\_\_\_\_\_\_\_\_\_\_\_\_\_\_\_\_\_\_\_\_\_\_\_\_\_\_\_\_\_\_\_\_\_\_\_\_\_\_\_\_\_\_\_\_\_\_\_\_\_\_\_\_\_\_\_\_\_\_\_\_\_\_\_\_\_\_\_\_\_\_\_\_\_\_\_\_\_\_\_\_\_\_\_\_\_\_\_\_\_\_\_\_\_\_\_\_\_\_\_\_\_\_\_\_\_\_\_\_\_\_\_\_\_\_\_\_\_\_\_\_\_\_\_\_\_\_\_\_\_\_\_\_\_\_\_\_\_\_\_\_\_\_\_\_\_\_\_\_\_\_\_\_\_\_\_\_\_\_\_\_\_\_\_\_\_\_\_\_\_\_\_\_\_\_\_\_\_\_\_\_\_\_\_\_\_\_\_\_\_\_\_\_\_\_\_\_\_\_\_\_\_\_\_\_\_\_\_\_\_\_\_\_\_\_\_\_\_\_\_\_\_\_\_\_\_\_\_\_\_\_\_\_\_\_\_\_\_\_\_\_\_\_\_\_\_\_\_\_\_\_\_\_\_\_\_\_\_\_\_\_\_\_\_\_\_\_\_\_\_\_\_\_\_\_\_\_\_\_\_\_\_\_\_\_\_\_\_\_\_\_\_\_\_\_\_\_\_\_\_\_\_\_\_\_\_\_\_\_\_\_\_\_\_\_\_\_\_\_\_\_\_\_\_\_\_\_\_\_\_\_\_\_\_\_\_\_\_\_\_\_\_\_\_\_\_\_\_\_\_\_\_\_\_\_\_\_\_\_\_\_\_\_\_\_\_\_\_\_\_\_\_\_\_\_\_\_\_\_\_\_\_\_\_\_\_\_\_\_\_\_\_\_\_\_\_\_\_\_\_\_\_\_\_\_\_\_\_\_\_\_\_\_\_\_\_\_\_\_\_\_\_\_\_\_\_\_\_\_\_\_\_\_\_\_\_\_\_\_\_\_\_\_\_\_\_\_\_\_\_\_\_\_\_\_\_\_\_\_\_\_\_\_\_\_\_\_\_\_\_\_\_\_\_\_\_\_\_\_\_\_\_\_\_\_\_\_\_\_\_\_\_\_\_\_\_\_\_\_\_\_\_\_\_\_\_\_\_\_\_\_\_\_\_\_\_\_\_\_\_\_\_\_\_\_\_\_\_\_\_\_\_\_\_\_\_\_\_\_\_\_\_\_\_\_\_\_\_\_\_\_\_\_\_\_\_\_\_\_\_\_\_\_\_\_\_\_\_\_\_\_\_\_\_\_\_\_\_\_\_\_\_\_\_\_\_\_\_\_\_\_\_\_\_\_\_\_\_\_\_\_\_\_\_\_\_\_\_\_\_\_\_\_\_\_\_\_\_\_\_\_\_\_\_\_\_\_\_\_\_\_\_\_\_\_\_\_\_\_\_\_\_\_\_\_\_\_\_\_\_\_\_\_\_\_\_\_\_\_\_\_\_\_\_\_\_\_\_\_\_\_\_\_\_\_\_\_\_\_\_\_\_\_\_\_\_\_\_\_\_\_\_\_\_\_\_\_\_\_\_\_\_\_\_\_\_\_\_\_\_\_\_\_\_\_\_\_\_\_\_\_\_\_\_\_\_\_\_\_\_\_\_\_\_\_\_\_\_\_\_\_\_\_\_\_\_\_\_\_\_\_\_\_\_\_\_\_\_\_\_\_\_\_\_\_\_\_\_\_\_\_\_\_\_\_\_\_\_\_\_\_\_\_\_\_\_\_\_\_\_\_\_\_\_\_\_\_\_\_\_\_\_\_\_\_\_\_\_\_\_\_\_\_\_\_\_\_\_\_\_\_\_\_\_\_\_\_\_\_\_\_\_\_\_\_\_\_\_\_\_\_\_\_\_\_\_\_\_\_\_\_\_\_\_\_\_\_\_\_\_\_\_\_\_\_\_\_\_\_\_\_\_\_\_\_\_\_\_\_\_\_\_\_\_\_\_\_\_\_\_\_\_\_\_\_\_\_\_\_\_\_\_\_\_\_\_\_\_\_\_\_\_\_\_\_\_\_\_\_\_\_\_\_\_\_\_\_\_\_\_\_\_\_\_\_\_\_\_\_\_\_\_\_\_\_\_\_\_\_\_\_\_\_\_\_\_\_\_\_\_\_\_\_\_\_\_\_\_\_\_\_\_\_\_\_\_\_\_\_\_\_\_\_\_\_\_\_\_\_\_\_\_\_\_\_\_\_\_\_\_\_\_\_\_\_\_\_\_\_\_\_\_\_\_\_\_\_\_\_\_\_\_\_\_\_\_\_\_\_\_\_\_\_\_\_\_\_\_\_\_\_\_\_\_\_\_\_\_\_\_\_\_\_\_\_\_\_\_\_\_\_\_\_\_\_\_\_\_\_\_\_\_\_\_\_\_\_\_\_\_\_\_\_\_\_\_\_\_\_\_\_\_\_\_\_\_\_\_\_\_\_\_\_\_\_\_\_\_\_\_\_\_\_\_\_\_\_\_\_\_\_\_\_\_\_\_\_\_\_\_\_\_\_\_\_\_\_\_\_\_\_\_\_\_\_\_\_\_\_\_\_\_\_\_\_\_\_\_\_\_\_\_\_\_\_\_\_\_\_\_\_\_\_\_\_\_\_\_\_\_\_\_\_\_\_\_\_\_\_\_\_\_\_\_\_\_\_\_\_\_\_\_\_\_\_\_\_\_\_\_\_\_\_\_\_\_\_\_\_\_\_\_\_\_\_\_\_\_\_\_\_\_\_\_\_\_\_\_\_\_\_\_\_\_\_\_\_\_\_\_\_\_\_\_\_\_\_\_\_\_\_\_\_\_\_\_\_\_\_\_\_\_\_\_\_\_\_\_\_\_\_\_\_\_\_\_\_\_\_\_\_\_\_\_\_\_\_\_\_\_\_\_\_\_\_\_\_\_\_\_\_\_\_\_\_\_\_\_\_\_\_\_\_\_\_\_\_\_\_\_\_\_\_\_\_\_\_\_\_\_\_\_\_\_\_\_\_\_\_\_\_\_\_\_\_\_\_\_\_\_\_\_\_\_\_\_\_\_\_\_\_\_\_\_\_\_\_\_\_\_\_\_\_\_\_\_\_\_\_\_\_\_\_\_\_\_\_\_\_\_\_\_\_\_\_\_\_\_\_\_\_\_\_\_\_\_\_\_\_\_\_\_\_\_\_\_\_\_\_\_\_\_\_\_\_\_\_\_\_\_\_\_\_\_\_\_\_\_\_\_\_\_\_\_\_\_\_\_\_\_\_\_\_\_\_\_\_\_\_\_\_\_\_\_\_\_\_\_\_\_\_\_\_\_\_\_\_\_\_\_\_\_\_\_\_\_\_\_\_\_\_\_\_\_\_\_\_\_\_\_\_\_\_\_\_\_\_\_\_\_\_\_\_\_\_\_\_\_\_\_\_\_\_\_\_\_\_\_\_\_\_\_\_\_\_\_\_\_\_\_\_\_\_\_\_\_\_\_\_\_\_\_\_\_\_\_\_\_\_\_\_\_\_\_\_\_\_\_\_\_\_\_\_\_\_\_\_\_\_\_\_\_\_\_\_\_\_\_\_\_\_\_\_\_\_\_\_\_\_\_\_\_\_\_\_\_\_\_\_\_\_\_\_\_\_\_\_\_\_\_\_\_\_\_\_\_\_\_\_\_\_\_\_\_\_\_\_\_\_\_\_\_\_\_\_\_\_\_\_\_\_\_\_\_\_\_\_\_\_\_\_\_\_\_\_\_\_\_\_\_\_\_\_\_\_\_\_\_\_\_\_\_\_\_\_\_\_\_\_\_\_\_\_\_\_\_\_\_\_\_\_\_\_\_\_\_\_\_\_\_\_\_\_\_\_\_\_\_\_\_\_\_\_\_\_\_\_\_\_\_\_\_\_\_\_\_\_\_\_\_\_\_\_\_\_\_\_\_\_\_\_\_\_\_\_\_\_\_\_\_\_\_\_\_\_\_\_\_\_\_\_\_\_\_\_\_\_\_\_\_\_\_\_\_\_\_\_\_\_\_\_\_\_\_\_\_\_\_\_\_\_\_\_\_\_\_\_\_\_\_\_\_\_\_\_\_\_\_\_\_\_\_\_\_\_\_\_\_\_\_\_\_\_\_\_\_\_\_\_\_\_\_\_\_\_\_\_\_\_\_\_\_\_\_\_\_\_\_\_\_\_\_\_\_\_\_\_\_\_\_\_\_\_\_\_\_\_\_\_\_\_\_\_\_\_\_\_\_\_\_\_\_\_\_\_\_\_\_\_\_\_\_\_\_\_\_\_\_\_\_\_\_\_\_\_\_\_\_\_\_\_\_\_\_\_\_\_\_\_\_\_\_\_\_\_\_\_\_\_\_\_\_\_\_\_\_\_\_\_\_\_\_\_\_\_\_\_\_\_\_\_\_\_\_\_\_\_\_\_\_\_\_\_\_\_\_\_\_\_\_\_\_\_\_\_\_\_\_\_\_\_\_\_\_\_\_\_\_\_\_\_\_\_\_\_\_\_\_\_\_\_\_\_\_\_\_\_\_\_\_\_\_\_\_\_\_\_\_\_\_\_\_\_\_\_\_\_\_\_\_\_\_\_\_\_\_\_\_\_\_\_\_\_\_\_\_\_\_\_\_\_\_\_\_\_\_\_\_\_\_\_\_\_\_\_\_\_\_\_\_\_\_\_\_\_\_\_\_\_\_\_\_\_\_\_\_\_\_\_\_\_\_\_\_\_\_\_\_\_\_\_\_\_\_\_\_\_\_\_\_\_\_\_\_\_\_\_\_\_\_\_\_\_\_\_\_\_\_\_\_\_\_\_\_\_\_\_\_\_\_\_\_\_\_\_\_\_\_\_\_\_\_\_\_\_\_\_\_\_\_\_\_\_\_\_\_\_\_\_\_\_\_\_\_\_\_\_\_\_\_\_\_\_\_\_\_\_\_\_\_\_\_\_\_\_\_\_\_\_\_\_\_\_\_\_\_\_\_\_\_\_\_\_\_\_\_\_\_\_\_\_\_\_\_\_\_\_\_\_\_\_\_\_\_\_\_\_\_\_\_\_\_\_\_\_\_\_\_\_\_\_\_\_\_\_\_\_\_\_\_\_\_\_\_\_\_\_\_\_\_\_\_\_\_\_\_\_\_\_\_\_\_\_\_\_\_\_\_\_\_\_\_\_\_\_\_\_\_\_\_\_\_\_\_\_\_\_\_\_\_\_\_\_\_\_\_\_\_\_\_\_\_\_\_\_\_\_\_\_\_\_\_\_\_\_\_\_\_\_\_\_\_\_\_\_\_\_\_\_\_\_\_\_\_\_\_\_\_\_\_\_\_\_\_\_\_\_\_\_\_\_\_\_\_\_\_\_\_\_\_\_\_\_\_\_\_\_\_\_\_\_\_\_\_\_\_\_\_\_\_\_\_\_\_\_\_\_\_\_\_\_\_\_\_\_\_\_\_\_\_\_\_\_\_\_\_\_\_\_\_\_\_\_\_\_\_\_\_\_\_\_\_\_\_\_\_\_\_\_\_\_\_\_\_\_\_\_\_\_\_\_\_\_\_\_\_\_\_\_\_\_\_\_\_\_\_\_\_\_\_\_\_\_\_\_\_\_\_\_\_\_\_\_\_\_\_\_\_\_\_\_\_\_\_\_\_\_\_\_\_\_\_\_\_\_\_\_\_\_\_\_\_\_\_\_\_\_\_\_\_\_\_\_\_\_\_\_\_\_\_\_\_\_\_\_\_\_\_\_\_\_\_\_\_\_\_\_\_\_\_\_\_\_\_\_\_\_\_\_\_\_\_\_\_\_\_\_\_\_\_\_\_\_\_\_\_\_\_\_\_\_\_\_\_\_\_\_\_\_\_\_\_\_\_\_\_\_\_\_\_\_\_\_\_\_\_\_\_\_\_\_\_\_\_\_\_\_\_\_\_\_\_\_\_\_\_\_\_\_\_\_\_\_\_\_\_\_\_\_\_\_\_\_\_\_\_\_\_\_\_\_\_\_\_\_\_\_\_\_\_\_\_\_\_\_\_\_\_\_\_\_\_\_\_\_\_\_\_\_\_\_\_\_\_\_\_\_\_\_\_\_\_\_\_\_\_\_\_\_\_\_\_\_\_\_\_\_\_\_\_\_\_\_\_\_\_\_\_\_\_\_\_\_\_\_\_\_\_\_\_\_\_\_\_\_\_\_\_\_\_\_\_\_\_\_\_\_\_\_\_\_\_\_\_\_\_\_\_\_\_\_\_\_\_\_\_\_\_\_\_\_\_\_\_\_\_\_\_\_\_\_\_\_\_\_\_\_\_\_\_\_\_\_\_\_\_\_\_\_\_\_\_\_\_\_\_\_\_\_\_\_\_\_\_\_\_\_\_\_\_\_\_\_\_\_\_\_\_\_\_\_\_\_\_\_\_\_\_\_\_\_\_\_\_\_\_\_\_\_\_\_\_\_\_\_\_\_\_\_\_\_\_\_\_\_\_\_\_\_\_\_\_\_\_\_\_\_\_\_\_\_\_\_\_\_\_\_\_\_\_\_\_\_\_\_\_\_\_\_\_\_\_\_\_\_\_\_\_\_\_\_\_\_\_\_\_\_\_\_\_\_\_\_\_\_\_\_\_\_\_\_\_\_\_\_\_\_\_\_\_\_\_\_\_\_\_\_\_\_\_\_\_\_\_\_\_\_\_\_\_\_\_\_\_\_\_\_\_\_\_\_\_\_\_\_\_\_\_\_\_\_\_\_\_\_\_\_\_\_\_\_\_\_\_\_\_\_\_\_\_\_\_\_\_\_\_\_\_\_\_\_\_\_\_\_\_\_\_\_\_\_\_\_\_\_\_\_\_\_\_\_\_\_\_\_\_\_\_\_\_\_\_\_\_\_\_\_\_\_\_\_\_\_\_\_\_\_\_\_\_\_\_\_\_\_\_\_\_\_\_\_\_\_\_\_\_\_\_\_\_\_\_\_\_\_\_\_\_\_\_\_\_\_\_\_\_\_\_\_\_\_\_\_\_\_\_\_\_\_\_\_\_\_\_\_\_\_\_\_\_\_\_\_\_\_\_\_\_\_\_\_\_\_\_\_\_\_\_\_\_\_\_\_\_\_\_\_\_\_\_\_\_\_\_\_\_\_\_\_\_\_\_\_\_\_\_\_\_\_\_\_\_\_\_\_\_\_\_\_\_\_\_\_\_\_\_\_\_\_\_\_\_\_\_\_\_\_\_\_\_\_\_\_\_\_\_\_\_\_\_\_\_\_\_\_\_\_\_\_\_\_\_\_\_\_\_\_\_\_\_\_\_\_\_\_\_\_\_\_\_\_\_\_\_\_\_\_\_\_\_\_\_\_\_\_\_\_\_\_\_\_\_\_\_\_\_\_\_\_\_\_\_\_\_\_\_\_\_\_\_\_\_\_\_\_\_\_\_\_\_\_\_\_\_\_\_\_\_\_\_\_\_\_\_\_\_\_\_\_\_\_\_\_\_\_\_\_\_\_\_\_\_\_\_\_\_\_\_\_\_\_\_\_\_\_\_\_\_\_\_\_\_\_\_\_\_\_\_\_\_\_\_\_\_\_\_\_\_\_\_\_\_\_\_\_\_\_\_\_\_\_\_\_\_\_\_\_\_\_\_\_\_\_\_\_\_\_\_\_\_\_\_\_\_\_\_\_\_\_\_\_\_\_\_\_\_\_\_\_\_\_\_\_\_\_\_\_\_\_\_\_\_\_\_\_\_\_\_\_\_\_\_\_\_\_\_\_\_\_\_\_\_\_\_\_\_\_\_\_\_\_\_\_\_\_\_\_\_\_\_\_\_\_\_\_\_\_\_\_\_\_\_\_\_\_\_\_\_\_\_\_\_\_\_\

符号说明:

- $N$——辅机总架数
- $r_{N}$——针对问题 2,辅机总架数为 $N$,每架辅机只能加油一次的最优作战半径
- $r_{N}^{*}$——$r_{N}$ 的下界
- $R_{N}$——针对问题 3,辅机总架数为 $N$,每架辅机可加油多次的最优作战半径
- $R_{N}^{*}$——$R_{N}$ 的下界
- $a_{n}$——采用集群算法,以 $(n+1)$ 为基数,当 $n$ 固定时每一单元前进的最大距离

问题 1 设飞机垂直起飞、垂直降落、空中转向、在地面或空中加油的耗时均忽略不计,每架飞机只能上天一次,在上述假设下的作战半径记为 $r_{n}$。当 $n=1,2,3,4$ 时,求作战半径 $r_{n}$。

解答:

下面给出问题 1 最优解的必要条件的一些命题,并且找出 $n=1,2,3,4$ 的最优解。

命题 1.1 主机与辅机的航线不可能出现折线。

证明:假设主机的航线出现折线,即主机回头时其它辅机给它加油。那么可以直接用辅机在加油处给主机加油,省去了中间来回的浪费。同样假设如果送的辅机 1 改变方向,则在返程时有辅机 2 给它加油让它变向,则它追赶不上主机,只能去接主机。这样辅机 1 来回也造成了浪费。

命题 1.2 负责送的辅机必须与主机同时出发。

证明:由上面的命题,辅机不能改变方向,当然只能往一定方向行驶。如果不和主机同时出发就追不上主机,起不到送主机的用处。

命题 1.3 负责加油的辅机只给其它飞机加油一次随即返回基地。

证明:如果一架辅机多次给其他辅机和主机加油,那么可以直接在最后一次给其他辅机和主机加油。当中间没有辅机回基地时,辅机和主机总的油量一定,

所以中间过程没必要加油,只有返回基地前进行重新分配。

命题 1.4 $r_n$ 达到最大的一个必要条件为:辅机可以分为两类,第一类专为主机前进服务,第二类专为主机返回服务,且第一类与第二类的架数的差趋向于零,即,趋向对称。

(i) 当 $n=2k$ 时,第一类和第二类的架数分别为 $k$.

(ii) 当 $n=2k-1$ 时,第一类和第二类的架数分别为 $k, k-1, k$ 或 $k, k-1$,且两者方案等价。

证明:(i) 当 $n=2k$ 时,考虑 $k\_k$($k$ 架送,$k$ 架接)方案和 $k+1\_k-1$($k+1$ 架送,$k-1$ 架接)方案:设飞机前进 $x$,与后 $1-x$ 的过程对称。

(2) 当 $n=2k-1$ 时,由 (1) 的证明可得。

命题 1.5 $k\_k$($k$ 架送,$k$ 架接)方案中 $r_n$ 要优于 $k\_k-1$($k$ 架送,$k-1$ 架接)方案。

命题 1.6 飞行程序是可逆的,即接与送的过程可以互换。

证明:对任何一个可行的飞行程序,只要把整个飞行程序倒过来,也是一个可行的飞行程序。如,$n=1$ 时,主机与副机同时起飞,飞行 $1/3$,给主机加油 $1/3$,辅机返回,主机飞行到 $2/3$ 处返回。反之,若主机先飞 $2/3$,返回,同时辅机起飞,在 $1/3$ 处相遇,加油,返回。

$n=1$ 时,只有一架辅机,只能送一次或者接一次。由于送和接的互换性,我们只考虑送一次。设辅机向前送主机的消耗的油量为 $x$,在 B 处给主机加油量为 $y$。如下图

\begin{figure}[h]
\centering
\includegraphics[width=0.8\textwidth]{image.png}
\end{figure}

\begin{tikzpicture}[scale=1.5]
    % 辅机和主机的路径
    \draw[dashed] (0,0) -- (0,3);
    \draw[dashed] (4,0) -- (4,3);
    \draw[dashed] (6,0) -- (6,3);
    
    % 辅机和主机的箭头
    \draw[->, thick] (0,2.5) -- (4,2.5) node[midway, above] {$x$};
    \draw[->, thick] (0,2) -- (4,2);
    \draw[->, thick] (0,1.5) -- (6,1.5) node[midway, above] {$z/3$};
    \draw[->, thick] (0,1) -- (6,1);
    \draw[->, thick] (0,0.5) -- (4,0.5) node[midway, above] {$x+z/3$};
    \draw[->, thick] (0,0) -- (6,0);
    
    % 标注
    \node at (0,3.2) {A};
    \node at (4,3.2) {B};
    \node at (6,3.2) {C};
    \node at (0,-0.2) {D};
    \node at (4,-0.2) {E};
    \node at (6,-0.2) {F};
    
    % 辅机和主机的标注
    \node at (-0.5,2.25) {辅机};
    \node at (-0.5,1.75) {主机};
    \node at (-0.5,0.75) {辅机};
    \node at (-0.5,0.25) {主机};
\end{tikzpicture}

\textbf{命题 1.7} 辅机和主机回到基地 A 的时候油量正好消耗完毕。

\textbf{证明:} 若不然,如上图辅机回基地的时候油量还有剩余 $z$,那么可以让辅机在 B 处不给主机加油,而是在耗油 $x+z/3$,即 C 处给主机加油 $y+z/3$,在 C 处再返回基地时辅机剩余的油量为 0。它比上次多行了相对于耗油 $2z/3$ 的距离,同时给主机 $z/3$。原来在 B 处给主机加油 $y$,现在 C 处给主机加油 $y+z/3$。第一次主机在 B 处的油量为 $1-x+y$,第二次主机在 C 处的油量为 $1-(x+z/3)+(y+z/3)=1-x+y$。由此可见,主机在这两处的油量相同,但是在后面的情况下主机行驶得更远了。所以要充分利用辅机的油量,在达到最优解时辅机回基地剩余的油量为 0。若主机回基地油有剩余,则主机同样可以再向前飞远点再回基地。

\begin{tikzpicture}
    \draw[dashed] (0,0) -- (0,4);
    \draw[dashed] (6,0) -- (6,4);
    \draw[->] (0,3) -- (6,3) node[midway, above] {x};
    \draw[->] (0,2) -- (6,2);
    \draw[->] (0,1) -- (6,1);
    \draw[->] (0,0) -- (6,0);
    \node at (-0.5,3.5) {A};
    \node at (6.5,3.5) {B};
    \node at (-0.5,2.5) {辅机};
    \node at (-0.5,1.5) {主机};
    \node at (-0.5,0.5) {辅机};
    \node at (-0.5,-0.5) {主机};
    \draw[dashed] (0,-4) -- (0,-8);
    \draw[dashed] (8,-4) -- (8,-8);
    \draw[->] (0,-5) -- (8,-5) node[midway, above] {x-z};
    \draw[->] (0,-6) -- (8,-6);
    \draw[->] (0,-7) -- (8,-7);
    \draw[->] (0,-8) -- (8,-8);
    \node at (-0.5,-4.5) {D};
    \node at (2,-4.5) {E};
    \node at (4,-4.5) {F};
    \node at (8.5,-4.5) {G};
    \node at (6.5,-5.5) {C};
    \node at (6.5,-6.5) {z};
    \node at (-0.5,-5.5) {辅机};
    \node at (-0.5,-6.5) {主机};
    \node at (-0.5,-7.5) {辅机};
    \node at (-0.5,-8.5) {主机};
\end{tikzpicture}

\textbf{命题 1.8} 辅机给主机加油时一定要让主机加满。

\textbf{证明:} 如若不然, 辅机在加油处给主机加油时不让主机加满, 如上图所示。同样如上所述辅机在在耗油 \(x\), 即 B 处给主机加油 \(y\), 主机在此处的油量为 \(1-x+y\), 这时是不满的。当然此时要使得辅机正好可以回去。我们考虑在 B 处前面一点 C 处的情况, 即如果在耗油 \(x-z\), 即 C 处给主机加油 \(y+3z\) 也可以保证辅机正好可以回去, 这时 \(z\) 是很小的, 则主机在该处也不会加满。那么这时主机行使到 B 处的油量为 \(1-x+y+3z\), 明显比前面的情况好。所以让主机加满时情况更好。下面我们来求解 \(n=1\) 时的最优解。

\begin{tikzpicture}
    \draw[->] (0,0) -- (4,0) node[midway, above] {出发};
    \draw[<-] (0,-1) -- (4,-1) node[midway, above] {返回};
    \node at (0,-2) {1 架辅机};
    \draw[dashed] (0,-3) -- (0,-7);
    \draw[dashed] (4,-3) -- (4,-7);
    \draw[dashed] (8,-3) -- (8,-7);
    \draw[->] (0,-4) -- (4,-4) node[midway, above] {1/3};
    \draw[->] (4,-5) -- (8,-5) node[midway, above] {1/3};
    \draw[->] (0,-6) -- (8,-6);
    \node at (-0.5,-3.5) {A};
    \node at (4.5,-3.5) {B};
    \node at (8.5,-3.5) {C};
    \node at (-0.5,-4.5) {辅机};
    \node at (-0.5,-5.5) {主机};
    \node at (-0.5,-6.5) {辅机};
    \node at (-0.5,-7.5) {主机};
\end{tikzpicture}

\(n=1\) 时, 由于满足上述条件, 第一辅机到达基地时油量要消耗完毕, 第二辅机给主机加油时要保证加满。也就是说辅机来回和送主机的路程消耗的油量都由

这个辅机提供。也就是 \( L/3 \)。最后要保证主机能够回来,计算可得主机向前行驶 \( L/3 \) 可以回到基地。整个过程如上图所示,主机向前行驶 \( \frac{L}{3} + \frac{L}{3} = \frac{2L}{3} \)。

\section*{2 架辅机}

\begin{tikzpicture}
    \draw[dashed] (0,0) -- (0,4);
    \draw[dashed] (4,0) -- (4,4);
    \draw[dashed] (8,0) -- (8,4);
    
    \node at (0,4.5) {A};
    \node at (4,4.5) {B};
    \node at (8,4.5) {C};
    
    \node at (2,3.5) {1/3};
    \node at (6,3.5) {1/2};
    \node at (6,0.5) {1/3};
    
    \draw[->, thick] (0,3) -- (4,3);
    \draw[->, thick] (0,2) -- (4,2);
    \draw[->, thick] (0,1) -- (4,1);
    \draw[->, thick] (0,0) -- (4,0);
    
    \draw[->, thick] (4,2) -- (8,2);
    \draw[->, thick] (4,1) -- (8,1);
    
    \node at (2,2.5) {辅机 1};
    \node at (2,1.5) {主机};
    \node at (2,0.5) {辅机 2};
\end{tikzpicture}

\( n=2 \) 时,只有两架辅机,由于送和接的互换性,我们只考虑一架辅机送一架辅机接。考虑到上述条件,一架送的飞机和前面分析相同,送主机到耗油 \( L/3 \),即 B 处,恰好可以返回基地。而此时主机正好加满。主机往前飞耗油 \( L/2 \) 到达 C 时返回到 B 处正好把油消耗完毕,这时接主机的另一架辅机经同样过程正好把主机一起带回基地。整个过程如上图所示,主机向前行驶 \( \frac{L}{3} + \frac{L}{2} = \frac{5L}{6} \)。

\( n=3 \) 时,有三架辅机,由于互换性,我们考虑两架送机和一架接机。设有辅机 1 和辅机 2。下面给出一个命题:

\begin{figure}[h]
    \centering
    \includegraphics[width=\textwidth]{image.png}
    \caption{示意图}
\end{figure}

命题 1.9 辅机在加油处给辅机加油时一定要让辅机加满。

证明:若不然,如上图所示下面我们假设辅机 1 在消耗油量 $x$,B 处给辅机 2 加油 $y$ 并不加满,又前面的讨论必须给主机加满。同样我们考虑把加油点往后退一点,即如果在耗油 $x-z$,即 H 处给辅机 2 加油 $y+4z$ 也可以保证辅机正好可以回去,这时给主机加的油量要比前面少 $z$ 才能保证主机加满,这里 $z$ 是很小的。当辅机 2 和主机再次行使到前面说的耗油量 B 处时,辅机 2 剩余油量为 $1-x+y+4z$,主机剩余油量 $1-z$。而前一种情况辅机 1 在消耗油量 B 处给辅机 2 加油 $y$,主机加满时,这时辅机 2 剩余油量为 $1-x+y$,主机剩余油量 1。这时总的剩余油量比后面的情况少了 $3z$。所以辅机给辅机加油时应当加满。

$n=3$ 时,只有两架送的辅机,在满足上面的命题条件下有两种情况:

1. 两个辅机同时在一点处给主机加满。
2. 辅机 1 在第一个加油点处给辅机 2 和主机加满,辅机 2 在第二个加油点处给主机加满。

下面再给出一个命题,来解决这两种情况。

\begin{tikzpicture}[scale=0.8]
    % 辅助线
    \draw[dotted] (0,0) -- (10,0);
    \draw[dotted] (0,2) -- (10,2);
    \draw[dotted] (0,4) -- (10,4);
    \draw[dotted] (0,6) -- (10,6);
    \draw[dotted] (0,8) -- (10,8);
    \draw[dotted] (0,10) -- (10,10);
    \draw[dotted] (0,12) -- (10,12);
    \draw[dotted] (0,14) -- (10,14);
    \draw[dotted] (0,16) -- (10,16);
    \draw[dotted] (0,18) -- (10,18);
    \draw[dotted] (0,20) -- (10,20);
    
    % 标记点
    \node at (0,20) {A};
    \node at (10,20) {B};
    \node at (15,20) {C};
    \node at (0,10) {D};
    \node at (5,10) {E};
    \node at (10,10) {F};
    
    % 辅机1
    \draw[->] (0,18) -- (5,18);
    \node at (2.5,19) {$2/5$};
    \node at (0,18.5) {辅机1};
    
    % 辅机2
    \draw[->] (0,16) -- (5,16);
    \node at (0,16.5) {辅机2};
    
    % 主机
    \draw[->] (0,14) -- (10,14);
    \node at (0,14.5) {主机};
    
    % 辅机1
    \draw[->] (0,12) -- (5,12);
    \node at (2.5,13) {$y$};
    \node at (0,12.5) {辅机1};
    
    % 辅机2
    \draw[->] (0,10) -- (5,10);
    \node at (0,10.5) {辅机2};
    
    % 主机
    \draw[->] (0,8) -- (10,8);
    \node at (0,8.5) {主机};
    
    % 辅机1
    \draw[->] (0,6) -- (5,6);
    \node at (2.5,7) {$z$};
    \node at (0,6.5) {辅机1};
    
    % 辅机2
    \draw[->] (0,4) -- (5,4);
    \node at (0,4.5) {辅机2};
    
    % 主机
    \draw[->] (0,2) -- (10,2);
    \node at (0,2.5) {主机};
    
    % 标记点
    \node at (5,20) {x};
    \node at (10,20) {$1/2$};
    \node at (0,10) {$1/4$};
    \node at (5,10) {$1/4$};
\end{tikzpicture}

\textbf{命题 1.10} 不存在两架或两架以上的辅机同时给辅机和主机加油让辅机和主机加满。

\textbf{证明:} 在上面第一种情况下,我们将证明它不是最优的。上面的图示为简单只给出了两架辅机的情况。下面我们给出较为一般的证明,设有 $n$ 架辅机,在耗油 $x$,即 B 处同时给主机加满油。每架辅机给主机的油量为 $x/n$,则要使每架辅机恰好可以回基地,需满足:$2x=1-x/n$,$x=\frac{1}{2+\frac{1}{n}}$。如果我们考虑 $n-1$ 架辅机给辅机 $n$ 和主机在耗油 $y$,即 E 处加满油,然后辅机 $n$ 在往前耗油 $z$,即 F 处给主机加满油正好可以回去。在 E 处,$n-1$ 架辅机每架分给主机和辅机 $n$ 的油量为 $2y/(n-1)$,则 $2y=1-2y/(n-1)$,$y=\frac{1}{2(1+\frac{1}{n-1})}$。这时辅机 $n$ 和主机的油量是满的。在 F 处要保证辅机 $n$ 恰好可以回基地,则 $3z+y=1$,则 $z=\frac{1}{3}-\frac{1}{6(1+\frac{1}{n-1})}$。比较

这两种情况主机的送主机结束加满后的距离,即 B 和 F 可得

\begin{align*}
y + z &= \frac{1}{2\left(1 + \frac{1}{n-1}\right)} + \frac{1}{3} - \frac{1}{6\left(1 + \frac{1}{n-1}\right)} \\
&= \frac{2 + \frac{1}{n-1}}{3\left(1 + \frac{1}{n-1}\right)} = \frac{4 + \frac{2}{n} + \frac{2}{n-1} + \frac{1}{n(n-1)}}{3\left(1 + \frac{1}{n-1}\right)\left(2 + \frac{1}{n}\right)} \, .
\end{align*}

所以

\[
n > \frac{3 + \frac{3}{n-1}}{3\left(1 + \frac{1}{n-1}\right)\left(2 + \frac{1}{n}\right)} = x
\]

架辅机在 \( x \) 处同时给主机加油不如 \( n-1 \) 架辅机给辅机 \( n \) 和主机在 E 处加满油,然后辅机 \( n \) 在往前耗油 F 处给主机加满油好。

**命题 1.11** 每次只有一架辅机同时给主机和辅机加油让主机和辅机加满。

综合上述命题,我们知道辅机在加油处一定要让主机和辅机加满,并且每次只有一架辅机给主机和辅机加油。下面我们给出 3 架辅机的方案,如下图

**3 架辅机**

\begin{tikzpicture}[scale=1.5]
    % Draw the dashed lines
    \draw[dashed] (0,0) -- (0,4);
    \draw[dashed] (2,0) -- (2,4);
    \draw[dashed] (4,0) -- (4,4);
    \draw[dashed] (6,0) -- (6,4);
    
    % Draw the labels
    \node at (0,4.5) {A};
    \node at (2,4.5) {B};
    \node at (4,4.5) {C};
    \node at (6,4.5) {D};
    \node at (0,-0.5) {E};
    \node at (2,-0.5) {F};
    
    % Draw the arrows
    \draw[->, thick] (0,3) -- (2,3) node[midway, above] {$1/4$};
    \draw[->, thick] (0,2) -- (4,2) node[midway, above] {$1/4$};
    \draw[->, thick] (0,1) -- (6,1) node[midway, above] {$5/12$};
    \draw[->, thick] (0,0) -- (2,0) node[midway, above] {$1/3$};
    
    % Draw the labels for the planes
    \node at (-0.5,3) {辅机 1};
    \node at (-0.5,2) {辅机 2};
    \node at (-0.5,1) {主机};
    \node at (-0.5,0) {辅机 3};
\end{tikzpicture}

\( n=3 \) 时,只有两架送的辅机,设为辅机 1 和辅机 2。则辅机 1 在耗油 \( L/4 \),即 B 处给辅机 2 和主机加满,辅机 2 继续向前耗油 \( L/4 \) 到达 C 给主机加满返回。下面只有一架辅机 3 到 F 处去接主机,过程和一架辅机去送主机类似,在耗油 \( L/3 \),即 F 处。那么主机只能继续向前耗油 \( 5L/12 \) 到达 D 处,整个过程如上图所示。

示,主机向前行驶 $\frac{L}{4} + \frac{L}{4} + \frac{5L}{12} = \frac{11L}{12}$。

\section*{4 架辅机}

\begin{figure}[h]
    \centering
    \includegraphics[width=\textwidth]{image.png}
    \caption{4 架辅机示意图}
\end{figure}

$n=4$ 时,有两架送的辅机,同样设为辅机 1 和辅机 2,还有两架接的辅机,设为辅机 3 和辅机 4。辅机 1 和辅机 2 送的过程和 $n=3$ 时类似。接的过程和送的是对称的。则辅机 1 在耗油 $L/4$,即 B 处给辅机 2 和主机加满,辅机 2 继续向前耗油 $L/4$ 在 C 处给主机加满返回。主机向前行使 $L/2$ 再回头把油量消耗完毕。那么用辅机 3 在耗油 $L/2$ 即 G 处去接主机,带回主机耗油 $L/4$ 行驶到 F 处,再用辅机 4 在耗油 F 处去接辅机 4 和主机。整个过程如上图所示主机向前行驶 $\frac{L}{4} + \frac{L}{4} + \frac{L}{2} = L$。

\textbf{命题 1.12} 根据过程的对称性即上面的分析,对于 $n=1, 2, 3, 4$,我们有
\begin{equation}
\begin{cases}
r_n = \frac{n}{n+2} + \frac{1}{2}, n=2k, k=1,2 \\
r_n = \frac{1}{2} \left[ \frac{2k}{2k+2} + \frac{(2k+1)}{(2k+1)+2} + 1 \right] = \frac{1}{2} (r_{2k} + r_{2k+2}), n=2k+1, k=0,1
\end{cases}
\end{equation}

下面我们给出 $n=5, 6, 7, 8$ 的方案,如下图所示:

\section*{5 架辅机}

\begin{figure}[h]
    \centering
    \includegraphics[width=\textwidth]{image2.png}
    \caption{5 架辅机示意图}
\end{figure}

\begin{tikzpicture}[scale=0.8]
    % Draw the dashed lines
    \draw[dashed] (0,0) -- (0,10);
    \draw[dashed] (4,0) -- (4,10);
    \draw[dashed] (6,0) -- (6,10);
    \draw[dashed] (8,0) -- (8,10);
    \draw[dashed] (16,0) -- (16,10);
    
    % Draw the labels for the dashed lines
    \node at (0,10.5) {A};
    \node at (4,10.5) {B};
    \node at (6,10.5) {C};
    \node at (8,10.5) {D};
    \node at (16,10.5) {E};
    
    % Draw the labels for the dashed lines at the bottom
    \node at (0,-0.5) {F};
    \node at (4,-0.5) {G};
    \node at (8,-0.5) {H};
    
    % Draw the arrows and labels for the auxiliaries
    \draw[->, thick] (0,9) -- (4,9) node[midway, above] {$1/4$};
    \node at (-1,9) {辅机1};
    
    \draw[->, thick] (0,8) -- (4,8) node[midway, above] {$1/12$};
    \node at (-1,8) {辅机2};
    
    \draw[->, thick] (0,7) -- (6,7) node[midway, above] {$5/18$};
    \node at (-1,7) {辅机3};
    
    \draw[->, thick] (0,6) -- (8,6) node[midway, above] {$1/3$};
    \node at (-1,6) {主机};
    
    \draw[->, thick] (0,5) -- (8,5) node[midway, above] {$1/4$};
    \node at (-1,5) {辅机4};
    
    \draw[->, thick] (0,4) -- (4,4) node[midway, above] {$1/4$};
    \node at (-1,4) {辅机5};
    
    % Draw the arrow for the main machine
    \draw[->, thick] (0,6) -- (16,6);
    
    % Draw the labels for the fractions
    \node at (16,9) {$8/18$};
    \node at (4,3) {$1/4$};
    \node at (8,3) {$1/4$};
\end{tikzpicture}

6 架辅机

\begin{tikzpicture}[scale=0.8]
    % Draw the horizontal lines
    \foreach \y in {0,1,...,6} {
        \draw[thick] (0,\y) -- (15,\y);
    }
    
    % Draw the vertical dotted lines
    \foreach \x in {0,3,5,8,10,15} {
        \draw[dotted] (\x,0) -- (\x,6);
    }
    
    % Draw the labels
    \node at (0,6.5) {A};
    \node at (3,6.5) {B};
    \node at (5,6.5) {C};
    \node at (8,6.5) {D};
    \node at (10,6.5) {E};
    \node at (15,6.5) {F};
    \node at (3,-0.5) {G};
    \node at (8,-0.5) {H};
    
    % Draw the fractions
    \node at (1.5,5.5) {$1/4$};
    \node at (4,4.5) {$1/12$};
    \node at (6.5,3.5) {$5/18$};
    \node at (12.5,1.5) {$1/2$};
    \node at (5.5,0.5) {$1/3$};
    \node at (1.5,-1.5) {$1/4$};
    
    % Draw the labels for each line
    \node at (-1,0) {辅机6};
    \node at (-1,1) {辅机5};
    \node at (-1,2) {辅机4};
    \node at (-1,3) {主机};
    \node at (-1,4) {辅机3};
    \node at (-1,5) {辅机2};
    \node at (-1,6) {辅机1};
    
    % Draw the arrows
    \draw[->, thick] (0,0) -- (3,0);
    \draw[->, thick] (0,1) -- (5,1);
    \draw[->, thick] (0,2) -- (8,2);
    \draw[->, thick] (0,3) -- (15,3);
    \draw[->, thick] (0,4) -- (8,4);
    \draw[->, thick] (0,5) -- (5,5);
    \draw[->, thick] (0,6) -- (3,6);
\end{tikzpicture}

7 架辅机

\begin{tikzpicture}[scale=0.8]
    % Draw the horizontal lines
    \foreach \y in {0,1,...,7} {
        \draw[thick] (0,\y) -- (15,\y);
    }
    
    % Draw the vertical dotted lines
    \foreach \x in {0,3,5,8,10,12,15} {
        \draw[dotted] (\x,0) -- (\x,7);
    }
    
    % Draw the labels
    \node at (0,7.5) {A};
    \node at (3,7.5) {B};
    \node at (5,7.5) {C};
    \node at (8,7.5) {D};
    \node at (10,7.5) {E};
    \node at (12,7.5) {F};
    \node at (15,7.5) {G};
    \node at (17.5,7.5) {H};
    \node at (20,7.5) {I};
    \node at (22.5,7.5) {J};
    
    % Draw the fractions
    \node at (1.5,7) {$1/5$};
    \node at (4,7) {$1/5$};
    \node at (6.5,7) {$14/45$};
    \node at (9,7) {$4/9$};
    \node at (13.5,7) {$1/3$};
    \node at (16,7) {$1/4$};
    \node at (18.5,7) {$1/12$};
    \node at (21,7) {$5/8$};
    
    % Draw the arrows
    \foreach \y in {0,1,...,7} {
        \draw[->, thick] (0,\y) -- (15,\y);
    }
    
    % Draw the labels for each line
    \node at (-1,0) {辅机 7};
    \node at (-1,1) {辅机 6};
    \node at (-1,2) {辅机 5};
    \node at (-1,3) {主机};
    \node at (-1,4) {辅机 4};
    \node at (-1,5) {辅机 3};
    \node at (-1,6) {辅机 2};
    \node at (-1,7) {辅机 1};
\end{tikzpicture}

8 架辅机

\begin{tikzpicture}[scale=0.8]
    % Draw the horizontal lines
    \foreach \y in {0,1,...,8} {
        \draw[thick] (0,\y) -- (15,\y);
    }
    
    % Draw the vertical dotted lines
    \foreach \x in {0,3,6,9,12,15} {
        \draw[dotted] (\x,0) -- (\x,8);
    }
    
    % Draw the arrows
    \draw[->, thick] (0,7) -- (3,7);
    \draw[->, thick] (0,6) -- (3,6);
    \draw[->, thick] (0,5) -- (3,5);
    \draw[->, thick] (0,4) -- (3,4);
    \draw[->, thick] (0,3) -- (3,3);
    \draw[->, thick] (0,2) -- (3,2);
    \draw[->, thick] (0,1) -- (3,1);
    \draw[->, thick] (0,0) -- (3,0);
    
    \draw[->, thick] (3,7) -- (6,7);
    \draw[->, thick] (3,6) -- (6,6);
    \draw[->, thick] (3,5) -- (6,5);
    \draw[->, thick] (3,4) -- (6,4);
    \draw[->, thick] (3,3) -- (6,3);
    \draw[->, thick] (3,2) -- (6,2);
    \draw[->, thick] (3,1) -- (6,1);
    \draw[->, thick] (3,0) -- (6,0);
    
    \draw[->, thick] (6,7) -- (9,7);
    \draw[->, thick] (6,6) -- (9,6);
    \draw[->, thick] (6,5) -- (9,5);
    \draw[->, thick] (6,4) -- (9,4);
    \draw[->, thick] (6,3) -- (9,3);
    \draw[->, thick] (6,2) -- (9,2);
    \draw[->, thick] (6,1) -- (9,1);
    \draw[->, thick] (6,0) -- (9,0);
    
    \draw[->, thick] (9,7) -- (12,7);
    \draw[->, thick] (9,6) -- (12,6);
    \draw[->, thick] (9,5) -- (12,5);
    \draw[->, thick] (9,4) -- (12,4);
    \draw[->, thick] (9,3) -- (12,3);
    \draw[->, thick] (9,2) -- (12,2);
    \draw[->, thick] (9,1) -- (12,1);
    \draw[->, thick] (9,0) -- (12,0);
    
    \draw[->, thick] (12,7) -- (15,7);
    \draw[->, thick] (12,6) -- (15,6);
    \draw[->, thick] (12,5) -- (15,5);
    \draw[->, thick] (12,4) -- (15,4);
    \draw[->, thick] (12,3) -- (15,3);
    \draw[->, thick] (12,2) -- (15,2);
    \draw[->, thick] (12,1) -- (15,1);
    \draw[->, thick] (12,0) -- (15,0);
    
    % Add labels
    \node at (1.5,7.5) {1/5};
    \node at (4.5,7.5) {1/5};
    \node at (7.5,7.5) {14/45};
    \node at (10.5,7.5) {1/2};
    
    \node at (1.5,-0.5) {1/5};
    \node at (4.5,-0.5) {1/5};
    \node at (7.5,-0.5) {14/45};
    
    \node at (-0.5,7) {辅机 1};
    \node at (-0.5,6) {辅机 2};
    \node at (-0.5,5) {辅机 3};
    \node at (-0.5,4) {辅机 4};
    \node at (-0.5,3) {主机};
    \node at (-0.5,2) {辅机 5};
    \node at (-0.5,1) {辅机 6};
    \node at (-0.5,0) {辅机 7};
    \node at (-0.5,-1) {辅机 8};
    
    \node at (1.5,8.5) {A};
    \node at (4.5,8.5) {B};
    \node at (7.5,8.5) {C};
    \node at (10.5,8.5) {D};
    \node at (13.5,8.5) {E};
    
    \node at (1.5,-1.5) {E};
    \node at (4.5,-1.5) {F};
    \node at (7.5,-1.5) {G};
    \node at (10.5,-1.5) {H};
\end{tikzpicture}

\textbf{问题 2} 在问题 1 的假设下,当 $n > 4$ 时,尽你的可能求出 $r_n$ (提示:先假设辅机可以分为两类,第一类专为主机前进服务,第二类专为主机返回服务,再考虑一般情形),或给出 $r_n$ 的上、下界;讨论当 $n \to \infty$ 的过程中 $r_n$ 与 $n$ 的渐近关系;试给出判断最优作战方案(主机能够飞到 $r_n$ 处)的必要条件或充分条件。

\textbf{解答:} 注:这里我们采用启发式集群算法,设计如下:

(1) 集群算法,当 $N$ 很大时,可将每 $n$ 架看成一个单元,每 $n$ 个单元再看成一个大的单元,一层一层以此类推。

(2) 当 $n=1$ 时,以 $n+1=2$ 为基数进行集群是不可能的。

集成方案如下图所示,我们给出了 $n=3$ 和 $n=5$ 的情况。$n=3$ 时我们给出了示例,共有两层,最后只有一个主机在前面。

组合原子:

\begin{figure}[h]
    \centering
    \begin{tikzpicture}[scale=0.8]
        % 辅机 1
        \draw[dashed] (0,0) -- (0,1);
        \draw[->] (0,0.5) -- (3,0.5);
        \node at (-0.5,0.5) {辅机 1};
        
        % 次主机
        \draw[dashed] (0,-1) -- (0,-2);
        \draw[->] (0,-1.5) -- (3,-1.5);
        \node at (-0.5,-1.5) {次主机};
        
        % 辅机 3
        \draw[dashed] (0,-2) -- (0,-3);
        \draw[->] (0,-2.5) -- (3,-2.5);
        \node at (-0.5,-2.5) {辅机 3};
        
        % 满油点
        \draw[dashed] (3,0) -- (3,-3);
        \node at (3.5,-1.5) {满油点};
        
        % 标记 A 和 B
        \node at (0,1.2) {A};
        \node at (3,1.2) {B};
    \end{tikzpicture}
    \caption{以 3 为基数的集群}
    \label{fig:cluster_3}
\end{figure}

\begin{figure}[h]
    \centering
    \begin{tikzpicture}[scale=0.8]
        % 辅机 1
        \draw[dashed] (0,0) -- (0,1);
        \draw[->] (0,0.5) -- (3,0.5);
        \node at (-0.5,0.5) {辅机 1};
        
        % 辅机 2
        \draw[dashed] (0,-1) -- (0,-2);
        \draw[->] (0,-1.5) -- (3,-1.5);
        \node at (-0.5,-1.5) {辅机 2};
        \node at (-0.5,-1.8) {(次主机)};
        
        % 辅机 3
        \draw[dashed] (0,-2) -- (0,-3);
        \draw[->] (0,-2.5) -- (3,-2.5);
        \node at (-0.5,-2.5) {辅机 3};
        
        % 辅机 4
        \draw[dashed] (0,-3) -- (0,-4);
        \draw[->] (0,-3.5) -- (3,-3.5);
        \node at (-0.5,-3.5) {辅机 4};
        
        % 主机
        \draw[dashed] (0,-4) -- (0,-5);
        \draw[->] (0,-4.5) -- (6,-4.5);
        \node at (-0.5,-4.5) {主机};
        
        % 辅机 5
        \draw[dashed] (0,-5) -- (0,-6);
        \draw[->] (0,-5.5) -- (3,-5.5);
        \node at (-0.5,-5.5) {辅机 5};
        
        % 辅机 6
        \draw[dashed] (0,-6) -- (0,-7);
        \draw[->] (0,-6.5) -- (3,-6.5);
        \node at (-0.5,-6.5) {辅机 6};
        
        % 辅机 7
        \draw[dashed] (0,-7) -- (0,-8);
        \draw[->] (0,-7.5) -- (3,-7.5);
        \node at (-0.5,-7.5) {辅机 7};
        \node at (-0.5,-7.8) {(次主机)};
        
        % 辅机 8
        \draw[dashed] (0,-8) -- (0,-9);
        \draw[->] (0,-8.5) -- (3,-8.5);
        \node at (-0.5,-8.5) {辅机 8};
        
        % 标记 A、B、C、D
        \node at (0,1.2) {A};
        \node at (3,1.2) {B};
        \node at (6,1.2) {C};
        \node at (9,1.2) {D};
        
        % 标记 1/3 和 1/2
        \node at (1.5,0.8) {$1/3$};
        \node at (4.5,0.8) {$1/3$};
        \node at (7.5,0.8) {$1/2$};
    \end{tikzpicture}
    \caption{以 3 为基数的集群}
    \label{fig:cluster_3_2}
\end{figure}

\begin{figure}[h]
    \centering
    \includegraphics[width=\textwidth]{image.png}
    \caption{以 5 为基数的集群}
\end{figure}

下表列出了 $n=1,\dots,8$ 的 $r_{n}^{*}$:

\begin{table}[h]
    \centering
    \begin{tabular}{|c|c|c|c|c|}
        \hline
        第 1 列 & 第 2 列 & 第 3 列 & 第 4 列 & 第 5 列 \\
        \hline
        $n$ & $n=2k$ 或 $2k-1$; $k/k$ 或 $k/k-1$ & $r_{n}^{*}$ (单位 L) & $a_{n}$ (单位 L) & $\frac{a_{n}}{\ln(n+1)}$ \\
        \hline
        \multirow{2}{*}{0} & 0 & \multirow{2}{*}{$1/2$} & \multirow{2}{*}{0} & \multirow{2}{*}{---} \\
        \cline{2-2}
        & 0 & & & \\
        \hline
        \multirow{2}{*}{1} & 1 & \multirow{2}{*}{$2/3$} & \multirow{2}{*}{0} & \multirow{2}{*}{---} \\
        \cline{2-2}
        & 0 & & & \\
        \hline
    \end{tabular}
    \caption{表 1}
\end{table}

\begin{table}
\centering
\begin{tabular}{|c|c|c|c|c|}
\hline
2 & 1 & 5/6 & 1/3 & 0.3034 \\ \cline{2-2}
 & 1 & & & \\ \hline
3 & 2 & 11/12 & 1/3 & \\ \cline{2-2}
 & 1 & & & \\ \hline
4 & 2 & 1 & 1/2 & 0.3107 \\ \cline{2-2}
 & 2 & & & \\ \hline
5 & 3 & 19/18 & 1/2 & \\ \cline{2-2}
 & 2 & & & \\ \hline
6 & 3 & 10/9 & 11/18 & 0.3140 \\ \cline{2-2}
 & 3 & & & \\ \hline
7 & 4 & 209/180 & 11/18 & \\ \cline{2-2}
 & 3 & & & \\ \hline
8 & 4 & 109/90 & 32/45 & 0.3236 \\ \cline{2-2}
 & 4 & & & \\ \hline
\end{tabular}
\caption{表一}
\end{table}

注:——表示集群时,不会以 2,4,6 为基数。

\textbf{命题 2.1} 存在一个可行方案 A,s.t. $r_{N}^{*}$ 的下界为 $r_{N}^{*}=a_{n}\log_{n+1}(N+1)+\frac{1}{2}$,单位为 L,其中,N 为辅机总架数,$1<n=2m<N$,m 为整数,(n+1) 为集群算法的基数,$a_{n}$ 为当 n 固定时每一单元前进的最大距离。

\textbf{证明:} 将 N+1 架总飞机以 (n+1) 为基数集群,则 $(n+1)^{k}=N+1$,其中 k 为层数,且在同一层的起点处每一单元的油量的总是满的,则当 n 固定时,$a_{n}$ 保持不变,主机在最后一层的油量是满的,所以可以飞行 1/2,

所以此方案的 $r_{N}^{*}$ 为
\begin{equation}
a_{n} * k + \frac{1}{2} = a_{n} * \log_{(n+1)}(N+1) + \frac{1}{2}
\end{equation}

而最优解优于此结果,即 $r_{N} \geq r_{N}^{*}$ 所以此结果为最优解的一个下界,问题得证。

\textbf{命题 2.2} 命题 2.1 中只能以奇数为基数,即 n=2m。

\textbf{证明:} 不妨设 n=2 和 3,由图 1 知,以 3 为基数进行集群,在 B 处辅机 2,主机和辅机 7 的状态同辅机 1,辅机 2 和辅机 3 在 A 处的状态,所以可以以 3 为基数进行集群;而由图 2 知,在 B, F, C 的任一处四架次主机的状态都不同于起始处 A 的状态,所以如果以 4 为基数进行集群,必然会造成油的大量浪费,则此方案不是一个较好的方案。命题得证。

\begin{figure}[h]
    \centering
    \includegraphics[width=0.8\textwidth]{image.png}
    \caption{}
    \label{fig:2}
\end{figure}

定理 1. $n$ 固定时,当 $N \to \infty$ 时,$\boldsymbol{r}_{N}^{*} \to \infty$,所以 $\boldsymbol{r}_{N} \to \infty, N \to \infty$。

证明:由命题 2.1 中的关系式即得。

命题 2.3 $\boldsymbol{r}_{n}$ 达到最大的必要条件为:

(a) 主机与辅机的航线不可能出现折线,即命题 1.1;

(b) 负责送的辅机必须与主机同时出发,即命题 1.2;

(c) 辅机和主机回到基地 A 的时候油量正好消耗完毕,即命题 1.7。

(d) 辅机可以分为两类,第一类专为主机前进服务,第二类专为主机返回服务,且第一类与第二类的架数的差趋向于零,即,趋向对称。

(i) 当 $n=2k$ 时,第一类和第二类的架数分别为 $k$。

(ii) 当 $n=2k-1$ 时,第一类和第二类的架数分别为 $k-1, k$ 或 $k, k-1$,且两者方案等价。即命题 1.4

(f) 在每一类中,对称的个数越多,$\boldsymbol{r}_{n}$ 越大。

命题 2.4 $\boldsymbol{r}_{N}$ 达到最大的充分条件为 $N=2^{n}-1$,其中 2 为采用集群算法以 2 为基数,$n$ 为某个参数。

证明:由命题 2.1,$\boldsymbol{r}_{N}=a_{1} \log _{2}(N+1)+\frac{1}{2}$,得 $2^{n}-1=N$,其中 $n=\left(\boldsymbol{r}_{\mathrm{N}}-\frac{1}{2}\right) / a_{1}$,然而,采用集群算法以 2 为基数进行集群是不可能的,所以 $N=2^{n}-1$ 是一个充

分条件。

讨论 $r_{N}^{*}$ 与 $\frac{a_{n}}{\ln(n+1)}$ 的性质:

性质 1. (1) 当基底 $n$ 固定时, $r_{N}^{*}$ 为 $N$ 的单增函数;

(2) 当辅机总数 $N$ 固定时, $r_{N}^{*}$ 随着 $n$ 的增大, 变化规律如表一第 5 列所示。

$a_{n}$ 的改进:

在我们的可行方案 A 中, 当 $N$ 很大时, 采用集群算法, 当 $a_{n}, n=2m$ 已知时, 我们可以算出以 $(2m+1)$ 为基数的任意一个 $r_{N}^{*}$; 但当 $n=2m$ 增大时, $a_{n}$ 不容易求出, 所以尽管我们采用了集群算法, 还是有相当数量的 $a_{n}$ 未知。

(1) 我们也可以看到以 $(2+1)$ 和 $(8+1)$ 为基数的 $r_{N}^{*}$ 存在着某种关系: $2 \ln 3=\ln 9$,表一中:

$\frac{a_{2}}{\ln (2+1)}=0.3034<\frac{a_{8}}{\ln (8+1)}=0.3236$, 差别很小, 所以我们可以用 $k a_{2}$ 代替 $a_{3^{k-1}}$, 用 $k a_{m}$ 代替 $a_{(m+1)^{k-1}}$, $m=2,4,6, k=1,2 \cdots \cdots$, $k$ 表示层数。

这样, 我们可以得到相当一部分 $a_{n}$, 尽管得到的 $r_{N}^{*}$ 有一定的影响, 但也是一个可行方案。

\begin{tabular}{|c|c|c|c|c|}
\hline $k a_{m}$ 代替 & \multicolumn{4}{|c|}{$k$} \\
\hline $a_{(m+1)^{k-1}}$ & 1 & 2 & 3 & 4 \\
\hline $k a_{2} \approx a_{3^{k-1}}$ & $a_{2}$ & $a_{8}$ & $a_{26}$ & $a_{80}$ \\
\hline $k a_{4} \approx a_{5^{k-1}}$ & $a_{4}$ & $a_{24}$ & $a_{124}$ & $a_{424}$ \\
\hline $k a_{6} \approx a_{7^{k-1}}$ & $a_{6}$ & $a_{48}$ & $a_{342}$ & \\
\hline $k a_{8} \approx a_{9^{k-1}}$ & $a_{8}$ & $a_{80}$ & $a_{728}$ & \\
\hline
\end{tabular}

\begin{table}
\centering
\begin{tabular}{|c|c|c|c|c|}
\hline
$k\boldsymbol{a}_{10} \approx \boldsymbol{a}_{11^{k-1}}$ & $\boldsymbol{a}_{10}$ & $\boldsymbol{a}_{120}$ & & \\
\hline
\end{tabular}
\caption{表二}
\end{table}

从表二可以看出,$\boldsymbol{a}_8$ 出现了两次,$\boldsymbol{a}_{80}$ 出现了两次,那么我们可以得出:用此方法来改进 $\boldsymbol{a}_n$,会得到几个不同的值,这是有可能的。由性质 1,选取最大值即可。

(2) 由 (1) 我们可以计算出相当一部分 $\boldsymbol{a}_n$,但是由表一和表二,$\boldsymbol{a}_{16}$,$\boldsymbol{a}_{17}$,$\boldsymbol{a}_{18}\dots$ 都未知,可由定理 2 的关系式 $\boldsymbol{r}_{2k-1}^* = \frac{1}{2}(\boldsymbol{r}_{2(k-1)}^* + \boldsymbol{r}_{2k}^*)$,再计算另外一部分 $\boldsymbol{a}_n$。

根据集群算法和 $\boldsymbol{a}_n$ 的改进,我们可以求出尽可能多的 $\boldsymbol{a}_n$,那么我们的方案就是可行的。

\textbf{问题 3}

若每架辅机可以多次上天,辅机从机场上空降落及在地面检修、加油、再起飞到机场上空的时间相当于飞行 $L/12$ 的时间,飞机第一次起飞、转向、在空中加油的耗时仍忽略不计,此时的作战半径记为 $R_n$,讨论与问题 1、问题 2 类似的问题。

\textbf{解答:} 同样我们给出 $n=1,2,3,4$ 的方案:

\textbf{1 架辅机的重复起飞方案}

\begin{figure}[h]
    \centering
    \includegraphics[width=0.8\textwidth]{image1.png}
    \caption{2 架辅机的重复起飞方案}
\end{figure}

\begin{figure}[h]
    \centering
    \includegraphics[width=0.8\textwidth]{image2.png}
    \caption{3 架辅机的重复起飞方案}
\end{figure}

\begin{tikzpicture}
    \draw[dotted] (0,0) -- (10,0);
    \draw[dotted] (0,1) -- (10,1);
    \draw[dotted] (0,2) -- (10,2);
    \draw[dotted] (0,3) -- (10,3);
    \draw[dotted] (0,4) -- (10,4);
    \draw[dotted] (0,5) -- (10,5);
    \draw[dotted] (0,6) -- (10,6);
    \draw[dotted] (0,7) -- (10,7);
    \draw[dotted] (0,8) -- (10,8);
    \draw[dotted] (0,9) -- (10,9);
    \draw[dotted] (0,10) -- (10,10);
    
    \draw[->, thick] (0,0.5) -- (2,0.5);
    \draw[->, thick] (0,1.5) -- (3,1.5);
    \draw[->, thick] (0,2.5) -- (4,2.5);
    \draw[->, thick] (0,3.5) -- (3,3.5);
    \draw[->, thick] (0,4.5) -- (10,4.5);
    \draw[->, thick] (0,5.5) -- (10,5.5);
    \draw[->, thick] (0,6.5) -- (10,6.5);
    \draw[->, thick] (0,7.5) -- (10,7.5);
    \draw[->, thick] (0,8.5) -- (10,8.5);
    \draw[->, thick] (0,9.5) -- (10,9.5);
    
    \node at (-0.5,0.5) {辅机1};
    \node at (-0.5,1.5) {辅机2};
    \node at (-0.5,2.5) {辅机3};
    \node at (-0.5,3.5) {辅机1};
    \node at (-0.5,4.5) {主机};
    \node at (-0.5,5.5) {辅机2};
    \node at (-0.5,6.5) {辅机1};
    \node at (-0.5,7.5) {辅机1};
    \node at (-0.5,8.5) {辅机3};
    
    \node at (1,0.5) {$1/5$};
    \node at (2,1.5) {$1/5$};
    \node at (3,2.5) {$1/3$};
    \node at (2,3.5) {$1/5$};
    \node at (10,4.5) {$14/45$};
    \node at (10,5.5) {$4/9$};
    \node at (10,6.5) {$1/5$};
    \node at (10,7.5) {$1/5$};
    \node at (10,8.5) {$1/5$};
    \node at (10,9.5) {$5/9$};
    
    \node at (0,10.5) {A};
    \node at (2,10.5) {B};
    \node at (3,10.5) {C};
    \node at (10,10.5) {D};
    \node at (10.5,10.5) {E};
    \node at (1, -0.5) {G};
    \node at (2, -0.5) {H};
    \node at (3, -0.5) {I};
    \node at (10, -0.5) {J};
\end{tikzpicture}

下面给出时间上能够接到的理由:

辅机1接辅机3:
\[ 1/5 + 1/12 + 1/3 < 1/5 + 2*14/45 + 1/15 \]

辅机2接主机:
\[ 2/5 + 1/12 + 3/5 < 14/54 + 1 \]

辅机3接主机:
\[ 2/5 + 14/45 + 1/12 + 2/5 < 1 + 1/5 \]

作战半径为:$1/5 + 1/5 + 14/45 + 4/9 = 52/45$

4架辅机的重复起飞方案

\begin{tikzpicture}[scale=0.8]
    % 辅机1
    \draw[thick, ->] (0,0) -- (2,0) node[midway, above] {$1/6$};
    \draw[dashed] (0,0) -- (0,-2);
    \draw[dashed] (2,0) -- (2,-2);
    \node[left] at (0,-1) {辅机1};
    
    % 辅机2
    \draw[thick, ->] (0,-2) -- (2,-2);
    \draw[dashed] (0,-2) -- (0,-4);
    \draw[dashed] (2,-2) -- (2,-4);
    \node[left] at (0,-3) {辅机2};
    
    % 辅机3
    \draw[thick, ->] (0,-4) -- (2,-4);
    \draw[dashed] (0,-4) -- (0,-6);
    \draw[dashed] (2,-4) -- (2,-6);
    \node[left] at (0,-5) {辅机3};
    
    % 辅机4
    \draw[thick, ->] (0,-6) -- (2,-6);
    \draw[dashed] (0,-6) -- (0,-8);
    \draw[dashed] (2,-6) -- (2,-8);
    \node[left] at (0,-7) {辅机4};
    
    % 辅机1
    \draw[thick, ->] (0,-8) -- (2,-8);
    \draw[dashed] (0,-8) -- (0,-10);
    \draw[dashed] (2,-8) -- (2,-10);
    \node[left] at (0,-9) {辅机1};
    
    % 主机
    \draw[thick, ->] (0,-10) -- (10,-10);
    \draw[dashed] (0,-10) -- (0,-12);
    \draw[dashed] (10,-10) -- (10,-12);
    \node[left] at (0,-11) {主机};
    
    % 辅机2
    \draw[thick, ->] (0,-12) -- (2,-12);
    \draw[dashed] (0,-12) -- (0,-14);
    \draw[dashed] (2,-12) -- (2,-14);
    \node[left] at (0,-13) {辅机2};
    
    % 辅机3
    \draw[thick, ->] (0,-14) -- (2,-14);
    \draw[dashed] (0,-14) -- (0,-16);
    \draw[dashed] (2,-14) -- (2,-16);
    \node[left] at (0,-15) {辅机3};
    
    % 辅机1
    \draw[thick, ->] (0,-16) -- (2,-16);
    \draw[dashed] (0,-16) -- (0,-18);
    \draw[dashed] (2,-16) -- (2,-18);
    \node[left] at (0,-17) {辅机1};
    
    % 辅机4
    \draw[thick, ->] (0,-18) -- (2,-18);
    \draw[dashed] (0,-18) -- (0,-20);
    \draw[dashed] (2,-18) -- (2,-20);
    \node[left] at (0,-19) {辅机4};
    
    % 标记
    \node[above] at (1,0) {A};
    \node[above] at (3,0) {B};
    \node[above] at (5,0) {C};
    \node[above] at (7,0) {D};
    \node[above] at (9,0) {E};
    \node[above] at (11,0) {F};
    \node[below] at (1,-20) {G};
    \node[below] at (3,-20) {H};
    \node[below] at (5,-20) {I};
    \node[below] at (7,-20) {J};
    \node[below] at (9,-20) {K};
    
    % 标记箭头
    \draw[thick, ->] (2,0) -- (4,0) node[midway, above] {$1/6$};
    \draw[thick, ->] (4,0) -- (6,0) node[midway, above] {$1/6$};
    \draw[thick, ->] (6,0) -- (8,0) node[midway, above] {$5/18$};
    \draw[thick, ->] (8,0) -- (10,0) node[midway, above] {$4/9$};
    
    \draw[thick, ->] (2,-2) -- (4,-2);
    \draw[thick, ->] (4,-2) -- (6,-2);
    \draw[thick, ->] (6,-2) -- (8,-2);
    \draw[thick, ->] (8,-2) -- (10,-2);
    
    \draw[thick, ->] (2,-4) -- (4,-4);
    \draw[thick, ->] (4,-4) -- (6,-4);
    \draw[thick, ->] (6,-4) -- (8,-4);
    \draw[thick, ->] (8,-4) -- (10,-4);
    
    \draw[thick, ->] (2,-6) -- (4,-6);
    \draw[thick, ->] (4,-6) -- (6,-6);
    \draw[thick, ->] (6,-6) -- (8,-6);
    \draw[thick, ->] (8,-6) -- (10,-6);
    
    \draw[thick, ->] (2,-8) -- (4,-8);
    \draw[thick, ->] (4,-8) -- (6,-8);
    \draw[thick, ->] (6,-8) -- (8,-8);
    \draw[thick, ->] (8,-8) -- (10,-8);
    
    \draw[thick, ->] (2,-10) -- (4,-10);
    \draw[thick, ->] (4,-10) -- (6,-10);
    \draw[thick, ->] (6,-10) -- (8,-10);
    \draw[thick, ->] (8,-10) -- (10,-10);
    
    \draw[thick, ->] (2,-12) -- (4,-12);
    \draw[thick, ->] (4,-12) -- (6,-12);
    \draw[thick, ->] (6,-12) -- (8,-12);
    \draw[thick, ->] (8,-12) -- (10,-12);
    
    \draw[thick, ->] (2,-14) -- (4,-14);
    \draw[thick, ->] (4,-14) -- (6,-14);
    \draw[thick, ->] (6,-14) -- (8,-14);
    \draw[thick, ->] (8,-14) -- (10,-14);
    
    \draw[thick, ->] (2,-16) -- (4,-16);
    \draw[thick, ->] (4,-16) -- (6,-16);
    \draw[thick, ->] (6,-16) -- (8,-16);
    \draw[thick, ->] (8,-16) -- (10,-16);
    
    \draw[thick, ->] (2,-18) -- (4,-18);
    \draw[thick, ->] (4,-18) -- (6,-18);
    \draw[thick, ->] (6,-18) -- (8,-18);
    \draw[thick, ->] (8,-18) -- (10,-18);
\end{tikzpicture}

下面给出时间上能够接到的理由:

辅机1接辅机4:
\[ 1/6 + 1/12 + 1/3 < 1/6 + 1/6 + 2*5/18 \]

辅机2接主机:
\[ 1/6 + 1/6 + 1/12 + 4/6 < 1/6 + 5/18 + 1 \]

辅机3接主机:
\[ 1/6 + 1/6 + 1/6 + 1/12 + 3/6 < 5/18 + 1 + 1/6 \]

辅机1接主机:
\[ 3/6 + 5/18 + 1/12 + 2/6 < 1 + 2/6 \]

作战半径为:\( 1/6 + 1/6 + 1/6 + 5/18 + 4/9 = 11/9 \)

5架辅机的重复起飞方案

\begin{tikzpicture}[scale=0.8]
    % 辅机1到辅机5
    \foreach \i in {1,...,5} {
        \node at (0, -2*\i) {辅机\i};
    }
    % 主机
    \node at (0, -12) {主机};
    % 辅机1到辅机5
    \foreach \i in {1,...,5} {
        \node at (0, -14-2*\i) {辅机\i};
    }
    
    % 横线
    \foreach \i in {1,...,7} {
        \draw[dotted] (\i*2, 0) -- (\i*2, -46);
    }
    
    % 辅机1到辅机5的箭头
    \foreach \i in {1,...,5} {
        \draw[->] (2, -2*\i) -- (4, -2*\i);
        \draw[->] (4, -2*\i) -- (6, -2*\i);
        \draw[->] (6, -2*\i) -- (8, -2*\i);
        \draw[->] (8, -2*\i) -- (10, -2*\i);
        \draw[->] (10, -2*\i) -- (12, -2*\i);
        \draw[->] (12, -2*\i) -- (14, -2*\i);
        \draw[->] (14, -2*\i) -- (16, -2*\i);
    }
    
    % 主机的箭头
    \foreach \i in {1,...,7} {
        \draw[->] (2, -12) -- (16, -12);
    }
    
    % 辅机1到辅机5的箭头
    \foreach \i in {1,...,5} {
        \draw[->] (2, -14-2*\i) -- (4, -14-2*\i);
        \draw[->] (4, -14-2*\i) -- (6, -14-2*\i);
        \draw[->] (6, -14-2*\i) -- (8, -14-2*\i);
        \draw[->] (8, -14-2*\i) -- (10, -14-2*\i);
        \draw[->] (10, -14-2*\i) -- (12, -14-2*\i);
        \draw[->] (12, -14-2*\i) -- (14, -14-2*\i);
        \draw[->] (14, -14-2*\i) -- (16, -14-2*\i);
    }
    
    % 标注
    \node at (1, -1) {A};
    \node at (3, -1) {B};
    \node at (5, -1) {C};
    \node at (7, -1) {D};
    \node at (9, -1) {E};
    \node at (11, -1) {F};
    \node at (13, -1) {G};
    
    \node at (1, -45) {I};
    \node at (3, -45) {J};
    \node at (5, -45) {K};
    \node at (7, -45) {M};
    \node at (9, -45) {P};
    \node at (11, -45) {Q};
    
    % 数值标注
    \node at (3, -3) {$1/7$};
    \node at (5, -3) {$1/7$};
    \node at (7, -3) {$1/7$};
    \node at (9, -3) {$19/84$};
    \node at (11, -3) {$19/84$};
    \node at (13, -3) {$5/12$};
    
    \node at (3, -43) {$1/7$};
    \node at (5, -43) {$1/7$};
    \node at (7, -43) {$1/7$};
    \node at (9, -43) {$1/7$};
    \node at (11, -43) {$1/7$};
    
    \node at (3, -13) {$1/3$};
    
    % 辅助线
    \draw[dotted] (0, -10) -- (16, -10);
    \draw[dotted] (0, -14) -- (16, -14);
\end{tikzpicture}

下面给出时间上能够接到的理由:

辅机1接辅机4:
\[ 1/7 + 1/12 + 1/3 < 1/7 + 1/7 + 2*19/84 + 3/7 - 1/3 \]

辅机2接辅机5:
\[ 1/7 + 1/7 + 1/12 + 1/3 < 1/7 + 2*19/84 + 3/7 - 1/3 \]

辅机3接主机:
\[ 1/7 + 1/7 + 1/7 + 1/12 + 5/7 < 2*19/84 + 1 + 1/7 \]

\begin{table}
\centering
\begin{tabular}{|c|c|c|c|c|c|}
\hline
第1列 & 第2列 & 第3列 & 第4列 & 第5列 & 第6列 \\
\hline
n架辅机 & 辅机编号 & 每架辅机使用次数 (单位为次) & 最远半径 $R_{n}^{*}$ (单位为L) & $b_{n}$ (单位为L) & $\frac{b_{n}}{\ln(n+1)}$ \\
\hline
1 & P[1,1] & 2 & 5/6 & 1/3 & 0.3034 \\
\hline
\multirow{2}{*}{2} & P[2,1] & 2 & \multirow{2}{*}{1} & \multirow{2}{*}{1/2} & \multirow{2}{*}{0.3107} \\
\cline{2-3}
 & P[2,2] & 2 & & & \\
\hline
\multirow{3}{*}{3} & P[3,1] & 3 & \multirow{3}{*}{52/45} & \multirow{3}{*}{32/45} & \multirow{3}{*}{0.5130} \\
\cline{2-3}
 & P[3,2] & 2 & & & \\
\cline{2-3}
 & P[3,3] & 2 & & & \\
\hline
\multirow{4}{*}{4} & P[4,1] & 3 & \multirow{4}{*}{11/9} & \multirow{4}{*}{7/9} & \multirow{4}{*}{0.4833} \\
\cline{2-3}
 & P[4,2] & 2 & & & \\
\cline{2-3}
 & P[4,3] & 2 & & & \\
\cline{2-3}
 & P[4,4] & 2 & & & \\
\hline
\multirow{5}{*}{5} & P[5,1] & 3 & \multirow{5}{*}{109/84} & \multirow{5}{*}{37/42} & \multirow{5}{*}{0.4917} \\
\cline{2-3}
 & P[5,2] & 3 & & & \\
\cline{2-3}
 & P[5,3] & 3 & & & \\
\cline{2-3}
 & P[5,4] & 3 & & & \\
\cline{2-3}
 & P[5,5] & 2 & & & \\
\hline
\end{tabular}
\caption{表三}
\end{table}

命题 3.1. 存在一个可行方案 $B$,s.t. $R_{N}$ 的下界为
\[
R_{N}^{*}=a_{n}\log_{n+1}(N+1)+\frac{1}{2}+b_{n}-a_{n} \text{,单位为L,其中,N为辅机总架数,} b_{n} \text{为在辅}
\]

机可多次使用时,当 $n$ 固定时第一层单元前进的最大距离,$\boldsymbol{R}_{N}^{*}$ 为采用集群算法之后主机的最大半径,$(n+1)$ 为基数,$1 \leqslant n < N$,$n=2, 4, \cdots$。

证明:将 $N+1$ 架总飞机以 $(n+1)$ 为基数集群,则 $(n+1)^{k+1}=N+1$,其中 $k+1$ 为层数,第一层我们采用可重复起飞的方案,后面的层数中我们还是用问题 2 的方案,因为如果直接按照重复起飞的方案集群,在下一层将无法提供 $L$ 的油量供下一层起始。则当 $n$ 固定时,$a_{n}$ 和 $b_{n}$ 保持不变,主机在最后一层的油量是满的,所以可以飞行 $1/2$。后面 $k$ 层用问题 2 的方案,我们有 $(n+1)^{k}=N_{1}+1$,其中 $N_{1}$ 为后面 $k$ 层的飞机数。

\begin{equation}
\begin{aligned}
\boldsymbol{R}_{N}^{*} &= a_{n} * k + \frac{1}{2} + b_{n} = a_{n} * \lfloor \log_{(n+1)}(N+1) - 1 \rfloor + \frac{1}{2} + b_{n} \\
&= a_{n} * \log_{(n+1)}(N+1) + \frac{1}{2} + b_{n} - a_{n}
\end{aligned}
\end{equation}

而最优解优于此结果,即 $R_{N} \geqslant \boldsymbol{R}_{N}^{*}$,所以此结果为最优解的一个下界,问题得证。

讨论 $\boldsymbol{R}_{N}^{*}$ 与 $\frac{b_{n}}{\ln(n+1)}$ 的性质:

性质 2

(1) 当基底 $n$ 固定时,$\boldsymbol{R}_{N}^{*}$ 为 $n$ 的单增函数;

(2) 当辅机总数 $N$ 固定时,$\boldsymbol{R}_{N}^{*}$ 随着 $n$ 的增大,变化规律如表三第 6 列所示。

$b_{n}$ 的改进的讨论同 $a_{n}$ 的改进。

问题 4 若另有 2 个待建的空军基地(或航空母舰)$A_{1}, A_{2}$,有 $n$ 架辅机,主机从基地 $A$ 起飞,向一给定的方向飞行,必须在基地 $A$ 降落,辅机可在任一基地待命,可多次起飞,且可在任一基地降落。其他同问题 3 的假设,讨论 $A_{1}, A_{2}$ 的选址和主机的作战半径 $R_{n}^{*}$。

解答:

\begin{figure}[h]
    \centering
    \includegraphics[width=\textwidth]{image.png}
    \caption{基地 A、B、C 的加油范围示意图}
\end{figure}

设共有 $N$ 架辅机,基地 $A$ 有 $N_A$ 架辅机。在基地 $A$,$N_A$ 架辅机只需要考虑可以将主机送到的最远距离,并且可以保证 $N_A$ 架辅机恰好返回过来。

\textbf{命题 4.1} 辅机不能直接随主机到达基地 $B$。

\textbf{证明:} 因为从基地 $B$ 接回主机和辅机的代价太大,明显接回主机就可以了。我们考虑将主机在 $D$ 处加满,然后又由主机本身向前消耗完 $L$ 到达 $E$ 处。在此处我们只考虑单程,即不要把主机接回来。送完主机后辅机全部回来,等主机最后回来的时候同样过程把它接回来。

\textbf{命题 4.2} 存在一个可行方案 $B$,s.t. $r_A$ 的下界为 $r_A = a_n * \log_{(n+1)} \frac{N_A}{n} + b_n$,单位为 $L$,其中,$N_A$ 为基地 $A$ 处辅机总架数,$b_n$ 为在辅机可多次使用时,当 $n$ 固定时第一层单元前进的最大距离,此处 $r_A$ 为采用集群算法之后主机单程到达 $D$ 的距离(不考虑接回主机),$(n+1)$ 为基数,$1 \leq n < N$,$n = 2, 4, \dots$(这里我们只考虑将主机加满送到 $D$ 处)。

\textbf{证明:} 将 $N+1$ 架总飞机以 $(n+1)$ 为基数集群,设有 $k+1$ 层。由于考虑单程,所以后面 $k$ 层只需要一半就可以了。我们有 $(n+1)^k = \frac{N_1}{2} + 1$,其中 $N_1$ 为后面 $k$ 层的飞机数。$N_A = (n+1)^{k+1} - 1 - \frac{N_1}{2} = (n+1)^{k+1} - (n+1)^k = n(n+1)^k$。

同样第一层我们采用可重复起飞的方案,后面的 $k$ 层中我们还是用问题 2 的方案。则当 $n$ 固定时,$a_n$ 和 $b_n$ 保持不变,主机在最后一层的油量是满的,所以可

以飞行 $1/2$。

所以此方案的为 $r_{A}=a_{n} * k + b_{n} = a_{n} * \log_{(n+1)} \frac{N_{A}}{n} + b_{n}$。

问题得证。

对于基地 B,同样考虑用 $N_{A}$ 加辅机将主机从 E 处接回。不和主机一起回来的可以重复利用。在后面的 k 层中每一块 (n+1) 中可重复利用比主机早回来的比例为 $\frac{n/2}{n+1}$ (n=2,4,…)。符号同上面,设 $N_{1}$ 为后面 k 层的飞机数,则有 $\frac{n/2}{n+1} N_{1}$ 架飞机可以早点回基地 B 再次利用。$(n+1)^{k} = \frac{N_{1}}{2} + 1$,$N_{1} = 2(n+1)^{k} - 2 = 2 \frac{N_{A}}{n} - 2$。则早些回基地 B 可以重复利用的大约为

$\Delta N_{B} = \frac{N_{A}}{n+1}$。此时 B 处应剩余 $N_{A} - \Delta N_{B} = N_{A} - \frac{N_{A}}{n+1} = \frac{n}{n+1} N_{A}$。所以在 B 处的辅机总数为 $N_{B} = 2N_{A} - \Delta N_{B} = \frac{2n+1}{n+1} N_{A}$。

对于基地 C,同样我需要 $N_{A}$ 架辅机去接从 B 处开来的主机。和上面一样回收利用早点回来的辅机为 $\Delta N_{B}$,和 B 处辅机量一样,所以在 C 处的辅机总数为

$N_{C} = 2N_{A} - \Delta N_{B} = \frac{2n+1}{n+1} N_{A}$。

综上可知,在 A,B,C 处的辅机分配为 $N_{A}$,$N_{B} = N_{C} = \frac{2n+1}{n+1} N_{A}$。设共有 N 架辅机,则在 A,B,C 处的辅机分配为 $N_{A} = \frac{n+1}{5n+3} N$,$N_{B} = \frac{2n+1}{5n+3} N$,$N_{C} = \frac{2n+1}{5n+3} N$。

基地 A 和 B 之间的距离为

$d_{AB} = 2r_{A} + 1 = 2a_{n} * \log_{(n+1)} \frac{N_{A}}{n} + 2b_{n} + 1 = 2a_{n} * \log_{(n+1)} \frac{(n+1)N}{n(5n+3)} + 2b_{n} + 1$,基地 B 和 C 之间的距离也为 $d_{BC} = 2a_{n} * \log_{(n+1)} \frac{(n+1)N}{n(5n+3)} + 2b_{n} + 1$。这时主机的最大作战半径为

\begin{align*}
R_{N}^{*} &= d_{AB} + d_{BC} + R_{N_{C}} = 4a_{n} * \log_{(n+1)} \frac{(n+1)N}{n(5n+3)} + 4b_{n} + 2 + a_{n} * \log_{(n+1)} \frac{(n+1)N}{5n+3} + \frac{1}{2} + b_{n} - a_{n} \\
&= 5a_{n} * \log_{(n+1)} \frac{(n+1)N}{5n+3} + 5b_{n} + \frac{5}{2} - a_{n}(4\log_{n+1} n + 1).
\end{align*}

问题 5 设 $ABCD$ 为矩形,$AB = 4L$,$AD = 2L$,$A, B, D$ 为三个空军基地,主机从 $A$ 起飞,到 $C$ 执行任务(执行任务时间仍忽略不计)再返回 $A$。假设辅机起飞、降落的基地可任意选择,其他同问题 3 的假设,试按最快到达并返回和最少辅机架数两种情况给出你的作战方案。

解答: 1) 最快到达并返回

首先我们考虑最快到达的情况

\begin{figure}[h]
\centering
\includegraphics[width=\textwidth]{image.png}
\caption{矩形 $ABCD$ 和相关点}
\end{figure}

首先我们考虑最快到达并返回的情况,由于最后随着飞机架数的增加最大半径的增长速度越来越慢,所以我们尽可能使最大半径最小。由于最快到达并返回,主机的速度一定,当然沿着 $AC$ 这条直线最好。开始离 $A$ 点较近,当然主机的前进由 $A$ 提供比较合适。主机加满可以往前行驶 $L$,最后我让主机行驶 $L$ 代价较小。因此我们让基地 $A$ 的辅机给主机加满到 $E$ 处,而基地 $D$ 的辅机在主机行驶到没油的 $F$ 处给主机加油。所以 $d_{AE} = d_{DF}$,$d_{EF} = 1$,这里我们以耗油量 $L$ 行驶的距离为单位 1。设 $d_{AE} = d_{DF} = x$,则有 $d_{EO} = \sqrt{5} - x$,

\begin{align*}
d_{OF} &= 1 - d_{EO} = x + 1 - \sqrt{5}, \\
d_{OH} &= \frac{3}{5}\sqrt{5}, \\
d_{FH} &= \frac{3}{5}\sqrt{5} - d_{FO} = \frac{8}{5}\sqrt{5} - x - 1, \\
d_{DH} &= \frac{4}{5}\sqrt{5},
\end{align*}

我们有 $d_{DH}^{2} + d_{FH}^{2} = d_{DF}^{2}$,我们得

\begin{equation}
\left(\frac{8}{5}\sqrt{5}-x-1\right)^2 + \left(\frac{4}{5}\sqrt{5}\right)^2 = x^2, \text{ 解得 } x = \frac{\frac{8}{\sqrt{5}}-1)^2 + \frac{16}{5}}{2\left(\frac{8}{\sqrt{5}}-1\right)} \approx 1.9096.
\end{equation}

基地 B 离这条线太远,不起作用。由于在 H 处离基地 D 最近,所以代价小,我们考虑在 H 处让主机加满。$d_{OF} \approx 0.6735$, $d_{FH} \approx 0.6681$, $d_{HC} \approx 0.8944$。因此我们考虑在 F 处让主机加油使得主机可以行驶 $d_{FH}$ 的距离,在 H 处让主机加满。由于到达 C 点后要返回,而 $d_{HC}$ 的距离较大,不能直接从 C 返回行驶到 H,那么在 HC 之间必须选取加油点 I。当然 I 选取得要离 H 充分性。经选择可知 $d_{IC} = \frac{1}{2}$, $d_{DI} \approx 1.8318$ 时,最为合适。如果我们在未到 I 处给辅机加油,则到达 C 处再次返回到没油时加油点必须设置在比 I 离 C 更近的点,这样造成了浪费。这样我们把所有最佳的加油点已经选取好了,下面就要在每个基地选取合适的辅机数量。

由前面的讨论可知,从 A 送单程送主机(主机不返回)到 E 处让主机加满油所需的辅机量为 $N_A$,则
\begin{equation}
d_{AE} = a_n \log_{(n+1)} \frac{N_A}{n} + b_n, \quad N_A = n*(n+1)^{\frac{d_{AE} - b_n}{a_n}}.
\end{equation}

如果我们取 $n=4$,也就是说以 5 为集群中一块的辅机数量。$a_4 = \frac{1}{2}$, $b_4 = \frac{7}{9}$。则
\begin{equation}
N_A = 4*5^{2\left(\frac{d_{AE} - 7}{9}\right)} \approx 153,
\end{equation}
就是我们用 153 架辅机去送主机到 E 处让主机加满。下面我们考虑用 $N_{DF}$ 架辅机去 F 处给主机加油,同样我们需要 153 架主机,即 $N_{DF} = 153$。接着考虑到 H 处给主机加满所需的辅机数
\begin{equation}
N_{DH} = 4*5^{2\left(\frac{d_{DH} - 7}{9}\right)} \approx 104.
\end{equation}
在 I 处加油需要 $N_{DI} = 4*5^{2\left(\frac{d_{DI} - 7}{9}\right)} \approx 120$,到达 C 处再回到 I 处油量消耗完毕,这时在 F 处给主机加油的 153 架辅机已经回到基地,并且可以到达 I 处给主机再次加油。因此,在 D 点所需要的辅机数一共为
\begin{equation}
N_D = N_{DF} + N_{DH} + N_{DI} \approx 153 + 104 + 120 = 377.
\end{equation}

最后我们来考虑

\section*{2) 最少辅机架数}

我们需要考虑从 A 到 C 的最少辅机架数。由于最后随着飞机架数的增加最大半径的增长速度越来越慢,所以我们尽可能使最大半径最小。

\begin{figure}[h]
    \centering
    \includegraphics[width=\textwidth]{image.png}
    \caption{示意图}
\end{figure}

我们对于 A,B,D 为圆心做同样半径的圆,当圆逐渐增大的时候,我们需要在这三个圆内找一条连接 A 和 C 的线。这样使最大半径最小。可以发现当半径为 \( d_{AE} = 2L \) 时,我们可以通过 \( A \to D \to C \) 从 A 到达 C 处。下面同样先考虑在 A 处有 \( N_A \) 架辅机将主机送到 F 处并让其加满,\( d_{AF} = \frac{3}{2}L \) 同样我们取 \( n = 4 \),根据前面的结果所需辅机架数为
\[
N_A = n * (n + 1) \left( \frac{d_{AF} - b_n}{a_n} \right)^n,
\]
则
\[
N_A = 4 * 5^{2 \left( \frac{d_{AF} - 7}{9} \right)} \approx 41.
\]
我们考虑对称,用 \( N_A \) 架辅机去接把主机正好接到基地 D。同样基于问题 4 的讨论,回基地 D 可以重复利用的大约为
\[
\Delta N_D = \frac{N_A}{n + 1},
\]
下面要考虑将主机从 D 送到 C 并可以返回所需的辅机数 \( N_{DC} \)。\( d_{DC} = 2L \),把主机从 D 送到 C 并可以返回所需的辅机数 \( N_{DC} \) 满足问题 3 中研究过的情况
\[
d_{DC} = a_n * \log_{(n+1)} (N_{DC} + 1) + \frac{1}{2} + b_n - a_n.
\]

\begin{equation}
N_{DC} = (n+1)^{\frac{d_{DC} + \frac{1}{2} + a_n - b_n}{a_n}} - 1,
\end{equation}
取 \( n = 4 \),我们有 \( N_{DC} = 5^{\frac{2(d_{DC} - 7)}{9}} - 1 \approx 51 \)。最后返回,一部分早些返回的和开始主机从 G 一起返回停在 D 处的辅机一起在 D 处送主机。最终让 D 回到 A。整个过程需要基地 A 的辅机数 \( N_A \approx 41 \),需要基地 B 的辅机数 \( N_B = N_A - \frac{1}{n+1} N_A + N_{DC} \approx 84 \) 架。

\section*{参考文献}
[1] 洪冠新,金长江.空中加油调度的研究,飞行力学,1997,3。
[2] 魏权龄等.运筹学通论.北京:中国人民大学出版社,2001。
[3] 孙金标,施克如,王克格.空中加油的最优化研究,飞行力学,2000,12。
[4] 姜启源.数学模型.北京:高等教育出版社,2001。