\documentclass{article}
\usepackage{amsmath}
\usepackage{amssymb}

\title{微蜂窝环境中无线接收信号的特性分析}
\author{}
\date{}

\begin{document}

\maketitle

\begin{abstract}
为研究微蜂窝环境中无线接收信号的特性分析问题,本文综合运用图论法、数值计算方法、统计学原理和信号处理技术,建立了信号传播预测模型,并基于此进行了仿真计算及统计特性分析,为现实中指导网络规划提供了较为准确的理论依据。

针对问题一:首先定义建筑物的各个拐角和墙面元素,每个元素作为一个节点,然后对点与点、点与边、边与边的可见性进行了定义,建立连通图。通过搜索算法初步找到起始点之间的所有路径,然后建立视距传播模型与镜像反射传播模型作为判据的,选出能够成功实现信号传播的路径,得到由发射机 \( \text{Tx}(500, 200) \) 和接收机 \( \text{Rx}(250, 350) \) 的主要传播路径有 74 条。

针对问题二:借助问题一建立的模型,对 25 种收发机组合方式进行了仿真计算,将主要路径分别可视化展示在二维建筑平面图中。得出了传播路径最多的组合方式为 \( \text{R}(450, 275) \)、\( \text{T}(400, 350) \),路径数为 384 条,传播路径最少的组合方式为 \( \text{T}(500, 350) \)、\( \text{R}(450, 200) \),路径数为 54 条。

针对问题三:针对 25 种收发机组合的信号传播路径数,考虑收发机的排列位置对收发机之间传播路径数的规律进行了归纳总结,并给出了解释。

针对问题四,建立了发射信号模型,场强衰减模型和多波干涉下信号接收模型,以此求解任意组合任意路径下的信号场强。在同一收发机组合内对经不同路径到达接收机的信号强度分析期望和方差,得到经不同路径到达的场强的概率密度分布,重点分析了同一接收机接收到的来自不同发射机的信号场强之间的相关性。

针对问题五:建立了针对宽带多波干涉现象的数学模型,分析了同一频率、不同路径之间和同一路径、不同频率信号之间的相关性。
\end{abstract}

\begin{center}
- 2 -
\end{center}

\section*{目录}

\begin{itemize}
    \item[1.] 绪论 \dotfill 5
    \begin{itemize}
        \item[1.1] 问题背景 \dotfill 5
        \item[1.2] 发展技术与研究现状 \dotfill 6
        \item[1.3] 问题提出 \dotfill 6
        \item[1.4] 主要工作 \dotfill 6
    \end{itemize}
    \item[2.] 名词解释、符号说明及基本假设 \dotfill 8
    \begin{itemize}
        \item[2.1] 名词解释 \dotfill 8
        \item[2.2] 符号说明 \dotfill 9
        \item[2.3] 基本假设 \dotfill 9
    \end{itemize}
    \item[3.] 问题分析 \dotfill 11
    \begin{itemize}
        \item[3.1] 问题一分析 \dotfill 11
        \item[3.2] 问题二分析 \dotfill 11
        \item[3.3] 问题三分析 \dotfill 11
        \item[3.4] 问题四分析 \dotfill 12
        \item[3.5] 问题五分析 \dotfill 12
    \end{itemize}
    \item[4.] 问题一的建模、求解 \dotfill 13
    \begin{itemize}
        \item[4.1] 模型的建立 \dotfill 13
        \begin{itemize}
            \item[4.1.1] 树结构模型的建立 \dotfill 13
            \item[4.1.2] 视距传播模型的建立 \dotfill 16
            \item[4.1.3] 反射传播模型的建立 \dotfill 17
        \end{itemize}
        \item[4.2] 模型的求解 \dotfill 20
        \item[4.3] 结果仿真 \dotfill 22
        \item[4.4] 模型小结 \dotfill 24
    \end{itemize}
    \item[5.] 问题二的建模、求解 \dotfill 25
    \begin{itemize}
        \item[5.1] 模型的建立、求解 \dotfill 25
        \item[5.2] 结果仿真 \dotfill 25
        \item[5.3] 模型小结 \dotfill 30
    \end{itemize}
    \item[6.] 问题三的求解 \dotfill 31
    \begin{itemize}
        \item[6.1] 数据的整理 \dotfill 31
        \item[6.2] 规律的阐述 \dotfill 32
        \begin{itemize}
            \item[6.2.1] 以发射机为研究对象对规律进行阐述 \dotfill 32
            \item[6.2.2] 以接收机为研究对象对规律进行阐述 \dotfill 33
        \end{itemize}
        \item[6.3] 规律的解释 \dotfill 34
    \end{itemize}
    \item[7.] 问题四的建模、求解 \dotfill 35
    \begin{itemize}
        \item[7.1] 多波干涉信号问题 \dotfill 35
        \begin{itemize}
            \item[7.1.1] 模型的建立 \dotfill 35
            \begin{itemize}
                \item (1) 发射信号模型 \dotfill 35
                \item (2) 场强衰减模型 \dotfill 35
                \item (3) 多波干涉下接收信号模型 \dotfill 39
            \end{itemize}
            \item[7.1.2] 多波干涉信号的求解 \dotfill 40
            \item[7.1.3] 模型小结 \dotfill 40
        \end{itemize}
        \item[7.2] 统计特性分析 \dotfill 41
        \begin{itemize}
            \item[7.2.1] 统计特性分析模型 \dotfill 41
            \item[7.2.2] 统计特性分析 \dotfill 42
        \end{itemize}
    \end{itemize}
    \item[8.] 问题五的建模、求解 \dotfill 44
    \item[参考文献] \dotfill 47
\end{itemize}

\section{一、绪论}

\subsection{1.1 问题背景}

随着通信技术的快速发展,能否更好地掌握无线移动通信的传播特性对无线通信网络的设计、规划和优化有着十分重要的意义。目前,为了应对频谱资源紧缺的问题,充分利用有限频率资源并增加通信网的容量,个人通信系统(PCS)的无线电覆盖通常采用微蜂窝、微微蜂窝结构。由于该结构具有传播距离短、基站高度低的特点,因而许多传统的传播预测模型已不再适合于微蜂窝传播环境。

射线跟踪法被认为是最适合用来研究微蜂窝结构的电波传播特性方法。这种方法用射线来表示电磁波束的传播,在确定了收发天线的位置以及周围环境等特征后,可根据电磁波的反射、绕射、透射、散射等波动现象借助计算机精确定每一条射线的传播路径,进而准确预测蜂窝区的场强(功率)分布,得到路径损耗、功率延迟谱等对移动通信工程设计具有重要实用意义的信道参数 \cite{ref1}。

射线跟踪法的难点在于确定每一条射线的传播路径,目前应用较多的传播路径技术有两种:一种是强力射线跟踪法,而另一种是镜像理论。强力射线跟踪法的优点在于它能用于任何复杂的传播环境中。而它的缺点是:由于在发射机处要每隔一定的角度发射一束射线并要跟踪这束射线,并且在接收机处要利用接收球进行接收测试,所以需要大量的计算时间和计算机内存;而且预测精度与发射源射线的角度间隔和接收球的半径密切相关。镜像法的理论是大家熟知的一种电磁理论,其理论根据是唯一性定理。镜像法是一种点对点的预测方法,它具有较高的预测精度,而且它一开始就舍弃了那些不能到达接收机的射线,所以它具有较高的计算效率。然而该方法的致命弱点是选择产生镜像的散射点十分困难。这样该方法只能用于一些简单的传播环境中,如用在设计良好的规则街道中。

通常,无线传播模型是无线网络规划的基础,电波传播模型的研究就是为了给运营商在网络规划的初期阶段提供一个较为准确的理论依据,以便于指导网络的规划。传播模型的准确与否将影响到链路预算小区半径,影响电波传播及干扰的计算,从而影响规划仿真的效果。一般情况下无线传播预测和覆盖预测都是基于经验模型进行的,它建立在统计分析的基础上,具有一定的普遍性。但任何一个经验传播模型都是根据其特定的预设环境给出的,有其独特的地形、地貌和地物特征。所以,这就有必要对特定环境进行无线传播测量,来确定该环境的无线传播模型,确保覆盖预测和系统性能仿真的准确性 \cite{ref2}。

随着越来越多的人使用移动通信,需要建立的辐射源越来越多,相互之间的干扰也越来越严重。多波干涉现象可理解为频率相同、振动方向相同、初始相位相同的简谐波在各接收点相遇叠加,出现某些接收点振动始终加强、而在另一些接收点振动始终减弱的现象。通过电场预测,可以在基站建立之前,预测对周围基站的影响,进而进行调整,这样可有效避免一些不必要的浪费和损失。

\subsection{1.2 发展技术与研究现状}

从七十年代开始,国内外对电波传播预测的研究工作就广泛展开,并建立了许多分析模型。根据辐射源的覆盖范围,这些模型可以大致分为大区模型、小区模型、微区模型。

近年来,微蜂窝技术由于采用了频谱复用技术而得到广泛应用,小区和微小区情况下的电波传播特性及分析模型正在成为国内外研究的热点。尤其对于城市小区及微区电波传播特性的分析与建模,在国内外著名刊物上都有不少相关论文发表。国内外很多学者都给出了在小区环境下电波传播的理论分析方法。

小区和微区模型远没有大区模型那么成熟,现在已经提出了几种有效的方法和工具,主要有设计线追踪模型、时域有限差分模型、人工神经网络模型等,应用较多的是射线追踪方法。但至今还没有提出一个精度较高的实用的模型。

近两年来,国内外又提出了一些新的预测模型及预测方法,如 Shinichi Ichitisubo 等人提出的多径传播模型、张卫等人提出的快速二维绕线模型、华东师范大学廖斌等人提出的基于虚拟源树的射线跟踪算法、东南大学顾晓龙等人提出的利用可见性概念改进的基于镜像原理的射线追踪法。Danlio Erricolo 等人还对不规则街区的电波传播进行了二维仿真 \cite{ref3}。

上述方法都存在着各自的优缺点和适用范围,由于城市环境十分复杂,上述模型有的精确较高、但是运算量却很大(如 FDTD 方法);有的方法或模型的计算速度较快,但其精确度却偏低,如两射线追踪模型,只考虑了直射和一次反射的贡献,而忽略了更高次反射和绕射,因此虽然计算速度快,但精度较低。寻找一种计算精度较高同时计算速度较快的新型计算模型迫在眉睫。

\subsection{1.3 问题提出}

问题详见全国第十届研究生数学建模竞赛 C 题。

\subsection{1.4 主要工作}

针对问题一:

(1)建立树结构模型

将建筑物简化为多边形,以多边形的点和边作为元素形成集合。将发射机作为树的源点,接收机作为树结构末端,集合元素作为树结构的结点。对点与点、点与边、边与边的可见性进行定义,建立表明各元素间可见性关系的邻接矩阵。通过邻接矩阵形成连通图或树状图结构。

(2) 建立视距传播模型

首先对视距传播概念进行定义,根据定义,取二维平面内任意两点与一边,依据几何关系对模型进行建立及求解。

(3) 建立反射传播模型

从两种基本情况(经一次反射、经两次反射)入手,依据几何关系建立并求解反射传播模型。

(4) 仿真分析

结合建立的模型编写程序,对问题进行仿真计算,将主要路径可视化展示在二维建筑平面图中并对模型的可靠性进行评价总结。

针对问题二:

(1) 借助问题一建立的模型,对 25 种收发机组合方式进行仿真计算,将主要路径分别可视化展示在二维建筑平面图中。

(2) 对 25 种收发机组合方式下的传播路径进行归纳分析,得出传播路径最多的组合方式与传播路径最少的组合方式。

针对问题三:

(1) 将问题二中仿真计算的结果进行整理,建立相应的统计折线图。

(2) 分别以发射机、接收机为研究对象,结合统计折线图对收发机之间传播路径的规律进行阐述。

(3) 针对提出的规律做出解释并验证。

针对问题四:

(1) 根据电磁波的传播规律和文中的相关假设要求,建立发射信号模型,信号传播衰减模型和接收到的多波干涉信号模型,求解得到所有组合情况下接收到的多波干涉信号。

(2) 在 (1) 的基础上对多波干涉信号振幅进行统计学分析,主要是对每种情况下经不同路径到达接收机的信号强度求期望和方差,得到经不同路径到达的场强的概率密度分布,重点分析了同一接收机接收到的来自不同发射机的信号之间的相关性。

针对问题五:

(1) 在问题四模型建立的基础上,进一步对 101 个单频信号形成的多波干涉现象进行了研究,建立了宽带多波干涉现象的数学模型。

(2) 分析合成了波的包络统计特性,重点分析了同频率不同路径信号之间和同一路径、不同频率信号之间的相关性。

\section{二、名词解释、符号说明及基本假设}

\subsection{2.1 名词解释}

1) 微蜂窝:

微蜂窝(microcell)是在宏蜂窝的基础上发展起来的一门技术。与宏蜂窝相比,它的发射功率较小,一般在 2W 左右;覆盖半径大约为 100m~1km。微蜂窝小区天线的发射功率小,允许较小的频率复用距离,使每个单元的信道数增多,可以较好的解决容量和通信质量的问题 \cite{ref4}。

2) 射线跟踪模型:

射线跟踪是一种被广泛用于移动通信和个人通信环境中的预测无线电波传播特性的技术,可以用来辨认出多径信道中收发之间所有可能的射线路径。一旦所有可能的射线被辨认出后,就可根据电波传播理论来计算每条射线的幅度、相位、延迟和极化,然后结合天线方向图和系统带宽就可得到接收点的所有射线的相干合成结果。

3) 一阶矩、二阶矩:

一阶矩是随机变量的期望,二阶矩是随机变量平方的期望,以此可以类推高阶的矩。

4) 反射:

波在传播过程中从一种媒质射向另一种媒质时,在两种媒质的界面上有部分波返回原媒质的现象。其传播路径如图 2-1-1 所示。

\begin{figure}[h]
    \centering
    \includegraphics[width=0.6\textwidth]{reflection_diagram.png}
    \caption{反射的传播规律}
    \label{fig:reflection}
\end{figure}

5) 绕射:

波在传播过程中经过障碍物边缘或孔隙时所发生的传播方向弯曲现象。孔隙越小,波长越大,这种现象就越显著。其传播路径如图 2-1-2 所示。

\begin{figure}[h]
    \centering
    \includegraphics[width=0.8\textwidth]{image.png}
    \caption{绕射的传播规律}
    \label{fig:绕射传播规律}
\end{figure}

6) 绕射系数:
    绕射系数反映绕射后绕射波强度的衰减程度。

7) 多波干涉现象:
    频率相同、振动方向相同、初始相位相同的简谐波在各接收点相遇叠加,出现某些接收点振动始终加强、而在另一些接收点振动始终减弱的现象。

8) 相关性:
    用相关系数描述,反映了两变量线性相关的密切程度。

\subsection{2.2 符号说明}

\begin{itemize}
    \item $D_{l}$: 视距传播衰减系数
    \item $D_{r}$: 反射系数
    \item $D_{d}$: 绕射系数
    \item $f$: 所有路径信号干涉叠加后的场强
    \item $Q$: 到达某接收点的传播途径总数
    \item $\rho$: 相关性系数
\end{itemize}

\subsection{2.3 基本假设}

1) 城市环境下的微蜂窝主要指高楼密集区,基站天线(发射机)低于周围建筑物的高度,电波是在建筑物的“峡谷”当中传播。电波经过屋顶绕射后再到达地面接收点的射线路经数量非常少,而且其场强与经过建筑物多次反射和绕射的路径相比很小,忽略经屋顶绕射和经地面反射的传播路径。

2) 微蜂窝环境下假设建筑物的高度高于基站天线的高度,电波的传播路径由三维问题简化为二维问题。

3) 在微蜂窝环境下,无线电波的传播只考虑两种传播机制:反射和绕射。

4) 对于城市微蜂窝的二维模型,建筑群划分为一定的“块”,建筑物简化为“多边形”,多边形的“边”代表建筑物的表面,多边形的“顶点”则代表了建筑物的拐角。

5) 将所有建筑物定义为理想电介质,电磁波传播到建筑物表面时,80\%的能量被反射回来,其余能量进入建筑物内,考虑到进入建筑物内电磁波的能量衰减,剩余的能量很小故不再考虑。

6) 由于经过多次反射或绕射后能量衰减情况较为严重,在此仅考虑下列收、发射机间的传播路径:
\begin{itemize}
    \item[A.] 只存在反射,且反射次数不超过7次。
    \item[B.] 只存在绕射,且绕射次数不超过2次。
    \item[C.] 一次绕射与一次、两次、三次或四次反射的任意组合。
    \item[D.] 两次绕射和一次反射的任意组合。
\end{itemize}

\section{三、问题分析}

\subsection{3.1 问题一分析}

题目中指出:根据实际情况,选择定量化的精细程度,然后跟踪确定发射机 $T_x$ 和接收机 $R_x$ 间的主要传播路径,并可视化展示在图中。

本题中我们可以采用运筹学中图论的方法,首先根据已知建筑物二维视图数据,在 Matlab 中绘制出建筑群的平面示意图。然后将建筑物的拐角以及表面用点元素和边元素的形式形成集合。在题目假设和定义可见性的基础上,我们可以提出一些判据模型来帮助我们建立所需要的树结构。由于射线在经过点和边时将分别发生绕射和反射现象,因此我们需要建立二维空间中点与点、点与边、边与边的可见性判据,并建立判定反射路径是否有效的判据。这些判据需要根据几何关系加以确定。

当我们将判据模型建立好之后,就可以得到各元素间的可见性关系。可见性关系又可以通过连通图和树状图方便地表现。在解答问题的过程中,多叉树结构作为载体存储了发射机、接收机、反射源(建筑物表面)、绕射源(建筑物拐角)的信息,判定模型可以作为判据判断结点是否有效,通过判断条件对树的遍历即可确定射线传播的所有有效路径。将上述过程通过编写 Matlab 程序实现,对程序进行运行仿真即可在二维图上得到所有有效路径的可视化表现。

\subsection{3.2 问题二分析}

题目中指出:现有 25 种发射机-接收机组合,找出哪一个组合收发机间的传播路径最多,哪一个组合最少。

本题的求解是建立在问题一工作基础上的,问题一中我们已经对模型进行了建立与求解,只要我们利用问题一建立的树结构模型及判据模型,根据不同路径上发射机和接收机的具体位置就可以方便地求出 25 种发射机-接收机组合下的传播路径。通过对数据进行分析比较即可得到哪一组合收发机间的传播路径最多或哪一组合收发机间的传播路径最少。

\subsection{3.3 问题三分析}

题目中指出:将问题二中所有发射机-接收机组合的传播路径进行对比,尽可能多的寻找规律。

本题的求解建立在问题二的工作基础之上,将问题二中的 25 种发射机-接收机组合的传播路径进行对比,参考不同组合方式下传播路径的数量统计表,根据表中数据分别以发射机、接收机为研究对象,建立相应的统计折线图。根据折线图和收发机的布置位置图做进一步分析,把观察到的结果进行总结归纳形成一定的规律。尝试对规律进一步解释。

\subsection{3.4 问题四分析}

在前三个问题中,针对 25 种发射机-接收机组合,我们分别找到了各自的主要传播路径,并找出了其中的传播规律。问题四是建立前三问的基础之上,针对每条路经、每种组合场强衰减、相位变化等信息进行进一步的分析求解。因此,需要针对发射信号、信号各种传播机制和多波干涉形成的接收信号建立数学模型,进行量化分析,求解信号分别经过不同路径到达接收机的场强,以及经过不同路径到达的信号的场强干涉叠加结果。为对经过不同路径到达接收机形成的多波干涉的振幅 $|f(w, t)|$ 进行统计学分析,对某种情况经不同路径到达接收机的信号变量求解期望和方差,并对各路径到达信号的场强求解概率密度分布,相关性主要考虑接收机接收到的不同发射机发射的信号之间的关系,固定一接收机,分别计算不同发射机发射信号后接收机接收到的信号相关性。

\subsection{3.5 问题五分析}

问题四主要考虑单一频率情况,问题五是在问题四的基础上在信号的发射与传播上进一步增加了频率这一变量,单一频率的不同波相遇会形成稳定的干涉,而不同频率的波经过不同路径到达接收点时会形成复杂的宽带多波干涉现象,出现包络现象,通过对多频率不同路径的传播信号进行仿真求和,得到叠加后的信号,针对信号特点参照双频干涉形成拍的数学模型对宽带多波干涉现象进行数学建模,并对合成波的包络特性进行分析。

\section{四、问题一的建模、求解}

\subsection{4.1 模型的建立}

基本模型建立步骤(流程图)如下:

\begin{figure}[h]
\centering
\includegraphics[width=0.8\textwidth]{model_flowchart.png}
\caption{模型建立流程}
\label{fig:model_flowchart}
\end{figure}

\subsubsection{4.1.1 树结构模型的建立}

射线追踪法的关键在于准确地找寻出有效射线路径。本题采用图论的计算方法,图论是应用十分广泛的运筹学分支,我们将建筑物的表面与拐角用树的结点形式表示,发射机作为树的源点。多叉树结构作为载体存储了发射机、接收机、反射源(建筑物表面)、绕射源(建筑物拐角)的信息,通过一定的判据判断结点是否有效,即可确定射线传播的有效路径。从发射机所在的源点到指定的场点可能存在众多的射线,射线追踪的过程则对应于树的遍历。

具体分析:根据题目所给出的“城市微小区地图对应的数据.txt”,利用 Matlab 绘制出建筑物二维视图,如图 \ref{fig:building_2d_view} 所示。

\begin{figure}[h]
\centering
\includegraphics[width=0.8\textwidth]{building_2d_view.png}
\caption{建筑物二维视图}
\label{fig:building_2d_view}
\end{figure}

建筑物用“多边形”定义,多边形的“边”定义为建筑物的表面,多边形的“顶点”定义为建筑物的拐角。分别建立由“边”和“顶点”组成的集合 P、W。观察可发现建筑群共有 80 个“顶点”和 80 条“边”。

对应建立的集合为 $P\{P_1, P_2, \ldots, P_{80}\}$、$W\{W_1, W_2, \ldots, W_{80}\}$。

树结构的源点(发射机)用 $T_x$ 表示,场点用 $R_x$ 表示。射线由源点出发,经反射面和绕射面作用后最终到达场点。下面简要叙述树结构模型的建立过程,通过树状图结构不难发现,如果某一个树结点不产生后继树结点,就可在树结构中去除其后继树结点之间的路径,与该路径连接的所有后代树结点也都可去除,从而使树结构得到简化 \footnote{[5]}。由于建筑物之间的遮挡会生成一些无效路径,为了去除这些路径需要对树结构添加可见性判据:

(1) 两点间的可见性判据:如果二维平面中相异两点 X、Y 不与任何边相交,则认为点 X、Y 是可见的。

(2) 点与边的可见性判据:如果二维平面中某边上存在相异于点 X 的一点 Y,使得点 X 与点 Y 可见,则认为该边与点 X 是可见的。

(3) 边与边的可见性判据:如果二维平面中一边上存在一点 X,使得 X 与另一边可见,则称两边是可见的。

现定义函数 $F = f(x, y)$,其中
\[
\begin{cases}
x = P_i, W_j, T_x, R_x, & P_i, P_l \in \{P\}, \, W_j, W_v \in \{W\}, \\
y = P_l, W_v, T_x, R_x
\end{cases}
\]
又 $i \neq l$, $j \neq v$。若从集合 $P \cup W \cup \{T_x, R_x\}$ 中任取两不同元素满足上述可见性判据,则函数值 $F = 1$,否则 $F = 0$。

依据上述定义可建立反映各元素间可见性的邻接矩阵:
\[
\begin{array}{c|cccccccccc}
i, j \backslash l, v & P_1 & P_2 & \cdots & P_{80} & W_1 & W_2 & \cdots & W_{80} & T_x & R_x \\
\hline
P_1 & F_{11} & F_{12} & F_{13} & F_{14} & \cdots \\
P_2 & F_{21} & F_{22} & F_{23} & & & & & & & \\
\vdots & & & & & & & & & & \\
P_{80} & F_{31} & F_{32} & F_{33} & & & & & & & \\
W_1 & F_{41} & F_{42} & & & & & & & & \\
W_2 & F_{51} & & & & & & & & & \\
\vdots & \vdots & & & & & & & & & \\
W_{80} & & & & & & & & & & \\
T_x & & & & & & & & & & \\
R_x & & & & & & & & & &
\end{array}
\]

根据运筹学中图论的方法,结合邻接矩阵,上式中第一行、第一列各元素之间的可见性关系也可由连通图表现出来,图 4-1-3 为一种简单的连通图示意图,通过连通图可以清晰、方便地观察出各元素之间的特定关系。

\begin{figure}[h]
    \centering
    \includegraphics[width=0.8\textwidth]{image1.png}
    \caption{连通图示意图}
    \label{fig:connected_graph}
\end{figure}

树状图的建立就是以连通图为基础的,通过对连通图的适当变化,即可将得到的连通图转换为树状图。图 \ref{fig:tree_diagram} 即为连通图对应的树状图示意图。

\begin{figure}[h]
    \centering
    \includegraphics[width=0.8\textwidth]{image2.png}
    \caption{树状图示意图}
    \label{fig:tree_diagram}
\end{figure}

图 \ref{fig:tree_diagram} 所示的树状图仅是一种示意图,本题中需考虑的元素共有 162 个(80 个点、80 条边、接收源与发射源),树状图结点数目将随层数呈指数形式急速增加,这将使计算变得极为复杂。由题目假设,当绕射和反射次数较多时能量迅速衰减,因此不予考虑,本题中仅考虑以下四种情况:
\begin{enumerate}
    \item 只存在反射,且反射次数不超过 7 次。
    \item 只存在绕射,且绕射次数不超过 2 次。
    \item 一次绕射与一次、两次、三次或四次反射的任意组合。
    \item 两次绕射和一次反射的任意组合。
\end{enumerate}

射线经过点发生绕射,经过边发生发射,因此我们需按上述假设要求控制树结构各分支中点元素与边元素出现的个数,这将在计算编程中予以实现。

面我们将对可见性判据进行建模分析。

\subsubsection{4.1.2 视距传播模型的建立}

为了对二维平面内任意两点进行可见性判断,我们提出了一种判定视距传播的数学模型。

具体分析:取建筑群二维平面中任意两点,若两点连线不被任意一墙面(边)截断,则两点间信号的传递定义为视距传播(或可见性传播),反之为非视距传播(不可见性传播)。如图 4-1-5 所示,两点间为非视距传播。图 4-1-6 中所示两点间为视距传播。

\begin{figure}[h]
    \centering
    \includegraphics[width=0.8\textwidth]{image1.png}
    \caption{图 4-1-5 非视距传播示意图}
\end{figure}

\begin{figure}[h]
    \centering
    \includegraphics[width=0.8\textwidth]{image2.png}
    \caption{图 4-1-6 视距传播示意图}
\end{figure}

\textbf{视距传播的判定:}

如图 4-1-7 所示,设点 $P_1$ 表示为发射机,点 $P_2$ 表示为接收机,$M$、$N$ 表示为建筑物拐角点,线段 $\overline{MN}$ 表示为建筑物墙面。线段 $\overline{MN}$ 与线段 $\overline{P_1P_2}$ 的连线交于点 $Q$。设 $\angle NP_1L$ 为 $\alpha_N$、$\angle P_2P_1L$ 为 $\alpha_M$、$\angle MP_1L$ 为 $\alpha_P$。

\begin{figure}[h]
    \centering
    \includegraphics[width=0.8\textwidth]{image3.png}
    \caption{图 4-1-7 判定视距传播模型}
\end{figure}

则判断两点间为非视距传播的判据为

\begin{align*}
F(x) &= 2j\sqrt{x}\exp(jx)\left(\int_{0}^{+\infty}e^{-j\tau^2}d\tau - \int_{0}^{\sqrt{x}}e^{-j\tau^2}d\tau\right) \\
&= 2j\sqrt{x}\exp(jx)\left(\frac{\sqrt{\pi}}{2}e^{-j\pi/4} - \int_{0}^{\sqrt{x}}e^{-j\tau^2}d\tau\right) \tag{7-1-13} \\
\alpha^{\pm}(\beta) &= 2\cos^2\left(\frac{2n\pi N^{\pm} - \beta}{2}\right) \tag{7-1-14}
\end{align*}

\begin{align}
2n\pi N^+ - \beta &= \pi \tag{7-1-15} \\
2n\pi N^- - \beta &= -\pi \tag{7-1-16}
\end{align}

由几何关系可得:
\begin{equation}
S_{\Box_{MP_1N}} = \frac{1}{2} \overline{P_1M} \overline{P_1N} \sin(\alpha_M - \alpha_N) \tag{4-1-3}
\end{equation}
\begin{equation}
S_{\Box_{QP_1N}} = \frac{1}{2} \overline{P_1Q} \overline{P_1N} \sin(\alpha_P - \alpha_N) \tag{4-1-4}
\end{equation}
\begin{equation}
S_{\Box_{MP_1P_Q}} = \frac{1}{2} \overline{MP_1} \overline{P_1Q} \sin(\alpha_M - \alpha_P) \tag{4-1-5}
\end{equation}
\begin{equation}
S_{\Box_{MP_1N}} = S_{\Box_{MP_1Q}} + S_{\Box_{QP_1N}} \tag{4-1-6}
\end{equation}

联立上式可解得:
\begin{equation}
\overline{P_1Q} = \frac{\overline{MP_1} \overline{P_1N} \sin(\alpha_M - \alpha_N)}{\overline{MP_1} \sin(\alpha_M - \alpha_P) + \overline{P_1N} \sin(\alpha_P - \alpha_N)} \tag{4-1-7}
\end{equation}

将式 4-1-1 与式 4-1-2 作为视距传播的判据模型,遍历建筑群的所有墙面,将集合 \(W\) 中的元素一一验证计算,即可判别出建筑群二维平面中发射机与接收机之间是否为视距传播。

\subsubsection{4.1.3 反射传播模型的建立}

为了进一步排除射线的无效传播路径,按照射线反射的规律筛选出有效传播路径,我们提出了一种判定反射传播的数学模型。

具体分析:为了更好地说明数学模型的建立过程,在此我们详细分析了两种基本情况下判断目标路径是否为有效反射路径的方法,其他情况可按照这两种情况加以类推。

第一种情况:如图 4-1-8 所示,已知点 \(P_1\)、\(P_2\) 位置以及边 \(W_1\)、\(W_2\) 位置,我们将判断点 \(P_1\) 通过边 \(W_1\) 经过一次反射能否到达点 \(P_2\)。点 \(P_1'\) 为点 \(P_1\) 关于边 \(W_1\) 的镜像点。

\begin{figure}[h]
    \centering
    \includegraphics[width=0.8\textwidth]{reflection_diagram_1.png}
    \caption{判定反射传播示意图一}
    \label{fig:reflection_diagram_1}
\end{figure}

针对一次反射的情况,可以分以下几个步骤来判断传播路径是否为有效传播路径:

步骤一:作点 $P_1$ 关于边 $W_1$ 的镜像点 $P_1'$。

步骤二:连接点 $P_1'$ 与点 $P_2$,得到线段 $\overline{P_1'P_2}$ 与边 $W_1$ 的交点 $Q$。此时判断点 $Q$ 与点 $P_2$ 是否可见并判断点 $Q$ 是否在边 $W_1$ 上,若判断结果为 “是” 则进行下一步骤,否则终止。

步骤三:判断点 $Q$ 与点 $P_1$ 是否可见,若判断结果为 “是” 则判定反射传播路径有效,否则终止。

第二种情况:如图 4-1-9 所示,已知点 $P_1$、$P_2$ 位置以及边 $W_1$、$W_2$、$W_3$ 位置,我们将判断点 $P_1$ 通过边 $W_1$ 与边 $W_2$ 经过两次反射能否到达点 $P_2$。点 $P_1'$ 为点 $P_1$ 关于边 $W_1$ 的镜像点。点 $P_1''$ 为点 $P_1'$ 关于边 $W_2$ 的镜像点。

\begin{figure}[h]
    \centering
    \includegraphics[width=0.8\textwidth]{image.png}
    \caption{判定反射传播示意图二}
\end{figure}

针对两次反射的情况,可以分以下几个步骤来判断传播路径是否为有效传播路径:

步骤一:作点 $P_1$ 关于边 $W_1$ 的镜像点 $P_1'$。

步骤二:作点 $P_1'$ 关于边 $W_2$ 的镜像点 $P_1''$。

步骤三:连接点 $P_1''$ 与点 $P_2$,得到线段 $\overline{P_1''P_2}$ 与边 $W_3$ 的交点 $Q_2$。此时判断点 $Q_2$ 与点 $P_2$ 是否可见并判断点 $Q_2$ 是否在边 $W_3$ 上,若判断结果为 “是” 则进行下一步骤,否则终止。

步骤四:连接点 $Q_2$ 与点 $P_1'$,得到线段 $\overline{Q_2P_1'}$ 与边 $W_1$ 的交点 $Q_1$。此时判断点 $Q_1$ 与点 $Q_2$ 是否可见并判断点 $Q_1$ 是否在边 $W_1$ 上,若判断结果为 “是” 则进行下一步骤,否则终止。

步骤五:判断点 $Q_1$ 与点 $P_1$ 是否可见,若判断结果为 “是” 则判定反射传播路径有效,否则终止。

此处我们仅分析了经过一次反射与两次反射传播路径是否为有效传播路径的情况,经过更多次反射的判定方法可按照这两种类推。

下面我们建立二维空间中求点关于边的镜像点的数学模型。

已知空间中任一点 \( P(x, y) \) 坐标与边上两端点 \( M \)、\( N \) 坐标,求点 \( P(x, y) \) 关于边 \( \overline{MN} \) 的镜像点 \( P'(x, y) \) 坐标。如图 4-1-10 所示,点 \( V \) 为线段 \( \overline{PP'} \) 与边 \( \overline{MN} \) 的交点。

\begin{figure}[h]
    \centering
    \includegraphics[width=0.6\textwidth]{image.png}
    \caption{求点关于边镜像点的模型}
    \label{fig:4-1-10}
\end{figure}

设 \( \overrightarrow{OM} = \mathbf{a} \)、\( \overrightarrow{ON} = \mathbf{b} \)、\( \overrightarrow{OP} = \mathbf{c} \),由几何关系可得:

\begin{equation}
    \overline{MN} \square \overline{PP'} = 0 \tag{4-1-8}
\end{equation}

\begin{equation}
    \left| \overline{MV} \right| = \frac{\overline{MN} \square \overline{MP}}{\left| \overline{MN} \right|} = \frac{(\mathbf{b} - \mathbf{a}) \square (\mathbf{c} - \mathbf{a})}{\left| \mathbf{b} - \mathbf{a} \right|} \tag{4-1-9}
\end{equation}

\begin{equation}
    \overrightarrow{OV} = \overrightarrow{OM} + \overrightarrow{MV} \tag{4-1-10}
\end{equation}

设 \( \overrightarrow{OV} = \mathbf{d} \),因此有:

\begin{equation}
    \mathbf{d} = \mathbf{a} + (\mathbf{b} - \mathbf{a}) \square (\mathbf{c} - \mathbf{a}) \square \frac{\mathbf{b} - \mathbf{a}}{\left| \mathbf{b} - \mathbf{a} \right|^2} \tag{4-1-11}
\end{equation}

故可以得到:

\begin{equation}
    \overrightarrow{OP'} = \overrightarrow{OP} + \overrightarrow{PP'} = \overrightarrow{OP} + 2 \overrightarrow{PQ} = \mathbf{c} + 2 (\mathbf{d} - \mathbf{c}) = 2 \mathbf{d} - \mathbf{c} \tag{4-1-12}
\end{equation}

由上式即可求出点 \( P' \) 坐标。

下面我们建立判定镜像点和接收点连线与过边直线交点是否在边上的数学模型。

已知空间中一镜像点 \( P_2 \) 坐标和接收点 \( P_1 \) 坐标,并知道反射边端点 \( M \)、\( N \) 坐标。如图 4-1-11 所示,点 \( Q \) 为镜像点和接收点连线与过边 \( \overline{MN} \) 直线交点。

\begin{figure}[h]
    \centering
    \includegraphics[width=0.5\textwidth]{image.png}
    \caption{判定镜像点和接收点连线与过边直线交点是否在边上的数学模型}
    \label{fig:4-1-11}
\end{figure}

设 $\overrightarrow{OP_1} = \mathbf{a}$、$\overrightarrow{OP_2} = \mathbf{b}$、$\overrightarrow{OM} = \mathbf{m}$、$\overrightarrow{ON} = \mathbf{n}$、$\overrightarrow{OQ} = \mathbf{q}$,由几何关系可得:
\begin{align*}
F(x) &= 2j\sqrt{x}\exp(jx)\left(\int_{0}^{+\infty}e^{-j\tau^2}d\tau - \int_{0}^{\sqrt{x}}e^{-j\tau^2}d\tau\right) \\
&= 2j\sqrt{x}\exp(jx)\left(\frac{\sqrt{\pi}}{2}e^{-j\pi/4} - \int_{0}^{\sqrt{x}}e^{-j\tau^2}d\tau\right) \tag{7-1-13} \\
\alpha^{\pm}(\beta) &= 2\cos^2\left(\frac{2n\pi N^{\pm} - \beta}{2}\right) \tag{7-1-14}
\end{align*}
其中 $\lambda$、$\mu \in \mathbb{R}$,解得:
\begin{equation}
    \begin{cases}
        \mathbf{q} - \mathbf{n} = \lambda (\mathbf{m} - \mathbf{n}) \\
        \mathbf{q} - \mathbf{a} = \mu (\mathbf{b} - \mathbf{a})
    \end{cases} \tag{4-1-15}
\end{equation}
\begin{equation}
    \begin{cases}
        \mathbf{q} + \lambda (\mathbf{n} - \mathbf{m}) = \mathbf{n} \\
        \mathbf{q} + \mu (\mathbf{a} - \mathbf{b}) = \mathbf{a}
    \end{cases} \tag{4-1-16}
\end{equation}

将已知向量坐标代入方程组 4-1-16,可得式:
\begin{equation}
    \begin{pmatrix}
        1 & 0 & n_x - m_x & 0 \\
        0 & 1 & n_y - m_y & 0 \\
        1 & 0 & 0 & a_x - b_x \\
        0 & 1 & 0 & a_y - b_y
    \end{pmatrix}
    \begin{pmatrix}
        q_x \\
        q_y \\
        \lambda \\
        \mu
    \end{pmatrix}
    =
    \begin{pmatrix}
        n_x \\
        n_y \\
        a_x \\
        a_y
    \end{pmatrix} \tag{4-1-17}
\end{equation}

由方程组 4-1-17 可分别解得 $q_x$、$q_y$、$\lambda$、$\mu$,当判断镜像点和接收点连线与过反射边直线交点是否在反射边上时,仅需满足:
\begin{equation}
    \begin{cases}
        0 < \lambda < 1 \\
        0 < \mu < 1
    \end{cases} \tag{4-1-18}
\end{equation}

\subsection{4.2 模型的求解}

按照题目要求,我们首先提取出城市建筑群中建筑物的点(拐角)元素和边(表面)元素,建立相应的点集合 $P\{P_1, P_2, \ldots, P_{80}\}$ 与边集合 $W\{W_1, W_2, \ldots, W_n\}$。各元素在二维图中的对应位置如图 4-2-1 所示。

\begin{figure}[h]
    \centering
    \includegraphics[width=\textwidth]{image.png}
    \caption{点、边元素在二维图中的对应位置图}
    \label{fig:point_edge}
\end{figure}

然后利用视距传播模型作为判据,我们建立了反映各元素间可见性的邻接矩阵,由于矩阵规模较大(182 行×182 列),这里我们只选出矩阵的一部分结合对应的二维图对邻接矩阵的可靠性进行验证说明。

选出的部分邻接矩阵为:
\begin{equation}
\begin{array}{c|ccccccc}
i,j\backslash l,v & P_{11} & P_{12} & P_{13} & P_{14} & P_{15} & P_{16} & P_{17} \\
\hline
P_{11} & 0 & 0 & 1 & 1 & 0 & 0 & 1 \\
P_{12} & 0 & 0 & 1 & 1 & 0 & 0 & 0 \\
P_{13} & 1 & 1 & 0 & 0 & 0 & 0 & 1 \\
P_{14} & 1 & 1 & 0 & 0 & 0 & 0 & 1 \\
P_{15} & 0 & 0 & 0 & 0 & 0 & 0 & 1 \\
P_{16} & 0 & 0 & 0 & 0 & 0 & 0 & 0 \\
P_{17} & 1 & 0 & 1 & 1 & 1 & 0 & 0 \\
\end{array}
\tag{4-2-1}
\end{equation}

矩阵在二维图中的对应区域如图 4-2-2 所示,观察邻接矩阵可以发现,函数值 $F=1$ 的点组为 $(P_{11},P_{13})$、$(P_{11},P_{14})$、$(P_{11},P_{17})$、$(P_{12},P_{13})$、$(P_{12},P_{14})$、$(P_{13},P_{11})$、$(P_{13},P_{12})$、$(P_{13},P_{17})$、$(P_{14},P_{11})$、$(P_{14},P_{12})$、$(P_{14},P_{17})$、$(P_{15},P_{17})$、$(P_{17},P_{11})$、$(P_{17},P_{13})$、$(P_{17},P_{14})$、$(P_{17},P_{15})$,说明点组中的两点为可见的,两点间的射线传播方式为视距传播,反映在图 4-2-2 中可由点与点之间的连线不被任意边截断。例如可见点组 $(P_{11},P_{13})$ 与 $(P_{13},P_{11})$,表现在二维图中即为 P13 与 P11 的连线不被任意建筑物的边截断,证明了邻接矩阵中该点组的可见性表达是正确的。

\begin{figure}[h]
    \centering
    \includegraphics[width=\textwidth]{image.png}
    \caption{选出的邻接矩阵在二维图中的对应位置图}
    \label{fig:adjacency_matrix}
\end{figure}

在本问题中,通过视距传播模型的判据建立起了邻接矩阵,通过邻接矩阵可以建立相应的连通图与树状图。再结合判定反射传播的判据,并考虑到题目的基本假设,我们就可以在 Matlab 中进行仿真,从而得到从发射机到接收机之间所有的有效射线路径。

\subsection{4.3 结果仿真}

按照以上分析方法,在 Matlab 中对各个判定模型建立相应的仿真程序并运行仿真,得到的结果如下:

表 4-3-1 显示的为未经过反射传播模型判定得到的发射机到接收机间的射线传播路径。可观察到共有 8870 条传播路径,其中大部分为未经反射条件排除的无效路径。

\begin{tabular}{c c c c c c}
\hline
\multirow{2}{*}{\begin{tabular}{c}
接收机 \\
发射机
\end{tabular}

\begin{table}[h]
\centering
\caption{表 4-3-2 经反射传播判定得到的发射机到接收机间的射线传播路径表}
\begin{tabular}{c l}
\hline
路径序号 & 路径 \\
\hline
1 & $T_x \to P_2 \to P_{40} \to W_{50} \to R_x$ \\
2 & $T_x \to P_2 \to P_{40} \to R_x$ \\
3 & $T_x \to P_2 \to P_{48} \to W_{50} \to R_x$ \\
4 & $T_x \to P_2 \to P_{48} \to R_x$ \\
5 & $T_x \to P_2 \to P_{50} \to P_{47} \to R_x$ \\
6 & $T_x \to P_2 \to P_{50} \to R_x$ \\
$\vdots$ & $\vdots$ \\
72 & $T_x \to W_{28} \to P_{15} \to P_{40} \to R_x$ \\
73 & $T_x \to W_{28} \to P_{15} \to P_{48} \to R_x$ \\
74 & $T_x \to W_{28} \to P_{15} \to P_{50} \to R_x$ \\
\hline
\end{tabular}
\end{table}

表 4-3-2 显示的为经过反射传播模型判定后发射机到接收机间的射线传播路径。可观察到共有 74 条传播路径,74 条路径均为有效传播路径。

有效传播路径在二维图中的可视化如图 4-3-1 所示。

\begin{figure}[h]
    \centering
    \includegraphics[width=\textwidth]{image.png}
    \caption{有效传播路径在二维图中的可视化图}
    \label{fig:4-3-1}
\end{figure}

\subsection{4.4 模型小结}

本题中建立的模型较好地判断出了二维建筑群中射线由发射机到接收机的全部有效传播路径,实现了对射线传播的追踪。

模型使用多叉树结构图来存储发射机、接收机、点元素(绕射源)、边元素(反射源)的全部信息,通过视距传播模型判据与反射传播模型判据对树结点进行有效性判断,实现了对整个射线传播过程的跟踪。

观察仿真结果,我们注意到经过视距传播模型判据判断后仍有大量无效路径存在,在计算中我们仍需要利用反射传播模型判据对这些无效路径一一判定,这必将影响到计算速度,增加了计算的时间和难度。因此我们在后续工作中有必要对视距传播模型判据进行加强,这就涉及到对可见性的重新定义(建筑物之间的遮挡将导致建筑物的边仅有部分可作为有效反射源)。对判据的加强将会使计算过程更加高效,降低解决问题的复杂度。

\section{五、问题二的建模、求解}

\subsection{5.1 模型的建立、求解}

本题的求解是建立在问题一工作基础上的,利用问题一建立的树结构模型及判据模型,根据不同路径上发射机和接收机的具体位置可以方便地求出 25 种发射机-接收机组合下的传播路径。通过对数据进行分析比较即可得到哪一组合收发机间的传播路径最多或哪一组合收发机间的传播路径最少。

发射机、接收机在二维图上的具体位置如图 5-1-1 所示。

\begin{figure}[h]
    \centering
    \includegraphics[width=\textwidth]{image.png}
    \caption{发射机、接收机的具体位置图}
    \label{fig:5-1-1}
\end{figure}

\subsection{5.2 结果仿真}

按照以上分析方法,利用问题一中用到的 Matlab 仿真程序,对 25 种收发机组合一一进行仿真计算,得到的传播路径仿真图如下:

以 T1 为发射机,R1、R2 为接收机的传播路径仿真图如图 5-2-1,图 5-2-2 所示。

\begin{figure}[h]
    \centering
    \includegraphics[width=0.45\textwidth]{image1.png}
    \caption{T1R1}
    \label{fig:5-2-1}
\end{figure}
\begin{figure}[h]
    \centering
    \includegraphics[width=0.45\textwidth]{image2.png}
    \caption{T1R2}
    \label{fig:5-2-2}
\end{figure}

以 T1 为发射机,R3、R4 为接收机的传播路径仿真图如图 \ref{fig:5-2-3},图 \ref{fig:5-2-4} 所示。

\begin{figure}[h]
    \centering
    \includegraphics[width=0.45\textwidth]{image3.png}
    \caption{T1R3}
    \label{fig:5-2-3}
\end{figure}
\begin{figure}[h]
    \centering
    \includegraphics[width=0.45\textwidth]{image4.png}
    \caption{T1R4}
    \label{fig:5-2-4}
\end{figure}

以 T1 为发射机,R5 为接收机和以 T2 为发射机,R1 为接收机的传播路径仿真图如图 \ref{fig:5-2-5},图 \ref{fig:5-2-6} 所示。

\begin{figure}[h]
    \centering
    \includegraphics[width=0.45\textwidth]{image5.png}
    \caption{T1R5}
    \label{fig:5-2-5}
\end{figure}
\begin{figure}[h]
    \centering
    \includegraphics[width=0.45\textwidth]{image6.png}
    \caption{T2R1}
    \label{fig:5-2-6}
\end{figure}

以 T2 为发射机,R2、R3 为接收机的传播路径仿真图如图 5-2-7,图 5-2-8 所示。

\begin{figure}[h]
    \centering
    \includegraphics[width=0.45\textwidth]{image1.png}
    \caption{T2R2}
    \label{fig:5-2-7}
\end{figure}
\begin{figure}[h]
    \centering
    \includegraphics[width=0.45\textwidth]{image2.png}
    \caption{T2R3}
    \label{fig:5-2-8}
\end{figure}

以 T2 为发射机,R4、R5 为接收机的传播路径仿真图如图 \ref{fig:5-2-9},图 \ref{fig:5-2-10} 所示。

\begin{figure}[h]
    \centering
    \includegraphics[width=0.45\textwidth]{image3.png}
    \caption{T2R4}
    \label{fig:5-2-9}
\end{figure}
\begin{figure}[h]
    \centering
    \includegraphics[width=0.45\textwidth]{image4.png}
    \caption{T2R5}
    \label{fig:5-2-10}
\end{figure}

以 T3 为发射机,R1、R2 为接收机的传播路径仿真图如图 \ref{fig:5-2-11},图 \ref{fig:5-2-12} 所示。

\begin{figure}[h]
    \centering
    \includegraphics[width=0.45\textwidth]{image5.png}
    \caption{T3R1}
    \label{fig:5-2-11}
\end{figure}
\begin{figure}[h]
    \centering
    \includegraphics[width=0.45\textwidth]{image6.png}
    \caption{T3R2}
    \label{fig:5-2-12}
\end{figure}

以 T3 为发射机,R3、R4 为接收机的传播路径仿真图如图 5-2-13,图 5-2-14 所示。

\begin{figure}[h]
    \centering
    \includegraphics[width=0.45\textwidth]{image1.png}
    \caption{T3R3}
    \label{fig:5-2-13}
\end{figure}
\begin{figure}[h]
    \centering
    \includegraphics[width=0.45\textwidth]{image2.png}
    \caption{T3R4}
    \label{fig:5-2-14}
\end{figure}

以 T3 为发射机,R5 为接收机和以 T4 为发射机,R1 为接收机的传播路径仿真图如图 \ref{fig:5-2-15},图 \ref{fig:5-2-16} 所示。

\begin{figure}[h]
    \centering
    \includegraphics[width=0.45\textwidth]{image3.png}
    \caption{T3R5}
    \label{fig:5-2-15}
\end{figure}
\begin{figure}[h]
    \centering
    \includegraphics[width=0.45\textwidth]{image4.png}
    \caption{T4R1}
    \label{fig:5-2-16}
\end{figure}

以 T4 为发射机,R2、R3 为接收机的传播路径仿真图如图 \ref{fig:5-2-17},图 \ref{fig:5-2-18} 所示。

\begin{figure}[h]
    \centering
    \includegraphics[width=0.45\textwidth]{image5.png}
    \caption{T4R2}
    \label{fig:5-2-17}
\end{figure}
\begin{figure}[h]
    \centering
    \includegraphics[width=0.45\textwidth]{image6.png}
    \caption{T4R3}
    \label{fig:5-2-18}
\end{figure}

以 T4 为发射机,R4、R5 为接收机的传播路径仿真图如图 5-2-19,图 5-2-20

所示。

\begin{figure}[h]
    \centering
    \includegraphics[width=0.45\textwidth]{image1.png}
    \caption{T4R4}
    \label{fig:5-2-19}
\end{figure}
\begin{figure}[h]
    \centering
    \includegraphics[width=0.45\textwidth]{image2.png}
    \caption{T4R5}
    \label{fig:5-2-20}
\end{figure}

以 T5 为发射机, R1、R2 为接收机的传播路径仿真图如图 \ref{fig:5-2-21}, 图 \ref{fig:5-2-22} 所示。

\begin{figure}[h]
    \centering
    \includegraphics[width=0.45\textwidth]{image3.png}
    \caption{T5R1}
    \label{fig:5-2-21}
\end{figure}
\begin{figure}[h]
    \centering
    \includegraphics[width=0.45\textwidth]{image4.png}
    \caption{T5R2}
    \label{fig:5-2-22}
\end{figure}

以 T5 为发射机, R3、R4 为接收机的传播路径仿真图如图 \ref{fig:5-2-23}, 图 \ref{fig:5-2-24} 所示。

\begin{figure}[h]
    \centering
    \includegraphics[width=0.45\textwidth]{image5.png}
    \caption{T5R3}
    \label{fig:5-2-23}
\end{figure}
\begin{figure}[h]
    \centering
    \includegraphics[width=0.45\textwidth]{image6.png}
    \caption{T5R4}
    \label{fig:5-2-24}
\end{figure}

以 T5 为发射机,R5 为接收机的传播路径仿真图如图 5-2-25。

\begin{figure}[h]
    \centering
    \includegraphics[width=0.8\textwidth]{image.png}
    \caption{T5R5}
    \label{fig:5-2-25}
\end{figure}

将 25 种组合方式下收发机的传播路径数量统计成表,如表 5-2-1 所示。

\begin{table}[h]
    \centering
    \caption{不同组合方式下收发机传播路径数量统计表}
    \label{tab:5-2-1}
    \begin{tabular}{c c c c c c}
        \hline
        \diagbox{T}{R} & R1 & R2 & R3 & R4 & R5 \\
        \hline
        T1 & 97 & 137 & 97 & 89 & 83 \\
        T2 & 82 & 123 & 82 & 75 & 65 \\
        T3 & 153 & 348 & 216 & 234 & 225 \\
        T4 & 186 & 268 & 208 & 198 & 184 \\
        T5 & 83 & 116 & 75 & 71 & 54 \\
        \hline
    \end{tabular}
\end{table}

通过表中数据可以看出,T3 与 R2 组合方式下共有 348 条传播路径,是传播路径最多的组合。T5 与 R5 组合方式下共有 54 条传播路径,是传播路径最少的组合。

\subsection{5.3 模型小结}

本题中涉及模型为问题一中已经建立和求解的模型。通过对 25 种不同组合方式下收发机的传播路径进行仿真计算,可以进一步验证问题一中模型建立的准确性与可靠性。另外,通过对不同布置方式下得到数据的进一步分析亦可以寻找一定的规律,为问题三的解决奠定基础。

\section{六、问题三的求解}

本题的求解建立在问题二的工作基础之上,将问题二中的 25 种发射机-接收机组合的传播路径进行对比,参考不同组合方式下传播路径的数量统计表,寻找尽可能多的规律。

\subsection{6.1 数据的整理}

在上一节中,表 5-2-1 已经将 25 种组合方式下收发机的传播路径数量作了统计,为了方便观察和寻找规律,我们现将表 5-2-1 中数据以折线图的形式表现出来。以发射机作为研究对象,观察不同发射机间传播路径数量的多少可参考图 6-1-1。以接收机作为自变量,观察不同接收机间传播路径数量的多少可参考图 6-1-2。

\begin{figure}[h]
    \centering
    \includegraphics[width=\textwidth]{image1.png}
    \caption{图 6-1-1 发射机-接收机传播路径折线图}
\end{figure}

\begin{figure}[h]
    \centering
    \includegraphics[width=\textwidth]{image2.png}
    \caption{图 6-1-2 接收机-发射机传播路径折线图}
\end{figure}

\subsection{6.2 规律的阐述}

在对规律进行阐述的过程中往往涉及对传播路径数量的比较问题,此时我们可以采用控制变量的方法,首先固定某一接收机(或发射机)位置不变,比较各发射机(接收机)与之组合时的传播路径数量。各发射机与接收机位置的布置情况见图 6-2-1 所示。

\begin{figure}[h]
    \centering
    \includegraphics[width=\textwidth]{image.png}
    \caption{发射机与接收机位置布置情况图}
    \label{fig:6-2-1}
\end{figure}

\subsubsection{6.2.1 以发射机为研究对象对规律进行阐述}

首先我们以发射机为研究对象,通过观察比较不同位置发射机的传播路径多少来对规律进行阐述。

读图分析:观察图 6-1-1,不难发现发射机 T3、T4 的传播路径数较多。发射机 T1、T2、T5 的传播路径数相比之下较少,且三者的路径数量基本相同。随着接收机位置的变化,各个发射机传播路径数量的折线图形状、走向基本一致。

将观察到的现象及分析原因阐述如下:

(1) 选择 T1、T2 为观察对象,T2 比 T1 更靠近通往接收机的出口,但由折线图知 T1 的传播路径数均比 T2 略大,分析原因为:靠近 T1 建筑物的反射边比 T2 的反射边更长,因此可将更多信号经建筑物边反射传播出去。

(2) 选择 T1、T5 为观察对象,T5 比 T1 更靠近通往接收机的出口,且 T5 周围的绕射点比 T1 多得多,由折线图知 T1 的传播路径数均比 T5 略大。分析原因可能为:在狭长的街道中,较长的建筑物反射边比绕射点对信号的传播贡献更大。

(3) 选择 T2、T3、T4 为研究对象,可观察到除接收机 R1 处,其余各处

T3 的传播路径数均大于 T4,T4 的传播路径数均明显大于 T2。观察到 T3 处于交叉路口处,T4 处于丁字路口处,T2 处于狭长路段中。分析原因为:宽敞、视野性较好的交叉路口比狭长、封闭的路段对信号有效传播的贡献更大。

总结出的基本规律阐述如下:

(1)狭长路段中,发射机应合理布置在道路深处以充分利用两侧建筑物的反射边使信号通过反射有效传播。

(2)建筑群中,较长的反射边比绕射点更有利于信号的有效传播。

(3)建筑群中,将发射机布置在交叉路口处会大幅提高信号传播的效率。

\subsubsection{6.2.2 以接收机为研究对象对规律进行阐述}

我们以接收机为研究对象,通过观察比较不同位置接收机的传播路径多少来对规律进行阐述。

读图分析:观察图 6-1-2,不难发现接收机 R2 的传播路径数较多。接收机 R1、R3、R4、R5 的传播路径数相比之下较少,且三者的路径数量基本相同。随着发射机位置的变化,各个接收机传播路径数量的折线图形状、走向基本一致。

将观察到的现象及总结出的基本规律阐述如下:

很明显,通过读图可以发现 R2 处于交叉路口处,且其传播路径数远大于其他接收机布置位置。

总结出的基本规律为:建筑群中,将接收机布置在交叉路口处会大幅提高信号传播的效率。布置在其他位置一般不会对信号传播造成较大影响。

除了上述内容,我们还另外补充一点规律。如图 6-2-2 所示。注意到绿色标记处与黄色标记处不存在有效传播信号。

总结出的规律为:不规则建筑物表面或过于狭窄的通道处不利于信号的有效传播,会“吞噬”信号。

\begin{figure}[h]
    \centering
    \includegraphics[width=\textwidth]{image.png}
    \caption{图 6-2-2 补充规律示意图}
\end{figure}

\subsection{6.3 规律的解释}

前面我们针对问题二的仿真结果归纳了几点规律,下面我们将对规律提出自己的解释。为了方便说明,我们附上问题二中最多传播路径和最少传播路径的收发机组合二维位置图,如图 6-3-1 与图 6-3-2 所示。

\begin{figure}[h]
    \centering
    \includegraphics[width=0.45\textwidth]{image1.png}
    \caption{最多传播路径图 (T3R2)}
    \label{fig:6-3-1}
\end{figure}
\begin{figure}[h]
    \centering
    \includegraphics[width=0.45\textwidth]{image2.png}
    \caption{最少传播路径图 (T5R5)}
    \label{fig:6-3-2}
\end{figure}

观察图 6-3-1 发现,T3 与 R2 均布置在交叉路口处,且经过仿真计算得到其传播路径数是最多的,这也验证了上一节总结的规律。另一方面,由于之前我们以可见性判据为基础在问题一中建立了树结构模型,此处 T3 与 R2 的可见元素(可见点与可见边)数量必然影响到其有效信号传播的数量,因此我们假设影响到收发机组合传播路径数多少的根本因素是发射机与接收机的可见元素数量。

利用 Matlab 程序可以一一找到五个发射机与五个接收机位置处的可见元素数量,将数据进行整理如表 6-3-1 所示。

\begin{table}[h]
    \centering
    \caption{收发机可见元素数量表}
    \label{tab:6-3-1}
    \begin{tabular}{|c|c|c|c|c|c|c|c|c|c|}
        \hline
        发射/接收机 & T1 & T2 & T3 & T4 & T5 & R1 & R2 & R3 & R4 & R5 \\
        \hline
        可见元素个数 & 18 & 18 & 38 & 22 & 27 & 21 & 39 & 26 & 25 & 25 \\
        \hline
    \end{tabular}
\end{table}

通过对表中数据的分析可知,T3 的可见元素为 38,R2 的可见元素为 39,T5 的可见元素为 27,R5 的可见元素为 25。因此可以看到 T3 与 R2 的可见元素数远大于 T5 与 R5,这也验证了我们对规律的解释。根据这一方法,我们还可以用其他组合方式进行验证,在此不做赘述。

\section{七、问题四的建模、求解}

本题是在前三问的求解基础上——已知 25 种组合每种情况的信号主要传播路径情况下,需要对发射信号、信号各种传播机制和多波干涉形成的接收信号建立数学模型,然后对每种组合每条路径信号的电场强度衰减、相位变化等情况进行分析求解,再对由于不同路径传播而造成干涉现象得到的信号振幅 $|Y(w,t)|$ 进行统计学分析。

\subsection{7.1 多波干涉信号问题}

\subsubsection{7.1.1 模型的建立}

信号经发射机发射后,均匀向四周发射,信号在传播过程中,主要有三种传播机制:反射、绕射和视距传播。为降低计算难度,假设本文求解的渥太华市区部分区域二维视图内的所有建筑物为理想电介质。

\paragraph{(1) 发射信号模型}

为对信号进行具体的场强衰减计算和能量估计,需对信号源进行模型假设,假设信号源(发射机)发出的信号场强为简谐正弦波,表达式为:
\begin{equation}
E_{0}(w,t) = A_{0} \cos(wt + \varphi_{0})
\tag{7-1-1}
\end{equation}
使用复基带符号,可以写成:
\begin{equation}
E_{0}(w,t) = A_{0} e^{(wt + \varphi_{0})}
\tag{7-1-2}
\end{equation}
式中,$A_{0}$ 表示发射信号的场强振幅,$w$ 表示发射信号的频率,$t$ 表示时间,$\varphi_{0}$ 表示初始相位,本文考虑发射的信号初始相位均为零。用复数表示时,场的实部 $\operatorname{Re}\{E\}$ 就等于场强在时间 $t$ 时的瞬时值。

\paragraph{(2) 场强衰减模型}

\textbf{a、视距传播情况}

\begin{figure}[h]
    \centering
    \begin{tikzpicture}[scale=0.8]
        % (a) 两点之间无阻挡
        \begin{scope}[shift={(-5,0)}]
            \filldraw (0,0) circle (2pt) node[below] {0};
            \filldraw (5,0) circle (2pt) node[below] {2};
            \draw[->, thick] (0,0) -- (5,0);
            \node at (2.5,-1.5) {(a)};
        \end{scope}

        % (b) 两点之间有阻挡
        \begin{scope}[shift={(5,0)}]
            \filldraw (0,0) circle (2pt) node[below] {0};
            \filldraw (5,0) circle (2pt) node[below] {2};
            \filldraw (4,-2) circle (2pt) node[below] {1};
            \draw[->, thick] (0,0) -- (5,0);
            \draw[dashed, thick] (0,0) -- (4,-2) -- (5,0);
            \node at (2.5,-1.5) {(b)};
        \end{scope}
    \end{tikzpicture}
    \caption{视距传播图示}
    \label{fig:7-1-1}
\end{figure}

视距传播包括两种情况:(a) 为两点之间无任何阻挡物直接能通视;(b)

电磁波能够从发射点到接收点,只经过建筑物表面反射;如图 7-1-1 所示(a)和(b)两种情况从发射点 0 到接收点 2 都视为视距传播。发射机发出信号后,信号是向四周传播的,三维情况下是成球形膨胀,二维情况下是成圆形扩散,故在本文处理问题时,信号场强的大小是与传播的距离成反比的。场强关系可表示为:

\begin{equation}
E_{2} = E_{0} D_{l}
\tag{7-1-3}
\end{equation}

式中,k 为波数,其计算公式为 \( k = 2\pi / \lambda \)(式中 \(\lambda\) 为波长),\( E_{0} \) 为发射点的电场强度,\( E_{2} \) 为接收到的信号的电场强度,\( D_{l} \) 为视距传播衰减系数。

\begin{equation}
D_{l} = \frac{e^{-jkr}}{r}
\tag{7-1-4}
\end{equation}

\( r \) 为传播的路径长度,图 7-1-1(a)中即为 0 点到 2 点的直线距离;(b)为从 0 点到 2 点相对于建筑物表面的镜像点 1 的直线距离。

当公式中 \( r \leq 1 \, \text{m} \) 时,发现接收到的电场强度没有减小反而增大了,这与实际情况不符,因此在 \( r_{02} \leq 1 \, \text{m} \) 时此公式不适用,且从远及近越接近传播距离为 \( 1 \, \text{m} \) 时求得的电场强度和实际电场强度相差越大。微蜂窝环境下实际情况发射机要保证信号能量的有效传递,不会安装在建筑物近处,而是尽可能的与周围建筑物保持一定距离,至少几米,十几米甚至几十米,这时候公式能够满足实际电场强度的衰减情况。

\textbf{b、反射传播情况}

当电磁波传播到不同介质交接处时,会产生反射。反射时传播路径,如图 7-1-2(a) 所示,产生反射时入射角 \( \theta_{i} \) 等于反射角 \( \theta_{r} \)。

\begin{figure}[h]
    \centering
    \includegraphics[width=0.8\textwidth]{reflection_scattering_diagram.png}
    \caption{反射和绕射传播机制下的传播规律示意图}
    \label{fig:reflection_scattering}
\end{figure}

图 7-1-2 反射和绕射传播机制下的传播规律示意图

当电磁波传播到理想电介质表面,80\% 的能量按照如图 7-1-2(a)所示的反射传播规律被反射出来,其余能量进入新介质继续传播(即发生透射,透射到建筑物的能量不再考虑,原因是建筑物结构复杂,透射波进入后,会发生多次的反射或透射,待到电磁波从建筑物内传播出来时能量已经衰减到很小了)。反射前后能量关系可表示为:

\begin{equation}
E_{r} = E_{in} D_{r}
\tag{7-1-5}
\end{equation}

式中,$E_{in}$ 表示电磁波刚到达建筑物时的电场强度,$E_{r}$ 表示电磁波反射结束后刚刚离开介质表面时的电场强度,$D_{r}$ 为反射系数,$D_{r} = 0.8e^{-j\pi}$。

\textbf{c、绕射传播情况}

绕射是指在电磁波传播路径上,当电波被尺寸较大(与波长相比)的障碍物遮挡时,电磁波改变传播方向的现象。电磁波在如图 7-1-2 (b) 中建筑物顶点(可称为劈)发生绕射时的能量关系式可表示为:

\begin{equation}
E_{d} = E_{in} D_{d}
\tag{7-1-6}
\end{equation}

式中,$E_{d}$ 表示绕射后绕射波强度($E_{d}$ 与绕射后接收点的位置及绕射后电磁波的传播方向与墙面的夹角有关系),$E_{in}$ 表示电磁波刚到达建筑物时的电场强度,$D_{d}$ 为绕射系数。

\begin{figure}[h]
    \centering
    \includegraphics[width=\textwidth]{image1.png}
    \caption{三维情况下劈的绕射示意图}
    \label{fig:3d_diffraction}
\end{figure}

\begin{figure}[h]
    \centering
    \includegraphics[width=0.5\textwidth]{image2.png}
    \caption{二维情况下劈的绕射示意图}
    \label{fig:2d_diffraction}
\end{figure}

为了求解绕射后电磁波的能量变化情况,需要计算绕射系数 $D$。由三维情

况下劈的绕射情况如图 7-1-3 所示,绕射系数的计算公式 \cite{ref6}:

\begin{equation}
D_{s,h} = \frac{-e^{-j\pi/4}}{2n\sqrt{2\pi k}\sin\beta_0} \cdot \left\{ \cot\left(\frac{\pi+\beta^-}{2n}\right)F(kLa^+(\beta^-)) + \cot\left(\frac{\pi-\beta^-}{2n}\right)F(kLa^-(\beta^-)) \right.
\end{equation}
\begin{equation}
\left. - \left[ \cot\left(\frac{\pi+\beta^+}{2n}\right)F(kLa^+(\beta^+)) + \cot\left(\frac{\pi-\beta^+}{2n}\right)F(kLa^-(\beta^+)) \right] \right\}
\tag{7-1-7}
\end{equation}

在二维假设下:

(1) 公式中的 $\sin\beta_0$ 的 $\beta_0 = 90^\circ$,因此 $\sin\beta_0 = 1$。

(2) $D_s$:软边界绕射系数(入射电场平行于 $\hat{s}$ 和构成的平面 $\hat{e}$),$D_h$:硬边界绕射系数(入射电场平行于 $\hat{s}$ 和构成的平面 $\hat{e}$),$\hat{I} = -\hat{s}$,$\hat{I}$ 是入射方向单位矢量,$\hat{e}$ 绕射点处的切线。对题目中的入射电场进行了一个预先判断,应该采用软边界绕射系数。

因此,根据 (1)(2) 两种情况得到二维假设下的绕射系数(题目中所给情况):

\begin{equation}
D_d = \frac{-e^{-j\pi/4}}{2n\sqrt{2\pi k}} \cdot \left\{ \cot\left(\frac{\pi+\beta^-}{2n}\right)F(kL\alpha^+(\beta^-)) + \cot\left(\frac{\pi-\beta^-}{2n}\right)F(kL\alpha^-(\beta^-)) \right.
\end{equation}
\begin{equation}
\left. - \left[ \cot\left(\frac{\pi+\beta^+}{2n}\right)F(kL\alpha^+(\beta^+)) + \cot\left(\frac{\pi-\beta^+}{2n}\right)F(kL\alpha^-(\beta^+)) \right] \right\}
\tag{7-1-8}
\end{equation}

式中,$L$ 是绕射点到场点之间的距离,$n = 2 - \alpha_3/\pi$($\alpha_3$ 的定义如图 7-1-4 所示,单位弧度),$F(x)$ 是用来修正 Keller 非一致性解的过渡函数,它是菲涅尔积分的一种变形,它的定义是

\begin{equation}
F(x) = 2j\sqrt{x}\exp(jx)\int_{\sqrt{x}}^{+\infty} \exp(-j\tau^2)d\tau
\tag{7-1-9}
\end{equation}

过渡函数定义域在 $(0, +\infty)$ 上,

(1) 当 $0 \leq x < 0.001$ 时,过渡函数可近似写成

\begin{equation}
F(x) = \left( \sqrt{\pi x} - 2xe^{j\pi/4} - \frac{2}{3}x^2e^{-j\pi/4} \right)e^{j(\pi/4+x)}
\tag{7-1-10}
\end{equation}

(2) 当 $x > 10$ 时,过渡函数可近似写成

\begin{equation}
F(x) \approx 1 + j\frac{1}{2x} - \frac{3}{4x^2} - j\frac{15}{8x^3} + \frac{75}{16x^4}
\tag{7-1-11}
\end{equation}

(3) 当 $0.001 \leq x \leq 10$ 时,需要直接计算 (7-1-8) 式中的积分。由于积分的上限是 $+\infty$,难以用数值方法实现。可以用公式:

\begin{equation}
\int_{-\infty}^{\infty} \exp(-au^2 + 2bu)du = \sqrt{\frac{\pi}{a}}\exp\left(\frac{b^2}{a}\right)
\tag{7-1-12}
\end{equation}

可以得出:

\begin{align*}
F(x) &= 2j\sqrt{x}\exp(jx)\left(\int_{0}^{+\infty}e^{-j\tau^2}d\tau - \int_{0}^{\sqrt{x}}e^{-j\tau^2}d\tau\right) \\
&= 2j\sqrt{x}\exp(jx)\left(\frac{\sqrt{\pi}}{2}e^{-j\pi/4} - \int_{0}^{\sqrt{x}}e^{-j\tau^2}d\tau\right) \tag{7-1-13} \\
\alpha^{\pm}(\beta) &= 2\cos^2\left(\frac{2n\pi N^{\pm} - \beta}{2}\right) \tag{7-1-14}
\end{align*}

其中, $\beta = \beta^{\pm} = \alpha_2 \pm \alpha_1$ ($\alpha_1$ 和 $\alpha_2$ 分别为入射角和绕射角, 其定义如图 7-1-4 所示, 这两个角的值以劈上任一边为参考)。

$N^{\pm}$ 是最接近满足下列方程的整数
\begin{align}
2n\pi N^+ - \beta &= \pi \tag{7-1-15} \\
2n\pi N^- - \beta &= -\pi \tag{7-1-16}
\end{align}

\paragraph{(3) 多波干涉下接收信号模型}

两个或两个以上的波相遇时, 在一定情况下会相互影响, 这种现象叫干涉现象。声波、光波和其他电磁波等都有此现象。

对于问题二中提到的 CD 路径上的所有接收点, 从发射机出发的电波都有多个传播途径 (可能是多次反射传播、多次绕射传播或是反射与绕射的任意组合传播) 到达这些点, 这些频率相同、振动方向相同、初相位相同的简谐波(即正弦波)在各个接收点相遇叠加, 出现某些接收点振动始终加强、而在另一些接收点振动始终减弱的现象, 形成多波干涉。

针对多波干涉现象, 建立经多波干涉信号的数学模型, 每条路径上场强表达式为:
\begin{equation}
f_i = E_i = E_0 D_i \tag{7-1-17}
\end{equation}
那么有
\begin{equation}
A_i = A_0 |D_i| \tag{7-1-18}
\end{equation}
所有路径干涉叠加后的场强为:
\begin{equation}
f = \sum_{i=1}^{Q} f_i = \sum_{i=1}^{Q} E_0 D_i \tag{7-1-19}
\end{equation}
式中, $Q$ 为到达某接收点的传播途径总数; $D_i$ 为第 $i$ 条传播路径上总的衰减系数; $E_i$ 为第 $i$ 条传播路径上到达接收机时信号的场强; $A_i$ 为第 $i$ 条传播路径上到达接收机时信号的振幅; $f$ 为所有路径信号干涉叠加后的场强。

说明: $D_i$ 与具体传播路径有关, 信号从发射源发出, 沿不同传播路径经过视距传播及多次反射、绕射, 其总衰减系数 $D_i$ 为各情况衰减系数的乘积。

\subsubsection{7.1.2 多波干涉信号的求解}

前三个问题中,我们已经找到了每种组合的信号传播路径,依照上面理论分析建立的相关模型,对照公式(7-1-17)可求得信号经不同路经到达的分场强,然后根据公式(7-1-19)再求和,即可得到接收机收到的信号场强情况。

针对发射机和接收机组合为 T1R1,选取频率为 2000MHz 的电磁波,发射机场强设为 10,发射机发射该简谐正弦波信号如图 7-1-5,在接收机收到的信号如图 7-1-6 所示。

\begin{figure}[h]
    \centering
    \includegraphics[width=\textwidth]{image1.png}
    \caption{发射机发射的信号波图}
    \label{fig:7-1-5}
\end{figure}

\begin{figure}[h]
    \centering
    \includegraphics[width=\textwidth]{image2.png}
    \caption{接收机发射的信号波图}
    \label{fig:7-1-6}
\end{figure}

由图可以看出,发射机发射的信号经过多种不同路径到达接收机后叠加,接收到的信号为一稳定的多波干涉信号,该信号保持原有信号的频率,其幅值发生衰减,相位发生变化。

\subsubsection{7.1.3 模型小结}

由上面的理论分析建模及仿真求解可以得出结论:

(1)对于某一组合,发射机和接收机位置固定后,信号的传播路径就确定了,同时对于每种传播路径的信号衰减系数也确定了。

(2)在本题假设下,场强的衰减情况只与路径有关,收发机位置固定后,

接收机收到信号的强度也定了,与频率无关。

(3) 对于同一种频率,接收机接收到经过不同路径收到的信号叠加干涉后形成稳定的场。接收到的多波干涉信号只与频率和相位有关系。

\subsection{7.2 统计特性分析}

\subsubsection{7.2.1 统计特性分析模型}

在某一发射机和接收机组合内,对于多波干涉场强求和公式 \( f \) 有随机变量 \( f_i \),随时间的变化而呈现周期性的变化,由于路径的不同造成振幅的衰减和相位不同,任选一时刻 \( t \),得到各条路径到达接收机的信号强度 \( f_i \),可求所有路径的到达接收机时信号强度的期望:

\[
E(f) = \frac{1}{Q} \sum_{i=1}^{Q} f_i \tag{7-2-1}
\]

不同组合可求得不同的期望,例如 \( E(f^j) \) 表示第 \( j \) 种组合信号经不同路径到达接收机时信号强度的期望。信号经过不同路径到达接收机时信号的场强也往往是不同的,对所有的路径,可以统计出现在不同场强区间的路径数,进而可得到接收机收到信号的场强概率密度分布情况 \({}^{[7]}\)。

为进一步查看各路径信号场强的相对于期望的偏离程度,对各路径信号进行二阶中心矩分析,表达式为:

\[
\mu = E(f_i - E(f))^2 \tag{7-2-2}
\]

对于不同种组合之间,比如同一接收机收到来自不同发射机发射过来的信号,接收机在接收信号时来自不同发射机的信号之间会产生影响。设两种组合接收到的信号分别为 \( f^1 \) 和 \( f^2 \),探讨对于不同组合信号的沿着各自传播路径传播时信号 \( f_i^1 \) 和 \( f_j^2 \) 之间的关系,可用两两路径信号场强的乘积 \( f_i^1 f_j^2 \) 来表示各路径间传播信号的关系,并分别求出相对应的概率,由期望公式求得:

\[
E(f^1 f^2) = \sum_{i}^{Q_1} \sum_{j}^{Q_2} f_i^1 f_j^2 P_{f_i^1 f_j^2} \tag{7-2-3}
\]

式中,\( P_{f_i^1 f_j^2} = P_{f_i^1} P_{f_j^2} \),\( i = 1, 2, \ldots, Q_1 \); \( j = 1, 2, \ldots, Q_2 \)。

在同一收发机组合内,变量场强 \( f \) 方差求解公式为:

\[
D(f) = E(f^2) - (E(f))^2 \tag{7-2-4}
\]

两种不同组合协方差表示为:

\[
\text{cov}(f^1, f^2) = E(f^1 f^2) - E(f^1) E(f^2) \tag{7-2-5}
\]

由公式 (7-2-1)、(7-2-2) 和 (7-2-4) 可求得不同收发机组合间干涉信号的相关性 \({}^{[8]}\):

\[
\rho(f^1, f^2) = \frac{\text{cov}(f^1, f^2)}{\sqrt{D(f^1)} \sqrt{D(f^2)}} \tag{7-2-6}
\]

\subsubsection{7.2.2 统计特性分析}

选取频率为 \(2000 \mathrm{MHz}\) 的电磁波,发射机场强设为 10,时间 \(t\) 取 0 时刻,对于 25 种发射机和接收机组合内不同传播路径到达接收机时信号场强的期望和方差分别列于表 7-2-1 和表 7-2-2。

\textbf{表 7-2-1 25 种发射机和接收机组合接收信号场强的期望}

\begin{tabular}{c c c c c c}
\hline
\multirow{2}{*}{\begin{tabular}{c}
接收机 \\
发射机
\end{tabular}} & \multicolumn{5}{c}{R1 R2 R3 R4 R5} \\
\cline{2-6}
 & & & & & \\
\hline
T1 & 0.0153 & 0.0405 & -0.0771 & -0.0831 & 0.0777 \\
T2 & 0.0029 & -0.0127 & 0.0098 & -0.1303 & 0.0312 \\
T3 & 0.0023 & -0.0164 & -0.0009 & -0.2580 & 0.0196 \\
T4 & 0.0005 & -0.0162 & -0.0149 & 0.1172 & 0.0231 \\
T5 & 0.0062 & -0.0012 & 0.0100 & -0.0805 & 0.0197 \\
\hline
\end{tabular}

\textbf{表 7-2-2 25 种发射机和接收机组合接收信号场强的方差}

\begin{tabular}{c c c c c c}
\hline
\multirow{2}{*}{\begin{tabular}{c}
接收机 \\
发射机
\end{tabular}} & \multicolumn{5}{c}{R1 R2 R3 R4 R5} \\
\cline{2-6}
 & & & & & \\
\hline
T1 & 0.0338 & 0.1122 & 0.2002 & 1.7791 & 0.4655 \\
T2 & 0.0103 & 0.0114 & 0.0237 & 2.9057 & 0.0736 \\
T3 & 0.0110 & 0.0112 & 0.0107 & 6.9523 & 0.0464 \\
T4 & 0.0115 & 0.0115 & 0.0344 & 2.5428 & 0.0490 \\
T5 & 0.0052 & 0.0053 & 0.0973 & 1.2137 & 0.0747 \\
\hline
\end{tabular}

由 T3 发射信号,R1、R2、R3、R4、R5 接收信号时,五种组合的概率密度分布如图 7-2-1、7-2-2、7-2-3、7-2-4、7-2-5 所示。

\begin{figure}[h]
\centering
\includegraphics[width=0.45\textwidth]{image1.png}
\caption{T3-R1 接受信号的概率分布}
\end{figure}

\begin{figure}[h]
\centering
\includegraphics[width=0.45\textwidth]{image2.png}
\caption{T3-R2 接受信号的概率密度分布}
\end{figure}

\begin{figure}[h]
\centering
\includegraphics[width=0.45\textwidth]{image3.png}
\caption{T3-R3 接受信号的概率分布}
\end{figure}

\begin{figure}[h]
\centering
\includegraphics[width=0.45\textwidth]{image4.png}
\caption{T3-R4 接受信号的概率密度分布}
\end{figure}

\begin{figure}[h]
    \centering
    \includegraphics[width=0.6\textwidth]{image.png}
    \caption{T3-R5 接受信号的概率分布}
    \label{fig:signal_distribution}
\end{figure}

信号间的相关性主要考虑接收到来自不同发射机的信号间的相关性,表 \ref{tab:correlation_coefficients} 列出的是 T1、T2、T3、T4、T5 发出的信号被 R2 接收,两两发射机到达接收机信号的相关性。

\begin{table}[h]
    \centering
    \caption{25 种发射机和接收机组合接收信号场强的期望}
    \label{tab:correlation_coefficients}
    \begin{tabular}{c c c c c c}
        \hline
        相关系数 & T1R2 & T2R2 & T3R2 & T4R2 & T5R2 \\
        \hline
        T1R2 & 1.000 & 0.652 & 0.021 & 0.005 & 0.032 \\
        T2R2 & 0.652 & 1.000 & 0.069 & 0.012 & 0.068 \\
        T3R2 & 0.021 & 0.069 & 1.000 & 0.005 & 0.046 \\
        T4R2 & 0.005 & 0.012 & 0.005 & 1.000 & 0.003 \\
        T5R2 & 0.032 & 0.068 & 0.046 & 0.003 & 1.000 \\
        \hline
    \end{tabular}
\end{table}

因为幅值样本选取某一特定时刻接收机的瞬时值,之前分析得出了路径传播对信号频率没有影响,因此对样本值的影响因素包括衰减程度和相位变化。上表分析相关系数的变化规律,T1 和 T2 处于同一个巷道之内,相关系数最高,而其它的情况都远低于这一值,说明信号在同巷道传播的时候相关性变化较小,路口虽然有利于信号的传播,却会产生较大的衰减和相位变化。

\section{八、问题五的建模、求解}

本问题是在问题四的基础上增加了波频率这一变量,由此出现了更加复杂的多频宽带干涉现象。

\subsection{8.1 宽带多频干涉}

\subsubsection{8.1.1 数学模型建立}

(1) 发射信号模型

为对信号进行具体的场强衰减计算和能量估计,需对信号源进行模型假设,假设信号源(发射机)发出的信号场强为简谐正弦波,表达式为:
\begin{equation}
\overline{E}_{0}(w_{i}, t)=\overline{A}_{0} \cos \left(w_{i} t+\varphi_{0}\right)
\tag{8-1-1}
\end{equation}
使用复基带符号,可以写成:
\begin{equation}
\overline{E}_{0}\left(w_{i}, t\right)=\overline{A}_{0} e^{\left(w_{i} t+\varphi_{0}\right)}
\tag{8-1-2}
\end{equation}
式中,$A_{0}$ 表示发射信号的场强振幅,$w_{i}$ 表示发射信号的频率,$t$ 表示时间,$\varphi_{0}$ 表示初始相位,本文考虑发射的信号初始相位均为零。用复数表示时,场的实部 $\operatorname{Re}\left\{\overline{E}\right\}$ 就等于场强在时间 $t$ 时的瞬时值。

(2) 场强衰减模型

此模型参照问题四中场强衰减模型的建立。

(3) 多频干涉模型

两个或两个以上的不同频率的波相遇时,会相互影响,形成复杂的宽带多波干涉现象。声波、光波和其他电磁波等都有此现象。

对于问题二中提到的 CD 路径上的所有接收点,从发射机出发的电波都有多个传播途径(可能是多次反射传播、多次绕射传播或是反射与绕射的任意组合传播)到达这些点,这些频率不同、振动方向相同、初相位相同的简谐波(即正弦波)在各个接收点相遇叠加,形成较为复杂的波形,出现包络现象,即为宽带多波干涉。

针对宽带多波干涉现象,建立宽带多波干涉信号的数学模型,不同频率信号每条路经上场强表达式为:
\begin{equation}
\overline{f}_{i}=\overline{E}_{i}=\sum_{j=1}^{J} \overline{E}_{0} \overline{D}_{i}
\tag{8-1-3}
\end{equation}
那么有

\begin{equation}
\overline{A}_{i}=\overline{A}_{0}\left|\overline{D}_{i}\right|
\tag{8-1-4}
\end{equation}

所有路径干涉叠加后的场强为:

\begin{equation}
\overline{f}=\sum_{i=1}^{\overline{Q}} \overline{f}_{i}=\sum_{i=1}^{\overline{Q}} \overline{E}_{0} \overline{D}_{i}
\tag{8-1-5}
\end{equation}

式中, \(J\) 表示波含有的频率个数; \(\overline{Q}\) 为到达某接收点的传播途径总数; \(\overline{D}_{i}\) 为第 \(i\) 条传播路径上总的衰减系数; \(\overline{E}_{i}\) 为第 \(i\) 条传播路径上所有频率波到达接收机时的场强; \(\overline{A}_{i}\) 为第 \(i\) 条传播路径上所有频率波到达接收机时信号的振幅; \(\overline{f}\) 为所有路径信号干涉叠加后的场强。

说明: \(\overline{D}_{i}\) 与具体传播路径和波频率有关, 含有多种频率的信号从发射源发出, 沿不同传播路径经过视距传播及多次反射、绕射, 其总衰减系数 \(\overline{D}_{i}\) 为各情况衰减系数的乘积。

\subsubsection{8.1.2 宽带多波干涉求解}

发射机发出不同频率的信号, 经过不同的路径到达接收机, 形成宽带多波干涉现象。对照公式 (8-1-3) 可求得不同频率的信号经不同路经到达接收机的场强, 然后根据公式 (8-1-5) 再求和, 即可得到接收机收到的信号场强情况。

针对问题一中发射机和接收机位置配属, 发射频率为 \(2000 \mathrm{MHz}\) 到 \(2100 \mathrm{MHz}\), 间隔近似为 \(1 \mathrm{MHz}\) 共 101 个电磁波, 发射机场强设为 10 , 经过发射机发射后, 在接收机收到的信号如图 8-1-1 所示。

\begin{figure}[h]
\centering
\includegraphics[width=\textwidth]{image.png}
\caption{接收机接收到的信号}
\end{figure}

从图中可以看出, 宽带多波干涉信存在着较为复杂的干涉, 出现明显的包络现象。

\subsection{8.2 包络统计特性}

本问题中对于合成波的包络特性进行分析, 在同一频率、不同路径信号之

间的相关性分析和问题四的分析思路基本一致,因此重点在于同一路径、不同频率信号之间的相关性分析上。

发射机发出一些列功率相同、频率不同、初始相位相同的电磁波,各种频率的波经过不同的路径到达接收机。在同一路径中,不同频率的信号到达接收机时因相位不同而导致场强不同(不同频率信号沿同一路径衰减后场强的振幅是相同的)。对不同频率信号到达接收机场强求期望:

\begin{equation}
E(\overline{f}) = \frac{1}{J} \sum_{j=1}^{J} \overline{f}_i
\tag{8-2-1}
\end{equation}

对于不同频率,到达接收机时的场强是不一样的,可以统计出在不同场强区域不同频率波出现的次数,进而得到接收机机收到信号的场强概率密度分布情况。

为进一步分析各频率信号场强的相对于期望的偏离程度,对各频率信号进行二阶中心矩分析,表达式为:

\begin{equation}
\overline{\mu} = E(\overline{f}_i - E(\overline{f}))^2
\tag{8-2-2}
\end{equation}

\section{参考文献}

[1] 李清,基于射线跟踪模型的室外微小区电波传播预测研究,西安电子科技大学硕士论文:1-5,2011。

[2] 姚翠萍,基于加速多镜像法对城市微蜂窝小区电波传播的分析计算,电子科技大学硕士论文:2-4,2007。

[3] 刘斐,基于射线追踪法的电波特性研究,煤炭技术,2010年第五期:210-211,2010。

[4] 廖斌等,基于虚拟源树的射线跟踪算法的研究,华东师范大学学报,2008年第三期:103-105,2008。

[5] 顾晓龙等,利用可见性概念改进基于镜像原理的射线追踪法,电波科学学报,2001年第四期:464-467,2001。

[6] ROBERT G. KOUYOUmjIAN, A Uniform Geometrical Theory of Diffraction for an Edge in a Perfectly Conducting Surface, Proceedings of the IEEE Vol. 62 No11: 1448-1451。

[7] 牛映武,运筹学,西安:西安交通大学出版社,1999。

[8] 田斌,任光亮,无线通信,北京:电子工业出版社,2009。

[9] 王福昌,熊兆飞,通信原理,北京:清华大学出版社,2006。

[10] 王晓鸥,万伟,大学物理学,北京:机械工业出版社,2009。

[11] 胡光书,数字信号处理,北京:清华大学出版社,2003。

[12] 高峰,高泽华,无线城市:电信级Wi-Fi网络建设与运营,北京:人民邮电出版社,2012。

[13] 胡建村,韦再雪,蜂窝网络高级规划与优化,北京:机械工业出版社,2012。

\end{document}