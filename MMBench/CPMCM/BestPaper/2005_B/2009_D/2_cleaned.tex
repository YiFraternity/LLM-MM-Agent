\title{全国第六届研究生数学建模竞赛}

\begin{document}

\maketitle

\begin{abstract}
本文针对 110 警车调度问题,引入了图论中的最短路算法以及计算几何的相关理论,建立了车辆调配模型、巡逻路线模型以及基于模糊数学的评价指标模型。另外,我们用 C++ 编写了一个可视化的软件,不仅实现了手动描点,自动求出覆盖线段集合的功能,同时利用计算机模拟警车的巡逻路线,最后通过计算机检验得到的结果,其合理性和实用性都令人满意。

针对问题一,通过人机结合,配置 17 辆警车就能实现 D1 的目标,很好地兼顾了警车巡逻的运行成本,减少公安部门车辆和人员等的投入。

针对问题二,采用模糊数学相关理论使评价指标实现了从定性到定量的过程。

针对问题三和六,在 D1 的基础上,兼顾了巡逻效果的显著性,采用最少被巡逻道路优先的贪心算法建立了动态巡逻模型,得到了合理的全区域巡逻调度方案。在此方案中,我们动用了 30 辆警车完成了问题一的目标,并且巡逻效果显著。同时额外考虑了案发事件概率不均匀分布的情况,建立了改进模型。

针对问题四,在完成问题三指标的基础上,为了尽可能得兼顾巡逻车辆的隐蔽性,使巡逻效果进一步提高,我们加入了随机性,采用轮盘赌算法进一步提高了隐蔽性。

针对问题五,采用最远距离道路优先的贪心策略,使模型尽可能的满足了条件 D1、D2。

针对问题七,提出了一些额外因素及其解决方案,进一步完善了模型,使模型更贴近现实。
\end{abstract}

\textbf{关键词:} 计算几何 \quad 最短路算法 \quad KMP 贪心算法 \quad 模糊数学

\section{全国第六届研究生数学建模竞赛}

\begin{center}
\includegraphics[width=0.5\textwidth]{image.png}
\end{center}

\section{题目 110 警车配置及巡逻方案}

\section{参赛队号 1038417}

\begin{tabular}{l l}
\textbf{队员姓名} & \textbf{林阳斌 陈碧黎 苏圳泷} \\

\end{tabular}

\section{110 警车配置及巡逻方案}

\subsection{1. 问题重述}

110 警车在街道上巡逻,既能够对违法犯罪分子起到震慑作用,降低犯罪率,又能够增加市民的安全感,同时也加快了接处警(接受报警并赶往现场处理事件)时间,提高了反应时效,为社会和谐提供了有力的保障。

考虑某城市内一区域,为简化问题,假定所有事发现场均在图中的道路上。该区域内有三个重点部位需要重点防范,给出了相应的三个坐标,在图中表示为红点部位,而蓝色部分为水域。城市的道路四通八达,整个城市交通图为一个连通图,顶点为路口,边为道路。相邻两个交叉路口之间的道路近似认为是直线。题目的数据给出了每一个交叉点的坐标和道路线段端点。

该城市拟增加一批配备有 GPS 卫星定位系统及先进通讯设备的 110 警车。设 110 警车的平均巡逻速度为 $20 \mathrm{~km} / \mathrm{h}$,接警后的平均行驶速度为 $40 \mathrm{~km} / \mathrm{h}$。警车配置及巡逻方案要尽量满足以下要求:

D1. 警车在接警后三分钟内赶到现场的比例不低于 $90\%$;而赶到重点部位的时间必须在两分钟之内;

D2. 使巡逻效果更显著;

D3. 警车巡逻规律应有一定的隐蔽性。

考虑以下问题:

一. 若要求满足 D1,则该区最少需要配置多少辆警车巡逻?

二. 给出评价巡逻效果显著程度的有关指标。

三. 给出满足 D1 且尽量满足 D2 条件的警车巡逻方案及其评价指标值。

四. 在第三问的基础上,再考虑 D3 条件,给出你们的警车巡逻方案及其评价指标值。

五. 如果该区域仅配置 10 辆警车,应如何制定巡逻方案,使 D1、D2 尽量得到满足?

六. 若警车接警后的平均行驶速度提高到 $50 \mathrm{~km} / \mathrm{h}$,考虑问题三。

七. 还需要考虑哪些因素、哪些情况?给出相应的解决方案。

\subsection{2. 问题的条件和假设}

\begin{enumerate}
    \item 重点区域不一定在道路上,而非重点区域的事发现场都在道路上
    \item 相邻两个交叉路口之间的道路近似认为是直线
    \item 110 警车的平均巡逻速度为 $20 \mathrm{~km} / \mathrm{h}$,接警后的平均行驶速度为 $40 \mathrm{~km} / \mathrm{h}$
    \item 接警后警车在 GPS 汽车导航中不会遇到堵车等其他非案发事件的突发事件,出行路线的路面状况是通畅的
    \item 110 警车的出行道路均是双行道
    \item 在同一时间内,若发生两起以上事故,其发生地点的距离比较远
    \item 当警车在巡逻的过程中接到任务后,将根据 GPS 卫星定位系统选择最近路线前往案发现场
\end{enumerate}

\subsection{3. 符号约定和名词解释}

\begin{itemize}
    \item $I_i$: 第 $i$ 个交叉点, $i=1,2,3,4\ldots$
    \item $R_i$: 第 $i$ 条道路, $i=1,2,3,4\ldots$
    \item $\text{Dis}(a,b)$: 点 $a$ 与点 $b$ 之间的欧式距离
    \item $\text{Imin}(i,j)$: 交叉点 $i$ 与交叉点 $j$ 之间的最短距离
    \item $\text{Rmin}(a,b)$: 道路上任意两点 $a$ 和 $b$ 的最短距离
    \item $\text{CS}(P,L)$: 距离点 $P$ 小于 $L$ 的线段集合
    \item $S$: 巡逻效果显著程度总评价指标值
    \item $A(p)$: 点 $p$ 事件响应时间, 表示地点 $p(i,j)$ 发生事件报警后, 警车到达现场所用的时间
    \item $SA$: 事件响应时间
    \item $B(p)$: 点 $p$ 事件响应能力值
    \item $H$: 隐蔽性评价指标值, 表示 110 警车的配置和巡逻方案的隐蔽程度
    \item $Z$: 震慑性评价指标值, 表示 110 警车的配置和巡逻方案的震慑作用
    \item $J$: 经济效益评价指标值, 表示警车配置所需资金(购买警车费用)及所有警车巡逻中所消耗的油费等开销对经济效益的影响程度
    \item $\overline{\text{EA}}$: 平均事件响应时间
    \item $N$: 动用的 110 警车数
    \item $C$: 权重向量
    \item $G$: 隶属度向量
    \item $U$: 区域权值向量
\end{itemize}

\section{4. 问题一建模}

\subsection{4.1 模型假设}

\begin{enumerate}
    \item 110 警车在巡逻和接警后的行驶过程中均保持匀速行驶
    \item 110 警车初始位置为固定点
    \item 当警车处理完其管辖区域的案件之后要返回原来的固定点
    \item 事件发生均匀地分布在每个点上
\end{enumerate}

\subsection{4.2 问题分析}

我们的主要任务是将配置有 GPS 卫星定位系统及先进通讯设备的 110 警车合理地分配到该城市中, 使得当案件发生时, 巡逻的警车能及时赶到案发现场并进行处理; 在该前提下, 求出满足条件所需要的最少警车数。

对于重点位置, 要求必须在 2 分钟内到达, 因此我们可以考虑一个距离重点位置路程为 2 分钟的点的集合; 为了描述方便, 我们定义覆盖点集合 $\text{CS}(p_i, L)$, 它的形式化的定义如下:
\[
\text{CS}(p_i, L) = \{ q_j \mid R_{\text{min}}(p_i, q_j) \leq L \}
\]
现在要满足到达重点位置 $p_i$ 的时间不超过 2 分钟, 只需令 $L = v * t = 1.333 \, \text{km}$,

然后任选 $\mathrm{CS}\left(p_{i}, \mathrm{~L}\right)$ 中的一个点作为警车的初始固定点,这样就能保证在 2 分钟内到达重点位置。

对于非重点位置,需要 90\% 以上的概率到达事发现场;由于已经固定了警车的初始位置,因而对于每个初始位置 $p_{i}$,计算 $\bigcup\left(\mathrm{CS}\left(p_{i}, \mathrm{~L}\right)\right)$,其中 $\mathrm{L}=2 \mathrm{~km}$ 即接警后 3 分钟内警车可以行驶的路程。如果地图上 90\% 的点都在 $\bigcup\left(\mathrm{CS}\left(p_{i}, \mathrm{~L}\right)\right)$ 内,则说明该方案满足条件。

现在要使得警车尽量少,即要令每辆警车的 $\mathrm{CS}\left(p_{i}, \mathrm{~L}\right)$ 尽可能大,而交集尽可能的少。在这里,把每辆巡逻警车的 $\mathrm{CS}\left(p_{i}, \mathrm{~L}\right)$ 划为自己的管辖区,然后让每辆警车在一个位置静止候命,因为这样能够比运动巡逻状态更好地控制管辖区,保证在管辖区内均能及时到达事件现场。由于我们已经假设,同时发生两起事件的概率非常小,并且如果同时发生,它们相隔的距离较远,即可以认为它们不在同一个警车的管辖区内,不会影响两个管辖区的警车调度。

综上所述,我们采用如下方案:将警车固定于点 $p_{i}$ 并以 $\mathrm{CS}\left(p_{i}, \mathrm{~L}\right)$ 作为其管辖范围,其中 $L=v * t$,表示警车以速度 $v$ 在 $t$ 分钟内能到达的点的集合。对于重点区域,题目要求在 2 分钟内到达,而接警后速度为 $40 \mathrm{~km} / \mathrm{h}$,因此 $L=v * t=1.333 \mathrm{~km}$;而对于非重点区域,要求在 3 分钟内到达,因此 $L=2 \mathrm{~km}$。

同时为了满足下一次事件发生时能以相同的概率处理事件,我们假定警车处理完该区域的案件之后立即返回原来的固定点 $P_{i}$。

我们采用人机结合的方式,用计算机求出其管辖区 $\mathrm{CS}\left(p_{i}, \mathrm{~L}\right)$,并结合人工分析找出警车的固定点 $p_{i}$ 来求得最少的警车数。

\subsection{4.3 模型建立与模型求解}

我们在 vs2008 平台下采用 C++ 语言编写了一个可以自动求出 $\mathrm{CS}\left(p_{i}, \mathrm{~L}\right)$ 的程序,用户能够根据鼠标点击城市的坐标来自动计算出该点(即警车)的覆盖点集合 $\mathrm{CS}\left(p_{i}, \mathrm{~L}\right)$ 并将覆盖区域以可视化形式描绘出来,然后采用人工的方法选取警车的位置,程序还提供了统计 $\bigcup\left(\mathrm{CS}\left(p_{i}, \mathrm{~L}\right)\right)$ 的长度与地图道路总长度的比值 $F$,如果 $F$ 大于等于 90%,则说明警车可以在指定分钟内到达地图 90% 的区域,也就是说能及时处理案件的概率大于 90%。需要解释的一点是,由于 $\mathrm{CS}\left(p_{i}, \mathrm{~L}\right)$ 中的点都是连续的,因此实际上 $\mathrm{CS}\left(p_{i}, \mathrm{~L}\right)$ 也是线段的集合,因此只需要累加 $\mathrm{CS}\left(p_{i}, \mathrm{~L}\right)$ 中每条线段的长度就可以得到 $\mathrm{CS}\left(p_{i}, \mathrm{~L}\right)$ 的总长度。

下列是我们的算法:

(1) 计算 $I_{\min }(i, j)$ 的方法:

采用 FLOYD 算法,在 $O\left(n^{3}\right)$ 的时间内计算出所有两两交叉点间的最短距离,并将其保存于矩阵 $I_{\min }(i, j)$ 中,其中 $n$ 为交叉点个数,$i$ 和 $j$ 表示交叉点的编号。

FLOYD 算法的伪代码如下:

\begin{verbatim}
For u = 0 to n
    For v = 0 to n
        For w = 0 to n
            If (I_min(v,u) + I_min(u,w) < I_min(v,w))
                do I_min(v,w) = I_min(v,u) + I_min(u,w)
\end{verbatim}

初始时,若 \(i\) 和 \(j\) 有道路相连则 \(I_{\min}(i,j) = \text{Dis}(i,j)\),否则令 \(I_{\min}(i,j)\) 为无穷大。

(2) 计算 \(R_{\min}(a,b)\) 的方法:

令 \(i\) 和 \(j\) 分别是点 \(a\) 和点 \(b\) 的编号

分情况讨论:
\begin{itemize}
    \item 若 \(a\) 和 \(b\) 都在同一条道路上,则 \(R_{\min}(a,b) = \text{Dis}(a,b)\)
    \item 若 \(a\) 和 \(b\) 都是道路交叉点,则 \(R_{\min}(a,b) = I_{\min}(i,j)\)
    \item 若 \(a\) 为交叉点,\(b\) 不是交叉点,则
  先求出 \(b\) 点所在道路 \(r\),设 \(r\) 的两端点分别为 \(a1\) 和 \(b1\),
  则 \(R_{\min}(a,b) = \min(I_{\min}(a,a1) + \text{Dis}(a1,b), I_{\min}(a,b1) + \text{Dis}(b1,b))\)
    \item 若 \(b\) 为交叉点,\(a\) 不是交叉点,则 \(R_{\min}(a,b) = R_{\min}(b,a)\)
    \item 若 \(a\) 和 \(b\) 都不是道路交叉点,则
  设 \(b\) 点所在的道路 \(r\) 的两个端点分别为 \(a1\) 和 \(b1\),
  \(R_{\min}(a,b) = \min(R_{\min}(a,a1) + \text{Dis}(a1,b), R_{\min}(a,b1) + \text{Dis}(b1,b))\)
\end{itemize}

(3) 计算 \(CS(p_i, L)\) 的方法:

\begin{itemize}
    \item 步骤一:计算地图上距离点 \(p\) 最近的点 \(O\),及点 \(O\) 所在的道路编号 \(r\)
  \begin{enumerate}
      \item 若 \(p\) 在道路 \(i\) 上,则 \(r = i\);
      \item 若 \(p\) 不在任何道路上,则遍历所有道路,寻找与 \(p\) 点欧式距离最近的点 \(O\)。
     寻找 \(p\) 点在道路 \(i\) 的最近点 \(q_i\) 的方法如下图 4-1 所示:
  \end{enumerate}
\end{itemize}

\begin{figure}[h]
    \centering
    \includegraphics[width=0.8\textwidth]{image.png}
    \caption{点 \(p\) 到道路的最近点 \(i\)}
    \label{fig:4-1}
\end{figure}

我们将道路 \(i\) 看作是一个线段 \(S_i\)

点 \(p\) 到线段 \(S_i\) 的最近的点 \(q_i\) 只可能有两种情况:
\begin{enumerate}
    \item 点 \(p\) 关于线段 \(S_i\) 的垂足在 \(S_i\) 上,则 \(q_i\) 即为垂足,如上图 4-1 左边显示。
    \item 若垂足不在线段 \(S_i\) 上,则认为点 \(q_i\) 为 \(S_i\) 两个端点中距离 \(p\) 较近的那个点,如上图 4-1 右边显示。
\end{enumerate}

定义 \(r\) 如下,该式子表示 \(r\) 等于使 \(\text{Dis}(p, q_i)\) 为最小值的 \(i\)
\[
r = \min_i \{\text{Dis}(p, q_i)\}
\]
最后,所有的最近点 \(O\) 即为 \(q_r\);

\begin{itemize}
    \item 步骤二:对每条道路 \( r_i \),计算 \( r_i \) 上距离 O 点长度为 \( L - \text{Dis}(P, O) \) 的线段集合,如图 4-2 所示:
\end{itemize}

\begin{figure}[h]
    \centering
    \includegraphics[width=0.8\textwidth]{image.png} % 替换为实际图片路径
    \caption{线段集合示意图}
    \label{fig:4-2}
\end{figure}

实线表示实际的道路,虚线表示连接两个交叉点的最短路径

定义道路 \( r_i \) 的两个端点为 \( a_i \) 和 \( b_i \)

分情况讨论:

先考虑 \( a_i \) 如下

\begin{itemize}
    \item 若 \( L - \text{R}_{\min}(O, a_i) > \text{Dis}(a_i, b_i) \),说明道路 \( r_i \) 与 O 的距离都小于 L,因此将线段 \( (a_i, b_i) \) 加入到 \( \text{CS}(P, L) \) 中
    \item 若 \( L - \text{R}_{\min}(O, a_i) \leq 0 \),则说明 i 无法到达点 \( a_i \),不再继续
    \item 若 \( 0 < L - \text{R}_{\min}(O, a_i) \leq \text{Dis}(a_i, b_i) \),则说明 \( r_i \) 至少有部分与 O 的距离小于 L
    \item 此时令 \( d = L - \text{Dis}(a_i, b_i) \),然后计算从 \( b_i \) 开始沿向量 \( \text{V}(a_i, b_i) \) 行走长度 \( d \) 所到达的点 A。点 A 的坐标可按如下公式计算:
\end{itemize}

令向量 \( \text{V}(a_i, b_i) \) 的单位向量为 \( \textbf{e} = \textbf{V} / |\textbf{V}| \),其中 \( |\textbf{V}| \) 表示向量的长度

则 \( \textbf{A} = a_i + \textbf{e} * d \)

最后 \( (a_i, A) \) 即为道路 \( r_i \) 被覆盖的集合,将 \( (a_i, A) \) 加入到 \( \text{CS}(P, L) \) 中,然后再用同样的方法考虑点 \( b_i \)。

这里有一个特殊情况要说明:

即对于 O 在道路 \( r_i \) 上的情况,且道路 \( r_i \) 的长度特别长,如图 4-3 所示

\begin{figure}[h]
    \centering
    \includegraphics[width=\textwidth]{image1.png}
    \caption{特殊情况}
    \label{fig:4-3}
\end{figure}

此时道路 $r_i$ 在 $CS(P, L)$ 的区域即为 $(A, B)$,其中
\begin{align*}
f'(x) &= \left(e^{-x^2/6}\right)' = -\frac{1}{3}xe^{-x^2/6} \\
f''(x) &= \left(-\frac{1}{3}xe^{-x^2/6}\right)' = -\frac{1}{3}\left(e^{-x^2/6} - \frac{1}{3}xe^{-x^2/6}\right)
\end{align*}

\begin{itemize}
    \item 步骤三:在第三步的操作中我们有可能加入了一些有重复部分的线段,我们将重复部分的线段去掉,以避免重复运算。
\end{itemize}

计算完毕。

\subsubsection{计算最小警车数}

在前三步的基础上,利用计算机计算出 $CS(P, L)$,如图 \ref{fig:4-4}:

\begin{figure}[h]
    \centering
    \includegraphics[width=\textwidth]{image2.png}
    \caption{覆盖线段集合}
    \label{fig:4-4}
\end{figure}

其中红色点代表交叉点,黑色代表道路,蓝色代表覆盖线段集合。

假设用户鼠标的点击位置为 $P$,左边的白色框内显示的是位于道路上的,距离 $P$ 点最近的点 $Q$ 的坐标,坐标值为 $(4114.90, 2981.39)$。

定义 $L$ 为警车接警后三分钟内行驶的长度,即 $L = 3 * 60 * 40 / 3.6 = 2000$(米)。

图中的蓝色线段即为计算机计算出的覆盖线段集合 $CS(P, L)$,左下角为计算机计算出的覆盖率,也就是警车能及时到达的区域与总区域的比例。这里要说明

\begin{figure}[h]
    \centering
    \includegraphics[width=\textwidth]{image1.png}
    \caption{重点区域的覆盖线段集合 黄色代表重点区域覆盖线段集合}
    \label{fig:4-5}
\end{figure}

我们发现这 3 个区域间没有交集,因此先选择 3 辆警车分别位于三个黄色区域内,然后再放置其他的警车。

经过实验发现,达到覆盖率为 90\% 的最少警车数为 17 辆。如图 \ref{fig:4-6} 所示:

\begin{figure}[h]
    \centering
    \includegraphics[width=\textwidth]{image2.png}
    \caption{最少辆车的分布情况 左下角显示覆盖率为 90.22\%,超过 90\%}
    \label{fig:4-6}
\end{figure}

\begin{tabular}{l l}
\textbf{队员姓名} & \textbf{林阳斌 陈碧黎 苏圳泷} \\

\end{tabular}

\section{问题二建模}

\subsection{符号约定}

以下是本问中特殊用到的符号,具体定义看前面的第3点符号约定

\begin{itemize}
    \item $A_{1}(p)$:重点区域中,点$p$的事件响应时间
    \item $A_{2}(p)$:非重点区域中,点$p$的事件响应时间
    \item $SA_{2}$:非重点区域的事件响应时间评价最终值
    \item $B_{1}(p)$:重点区域中,点$p$的事件响应能力值
    \item $B_{2}(p)$:非重点区域中,点$p$的事件响应能力值
    \item $H$:隐蔽性评价值
    \item $Z$:震慑性评价值
    \item $\underline{EA_{1}}$:重点区域的平均事件响应时间
    \item $\underline{EA_{2}}$:非重点区域的平均事件响应时间
    \item $EA(x)$:巡逻方案$x$的平均事件响应时间
    \item $N$:动用的110警车数
    \item $u_{1}$:重点区域的权值
\end{itemize}

\subsection{5.2 模型分析与模型建立}

问题二的主要任务是给出评价巡逻效果显著程度的有关指标,在这里我们将采用模糊数学模型来完成这个任务。

在模糊数学模型中,隶属函数的选择非常重要,隶属函数 \footnote{从理论上讲是任何正规的凸函数即可。在实际中,一般采用三角形、正态分布函数等几种形状。在实际应用中,并不强求绝对准确的隶属函数,模糊系统既然是模糊的,其语言变量的隶属函数也允许有一定的裕度。} 本问中,将寻找各个指标的隶属度函数,从而建立评价巡逻效果显著程度的模糊数学模型。

我们的评价是在执勤期间任意连续 1 小时内对巡逻方案进行各个指标的评价。

经过整体分析,我们认为影响巡逻效果显著程度的主要因素包括以下 5 点:

(1) 事件响应能力 B,表示当事件发生报警后能否在指定时间内到达指定区域的能力

① 对于重点区域,由于重要性及题意要求,如果发生事件后警车无法在 2 分钟内到达,那么就认为该方案的总体指标值 S 为 0,无需再计算其他的指标,而对于能够在 2 分钟内到达的情况则计算它的事件响应能力,具体如下,这种思想源于奖学金评定,如果某位同学某门课不及格,则将不参与奖学金的评定。

② 对于非重点区域,构造如下函数:
\[
B(A_2(p)) =
\begin{cases}
1, & A_2(p) \leq 3 \\
0, & A_2(p) > 3
\end{cases}
\]
考虑用如下形式的隶属函数来表示事件响应能力。
\[
B_2 = \frac{\sum_{i=1}^n B(A_2(p_i))}{n}
\]
当 \( n \) 足够大时,\( B_2 \) 可以较准确的反映出巡逻警车对任意发生的事件的平均响应能力。不妨设 \( n = 300 \),于是有
\[
B_2 = \frac{\sum_{i=1}^{300} B(A_2(p_i))}{300}
\]

(2) 事件响应时间 (A),表示当事件发生后,警车到达现场所用的时间;由于本问题有分重点区域和非重点区域,分别用 \( A_1(p) \) 和 \( A_2(p) \) 来表示点 \( p \) 的事件响应时间。

① 对于重点区域,计算第 \( x \) 套方案 3 个重点区域的平均事件响应时间 \( \text{EA}_1(x) \)
\[
\text{EA}_1(x) = \frac{A_1(p_1) + A_1(p_2) + A_1(p_3)}{3}
\]

(2) 对于非重点区域任意 \( n \) 个点,计算每个点若发生事件所需响应时间,然后计算巡逻方案 \( x \) 在非重点区域内平均事件响应时间 \( EA_2(x) \):

\[
EA_2(x) = \frac{\sum_{p=1}^n A_2(p)}{n}
\]

为了使计算所得的平均时间能够尽量反映总体响应时间,可以把 \( n \) 设为足够大,不妨设 \( n = 300 \),于是有:

\[
EA_2(x) = \frac{\sum_{p=1}^{300} A_2(p)}{300}
\]

根据 \( EA_1(x) \) 和 \( EA_2(x) \) 的值,我们可以选择适当的隶属函数来对响应时间做一些适当的定量分析,(如下图 5-1):

当警车能在 \( 0 \sim 1 \) 分钟内到达的效果都是非常好的,基本可以认为相差不大,这个区间段的指标值变化可以比较平稳,当警车在 \( 1 \sim 3 \) 分钟内到达,这段时间比较宝贵,往往对犯罪事件起到至关重要的作用,因此可以认为这个区间段的变化比较剧烈,而当警车在大于 3 分钟之后到达,基本可以认为错过了最佳接处警时间,对事件处理及时性的影响差别不大。根据以上经验,构造隶属函数 \( f(x) \) 如下:

\[
f(x) = e^{-x^2/6}
\]

\( x \) 表示单位事件响应时间(分钟),\( f(x) \) 为评价指标值,值域为 \([0, 1]\)。其函数图像如下图 5-1 所示:

\begin{figure}[h]
    \centering
    \includegraphics[width=\textwidth]{image.png}
    \caption{隶属函数}
    \label{fig:5-1}
\end{figure}

可以对其合理性进行简单的分析:

\begin{align*}
f'(x) &= \left(e^{-x^2/6}\right)' = -\frac{1}{3}xe^{-x^2/6} \\
f''(x) &= \left(-\frac{1}{3}xe^{-x^2/6}\right)' = -\frac{1}{3}\left(e^{-x^2/6} - \frac{1}{3}xe^{-x^2/6}\right)
\end{align*}

拐点为 \(x=3\),\(f'(x)\) 为 \([0, 3]\) 上的减函数,在 \([0, 1]\) 上 \(f'(x)\) 的变化不大,在 \([1, 3]\) 上 \(f'(x)\) 不断减小,\(f(x)\) 变化逐渐剧烈。在拐点处以后,\(f'(x)\) 不断增加,\(f(x)\) 的变化放缓,基本上符合事实。因此,事件响应时间评价值 \(SA\) 可采用隶属函数:
\[
SA = e^{-(\text{EA})^2/6}
\]
其中重点部位事件响应时间评价值为:
\[
SA_1 = e^{-(\text{EA}_1)^2/6}
\]
非重点部位事件响应时间评价值为:
\[
SA_2 = e^{-(\text{EA}_2)^2/6}
\]

经过我们的分析,由于城市的犯罪事件发生概率分布不一致,主要分为重点区域和非重点区域,因此我们认为不同区域对总体的指标值贡献的程度是不一样的,我们给它们赋予不同的权重,我们知道,重点区域是犯罪事件发生比较密集的地方,因此处理重点区域应比处理非重点区域来的更为重要,所以我们认为重点区域的权值应大于非重点区域,根据历史经验,我们假定重点区域权值 \(u_1\) 为 0.7,非重点区域 \(u_2\) 为 0.3,\(U\) 表示区域权值向量
\[
U = (u_1, u_2) = (0.7, 0.3)
\]
因此,平均的事件响应时间 \(\overline{SA}\) 为:
\[
\overline{SA} = U * (SA_1, SA_2)^T = (u_1, u_2) * (SA_1, SA_2)^T
\]

(3) 隐蔽性 (H),表示 110 警车巡逻方案的隐蔽程度。

如果警车的巡逻路线越无规律,则它的隐蔽性越好。反之如果警车巡逻呈现容易辨认的规律,则认为它的隐蔽性不好,这样认定是完全合理的。因此我们定义一个街道的访问序列,该序列是采集警车在执勤期间任意 4 小时内、时间间隔为 1 分钟时所处的街道标号。

若警车在连续的两个时间单位内处于相同的街道,则只选择一个,如:警车 1 在前 5 分钟内所处的街道编号如下表 5-a 所示:

\begin{table}[h]
\centering
\begin{tabular}{|c|c|c|c|c|c|}
\hline
时间(分钟) & 1 & 2 & 3 & 4 & 5 \\
\hline
街道编号 & 1 & 2 & 2 & 1 & 3 \\
\hline
\end{tabular}
\caption{表 5-a}
\end{table}

其中 2 分钟和 3 分钟的街道编号相同,我们不把后者的编号重复记录,因此最终获得街道的访问序列为 1213。可以根据这个街道序列的周期性来体现隐蔽性。比如:1213 无周期性,而街道序列 12341234123 可以认为具有周期性,因为它具有周期序列 1234(末尾的序列 123 由于与开头相同,因此被认为也属于周期序列中)。

我们可以根据街道序列的周期性来衡量隐蔽性这一个指标值。由于周期性

越大表示重复度越高,隐蔽性就越弱,采用下面的隶属函数来表示:
\[
H = 1 - \frac{cl}{l}, \text{ 其中 } cl \text{ 为周期序列的长度, } l \text{ 为序列总长度 }
\]

下面给出周期序列的计算方法:

值得一提的是,我们利用 KMP 算法的 next 数组来计算周期序列的长度(即 \( cl \)),时间复杂度为 \( O(l^2) \),\( l \) 为序列长度。

算法过程如下:

首先利用 KMP 算法求出 next 数组,接着,

\[
cl = l, s \text{ 为道路的访问序列}
\]

\begin{verbatim}
For i = l DownTo l / 2 + 1
    k = 0;
    For j = i to l
        If s[j] != s[k] break;
    End for
    if (j < l) continue;
    t = i - next[i - 1];
    if (i % t == 0 && i / t != 1)
        cl = min(t, cl);
    End for
\end{verbatim}

这里利用了 next[i] 为表示离 i 最近的 j 使得串 \( S_1 \ldots S_j = S_{i-j+1} \ldots S_i \),因此 \( i - \text{next}[i-1] \) 表示的即为 \( s_1 \ldots s_i \) 的重复串的长度(注意要满足 \( i - \text{next}[i-1] \) 可以整除 i)。由于串的末尾可能出现不到一个周期的情况,如例子:“12341234123”,最后的“123”不到一个周期。因此我们可以通过枚举 i 的方法来枚举每一种不满足一个周期的情况,以此来求出 \( cl \)。

这样就可以通过求出 \( cl \),将其带入公式:\( H = 1 - \frac{cl}{l} \) 计算出隐蔽性。

(4) 震慑性(Z),表示 110 警车的配置和巡逻方案对整个区域起到震慑作用,达到降低犯罪率的效果。一般地讲,震慑性可以通过所有警车巡逻覆盖的街道长度(重复经过不再次计入,记做 h)占总街道长度 L 的比例来衡量,因此构造下面的隶属函数:
\[
Z = \frac{h}{L}
\]

(5) 经济效益(M),表示某种巡逻方案所花费的经济费用带来的效益,因为每辆警车都是以巡逻速度在不断行驶,所以油费开销、车辆折旧以及人力资源等费用可以用警车数量来直接衡量,因此经济开销可以只考虑警车的车辆数。考虑到我们每个指标给予 10 分的分值,结合第一步所求的最少警车,我们认为该城市能动用的警车数最多为 50 辆,一旦超过 50 辆,产生的经济效益为 0,根据这个情况,我们定义一个简单的线性隶属函数来衡量经济效益 M:

\[
M =
\begin{cases}
-\frac{N}{50} + 1 & 0 \leq N \leq 50, N \text{为正整数} \\
0 & N > 50, N \text{为正整数}
\end{cases}
\]

到此,我们已经找到衡量巡逻方案的指标为事件响应能力(B)、事件响应时间(A)、隐蔽性(H)、震慑性(Z)、经济效益(M),我们可以根据隶属函数计算每个方案对应于考虑因素的不同隶属度,如下表5-b所示:

\begin{table}[h]
\centering
\begin{tabular}{|c|c|c|c|}
\hline
方案 & 1 & 2 & $\dots$ \\ \hline
事件响应能力(B) & b & $\dots$ & $\dots$ \\ \hline
事件响应时间(A) & a & $\dots$ & $\dots$ \\ \hline
隐蔽性(H) & h & $\dots$ & $\dots$ \\ \hline
震慑性(Z) & z & $\dots$ & $\dots$ \\ \hline
经济效益(M) & m & $\dots$ & $\dots$ \\ \hline
\end{tabular}
\caption{表5-b 隶属度向量}
\end{table}

对于每个方案,定义一个隶属度向量 $G$,表示方法如下:
\[
G = (b, a, h, z, m)
\]

但是,我们知道每个指标对于评价巡逻方案的重要性不一样,或者认为地位高低不一样,为此,引入权重系数一列来表示每个指标在巡逻方案中的重要性。根据历史和专家经验,得到的权重向量为:
\[
C = (c_1, c_2, c_3, c_4, c_5) = (0.25, 0.3, 0.1, 0.25, 0.1)
\]

利用上述的隶属度向量和权重向量,我们就可以求评价向量:
\[
S = C * G^T = (c_1, c_2, c_3, c_4, c_5) * (b, a, h, z, m)^T
\]

因此,整个评价指标值的过程由下列流程图5-2来表示

\begin{figure}[h]
\centering
\includegraphics[width=0.8\textwidth]{image.png} % 替换为实际的图像文件名
\caption{流程图5-2}
\end{figure}

\begin{figure}[h]
    \centering
    \includegraphics[width=\textwidth]{image.png}
    \caption{计算评价指标值流程图}
    \label{fig:flowchart}
\end{figure}

\section{问题三建模}

\subsection{符号约定}

\begin{itemize}
    \item $Cr[m,n]$:表示第 $m$ 个区域的第 $n$ 辆车
    \item $P_{i}$:表示区域 $i$ 的事件发生概率,$i=4\ldots14$
    \item $Count_{i}$:表示道路 $i$ 的巡逻次数累计值
\end{itemize}

\subsection{模型分析、建立与求解}

在第一问中,我们仅仅考虑如何使用最少的警车满足条件 D1,因此让每辆警车都各自固定守候在某一个点上,将该点的覆盖线段集合作为该警车的管辖区域,以保证在规定时间内到达其管辖区的任何点。可是静止等候的策略虽然理论上能保证办案效率,却无法带来巡逻所起到的警示效果等弊端。

现在考虑在保证 D1 的基础上,使巡逻效果更显著的方案。通过上面分析,考虑让警车以平均巡逻速度巡逻街道,但是又不能使全部警车按照无规则无次序的巡逻路线行驶,因此问题转化为制定合理的巡逻路线以求达到最优的巡逻效果。为了制定好的巡逻方案,首先需要划分区域,根据整个城市的道路特征、交叉点密集程度等因素来对城市划分区域,划分结果如下图 \ref{fig:region划分}:

\begin{figure}[h]
    \centering
    \includegraphics[width=\textwidth]{image.png}
    \caption{区域划分结果}
    \label{fig:region划分}
\end{figure}

\begin{figure}[h]
    \centering
    \includegraphics[width=\textwidth]{image.png}
    \caption{区域划分图}
    \label{fig:region_division}
\end{figure}

说明:每个区域块用颜色区分开来,数字代表每个区域的区域号。

在本模型中,我们假定非重点区域事件发生的概率分布都是一致的,结合我们编写的可视化软件进行不断测试实最终得出能满足题目要求的最少车辆为:30辆警车。下面重点说明 30 辆警车巡逻方案的制定。

为了使巡逻效果更显著,我们结合贪心算法的思想,来安排巡逻路线。贪心算法\cite{greedy_algorithm}总是作出在当前看来最好的选择,也就是说贪心算法并不从整体最优考虑,它所作出的选择只是在某种意义上的局部最优选择;但对许多问题它能产生整体最优解,如单源最短路径问题,最小生成树问题等。在一些情况下,即使贪心算法不能得到整体最优解,其最终结果却是最优解的很好近似。

根据上面区域划分图的结果,为每个区域安排警车,其数量可以大于 1,即可由多辆警车以巡逻状态一同管辖该区域。

假设下图 6-2 为某个划分区域,我们用它来说明如何制定巡逻路线。首先我们给每条道路分配一个整型值 $Count_{i}$ 存储当前该道路被巡逻过的累计次数,每当警车巡逻一次,该值加 1,初始时 $Count_{i} = 0$。

图中每条道路由 $(a_{i}, c_{i}), i=1,2,3……17$ 来标识,其中 $a_{i}$ 为道路编号,$c_{i}$ 为道路 $a_{i}$ 的 $Count_{i}$ 值。每次警车行驶到交叉点时,将判断该交叉点延伸的每条道路的 $Count_{i}$ 值,采用贪心算法思想,优先选择被巡逻次数最少的道路。例如:图 6-2 中红色区域代表某一时刻警车走到该位置,这个时候警车就需要判断道路 $a_{10}$、$a_{11}$、$a_{12}$、$a_{17}$ 的 $Count_{i}$ 值 ($c_{i}$) 大小选择具有最小 $Count$ 值的街道行走,同时需要提到的是,由于每个管辖区域可能有多辆警车,因此如果该道路上已经有其他警车在巡逻,则放弃走这条道路。

\begin{figure}[h]
    \centering
    \includegraphics[width=0.8\textwidth]{image.png}
    \caption{某一划分区域}
    \label{fig:region}
\end{figure}

具体算法分析与步骤如下:m 代表区域编号,n 代表该区域的第 n 辆警车
\begin{enumerate}
    \item 遍历城市中的每个区域,m=1...14
    \item 遍历每个区域中的每辆警车 Cr[m,n],为其随机指派一个起始位置,开始巡逻该位置所在的道路 a,同时 $Count_a$ 值增加 1
    \item 执行巡逻,巡逻中遇到管辖的区域边界或者交叉点时,检查是否有相交道路,如果没有,返回反向行驶,然后继续巡逻,执行步骤 3。如果有,转到步骤 4
    \item 如果某条道路已有警车在巡逻,则放弃巡逻该道路,找出当前无警车巡逻的道路中 $Count_i$ 值最大的道路 Cmax,巡逻道路 Cmax
    \item 若指挥中心接警后需要安排该警车去解决时,则通过 GPS 寻找最优路线赶赴现场
    \item 当该警车的警员下班时,则回到警局
\end{enumerate}

另外,我们从指挥中心的角度描述该方案的实施:
指挥中心统一调度每辆警车巡逻路线,指挥中心每分钟采集一份整个区域各辆警车状态和道路状态的数据,然后经过计算机的处理,分析出每辆警车下一步应该走的道路编号,最后反馈给每辆警车。

而从警车的角度来看,如果警车行驶到交叉口还未收到来自指挥中心的指示,则警车会在交叉口某处等候直到收到命令,这种等待一般不超过一分钟。而采用这种在交叉口等待片刻的策略,是有一定道理的,因为交叉口位置的特殊,每个道路一般都可以从远处看到交叉口,由于警车特征比较鲜明,无疑增加一定的震慑效果,再者交叉口一般是交通要道,可以较快速到达其他地点。

另外需要说明的是,由于采用街道的巡逻次数累计值 $Count_i$,计算机经过长期连续的运行,会使得 $Count_i$ 值变得非常大,到一定程度后,巡逻路线更易形成周期性的特点,因此,规定每条道路的巡逻次数累计值 $Count_i$ 要在一定时间内进行清零,防止巡逻趋于规律性,一般是一周清零一次即可。

现给出本问题结果的前 3 组数据:
30,1.0595,1.68909,0.862507,1,0.91,0.972139,0.4

1,(8082,4662),(6624,1332),(5562,4608),(9630,7992),(9144,7650),(13050,5346),(9162,7992),(8748,252),(8748,252),(13266,5292),(9180,2664),(8262,2358),(7506,3618),(6390,4086),(5526,2340),(8118,7974),(5580,6246),(8064,7146),(2268,6858),(4644,7110),(3042,5508),(2088,4266),(1962,3636),(1638,1656),(2178,864),(5184,1908),(6066,1278),(5796,936),(4500,3150),(4356,3348)

\begin{align*}
2, (8415.34, 4662), (6704.85, 1008.62), (5598, 4356), (9595.1, 7660.5), (9477.34, 7650), (13266, 5292), (9144.48, 7659.12), (9081.29, 246.351), (9081.29, 246.351), (13243, 4959.46), (9198, 2340), (8595.27, 2351.59), (7697.83, 3345.4), (6408, 3852), (5512.12, 2006.95), (8118, 7668), (5912.99, 6230.86), (7730.78, 7137.23), (2487.5, 6607.14), (4977.34, 7110), (3042, 5220), (2022.63, 3939.14), (1898.16, 3308.83), (1743.41, 1339.77), (2510.51, 840.489), (5058, 1800), (5859.45, 1016.37), (5832, 738), (4167.08, 3133.35), (4644, 3330)
\end{align*}

\begin{align*}
3, (8748.67, 4662), (6768, 756), (5616, 4050), (9594, 7650), (9594, 7650), (13465, 5559.41), (9144, 7650), (9414.58, 240.702), (9414.58, 240.702), (13220, 4626.92), (8864.73, 2346.41), (8928.55, 2345.18), (7848, 3132), (6112.72, 4006.67), (5508, 1908), (8118, 7974), (6245.98, 6215.73), (7397.56, 7128.46), (2707.01, 6356.28), (5022, 7110), (2709.18, 5201.51), (1962, 3636), (1834.33, 2981.67), (1848.82, 1023.54), (2843.01, 816.979), (5184, 1908), (5796, 936), (5498.66, 738), (4140, 3132), (4356, 3348)
\end{align*}

本题目中具体巡逻方案效果可以参考文件 1038417-Result3.txt。

\subsection{6.3 模型的评价指标值求解}

问题二已给出了方案的各个评价指标,并建立了基于模糊数学理论的评价模型。同时也将这个模糊评价的计算编写在软件中,运行各个方案就能输出各个评价指标值。现根据里面的理论和公式给出每个指标值的计算过程。

\subsubsection{6.3.1 事件响应能力求解}

我们在划分区域时就确保重点区域有一辆警车在附近巡逻,因此都能满足 2 分钟内到达,所以总评价指标值不为 0,因此继续计算其他各项指标值。

根据公式:
\begin{align*}
2, (8415.34, 4662), (6704.85, 1008.62), (5598, 4356), (9595.1, 7660.5), (9477.34, 7650), (13266, 5292), (9144.48, 7659.12), (9081.29, 246.351), (9081.29, 246.351), (13243, 4959.46), (9198, 2340), (8595.27, 2351.59), (7697.83, 3345.4), (6408, 3852), (5512.12, 2006.95), (8118, 7668), (5912.99, 6230.86), (7730.78, 7137.23), (2487.5, 6607.14), (4977.34, 7110), (3042, 5220), (2022.63, 3939.14), (1898.16, 3308.83), (1743.41, 1339.77), (2510.51, 840.489), (5058, 1800), (5859.45, 1016.37), (5832, 738), (4167.08, 3133.35), (4644, 3330)
\end{align*}
得到出非重点区域的响应能力为:$B_2 = 0.9033$

\subsubsection{6.3.2 事件响应时间求解:}

(1) 对于重点区域,根据公式:
\begin{align*}
3, (8748.67, 4662), (6768, 756), (5616, 4050), (9594, 7650), (9594, 7650), (13465, 5559.41), (9144, 7650), (9414.58, 240.702), (9414.58, 240.702), (13220, 4626.92), (8864.73, 2346.41), (8928.55, 2345.18), (7848, 3132), (6112.72, 4006.67), (5508, 1908), (8118, 7974), (6245.98, 6215.73), (7397.56, 7128.46), (2707.01, 6356.28), (5022, 7110), (2709.18, 5201.51), (1962, 3636), (1834.33, 2981.67), (1848.82, 1023.54), (2843.01, 816.979), (5184, 1908), (5796, 936), (5498.66, 738), (4140, 3132), (4356, 3348)
\end{align*}
程序实现计算在一个小时内的各个时刻的 $\text{EA}_1(x)$,最后取平均值计算平均值得:$\overline{\text{EA}_1(x)} = 1.06$

代入隶属函数:
\begin{align*}
\overline{SA} &= \mathbf{U} * (SA_1, SA_2)^T = (u_1, u_2) * (SA_1, SA_2)^T \\
&= (0.7, 0.3) * (0.89, 0.69)^T = 0.83
\end{align*}
得出重点区域的事件响应时间的评价指标值为:$\text{SA}_1 = 0.83$

(2) 对于非重点区域,根据公式:
\[
E A_{2}(x)=\frac{\sum_{p=1}^{300} \mathbf{A}_{2}(p)}{300}
\]
程序在任意时间任意地点找出 300 个点的事件响应时间,取平均值,得到最后结果为: $\overline{\mathbf{E A}}_{2}=1.65$

代入隶属函数: $S A=e^{-(\mathbf{E A})^{2} / 6}$

得出重点区域的事件响应时间的评价指标值为: $\mathbf{S A}_{2}=0.64$

最后,计算平均的事件响应时间 $\overline{S A}$ 为:
\[
\begin{aligned}
\overline{S A} & =\mathbf{U} *\left(\mathbf{S A}_{1}, S A_{2}\right)^{T}=\left(u_{1}, u_{2}\right) *\left(\mathbf{S A}_{1}, S A_{2}\right)^{T} \\
& =(0.7,0.3) *(0.83,0.64)^{T}=0.773
\end{aligned}
\]

\subsubsection{6.3.3 隐蔽性评价指标值求解}

利用公式 $H=1-\frac{c l}{l} \quad c l$ 为周期序列的长度,$l$ 为序列总长度

来计算隐蔽性,求得: $H=0.97$

\subsubsection{6.3.4 震慑性评价指标值求解}

通过程序,我们计算该方案在执勤期间任意一个小时内,所有警车巡逻覆盖的街道长度(重复经过不再次计入)记做 $h$,其占总街道长度 $L$ 的比例作为震慑性的依据,由震慑性的隶属函数:
\[
Z=\frac{h}{L}
\]

得到该方案的震慑性评价指标值为: $Z=0.9076$

\subsubsection{6.3.5 经济效益评价指标值求解}

该方案所配置的警车数量为 30 辆,代入线性隶属函数:
\[
M=\begin{cases}-\frac{N}{50}+1 & 0 \leq N \leq 50, N \text { 为正整数 } \\ 0 & N>50, N \text { 为正整数 }\end{cases}
\]

得到:
\[
M=-\frac{30}{50}+1=0.4
\]

\subsubsection{6.3.6 总结}

我们通过问题二中建立的模糊数学关系模型,求得隶属度向量:
\[ G = (b, a, h, z, m) \]
最终值为:
\[ G = (b, a, h, z, m) = (0.9033, 0.773, 0.97, 0.9076, 0.4) \]
而权重向量为:
\[ C = (c_1, c_2, c_3, c_4, c_5) = (0.25, 0.3, 0.1, 0.25, 0.1) \]
利用上述的隶属度向量和权重向量,我们就可以求评价向量,
\[ S = C * G^T = (c_1, c_2, c_3, c_4, c_5) * (b, a, h, z, m)^T \]
最后求得整体指标值:$ S = 0.8216 $
总之,问题三中所求的各个指标值为隶属度向量:
\[ G = (b, a, h, z, m) = (0.9033, 0.773, 0.97, 0.9076, 0.4) \]
而最终整体指标值 $ S = 0.8216 $

\subsection{6.4 模型改进}

上面的模型是假定每个非重点区域的案件发生概率是一致的。然而在现实中,每个区域由于其地理位置、城区的性质等差异,发生案件的概率也会不同,一般市中心发生抢劫、盗窃、纠纷等需要及时解决的案件比较多,而郊区发生概率比较小,因此我们可以假设每个区域事件发生的概率不一样,根据历史经验并结合区域地理位置、繁华程度和交通状况等因素来大概估计每个区域发生事件的概率 $ P_i $,得到下面的事件发生概率分布表6-a:

\begin{table}[h]
\centering
\begin{tabular}{|c|c|c|c|c|c|c|c|c|c|c|c|}
\hline
区域号 & 4 & 5 & 6 & 7 & 8 & 9 & 10 & 11 & 12 & 13 & 14 \\ \hline
事件发生概率 $ P_i $ & 0.06 & 0.03 & 0.005 & 0.08 & 0.165 & 0.1 & 0.1 & 0.1 & 0.08 & 0.12 & 0.16 \\ \hline
\end{tabular}
\caption{非重点区域事件发生概率分布表}
\end{table}

(1) 为了满足条件D1,我们随机选择300个事件,根据每个区域的事件发生概率 $ P_i $ 为这300个事件分配区域,分配结果如下表6-b:

\begin{table}[h]
\centering
\begin{tabular}{|c|c|c|c|c|c|c|c|c|c|c|c|c|}
\hline
区域号 & 1 & 2 & 3 & 4 & 5 & 6 & 7 & 8 & 9 & 10 & 11 & 12 & 13 & 14 \\ \hline
分配车辆数 & 1 & 1 & 1 & 2 & 1 & 1 & 2 & 3 & 3 & 2 & 2 & 1 & 3 & 2 \\ \hline
\end{tabular}
\caption{}
\end{table}

最终得到警车的总数为:25辆。小于未改进模型的车辆数(30辆)。

\subsection{6.5 改进模型的评价指标值}

如上面的 6.4 点,我们给出改进模型的各个指标值的计算过程。

\subsubsection{6.5.1 事件响应能力求解}

我们在划分区域时就确保重点区域有一辆警车在附近巡逻,因此都能满足 2 分钟内到达,所以总评价指标值不为 0,因此继续计算其他各项指标值。

根据公式:
\[
B_{2}(p) = B(A_{2}(p)) =
\begin{cases}
1, & A_{2}(p) \leq 3 \\
0, & A_{2}(p) > 3
\end{cases}
\]
求得:$\mathbf{B}_{2} = 0.914$

\subsubsection{6.5.2 事件响应时间求解}

(1) 对于重点区域,根据公式:
\[
\mathrm{EA}_{1}(x) = \frac{A_{1}(p_{1}) + A_{1}(p_{2}) + A_{1}(p_{3})}{3}
\]
计算平均值得:$\overline{\mathrm{EA}}_{1}(x) = 1.08$

代入隶属函数:$SA = e^{-(\mathrm{EA})^2 / 6}$ 得出重点区域的事件响应时间的评价指标值为:
\[
\mathrm{SA}_{1} = 0.82
\]

(2) 对于非重点区域,根据公式:
\[
\mathrm{EA}_{2}(x) = \frac{\sum_{p=1}^{300} A_{2}(p)}{300}
\]
取平均值,得到最后结果为:$\overline{\mathrm{EA}}_{2} = 1.73$

代入隶属函数:$SA = e^{-(\mathrm{EA})^2 / 6}$ 得出重点区域的事件响应时间的评价指标值为:
\[
\mathrm{SA}_{2} = 0.61
\]

最后,计算平均的事件响应时间 $\overline{SA}$ 为:
\[
\overline{SA} = \mathbf{U} * (\mathrm{SA}_{1}, \mathrm{SA}_{2})^T = (u_{1}, u_{2}) * (\mathrm{SA}_{1}, \mathrm{SA}_{2})^T
\]
\[
= (0.7, 0.3) * (0.82, 0.61)^T = 0.757
\]

\subsubsection{6.5.3 隐蔽性评价指标值求解}

利用公式 $H = 1 - \frac{cl}{l}$,$cl$ 为周期序列的长度,$l$ 为序列总长度来计算隐蔽性,求得:$H = 0.97$

\subsubsection{6.5.4 震慑性评价指标值求解}

通过程序,我们计算该方案在执勤期间任意一个小时内,所有警车巡逻覆盖的街道长度(重复经过不再次计入)记做 $h$,其占总街道长度 $L$ 的比例作为震慑性的依据,由震慑性的隶属函数:
\[
Z = \frac{h}{L}
\]
得到该方案的震慑性评价指标值为:$Z = 0.8014$

\subsubsection{6.5.5 经济效益评价指标值求解}

该方案所配置的警车数量为 25 辆,代入线性隶属函数:
\[
M =
\begin{cases}
-\frac{N}{50} + 1 & 0 \leq N \leq 50, N \text{ 为正整数} \\
0 & N > 50, N \text{ 为正整数}
\end{cases}
\]
得到:
\[
M = -\frac{25}{50} + 1 = 0.5
\]

\subsubsection{6.5.6 总结}

我们通过问题二中建立的模糊数学关系模型,求得隶属度向量
\[
G = (b, a, h, z, m)
\]
最终值为
\[
G = (b, a, h, z, m) = (0.914, 0.757, 1.0, 0.8014, 0.5)
\]
而权重向量为:
\[
C = (c_1, c_2, c_3, c_4, c_5) = (0.25, 0.3, 0.1, 0.25, 0.1)
\]
利用上述的隶属度向量和权重向量,我们就可以求评价向量,
\[
S = C * G^T = (c_1, c_2, c_3, c_4, c_5) * (b, a, h, z, m)^T
\]
最后求得 $S = 0.8060$

前面无考虑概率分布不均匀的模型的 $S$ 值为 0.8216,明显大于改进模型的值 0.8060,前面一种情况需要 30 辆警车才能达到超过 90\% 的覆盖率,而改进模型只

需要 25 辆车就能达到该目标,而两者的最终评价值相异并不大。

\section{7. 问题建模}

\subsection{7.1 问题分析}

本问要求在第三问的基础上兼顾 D3 的隐蔽性的条件。本节将巡逻的隐蔽性解释为巡逻路线的无规律性。之前的模型给出的巡逻方案,虽然已经能够满足 D1 和 D2 两个条件,但若是警车巡逻的路线有迹可寻,则效果便会打些折扣。所以我们对之前的模型进行改良,将隐蔽性因素考虑进去。为了提高巡逻方案的隐蔽性,必须让巡逻的路线呈现更无规律的变化,基于这一点考虑,我们可以采用轮盘赌算法。

\subsubsection{7.1.1 轮盘赌算法}

轮盘赌算法是一种常用的随机选择方法,类似于博彩游戏中的轮盘赌,每次警车走到交叉点的时候,它就需要选择下一条巡逻的街道,而每条街道都有一个巡逻次数的累计值 $Count$,轮盘赌思想是按照要进行选择的街道的在一圆盘上进行比例划分,每次转动圆盘后待圆盘停止后指针停靠扇区对应的街道即为选择的道路。显然,概率值越高的,其在圆盘中所占的面积越大,被选中的机会也就越多。如下图 8-1 所示:其中 A,B,C,D 为不同的道路的比例分区。

\begin{figure}[h]
    \centering
    \includegraphics[width=0.5\textwidth]{roulette_wheel.png}
    \caption{轮盘赌示意图}
    \label{fig:roulette}
\end{figure}

\subsection{7.2 模型建立与求解}

\subsubsection{7.2.1 模型建立}

我们结合问题三的思路,给出了两种模型,一种是假定非重点区域的案发事件概率分布一致,另一种是假定非重点区域中每个划分区域的案发事件概率是不一致的。为了提高隐蔽性,也就是让警车的巡逻更无规律性,采用轮盘赌算法,即当巡逻车到达交叉点时,根据 $Count_i$ 值给予可行的每条道路一个概率,$Count_i$ 越低说明这条道路被巡逻的概率越高。

为每条可行道路设置概率 $P_i$ 的算法如下:

\begin{enumerate}
    \item 计算所有可行道路中 $Count_i$ 的最大值,记为 $MaxC$,记 $Vi = MaxC - Count_i$
    \item 记 $T = \sum_i V_i$,则 $P_i = \frac{V_i}{T}$;
\end{enumerate}

根据 $P_{i}$ 的概率选择算法如下

\begin{enumerate}
    \item 随机产生浮点数 K
    \item 定义 N = 可行道路总数 $t = 0$;
    \item FOR $i = 0$ to $n$
   \begin{verbatim}
   t += p[i];
   if (t >= k)
       Choose(roads[i]) // 选择第 i 条路
   end
   \end{verbatim}
\end{enumerate}

\subsection{7.2.2 模型求解}

我们按照上面的算法步骤编程实现了上述的两种模型,图 7-2 为计算结果

\begin{figure}[h]
    \centering
    \includegraphics[width=0.45\textwidth]{image1.png}
    \includegraphics[width=0.45\textwidth]{image2.png}
    \caption{两种模型程序运行结果}
    \label{fig:7-2}
\end{figure}

从运行结果看,两个模型的重点区域事件被及时处理的概率都达到了 100%,这是因为我们已经给每个重点区域分别安排了一辆警车在覆盖线段集合内巡逻所起的效果;另外非重点区域事件被及时处理的概率也都达到了 90\%(特别说明,由于随机性影响,操作该程序时,并不能马上得到超过 90\% 的结果,多次运行即可找到达到 90\% 的运行结果),另外,隐蔽性指标也都达到了 1.00,说明我们的模型已经满足条件 D3。整体上,都达到了题目给的要求。

\section{8. 问题五建模}

\subsection{8.1 问题分析}

针对该问题,由于只有 10 辆警车可供调度,相对于第一个模型得出的 17 辆与第三个模型得出的 30 辆都有较大差距,不可能满足 D1,D2 条件,只能尽可能地去接近该目标。前面几个模型基本上得出了静止候命的策略比运动巡逻的策略更能满足 D1,而运动巡逻策略比静止候命策略更能满足 D2 的结论。两个目标难以兼得,应尽可能地找到均衡点解决问题。我们要寻求的是既能保证重要区域安全,同时使巡逻覆盖范围尽可能大,再兼顾巡逻效果显著性的方案。

我们制定这样一个方案:每个重点区域各分配一辆警车,其余 7 辆随机分配到其他区域。每辆警车按照这样的策略巡逻:若是行驶到每个交叉点处,在选择往何方向行驶时,选择与其他车辆最短距离总和最大的街道行驶。虽然有些笼统,但这样基本上能尽可能地保证车与车之间的距离不会太小,从而使车辆的覆盖区

域相交较少,总覆盖率得到较大满足。

\subsection{8.2 符号说明}

\begin{itemize}
    \item $p_{ik}$: 警车 $i$ 在 $k$ 时刻所处的坐标点,$i=1,2,\dots,10$
    \item $l_{j}$: 第 $j$ 条路
    \item $L(a_{i}, l_{j})$: 点 $a_{i}$ 到 $l_{j}$ 的中点的最短路径
    \item $B_{i}(p_{ik})$: 警车 $i$ 在点 $p_{ik}$ 处的覆盖区域
\end{itemize}

\subsection{8.3 模型建立与求解}

目标函数: $\max \bigcup_{i=1}^{10} B_{i}(p_{ik})$,

$p_{ik}$ 在警车 $i$ 所行驶的路上随时间 $k$ 的变化连续变化,应用从局部最优实现总体最优的原理,在每个交叉点处,目标函数的求解可以转化为警车 $j$ 的决策函数 $\sum_{\substack{0 \leq i \leq 10 \\ i \neq j}} L(a_{i}, l_{j})$ 的最大化问题,其中,$p_{ik}$,$i=1,2,3$ 在任意时刻均处于重点区域,$p_{i0}$,$i=4,5,\dots,10$,初始位置随机分布在重点区域之外。

我们采用贪心算法来求解该模型。贪心准则为:当行驶至某个交叉点处,警车 $j$ 将选择使得 $\sum_{\substack{0 \leq i \leq 10 \\ i \neq j}} L(a_{i}, l_{j})$ 最大化的路径为前进方向。值得注意的是,警车 $1,2,3$ 在任何时刻,均只限制在重点区域内巡逻。

\subsection{8.4 模型分析}

该方案的巡逻效果存在一定的随机性,据随机一组模型模拟,得到的指标如下表 8-a 所示:

\begin{table}[h]
\centering
\begin{tabular}{|l|l|}
\hline
重点区域平均响应时间 & 0.98(分钟) \\
\hline
非重点区域平均响应时间 & 2.94(分钟) \\
\hline
道路巡逻覆盖率 & 15.53\% \\
\hline
重点区域事件被处理概率 & 100\% \\
\hline
非重点区域事件被处理概率 & 56\% \\
\hline
隐藏性指标 & 1.00 \\
\hline
\end{tabular}
\caption{表 8-a}
\end{table}

该模型在重点区域响应时间较短,响应能力也得到保证,且具有极高的隐藏性,但由于其具有较高随机性,因此非重点区域平均响应时间、事件处理概率以及道路巡逻覆盖率波动较大,很难对其给出一个定性的评价。

\section{问题六建模}

\subsection{问题分析与求解}

本问题将接警后的驾驶速度由 $40 \mathrm{~km} / \mathrm{h}$ 变为 $50 \mathrm{~km} / \mathrm{h}$, 这样警车就能够更快更及时地赶到案发现场, 我们将这个速度应用于问题三建立的模型中, 用问题三所编写的软件求出的最后评价指标值。

\subsection{问题求解}

\subsubsection{事件响应能力求解}

我们在划分区域时就确保重点区域有一辆警车在附近巡逻, 因此都能满足 2 分钟内到达, 所以总评价指标值不为 0, 因此继续计算其他各项指标值。对于该巡逻方案, 根据公式:

\[
B\left(A_{2}(p)\right)= \begin{cases}1, A_{2}(p) \leq 3 \\ 0, A_{2}(p) > 3\end{cases}
\]

经过编写程序, 结果显示非重点区域的响应能力为: $B_{2}=0.95$

\subsubsection{事件响应时间求解:}

(1) 对于重点区域, 根据公式:

\[
\mathrm{EA}_{1}(\mathrm{x})=\frac{A_{1}\left(p_{1}\right)+A_{1}\left(p_{2}\right)+A_{1}\left(p_{3}\right)}{3}
\]

我们编程实现计算在一个小时内的各个时刻的 $\mathrm{EA}_{1}(\mathrm{x})$, 最后取平均值计算平均值得: $\overline{\mathrm{EA}}_{1}(\mathrm{x})=0.85$

代入隶属函数: $S A=e^{-(\mathrm{EA})^{2} / 6}$

得出重点区域的事件响应时间的评价指标值为: $\quad S A_{1}=0.89$

(2) 对于非重点区域, 根据公式:

\[
E A_{2}(\mathrm{x})=\frac{\sum_{p=1}^{300} A_{2}(\mathrm{p})}{300}
\]

我们编写程序实现任意时间内找出 300 个点的事件响应时间的平均值, 得到最后结果为: $\overline{\mathrm{EA}}_{2}=1.48$

代入隶属函数: $S A=e^{-(\mathrm{EA})^{2} / 6}$

得出重点区域的事件响应时间的评价指标值为: $\quad S A_{2}=0.69$

最后, 计算平均的事件响应时间 $\overline{S A}$ 为:

\begin{align*}
\overline{SA} &= \mathbf{U} * (SA_1, SA_2)^T = (u_1, u_2) * (SA_1, SA_2)^T \\
&= (0.7, 0.3) * (0.89, 0.69)^T = 0.83
\end{align*}

\subsubsection{9.2.3 隐蔽性评价指标值求解}

利用公式 \( H = 1 - \frac{cl}{l} \)(\( cl \) 为周期序列的长度,\( l \) 为序列总长度)来计算隐蔽性,求得:\( H = 1.0 \)

\subsubsection{9.2.4 震慑性评价指标值求解}

通过程序,我们计算该方案在执勤期间任意一个小时内,所有警车巡逻覆盖的街道长度(重复经过不再次计入)记做 \( h \),其占总街道长度 \( L \) 的比例作为震慑性的依据,由震慑性的隶属函数:
\[
Z = \frac{h}{L}
\]
得到该方案的震慑性评价指标值为:\( Z = 0.6066 \)

\subsubsection{9.2.5 经济效益评价指标值求解}

该方案所配置的警车数量为 26 辆,由前面的定义,代入线性隶属函数:
\[
M =
\begin{cases}
-\frac{N}{50} + 1 & 0 \leq N \leq 50, N \text{ 为正整数} \\
0 & N > 50, N \text{ 为正整数}
\end{cases}
\]
得到 \( M = -\frac{26}{50} + 1 = 0.48 \)

\subsubsection{9.2.6 总结}

我们通过问题二中建立的模糊数学关系模型,求得隶属度向量
\[
G = (b, a, h, z, m)
\]
也即各个指标值为:
\[
G = (b, a, h, z, m) = (0.95, 0.83, 1.0, 0.6066, 0.48)
\]
而权重向量为:
\[
C = (c_1, c_2, c_3, c_4, c_5) = (0.25, 0.3, 0.1, 0.25, 0.1)
\]
利用上述的隶属度向量和权重向量,我们就可以求评价向量:

\begin{align*}
4. (8118.00, 6282.00) & \quad 5. (4086.00, 2070.00) & 6. (1962.00, 3636.00) \\
7. (6642.00, 270.00) & \quad 8. (11574.00, 7164.00) & 9. (11952.00, 2142.00) \\
10. (5616.00, 7920.00) & \quad 11. (9162.00, 5256.00) & 12. (1314.00, 6480.00) \\
13. (13266.00, 5292.00) & \quad 14. (8748.00, 252.00) & 15. (9162.00, 7992.00) \\
16. (306.00, 1386.00) & \quad 17. (11880.00, 198.00)
\end{align*}

最后求得 \( S = 0.7882 \)

\section{问题七}

\subsection{问题分析}

前面我们为了简化模型,做了一些假设,但是由于现实世界存在很多不稳定因素,导致一些假设太过理想化,因此在这一问中,我们将上述简化的因素考虑进去,提出一些解决方案,另外我们在前几问中为了满足题目给定的条件,采取的模型可能会以牺牲其他重要指标为代价来满足题目所需的条件,因此在这一步中,我们也会提到一些改进其他指标的方案;最后,由于我们在前面的巡逻方案中都是每辆警车的行驶都是独立的,没有考虑到警车之间的协作关系,因此我们提出了一种警车协作的方案来提高巡逻效果。

下面,我们将提供一些改进模型以及它们的解决方案

\subsection{不稳定因素的考虑与解决方案}

我们在前面假设派遣出的警车的出行状况都是畅通的,但是由于现实世界存在很多不确定因素,比如天气恶劣、交通堵塞、警车故障、驾驶员身体状况等使得该警车无法及时到达案发现场或者根本无法到达现场,这时候指挥中心就需要再重新调度另外一辆警车来处理事件,这时就需要制定一些方案来处理这些情况。

当指挥中心接到报警后派遣警车 A 去处理该事件,而 A 警车由于以上状况无法在最佳的接处警时间内到达时,指挥中心应当及时联系 A 警车了解情况,如果确定无法到达,则指挥中心应当再派遣另外一辆警车。因此我们可以制定以下步骤来检测这一过程。

步骤一:指挥中心派遣一辆警车 A 后,及时获取报警人反馈意见,如果发现警车 A 无法在最佳接处警时间按到达,则转下一步

步骤二:获取案发现场地点 \( Q \),检测当前离点 \( Q \) 最近的警车前往处理。

\subsection{计算机智能处理的考虑与解决方案}

在第一问中,我们刚开始是根据一个贪心算法来自动获取管辖区域,该算法如下描述:

[1] 从数据文件中获取交叉点的位置坐标

[2] 遍历所有的交叉点 \( p \),找出该位置坐标的覆盖线段集合,计算其总长度,并将其保存于 \( \text{LCS}[p] \) 中。

[3] 依次找出 \( \text{LCS}[p] \) 中找到最大无重复覆盖的值,作为警车固定静候的位置

以下是我们用程序编程实现后找到的 17 辆警车安置点(和手动结果一样),得到的覆盖率为 91.29%,以下前面的数字代表位置序号,括号部分代表坐标。

1. (5616.00, 4050.00) \quad 2. (7452.00, 2988.00) \quad 3. (3726.00, 6012.00)

\begin{align*}
4. (8118.00, 6282.00) & \quad 5. (4086.00, 2070.00) & 6. (1962.00, 3636.00) \\
7. (6642.00, 270.00) & \quad 8. (11574.00, 7164.00) & 9. (11952.00, 2142.00) \\
10. (5616.00, 7920.00) & \quad 11. (9162.00, 5256.00) & 12. (1314.00, 6480.00) \\
13. (13266.00, 5292.00) & \quad 14. (8748.00, 252.00) & 15. (9162.00, 7992.00) \\
16. (306.00, 1386.00) & \quad 17. (11880.00, 198.00)
\end{align*}

虽然能找到解,并且等同于我们手动找到的最优解,但是其搜索空间大,耗费时间长,经过测试,计算机要花费十几分钟才能输出解,而且解的质量(覆盖率)不高。我们发现,利用 vs2008 编写的手动描点显示覆盖线段集的软件,采用手动来划分区域能得到更好的覆盖率,且速度比较快,同时能考虑到其他无法用计算机限制的因素,因此第一问我们采用了手动划分区域获得了 17 辆警车的最优解。但是如果为了让本模型完全让计算机来完成,我们可以考虑第一种方法,以牺牲时间为代价来获取解。

\subsection{10.4 警车协作模型的建立与解决方案}

我们将这个模型建立在第三问获取的 25 辆车的基础上,考虑 25 辆警车的配置背景来说明问题,由于我们之前的划分区域比较小,因此可以考虑将区域合并以求得巡逻更具灵活性,同时先前模型的巡逻方案的一个前提是每辆警车的行驶都是独立的,即几乎不受其他辆警车的影响,没有考虑和其他警车进行配合,因此在本模型中,我们提出了警车协作模型以求得更好的巡逻效果。

\subsubsection{10.4.1 重新划分区域}

如图 10-1 所示的是先前的划分区域效果图:

\begin{figure}[h]
\centering
\includegraphics[width=\textwidth]{image.png}
\caption{区域划分图}
\end{figure}

对于包含重点区域的管辖区,为了更及时地处理事件,我们不对其合并区域,因为警车巡逻之间的独立性对于案件发生概率较大的区域来说更能获得好的巡逻效果;另外考虑到区域 14 在湖泊边,地理位置与外界比较独立,也可以归为一个区域,现重命名各区域:
\begin{itemize}
    \item 区域 1 重命名为 A,
    \item 区域 2 重命名为 B,
    \item 区域 3 重命名为 C,
    \item 区域 14 重名为 D;
\end{itemize}

而对于非重点区域,我们考虑区域合并,如下面所述:
\begin{itemize}
    \item 合并区域 10 和 13 为区域 E,
    \item 合并区域 4,6,5 为区域 F,
    \item 合并区域 11,12 为区域 G,
    \item 合并区域 8,9,7 为区域 H;
\end{itemize}

为了简化问题,我们提出最简单的警车协作模型,就是两辆警车的协作,定义一个数据结构 \( X(\text{Car1}, \text{Car2}, i) \) 代表区域 \( i \) 中具有协作关系的两辆警车 Car1 和 Car2 的协作程度,协作的目的就是在巡逻过程中,尽量让两辆巡逻车的覆盖线段集合的并集保持最大,为了更好地描述协作程度,我们规定 Car1 和 Car2 的覆盖线段集合分别为 \( CS(\text{Car1}) \) 和 \( CS(\text{Car2}) \),定义它们的并集为:
\[
PCS(\text{Car1}, \text{Car2}) = CS(\text{Car1}) \bigcup CS(\text{Car2})
\]
另外,我们令每个区域的道路总长为 \( L_i \),其中 \( i \) 表示区域的编号。由于 PCS 从一定程度上反映了协作效果,因此我们可以合理地定义下面的函数来衡量协作程度:
\[
X(\text{Car1}, \text{Car2}, i) = \frac{PCS(\text{Car1}, \text{Car2})}{L_i}
\]
在以后的研究中,我们可以通过设计算法使得每个区域中的两辆警车尽可能每次选择使 \( X(\text{Car1}, \text{Car2}, i) \) 最大的道路行驶,来更好完善该模型,使得该问题的解更理想更符合实际。

\section{参考文献}

[1] 梁国业,廖建平. 数学建模. 北京:冶金工业出版社,2004(343 页)
[2] 百度. 百度百科贪心算法,http://baike.baidu.com/view/298415.htm,2009-9-21
[3] 严蔚敏,陈文博. 数据结构及应用算法教程. 北京:清华大学出版社,2001
[4] 姜启源. 数学模型. 北京:高等教育出版社,1993
[5] 白其峥. 数学建模案例分析. 北京:北京海洋出版社,2000
[6] 张德富. 高级算法. 北京:国防大学出版社,2004

\end{document}