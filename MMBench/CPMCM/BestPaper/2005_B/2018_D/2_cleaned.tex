\begin{tabular}{ll}
学校 & 重庆大学 \\
\hline
参赛队号 & 18106110053 \\
\hline
队员姓名 & 1. 张路 \\
& 2. 杨丰祥 \\
& 3. 黄金辉 \\
\end{tabular}

\title{基于卫星高度计海面高度异常资料获取潮汐调和常数方法及应用}

\maketitle

\begin{abstract}
利用 T/P 卫星高度计资料分析海洋潮汐潮流现象,是动力海洋学研究的重要内容。在描述潮汐潮流特征过程中,潮汐调和常数的获取是重要的一项科学研究,直接影响海洋潮汐同潮图的绘制,为波浪、风暴潮、环流、水团等其他海洋现象的研究进展提供资料,是加强国防建设、促进海洋能源开发、保护环境、建设海港和保护海岸中不可或缺的一部分。本文基于卫星高度计资料探究潮汐调和常数计算方法及其应用,主要解决以下问题:

\textbf{问题 1} 针对整个卫星下观测点的海面高度异常值,利用调和分析方法对各个主要分潮 ($M_{2}, S_{2}, K_{1}, O_{1}$) 的潮汐调和常数进行了求取,求解过程中,基于达尔文-杜森模型,选取九个分潮 ($M_{2}, S_{2}, K_{1}, O_{1}, N_{2}, K_{2}, P_{1}, Q_{1}, S_{a}$) 进行数据拟合,并结合最小二乘法进行求解,最后通过验潮站数据和上下行轨道交叉点对结果进行验证:与 20 个验潮站调和常数结果相比,四个主要分潮绝均差分别为:$M_{2}, S_{2}, K_{1}, O_{1}$ 分潮振幅绝均差为 16.71、7.33、15.89 和 12.21cm;迟角绝均差为 6.94°、10.84°、9.15° 和 9.72°。20 个交叉点的各个分潮振幅和迟角绝均差:$M_{2}, S_{2}, K_{1}, O_{1}$ 振幅差 $\Delta H$ 均方根为 0.65、1.09、0.83、0.81cm,矢量差 $\Delta$ 均方根为 0.85、1.02、0.79、0.78。

\textbf{问题 2} 通过绘制四大主要分潮 ($M_{2}, S_{2}, K_{1}, O_{1}$) 潮汐调和常数沿卫星轨道的分布图,发现潮汐调和常数分布在空间上存在由内潮对正压潮调制引起的细节构,这主要表现为局部区域存在较大的数值波动。鉴于此,本文考虑使用多项式拟合方法,拟合正压潮的调和常数,并在此基础上,通过坐标变换获得内潮的调和常数,实现正压潮与内潮的分离,并分析了分离后的正压潮的调和常数的分布规律,得到同潮图插值经验规则。

\textbf{问题 3} 进一步针对南海所有卫星观测轨迹下计算得到的正压潮调和常数,利用克里金插值法分析各个分潮上的同振幅、同迟角的位置,并结合问题 2 中的经

验插值规则,从而获得各主要分潮正压潮的同迟角与同幅值分布图。通过与验潮站数据进行对比分析,结果表明:8大分潮正压潮幅值、迟角计算结果与验潮站对比的绝均差分别为:$M_{2}$、$S_{2}$、$K_{1}$、$O_{1}$、$N_{2}$、$K_{2}$、$P_{1}$ 和 $Q_{1}$ 分潮振幅绝均差为 9.68、15.22、10.94、12.59、13.64、12.86、11.06 和 17.41cm;迟角绝均差为 $8.81^\circ$、$8.65^\circ$、$11.81^\circ$、$8.85^\circ$、$9.45^\circ$、$8.53^\circ$、$10.35^\circ$ 和 $7.35^\circ$。

\textbf{问题 4} 针对前两问拟合和插值过程中可能存在的拟合残差较大的问题,本部分主要从两个方面出发:1) 针对不同次数下的多项式拟合分潮调和常数进行正压潮和内潮的分离分析讨论;2) 针对不同插值方法和方式的同潮图绘制的分析讨论。得到如下启示:1) 可以分上下行轨道分方法插值,设置权重,增加下行轨道对插值结果的权重比例,降低上行轨道振幅数据对插值结果的影响,从而可以获得较为完备的各个分潮的同潮分布图;2) 依据不同分潮类型的实际情况,进行不同的插值方式研究,最后进行叠加,从而达到完整的同潮图绘制。

关键词:卫星高度计;海面高度异常;潮汐调和常数;最小二乘法;调和分析
\end{abstract}

\tableofcontents

\section{一、问题背景及重述}
\subsection{问题背景}

海洋潮汐是在天体引潮力作用下形成的长周期波动现象,在水平方向上表现为潮流的涨落,在铅直方向上则表现为潮位的升降。潮汐潮流运动是海洋中的基本运动之一,它是动力海洋学研究的重要组成部分,对它的研究直接影响着波浪、风暴潮、环流、水团等其他海洋现象的研究,在大陆架浅海海洋中,对潮汐潮流的研究更具重要性。

海岸附近和河口区域是人类进行生产活动十分频繁的地带,而这个地带的潮汐现象非常显著,它直接或间接地影响着人们的生产和生活。潮汐潮流工作的开展和研究,可为国防建设、交通航运、海洋资源开发、能源利用、环境保护、海港建设和海岸防护提供资料。例如,沿海地区的海滩围垦、农田排灌,水产的捕捞和养殖,制盐,海港的选址及建设,以至于潮能发电等活动,无不与潮汐潮流现象有着密切的关系。

区域海洋潮汐的数值模拟需要提供开边界的水位调和常数,而开边界的水位调和常数,或者来源于观测、或者来源于全球海洋潮汐的数值模拟;而全球海洋潮汐的数值模拟,相当耗费资源。虽然目前有国外学者或研究机构,能够提供区域海洋潮汐的调和常数,但实质上的评价结果难以令人满意。

从区域海洋潮汐的数值模拟的现状来讲,四个主要分潮($M_{2}$、$S_{2}$、$K_{1}$、$O_{1}$)的单一分潮的数值模拟与同化可以得到令人满意的结果,但其它分潮($N_{2}$、$K_{2}$、$P_{1}$、$Q_{1}$等)的单一分潮的数值模拟与同化,结果却差强人意;这意味着其它分潮的数值模拟,只有与四个主要分潮同时进行数值模拟,才能得到可以接受的结果。从具体操作来讲,其它分潮由于相对较弱,导致模拟结果的精度难以提高。

长周期分潮($S_{a}$、$S_{sa}$、$M_{m}$、$M_{f}$)的获取,目前已有基于全球长周期分潮数值模拟手段的报道,但其面临的困境,与其它较弱分潮面临的困境没有差别。

从各分潮的调和常数获取的发展史来说,通过对已有观测结果进行插值曾经是首选,但发展过程中逐渐被数值模拟方法所取代。高度计资料的出现,引发部分学者开展了插值方法的研究,并取得了一些值得一提的结果,尽管被所谓的主流方式淹没,但也难掩其光芒所在。鉴于目前已有高度计资料作为支持,其它分潮及长周期分潮的调和常数获取的插值方法研究大有可为。

\subsection{资料描述}

\subsubsection{地形数据}

地形数据来自 ETOP5,全球的分辨率为 $5' \times 5'$,图 1.1 的区域是 $2^{\circ} \sim 25^{\circ} \text{N}$,$99^{\circ} \sim 122^{\circ} \text{E}$。

\begin{figure}[h]
    \centering
    \includegraphics[width=\textwidth]{image1.png}
    \caption{南海地形图}
    \label{fig:1.1}
\end{figure}

\subsubsection{验潮站资料}

中国近海及周边海域 770 个验潮点的资料,和 56 个验潮点的资料(是国际上公开的长期验潮站数据分析得到的调和常数),包括 9 个分潮(\(M_{2}\)、\(S_{2}\)、\(K_{1}\)、\(O_{1}\)、\(N_{2}\)、\(K_{2}\)、\(P_{1}\)、\(Q_{1}\)、\(S_{a}\))的潮汐调和常数。

图 \ref{fig:1.2} 显示了上述资料点所在的位置,从图中可以看出上述验潮点主要分布在近岸或岛屿附近。

\begin{figure}[h]
    \centering
    \includegraphics[width=\textwidth]{image2.png}
    \caption{验潮站资料的分布图}
    \label{fig:1.2}
\end{figure}

\subsubsection{TOPEX/POSEIDON 卫星高度计资料}

TOPEX/POSEIDON 卫星是由美国国家航空航天局和法国空间局联合于 1992 年 8 月 10 日发射的,是世界上第一颗专门为研究世界大洋环流而设计的高度计卫星。其轨道高度达 1336km,倾角为 \(66^\circ\),覆盖面大,保证了资料的连续性。轨道的交点周期(绕地球一圈的时间)为 6745.8s,轨道运行 127 圈以后精确重复,轨道重复周期为 9.9156 天。相邻最近的轨道之间在赤道上的间隔为 \(360^\circ / 127 = 2.835^\circ\)。卫星在一个周期内的每一圈分为上行轨和下行轨两条轨道,一

个完整的周期内共有 254 条轨道,沿轨道的两个相邻的星下观测点的距离 5.75km。高度计系统的定规精度和测高精度较以前有显著提高,其测量精度约为 5cm,是目前观测海面高度精度最高的卫星。

当然,本文只是涉及到 TOPEX/POSEIDON 卫星高度计资料与潮汐相关的研究,即海面高度异常产品。

\subsubsection{南海高度计资料}

图 1.3 的给出了 $2^{\circ} \sim 25^{\circ} \mathrm{N}, 99^{\circ} \sim 122^{\circ} \mathrm{E}$,TOPEX/POSEIDON 卫星高度计星下观测点所在的轨道。一共有超过 4000 个数据点,每个点都对应一个海面高度异常的时间序列,从 1992 年到 2017 年,时间跨度为 25 年。

\begin{figure}[h]
    \centering
    \includegraphics[width=\textwidth]{image.png}
    \caption{南海 TOPEX/POSEIDON 高度计资料的星下轨迹}
    \label{fig:1.3}
\end{figure}

\section{1.3 问题提出}

\subsection{1.3.1 问题 1:潮汐调和常数求取与评价}

根据沿轨道的星下观测点的海面高度异常值,提取所有星下观测点各主要分潮($M_{2}$、$S_{2}$、$K_{1}$、$O_{1}$)的潮汐调和常数,注意能有效提取那些分潮的潮汐调和常数取决于相应的资料长度;对提取的潮汐调和常数,应利用潮汐验潮点的调和常数给予评价或检验,并给出评价结果的分析或评价。

\subsection{1.3.2 问题 2:正压潮和内潮的分离}

得到所有星下观测点各主要分潮($M_{2}$、$S_{2}$、$K_{1}$、$O_{1}$)的潮汐调和常数,沿轨道作图后,可发现潮汐调和常数在沿轨道方向,在空间有细结构,而此细结构是内潮对正压潮的调制;请设法对沿轨道的各分潮的潮汐调和常数进行正压潮和内潮的分离。

\subsection{1.3.3 问题 3:分潮特征同潮图分析}

设计数据插值或拟合方法给出南海的各主要分潮的同潮图,并利用潮汐验潮点的调和常数给予评价或检验,并给出评价结果的分析或评价。

\subsection{1.3.4 问题 4:潮汐调和分析优化过程讨论}

如果你们还有时间和兴趣,还可考虑下列:

如果在对沿轨道的潮汐调和常数分离、插值或拟合的过程中,利用了特定的函数进行拟合,是否能够确定出需利用的特定函数的最佳(高)次数?上述结论是否对第 3 问有启示或帮助。

\section{二、问题分析}

\subsection{2.1 问题 1:潮汐调和分析和评价方法}

问题 1 中主要针对沿轨道的星下观测点的海面高度异常值,需要提取星下测点各个主要分潮的潮汐调和常数。需要对潮汐调和常数的获取,进行调和分析,利用最小二乘法求解线性线性方程组,最终需要借助可行的评价方法对求得的分潮调和常数进行对比和验证,本文主要利用上下行轨道交叉点调和常数和验潮站的调和常数分别进行验证和对比分析。

\subsection{2.2 问题 2:沿轨道正压潮和内潮分离方法}

问题 2 中主要利用沿轨道做出的各主要分潮的潮汐调和常数(分潮振幅和迟角)分析图,得到潮汐调和常数在轨道方向上,有内潮对正压潮调制引起的细结构这一结论。鉴于此,需要对正压潮和内潮进行分离。这方面的研究主要为特定形式的多项式拟合方法,对正压潮和内潮(斜压潮)时间序列进行调和分析,分离出正压潮和内潮的主要成分即调和常数参量。从而得到较为平滑的正压潮调和常数参量,为第三问同潮图的绘制提供基础资料。

\subsection{2.3 问题 3:分潮特征同潮图分析}

问题 3 主要是设计数据插值或拟合方法以得到南海的各主要分潮的同潮图。利用第二问中分离得到的平滑的正压潮的潮汐调和常数的参量,结合自然邻居插值方法,分别对各个主要分潮($M_{2}$、$S_{2}$、$K_{1}$、$O_{1}$)和另外 4 个分潮($N_{2}$、$K_{2}$、$P_{1}$、$Q_{1}$)的调和参数进行插值,从而绘制出其正压潮振幅和迟角的同潮分布图,最后利用验潮站的数据,对插值后得到的靠近验潮站的数据进行评价和检验。

\subsection{2.4 问题 4:潮汐调和分析优化过程讨论}

问题 4 主要针对沿轨道的潮汐调和常数分离、插值或拟合过程中,使用了特定格式的拟合函数,主要从两个方面进行分析和讨论:1)对于分离过程中,拟合正压潮的多项式,分析了 5、7、9、11、13 次多项式的拟合结果,优选出(多项式最佳形式)最佳次数下的多项式;2)从空间插值法进行改进,利用部分加权的方法对时间序列差别较大的数据设置权值(例如:上下行轨道不同权值插值),从而利于各个分潮同潮图的特征展示。

\begin{figure}[h]
\centering
\includegraphics[width=\textwidth]{research_flowchart.png}
\caption{研究技术路线图}
\label{fig:research_flowchart}
\end{figure}

\section{三、模型假设}

为了使实际问题的解决具有可行性,同时易于数学模型的建立,在针对性建模之前需要做一定的假设。

\begin{enumerate}
    \item 对于 T/P 卫星高度计的测量误差,符合正态分布,不存在测量错误;
    \item 在分析实际潮汐的分潮过程中,不考虑非天文分潮;
    \item 逐时观测记录完整并且连续,允许潮时误差;
    \item 潮高水位平均水位可取近似值;
    \item 潮汐调和常数的差比关系用近似代替。
\end{enumerate}

\section{四、名词解释及符号说明}

\begin{tabular}{ll}
学校 & 重庆大学 \\
\hline
参赛队号 & 18106110053 \\
\hline
队员姓名 & 1. 张路 \\
& 2. 杨丰祥 \\
& 3. 黄金辉 \\
\end{tabular}

\section{五、模型的建立与求解}

\subsection{问题一:潮汐调和分析方法}

\subsubsection{卫星测高分潮混叠}

执行重复轨迹任务的测高卫星,连续两次通过某上升轨迹或下降轨迹的时间间隔为其重复周期,也即对海面高的采样间隔 $T_{\text {satPeriod }}$。根据 Nyquist 采样定理,当采样时间间隔大于信号的半周期时,会产生频率混叠效应。测高卫星的时间序列可恢复信号的最高频率为 Nyquist 频率,为 $1/(2T_{\text {satPeriod }})$,对应的信号周期为 $2T_{\text {satPeriod }}$。对于 T/P 卫星而言 $T_{\text {satPeriod }}$、为 9.9156 天,在此期间卫星绕地球旋转 127 圈,其结果是直接探测信号的周期在 20 天以上,远大于潮汐主要半日分潮和日分潮的周期,其相应的折叠频率远低于主要潮汐频率,因而其采样必然产生潮致高频混淆。

折叠频率对应的最小周期为混淆周期,对于周期为 $T_{\text {TidePeriod }}$(小时)的分潮而言,分潮的混淆周期 $T_{\text {AliasPeriod }}$ 为:

\begin{equation}
t = T_{\text {satPeriod }} \times 24 / T_{\text {TidePeriod }}
\tag{5.1}
\end{equation}

\begin{equation}
t = t - [t]
\tag{5.2}
\end{equation}

$[t]$ 表示不大于 $t$ 的最大整数,且

\[
t =
\begin{cases}
t & t < 0.5 \\
1 - t & t > 0.5
\end{cases}
\]

所以对于分潮 $T_{\text {TidePeriod }}$ 而言,其混淆周期 $T_{\text {AliasPeriod }}$ 为:

\begin{equation}
T_{\text {AliasPeriod }} = T_{\text {SatPeriod }} / t
\tag{5.3}
\end{equation}

\begin{tabular}{ll}
Lagrange 插值 & $p_n(x) = \sum_{i=1}^n f(x_i) l_i(x)$ \\
Newton 插值 & $N_n(x) = f\left[x_0\right] + f\left[x_0, x_1\right] \omega(x) + \cdots + f\left[x_0, x_1, \ldots, x_n\right] \omega_n(x)$ \\
Hermite 插值 & $H_{2n+1}(x) = \sum_{k=0}^n y_k \left[1 - 2l_k'(x_k)(x - x_k)\right] l_k^2(x) + \sum_{k=0}^n y_k'(x - x_k) l_k^2(x)$ \\
三次样条插值 & $S_3(x) = P_3(x) + \sum_{i=1}^{n-1} C_i (x - x_i)_+^3$
\end{tabular}

与以往卫星相比,随着设备精度的提高和校正工作的改进,T/P 卫星的径向轨道误差有大幅度降低,且其轨道设计使 8 个主要分潮的混叠周期均低于半年。对于混淆频率为 $T_1$ 和 $T_2$ ($T_1 > T_2$) 的两个潮汐信号,实现可靠分离所需要的采样时间 $T_d$ 为:

\begin{equation}
T_d = T_1 \times T_2 / (T_1 - T_2) / 4
\tag{5.4}
\end{equation}

上式中“4”,表明要区分这两个分潮,采样周期至少要是信号周期的四分之一。由此计算的在 T/P 采样情况下 8 个主要潮汐分潮之间的分辨时间如下表 5.2 所示。

\begin{table}
\centering
\caption{主要分潮基本分辨时间/天}
\begin{tabular}{|c|c|c|c|c|c|c|c|c|}
\hline
 & $M_{2}$ & $S_{2}$ & $N_{2}$ & $K_{2}$ & $O_{1}$ & $P_{1}$ & $K_{1}$ & $Q_{1}$ \\ \hline
$M_{2}$ & 0 & 271.7 & 61 & 54.9 & 43.3 & 51.5 & 24.2 & 148.8 \\ \hline
$S_{2}$ &  & 0 & 78.8 & 45.7 & 51.5 & 43.3 & 22.2 & 96.1 \\ \hline
$N_{2}$ &  &  & 0 & 28.9 & 148.8 & 27.9 & 17.3 & 43.3 \\ \hline
$K_{2}$ &  &  &  & 0 & 24.2 & 844.2 & 43.3 & 87 \\ \hline
$O_{1}$ &  &  &  &  & 0 & 23.5 & 15.5 & 33.5 \\ \hline
$P_{1}$ &  &  &  &  &  & 0 & 45.7 & 78.8 \\ \hline
$K_{1}$ &  &  &  &  &  &  & 0 & 28.9 \\ \hline
$Q_{1}$ &  &  &  &  &  &  &  & 0 \\ \hline
\end{tabular}
\end{table}

从上表可以看出,区分 $M_{2}$ 和 $S_{2}$ 分潮,需要 0.7 年 (271.7 天) 的数据;区分 $K_{1}$ 和 $P_{1}$ 分潮,需要约 2.3 年 (844.2 天) 的数据,该时间是 8 个主要分潮中分辨两个分潮所需要的最长时间。因此,要获取 $M_{2}$、$S_{2}$、$N_{2}$、$K_{1}$、$O_{1}$、$Q_{1}$、$K_{1}$ 和 $P_{1}$ 稳定的调和常数,至少要 100 个采样点 (约 1000 天)。因此,本文采用的 T/P 卫星高度计时间序列的长度足够对以上主要分潮进行调和分析。

\subsection{5.1.2 利用达尔文-杜森模型 ($f/u$) 进行潮汐的调和分析}

实际潮汐的分潮从其来源看可分为以下四种:天文分潮、气象分潮、天文-气象分潮和浅水分潮。

从分潮的频率分布来看,分潮在频率上的分布是极不均匀的,而是分成族、群和亚群。在 Doodson 展开中,按 Doodson 数 $\mu_{1}$ 区分潮族,按 $\mu_{2}$ 区分群,按 $\mu_{3}$ 区分亚群。在潮族中一般分为长周期分潮族 ($\mu_{1}=0$)、全日分潮族 ($\mu_{1}=1$)、半日分潮族 ($\mu_{1}=2$)、三分日分潮族 ($\mu_{1}=3$) 直到十二分日分潮族 ($\mu_{1}=12$),共 13 个潮族。在每一个潮族中,具有不同数量的群和亚群。

在亚群中的各个分潮的角速度是非常接近的,彼此之间只有微小的差异。因此,在资料长度有限的情况下,亚群中的各个分潮是无法区分的。因此,在实际的潮汐分析中,往往将一个亚群合成一个分潮,此时这一分潮的振幅和迟角不再是常数,而是随着升交点的黄经十分缓慢地变化,一般在较短的时间内可近似看作不变。这样的分潮实质上是准调和的,但习惯上仍叫做调和分潮。

实际水位可以看作是很多个调和分潮迭加的结果,但是在实际分析中只能选取其中有限个较主要的分潮。假设我们选取了 $J$ 个分潮,对于任一点的潮位表达式为:
\begin{equation}
h = S_{0} + \sum_{j=1}^{J} f_{j} h_{j} \cos (\nu_{j} + u_{j} - g_{j}) = S_{0} + \sum_{j=1}^{J} f_{j} h_{j} \cos (\sigma_{j} t + \nu_{0j} + u_{j} - g_{j})
\tag{5.1}
\end{equation}
其中,$S_{0}$ 为余水位,$f_{j}$ 为交点因子,$u_{j}$ 为交点订正角,$h_{j}, g_{j}$ 为分潮的调和常数(振幅和迟角)。

\subsubsection{(1) 分潮角速度的计算}
\begin{equation}
\sigma = \mu_{1} \tau + \mu_{2} s + \mu_{3} h' + \mu_{4} p + \mu_{5} N' + \mu_{6} p'
\tag{5.2}
\end{equation}
其中:$\sigma$ 为分潮的角速度,$\mu_{1}, \mu_{2}, \mu_{3}, \mu_{4}, \mu_{5}, \mu_{6}$ 为 Doodson 数,

\begin{itemize}
    \item $\tau = 14.49205211$
    \item $s = 0.54901653$
    \item $h' = 0.04106864$ \hfill (单位:度/平太阳时)
    \item $p = 0.00464183$
    \item $N' = 0.00220641$
    \item $p' = 0.00000196$
\end{itemize}

\subsubsection{分潮初相位的计算}

$Y$ 年 $M$ 月 $D$ 日 $t$ 时刻(实际计算中是观测数据的起始时间)的天文初相角:

\begin{equation}
\nu_0 = \mu_1 \tau + \mu_2 s + \mu_3 h' + \mu_4 p + \mu_5 N' + \mu_6 p' + \mu_0 90
\tag{5.3}
\end{equation}

其中:$\mu_0, \mu_1, \mu_2, \mu_3, \mu_4, \mu_5, \mu_6$ 为 Doodson 数,

\begin{equation}
\begin{cases}
s = 277.02 + 129.3848 (Y - 1900) + 13.1764 \left( n + i + \frac{t}{24} \right) \\
h' = 280.19 - 0.2387 (Y - 1900) + 0.9857 \left( n + i + \frac{t}{24} \right) \\
p = 334.39 + 40.6625 (Y - 1900) + 0.1114 \left( n + i + \frac{t}{24} \right) \\
N' = 100.84 + 19.3282 (Y - 1900) + 0.0530 \left( n + i + \frac{t}{24} \right) \\
p' = 281.22 + 0.0172 (Y - 1900) + 0.00005 \left( n + i + \frac{t}{24} \right) \\
\tau = 15t - s + h'
\end{cases}
\tag{5.4}
\end{equation}

式中 $i$ 为 1900 年至 $Y$ 年的闰年数,$i = \text{int} \left( \frac{Y - 1901}{4} \right)$;$n$ 为从 $Y$ 年 1 月 1 日开始计算的累积日期序数,1 月 1 日的日期序数为 0,$t$ 为时间(单位:小时)。以上各式中的单位是度。

\begin{table}[h]
\centering
\caption{部分分潮的 Doodson 数、分潮角速度和交点因子与订正角}
\begin{tabular}{c|cccccc|c|cc}
\hline
分潮符号 & \multicolumn{6}{c|}{Doodson 数} & \multicolumn{1}{c|}{分潮角速度} & \multicolumn{2}{c}{交点因子与订正角} \\
 & $\mu_1$ & $\mu_2$ & $\mu_3$ & $\mu_4$ & $\mu_5$ & $\mu_6$ & 单位:度/平太阳时 & $f$ & $u$ \\
\hline
$S_a$ & 0 & 0 & 1 & 0 & 0 & 0 & 0.0410686 & 1 & 0 \\
$S_{sa}$ & 0 & 0 & 2 & 0 & 0 & 0 & 0.0821373 & 1 & 0 \\
$M_m$ & 0 & 1 & 0 & -1 & 0 & 0 & 0.5443747 & $M_m$ & $M_m$ \\
$MS_f$ & 0 & 2 & -2 & 0 & 0 & 0 & 1.0158958 & $M_2$ & $-M_2$ \\
$M_f$ & 0 & 2 & 0 & 0 & 0 & 0 & 1.0980331 & $M_f$ & $M_f$ \\
$Q_1$ & 1 & -2 & 0 & 1 & 0 & -1 & 13.3986609 & $O_1$ & $O_1$ \\
\hline
\end{tabular}
\end{table}

\begin{table}
\centering
\begin{tabular}{c|c c c c c c|c|c}
\hline
分潮符号 & \multicolumn{6}{c|}{Doodson数} & 分潮角速度 & 交点因子与订正角 \\
 & $\mu_{1}$ & $\mu_{2}$ & $\mu_{3}$ & $\mu_{4}$ & $\mu_{5}$ & $\mu_{6}$ & 单位:度/平太阳时 & \\
\hline
$O_{1}$ & 1 & -1 & 0 & 0 & 0 & 0 & 13.9430356 & $O_{1}$ & $O_{1}$ \\
$M_{1}$ & 1 & 0 & 0 & 0 & 0 & 0 & 14.4920521 & $M_{1}$ & $M_{1}$ \\
$P_{1}$ & 1 & 1 & -2 & 0 & 0 & 0 & 14.9589314 & $P_{1}$ & $P_{1}$ \\
$S_{1}$ & 1 & 1 & -1 & 0 & 0 & 0 & 15.0000000 & 1 & 0 \\
$K_{1}$ & 1 & 1 & 0 & 0 & 0 & 0 & 15.0410686 & $K_{1}$ & $K_{1}$ \\
$J_{1}$ & 1 & 2 & 0 & -1 & 0 & 0 & 15.5854434 & $J_{1}$ & $J_{1}$ \\
$OO_{1}$ & 1 & 3 & 0 & 0 & 0 & 0 & 16.1391017 & $OO_{1}$ & $OO_{1}$ \\
$N_{2}$ & 2 & -1 & 0 & 1 & 0 & 0 & 28.4397295 & $M_{2}$ & $M_{2}$ \\
$M_{2}$ & 2 & 0 & 0 & 0 & 0 & 0 & 28.9841042 & $M_{2}$ & $M_{2}$ \\
$L_{2}$ & 2 & 1 & 0 & -1 & 0 & 0 & 29.5284789 & $L_{2}$ & $L_{2}$ \\
$S_{2}$ & 2 & 2 & -2 & 0 & 0 & 0 & 30.0000000 & 1 & 0 \\
$k_{2}$ & 2 & 2 & 0 & 0 & 0 & 0 & 30.0821373 & $k_{2}$ & $k_{2}$ \\
$M_{4}$ & 4 & 0 & 0 & 0 & 0 & 0 & 57.9682085 & $M_{2}^{2}$ & $2M_{2}$ \\
$MS_{4}$ & 4 & 2 & -2 & 0 & 0 & 0 & 58.9841043 & $M_{2}$ & $M_{2}$ \\
$M_{6}$ & 6 & 0 & 0 & 0 & 0 & 0 & 86.9523127 & $M_{2}^{3}$ & $3M_{2}$ \\
\hline
\end{tabular}
\caption{表中交点因子及交点订正角的含义说明:例如,表中 $M_{6}$ 分潮的交点因子是 $M_{2}$ 分潮的交点因子的三次方,$M_{6}$ 分潮的交点订正角是 $M_{2}$ 分潮的交点订正角的三倍。}
\end{table}

(3) $f_{j}$ 和 $u_{j}$ 的计算

由于 $f_{j}$ 和 $u_{j}$ 随时间变化非常缓慢,一般情况下取资料序列的中间时刻计算。各分潮的 $f_{j}$、$u_{j}$ 的具体计算公式如下:

\begin{equation}
\left\{
\begin{aligned}
f \cos u &= \sum_{m=1}^{M} \rho_{m} \cos (\Delta \mu_{4}^{m} p + \Delta \mu_{5}^{m} N') \\
f \sin u &= \sum_{m=1}^{M} \rho_{m} \sin (\Delta \mu_{4}^{m} p + \Delta \mu_{5}^{m} N')
\end{aligned}
\right.
\tag{5.5}
\end{equation}

其中,$\rho_{m}$、$\Delta \mu_{4}^{m}$、$\Delta \mu_{5}^{m}$ 和 Doodson 数见表 5.4。

\begin{table}
\centering
\caption{$\rho_{m}$、$\Delta\mu_{4}^{m}$、$\Delta\mu_{5}^{m}$ 列表}
\begin{tabular}{c c | c | c | c | c | c | c | c | c | c | c}
\hline
$\Delta\mu_{4}$ & $\Delta\mu_{5}$ & $M_{m}$ & $M_{f}$ & $O_{1}$ & $P_{1}$ & $K_{1}$ & $J_{1}$ & $OO_{1}$ & $M_{2}$ & $L_{2}$ & $k_{2}$ \\
\hline
-2 & -1 & & 2.3e-3 & & 2e-4 & & 3.7e-3 & & & & \\
-2 & 0 & & 4.32e-2 & & & & 0.149 & & & & \\
-2 & 1 & & 2.8e-3 & & & & 2.96e-2 & & & & \\
0 & -2 & 8e-4 & & 5.8e-3 & 8e-4 & 1e-4 & & 5e-4 & & & 1.28e \\
0 & -1 & -6.57e-2 & & 1.885e-1 & 1.12e-2 & 1.98e-2 & 2.94e-2 & 3.73e-2 & 3.66e-2 & & -2 \\
0 & 0 & 1 & & 1 & 1 & 1 & 1 & 1 & 2 & & 1 \\
0 & 1 & -0.0649 & 0.4143 & & 0.1356 & & 0.1980 & 0.6398 & & & 0.298 \\
0 & 2 & & 3.87e-2 & & & & 4.7e-3 & 1.342e-1 & & & 3.24e-2 \\
0 & 3 & & -8e-4 & & & & 0.0086 & & & & \\
2 & -1 & & & 2e-4 & & & & & 0.0047 & & \\
2 & 0 & -5.34e-2 & & -6.4e-3 & -1.5e-3 & & 1.52e-2 & 6e-4 & 2.505e-1 & & \\
2 & 1 & -2.18e-2 & & -1e-3 & 3e-4 & & 9.8e-3 & 2e-4 & 1.102e-1 & & \\
2 & 2 & -5.9e-3 & & & & & 5.7e-3 & & 1.56e-2 & & \\
\hline
\end{tabular}
\end{table}

对于 $M_{m}$、$Mf$、$O_{1}$、$P_{1}$、$K_{1}$、$J_{1}$、$OO_{1}$、$M_{2}$、$L_{2}$、$k_{2}$ 分潮的 $f$ 和 $u$ 依照上式计算,其他分潮由这些分潮组合计算,但 $M_{1}$ 分潮的 $f$ 和 $u$ 由以下公式计算得出:

\begin{equation}
f\cos u = -0.008\cos(-p-2N') + 0.094\cos(-p-N') + 0.510\cos p - 0.041\cos(p-N') \tag{5.6}
\end{equation}
\begin{equation}
+ 1.418\cos p + 0.284\cos(p+N') - 0.008\cos(p+2N')
\end{equation}

\begin{equation}
f\sin u = -0.008\sin(-p-2N') + 0.094\sin(-p-N') - 0.510\sin p - 0.041\sin(p-N') \tag{5.7}
\end{equation}
\begin{equation}
+ 1.418\sin p + 0.284\sin(p+N') - 0.008\sin(p+2N')
\end{equation}

\subsubsection{(4) 最小二乘法提取分潮调和常数}

在进行潮汐调和分析时,对某一确定的分潮:

\begin{equation}
f_{j}H_{j}\cos(\sigma_{j}t+V_{0j}+u_{j}-g_{j}) \tag{5.8}
\end{equation}

可化为如下形式:

\begin{equation}
f_{j}\cos(\sigma_{j}t+V_{0j}+u_{j})H_{j}\cos g_{j} + f_{j}\sin(\sigma_{j}t+V_{0j}+u_{j})H_{j}\sin g_{j} \tag{5.9}
\end{equation}

其中,$f_{j}$、$\sigma_{j}$、$t$、$u_{j}$、$V_{0j}$ 均为已知或可通过简单计算得出。对应 $J$ 个分潮,则有:

\begin{equation}
h = S_{0} + \sum_{j=1}^{J}f_{j}\cos(\sigma_{j}t+V_{0j}+u_{j})H_{j}\cos g_{j} + \sum_{j=1}^{J}f_{j}\sin(\sigma_{j}t+V_{0j}+u_{j})H_{j}\sin g_{j} \tag{5.10}
\end{equation}

如果在 $n$ 个时刻 $t=t_{1}, t_{2}, \cdots, t_{n}$,有 $n$ 个潮高观测值 $h=h_{1}, h_{2}, \cdots, h_{n}$,那么,就可以建立如下由 $n$ 个方程构成的方程组:

\begin{equation}
\left\{
\begin{aligned}
S_{0}+\sum_{j=1}^{J} x_{j} f_{j} \cos \left(\sigma_{j} t_{1}+V_{0}+u_{j}\right)+\sum_{j=1}^{J} y_{j} f_{j} \sin \left(\sigma_{j} t_{1}+V_{0}+u_{j}\right) &= h_{1} \\
S_{0}+\sum_{j=1}^{J} x_{j} f_{j} \cos \left(\sigma_{j} t_{2}+V_{0}+u_{j}\right)+\sum_{j=1}^{J} y_{j} f_{j} \sin \left(\sigma_{j} t_{2}+V_{0}+u_{j}\right) &= h_{2} \\
& \vdots \\
S_{0}+\sum_{j=1}^{J} x_{j} f_{j} \cos \left(\sigma_{j} t_{n}+V_{0}+u_{j}\right)+\sum_{j=1}^{J} y_{j} f_{j} \sin \left(\sigma_{j} t_{n}+V_{0}+u_{j}\right) &= h_{n}
\end{aligned}
\right.
\tag{5.11}
\end{equation}

方程组中 $x_{j}$ 对应 $H_{j} \cos g_{j}$,$y_{j}$ 对应 $H_{j} \sin g_{j}$,它们和 $S_{0}$ 共同构成了方程组中的全部未知量。潮汐调和分析的目的正是求出 $x_{j}$ 与 $y_{j}$,从而求出各个分潮的调和常数 $H$ 和 $g$。

为了表示的方便,将方程组写成如下形式:

\begin{equation}
\left\{
\begin{aligned}
S_{0}+a_{11} x_{1}+a_{12} x_{2}+\cdots+a_{1m} x_{m} &= h_{1} \\
S_{0}+a_{21} x_{1}+a_{22} x_{2}+\cdots+a_{2m} x_{m} &= h_{2} \\
& \vdots \\
S_{0}+a_{n1} x_{1}+a_{n2} x_{2}+\cdots+a_{nm} x_{m} &= h_{n}
\end{aligned}
\right.
\tag{5.12}
\end{equation}

其中 $m=2J$,即所选分潮数的两倍,$S_{0}$ 与 $x_{j}$ 为待求解的未知数。

为了尽量减小噪声 $r$ 对分析结果的影响,使调和常数尽可能接近真值,在实际潮汐分析中,总是希望使用更多的观测数据。因此,方程的数量 $n$ 一般远大于未知数的数量 $m+1$。对于这样的矛盾方程组,可以用最小二乘法来求解。

最小二乘法的思想是,寻求一组解,使拟合值与实际值之差的平方和,即下式的值达到最小。

\begin{equation}
\Delta=\sum_{i=1}^{n}\left(a_{i1} x_{1}+a_{i2} x_{2}+\cdots+a_{im} x_{m}+S_{0}-h_{i}\right)^{2}
\tag{5.13}
\end{equation}

根据多元函数微分的理论,这要求 $\Delta$ 相对于各未知数的偏导数均为 0,即:

\begin{equation}
\frac{\partial \Delta}{\partial S_{0}}=\frac{\partial \Delta}{\partial x_{1}}=\frac{\partial \Delta}{\partial x_{2}}=\cdots=\frac{\partial \Delta}{\partial x_{m}}=0
\tag{5.14}
\end{equation}

这样就可得到关于分潮调和常数的线性方程组,进而可提取出分潮的调和常数,具体流程可以参考以下算法设计。

\subsection{5.1.3 高度计资料的调和分析算法设计}

分别利用上述研究方法,对卫星高度计资料进行分潮分析,得到 4 个主要分潮($M_{2}, S_{2}, K_{1}, O_{1}$)和另外 4 个分潮($N_{2}, K_{2}, P_{1}, Q_{1}$)的潮汐调和常数,根据文献[]表示,对于余水位 $S_{0}$ 大于 $1 \mathrm{~cm}$ 的点需要被剔除,从而可以得到各分潮潮汐调和常数(振幅 $H$ 和迟角 $g$)的结果如表所示。其中:$M_{2}, S_{2}, N_{2}, K_{2}$ 为半日分潮;$K_{1}, O_{1}, P_{1}, Q_{1}, M_{1}$ 为全日分潮。

\subsubsection{问题:分潮调和常数获取与评价算法设计}

步骤 1:根据 T/P 卫星测量水深数据进行数据预处理,得到各测点经纬度坐标和测量水深数据。

步骤 2:计算分潮角速度、分潮初相位及 $f$、$u$ 的计算;

步骤 3:建立各个分潮的调和常数求解方程组,利用最小二乘法对线性方程组进行求解;

步骤 4:利用上诉求解得到的结果,根据坐标变换得到各个分潮的调和常数(振幅 $H$ 和迟角 $G$)。

\subsection{5.1.4 结果对比分析}

在对卫星高度计计算得到的各主要分潮($M_{2}$、$S_{2}$、$K_{1}$、$O_{1}$)的潮汐调和参数进行评价和检验过程中,主要分为两种方式进行对比:(1)与验潮站的分潮调和常数结果对比:选出离卫星高度计测量近的验潮站作为对比评价的目标,需要筛选出能够用于评价的验潮站点坐标,并对比分析靠离卫星高度计测量附近的验潮站分潮调和数据;(2)对比轨道交点处各分潮调和常数偏差:对比分析上行轨道和下行轨道交叉点的 8 个主要分潮的潮汐调和常数的误差。

\subsubsection{(1) 与验潮站的分潮调和常数结果对比}

\begin{enumerate}
    \item \textbf{验潮站点选择:} 利用卫星高度计测量数据和验潮站数据,以卫星高度计距离测量位置最近为准则选择验潮站,通过筛选得到用于结果对比的验潮站的位置如图 5.1 所示。
\end{enumerate}

\begin{figure}[h]
    \centering
    \includegraphics[width=0.8\textwidth]{image.png}
    \caption{验潮站点和卫星高度计测量轨道示意图}
    \label{fig:5.1}
    \begin{center}
        (红色代表本文参考验潮站,蓝色代表未使用验潮站)
    \end{center}
\end{figure}

\begin{enumerate}
    \setcounter{enumi}{1}
    \item \textbf{结果对比分析:} 对比验证给出了模拟海区内位于中国南沿海和世界沿海共 826 个验潮站 4 个主要分潮调和常数(振幅 $H$ 和迟角 $G$)的模拟结果(见表 5.5)。验证结果显示:$M_{2}$ 分潮振幅和迟角的平均绝对误差分别为 $16.71\,\text{cm}$ 和 $6.94^\circ$;$S_{2}$ 分潮振幅和迟角的平均绝对误差分别为 $7.33\,\text{cm}$ 和 $10.84^\circ$;$K_{1}$ 分潮振幅和迟角的平均绝对误差分别为 $15.89\,\text{cm}$ 和 $9.15^\circ$;$O_{1}$ 分潮振幅和迟角的平均绝对误差分别为 $12.21\,\text{cm}$ 和 $9.72^\circ$。总体来看,模拟结果与实测结果符合较好,能够模拟出南
\end{enumerate}

\begin{table}
\centering
\begin{tabular}{c c c c c c c c c}
\hline
\multirow{2}{*}{分潮类型} & \multicolumn{2}{c}{验潮站} & \multicolumn{3}{c}{$H/cm$} & \multicolumn{3}{c}{$G/^{\circ}$} \\
\cline{2-9}
 & 纬度 & 经度 & 观察值 & 计算值 & 绝均差 & 观察值 & 计算值 & 绝均差 \\
\hline
\multirow{8}{*}{$O_{1}$} & 23.183 & 120.067 & 13.2 & 2.86 & 10.34 & 267.3 & 268.80 & 1.50 \\
 & 23.45 & 120.133 & 15.7 & 2.86 & 12.84 & 255.5 & 268.80 & 13.30 \\
 & 22.617 & 120.25 & 15.9 & 2.86 & 13.04 & 260.3 & 268.80 & 8.50 \\
 & 22.617 & 120.267 & 15.2 & 2.86 & 12.34 & 248.86 & 268.80 & 19.94 \\
 & 17.783 & 120.417 & 18.5 & 9.97 & 8.53 & 264.1 & 274.57 & 10.47 \\
 & 24.267 & 120.517 & 17.6 & 8.35 & 9.25 & 142.7 & 143.30 & 0.60 \\
 & 13.9 & 121.6 & 28.7 & 9.97 & 18.73 & 279.9 & 274.57 & 5.33 \\
 & 6.55 & 121.867 & 19.5 & 6.91 & 12.59 & 269.7 & 287.82 & 18.12 \\
\hline
\multirow{4}{*}{$K_{1}$} & 120.07 & 16.142 & 19 & 2.86 & 16.14 & 278.3 & 268.80 & 9.50 \\
 & 120.15 & 17.142 & 20 & 2.86 & 17.14 & 278.2 & 268.80 & 9.40 \\
 & 120.27 & 13.142 & 16 & 2.86 & 13.14 & 281.3 & 268.80 & 12.50 \\
 & 120.75 & 17.142 & 20 & 2.86 & 17.14 & 263.6 & 268.80 & 5.20 \\
\hline
\multirow{5}{*}{$S_{2}$} & 4.38 & 113.98 & 8 & 5.25 & 2.75 & 22 & 24.50 & 2.50 \\
 & 4.58 & 113.98 & 8 & 5.25 & 2.75 & 10 & 24.50 & 14.50 \\
 & 4.78 & 119.43 & 26.6 & 6.97 & 19.63 & 211.1 & 202.68 & 8.42 \\
 & 23.00 & 120.15 & 6 & 2.86 & 3.14 & 278 & 268.80 & 9.20 \\
 & 18.87 & 121.28 & 10.2 & 1.82 & 8.38 & 205.4 & 185.83 & 19.57 \\
\hline
\multirow{3}{*}{$M_{2}$} & 11.17 & 108.7 & 21.5 & 1.58 & 19.92 & 345.5 & 350.37 & 4.87 \\
 & 22.62 & 120.25 & 15.9 & 2.86 & 13.04 & 262.7 & 268.80 & 6.10 \\
 & 18.62 & 121.1 & 19 & 1.82 & 17.18 & 195.7 & 185.83 & 9.87 \\
\hline
\end{tabular}
\end{table}

\begin{figure}[h]
\centering
\includegraphics[width=0.8\textwidth]{chart_image.png}
\caption{各个分潮与验潮站的调和常数(振幅$H$和迟角$G$)绝均差结果}
\end{figure}

图5.2 各个分潮与验潮站的调和常数(振幅$H$和迟角$G$)绝均差结果

针对文献[5],为了进一步检验数据,评价指标可参考:

\begin{equation}
d_{rmsH} = \left[ \frac{1}{K} \sum_{k=1}^{K} (H_{a,k} - H_{d,k})^2 \right]^{1/2} \tag{5.14}
\end{equation}

\begin{equation}
d_{rmsG} = \left[ \frac{1}{K} \sum_{k=1}^{K} (G_{a,k} - G_{d,k})^2 \right]^{1/2}
\tag{5.15}
\end{equation}

式中 \(d_{rmsH}\)、\(d_{rmsG}\) 分别表示振幅 \(H\)、迟角 \(G\) 的的均方根,\(H_a\)、\(H_d\) 分别表示 T/P-J 的观察值和计算值,同理 \(G_a\)、\(G_d\),\(K\) 是整个数据集比较的个数。

\begin{equation}
h = H_a \cos(\omega t - G_a)
\tag{5.16}
\end{equation}

式中 \(t\) 表示时间,\(\omega\) 表示相位。

\begin{equation}
h_{rms} = \left[ \frac{1}{2K} \sum_{k=1}^{K} H_{a,k}^2 \right]^{1/2}
\tag{5.17}
\end{equation}

\begin{equation}
d_{rms} = \left\{ \frac{1}{2K} \sum_{k=1}^{K} \left[ (H_{a,k} \cos G_{a,k} - H_{d,k} \cos G_{d,k})^2 + (H_{a,k} \sin G_{a,k} - H_{d,k} \sin G_{d,k})^2 \right] \right\}^{1/2}
\tag{5.18}
\end{equation}

\begin{equation}
r = d_{rms} / h_{rms}
\tag{5.19}
\end{equation}

\begin{equation}
d_{rss} = \left\{ \frac{1}{2K} \sum_{j=1}^{J} \sum_{k=1}^{K} \left[ (H_{a,k,j} \cos G_{a,k,j} - H_{d,k,j} \cos G_{d,k,j})^2 + (H_{a,k,j} \sin G_{a,k,j} - H_{d,k,j} \sin G_{d,k,j})^2 \right] \right\}^{1/2}
\tag{5.20}
\end{equation}

\begin{equation}
r_s = d_{rss} / h_{rss}
\tag{5.21}
\end{equation}

式中 \(h_{rss} = \left[ \frac{1}{2K} \sum_{j=1}^{J} \sum_{k=1}^{K} H_{a,k,j}^2 \right]^{1/2}\)。

\subsubsection{(2) 轨道交点处各分潮调和常数的比较}

另一方面,为了进一步验证调和分析的结果,分析轨道交叉点的上下行轨道的调和常数进行了对比分析。对其中 4 个分潮,其中半日潮 \((M_2, S_2)\) 和全日潮 \((K_1, O_1)\) 的分潮潮汐调和常数进行了对比,对比结果如表 5.6 所示。

上下行轨道主要有 20 条:其中,上行轨道 9 条,下行轨道 11 条。交叉点数据为:23 个。

\begin{figure}[h]
    \centering
    \includegraphics[width=\textwidth]{image.png}
    \caption{南海卫星上下行轨道分布}
    \label{fig:5.3}
\end{figure}

\begin{table}
\centering
\caption{轨道各交点$M_{2}$、$S_{2}$、$K_{1}$、$O_{1}$分潮振幅与迟角间的比较}
\begin{tabular}{c c | c c | c c | c c | c c}
\hline
\multicolumn{2}{c|}{交点位置} & \multicolumn{2}{c|}{$M_{2}$} & \multicolumn{2}{c|}{$S_{2}$} & \multicolumn{2}{c|}{$K_{1}$} & \multicolumn{2}{c}{$O_{1}$} \\
\hline
$^{\circ}$E & $^{\circ}$N & $\Delta H/cm$ & $\Delta$ & $\Delta H/cm$ & $\Delta$ & $\Delta H/cm$ & $\Delta$ & $\Delta H/cm$ & $\Delta$ \\
\hline
99.25 & 6.027 & 1.092 & 2.37 & 2.922 & 2.94 & 3.038 & 3.48 & 1.838 & 4.12 \\
100.5 & 10.6 & -0.501 & 0.57 & -0.771 & 3.85 & 0.0002 & 1.02 & 0.708 & 1.47 \\
103.5 & 9.631 & 0.372 & 0.43 & 0.551 & 0.60 & -0.068 & 0.25 & 0.842 & 0.87 \\
104.8 & 6.087 & -0.803 & 1.10 & -0.189 & 0.38 & -0.199 & 0.73 & -0.622 & 1.75 \\
107.8 & 5.718 & 0.504 & 0.82 & -0.564 & 0.68 & 0.546 & 0.89 & 0.859 & 1.01 \\
110.5 & 6.102 & 0.238 & 1.12 & -0.555 & 0.86 & -0.920 & 1.30 & -0.729 & 1.30 \\
109.1 & 10.05 & 0.076 & 0.82 & -0.899 & 1.42 & 0.562 & 0.97 & 0.276 & 0.92 \\
109.3 & 16.87 & -0.093 & 0.35 & 0.333 & 0.45 & -0.598 & 0.66 & -0.237 & 0.25 \\
110.5 & 13.65 & 0.422 & 1.00 & -0.317 & 0.35 & -0.437 & 0.53 & 0.357 & 0.36 \\
111.9 & 10.06 & 0.192 & 2.44 & 0.376 & 0.62 & -1.060 & 1.26 & 0.473 & 0.83 \\
113.3 & 6.269 & -0.839 & 2.70 & 1.829 & 2.91 & 0.325 & 2.00 & 1.480 & 1.67 \\
111.9 & 17.39 & -1.240 & 1.29 & -0.131 & 0.24 & -0.342 & 0.35 & -0.587 & 0.70 \\
113.3 & 13.77 & -0.018 & 0.83 & 0.380 & 0.42 & -0.214 & 0.25 & 0.442 & 0.79 \\
114.7 & 10.14 & 1.315 & 2.68 & 0.277 & 0.63 & 0.479 & 1.32 & -0.816 & 1.72 \\
114.9 & 16.96 & 0.410 & 0.41 & 0.515 & 0.81 & 0.208 & 0.53 & -0.447 & 0.46 \\
116.1 & 13.29 & -0.297 & 0.86 & 0.441 & 1.03 & 0.670 & 2.17 & 0.273 & 1.40 \\
117.6 & 10 & 0.953 & 1.71 & -1.310 & 1.47 & 0.722 & 0.74 & 0.194 & 1.41 \\
119 & 6.14 & -0.047 & 1.65 & -1.138 & 2.43 & 1.036 & 1.04 & 0.852 & 1.12 \\
116.3 & 20.38 & -0.006 & 0.52 & -0.002 & 0.30 & -0.642 & 0.67 & -0.279 & 0.88 \\
117.6 & 16.99 & -1.073 & 3.08 & 1.552 & 1.84 & 0.078 & 2.16 & 1.196 & 1.46 \\
119 & 13.84 & 0.351 & 0.37 & 0.973 & 0.99 & 0.092 & 0.62 & -1.300 & 1.41 \\
120.4 & 10.01 & 0.024 & 1.25 & -2.179 & 2.27 & 0.646 & 1.90 & 1.035 & 1.07 \\
119.2 & 20.89 & 0.523 & 0.63 & -0.178 & 0.39 & 0.192 & 0.27 & -0.355 & 0.40 \\
\hline
均方根 & & 0.65 & 0.85 & 1.09 & 1.02 & 0.83 & 0.79 & 0.81 & 0.78 \\
\hline
\end{tabular}
\end{table}

其中,在表5.6中$\Delta H$表示交点处的振幅差,$\Delta$为上行轨道与下行轨道在交点处分潮调和常数的矢量差异,$\Delta$定义为:

\begin{equation}
\Delta = \left[ \left( H_{a} \cos g_{a} - H_{d} \cos g_{d} \right)^{2} + \left( H_{a} \sin g_{a} - H_{d} \sin g_{d} \right)^{2} \right]^{\frac{1}{2}}
\tag{5.15}
\end{equation}

下标$a$,$d$分别代表上、下行轨道。

由表5.6和图可知,$M_{2}$、$S_{2}$、$K_{1}$、$O_{1}$振幅差$\Delta H$均方根为0.65、1.09、0.83、0.81cm,其平均值为0.805cm,最大值和最小值分别为$K_{1}$分潮的1.09cm、$M_{2}$分潮的0.65cm;$M_{2}$、$S_{2}$、$K_{1}$、$O_{1}$矢量差$\Delta$均方根为0.85、1.02、0.79、0.78,其平均值为0.86,最大值和最小值分别为$S_{2}$分潮的1.02、$O_{1}$分潮的0.78。上述结果表明,上下行轨道在轨道交叉点出的调和分析结果偏差很小,说明各个分潮潮汐调和分析结果具有较高的可靠性。

\subsubsection{分潮潮汐调和常数沿轨道分布结果图}

从整体的主要分潮潮汐调和常数沿轨道上的结果分布图可以看出,从T/P获取得到的各个分潮($M_{2}$、$S_{2}$、$K_{1}$、$O_{1}$)的潮汐调和常数(振幅$H$和迟角$G$),从图中可以看出$M_{2}$振幅的范围主要在0~40cm,大部分处于10cm以下,迟角跨度

较大,沿轨道分布极为不均,查阅文献表明,需要对其进行内潮分离,得到潮汐正压潮的特征,从而为绘制同潮图做好资料准备。同样对 $S_{2} 、 K_{1} 、 O_{1}$ 分潮的潮汐调和常数具有类似的结论。

\begin{figure}[h]
    \centering
    \includegraphics[width=0.45\textwidth]{image1.png}
    \caption{$M_{2}$ 分潮振幅 $H$ 沿轨道分布示意图}
    \label{fig:5.4}
\end{figure}
\begin{figure}[h]
    \centering
    \includegraphics[width=0.45\textwidth]{image2.png}
    \caption{$M_{2}$ 分潮迟角 $G$ 沿轨道分布示意图}
    \label{fig:5.5}
\end{figure}

\begin{figure}[h]
    \centering
    \includegraphics[width=0.45\textwidth]{image3.png}
    \caption{$K_{1}$ 分潮振幅 $H$ 沿轨道分布示意图}
    \label{fig:5.6}
\end{figure}
\begin{figure}[h]
    \centering
    \includegraphics[width=0.45\textwidth]{image4.png}
    \caption{$K_{1}$ 分潮迟角 $G$ 沿轨道分布示意图}
    \label{fig:5.7}
\end{figure}

\begin{figure}[h]
    \centering
    \includegraphics[width=0.45\textwidth]{image5.png}
    \caption{$O_{1}$ 分潮振幅 $H$ 沿轨道分布示意图}
    \label{fig:5.8}
\end{figure}
\begin{figure}[h]
    \centering
    \includegraphics[width=0.45\textwidth]{image6.png}
    \caption{$O_{1}$ 分潮迟角 $G$ 沿轨道分布示意图}
    \label{fig:5.9}
\end{figure}

\begin{figure}[h]
    \centering
    \includegraphics[width=0.45\textwidth]{image1.png}
    \caption{$S_{2}$ 分潮振幅 $H$ 沿轨道分布示意图}
    \label{fig:5.10}
\end{figure}
\begin{figure}[h]
    \centering
    \includegraphics[width=0.45\textwidth]{image2.png}
    \caption{$S_{2}$ 分潮迟角 $G$ 沿轨道分布示意图}
    \label{fig:5.11}
\end{figure}

\section{问题二:正压潮和内潮的分离}

由第一问得到各个主要分潮在轨道上的分布示意图,分析发现,各个分潮的振幅和迟角在轨道上具有一定震荡现象,是由于内潮对正压潮具有调制作用,而为了分离正压潮和内潮,许多学者对其进行了大量的研究,目前,发现通过多项式拟合获得的结果,不受内潮的影响,于是,可以获得正压潮,得到分潮潮汐调和常数稳定的部分,为同潮图的绘制提供有效的参数支持。首先,针对不同轨道的分潮潮汐调和参数进行了相应的图形绘制,发现:上下行轨道的有细结构差别较大,下行轨道具有明显地有细结构,而上行轨道的有细结构不明显,从而为多项式拟合的结果提供了一定的参考支持。

\subsection{沿轨道分析}

\subsubsection{上行轨道分析}

利用轨道分割法,获得各个轨道上各个主要分潮潮汐调和常数沿上行轨道的分布,从图 \ref{fig:5.12}-图 \ref{fig:5.15} 可以看出,各个分潮 ($M_{2}$、$S_{2}$、$K_{1}$、$O_{1}$) 振幅在沿纬度方向上,具有明显地细条结构,虽然在部分纬度跨越时,振幅波动变化较大。

\begin{figure}[h]
    \centering
    \includegraphics[width=0.45\textwidth]{image3.png}
    \caption{1 号轨道 $S_{2}$ 分潮振幅 $H$ 沿轨道分布}
    \label{fig:5.12}
\end{figure}
\begin{figure}[h]
    \centering
    \includegraphics[width=0.45\textwidth]{image4.png}
    \caption{1 号轨道 $K_{1}$ 分潮振幅 $H$ 沿轨道分布}
    \label{fig:5.13}
\end{figure}

\begin{figure}[h]
    \centering
    \includegraphics[width=0.45\textwidth]{image1.png}
    \caption{1号轨道 $O_{1}$ 分潮振幅 $H$ 沿轨道分布}
    \label{fig:5.14}
\end{figure}
\begin{figure}[h]
    \centering
    \includegraphics[width=0.45\textwidth]{image2.png}
    \caption{12号轨道 $M_{2}$ 分潮振幅 $H$ 沿轨道分布}
    \label{fig:5.15}
\end{figure}

\subsubsection{下行轨道分析}

利用轨道分割法,获得各个轨道上各个主要分潮潮汐调和常数沿下行轨道的分布,从图 \ref{fig:5.16}-图 \ref{fig:5.19} 可以看出,各个分潮 ($M_{2}$、$S_{2}$、$K_{1}$、$O_{1}$) 振幅在沿纬度方向上,具有明显地细条结构,虽然在部分纬度跨越时,振幅波动变化较大。

\begin{figure}[h]
    \centering
    \includegraphics[width=0.45\textwidth]{image3.png}
    \caption{17号轨道 $K_{1}$ 分潮振幅 $H$ 沿轨道分布}
    \label{fig:5.16}
\end{figure}
\begin{figure}[h]
    \centering
    \includegraphics[width=0.45\textwidth]{image4.png}
    \caption{18号轨道 $S_{2}$ 分潮振幅 $H$ 沿轨道分布}
    \label{fig:5.17}
\end{figure}
\begin{figure}[h]
    \centering
    \includegraphics[width=0.45\textwidth]{image5.png}
    \caption{19号轨道 $M_{2}$ 分潮振幅 $H$ 沿轨道分布}
    \label{fig:5.18}
\end{figure}
\begin{figure}[h]
    \centering
    \includegraphics[width=0.45\textwidth]{image6.png}
    \caption{20号轨道 $O_{1}$ 分潮振幅 $H$ 沿轨道分布}
    \label{fig:5.19}
\end{figure}

此细结构是内潮对正压潮的调制,为了剥离各个分潮的潮汐调和常数中的内潮的影响,从而需要利用多项式拟合的方法,针对各个分潮分别进行沿轨道上进行分离分析,以下为多项式拟合方法的过程。

\subsection{5.2.2 多项式拟合方法}

利用调和分析获得各个主要分潮的潮汐调和常数,得到某一条轨道上的任一观测点 \( P(x, y) \)(\( x \) 和 \( y \) 分别为点 \( p \) 的经、纬度)的振幅值 \( H \) 和迟角值 \( g \)。首先将 \( H \)、\( g \) 转化为 \( H \cos g \) 和 \( H \sin g \),然后分别对 \( H \cos g \) 和 \( H \sin g \) 沿轨进行 \( n \) 次多项式拟合(\( n=1, 2, \ldots \)),最后通过坐标转换得到 \( M_2 \) 内潮的海表面振幅和迟角。以拟合 \( H \cos g \) 为例说明。

设拟合多项式为:
\begin{equation}
f(y) = a_0 + a_1 + a_2 + \cdots + a_n y^n
\tag{5.15}
\end{equation}

构造代价函数:
\begin{equation}
J(a_n) = \sum_{k=1}^N \left[ (a_0 + a_1 y_k + a_2 y_k^2 + \cdots + a_n y_k^n) - H_k \cos g_k \right]^2, k=1, \ldots, N
\tag{5.16}
\end{equation}

式中,\( y_k \) 代表轨道任一观测点的纬度。对式子 (5.17) 分别求 \( a_0 \),\( a_1 \),\( a_2 \),\ldots,\( a_n \) 的编号,并使其等于 0,
\begin{equation}
\left.
\begin{aligned}
\sum_{k=1}^N \left[ (a_0 + a_1 y_k + a_2 y_k^2 + \cdots + a_n y_k^n) - H_k \cos g_k \right] &= 0 \\
\sum_{k=1}^N y_k \left[ (a_0 + a_1 y_k + a_2 y_k^2 + \cdots + a_n y_k^n) - H_k \cos g_k \right] &= 0 \\
\sum_{k=1}^N y_k^2 \left[ (a_0 + a_1 y_k + a_2 y_k^2 + \cdots + a_n y_k^n) - H_k \cos g_k \right] &= 0 \\
\vdots \\
\sum_{k=1}^N y_k^n \left[ (a_0 + a_1 y_k + a_2 y_k^2 + \cdots + a_n y_k^n) - H_k \cos g_k \right] &= 0
\end{aligned}
\right\}
\tag{5.17}
\end{equation}

为便于阐述,引入记号:
\begin{equation}
M_{i,j} = \sum_{k=1}^N y_k^{i+j-2}, b_i = \sum_{k=1}^N y_k^{i-1} H_k \cos g_k
\tag{5.18}
\end{equation}
其中 \( i \),\( j=1, 2, \ldots, n+1 \)。

式子 (5.18) 表示为
\begin{equation}
\begin{bmatrix}
M_{1,1} & M_{1,2} & \cdots & M_{1,n+1} \\
M_{2,1} & M_{2,2} & \cdots & M_{2,n+1} \\
\cdots & \cdots & \cdots & \cdots \\
M_{n+1,1} & M_{n+1,2} & \cdots & M_{n+1,n+1}
\end{bmatrix}
\begin{bmatrix}
a_0 \\
a_1 \\
\cdots \\
a_n
\end{bmatrix}
=
\begin{bmatrix}
b_1 \\
b_2 \\
\cdots \\
b_{n+1}
\end{bmatrix}
\tag{5.19}
\end{equation}

通过求解上述线性方程组可得系数 \( a_0 \),\( a_1 \),\( a_2 \),\ldots,\( a_n \) 的值,并代入式 (5.16) 得到正压潮值 \( f(y_k) \),记 \( H_k \cos g_k \) 与其拟合值 \( f(y_k) \) 之差为:
\begin{equation}
\tilde{f}(y_k) = H_k \cos g_k - f(y_k)
\tag{5.20}
\end{equation}
\( \tilde{f}_k \) 为内潮贡献。同理,对 \( H \sin g \) 进行多形式拟合可得:
\begin{equation}
\tilde{g}(y_k) = H_k \sin g_k - \hat{g}(y_k)
\tag{5.21}
\end{equation}
通过坐标转换可求得内潮海表面振幅 \( \tilde{H}_k \)、迟角 \( \tilde{g}_k \)。

\subsection{5.2.3 拟合结果分析}

利用多项式拟合方法分别对南海海区内 20 条轨道 8 个主要分潮的潮汐调和常数沿轨进行 3-15 次多项式拟合(这里在第四问重点分析),从而获得了正压潮和内潮的分离。由此得到各个轨道的最佳多项式次数拟合如表 5.33 所示(问题四)。

从而利用得到的正压潮和内潮的贡献,通过坐标变换分别可以得到正压潮和内潮的调和振幅 $H$ 和迟角 $G$,从而可以达到对正压潮和内潮的分离,针对 20 条轨道分别进行正压潮的分离,此处,列举了上下行轨道各一条进行分析和讨论。1 号上行轨道分离结果,如图 5.21-图 5.25;20 号下行轨道分离结果如图 5.26-图 5.31 所示。

\subsubsection{(1) 上行轨道正压潮和内潮分离结果}

分别对 9 条上行轨道进行了正压潮和内潮的分离,本部分列举了 1 号上行轨道的各个分潮的正压潮和内潮的调和常数(振幅 $H$ 和迟角 $G$)。可以看出,$M_{2}$ 分潮的正压潮的振幅处于 0.5~3.5cm 之间,中间存在波峰和波谷,$6^{\circ} \mathrm{N}$ 处存在较为峰尖,对于后面的插值来说,具有较为剧烈的影响。对于 $6.2^{\circ} \mathrm{N}$ 的振幅和迟角值,后期在进行插值时,为了使得同潮图具有连续性并能够体现海洋潮汐的特征,则需要对这种“尖端”进行另外处理(例如剔除)。

\begin{figure}[h]
    \centering
    \includegraphics[width=\textwidth]{image.png}
    \caption{1 号上行轨道 $M_{2}$ 分潮正压潮和内潮分离情况}
    \label{fig:5.20}
\end{figure}

图 5.20 1 号上行轨道 $M_{2}$ 分潮正压潮和内潮分离情况

$S_{2}$ 分潮的正压潮的振幅处于 0.5~2.5cm 之间,中间存在多个波峰和波谷,同样针对 $6^{\circ} \mathrm{N}$ 前后处存在较为峰尖,具有较为剧烈的影响。对于 $6^{\circ} \mathrm{N}$ 左右的振幅和迟角值,同时存在这种“尖端”的现象的,同样插值时,剔除处理。

\begin{figure}[h]
    \centering
    \includegraphics[width=\textwidth]{image1.png}
    \caption{1号上行轨道 $S_{2}$ 分潮正压潮和内潮分离情况}
    \label{fig:1}
\end{figure}

$K_{1}$ 分潮的正压潮的振幅处于 $0 \sim 5.0 \, \text{cm}$ 之间,中间存在多个波峰和波谷,在 $2.3^{\circ} \, \text{N}$ 处存在较为峰尖,具有较为剧烈的影响。对于 $2.3^{\circ} \, \text{N}$ 左右的振幅和迟角值,同时存在这种“尖端”的现象的,同样插值时,剔除处理。

\begin{figure}[h]
    \centering
    \includegraphics[width=\textwidth]{image2.png}
    \caption{正压潮振幅沿轨道分布}
    \label{fig:2}
\end{figure}

\begin{figure}[h]
    \centering
    \includegraphics[width=\textwidth]{image1.png}
    \caption{1号上行轨道 $K_{1}$ 分潮正压潮和内潮分离情况}
    \label{fig:5.22}
\end{figure}

$O_{1}$ 分潮的正压潮的振幅处于 0.5~3.5cm 之间,中间存在多个波峰和波谷,多数处于 0.5cm~2cm 范围内,在 $6^{\circ}\mathrm{N}$ 处存在较为“不可导点”,对插值具有较为剧烈的影响。

\begin{figure}[h]
    \centering
    \includegraphics[width=\textwidth]{image2.png}
    \caption{1号上行轨道 $O_{1}$ 分潮正压潮和内潮分离情况}
    \label{fig:5.23}
\end{figure}

\subsubsection{下行轨道正压潮和内潮分离情况}

分别对 11 条下行轨道进行了正压潮和内潮的分离,本部分列举了 20 号下行轨道的各个分潮的正压潮和内潮的调和常数(振幅 $H$ 和迟角 $G$)。可以看出,$M_{2}$ 分潮的正压潮的振幅处于 0~7cm 之间,中间存在一个波峰和一个波谷,$6^{\circ}\mathrm{N}$ 处

\begin{figure}[h]
    \centering
    \includegraphics[width=\textwidth]{image1.png}
    \caption{20号上行轨道M2分潮正压潮和内潮分离情况}
    \label{fig:20_track_M2}
\end{figure}

针对 $S_{2}$ 分潮的正压潮的振幅处于 $0 \sim 25 \, \text{cm}$ 之间,中间存在一个波谷,整体处于平滑曲线,但对于迟角从 $0^\circ$ 变到 $358^\circ$,这种情况,在插值时,可以简化为 $0^\circ$,从而达到一定的平滑处理。

\begin{figure}[h]
    \centering
    \includegraphics[width=\textwidth]{image2.png}
    \caption{20号轨道正压潮}
    \label{fig:20_track_positive_pressure_tide}
\end{figure}

\begin{figure}[h]
    \centering
    \includegraphics[width=0.45\textwidth]{image1.png}
    \caption{(c) 内潮振幅沿轨道分布}
    \label{fig:5.25c}
\end{figure}
\begin{figure}[h]
    \centering
    \includegraphics[width=0.45\textwidth]{image2.png}
    \caption{(d) 内潮迟角沿轨道分布}
    \label{fig:5.25d}
\end{figure}

图 5.25 20 号上行轨道 $S_{2}$ 分潮正压潮和内潮分离情况

同样,针对 $K_{1}$ 分潮的正压潮的振幅处于 0~20cm 之间,整体处于平滑曲线,但对于迟角从 $0^{\circ}$ 变到 $358^{\circ}$,这种情况,在插值时,可以简化为 $0^{\circ}$,从而达到一定的平滑处理。

\begin{figure}[h]
    \centering
    \includegraphics[width=0.45\textwidth]{image3.png}
    \caption{(a) 正压潮振幅沿轨道分布}
    \label{fig:5.26a}
\end{figure}
\begin{figure}[h]
    \centering
    \includegraphics[width=0.45\textwidth]{image4.png}
    \caption{(b) 正压潮迟角沿轨道分布}
    \label{fig:5.26b}
\end{figure}
\begin{figure}[h]
    \centering
    \includegraphics[width=0.45\textwidth]{image5.png}
    \caption{(c) 内潮振幅沿轨道分布}
    \label{fig:5.26c}
\end{figure}
\begin{figure}[h]
    \centering
    \includegraphics[width=0.45\textwidth]{image6.png}
    \caption{(d) 内潮迟角沿轨道分布}
    \label{fig:5.26d}
\end{figure}

图 5.26 20 号上行轨道 $K_{1}$ 分潮正压潮和内潮分离情况

针对 $O_{1}$ 分潮的正压潮的振幅处于 0~12cm 之间,中间存在较小的波动,整体处于平滑曲线,但对于迟角从 $358^{\circ}$ 变到 $0^{\circ}$,这种情况,在插值时,可以简化为 $350^{\circ}$,从而达到一定的平滑处理。

\begin{figure}[h]
    \centering
    \includegraphics[width=\textwidth]{image.png}
    \caption{20号上行轨道 $O_{1}$ 分潮正压潮和内潮分离情况}
    \label{fig:5.27}
\end{figure}

\section{问题三:分潮特征同潮图分析}

由问题二得到各个分潮的正压潮的潮汐调和常数沿轨道上的分布,可以利用相对应的插值方法,对整个坐标系下进行插值,从而获得各个分潮的潮汐调和常数(振幅 $H$ 和迟角 $G$)的同潮图(北京时间)。查阅参考文献,目前主要用于插值法获得同潮分布图的常规方法有:牛顿(Newton)、格雷格里(Gregory)和拉格朗日(Lagrange)。在本次潮汐调和分析过程中,这三种插值方法对数据具有较为明显的依赖性,需要提供的数据具有一定规律的连续性,而由于内潮对正压潮具有调制作用,需要对内潮进行剥离,从而获得具有稳定信号的正压潮的调和常数,并有利于同潮图的绘制。

\subsection{插值方法}

常规的插值主要是利用特定的公式进行插值推导,假设函数 $y=f(x)$ 是定义在区间 $[a,b]$ 上的函数,在区间 $[a,b]$ 上取 $n+1$ 个互异的点 $x_{0}, x_{1}, \ldots, x_{n}$,这些点 $y=f(x)$ 在上对应的函数值分别为 $y_{0}, y_{1}, \ldots, y_{n}$。

\begin{table}[h]
    \centering
    \begin{tabular}{|c|c|c|c|c|c|}
        \hline
        $x_{i}$ & $x_{0}$ & $x_{1}$ & $x_{2}$ & $\cdots$ & $x_{n}$ \\
        \hline
        $f(x_{i})$ & $f(x_{0})$ & $f(x_{1})$ & $f(x_{2})$ & $\cdots$ & $f(x_{n})$ \\
        \hline
    \end{tabular}
    \caption{差值数据对应表}
    \label{tab:5.7}
\end{table}

那么如果想要知道 $y=f(x)$ 在其他点的值,就需要构造一个函数 $g(x)$,使它满足下式:

\begin{equation}
g(x_i) = f(x_i), i = 0, 1, \ldots, n
\tag{5.22}
\end{equation}

此时点 \(x_0\), \(x_1\), \(\ldots\), \(x_n\) 被称为插值基点。区间 \([min(x_0, x_1, \ldots, x_n), max(x_0, x_1, \ldots, x_n)]\) 被称为差值区间。函数 \(g(x)\) 被称为原函数 \(f(x)\) 的插值函数。差值得误差一般可以表示为式子 5.23

\begin{equation}
r(x) = f(x) - g(x)
\tag{5.23}
\end{equation}

插值的基本过程就是通过已知的插值基点来构建原函数 \(f(x)\) 的插值函数 \(g(x)\),此时其他未知点的函数值就可以用差值函数 \(g(x)\) 的函数值去替代。

目前主要的插值函数有:

\begin{tabular}{ll}
Lagrange 插值 & $p_n(x) = \sum_{i=1}^n f(x_i) l_i(x)$ \\
Newton 插值 & $N_n(x) = f\left[x_0\right] + f\left[x_0, x_1\right] \omega(x) + \cdots + f\left[x_0, x_1, \ldots, x_n\right] \omega_n(x)$ \\
Hermite 插值 & $H_{2n+1}(x) = \sum_{k=0}^n y_k \left[1 - 2l_k'(x_k)(x - x_k)\right] l_k^2(x) + \sum_{k=0}^n y_k'(x - x_k) l_k^2(x)$ \\
三次样条插值 & $S_3(x) = P_3(x) + \sum_{i=1}^{n-1} C_i (x - x_i)_+^3$
\end{tabular}

\begin{equation}
\tag{5.24}
\end{equation}
\begin{equation}
\tag{5.25}
\end{equation}
\begin{equation}
\tag{5.26}
\end{equation}
\begin{equation}
\tag{5.27}
\end{equation}

本文采用克里金估值方法:确定诸权系数 \(\lambda_n\),使估计值成为观测值的无偏估计量,且估计方差最小,且无偏条件如式 5.28 所示。

\begin{equation}
\sum_{a=1}^n \lambda_a = 1
\tag{5.28}
\end{equation}

估计方差最小条件:在不存在偏差且估值中权值之和为 1,则该估值是无偏的,围绕该真值的估值离散程度称为估计方差,估计方差的计算公式为:

\begin{equation}
\sigma_E^2 = C(x, x) + \sum_{i=1}^n \sum_{j=1}^n \lambda_i \lambda_j C(x_i, x_j) - 2 \sum_{i=1}^n \lambda_i C(x, x_j)
\tag{5.29}
\end{equation}

普通克里金方程组所要解决的问题是:求解在无偏条件下,使估计方差达到极小时的诸权重系数。无偏估计公式为:

\begin{equation}
F = \sigma_E^2 - 2\mu \left(\sum_{i=1}^n \lambda_i - 1\right)
\tag{5.30}
\end{equation}

式中:\(F\) 为 \(n\) 个权值系数 \(\lambda_i\) 和 \(\mu\) 的 \((n+1)\) 元函数;\(\mu\) 为拉格朗日乘数。求其偏导数即可获得普通克里金方程组:

\begin{equation}
\begin{cases}
\frac{\partial F}{\partial \lambda_i} = 2 \sum_{j=1}^n \lambda_j C(x_i, x_j) - 2\bar{C}(x, x) - 2\mu = 0, i = 1, 2, \ldots, n \\
\frac{\partial F}{\partial \lambda_i} = -2 \left(\sum_{i=1}^n \lambda_i - 1\right) = 0
\end{cases}
\tag{5.31}
\end{equation}

利用协方差函数与变差函数之间的关系:

\begin{equation}
\gamma(h) = C(0) - C(h)
\tag{5.32}
\end{equation}

最终式 (4) 可以变成函数表达式:

\begin{equation}
\begin{cases}
\sum_{j=1}^n \lambda_j \gamma(x_i, x_j) + \mu = 2\gamma(x_i, x) \\
\sum_{i=1}^n \lambda_i = 1
\end{cases}
\tag{5.33}
\end{equation}

利用上述克里金插值法,同时结合问题二分析得到的经验插值规则:

1) 迟角从 \(0^\circ\) 瞬间变换到 \(350^\circ \sim 360^\circ\) 时(极小到极大值骤变过程),按照变换前进行微小变化插值;

2) 振幅和迟角分别或者同时存在 “尖端”,即不可导点时,可以向前或者向后进行微小变化插值,从而可以防止出现骤变数据。

\subsection{5.3.2 分潮同潮图分析}

(1) 半日分潮同潮图分析

根据 \(f\) 、 \(u\) 模型,利用卫星高度计资料计算出南海研究区域各个分潮的调和常数,获取了 T/P 卫星观察下的 8 个主要分潮的调和数据。依据前面的加权插值方法,绘制了图 5.9-图 5.12,4 个半日分潮 \(\left(M_{2} 、 S_{2} 、 K_{2} 、 N_{2}\right)\) 的振幅 \(H\) 和迟角 \(G\) 分布的同潮图(北京时间)。

从同潮图可以看出,南海 \(M_{2}\) 分潮的振幅均在 \(10 \mathrm{~cm}\) 左右,仅部分海岸的 \(M_{2}\) 分潮的振幅较大,达到了 \(14 \sim 18 \mathrm{~cm}\) 的范围,最大振幅接近 \(18 \mathrm{~cm}\) ;沿岸线的分潮振幅整体比海洋区域内部的振幅大,海洋区域的分潮振幅变化不明显,整体图具有较好的美观,与目前的同潮图对比,发现具有较高的重叠性;迟角的同潮图可以看出,利用自然邻居插值法得到结果,在卫星轨道上,插值结果受到严重的影响,这与自然邻居插值的形式有关,从而需要提高插值形式,针对不同轨道的迟角,需要利用不同的插值方式进行插值,例如加权插值等方式。

\begin{figure}[h]
    \centering
    \includegraphics[width=\textwidth]{image1.png}
    \caption{\(M_{2}\) 分潮的振幅 \(H\) 和迟角 \(G\) 分布的同潮图(\(a\) :振幅 \(H\) 同潮图,\(b\) :迟角 \(G\) 同潮图)}
    \label{fig:5.28}
\end{figure}

\begin{figure}[h]
    \centering
    \includegraphics[width=\textwidth]{image2.png}
    \caption{\(S_{2}\) 分潮的振幅 \(H\) 和迟角 \(G\) 分布的同潮图(\(c\) :振幅 \(H\) 同潮图,\(d\) :迟角 \(G\) 同潮图)}
    \label{fig:5.29}
\end{figure}

\begin{figure}[h]
    \centering
    \includegraphics[width=\textwidth]{image1.png}
    \caption{图 5.30 $K_{2}$ 分潮的振幅 $H$ 和迟角 $G$ 分布的同潮图($e$: 振幅 $H$ 同潮图,$f$: 迟角 $G$ 同潮图)}
\end{figure}

\begin{figure}[h]
    \centering
    \includegraphics[width=\textwidth]{image2.png}
    \caption{图 5.31 $N_{2}$ 分潮的振幅 $H$ 和迟角 $G$ 分布的同潮图($g$: 振幅 $H$ 同潮图,$h$: 迟角 $G$ 同潮图)}
\end{figure}

\subsection{全日分潮同潮图分析}

同样根据 $f$、$u$ 模型,利用卫星高度计资料计算出南海研究区域各个分潮的调和常数,获取了 T/P 卫星观察下的 8 个主要分潮的调和数据。依据前面的加权插值方法,绘制了图 5.9-图 5.12,4 个全日分潮($K_{1}$、$O_{1}$、$P_{1}$、$Q_{1}$)的振幅 $H$ 和迟角 $G$ 分布的同潮图(北京时间)。

从同潮图可以看出,南海 $M_{2}$ 分潮的振幅均在 $10\,\text{cm}$ 左右,仅部分海岸的 $M_{2}$ 分潮的振幅较大,达到了 $14\sim18\,\text{cm}$ 的范围,最大振幅接近 $18\,\text{cm}$;沿岸线的分潮振幅整体比海洋区域内部的振幅大,海洋区域的分潮振幅变化不明显,整体图具有较好的美观,与目前的同潮图对比,发现具有较高的重叠性;迟角的同潮图可以看出,利用自然邻居插值法得到结果,在卫星轨道上,插值结果受到严重的影响,这与自然邻居插值的形式有关,从而需要提高插值形式,针对不同轨道的迟角,需要利用不同的插值方式进行插值,例如加权插值等方式。

\begin{figure}[h]
    \centering
    \includegraphics[width=\textwidth]{image1.png}
    \caption{$K_{1}$ 分潮的振幅 $H$ 和迟角 $G$ 分布的同潮图($a$: 振幅 $H$ 同潮图,$b$: 迟角 $G$ 同潮图)}
    \label{fig:5.32}
\end{figure}

\begin{figure}[h]
    \centering
    \includegraphics[width=\textwidth]{image2.png}
    \caption{$O_{1}$ 分潮的振幅 $H$ 和迟角 $G$ 分布的同潮图($c$: 振幅 $H$ 同潮图,$d$: 迟角 $G$ 同潮图)}
    \label{fig:5.33}
\end{figure}

\begin{figure}[h]
    \centering
    \includegraphics[width=\textwidth]{image3.png}
    \caption{$P_{1}$ 分潮的振幅 $H$ 和迟角 $G$ 分布的同潮图($e$: 振幅 $H$ 同潮图,$f$: 迟角 $G$ 同潮图)}
    \label{fig:5.34}
\end{figure}

\begin{figure}[h]
    \centering
    \includegraphics[width=\textwidth]{image1.png}
    \caption{图 5.35 $Q_{1}$ 分潮的振幅 $H$ 和迟角 $G$ 分布的同潮图(g:振幅 $H$ 同潮图,h:迟角 $G$ 同潮图)}
\end{figure}

\subsection{5.3.3 与潮汐验潮站结果对比}

利用最近原则,对比分析附近的验潮站的分潮调和数据,得到同潮图与验潮站的分潮调和数据检验对比分析的结果,如表 5.8 所示,通过对比可以,得出:

$M_{2}$ 分潮的 6 个验潮站的数据对比下,振幅 $H$ 误差平均值为 8.72cm,迟角 $G$ 误差平均值为 $8.81^{\circ}$;$S_{2}$ 分潮的 6 个验潮站的数据对比下,振幅 $H$ 误差平均值为 15.22cm,迟角 $G$ 误差平均值为 $8.65^{\circ}$;$K_{1}$ 分潮的 8 个验潮站的数据对比下,振幅 $H$ 误差平均值为 10.94cm,迟角 $G$ 误差平均值为 $11.81^{\circ}$;$O_{1}$ 分潮的 9 个验潮站的数据对比下,振幅 $H$ 误差平均值为 12.59cm,迟角 $G$ 误差平均值为 $8.89^{\circ}$;$N_{2}$ 分潮的 10 个验潮站的数据对比下,振幅 $H$ 误差平均值为 13.64cm,迟角 $G$ 误差平均值为 $9.45^{\circ}$;$K_{2}$ 分潮的 13 个验潮站的数据对比下,振幅 $H$ 误差平均值为 12.86cm,迟角 $G$ 误差平均值为 $8.53^{\circ}$;$P_{1}$ 分潮的 8 个验潮站的数据对比下,振幅 $H$ 误差平均值为 11.06cm,迟角 $G$ 误差平均值为 $10.35^{\circ}$;$Q_{1}$ 分潮的 2 个验潮站的数据对比下,振幅 $H$ 误差平均值为 17.41cm,迟角 $G$ 误差平均值为 $7.35^{\circ}$;

\begin{table}[h]
\centering
\caption{表 5.8 同潮图与验潮站的分潮调和数据对比分析}
\begin{tabular}{|c|c|c|c|c|c|c|c|c|}
\hline
\multirow{2}{*}{分潮类型} & \multicolumn{2}{c|}{验潮站} & \multicolumn{3}{c|}{振幅 $H$/cm} & \multicolumn{3}{c|}{迟角 $G/^{\circ}$} \\
\cline{2-9}
 & 经度 & 纬度 & 观察值 & 计算值 & MSEH & 观察值 & 计算值 & MSEG \\
\hline
\multirow{6}{*}{$M_{2}$} & 107.033 & 20.717 & 4.4 & 5.56 & 1.16 & 130.7 & 120.00 & 10.70 \\
\hline
 & 107.367 & 21.033 & 13.5 & 10.42 & 3.08 & 193.1 & 197.49 & 4.39 \\
\hline
 & 120.417 & 17.783 & 7.7 & 2.18 & 5.52 & 228 & 238.74 & 10.74 \\
\hline
 & 120.567 & 18.483 & 7.6 & 0.89 & 6.71 & 191.7 & 191.32 & 0.38 \\
\hline
 & 120.683 & 22.067 & 20 & 1.90 & 18.10 & 200.5 & 188.75 & 11.75 \\
\hline
 & 121.1 & 18.617 & 19 & 1.25 & 17.75 & 195.7 & 180.78 & 14.92 \\
\hline
\multirow{6}{*}{$S_{2}$} & 109.533 & 18.217 & 20.3 & 4.95 & 15.35 & 303.4 & 304.22 & 0.82 \\
\hline
 & 120.217 & 5.35 & 13.5 & 3.35 & 10.15 & 210.4 & 207.77 & 2.63 \\
\hline
 & 120.267 & 22.683 & 15 & 2.17 & 12.83 & 230.1 & 222.89 & 7.21 \\
\hline
 & 120.683 & 22.067 & 20 & 1.71 & 18.29 & 200.5 & 183.12 & 17.38 \\
\hline
 & 120.883 & 5.7 & 19.6 & 3.02 & 16.58 & 176.1 & 184.44 & 8.34 \\
\hline
 & 121.467 & 6.733 & 19.6 & 1.50 & 18.10 & 242.9 & 258.40 & 15.50 \\
\hline
\multirow{2}{*}{$K_{1}$} & 102.167 & 6.2 & 18 & 0.65 & 17.35 & 294.49 & 308.52 & 14.03 \\
\hline
 & 106.467 & 17.7 & 17.6 & 0.80 & 16.80 & 41.4 & 59.55 & 18.15 \\
\hline
\end{tabular}
\end{table}

\begin{table}
\centering
\begin{tabular}{|c|c|c|c|c|c|c|c|c|}
\hline
\multirow{2}{*}{分潮类型} & \multicolumn{2}{c|}{验潮站} & \multicolumn{3}{c|}{振幅$H/cm$} & \multicolumn{3}{c|}{迟角$G/^{\circ}$} \\
\cline{2-9}
 & 经度 & 纬度 & 观察值 & 计算值 & MSEH & 观察值 & 计算值 & MSEG \\
\hline
 & 106.667 & 20.867 & 4.4 & 1.14 & 3.26 & 98.78 & 92.58 & 6.20 \\
\hline
 & 106.8 & 20.667 & 5.4 & 1.14 & 4.26 & 75.98 & 92.58 & 16.60 \\
\hline
 & 107.067 & 20.95 & 6.7 & 1.35 & 5.35 & 143.7 & 156.09 & 12.39 \\
\hline
 & 120.417 & 17.783 & 7.7 & 2.35 & 5.35 & 228 & 236.76 & 8.76 \\
\hline
 & 120.683 & 22.067 & 20 & 2.14 & 17.86 & 200.5 & 192.41 & 8.09 \\
\hline
 & 120.883 & 5.7 & 19.6 & 2.34 & 17.26 & 176.1 & 165.86 & 10.24 \\
\hline
\multirow{10}{*}{$O_{1}$} & 106.667 & 20.867 & 4.4 & 0.66 & 3.74 & 98.78 & 116.64 & 17.86 \\
\hline
 & 107.067 & 20.95 & 6.7 & 0.76 & 5.94 & 143.7 & 139.29 & 4.41 \\
\hline
 & 110.317 & 20.05 & 14 & 1.57 & 12.43 & 253.84 & 264.59 & 10.75 \\
\hline
 & 119.9 & 16.4 & 9.6 & 1.88 & 7.72 & 270.1 & 274.21 & 4.11 \\
\hline
 & 120.283 & 14.817 & 17 & 3.08 & 13.92 & 284.3 & 273.67 & 10.63 \\
\hline
 & 120.583 & 22.35 & 20.9 & 1.68 & 19.22 & 217.2 & 204.99 & 12.21 \\
\hline
 & 120.683 & 22.067 & 20 & 1.65 & 18.35 & 200.5 & 208.91 & 8.41 \\
\hline
 & 120.883 & 5.7 & 19.6 & 3.04 & 16.56 & 176.1 & 165.12 & 10.98 \\
\hline
 & 120.967 & 14.583 & 19.1 & 3.71 & 15.39 & 294.5 & 294.18 & 0.32 \\
\hline
\multirow{12}{*}{$N_{2}$} & 107.367 & 21.033 & 13.5 & 4.70 & 8.80 & 193.1 & 201.91 & 8.81 \\
\hline
 & 113.983 & 4.383 & 15.8 & 5.86 & 9.94 & 341.9 & 337.49 & 4.41 \\
\hline
 & 113.983 & 4.583 & 17 & 6.12 & 10.88 & 335 & 319.20 & 15.80 \\
\hline
 & 120.1 & 16.067 & 8.8 & 1.34 & 7.46 & 267.7 & 281.24 & 13.54 \\
\hline
 & 120.267 & 22.683 & 15 & 2.65 & 12.35 & 230.1 & 215.31 & 14.79 \\
\hline
 & 120.433 & 22.467 & 20 & 2.27 & 17.73 & 229.02 & 212.45 & 16.57 \\
\hline
 & 120.583 & 22.35 & 20.9 & 2.21 & 18.69 & 217.2 & 222.04 & 4.84 \\
\hline
 & 120.683 & 22.067 & 20 & 1.70 & 18.30 & 200.5 & 199.14 & 1.36 \\
\hline
 & 120.717 & 22.45 & 17.9 & 2.28 & 15.62 & 235.3 & 230.01 & 5.29 \\
\hline
 & 120.883 & 5.7 & 19.6 & 2.94 & 16.66 & 176.1 & 167.05 & 9.05 \\
\hline
\multirow{13}{*}{$K_{2}$} & 106.667 & 20.867 & 4.4 & 1.32 & 3.08 & 98.78 & 97.27 & 1.51 \\
\hline
 & 107.367 & 21.033 & 13.5 & 1.31 & 12.19 & 193.1 & 180.50 & 12.60 \\
\hline
 & 109.533 & 18.217 & 20.3 & 2.26 & 18.04 & 303.4 & 321.59 & 18.19 \\
\hline
 & 120.1 & 16.067 & 8.8 & 0.98 & 7.82 & 267.7 & 256.37 & 11.33 \\
\hline
 & 120.267 & 22.617 & 15.4 & 1.62 & 13.78 & 236.12 & 240.14 & 4.02 \\
\hline
 & 120.267 & 22.683 & 15 & 1.39 & 13.61 & 230.1 & 241.81 & 11.71 \\
\hline
 & 120.3 & 16.617 & 7.6 & 2.58 & 5.02 & 264.2 & 246.42 & 17.78 \\
\hline
 & 120.417 & 17.783 & 7.7 & 2.71 & 4.99 & 228 & 234.13 & 6.13 \\
\hline
 & 120.433 & 22.467 & 20 & 1.71 & 18.29 & 229.02 & 239.41 & 10.39 \\
\hline
 & 120.583 & 22.35 & 20.9 & 1.75 & 19.15 & 217.2 & 232.81 & 15.61 \\
\hline
 & 120.683 & 22.067 & 20 & 1.59 & 18.41 & 200.5 & 200.04 & 0.46 \\
\hline
 & 120.717 & 22.45 & 17.9 & 1.78 & 16.12 & 235.3 & 235.40 & 0.10 \\
\hline
 & 120.883 & 5.7 & 19.6 & 2.96 & 16.64 & 176.1 & 174.99 & 1.11 \\
\hline
$P_{1}$ & 106.667 & 20.867 & 4.4 & 3.00 & 1.40 & 98.78 & 101.99 & 3.21 \\
\hline
\end{tabular}
\end{table}

\begin{table}
\centering
\begin{tabular}{c c c c c c c c c}
\hline
\multirow{2}{*}{分潮类型} & \multicolumn{2}{c}{验潮站} & \multicolumn{3}{c}{振幅H/cm} & \multicolumn{3}{c}{迟角G/°} \\
\cline{2-9}
 & 经度 & 纬度 & 观察值 & 计算值 & MSEH & 观察值 & 计算值 & MSEG \\
\hline
 & 107.033 & 20.717 & 4.4 & 2.63 & 1.77 & 130.7 & 111.80 & 18.90 \\
 & 109.533 & 18.217 & 20.3 & 1.42 & 18.88 & 303.4 & 307.92 & 4.52 \\
 & 120.217 & 5.35 & 13.5 & 2.39 & 11.11 & 210.4 & 192.92 & 17.48 \\
 & 120.267 & 22.683 & 15 & 1.86 & 13.14 & 230.1 & 240.44 & 10.34 \\
 & 120.3 & 16.617 & 7.6 & 1.51 & 6.09 & 264.2 & 248.54 & 15.66 \\
 & 120.583 & 22.35 & 20.9 & 1.73 & 19.17 & 217.2 & 205.39 & 11.81 \\
 & 120.883 & 5.7 & 19.6 & 2.70 & 16.90 & 176.1 & 175.22 & 0.88 \\
\hline
\multirow{2}{*}{$Q_{1}$} & 109.533 & 18.217 & 20.3 & 2.14 & 18.16 & 303.4 & 289.18 & 14.22 \\
 & 120.883 & 5.7 & 19.6 & 2.94 & 16.66 & 176.1 & 176.58 & 0.48 \\
\hline
\end{tabular}
\end{table}

\begin{figure}[h]
\centering
\includegraphics[width=\textwidth]{image.png}
\caption{各个分潮与验潮站的调和常数(振幅H和迟角G)绝均差结果}
\end{figure}

\section{问题四:潮汐调和分析优化过程讨论}

针对沿轨道的潮汐调和常数分离、插值和拟合过程中,分布图和同潮时图绘制的精度,主要引入了一定特定函数多项式进行拟合,为了提高精度,多项式次数越高,对正压潮分离使用了特定的多项式次数进行拟合,本部分主要从两个方面出发:1) 针对不同次数下的多项式拟合分潮调和常数进行正压潮和内潮的分离分析讨论;2) 针对不同插值方法和方式的同潮图绘制的分析讨论。最终实现海洋潮汐调和常数。

\subsection{不同次数下的多项式拟合的分析讨论}

本部分主要针对南海海区内20条轨道8个主要分潮的潮汐调和常数沿轨进行3-15次多项式拟合,针对上下行轨道分别进行了多次多项式拟合,拟合结果如下图5.37-图5.40所示。

\begin{figure}[h]
    \centering
    \includegraphics[width=\textwidth]{image1.png}
    \caption{(a) 1 号上行轨道振幅 H 拟合对比图 \hspace{1cm} (b) 2 号下行轨道振幅 H 拟合对比图}
    \label{fig:5.37}
\end{figure}

图 5.37 $M_{2}$ 分潮的不同次数多项式拟合对比图

\begin{figure}[h]
    \centering
    \includegraphics[width=\textwidth]{image2.png}
    \caption{(a) 1 号上行轨道振幅 $H$ 拟合对比图 \hspace{1cm} (b) 2 号上行轨道振幅 $H$ 拟合对比图}
    \label{fig:5.38}
\end{figure}

图 5.38 $S_{2}$ 分潮的不同次数多项式拟合对比图

\begin{figure}[h]
    \centering
    \includegraphics[width=\textwidth]{image3.png}
    \caption{(a) 1 号上行轨道振幅 H 拟合对比图 \hspace{1cm} (b) 2 号下行轨道振幅 H 拟合对比图}
    \label{fig:5.39}
\end{figure}

图 5.39 $K_{1}$ 分潮的不同次数多项式拟合对比图

\begin{table}[h]
\centering
\caption{不同次数下分潮振幅的均方差}
\begin{tabular}{|c|c|c|c|c|c|c|}
\hline
\multirow{2}{*}{轨道编号} & \multicolumn{5}{c|}{不同次数下分潮振幅的均方差} \\
\cline{2-6}
& 5 & 7 & 9 & 11 & 13 \\
\hline
\multirow{10}{*}{上行轨道} & 2 & 1.62 & 1.53 & 1.54 & 1.54 & 1.56 \\
\cline{2-6}
& 4 & 0.27 & 0.27 & 0.27 & 1.01 & 1.00 \\
\cline{2-6}
& 5 & 1.22 & 1.23 & 1.02 & 1.41 & 1.42 \\
\cline{2-6}
& 7 & 0.43 & 0.41 & 0.41 & 1.13 & 1.15 \\
\cline{2-6}
& 14 & 0.29 & 0.25 & 0.24 & 0.98 & 0.98 \\
\cline{2-6}
& 15 & 0.74 & 0.67 & 0.53 & 1.19 & 1.17 \\
\cline{2-6}
& 16 & 0.50 & 0.33 & 0.32 & 1.06 & 1.06 \\
\cline{2-6}
& 17 & 0.84 & 0.82 & 0.77 & 1.31 & 1.31 \\
\cline{2-6}
& 18 & 0.29 & 0.19 & 0.15 & 0.83 & 0.83 \\
\cline{2-6}
& 19 & 0.86 & 0.74 & 0.71 & 1.30 & 1.29 \\
\cline{2-6}
& 20 & 0.53 & 0.47 & 0.35 & 1.09 & 1.09 \\
\hline
\multirow{8}{*}{下行轨道} & 1 & 3.67 & 3.59 & 3.56 & 1.94 & 1.93 \\
\cline{2-6}
& 3 & 1.42 & 1.41 & 1.41 & 1.54 & 1.54 \\
\cline{2-6}
& 6 & 1.37 & 1.38 & 1.36 & 1.53 & 1.52 \\
\cline{2-6}
& 8 & 1.82 & 1.82 & 1.81 & 1.64 & 1.65 \\
\cline{2-6}
& 9 & 0.34 & 0.33 & 0.32 & 1.06 & 1.06 \\
\cline{2-6}
& 10 & 1.27 & 1.27 & 1.27 & 1.50 & 1.50 \\
\cline{2-6}
& 11 & 0.97 & 0.96 & 0.96 & 1.40 & 1.40 \\
\cline{2-6}
& 12 & 0.89 & 0.89 & 0.89 & 1.37 & 1.37 \\
\cline{2-6}
& 13 & 0.90 & 0.90 & 0.89 & 1.37 & 1.37 \\
\hline
\multicolumn{2}{|c|}{均方根} & 1.01 & 0.97 & 0.94 & 1.31 & 1.31 \\
\hline
\end{tabular}
\end{table}

由表5.9可知,沿1-20号轨道分别进行5、7、9、11、13次多项式拟合得到的振幅平均均方差分别为1.01cm、0.97cm、0.94cm、1.31cm、1.31cm,9次多项式拟合均方差最小。由图5.41-图5.42可以看出,虽然5、7、11、13次多项式的拟合结果都反映海表面$M_{1}$分潮振幅的变化趋势,但不同次数多项式的拟合结果与海表面$M_{1}$分潮振幅的符合程度存在明显差异,其中,基本上下行轨道最佳次数稳定在9次方前后,部分下行轨道为了方便计算,可以采用较为低次的多项式。

\begin{figure}[h]
    \centering
    \includegraphics[width=\textwidth]{image1.png}
    \caption{上行轨道最佳次数多项式分析(部分轨道)}
    \label{fig:5.41}
\end{figure}

\begin{figure}[h]
    \centering
    \includegraphics[width=\textwidth]{image2.png}
    \caption{下行轨道最佳次数多项式分析(部分轨道)}
    \label{fig:5.42}
\end{figure}

从上行和下行轨道的各个次数的多项式拟合方面综合分析,下行轨道的在较低的次数拟合效果较好,且趋势明显;而上行轨道的振幅 \( H \) 沿轨道的分布较为发散,不太具有一定的规律,在这种情况下,进行多项式拟合,效果不明显,从而对后续插值获得同潮图的工作带来一定的难度。启示:1) 可以分上下行轨道分方法插值,设置权重,增加下行轨道对插值结果的权重比例,降低上行轨道振幅数据对插值结果的影响,从而可以获得较为完备的各个分潮的同潮分布图;2) 依据不同分潮类型的实际情况,进行不同的插值方式研究,最后进行叠加,从而达到完整的同潮图绘制。

\subsection{5.4.2 不同插值方法的同潮图绘制的分析讨论}

由于时间的关系,只从理论角度出发,研究局部加权的插值方法。参考文献 [14] 可以利用局部加权二次多项式来对数据进行插值处理。假设在网格点 \((x_n, y_n)\) 的影响距离内存在 \( k \) 个数据点,为了将每个网格点映射变量 \( z \) 中,引入插值函数

式子 5.34:
\begin{equation}
\hat{z}_{k}=b_{1}+b_{2} \Delta x_{k}+b_{3} \Delta y_{k}+b_{4} \Delta x_{k}^{2}+b_{5} \Delta x \Delta y+b_{6} \Delta y_{k}^{2}, k=1, \ldots, K
\tag{5.34}
\end{equation}
式中 \(\Delta x_{k}=x_{k}-x_{n}\); \(\Delta y_{k}=y_{k}-y_{n}\); \(x_{k}\) 和 \(y_{k}\) 表示 T/P-J 第 \(k\) 个点的轨道坐标。

用指数函数 (5.35) 将这个等式加权:
\begin{equation}
w_{k}=\exp \left(-\frac{r_{i k}}{L}\right)
\tag{5.35}
\end{equation}
式子中 \(r_{i k}=\sqrt{\left(\Delta x_{k}^{2}+\Delta y_{k}^{2}\right)}\); \(L\) 为 1.0, 得到了一组确定的方程 (5.36):
\begin{equation}
M B=Z
\tag{5.36}
\end{equation}
式中 \(B=\left(b_{1}, b_{2}, \ldots, b_{6}\right)^{T}\), \(Z=\left(z_{1}, z_{2}, \ldots, z_{k}\right)^{T}\), \(M=\left[\begin{array}{llllll}w_{1} & m_{1} & n_{1} & o_{1} & p_{1} & r_{1} \\w_{2} & m_{2} & n_{2} & o_{2} & p_{2} & r_{2} \\\cdots & \cdots & \cdots & \cdots & \cdots & \cdots \\w_{K} & m_{K} & n_{K} & o_{K} & p_{K} & r_{K}\end{array}\right]\)

式中 \(m_{1}=w_{1} \Delta x\); \(\quad m_{2}=w_{2} \Delta x_{2}\); \(\quad m_{K}=w_{K} \Delta x_{K}\); \(\quad n_{1}=w_{1} \Delta y_{1}\); \(\quad n_{2}=w_{2} \Delta y_{2}\); \(\quad n_{K}=w_{K} \Delta y_{K}\); \(\quad o_{1}=w_{1} \Delta x_{1} \Delta y_{1}\); \(\quad o_{2}=w_{2} \Delta x_{2} \Delta y_{2}\); \(\quad o_{K}=w_{K} \Delta x_{K} \Delta y_{K}\); \(\quad p_{1}=w_{1} \Delta x_{1}^{2}\); \(\quad p_{2}=w_{2} \Delta x_{2}^{2}\); \(\quad p_{K}=w_{K} \Delta x_{K}^{2}\); \(\quad r_{1}=w_{1} \Delta y_{1}^{2}\); \(\quad r_{2}=w_{2} \Delta y_{2}^{2}\); \(\quad r_{K}=w_{K} \Delta y_{K}^{2}\)。

应用最小二乘法进行拟合可得到方程 (17):
\begin{equation}
\left(M^{T} M\right) B=M^{T} Z
\tag{5.37}
\end{equation}
式中上标 \(T\) 表示转置; 可以用标准算法解出方程 (5.37), 在 \((x_{n}, y_{n})\) 的 \(\hat{z}\) 预测值为 \(b_{1}\)。

\section{六、模型评价}

\subsection{6.1 模型优点}

针对问题一, 利用 T/P 卫星所测得的数据已能完全克服主要分潮间的混淆影响, 运用沿迹调和分析法可获得沿迹高分辨率的潮汐参数分布, 连续最小二乘调和分析方法, 克服了传统的基于离散数据的最小二乘法建立起来的潮汐调和分析方法的诸多不足, 特别是所建立起来的模型不需要等时间间隔的观测数据, 但其法方程组系数矩阵的形成依旧快速。且能更好地保证用大量分潮做调和分析时法方程组系数矩阵的计算非奇异性和计算收敛性。能够分离处理正压潮和斜压潮的主要成分及其椭圆参量, 丰富了观测资料的数量, 其得到的振幅基本具有等精度, 与实测资料与遥感资料分析的分潮振幅相比较, 本方法更能有效的分离潮汐信息。

针对问题二, 沿轨道分析了各个分潮的潮汐调和常数分布情况, 有细结构, 利用多项式拟合的方法分别对 \(H \text { sing}\) 与 \(H \cos g\) 进行拟合, 利用内潮和正压潮的叠加为原始的调和常数的原理, 从而达到正压潮和内潮的分离, 获得较为平滑的正压潮调和常数参数, 模型具有的稳定性。

针对问题三, 通过利用问题二获得的各个分潮沿轨道上的正压潮的调和常数数据出发, 通过克里金插值方法, 绘制出了不同分潮的同潮图, 利用最近的验潮站的潮汐调和数据对比评价, 发现本模型具有较好的拟合结果。

\subsection{6.2 模型缺点}

针对问题一, T/P 卫星在深海中的潮汐量值均方差为 \(5 \mathrm{~cm}\), 而由于潮汐的潜水效应, 导致其精度大大降低。最小二乘调和分析法得到的迟角的精度不均匀,

有时不太稳定。

针对问题二, 正压潮和内潮的分离模型, 主要依据内潮对正压潮的调制作用, 该作用与实际情况下的潮汐特征具有一定的差异性, 需要更为具有理论和实际操作意义的数据分离方法。

针对问题三, 本文采用空间插值法对问题二平滑出来的正压潮的调和常数进行空间插值, 从而获得各个分潮在整个空间上的同潮分布图, 插值的效果与数据平滑程度和的要求较高, 对数据具有较强依耐性。

\subsection{6.3 模型改进}

在针对问题一和问题四中, 可以利用数值模拟的方法对 T/P 数据进行拟合和模拟, 本次由于时间的原因, 数值模拟动力学方法较为复杂, 短时间内不能够得到结果。可以考虑数据资料完善, 利用数值模拟动力学的模型, 对该问题进行求解和改进。

\subsection{6.4 模型推广}

该模型不仅能够在潮汐调和分析方面具有较广的普适性, 而且可以拓宽多项式拟合和数据分析等领域的运用, 为波浪、风暴、环流、水团等其他海洋现象的研究, 提供技术支持。

\section{七、参考文献}

[1]. 陈宗镛, 潮汐学, 北京: 科学出版社, 1980.

[2]. 方国洪, 郑文振, 陈宗镛, 王骥, 潮汐和潮流的分析与预报, 北京: 海洋出版社, 1986.

[3]. 黄祖珂, 黄磊, 潮汐原理与计算, 青岛, 中国海洋大学出版社, 2005

[4]. 孙丽艳: 渤黄东海潮汐底摩擦系数的优化研究 [硕士学位论文]. 青岛: 中国海洋大学海洋环境学院, 2006.

[5]. 范丽丽: 风暴潮数值同化研究和高度计资料拟合方法研究 [硕士学位论文]. 青岛: 中国海洋大学海洋环境学院, 2011.

[6]. 李培良. 渤黄东海潮波同化数值模拟和潮能耗散的研究[D]. 中国海洋大学, 2002.

[7]. 王延强, 仉天宇, 朱学明. 基于 18.6 年卫星高度计资料对南海潮汐的分析与研究[J]. 海洋预报, 2014, 31(2):35-40.

[8]. 赵云霞, 魏泽勋, 王新怡. 利用 T/P 卫星高度计资料调和分析南海潮汐信息 [J]. 海洋科学, 2012, 36(5):10-17.

[9]. 胡景. 卫星高度计数据提取海洋潮汐信息及气候变化研究[D]. 中国海洋大学, 2007.

[10]. 王斌, 张晓爽, 吕咸青, 等. 高度计资料提取内潮信号的方法[J]. 解放军理工大学学报(自然科学版), 2015(3):266-272.

[11]. 唐岩, 刘雁春, 暴景阳, 等. 关于潮流(准)调和分析方法中的几个问题[J]. 测绘科学, 2010(s1):33-35.

[12]. 刘晓宇. 基于卫星测高数据反演海水深度潮汐动力学方法研究[D]. 南京师范大学, 2012.

[13]. 孙新轩, 许军, 暴景阳. 利用 T/P 卫星测高数据提取潮汐参数的研究[J]. 测绘通报, 2006(7):16-18.

[14]. 付延光, 周兴华, 许军, 等. 利用 TOPEX/Poseidon 和 Jason-1 高度计数据提取中国南海潮汐信息[J]. 武汉大学学报(信息科学版), 2018(6).