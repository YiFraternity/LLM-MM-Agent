\begin{center}
\textbf{全国第八届研究生数学建模竞赛}
\end{center}

\begin{center}
南京邮电大学
\end{center}

\begin{tabular}{l l}
学校 & 10293009 \\
\hline
参赛队号 & \\
\hline
队员姓名 & 1. 叶强 \\
& 2. 顾芹芹 \\
& 3. 施娟娟 \\
\hline
\end{tabular}

\begin{flushright}
参赛密码 \underline{\hspace{2cm}} \\
(由组委会填写)
\end{flushright}

\begin{center}
\textbf{参赛密码} \underline{\hspace{2cm}} \\
(由组委会填写)
\end{center}

\begin{center}
\textbf{全国第八届研究生数学建模竞赛}
\end{center}

\begin{center}
\includegraphics[width=0.5\textwidth]{logo.png}
\end{center}

\begin{flushleft}
\textbf{题目} \hspace{1cm} 小麦发育后期茎秆抗倒性的数学模型
\end{flushleft}

\begin{center}
\textbf{摘 \hspace{2cm} 要:}
\end{center}

倒伏不但造成小麦减产,而且影响小麦的籽粒品质。因此要实现小麦高产优质的跨越,就必须解决或尽量减少小麦的倒伏问题。论文通过研究相关试验参考数据,寻找小麦抗倒伏能力最佳的茎秆性状的同时通过物理力学类比研究小麦抗倒伏性。

针对问题一,我们建立的抗倒伏指数公式为:
\[
\text{茎秆抗倒伏指数} = \frac{\text{茎秆机械强度}}{\text{茎秆鲜重} \times \text{茎秆重心高度}}
\]

首先根据 2007 年的数据,计算出 2007 年各品种小麦的抗倒伏指数见表 5 至表 8。然后,依据 2007 年数据,通过非线性曲线拟合的方法建立茎秆机械强度与茎杆高度、重量和各节壁厚的函数模型,估算出 2008 和 2011 年的茎杆机械强度。同时,引入重心高度计算公式,计算出 2008 年各品种小麦的重心高度。最后,代入抗倒伏指数公式算出 2008 年和 2011 年各品种小麦的抗倒伏指数,见表 9 和表 10。

针对问题二,首先是相关性指标的分析,我们采用多元回归分析法,探讨得到小麦茎杆性状引起倒伏的相关性顺序为:穗长 > 株高 > 基部第一二节长度比 > 基部第二节壁厚 > 基部第一节壁厚 > 穗重 > 基部第一节长 = 基部第二节长。我们用相关系数来表征和判断小麦的鲜重、壁厚、株高、粗、穗长等因素的相关性,得到鲜重与壁厚、株高、粗、穗长成正比,小麦的壁厚与粗和穗长成正比与株高成反比,株高与小麦的穗长和粗成正比,小麦的粗与穗长也成正比。2008 年国信 1 号和智 9998 倒伏的原因是基部第一、二节的壁厚太薄,株高较高。

针对问题三,以抗倒伏指数最大为目标,给定穗重的前提下优化小麦每节茎秆高度和重量、穗长,得到小麦的理想株型结构见表 15。

针对问题四,我们将麦秆当作弹性材料处理,考虑麦秆根倒情况,建立小麦麦穗在单独穗重作用下的最大抗弯曲正应力和抗倒伏指数的表达式:见式(18)、(29);以及在穗重和风载联合作用下的最大抗弯曲正应力和抗倒伏指数的表达式:见式(24)、(29);考虑麦秆茎倒情况,在同样的两种受力模型下,得出麦秆的最大抗弯曲正应力和抗倒伏指数的表达式分别见式(20)、(29),式(28)、(29)。

针对问题五,1. 我们通过对问题四模型的求解,计算出了 2007 年蜡熟期各品种小麦,在麦秆茎倒情况下,麦穗自重单独作用的最大抗弯曲正应力与抗倒伏指数,见表 17;以及麦穗自重和风载共同作用的最大抗弯曲正应力和抗倒伏指数,见表 19。2. 通过风载单独作用下麦穗临界力的公式和风压伯努力公式计算出 2007 年蜡熟期各品种小麦的抗倒伏风速,见(27)式,计算结果见表 18。

针对第六问,我们先总结了前五问所建模型的方法,分析和评价了我们所建模型的优劣,对于现有的通过实验和理论计算两种得出小麦抗倒伏指数的方法提出了我们自己的看法,并提出了修改意见,制定了一套相对完整的实验方案和分析方法,并给出了育种实践中的建议。

关键词:机械强度,抗倒伏指数,茎秆临界力,抗倒伏风速

\section*{一、问题重述}

小麦高产、超高产的研究始终是小麦育种家关注的热点问题。随着产量的增加,小麦的单茎穗重不断增加。但穗重的增加同时使茎秆的负荷增大,导致容易倒伏。倒伏不但造成小麦减产,而且影响小麦的籽粒品质。因此要实现小麦高产优质的跨越,就必须解决或尽量减少小麦的倒伏问题。

小麦倒伏从形式上可分为“根倒”和“茎倒”,一般都发生在小麦发育后期。“根倒”主要与小麦种植区域的土壤品种与结构特性有关,本题不做讨论。“茎倒”是高产小麦倒伏的主要形式,尤其是发生时间较早的“茎倒”,往往造成大幅度的减产。“茎倒”的原因是茎秆与穗的自重和风载作用的迭加超过了小麦茎秆的承受能力。

解决倒伏问题的方法之一就是针对不同的产量,寻找小麦抗倒伏能力最佳的茎秆性状(包括株高、茎长、各节间长、各节茎外径、壁厚、茎秆自重、穗长、穗重等)。

通过物理力学类比研究小麦抗倒伏性是一个新方向,已有一些工作。值得我们进行探讨。题目的附件中收集了一批各个品种小麦的茎秆性状、产量、倒伏情况的数据。请你们就已有数据解决以下几个问题:

(1) 依据有些论文中判断茎秆抗倒性的抗倒伏指数公式:
\[
\text{茎秆抗倒伏指数} = \text{茎秆鲜重} \times \text{茎秆重心高度} / \text{茎秆机械强度}
\]
对提供的数据,建立各品种小麦的茎秆抗倒指数公式。对于缺乏有关参数的年份,可进行合理的假设,如通过已知数据求茎秆机械强度与茎秆粗厚的关系。

(2) 研究抗倒伏指数与茎秆外部形态特征之间的关系。即给出抗倒伏指数与株高、穗长、各节间长、节间长度比、各节壁厚、穗重、鲜重等茎秆性状在最易引起倒伏期的相关性指标。

判断小麦茎秆性状的各个因素之间是否有相关性?并对 2008 年国信 1 号与智 9998 品种的小麦都发生倒伏,其他品种没有发生倒伏的原因给出判断。

(3) 探讨单穗重分别是 1.19g,2.06g,2.46g,2.56g,2.75g,2.92g 时小麦的理想株型结构。

(4) 将茎秆按刚/弹性材料处理,研究小麦茎秆在麦穗自重和风载作用下应力的基本规律,引用、修改附件三文献中力学公式或自己另行推导,建立小麦茎秆抗倒伏的数学模型。

(5) 应用 (4) 力学模型中的抗弯刚度 \(EI\),麦穗自重下和风载作用下的公式对 2007 年腊熟期各品种数据进行计算,有些参数可依据需要作某些假设。

因腊熟期小麦叶片、叶鞘多已脱落,可设风力仅对单穗穗头起作用,暂时忽略风力对小麦茎秆作用。试计算在 2007 年数据中腊熟期各品种的抗倒伏风速(取小数后两位)。

(6) 总结所建模型及分析结果,提出值得考虑的问题。同时请你为 2012 年制定完整的试验方案及数据分析方法。并给小麦育种家在育种实践中提出合理的建议。

\section*{二、问题分析}

针对问题一,依据我们查阅资料以及经验总结,我们建立以题目所给抗倒伏指数公式的倒数作为做题的抗倒伏指数公式,即

\[
\text{茎秆抗倒伏指数} = \frac{\text{茎秆机械强度}}{\text{茎秆鲜重} \times \text{茎秆重心高度}} \quad \left( C = \frac{M}{F \times H} \right)
\]

在该公式中,计算各年份各品种小麦的抗倒伏指数需要三个变量:茎秆机械强度,茎秆鲜重,茎秆重心高度。但是只有 2007 年的数据统计中完整的包含该三要素,2011 年数据缺失茎秆机械强度,2008 年的统计数据中既缺失机械强度也缺失重心高度。因此我们首先根据 2007 年的数据进行假设推理,得到机械强度和重心高度与各年统计数据中共有的数据选项之间的联系,从而可以从 2008 以及 2011 年的已有数据推导出其机械强度或重心高度,最终利用抗倒伏指数公式求出各年份的各品种小麦的抗倒伏指数。

针对问题二,根据第一问求出的抗倒伏指数计算它与茎秆的外部形态特征的相关指标,然后以此为依据对国信 1 号和智 9998 品种小麦发生倒伏的原因进行判断。查阅相关资料,我们初步选取株高,穗长,基部第一节壁厚、基部第二节壁厚,基部第一、二节长之比,穗重这六个茎秆性状与抗倒伏指数作多元回归分析。

针对问题三,探讨不同穗重小麦的理想株型结构,利用最优化的模型,以抗倒伏指数最高为目标,在给定穗重的条件下,优化小麦的株型结构。

针对问题四,我们将小麦茎秆按弹性结构处理,首先研究小麦麦穗在根倒和茎倒两种倒伏状态下,受麦穗自重单独作用时麦秆的应力规律。我们将其应力模型归纳成计算端约束为下端固定,上端自由并在自由端受轴向压力作用时等直细长杆在临界失稳平衡状态下的所受的最大抗弯折正应力,即麦秆的机械强度,然后根据机械强度和抗倒伏指数的换算公式可以得到麦秆的抗倒伏指数计算公式。然后我们将该应力模型推广,考虑小麦麦穗在根倒和茎倒两种倒伏状态下,受麦穗自重和风载共同作用下的应力模型,我们首先根据附件三中的公式,得出风载单独作用时麦穗顶部的临界力计算公式,进而得出该临界力作用在麦秆弯折处的弯矩,再加上麦穗穗重单独作用下在弯折处产生的弯矩,可以算出这两个临界力共同作用时产生的合弯矩,进而算出麦秆的最大抗弯折正应力和麦秆的抗倒伏指数。

针对问题五,我们仅计算麦秆茎倒的情况,对于第五问的第(1)问我们根据上一问所建立的茎秆在麦穗自重单独作用下以及麦穗自重和风载共同作用下的应力模型分别算出以上两种情况下麦秆弯折处的最大抗弯曲应力进而算出 2007 年蜡熟期各品种小麦的抗倒伏指数。对于第五问的第(2)问,我们先算出风载单独作用在麦穗顶部时的临界力,然后再计算出蜡熟期小麦穗头的迎风面积,用该临界力除以迎风面积就可以得到作用在小麦穗头的风压,再代入计算风压的伯努力公式可以算得各个小麦品种的抗倒伏风速。

针对问题六,我们首先对问题一到问题五每一问建模的方法做了归纳和总结,然后分析了每一问模型的考虑到和没有考虑到的因素,以及算法的优缺点,提出了模型改进的建议,最后,我们针对前五问在分析、实验和求解麦秆抗倒伏指数所采用的实验方案、数据分析方法的不足提出了我们自己的修改意见和补充,目的是联合已有方案中的优势,拟作为未来 2012 年制定完整的实验方案和数据分析方法提供依据,并给小麦育种专家提供了育种实践中的建议。

\section*{三、模型假设}

1、假设 2008 年各小麦的数据样本在小麦的腊熟期所测得;

\section*{四、符号说明}

\begin{itemize}
    \item $C$: 茎秆抗倒伏指数;
    \item $F$: 茎秆鲜重;
    \item $H_{1}$: 茎秆重心高度;
    \item $M$: 茎秆机械强度;
    \item $d_{i}$: 茎秆每一节的直径;
    \item $w_{i}$: 茎秆每一节的壁厚;
    \item $l_{i}$: 茎秆每一节的长度;
    \item $l_{0}$: 穗头的长度
    \item $l$: 表示麦穗顶部到固定点的轴向距离
    \item $m$: 为麦穗的质量
    \item $\overline{m}$: 为麦秆的单位长度的质量,
    \item $h$: 为麦穗穗头的高度,
    \item $L_{1}$: 为麦秆的高度
    \item $w$: 为杆上任意横截面处杆的挠度,
    \item $x$: 表示麦秆横截面上某点到固定点处的距离,
    \item $\delta$: 表示临界力作用点处,麦穗顶部的最大挠度。
    \item $E$: 表示麦秆的弹性模量,
    \item $I$: 表示麦秆横截面的惯性矩
    \item $F_{cr}$: 表示麦穗自重作用下麦秆的临界力
    \item $W$: 表示试样抗弯截面系数,
    \item $D$: 表示茎秆外茎,
    \item $t$: 表示茎秆壁厚
    \item $M_{1}^{'}$: 表示茎倒情况下麦穗自重作用下麦秆折断处的弯矩
    \item $\sigma_{1}^{'}$: 茎倒情况下麦穗自重作用下麦秆最大抗弯曲正应力
    \item $\sigma_{1}$: 麦穗自重根倒情况下茎秆弯曲至破坏时所能承受的最大抗弯曲正应力
    \item $M_{1}$: 麦穗自重根倒情况下临界力产生的倒伏弯矩
    \item $P_{cr}$: 风载单独作用下的临界力
    \item $E$: 表示弹性模量
    \item $I$: 表示惯性矩
    \item $M_{0}$: 根倒情况下风力及茎秆重力产生的倒伏弯矩
    \item $\sigma_{0}$: 麦穗自重和风载联合作用下弯曲至破坏时所能承受的最大抗弯曲正应力
    \item $M_{2}^{'}$: 风载单独作用时麦秆顶端受力点处的弯矩
    \item $M^{'}$: 茎倒情况下风力及茎秆重力产生的倒伏弯矩
    \item $\sigma^{'}$: 茎倒情况下麦穗自重和风力联合作用下麦秆所能承受的最大抗弯曲正应力
\end{itemize}

\begin{itemize}
    \item $g_{i}$: 茎秆每一节的重量;
    \item $g_{0}$: 穗的重量;
    \item $P_{w}$: 风力对单穗穗头的作用力
    \item $S$: 穗头迎风面积
    \item $m_{s}$: 为小麦单穗鲜重
\end{itemize}

\section{模型建立与求解}

\subsection{问题一}

\subsubsection{问题一的模型建立}

我们通过阅读文献,经过分析将茎秆抗倒伏指数公式模型修正为:
\[
\text{茎秆抗倒伏指数} = \frac{\text{茎秆机械强度}}{\text{茎秆鲜重} \times \text{茎秆重心高度}} \quad \left( C = \frac{M}{F \times H_{1}} \right)
\]
其中茎秆机械强度 $M$ 的单位为 kg;鲜重 $F$ 的单位为 g;重心高度 $H_{1}$ 的单位为 cm。

从各年份数据的完整性看,只有 2007 年的数据是完整的,因此我们可以直接通过抗倒伏指数公式计算出 2007 年的各品种小麦的抗倒伏指数。

但是,由于 2008 年的数据缺少小麦茎秆机械强度和茎秆重心高度,2011 年的数据缺少小麦的茎秆机械强度,因此解决此问题需要分三步:
\begin{enumerate}
    \item[(a)] 通过 2007 年数据拟合出茎秆机械强度与茎秆粗厚、长度以及重量的函数关系,从而计算 2008 及 2011 年各小麦的茎秆机械强度。
    \item[(b)] 建立茎秆的重心高度与茎秆各节长度和重量的函数关系,计算出 2008 年各小麦的茎秆重心高度。
    \item[(c)] 根据抗倒伏指数公式求出各品种小麦的抗倒伏指数。
\end{enumerate}

\subsubsection{茎秆机械强度模型}

我们采用曲线拟合的方法来求解茎秆机械强度,将茎秆临界力视为茎秆的机械强度,通过附件分析可以看出茎秆机械强度与茎秆长度即穗位的高度成反比,与茎秆厚度和茎秆截面直径成正比。因此,我们首先构造了一个中间变量 $temp$ 表征与机械强度的直接关系,令
\begin{equation}
temp = \sum_{i} \frac{d_{i} \cdot w_{i}}{l_{i}^{2}}
\tag{1}
\end{equation}
其中 $d_{i}$ 为各品种小麦的第 $i$ 节茎秆的直径(即数据表中的粗细),$w_{i}$ 为小麦的第 $i$ 节茎秆的壁厚,$l_{i}$ 小麦的第 $i$ 节茎秆的长度。首先我们筛选 2007 年数据,将一些有数据缺失的样本或某些元素数据明显异常的样本(如某节的茎秆粗低于该品种茎秆粗的平均值一定范围)滤除,再计算已选的样本的中间变量 $temp$ 的值,并运用 MATLAB 软件将这些中间变量 $temp$ 的值与其对应的机械强度值拟合出一条曲线。最后根据这条拟合线推算 2008 年以及 2011 年各品种小麦样本的机械强度。

\subsubsection{茎秆重心高度模型}

对这部分的求解我们首先假设出重心高度的计算公式,再进行验证。

2008 年的样本数据中没有茎秆的重心高度,明显地,茎秆的重心高度与其每一节茎秆的长度、重量和麦穗的长度和重量具有一定关系,由于每一节的茎秆(或穗)的密度相同,因此各节的重心高度可以认为在其中心位置,对每一节的

重心高度求加权平均就可以等效整株茎秆(含穗)的重心高度,因此我们假设重心高度计算公式为(小麦结构如右图所示):

\begin{equation}
H_{1} = \frac{\sum_{i=1}^{n} \left( g_{i} * \left( \sum_{j=1}^{i-1} l_{j} + l_{i}/2 \right) \right)}{\sum_{i=1}^{n} g_{i}}
\tag{2}
\end{equation}

其中 $H_{1}$ 为茎秆重心高度,$n$ 为小麦的茎秆节数(这边将穗当做茎秆的一节),$l_{i}$ 为茎秆基部第 $i$ 节的长度(即 $l_{n}$ 为穗的长度),$g_{i}$ 为茎秆基部第 $i$ 节的重量($g_{n}$ 为穗的重量)。

将假设的重心高度公式应用于 2011 年数据样本,将 2011 年的每个有效小麦茎秆的长度重量和麦穗的长度重量带入到上述公式得出其重心高度进行验证,与实验测出的重心高度值相差 $\pm 1 \, \text{cm}$,表 1 展示部分验证结果。

\begin{table}
\centering
\caption{2011 年矮抗 58 重心高度}
\begin{tabular}{|c|c|c|c|c|c|c|c|c|}
\hline
测量值 (cm) & 34.10 & 33.10 & 31.40 & 34.80 & 33.10 & 28.64 & 32.60 & 28.25 \\
\hline
理论值 (cm) & 33.40 & 32.42 & 30.69 & 34.28 & 32.13 & 28.04 & 31.61 & 27.58 \\
\hline
误差 (cm) & +0.7 & +0.68 & +0.71 & +0.52 & -0.13 & +0.60 & +0.99 & +0.67 \\
\hline
\end{tabular}
\end{table}

因此,茎秆重心高度模型确定为我们所假设的公式。

(3)抗倒伏指数模型

在各年份各品种小麦的重心高度,机械强度,茎秆鲜重数据完整后直接利用抗倒伏指数公式:

\begin{equation}
\text{茎秆抗倒伏指数} = \frac{\text{茎秆机械强度}}{\text{茎秆鲜重} \times \text{茎秆重心高度}}
\end{equation}

\subsection*{5.1.2 问题一的模型求解}

(1)机械强度的拟合曲线方程

在拟合的过程中,我们采用精确度较高的多项式曲线拟合,为了使拟合的多项式系数在可控范围呢,并且使拟合的结果更接近实验值,我们对拟合的输入数据建立范德蒙多矩阵,将给定的数据代入范德蒙多矩阵关系式中,以使得计算出来的多项式系数向量的均方误差最小。为了使拟合的多项式系数范围不至于过大,不超过精度类型所能进行的数学运算极限,我们采用均值和标准差的数据缩放方法,将输入数据按比例缩小,将数据点减去均值,相当与将坐标原点平移到均值处,除以标准偏差意味着减小数据点之间的差距。由于我们采用了最小均方误差非线性拟合法,拟合的精度较高,因此对 2008 和 2011 年机械强度的预测数据也具有一定的说服力。

由 2007 年的数据,不同品种小麦的机械强度与中间变量 $temp$(即茎秆粗细,茎秆长度,壁厚的函数)所拟合出的曲线如下:

(a)矮抗

\begin{figure}[h]
\centering
\includegraphics[width=0.8\textwidth]{image.png}
\end{figure}

\begin{figure}[h]
    \centering
    \includegraphics[width=\textwidth]{image1.png}
    \caption{矮抗 58 拟合曲线}
    \label{fig:1}
\end{figure}

该拟合曲线的阶数为 15,横轴 \( x \) 表示 \( temp \),纵轴 \( y \) 表示茎秆机械强度,我们拟合出的解析表达式为: 
\[
\sum_{i=0}^{15} p_i x^i
\]
其中拟合系数为:
\[
[p_{15}, \ldots, p_0] = [0.055565, -0.23009, -0.86052, 3.3056, 2.8024, -13.063, 3.8874, 22.526, 3.615, -18.872, -2.973, 7.7124, 1.4803, -1.4435, -0.11603, 0.69243]
\]

\subsubsection{(b) 新麦}

\begin{figure}[h]
    \centering
    \includegraphics[width=\textwidth]{image2.png}
    \caption{新麦的拟合曲线}
    \label{fig:2}
\end{figure}

该拟合曲线的阶数为 12,横轴 \( x \) 表示 \( temp \),纵轴 \( y \) 表示茎秆机械强度,我们拟合出的表达式是:
\[
\sum_{i=0}^{12} p_i x^i
\]
其中拟合系数为:
\[
[p_{12}, \ldots, p_0] = [-0.1111, 0.3072, 0.6739, -2.104, -1.534, 5.438, 1.650, -6.544, -0.8557, 3.601, 0.1626, -0.6979, 0.4111]
\]

(c) 周麦

\begin{figure}[h]
    \centering
    \includegraphics[width=0.8\textwidth]{placeholder.png} % 替换为实际图像文件名
    \caption{周麦的拟合曲线}
    \label{fig:curve_fitting}
\end{figure}

该拟合曲线的阶数为 10,横轴 \( x \) 表示 \( temp \),纵轴 \( y \) 表示茎秆机械强度,我们得出的拟合表达式是:\(\sum_{i=0}^{10} p_i x^i\),其中拟合系数为:

\[
[p_{10}, \ldots, p_0] = [-1.175, 5.530, 2.759, -24.256, -15.617, 17.973, 11.925, -3.880, -2.504, 0.386, 0.666]
\]

(2) 根据曲线方程估算其余小麦茎秆机械强度

由于2007年的小麦品种与2008年的小麦品种不同,因此我们先根据上面三条拟合曲线分别计算2008年六种小麦的茎秆机械强度,然后取三个值的期望来作为最后六种小麦的茎秆机械强度。

\begin{table}[h]
    \centering
    \caption{2008年六种小麦的机械强度}
    \label{tab:2008_strength}
    \begin{tabular}{|c|c|c|c|c|c|c|}
        \hline
        小麦品种 & 矮抗58 & 周18 & 国信 & 智9998 & 联丰85 & TM \\
        \hline
        机械强度 (kg) & 0.6833 & 0.58178 & 0.58689 & 0.58835 & 0.58615 & 0.58611 \\
        \hline
    \end{tabular}
\end{table}

类似的,根据上面三条拟合曲线分别计算2011年七种小麦的机械强度,然后取三个值的期望,作为最后七种小麦的机械强度。

\begin{table}[h]
    \centering
    \caption{2011年七种小麦的机械强度}
    \label{tab:2011_strength}
    \begin{tabular}{|c|c|c|c|c|c|c|c|}
        \hline
        小麦品种 & 矮58 & 平安6 & 温麦6 & 郑麦9023 & 周麦18 & 周麦22 & 豫麦18 \\
        \hline
        机械强度 (kg) & 0.68874 & 0.58789 & 0.58743 & 0.58919 & 0.58591 & 0.58853 & 0.58796 \\
        \hline
    \end{tabular}
\end{table}

(3) 将该重心高度模型用于求解2008年的小麦重心高度,结果如表4:

\begin{table}
\centering
\begin{tabular}{|c|c|c|c|c|c|c|}
\hline
小麦品种 & 矮抗58 & 周18 & 国信1号 & 智9998 & 联丰85 & TM \\
\hline
重心高度 (cm) & 52.12 & 56.87 & 64.79 & 58.13 & 55.47 & 60.78 \\
\hline
\end{tabular}
\caption{六种小麦的重心高度}
\end{table}

\section*{(4) 茎秆抗倒伏指数:}

(a) 2007年各品种各时期的小麦的茎秆鲜重、重心高度以及机械强度的数据均是实验得出的,直接代入到茎秆抗倒伏指数公式中求出每个样本的抗倒伏指数,作平均后即为该品种在该时期的茎秆抗倒伏指数,具体结果如表5-表8:

\begin{table}
\centering
\begin{tabular}{|c|c|c|c|c|}
\hline
 & 蜡熟期 & 乳熟期 & 灌浆期 & 开花期 \\
\hline
茎秆抗倒伏指数 ($\times 10^{-3}$) & 1.337 & 1.708 & 2.724 & 2.538 \\
\hline
\end{tabular}
\caption{矮抗58:(播种时间:2006.10.10)}
\end{table}

\begin{table}
\centering
\begin{tabular}{|c|c|c|c|c|}
\hline
 & 蜡熟期 & 乳熟期 & 灌浆期 & 开花期 \\
\hline
茎秆抗倒伏指数 ($\times 10^{-3}$) & 1.436 & 1.490 & 2.352 & 3.508 \\
\hline
\end{tabular}
\caption{矮抗58:(播种时间:2006.10.25)}
\end{table}

\begin{table}
\centering
\begin{tabular}{|c|c|c|c|c|}
\hline
 & 蜡熟期 & 乳熟期 & 灌浆期 & 开花期 \\
\hline
茎秆抗倒伏指数 ($\times 10^{-3}$) & 0.855 & 1.148 & 1.452 & 2.186 \\
\hline
\end{tabular}
\caption{新麦208:(播种时间:2006.10.25)}
\end{table}

\begin{table}
\centering
\begin{tabular}{|c|c|c|c|c|}
\hline
 & 蜡熟期 & 乳熟期 & 灌浆期 & 开花期 \\
\hline
茎秆抗倒伏指数 ($\times 10^{-3}$) & 1.088 & 1.306 & 2.080 & 1.952 \\
\hline
\end{tabular}
\caption{周麦18:(播种时间:2006.10.25)}
\end{table}

(b) 2008年各品种小麦的机械强度、重心高度已由上面曲线拟合的方法得到,分别代入到抗倒伏指数公式求出每个样本的抗倒伏指数,最后作平均就是2008年六种小麦分别的抗倒伏指数,结果如表9:

\begin{table}
\centering
\begin{tabular}{|c|c|c|c|c|c|c|}
\hline
小麦品种 & 矮抗58 & 周18 & 国信 & 智9998 & 联丰85 & TM \\
\hline
抗倒伏指数 ($\times 10^{-3}$) & 3.8523 & 2.8181 & 3.2377 & 3.4436 & 2.9909 & 1.9997 \\
\hline
\end{tabular}
\caption{2008年六种小麦的抗倒伏指数}
\end{table}

(c) 2011年各品种小麦茎秆重心高度是已知的,机械强度已由上面得到,

将各参数分别代入抗倒伏指数公式求出每个样本的抗倒伏指数,最后作平均就可以得到 2011 年七种小麦的抗倒伏指数。

\begin{table}[h]
\centering
\caption{2011 年七种小麦的抗倒伏指数}
\begin{tabular}{|c|c|c|c|c|c|c|c|}
\hline
小麦品种 & 矮抗 58 & 平安 6 & 温麦 & 郑麦 9023 & 周麦 18 & 周麦 22 & 豫麦 18 \\
\hline
抗倒伏指数 ($\times 10^{-3}$) & 2.854 & 2.100 & 2.2007 & 2.659 & 2.166 & 2.636 & 2.290 \\
\hline
\end{tabular}
\end{table}

\subsection{问题二}

\subsubsection{模型建立}

(1) 抗倒伏指数与茎秆外部形态特征的关系模型

小麦的抗倒伏指数与茎秆外部形态特征,如株高、穗长、各节间长、节间长度比、各节壁厚、穗重、鲜重之间都有一定的关系,是一个典型的多元统计问题。设小麦的抗倒伏指数是因变量 $y$,考虑如下 8 个自变量:株高 $x_1$,穗长 $x_2$,基部第一节长 $x_3$,基部第二节长 $x_4$,第一二节长度比 $x_5$,第一节壁厚 $x_6$,第二节壁厚 $x_7$,穗重 $x_8$,建立多元回归数学模型为:
\[
y = \beta_0 + \beta_1 x_1 + \beta_2 x_2 + \beta_3 x_3 + \beta_4 x_4 + \beta_5 x_5 + \beta_6 x_6 + \beta_7 x_7 + \beta_8 x_8 + \varepsilon
\]
$\beta_i$ 就即为回归系数,这个值的正负表征变量同抗倒伏指数成正成反比的关系,绝对值大小即表征了该变量的的相关程度,系数越大相关程度越高。

(2) 茎秆性状的各因素是否有相关性模型

相关系数是在线性相关的情况下,用来说明相互关系密切程度的统计分析指标。本问中我们采用相关系数的概念来判断小麦茎秆性状的各个因素之间的相关性:
\[
\rho = \frac{n \sum_{i=1}^n x_i y_i - \sum_{i=1}^n x_i \sum_{i=1}^n y_i}{\sqrt{\left[n \sum_{i=1}^n x_i^2 - \left(\sum_{i=1}^n x_i\right)^2\right]\left[n \sum_{i=1}^n y_i^2 - \left(\sum_{i=1}^n y_i\right)^2\right]}} \quad -1 \leq \rho \leq +1
\tag{3}
\]
$\rho$ 即为变量 $X$ 和变量 $Y$ 的相关性系数,变量 $X$ 代表茎秆性状的一个特征(如株高)数据,$Y$ 代表茎秆性状除 $X$ 外的其他任一特征数据,$i$ 表示 $X$ 和 $Y$ 数据样本个数。

\subsubsection{模型求解}

(1) 抗倒伏指数与茎秆外部形态特征的关系模型求解

考虑本问题没有区分品种,因此我们将 2008 年 6 个品种小麦的有效样本数据集合中的株高,穗长,基部第一节长,基部第二节长,第一二节长度比,第一节壁厚,第二节壁厚,穗重数据分别求其平均值,然后以这 6 种品种小麦各自的平均株高,穗长,基部第一节长,基部第二节长,第一二节长度比,第一节壁厚,第二节壁厚,穗重数据(如表 11)为自变量,以它们各自的抗倒伏指数作为因变量进行回归性分析。

\begin{table}
\centering
\caption{6组观测数据}
\begin{tabular}{|c|c|c|c|c|c|c|c|c|}
\hline
 & 株高 & 穗长 & 第一 & 第二 & 第一 & 第二 & 穗重 \\
 & (cm) & (cm) & 节长 & 节长 & 节长度 & 节壁 & 节壁 & (g) \\
 & & & (cm) & & 比 & 厚 & 厚 & \\
 & & & & & & (mm) & (mm) & \\
\hline
矮抗 & 58.38 & 8.53 & 4.550 & 7.410 & 0.61403 & 0.55 & 0.55 & 2.460 \\
58 & 0 & & & & 5 & & & \\
\hline
周18 & 62.46 & 9.02 & 4.250 & 6.470 & 0.65687 & 0.55 & 0.575 & 2.750 \\
 & & & & & 8 & & & \\
\hline
国信 & 74.78 & 8.77 & 6.030 & 7.030 & 0.85775 & 0.375 & 0.325 & 1.910 \\
 & & & & & 2 & & & \\
\hline
智 & 66.69 & 8.3 & 6.900 & 7.670 & 0.89960 & 0.45 & 0.425 & 2.060 \\
9998 & & & & & 9 & & & \\
\hline
联丰 & 61.65 & 9.11 & 6.190 & 7.240 & 0.85497 & 0.575 & 0.5 & 2.560 \\
85 & & & & & 2 & & & \\
\hline
TM & 75.92 & 15.04 & 5.910 & 8.610 & 0.68641 & 0.525 & 0.575 & 2.920 \\
 & & & & & 1 & & & \\
\hline
\end{tabular}
\end{table}

代入模型得到以下方程组:
\begin{align*}
3.8523 &= \beta_0 + 58.38\beta_1 + 8.53\beta_2 + 4.55\beta_3 + 7.41\beta_4 + 0.61\beta_5 + 0.55\beta_6 + 0.55\beta_7 + 2.46\beta_8 + \varepsilon_1 \\
2.8181 &= \beta_0 + 62.46\beta_1 + 9.02\beta_2 + 4.25\beta_3 + 6.47\beta_4 + 0.65\beta_5 + 0.55\beta_6 + 0.57\beta_7 + 2.75\beta_8 + \varepsilon_2 \\
3.2377 &= \beta_0 + 74.78\beta_1 + 8.77\beta_2 + 6.03\beta_3 + 7.03\beta_4 + 0.85\beta_5 + 0.375\beta_6 + 0.325\beta_7 + 1.91\beta_8 + \varepsilon_3 \\
3.4436 &= \beta_0 + 66.69\beta_1 + 8.3\beta_2 + 6.9\beta_3 + 7.67\beta_4 + 0.89\beta_5 + 0.45\beta_6 + 0.425\beta_7 + 2.06\beta_8 + \varepsilon_4 \\
2.9909 &= \beta_0 + 61.65\beta_1 + 9.11\beta_2 + 6.19\beta_3 + 7.24\beta_4 + 0.85\beta_5 + 0.575\beta_6 + 0.5\beta_7 + 2.56\beta_8 + \varepsilon_5 \\
1.9997 &= \beta_0 + 75.92\beta_1 + 15.04\beta_2 + 5.91\beta_3 + 8.61\beta_4 + 0.68\beta_5 + 0.525\beta_6 + 0.575\beta_7 + 2.92\beta_8 + \varepsilon_6
\end{align*}

在古典线性回归模型中,根据中心极限定理通常假定 $\varepsilon_1, \varepsilon_2, \ldots, \varepsilon_6$ 独立且 $\varepsilon_i \sim N(0, \sigma^2), i = 1, \ldots, 6$,在一般情况下,只假定模型满足:$E(\varepsilon_i) = 0, D(\varepsilon_i) = \sigma^2$(对所有 $i$)及 $Cov(\varepsilon_i, \varepsilon_s) = 0, t \neq s$。可用矩阵表示为:
\[
\begin{bmatrix}
3.8523 \\
2.8181 \\
3.2377 \\
3.4436 \\
2.9909 \\
1.9997
\end{bmatrix}
=
\begin{bmatrix}
1 & 58.38 & 8.53 & 4.550 & 7.410 & 0.614035 & 0.55 & 0.55 & 2.460 \\
1 & 62.46 & 9.02 & 4.250 & 6.470 & 0.656878 & 0.55 & 0.575 & 2.750 \\
1 & 74.78 & 8.77 & 6.030 & 7.030 & 0.857752 & 0.375 & 0.325 & 1.910 \\
1 & 66.69 & 8.3 & 6.900 & 7.670 & 0.899609 & 0.45 & 0.425 & 2.060 \\
1 & 61.65 & 9.11 & 6.190 & 7.240 & 0.854972 & 0.575 & 0.5 & 2.560 \\
1 & 75.92 & 15.04 & 5.910 & 8.610 & 0.686411 & 0.525 & 0.575 & 2.920
\end{bmatrix}
\begin{bmatrix}
\beta_0 \\
\beta_1 \\
\beta_2 \\
\beta_3 \\
\beta_4 \\
\beta_5 \\
\beta_6 \\
\beta_7 \\
\beta_8
\end{bmatrix}
+
\begin{bmatrix}
\varepsilon_1 \\
\varepsilon_2 \\
\varepsilon_3 \\
\varepsilon_4 \\
\varepsilon_5 \\
\varepsilon_6
\end{bmatrix}
\]

简记为:$Y = X\beta + \varepsilon$。

用最小二乘法求 $\beta$ 的估计,为此解方程组:
\[
\frac{\partial Q}{\partial \beta} = 0
\]

其中, \( Q = \sum_{a=1}^{n} (y_a - \hat{y}_a)^2 = \sum_{a=1}^{n} \varepsilon_i^2 \) 为误差平方和,其中

\[
\hat{y}_a = \beta_0 + \beta_1 x_{a1} + \beta_2 x_{a2} + \beta_3 x_{a3} + \beta_4 x_{a4} + \beta_5 x_{a5} + \beta_6 x_{a6} + \beta_7 x_{a7} + \beta_8 x_{a8}.
\]

可以得到: \(\hat{\beta} = (X'X)^{-1} X'Y\)

解得

\[
\hat{\beta} =
\begin{bmatrix}
\beta_0 \\
\beta_1 \\
\beta_2 \\
\beta_3 \\
\beta_4 \\
\beta_5 \\
\beta_6 \\
\beta_7 \\
\beta_8
\end{bmatrix}
=
\begin{bmatrix}
0 \\
-1.346952 \\
-5.59185 \\
0.000 \\
0.000 \\
1.296912 \\
0.48031 \\
0.52966 \\
-0.012
\end{bmatrix}
\]

所以,株高、穗长、基部第一二节长度比的回归系数分别为 \(-1.347\)、\(-5.592\)、\(1.297\),系数比较大,与抗倒伏指数的相关性大,基部第一节壁厚、基部第二节壁厚、穗重的回归系数分别为 \(0.48031\)、\(0.52966\)、\(-0.012\),系数相对较小,与抗倒伏指数的相关性稍微弱一点,基部第一节长与基部第二节长的回归系数均为 \(0\),不能认为其与抗倒伏指数无关,只是相关性很小。小麦茎杆性状引起倒伏的相关性大小为:穗长 \(>\) 株高 \(>\) 基部第一二节长度比 \(>\) 基部第二节壁厚 \(>\) 基部第一节壁厚 \(>\) 穗重 \(>\) 基部第一节长 \(=\) 基部第二节长。

(2) 茎秆性状的各因素是否有相关性模型求解

将 2008 年各品种的有效数据样本取出,求得这些品种小麦的平局株高、平均粗、平均穗长、平均壁厚、平均秆重(表 12)作为 \(X\),\(Y\) 求相关系数,结果见表 13。

表 12 各品种小麦的形态特征

\begin{tabular}{|c|c|c|c|c|c|c|}
\hline 小麦品种 & 矮抗 58 & 周 18 & 国信 & 智 9998 & 联丰 85 & TM \\
\hline 株高 (cm) & 58.38 & 62.46 & 74.78 & 66.69 & 61.65 & 75.92 \\
\hline 粗 (cm) & 0.45 & 0.43 & 0.4 & 0.38 & 0.47 & 0.64 \\
\hline 穗长 (cm) & 8.53 & 9.02 & 8.77 & 8.30 & 9.11 & 15.04 \\
\hline 壁厚 (mm) & 0.58 & 0.57 & 0.42 & 0.43 & 0.56 & 0.55 \\
\hline 杆重 (g) & 0.94 & 0.92 & 0.94 & 0.88 & 0.98 & 1.91 \\
\hline
\end{tabular}

代入相关系数的公式中,得到茎秆各特征之间的相关系数如下表:

表 13 茎秆各特征之间的相关系数

\begin{tabular}{|c|c|c|c|c|c|}
\hline 茎秆特征 & 杆重 & 壁厚 & 株高 & 粗 & 穗长 \\
\hline 杆重 & 1 & 0.2505 & 0.6111 & 0.9575 & 0.9968 \\
\hline 壁厚 & 0.2505 & 1 & -0.5452 & 0.4966 & 0.2670 \\
\hline 株高 & 0.6111 & -0.5452 & 1 & 0.3951 & 0.6115 \\
\hline 粗 & 0.9575 & 0.4966 & 0.3951 & 1 & 0.9568 \\
\hline 穗长 & 0.9968 & 0.2670 & 0.6115 & 0.9568 & 1 \\
\hline
\end{tabular}

\begin{table}
\centering
\begin{tabular}{|c|c|c|c|c|c|c|}
\hline
 & & 株高(cm) & 穗长(cm) & 第一-二节长度比 & 第一节壁厚(mm) & 第二节壁厚(mm) \\
\hline
\multirow{2}{*}{发生倒伏品种} & 国信一号 & 74.78 & 8.77 & 0.857752 & 0.375 & 0.325 \\
\cline{2-7}
 & 智9998 & 66.69 & 8.3 & 0.899609 & 0.45 & 0.425 \\
\hline
\multirow{4}{*}{未发生倒伏品种} & 矮抗 & 58.380 & 8.53 & 0.614035 & 0.55 & 0.55 \\
\cline{2-7}
 & 周18 & 62.46 & 9.02 & 0.656878 & 0.55 & 0.575 \\
\cline{2-7}
 & 联丰85 & 61.65 & 9.11 & 0.854972 & 0.575 & 0.5 \\
\cline{2-7}
 & TM & 75.92 & 15.04 & 0.686411 & 0.525 & 0.575 \\
\hline
未倒伏品种数据平均值 & & 64.60 & 10.43 & 0.703074 & 0.55 & 0.55 \\
\hline
\end{tabular}
\caption{2008年六种小麦的部分形态特征对比}
\end{table}

的株高、穗长、节间长度比、各节壁厚等茎秆的外部形态特征。  
将重心公式 $H = \frac{\sum\limits_{i=1}^{n} \left( g_{i} * \left( \sum\limits_{j=1}^{i-1} l_{j} + l_{i} / 2 \right) \right)}{\sum\limits_{i=1}^{n} g_{i}}$ 代入抗倒伏指数公式 $C = \frac{M}{F \times H}$ 得

\begin{equation}
C = \frac{M * W}{(G + W) \sum\limits_{i=1}^{n} \left( g_{i} * \left( \sum\limits_{j=1}^{i-1} l_{j} + l_{i} / 2 \right) \right)}
\tag{4}
\end{equation}

其中 $M$ 为茎秆机械强度,机械强度 $M$ 从表 1 可以得到,由于表 1 中有 6 个机械强度,所以最后的最优解由 6 个结果取平均值。$W$ 为茎秆的秆重,$G$ 为穗重,$g_{i}$ 为茎秆每一节的重量,$l_{i}$ 为茎秆每一节的重量。这里假设小麦是五节的,所以 $n = 5$。本题的数学模型为:

\begin{equation}
\text{Max } \mathrm{C} = \frac{M * W}{(G + W) \sum_{i=1}^{n} \left( g_i * \left( \sum_{j=1}^{i-1} l_j + l_i / 2 \right) \right)}
\tag{5}
\end{equation}

\begin{equation}
\begin{aligned}
s.t. \quad & \begin{cases}
n = 5 \\
0 \leq G = t \\
0 \leq W \leq 2 \\
2 \leq l_1 \leq 7 \\
2 \leq l_2 \leq 12 \\
2 \leq l_3 \leq 13.7 \\
6 \leq l_4 \leq 17 \\
2 \leq l_5 \leq 33.8 \\
0.08 \leq g_1 \leq 0.91 \\
0.08 \leq g_2 \leq 0.74 \\
0.08 \leq g_3 \leq 1.54 \\
0.08 \leq g_4 \leq 2.64 \\
0.08 \leq g_5 \leq 1.75 \\
\sum_{i=1}^{n} g_i = W \\
1.5 < l_2 / l_1 < 2.0 \\
2.0 < l_3 / l_1 < 2.5 \\
3.0 < l_4 / l_1 < 3.5 \\
5.0 < l_5 / l_1 < 5.5
\end{cases}
\end{aligned}
\end{equation}

其中 $l_i (i = 1..5), g_i (i = 1..5)$ 的范围是根据2008年测试数据的实际经验得来的。最后4个节间比的约束条件参考文献\cite{ref10}。

\subsection{5.3.2 模型的求解}

由于要求的是在单穗重量分别是 $1.19\,\mathrm{g}, 2.06\,\mathrm{g}, 2.46\,\mathrm{g}, 2.56\,\mathrm{g}, 2.75\,\mathrm{g}, 2.92\,\mathrm{g}$ 时的小麦的理想株型结构,所以令 $t$ 分别为 $1.19, 2.06, 2.46, 2.56, 2.75, 2.92$ 时,求模型的最优解。通过 lingo 得到每种情况下小麦的理想株型结构为:

\begin{table}
\centering
\caption{在不同单穗重下小麦的理想结构}
\begin{tabular}{|c|c|c|c|c|c|c|}
\hline
单穗重(g) & 1.19 & 2.06 & 2.46 & 2.56 & 2.75 & 2.92 \\
\hline
第一节长(cm) & 2.8 & 4.9 & 5.88 & 6.19 & 6.25 & 6.91 \\
\hline
第一节重(g) & 0.14 & 0.20 & 0.23 & 0.24 & 0.27 & 0.37 \\
\hline
第二节长(cm) & 4.03 & 6.83 & 7.27 & 7.41 & 8.47 & 8.61 \\
\hline
第二节重(g) & 0.13 & 0.23 & 0.22 & 0.23 & 0.25 & 0.323 \\
\hline
第三节长(cm) & 4.26 & 7.54 & 8.53 & 8.64 & 9.12 & 11.98 \\
\hline
第三节重(g) & 0.11 & 0.165 & 0.17 & 0.17 & 0.19 & 0.307 \\
\hline
第四节长(cm) & 7.88 & 10.43 & 13.43 & 14.66 & 16.09 & 16.30 \\
\hline
第四节重(g) & 0.08 & 0.14 & 0.17 & 0.14 & 0.15 & 0.285 \\
\hline
第五节长(cm) & 12.10 & 19.37 & 23.03 & 25.91 & 26.24 & 29.88 \\
\hline
第五节重(g) & 0.05 & 0.09 & 0.1 & 0.13 & 0.14 & 0.226 \\
\hline
\end{tabular}
\end{table}

此图为单穗重为1.19g时的小麦理想结构模型。

\subsection{5.4 问题4的模型建立与求解}

\subsubsection{5.4.1 模型的建立}

首先,根据题目的要求,我们将麦秆抽象成理想的线弹性杆,将第四问要求的应力模型归纳为线弹性稳定问题。

设 $m$ 为麦穗的质量,$\overline{m}$ 为麦秆的单位长度的质量,$H$ 为麦穗穗头的高度,$L_{1}$ 为麦秆的高度,我们将麦秆质量的四分之一折算到上部麦穗穗头重心处,即将麦秆简化为无重的等截面弹性杆,如图所示:

\begin{figure}[h]
    \centering
    \includegraphics[width=0.4\textwidth]{image1.png}
    \caption{图4}
\end{figure}
\begin{figure}[h]
    \centering
    \includegraphics[width=0.4\textwidth]{image2.png}
    \caption{图5}
\end{figure}

其中: \( m^{*} = m + \frac{\overline{m \cdot L_{1}}}{4}, \quad L = L_{1} + 0.5H \)

1. 麦穗自重下的应力模型(根倒情况)

对于仅考虑麦穗自重下的所能承受的最大临界正应力,我们将该问题归结为求杆端约束为一端固定,另一端自由的等直细长压杆在自由端受轴向压力作用时所受的临界力问题 \({}^{[2]}\),如图 6 所示:

\begin{figure}[h]
\centering
\includegraphics[width=0.5\textwidth]{image.png}
\caption{图 6}
\end{figure}

### 1.1 麦穗自重下的临界力

如图所示,在临界力 \( F_{cr} \) 作用下,杆失稳时有可能在 \( x-y \) 平面内维持微弯状态下的平衡,其弯曲刚度为 \( EI \)。根据杆端约束情况,由临界力 \( F_{cr} \) 引起的杆任意 \( x \) 横截面上的弯矩为:

\begin{equation}
M(x) = -F_{cr} (\delta - w)
\tag{6}
\end{equation}

式中,\( w \) 为杆上任意横截面处杆的挠度,\( x \) 表示麦秆横截面上某点到固定点处的距离,\( \delta \) 表示临界力作用点处,麦穗顶部的最大挠度。

将(6)式代入杆的挠曲线近似微分方程可得:

\begin{equation}
EI w'' = -M = F_{cr} (\delta - w)
\tag{7}
\end{equation}

其中,\( E \) 表示麦秆的弹性模量,\( I \) 表示麦秆横截面的惯性矩。

将(7)式简化得:

\begin{equation}
w'' + k^2 w = k^2 \delta
\tag{8}
\end{equation}

其中,\( k^2 = \frac{F_{cr}}{EI} \)。因此,该微分方程的通解为:

\begin{equation}
w = A \sin kx + B \cos kx + \delta
\tag{9}
\end{equation}

(9)式中的待定常数 \( A, B, k \) 可以通过挠曲线的边界条件确定:

(i) 当 \( x = 0 \) 时,挠度 \( w = 0 \)

(ii) 当 \( x = 0 \) 时,\( w' = 0 \)

(iii) 当 \( x = l \) 时,挠度 \( w = \delta \) 取得最大值(其中 \( l \) 表示麦穗顶部到固定点的轴向距离)。

将(9)式取一阶导数得:

\begin{equation}
w' = A k \cos kx - B k \sin kx
\tag{10}
\end{equation}

再将边界条件(i)、(ii)代入(10)式可得 \( A = 0 \),\( B = -\delta \);再将他们代入(9)

式可得:
\begin{equation}
w = \delta(1 - \cos kx)
\tag{11}
\end{equation}
将边界条件 (iii) 代入 (11) 式,可得:
\begin{equation}
\delta = \delta(1 - \cos kl)
\tag{12}
\end{equation}
由上式可知,挠曲线可以成立的条件为:
\begin{equation}
\cos kl = 0
\tag{13}
\end{equation}
因此可得:
\begin{equation}
kl = n\pi / 2 \quad (n = 1, 3, 5, \ldots)
\tag{14}
\end{equation}
由 (14) 式的最小解 $kl = \frac{\pi}{2}$,可得到压杆临界力 $F_{cr}$ 的欧拉公式为:
\begin{equation}
F_{cr} = EI \cdot k^2 = EI \cdot \left(\frac{\pi}{2l}\right)^2 = \frac{\pi^2 EI}{4l^2}
\tag{15}
\end{equation}
将 $k = \frac{\pi}{2l}$ 代入 (11) 式,可以得到此压杆的挠曲线方程:
\begin{equation}
w = \delta(1 - \cos \frac{\pi x}{2l})
\tag{16}
\end{equation}

### 1.2 麦穗自重下的极限抗弯曲正应力

我们在推导中心受压直杆临界力的欧拉公式时,假定材料是在线弹性范围内工作的,当压杆受临界力 $F_{cr}$ 作用而在直线平衡形态下维持不稳定平衡时,该临界力产生的倒伏弯矩为:
\begin{equation}
M_1 = \frac{F_{cr} \cdot \delta}{4} = \frac{\pi^2 EI \delta}{16l^2}
\tag{17}
\end{equation}
因此,在麦穗自重作用下,茎杆弯曲至破坏时所能承受的最大抗弯曲正应力:
\begin{equation}
\sigma_1 = \frac{M_1}{W} = \frac{\frac{\pi^2 EI \delta}{16l^2}}{\frac{\pi}{32D}\left[D^4 - \left(D - 2t\right)^4\right]} = \frac{2\pi D \delta EI}{l^2\left[D^4 - \left(D - 2t\right)^4\right]}
\tag{18}
\end{equation}
其中,$W$ 表示试样抗弯截面系数,$D$ 表示茎杆外径,$t$ 表示茎杆壁厚,$l$ 表示麦穗顶部到固定点的轴向距离。

### 2. 麦穗自重下的应力模型(茎倒情况)

我们考虑,麦秆在基部第二节折断的情况,因此,在麦秆顶部临界力的作用下,该折断处的弯矩为 $^{[5-7]}$:
\begin{equation}
M_1' = \frac{F_{cr} \cdot \left(\delta - w(x)\right)}{4} = \frac{F_{cr} \cdot \left(\delta - \delta(1 - \cos \frac{\pi x}{2l})\right)}{4} = \frac{1}{4} F_{cr} \delta \cos \frac{\pi x}{2l} = \frac{\pi^2 EI \delta}{16l^2} \cos \frac{\pi x}{2l}
\tag{19}
\end{equation}
其中,$x$ 表示基部第二节到固定点的距离。

根据最大抗弯曲正应力计算公式可得:
\begin{equation}
\sigma_1' = \frac{M_1'}{W} = \frac{\frac{\pi^2 EI \delta}{16l^2} \cos \frac{\pi x}{2l}}{\frac{\pi}{32D}\left[D^4 - \left(D - 2t\right)^4\right]} = \frac{2D \pi EI \delta \cos \frac{\pi x}{2l}}{l^2\left[D^4 - \left(D - 2t\right)^4\right]}
\tag{20}
\end{equation}

### 3. 麦穗自重和风载联合作用下的应力模型(根倒):

根据附件三中的风载单独作用下的临界力公式,可得:

\begin{equation}
P_{cr} = \frac{EI\pi^4\delta}{32l^3}
\tag{21}
\end{equation}

其中,$\delta$ 表示临界力作用点处,麦穗顶部的最大挠度,$EI$ 表示弹性模量与惯性矩的乘积,$l$ 表示麦穗顶部到固定点的轴向距离。

则该临界力产生的根倒倒伏弯矩为:
\begin{equation}
M_2 = \frac{P_{cr} \cdot l}{4} = \frac{\frac{EI\pi^4\delta}{32l^3} \cdot l}{4} = \frac{EI\pi^4\delta}{128l^2}
\tag{22}
\end{equation}

在麦穗自重和风载联合作用下茎杆的受力情况,如图所示 [3]

\begin{figure}[h]
    \centering
    \includegraphics[width=0.8\textwidth]{image.png} % 替换为实际图片路径
    \caption{}
    \label{fig:7}
\end{figure}

因此,我们可以得到风力及茎杆重力产生的倒伏弯矩 $M_0$:

\begin{equation}
M_0 = M_1 + M_2 = \frac{F_{cr} \cdot \delta}{4} + \frac{P_{cr} \cdot l}{4} = \frac{\pi^2EI\delta}{16l^2} + \frac{EI\pi^4\delta}{128l^2} = \frac{\pi^2EI\delta\left(8 + \pi^2\right)}{128l^2}
\tag{23}
\end{equation}

再运用横截面上的压应力可按材料弯曲至破坏时所能承受的最大抗弯曲正应力公式可得:
\begin{equation}
\sigma_0 = \frac{M_0}{W} = \frac{\frac{\pi^2EI\delta\left(8 + \pi^2\right)}{128l^2}}{\frac{\pi}{32D}\left[D^4 - \left(D - 2t\right)^4\right]} = \frac{\pi D\delta EI\left(8 + \pi^2\right)}{4l^2\left[D^4 - \left(D - 2t\right)^4\right]}
\tag{24}
\end{equation}

\section*{4. 麦穗自重下和风力联合作用下的应力模型(茎倒情况)}

对于茎倒情况,我们考虑麦秆在基部第二节处的折断情况,因此我们分别讨论麦穗自重和风力作用下茎杆顶端受轴向压力处最大可承受弯矩的变化:

(1) 麦穗自重作用下麦秆顶端受力点处的弯矩:
\begin{equation}
M_1' = \frac{F_{cr} \cdot \left(\delta - w(x)\right)}{4} = \frac{F_{cr} \cdot \left(\delta - \delta\left(1 - \cos\frac{\pi x}{2l}\right)\right)}{4} = \frac{1}{4}F_{cr}\delta\cos\frac{\pi x}{2l} = \frac{\pi^2EI\delta}{16l^2}\cos\frac{\pi x}{2l}
\tag{25}
\end{equation}

(2) 风载单独作用时麦秆顶端受力点处的弯矩:

\begin{equation}
M_{2}^{'} = \frac{P_{cr} \left( l - x \right)}{4} = \frac{EI \pi^{4} \delta}{32 l^{3}} \cdot \frac{\left( l - x \right)}{4} = \frac{EI \pi^{4} \delta \left( l - x \right)}{128 l^{3}}
\tag{26}
\end{equation}

因此,麦穗自身重力和风力联合作用在麦秆顶部时,茎秆基部第二节中间折断点处的合倒伏弯矩 \( M' \) 为:

\begin{equation}
M' = M_{1}^{'} + M_{2}^{'} = \frac{F_{cr} \cdot \left( \delta - w(x) \right)}{4} + \frac{P_{cr} \left( l - x \right)}{4} = \frac{\pi^{2} EI \delta}{16 l^{2}} \cos \frac{\pi x}{2l} + \frac{EI \pi^{4} \delta \left( l - x \right)}{128 l^{3}}
\tag{27}
\end{equation}

所以,麦秆所能承受的最大抗弯曲正应力为:

\begin{equation}
\sigma' = \frac{M'}{W} = \frac{\frac{\pi^{2} EI \delta}{16 l^{2}} \cos \frac{\pi x}{2l} + \frac{EI \pi^{4} \delta \left( l - x \right)}{128 l^{3}}}{\frac{\pi}{32 D} \left[ D^{4} - \left( D - 2t \right)^{4} \right]} = \frac{8 l \pi^{2} EI \delta \cos \frac{\pi x}{2l} + EI \pi^{4} \delta \left( l - x \right)}{\frac{4 \pi l^{3}}{D} \left[ D^{4} - \left( D - 2t \right)^{4} \right]}
\tag{28}
\end{equation}

根据以上所算得的根据不同外力作用下(麦穗自重和风载)麦秆的弯曲至倒伏(根倒情况和茎倒情况)时所能承受的最大抗弯曲正应力(\(\sigma_{1}\), \(\sigma_{1}^{'}\), \(\sigma\), \(\sigma^{'}\))的值,代入抗倒伏指数计算公式,可分别得出,小麦在麦穗自重单独作用下、麦穗自重和风载联合作用下的茎秆抗倒伏数学模型:

\begin{equation}
VLI = \sqrt{\frac{S}{\left( m^{*} g \cdot L \right)}}
\tag{29}
\end{equation}

其中,\( S \) 为茎秆机械强度,它可以取最大弯曲正应力变量 \(\sigma_{1}\), \(\sigma_{1}^{'}\), \(\sigma\), \(\sigma^{'}\) 中的任意一个。

\subsection*{5.5 第五问的模型建立与求解}

\subsubsection{5.5.1 模型的建立}

该问在忽略风力对小麦茎秆的作用下,只考虑风力对单穗穗头的作用,我们知道在临界抗倒伏风速条件下,风力对单穗穗头的作用力 \( P_{w} \) 应当等于单穗穗头受到的抗弯折临界作用力 \( P_{cr} \),即:

\begin{equation}
P_{cr} = P_{w}
\end{equation}

由于穗头上的风压等于风力除以穗头迎风面积 \({}^{[4]}\),即:\( W = \frac{P_{w}}{S} \),其中,在蜡熟期小麦的穗头迎风面积和小麦穗重有一定关系,可假定单穗鲜重与穗头迎风面积的关系为:

\begin{equation}
S = a + b \cdot m_{s}
\tag{30}
\end{equation}

其中 \( m_{s} \) 为小麦单穗鲜重,\( a \),\( b \) 为转换系数,\( a = 0.4841 \),\( b = 0.2238 \)。面积单位为 \( cm^{2} \),单穗穗重单位为 \( g \)。

然后根据附件中给出的风压的贝努利公式和风载单独作用下茎秆顶部所受的临界力 \( P_{cr} \) 公式:

\begin{equation}
W = \frac{1}{2} \rho v^{2} = \frac{\gamma}{2g} v^{2} = \frac{v^{2}}{1630} \left( kN / m^{2} \right)
\tag{31}
\end{equation}

其中 \( v \) 为风速,\( \rho = 1.25 kg / m^{3} \) 为标准的空气密度,\( \gamma = 0.012018 kN / m^{3} \) 为空气单位体积的重力,\( g = 9.8 m / s^{2} \) 重力加速度。

我们可以导出抗最终的倒伏风速 \( v \) 的计算公式:

\[
v = \sqrt{\frac{2 \frac{P_{w}}{a + b \cdot m_{s}}}{\rho}} = \sqrt{\frac{2 \frac{P_{cr}}{a + b \cdot m_{s}}}{\rho}} = \sqrt{\frac{2 P_{cr}}{\rho \left( a + b \cdot m_{s} \right)}}
\tag{32}
\]

### 5.5.2 模型的求解:

我们在进行模型求解时,仅考虑麦秆茎倒的情况,对于麦秆根倒的情况我们只建立理论模型,不予求解,因为根倒主要与小麦种植区域的土壤品种与结构特性有关。在求解的过程中,对于弹性模量的计算,我们利用麦秆弹性测定法,测定出来每一节麦秆测压两次后,在去除变形力样品恢复到变形前的高度之比,然后利用该弹性测定值与麦秆每一节的弹性模量之间的线性关系,得出每一节麦秆的弹性模量,在求期望得出麦秆弹性模量的均值。弹性模量的求解结果如表 16 所示:

\begin{table}[h]
\centering
\caption{2007 年不同品种小麦弹性模量}
\begin{tabular}{|c|c|c|c|}
\hline
小麦品种 & 矮抗 58 & 新麦 208 & 周麦 18 \\
弹性模量 \( E \) & & & \\
单位:MPa & & & \\
\hline
 & 1205.3 & 1019.3 & 1254.5 \\
\hline
\end{tabular}
\end{table}

由于机械强度就是茎秆抗折力,即采用拉力法,取基部第二茎节间(去叶鞘)、两端放于高 \( 50 \) cm、间隔 \( 5 \) cm 的支撑木架凹槽内,在其中部挂一弹簧秤,向下缓慢用力拉称,使茎秆折断所用的力加上弹簧称的重量。所以采用资料中完熟期节间弹性模量值表中,第二节的杨氏模量 \( 6.59 \) Gpa 作为小麦的机械强度,即 \( E = 6.59 \) Gpa。从 2011 年的测试数据 1 得到周麦穗长 \( H = 8.93 \) cm,穗重 \( m = 2.81 \) g,基部第一节长 \( 6.615 \) cm;矮抗穗长 \( H = 7.97 \) cm,穗重 \( m = 2.03 \) g,基部第一节长 \( 5.34 \) cm;然后从 2007 年中周麦与矮抗的穗长穗重以及基部第一节长和各节长粗的关系,进行合理推算,假设新麦穗长 \( H = 8.5 \) cm,穗重 \( m = 2.47 \) g,第一节长 \( 7 \) cm。

#### (1)麦穗自重作用下最大抗弯曲正应力与抗倒伏指数的求解

利用第四问中求出的麦穗自重作用下的应力模型中的麦秆折断处的弯矩计算公式(19),最大抗弯曲正应力计算公式(20),压杆临界力计算公式(15)以及茎秆抗倒伏计算公式(29)来计算最大抗弯曲正应力和抗倒伏指数,如表 17 所示:

其中挠度的取值,我们参考了一些文献中提到的麦穗在临界力作用下挠度的经验值,即茎倒临界状态下麦穗顶部偏离轴向的角度通常可取 \( 30 \) 度~\( 45 \) 度。(以下各表中矮抗 1 是指播种时间为 2006.10.10 的小麦,矮抗 2 是指播种时间为 2006.10.25 的小麦)

\begin{table}[h]
\centering
\caption{麦穗自重单独作用时的相关指标计算}
\begin{tabular}{|c|c|c|c|c|}
\hline
小麦品种 & 周麦 18 & 新麦 208 & 矮抗 1 & 矮抗 2 \\
指标 & & & & \\
\hline
最大抗弯曲正应力 \((\mathrm{N} / \mathrm{m}^{2})\) & 3.171090241 & 3.435395742 & 4.229989642 & 4.686246358 \\
\hline
抗倒伏指数 & 0.001091477 & 0.001261787 & 0.001871717 & 0.001939001 \\
\hline
\end{tabular}
\end{table}

\section*{(2) 风载单独作用时临界力和倒伏风速的求解}

利用风载单独作用下临界力计算公式 (21) 和抗倒伏风速计算公式 (32) 来计算临界力和抗倒伏风速,如表 18 所示:

\begin{table}[h]
\centering
\begin{tabular}{|c|c|c|c|c|}
\hline
小麦品种指标 & 周麦 18 & 新麦 208 & 矮抗 1 & 矮抗 2 \\
\hline
临界力 \(P_{\mathrm{cr}}\) (N) & 0.20598 & 0.201942 & 0.557976 & 0.72085 \\
\hline
风速 (m/s) & 17.3685 & 17.8173 & 31.1318 & 35.3850 \\
\hline
风级 & 8 & 8 & 大于 12 级 & 大于 12 级 \\
\hline
\end{tabular}
\caption{风载单独作用时的相关指标计算}
\end{table}

\section*{(3) 麦穗自重和风载共同作用下最大抗弯曲正应力和抗倒伏指数的求解}

利用麦穗自重下和风力联合作用下的应力模型公式 (25) - (28) 以及茎杆抗倒伏计算公式 (29) 计算最大抗弯曲正应力和抗倒伏指数,如表 19 所示:

\begin{table}[h]
\centering
\begin{tabular}{|c|c|c|c|c|}
\hline
小麦品种指标 & 周麦 18 & 新麦 208 & 矮抗 1 & 矮抗 2 \\
\hline
最大抗弯曲正应力 (N/m²) & 6.630771651 & 7.144209749 & 8.812934861 & 9.783728222 \\
\hline
抗倒伏指数 & 0.002282 & 0.002624 & 0.0039 & 0.004048 \\
\hline
\end{tabular}
\caption{自重和风载共同作用}
\end{table}

由以上表 17 和表 19 的计算结果我们可以看出无论是在麦穗自重单独作用下还是在麦穗自重和风载共同作用下,矮抗 58 的抗倒伏指数最大,而新麦 208 和周麦 18 的倒伏指数很接近。这个结果从理论与 2007 各个品种小麦抗倒伏指数的实验数据是吻合的。

\subsection*{5.6 第六问的分析与求解}

\subsubsection{5.6.1 第一问模型的总结与结果分析}

在问题一模型的建立过程中,我们首先对 2007 年各个时期各品种小麦茎杆的机械强度与小麦茎秆性状特征(秆长、秆粗和壁厚等)因素之间的关系进行了分析,从统计意义上得出了机械强度与这几个因素之间的比例关系,并构造出了一个与机械强度有直接正相关关系的临时变量 \( \text{temp} = \sum_i \frac{d_i w_i}{l_i^2} \),(这里 \(d_i\) 为各品种小麦的第 \(i\) 节茎秆的直径,\(w_i\) 为小麦的第 \(i\) 节茎秆的壁厚,\(l_i\) 小麦的第 \(i\) 节茎秆的长度。)然后,我们通过对 2007 年三个小麦品种(矮抗 58,新麦 208 和周麦 18)各个生长时期小麦样本计算出来的临时变量 temp 与所给的机械强度之间的关系进行非线性拟合,分别得出了 2007 年三个小麦品种的茎秆性状(秆长、秆粗和 23)

壁厚等)与机械强度的三个非线性拟合解析表达式。因此,对于缺乏茎秆机械强度数据的 2008 年和 2011 年,我们根据 2008 年和 2011 年各个小麦品种所有单穗样本的茎杆性状(秆长、秆粗和壁厚等)可以类似的算出其与茎杆机械强度相关的临时变量 $temp$ 的值,再将该值分别代入之前拟合的 2007 年小麦茎秆形状与机械强度的三个非线性解析表达式,从而得到三种机械强度的预测值,再将这三种预测的值求期望得到最终预测出的机械强度的值。最后将得到所有 2008 和 2011 年所有小麦品种样本的机械强度预测值代入小麦茎秆抗倒伏指数公式中:

\[
\text{茎杆抗倒伏指数} = \frac{\text{茎杆机械强度}}{\text{茎杆鲜重} \times \text{茎杆重心高度}} \quad \left( C = \frac{M}{F \times H} \right)
\]

得到各个品种所有小麦样本的抗倒伏指数,然后对每个品种的样本数量做平均得到最终 2008 和 2011 年的各个品种小麦的抗倒伏指数。

由于,2007 年数据中已经给出了所有小麦单穗样本的机械强度,因此,对于 2007 年各品种小麦的抗倒伏指数可以直接运用上述公式算得。

\subsection*{5.6.2 第一问模型优劣的评价:}

首先,我们很好的对 2007 年小麦茎秆性状指标和机械强度指标进行了分析,建立了一个由这些茎杆形状指标组成的并且与机械强度正相关的综合指标,用该综合指标与机械强度进行非线性拟合,在拟合的过程中,我们采用精确度较高的多项式曲线拟合,为了使拟合的多项式系数在可控范围呢,并且使拟合的结果更接近实验值,我们对拟合的输入数据建立范德蒙多矩阵,将给定的数据代入范德蒙多矩阵关系式中,以使得计算出来的多项式系数向量的均方误差最小。为了使拟合的多项式系数范围不至于过大,不超过精度类型所能进行的数学运算极限,我们采用均值和标准差的数据缩放方法,将输入数据按比例缩小,将数据点减去均值,相当与将坐标原点平移到均值处,除以标准偏差意味着减小数据点之间的差距。由于我们采用了最小均方误差非线性拟合法,拟合的精度较高,因此对 2008 和 2011 年机械强度的预测数据也具有一定的说服力。

其次,我们拟合的预测解析式可以很好的反映 2007 年的三种小麦品种的机械强度,因此,为了最大程度的预测 2008 年和 2011 年更多不同品种小麦的机械强度,我们很好的综合了 2007 年的这三个预测表达式,利用每个解析式求出相应的预测机械强度,再将求出的三个预测强度做算术平均,一定程度的避免了,单个小麦品种的在预测方面的局限性,最大程度的控制了预测误差。

第三,引入重心高度计算公式,估算 2008 年小麦样本的重心高度,同时我们验证了这个重心高度公式的合理性,将该公式应用在 2011 年小麦样本计算出其理论重心高度,与实际测量的重心高度相差微小,认为可以 2008 年的理论计算的重心高度近似实际重心高度,减小误差。

但是由于 2007 年的实验数据相对缺乏,小麦品种较少,且和 2008 与 2011 年的品种不一样。因此,无法考虑各品种之间的差异,而且,2008 年和 2011 年数据都没有标记小麦测试样本所处的成长期。我们只能将 2007 年的所有成长期的小麦样本数据作为输入数据参与拟合解析式的建立。事实上,如果实验数据允许,我们可以根据 2007 年的各个成熟期的机械强度的拟合表达式预测出 2008 和 2011 年各品种小麦在相同时期下的机械强度。在这种情况下可以进一步减小预测误差。

因此,总的来说我们的模型最大程度的利用了实验数据和仿真环境,尽可能的对 2008 和 2011 年的各品种小麦机械强度和抗倒伏指数进行了合理的预测和计算。

\subsection*{5.6.3 第二问模型总结和结果分析}

我们采用多元回归分析法,探讨得到小麦麦茎杆性状引起倒伏的相关性顺序为:穗长 > 株高 > 基部第一二节长度比 > 基部第二节壁厚 > 基部第一节壁厚 > 穗重 > 基部第一节长 = 基部第二节长。然后是判断各因素间的相关性,我们用相关系数来表征和判断,得到小麦的鲜重与壁厚、株高、粗、穗长成正比,小麦的壁厚与粗和穗长成正比,而与株高成反比,株高与小麦的穗长和粗成正比,小麦的粗与穗长也成正比,最后结论为国信 1 号和智 9998 的第一二节的壁厚太小的同时株高比较大,导致了他们的倒伏特性。由于时间有限,我们仅考虑了以上几个茎杆性状,可能不够全面。

\subsection*{5.6.4 第二问模型优劣的评价}

采用多元回归分析法通过分析大量的样本数据,确定倒伏指数与茎秆外部形态特征之间的数学关系式,对所确定的数学关系式的可信程度进行统计检验,可以很容易地区分出对抗倒伏指数影响较大的是株高、穗长,对倒伏指数影响最小的是第一二节茎秆的长度。利用所确定的数学关系式,可以根据一个或几个变量的值来预测或控制另一个特定变量的取值,但是由于时间原因这里没有给出这种预测和控制。在判断小麦茎秆性状的各个因素之间的相关性时,采用的计算相关系数的方法是很合理的,它可以很直观的反映出各因素之间的关系。

\subsection*{5.6.5 第三问模型的总结和结果分析}

采用最优化方法,以抗倒伏指数最大为目标,给定穗重的前提下优化小麦每节茎秆高度和重量、穗长,从而得到小麦的理想株型结构,结果中的株型结构我们仅列出了茎秆高度和重量、穗长,时间和数据限制,缺少了茎秆粗度,但是从第二问的分析里可以估计茎秆粗度越大越好。

\subsection*{5.6.6 第三问模型优劣的评价}

利用最优化模型,求取得到的理想株型结构可以使得抗倒伏指数最大化,提升小麦茎秆的抗倒伏能力,但是缺陷在于茎秆的粗度在此模型中体现出来,需要进一步对模型修正和优化。

\subsection*{5.6.7 第四问模型的总结与结果分析}

关于第四问的模型建立,我们首先根据题目的要求将茎秆当作弹性材料处理,将麦秆抽象成一个下端固定,上端自由并在自由端受轴向压力和水平方向风载作用下的等直细长压杆。为了进一步简化模型,我们将麦秆的秆重的四分之一折算到穗头重心处,使得将麦秆简化为一个无重的等截面弹性杆。全部杆重和穗重全部集中在穗头重心处得到一个等效的麦穗质量。

接下来我们分麦秆根倒和茎倒两种情况,分别建立了麦秆在等效的麦穗重力作用下的应力模型。根据材料力学的知识,我们自己推导了杆约束为一端固定、另一端自由的情况下,在轴向临界力作用下(临界倒伏时的麦穗重力)麦秆倒伏折断点所受到的最大弯曲正应力的公式,即茎秆机械强度的计算公式。再根据茎

杆机械强度和抗倒伏指数的关系建立仅有麦穗自重作用下的抗倒伏数学模型。

然后我们把上述模型进行了推广,考虑到麦穗自重和风载同时作用下麦秆根倒和茎倒的应力模型。其中假设麦穗自重产生的临界力沿轴线向下,忽略风力对小麦茎杆的作用,只考虑风载作用在麦穗上的力,则假设风载在穗头上产生的临界力方向水平。通过将这两种临界力在麦秆折断点处产生的弯矩进行叠加,可以得到推导得到麦秆根倒和茎倒两种情况下麦秆折断点出的最大抗弯曲应力,即麦秆的机械强度从而得到麦秆的抗倒伏数学模型。

\subsection*{5.6.8 第四问模型优劣的评价:}

1. 在模型假设方面,我们应题目要求对麦秆做了弹性处理,即将麦秆当成等截面的弹性杆,并将其杆中折算到了穗头重心处。这种合理的假设方式对接下来模型的建立与简化起到了关键的作用。

2. 我们首先假设了无风状态下,考虑弹性杆仅在沿轴线方向的穗重作用下所受的临界力情况,经查阅相关力学知识,我们将其等效为材料力学里经典的杆端约束下细长压杆所受的临界力的模型,并推导出了临界力的欧拉公式和压杆长度因数。将该问题很好的归纳为了细长受压直杆在临界力作用下,处于失稳弯曲临界平衡状态下的线弹性稳定问题。

3. 其次我们推广了该模型,引入了风载作用力,推导出了在麦穗自重和风载联合作用下的麦秆所受的最大抗弯曲正应力以及抗倒伏指数,这个推广的模型具有普遍意义上的实用价值。

\subsection*{5.6.9 第五问模型的总结与结果分析:}

通过第四问建立的抗倒伏模型,我们可以分别计算出仅在麦穗自重下以及在麦穗和风载共同作用下,2007 年蜡熟期所有小麦品种的机械强度(最大抗弯曲正应力)和抗倒伏指数,其中在算弹性模量的时候,我们利用麦秆弹性测定法,测定出来每一节麦秆测压两次后,在去除变形力样品恢复到变形前的高度之比,然后利用该弹性测定值与麦秆每一节的弹性模量之间的线性关系,得出每一节麦秆的弹性模量,在求期望得出麦秆弹性模量的均值。

在计算各品种麦秆的抗倒伏风速的时候,我们首先利用风载对麦穗所作用的临界力以及麦穗的迎风面积计算出麦穗上所作用的风压,然后利用风压的伯努力公式求出小麦的抗倒伏风速。

根据计算出来的麦秆机械强度和抗倒伏指数的值,我们可以发现在单独麦穗自重作用以及麦穗自重和风载共同作用下,矮抗的机械强度和抗倒伏指数最大,新麦和周麦的抗倒伏指数很接近,并且都远小于矮抗的值。这个结果从理论上很好的验证了第一问从实验中得出的三种小麦的机械强度和抗倒伏指数的值。证明了我们通过模型算出来的结果和实际实验中的值是吻合的。

\subsection*{5.6.10 第五问模型优劣的评价:}

在第五问的计算过程中,我们很好的利用了麦秆弹性测定法中,麦秆每一节的弹性测定值与该节弹性模量的线性关系得出弹性模量的平均值,很好的回避了附件一中给出的弹性模量的复杂计算公式。而且我们用理论模型算出来 2007 年蜡熟期的茎杆机械强度和抗倒伏指数和 2007 年的实验数据相吻合,即验证了我们模型的正确性。

\subsection*{5.6.11 2012 年制定的完整实验方案和数据分析方法:}

在未来 2012 年实验方案对集中在对小麦茎秆机械强度的测定和估算的方法上:

\section*{参考文献:}

[1] 袁志华,李云东,陈合顺,麦类作物的力学模型即振动分析,河南科学,第 2 卷第 1 期:11-13, 2002

[2] 孙训方等,材料力学(I),北京:高等教育出版社,2002

[3] 袁志华,郑桂梅,苏宗伟,典型风荷载对小麦茎杆倒伏的影响,河南科学,第 23 卷第 1 期:54-55,2005

[4] 裘昭峰,翟立业,小麦穗和芒表面积的估测,作物学报,第 11 卷:138-144,1984.

[5] 胡婷,小麦茎秆抗倒伏的力学原理探讨,中国农业大学博士学位论文,2007

[6] 袁志华,冯宝萍,赵安庆,梁爱琴,作物茎秆抗倒伏的力学分析及综合评价探讨,农业工程学报,第 18 卷第 6 期:30-31,2002

[7] 孙凡,作物茎秆抗倒伏的力学研究,西南农业大学学报,第 16 卷第 2 期:183-186,1994

\begin{enumerate}
    \item 非线性拟合的程序
    \begin{verbatim}
x=[4.75E-04
4.03E-04
3.70E-04
2.90E-04
3.45E-04
2.71E-04
2.95E-04
2.98E-04
3.37E-04
4.12E-04
6.03E-04
4.03E-04
4.09E-04
4.42E-04
5.06E-04
3.24E-04
4.60E-04
3.87E-04
5.53E-04
4.37E-04
6.53E-04
3.99E-04
4.46E-04
3.40E-04
5.12E-04
4.14E-04
3.29E-04
4.77E-04
4.91E-04
6.34E-04
6.28E-04
3.68E-04
4.89E-04
7.85E-04
6.40E-04
6.83E-04
7.85E-04]
    \end{verbatim}
\end{enumerate}

\section*{附录}

[8] Duane Hanselman, 朱仁峰, 精通 Matlab 7, 北京: 清华大学出版社, 2007

[9] 张宏伟 牛广志, Lingo 8.0 及其在环境系统优化中的应用, 天津: 天津大学出版, 2005.

[10] 董琦, 王爱萍, 梁素明, 小麦基部茎节形态结构特征与抗倒性的研究, 山西农业大学学报, 188-191, 2003

\begin{align*}
&8.02E-04 \\
&4.98E-04 \\
&4.67E-04 \\
&5.57E-04 \\
&4.93E-04 \\
&8.39E-04 \\
&8.77E-04 \\
&4.61E-04 \\
&4.86E-04 \\
&5.16E-04 \\
&4.48E-04 \\
&4.72E-04 \\
&6.24E-04 \\
&5.65E-04 \\
&9.72E-04 \\
&5.69E-04 \\
&6.19E-04 \\
&8.78E-04 \\
&9.83E-04 \\
&5.75E-04 \\
&5.79E-04 \\
&6.86E-04 \\
&8.57E-04 \\
&8.24E-04 \\
&1.74E-03 \\
&5.22E-04 \\
&9.61E-04 \\
&8.86E-04 \\
&8.56E-04 \\
&6.58E-04 \\
&8.54E-04 \\
&1.02E-03 \\
&9.04E-04 \\
&] ; \\
&y=[0.80 \\
&0.45 \\
&0.40 \\
&0.35 \\
&0.50 \\
&0.30 \\
&0.40 \\
&0.25 \\
&0.60 \\
&0.50
\end{align*}

\begin{itemize}
    \item 0.80
    \item 0.45
    \item 0.60
    \item 0.40
    \item 0.45
    \item 0.60
    \item 0.30
    \item 0.55
    \item 0.60
    \item 0.40
    \item 0.90
    \item 0.30
    \item 0.50
    \item 0.60
    \item 0.70
    \item 0.55
    \item 0.40
    \item 0.75
    \item 0.70
    \item 0.40
    \item 0.45
    \item 0.40
    \item 0.30
    \item 0.70
    \item 0.60
    \item 0.55
    \item 0.60
    \item 0.80
    \item 0.30
    \item 0.60
    \item 1.20
    \item 0.80
    \item 0.80
    \item 1.50
    \item 0.60
    \item 0.50
    \item 0.65
    \item 0.55
    \item 0.60
    \item 0.80
    \item 0.75
    \item 0.85
    \item 0.70
    \item 0.60
\end{itemize}

\begin{verbatim}
0.80
0.90
0.60
0.45
0.60
0.80
0.70
0.75
0.40
1.00
1.10
1.10
0.70
0.80
1.50
1.40
];
n=15;
xm=mean(x);
s=std(x);
z=(x-xm)/s;
m=length(z);
V=ones(m,n+1);
for i=n:-1:1
    V(:,i)=z.*V(:,i+1);
end
p=(V\y)';
% p=polyfit(x,y,n);
xi=linspace(0.000,0.001,100);
zi=(xi-xm)/s;
yi=polyval(p,zi);
plot(x,y,'o',xi,yi,'k-')
xlabel('x'),ylabel('y=f(x)')
title('Fifteenth Order Curve Fitting')
axis([0.000,0.001,0,2.0])

% p=[1.2,2.5,3.6,7.9]';
% c=[2.8,4,6,8]';
% plot(p,c,'o'),grid on
% X=[ones(size(p)) p p.^2];
% a=X\c
% T=[0:0.5:5]';
% Y=[ones(size(T)) T T.^2]*a;
% plot(T,Y,p,c,'o'),grid on
\end{verbatim}

\begin{verbatim}
2. 计算弹性模量
x=[0.761, 0.988];
y=[355, 1443];
n=1;
p=polyfit(x, y, 1);
%xi=[0.908 0.877 0.961 0.906 0.961 0.911 0.981 0.966 0.949 0.964];
%xi=[0.966 0.928 0.933 0.961 0.958 0.946];
xi=[0.761 0.912 0.924 0.988 0.913];
yi=polyval(p, xi);
sum=0;
for i=1:length(yi)
    sum=sum+yi(i);
end
avyi=sum/length(yi);
3. 计算2008和2011的机械强度
function [s]=chuan1(xn, b)
for i=1:length(xn)
    s(i)=0;
    for n=1:16
        s(i)=s(i)+b(n)*xn(i)^(16-n);
    end
end
%sum=0;
%for i=1:length(xn)
%    sum=sum+s(i);
%end
%avl=sum/length(xn);
end
function [s4]=chuan2(xn, b1, b2, b3)
for i=1:length(xn)
    s1(i)=0;
    for n=1:16
        s1(i)=s1(i)+b1(n)*xn(i)^(16-n);
    end
    s2(i)=0;
    for n=1:13
        s2(i)=s2(i)+b2(n)*xn(i)^(13-n);
    end
    s3(i)=0;
    for n=1:11
        s3(i)=s3(i)+b3(n)*xn(i)^(11-n);
    end
end
\end{verbatim}

\begin{verbatim}
s4(i)=(s1(i)+s2(i)+s3(i))/3;
end
% sum1=0;
% sum2=0;
% sum3=0;
% for i=1:length(xn)
%     sum1=sum1+s1(i);
%     sum2=sum2+s2(i);
%     sum3=sum3+s3(i);
% end
% a1=sum1/length(xn);
% a2=sum2/length(xn);
% a3=sum3/length(xn);
% avn=(a1+a2+a3)/3;
end
p1=[0.055565 -0.23009 -0.86052 3.3056 2.8024 -13.063
    -3.8874 22.526 3.615 -18.872 -2.9732 7.7124 1.4803
    -1.4435 -0.11603 0.69243]; %aikang
p2=[-0.11112 0.30724 0.67395 -2.1046 -1.5349 5.4384 1.6505 -6.544
    -0.85578 3.6018 0.16262 -0.6979 0.41109]; %xinmai
p3=[-1.1755 5.5308 2.7599 -24.256 -15.617 17.973 11.925 -3.8801
    -2.5048 0.3868 0.66668]; %zhoumai
x1=[0.016269
    0.017471
    0.020827
    0.019201
    0.019319
    0.026463
    0.020367
    0.017498
    0.036535
    0.03091
    0.040725
    0.031254
    0.028367
    0.054947
    0.016829
    0.019925
    0.018989
    0.152846
    0.153246
    0.103044
    0.024801
    0.070748]
\end{verbatim}

\begin{align*}
0.02214 \\
0.025278 \\
0.021866 \\
0.049869 \\
0.029023 \\
0.076449 \\
0.374532 \\
0.128221 \\
0.010829 \\
0.009565 \\
0.010778 \\
0.010069 \\
0.011685 \\
0.013175 \\
0.009785 \\
0.024 \\
0.00934 \\
0.004754 \\
] \\
x2=[0.021298 \\
0.022345 \\
0.021505 \\
0.019099 \\
0.025149 \\
0.019556 \\
0.027735 \\
0.022734 \\
0.022867 \\
0.019646 \\
0.023465 \\
0.024568 \\
0.022546 \\
0.029504 \\
0.081788 \\
0.02822 \\
0.020156 \\
0.024178 \\
0.03375 \\
0.021016 \\
0.02496 \\
0.028076 \\
0.044221 \\
0.040893 \\
0.040849
\end{align*}

\begin{verbatim}
0.027171
0.025809
0.02588
0.033263
0.08032
0.101435
0.096447
0.064358
0.049304
0.058444
0.172507
0.065068
0.01242
0.011513
0.010179
];
x3=[0.014545
0.009197
0.010903
0.011987
0.009706
0.010267
0.01533
0.00803
0.010751
0.020782
0.015999
0.016084
0.014271
0.010439
0.015483
0.013555
0.013645
0.01532
0.013247
0.009578
0.014156
0.014802
0.008468
0.008897
0.013976
0.014575
0.011112
0.012286
\end{verbatim}

\begin{itemize}
    \item 0.016598
    \item 0.019595
    \item 0.016293
    \item 0.07651
    \item 0.016563
    \item 0.011725
    \item 0.008454
    \item 0.054032
    \item 0.090856
    \item 0.010063
    \item 0.006696
    \item 0.009548
    \item ]
    \item ;
    \item x4=[0.008508
    \item 0.01336
    \item 0.008474
    \item 0.011697
    \item 0.008944
    \item 0.013809
    \item 0.009778
    \item 0.011323
    \item 0.008238
    \item 0.009342
    \item 0.009012
    \item 0.011154
    \item 0.008014
    \item 0.010751
    \item 0.00955
    \item 0.00981
    \item 0.008292
    \item 0.007892
    \item 0.009125
    \item 0.009408
    \item 0.015391
    \item 0.008538
    \item 0.007667
    \item 0.008722
    \item 0.010738
    \item 0.014501
    \item 0.009987
    \item 0.007956
    \item 0.009942
    \item 0.008059
    \item 0.008601
\end{itemize}

\begin{verbatim}
0.007962
0.012461
0.025247
0.008329
0.007919
0.010155
0.022252
0.014749
0.018919
];
x5=[0.023297
0.018273
0.017159
0.015932
0.017969
0.024553
0.033796
0.019628
0.016767
0.017337
0.015271
0.015198
0.03045
0.020654
0.017427
0.021131
0.019705
0.017878
0.017641
0.018487
0.026643
0.018485
0.018175
0.019097
0.0324
0.035165
0.044792
0.02984
0.026296
0.045142
0.018205
0.015529
0.022688
0.033457
\end{verbatim}

\begin{itemize}
    \item 0.020671
    \item 0.01813
    \item 0.020371
    \item 0.026056
    \item 0.019952
    \item 0.012382
    \item ]
    \item x6=[0.04
    \item 0.018492
    \item 0.017896
    \item 0.01728
    \item 0.024208
    \item 0.024433
    \item 0.025475
    \item 0.013126
    \item 0.03044
    \item 0.019442
    \item 0.030158
    \item 0.013761
    \item 0.02172
    \item 0.01866
    \item 0.019991
    \item 0.01991
    \item 0.016898
    \item 0.026395
    \item 0.030994
    \item 0.026078
    \item 0.024173
    \item 0.023702
    \item 0.019104
    \item 0.024972
    \item 0.019037
    \item 0.027156
    \item 0.027828
    \item 0.015876
    \item 0.01493
    \item 0.018939
    \item 0.019368
    \item 0.018753
    \item 0.025585
    \item 0.033008
    \item 0.022151
    \item 0.025173
    \item 0.024334
\end{itemize}

\begin{verbatim}
0.025573
0.037042
0.027711
0.021363
0.008385
];
v108=chuan1(x1,p1);
v208=chuan2(x2,p1,p2,p3);
v308=chuan2(x3,p1,p2,p3);
v408=chuan2(x4,p1,p2,p3);
v508=chuan2(x5,p1,p2,p3);
v608=chuan2(x6,p1,p2,p3);
% av1=chuan1(x1,p1);
% av2=chuan2(x2,p1,p2,p3);
% av3=chuan2(x3,p1,p2,p3);
% av4=chuan2(x4,p1,p2,p3);
% av5=chuan2(x5,p1,p2,p3);
% av6=chuan2(x6,p1,p2,p3);
\end{verbatim}