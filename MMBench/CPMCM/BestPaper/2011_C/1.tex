\begin{center}
\textbf{全国第八届研究生数学建模竞赛}
\end{center}

\begin{tabular}{l l}
\hline
学校 & 南京航空航天大学 \\
\hline
参赛队号 & 10287012 \\
\hline
队员姓名 & 1. 蒋雪峰 \\
 & 2. 赵健伍 \\
 & 3. 杨璐 \\
\hline
\end{tabular}

\begin{center}
参赛密码 \_\_\_\_\_\_\_\_\_\_\_\_\_\_\_\_\_\_\_\_\_\_\_\_\_\_\_\_\_\_\_\_\_\_\_\_\_\_\_\_\_\_\_\_\_\_\_\_\_\_\_\_\_\_\_\_\_\_\_\_\_\_\_\_\_\_\_\_\_\_\_\_\_\_\_\_\_\_\_\_\_\_\_\_\_\_\_\_\_\_\_\_\_\_\_\_\_\_\_\_\_\_\_\_\_\_\_\_\_\_\_\_\_\_\_\_\_\_\_\_\_\_\_\_\_\_\_\_\_\_\_\_\_\_\_\_\_\_\_\_\_\_\_\_\_\_\_\_\_\_\_\_\_\_\_\_\_\_\_\_\_\_\_\_\_\_\_\_\_\_\_\_\_\_\_\_\_\_\_\_\_\_\_\_\_\_\_\_\_\_\_\_\_\_\_\_\_\_\_\_\_\_\_\_\_\_\_\_\_\_\_\_\_\_\_\_\_\_\_\_\_\_\_\_\_\_\_\_\_\_\_\_\_\_\_\_\_\_\_\_\_\_\_\_\_\_\_\_\_\_\_\_\_\_\_\_\_\_\_\_\_\_\_\_\_\_\_\_\_\_\_\_\_\_\_\_\_\_\_\_\_\_\_\_\_\_\_\_\_\_\_\_\_\_\_\_\_\_\_\_\_\_\_\_\_\_\_\_\_\_\_\_\_\_\_\_\_\_\_\_\_\_\_\_\_\_\_\_\_\_\_\_\_\_\_\_\_\_\_\_\_\_\_\_\_\_\_\_\_\_\_\_\_\_\_\_\_\_\_\_\_\_\_\_\_\_\_\_\_\_\_\_\_\_\_\_\_\_\_\_\_\_\_\_\_\_\_\_\_\_\_\_\_\_\_\_\_\_\_\_\_\_\_\_\_\_\_\_\_\_\_\_\_\_\_\_\_\_\_\_\_\_\_\_\_\_\_\_\_\_\_\_\_\_\_\_\_\_\_\_\_\_\_\_\_\_\_\_\_\_\_\_\_\_\_\_\_\_\_\_\_\_\_\_\_\_\_\_\_\_\_\_\_\_\_\_\_\_\_\_\_\_\_\_\_\_\_\_\_\_\_\_\_\_\_\_\_\_\_\_\_\_\_\_\_\_\_\_\_\_\_\_\_\_\_\_\_\_\_\_\_\_\_\_\_\_\_\_\_\_\_\_\_\_\_\_\_\_\_\_\_\_\_\_\_\_\_\_\_\_\_\_\_\_\_\_\_\_\_\_\_\_\_\_\_\_\_\_\_\_\_\_\_\_\_\_\_\_\_\_\_\_\_\_\_\_\_\_\_\_\_\_\_\_\_\_\_\_\_\_\_\_\_\_\_\_\_\_\_\_\_\_\_\_\_\_\_\_\_\_\_\_\_\_\_\_\_\_\_\_\_\_\_\_\_\_\_\_\_\_\_\_\_\_\_\_\_\_\_\_\_\_\_\_\_\_\_\_\_\_\_\_\_\_\_\_\_\_\_\_\_\_\_\_\_\_\_\_\_\_\_\_\_\_\_\_\_\_\_\_\_\_\_\_\_\_\_\_\_\_\_\_\_\_\_\_\_\_\_\_\_\_\_\_\_\_\_\_\_\_\_\_\_\_\_\_\_\_\_\_\_\_\_\_\_\_\_\_\_\_\_\_\_\_\_\_\_\_\_\_\_\_\_\_\_\_\_\_\_\_\_\_\_\_\_\_\_\_\_\_\_\_\_\_\_\_\_\_\_\_\_\_\_\_\_\_\_\_\_\_\_\_\_\_\_\_\_\_\_\_\_\_\_\_\_\_\_\_\_\_\_\_\_\_\_\_\_\_\_\_\_\_\_\_\_\_\_\_\_\_\_\_\_\_\_\_\_\_\_\_\_\_\_\_\_\_\_\_\_\_\_\_\_\_\_\_\_\_\_\_\_\_\_\_\_\_\_\_\_\_\_\_\_\_\_\_\_\_\_\_\_\_\_\_\_\_\_\_\_\_\_\_\_\_\_\_\_\_\_\_\_\_\_\_\_\_\_\_\_\_\_\_\_\_\_\_\_\_\_\_\_\_\_\_\_\_\_\_\_\_\_\_\_\_\_\_\_\_\_\_\_\_\_\_\_\_\_\_\_\_\_\_\_\_\_\_\_\_\_\_\_\_\_\_\_\_\_\_\_\_\_\_\_\_\_\_\_\_\_\_\_\_\_\_\_\_\_\_\_\_\_\_\_\_\_\_\_\_\_\_\_\_\_\_\_\_\_\_\_\_\_\_\_\_\_\_\_\_\_\_\_\_\_\_\_\_\_\_\_\_\_\_\_\_\_\_\_\_\_\_\_\_\_\_\_\_\_\_\_\_\_\_\_\_\_\_\_\_\_\_\_\_\_\_\_\_\_\_\_\_\_\_\_\_\_\_\_\_\_\_\_\_\_\_\_\_\_\_\_\_\_\_\_\_\_\_\_\_\_\_\_\_\_\_\_\_\_\_\_\_\_\_\_\_\_\_\_\_\_\_\_\_\_\_\_\_\_\_\_\_\_\_\_\_\_\_\_\_\_\_\_\_\_\_\_\_\_\_\_\_\_\_\_\_\_\_\_\_\_\_\_\_\_\_\_\_\_\_\_\_\_\_\_\_\_\_\_\_\_\_\_\_\_\_\_\_\_\_\_\_\_\_\_\_\_\_\_\_\_\_\_\_\_\_\_\_\_\_\_\_\_\_\_\_\_\_\_\_\_\_\_\_\_\_\_\_\_\_\_\_\_\_\_\_\_\_\_\_\_\_\_\_\_\_\_\_\_\_\_\_\_\_\_\_\_\_\_\_\_\_\_\_\_\_\_\_\_\_\_\_\_\_\_\_\_\_\_\_\_\_\_\_\_\_\_\_\_\_\_\_\_\_\_\_\_\_\_\_\_\_\_\_\_\_\_\_\_\_\_\_\_\_\_\_\_\_\_\_\_\_\_\_\_\_\_\_\_\_\_\_\_\_\_\_\_\_\_\_\_\_\_\_\_\_\_\_\_\_\_\_\_\_\_\_\_\_\_\_\_\_\_\_\_\_\_\_\_\_\_\_\_\_\_\_\_\_\_\_\_\_\_\_\_\_\_\_\_\_\_\_\_\_\_\_\_\_\_\_\_\_\_\_\_\_\_\_\_\_\_\_\_\_\_\_\_\_\_\_\_\_\_\_\_\_\_\_\_\_\_\_\_\_\_\_\_\_\_\_\_\_\_\_\_\_\_\_\_\_\_\_\_\_\_\_\_\_\_\_\_\_\_\_\_\_\_\_\_\_\_\_\_\_\_\_\_\_\_\_\_\_\_\_\_\_\_\_\_\_\_\_\_\_\_\_\_\_\_\_\_\_\_\_\_\_\_\_\_\_\_\_\_\_\_\_\_\_\_\_\_\_\_\_\_\_\_\_\_\_\_\_\_\_\_\_\_\_\_\_\_\_\_\_\_\_\_\_\_\_\_\_\_\_\_\_\_\_\_\_\_\_\_\_\_\_\_\_\_\_\_\_\_\_\_\_\_\_\_\_\_\_\_\_\_\_\_\_\_\_\_\_\_\_\_\_\_\_\_\_\_\_\_\_\_\_\_\_\_\_\_\_\_\_\_\_\_\_\_\_\_\_\_\_\_\_\_\_\_\_\_\_\_\_\_\_\_\_\_\_\_\_\_\_\_\_\_\_\_\_\_\_\_\_\_\_\_\_\_\_\_\_\_\_\_\_\_\_\_\_\_\_\_\_\_\_\_\_\_\_\_\_\_\_\_\_\_\_\_\_\_\_\_\_\_\_\_\_\_\_\_\_\_\_\_\_\_\_\_\_\_\_\_\_\_\_\_\_\_\_\_\_\_\_\_\_\_\_\_\_\_\_\_\_\_\_\_\_\_\_\_\_\_\_\_\_\_\_\_\_\_\_\_\_\_\_\_\_\_\_\_\_\_\_\_\_\_\_\_\_\_\_\_\_\_\_\_\_\_\_\_\_\_\_\_\_\_\_\_\_\_\_\_\_\_\_\_\_\_\_\_\_\_\_\_\_\_\_\_\_\_\_\_\_\_\_\_\_\_\_\_\_\_\_\_\_\_\_\_\_\_\_\_\_\_\_\_\_\_\_\_\_\_\_\_\_\_\_\_\_\_\_\_\_\_\_\_\_\_\_\_\_\_\_\_\_\_\_\_\_\_\_\_\_\_\_\_\_\_\_\_\_\_\_\_\_\_\_\_\_\_\_\_\_\_\_\_\_\_\_\_\_\_\_\_\_\_\_\_\_\_\_\_\_\_\_\_\_\_\_\_\_\_\_\_\_\_\_\_\_\_\_\_\_\_\_\_\_\_\_\_\_\_\_\_\_\_\_\_\_\_\_\_\_\_\_\_\_\_\_\_\_\_\_\_\_\_\_\_\_\_\_\_\_\_\_\_\_\_\_\_\_\_\_\_\_\_\_\_\_\_\_\_\_\_\_\_\_\_\_\_\_\_\_\_\_\_\_\_\_\_\_\_\_\_\_\_\_\_\_\_\_\_\_\_\_\_\_\_\_\_\_\_\_\_\_\_\_\_\_\_\_\_\_\_\_\_\_\_\_\_\_\_\_\_\_\_\_\_\_\_\_\_\_\_\_\_\_\_\_\_\_\_\_\_\_\_\_\_\_\_\_\_\_\_\_\_\_\_\_\_\_\_\_\_\_\_\_\_\_\_\_\_\_\_\_\_\_\_\_\_\_\_\_\_\_\_\_\_\_\_\_\_\_\_\_\_\_\_\_\_\_\_\_\_\_\_\_\_\_\_\_\_\_\_\_\_\_\_\_\_\_\_\_\_\_\_\_\_\_\_\_\_\_\_\_\_\_\_\_\_\_\_\_\_\_\_\_\_\_\_\_\_\_\_\_\_\_\_\_\_\_\_\_\_\_\_\_\_\_\_\_\_\_\_\_\_\_\_\_\_\_\_\_\_\_\_\_\_\_\_\_\_\_\_\_\_\_\_\_\_\_\_\_\_\_\_\_\_\_\_\_\_\_\_\_\_\_\_\_\_\_\_\_\_\_\_\_\_\_\_\_\_\_\_\_\_\_\_\_\_\_\_\_\_\_\_\_\_\_\_\_\_\_\_\_\_\_\_\_\_\_\_\_\_\_\_\_\_\_\_\_\_\_\_\_\_\_\_\_\_\_\_\_\_\_\_\_\_\_\_\_\_\_\_\_\_\_\_\_\_\_\_\_\_\_\_\_\_\_\_\_\_\_\_\_\_\_\_\_\_\_\_\_\_\_\_\_\_\_\_\_\_\_\_\_\_\_\_\_\_\_\_\_\_\_\_\_\_\_\_\_\_\_\_\_\_\_\_\_\_\_\_\_\_\_\_\_\_\_\_\_\_\_\_\_\_\_\_\_\_\_\_\_\_\_\_\_\_\_\_\_\_\_\_\_\_\_\_\_\_\_\_\_\_\_\_\_\_\_\_\_\_\_\_\_\_\_\_\_\_\_\_\_\_\_\_\_\_\_\_\_\_\_\_\_\_\_\_\_\_\_\_\_\_\_\_\_\_\_\_\_\_\_\_\_\_\_\_\_\_\_\_\_\_\_\_\_\_\_\_\_\_\_\_\_\_\_\_\_\_\_\_\_\_\_\_\_\_\_\_\_\_\_\_\_\_\_\_\_\_\_\_\_\_\_\_\_\_\_\_\_\_\_\_\_\_\_\_\_\_\_\_\_\_\_\_\_\_\_\_\_\_\_\_\_\_\_\_\_\_\_\_\_\_\_\_\_\_\_\_\_\_\_\_\_\_\_\_\_\_\_\_\_\_\_\_\_\_\_\_\_\_\_\_\_\_\_\_\_\_\_\_\_\_\_\_\_\_\_\_\_\_\_\_\_\_\_\_\_\_\_\_\_\_\_\_\_\_\_\_\_\_\_\_\_\_\_\_\_\_\_\_\_\_\_\_\_\_\_\_\_\_\_\_\_\_\_\_\_\_\_\_\_\_\_\_\_\_\_\_\_\_\_\_\_\_\_\_\_\_\_\_\_\_\_\_\_\_\_\_\_\_\_\_\_\_\_\_\_\_\_\_\_\_\_\_\_\_\_\_\_\_\_\_\_\_\_\_\_\_\_\_\_\_\_\_\_\_\_\_\_\_\_\_\_\_\_\_\_\_\_\_\_\_\_\_\_\_\_\_\_\_\_\_\_\_\_\_\_\_\_\_\_\_\_\_\_\_\_\_\_\_\_\_\_\_\_\_\_\_\_\_\_\_\_\_\_\_\_\_\_\_\_\_\_\_\_\_\_\_\_\_\_\_\_\_\_\_\_\_\_\_\_\_\_\_\_\_\_\_\_\_\_\_\_\_\_\_\_\_\_\_\_\_\_\_\_\_\_\_\_\_\_\_\_\_\_\_\_\_\_\_\_\_\_\_\_\_\_\_\_\_\_\_\_\_\_\_\_\_\_\_\_\_\_\_\_\_\_\_\_\_\_\_\_\_\_\_\_\_\_\_\_\_\_\_\_\_\_\_\_\_\_\_\_\_\_\_\_\_\_\_\_\_\_\_\_\_\_\_\_\_\_\_\_\_\_\_\_\_\_\_\_\_\_\_\_\_\_\_\_\_\_\_\_\_\_\_\_\_\_\_\_\_\_\_\_\_\_\_\_\_\_\_\_\_\_\_\_\_\_\_\_\_\_\_\_\_\_\_\_\_\_\_\_\_\_\_\_\_\_\_\_\_\_\_\_\_\_\_\_\_\_\_\_\_\_\_\_\_\_\_\_\_\_\_\_\_\_\_\_\_\_\_\_\_\_\_\_\_\_\_\_\_\_\_\_\_\_\_\_\_\_\_\_\_\_\_\_\_\_\_\_\_\_\_\_\_\_\_\_\_\_\_\_\_\_\_\_\_\_\_\_\_\_\_\_\_\_\_\_\_\_\_\_\_\_\_\_\_\_\_\_\_\_\_\_\_\_\_\_\_\_\_\_\_\_\_\_\_\_\_\_\_\_\_\_\_\_\_\_\_\_\_\_\_\_\_\_\_\_\_\_\_\_\_\_\_\_\_\_\_\_\_\_\_\_\_\_\_\_\_\_\_\_\_\_\_\_\_\_\_\_\_\_\_\_\_\_\_\_\_\_\_\_\_\_\_\_\_\_\_\_\_\_\_\_\_\_\_\_\_\_\_\_\_\_\_\_\_\_\_\_\_\_\_\_\_\_\_\_\_\_\_\_\_\_\_\_\_\_\_\_\_\_\_\_\_\_\_\_\_\_\_\_\_\_\_\_\_\_\_\_\_\_\_\_\_\_\_\_\_\_\_\_\_\_\_\_\_\_\_\_\_\_\_\_\_\_\_\_\_\_\_\_\_\_\_\_\_\_\_\_\_\_\_\_\_\_\_\_\_\_\_\_\_\_\_\_\_\_\_\_\_\_\_\_\_\_\_\_\_\_\_\_\_\_\_\_\_\_\_\_\_\_\_\_\_\_\_\_\_\_\_\_\_\_\_\_\_\_\_\_\_\_\_\_\_\_\_\_\_\_\_\_\_\_\_\_\_\_\_\_\_\_\_\_\_\_\_\_\_\_\_\_\_\_\_\_\_\_\_\_\_\_\_\_\_\_\_\_\_\_\_\_\_\_\_\_\_\_\_\_\_\_\_\_\_\_\_\_\_\_\_\_\_\_\_\_\_\_\_\_\_\_\_\_\_\_\_\_\_\_\_\_\_\_\_\_\_\_\_\_\_\_\_\_\_\_\_\_\_\_\_\_\_\_\_\_\_\_\_\_\_\_\_\_\_\_\_\_\_\_\_\_\_\_\_\_\_\_\_\_\_\_\_\_\_\_\_\_\_\_\_\_\_\_\_\_\_\_\_\_\_\_\_\_\_\_\_\_\_\_\_\_\_\_\_\_\_\_\_\_\_\_\_\_\_\_\_\_\_\_\_\_\_\_\_\_\_\_\_\_\_\_\_\_\_\_\_\_\_\_\_\_\_\_\_\_\_\_\_\_\_\_\_\_\_\_\_\_\_\_\_\_\_\_\_\_\_\_\_\_\_\_\_\_\_\_\_\_\_\_\_\_\_\_\_\_\_\_\_\_\_\_\_\_\_\_\_\_\_\_\_\_\_\_\_\_\_\_\_\_\_\_\_\_\_\_\_\_\_\_\_\_\_\_\_\_\_\_\_\_\_\_\_\_\_\_\_\_\_\_\_\_\_\_\_\_\_\_\_\_\_\_\_\_\_\_\_\_\_\_\_\_\_\_\_\_\_\_\_\_\_\_\_\_\_\_\_\_\_\_\_\_\_\_\_\_\_\_\_\_\_\_\_\_\_\_\_\_\_\_\_\_\_\_\_\_\_\_\_\_\_\_\_\_\_\_\_\_\_\_\_\_\_\_\_\_\_\_\_\_\_\_\_\_\_\_\_\_\_\_\_\_\_\_\_\_\_\_\_\_\_\_\_\_\_\_\_\_\_\_\_\_\_\_\_\_\_\_\_\_\_\_\_\_\_\_\_\_\_\_\_\_\_\_\_\_\_\_\_\_\_\_\_\_\_\_\_\_\_\_\_\_\_\_\_\_\_\_\_\_\_\_\_\_\_\_\_\_\_\_\_\_\_\_\_\_\_\_\_\_\_\_\_\_\_\_\_\_\_\_\_\_\_\_\_\_\_\_\_\_\_\_\_\_\_\_\_\_\_\_\_\_\_\_\_\_\_\_\_\_\_\_\_\_\_\_\_\_\_\_\_\_\_\_\_\_\_\_\_\_\_\_\_\_\_\_\_\_\_\_\_\_\_\_\_\_\_\_\_\_\_\_\_\_\_\_\_\_\_\_\_\_\_\_\_\_\_\_\_\_\_\_\_\_\_\_\_\_\_\_\_\_\_\_\_\_\_\_\_\_\_\_\_\_\_\_\_\_\_\_\_\_\_\_\_\_\_\_\_\_\_\_\_\_\_\_\_\_\_\_\_\_\_\_\_\_\_\_\_\_\_\_\_\_\_\_\_\_\_\_\_\_\_\_\_\_\_\_\_\_\_\_\_\_\_\_\_\_\_\_\_\_\_\_\_\_\_\_\_\_\_\_\_\_\_\_\_\_\_\_\_\_\_\_\_\_\_\_\_\_\_\_\_\_\_\_\_\_\_\_\_\_\_\_\_\_\_\_\_\_\_\_\_\_\_\_\_\_\_\_\_\_\_\_\_\_\_\_\_\_\_\_\_\_\_\_\_\_\_\_\_\_\_\_\_\_\_\_\_\_\_\_\_\_\_\_\_\_\_\_\_\_\_\_\_\_\_\_\_\_\_\_\_\_\_\_\_\_\_\_\_\_\_\_\_\_\_\_\_\_\_\_\_\_\_\_\_\_\_\_\_\_\_\_\_\_\_\_\_\_\_\_\_\_\_\_\_\_\_\_\_\_\_\_\_\_\_\_\_\_\_\_\_\_\_\_\_\_\_\_\_\_\_\_\_\_\_\_\_\_\_\_\_\_\_\_\_\_\_\_\_\_\_\_\_\_\_\_\_\_\_\_\_\_\_\_\_\_\_\_\_\_\_\_\_\_\_\_\_\_\_\_\_\_\_\_\_\_\_\_\_\_\_\_\_\_\_\_\_\_\_\_\_\_\_\_\_\_\_\_\_\_\_\_\_\_\_\_\_\_\_\_\_\_\_\_\_\_\_\_\_\_\_\_\_\_\_\_\_\_\_\_\_\_\_\_\_\_\_\_\_\_\_\_\_\_\_\_\_\_\_\_\_\_\_\_\_\_\_\_\_\_\_\_\_\_\_\_\_\_\_\_\_\_\_\_\_\_\_\_\_\_\_\_\_\_\_\_\_\_\_\_\_\_\_\_\_\_\_\_\_\_\_\_\_\_\_\_\_\_\_\_\_\_\_\_\_\_\_\_\_\_\_\_\_\_\_\_\_\_\_\_\_\_\_\_\_\_\_\_\_\_\_\_\_\_\_\_\_\_\_\_\_\_\_\_\_\_\_\_\_\_\_\_\_\_\_\_\_\_\_\_\_\_\_\_\_\_\_\_\_\_\_\_\_\_\_\_\_\_\_\_\_\_\_\_\_\_\_\_\_\_\_\_\_\_\_\_\_\_\_\_\_\_\_\_\_\_\_\_\_\_\_\_\_\_\_\_\_\_\_\_\_\_\_\_\_\_\_\_\_\_\_\_\_\_\_\_\_\_\_\_\_\_\_\_\_\_\_\_\_\_\_\_\_\_\_\_\_\_\_\_\_\_\_\_\_\_\_\_\_\_\_\_\_\_\_\_\_\_\_\_\_\_\_\_\_\_\_\_\_\_\_\_\_\_\_\_\_\_\_\_\_\_\_\_\_\_\_\_\_\_\_\_\_\_\_\_\_\_\_\_\_\_\_\_\_\_\_\_\_\_\_\_\_\_\_\_\_\_\_\_\_\_\_\_\_\_\_\_\_\_\_\_\_\_\_\_\_\_\_\_\_\_\_\_\_\_\_\_\_\_\_\_\_\_\_\_\_\_\_\_\_\_\_\_\_\_\_\_\_\_\_\_\_\_\_\_\_\_\_\_\_\_\_\_\_\_\_\_\_\_\_\_\_\_\_\_\_\_\_\_\_\_\_\_\_\_\_\_\_\_\_\_\_\_\_\_\_\_\_\_\_\_\_\_\_\_\_\_\_\_\_\_\_\_\_\_\_\_\_\_\_\_\_\_\_\_\_\_\_\_\_\_\_\_\_\_\_\_\_\_\_\_\_\_\_\_\_\_\_\_\_\_\_\_\_\_\_\_\_\_\_\_\_\_\_\_\_\_\_\_\_\_\_\_\_\_\_\_\_\_\_\_\_\_\_\_\_\_\_\_\_\_\_\_\_\_\_\_\_\_\_\_\_\_\_\_\_\_\_\_\_\_\_\_\_\_\_\_\_\_\_\_\_\_\_\_\_\_\_\_\_\_\_\_\_\_\_\_\_\_\_\_\_\_\_\_\_\_\_\_\_\_\_\_\_\_\_\_\_\_\_\_\_\_\_\_\_\_\_\_\_\_\_\_\_\_\_\_\_\_\_\_\_\_\_\_\_\_\_\_\_\_\_\_\_\_\_\_\_\_\_\_\_\_\_\_\_\_\_\_\_\_\_\_\_\_\_\_\_\_\_\_\_\_\_\_\_\_\_\_\_\_\_\_\_\_\_\_\_\_\_\_\_\_\_\_\_\_\_\_\_\_\_\_\_\_\_\_\_\_\_\_\_\_\_\_\_\_\_\_\_\_\_\_\_\_\_\_\_\_\_\_\_\_\_\_\_\_\_\_\_\_\_\_\_\_\_\_\_\_\_\_\_\_\_\_\_\_\_\_\_\_\_\_\_\_\_\_\_\_\_\_\_\_\_\_\_\_\_\_\_\_\_\_\_\_\_\_\_\_\_\_\_\_\_\_\_\_\_\_\_\_\_\_\_\_\_\_\_\_\_\_\_\_\_\_\_\_\_\_\_\_\_\_\_\_\_\_\_\_\_\_\_\_\_\_\_\_\_\_\_\_\_\_\_\_\_\_\_\_\_\_\_\_\_\_\_\_\_\_\_\_\_\_\_\_\_\_\_\_\_\_\_\_\_\_\_\_\_\_\_\_\_\_\_\_\_\_\_\_\_\_\_\_\_\_\_\_\_\_\_\_\_\_\_\_\_\_\_\_\_\_\_\_\_\_\_\_\_\_\_\_\_\_\_\_\_\_\_\_\_\_\_\_\_\_\_\_\_\_\_\_\_\_\_\_\_\_\_\_\_\_\_\_\_\_\_\_\_\_\_\_\_\_\_\_\_\_\_\_\_\_\_\_\_\_\_\_\_\_\_\_\_\_\_\_\_\_\_\_\_\_\_\_\_\_\_\_\_\_\_\_\_\_\_\_\_\_\_\_\_\_\_\_\_\_\_\_\_\_\_\_\_\_\_\_\_\_\_\_\_\_\_\_\_\_\_\_\_\_\_\_\_\_\_\_\_\_\_\_\_\_\_\_\_\_\_\_\_\_\_\_\_\_\_\_\_\_\_\_\_\_\_\_\_\_\_\_\_\_\_\_\_\_\_\_\_\_\_\_\_\_\_\_\_\_\_\_\_\_\_\_\_\_\_\_\_\_\_\_\_\_\_\_\_\_\_\_\_\_\_\_\_\_\_\_\_\_\_\_\_\_\_\_\_\_\_\_\_\_\_\_\_\_\_\_\_\_\_\_\_\_\_\_\_\_\_\_\_\_\_\_\_\_\_\_\_\_\_\_\_\_\_\_\_\_\_\_\_\_\_\_\_\_\_\_\_\_\_\_\_\_\_\_\_\_\_\_\_\_\_\_\_\_\_\_\_\_\_\_\_\_\_\_\_\_\_\_\_\_\_\_\_\_\_\_\_\_\_\_\_\_\_\_\_\_\_\_\_\_\_\_\_\_\_\_\_\_\_\_\_\_\_\_\_\_\_\_\_\_\_\_\_\_\_\_\_\_\_\_\_\_\_\_\_\_\_\_\_\_\_\_\_\_\_\_\_\_\_\_\_\_\_\_\_\_\_\_\_\_\_\_\_\_\_\_\_\_\_\_\_\_\_\_\_\_\_\_\_\_\_\_\_\_\_\_\_\_\_\_\_\_\_\_\_\_\_\_\_\_\_\_\_\_\_\_\_\_\_\_\_\_\_\_\_\_\_\_\_\_\_\_\_\_\_\_\_\_\_\_\_\_\_\_\_\_\_\_\_\_\_\_\_\_\_\_\_\_\_\_\_\_\_\_\_\_\_\_\_\_\_\_\_\_\_\_\_\_\_\_\_\_\_\_\_\_\_\_\_\_\_\_\_\_\_\_\_\_\_\_\_\_\_\_\_\_\_\_\_\_\_\_\_\_\_\_\_\_\_\_\_\_\_\_\_\_\_\_\_\_\_\_\_\_\_\_\_\_\_\_\_\_\_\_\_\_\_\_\_\_\_\_\_\_\_\_\_\_\_\_\_\_\_\_\_\_\_\_\_\_\_\_\_\_\_\_\_\_\_\_\_\_\_\_\_\_\_\_\_\_\_\_\_\_\_\_\_\_\_\_\_\_\_\_\_\_\_\_\_\_\_\_\_\_\_\_\_\_\_\_\_\_\_\_\_\_\_\_\_\_\_\_\_\_\_\_\_\_\_\_\_\_\_\_\_\_\_\_\_\_\_\_\_\_\_\_\_\_\_\_\_\_\_\_\_\_\_\_\_\_\_\_\_\_\_\_\_\_\_\_\_\_\_\_\_\_\_\_\_\_\_\_\_\_\_\_\_\_\_\_\_\_\_\_\_\_\_\_\_\_\_\_\_\_\_\_\_\_\_\_\_\_\_\_\_\_\_\_\_\_\_\_\_\_\_\_\_\_\_\_\_\_\_\_\_\_\_\_\_\_\_\_\_\_\_\_\_\_\_\_\_\_\_\_\_\_\_\_\_\_\_\_\_\_\_\_\_\_\_\_\_\_\_\_\_\_\_\_\_\_\_\_\_\_\_\_\_\_\_\_\_\_\_\_\_\_\_\_\_\_\_\_\_\_\_\_\_\_\_\_\_\_\_\_\_\_\_\_\_\_\_\_\_\_\_\_\_\_\_\_\_\_\_\_\_\_\_\_\_\_\_\_\_\_\_\_\_\_\_\_\_\_\_\_\_\_\_\_\_\_\_\_\_\_\_\_\_\_\_\_\_\_\_\_\_\_\_\_\_\_\_\_\_\_\_\_\_\_\_\_\_\_\_\_\_\_\_\_\_\_\_\_\_\_\_\_\_\_\_\_\_\_\_\_\_\_\_\_\_\_\_\_\_\_\_\_\_\_\_\_\_\_\_\_\_\_\_\_\_\_\_\_\_\_\_\_\_\_\_\_\_\_\_\_\_\_\_\_\_\_\_\_\_\_\_\_\_\_\_\_\_\_\_\_\_\_\_\_\_\_\_\_\_\_\_\_\_\_\_\_\_\_\_\_\_\_\_\_\_\_\_\_\_\_\_\_\_\_\_\_\_\_\_\_\_\_\_\_\_\_\_\_\_\_\_\_\_\_\_\_\_\_\_\_\_\_\_\_\_\_\_\_\_\_\_\_\_\_\_\_\_\_\_\_\_\_\_\_\_\_\_\_\_\_\_\_\_\_\_\_\_\_\_\_\_\_\_\_\_\_\_\_\_\_\_\_\_\_\_\_\_\_\_\_\_\_\_\_\_\_\_\_\_\_\_\_\_\_\_\_\_\_\_\_\_\_\_\_\_\_\_\_\_\_\_\_\_\_\_\_\_\_\_\_\_\_\_\_\_\_\_\_\_\_\_\_\_\_\_\_\_\_\_\_\_\_\_\_\_\_\_\_\_\_\_\_\_\_\_\_\_\_\_\_\_\_\_\_\_\_\_\_\_\_\_\_\_\_\_\_\_\_\_\_\_\_\_\_\_\_\_\_\_\_\_\_\_\_\_\_\_\_\_\_\_\_\_\_\_\_\_\_\_\_\_\_\_\_\_\_\_\_\_\_\_\_\_\_\_\_\_\_\_\_\_\_\_\_\_\_\_\_\_\_\_\_\_\_\_\_\_\_\_\_\_\_\_\_\_\_\_\_\_\_\_\_\_\_\_\_\_\_\_\_\_\_\_\_\_\_\_\_\_\_\_\_\_\_\_\_\_\_\_\_\_\_\_\_\_\_\_\_\_\_\_\_\_\_\_\_\_\_\_\_\_\_\_\_\_\_\_\_\_\_\_\_\_\_\_\_\_\_\_\_\_\_\_\_\_\_\_\_\_\_\_\_\_\_\_\_\_\_\_\_\_\_\_\_\_\_\_\_\_\_\_\_\_\_\_\_\_\_\_\_\_\_\_\_\_\_\_\_\_\_\_\_\_\_\_\_\_\_\_\_\_\_\_\_\_\_\_\_\_\_\_\_\_\_\_\_\_\_\_\_\_\_\_\_\_\_\_\_\_\_\_\_\_\_\_\_\_\_\_\_\_\_\_\_\_\_\_\_\_\_\_\_\_\_\_\_\_\_\_\_\_\_\_\_\_\_\_\_\_\_\_\_\_\_\_\_\_\_\_\_\_\_\_\_\_\_\_\_\_\_\_\_\_\_\_\_\_\_\_\_\_\_\_\_\_\_\_\_\_\_\_\_\_\_\_\_\_\_\_\_\_\_\_\_\_\_\_\_\_\_\_\_\_\_\_\_\_\_\_\_\_\_\_\_\_\_\_\_\_\_\_\_\_\_\_\_\_\_\_\_\_\_\_\_\_\_\_\_\_\_\_\_\_\_\_\_\_\_\_\_\_\_\_\_\_\_\_\_\_\_\_\_\_\_\_\_\_\_\_\_\_\_\_\_\_\_\_\_\_\_\_\_\_\_\_\_\_\_\_\_\_\_\_\_\_\_\_\_\_\_\_\_\_\_\_\_\_\_\_\_\_\_\_\_\_\_\_\_\_\_\_\_\_\_\_\_\_\_\_\_\_\_\_\_\_\_\_\_\_\_\_\_\_\_\_\_\_\_\_\_\_\_\_\_\_\_\_\_\_\_\_\_\_\_\_\_\_\_\_\_\_\_\_\_\_\_\_\_\_\_\_\_\_\_\_\_\_\_\_\_\_\_\_\_\_\_\_\_\_\_\_\_\_\_\_\_\_\_\_\_\_\_\_\_\_\_\_\_\_\_\_\_\_\_\_\_\_\_\_\_\_\_\_\_\_\_\_\_\_\_\_\_\_\_\_\_\_\_\_\_\_\_\_\_\_\_\_\_\_\_\_\_\_\_\_\_\_\_\_\_\_\_\_\_\_\_\_\_\_\_\_\_\_\_\_\_\_\_\_\_\_\_\_\_\_\_\_\_\_\_\_\_\_\_\_\_\_\_\_\_\_\_\_\_\_\_\_\_\_\_\_\_\_\_\_\_\_\_\_\_\_\_\_\_\_\_\_\_\_\_\_\_\_\_\_\_\_\_\_\_\_\_\_\_\_\_\_\_\_\_\_\_\_\_\_\_\_\_\_\_\_\_\_\_\_\_\_\_\_\_\_\_\_\_\_\_\_\_\_\_\_\_\_\_\_\_\_\_\_\_\_\_\_\_\_\_\_\_\_\_\_\_\_\_\_\_\_\_\_\_\_\_\_\_\_\_\_\_\_\_\_\_\_\_\_\_\_\_\_\_\_\_\_\_\_\_\_\_\_\_\_\_\_\_\_\_\_\_\_\_\_\_\_\_\_\_\_\_\_\_\_\_\_\_\_\_\_\_\_\_\_\_\_\_\_\_\_\_\_\_\_\_\_\_\_\_\_\_\_\_\_\_\_\_\_\_\_\_\_\_\_\_\_\_\_\_\_\_\_\_\_\_\_\_\_\_\_\_\_\_\_\_\_\_\_\_\_\_\_\_\_\_\_\_\_\_\_\_\_\_\_\_\_\_\_\_\_\_\_\_\_\_\_\_\_\_\_\_\_\_\_\_\_\_\_\_\_\_\_\_\_\_\_\_\_\_\_\_\_\_\_\_\_\_\_\_\_\_\_\_\_\_\_\_\_\_\_\_\_\_\_\_\_\_\_\_\_\_\_\_\_\_\_\_\_\_\_\_\_\_\_\_\_\_\_\_\_\_\_\_\_\_\_\_\_\_\_\_\_\_\_\_\_\_\_\_\_\_\_\_\_\_\_\_\_\_\_\_\_\_\_\_\_\_\_\_\_\_\_\_\_\_\_\_\_\_\_\_\_\_\_\_\_\_\_\_\_\_\_\_\_\_\_\_\_\_\_\_\_\_\_\_\_\_\_\_\_\_\_\_\_\_\_\_\_\_\_\_\_\_\_\_\_\_\_\_\_\_\_\_\_\_\_\_\_\_\_\_\_\_\_\_\_\_\_\_\_\_\_\_\_\_\_\_\_\_\_\_\_\_\_\_\_\_\_\_\_\_\_\_\_\_\_\_\_\_\_\_\_\_\_\_\_\_\_\_\_\_\_\_\_\_\_\_\_\_\_\_\_\_\_\_\_\_\_\_\_\_\_\_\_\_\_\_\_\_\_\_\_\_\_\_\_\_\_\_\_\_\_\_\_\_\_\_\_\_\_\_\_\_\_\_\_\_\_\_\_\_\_\_\_\_\_\_\_\_\_\_\_\_\_\_\_\_\_\_\_\_\_\_\_\_\_\_\_\_\_\_\_\_\_\_\_\_\_\_\_\_\_\_\_\_\_\_\_\_\_\_\_\_\_\_\_\_\_\_\_\_\_\_\_\_\_\_\_\_\_\_\_\_\_\_\_\_\_\_\_\_\_\_\_\_\_\_\_\_\_\_\_\_\_\_\_\_\_\_\_\_\_\_\_\_\_\_\_\_\_\_\_\_\_\_\_\_\_\_\_\_\_\_\_\_\_\_\_\_\_\_\_\_\_\_\_\_\_\_\_\_\_\_\_\_\_\_\_\_\_\_\_\_\_\_\_\_\_\_\_\_\_\_\_\_\_\_\_\_\_\_\_\_\_\_\_\_\_\_\_\_\_\_\_\_\_\_\_\_\_\_\_\_\_\_\_\_\_\_\_\_\_\_\_\_\_\_\_\_\_\_\_\_\_\_\_\_\_\_\_\_\_\_\_\_\_\_\_\_\_\_\_\_\_\_\_\_\_\_\_\_\_\_\_\_\_\_\_\_\_\_\_\_\_\_\_\_\_\_\_\_\_\_\_\_\_\_\_\_\_\_\_\_\_\_\_\_\_\_\_\_\_\_\_\_\_\_\_\_\_\_\_\_\_\_\_\_\_\_\_\_\_\_\_\_\_\_\_\_\_\_\_\_\_\_\_\_\_\_\_\_\_\_\_\_\_\_\_\_\_\_\_\_\_\_\_\_\_\_\_\_\_\_\_\_\_\_\_\_\_\_\_\_\_\_\_\_\_\_\_\_\_\_\_\_\_\_\_\_\_\_\_\_\_\_\_\_\_\_\_\_\_\_\_\_\_\_\_\_\_\_\_\_\_\_\_\_\_\_\_\_\_\_\_\_\_\_\_\_\_\_\_\_\_\_\_\_\_\_\_\_\_\_\_\_\_\_\_\_\_\_\_\_\_\_\_\_\_\_\_\_\_\_\

\section*{全国第八届研究生数学建模竞赛}

\begin{center}
\includegraphics[width=0.5\textwidth]{image.png}
\end{center}

\section*{题目:对小麦发育后期茎秆抗倒性问题的研究}

\section*{摘要:}

本文是一个研究小麦发育后期茎秆抗倒性的问题。

针对问题一,我们运用了数理统计与分析的相关知识,建立了各品种小麦的抗倒伏指数模型。首先,我们对所需的数据进行了分析与处理;然后,利用 SPSS 软件和已有数据分别对影响小麦抗倒伏指数的三个因素(茎秆鲜重、重心高度和机械强度)进行了数据拟合分析,建立了相应的回归模型,并对模型进行了检验;接着,综合三个因素建立了一个更优化更实用的抗倒伏指数的总模型和三个具体的子模型。在求解过程中,我们因地制宜,用三个具体的子模型依次对 07 年、11 年和 08 年的各品种抗倒伏指数进行了求解,如矮抗 58 的抗倒伏指数为 0.626。

针对问题二,我们首先利用了 07 年的数据对最易引起倒伏的时期进行了判断与确定,得出最易引起倒伏期为蜡熟期;接着,分别对抗倒伏指数与各节间长、节间长度比、各节壁厚及各因素间的关系,建立了相关性系数模型并利用了 SPSS 软件对其相关性进行了分析,判断出了小麦茎秆性状的各个因素之间所存在的相关性,如株高与抗倒伏指数间呈极显著正相关;最后,对 08 年国信 1 号与智 9998 品种的小麦发生倒伏原因进行了判断与分析,得出其穗下第四节长比较大、株高比较大、穗下第一节壁厚比较小、穗重、鲜重比较大等因素是其小麦发生倒伏的主要原因。

针对问题三,我们建立了以抗倒伏能力最强和经济系数最大为目标的双目标优化模型。在求解过程中,利用了序贯式算法将双目标转化为了单目标;然后利用了 Lingo 9.0 版软件进行了编程求解。得出了不同产量下,小麦抗倒伏能力最佳的茎秆性状的定量标准。

针对问题四,我们结合力学知识,建立了小麦茎秆抗倒伏性能的综合评价模型。首先,综合考虑了麦穗自重和风载的作用,确定了其评价指标;然后,对数据指标进行了归一化处理,并运用了层次分析法(AHP)结合 MATLAB 软件确定了各评价指标的权重,且对其一致性进行了检验;最后建立了小麦抗倒伏性能的综合评价模型。

针对问题五,我们首先建立了风速与风载作用力间的关系;接着,建立了风速与小麦抗倒伏性能间的关系;然后,结合压杆弯曲的临界力,建立了抗倒伏

风速模型并对其进行了求解,如周 18 抗倒伏风速为 \(15.50 \, \text{m/s}\),为 7 级风。

针对问题六,首先,我们对前面所建的模型进行了总结与结果分析;随后,提出了一些值得深思与考虑的问题;然后,为 2012 年的实验方案及数据分析方法进行了制定;最后,结合我们所做的论文为小麦育种家提出了一些建议。

本文的特色在于,在问题一中,建立了一个更优化更实用的抗倒伏指数的总模型和三个具体的子模型,计算简便,并在建立回归模型后对模型进行了检验,验证了模型的可靠性;在问题二中,我们先利用了 07 年的数据对最易引起倒伏的时期进行了判断与确定;在问题三中,我们引入了经济系数因素,建立了双目标优化模型并利用了序贯式算法来转化为单目标;在问题四中,运用了层次分析法 (AHP) 结合 MATLAB 软件来确定各评价指标的权重,且对其一致性进行了检验。

关键词:回归模型 SPSS 优化模型 Lingo 综合评价模型 层次分析法 MATLAB

\section*{一 问题的提出}

小麦高产、超高产的研究始终是小麦育种家关注的热点问题。随着产量的增加,小麦的单茎穗重不断增加。但穗重的增加同时使茎秆的负荷增大,导致容易倒伏。倒伏不但造成小麦减产,而且影响小麦的籽粒品质。因此要实现小麦高产优质的跨越,就必须解决或尽量减少小麦的倒伏问题。

小麦倒伏从形式上可分为“根倒”和“茎倒”,一般都发生在小麦发育后期。“根倒”主要与小麦种植区域的土壤品种与结构特性有关,本题不做讨论。“茎倒”是高产小麦倒伏的主要形式,尤其是发生时间较早的“茎倒”,往往造成大幅度的减产。“茎倒”的原因是茎秆与穗的自重和风载作用的迭加超过了小麦茎秆的承受能力。

解决倒伏问题的方法之一就是针对不同的产量,寻找小麦抗倒伏能力最佳的茎秆性状(包括株高、茎长、各节间长、各节茎外径、壁厚、茎秆自重、穗长、穗重等)。各方面的专家通过分析影响小麦倒伏的各种因素,目前已经得到了一些结果,但是对抗倒伏能力最佳的茎秆性状还没有定论。

通过物理力学类比研究小麦抗倒伏性是一个新方向,已有一些工作。值得我们进行探讨。困难在于缺乏相关试验参考数据,我们只能在作较多假设下先进行粗略研究,为进一步试验提供根据。

题目的附件中收集了一批各个品种小麦的茎秆性状、产量、倒伏情况的数据。显然还不够完整,各年参数选取不一致,也有数据缺漏。但农业数据一年只有一次,短期内无法做到完整、全面、详尽,期望以后能逐渐完善。请就已有数据解决以下几个问题:

(1) 依据有些论文中判断茎秆抗倒性的抗倒伏指数公式:
\[
\text{茎秆抗倒伏指数} = \text{茎秆鲜重} \times \text{茎秆重心高度} / \text{茎秆机械强度}
\]
对提供的数据,建立各品种小麦的茎秆抗倒指数公式。对于缺乏有关参数的年份,可进行合理的假设,如通过已知数据求茎秆机械强度与茎秆粗厚的关系。

(2) 研究抗倒伏指数与茎秆外部形态特征之间的关系。即给出抗倒伏指数与株高、穗长、各节间长、节间长度比、各节壁厚、穗重、鲜重等茎秆性状在最易引起倒伏期的相关性指标。

判断小麦茎秆性状的各个因素之间是否有相关性?

并对 2008 年国信 1 号与智 9998 品种的小麦都发生倒伏,其他品种没有发生倒伏的原因给出判断。

(3) 探讨单穗重分别是 1.19g,2.06g,2.46g,2.56g,2.75g,2.92g 时小麦的理想株型结构。

(4) 将茎秆按刚/弹性材料处理,研究小麦茎秆在麦穗自重和风载作用下应力的基本规律,引用、修改附件三文献中力学公式或自己另行推导,建立小麦茎秆抗倒伏的数学模型。

(5) 应用(4)力学模型中的抗弯刚度 \( EI \),麦穗自重下和风载作用下的公式对2007年腊熟期各品种数据进行计算,有些参数可依据需要作某些假设。

因腊熟期小麦叶片、叶鞘多已脱落,可设风力仅对单穗穗头起作用,暂时忽略风力对小麦茎秆作用。试计算在2007年数据中腊熟期各品种的抗倒伏风速(取小数后两位)。

(6) 总结所建模型及分析结果,提出值得考虑的问题。同时请你为2012年制定完整的试验方案及数据分析方法。并给小麦育种家在育种实践中提出合理的建议。

\section*{二 模型的假设}

(1) 假设小麦倒伏的主要形式是“茎倒”,不考虑“根倒”形式;

(2) 假设茎秆机械强度与茎秆粗厚存在关联;

(3) 假设2011年数据中小麦穗下第四节拉力与茎秆机械强度等效;

(4) 假设小麦穗下第一至四节长度、平均粗度壁厚能够反映小麦真实情况;

(5) 假设弹性测定值和弹性模量值为线性关系;

(6) 假设小麦穗头迎风面积与穗重有一定关系;

(7) 假设小麦仅在蜡熟期易发生倒伏;

(8) 假设穗位高度等于茎秆的高度。

\section*{三 符号的说明}

\begin{itemize}
    \item \( l_i \):表示小麦茎秆穗下第 \( i \) 节的长度(cm)
    \item \( b_i \):表示小麦茎秆穗下第 \( i \) 节的粗度(mm)
    \item \( d \):表示小麦茎秆穗下第1-4节的平均壁厚(mm)
    \item \( w \):表示小麦茎秆穗鲜重
    \item \( G \):表示茎秆鲜重
    \item \( P \):表示茎秆机械强度
    \item \( h \):表示茎秆重心高度
    \item \( I \):表示茎秆抗倒伏指数
    \item \( a_i \):表示与茎秆鲜重相关的权重系数
\end{itemize}

\begin{itemize}
    \item $a_{i}^{\prime}$: 表示与茎秆机械强度相关的权重系数
    \item $a_{i}^{\prime\prime}$: 表示与茎秆重心高度相关的权重系数
    \item $Y$: 表示小麦茎秆抗倒伏性能;
    \item $E$: 表示弹性模量;
    \item $f$: 为弹性测定值;
    \item $\lambda$: 表示茎秆自重系数;
    \item $F$: 表示风载临界力;
    \item $J$: 表示惯性矩;
    \item $W$: 表示风压;
    \item $\nu$: 表示风速;
    \item $S$: 表示穗头迎风面积;
\end{itemize}

\section*{四、模型的建立与求解}

\subsection{问题一模型的建立与求解}

\subsubsection{问题分析}

第一问要求我们建立各品种小麦的抗倒伏性指数公式,用于判断不同品种小麦茎秆抗倒伏的能力。根据题中提供的抗倒伏指数公式,对于其所需数据齐全的年份和品种,我们可直接对其求解,例如 07 年三个品种所提供的数据比较完整,就可以直接得出该年三个品种的抗倒伏指数公式;然而对于有关参数缺失或未知的年份,我们就得利用已有数据来间接确定其未知的参数,例如 11 年缺失机械强度的相关数据,我们就可设想利用 07 年已知的机械强度数据进行线性拟合,推导出茎秆机械强度的回归模型,然后间接求 11 年各品种的抗倒伏指数公式;对于 08 年,由于 08 年的茎秆鲜重、重心高度和机械强度均未知,则得先利用 11 年的第二组数据拟合出穗干重与穗鲜重的关系模型,再结合 11 年第一组数据拟合得到茎秆鲜重和重心高度的回归模型,然后间接求 08 年各品种的抗倒伏指数公式。

最后,综合以上各模型,建立出小麦茎秆抗倒伏指数的总模型;根据已建立的茎秆抗倒伏指数总模型,在求解时,为了便于计算快捷,可针对不同情况建立其具体的模型与求解,例如可针对三个年份分别建立其具体模型,求解出各品种小麦的茎秆抗倒伏指数。

\subsubsection{模型的建立}

\paragraph{数据的分析与处理}

由于附件中收集的数据并不完整有缺漏,各年参数的选取也不一致,所以在利用数据搭建模型之前,需在附件提供的数据基础上进行分析、筛选与处理。

\begin{enumerate}
    \item[$1^{\circ}$] 对于缺失数据不能全面的反映模型的变化规律,所以在分析时应直接剔除;
    \item[$2^{\circ}$] 对于有错误数据应予以剔除或修正;
    \begin{enumerate}
        \item 数据数值与实际情况不符。由数据说明中给出的定义可知,单茎鲜重 = 穗
    \end{enumerate}
\end{enumerate}

鲜重+穗下各节鲜重。而在 2011 年 5 月 28 日的测量数据中,周麦 22 的单茎鲜重 $(=3.472 \mathrm{~g})<\text { 穗鲜重 } (=5.296 \mathrm{~g})$,显然与实际情况不相符合,应予以剔除。

(2) 数据数值有明显错误。在相同条件下测得的一系列数据中,有极个别数据与同组数据相差极大,不符合同组数据的变化趋势。如下图所示,在 2008 年国信的穗下第一节长度统计数据中,第 10 组的数值与同组数值相差过大,可认为是错误数据。

\begin{figure}[h]
\centering
\includegraphics[width=\textwidth]{image1.png}
\caption{穗下第一节长度数据统计}
\end{figure}

3° 在数据较多的情况下,对于可能引起较大误差的数据也可以剔除;

在 2011 年平安 6 号穗下第三节鲜重的统计数据中,如下图所示,对第一行的数据做分析,发现其值是其余值均值的 3 倍多,若去除该行数据,剩余的数据能更好的反映鲜重的规律。由于数据较多,去除一组误差较大的数据反而能够增强模型构建的准确性,所以将第一行的数据剔除。

\begin{figure}[h]
\centering
\includegraphics[width=\textwidth]{image2.png}
\caption{穗下第三节鲜重值判别}
\end{figure}

(2) 茎秆鲜重的回归模型建立

Step1:数据预处理

假设 $l_{i}$ 表示第 $i(i=1,2,3,4)$ 节的茎秆长度,$l$ 表示茎秆的总长,则有:

\[
l=\sum_{i=1}^{4} l_{i}
\]

对于茎秆的粗和壁厚,为了提高计算的精确性,常采用平均值代入,以 $b_{i}$ 和 $d_{i}$ 表示第 $i(i=1,2,3,4)$ 节的茎秆粗和壁厚,$b$ 和 $d$ 表示茎秆粗平均值和壁厚平均值,则有:

\begin{align*}
b &= \frac{1}{4} \sum_{i=1}^{4} b_i \\
d &= \frac{1}{4} \sum_{i=1}^{4} d_i
\end{align*}

Step2:运用 SPSS 软件对其进行数据拟合分析

设定其多元线性回归模型为:
\[
G = a_0 l + a_1 b + a_2 d + a_3 w + a_4
\]

运用 SPSS 软件对此进行拟合调试与分析,由于测试数据存在不完整、不全面及缺漏现象,这将会影响到整体的拟合优度,例如对全体数据一起拟合,其拟合优度就只有 67.5%,单独对矮抗 58 和周麦 18 号数据拟合,其拟合优度为 92.3%,可见其选择拟合的数据有一定的影响,对此我们对其分别拟合择优。

Step3:比较分析,确定最佳拟合模型

由于上分析可知,拟合数据的选择对拟合优度有着一定的影响,而由附件一的说明我们可知供试品种为矮抗 58,它为主要研究的对象,其他都为它的对照品种,对此我们在数据选择中重点考虑矮抗 58,同时考虑各种品种,分别对其拟合择优。

主要考虑的六种组合为:组合 1:采用矮抗 58 和平安 6 号的数据进行拟合;组合 2:采用矮抗 58 和郑麦 9023 的数据进行拟合;组合 3:采用矮抗 58 和周麦 18 的数据进行拟合;组合 4:采用矮抗 58 和周麦 22 的数据进行拟合的数据进行拟合;组合 5:采用矮抗 58 和豫麦 18 的数据进行拟合;组合 6:利用所有品种的数据进行拟合。

\begin{figure}[h]
\centering
\includegraphics[width=0.8\textwidth]{image.png}
\caption{茎秆鲜重拟合}
\end{figure}

综上,我们选择了拟合优度最佳的回归模型,其具体的模型为:
\[
G = 0.116l - 0.463b + 12.933d + 1.326w - 8.337
\]

Step4:模型检验

(1) 回归方程的拟合优度 \( R^2 \) 的检验:

\begin{table}[h]
\centering
\caption{拟合优度检验}
\begin{tabular}{|c|c|c|c|c|}
\hline
Mode & R & R Square & Adjusted R Square & Std. Error of the Estimate \\
\hline
1 & .922* & .850 & .834 & 3.35620 \\
\hline
\end{tabular}
\end{table}

由于回归方程的拟合优度 \( R^2 \) 越接近 1,拟合效果越好,此题中的拟合优度为 95.8%,可见拟合效果已较好。

\section*{②回归方程的显著性F与回归系数的显著性sig的检验:}

\begin{table}[h]
\centering
\caption{显著性F与sig的检验}
\begin{tabular}{|c|c|c|c|c|c|}
\hline
Model & Sum of Squares & df & Mean Square & F & Sig. \\
\hline
1 & Regression & 334.141 & 4 & 83.535 & 245.050 & .000* \\
 & Residual & 14.658 & 43 & .341 & & \\
 & Total & 348.800 & 47 & & & \\
\hline
\end{tabular}
\end{table}

由于回归方程的显著性F越大越好,回归系数的显著性sig<0.05就越显著,此题中F=245.050比较大,且sig=0.000,可见其显著性相当高。

\section*{③残差检验:}

\begin{figure}[h]
\centering
\includegraphics[width=0.8\textwidth]{residual_plot.png}
\caption{残差检验}
\end{figure}

当残差图均匀分布时效果最好,而此题中残差图分布得较均匀可见拟合得相当好。

\section*{(3) 茎秆重心高度的回归模型建立}

\textbf{Step1:运用SPSS软件对其进行数据拟合分析}

设定其多元线性回归模型为:
\[
h = a_0^{\prime\prime}l + a_1^{\prime\prime}b + a_2^{\prime\prime}d + a_3^{\prime\prime}w + a_4^{\prime\prime}
\]

运用SPSS软件对此进行拟合调试与分析,由于测试数据存在不完整、不全面及缺漏现象,这将会影响到整体的拟合优度,例如对全体数据一起拟合,其拟合优度就只有50.7\%,单独对矮抗58和平安6号数据拟合,其拟合优度为85\%,可见其选择拟合的数据有一定的影响,对此我们对其分别拟合择优。

\textbf{Step2:比较分析,确定最佳拟合模型}

由于上分析可知,拟合数据的选择对拟合优度有着一定的影响,而由附件一的说明我们可知供试品种为矮抗58,它为主要研究的对象,其他都为它的对照品种,对此我们在数据选择中重点考虑矮抗58,同时考虑各种品种,分别对其拟合择优。

主要考虑的六种组合为:组合1:采用矮抗58和周麦18的数据进行拟合;组合2:采用矮抗58和平安6号的数据进行拟合;组合3:采用矮抗58和郑麦9023的数据进行拟合;组合4:采用矮抗58和周麦22的数据进行拟合的数据进行拟合;组合5:采用矮抗58和豫麦18的数据进行拟合;组合6:利用以上所有品种的拟合。

\begin{figure}[h]
    \centering
    \includegraphics[width=\textwidth]{image1.png}
    \caption{重心高度拟合}
\end{figure}

综上,我们选择了拟合优度最佳的回归模型,其具体的模型为:
\begin{equation}
h = 0.459l - 2.998b - 30.506d + 6.336w + 28.425
\end{equation}

\textbf{Step 3:模型检验}

\textbf{(1) 回归方程的拟合优度 $R^2$ 的检验:}

\begin{table}[h]
    \centering
    \caption{拟合优度检验}
    \begin{tabular}{|c|c|c|c|c|}
    \hline
    Mode & R & R Square & Adjusted R Square & Std. Error of the Estimate \\
    \hline
    1 & .922* & .850 & .834 & 3.35620 \\
    \hline
    \end{tabular}
\end{table}

由于回归方程的拟合优度 $R^2$ 越接近 1,拟合效果越好,此题中的拟合优度为 95.8\%,可见拟合效果已较好。

\textbf{(2) 回归方程的显著性 F 与回归系数的显著性 sig 的检验:}

\begin{table}[h]
    \centering
    \caption{显著性 F 与 sig 的检验}
    \begin{tabular}{|c|c|c|c|c|c|}
    \hline
    Model & Sum of Squares & df & Mean Square & F & Sig. \\
    \hline
    1 Regression & 2362.833 & 4 & 590.708 & 52.442 & .000* \\
    Residual & 416.772 & 37 & 11.264 & & \\
    Total & 2779.605 & 41 & & & \\
    \hline
    \end{tabular}
\end{table}

由于回归方程的显著性 F 越大越好,回归系数的显著性 sig $< 0.05$ 就越显著,此题中 F = 54.442 比较大,且 sig = 0.000,可见其显著性相当高。

\textbf{(3) 残差检验:}

\begin{figure}[h]
    \centering
    \includegraphics[width=0.8\textwidth]{image2.png}
    \caption{残差检验}
\end{figure}

当残差图均匀分布时效果最好,而此题中残差图分布得较均匀可见拟合得相当好。

(4) 茎秆机械强度的回归模型建立

Step1:运用 SPSS 软件对其进行数据拟合分析,并确定最佳拟合模型

设定其多元线性回归模型为:
\[
P = a_{0}^{\prime} l_{4} + a_{1}^{\prime} b_{4} + a_{2}^{\prime} d + a_{3}^{\prime}
\]
运用 SPSS 软件对此进行拟合调试与分析,得到其拟合优度最佳的回归模型为:
\[
P = 0.014 l_{4} - 0.168 b_{4} - 0.308 d + 1.083
\]

Step2:模型检验

(1) 回归方程的拟合优度 $R^{2}$ 的检验:

\textbf{拟合优度检验}

\begin{tabular}{|c|c|c|c|c|}
\hline
Mode & R & R Square & Adjusted R Square & Std. Error of the Estimate \\
\hline
1 & .913$^{a}$ & .834 & .669 & .06145 \\
\hline
\end{tabular}

由于回归方程的拟合优度 $R^{2}$ 越接近 1,拟合效果越好,此题中的拟合优度为 83.4\%,可见拟合效果较好。

(2) 回归方程的显著性 F 与回归系数的显著性 sig 的检验:

\textbf{显著性 F 与 sig 检验}

\begin{tabular}{|c|c|c|c|c|c|}
\hline
Model & Sum of Squares & df & Mean Square & F & Sig. \\
\hline
1 & Regression & .057 & 3 & .019 & 5.040 & .018$^{a}$ \\
 & Residual & .011 & 3 & .004 & & \\
 & Total & .068 & 6 & & & \\
\hline
\end{tabular}

由于回归方程的显著性 F 越大越好,回归系数的显著性 sig $< 0.05$ 就越显著,此题中 F = 5.040 比较大,且 sig = 0.018,可见其显著性较高。

(3) 残差检验

\textbf{残差检验}

\begin{figure}[h]
\centering
\includegraphics[width=0.6\textwidth]{residual_plot.png}
\caption{Dependent Variable: 机械强度 P}
\end{figure}

当残差图均匀分布时效果最好,而此题中残差图分布得较均匀可见拟合得相当好。

\section*{(5) 穗鲜重与穗干重间的关系模型的建立}

\textbf{Step 1:数据预处理}

由11年第二组测量数据可以拟合穗鲜重与穗干重的函数关系,针对三个日期的数据分别进行,其中5.28日数据的拟合效果最好。

\textbf{Step 2:运用SPSS软件对其进行数据拟合分析}

我们将穗干重作为自变量,穗鲜重作为因变量,在SPSS中运用曲线回归,分别拟合出其线性模型、对数模型、逆模型、复合模型、指数模型、幂模型等多种曲线,其各种模型拟合结果和拟合曲线如下:

\textbf{各种模型的拟合结果}

\begin{table}[h]
\centering
\begin{tabular}{|c|c|c|c|c|c|c|c|c|c|}
\hline
\multirow{2}{*}{Equation} & \multicolumn{5}{c|}{Model Summary} & \multicolumn{4}{c|}{Parameter Estimates} \\
\cline{2-10}
 & R Square & F & df1 & df2 & Sig. & Constant & b1 & b2 & b3 \\
\hline
Linear & .838 & 10.367 & 1 & 2 & .084 & .169 & 2.762 &  &  \\
Logarithmic & .849 & 11.237 & 1 & 2 & .079 & 2.973 & 2.926 &  &  \\
Inverse & .857 & 11.993 & 1 & 2 & .074 & 6.071 & -3.055 &  &  \\
Quadratic & .878 & 3.582 & 2 & 1 & .350 & -8.800 & 20.136 & -8.152 & -2.443 \\
Cubic & .878 & 3.587 & 2 & 1 & .350 & -5.572 & 11.182 & .000 &  \\
Compound & .801 & 8.032 & 1 & 2 & .105 & 1.145 & 2.513 &  &  \\
Power & .811 & 8.566 & 1 & 2 & .100 & 2.919 & .976 &  &  \\
S & .818 & 9.013 & 1 & 2 & .095 & 2.105 & -1.019 &  &  \\
Growth & .801 & 8.032 & 1 & 2 & .105 & .136 & .922 &  &  \\
Exponential & .801 & 8.032 & 1 & 2 & .105 & 1.145 & .922 &  &  \\
Logistic & .801 & 8.032 & 1 & 2 & .105 & .873 & .398 &  &  \\
\hline
\end{tabular}
\caption*{The independent variable is 穗干重.}
\end{table}

\textbf{各种模型的拟合曲线}

\begin{figure}[h]
\centering
\includegraphics[width=0.8\textwidth]{image.png}
\caption*{穗干重}
\end{figure}

\textbf{Step 3:比较分析,确定最佳拟合模型}

由于回归方程的拟合优度 \( R^2 \) 越接近1,拟合效果越好;

对此通过以上指标对各种模型拟合结果进行对比分析后,可知拟合出的二次模型其 \( R^2 = 87.8\% \),为最佳拟合模型。即:

\[
w = -8.152w_{0}^2 + 20.136w_{0} - 8.8
\]

(6) 茎秆抗倒伏指数模型的建立

基于上述的分析,建立各品种小麦的茎秆抗倒指数模型如下:

\[
\left\{
\begin{aligned}
I &= \frac{Gh}{P} \\
h &= 0.459l - 2.998b - 30.506d + 6.336w + 28.425 \\
G &= 0.116l - 0.463b + 12.933d + 1.326w - 8.337 \\
P &= 0.014l_4 - 0.168b_4 - 0.308d + 1.083 \\
w &= -8.152w_0^2 + 20.136w_0 - 8.8 \\
l &= \sum_{i=1}^4 l_i \\
b &= \frac{1}{4} \sum_{i=1}^4 b_i \\
d &= \frac{1}{4} \sum_{i=1}^4 d_i
\end{aligned}
\right.
\]

模型说明:

- $I$ 为小麦茎秆抗倒伏指数;
- $h$ 为拟合得到的茎秆重心高度表达式;
- $G$ 为拟合得到的茎秆鲜重表达式;
- $P$ 为拟合得到的机械强度表达式;
- $w$ 为根据穗干重 $w_0$ 拟合得到的穗鲜重;
- $l$ 为茎秆各节长度之和,$l_i$ 表示茎秆第 $i$ ($i=1, 2, 3, 4$) 节长度;
- $b$ 为茎秆粗平均值,$b_i$ 表示茎秆第 $i$ ($i=1, 2, 3, 4$) 节粗;
- $d$ 为茎秆壁厚平均值,$d_i$ 表示茎秆第 $i$ ($i=1, 2, 3, 4$) 节壁厚;

① 对于 07 年各品种小麦的茎秆抗倒指数的具体模型:

由于 07 年的数据比较完整,茎秆鲜重、茎秆重心高度和茎秆机械强度都已知,则可直接根据茎秆抗倒伏指数公式 $I = \frac{Gh}{P}$ 直接求解出各品种抗倒伏指数,则对于 07 年各品种小麦的茎秆抗倒指数模型可简化为:

\[
I = \frac{Gh}{P}
\]

② 对于 11 年各品种小麦的茎秆抗倒指数的具体模型:

由于 11 年的数据中缺少机械强度 $P$ 的数据,而机械强度 $P$ 的回归模型已求出,则可利用已知数据先求解出机械强度 $P$,然后再运用茎秆抗倒伏指数公式来求解出各品种抗倒伏指数,则对于 11 年各品种小麦的茎秆抗倒指数模型可简化为:

\begin{equation}
\begin{cases}
I = \frac{Gh}{P} \\
P = 0.014l_{4} - 0.168b_{4} - 0.308d + 1.083 \\
d = \frac{1}{4}\sum_{i=1}^{4}d_{i}
\end{cases}
\end{equation}

③ 对于 08 年各品种小麦的茎秆抗倒指数的具体模型:

由于 08 年的数据缺失较多,机械强度 \( P \)、茎秆鲜重 \( G \) 和重心高度 \( h \) 都没直接告诉,则我们得先利用前面已求出的机械强度 \( P \) 的回归模型、茎秆鲜重 \( G \) 的回归模型及穗鲜重 \( w \) 与穗干重间的关系模型和重心高度 \( h \) 的回归模型,先分别求解出机械强度 \( P \)、茎秆鲜重 \( G \) 和重心高度 \( h \),然后再运用茎秆抗倒伏指数公式来求解出各品种抗倒伏指数,则对于 08 年各品种小麦的茎秆抗倒指数模型可简化为:

\begin{equation}
\begin{cases}
I = \frac{Gh}{P} \\
G = 0.116l - 0.463b + 12.933d + 1.326w - 8.337 \\
P = 0.014l_{4} - 0.168b_{4} - 0.308d + 1.083 \\
h = 0.459l - 2.998b - 30.506d + 6.336w + 28.425 \\
w = -8.152w_{0}^{2} + 20.136w_{0} - 8.8 \\
l = \sum_{i=1}^{4}l_{i} \\
b = \frac{1}{4}\sum_{i=1}^{4}b_{i} \\
d = \frac{1}{4}\sum_{i=1}^{4}d_{i}
\end{cases}
\end{equation}

\subsection*{4.1.3 模型的求解}

(1) 对 07 年各品种小麦的茎秆抗倒指数的求解

由于 07 年的数据比较完整,茎秆鲜重、茎秆重心高度和茎秆机械强度都已知,则可直接运用上面的对于 07 年各品种小麦的茎秆抗倒指数的具体模型直接求解出各品种抗倒伏指数,代入已知数值,可求得小麦矮抗 58、新麦 208 和周麦 18 的茎秆抗倒伏指数为:

\begin{align*}
\text{矮抗 58 的茎秆抗倒伏指数:} & \quad I = 0.758; \\
\text{新麦 208 的茎秆抗倒伏指数:} & \quad I = 1.202; \\
\text{周麦 18 的茎秆抗倒伏指数:} & \quad I = 0.966
\end{align*}

(2) 对 11 年各品种小麦的茎秆抗倒指数的求解

运用上面的对于 11 年各品种小麦的茎秆抗倒指数的具体模型,可利用已知数据先求解出机械强度 \( P \),然后再代入茎秆抗倒伏指数公式来求解出各品种抗倒伏指数。代入已知数值,即可求得小麦矮抗 58、平安 6 号、郑麦 9023、周麦 18、周麦 22 和豫麦 18 的茎秆抗倒伏指数为:

\begin{align*}
\text{矮抗 58 的茎秆抗倒伏指数:} & \quad I = 0.493; \\
\text{平安 6 号的茎秆抗倒伏指数:} & \quad I = 0.866;
\end{align*}

\begin{itemize}
    \item 郑麦9023的茎秆抗倒伏指数:$I=0.487$;
    \item 周麦18的茎秆抗倒伏指数:$I=0.624$;
    \item 周麦22的茎秆抗倒伏指数:$I=0.708$;
    \item 豫麦18的茎秆抗倒伏指数:$I=0.619$
\end{itemize}

(3) 对08年各品种小麦的茎秆抗倒指数的求解

运用上面的对于08年各品种小麦的茎秆抗倒指数的具体模型,可先利用前面已求出的机械强度$P$的回归模型、茎秆鲜重$G$的回归模型及穗鲜重$w$与穗干重间的关系模型和重心高度$h$的回归模型,先分别求解出机械强度$P$、茎秆鲜重$G$和重心高度$h$,然后再运用茎秆抗倒伏指数公式来求解出各品种抗倒伏指数。代入已知数值,即可求得小麦矮抗58、国信、智9998早和联丰85的茎秆抗倒伏指数为:

\begin{itemize}
    \item 矮抗58的茎秆抗倒伏指数:$I=1.478$;
    \item 国信的茎秆抗倒伏指数:$I=1.014$;
    \item 智9998早的茎秆抗倒伏指数:$I=1.021$;
    \item 联丰85的茎秆抗倒伏指数:$I=1.470$
\end{itemize}

(4) 最终求解结果

在spss软件分析的基础上,利用Excel及Mathematica软件对茎秆抗倒伏指数进行统计计算,对于有多个年份数据的小麦品种,求取的最终结果取各年份的期望值,即:

\[
I = \frac{1}{n} \sum_{i=1}^{n} I_i \quad (n \text{为年份数})
\]

则可得到求解结果如下表所示:

\begin{table}[h]
\centering
\begin{tabular}{|c|c|c|c|c|c|}
\hline
\multicolumn{6}{|c|}{抗倒伏指数最终求解结果} \\
\hline
矮抗58 & 周麦18 & 豫麦18 & 国信 & 智9998早 & 联丰85 \\
\hline
0.626 & 0.966 & 1.202 & 1.014 & 1.021 & 0.809 \\
\hline
平安6号 & 新麦208 & 郑麦9023 & 周麦22 & 温麦6号 & \\
\hline
0.866 & 1.278 & 0.487 & 0.708 & 0.639 & \\
\hline
\end{tabular}
\end{table}

在表中,第一行为小麦的品种,第二行对应求解的抗倒伏指数。因为小麦抗倒伏指数越小,表示其抗倒伏能力越强。由表可知郑麦9023的抗倒伏能力最强,其次是矮抗58,而新麦208号的抗倒伏能力最弱。

\section{问题二模型的建立与求解}

\subsection{数据处理}

(1) 小麦最易倒伏时期的确定

通过对07年的三个品种在四个生长时期的抗倒伏指数计算,比较抗倒伏指数的大小可得出结论。

(2) 影响抗倒伏指数的茎秆外部形态特征的确定

由第一问的总模型可确定茎秆长、各节间长、各节壁厚、茎秆鲜重为影响因素,此外考虑实际情况,还加入了节间长度比,穗长和穗重的影响。其中参考相关论文可知各节中穗下第四节对小麦的倒伏性影响最大,所以节间长度比采用 $l_{i}/l_{4} \ (i=1,2,3)$ 表示。

\subsection{最易引起倒伏期的判断与确定}

由于各品种小麦在不同时期的抗倒伏性能不同,其抗倒伏指数是变化的,对此我们可利用07年的数据分别对小麦开花期、灌浆期、乳熟期和蜡熟期的抗倒伏性研究,通过对各品种不同时期抗倒伏指数变化的分析比较,即可判断出小麦的最易引起倒伏期。代入数据,由问题一中的抗倒伏指数模型可求出各品种不同时期抗倒伏指数变化如下:

\begin{table}[h]
\centering
\begin{tabular}{|c|c|c|c|c|}
\hline
\multicolumn{5}{|c|}{ 各品种不同时期抗倒伏指数的变化 } \\
\hline
品种 & 开花期 & 灌浆期 & 乳熟期 & 蜡熟期 \\
\hline
矮抗58 & 0.351 & 0.419 & 0.672 & 0.758 \\
\hline
新麦208 & 0.488 & 0.761 & 0.892 & 1.202 \\
\hline
周麦18 & 0.521 & 0.675 & 0.799 & 0.966 \\
\hline
\end{tabular}
\end{table}

由表中可以看出,小麦的抗倒伏指数在开花期最小,然后逐渐增大,当小麦成熟时,蜡熟期时茎秆抗倒伏指数为最大。由于抗倒伏指数越大,抗倒伏能力越弱,可见最易引起倒伏期为蜡熟期。

\subsection{抗倒伏指数与茎秆外部形态特征间的相关性分析}

(1) 抗倒伏指数与各节间长的相关性分析

Step1:建立抗倒伏指数与各节间长的相关性系数模型

根据相关性分析法理论基础\cite{ref1}可知:设有随机变量 $X$ 与 $Y$,对其进行了 $n$ 次随机试验,得到的观测值分别为 $(X_{i}, Y_{i}) \ (i=1,2,\cdots n)$,$\overline{X}$、$\overline{Y}$ 分别为各自的平均值,$R$ 为随机变量 $X$ 与 $Y$ 对于样本 $(X_{i}, Y_{i}) \ (i=1,2,\cdots n)$ 的相关性系数,称之为样本相关系数。在实际中,常常用样本相关系数 $R$ 作为相关系数 $r$ 估计值。根据以上理论知识可建立以下相关性系数模型:

\begin{equation}
R = \frac{\sum\limits_{i=1}^{n} (X_i - \overline{X})(Y_i - \overline{Y})}{\sqrt{\sum\limits_{i=1}^{n} (X_i - \overline{X})^2 \sum\limits_{i=1}^{n} (Y_i - \overline{Y})^2}}
\end{equation}
\begin{equation}
\overline{X} = \frac{1}{n} \sum_{i=1}^{n} X_i
\end{equation}
\begin{equation}
\overline{Y} = \frac{1}{n} \sum_{i=1}^{n} Y_i
\end{equation}

Step 2: 利用 SPSS 软件对抗倒伏指数与各节间长间的关系进行相关性分析

利用 SPSS 软件中的 Correlate 菜单能对变量间的关系进行很好的相关性分析,对此我们利用已知数据对抗倒伏指数与各节间长间的关系进行相关性分析,可得到以下结果:

\textbf{抗倒伏指数与各节间长的相关性分析}

\begin{table}[h]
\centering
\begin{tabular}{|c|c|c|c|c|c|}
\hline
 & 抗倒伏指数 & 穗下第一节长 L1 & 穗下第二节长 L2 & 穗下第三节长 L3 & 穗下第四节长 L4 \\
\hline
抗倒伏指数 Pearson Correlation & 1 & -0.349 & -0.348 & 0.159 & 0.413 \\
Sig. (2-tailed) & & 0.293 & 0.294 & 0.640 & 0.207 \\
\hline
穗下第一节长 L1 Pearson Correlation & -0.349 & 1 & 0.623 & 0.411 & -0.327 \\
Sig. (2-tailed) & 0.293 & & 0.041 & 0.209 & 0.327 \\
\hline
穗下第二节长 L2 Pearson Correlation & -0.348 & 0.623 & 1 & 0.578 & 0.166 \\
Sig. (2-tailed) & 0.294 & 0.041 & & 0.062 & 0.626 \\
\hline
穗下第三节长 L3 Pearson Correlation & 0.159 & 0.411 & 0.578 & 1 & 0.578 \\
Sig. (2-tailed) & 0.640 & 0.209 & 0.062 & & 0.062 \\
\hline
穗下第四节长 L4 Pearson Correlation & 0.413 & -0.327 & 0.166 & 0.578 & 1 \\
Sig. (2-tailed) & 0.207 & 0.327 & 0.626 & 0.062 & \\
\hline
\end{tabular}
\end{table}

Step 3: 抗倒伏指数与各节间长间的相关性结果分析

从分析结果表中,我们可以看出穗下第四节长与抗倒伏指数间呈显著正相关,且相关系数最大;穗下第三节长与抗倒伏指数间呈正相关;穗下第二节长与抗倒伏指数间呈显著负相关;穗下第一节长与抗倒伏指数间也呈显著负相关。而且可知穗下第四节长与小麦抗倒伏指数最为密切。

(2) 抗倒伏指数与节间长度比的相关性分析

同理,根据相关性分析法理论基础,利用 SPSS 软件中的 Correlate 菜单能对变量间的关系进行相关性分析,结合才求出的穗下第四节长与小麦抗倒伏指数最为密切,对此我们采用穗下第四节长与各节间的比来作为重点探讨意义更大,为此采用已知数据对抗倒伏指数与节间长度比进行相关性分析,可得到以下结果:

\section*{抗倒伏指数与节间长度比的相关性分析}

\begin{table}[h]
\centering
\begin{tabular}{|c|c|c|c|c|}
\hline
 & 抗倒伏指数 & L1比L4 & L2比L4 & L3比L4 \\
\hline
抗倒伏指数 Pearson Correlation Sig. (2-tailed) & 1 & -.497 .120 & -.518 .043 & -.166 .626 \\
\hline
L1比L4 Pearson Correlation Sig. (2-tailed) & -.497 .120 & 1 & .849 .001 & .570 .067 \\
\hline
L2比L4 Pearson Correlation Sig. (2-tailed) & -.518 .043 & .849 .001 & 1 & .539 .087 \\
\hline
L3比L4 Pearson Correlation Sig. (2-tailed) & -.166 .626 & .570 .067 & .539 .087 & 1 \\
\hline
\end{tabular}
\end{table}

从分析结果表中,我们可以看出穗下第二、四节长比与抗倒伏指数间呈显著负相关,且相关系数绝对值最大;穗下第一、四节长比,穗下第三、四节长比与抗倒伏指数间呈负相关。而且可知穗下第二、四节长比与小麦抗倒伏指数最为密切。

\section*{(3) 抗倒伏指数与各节壁厚的相关性分析}

同理,根据相关性分析法理论基础,利用 SPSS 软件中的 Correlate 菜单能对变量间的关系进行相关性分析,采用已知数据对抗倒伏指数与各节壁厚进行相关性分析,可得到以下结果:

\section*{抗倒伏指数与各节壁厚的相关性分析}

\begin{table}[h]
\centering
\begin{tabular}{|c|c|c|c|c|c|}
\hline
 & 抗倒伏指数 & 第一节壁厚d1 & 第二节壁厚d2 & 第三节壁厚d3 & 第四节壁厚d4 \\
\hline
抗倒伏指数 Pearson Correlation Sig. (2-tailed) & 1 & -.296 .376 & -.171 .615 & -.261 .438 & -.129 .705 \\
\hline
第一节壁厚d1 Pearson Correlation Sig. (2-tailed) & -.296 .376 & 1 & .914 .000 & .774 .005 & .803 .003 \\
\hline
第二节壁厚d2 Pearson Correlation Sig. (2-tailed) & -.171 .615 & .914 .000 & 1 & .835 .001 & .884 .000 \\
\hline
第三节壁厚d3 Pearson Correlation Sig. (2-tailed) & -.261 .438 & .774 .005 & .835 .001 & 1 & .936 .000 \\
\hline
第四节壁厚d4 Pearson Correlation Sig. (2-tailed) & -.129 .705 & .803 .003 & .884 .000 & .936 .000 & 1 \\
\hline
\end{tabular}
\end{table}

从分析结果表中,我们可以看出各节壁厚与抗倒伏指数间呈显著负相关,且穗下第一节壁厚与小麦抗倒伏指数最为密切。

\section*{(4) 抗倒伏指数与小麦茎秆各因素间的相关性分析}

Step1:建立抗倒伏指数与小麦茎秆各因素间的相关性系数模型

根据相关性分析法理论基础可知:对有多个变量间的关系,可设其中任意两个随机变量 \(X\) 与 \(Y\),对其进行了 \(n\) 次随机试验,得到的观测值分别为 \((X_i, Y_i)\) \((i=1, 2, \cdots n)\),\(\overline{X}\)、\(\overline{Y}\) 分别为各自的平均值,\(R\) 为随机变量 \(X\) 与 \(Y\) 对于样本 \((X_i, Y_i)\) \((i=1, 2, \cdots n)\) 的相关性系数,称之为样本相关系数。在实际中,常常用样本相关系数 \(R\) 作为相关系数 \(r\) 估计值。

根据以上理论知识可建立以下相关性系数模型:

\begin{equation}
R = \frac{\sum\limits_{i=1}^{n} (X_i - \overline{X})(Y_i - \overline{Y})}{\sqrt{\sum\limits_{i=1}^{n} (X_i - \overline{X})^2 \sum\limits_{i=1}^{n} (Y_i - \overline{Y})^2}}
\end{equation}
\begin{equation}
\overline{X} = \frac{1}{n} \sum_{i=1}^{n} X_i
\end{equation}
\begin{equation}
\overline{Y} = \frac{1}{n} \sum_{i=1}^{n} Y_i
\end{equation}

对有多个变量间的关系,其中随机变量 \(X\) 与 \(Y\) 可表示其中任意两个随机变量。

Step2:利用 SPSS 对抗倒伏指数与小麦茎秆各因素间的关系进行相关性分析

利用 SPSS 软件中的 Correlate 菜单能对各变量间的关系进行很好的相关性分析,对此我们利用已知数据对抗倒伏指数与小麦茎秆各因素间的关系进行相关性分析,可得到以下结果:

\begin{table}[h]
\centering
\caption{抗倒伏指数与小麦茎秆各因素间的相关性分析}
\begin{tabular}{|c|c|c|c|c|c|c|c|c|c|}
\hline
 & 抗倒伏指数 & 穗下第四节长 L4 & L2比L4 & 第一节壁厚 d1 & 株高 & 穗长 & 穗重 & 茎秆鲜重 \\
\hline
抗倒伏指数 & Pearson Correlation & .522 & -.656 & -.450 & .772 & -.499 & .371 & .831 \\
 & Sig. (2-tailed) & .288 & .157 & .370 & .072 & .313 & .974 & .040 \\
\hline
穗下第四节长 L4 & Pearson Correlation & 1 & -.725 & .228 & .089 & -.203 & -.398 & .342 \\
 & Sig. (2-tailed) & & .103 & .663 & .867 & .700 & .434 & .507 \\
\hline
L2比L4 & Pearson Correlation & -.725 & 1 & -.272 & .611 & .732 & .652 & -.364 \\
 & Sig. (2-tailed) & .103 & & .602 & .197 & .098 & .161 & .478 \\
\hline
第一节壁厚 d1 & Pearson Correlation & .228 & -.272 & 1 & -.473 & -.053 & .155 & .178 \\
 & Sig. (2-tailed) & .663 & .602 & & .343 & .921 & .769 & .736 \\
\hline
株高 & Pearson Correlation & .772 & .089 & -.473 & 1 & .607 & .033 & -.454 \\
 & Sig. (2-tailed) & .072 & .867 & .343 & & .202 & .951 & .366 \\
\hline
穗长 & Pearson Correlation & -.499 & .732 & -.053 & .607 & 1 & .745 & -.505 \\
 & Sig. (2-tailed) & .313 & .098 & .921 & .202 & & .089 & .307 \\
\hline
穗重 & Pearson Correlation & .371 & .652 & .155 & .033 & .745 & 1 & -.066 \\
 & Sig. (2-tailed) & .974 & .161 & .769 & .951 & .089 & & .901 \\
\hline
茎秆鲜重 & Pearson Correlation & .831 & .342 & .178 & -.454 & -.505 & -.066 & 1 \\
 & Sig. (2-tailed) & .040 & .507 & .736 & .366 & .307 & .901 & \\
\hline
\end{tabular}
\end{table}

Step3:抗倒伏指数与小麦茎秆各因素间的相关性结果分析

从分析结果表中,我们可以看出穗下第四节长与抗倒伏指数间呈显著正相关;穗下第二、四节长比与抗倒伏指数间呈显著负相关;穗下第一节壁厚与抗倒伏指数间呈显著负相关;株高与抗倒伏指数间呈极显著正相关;穗长与抗倒伏指数间相关;穗重与抗倒伏指数间呈正相关;茎秆鲜重与抗倒伏指数间呈极显著正相关。

而且可知小麦茎秆各因素间与小麦抗倒伏指数的密切度大小,在正相关上:茎秆鲜重 > 株高 > 穗下第四节长 > 穗重;在负相关上:穗下第二、四节长比 > 穗长 > 穗下第一节壁厚。

4.2.4 小麦茎秆性状的各因素间的相关性分析

同理,根据相关性分析法理论基础,利用 SPSS 软件中的 Correlate 菜单能对变量间的关系进行相关性分析,采用已知数据对小麦茎秆性状的各因素间进行相关性分析,可得到以下结果:

\begin{table}
\centering
\begin{tabular}{c|c|c|c|c|c|c|c}
\hline
 & 穗下第四节长 & L2比L4 & 第一节壁厚d1 & 株高 & 穗长 & 穗重 & 茎秆鲜重 \\
 & L4 & & & & & & \\
\hline
穗下第四节长L4 & 1 & -.725 & .228 & -.089 & -.203 & -.398 & .342 \\
Pearson Correlation & & .103 & .663 & .867 & .700 & .434 & .507 \\
Sig. (2-tailed) & & & & & & & \\
\hline
L2比L4 & -.725 & 1 & -.272 & .611 & .732 & .652 & -.364 \\
Pearson Correlation & .103 & & .602 & .197 & .098 & .161 & .478 \\
Sig. (2-tailed) & & & & & & & \\
\hline
第一节壁厚d1 & .228 & -.272 & 1 & -.473 & -.053 & .155 & .178 \\
Pearson Correlation & .663 & .602 & & .343 & .921 & .769 & .736 \\
Sig. (2-tailed) & & & & & & & \\
\hline
株高 & -.089 & .611 & -.473 & 1 & .607 & .033 & -.454 \\
Pearson Correlation & .867 & .197 & .343 & & .202 & .951 & .366 \\
Sig. (2-tailed) & & & & & & & \\
\hline
穗长 & -.203 & .732 & -.053 & .607 & 1 & .745 & -.505 \\
Pearson Correlation & .700 & .098 & .921 & .202 & & .089 & .307 \\
Sig. (2-tailed) & & & & & & & \\
\hline
穗重 & -.398 & .652 & .155 & .033 & .745 & 1 & -.066 \\
Pearson Correlation & .434 & .161 & .769 & .951 & .089 & & .901 \\
Sig. (2-tailed) & & & & & & & \\
\hline
茎秆鲜重 & .342 & -.364 & .178 & -.454 & -.505 & -.066 & 1 \\
Pearson Correlation & .507 & .478 & .736 & .366 & .307 & .901 & \\
Sig. (2-tailed) & & & & & & & \\
\hline
\end{tabular}
\caption{小麦茎秆性状各因素间的相关性分析}
\end{table}

由于一般当相关系数绝对值大于0.4时,两者间就有着较强的相关性。从分析结果表中,我们可以看出株高与穗长、穗下第二四节长比间有着较强的正相关性;穗长与穗下第二四节长比、株高、穗重间有着较强的正相关性,与茎秆鲜重有着较强的负相关性;穗重与穗下第二四节长比、穗长有着较强的正相关性;茎秆鲜重与株高、穗长有着较强的相关性。

\subsection*{4.2.5 2008年国信1号与智9998品种的小麦发生倒伏原因的判断与分析}

首先,我们对2008年正常的几个品种各小麦茎秆外部形态与性状进行计算整理,得到以下,08年各品种小麦特性对比表。

\begin{table}
\centering
\begin{tabular}{c|c|c|c|c|c|c|c}
\hline
品种 & 穗下第四节长(cm) & 株高(cm) & 第二节长/第四节长 & 第一节壁厚(mm) & 穗长(cm) & 穗重(g) & 茎秆鲜重(g) \\
\hline
矮抗58 & 7.19 & 53.97 & 2.073 & 0.43 & 8.53 & 2.96 & 6.20 \\
\hline
周麦18 & 6.49 & 58.22 & 2.516 & 0.38 & 9.02 & & \\
\hline
国信 & 7.41 & 70.93 & 1.888 & 0.25 & 8.74 & 3.07 & 6.61 \\
\hline
智9998早 & 7.67 & 59.80 & 2.038 & 0.35 & 8.30 & 3.61 & 7.01 \\
\hline
联丰85 & 7.24 & 55.43 & 2.672 & 0.38 & 9.13 & 2.64 & 6.16 \\
\hline
\end{tabular}
\caption{08年各品种小麦特性对比表}
\end{table}

再结合我们研究出的抗倒伏指数与小麦茎秆各因素间的相关性分析结果,便很容易发现国信1号与智9998品种为什么发生倒伏,而其他品种没有发生倒伏的原因。

其原因如下:

(1) 对于穗下第四节长,我们通过比较可以很容易发现国信1号与智9998品种穗下第四节长比较大,大于了其余品种。而穗下第四节长与抗倒伏指数间呈显著正相关,即穗下第四节长越大,抗倒伏指数越大,其抗倒伏能力就越弱。

(2) 对于株高,我们通过比较可以很容易发现国信1号与智9998品种株高比较大,大于了其余品种。而株高与抗倒伏指数间呈极显著正相关,即株高越大,抗倒伏指数越大,其抗倒伏能力就越弱。

(3) 对于第二节长/第四节长,我们通过比较可以很容易发现国信1号与智19

9998 品种第二节长/第四节长比较小,小于了其余品种。而第二节长/第四节长与抗倒伏指数间呈显著负相关,即第二节长/第四节长越小,抗倒伏指数越大,其抗倒伏能力就越弱。

(4) 对于第一节壁厚,我们通过比较可以很容易发现国信 1 号与智 9998 品种第一节壁厚比较小,小于了其余品种。而第一节壁厚与抗倒伏指数间呈显著负相关,即第一节壁厚越小,抗倒伏指数越大,其抗倒伏能力就越弱。

(5) 对于穗重,我们通过比较可以很容易发现国信 1 号与智 9998 品种穗重比较大,大于了其余品种。而穗重与抗倒伏指数间呈极显著正相关,即穗重越大,抗倒伏指数越大,其抗倒伏能力就越弱。

(6) 对于茎秆鲜重,我们通过比较可以很容易发现国信 1 号与智 9998 品种茎秆鲜重比较大,大于了其余品种。而茎秆鲜重与抗倒伏指数间呈极显著正相关,即茎秆鲜重越大,抗倒伏指数越大,其抗倒伏能力就越弱。

\section{问题三模型的建立与求解}

\subsection{数据处理}

(1) 小麦穗干重的确定

对于小麦穗干重即为单穗籽粒重与小麦茎秆自重之差,即:

穗干重 = 单穗籽粒重 - 小麦茎秆自重

(2) 小麦穗鲜重的确定

由问题一求解结果知,小麦穗鲜重 $w$ 可由穗干重 $w_0$ 求解出,即:

\[
w = -8.152w_0^2 + 20.136w_0 - 8.8
\]

\subsection{模型的分析}

该问题要求我们探讨单穗重分别是 $1.19\,\text{g}$, $2.06\,\text{g}$, $2.46\,\text{g}$, $2.56\,\text{g}$, $2.75\,\text{g}$, $2.92\,\text{g}$ 时小麦的理想株型结构。

对此我们建立其相应的模型,其目的是为了对小麦株型结构进行合理的配置,以便可以根据已知的小麦单穗籽粒重来确定如何安排该种小麦的合理育种。结合实际,主要考虑制定合理的小麦育种方案,以满足如下两个需求因素:

目标1:抗倒伏系数最小(抗倒伏能力最强)

目标2:经济系数最大

\subsubsection{目标分析}

\subsubsection{目标1:抗倒伏系数最小(抗倒伏能力最强)}

要实现小麦高产优质的跨越,就必须解决或尽量减少小麦的倒伏问题。解决倒伏问题的方法之一就是针对不同的产量,寻找小麦抗倒伏能力最佳的茎秆性状(包括株高、茎长、各节间长、各节茎外径、壁厚、茎秆自重、穗长、穗重等)。

这里我们刻画的抗倒伏系数是为了说明不同单穗籽粒重小麦的理想株型,它可以用单穗籽粒重与穗鲜重之间的关系来求出对应的抗倒系数,从而用求出的抗倒系数来刻画。由于抗倒伏系数越小越好,其抗倒伏能力越强,所以抗倒伏系数最小可以表示为:

\[
\min \, I = \frac{G \cdot h}{P}
\]

其中,$I$ 表示不同单穗籽粒重小麦的抗倒伏系数;

$G$ 表示第 $i$ 种小麦茎秆鲜重;

$h$ 表示第 $i$ 种小麦茎秆重心高度;

$P$ 表示第 $i$ 种小麦茎秆机械强度。

\subsubsection{目标2:经济系数最大}

由于经济系数的定义为经济产量与生物学产量之比,根据植物种类、品种、自然条件和栽培措施而不同,是影响小麦高产的重要因素。同时它能刻画出产量及小麦具体种植分布关系。

这里我们刻画的经济系数是在考虑抗倒性系数以及平均亩穗数的影响下的经济系数,是在满足抗倒性系数越低,平均亩穗数越大的情况下来刻画的。由于在一定范围内,经济系数越大越好,所以小麦产量的经济系数最大可以表示为:

\[
\max \quad Z = \frac{\sum_{i} P_{\max} \cdot I}{\sum_{i} y \cdot \sum_{i} x}
\]

其中,$Z$ 表示小麦产量的经济系数;

\[
\sum_{i} P_{\max} \quad \text{表示第 } i \text{ 种小麦的最佳产量;}
\]

\[
\sum_{i} y \quad \text{表示第 } i \text{ 种小麦平均亩穗数;}
\]

\[
\sum_{i} x \quad \text{表示第 } i \text{ 种小麦单穗籽粒重;}
\]

约束分析:

(1) 小麦各节间长 $l_{i}$ 的约束

根据问题二中抗倒伏指数相关性指标的分析,可知小麦穗下第四节长与抗倒伏指数间呈显著正相关,且相关系数最大;小麦穗下第三节长与抗倒伏指数间呈正相关;小麦穗下第二节长与抗倒伏指数间呈显著负相关;小麦穗下第一节长与抗倒伏指数间也呈显著负相关,且有相关系数的大小可知,有约束:

\[
l_{4} < l_{3} < l_{2} < l_{1}
\]

(2) 小麦各节间长度比 $l_{i} / l_{j}$ 的约束

根据问题二中抗倒伏指数相关性指标的分析,可知小麦穗下第二、四节长比与抗倒伏指数间呈显著负相关,且相关系数绝对值最大;小麦穗下第一、四节长比,小麦穗下第三、四节长比与抗倒伏指数间呈负相关,于是有约束:

\[
\frac{l_{3}}{l_{4}} < \frac{l_{1}}{l_{4}} < \frac{l_{2}}{l_{4}}
\]

(3) 小麦各节壁厚 $d_{i}$ 的约束

根据问题二中抗倒伏指数相关性指标的分析,可知各节壁厚与抗倒伏指数间呈显著负相关,且小麦穗下第一节壁厚与小麦抗倒伏指数最为密切,于是有约束:

\[
d_{1} < d_{2} < d_{3} < d_{4}
\]

(4) 小麦茎秆各因素间的约束

根据问题二中抗倒伏指数相关性指标的分析,可知小麦茎秆各因素与抗倒伏

指数间呈正相关:茎秆鲜重 > 小麦穗下第四节长 > 小麦穗下第一节壁厚,于是有约束:
\[
G > l_4 > d_1
\]

(5) 最佳产量 \( P_{\max} \) 的约束

根据 2008 年补充说明,可知各种小麦的最佳产量在一定范围内,于是有约束:
\[
500 \leq P_{\max} \leq 700
\]

(6) 经济系数 \( Z \) 的约束

根据 2011 年补充说明,目前高产小麦的经济系数多在 0.45 左右,根据图表所列数据,于是有约束:
\[
Z \leq 0.5
\]

(7) 平均亩穗数 \( y \) 的约束

根据 2008 年补充说明,平均亩穗数对产量的影响大于单穗籽粒重对产量的影响,于是有约束:
\[
38 \leq y \leq 56
\]

(8) 单穗籽粒重 \( x \) 的约束

根据 2008 年补充说明,观察个品种单穗籽粒重数据,于是有约束:
\[
0.97 \leq x \leq 2.92
\]

4.3.3 模型的建立

针对这个多目标优化问题,我们基于 4.3.2 的分析,建立多目标规划模型如下:

\begin{align*}
\min \quad & I = \frac{G \cdot h}{P} \\
\max \quad & Z = \frac{\sum P_{\max} \cdot I}{\sum_{i} y \cdot \sum_{i} x}
\end{align*}

\begin{equation}
s.t.
\begin{cases}
l_{4} < l_{3} < l_{2} < l_{1} \\
\frac{l_{3}}{l_{4}} < \frac{l_{1}}{l_{4}} < \frac{l_{2}}{l_{4}} \\
d_{1} < d_{2} < d_{3} < d_{4} \\
G > l_{4} > d_{1} \\
500 \leq P_{\max} \leq 700 \\
Z \leq 0.5 \\
38 \leq y \leq 56 \\
0.97 \leq x \leq 2.92 \\
G = 0.116 \sum_{i=1}^{4} l_{i} - \frac{0.463}{4} \sum_{i=1}^{4} b_{i} + \frac{12.933}{4} \sum_{i=1}^{4} d_{i} + 1.326w - 8.337 \\
P = 0.014l_{4} - 0.168b_{4} - 0.308d + 1.083 \\
h = 0.459 \sum_{i=1}^{4} l_{i} - \frac{0.463}{4} \sum_{i=1}^{4} b_{i} - \frac{12.933}{4} \sum_{i=1}^{4} d_{i} + 6.336w + 28.425 \\
w = -8.152w_{0}^{2} + 20.136w_{0} - 8.8
\end{cases}
\end{equation}

模型说明:

- 目标1为(抗)倒伏系数最小;
- 目标2为经济系数最大;
- $\sum_{i} G$ 表示小麦茎秆鲜度;
- $P$ 表示小麦茎秆机械强度;
- $\sum_{i} h$ 表示小麦茎秆重心高度;
- $w$ 表示小麦穗鲜度;
- 约束1为小麦各节间长 $l_{i}$ 的约束;
- 约束2为小麦各节间长度比 $l_{i}/l_{j}$ 的约束;
- 约束3为小麦各节壁厚 $d_{i}$ 的约束;
- 约束4为小麦茎秆各因素间的约束;
- 约束5为最佳产量 $P_{\max}$ 的约束;
- 约束6为经济系数 $Z$ 的约束;
- 约束7为平均亩穗数 $y$ 的约束;

约束 8 为单穗籽粒重 \( x \) 的约束;

\subsection*{4.3.4 模型的求解}

(1) 模型的化简:

这是一个多目标决策问题,首先我们应将多目标规划模型转化为单目标规划模型来求解。

由于这两个目标函数的量纲存在不同,且权重系数合理确定比较难确定,不易用线性加权法对其化简。对此我们运用序贯法对其化简,因为由问题分析知道首先希望小麦的抗倒伏能力强,其次才会考虑它的经济系数在一定范围内取得最优值。

利用序贯法将模型划分层次如下:

第一级:
\[
\begin{aligned}
& \min \quad I = \frac{G \cdot h}{P} \\
& \begin{cases}
l_4 < l_3 < l_1 < l_2 \\
d_1 < d_2 < d_3 < d_4 \\
G > l_4 > d_1 \\
G = 0.116 \sum_{i=1}^4 l_i - \frac{0.463}{4} \sum_{i=1}^4 b_i + \frac{12.933}{4} \sum_{i=1}^4 d_i + 1.326w - 8.337 \\
P = 0.014l_4 - 0.168b_4 - 0.308d + 1.083 \\
h = 0.459 \sum_{i=1}^4 l_i - \frac{0.463}{4} \sum_{i=1}^4 b_i - \frac{12.933}{4} \sum_{i=1}^4 d_i + 6.336w + 28.425 \\
w = -8.152w_0^2 + 20.136w_0 - 8.8
\end{cases}
\end{aligned}
\]

求得的最优值设为 \( I_0 \);

第二级:
\[
\begin{aligned}
& \max \quad Z = \frac{\sum_i P_{\max} \cdot I_0}{\sum_i y \cdot \sum_i x} \\
& \begin{cases}
s.t2 \\
500 \leq P_{\max} \leq 700 \\
Z \leq 0.5 \\
38 \leq y \leq 56 \\
0.97 \leq x \leq 2.92
\end{cases}
\end{aligned}
\]

(2) 模型的求解结果:

对此我们运用 LINGO 9.0 版本软件对其进行编程求解,假设已知穗干重为 0.79g,即代入 $w_{0}=0.79g$(对应单穗重为 1.19g),($w_{0}$ 取它任何值的情况类同)。对其求解(其程序见附录),得到的结果为:

\[
I_{0}=0.443 \quad Z=0.43
\]

\[
l_{4}=6.42, \, d_{4}=0.5, \, G=5.574, \, P=0.466, \, h=37.02, \, w=2.02
\]

同理可得到其他单穗重时的结果如下:

\begin{table}[h]
\centering
\begin{tabular}{|c|c|c|c|c|c|c|c|}
\hline
单穗重 $w_{0}(\mathrm{~g})$ & 抗倒伏指数 $I_{0}$ & 经济系数 $Z$ & 穗下第四节长 $14(\mathrm{~cm})$ & 第四节壁厚 $d_{4}(\mathrm{~mm})$ & 茎秆鲜重 $G(\mathrm{~g})$ & 茎秆重心高度 $h(\mathrm{~cm})$ & 机械强度 $P$ \\
\hline
1.19 & 0.443 & 0.43 & 6.42 & 0.50 & 5.57 & 37.02 & 0.466 \\
\hline
2.06 & 0.456 & 0.46 & 6.83 & 0.29 & 4.72 & 51.52 & 0.533 \\
\hline
2.46 & 0.446 & 0.45 & 6.62 & 0.28 & 4.61 & 51.55 & 0.532 \\
\hline
2.56 & 0.451 & 0.46 & 6.51 & 0.31 & 4.77 & 49.04 & 0.520 \\
\hline
2.75 & 0.445 & 0.43 & 6.38 & 0.34 & 4.90 & 45.86 & 0.505 \\
\hline
2.92 & 0.448 & 0.44 & 6.26 & 0.41 & 5.21 & 41.73 & 0.485 \\
\hline
\end{tabular}
\end{table}

\section*{4.4 问题四}

\subsection*{4.4.1 小麦茎秆抗倒伏性能评价指标的建立}

对小麦抗倒伏性进行评价分析时,应先选出评价指标,而评价指标的选取主要应遵循以下原则:

(1) 科学性原则:这是确保评价结果准确合理的基础。指标体系的科学性原则包括以下两个方面:一方面是特征性,指标应反映评估对象的特征;另一方面是准确一致性,指标的概念要正确,含义要清晰,尽可能避免或减少主观判断,对难以量化的评估因素应采用定性和定量相结合的方法来设置指标。

(2) 突出主要指标的原则:在选择评价指标时,由于指标可以很多,但每个指标的贡献程度不同,因此不能任意罗列,否则将会增加评价的难度,而且容易冲淡对主要指标的注意力。所以,在选择评价指标时,必须认真分析,明确重点。

(3) 独立性原则:即各指标之间必须相互独立。当指标之间存在明显的相关关系时,应采取主要指标的原则。这样不仅可以减少主要指标的数量,而且不丢失必要的信息。

依据上述原则,通过查阅相关资料,并结合问题中要求考虑小麦茎秆在麦穗自重和风载作用下应力的基本规律,构建了小麦茎秆抗倒伏性能的评价指标体系,包括了以下 4 个评价指标:

- 指标 1:弹性模量指标;
- 指标 2:茎秆自重系数指标;
- 指标 3:风载临界力指标;
- 指标 4:惯性矩指标。

\section*{指标分析:}

(1) 弹性模量指标

弹性模量指的是与材料有关的比例常数。即材料在弹性变形阶段内,正应力和对应的正应变的比值,与小麦茎秆抗倒伏性能有着密切关系。

对于弹性模量的确定有多种方法,这里我们为简单起见,参考茎秆弹性测定说明中的提示,假定弹性测定值和弹性模量值为线性关系。以腊熟期计,由弹性测定值的最小 0.761 至最大 0.988 可对应于弹性模量值的最小 355 至最大 1443,其节与节之间的弹性测定值、弹性模量值以线性关系大小对应。

则可求得弹性测定值与弹性模量之间的比例关系如下:

\[
k = \frac{1443 - 355}{0.988 - 0.761} = 4792.95
\]

则可知表示弹性模量指标为:

\[
E = k \cdot f = 4792.95 \cdot f
\]

其中,$f$—弹性测定平均值,$k$—弹性测定值与弹性模量之间的比例关系,  
$E$—茎秆弹性模量(GPa)

(2) 茎秆自重系数指标

令 $q_{cr}$ 为临界状态时茎秆单位长度的自重。$P_{cr}$ 为临界状态时的穗重。由附件三有关力学的资料,知:

① 茎秆势能:
\[
\Pi = \frac{3EI\delta^2}{2l^3} - \frac{3}{16}q_{cr}\delta^2 - \frac{9\delta^2}{8l^6}\left(\frac{4}{3}l^2h^3 - lh^4 + \frac{1}{5}h^5\right)P_{cr}
\]
式中 $q_{cr}$—临界状态时茎秆单位长度的自重,$P_{cr}$—临界状态时的穗重,  
$\delta$—位移参数,它表示茎秆顶端处的水平位移,$l$—茎秆的高度,$x$—截面位置,$\nu$—挠度,表示 $x$ 截面处的水平位移,$l, \delta, x, \nu$ 的单位为 cm。

② 由势能驻值原理 $\frac{\partial \Pi}{\partial \delta} = 0$ 可得:
\[
q_{cr} + \frac{6P_{cr}}{l^6}\left(\frac{4}{3}l^2h^3 - lh^4 + \frac{1}{5}h^5\right) = \frac{8EI}{l^3}.
\]
化简可得:
\[
P_{cr} = \frac{8EI}{bl^2} - \frac{Q}{b}.
\]

③ 茎秆自重系数 $\lambda$
\[
\lambda^2 = \frac{bl^2}{I}
\]
单位为 $cm^{-1}$。茎秆自重系数 $\lambda$ 与秆长,穗位,截面尺寸和形状等茎秆性状有关。

可得农作物茎秆临界力的表达式:
\[
\begin{cases}
P_{cr} = \frac{8E}{\lambda^2} - \frac{Q}{b} \\
b = 6a^3\left(\frac{4}{3} - a + \frac{1}{5}a^2\right) \\
a = \frac{h'}{L}
\end{cases}
\]

式中,$l$——茎秆的高度,$h'$——穗位高,取为茎秆的高度,即 $h'=l$,即 $a=1$,$b=3.2$。

由上述分析可知,$P_{cr}$ 与 $Q$ 相比,$P_{cr}$ 为主要失稳因素,即有:
\[
P_{cr} \sim \frac{8E}{\lambda^2}
\]
则有茎秆自重系数 $\lambda$ 为:
\[
\lambda^2 = \frac{3.2 \cdot l^2}{J}
\]
\[
\lambda = \sqrt{\frac{3.2 \cdot l^2}{J}}
\]

由于作物品种不同,临界力不同,临界力越大,抗倒伏能量越强。同一作物,不同品种,$\lambda$ 越小,$P_{cr}$ 越大,抗倒伏能力越强。

(3) 风载临界力指标

由于附件三中的风载临界力中变量无法计算或计算很复杂,对此我们从风压和迎风面积着手,利用给定的参考附件,推出了一种新的风载临界力表达式:
\[
\begin{cases}
F = W \cdot S \\
W = \frac{1}{2} \rho v^2 = \frac{\gamma}{2g} v^2 = \frac{v^2}{1630} \, (kN/m^2) \\
S = 0.4841 + 0.2238 \cdot w
\end{cases}
\]
则可推导出风载临界力指标为:
\[
F = W \cdot S = \frac{v^2}{1630} \cdot (0.4841 + 0.2238 \cdot w)
\]
其中,$W$ 为风压,$v$ 为风速,$S$——穗头迎风面积,$\rho = 1.25 \, kg/m^3$ 为标准的空气密度,$\gamma = 0.012018 \, kN/m^3$ 为空气单位体积的重力,$g = 9.8 \, m/s^2$ 重力加速度。

(4) 惯性矩指标

对外圆直径为 $D$、小圆直径为 $d$ 的同心空心秆,其惯性矩 $J$ 为
\[
J = \frac{\pi}{64} (D^4 - d^4).
\]
进一步简化可得惯性矩指标为:
\[
J = \frac{\pi}{64} \big[b^4 - (b - 2d)^4\big]
\]

其中,$b$—茎秆外茎(粗),$d$—茎秆壁厚(mm)

\subsection*{4.4.2 数据指标的归一化处理}

由于所给的四项指标值具有不同的量纲,要得到综合的评价因子,首先必须通过一定方法对数据进行归一化处理。所以我们首先要对各项指标进行归一化处理。对此这里我们使用标准差方法对数据进行归一化处理:

\[
\begin{cases}
X_{ij}^{'} = \frac{X_{ij} - \overline{X}_{j}}{S_{j}} (i=1,2,\ldots,n; j=1,2,\ldots,m) \\
\overline{X}_{j} = \frac{1}{n} \sum_{i=1}^{n} X_{ij} \\
S_{j} = \left[ \frac{1}{n} \sum_{i=1}^{n} (X_{ij} - \overline{X}_{j})^2 \right]^{1/2} (j=1,2,\ldots,m)
\end{cases}
\]

其中,$X_{ij}^{'}$ ($i=1,2\cdots n, j=1,2\cdots m$) 的均值和均方差分别为 0 和 1,即 $X_{ij}^{'} \in [0,1]$ 是无量纲的,称之为 $X_{ij}$ 的标准观测值。

\subsection*{4.4.3 小麦茎秆抗倒伏性能评价指标权重的确定}

在评价小麦茎秆抗倒伏性能过程中,作比较、判断、评价分析时,小麦茎秆的弹性模量、茎秆自重系数、风载临界力和惯性矩指标的重要性(即通常所说的权重值问题),往往难以量化,用一般的数学方法解决有本质上的困难。

这里我们利用层次分析法(AHP)来确定各小麦茎秆抗倒伏性能指标的权重大小。这种方法把复杂问题中的各因素通过划分为相互联系的有序层次,使之条理化,并把数据、专家意见和分析者的主观判断直接而有效地结合起来,就每一层次的相对重要性给予定量表示,然后利用数学方法确定表达每一层次的相对重要性给予定量表示,然后利用数学方法确定表达每一层次全部要素的相对重要性权值。

\textbf{Step1:确定比例标度值}

我们在运用层次分析法(AHP)确定评价指标权重时,对所有因素进行两两相互对比,采用相对标度,以尽可能地减少性质不同的诸因素相互比较的困难,提高精度。为了易于确定 $U_{ij}$,在比较过程中,引入 1-9 标度法,其表示如下:

\begin{table}[h]
\centering
\caption{比例标度值及含义}
\begin{tabular}{|c|c|}
\hline
\textbf{定性结果} & \textbf{定量结果} \\
\hline
$B_{i}$ 与 $B_{j}$ 的影响相同 & $B_{i}:B_{j}=1:1$ \\
\hline
$B_{i}$ 比 $B_{j}$ 的影响稍强 & $B_{i}:B_{j}=3:1$ \\
\hline
$B_{i}$ 比 $B_{j}$ 的影响强 & $B_{i}:B_{j}=5:1$ \\
\hline
$B_{i}$ 比 $B_{j}$ 的影响明显强 & $B_{i}:B_{j}=7:1$ \\
\hline
$B_{i}$ 比 $B_{j}$ 的影响绝对强 & $B_{i}:B_{j}=9:1$ \\
\hline
$B_{i}$ 与 $B_{j}$ 的影响在上述两个等级之间 & $B_{i}:B_{j}=2,4,6,8:1$ \\
\hline
\end{tabular}
\end{table}

\begin{tabular}{|l|l|}
\hline
$B_{i}$ 与 $B_{j}$ 的影响和上述情况相反 & $B_{i}: B_{j}=1: 1, 2, \cdots, 9$ \\
\hline
\end{tabular}

\section*{Step2: 构造判断矩阵}

由于根据判断矩阵 P 评价因素重要性排序, 用方根法可求出最大特征根所对应的特征向量, 所求特征向量即为评价因素重要性排序, 也就是权数分配。其具体步骤如下, 其判断矩阵 P 为:

\[
P=\left[\begin{array}{cccc}
u_{11} & u_{12} & u_{13} & u_{14} \\
u_{21} & u_{22} & u_{23} & u_{24} \\
u_{31} & u_{32} & u_{33} & u_{34} \\
u_{41} & u_{42} & u_{43} & u_{44}
\end{array}\right]
\]

(1) 计算判断矩阵每一行元素的乘积 $U_{i}$

\[
U_{i}=\prod_{j=1}^{n} U_{i j} \quad(i, j=1,2,3,4)
\]

(2) 计算 $U_{i}$ 的 $n$ 次方根 $\bar{W}_{i}$:

\[
\bar{W}_{i}=\sqrt[n]{U_{i}}
\]

(3) 对向量 $\bar{W}=\left[\bar{W}_{1}, \bar{W}_{2}, \ldots \bar{W}_{n}\right]^{T}$ 作归一化处理, 即

\[
W_{i}=\bar{W}_{i} / \sum_{i=1}^{n} \bar{W}_{i}
\]

则所得的 $W=\left(W_{1}, W_{2}, \ldots W_{n}\right)^{T}$ 即为所求特征向量。

(4) 计算判断矩阵的最大特征根 $\lambda_{\max }$ :

\[
\lambda_{\max }=\frac{1}{n} \sum_{i=1}^{n} \frac{(P W)_{i}}{W_{i}}
\]

式中 $(P W)_{i}$ 表示向量 $P W$ 第 $i$ 个元素。

\[
P W=\left[\begin{array}{c}
(P W)_{1} \\
(P W)_{2} \\
(P W)_{3} \\
(P W)_{4}
\end{array}\right]=\left[\begin{array}{cccc}
u_{11} & u_{12} & u_{13} & u_{14} \\
u_{21} & u_{22} & u_{23} & u_{24} \\
u_{31} & u_{32} & u_{33} & u_{34} \\
u_{41} & u_{42} & u_{43} & u_{44}
\end{array}\right] \cdot\left[\begin{array}{c}
W_{1} \\
W_{2} \\
W_{3} \\
W_{4}
\end{array}\right]
\]

\section*{Step3: 分析计算小麦茎秆抗倒伏性能各评价指标权重}

根据题目表示出的严重性排序, 以及查阅相关资料, 可将这四个指标的影响程度大小进行了排序, 结果为: 弹性模量 > 茎秆自重系数 > 惯性矩 > 风载临界力。依据所述的判断矩阵标度法, 给小麦茎秆的四个指标构造判断矩阵 P 如下:

\[
P = \begin{bmatrix}
1 & 2 & 4 & 3 \\
\frac{1}{2} & 1 & 3 & 2 \\
\frac{1}{4} & \frac{1}{3} & 1 & \frac{1}{2} \\
\frac{1}{3} & \frac{1}{2} & 2 & 1
\end{bmatrix}
\]

\( u_1 \sim u_4 \) 依次表示弹性模量、茎秆自重系数、风载临界力、惯性矩。

由判断矩阵 \( P \) 可计算出权向量最大特征根 \( \lambda_{\max} \) 和其对应的向量 \( W^{(0)} = (W_1^{(0)}, W_2^{(0)}, W_3^{(0)}, W_4^{(0)}) \)。然后通过归一化,我们可得到标准权向量 \( W = (W_1, W_2, W_3, W_4) \),我们把该权向量定义为四个影响因素的类别权向量。

经计算得到弹性模量、茎秆自重系数、风载临界力、惯性矩的权重系数分别为 0.47,0.28,0.25,0.16。

\section*{Step4:一致性检验}

由于主观判断与客观理想之间存在偏差,因此需要对各比较判断矩阵进行一致性检验,检验构造的判断矩阵求出的特征向量(权值)是否合理。用一致性比例 \( CR \) 作为判断依据,\( CR \) 越小,表明判断矩阵的一致性越好,权重可接受性越强。计算公式为 \( CR = CI / RI \),其中 \( CI = (\lambda_{\max} - n) / (n - 1) \)(\( n \) 为判断矩阵阶数),\( RI \) 为判断矩阵的平均随机一致性指标,其值参见层次分析法(AHP)的平均随机一致性指标值。则有

\[
CR = 0.01
\]

由于,判断矩阵一致性比例小于 0.1,所以该判断矩阵具有良好的一致性,可以通过矩阵的相容性检验。因此,计算的权重是可以接受的。

\subsection*{4.4.4 综合评价指标函数的确定}

基于上述分析的同时,我们可建立其综合评价指标函数关系式为:

\[
Y = \omega_1 E + \omega_2 \lambda + \omega_3 F + \omega_4 J
\]

其中,\( Y \) 表示小麦茎秆抗倒伏性能,\( E \) 表示弹性模量,\( \lambda \) 表示茎秆自重系数,\( F \) 表示风载临界力,\( J \) 表示惯性矩。

\section*{4. 4. 5 小麦抗倒伏性能的综合评价模型的建立}

综合以上的指标分析、数据处理和权的求法的数学模型,我们可建立以下对小麦茎秆抗倒伏性能的综合评价模型:

\[
\left\{
\begin{aligned}
Y &= \omega_1 E + \omega_2 \lambda + \omega_3 F + \omega_4 J \\
E &= k \cdot f = 4792.95 \cdot f \\
\lambda &= \sqrt{\frac{3.2 \cdot l^2}{J}} \\
F &= W \cdot S \\
W &= \frac{1}{2} \rho v^2 = \frac{\gamma}{2g} v^2 = \frac{v^2}{1630} \, (kN/m^2) \\
S &= 0.4841 + 0.2238 \cdot w \\
J &= \frac{\pi}{64} \left[ b^4 - (b - 2d)^4 \right] \\
X_{ij}^{' } &= \frac{X_{ij} - \overline{X}_j}{s_j} \\
\overline{X}_j &= \frac{1}{n} \sum_{i=1}^n X_{ij} \\
s_j &= \left[ \frac{1}{n} \sum_{i=1}^n (X_{ij} - \overline{X}_j)^2 \right]^{1/2} \\
\lambda_{\text{max}} &= \frac{1}{n} \sum_{i=1}^n \frac{(PW)_i}{W_i}
\end{aligned}
\right.
\]

\section*{模型说明:}

- $Y$ 表示小麦茎秆抗倒伏性能;
- $E$ 表示弹性模量,
- $f$ 为弹性测定值;
- $\lambda$ 表示茎秆自重系数;
- $F$ 表示风载临界力;
- $J$ 表示惯性矩;
- $W$ 为风压, $\nu$ 表示风速;
- $S$ 表示穗头迎风面积; $s_j$ 为 $X_{ij}$ 对应标准差;
- $X_{ij}^{' }$ 是观测值 $X_{ij}$ 经归一化处理后得到的标准观测值;
- $\lambda_{\text{max}}$ 为权向量对应的最大特征根, $(PW)_i$ 表示向量 $PW$ 第 $i$ 个元素;

\section*{4.5 问题五模型的建立与求解}

\subsection*{4.5.1 问题的分析}

本问题要求我们根据问题四中建立的小麦茎秆抗倒伏模型,运用抗弯刚度 $EJ$ 公式以及麦穗自重下、风载作用下的公式对腊熟期各品种小麦的抗倒伏风速进行计算。

为此我们假设风力仅对单穗穗头起作用,利用风压的贝努利公式,将风速转化为风压,同时把穗头迎风面积与已知的单穗鲜重联系,利用风压和穗头迎风面积的乘积,得到风速对单穗穗头作用力的关系。最后根据压杆弯曲的临界力等于小麦茎秆抗倒伏作用力,求出各品种的抗倒伏风速。

\subsection*{4.5.2 模型的建立}

\subsubsection{(1) 风速与风载作用力关系的建立}

由问题四中风载临界力指标的确定与以上问题分析,我们可得到风速与风载作用力的关系为:
\[
\begin{cases}
F = W \cdot S \\
W = \frac{1}{2} \rho v^2 = \frac{\gamma}{2g} v^2 = \frac{v^2}{1630} \, (kN/m^2) \\
S = 0.4841 + 0.2238 \cdot w
\end{cases}
\]
式中 $W$ ——风压(Pa),$S$ ——穗头迎风面积($cm^2$)

其中① $W = \frac{1}{2} \rho v^2 = \frac{\gamma}{2g} v^2 = \frac{v^2}{1630} \, (kN/m^2)$

式中 $v$ ——风速,$\rho = 1.25 \, kg/m^3$ 为标准的空气密度,$\gamma = 0.012018 \, kN/m^3$ 为空气单位体积的重力,$g = 9.8 \, m/s^2$ 重力加速度。

② 穗头迎风面积 $S = 0.4841 + 0.2238 \cdot w$

代值求解可得到风速与风载作用力的关系具体为:
\[
F = W \cdot S = \frac{v^2}{1630} \cdot (0.4841 + 0.2238 \cdot w)
\]

\subsubsection{(2) 风速与小麦抗倒伏性能间关系的建立}

由问题四的分析及风速与风载作用力的关系可知,风速与小麦抗倒伏性能间关系为:

\begin{equation}
\left\{
\begin{aligned}
Y &= \omega_1 E + \omega_2 \lambda + \omega_3 F + \omega_4 J \\
E &= k \cdot f = 4792.95 \cdot f \\
\lambda &= \sqrt{\frac{3.2 \cdot l^2}{J}} \\
F &= W \cdot S = \frac{v^2}{1630} \cdot (0.4841 + 0.2238 \cdot w) \\
J &= \frac{\pi}{64} \left[ b^4 - (b - 2d)^4 \right]
\end{aligned}
\right.
\end{equation}

其中,$Y$ 表示小麦茎秆抗倒伏性能,$E$ 表示弹性模量,$\lambda$ 表示茎秆自重系数,$F$ 表示风载临界力,$J$ 表示惯性矩。

(3) 压杆弯曲的临界力的确定

根据题中给出的附件三及题设可知,将两头受与杆反向力作用的受压长杆视为轴向载荷的柱。设柱在固定点 $x = l$($l$——茎秆长度)处以铰链连接,在 $x = 0$ 处有一支承防止侧向变位,但允许自由转动及轴向挠曲,柱有一轴向载荷 $P$。

当柱的变位为 $w(x)$ 时 $P$ 产生的弯矩为 $P \cdot w$,于是可得二阶方程 $EJ \frac{d^2 w}{dx^2} = Pw$。

令 $\lambda = \frac{P}{EJ}$,方程成为 $w'' = \lambda w$,其边值条件为 $w(0) = w(l) = 0$。边值问题成为特征值问题,边值问题当且仅当 $\lambda = \frac{n^2 \pi^2}{l^2}$ ($n = \pm 1, \pm 2, \cdots$) 时有非零解

$$
w(x) = c \sin \left( \frac{n \pi}{l} x \right), \quad (c \neq 0, n = \pm 1, \pm 2, \cdots).
$$

因此,欧拉问题的解为临界力取 $P = EJ \lambda = EJ \frac{n^2 \pi^2}{l^2}$ ($n = \pm 1, \pm 2, \cdots$),而最小临界力为 $P = EJ \frac{\pi^2}{l^2}$。在临界力作用下杆会弯曲但两端保持不变。

则压杆弯曲的最小临界力为:

$$
P = EJ \frac{\pi^2}{l^2}
$$

(4) 抗倒伏风速模型的建立

由小麦抗倒伏性能与压杆弯曲的最小临界力的相互关系,并结合以上分析,可建立如下抗倒伏风速模型:

\begin{equation}
\left\{
\begin{aligned}
Y &= P \\
P &= EJ \frac{\pi^2}{l^2} \\
Y &= \omega_1 E + \omega_2 \lambda + \omega_3 F + \omega_4 J \\
E &= k \cdot f = 4792.95 \cdot f \\
\lambda &= \sqrt{\frac{3.2 \cdot l^2}{J}} \\
F &= W \cdot S = \frac{v^2}{1630} \cdot (0.4841 + 0.2238 \cdot w) \\
J &= \frac{\pi}{64} \left[ b^4 - (b - 2d)^4 \right]
\end{aligned}
\right.
\end{equation}

模型说明:
- $Y$ 表示小麦茎秆抗倒伏性能,$P$ 表示压杆弯曲的最小临界力,
- $E$ 表示弹性模量,$\lambda$ 表示茎秆自重系数,
- $F$ 表示风载临界力,$J$ 表示惯性矩。

\subsection*{4.5.3 模型的求解}

\textbf{Step1:求解 2007 年蜡熟期各品种的惯性矩}

惯性矩:
\[
J = \frac{\pi}{64} \left[ b^4 - (b - 2d)^4 \right]
\]
式中 $b$ ——茎秆外径(粗),$d$ ——茎秆壁厚(mm)

可得其惯性矩如下:

\begin{table}[h]
\centering
\caption{惯性矩计算表}
\begin{tabular}{|c|c|c|c|c|c|c|c|}
\hline
品种 & 穗下第一节 & 穗下第二节 & 穗下第三节 & 穗下第四节 & 平均粗度 & 平均壁厚 & 惯性矩 \\
\hline
58-1 & 1.48 & 2.79 & 2.78 & 2.70 & 1.76 & 0.48 & 1.58 \\
58-2 & 1.55 & 2.82 & 2.91 & 2.93 & 2.55 & 0.52 & 1.89 \\
周18 & 1.52 & 2.43 & 2.42 & 2.28 & 2.16 & 0.4 & 0.97 \\
新19 & 1.39 & 2.50 & 2.69 & 2.36 & 2.23 & 0.3 & 0.88 \\
\hline
\end{tabular}
\end{table}

\textbf{Step2:求解 2007 年蜡熟期各品种的弹性模量}

弹性模量:
\[
E = k \cdot f = 4792.95 \cdot f
\]

$f$ ——弹性测定平均值,$k$ ——弹性测定值与弹性模量之间的比例关系,$E$ ——茎秆弹性模量(GPa)

可得其弹性模量如下:
\begin{align*}
E_{\text{矮抗58-1}} &= 4792.95 \times 0.92 = 4409.514 \, \text{(GPa)} \\
E_{\text{矮抗58-2}} &= 4792.95 \times 0.95 = 4553.3025 \, \text{(GPa)} \\
E_{\text{周18}} &= 4792.95 \times 0.95 = 4553.3025 \, \text{(GPa)} \\
E_{\text{新19}} &= 4792.95 \times 0.9 = 4313.655 \, \text{(GPa)}
\end{align*}

Step 3:求解 2007 年蜡熟期各品种的茎秆自重系数

茎秆自重系数:
\[
\lambda = \sqrt{\frac{3.2 \cdot l^2}{J}}
\]

式中 $l$ ——茎秆的高度,$b = 6a^3 \left( \frac{4}{3} - a + \frac{1}{5}a^2 \right)$,$a = \frac{h'}{L}$,$l$ ——茎秆的高度,$h'$ ——穗位高,取为茎秆的高度,即 $h = l$,即 $a = 1$,$b = 3.2$

可得其茎秆自重系数如下:

\begin{tabular}{|c|c|c|c|c|c|c|}
\hline
品种 & 基部第一节长 & 基部第二节长 & 基部第三节长 & 基部第四节长 & 茎秆长度 & 惯性矩 & 茎秆自重系数 \\
\hline
58-1 & 21.36 & 16.21 & 10.18 & 9.36 & 57.11 & 1.58 & 81.37 \\
58-2 & 25.73 & 14.10 & 9.47 & 9.05 & 58.35 & 1.89 & 75.83 \\
周18 & 29.77 & 17.35 & 10.85 & 8.95 & 66.92 & 0.97 & 121.48 \\
新19 & 25.00 & 17.44 & 11.35 & 9.16 & 62.95 & 0.88 & 119.72 \\
\hline
\end{tabular}

Step 4:求解 2007 年蜡熟期各品种的最小压杆弯曲的临界力

\[
P = EJ \frac{\pi^2}{l^2}
\]

可得其最小压杆弯曲的临界力如下:

\begin{tabular}{|c|c|c|c|c|c|c|c|}
\hline
品种 & 基部第一节长 & 基部第二节长 & 基部第三节长 & 基部第四节长 & 茎秆长度 & 惯性矩 & 弹性模量 & 临界力 \\
\hline
58-1 & 21.36 & 16.21 & 10.18 & 9.36 & 57.11 & 1.58 & 4409.51 & 21.03 \\
58-2 & 25.73 & 14.10 & 9.47 & 9.05 & 58.35 & 1.89 & 4553.30 & 25.01 \\
周18 & 29.77 & 17.35 & 10.85 & 8.95 & 66.92 & 0.97 & 4553.30 & 9.74 \\
新19 & 25.00 & 17.44 & 11.35 & 9.16 & 62.95 & 0.88 & 4313.66 & 9.51 \\
\hline
\end{tabular}

Step5:求解 2007 年蜡熟期各品种的风载作用力

\[
F = W \cdot S = \frac{v^2}{1630} \cdot (0.4841 + 0.2238 \cdot w)
\]

可得其风载作用力为:

\[
F_{\text{矮抗58-1}} = \frac{2.287928v^2}{1630}
\]

\[
F_{\text{矮抗58-2}} = \frac{2.395352v^2}{1630}
\]

\[
F_{\text{周18}} = \frac{1.992512v^2}{1630}
\]

\[
F_{\text{新19}} = \frac{2.191694v^2}{1630}
\]

Step6:抗倒伏风速求解结果

由问题四模型 \( Y = \omega_1 \times J + \omega_2 \times E + \omega_3 \times \lambda + \omega_4 \times F \),其中 \(\omega_1 = 0.16\)、\(\omega_2 = 0.47\)、\(\omega_3 = 0.28\)、\(\omega_4 = 0.25\)

故

\[
Y = 0.16J + 0.47E + 0.28\lambda + 0.25F
\]

于是可得其抗倒伏性能为:

\[
Y_{\text{矮抗58-1}} = 4.60 + \frac{5.72v^2}{163}
\]

\[
Y_{\text{矮抗58-2}} = 4.57 + \frac{6.58v^2}{163}
\]

\[
Y_{\text{周18}} = 5.70 + \frac{4.98v^2}{163}
\]

\[
Y_{\text{新19}} = 5.52 + \frac{5.48v^2}{163}
\]

代入 \( Y = P \left( = EJ \frac{\pi^2}{l^2} \right) \),即可以求得风速,其求解结果为:

矮抗 58-1 抗倒伏风速为 21.64m/s,为 9 级风;

矮抗 58-2 抗倒伏风速为 22.50 m/s,为 9 级风;

周 18 抗倒伏风速为 15.50 m/s,为 7 级风;

新 19 抗倒伏风速为 14.89m/s,为 7 级风;

\section*{4.6 问题六}

\subsection*{4.6.1 模型总结}

本文研究了小麦发育后期茎秆抗倒性的数学模型,分别通过对实际测量数据(株高、穗长、各节间长、节间长度比、各节壁厚、穗重、鲜重等茎秆性状)的处理,以及将茎秆按刚/弹性材料处理,建立了小麦茎秆抗倒伏的数学模型,并对小麦从株型选择、麦穗自重和风载作用下的基本规律进行了研究。

实际测量数据处理建立的模型。由于附件中收集的数据并不完整有缺漏,各年参数的选取也不一致,我们对数据进行了分析、筛选与处理,建立了各品种小麦的抗倒伏性指数公式,用于判断不同品种小麦茎秆抗倒伏的能力。根据题中提供的抗倒伏指数公式,对于其所需数据齐全的年份和品种,我们可直接对其求解,例如 07 年三个品种所提供的数据比较完整,就可以直接得出该年三个品种的抗倒伏指数公式;然而对于有关参数缺失或未知的年份,我们就得利用已有数据来间接确定其未知的参数,例如 11 年缺失机械强度的相关数据,我们就可设想利用 07 年已知的机械强度数据进行线性拟合,推导出茎秆机械强度的回归模型,然后间接求 11 年各品种的抗倒伏指数公式;对于 08 年,由于 08 年的茎秆鲜重、重心高度和机械强度均未知,则得先利用 11 年的第二组数据拟合出穗干重与穗鲜重的关系模型,再结合 11 年第一组数据拟合得到茎秆鲜重和重心高度的回归模型,然后间接求 08 年各品种的抗倒伏指数公式。

茎秆按刚/弹性材料处理建立的模型。为了对小麦抗倒伏性进行评价分析时,通过科学性原则、突出主要指标原则、独立性原则,并结合问题中要求考虑小麦茎秆在麦穗自重和风载作用下应力的基本规律,构建了小麦茎秆抗倒伏性能的评价指标体系,包括弹性模量指标、茎秆自重系数指标、风载临界力指标、惯性矩指标。由于所给的四项指标值具有不同的量纲,本文使用标准差方法对数据进行归一化处理,并利用层次分析法(AHP)来确定各小麦茎秆抗倒伏性能指标的权重大小,进而建立小麦茎秆抗倒伏性能的综合评价模型。

\subsection*{4.6.2 结果分析}

根据所建模型以及抗倒性状的选择对形态及产量性状的影响,得到以下结果:

(1) 在茎秆形态性状的选择上,选择茎高较低的小麦株型将会使重心高度、机械强度、倒伏指数降低;

(2) 选择单株产量有可能使重心高度下降,机械强度升高,倒伏指数下降,茎秆密度增加,但是茎秆节间的单位长度干重将会降低,因此,选择单株产量较高的株系将会提高茎秆的抗倒伏性能。

\subsection*{4.6.3 值得考虑的问题提出}

(1) 由于小麦茎秆具有各项异性的物理性质,因而,对于同一种作物茎秆的生物力学性质(材料弹性模量、抗弯刚度、抗弯强度等),今后可考虑茎秆各生物力学指标在这些因素影响下的变化趋势。

(2) 运用相关性分析时,相关性系数的取值对关联度有明显影响。因此,根据育种目标的不同需要,相关性系数取值应随着适当调整,这有待于今后实践中进一步探索。另外,不同的评价角度选取相应的考查指标将产生不同的评价效果。

(3) 目前,对茎秆力学性质的试验研究还没有比较成熟和普遍适用的方法和设备,无法满足茎秆加工过程中的工况条件。大量研究表明,不同载荷形式作用于小麦茎秆时,茎秆的断裂机理不同,进一步探讨冲击载荷下茎秆的加工特性以及茎秆在高应变率条件下的力学性质是一种较新的研究思路。

\subsection{4.6.4 试验方案及数据分析方法}

由于小麦茎秆具有各项异性的物理性质,对同一种作物茎秆的生物力学性质(材料弹性模量、抗弯刚度、抗弯强度及剪切强度等)受茎秆含水率、茎秆外部形态、内部组织结构及田间管理等多种因素的综合影响,2012年实验时,我们不必仅仅局限于以茎秆的抗倒伏指数衡量各品种的优劣,而可以考虑更多的内外部因素来对品种的优劣进行综合评价。在此暂提出几个衡量因素:小麦的抗寒性,成穗率,抗病性,对水分和温度变化的适应性,品质优势等。

在2012的实验方案中,由于需要同时考虑几个衡量因素进行综合评价,首先固定其他因素而使某一因素变化,通过单变量变化的实验方案可得到各个衡量因素对小麦的影响。然后通过相关性分析可得到不同因素的加权系数,建立衡量小麦品种优劣的总模型。以上得到的是针对单一品种的模型,通过同样的方法可以得到不同品种的小麦的加权模型,通过纵向小麦品种模型优劣的比较,可以得到衡量指标比较全面的小麦优势品种,对于农民的选种、播种有一定的指导意义。

其中数据方法涉及到:数据的筛选与预处理、数据的回归分析与检验、数据的相关性分析

\subsection{4.6.5 总结及建议}

(1) 结合小麦的育种栽培,以试验测定(主要是物理机械方法,运用现代测试技术)、计算机处理为主,研究了小麦茎秆的常规力学性质(材料弹性模量、抗弯刚度、抗弯强度及剪切强度等),获取实际需要的各项生物力学指标值。

(2) 通过试验,得出了不同生长期内、不同节间小麦茎秆惯性矩、弹性模量、抗弯刚度、剪切强度等各生物力学指标的变化趋势。

(3) 依据评价指标的确定原则,并通过咨询、调查,确立了有关小麦茎秆弯折性能的四个评价指标,分别为茎秆弹性模量、茎秆自重系数、惯性矩、风载临界力。考虑这四个指标对小麦产量、茎秆弯折性能及其茎秆抗倒伏性的综合影响程度,对四指标的重要性进行了排序,结果为弹性模量 > 茎秆自重系数 > 风载临界力 > 惯性矩,并采用层次分析法确定了这四个弯折力学性能指标的权重系数大小为 0.47,0.28,0.25,0.16。

(5) 运用有关小麦茎秆弯折性能各生物力学指标,对小麦的倒伏特性进行定量评价分析,指出矮抗58的抗倒伏性较其他两品种好。今后,在小麦优种选育中就抗倒性来说,可考虑选用株高适宜、茎秆基部1、2节相对较短较粗的品种。

\section{五、模型的评价与推广}

\subsubsection{(1) 模型的评价}

在第一问中,建立了一个更优化的抗倒伏指数的总模型和三个具体的子模型,计算简便,并在建立回归模型后对模型进行了检验,验证了模型的可靠性;在问题二中,我们先利用了07年的数据对最易引起倒伏的时期进行了判断与确定;在问题三中,我们引入了经济系数因素,建立了双目标优化模型并利用了序贯式算法来转化为单目标;在问题四中,运用了层次分析法(AHP)结合MATLAB软件来确定各评价指标的权重,且对其一致性进行了检验。

\subsubsection{(2) 模型推广}

对于问题中,我们进行了很多假设,简化了问题的复杂化,但也过于理想化,如在计算风载力和茎秆自重系数时,若有更全的数据和相关资料,将能更好

\section*{六、参考文献}

[1] 袁志华, 冯宝萍, 赵安庆, 梁爱琴, 《作物茎杆抗倒伏的力学分析及综合评价探讨》, 农业工程学报, 第 18 卷第 6 期: 30-31, 2002 年

[2] 袁红梅, 《小麦茎秆生物力学性能试验与抗倒伏力学评价分析》, 山西农业大学, 2005 年

[3] 韩中庚, 《数学建模方法及其应用》, 北京, 高等教育出版社, 2005 年

[4] 章文波 陈红艳, 《实用数据统计分析及 SPSS12.0 应用》, 北京, 人民邮电出版社, 2006 年

[5] 徐瑞、黄兆东等, 《MATLAB 2007 科学计算与工程分析》, 北京, 科学出版社, 2008 年

[6] 徐芝纶, 《弹性力学》, 北京, 高等教育出版社, 1992 年

[7] 刘鸿文, 《材料力学》(第三版), 北京, 高等教育出版社, 1999 年

\section*{附录清单:}

附录一: 茎秆弹性测定数据

附录二: 抗倒伏指数与茎秆外部形态特性的相关性数据表

附录三: 茎秆鲜重的回归模型检验图

附录四: 抗倒伏指数与茎秆外部形态特征的相关性分析图

附录五: 问题三的 Lingo 求解程序

附录六: 力学资料和相关力学公式的说明

\section*{附录一: 茎秆弹性测定数据}

1、2007年5月25日用物性测定仪两次测定茎秆各节间的弹性图及对应的测定值之比得到的弹性测定值(腊熟期):

\textbf{表1 弹性测定值(表中空白处表示测量样品茎杆没有这一节)}

\begin{tabular}{|c|c|c|c|c|c|c|}
\hline 品种 & 基部第一节 & 基部第二节 & 基部第三节 & 基部第四节 & 基部第五节 & 基部第六节 \\
\hline 58-1 & 0.908 & 0.877 & 0.961 & 0.906 & 0.961 & \\
\hline 58-2 & 0.911 & 0.981 & 0.966 & 0.949 & 0.964 & \\
\hline 周18 & 0.966 & 0.928 & 0.933 & 0.961 & 0.958 & 0.946 \\
\hline 新19 & 0.761 & 0.912 & 0.924 & 0.988 & 0.913 & \\
\hline 温6 & 0.982 & 0.826 & 0.909 & 0.891 & 0.959 & 0.957 \\
\hline
\end{tabular}

2. 弹性测定值与弹性模量的关系

弹性测定值由物性测定仪测定,而其他文献测定弹性模量是在 INSTRON 5544 材料性能试验机或更先进的使用力传感器和位移传感器的微机控制电子万能试验机上进行,用三点弯曲法通过力与位移的测量计算弹性模量。由物性测定仪测定的弹性测定值转为弹性模量值是个值得探讨的问题。

现为简单起见,可假定弹性测定值和弹性模量值为线性关系。以腊熟期计,由弹性测定值的最小0.761至最大0.988可对应于弹性模量值的最小355至最大1443,其节与节之间的弹性测定值、弹性模量值以线性关系大小对应。自然,亦可提出其他更合理的对应关系。

\begin{table}
\centering
\begin{tabular}{|c|c|c|c|c|c|c|}
\hline
 & 各品种抗倒 & 穗下第一节 & 穗下第二节 & 穗下第三节 & 穗下第四节 & 株高(cm) \\
 & 伏指数 & 长(cm) & 长(cm) & 长(cm) & 长(cm) & \\
\hline
新麦208 & 1.20191 & 25.00 & 17.44 & 11.35 & 9.16 & 62.95 \\
\hline
国信 & 1.01402 & 32.91 & 19.21 & 11.62 & 7.19 & 70.93 \\
\hline
智9998早 & 1.02105 & 25.92 & 15.63 & 10.58 & 7.67 & 59.80 \\
\hline
联丰85 & 1.46961 & 25.88 & 13.67 & 8.64 & 7.24 & 55.43 \\
\hline
平安6号 & 0.86644 & 24.88 & 17.04 & 7.51 & 7.33 & 56.77 \\
\hline
温麦6号 & 1.27766 & 26.06 & 16.02 & 8.47 & 6.82 & 57.37 \\
\hline
郑麦9023 & 0.48746 & 30.54 & 17.78 & 9.91 & 7.13 & 65.35 \\
\hline
周麦22 & 0.70808 & 26.02 & 16.32 & 8.71 & 6.33 & 57.37 \\
\hline
预麦18 & 0.61877 & 28.81 & 15.42 & 7.74 & 6.40 & 58.37 \\
\hline
周麦18 & 0.79486 & 28.60 & 17.36 & 9.80 & 7.45 & 63.20 \\
\hline
矮抗58 & 0.90980 & 23.15 & 15.03 & 8.71 & 7.62 & 54.52 \\
\hline
\end{tabular}
\end{table}

\begin{table}
\centering
\begin{tabular}{|c|c|c|c|c|c|c|c|}
\hline
 & 第一节长度 & 第二节长度 & 第三节长度 & 第一节 & 第二节 & 第三节 & 第四节 \\
 & /第四节长度 & /第四节长度 & /第四节长度 & 壁厚 & 壁厚 & 壁厚 & 壁厚 \\
 & & & & (mm) & (mm) & (mm) & (mm) \\
\hline
新麦208 & 2.729 & 1.903 & 1.239 & 0.30 & 0.30 & 0.30 & 0.30 \\
\hline
国信 & 4.577 & 2.672 & 1.616 & 0.25 & 0.33 & 0.38 & 0.45 \\
\hline
智9998早 & 3.379 & 2.038 & 1.379 & 0.35 & 0.43 & 0.45 & 0.43 \\
\hline
联丰85 & 3.575 & 1.888 & 1.193 & 0.38 & 0.50 & 0.58 & 0.63 \\
\hline
平安6号 & 3.393 & 2.324 & 1.024 & 0.51 & 0.56 & 0.71 & 0.71 \\
\hline
温麦6号 & 3.821 & 2.349 & 1.241 & 0.46 & 0.51 & 0.61 & 0.68 \\
\hline
郑麦9023 & 4.284 & 2.494 & 1.390 & 0.41 & 0.43 & 0.46 & 0.51 \\
\hline
周麦22 & 4.112 & 2.579 & 1.376 & 0.42 & 0.50 & 0.58 & 0.55 \\
\hline
预麦18 & 4.502 & 2.410 & 1.209 & 0.44 & 0.51 & 0.84 & 0.73 \\
\hline
周麦18 & 3.968 & 2.428 & 1.341 & 0.42 & 0.45 & 0.50 & 0.55 \\
\hline
矮抗58 & 3.104 & 2.018 & 1.155 & 0.43 & 0.50 & 0.51 & 0.55 \\
\hline
\end{tabular}
\end{table}

\begin{table}
\centering
\begin{tabular}{|c|c|c|c|}
\hline
 & 穗长(cm) & 穗重(g) & 茎秆鲜重(g) \\
\hline
新麦208 & & & 7.63 \\
\hline
国信 & 8.74 & 3.07 & 6.61 \\
\hline
智9998早 & 8.30 & 3.61 & 7.01 \\
\hline
联丰85 & 9.13 & 2.64 & 6.16 \\
\hline
平安6号 & 9.17 & 3.07 & 6.24 \\
\hline
温麦6号 & 8.60 & & \\
\hline
郑麦9023 & 9.63 & 2.59 & 4.13 \\
\hline
周麦22 & 9.42 & 3.64 & 5.45 \\
\hline
\end{tabular}
\end{table}

\section*{附录三:茎秆鲜重的回归模型检验}

\textbf{拟合优度检验}

\begin{tabular}{|c|c|c|c|c|}
\hline Mode & R & R Square & Adjusted R Square & Std. Error of the Estimate \\
\hline 1 & .922$^{*}$ & .850 & .834 & 3.35620 \\
\hline
\end{tabular}

\textbf{显著性F与sig的检验}

\begin{tabular}{|c|c|c|c|c|c|}
\hline Model & Sum of Squares & df & Mean Square & F & Sig. \\
\hline 1 Regression & 334.141 & 4 & 83.535 & 245.050 & .000$^{*}$ \\
Residual & 14.658 & 43 & .341 & & \\
Total & 348.800 & 47 & & & \\
\hline
\end{tabular}

\textbf{残差检验}

\textbf{Dependent Variable: 茎秆鲜重G}

\begin{figure}[h]
\centering
\includegraphics[width=0.8\textwidth]{image.png}
\caption{Observed Cum Prob}
\end{figure}

\section*{附录四:抗倒伏指数与茎秆外部形态特征的相关性分析}

\textbf{抗倒伏指数与各节间长的相关性分析}

\begin{tabular}{|c|c|c|c|c|c|}
\hline & 抗倒伏指数 & 穗下第一节长 L1 & 穗下第二节长 L2 & 穗下第三节长 L3 & 穗下第四节长 L4 \\
\hline 抗倒伏指数 Pearson Correlation & 1 & -.349 & -.348 & .159 & .413 \\
Sig. (2-tailed) & & .293 & .294 & .640 & .207 \\
\hline 穗下第一节长 L1 Pearson Correlation & -.349 & 1 & .623 & .411 & -.327 \\
Sig. (2-tailed) & .293 & & .041 & .209 & .327 \\
\hline 穗下第二节长 L2 Pearson Correlation & -.348 & .623 & 1 & .578 & .166 \\
Sig. (2-tailed) & .294 & .041 & & .062 & .626 \\
\hline 穗下第三节长 L3 Pearson Correlation & .159 & .411 & .578 & 1 & .578 \\
Sig. (2-tailed) & .640 & .209 & .062 & & .062 \\
\hline 穗下第四节长 L4 Pearson Correlation & .413 & -.327 & .166 & .578 & 1 \\
Sig. (2-tailed) & .207 & .327 & .626 & .062 & \\
\hline
\end{tabular}

\section*{抗倒伏指数与节间长度比的相关性分析}

\begin{table}
\centering
\begin{tabular}{|c|c|c|c|c|}
\hline
 & 抗倒伏指数 & L1比L4 & L2比L4 & L3比L4 \\
\hline
抗倒伏指数 Pearson Correlation Sig.(2-tailed) & 1 & -.497 .120 & -.518 .043 & -.166 .626 \\
\hline
L1比L4 Pearson Correlation Sig.(2-tailed) & -.497 .120 & 1 & .849 .001 & .570 .067 \\
\hline
L2比L4 Pearson Correlation Sig.(2-tailed) & -.518 .043 & .849 .001 & 1 & .539 .087 \\
\hline
L3比L4 Pearson Correlation Sig.(2-tailed) & -.166 .626 & .570 .067 & .539 .087 & 1 \\
\hline
\end{tabular}
\end{table}

\section*{抗倒伏指数与各节壁厚的相关性分析}

\begin{table}
\centering
\begin{tabular}{|c|c|c|c|c|c|}
\hline
 & 抗倒伏指数 & 第一节壁厚d1 & 第二节壁厚d2 & 第三节壁厚d3 & 第四节壁厚d4 \\
\hline
抗倒伏指数 Pearson Correlation Sig.(2-tailed) & 1 & -.296 .376 & -.171 .615 & -.261 .438 & -.129 .705 \\
\hline
第一节壁厚d1 Pearson Correlation Sig.(2-tailed) & -.296 .376 & 1 & .914 .000 & .774 .005 & .803 .003 \\
\hline
第二节壁厚d2 Pearson Correlation Sig.(2-tailed) & -.171 .615 & .914 .000 & 1 & .835 .001 & .884 .000 \\
\hline
第三节壁厚d3 Pearson Correlation Sig.(2-tailed) & -.261 .438 & .774 .005 & .835 .001 & 1 & .936 .000 \\
\hline
第四节壁厚d4 Pearson Correlation Sig.(2-tailed) & -.129 .705 & .803 .003 & .884 .000 & .936 .000 & 1 \\
\hline
\end{tabular}
\end{table}

\section*{抗倒伏指数与小麦茎秆各因素间的相关性分析}

\begin{table}
\centering
\begin{tabular}{|c|c|c|c|c|c|c|c|c|}
\hline
 & 抗倒伏指数 & 穗下第四节长L4 & L2比L4 & 第一节壁厚d1 & 株高 & 穗长 & 穗重 & 茎秆鲜重 \\
\hline
抗倒伏指数 Pearson Correlation Sig.(2-tailed) & 1 & .522 .288 & -.656 .157 & -.450 .370 & .772 .072 & -.499 .313 & .371 .974 & .831 .040 \\
\hline
穗下第四节长L4 Pearson Correlation Sig.(2-tailed) & .522 .288 & 1 & -.725 .103 & .228 .663 & .089 .867 & -.203 .700 & -.398 .434 & .342 .507 \\
\hline
L2比L4 Pearson Correlation Sig.(2-tailed) & -.656 .157 & -.725 .103 & 1 & -.272 .602 & .611 .197 & .511 .197 & .732 .098 & .652 .161 \\
\hline
第一节壁厚d1 Pearson Correlation Sig.(2-tailed) & -.450 .370 & .228 .663 & -.272 .602 & 1 & -.473 .343 & -.473 .343 & .155 .769 & .178 .736 \\
\hline
株高 Pearson Correlation Sig.(2-tailed) & .772 .072 & .089 .867 & .611 .197 & -.473 .343 & 1 & .607 .202 & .033 .951 & -.454 .366 \\
\hline
穗长 Pearson Correlation Sig.(2-tailed) & -.499 .313 & -.203 .700 & .732 .098 & -.053 .921 & .607 .202 & 1 & .745 .089 & -.505 .307 \\
\hline
穗重 Pearson Correlation Sig.(2-tailed) & .371 .974 & -.398 .434 & .652 .161 & .155 .769 & .033 .951 & .745 .089 & 1 & -.066 .901 \\
\hline
茎秆鲜重 Pearson Correlation Sig.(2-tailed) & .831 .040 & .342 .507 & -.364 .478 & .178 .736 & -.454 .366 & -.505 .307 & -.066 .901 & 1 \\
\hline
\end{tabular}
\end{table}

\begin{table}
\centering
\caption{小麦茎秆性状各因素间的相关性分析}
\begin{tabular}{c|c|c|c|c|c|c|c}
\hline
 & 穗下第四节长L4 & L2比L4 & 第一节壁厚d1 & 株高 & 穗长 & 穗重 & 茎秆鲜重 \\
\hline
穗下第四节长L4 & \begin{tabular}{l} Pearson Correlation \\ Sig.(2-tailed) \end{tabular} & 1 & -.725 & .228 & -.089 & -.203 & -.398 \\
 & & & .103 & .663 & .867 & .700 & .434 \\
\hline
L2比L4 & \begin{tabular}{l} Pearson Correlation \\ Sig.(2-tailed) \end{tabular} & -.725 & 1 & -.272 & .611 & .732 & .652 \\
 & & .103 & & .602 & .197 & .098 & .161 \\
\hline
第一节壁厚d1 & \begin{tabular}{l} Pearson Correlation \\ Sig.(2-tailed) \end{tabular} & .228 & -.272 & 1 & -.473 & -.053 & .155 \\
 & & .663 & .602 & & .343 & .921 & .769 \\
\hline
株高 & \begin{tabular}{l} Pearson Correlation \\ Sig.(2-tailed) \end{tabular} & -.089 & .611 & -.473 & 1 & .607 & .033 \\
 & & .867 & .197 & .343 & & .202 & .951 \\
\hline
穗长 & \begin{tabular}{l} Pearson Correlation \\ Sig.(2-tailed) \end{tabular} & -.203 & .732 & -.053 & .607 & 1 & .745 \\
 & & .700 & .098 & .921 & .202 & & .089 \\
\hline
穗重 & \begin{tabular}{l} Pearson Correlation \\ Sig.(2-tailed) \end{tabular} & -.398 & .652 & .155 & .033 & .745 & 1 \\
 & & .434 & .161 & .769 & .951 & .089 & \\
\hline
茎秆鲜重 & \begin{tabular}{l} Pearson Correlation \\ Sig.(2-tailed) \end{tabular} & .342 & -.364 & .178 & -.454 & -.505 & -.066 \\
 & & .507 & .478 & .736 & .366 & .307 & .901 \\
\hline
\end{tabular}
\end{table}

\section*{附录五:问题三的Lingo求解程序}

\subsection*{第一级模型min I的求解程序}
\begin{verbatim}
model:
min=I;
I=G*h/P;
G=0.116*(l1+l2+l3+l4)-0.463/4*(b1+b2+b3+b4)+12.933/4*(d1+d2+d3+d4)+1.
326*w-8.337;
P=0.014*14-0.168*b4-0.308*d+1.083;
h=0.459*(l1+l2+l3+l4)-2.998/4*(b1+b2+b3+b4)-30.506/4*(d1+d2+d3+d4)+6.
336*w+28.425;
w0=1.19;
w=-8.152*w0^2+20.136*w0-8.8;
14<13;
13<11;
11<12;
d4<d2;
d2<d3;
d3<d1;
d1<14;
b1<b2;
b2<b3;
b3<b4;
b4<14;
h>14;
G>w0;
G>w;
end
\end{verbatim}

\subsection*{第二级模型Z的求解程序}
\begin{verbatim}
model:
max=Pmax*0.443/(y*x);
\end{verbatim}

\begin{verbatim}
Pmax >= 500;
Pmax <= 700;
y >= 38;
y <= 56;
x >= 0.97;
x <= 2.92;
y * x / Pmax < 0.5;
end
\end{verbatim}

\section*{附录六:力学资料和相关力学公式的说明}

(1) 空心杆惯性矩与三点弯曲测弹性模量

对外圆直径为 \(D\)、小圆直径为 \(d\) 的同心空心杆,其惯性矩 \(I\) 为
\[
I = \frac{\pi}{64}(D^4 - d^4).
\]

三点弯曲试验测定弹性模量 \(E\) 根据材料力学(刘鸿文. 材料力学(第三版上册). 高等教育出版社, 1999, 60~173)知识,有如下公式:
\[
E = \frac{Pl^3}{48yI}, \quad I = \frac{\pi}{64}\left[D^4 - (D - 2t)^4\right].
\]

\(E\) --- 茎秆弹性模量 (GPa), \(y\) --- 弯曲挠度 (mm), \(P\) --- 载荷 (N), \(l\) --- 跨度 (mm), \(D\) --- 茎秆外径, \(t\) --- 茎秆壁厚 (mm), \(I\) --- 茎秆截面惯性矩 \((mm^4)\)。

抗弯刚度:常用弹性模量与惯性矩的乘积 \(EI\) 来表示。\(EI\) 越大,梁轴线变形后的曲率越小。

抗弯强度:材料弯曲至破坏时所能承受的最大弯曲正应力 \(\sigma\):
\[
\sigma = \frac{M}{W}, \quad M = \frac{Pl}{4}, \quad W = \frac{\pi}{32D}\left[D^4 - (D - 2t)^4\right].
\]
其中,\(M\) --- 最大弯矩 (N·mm); \(W\) --- 试样抗弯截面系数 \((mm^3)\)

(2) 茎秆临界力

令 \(q_{cr}\) 为临界状态时茎秆单位长度的自重。\(P_{cr}\) 为临界状态时的穗重。茎秆在临界力 \(q_{cr}, P_{cr}\) 作用下;在微弯状态下处于不稳定平衡;其挠曲线近似方程(孙训方等. 材料力学(下). 高等教育出版社, 1987, 238~256)为
\[
v = \frac{\delta}{2L^3}(3Lx^2 - x^3),
\]
式中 \(\delta\) --- 位移参数,它表示茎秆顶端处的水平位移;\(L\) --- 茎秆的高度;\(x\) --- 截面位置;\(v\) --- 挠度,表示 \(x\) 截面处的水平位移;\(L, \delta, x, v\) 的单位为 cm。

茎秆势能
\[
\Pi = \frac{3EI\delta^2}{2L^3} - \frac{3}{16}q_{cr}\delta^2 - \frac{9\delta^2}{8L^6}\left(\frac{4}{3}L^2h^3 - Lh^4 + \frac{1}{5}h^5\right)P_{cr}.
\]
式中 $E$ --- 茎秆的弹性模量,Pa;$I$ --- 截面的惯性矩,$cm^4$。

由势能驻值原理 $\frac{\partial \Pi}{\partial \delta} = 0$ 可得
\[
q_{cr} + \frac{6P_{cr}}{L^6}\left(\frac{4}{3}L^2h^3 - Lh^4 + \frac{1}{5}h^5\right) = \frac{8EI}{L^3}.
\]

令 $Q = q_{cr}L$, $Q$ 为在临界平衡状态时的茎秆自重,$b = 6a^3\left(\frac{4}{3} - a + \frac{1}{5}a^2\right)$, $a = \frac{h}{L}$. $h$ 为穗位高。代入可得
\[
P_{cr} = \frac{8EI}{bL^2} - \frac{Q}{b}.
\]

引入茎秆系数 $\lambda$, $\lambda^2 = \frac{bL^2}{I}$, 单位 $cm^{-1}$。茎秆系数 $\lambda$ 与杆长、穗位、截面尺寸和形状等茎秆性状有关。可得农作物茎秆临界力的表达式
\[
P_{cr} = \frac{8E}{\lambda^2} - \frac{Q}{b}.
\]

注:穗位高可取为茎秆的高度,即 $h = L$。

(3)风载单独作用时植株临界力

茎秆在临界力 $q_{cr}$ 作用下处于不稳定,其挠曲线近似方程可按下式来表达(徐芝纶. 弹性力学(上册). 高等教育出版社, 1992, 340~370):

挠曲线近似
\[
\sigma = \frac{M}{W}, \quad v(x) = a_l\left(1 - \cos\frac{\pi x}{2l}\right).
\]

其中,$l$ --- 茎秆长度(不包括穗头长度);$a_l$ --- 距固定端距离 $l$ 处的挠度;$x$ --- 作用力距固定端的距离。

由最小势能原理 $\delta \Pi = 0$ 可得:

弯曲变形能

\begin{equation}
U = \frac{EI}{2} \int_{0}^{l} \left( \frac{d^2 v}{dx^2} \right)^2 dx = \frac{EI}{2} a_l^2 \left( \frac{\pi}{2l} \right)^2 \int_{0}^{l} \frac{\pi x}{2l} dx = \frac{EI \pi^4 a_l^2}{64l^3}.
\end{equation}

外力势能 \quad \( V = q a_l, \quad (x = l) \).

总势能 \(\Pi = U - V = \frac{EI \pi^4 a_l^2}{64l^3} - q a_l\).

即 \(\frac{\partial \Pi}{\partial a_l} = \frac{EI \pi^4 a_l}{32l^3} - q = 0\).

可求得单一植株弯折时所需最大力 \( q = \frac{EI \pi^4 a_l}{32l^3} \).

注: \( a_l \) 可按植株弯折时的弯折长度取值.

\section*{(4) 风级与风速对照表}

\begin{table}[h]
\centering
\begin{tabular}{|c|c|}
\hline
风级 & 风速(米/秒) \\
\hline
0 & 0-0.2 \\
\hline
1 & 0.3-1.5 \\
\hline
2 & 1.6-3.3 \\
\hline
3 & 3.4-5.4 \\
\hline
4 & 5.5-7.9 \\
\hline
5 & 8.0-10.7 \\
\hline
6 & 10.8-13.8 \\
\hline
7 & 13.9-17.1 \\
\hline
8 & 17.2-20.7 \\
\hline
9 & 20.8-24.4 \\
\hline
10 & 24.5-28.4 \\
\hline
11-12 & 大于28.4 \\
\hline
\end{tabular}
\end{table}

风压的贝努利公式:
\[
w = \frac{1}{2} \rho v^2 = \frac{\gamma}{2g} v^2 = \frac{v^2}{1630} \, (kN/m^2).
\]
其中 $w$ 为风压,$v$ 为风速,$\rho = 1.25 \, kg/m^3$ 为标准的空气密度,$\gamma = 0.012018 \, kN/m^3$ 为空气单位体积的重力,$g = 9.8 \, m/s^2$ 重力加速度。