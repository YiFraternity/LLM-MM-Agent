\documentclass{article}
\usepackage{amsmath}
\usepackage{amssymb}

\title{基于部件法的变循环发动机建模法}
\author{}
\date{}

\begin{document}

\maketitle

\begin{center}
\textbf{第十届华为杯全国研究生数学建模竞赛}
\end{center}

\begin{table}
\begin{tabular}{|c|l|}
\hline
$CDFS$ & 核心驱动风扇级 \\
\hline
$H$ & 发动机飞行高度 \\
\hline
$Ma$ & 飞行马赫数 \\
\hline
$n_{L}$ & 低压转速(风扇、低压涡轮物理转速) \\
\hline
$n_{H}$ & 高压转速(高压压气机、CDFS、高压涡轮物理转速) \\
\hline
$Z_{i}$ & 压比函数值,$i$为CL、CDFS、CH、TH、TL时分别表示风扇、 \\
 & CDFS、高压压气机、高压涡轮、低压涡轮 \\
\hline
$\alpha_{i}$ & 导叶角,$i$为CL、CDFS、CH、TH、TL时分别表示风扇、CDFS \\
 & 、高压压气机、低压涡轮 \\
\hline
$T_{4}^{*}$ & 主燃烧室出口温度 \\
\hline
$A_{8}$ & 尾喷管喉道面积 \\
\hline
$N_{i}$ & 功率,$i$为CL、CDFS、CH、TH、TL时分别表示风扇、CDFS、 \\
 & 高压压气机、高压涡轮、低压涡轮 \\
\hline
$W_{aS}$ & 截面S处的纯空气流量,截面序号参照原题目 \\
\hline
$W_{gS}$ & 截面S处的总气体流量,包括纯空气与燃气 \\
\hline
$p$ & 静压 \\
\hline
$ERR$ & 平衡方程的残差量 \\
\hline
$pr$ & 增压比 \\
\hline
$\eta$ & 效率 \\
\hline
$F_{s}$ & 发动机单位推力 \\
\hline
$sfc$ & 耗油率 \\
\hline
\end{tabular}
\end{table} \\
\end{tabular}

\begin{abstract}
本文采用部件级建模法精确模拟发动机的各个部件,依据各部件匹配工作时的 7 个平衡方程,对发动机的性能进行模拟。

针对问题一,为了求解风扇和 CDFS 的出口总温、总压和流量,建立模型对这两个部件的特性进行精确模拟,利用给定的发动机飞行高度和飞行马赫数,求解出风扇的出口总温、总压、流量分别为 379.4985、1.3087、19.0483,CDFS 的出口总温、总压、流量分别为 420.5365、1.8012、17.164。分析得出,气流在进入风扇和 CDFS 两个压气机部件至流出过程中,总温、总压增大,而气体从风扇流入到 CDFS 的过程中,总温、总压亦增大,流量减小。此结论符合压气机压缩气体导致温度升高、压强增大、流量减小的功能特点。

针对问题二,根据发动机整机模型,由七个参数值可计算出平衡残差量。以平衡残差量最小为原则,对离散化的待估参数进行变域、变步长的搜索,根据当前的最优解与次优解确定下一步的搜索域与搜索步长,逐步缩小搜索范围、减小搜索步长,搜索的终止条件设为:(1)高压转速、压比函数值的搜索步长减小至 0.01,主燃烧室出口温度的搜索步长减小至 10;(2)最优解与次优解相同。搜索的终止条件保证了解的精度与收敛性。依此算法搜索得到高压转速、压比函数值(风扇、CDFS、高压压气机、高压涡轮、低压涡轮)、主燃烧室出口温度的最优解分别为 1.00, 0.33, 0.43, 0.53, 0.14, 0.12, 1520,此时平衡方程残差量为 0.2550。逐步搜索过程中参数的解与平衡方程的残差趋于固定值,参数的解为模型的收敛解。

针对问题三(1),为了保证发动机性能最优,求解 CDFS 导叶角度、低压涡轮导叶角度和喷管喉道面积 3 个变量,实质上是一个优化的问题。本文建立优化模型,采用单位推力和耗油率的线性组合构建一个新的性能评价指标
\[ A = \lambda_1 \hat{F}s + \lambda_2 s\hat{f}c \quad (\lambda_1, \lambda_2 \text{ 为比例系数}) \]
以其最小值作为目标函数,同时借鉴

问题二中求解非线性方程组的方法,利用参数遍历法对模型进行解算。最终得到 CDFS 导叶角度、低压涡轮导叶角度和喷管喉道面积 3 个量分别为 3, 15, 6952.496 时,发动机的性能最优,此时单位推力和耗油率分别为 1293.092, 0.000239。同时,通过对遍历过程中部分参数对应的发动机的性能大小,分析得出规律:低压转速对发动机的性能无太大影响,提高主燃烧室的出口温度可以有效降低耗油率,增大风扇的压比函数值则能有效地增强单位推动力、降低耗油率。

针对问题三(2),探索 CDFS 导叶角、低压涡轮导叶角和尾喷管喉部面积在发动机性能最优条件下随飞行马赫数的变化规律,基于工作点的变步长的搜索方法,以发动机性能局部最优作为约束条件,以马赫数、CDFS 导叶角、低压涡轮导叶角为输入值,以尾喷管喉部面积、局部最优时对应的马赫数、CDFS 导叶角、低压涡轮导叶角为输出值,建立了变步长最优化模型。得出的结果显示,在某个具体工作点时发动机性能最优的条件下,当马赫数增加时,CDFS 导叶角、低压涡轮导叶角为恒定值,相关系数为 0;而尾喷管喉部面积随马赫数的增大呈现阶梯性递减的情况,当马赫数增加到某个具体的值时,面积保持恒定。这与整机模型中尾喷管喉部面积的规律描述相符。在本文给出的工作点 1 下,压比函数值处在中位,CDFS 导叶角的值恒为 35,低压涡轮导叶角恒为 15,尾喷管喉部面积从 4109.696 递减到 4087.818 后保持恒定;工作点 2 下,压比函数值处在高位,CDFS 导叶角的值恒为 29,低压涡轮导叶角恒为 15,尾喷管喉部面积从 3336.678 递减到 3283.023 后保持恒定;工作点 3 下,压比函数值处在低位,CDFS 导叶角的值恒为 29,低压涡轮导叶角恒为 14,尾喷管喉部面积不变,为 3369.63。

关键词:变循环发动机;部件法建模;平衡方程;变域变步长搜索
\end{abstract}

\section{问题重述}

变循环发动机由于其高单位推力、低巡航油耗的双重性能优势, 而倍受当前军事航空领域研究人员的关注 \footnote{[1]}, 是目前航空发动机的重要研究方向。

本题给定变循环发动机的构造及基本原理, 以及变循环发动机部件建模法的思路, 并且给定了发动机各部件匹配工作时的平衡方程以及相关参数说明。在附录 1 中分别对变循环发动机每个部件的计算公式进行了逐一介绍; 附录 2 给出了工质热物理性质参数, 包括空气、纯燃气、燃气的焓和熵等信息; 附录 3 对气动函数、气体常数 $R$、流量系数 $k_{m}$ 的相关公式和压气机、涡轮机的相关基本概念进行了说明; 附录 4 给出了各压气机和高、低压涡轮的特性数据。

根据已知内容, 需要解决的问题如下:

\subsection*{问题一}

1) 画出附录 4 中风扇特性数据表中流量随压比函数值变化的图形。

2) 设在发动机飞行高度 $H=11 \mathrm{~km}$, 飞行马赫数 $Ma=0.8$ 的亚音速巡航点, 导叶角度均设置为 $0^{\circ}$, 风扇和 CDFS 的物理转速都为 0.95 , 风扇和 CDFS 的压比函数值都为 0.5 , 求风扇和 CDFS 的出口总温、总压和流量。

\subsection*{问题二}

设在发动机飞行高度 $H=11 \mathrm{~km}$, 飞行马赫数 $Ma=0.8$ 的亚音速巡航点, 采用双涵道模式, 导叶角度均设置为 $0^{\circ}$, 选择活门完全打开, 副外涵道面积设为 $1.8395 \mathrm{e}+003$, 后混合器出口总面积设置为 $2.8518 \mathrm{e}+004$, 尾喷管喉道面积 $A_{8}=9.5544 \mathrm{e}+003, n_{L}=0.85$。运用或设计适当的算法求解由发动机 7 个平衡方程组成的非线性方程组。并且陈述及解释算法的关键步骤, 讨论算法的有效性。

\subsection*{问题三}

1) 设在发动机飞行高度 $H=11 \mathrm{~km}$, 飞行马赫数 $Ma=1.5$ 的超音速巡航点, 发动机采用单涵道模式, 将选择活门面积设置为 0 , 风扇导叶角度、高压压气机导叶角度、高压涡轮导叶角度均设置为 $0^{\circ}$, 后混合器面积设置为 $2.8518 \mathrm{e}+004$。为使发动机的性能最优, 发动机 CDFS 导叶角度、低压涡轮导叶角度和喷管喉道面积 3 个量应设置为多少?

2) 如果发动机采用单涵道模式, 将选择活门面积设置为 0 , 风扇导叶角度、高压压气机导叶角度、高压涡轮导叶角度均设置为 $0^{\circ}$, 后混合器出口总面积设置为 $2.8518 \mathrm{e}+004$, 后混合器内、外涵道面积可调 (即不受附录 1 后混合器给定的内、外涵道面积值的约束)。研究发动机飞行高度 $H=11 \mathrm{~km}$, 飞行马赫数从 $Ma=1.1$ 变化到 $Ma=1.6$, 发动机特性最优时, CDFS 导叶角度、低压涡轮导叶角度, 尾喷管喉道面积随飞行马赫数的变化规律。

\section{基本假设}

1) 经由高压压气机后不经主燃烧室的冷却气体的空气流量忽略不计;

2) 气流在各个部件中的流动均按一维处理,即截面气流参数作平均处理;

3) 燃气经过加力燃烧室已燃烧充分,尾喷管喷出气体不包含燃气。

\section{符号说明}
\begin{table}
\begin{tabular}{|c|l|}
\hline
$CDFS$ & 核心驱动风扇级 \\
\hline
$H$ & 发动机飞行高度 \\
\hline
$Ma$ & 飞行马赫数 \\
\hline
$n_{L}$ & 低压转速(风扇、低压涡轮物理转速) \\
\hline
$n_{H}$ & 高压转速(高压压气机、CDFS、高压涡轮物理转速) \\
\hline
$Z_{i}$ & 压比函数值,$i$为CL、CDFS、CH、TH、TL时分别表示风扇、 \\
 & CDFS、高压压气机、高压涡轮、低压涡轮 \\
\hline
$\alpha_{i}$ & 导叶角,$i$为CL、CDFS、CH、TH、TL时分别表示风扇、CDFS \\
 & 、高压压气机、低压涡轮 \\
\hline
$T_{4}^{*}$ & 主燃烧室出口温度 \\
\hline
$A_{8}$ & 尾喷管喉道面积 \\
\hline
$N_{i}$ & 功率,$i$为CL、CDFS、CH、TH、TL时分别表示风扇、CDFS、 \\
 & 高压压气机、高压涡轮、低压涡轮 \\
\hline
$W_{aS}$ & 截面S处的纯空气流量,截面序号参照原题目 \\
\hline
$W_{gS}$ & 截面S处的总气体流量,包括纯空气与燃气 \\
\hline
$p$ & 静压 \\
\hline
$ERR$ & 平衡方程的残差量 \\
\hline
$pr$ & 增压比 \\
\hline
$\eta$ & 效率 \\
\hline
$F_{s}$ & 发动机单位推力 \\
\hline
$sfc$ & 耗油率 \\
\hline
\end{tabular}
\end{table}

\section{问题分析}

对变循环发动机性能的研究可以采用实验法和模拟计算法。由于实验法需要对复杂的设备进行研制,成本较高,在实际情况中并不常用。同时随着计算机的迅猛发展,采用数学模型来模拟仿真发动机的精度愈来愈高,发动机仿真技术得到了广泛的应用。

变循环发动机由进气道、压气机、主燃烧室、涡轮、混合器和尾喷管等部件组成。本文采用部件法对变循环发动机进行建模,即利用计算机对各个部件进行精确的模拟,最终模拟出整个发动机的性能。

\subsection{变循环发动机的基本工作原理}

图 4.1 表示了变循环发动机工作时气体在各个部件之间的流向。箭头表示气体流向(虚线箭头①表示从高压压气机出来不流经主燃烧室的分流气体,即冷却气体,根据假设,此部分气体忽略不计);虚线表示同轴的部件,风扇与低压涡轮同轴转动,CDFS、高压压气机与高压涡轮同轴转动。

\begin{figure}[h]
\centering
\includegraphics[width=\textwidth]{image.png}
\caption{变循环发动机的工作原理}
\end{figure}

变循环发动机有两种工作模式:单外涵模式和双外涵模式,单外涵模式常用于飞机的加速、爬升或超音速巡航等需要强大推力的飞行状态,双外涵模式常用于起飞、下降和亚声速巡航。表 4.1 显示了两种工作模式下的部件设置和特点。

\begin{tabular}{|c|c|c|c|}
\hline 参数 & 取值范围 & 参数 & 取值范围 \\
\hline $n_{L}$ & $0-1$ & $Z_{T L}$ & $0-1$ \\
\hline $n_{H}$ & $0-1$ & $T_{4}^{*}$ & $<2000$ \\
\hline $Z_{C L}$ & $0-1$ & $\alpha_{L}$ & $0$ \\
\hline $Z_{C D F S}$ & $0-1$ & $\alpha_{C D F S}$ & $-5-35$ \\
\hline $Z_{C H}$ & $0-1$ & $\alpha_{H}$ & $0$ \\
\hline $Z_{T H}$ & $0-1$ & $\alpha_{C H}$ & $-5-15$ \\
\hline
\end{tabular}

\subsection{对问题一的分析}

对于问题一(1),要求画出风扇特性数据表中流量随压比函数值变化的图形,附录 4 中给定了九组不同换算转速下的增压比、流量和效率数据。根据附录 3 中对压气机压比函数值 $zz$ 的定义,即

\begin{equation}
zz = \frac{pr - pr_{\min}}{pr_{\max} - pr_{\min}}
\tag{4.1}
\end{equation}

其中,$pr_{\min}$ 和 $pr_{\max}$ 分别为增压比的最大值和最小值。

对于问题一(2),风扇和 CDFS 的出口总温、总压和流量的计算可参考附录 1 关于进气道、风扇、和 CDFS 等部件的相关公式。

\subsection{对问题二的分析}

题目给出了发动机的 12 个设计参数,在问题二中,有 7 个设计参数未知,应根据 7 个平衡方程构成的非线性方程组求解未知参数,7 个平衡方程如式 (4.2) 所示。

\begin{equation}
\left\{
\begin{aligned}
N_{CL} - N_{TL} \eta_{mL} &= 0 \\
N_{CH} + N_{CDFS} - N_{TH} \eta_{mH} &= 0 \\
W_{g41} - W_{g41}' &= 0 \\
W_{g45} - W_{g45}' &= 0 \\
p_{61} - p_{62} &= 0 \\
A_8 - A_8' &= 0 \\
W_{a2} - W_{a21} - W_{a13} &= 0
\end{aligned}
\right.
\tag{4.2}
\end{equation}

其中,$N_{CL}$、$N_{TL}$、$N_{CH}$、$N_{CDFS}$、$N_{TH}$ 分别是风扇、低压涡轮、高压压气机、CDFS 和高压涡轮的功率,$\eta_{mL}$、$\eta_{mH}$ 分别为中间轴和高速轴的机械效率;$W_{g41}$、$W_{g41}'$ 分别是高压涡轮进口截面气体流量和通过高压涡轮特性数据线性插值得到的高压涡轮流量;$W_{g45}$、$W_{g45}'$ 分别是低压涡轮进口截面流量和通过低压涡轮特性数据线性插值得到的低压涡轮流量;$W_{a2}$、$W_{a13}$、$W_{a21}$ 分别为风扇出口、副外涵和 CDFS 进口的流量;$p_{61}$、$p_{62}$ 分别是后混合器内涵道和主外涵道的静压;$A_8'$、$A_8$ 分别为给定的尾喷管喉道面积和按附录 1 尾喷管的有关公式计算出的尾喷管喉道面积。

求解非线性方程组一般采用牛顿—拉夫森法、逆牛顿法等,然而问题二中的平衡方程难以展开为具体的表达式,常规方法并不适用。本文采用部件法对变循环发动机进行建模,即利用计算机对各个部件进行精确的模拟,最终模拟出整个发动机的性能。基于部件法的建模,可将 7 个未知参数作为模型输入量,利用模型输出量构建求解的准则。

式 (4.3) 表示了平衡方程组的不符值(以下简称为平衡残差量):

\begin{equation}
ERR = \sum_{i=1}^{7} |err(\mathbf{i})|
\tag{4.3}
\end{equation}

其中,$|err(\mathbf{i})|$ 表示对 $err(\mathbf{i})$ 取绝对值,$err(\mathbf{i})$ 的意义如式 (4.4) 所示:

\begin{equation}
\left\{
\begin{aligned}
err(1) &= \frac{N_{CL} - N_{TL} \eta_{mL}}{N_{TL}} \\
err(2) &= \frac{N_{CH} + N_{CDFS} - N_{TH} \eta_{mH}}{N_{TH}} \\
err(3) &= \frac{W_{g41} - W'_{g41}}{W_{g41}} \\
err(4) &= \frac{W_{g45} - W'_{g45}}{W_{g45}} \\
err(5) &= \frac{p_{61} - p_{62}}{p_{61}} \\
err(6) &= \frac{A_{8} - A'_{8}}{A_{8}} \\
err(7) &= \frac{W_{a2} - W_{a21} - W_{a13}}{W_{a2}}
\end{aligned}
\right.
\tag{4.4}
\end{equation}

式 (4.4) 右侧为对应平衡方程的残差量,各量的意义见式 (4.2)。

$ERR$ 反映了发动机偏离稳态的程度,当 $ERR$ 接近 0 时,平衡方程组近似成立,发动机处于近似稳态。由此考虑将平衡残差量作为模型的输出量,以平衡残差量最小为准则确定问题二中 7 个参数的解。

\subsection{对问题三的分析}

根据图 4.4 变循环发动机的工作模式图和表 4.1,变循环发动机在单涵道工作模式下,模式转换活门关闭,此时风扇的出口气流全部进入 CDFS,而不再通过副外涵道,同时前混合器和后混合器面积均关小以应对改变的气流。

\subsubsection*{(1)对发动机性能指标的评价}

附录 1 给出了发动机的性能参数,包括推力、单位推力和耗油率。本文采用单位推力和耗油率两项指标。

单位推力:
\[
Fs = F / W_{a},
\]

耗油率:
\[
sfc = \frac{3600 W_{f}}{F},
\]

其中,$F$ 是推力,可按附录 1 中的公式计算得到,$W_{a}$ 是总的空气流量,$W_{f}$ 是主燃烧室的燃油流量。

本文在判断发动机性能的优劣时将这两项指标进行组合构建一个新的指标 $A$,见式 (4.5)。
\begin{equation}
A = \lambda_{1} \hat{Fs} + \lambda_{2} \hat{sfc}
\tag{4.5}
\end{equation}

其中,$\lambda_{1}$、$\lambda_{2}$ 为比例系数,根据发动机的高性能取决于高单位推力和低耗油率,在两者之间不强调偏重时,一般取 $\lambda_{1} = 1$,$\lambda_{2} = -1$;$\hat{Fs}$、$\hat{sfc}$ 分别表示单位推

力和低耗油率的标准化值,即 $\hat{F}_{S}=\frac{F_{S}-\tilde{F}_{S}}{\sigma_{F_{S}}}, \quad \hat{s f c}=\frac{s f c-\tilde{s f c}}{\sigma_{s f c}}$,$\tilde{F}_{S} 、 \tilde{s f c}$ 分别为相应的均值,$\sigma_{F_{S}} 、 \sigma_{s f c}$ 分别为相应的标准差。

如此,在本问题中关于发动机性能最优的问题即转化为求 $A$ 的最大值的问题。

\subsubsection*{(2)对于问题三 1)}

本问题要求确定 CDFS 导叶角度、低压涡轮导叶角度和喷管喉道面积 3 个参数,以使发动机的性能最优。已知给定了风扇和高压压气机的导叶角度 $\alpha_{L} 、 \alpha_{H}$,按照 4.5 节的分析,表 4.2 中 12 个参数有 10 个未知量。根据实际情况,低压转速和高压转速最大值均为 1,主燃烧室出口温度不超过 2000,压比函数值为标准化变量,其范围在 $0-1$ 之间,CDFS 和低压涡轮的导叶角亦在附录 2 中给定了范围。各参数的具体取值或范围见表 4.2。

表 4.2 参数的取值范围

\begin{tabular}{|c|c|c|c|}
\hline 参数 & 取值范围 & 参数 & 取值范围 \\
\hline $n_{L}$ & $0-1$ & $Z_{T L}$ & $0-1$ \\
\hline $n_{H}$ & $0-1$ & $T_{4}^{*}$ & $<2000$ \\
\hline $Z_{C L}$ & $0-1$ & $\alpha_{L}$ & $0$ \\
\hline $Z_{C D F S}$ & $0-1$ & $\alpha_{C D F S}$ & $-5-35$ \\
\hline $Z_{C H}$ & $0-1$ & $\alpha_{H}$ & $0$ \\
\hline $Z_{T H}$ & $0-1$ & $\alpha_{C H}$ & $-5-15$ \\
\hline
\end{tabular}

本问题在问题二基础上,未知变量由 7 个增加至 10 个,且具有描述发动机性能的目标函数,最终均是通过求解非线性平衡方程组以获取参数值。因此,除了求解平衡方程,本问题可归结为优化问题。

\subsubsection*{(3)对于问题三 2)}

该问题要求在发动机处于最优的情况下,考虑 CDFS 导叶角度、低压涡轮导叶角度,尾喷管喉道面积随飞行马赫数的变化规律。为了计算出飞机的性能参数,除了题中给定的活门面积,风扇导叶角度、高压压气机导叶角度、高压涡轮导叶角度、后混合器出口总面积等参数外,仍然需要以下参数的输入:低压转速、高压转速、风扇压比函数值、CDFS 压比函数值、高压压气机压比函数值、主燃烧室出口温度、高压涡轮压比函数值、低压涡轮压比函数值、CDFS 导叶角、低压涡轮导叶角。显然,如果采用在各个变量的空间内遍历 $n$ 次,其算法复杂度高达 $n^{10}$。为此,仍然参照第三题的思路,给出某个具体工作点下的局部最优方案策略。

这一题,将马赫数也作为变量,然后计算出每个马赫数下,求得使发动机性能达到最优的一组 CDFS 导叶角度、低压涡轮导叶角度,尾喷管喉道面积。然后考虑每个马赫数同与其对应的那一组 CDFS 导叶角度、低压涡轮导叶角度,尾喷管喉道面积的关系。因此需要对每个输入的马赫数采用最优策略,同时为了确保发动机能正常工作,应该使得计算的残差不超过阈值,在这种情况下计算发动机的性能。

\section{模型建立与求解}

\subsection{问题一的模型建立与求解}

\subsubsection{问题一(1)的模型建立与求解}

根据 4.2 中的分析,可建立如下模型:
\[
W_{cL}^{i} = f\left(zz_{cL}^{i}\right)
\]
其中,
\[
\begin{cases}
pr_{\min}^{i} = \min(pr_{cL}^{i}) \\
pr_{\max}^{i} = \max(pr_{cL}^{i}) \\
zz_{cL}^{i} = \frac{pr_{cL}^{i} - pr_{\min}^{i}}{pr_{\max}^{i} - pr_{\min}^{i}} \\
i = 1, 2, 3 \ldots 9
\end{cases}
\]

\(i = 1, 2, 3 \ldots 9\) 表示风扇对应不同换算转速的九组数据序号;

\(pr_{cL}^{i}\) 为第 \(i\) 组数据中的增压比序列;

\(pr_{\min}^{i}\) 为第 \(i\) 组数据中的增压比序列中的最小值,\(pr_{\max}^{i}\) 为第 \(i\) 组数据中的增压比序列中的最大值;

\(zz_{cL}^{i}\) 表示第 \(i\) 组数据的压比函数值;

\(W_{cL}^{i}\) 表示第 \(i\) 组数据的流量。

模型即绘制函数 \(W_{cL}^{i} = f\left(zz_{cL}^{i}\right)\) 在特定数值范围内的图形,首先计算每组数据中的压比函数值,然后以流量为因变量,压比函数值为自变量绘制九条曲线。利用 matlab 软件绘制得流量随压比函数值变化的图形如图 5.1 所示。

\begin{figure}[h]
    \centering
    \includegraphics[width=\textwidth]{image.png}
    \caption{风扇特性的流量随压比函数值的变化}
    \label{fig:5.1}
\end{figure}

分析图 5.1 可得到如下几条规律:

(1) 在换算转速一定时,随着压比函数值在一定范围内的增大,流量呈递减趋势;

(2) 在换算转速一定时,压比函数值增大到一定程度时,同一压比函数值可能对应不同的流量;

(3) 随着换算转速的增加,流量逐渐增大。

结合压比函数值的概念和发动机的工作原理,压比函数值是增压比的标准化值,增压比定义为本级出口气流的总压与进口气流的总压之比。换算转速一定时,增压比越大,即出口气流总压所占的比例增大时,会导致风扇的流量减小,与第一条规律相吻合。

\subsubsection{问题一(2)的模型建立与求解}

根据 4.2 节的分析,对本问题可建立如下模型:

总压 \( p_i \):
\[
p_i = p_i^* \cdot pr_i
\]

总温 \( T_i \):
\[
\begin{cases}
\psi_{i,ei} = \psi_i^* + \frac{R}{M} \ln pr_i \\
\psi_{i,ei} = \psi(T_{i,ei}) \\
h_{i,ei} = h(T_{i,ei}) \\
\eta_i = \frac{h_{i,ei} - h_i^*}{h_i - h_i^*} \\
h_i = h(T_i)
\end{cases}
\]

流量 \( W_i \):
\[
W_i = W_{ci}^* \cdot \sqrt{\frac{T_{i,d}^*}{T_i^*}} \cdot \frac{p_i^*}{p_{i,d}^*}
\]

其中,\( \psi_i^* \)、\( h_i^* \) 分别为进口熵和进口焓,可通过进口总温 \( T_i^* \) 计算而得;\( \psi_{i,ei} \)、\( h_{i,ei} \) 分别为风扇出口理想熵和理想焓;\( h_i \) 是风扇出口焓;\( M = 1 \),\( R \) 是气体常数,在附录 3 中给出了说明;\( T_{i,d}^* \) 和 \( p_{i,d}^* \) 为特定值,在附录 1 中给出了其数值;\( i \) 为 \( CL \)、\( CDFS \) 分别表示风扇和 \( CDFS \)。

风扇和 \( CDFS \) 的出口总温、总压和流量的计算流程如下:

(1) 利用已知发动机的飞行高度 \( H \) 和飞行马赫数 \( Ma \),计算风扇进口参数;

(2) 由风扇的换算转速和给定的导叶角度、压比函数值,对风扇的特性数据线性插值,计算风扇的增压比、效率和换算流量;

(3) 按附录 1 风扇部分的公式,计算风扇的出口总温 \( T_{CL} \)、总压 \( p_{CL} \) 和流量 \( W_{CL} \)。

(4) 将风扇的出口总温 \( T_{CL} \)、总压 \( p_{CL} \) 作为 \( CDFS \) 进口参数,根据题目附录 1 中 \( CDFS \) 部件的有关公式计算出 \( CDFS \) 的出口总温 \( T_{CDFS} \)、总压 \( p_{CDFS} \) 和流量 \( W_{CDFS} \),解得结果如表 5.1 所示。

\begin{table}
\centering
\caption{风扇和 CDFS 的总温、总压、流量}
\begin{tabular}{|c|c|c|c|c|c|}
\hline
压气机部件 & \multicolumn{2}{c|}{总温} & \multicolumn{2}{c|}{总压} & 流量 \\
\cline{2-6}
 & 进口 & 出口 & 进口 & 出口 & 出口 \\
\hline
风扇 & 244.3812 & 379.4985 & 0.3447 & 1.3087 & 19.0483 \\
\hline
CDFS & 379.4985 & 420.5365 & 1.3087 & 1.8012 & 17.164 \\
\hline
\end{tabular}
\end{table}

分析表 5.1 的结果可知,对于风扇和 CDFS 两个压气机部件,气流从流入到流出过程中总温、总压增大,而气体从风扇流入到 CDFS 的过程中,总温、总压亦增大,流量减小。此结论符合压气机的功能特点,即压气机对气体进行压缩,从而导致其温度升高、压强增大、流量减小。

\subsection{问题二的模型建立与求解}

\subsubsection{问题二模型的建立}

考虑到发动机模型的复杂性,问题二给出的 7 个平衡方程难以展开成试给参数的表达式。本文对发动机做部件级建模,模拟发动机各个部件的特性,各部件的模型均有输入输出参数作为外部接口,这些参数包括温度、压强、流量、面积等,利用这些接口可以实现各部件的连接,从而得到发动机整机模型。所建立的模型可用图 5.2 表示。

针对问题二的特点,将七个未知参数值 $(n_{H}, Z_{CL}, Z_{CDFS}, Z_{CH}, Z_{TH}, Z_{TL}, T_{4}^{*})$ 作为发动机整机模型的输入量,输出量为平衡方程的残差量,平衡残差量可用七个参数值计算得到,用式 XX 表示:

\begin{equation}
ERR = \sum_{i=1}^{7} |err(i)| = f(n_{H}, Z_{CL}, Z_{CDFS}, Z_{CH}, Z_{TH}, Z_{TL}, T_{4}^{*})
\tag{5.1}
\end{equation}

发动机处于稳态时 $ERR$ 的理论值为 0。但因模型误差的存在,认为只要 $ERR$ 足够小发动机便处于稳态。因此平衡方程组的求解可以转化为确定设计参数的值

\begin{equation}
(n_{Hop t}, Z_{CLop t}, Z_{CDFSop t}, Z_{CHop t}, Z_{THop t}, Z_{TLop t}, T_{4op t}^{*})
\tag{5.2}
\end{equation}

使得下式成立:

\begin{equation}
Min(ERR) = f(n_{Hop t}, Z_{CLop t}, Z_{CDFSop t}, Z_{CHop t}, Z_{THop t}, Z_{TLop t}, T_{4op t}^{*})
\tag{5.3}
\end{equation}

$Min$ 表示取最小值,此时的参数值即为平衡残差量最小原则下的模型最优解。

\subsubsection{问题二模型的求解与分析}

本文将 7 个参数作离散化处理,由参数值求出平衡残差量,以平衡残差量最小为求解原则可搜索出参数的最优解。理论上可以穷举参数所有可能的取值,但问题二模型复杂,搜索的计算量巨大,本文提出变搜索域、变步长的搜索方案,逐步缩小搜索域、减小步长。缩小搜索域可以缩小参数的取值区间,减小计算量,适当减小步长可以提高求解参数的精度。

根据参数的最优解 $\beta_{1}$ 与次优解 $\beta_{2}$ 确定下一步的搜索域,如果最优解与次优解相同,则搜索域为该值的邻域 $[\beta_{1} - \sigma, \beta_{1} + \sigma]$,$\sigma$ 视步长而定,如果最优解与次优解不同,则搜索域为 $[\beta_{1}, \beta_{2}]$。搜索步长应逐级减小,同时受搜索域的限制,

\begin{figure}[h]
    \centering
    \includegraphics[width=\textwidth]{image.png}
    \caption{发动机部件级建模与计算流程}
    \label{fig:engine_flowchart}
\end{figure}

搜索域较大时搜索步长不宜过小,否则会增加计算量,本文采用的搜索步长依次为 $0.3, 0.1, 0.03, 0.01$。

搜索终止的条件是:(1) 转速、压比函数值等六个参数的搜索步长达到 $0.01$

\begin{table}
\caption{参数的逐级搜索过程}
\label{tab:search_process}
\begin{tabular}{c c c c c}
\hline
\multicolumn{5}{c}{第一步搜索} \\
\hline
参数 & 搜索域 & 搜索步长 & 最优解 & 次优解 \\
\hline
\(n_{H}\) & \(0:1\) & \(0.3\) & \(0.9\) & \(0.9\) \\
\(Z_{CL}\) & \(0:1\) & \(0.3\) & \(0.5\) & \(0.5\) \\
\(Z_{CDFS}\) & \(0:1\) & \(0.3\) & \(0.1\) & \(0.5\) \\
\(Z_{CH}\) & \(0:1\) & \(0.3\) & \(0.5\) & \(0.5\) \\
\(Z_{TH}\) & \(0:1\) & \(0.3\) & \(0.1\) & \(0.1\) \\
\(Z_{TL}\) & \(0:1\) & \(0.3\) & \(0.1\) & \(0.1\) \\
\(T_{4}^{*}\) & \(1000:2000\) & \(300\) & \(1400\) & \(1400\) \\
\hline
\multicolumn{5}{c}{第一步平衡残差量:最优解 0.8032 次优解 0.8051} \\
\hline
\multicolumn{5}{c}{第二步搜索} \\
\hline
参数 & 搜索域 & 搜索步长 & 最优解 & 次优解 \\
\hline
\(n_{H}\) & \(0.8:1\) & \(0.1\) & \(1.0\) & \(1.0\) \\
\(Z_{CL}\) & \(0.4:0.6\) & \(0.1\) & \(0.4\) & \(0.4\) \\
\(Z_{CDFS}\) & \(0.1:0.5\) & \(0.1\) & \(0.5\) & \(0.4\) \\
\(Z_{CH}\) & \(0.4:0.6\) & \(0.1\) & \(0.5\) & \(0.5\) \\
\(Z_{TH}\) & \(0.1:0.2\) & \(0.1\) & \(0.1\) & \(0.1\) \\
\(Z_{TL}\) & \(0.1:0.2\) & \(0.1\) & \(0.1\) & \(0.1\) \\
\(T_{4}^{*}\) & \(1000:2000\) & \(100\) & \(1500\) & \(1500\) \\
\hline
\multicolumn{5}{c}{平衡残差量:最优解 0.5394 次优解 0.5422} \\
\hline
\multicolumn{5}{c}{第三步搜索} \\
\hline
参数 & 搜索域 & 搜索步长 & 最优解 & 次优解 \\
\hline
\(n_{H}\) & \(0.94:1\) & \(0.03\) & \(1.00\) & \(1.00\) \\
\(Z_{CL}\) & \(0.34:0.44\) & \(0.03\) & \(0.34\) & \(0.34\) \\
\(Z_{CDFS}\) & \(0.3:0.5\) & \(0.1\) & \(0.50\) & \(0.50\) \\
\(Z_{CH}\) & \(0.44:0.54\) & \(0.03\) & \(0.53\) & \(0.50\) \\
\(Z_{TH}\) & \(0.04:0.14\) & \(0.03\) & \(0.13\) & \(0.13\) \\
\(Z_{TL}\) & \(0.04:0.14\) & \(0.03\) & \(0.13\) & \(0.13\) \\
\(T_{4}^{*}\) & \(1440:1540\) & \(30\) & \(1530\) & \(1530\) \\
\hline
\multicolumn{5}{c}{平衡残差量:最优解 0.2933 次优解 0.3044} \\
\hline
\multicolumn{5}{c}{第四步搜索} \\
\hline
参数 & 搜索域 & 搜索步长 & 最优解 & 次优解 \\
\hline
\(n_{H}\) & \(0.98:1\) & \(0.01\) & \(1.00\) & \(1.00\) \\
\(Z_{CL}\) & \(0.33:0.35\) & \(0.01\) & \(0.33\) & \(0.34\) \\
\(Z_{CDFS}\) & \(0.44:0.54\) & \(0.03\) & \(0.44\) & \(0.44\) \\
\hline
\end{tabular}
\end{table}

\begin{table}
\centering
\begin{tabular}{c c c c c}
\hline
$Z_{CH}$ & 0.5:0.53 & 0.01 & 0.53 & 0.53 \\
\hline
$Z_{TH}$ & 0.12:0.14 & 0.01 & 0.14 & 0.14 \\
\hline
$Z_{TL}$ & 0.12:0.14 & 0.01 & 0.12 & 0.12 \\
\hline
$T_{4}^{*}$ & 1520:1540 & 10 & 1520 & 1520 \\
\hline
\multicolumn{5}{c}{平衡残差量:最优解 0.2550 次优解 0.2562} \\
\hline
\end{tabular}
\end{table}

\textbf{第五步搜索}

\begin{table}
\centering
\begin{tabular}{c c c c c}
\hline
参数 & 搜索域 & 搜索步长 & 最优解 & 次优解 \\
\hline
$n_{H}$ & 1 & 0 & 1.00 & 1.00 \\
\hline
$Z_{CL}$ & 0.33:0.34 & 0.01 & 0.33 & 0.33 \\
\hline
$Z_{CDFS}$ & 0.42:0.44 & 0.01 & 0.43 & 0.44 \\
\hline
$Z_{CH}$ & 0.53 & 0 & 0.53 & 0.53 \\
\hline
$Z_{TH}$ & 0.14 & 0 & 0.14 & 0.14 \\
\hline
$Z_{TL}$ & 0.12 & 0 & 0.12 & 0.12 \\
\hline
$T_{4}^{*}$ & 1520 & 0 & 1520 & 1520 \\
\hline
\multicolumn{5}{c}{平衡残差量:最优解 0.2550 次优解 0.2550} \\
\hline
\end{tabular}
\end{table}

图5.3给出了逐步搜索过程中参数取值、平衡残差量的变化,参数序号与表5.1中的顺序相同。为方便显示,对纵坐标值做了减均值或乘常数处理,并对曲线做了平滑处理。由曲线的趋势可以看出,逐步搜索时参数与平衡量残差趋于定值,此时的解可视为模型的收敛解。

\begin{figure}[h]
\centering
\includegraphics[width=\textwidth]{image.png}
\caption{逐步搜索时参数与平衡残差量的变化趋势}
\end{figure}

\subsection{问题三的模型建立与求解}

\subsubsection{问题三(1)的模型建立与求解}

对于问题三(1),根据 4.4 节的分析,本问题可归结为优化问题。即可建立目标函数:

\[
\min A
\]

其中,$A = \lambda_1 \hat{F}s + \lambda_2 sfc$。$\lambda_1$、$\lambda_2$ 为比例系数,一般取 $\lambda_1 = 1$,$\lambda_2 = -1$;$\hat{F}s$、$sfc$ 分别表示单位推力和低耗油率的标准化值。

约束条件采用式 (5.3):

\[
Min(ERR) = f(n_{Hopt}, Z_{CLopt}, Z_{CDFSopt}, Z_{CHopt}, Z_{THopt}, Z_{TLopt}, T_{4opt}^*)
\]

根据 4.4 节的分析,本问题的求解过程即对 7 个非线性方程求解 10 个参数,以满足目标函数。由于已知每个变量的取值范围,采用遍历法求解是较便捷的方法。

由于未知参数的数量过多,在具体执行过程中,仅对待求参数 CDFS 导叶角度 $\alpha_{CDFS}$ 和低压涡轮导叶角度 $\alpha_{CH}$ 进行逐一遍历,另一待求参数喷管喉道面积 $A_8$ 可由 10 个参数求出。而对其他 8 个未知参数 $n_L$、$n_H$、$Z_{CL}$、$Z_{CDFS}$、$Z_{CH}$、$Z_{TH}$、$Z_{TL}$、$T_4^*$,采取问题二中求解平衡方程组所得的参数值作为初值,通过改变个别的大小来体现其对目标函数即发动机性能大小的影响。

\begin{figure}[h]
    \centering
    \includegraphics[width=\textwidth]{Fs_sfc_plot.png}
    \caption{初始遍历所得的单位推力和耗油率}
    \label{fig:Fs_sfc}
\end{figure}

\begin{figure}[h]
    \centering
    \includegraphics[width=\textwidth]{residual_plot.png}
    \caption{初始遍历平衡方程的残差}
    \label{fig:residual}
\end{figure}

图 5.4 显示了 8 个未知参数取值为问题二中给定值及解算结果时,对 \(\alpha_{\text{CDFS}}\) 和 \(\alpha_{CH}\) 遍历所得的 190 种符合条件的情况下,单位推力和耗油率的取值情况。图 5.5 为七个平衡方程的残差情况。

从图 6.3 中可以看出,单位推力 \(Fs\) 和耗油率 \(sfc\) 呈现周期性的变化。每个周期内的变化情况为:在 CDFS 导叶角度 \(\alpha_{\text{CDFS}}\) 一定时,随着低压涡轮导叶角度 \(\alpha_{CH}\) 的增加,单位推力 \(Fs\) 逐渐增大,耗油率 \(sfc\) 逐渐降低;周期间的变化情况为:在低压涡轮导叶角度 \(\alpha_{CH}\) 一定时,随着 CDFS 导叶角度 \(\alpha_{\text{CDFS}}\) 的增加,单位推力 \(Fs\) 逐渐增大,耗油率 \(sfc\) 逐渐降低。即总体变化趋势为单位推力和耗油率随着 \(\alpha_{\text{CDFS}}\) 和 \(\alpha_{CH}\) 的增大分别递增和递减,此时性能愈高。

\section{表 5.3 参数取值情况}

\begin{table}[h]
\centering
\begin{tabular}{|c|c|c|c|c|c|}
\hline
参数 & 取值 1 & 取值 2 & 取值 3 & 取值 4 & 取值 5 \\
\hline
\(n_{L}\) & 0.85 & 0.95 & 0.85 & 0.85 & 0.85 \\
\hline
\(n_{H}\) & 1 & 1 & 1 & 1 & 1 \\
\hline
\(Z_{CL}\) & 0.33 & 0.33 & 0.33 & 0.6 & 1 \\
\hline
\(Z_{\text{CDFS}}\) & 0.43 & 0.43 & 0.43 & 0.43 & 0.43 \\
\hline
\(Z_{CH}\) & 0.53 & 0.53 & 0.53 & 0.53 & 0.53 \\
\hline
\(Z_{TH}\) & 0.14 & 0.14 & 0.14 & 0.14 & 0.14 \\
\hline
\(Z_{TL}\) & 0.12 & 0.12 & 0.12 & 0.12 & 0.12 \\
\hline
\(T_{4}^{*}\) & 1520 & 1520 & 2000 & 2000 & 2000 \\
\hline
\end{tabular}
\end{table}

更改 8 个未知参数的值,重新对 \(\alpha_{\text{CDFS}}\) 和 \(\alpha_{CH}\) 遍历计算,则可得到不同的单位推力和耗油率值。由于参数众多,本文仅选用其中五组情况,如表 5.3 所示,对三个待求参数对发动机性能的影响进行分析说明。得到相应的结果为图 5.6 所示,其中 \(Fs\) 和 \(sfc\) 的标号分别与表 5.3 中取值 1 到取值 5 一一对应。

\begin{figure}[h]
\centering
\includegraphics[width=\textwidth]{Fs_plot.png}
\caption{Fs}
\end{figure}

\begin{figure}[h]
    \centering
    \includegraphics[width=\textwidth]{image.png}
    \caption{不同参数对应的单位推力和耗油率}
    \label{fig:5.6}
\end{figure}

根据表 5.3 中参数的设置对图 \ref{fig:5.6} 进行分析可知,取值 1 到取值 2,低压转速增加时,单位推力无明显变化,而耗油率有小幅度降低;取值 1 到取值 3,随着主燃烧室出口温度的增高,单位推力有小幅度增加,耗油率大幅度降低;取值 3、4、5 则说明风扇的压比函数值增大时,单位推力明显增大,耗油率明显降低。即可得出结论,低压转速对发动机的性能无太大影响,主燃烧室的出口温度主要影响耗油率,而风扇的压比函数值则对发动机的性能具有较大的影响,其数值愈大,发动机的性能愈优。

利用遍历法对所建立的模型进行求解,最终解得在各参数为表 5.3 中取值 5 的情况下,发动机具有最佳性能,此时 CDFS 导叶角度 $\alpha_{\text{CDFS}}$、低压涡轮导叶角度 $\alpha_{CH}$ 和喷管喉道面积 $A_{8}$ 3 个量分别为:
\begin{equation}
\alpha_{\text{CDFS}} = 3, \alpha_{CH} = 15, A_{8} = 6952.496
\end{equation}
发动机的单位推力和油耗率分别为:
\begin{equation}
Fs = 1293.092, sfc = 0.000239
\end{equation}

\subsubsection{问题三(2)的模型建立与求解}

为了便于建模,首先对变量分类如表 5.4 所示。

先取定不同的工作状态,即将上述表格中的工作点的变量赋值,按照其他参考文献中一般给定的值赋值。在此处,工作状态(工作点)在低压转速与高压转速设为定值,变动压比函数值,分别让函数值处在低位,中位和高位,以建立三个工作点进行分析。

设工作点表达成如下的函数:
\begin{equation}
F^* = F(n_L, n_H, Z_{CL}, Z_{CDFS}, Z_{CH}, Z_{TH}, Z_{TL}, T_4^*)
\end{equation}
其中 $F^*$ 是当工作点变量取定时的具体的工作状态(工作点)。

构建飞机性能最优函数

现设飞机的性能参数 $Fs$ 和 $sfc$,参照问题三的第一问,可以构建飞机性能最优函数如下:
\begin{equation}
A = \lambda_1 \hat{Fs} + \lambda_2 \hat{sfc}
\end{equation}
其中 $\lambda_1$、$\lambda_2$ 为比例系数,根据发动机的高性能取决于高单位推力和低耗油率,在两者之间不强调偏重时,一般取 $\lambda_1 = 1$,$\lambda_2 = -1$;$\hat{Fs}$、$\hat{sfc}$ 分别表示单位

推力和低耗油率的标准化值,即 $\hat{F}_{S}=\frac{F_{S}-\bar{F}_{S}}{\sigma_{F_{S}}}, \quad \hat{s f c}=\frac{s f c-\bar{s f c}}{\sigma_{s f c}}$,$\bar{F}_{S} 、 \bar{s f c}$ 分别为相应的均值,$\sigma_{F_{S}} 、 \sigma_{s f c}$ 分别为离散后计算出来的相应标准差。

\begin{table}[h]
\centering
\caption{变量分类情况}
\begin{tabular}{|c|c|c|c|}
\hline
参数 & 说明 & 类型或值 & 范围 \\
\hline
$n_{L}$ & 低压转速 & 工作点 预设 0.85 & $[0,1]$ \\
\hline
$n_{H}$ & 高压转速 & 工作点 预设 0.95 & $[0,1]$ \\
\hline
$Z_{C L}$ & 风扇压比函数值 & 工作点,预设 0.8 & $[0,1]$ \\
\hline
$Z_{C D F S}$ & CDFS 压比函数值 & 工作点,预设 0.8 & $[0,1]$ \\
\hline
$Z_{C H}$ & 高压压气机压比函数值 & 工作点,预设 0.8 & $[0,1]$ \\
\hline
$Z_{T H}$ & 高压涡轮压比函数值 & 工作点,预设 0.8 & $[0,1]$ \\
\hline
$Z_{T L}$ & 低压涡轮压比函数值 & 工作点,预设 0.8 & $[0,1]$ \\
\hline
$T_{4}^{*}$ & 主燃烧室出口温度 & 工作点,预设 1800 & $\leq 2000$ \\
\hline
$\alpha_{L}$ & 风扇导叶角 & 定值,为 0 & $[-5,15]$ \\
\hline
$\alpha_{H}$ & 高压压气机导叶角 & 定值,为 0 & $[-5,15]$ \\
\hline
$\alpha_{T H}$ & 高压涡轮导叶角 & 定值,为 0 & $[-5,15]$ \\
\hline
$A_{6}$ & 后混合器出口总面积(必要时保持不变) & 定值,为 2851.8 & 2851.8 \\
\hline
$A_{225}$ & 选择活门面积 & 定值,$A_{225}=0$ & 0 \\
\hline
$F_{S}$ & 发动机的单位推力 & 性能参数,因变量 & 大于 0 \\
\hline
$s f c$ & 油耗比 & 性能参数,因变量 & 大于 0 \\
\hline
$M_{a}$ & 马赫数 & 自变量 & $[1.1,1.6]$ \\
\hline
$\alpha_{C D F S}$ & CDFS 导叶角 & 自变量 & $[-5,35]$ \\
\hline
$\alpha_{T L}$ & 低压涡轮导叶角 & 自变量 & $[-5,15]$ \\
\hline
$A_{8}$ & 尾喷管面积 & 因变量 & 大于 0 \\
\hline
\end{tabular}
\end{table}

设定马赫数 $M_{a}$ 后,则发动机在特定的工作点状态下,飞机的性能参数和尾喷管面积可以表示为如下函数:
\[
\left\{
\begin{aligned}
F_{S} &= f\left(F^{*}, M_{a}, \alpha_{C D F S}, \alpha_{T L}\right) \\
s f c &= f\left(F^{*}, M_{a}, \alpha_{C D F S}, \alpha_{T L}\right) \\
A_{8} &= f\left(F^{*}, M_{a}, \alpha_{C D F S}, \alpha_{T L}\right)
\end{aligned}
\right.
\]

于是马赫数 $M_{a}$ 在中间目标函数 $Min\{\mathbf{A}\}$ 的条件下可以选出其最小值对应的自变量,计这些自变量为: $\alpha_{CDFS}^{*}, \alpha_{TL}^{*}$,与之对应的尾喷管面积为 $A_{8}^{*}$。因此便可以得到下列表达式:

\[
\begin{cases}
\alpha_{CDFS}^{*} = g_{1}(\text{Ma}) \\
\alpha_{TL}^{*} = g_{2}(\text{Ma}) \\
A_{8}^{*} = g_{3}(\text{Ma})
\end{cases}
\]

此即为 CDFS 导叶角度、低压涡轮导叶角度,尾喷管喉道面积随飞行马赫数的变化规律。

综上所述,模型可建立如下:

目标函数:

\[
\begin{cases}
Min\{\mathbf{A}\} \\
\alpha_{CDFS}^{*} = g_{1}(\text{Ma}) \\
\alpha_{TL}^{*} = g_{2}(\text{Ma}) \\
A_{8}^{*} = g_{3}(\text{Ma})
\end{cases}
\]

约束条件为:

\[
\begin{cases}
F^{*} = F(n_{L}, n_{H}, Z_{CL}, Z_{CDFS}, Z_{CH}, Z_{TH}, Z_{TL}, T_{4}^{*}) \\
Fs = f_{1}(F^{*}, \text{Ma}, \alpha_{CDFS}, \alpha_{TL}) \\
sfc = f_{2}(F^{*}, \text{Ma}, \alpha_{CDFS}, \alpha_{TL}) \\
A_{8} = f_{3}(F^{*}, \text{Ma}, \alpha_{CDFS}, \alpha_{TL}) \\
A = \lambda_{1} \hat{F}s + \lambda_{2} \hat{s}fc
\end{cases}
\]

模型的求解与分析

1) 从算法上考虑,此处用局部最优考虑为原则代替全局最优,需要分析不同工作点下 CDFS 导叶角度、低压涡轮导叶角度,尾喷管喉道面积随飞行马赫数的变化规律。

2) 采用变步长,首先以大步长在 CDFS 导叶角度、低压涡轮导叶角度和马赫数的空间内搜索,对初步结果进行分析,然后再减少步长再在较大的区域内进行搜索。

根据前述模型建立时的表述,具体的工作点可以设置如表 5.5。

\begin{table}[h]
\centering
\caption{工作点设置情况}
\begin{tabular}{c c c c c c c c}
\hline \hline
\multicolumn{8}{c}{工作点 1} \\
\hline
$n_{L}$ & $n_{H}$ & $Z_{CL}$ & $Z_{CDFS}$ & $Z_{CH}$ & $Z_{TH}$ & $Z_{TL}$ & $T_{4}^{*}$ \\
0.85 & 0.95 & 0.8 & 0.6 & 0.8 & 0.4 & 0.3 & 1650 \\
\hline \hline
\multicolumn{8}{c}{工作点 2} \\
\hline
$n_{L}$ & $n_{H}$ & $Z_{CL}$ & $Z_{CDFS}$ & $Z_{CH}$ & $Z_{TH}$ & $Z_{TL}$ & $T_{4}^{*}$ \\
0.85 & 0.95 & 0.8 & 0.8 & 0.8 & 0.8 & 0.8 & 1800 \\
\hline \hline
\multicolumn{8}{c}{工作点 3} \\
\hline
\end{tabular}
\end{table}

\begin{table}[h]
\centering
\begin{tabular}{c c c c c c c}
$n_{L}$ & $n_{H}$ & $Z_{CL}$ & $Z_{CDFS}$ & $Z_{CH}$ & $Z_{TH}$ & $Z_{TL}$ & $T_{4}^{*}$ \\
0.85 & 0.95 & 0.4 & 0.3 & 0.4 & 0.3 & 0.4 & 1800 \\
\end{tabular}
\end{table}

在对不同工作点的计算时,采用的步长如表5.6。

表5.6 步长设置情况

\begin{table}[h]
\centering
\begin{tabular}{c c c c}
 & 马赫数 & CDFS导叶角 & 低压涡轮导叶角 \\
步长 & 0.01 & 1 & 1 \\
计算次数 & 51 & 41 & 21 \\
\end{tabular}
\end{table}

以此种步长计算时,得到三个工作点的情况如表5.7。

表5.7 工作点计算数据

\begin{table}[h]
\centering
\begin{tabular}{c c c c c}
\hline
\multicolumn{5}{c}{工作点1} \\
\hline
马赫数 & 发动机性能参数 & CDFS导叶角 & 低压涡轮导叶角 & 尾喷管喉部面积 \\
\hline
1.1 & 3.733411 & 35 & 15 & 3336.678 \\
1.11 & 3.604633 & 35 & 15 & 3327.756 \\
1.12 & 3.624741 & 35 & 15 & 3318.824 \\
1.13 & 3.718616 & 35 & 15 & 3309.884 \\
1.14 & 3.31263 & 35 & 15 & 3309.884 \\
1.15 & 3.476125 & 35 & 15 & 3309.884 \\
1.16 & 3.640984 & 35 & 15 & 3283.023 \\
1.17 & 3.384982 & 35 & 15 & 3283.023 \\
1.18 & 3.531357 & 35 & 15 & 3283.023 \\
\ldots & \ldots & 35 & 15 & 3283.023 \\
\hline
\multicolumn{5}{c}{工作点2} \\
\hline
马赫数 & 发动机性能参数 & CDFS导叶角 & 低压涡轮导叶角 & 尾喷管喉部面积 \\
\hline
1.1 & 3.725649 & 29 & 15 & 4109.696 \\
1.11 & 3.845916 & 29 & 15 & 4098.765 \\
1.12 & 3.721732 & 29 & 15 & 4087.818 \\
1.13 & 3.290368 & 29 & 15 & 4087.818 \\
1.14 & 3.679136 & 29 & 15 & 4087.818 \\
1.15 & 3.365985 & 29 & 15 & 4087.818 \\
1.16 & 3.658017 & 29 & 15 & 4087.818 \\
1.17 & 3.290452 & 29 & 15 & 4087.818 \\
1.18 & 3.376331 & 29 & 15 & 4087.818 \\
\hline
\multicolumn{5}{c}{工作点3} \\
\hline
马赫数 & 发动机性能参数 & CDFS导叶角 & 低压涡轮导叶角 & 尾喷管喉部面积 \\
\hline
1.1 & 3.4795411 & 29 & 14 & 3369.63 \\
1.11 & 3.398847 & 29 & 14 & 3369.63 \\
1.12 & 3.319939 & 29 & 14 & 3369.63 \\
\hline
\end{tabular}
\end{table}

\begin{table}
\centering
\begin{tabular}{|c|c|c|c|c|}
\hline
1.13 & 3.248891 & 29 & 14 & 3369.63 \\
\hline
1.14 & 3.275426 & 29 & 14 & 3369.63 \\
\hline
1.15 & 3.224918 & 29 & 14 & 3369.63 \\
\hline
1.16 & 3.171814 & 29 & 14 & 3369.63 \\
\hline
1.17 & 3.118685 & 29 & 14 & 3369.63 \\
\hline
1.18 & 3.069921 & 29 & 14 & 3369.63 \\
\hline
1.19 & 3.091455 & 29 & 14 & 3369.63 \\
\hline
1.2 & 3.055757 & 29 & 14 & 3369.63 \\
\hline
1.21 & 3.016904 & 29 & 14 & 3369.63 \\
\hline
1.22 & 2.977054 & 29 & 14 & 3369.63 \\
\hline
1.23 & 2.940393 & 29 & 14 & 3369.63 \\
\hline
1.24 & 2.957802 & 29 & 14 & 3369.63 \\
\hline
1.25 & 2.931563 & 29 & 14 & 3369.63 \\
\hline
\ldots & \ldots & \ldots & \ldots & \ldots \\
\hline
\end{tabular}
\end{table}

以工作点2为例,马赫数在[1.1,1.13]时,尾喷管的喉部面积出现变化,当马赫数超过1.13后,面积值稳定。而在马赫数的所有区间取值上,对应的CDFS导叶角和低压涡轮导叶角都分别恒定为29和15.因此取马赫数在[1.1,1.13]得区间,缩小马赫数步长为0.001,计算得到结果见表5.8。

表5.8 工作点计算结果

\begin{table}
\centering
\begin{tabular}{|c|c|c|c|c|}
\hline
\multicolumn{5}{|c|}{工作点2} \\
\hline
马赫数 & 发动机性能参数 & CDFS导叶角 & 低压涡轮导叶角 & 尾喷管喉部面积 \\
\hline
1.1 & 3.725649 & 29 & 15 & 4109.696 \\
\hline
1.101 & 3.746223 & 29 & 15 & 4108.604 \\
\hline
1.102 & 3.414043 & 29 & 15 & 4108.604 \\
\hline
1.103 & 3.638115 & 29 & 15 & 4106.419 \\
\hline
1.104 & 3.278499 & 29 & 15 & 4106.419 \\
\hline
1.105 & 3.358757 & 29 & 15 & 4106.419 \\
\hline
1.106 & 3.449559 & 29 & 15 & 4106.419 \\
\hline
1.107 & 3.461889 & 29 & 15 & 4106.419 \\
\hline
1.108 & 3.272483 & 29 & 15 & 4106.419 \\
\hline
1.109 & 3.377166 & 29 & 15 & 4106.419 \\
\hline
1.11 & 3.845916 & 29 & 15 & 4098.765 \\
\hline
1.111 & 3.440899 & 29 & 15 & 4098.765 \\
\hline
1.112 & 3.612742 & 29 & 15 & 4098.765 \\
\hline
1.113 & 3.3094 & 29 & 15 & 4098.765 \\
\hline
1.114 & 3.248148 & 29 & 15 & 4098.765 \\
\hline
1.115 & 3.534824 & 29 & 15 & 4098.765 \\
\hline
1.116 & 3.324242 & 29 & 15 & 4098.765 \\
\hline
1.117 & 3.265961 & 29 & 15 & 4098.765 \\
\hline
1.118 & 3.185355 & 29 & 15 & 4098.765 \\
\hline
1.119 & 3.185792 & 29 & 15 & 4098.765 \\
\hline
\end{tabular}
\end{table}

\begin{table}
\centering
\begin{tabular}{c c c c c}
\hline
1.12 & 3.721732 & 29 & 15 & 4087.818 \\
\hline
1.121 & 3.412177 & 29 & 15 & 4087.818 \\
\hline
1.122 & 3.22844 & 29 & 15 & 4087.818 \\
\hline
1.123 & 3.338805 & 29 & 15 & 4087.818 \\
\hline
1.124 & 3.542676 & 29 & 15 & 4087.818 \\
\hline
1.125 & 3.25596 & 29 & 15 & 4087.818 \\
\hline
1.126 & 3.351269 & 29 & 15 & 4087.818 \\
\hline
1.127 & 3.203746 & 29 & 15 & 4087.818 \\
\hline
1.128 & 3.205152 & 29 & 15 & 4087.818 \\
\hline
1.129 & 3.034802 & 29 & 15 & 4087.818 \\
\hline
1.13 & 3.290368 & 29 & 15 & 4087.818 \\
\hline
1.14 & 3.679136 & 29 & 15 & 4087.818 \\
\hline
$\cdots$ & $\cdots$ & 29 & 15 & 4087.818 \\
\hline
\end{tabular}
\end{table}

参照工作点2的处理,可以得到其余两个工作点的数据。可以分别绘图如图5.7和图5.8。

\begin{figure}[h]
\centering
\includegraphics[width=\textwidth]{image1.png}
\caption{工作点1—马赫数与发动机性能、尾喷管面积关系}
\end{figure}

\begin{figure}[h]
\centering
\includegraphics[width=\textwidth]{image2.png}
\caption{工作点2—马赫数与发动机性能、尾喷管面积关系}
\end{figure}

结果分析:
\begin{enumerate}
    \item 对于每一个具体的工作点,在发动机性能最优的条件约束下,CDFS导叶角和低压涡轮导叶角的值并不随马赫数的变化而变化。在性能最优的情况下,CDFS导叶角和低压涡轮导叶角都处在与各自工作点相对应的最优值。表明CDFS导叶角和低压涡轮导叶角的值与马赫数的相关程度为0.23。
    \item 对于每一个具体的工作点,在发动机性能最优的条件约束下,尾喷管的喉部面积随马赫数的出现阶梯性递减的规律,即当马赫数处在某个区间内,尾部喷管面积不变;当马赫数增加到一定值时,尾喷管喉部面积也处在定值。其中,当马赫数从 1.1 开始增加时,尾部喷管面积对马赫数的敏感度较高,当马赫数足够大时,尾部喷管面积较小。工作点 1 的马赫数与尾喷管喉道面积的相关系数为 -0.54852。工作点 2 的马赫数与尾喷管喉道面积的相关系数为 -0.69415。
    \item 为了更好的分析,本文也给出了在具体工作点下发动机性能和马赫数的关系。油耗比是燃油消耗率,它表明发动机每发出单位有效功率,在单位时间内所消耗的燃油质量。当马赫数增加时,发动机的性能的最优性能出现震荡递减的情况,这可以解释为,当马赫数增加时,单位时间内发出有效功率时所消耗的燃油增加,燃油燃烧不够充分。为了保证大功率的输出,需要提高单位时间的耗油量。
\end{enumerate}

根据数据可以看出,当工作点不同时,CDFS 导叶角和低压涡轮导叶角的值与马赫数相互之间的关系不变,但是 CDFS 导叶角的值与具体的工作点有关,工作点 1 下,压比函数值处在中位,CDFS 导叶角的值恒为 35;工作点 2 下,压比函数值处在高位,CDFS 导叶角的值恒为 29;工作点 3 下,压比函数值处在低位,CDFS 导叶角的值恒为 。所以,相比马赫数,CDFS 导叶角与工作点的数值存在比较强的相关关系。

\section{模型评价}

模型一对风扇和 CDFS 两个部件分别进行建模,精确地模拟其特性,此方法不仅解决了实验法高昂成本的问题,同时保证了高仿真精度,具有广泛的应用前景。

模型二对发动机的部件级建模可以精确地模拟各部件的特性,而各部件的连接使整机模型得以建立,此种建模方式易于理解且可以非常有效地模拟发动机的实际工作状态。但由于部件级建模的封装特性,部件间仅依靠接口连接,部件间的参数关系复杂,很难建立可用常规方法求解的方程组。问题二在部件级建模的基础上,将待估参数值离散化,对其进行变域、变步长的搜索,相对于定域、定步长的穷举法,此种搜索方法既能提高计算效率,又能保证计算精度,本文求解结果的收敛性也反映出了问题二的模型与算法的有效性。

对模型三(1),建立优化模型,同时求解平衡方程组,采取遍历法逐个代入参数解算,此模型简捷方便,较容易理解。但是计算量较大,假设对第 $i$ 个参数变量遍历 $n_i$ 次,$i=1,2,\ldots 10$,则计算复杂度为 $O(\prod_{i=1}^{10} n_i)$。因而寻求更为高效的算法是本问题需要进一步研究的内容。

模型三(2)给出在工作点一定时发动机性能最优条件下的计算模型。从数据计算可以看出,工作点对模型的影响比较大。因此,它只能是一个局部最优的计算。本模型采用遍历搜索的算法,这是一个很普通的算法,而且搜索的结果依赖于步长的设计。本模型的优点在于形式简洁,高效。不足之处在于模型的核心部分采用了与前几题相同的部件函数,如果该部件函数部分存在模型误差,也会影响本题模型的计算结果。

[REFERENCES:1]

\end{document}