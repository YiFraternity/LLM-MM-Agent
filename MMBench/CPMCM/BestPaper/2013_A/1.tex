\begin{center}
\textbf{第十届华为杯全国研究生数学建模竞赛}
\end{center}

\begin{table}[h]
\centering
\begin{tabular}{l l}
学校 & 大连理工大学 \\
\hline
参赛队号 & 10141011 \\
\hline
队员姓名 & 1. 尹进 \\
 & 2. 杨东生 \\
 & 3. 卢梦凯 \\
\hline
\end{tabular}
\end{table}

\section{题目}
变循环发动机部件法模型的求解及优化

\begin{abstract}
本文对变循环发动机部件法模型的求解和优化相关问题进行了研究,取得了以下几方面的成果。

1. 进气道、风扇、核心驱动风扇级(CDFS)数学模型的建立。
   \begin{itemize}
   \item 得到了风扇特性数据中流量随压比函数值的变化曲线。在给定飞行高度、马赫数、导叶角度、风扇和 CDFS 的物理转速及压比函数值的情况下,通过建立部件级传递关系模型,计算出了风扇和 CDFS 的出口总温、总压和流量。
   \end{itemize}

2. 变循环发动机非线性方程组的建立和求解。
   \begin{itemize}
   \item 建立了变循环发动机各部件的数学模型,并由发动机的 7 个平衡方程得到了变循环发动机的非线性方程组。并采用 3 种方法对该模型进行求解。
   \item 遗传算法。该算法性能稳定,具有全局搜索能力,能保证结果的正确性,特别是对本模型的强非线性、难得到变量导数等特点具有优势,但其速度较慢。在求解过程中可能会遇到内部一维非线性方程无解情况,对于这种情况需要重新生成种群。考查遗传算法的适应性和稳定性可以发现,其不具有初值敏感性,每次计算的结果都基本一致,并且在前几代收敛速度较快,最大迭代次数为 50 代时可以获得近似最优解。
   \item 信赖域算法。采用 MATLAB 软件的相关模块进行了计算。该算法在每次迭代中给出一个信赖域,在这个邻域内求解子问题得到试探步长,接着用某一评价函数来决定是否接受该试探步以及确定下一次迭代的信赖域。该方法求解本文模型时对初值较敏感,选择合适的初值,迭代 200 次左右可得到最优解。
   \item 提出了一种拟牛顿优选初值方法。该方法建立在拟牛顿方法的基础上,在迭代过程中,可以选取计算结果中出现的最优值来作为下一个迭代过程的初始向量来进行迭代。其优选出来的初始向量 $\mathbf{X}_{ip}$ 的初始 Jacobian 矩阵及其迭代过程中 $\mathbf{H}_i$ 分别由下 2 式形成更新:
   \end{itemize}

\[
\mathbf{K}(\mathbf{X}_{ip}) = \begin{bmatrix}
\frac{f_1(\mathbf{X}_{ip}^1) - f_1(\mathbf{X}_{ip})}{\Delta n_H} & \cdots & \frac{f_1(\mathbf{X}_{ip}^7) - f_1(\mathbf{X}_{ip})}{\Delta T_4^*} \\
\vdots & & \vdots \\
\frac{f_7(\mathbf{X}_{ip}^1) - f_7(\mathbf{X}_{ip})}{\Delta n_H} & \cdots & \frac{f_7(\mathbf{X}_{ip}^7) - f_7(\mathbf{X}_{ip})}{\Delta T_4^*}
\end{bmatrix}
\]

\[
\mathbf{H}_{i+1} = \mathbf{H}_i - \frac{(\mathbf{H}_i \mathbf{y}_i - \mathbf{t}_i \mathbf{p}_i) \mathbf{p}_i^T \mathbf{H}_i}{\mathbf{p}_i^T \mathbf{H}_i \mathbf{y}_i}
\]

为了防止在迭代过程中可能出现的迭代循环现象,在算法中加入了随机的步长系数来消除这一现象。该方法较前两种算法在初值敏感性、迭代收敛性能上都有很好的提高。文中给的算例结果很好的说明了所提出方法的有效性与稳定性。

3. 发动机性能的几何调节优化设计。

\begin{itemize}
\item 对于第一小问,首先将几何调节的优化问题分为内层非线性方程的求解以及外层的优化迭代,根据发动机非线性方程组的特点(高度非线性,很难得到变量导数),内层采用性能稳定的遗传算法求解非线性方程组(补充低压转速作为中间过程变量)。另外提出一种代理模型技术提高求解效率,通过采用拉丁法生成设计点,求解非线性方程组获得样本响应值,然后利用神经网络模型(模拟非线性较好)对样本对进行训练,从而生成近似代理模型。外层对代理模型进行优化设计,以最大单位推力为设计目标以及三个设计变量,同样采用遗传算法优化。最后对收敛解进行非代理模型(遗传算法求解)验证,两者结果相差很小,表明这种优化策略的可行性以及正确性。
\item 对于第二小问,将飞行马赫数进行离散,同时将后混合器内涵道面积作为中间过程变量,对每个离散点采用相同的优化策略,从而可以获得三个设计变量随飞行马赫数的变化规律。计算结果表明 CDFS 导叶角度、低压涡轮导叶角度以及尾喷管喉道面积都是随着飞行马赫数先增大后减小,而最大单位推力是随飞行马赫数单调下降的。
\end{itemize}

关键词:发动机部件法建模;拟牛顿方法;优选初值;遗传算法;神经网络;代理模型

\section{问题背景}

双涵道变循环发动机因可以提供高单位推力时的涡喷循环,也可以提供低马赫数和长航程时的低耗油率涡扇循环,受到了各航空强国的重视。发动机的特性可用实验和计算方法获得,然而实验方法需要研制复杂的设备、投入巨额资金和消耗巨大的能源而不能经常采用。计算机仿真在一定程度上弥补了实验方法的不足,尤其在发动机型号研制过程中,发动机的计算机仿真技术发挥了不可替代的作用。

燃气涡轮发动机是由进气道、压气机、混合器、主燃烧室、涡轮、喷管等部件组成[1],通过计算机对这些部件的性能进行模拟,从而准确的模拟整个发动机的性能。

\section{模型假设}

\begin{enumerate}
\item 题目中所列数据均真实可靠;
\item 发动机各部件建模按照题目中给定的方法;
\item 气流在各个部件的流动都按准一维流动处理,即各个截面气流参数作平均处理。
\end{enumerate}

\section{符号说明}

\begin{table}[h]
\centering
\begin{tabular}{l l}
学校 & 大连理工大学 \\
\hline
参赛队号 & 10141011 \\
\hline
队员姓名 & 1. 尹进 \\
 & 2. 杨东生 \\
 & 3. 卢梦凯 \\
\hline
\end{tabular}
\end{table}

\begin{table}
\centering
\caption{进气道和压气机部件输入输出数据}
\begin{tabular}{c c c}
\hline
部件 & 输入 & 输出 \\
\hline
进气道 & 飞行高度 $H$ & 出口总温 $T_{1}^{*}$ \\
 & 飞行马赫数 $Ma$ & 出口总压 $P_{1}^{*}$ \\
\hline
压气机 & 进口总温 $T_{in}^{*}$ & 出口总温 $T_{out}^{*}$ \\
 & 进口总压 $P_{in}^{*}$ & 出口总压 $P_{out}^{*}$ \\
 & 压比函数值 $zz$ & 出口流量 $W_{a}$ \\
 & 物理转速 $n$ & 功率 $N_{c}$ \\
 & 导叶角度 $\alpha$ & \\
\hline
\end{tabular}
\end{table}

\section{问题 1 模型的建立与求解}

\subsection{问题 1 重述}

1) 请画出附录 4 中风扇特性数据表中流量随压比函数值变化的图形。

2) 设在发动机飞行高度 $H=11km$,飞行马赫数 $Ma=0.8$ 的亚音速巡航点,导叶角度均设置为 $0^\circ$,风扇和 CDFS 的物理转速都为 0.95,风扇和 CDFS 的压比函数值都为 0.5,求风扇和 CDFS 的出口总温、总压和流量。

\subsection{风扇流量随压比函数值的关系}

由压气机增压比 $pr$ 和压比函数值 $zz$ 的关系

\begin{align*}
0 < n_{L} & \leq 1 \\
0 < n_{H} & \leq 1 \\
0 < Z_{CL} & \leq 1 \\
0 < Z_{CDFS} & \leq 1 \\
0 < Z_{CH} & \leq 1 \\
0 < Z_{TH} & \leq 1 \\
0 < Z_{TL} & \leq 1 \\
0 < T_{4}^{*} & < 2000 \\
0 \leq A_{62} & \leq 23212
\end{align*}

可把附录 4 中风扇特性数据表中流量随增压比 $pr$ 的关系换算为流量和压比函数值的变化关系。得到的曲线和曲面图如图 1 和 2 所示。

\begin{figure}[h]
    \centering
    \includegraphics[width=\textwidth]{image.png}
    \caption{不同转速情况下风扇流量随压比函数值变化曲线}
\end{figure}

\begin{figure}[h]
    \centering
    \includegraphics[width=\textwidth]{image.png}
    \caption{风扇流量随压比函数值和转速的变化曲面}
    \label{fig:2}
\end{figure}

从图 1 和图 2 可以看出,在相同的压比函数值的情况下,流量随转速的增加而增加。从图 2 可以看出,在转速从 0.4 到 1 变化时,流量的增加并不是线性增加,而是速度越来越大。当转速超过 1 时,流量的增加速度变缓。

在相同的转速情况下,压比函数值随着流量的减小而增加,增加速度越来越慢。当转速从 0.81 到 1 之间变化时,压比函数值增加到最大值之后,出现随着流量的减小而减小的趋势。

即某些给定的转速和压比函数值点,流量是不唯一的。在后续计算中,取较高的流量。

\subsection{风扇和 CDFS 的出口总温、总压和流量的求解}

若要计算风扇和 CDFS 的出口总温、总压和流量,需要对进气道、风扇和 CDFS 进行建模,采用附录 1 中的计算公式可对这三个模块进行建模。其中风扇和 CDFS 为压气机部件,均采用压气机部件模型。表 1 给出了进气道和压气机的输入和输出参数。

\begin{table}
\centering
\caption{进气道和压气机部件输入输出数据}
\begin{tabular}{c c c}
\hline
部件 & 输入 & 输出 \\
\hline
进气道 & 飞行高度 $H$ & 出口总温 $T_{1}^{*}$ \\
 & 飞行马赫数 $Ma$ & 出口总压 $P_{1}^{*}$ \\
\hline
压气机 & 进口总温 $T_{in}^{*}$ & 出口总温 $T_{out}^{*}$ \\
 & 进口总压 $P_{in}^{*}$ & 出口总压 $P_{out}^{*}$ \\
 & 压比函数值 $zz$ & 出口流量 $W_{a}$ \\
 & 物理转速 $n$ & 功率 $N_{c}$ \\
 & 导叶角度 $\alpha$ & \\
\hline
\end{tabular}
\end{table}

\subsubsection{进气道部件建模}

\begin{enumerate}
\item 由飞行高度 $H$ 计算标准大气条件下环境压力 $p_{0}$ 和环境温度 $T_{0}$,即
\begin{equation}
\begin{cases}
p_{0}=1.01325 \times\left(1-\frac{H}{44.308}\right)^{5.2553} \\
T_{0}=288.15-6.5 \times H
\end{cases}
\tag{2}
\end{equation}
其中,高度 $H$ 的单位为 $km$,温度的单位为 $K$,压力的单位为 $bar$。

\item 由飞行马赫数 $Ma$ 和空气绝热指数 $\gamma$ 计算进气道进口的总温总压
\begin{equation}
\begin{cases}
T_{0}^{*}=T_{0}\left(1+\frac{\gamma-1}{2} Ma^{2}\right) \\
p_{0}^{*}=p_{0}\left(1+\frac{\gamma-1}{2} Ma^{2}\right)^{\frac{\gamma}{\gamma-1}}
\end{cases}
\tag{3}
\end{equation}
其中空气的绝热指数 $\gamma=1.4$。

\item 由马赫数确定计算进气道总压恢复系数 $\sigma_{i}$
\begin{equation}
\begin{cases}
M_{H} \leq 1: \sigma_{i}=1.0 \\
M_{H}>1: \sigma_{i}=1.0-0.075\left(M_{H}-1\right)^{1.35}
\end{cases}
\tag{4}
\end{equation}

\item 计算进气道出口总温总压:
\end{enumerate}

\subsubsection{压气机部件建模}

\begin{enumerate}
\item 由物理转速 \( n \) 和进口总温 \( T_{in}^* \) 计算压气机换算转速

\begin{equation}
n_{cor} = n \sqrt{\frac{T_{in,d}^*}{T_{in}^*}}
\tag{6}
\end{equation}

其中,风扇:\( T_{in,d}^* = 288.15 \),CDFS:\( T_{in,d}^* = 428.56862609 \),高压压气机:\( T_{in,d}^* = 473.603961 \)。

\item 计算压气机增压比、效率和换算流量

\begin{enumerate}
\item 利用线性插值法由压气机(风扇、CDFS 和高压压气机)特性数据表计算出换算转速为 \( n_{cor} \)、压比函数值为 \( z \) 时的特性图上的增压比 \( pr_{c,map} \)、效率 \( \eta_{c,map} \) 和换算流量 \( W_{c,map} \)。

\item 将 \( pr_{c,map} \)、\( \eta_{c,map} \) 和 \( W_{c,map} \) 代入下式修正

\begin{equation}
\begin{cases}
pr_c = C_{pr} (pr_{c,map} - 1)(1 + \frac{k_{pr}}{100} \alpha) + 1 \\
W_c = C_w W_{c,map} (1 + \frac{k_w}{100} \alpha) \\
\eta_c = C_{\eta} \eta_{c,map} (1 + \frac{k_{\eta}^2}{100} \alpha)
\end{cases}
\tag{7}
\end{equation}

其中 \( k_{pr} \)、\( k_w \)、\( k_{\eta} \) 分别是增压比、效率和换算流量的修正系数。CDFS 导叶角变化范围:\( -5^\circ \leq \alpha \leq 35^\circ \),风扇和高压压气机的导叶角变化范围:\( -5^\circ \leq \alpha \leq 15^\circ \);风扇、CDFS 和高压压气机的 \( C_{pr} \)、\( C_w \) 和 \( C_{\eta} \) 参数均已给定。
\end{enumerate}

\item 计算压气机出口参数

压气机出口总压:\( p_{out}^* = p_{in}^* \cdot pr_c \)

计算进口熵:\( \psi_{in} = \psi(T_{in}^*) \),进口焓:\( h_{in} = h(T_{in}^*) \)

压气机出口理想熵:\( \psi_{out,ei} = \psi_{in} + \frac{R}{M} \ln pr_c \),这里 \( M = 1 \),\( R \) 是气体常数

由压气机出口理想熵 \(\psi_{\text{out, ei}} = \psi(T_{\text{out, ei}}^*)\),计算压气机出口理想总温:\(T_{\text{out, ei}}^*\)

计算压气机出口理想焓:\(h_{\text{out, ei}} = h(T_{\text{out, ei}}^*)\)

根据公式 \(\eta_c = \frac{h_{\text{out, ei}} - h_{\text{in}}}{h_{\text{out}} - h_{\text{in}}}\) 计算压气机出口焓 \(h_{\text{out}}\)

由压气机出口焓 \(h_{\text{out}} = h(T_{\text{out}}^*)\) 求压气机出口总温:\(T_{\text{out}}^*\)

计算压气机流量:
\begin{equation}
W_a = W_c \cdot \sqrt{\frac{T_{\text{in, d}}^*}{T_{\text{in}}^*}} \cdot \frac{p_{\text{in}}^*}{p_{\text{in, d}}^*}
\tag{8}
\end{equation}
其中,风扇:\(p_{\text{in, d}}^* = 1.01325\),CDFS:\(p_{\text{in, d}}^* = 3.5464\),高压压气机:\(p_{\text{in, d}}^* = 4.8860\)。

计算压气机功和功率:
\begin{equation}
\begin{cases}
l_c = h_{\text{out}} - h_{\text{in}} \\
N_c = W_a \cdot l_c
\end{cases}
\tag{9}
\end{equation}
\end{enumerate}

\subsubsection{进气道、风扇和 CDFS 整体建模}

进气道、风扇和 CDFS 的整体模型如图 3 所示。

\begin{figure}[h]
\centering
\includegraphics[width=0.8\textwidth]{image.png} % 替换为实际图像文件名
\caption{进气道、风扇和 CDFS 整体模型}
\end{figure}

其中的矩形表示不同的部件,箭头方向表示数据的输入和输出。

由第一问中的已知条件:
\begin{itemize}
\item 飞行高度 \(H = 11 \, \text{km}\),飞行马赫数 \(Ma = 0.8\),导叶角度 \(\alpha_L\)、\(\alpha_{CDFS}\) 均为 0,物理转速 \(n_L\) 和 \(n_H\) 为 0.95,压比函数值 \(Z_{CL}\) 和 \(Z_{CDFS}\) 均为 0.5。带入上述模型中,得到风扇和 CDFS 的出口总温、总压和流量如表 2 所示。
\end{itemize}

\begin{table}[h]
\centering
\caption{风扇和 CDFS 的出口总温、总压和流量}
\begin{tabular}{c c c c}
\hline
 & \(T^*\) & \(P^*\) & \(W_a\) \\
\hline
风扇 & 379.49 & 1.3086 & 19.048 \\
CDFS & 420.52 & 1.8011 & 17.163 \\
\hline
\end{tabular}
\end{table}

\section{问题 2 模型的建立与求解}

\subsection{问题 2 重述}

设在发动机飞行高度 $H=11km$,飞行马赫数 $Ma=0.8$ 的亚音速巡航点,采用双涵道模式,导叶角度均设置为 $0^\circ$,选择活门完全打开,副外涵道面积设为 $1.8395e+003$,后混合器出口总面积设置为 $2.8518e+004$,尾喷管喉道面积 $A_{8}=9.5544e+003$,$n_{L}=0.85$。请运用或设计适当的算法求解由发动机 7 个平衡方程组成的非线性方程组。要求陈述算法的关键步骤及其解释,尽可能讨论算法的有效性。

\subsection{主燃烧室、涡轮、前混合器、后混合器和尾喷管部件建模}

各部件的输入输出数据见表 3 所示。

\begin{table}[h]
\centering
\caption{各部件输入输出数据}
\begin{tabular}{c l l}
\hline
部件 & 输入 & 输出 \\
\hline
主燃烧室 & 进口总温 $T_{3in}^{*}$ & 油气比 $f_{b}$ \\
& 进口总压 $P_{3in}^{*}$ & 燃油流量 $W_{f}$ \\
& 空气流量 $W_{a3in}$ & 出口总压 $P_{4}^{*}$ \\
& 出口温度 $T_{4}^{*}$ & \\
\hline
涡轮 & 进口总温 $T_{in}^{*}$ & 出口总温 $T_{out}^{*}$ \\
& 进口总压 $P_{in}^{*}$ & 出口总压 $P_{out}^{*}$ \\
& 压比函数值 $zz$ & 出口流量 $W_{g}$ \\
& 物理转速 $n$ & 功率 $N_{T}$ \\
& 导叶角度 $\alpha$ & \\
\hline
前混合器 & 副外涵出口总温 $T_{225}^{*}$ & 出口总温 $T_{15}^{*}$ \\
& 副外涵出口总压 $P_{225}^{*}$ & 出口总压 $P_{15}^{*}$ \\
& CDFS 涵道出口总温 $T_{125}^{*}$ & 出口流量 $W_{g15}$ \\
& CDFS 涵道出口总压 $P_{125}^{*}$ & \\
& CDFS 涵道出口流量 $W_{g125}$ & \\
\hline
后混合器 & 低压涡轮出口总温 $T_{5}^{*}$ & 内涵静压 $p_{61}$ \\
& 低压涡轮出口总压 $P_{5}^{*}$ & 外涵静压 $p_{62}$ \\
& 低压涡轮出口流量 $W_{g5}$ & 出口总温 $T_{6}^{*}$ \\
& 主外涵出口总温 $T_{62}^{*}$ & 出口总压 $P_{6}^{*}$ \\
& 主外涵出口总压 $P_{62}^{*}$ & 出口流量 $W_{g6}$ \\
& 主外涵出口流量 $W_{g62}$ & \\
\hline
尾喷管 & 进口总温 $T_{6}^{*}$ & 出口面积 $A_{9}$ \\
& 进口总压 $P_{6}^{*}$ & 出口压力 $p_{9}$ \\
& 进口流量 $W_{g6}$ & 出口气流速度 $c_{9}$ \\
& 油气比 $f_{b}$ & 喉道面积 $A_{8}$ \\
& 飞行高度 $H$ & \\
\hline
\end{tabular}
\end{table}

\subsubsection{主燃烧室部件建模}

\begin{enumerate}
\item 根据公式 $f_{b}=\frac{h_{4}-h_{3}}{\xi_{b} H_{u}+h_{3}}$ 求出主燃烧室出口油气比,其中,$h_{3}$ 和 $h_{4}$ 分别为主燃烧室进出口焓,燃烧效率 $\xi_{b}=0.99$,燃油热值 $H_{u}=42900000$。

\item 计算燃油流量 $W_{f}=W_{a 3} \cdot f_{b}$。

\item 计算出口总压 $p_{4}^{*}=p_{3 in}^{*} \sigma_{b}$,其中主燃烧室总压恢复系数 $\sigma_{b}=0.98$。
\end{enumerate}

\subsubsection{涡轮部件建模}

\begin{enumerate}
\item 求涡轮换算转速

\begin{equation}
n_{\text {cor }}=n \sqrt{\frac{T_{\text {in }, d}^{*}}{T_{\text {in }}^{*}}}
\tag{10}
\end{equation}

其中,高压涡轮:$T_{\text {in }, d}^{*}=1850$,低压涡轮:$T_{\text {in }, d}^{*}=1.5405 \mathrm{e}+003$。

\item 求涡轮的增压比、效率和换算流量

\begin{enumerate}
\item 利用已知特性数据采用线性插值法计算出涡轮的换算转速为 $n_{\text {cor }}$、压比函数值为 $z z$ 时的特性图上的增压比 $p r_{c, \text { map }}$、效率 $\eta_{c, \text { map }}$ 和换算流量 $W_{c, \text { map }}$。

\item 将 (1) 求的特性图上的增压比 $p r_{c, \text { map }}$、效率 $\eta_{c, \text { map }}$ 和换算流量 $W_{c, \text { map }}$ 代入 (2.12) 修正后得到涡轮的增压比、效率和换算流量:

\begin{equation}
\left\{\begin{aligned}
p r_{c} &=C_{p r}\left(p r_{c, \text { map }}-1\right)\left(1+\frac{k_{p r}}{100} \alpha\right)+1 \\
W_{c} &=C_{w} W_{c, \text { map }}\left(1+\frac{k_{w}}{100} \alpha\right) \\
\eta_{c} &=C_{\eta} \eta_{c, \text { map }}\left(1+\frac{k_{\eta}^{2}}{100} \alpha\right)
\end{aligned}\right.
\tag{11}
\end{equation}

其中 $k_{\text {pr}}$、$k_{w}$、$k_{\eta}$ 分别是给定的涡轮增压比、效率和换算流量的修正系数。高、低压涡轮导叶角变化范围:$-5^{\circ} \leq \alpha \leq 15^{\circ}$;高低压涡轮的 $C_{p r}$、$C_{w}$ 和 $C_{\eta}$ 参数均已给定。
\end{enumerate}

\item 根据涡轮换算流量计算涡轮流量:

\begin{equation}
W_{g} = W_{c} \cdot \sqrt{\frac{T_{in,d}^{*}}{T_{in}^{*}}} \cdot \frac{p_{in}^{*}}{p_{in,d}^{*}}
\tag{12}
\end{equation}

其中,高压涡轮:$p_{in,d}^{*} = 28.7297$,低压涡轮:$p_{in,d}^{*} = 11.3371$。

\item 涡轮出口总压 $p_{out}^{*} = p_{in}^{*} / pr_{c}$。

\item 涡轮出口总温 $T_{out}^{*}$ 根据下面公式求出。
\begin{equation}
pr_{c} = \left[ 1 - \left( 1 - \frac{T_{out}^{*}}{T_{in}^{*}} \right) / \eta_{c} \right]^{-\overline{Cp}/R}
\tag{13}
\end{equation}

其中:高压涡轮平均等压比热 $\overline{CP} = 1.2988 \times 10^{3}$,低压涡轮平均等压比热 $\overline{CP} = 1.2745 \times 10^{3}$,$R$ 为气体常数。

\item 求涡轮进口焓 $h_{in} = h(T_{in}^{*}, f_{b})$,其中 $f_{b}$ 为涡轮进口油气比。

\item 求涡轮出口焓 $h_{out} = h(T_{out}^{*}, f_{b})$。

\item 涡轮功和功率:
\begin{equation}
\begin{cases}
l_{T} = h_{in} - h_{out} \\
N_{T} = W_{g} \cdot l_{T} \eta_{m}
\end{cases}
\tag{14}
\end{equation}

其中,$\eta_{m} = 0.99$ 为涡轮机械效率。
\end{enumerate}

\subsubsection{前混合器部件建模}

\begin{figure}[h]
    \centering
    \includegraphics[width=0.8\textwidth]{image.png}
    \caption{前混合器示意图}
    \label{fig:premixer}
\end{figure}

前混合器如图\ref{fig:premixer}所示,其中副外涵和CDFS涵道出口面积已给定,分别为

$A_{225}=1.8395 \mathrm{e}+003$ 和 $A_{125}=608.4252$ 。

\begin{enumerate}
\item CDFS 涵道气流根据流量公式 $W_{g^{125}}=k_{m} \frac{p_{125}^{*}}{\sqrt{T_{125}^{*}}} A_{125} q\left(\lambda_{125}\right)$ 求出 $q\left(\lambda_{125}\right)$ 和 $\lambda_{125}$ 。

\item 由 $p_{125}=p_{125}^{*} \cdot \pi\left(\lambda_{125}\right)$ 求出 CDFS 涵道出口静压 $p_{125}$ 。

\item 由前混合器静压平衡 $p_{225}=p_{125}$ 和 $p_{225}=p_{225}^{*} \cdot \pi\left(\lambda_{225}\right)$, 求出 $\pi\left(\lambda_{225}\right)$ 和 $\lambda_{225}$, $p_{225}^{*}$ 为副外涵道出口总压。

\item 由流量公式 $W_{g^{225}}=k_{m} \frac{p_{225}^{*}}{\sqrt{T_{225}^{*}}} A_{225} q\left(\lambda_{225}\right)$ 计算出副外涵道出口的流量。其中副外涵面积(选择活门面积) $A_{225}=1.8395 \mathrm{e}+003$ 。

\item 由下列公式求出前混合器出口总温 $T_{15}^{*}$ 、总压 $p_{15}^{*}$ 、流量 $W_{g^{15}}$ 。

\begin{equation}
\left\{\begin{aligned}
W_{g^{225}}+W_{g^{125}} & =W_{g^{15}} \\
W_{g^{225}} h_{225}+W_{g^{125}} h_{125} & =W_{g^{15}} h_{15} \\
p_{225}^{*} f\left(\lambda_{225}\right) A_{225}+P_{125}^{*} f\left(\lambda_{125}\right) A_{125} & =p_{15}^{*} f\left(\lambda_{15}\right)\left(A_{225}+A_{125}\right) \\
W_{g^{15}} & =k_{m} \frac{p_{15}^{*}}{\sqrt{T_{15}^{*}}} A_{15} q\left(\lambda_{15}\right)
\end{aligned}\right.
\tag{15}
\end{equation}
\end{enumerate}

\subsubsection{后混合器部件建模}

\begin{enumerate}
\item 内涵气流根据流量公式 $W_{g^{61}}=k_{m} \frac{p_{61}^{*}}{\sqrt{T_{61}^{*}}} A_{61} q\left(\lambda_{61}\right)$ 求出 $q\left(\lambda_{61}\right)$ 和 $\lambda_{61}$, 其中内涵出口面积 $A_{61}=5.3061 \mathrm{e}+003$, $p_{61}^{*}$ 为内涵出口总压, $T_{61}^{*}$ 为内涵出口总温;

\item 外涵气流根据流量公式 $W_{g^{62}}=k_{m} \frac{p_{62}^{*}}{\sqrt{T_{62}^{*}}} A_{62} q\left(\lambda_{62}\right)$ 求出 $q\left(\lambda_{62}\right)$ 和 $\lambda_{62}$, 其中外涵出口面积 $A_{62}=2.3212 \mathrm{e}+004$, $p_{62}^{*}$ 为外涵出口总压, $T_{62}^{*}$ 为外涵出口总温;

\item 计算内涵静压 $p_{61}=p_{61}^{*} \cdot \pi\left(\lambda_{61}\right)$, 计算外涵静压 $p_{62}=p_{62}^{*} \cdot \pi\left(\lambda_{62}\right)$;

\item 由下列公式求出混合器出口总温 $T_{6}^{*}$ 、总压 $p_{6}^{*}$ 、流量 $W_{g^{6}}$ 。

\begin{equation}
\left\{
\begin{aligned}
W_{g61} + W_{g62} &= W_{g6} \\
W_{g61} h_{61} + W_{g62} h_{62} &= W_{g6} h_{6} \\
p_{61}^* f(\lambda_{61}) A_{61} + P_{62}^* f(\lambda_{62}) A_{62} &= p_6^* f(\lambda_6) (A_{61} + A_{62}) \\
W_{g6} &= k_m \frac{p_6^*}{\sqrt{T_6^*}} A_6 q(\lambda_6)
\end{aligned}
\right.
\tag{16}
\end{equation}
\end{enumerate}

\subsubsection{尾喷管部件建模}

\begin{enumerate}
\item 计算尾喷管喉道面积 $A_8$,出口面积 $A_9$。

假设尾喷管始终处于临界或超临界状态,即速度系数 $\lambda_8 = 1$。

\begin{enumerate}
\item 由流量公式 $W_{g8} = k_m \frac{p_8^*}{\sqrt{T_8^*}} A_8 q(\lambda_8)$ 计算出 $A_8$。

\item $p_9 = p_0$,计算出 $\pi(\lambda_9) = p_9 / p_9^*$,并求出 $\lambda_9$。

\item 由流量公式 $W_{g9} = k_m \frac{p_9^*}{\sqrt{T_9^*}} A_9 q(\lambda_9)$ 计算出 $A_9$。

\item 判断 $\frac{A_9}{A_8} > ar$(这里 $ar = 3$),如果是,则 $A_9 = 3A_8$,利用流量公式重新计算 $\lambda_9$。
\end{enumerate}

\item 计算尾喷管出口静温 $T_9 = T_7^* \tau(\lambda_9)$。

\item 尾喷管出口气流速度 $c_9 = \varphi_{nz} \sqrt{2 \left( h_9^* - h_9 \right)}$,其中:$\varphi_{nz} = 0.98$,焓 $h_9^*$ 和焓 $h_9$ 分别由尾喷管出口总温和静温求出。
\end{enumerate}

\subsection{发动机整体建模}

发动机的整体模型如图 5 所示。

\begin{figure}[h]
    \centering
    \includegraphics[width=\textwidth]{engine_model_diagram.png}
    \caption{发动机整体模型示意图}
    \label{fig:engine_model}
\end{figure}

如图所示,在飞行高度 $H$ 和飞行马赫数 $Ma$ 已知的情况下,发动机模型有多个输入参数,比如压气机和涡轮的风扇导叶角、压比函数值、转速及主燃烧室出口温度,如图中红色箭头和符号所示。在给定输入参数的情况下,可以得到输出参数,如压气机和涡轮的功率、流量和压强等。

发动机各部件在匹配工作时,输入参数应使如下平衡方程满足。

\begin{enumerate}
\item 低压轴功率平衡
\begin{equation}
N_{CL} - N_{TL} \eta_{mL} = 0
\tag{17}
\end{equation}
其中 $\eta_{mL} = 0.99$ 为中间轴机械效率。

\item 高压轴功率平衡
\begin{equation}
N_{CH} + N_{CDFS} - N_{TH} \eta_{mH} = 0
\tag{18}
\end{equation}
其中 $\eta_{mH} = 0.99$ 是高速轴的机械效率。

\item 高压涡轮进口截面流量平衡
\begin{equation}
W_{g41} - W'_{g41} = 0
\tag{19}
\end{equation}
其中 $W_{g41}$ 是高压涡轮进口截面气体流量,即主燃烧室出口气体流量和冷却空气流量,$W'_{g41}$ 是通过高压涡轮特性数据线性插值得到的高压涡轮流量,这里忽略冷却的空气流量。

\item 低压涡轮进口截面流量平衡
\begin{equation}
W_{g45} - W'_{g45} = 0
\tag{20}
\end{equation}
与高压涡轮进口截面流量平衡类似。

\item 后混合器静压平衡

\begin{equation}
p_{61} - p_{62} = 0
\tag{21}
\end{equation}

其中 $p_{61}$ 和 $p_{62}$ 分别是后混合器内、外涵道(主外涵道)的静压,二者应平衡。

\item 尾喷管面积平衡

\begin{equation}
A_8 - A_8' = 0
\tag{22}
\end{equation}

其中 $A_8'$ 为给定的尾喷管 8 截面的面积,这里 $A_8' = 9.4575 \times 10^3$,$A_8$ 为计算出的尾喷管 8 截面的面积,二者应相等。

\item 风扇出口流量平衡

\begin{equation}
W_{a2} - W_{a21} - W_{a13} = 0
\tag{23}
\end{equation}

其中风扇出口的流量 $W_{a2}$ 分流为副外涵流量 $W_{a13}$ 和 CDFS 进口流量 $W_{a21}$,三者之间应存在平衡关系。
\end{enumerate}

\subsection{发动机非线性方程组的建立}

在问题二中,已知飞行高度 $H = 11 \, \text{km}$ 和飞行马赫数 $Ma = 0.8$ 的情况下,又给定了下列已知参数

\begin{table}[h]
\centering
\caption{问题二中给定参数}
\begin{tabular}{c c}
\hline \hline
变量 & 值 \\
\hline
$\alpha_L, \, \alpha_{CDFS}, \, \alpha_H, \, \alpha_{TH}, \, \alpha_{CH}$ & 0 \\
$n_L$ & 0.85 \\
\hline \hline
\end{tabular}
\end{table}

即在整体模型中的 13 个自变量中给定了 6 个,剩余 7 个自变量,由 7 个平衡方程可组成 7 个自由度的非线性方程组。

由于各方程中为不同的量,各量之间量级相差较大,为了更好的判断各方程是否满足平衡,把各方程做如下处理

\begin{equation}
\begin{aligned}
& \frac{N_{CL} - N_{TL} \eta_{mL}}{\left( N_{CL}^2 + \left( N_{TL} \eta_{mL} \right)^2 \right)^{1/2}} = 0, \quad \frac{N_{CH} + N_{CDFS} - N_{TH} \eta_{mH}}{\left( \left( N_{CH} + N_{CDFS} \right)^2 + \left( N_{TH} \eta_{mH} \right)^2 \right)^{1/2}} = 0 \\
& \frac{W_{g41} - W_{g41}'}{\left( W_{g41}^2 + W_{g41}'^2 \right)^{1/2}} = 0, \quad \frac{W_{g45} - W_{g45}'}{\left( W_{g45}^2 + W_{g45}'^2 \right)^{1/2}} = 0 \\
& \frac{p_{61} - p_{62}}{\left( p_{61}^2 + p_{62}^2 \right)^{1/2}} = 0, \quad \frac{A_8 - A_8'}{\left( A_8^2 + A_8'^2 \right)^{1/2}} = 0, \quad \frac{W_{a2} - W_{a21} - W_{a13}}{\left( W_{a2}^2 + \left( W_{a21} + W_{a13} \right)^2 \right)^{1/2}} = 0
\end{aligned}
\tag{24}
\end{equation}

\subsection{模型的遗传算法求解}

由于本问题的非线性方程较难求出未知量的梯度信息,可以将其转化为最优化问题,然后用零阶优化算法(遗传算法)求解。转换后的优化模型为:

\begin{equation}
\begin{aligned}
& \min \quad \sqrt{\sum_{i=1}^{7} f_{i}^{2}} \\
& \text{s.t.} \quad f_{1} = (N_{CL} - N_{TL} \eta_{mL}) / \sqrt{N_{CL}^{2} + N_{TL}^{2} \eta_{mL}^{2}} \\
& \quad \quad \quad f_{2} = (N_{CH} + N_{CDFS} - N_{TH} \eta_{mH}) / \sqrt{(N_{CH} + N_{CDFS})^{2} + N_{TH}^{2} \eta_{mH}^{2}} \\
& \quad \quad \quad f_{3} = (W_{g41} - W_{g41}') / \sqrt{W_{g41}^{2} + W_{g41}'^{2}} \\
& \quad \quad \quad f_{4} = (W_{g45} - W_{g45}') / \sqrt{W_{g45}^{2} + W_{g45}'^{2}} \\
& \quad \quad \quad f_{5} = (p_{61} - p_{62}) / \sqrt{p_{61}^{2} + p_{62}^{2}} \\
& \quad \quad \quad f_{6} = (A_{8} - A_{8}') / \sqrt{A_{8}^{2} + A_{8}'^{2}} \\
& \quad \quad \quad f_{7} = (W_{a2} - W_{a21} - W_{a13}) / \sqrt{W_{a2}^{2} + (W_{a21} + W_{a13})^{2}} \\
& \quad \quad \quad 0 < n_{L} \leq 1 \\
& \quad \quad \quad 0 < n_{H} \leq 1 \\
& \quad \quad \quad 0 < Z_{CL} \leq 1 \\
& \quad \quad \quad 0 < Z_{CDFS} \leq 1 \\
& \quad \quad \quad 0 < Z_{CH} \leq 1 \\
& \quad \quad \quad 0 < Z_{TH} \leq 1 \\
& \quad \quad \quad 0 < Z_{TL} \leq 1 \\
& \quad \quad \quad 0 < T_{4}^{*} < 2000
\end{aligned}
\tag{34}
\end{equation}

其中,$f_{i}$ 为各个方程的残差,为了消除各个方程量级的差别,对残差进行归一化处理;优化目标为各个方程残差的二范数最小;七个未知量为设计变量,取值范围由赛题附录 1 中给出。另外,导叶角度均设为 0,低压转速 $n_{L}$ 为 0.85,题干中已给出。

遗传算法是一种智能算法,又称为现代启发式算法,是一种理论上具有全局优化性能、通用性强、不需要函数的导数且适用于并行处理的随机化搜索算法。这种算法具有严密的理论依据,而不是单纯凭借经验,理论上可以在一定时间内找到最优解或近似最优解,并且不具有初值敏感性(初始解是随机生成的),适应性和稳定性更强。由于传统梯度算法对于函数不具有可导性(离散变量),求解导数的复杂性,以及容易陷入局部最优解等问题会面临困难,另外对于强非线性问题,遗传算法更具有其优势。

遗传算法模拟自然选择和自然遗传过程中发生的繁殖、交叉和基因突变现象,在每次迭代中都保留一组候选解,并按某种指标从解群中选取较优的个体,利用遗传算子(选择、交叉和变异)对这些个体进行组合,产生新一代的候选解群,重复此过程,直到满足某种收敛指标为止。算法的计算流程图如图 6 所示,算法的关键步骤是:

\begin{enumerate}
\item 编码(产生初始种群)

通过某种编码机制把对象抽象成由特定符号按一定顺序排成的串,本文使用二进制串进行编码。并采用随机方法生成若干个体的集合,即作为初始种群,本文的采用 40 作为种群规模[5]。

\item 计算适应度函数

遗传算法对一个个体(解)的好坏用适应度函数值来评价,适应度函数值越大,解的质量越好,它是进化过程的驱动力,也是进行自然选择的唯一标准。本文采用目标值(七个方程总残差)在种群内归一化后作为适应度函数。另外,在计算过程中,如果内部一些一维非线性方程的求解(例如已知气动函数求解速度参数)可能会遇到在实数范围内无解,并且得不到适应度函数,这样的种群个体要进行重新随机生成,并重新计算适应度函数。

\item 遗传算子(选择、交叉、变异)

选择算子是使用选择运算来实现对群体中的个体进行优胜劣汰操作,适应度高的个体被遗传到下一代群体中的概率大,选择的任务就是按某种方法从父代群体中选取一些个体,遗传到下一代群体。本文选择算子采用轮盘赌方法,其基本思想是各个个体被选中的概率与其适应度函数值大小成正比,本文采用 90\% 的选择率。

交叉算子是指对两个相互配对的染色体依据交叉概率按某种方式相互交换其部分基因,从而形成两个新的个体,是产生新个体的主要方法,在遗传算法中起重要作用。本文采用单点交叉算子,交叉概率为 0.7。

变异算子是指依据变异概率将个体编码串中的某些基因值用其它基因值来替换,从而形成一个新的个体,它是产生新个体的辅助方法,决定了遗传算法的局部搜索能力,同时保持种群的多样性。交叉运算和变异运算的相互配合,可以共同完成对搜索空间的全局搜索和局部搜索。本文采用基本位变异算子,变异概率为 0.9。
\end{enumerate}

\begin{figure}[h]
\centering
\begin{tikzpicture}[node distance=2cm, auto]
    \tikzstyle{block} = [rectangle, draw, text width=8em, text centered, rounded corners, minimum height=4em]
    \tikzstyle{line} = [draw, -latex']
    \tikzstyle{branch} = [fill,shape=circle,minimum size=5pt,inner sep=0pt]

    % 节点
    \node [block] (init) {产生初始种群};
    \node [block, below of=init] (check) {是否满足停止准则};
    \node [block, below of=check] (fitness) {计算个体适应度值};
    \node [block, below of=fitness] (select) {比例选择运算};
    \node [block, below of=select] (crossover) {单点交叉运算};
    \node [block, below of=crossover] (mutation) {基本位变异运算};
    \node [block, below of=mutation] (newpop) {产生新一代种群};
    \node [block, left of=check, node distance=5cm] (output) {输出结果并结束};

    % 箭头
    \path [line] (init) -- (check);
    \path [line] (check) -- node [near start] {否} (fitness);
    \path [line] (fitness) -- (select);
    \path [line] (select) -- (crossover);
    \path [line] (crossover) -- (mutation);
    \path [line] (mutation) -- (newpop);
    \path [line] (newpop) |- (check);
    \path [line] (check) -| node [near start] {是} (output);

    % 虚线框
    \draw [red, dashed] (select.north west) rectangle (mutation.south east);
\end{tikzpicture}
\caption{遗传算法计算流程图}
\end{figure}

本文采用最大迭代次数作为停止准则。

首先讨论算法的稳定性和适应性,将最大迭代次数设为 50,分别计算四次,考查这几次计算的收敛性,如图 7 所示,以及计算结果对比,如表 5,表 6 所

\begin{figure}[h]
    \centering
    \begin{subfigure}[t]{0.45\textwidth}
        \includegraphics[width=\textwidth]{image_a.png}
        \caption{第一次计算收敛曲线}
    \end{subfigure}
    \hfill
    \begin{subfigure}[t]{0.45\textwidth}
        \includegraphics[width=\textwidth]{image_b.png}
        \caption{第二次计算收敛曲线}
    \end{subfigure}
    \vspace{1em}
    \begin{subfigure}[t]{0.45\textwidth}
        \includegraphics[width=\textwidth]{image_c.png}
        \caption{第三次计算收敛曲线}
    \end{subfigure}
    \hfill
    \begin{subfigure}[t]{0.45\textwidth}
        \includegraphics[width=\textwidth]{image_d.png}
        \caption{第四次计算收敛曲线}
    \end{subfigure}
    \caption{四次计算遗传算法收敛曲线}
    \label{fig:convergence_curves}
\end{figure}

\begin{table}[h]
    \centering
    \caption{各个方程收敛时的各方程归一化残差及总残差对比}
    \label{tab:residuals}
    \begin{tabular}{c c c c c c c c}
        \hline
        ID & 方程1 & 方程2 & 方程3 & 方程4 & 方程5 & 方程6 & 方程7 & 总残差 \\
        \hline
        1 & -0.0085 & -0.0044 & -0.0163 & 0.0300 & -0.0218 & 0.0602 & 0.0227 & 0.0767 \\
        2 & -0.0452 & -0.0170 & 0.0690 & -0.0048 & -0.0460 & 0.0919 & 0.0250 & 0.1353 \\
        3 & -0.0097 & -0.0060 & 0.0423 & -0.0116 & -0.0102 & 0.0799 & 0.0193 & 0.0944 \\
        4 & 0.0261 & 0.0140 & 0.0489 & 0.0029 & -0.0221 & 0.0534 & 0.0402 & 0.0908 \\
        \hline
    \end{tabular}
\end{table}

\begin{table}[h]
    \centering
    \caption{收敛时七个未知量以及发动机性能对比}
    \label{tab:performance}
    \begin{tabular}{c c c c c c c c c c}
        \hline
        ID & $n_{H}$ & $Z_{CL}$ & $Z_{CDFS}$ & $Z_{CH}$ & $Z_{TH}$ & $Z_{TL}$ & $T_{4}^{*}$ & 单位推力 & 耗油率 \\
        \hline
        1 & 0.9023 & 0.2495 & 0.4391 & 0.4345 & 0.2172 & 0.1413 & 1448.9 & 550.10 & 0.1006 \\
        2 & 0.8353 & 0.4398 & 0.6398 & 0.5168 & 0.1205 & 0.1625 & 1644.3 & 590.33 & 0.1015 \\
        3 & 0.8389 & 0.4494 & 0.6779 & 0.5164 & 0.1322 & 0.1483 & 1552.9 & 568.50 & 0.0967 \\
        4 & 0.8388 & 0.4873 & 0.8458 & 0.4703 & 0.1368 & 0.1415 & 1510.8 & 562.38 & 0.0972 \\
        \hline
    \end{tabular}
\end{table}

计算结果可以看出,四次计算的初始值都是随机生成的,收敛解基本一致,并都保持非常低的总残差,并且四次计算得到的单位推力和耗油率都相差不大,可以说明遗传算法并没有初值敏感性,适应性和通用性强,并且算法稳定,适合不同参数的多次计算以及后续的参数优化设计。

然后讨论算法的收敛性,将最大迭代次数设为 300,图 8 为收敛曲线,考

\begin{figure}[h]
    \centering
    \includegraphics[width=\textwidth]{image.png} % 替换为实际图像文件名
    \caption{最大迭代次数为300次时的收敛曲线}
    \label{fig:convergence_curve}
\end{figure}

\begin{table}[h]
    \centering
    \caption{最大迭代次数为300次时的各方程归一化残差及总残差}
    \label{tab:residuals}
    \begin{tabular}{c c c c c c c c}
        \hline
        方程1 & 方程2 & 方程3 & 方程4 & 方程5 & 方程6 & 方程7 & 总残差 \\
        \hline
        0.0044 & 0.0082 & -0.0106 & 0.0220 & 0.0010 & 0.0430 & 0.0099 & 0.0513 \\
        \hline
    \end{tabular}
\end{table}

\begin{table}[h]
    \centering
    \caption{最大迭代次数为300次时七个未知量以及发动机性能}
    \label{tab:engine_performance}
    \begin{tabular}{c c c c c c c c}
        \hline
        $n_{H}$ & $Z_{CL}$ & $Z_{CDFS}$ & $Z_{CH}$ & $Z_{TH}$ & $Z_{TL}$ & $T_{4}^{*}$ & 耗油率 \\
        \hline
        0.8817 & 0.3351 & 0.5021 & 0.4611 & 0.2010 & 0.1309 & 1431.5 & 558.35 & 0.0977 \\
        \hline
    \end{tabular}
\end{table}

从计算结果可以看出,迭代次数越大,总残差越小,但是基本前50次迭代速度很快,后面的迭代时总残差变化不是很大,而且和最大迭代次数为50次相比,得到的收敛解基本一致,单位推力和耗油率也相对变化很小,也说明遗传算法的有效性,具有全局搜索能力,可以找到近似最优解。

\subsection{模型的信赖域算法}

信赖域方法和线性搜索方法是求解非线性优化问题的两类主要数值算法。信赖域方法从给定的初始解出发,通过迭代、改进直到获得满意的近似最优解为止。信赖域方法具有较强的可靠性、有效性和收敛性。

与线性搜索方法相比,信赖域方法直接通过模型求解得到试探步长,而不是先确定搜索方向,再寻找步长。

调用 MATLAB 软件的信赖域 Dogleg 路径算法相关模块 \cite{cite1, cite2},选取不同的初值进行计算,如表 \ref{tab:initial_values} 所示。计算结果如表 \ref{tab:results} 所示,其对应的各方程的残差如表 \ref{tab:equation_residuals} 所示。各个情况下的方程总残差的收敛情况如图 \ref{fig:residual_convergence} 所示。

\begin{table}
\centering
\caption{表9 各计算情况的初值}
\begin{tabular}{c c c c c c c}
\hline
ID & $n_{H}$ & $Z_{CL}$ & $Z_{CDFS}$ & $Z_{CH}$ & $Z_{TH}$ & $Z_{TL}$ & $T_{4}^{*}$ \\
\hline
1 & 0.9 & 0.4 & 0.4 & 0.1 & 0.1 & 0.1 & 1840 \\
2 & 0.9 & 0.4 & 0.4 & 0.2 & 0.2 & 0.2 & 1840 \\
3 & 0.9 & 0.5 & 0.5 & 0.3 & 0.3 & 0.3 & 1840 \\
4 & 0.9 & 0.5 & 0.5 & 0.3 & 0.3 & 0.3 & 1940 \\
5 & 0.9 & 0.5 & 0.5 & 0.4 & 0.4 & 0.4 & 1940 \\
6 & 0.85 & 0.4 & 0.4 & 0.4 & 0.2 & 0.2 & 1940 \\
\hline
\end{tabular}
\end{table}

\begin{table}
\centering
\caption{表10 各计算情况对应的计算结果}
\begin{tabular}{c c c c c c c c}
\hline
ID & $n_{H}$ & $Z_{CL}$ & $Z_{CDFS}$ & $Z_{CH}$ & $Z_{TH}$ & $Z_{TL}$ & $T_{4}^{*}$ \\
\hline
1 & 0.85639 & 0.63037 & 0.95008 & 0.50293 & 0.17132 & 0.12949 & 1450.4 \\
2 & 0.90656 & 0.40553 & 0.41159 & 0.20708 & 0.19602 & 0.17513 & 1833 \\
3 & 0.85638 & 0.63002 & 0.94838 & 0.5029 & 0.17133 & 0.12951 & 1450.3 \\
4 & 0.85639 & 0.63037 & 0.95008 & 0.50293 & 0.17132 & 0.12949 & 1450.4 \\
5 & 0.85639 & 0.63037 & 0.95008 & 0.50293 & 0.17132 & 0.12949 & 1450.4 \\
6 & 0.84332 & 0.72987 & 0.96375 & 0.46427 & 0.15494 & 0.15052 & 1605 \\
\hline
\end{tabular}
\end{table}

\begin{table}
\centering
\caption{表11 各计算情况对应的各方程的归一化残差}
\begin{tabular}{c c c c c c c c}
\hline
ID & 方程1 & 方程2 & 方程3 & 方程4 & 方程5 & 方程6 & 方程7 \\
\hline
1 & -7.52E-12 & 6.27E-13 & -3.26E-14 & -3.32E-14 & -1.21E-10 & -7.68E-10 & 3.05E-09 \\
2 & -7.32E-02 & -1.19E-02 & 1.66E-01 & 1.79E-01 & 2.38E-02 & 5.24E-02 & -9.75E-02 \\
3 & -2.26E-06 & 7.41E-07 & -4.78E-07 & -4.25E-07 & 2.34E-03 & 1.05E-02 & -3.82E-02 \\
4 & 2.56E-12 & -1.84E-13 & 1.08E-14 & 1.13E-14 & -4.97E-12 & -3.55E-11 & 1.43E-10 \\
5 & 1.38E-12 & 9.68E-14 & -5.67E-15 & -5.89E-15 & 2.87E-12 & 4.54E-11 & -1.83E-10 \\
6 & -3.04E-02 & -2.78E-02 & 6.53E-02 & 3.81E-02 & -2.50E-02 & 3.27E-02 & -5.62E-02 \\
\hline
\end{tabular}
\end{table}

\begin{figure}[h]
    \centering
    \includegraphics[width=\textwidth]{image_a.png}
    \caption{情况 1~6 分别对应的方程总残差收敛情况}
    \label{fig:9}
\end{figure}

图 9 情况 1~6 分别对应的方程总残差收敛情况,其中方程总残差定义为方程各个残差的模。

由上述结果来看,第 1、3、4 和 5 这几种情况,均收敛到了较为理想的结果。而 2 和 6 收敛解的误差相对较大。

从计算情况来看,该方法对初值较为敏感,对于某些初值,收敛不到最优结果。从迭代步数来看,该方法迭代次数较多,约在 200 次左右,对某些初值,出现了迭代 500 多次的情况。

\subsection{模型的拟牛顿优选初值求解方法}

针对上述非线性模型,本文提出了一种基于拟牛顿方法,在迭代过程中可以优选初值的一种求解方法,模型的非线性方程组可以表示如下:

\begin{equation}
\begin{cases}
f_{1}\left(n_{H}, Z_{CL}, Z_{CDFS}, Z_{CH}, Z_{TH}, Z_{TL}, T_{4}^{*}\right)=0 \\
f_{2}\left(n_{H}, Z_{CL}, Z_{CDFS}, Z_{CH}, Z_{TH}, Z_{TL}, T_{4}^{*}\right)=0 \\
\vdots \\
f_{7}\left(n_{H}, Z_{CL}, Z_{CDFS}, Z_{CH}, Z_{TH}, Z_{TL}, T_{4}^{*}\right)=0
\end{cases}
\tag{26}
\end{equation}

记为向量形式为

\begin{equation}
\mathbf{F}(\mathbf{X})=\mathbf{0}
\tag{27}
\end{equation}

可以表示为在 $\mathbf{X}_{0}$ 处的展开形式

\begin{equation}
\mathbf{F}(\mathbf{X})=\mathbf{F}(\mathbf{X}_{0})+\mathbf{K}(\mathbf{X}_{0})(\mathbf{X}-\mathbf{X}_{0})
\tag{28}
\end{equation}

$\mathbf{K}(\mathbf{X}_{0})$ 代表系统在 $\mathbf{X}_{0}$ 处的近似雅克比矩阵。在计算过程中,需要迭代进行,具体过程如下:

\begin{equation}
\left(\mathbf{X}_{i+1}-\mathbf{X}_{i}\right)=\mathbf{H}\left(\mathbf{X}_{i}\right)\left(\mathbf{F}\left(\mathbf{X}_{i+1}\right)-\mathbf{F}\left(\mathbf{X}_{i}\right)\right)
\tag{29}
\end{equation}

其中 $\mathbf{H}\left(\mathbf{X}_{i}\right)=\mathbf{K}\left(\mathbf{X}_{i}\right)^{-1}$,在迭代过程中,更新的梯度阵有利于加速收敛速度,本文中,我们采取如下的更新策略来更新 $\mathbf{K}\left(\mathbf{X}_{i}\right)$

\begin{equation}
\begin{aligned}
\mathbf{H}_{i+1} & =\mathbf{H}_{i}-\frac{\left(\mathbf{H}_{i} \mathbf{y}_{i}-\mathbf{t}_{i} \mathbf{p}_{i}\right) \mathbf{p}_{i}^{T} \mathbf{H}_{i}}{\mathbf{p}_{i}^{T} \mathbf{H}_{i} \mathbf{y}_{i}} \\
\mathbf{y}_{i} & =\mathbf{F}\left(\mathbf{X}_{i+1}\right)-\mathbf{F}\left(\mathbf{X}_{i}\right) \\
\mathbf{X}_{i+1} & =\mathbf{X}_{i}+t_{i} \mathbf{p}_{i} \\
\mathbf{p}_{i} & =\mathbf{H}_{i} \mathbf{F}\left(\mathbf{X}_{i}\right)
\end{aligned}
\tag{30}
\end{equation}

对于 $\mathbf{K}\left(\mathbf{X}_{0}\right)$ 的求取,采用

\begin{equation}
\begin{aligned}
\mathbf{X}_{0}^{1} & =\left\{n_{H}^{0}+\Delta n_{H}, Z_{CL}^{0}, Z_{CDFS}^{0}, Z_{CH}^{0}, Z_{TH}^{0}, Z_{TL}^{0}, T_{4}^{*0}\right\} \\
\mathbf{X}_{0}^{2} & =\left\{n_{H}^{0}, Z_{CL}^{0}+\Delta Z_{CL}, Z_{CDFS}^{0}, Z_{CH}^{0}, Z_{TH}^{0}, Z_{TL}^{0}, T_{4}^{*0}\right\} \\
& \vdots \\
\mathbf{X}_{0}^{7} & =\left\{n_{H}^{0}, Z_{CL}^{0}, Z_{CDFS}^{0}, Z_{CH}^{0}, Z_{TH}^{0}, Z_{TL}^{0}, T_{4}^{*0}+\Delta T_{4}^{*}\right\}
\end{aligned}
\tag{31}
\end{equation}

根据 $\mathbf{X}_{0}, \mathbf{X}_{0}^{1}, \mathbf{X}_{0}^{2}, \ldots, \mathbf{X}_{0}^{7}$,可得到 $\mathbf{F}\left(\mathbf{X}_{0}\right), \mathbf{F}\left(\mathbf{X}_{0}^{1}\right), \mathbf{F}\left(\mathbf{X}_{0}^{2}\right), \ldots, \mathbf{F}\left(\mathbf{X}_{0}^{7}\right)$,根据雅克比导数的定义的可以得到
\begin{equation}
\mathbf{K}(\mathbf{X}_0) =
\begin{bmatrix}
\frac{f_1(\mathbf{X}_0^1) - f_1(\mathbf{X}_0)}{\Delta n_H} & \frac{f_1(\mathbf{X}_0^2) - f_1(\mathbf{X}_0)}{\Delta Z_{CL}} & \cdots & \frac{f_1(\mathbf{X}_0^7) - f_1(\mathbf{X}_0)}{\Delta T_4^*} \\
\frac{f_2(\mathbf{X}_0^1) - f_2(\mathbf{X}_0)}{\Delta n_H} & \frac{f_2(\mathbf{X}_0^2) - f_2(\mathbf{X}_0)}{\Delta Z_{CL}} & \cdots & \frac{f_2(\mathbf{X}_0^7) - f_2(\mathbf{X}_0)}{\Delta T_4^*} \\
\vdots & \vdots & & \vdots \\
\frac{f_7(\mathbf{X}_0^1) - f_7(\mathbf{X}_0)}{\Delta n_H} & \frac{f_7(\mathbf{X}_0^2) - f_7(\mathbf{X}_0)}{\Delta Z_{CL}} & \cdots & \frac{f_7(\mathbf{X}_0^7) - f_7(\mathbf{X}_0)}{\Delta T_4^*}
\end{bmatrix}
\tag{32}
\end{equation}

算法实施步骤:

Step 1: 给定初值 $\mathbf{X}_0$,设置算法参数,最大迭代步 MaxItr,收敛精度 Tol,间隔步数 dN,步长系数 $\Delta X$,搜索系数 $t_i$,总残差 EPS。

Step 2: 根据给定的初值 $\mathbf{X}_0$ 和步长系数 $\Delta X$(如:$\Delta n_H = \Delta X \cdot n_H^0$),生成各个 $\mathbf{X}_0^1, \mathbf{X}_0^2, \ldots, \mathbf{X}_0^7$,调用模型函数 main1_v4(见附录 A),生成 $f_1, f_2, \ldots, f_7$ 的值。然后根据方程 XX 求得初始的 $\mathbf{K}(\mathbf{X}_0)$,进而求得初始 $\mathbf{H}(\mathbf{X}_0)$

Step 3: 循环迭代

While(EPS>Tol&&itn<MaxItr)

Step 3-1: 根据求 $\mathbf{p}_i = \mathbf{H}_i \mathbf{F}(\mathbf{X}_i)$,$\mathbf{X}_{i+1} = \mathbf{X}_i + t_i \mathbf{p}_i$,此处对各变量上下限进行限制。调用模型函数 main1_v4,得到 $\mathbf{F}(\mathbf{X}_{i+1})$,继而得到 $\mathbf{y}_i = \mathbf{F}(\mathbf{X}_{i+1}) - \mathbf{F}(\mathbf{X}_i)$,更新下一个迭代步的 $\mathbf{H}_{i+1} = \mathbf{H}_i - \frac{(\mathbf{H}_i \mathbf{y}_i - t_i \mathbf{p}_i) \mathbf{p}_i^T \mathbf{H}_i}{\mathbf{p}_i^T \mathbf{H}_i \mathbf{y}_i}$。

If(mod(itn,dN)==0)

Step 3-2: 从前述运行结果 $\mathbf{X}_0 \square \mathbf{X}_{itn}$ 中,选取使得总残差最小的初值 $\mathbf{X}_{ip}$,对求解 $\mathbf{H}_{ip}$ 进行类似于 Step2 的操作,其中对步长系数 $\Delta X$ 的选取采用随机数的策略,用于防止在迭代过程可能出现的收敛循环现象。

endIf

\begin{table}
\centering
\caption{各计算情况的初值、参数与迭代信息}
\begin{tabular}{c c c c c c c c c c c}
\hline
ID & $n_{H}$ & $Z_{CL}$ & $Z_{CDFS}$ & $Z_{CH}$ & $Z_{TH}$ & $Z_{TL}$ & $T_{4}^{*}$ & $dN$ & $t_{i}$ & $\Delta X$ & $N_{iter}$ & $Pos$ \\
\hline
1 & 0.9 & 0.4 & 0.4 & 0.1 & 0.1 & 0.1 & 1840 & 20 & -0.1 & -2\% & 190 & 190 \\
2 & 0.9 & 0.4 & 0.4 & 0.2 & 0.2 & 0.2 & 1840 & 20 & -0.1 & -2\% & 162 & 162 \\
3 & 0.9 & 0.5 & 0.5 & 0.3 & 0.3 & 0.3 & 1840 & 20 & -0.1 & -2\% & 200 & 200 \\
4 & 0.9 & 0.5 & 0.5 & 0.3 & 0.3 & 0.3 & 1940 & 20 & -0.04 & -2\% & 200 & 185 \\
5 & 0.9 & 0.5 & 0.5 & 0.4 & 0.4 & 0.4 & 1940 & 20 & -0.04 & -2\% & 200 & 176 \\
6 & 0.85 & 0.4 & 0.4 & 0.4 & 0.2 & 0.2 & 1940 & 20 & -0.04 & -2\% & 200 & 160 \\
\hline
\end{tabular}
\end{table}

\textbf{注:} $N_{iter}$ 代表收敛步数,$Pos$ 表示最优解出现的步数。

\begin{table}
\centering
\caption{各计算情况对应的计算结果}
\begin{tabular}{c c c c c c c}
\hline
ID & $n_{H}$ & $Z_{CL}$ & $Z_{CDFS}$ & $Z_{CH}$ & $Z_{TH}$ & $Z_{TL}$ & $T_{4}^{*}$ \\
\hline
1 & 0.8564 & 0.6302 & 0.9500 & 0.5029 & 0.1713 & 0.1295 & 1450.3923 \\
2 & 0.8564 & 0.6304 & 0.9501 & 0.5029 & 0.1713 & 0.1295 & 1450.3846 \\
3 & 0.8564 & 0.6304 & 0.9501 & 0.5029 & 0.1713 & 0.1295 & 1450.4121 \\
4 & 0.8565 & 0.6296 & 0.9486 & 0.5023 & 0.1716 & 0.1297 & 1452.3206 \\
5 & 0.8569 & 0.6466 & 0.9479 & 0.5016 & 0.1734 & 0.1301 & 1456.8070 \\
6 & 0.8564 & 0.6236 & 0.9494 & 0.5017 & 0.1716 & 0.1298 & 1459.2790 \\
\hline
\end{tabular}
\end{table}

\begin{table}
\centering
\caption{各计算情况对应的各方程的归一化残差}
\begin{tabular}{c c c c c c c c}
\hline
ID & 方程1 & 方程2 & 方程3 & 方程4 & 方程5 & 方程6 & 方程7 \\
\hline
1 & 2.42E-06 & 9.82E-06 & 8.11E-06 & 6.27E-06 & 2.42E-05 & 4.23E-05 & -7.64E-05 \\
2 & 1.46E-06 & -4.72E-07 & 5.50E-07 & 5.60E-07 & 2.19E-06 & 2.01E-05 & -8.62E-05 \\
3 & -7.68E-06 & -7.64E-06 & 8.20E-06 & 1.30E-05 & 2.60E-05 & 7.41E-06 & -4.09E-05 \\
4 & -8.83E-04 & -6.08E-04 & 5.81E-04 & 8.80E-04 & 4.49E-04 & 8.48E-04 & -3.39E-03 \\
5 & -3.88E-03 & -3.48E-03 & 1.51E-03 & 3.52E-04 & -9.25E-04 & -3.30E-03 & -4.16E-03 \\
6 & -7.56E-04 & -2.47E-03 & 2.81E-03 & 2.90E-03 & 5.07E-03 & 8.62E-04 & 1.90E-03 \\
\hline
\end{tabular}
\end{table}

\begin{figure}[h]
    \centering
    \begin{subfigure}[t]{0.45\textwidth}
        \centering
        \includegraphics[width=\textwidth]{image_a.png}
        \caption{}
        \label{fig:a}
    \end{subfigure}
    \hfill
    \begin{subfigure}[t]{0.45\textwidth}
        \centering
        \includegraphics[width=\textwidth]{image_b.png}
        \caption{}
        \label{fig:b}
    \end{subfigure}
    
    \begin{subfigure}[t]{0.45\textwidth}
        \centering
        \includegraphics[width=\textwidth]{image_c.png}
        \caption{}
        \label{fig:c}
    \end{subfigure}
    \hfill
    \begin{subfigure}[t]{0.45\textwidth}
        \centering
        \includegraphics[width=\textwidth]{image_d.png}
        \caption{}
        \label{fig:d}
    \end{subfigure}
    
    \begin{subfigure}[t]{0.45\textwidth}
        \centering
        \includegraphics[width=\textwidth]{image_e.png}
        \caption{}
        \label{fig:e}
    \end{subfigure}
    \hfill
    \begin{subfigure}[t]{0.45\textwidth}
        \centering
        \includegraphics[width=\textwidth]{image_f.png}
        \caption{}
        \label{fig:f}
    \end{subfigure}
    \caption{各计算情况迭代过程中的总残差变化}
    \label{fig:10}
\end{figure}

考察 2:采用初始值为表 12 中第 1 组与第 6 组数据,其得到的总残差随迭代步的变化如图 11 所示。

\begin{figure}[h]
    \centering
    \includegraphics[width=\textwidth]{image1.png}
    \caption{总残差随迭代步的变化}
    \label{fig:11}
\end{figure}

考察 3:采用初始值为表 12 中第 1 组数据,设置 dN=20,算法步骤 Step3-2 中的步长系数 \(\Delta X\) 取为定值-2\%时,其得到的总残差随迭代步的变化如图 12 所示。从图中可以清楚的观察到收敛循环现象,这是因为最优的可选初值出现在 46 步,导致过后每个 20 个迭代步所求得的初始 \(\mathbf{H}_{ip}\) 都一致,且在 20 个迭代步内没出现优于第 46 步值的时候,会出现此现象。因此,随机的步长系数策略有利于消除这一现象。

\begin{figure}[h]
    \centering
    \includegraphics[width=\textwidth]{image2.png}
    \caption{总残差随迭代步的变化}
    \label{fig:12}
\end{figure}

\subsection{三种求解算法的比较}

本章用 3 种非线性方程求解算法对所提出的模型进行了求解,现总结如下:

\begin{enumerate}
\item 遗传算法。该算法虽然性能稳定,具有全局搜索能力,但其在计算过程中获得的最优解总残差相对后 2 种方法来说较大,为了获得更优的解需要更多的迭代次数,但该算法的优势在于其不具有初值敏感性,相对比较稳定。

\item 信赖域算法。该算法在每次迭代中给出一个信赖域,在这个邻域内求解子问题得到试探步长,接着用某一评价函数来决定是否接受该试探步以及确定下一次迭代的信赖域。在该方法求解使用中发现其对初值较敏感,需要在选择相对合适的初值时,才能得到最优解。

\item 对于本文提出改进的拟牛顿优选初值方法。在迭代过程中,可以选取计算结果中出现的最优值来作为下一个迭代过程的初始向量来进行迭代。为了防止在迭代过程中可能出现的迭代循环现象,在算法中加入了随机的步长系数来消除这一现象。该方法较前两种算法在初值敏感性,迭代收敛性能上都有了很好的提高。文中给的算例结果也很好的说明了所提出方法的有效性与稳定性。
\end{enumerate}

\section{问题 3 模型的建立与求解}

\subsection{问题 3 重述}

设在发动机飞行高度 $H=11km$,飞行马赫数 $Ma=1.5$ 的超音速巡航点,发动机采用单涵道模式,将选择活门面积设置为 0,风扇导叶角度、高压压气机导叶角度、高压涡轮导叶角度均设置为 $0^\circ$,后混合器面积设置为 2.8518e+004。请问发动机 CDFS 导叶角度、低压涡轮导叶角度和喷管喉道面积 3 个量为多少时,发动机的性能最优?

\subsection{问题分析}

涡喷发动机的超音速性能好,能够提供足够大的单位推力,并且马赫数超过 1.8 以后涡扇发动机的低油耗不再具有优势;亚声速飞行时,涡扇发动机具有良好的经济性,低马赫数飞行时耗油率显著小于涡喷发动机,但单位推力较小。因此,对于持续的超声速巡航飞行任务,适宜选择高单位推力的涡喷循环,反之,对于强调低马赫和长航程的任务,这时就要选择低油耗的涡扇循环;而双涵道变循环发动机可以同时具备高速时的大推力与低速时的低油耗。

而问题中考查飞行马赫数 $Ma=1.5$ 时的超音速巡航点,发动机采用单涵道模式(即选择活门面积设置为 0),意味着此时发动机性能更体现在单位推力上,也就是要保持高单位推力状态作为优化目标。据此我们可以提出优化模型:

\begin{equation}
\left\{
\begin{aligned}
\max \quad & Fs(\alpha_{CDFS}, \alpha_{CH}, A_8) \\
s.t. \quad & N_{CL} - N_{TL} \eta_{mL} = 0 \\
& N_{CH} + N_{CDFS} - N_{TH} \eta_{mH} = 0 \\
& W_{g41} - W'_{g41} = 0 \\
& W_{g45} - W'_{g45} = 0 \\
& p_{61} - p_{62} = 0 \\
& A_8 - A'_8 = 0 \\
& W_{a2} - W_{a21} - W_{a13} = 0 \\
& -5 \leq \alpha_{CDFS} \leq 35 \\
& -5 \leq \alpha_{CH} \leq 35 \\
& 0 < A_8 < 28518
\end{aligned}
\right.
\tag{33}
\end{equation}

其中,最大单位推力 $Fs$ 作为优化目标,七个平衡方程为约束条件;设计变量为三个,分别为:CDFS 导叶角 $\alpha_{CDFS}$,低压涡轮导叶角 $\alpha_{CH}$,喷管喉道面积 $A_{8}$;$\alpha_{CDFS}$ 和 $\alpha_{CH}$ 的取值范围由赛题附录 1 中给出,$A_{8}$ 的取值范围不能大于后混合器的面积;而低压转速 $n_{L}$,高压转速 $n_{H}$,风扇压比函数值 $Z_{CL}$,CDFS 压比函数值 $Z_{CDFS}$,高压压气机压比函数值 $Z_{CH}$,高压涡轮压比函数值 $Z_{TH}$,低压涡轮压比函数值 $Z_{TL}$,主燃烧室出口温度 $T_{4}^{*}$ 八个变量为中间过程变量。

风扇导叶角度 $\alpha_{L}$,高压压气机导叶角度 $\alpha_{H}$,高压涡轮导叶角度 $\alpha_{CL}$ 为给定值,均设为 0。

此优化问题的求解可以分为两部分进行处理,内层为非线性方程组的求解,外层为优化算法迭代。

\subsubsection{内层的非线性方程组求解}

此非线性方程组是由八个中间过程变量作为未知量,七个平衡方程构成。通过上节讨论的非线性方程组求解算法,针对此问题未知量多于方程个数,我们采用速度较慢,但精度高、算法稳定的遗传算法进行求解。将非线性方程组的求解转化为优化问题,优化模型为:

[MATHENV:33]

其中,$f_{i}$ 为各个方程的残差,为了消除各个方程量级的差别,对残差进行归一化处理;优化目标为各个方程残差的二范数最小;八个未知量为设计变量,取值范围由赛题附录 1 中给出。

由于此优化问题的设计变量灵敏度很难给出,所以采用零阶优化算法(遗传算法)来求解,在求解过程中,内部一些一维非线性方程的求解(例如已知气动函数求解速度参数)可能会遇到在实数范围内无解,这样的种群要进行重新生成。

\subsubsection{外层的优化算法迭代}

由于内层求解非线性方程组采用的是遗传算法,速度上是较慢的,如果外层迭代继续采用遗传算法,计算时间是不可接受的,所以我们采用代理模型来近似代替内层求解。

在进行大型复杂系统与多学科优化时,常常会因迭代次数过多、计算量太大、试验次数多导致优化周期相对漫长而无法实施,浪费计算与试验成本。代理模型就是为了处理这种问题,所谓代理模型,是指在不降低精度的情况下构造的一个计算量小、计算周期短但计算结果与数值分析或物理试验结果相近的数学模型来“代理”原来比较复杂的问题。

代理模型包含试验设计与近似方法两部分内容,构造代理模型的三个步骤为:

(1)通过试验设计方法,在设计空间中确定构造模型所用样本点。撒点的好坏决定了所求解的模型是否更接近实际。

(2)利用分析软件或试验方法确定各样本点处的响应值。

(3)以一部分样本对为基础,寻找一个合适的数学近似模型,使拟合函数值与样本测试值符合的最好,并利用剩余的样本对对模型进行检验。如果模型拟合与预测精度都满足要求,就结束;否则构造新的数学模型,直到其拟合与预测精度满足要求为止。

通常的取样方法有:全析因试验设计,部分析因试验设计,拉丁超立方取样,正交试验设计以及均匀试验设计。考虑到样本点不能太多(效率问题),并且要分布的合理(精度问题),本文选择拉丁超立方取样作为试验设计方法,拉丁法在估计响应函数值的平均值、方差和分布函数等方面精度比较高,能处理设计变量很多的问题,计算量不大,操作简单,样本数量可控,并且每个设计变量在其变化范围内均匀分层取样,设计点是这些样本的随机组合。

本文在三维空间内(三个设计变量)选取 100 个设计点,分布图如图 13 所示,然后通过遗传算法求解非线性方程组获取各样本点处的响应值(单位推力),从而构成样本对。其中一个设计点的收敛曲线如图 14 所示,计算结果和响应值如表 15 所示。

\begin{figure}[h]
\centering
\includegraphics[width=\textwidth]{image.png}
\caption{拉丁法取样点在三维空间中的分布}
\end{figure}

\begin{table}
\centering
\caption{前十个数据点的计算结果以及响应值}
\begin{tabular}{c c c c c c}
\hline
ID & CDFS导叶角度 & 低压涡轮导叶角度 & 喷管喉道面积 & 收敛时总残差 & 收敛时单位推力 & 收敛时耗油率 \\
\hline
1 & 23.61 & 6.27 & 8574.07 & 0.09 & 237.74 & 0.12 \\
2 & 30.12 & 13.80 & 25497.35 & 0.56 & 638.73 & 0.09 \\
3 & -0.13 & 12.51 & 7552.71 & 0.06 & 320.48 & 0.12 \\
4 & -4.19 & 4.77 & 7281.52 & 0.10 & 328.63 & 0.12 \\
5 & 33.77 & 8.82 & 12399.83 & 0.28 & 400.46 & 0.10 \\
6 & 17.85 & 9.50 & 18429.20 & 0.47 & 522.38 & 0.09 \\
7 & 8.19 & 12.06 & 16276.11 & 0.40 & 795.42 & 0.10 \\
8 & 29.60 & -2.95 & 10232.11 & 0.18 & 402.06 & 0.09 \\
9 & 1.84 & 9.20 & 1965.78 & 0.63 & 408.84 & 0.13 \\
10 & 30.86 & 12.33 & 4785.63 & 0.26 & 337.58 & 0.11 \\
\hline
\end{tabular}
\end{table}

从表中数据可以看出,大部分设计点收敛后的总残差都维持在一个较低的水平,而不同设计点收敛时的单位推力变化较大,但耗油率相对来说变化不是很大,也验证了外层优化问题所提目标的合理性,即高马赫数飞行时,主要考查高单位推力。

代理模型的近似函数构造方法有:响应面方法(RSM),Kriging方法,径向基函数方法(RBF)以及神经网络模型。考虑本文问题非线性较强,问题较复杂,所以采用适应性良好、能更好模型非线性问题的神经网络模型来近似。

\subsubsection{BP 神经网络}

BP 神经网络是一种按误差逆传播算法训练的多层前馈网络,是目前应用最广泛的神经网络模型之一。其在模拟的训练规则有许多种[6-7],如梯度下降法、动量法、学习率自适应调整法、Quasi-Newton 优化方法及 Levenberg-Marquardt 优化方法等。BP 神经网络在设置完相应的网络结构、传输函数、训练方法、权值函数等一系列参数后,就可以按照相应的学习方法去调整权值系数跟阈值,最终得到想应的模型。其主要思想是:通过 BP 神经网络对训练样本进行网络训练,使网络输出达到或接近目标值,从而构造了对于特定任务具有合适的网络结构及网络权重和阈值的 BP 网络;之后,再应用该 BP 网络对相似的输入样本计算其输出值,达到预报目的。因而,BP 网络预报最重要的环节是建立适合特定任务的 BP 网络,也即找出特定的网络权重和阈值。其具体过程见下图。

\begin{figure}[h]
\centering
\includegraphics[width=0.8\textwidth]{neural_network_training_flowchart.png}
\caption{神经网络训练流程图}
\end{figure}

在 BP 模型建立前我们应该注意的 2 个问题:

(1)BP 网络存在局部最小点。BP 算法的实质是把一组样本输入输出问题,转化为一个非线性优化问题。但采用非线性梯度优化算法,容易形成局部极小,会使网络不能完全得到训练,从而影响整体优化。而 BP 网络收敛到局部极小主要是由于初始权重和阈值的设置不合适所引起的。若网络确实出现局部极小时,则只能重新训练,产生合适的随机权重,使 BP 网络最终达到整体优化的目的。

(2)网络隐含层节点数的确定。网络的输入输出节点由所要解决的问题而定。但隐含层节点的选取往往是根据经验。BP 网络隐含层节点数的选择对网络性能的影响很大。

BP 模型的建立需要大量的数据调试与反复试验,才能确定整个模型的参数设置与结构。本文中 BP 神经网络模型的建立是基于 matlab 神经网络仿真工具箱来实现的。

网络采用四层结构,一个输入层,两个隐含层,一个输出层。其中,输入向量为 3 个,隐含层神经元个数确定为 9 个与 6 个,输出目标为 1 个。模型建立采用参数设置:隐含层与输出层的传输函数都采用 Log-Sigmoid, Log-Sigmoid, 训练方法采用 trainlm(Levenberg-Marquardt 方法),权值函数设置为 negdist(归一化列权值初始化函数),学习函数设为 learngdm(附加动量因子的梯度下降权值/阈值学习函数),初始化参数采用 initwb 函数。图 16 给出了 100 对样本数据进行 BP 神经网络训练后的回归曲线图。

\begin{figure}[h]
    \centering
    \includegraphics[width=\textwidth]{image.png}
    \caption{BP 神经网络训练 100 对样本数据的回归曲线图}
    \label{fig:16}
\end{figure}

对神经网络训练得到的近似代理模型进行优化分析,同样采用遗传算法,收敛曲线如图 16 所示。

\begin{figure}[h]
    \centering
    \includegraphics[width=\textwidth]{image1.png}
    \caption{代理模型遗传算法优化收敛曲线}
    \label{fig:17}
\end{figure}

得到收敛解为,CDFS 导叶角度为 29.34,低压涡轮导叶角度为 10.32,尾喷管喉道面积为 19384,得到发动机的最大推力为 953.08。

对这组收敛解代入遗传算法进行验证,收敛曲线如图 \ref{fig:17} 所示,可以得到各组中间过程变量,计算结果如表 16 所示。

\begin{figure}[h]
    \centering
    \includegraphics[width=\textwidth]{image2.png}
    \caption{遗传算法验证的收敛曲线}
    \label{fig:18}
\end{figure}

\begin{table}
\centering
\caption{遗传算法验证得到的中间过程变量结果}
\begin{tabular}{c c c c c c c c}
\hline
变量名 & 低压转速 & 高压转速 & 风扇压比函数值 & CDFS压比函数值 & 高压压气机压比函数值 & 高压涡轮压比函数值 & 低压涡轮压比函数值 & 主燃烧室出口温度 \\
\hline
计算结果 & 0.7485 & 0.8524 & 0.5220 & 0.0073 & 0.3200 & 0.4398 & 0.3784 & 1032.6 \\
\hline
\end{tabular}
\end{table}

遗传算法计算得到的单位推力为 871.49,耗油率为 0.1722,和神经网络代理模型得到的结果是比较符合的。说明了这种优化策略的合理性。

\section{问题 4 模型的建立与求解}

\subsection{问题 4 重述}

试研究发动机飞行高度 $H=11km$,飞行马赫数从 $Ma=1.1$ 变化到 $Ma=1.6$,发动机特性最优时,CDFS 导叶角度、低压涡轮导叶角度,尾喷管喉道面积随飞行马赫数的变化规律。此时发动机采用单涵道模式,将选择活门面积设置为 0,风扇导叶角度、高压压气机导叶角度、高压涡轮导叶角度均设置为 $0^\circ$,后混合器出口总面积设置为 $2.8518e+004$,后混合器内、外涵道面积可调(即不受附录 1 后混合器给定的内、外涵道面积值的约束)。

\subsection{问题分析}

此问题同样考查高飞行马赫数,单涵道模式下的优化问题,需要进一步考查在发动机性能最优时,CDFS 导叶角度、低压涡轮导叶角度、尾喷管喉道面积随飞行马赫数的变化规律。本文针对飞行马赫数离散成数据点,即分别考查马赫数为 1.1,1.2,1.3,1.4,1.5,1.6 时的最优 CDFS 导叶角度、低压涡轮导叶角度、尾喷管喉道面积。每一个优化问题的模型都与公式 (34) 一致,但对于内层非线性方程组的求解模型有一些区别,本问题要求后混合器内、外涵道面积可调(即不受赛题附录 1 后混合器给定的内、外涵道面积值的约束),这意味着非线性方程组的未知数变为九个,同样将这个非线性方程组转化为优化问题,并采用遗传算法进行求解,则优化模型为:

[MATHENV:34]

\begin{align*}
0 < n_{L} & \leq 1 \\
0 < n_{H} & \leq 1 \\
0 < Z_{CL} & \leq 1 \\
0 < Z_{CDFS} & \leq 1 \\
0 < Z_{CH} & \leq 1 \\
0 < Z_{TH} & \leq 1 \\
0 < Z_{TL} & \leq 1 \\
0 < T_{4}^{*} & < 2000 \\
0 \leq A_{62} & \leq 23212
\end{align*}

其中,$A_{62}$ 为后混合器外涵道出口面积,其范围为后混合器关闭和打开到最大的数值。

分别对马赫数为 1.1,1.2,1.3,1.4,1.5,1.6 时,利用同样的优化分析思路,内层通过遗传算法生成样本对构造代理模型,外层采用遗传算法进行优化分析设计。

首先采用拉丁法生成 100 个设计点,对这些设计点进行遗传算法求解非线性方程组,得到响应值(单位推力),从而得到不同马赫数下的样本对,根据这些样本对进行神经网络训练,得到不同马赫数下的神经网络近似模型,针对这些近似模型再采用遗传算法进行优化设计,得到不同马赫数下的最优设计变量(CDFS 导叶角度、低压涡轮导叶角度、尾喷管喉道面积)。收敛曲线如图 18 所示,计算结果如表 17 所示,不同设计变量随马赫数的变化曲线如图 19 所示。

\begin{figure}[h]
    \centering
    \includegraphics[width=\textwidth]{image1.png}
    \caption{不同马赫数的收敛曲线}
    \label{fig:convergence_curves}
\end{figure}

\begin{table}[h]
    \centering
    \caption{不同参量随飞行马赫数的变化对比}
    \label{tab:parameters_vs_mach}
    \begin{tabular}{c c c c c}
        \hline
        飞行马赫数 & CDFS导叶角度 & 低压涡轮导叶角度 & 尾喷管喉道面积 & 最大单位推力 \\
        \hline
        1.1 & -5.00 & 7.45 & 10884.07 & 999.61 \\
        1.2 & -4.84 & 15 & 12409.77 & 995.41 \\
        1.3 & 19.68 & 9.48 & 16904.43 & 958.33 \\
        1.4 & 28.30 & 2.84 & 21465.23 & 954.05 \\
        1.5 & 27.18 & 0.97 & 18328.00 & 952.32 \\
        1.6 & 10.61 & -1.55 & 16186.74 & 883.78 \\
        \hline
    \end{tabular}
\end{table}

\begin{figure}[h]
    \centering
    \begin{subfigure}[t]{0.45\textwidth}
        \includegraphics[width=\textwidth]{image_a.png}
        \caption{}
        \label{fig:a}
    \end{subfigure}
    \hfill
    \begin{subfigure}[t]{0.45\textwidth}
        \includegraphics[width=\textwidth]{image_b.png}
        \caption{}
        \label{fig:b}
    \end{subfigure}
    
    \begin{subfigure}[t]{0.45\textwidth}
        \includegraphics[width=\textwidth]{image_c.png}
        \caption{}
        \label{fig:c}
    \end{subfigure}
    \hfill
    \begin{subfigure}[t]{0.45\textwidth}
        \includegraphics[width=\textwidth]{image_d.png}
        \caption{}
        \label{fig:d}
    \end{subfigure}
    \caption{不同参量(a) CDFS 导叶角 (b) 低压涡轮导叶角 (c) 尾喷管喉道面积 (d) 最大单位推力随飞行马赫数的变化规律曲线}
    \label{fig:20}
\end{figure}

从计算结果可以看出,CDFS 导叶角度、低压涡轮导叶角度以及尾喷管喉道面积都是随着飞行马赫数先增大后减小,而最大单位推力是随飞行马赫数单调下降的,这也和文献[2]中的结果一致,说明了文中算法的有效性。

\section{总结}

\begin{enumerate}
\item 通过发动机各个部件级传递关系进行建模,形成进道口,风扇,CDFS,高压压气机,前混合器,主燃烧室,高压涡轮,低压涡轮,后混合器,尾喷管喉道等传递模型。将中间过程变量转换为七个未知量,根据七个平衡方程建立非线性方程组。

\item 分别采用遗传算法、matlab 自带的可信域算法、以及提出的拟牛顿优选初值方法来求解非线性方程组。通过考查和讨论算法的有效性可知,遗传算法性能稳定,具有全局搜索能力,能保证结果的正确性,特别是处理强非线性,难于得到变量导数等问题具有优势,但其速度较慢。matlab 自带的可信域算法具有初值敏感性,好的初始解会有较好的收敛效果,并且收敛速度很快。而拟牛顿优选初值方法在初值敏感性,迭代收敛性能上都有很好的提高,并且求解速度较快,同时也可以和遗传算法结合使用。三种算法都能求解出本问题,并且都有各自的特点和适用性。

\item 对于几何调节的优化问题可以分为内层非线性方程的求解以及外层的优化迭代,内层采用性能稳定的遗传算法求解非线性方程组。另外提出一种代理模型技术提高求解效率,通过采用拉丁法生成设计点,求解非线性方程组获得样本响应值,然后利用神经网络模型对样本对进行训练,从而生成近似代理模型。外层对代理模型进行优化设计。最后对收敛解进行非代理模型验证,两者结果相差很小,表明这种优化策略的可行性以及正确性。

\item 为了考查飞行马赫数的影响,将其进行离散,同时将后混合器内涵道面积作为中间过程变量,对每个离散点采用相同的优化策略,从而可以获得三个设计变量随飞行马赫数的变化规律。计算结果表明 CDFS 导叶角度、低压涡轮导叶角度以及尾喷管喉道面积都是随着飞行马赫数先增大后减小,而最大单位推力是随飞行马赫数单调下降的。
\end{enumerate}

[REFERENCES:1]