\documentclass{article}
\usepackage{amsmath}
\usepackage{amssymb}

\title{多无人机协同任务规划}
\author{}
\date{}

\begin{document}

\maketitle

\begin{center}
\textbf{“华为杯”第十三届全国研究生数学建模竞赛}
\end{center}

\tableofcontents

\begin{abstract}
为了解决多机型、多任务、多攻击目标的无人机调度规划问题,充分发挥了经典线性规划方法与人工智能算法各自的优势,综合“零整互化”、“目标逆推”及“分步优化”三大思想,建立了无人机系统资源分配、任务分配、航线规划的分析模型。

针对问题 1,构建了基于 2-opt 算法的单目标优化模型,首先化零为整,得出 10 个目标群各自的中心点,遍历完这些中心点之间所有的路线组合后,利用 2-opt 启发式算法筛选出侦察这 10 个中心点的最短路径,以此进一步化整为零,筛选出经过 68 个目标点的侦察路线。分别从 P01、P03 基地调用载有 S-1、S-2 载荷的 FY-1 型无人机各一架次执行侦察任务,共需 4 架次 FY-1 型无人机完成对 68 个目标的侦察。此时,无人机在防御方雷达有效探测范围内的总飞行距离为 $2520.52 \, \text{km}$,滞留时间累计 $12.60 \, \text{h}$,为侦察无人机滞留防御方雷达有效探测范围内时间总和最小的方案。

针对问题 2,基于蒙特卡罗法模拟解决该问题,构建概率优化模型,根据目标建模的思想,判别依概率产生的假设结果与目标要求的差异性,继而寻找可供 FY-2 型通信中继无人机一次性服务多架 FY-1 型无人机的飞行路线。求解可知,为保证空中飞行的侦察无人机随时与 FY-2 型无人机的通信,应至少安排两架次的 FY-2 型通信中继无人机各与每条航线上的侦察无人机通信,方可完成侦察任务。

针对问题 3 和问题 4,根据分步优化的思想,利用几何分析方法,构建了无人机任务分配的线性规划模型,为进攻方火力打击任务规划出了无人机飞行路线、携带炸弹清单、攻击目标清单等具体方案。问题 3 中,首先分派一定数量的 FY-3 型无人机集中火力优先攻击 10 个目标群中的雷达,无人机滞留防御方雷达有效探测范围内的总时间最小为 $4.67h$,然后对剩余目标的攻击需要安排 FY-1 型无人机 8 架次、FY-3 型无人机 56 架次,完成对所有地面目标火力打击任务的总时间为 $1.4605h$。问题 4 中,首先集中火力优先攻击远程雷达,再攻击各目标群的防御雷达,最后对剩余目标进行攻击。利用具有较好全局搜索能力的粒子群算法模型确定了远程搜索雷达的部署位置,解算可知最少需要安排 FY-1 型无人机 8 架次、FY-3 型无人机 59 架次,完成对所有地面目标火力打击任务的最长时间小于 $3.55h$,满足 $7h$ 的上限要求。

对于问题 5,对本文算法复杂度进行分析。在求解无人机最优航线时,采用分层思想,先对 10 个目标群的中心点寻找最佳侦察路线与调度策略,只需比较 $2^9$ 次组合,避免了一次性处理 $2^{67}$ 次组合,增强了任务规划的时效性。粒子群算法的时间复杂度与迭代粒子数、迭代次数、迭代时间有关,在算法执行的过程中动态改变迭代粒子数可有效提高任务规划的效率。从本文构建的数学模型和对模型解算的结果可以看出,影响无人机作战能力的重要因素有:最大巡航速度、巡航高度、最长巡航时间、机动性能、机载设备(机载传感器的精度、可挂载的武器数量等)、隐身能力(反雷达侦察能力的大小)以及与基地的通信能力等。
关键词:无人机,协同作战,任务规划,2-opt 算法,粒子群算法
\end{abstract}




\section{一、问题重述}

无人机(Unmanned Aerial Vehicle, UAV)是一种具备自主飞行和独立执行任务能力的新型作战平台,不仅能够执行军事侦察、监视、搜索、目标指向等非攻击性任务,而且还能够执行对地攻击和目标轰炸等作战任务。随着无人机技术的快速发展,越来越多的无人机将应用在未来战场。

某无人机作战部队现配属有 P01~P07 等 7 个无人机基地,各基地均配备一定数量的 FY 系列无人机。其中 FY-1 型无人机主要担任目标侦察和目标指示,FY-2 型无人机主要担任通信中继,FY-3 型无人机用于对地攻击。FY-1 型无人机的巡航飞行速度为 $200 \mathrm{~km} / \mathrm{h}$,最长巡航时间为 $10 \mathrm{~h}$,巡航飞行高度为 $1500 \mathrm{~m}$;FY-2 型、FY-3 型无人机的巡航飞行速度为 $300 \mathrm{~km} / \mathrm{h}$,最长巡航时间为 $8 \mathrm{~h}$,巡航飞行高度为 $5000 \mathrm{~m}$。受燃料限制,无人机在飞行过程中尽可能减少转弯、爬升、俯冲等机动动作,一般来说,机动时消耗的燃料是巡航的 $2 \sim 4$ 倍。最小转弯半径 $70 \mathrm{~m}$。

FY-1 型无人机可加载 S-1、S-2、S-3 三种载荷。其中载荷 S-1 系成像传感器,采用广域搜索模式对目标进行成像,传感器的成像带宽为 $2 \mathrm{~km}$;载荷 S-2 系光学传感器,为达到一定的目标识别精度,对地面目标拍照时要求距目标的距离不超过 $7.5 \mathrm{~km}$,可瞬时完成拍照任务;载荷 S-3 系目标指示器,为制导炸弹提供目标指示时要求距被攻击目标的距离不超过 $15 \mathrm{~km}$。由于各种技术条件的限制,该系列无人机每次只能加载 S-1、S-2、S-3 三种载荷中的一种。为保证侦察效果,对每一个目标需安排 S-1、S-2 两种不同载荷各自至少侦察一次,两种不同载荷对同一目标的侦察间隔时间不超过 4 小时。

为保证执行侦察任务的无人机与地面控制中心的联系,需安排专门的 FY-2 型无人机担任通信中继任务,通信中继无人机与执行侦察任务的无人机的通信距离限定在 $50 \mathrm{~km}$ 范围内。通信中继无人机正常工作状态下可随时保持与地面控制中心的通信。

FY-3 型无人机可携带 6 枚 D-1 或 D-2 两种型号的炸弹。其中 D-1 炸弹系某种类型的“灵巧”炸弹,采用抛投方式对地攻击,即投放后炸弹以飞机投弹时的速度作抛物运动,当炸弹接近目标后,可主动寻的攻击待打击的目标,因此炸弹落点位于目标中心 $100 \mathrm{~m}$ 范围内可视为有效击中目标。D-2 型炸弹在激光制导模式下对地面目标进行攻击,其飞行速度为 $200 \mathrm{~m} / \mathrm{s}$,飞行方向总是指向目标。攻击同一目标的 D-2 型炸弹在整个飞行过程中需一架 FY-1 型无人机加载载荷 S-3 进行全程引导,直到命中目标。由于某些技术上的限制,携带 D-2 型炸弹的无人机在投掷炸弹时要求距目标 $10 \mathrm{~km} \sim 30 \mathrm{~km}$,并且要求各制导炸弹的发射点到目标点连线的大地投影不交叉(以保证弹道不交叉)。为达到一定的毁伤效果,对每个目标(包括雷达站和远程搜索雷达)需成功投掷 10 枚 D-1 型炸弹,而对同一目标投掷 2 枚 D-2 型炸弹即可达到相同的毁伤效果。

多架该型无人机在同时执行任务时可按照一定的编队飞行,但空中飞行时两机相距要求 $200 \mathrm{~m}$ 以上。由于基地后勤技术保障的限制,同一基地的两架无人机起飞时间间隔和降落回收的时间间隔要求在 3 分钟以上。无人机执行完任务后需返回原基地。

根据任务要求,需完成侦察和打击的目标有 A01~A10 等 10 个目标群,每个目标群包含数量不等的地面目标,每个目标群均配属有雷达站,各目标群配属雷达对 FY 型无人机的有效探测距离为 $70 \mathrm{~km}$。

需研究下列问题:

(1) 一旦有侦察无人机进入防御方某一目标群配属雷达探测范围,防御方 10 个目标群的配属雷达均开机对空警戒和搜索目标,并会采取相应对策,包括发射导弹对无人机进行摧毁等,因此侦察无人机滞留防御方雷达探测范围内时间越长,被其摧毁的可能性就越大。现需为 FY-1 型无人机完成 10 个目标群(共 68 个目标)的侦察任务拟制最佳的路线和无人机调度策略(包括每架无人机起飞基地、加载的载荷、起飞时间、航迹和侦察的目标),以保证侦察无人机滞留防御方雷达有效探测范围内的时间总和最小。

(2) FY-1 型无人机对目标进行侦察时,须将侦察信息实时通过 FY-2 型无人机传回地面控制中心。鉴于 50km 通信距离的限制,需安排多架 FY-2 型无人机升空,以保证空中飞行的侦察无人机随时与 FY-2 型无人机的通信。FY-2 型无人机可同时与多架在其有效通信范围的侦察无人机通信并转发信息。为完成问题 (1) 的侦察任务,至少安排多少架次的 FY-2 型通信中继无人机。

(3) 所有 FY-1 型无人机现已完成侦察任务并返回基地,均可加载载荷 S-3 用于为制导炸弹提供目标指示。现要求在 7 个小时内(从第一架攻击无人机进入防御方雷达探测范围内起,到轰炸完最后一个目标止)完成对 10 个目标群所有 68 个地面目标的火力打击任务,如何进行任务规划以保证攻击方的无人机滞留防御方雷达有效探测范围内的时间总和最小?请给出具体的无人机任务规划结果(包括每架无人机飞行路线、FY-3 型无人机携带炸弹的具体清单和攻击的目标清单)。

(4) 由相关信息渠道获知在 A02、A05、A09 周边可能还配置有三部远程搜索雷达,该雷达对 FY 型无人机的有效作用距离是 200km。这三部雷达的工作模式是相继开机工作,即只有首先开机的雷达遭到攻击后才开启第二部雷达,同样只有第二部雷达被攻击后才开启第三部雷达。远程搜索雷达一旦开机工作,攻击方无人机群即可获知信号并锁定目标,而后安排距其最近的无人机对其摧毁。请基于防御方部署远程搜索雷达的情形重新考虑问题 (3)。

(5) 请对求解模型的算法的复杂度进行分析;并讨论如何有效地提高算法的效率,以增强任务规划的时效性。基于你们小组构建的数学模型和对模型解算的结果,讨论哪些技术参数的提高将显著提升无人机的作战能力?

\section{二、问题分析}

对于问题 1,为 FY-1 型无人机完成 10 个目标群(共 68 个目标)的侦察任务拟制最佳的路线和无人机调度策略(包括每架无人机起飞基地、加载的载荷、起飞时间、航迹和侦察的目标),要保证侦察无人机滞留防御方雷达有效探测范围内的总时间最小,即要保证无人机进入雷达有效探测范围并侦察到第一个目标、侦察完第一个目标并按任务安排侦察完所有目标、侦察完最后一个目标并离开雷达有效探测范围这三个阶段用时的总和最小。根据“零整互化”的思想,首先化零为整,找到经过 10 个目标群中心点的最短路线,再化整为零,筛选出经过 68 个目标点的侦察路线。

对于问题 2,由于 FY-1 型无人机对目标进行侦察时须将侦察信息实时通过 FY-2 型无人机传回地面控制中心,且出于 50km 通信距离的限制,为了尽可能地安排较少架次的 FY-2 型通信中继无人机配合 FY-1 型无人机执行任务,只要求 FY-2 型通信中继无人机的飞行路线尽可能地经过多个 FY-1 型无人机周围 50km 的区域之中。问题即转化为寻找 FY-1 型无人机各条飞行路线周围 50km 的重叠区域,然后再合理地分配 FY-2 型通信中继无人机的架次。利用概率优化的方法取出符合约束条件的若干点,即可得出 FY-2 型通信中继无人机在公共区域内的飞行路线。

对于问题 3,要求在 7 个小时内(从第一架攻击无人机进入防御方雷达探测范围内起,到轰炸完最后一个目标止)完成对 10 个目标群所有 68 个地面目标的火力打击任务,并要保证攻击方的无人机滞留防御方雷达有效探测范围内的总时间最小。要保证无人机暴露在敌方雷达探测范围内的总时间最小,根据分步优化的思想,将这一问题的解决分为两个步骤,首先在已知敌方雷达坐标的前提下,进攻方应优先对敌方雷达设备重点进行破坏,而后再攻击其余非雷达设备。

对于问题 4,同样应分步来解决。由于在 A02、A05、A09 周边可能还配置有三部远程搜索雷达,该雷达对 FY 型无人机的有效作用距离是 200km,工作模式是相继开机工作,首先要站在防御方的角度部署远程搜索雷达,每个远程搜索雷达部署的原则是,尽可能多地将己方目标覆盖在 200km 的范围之内,且距离最近的己方目标须分别隶属于 A02、A05、A09 目标群。根据该原则可规划出这三个目标群周边远程搜索雷达的部署位置,且远程雷达的开启顺序应根据覆盖范围内目标个数的多少决定。远程雷达分布及开启情况确定之后,再进行无人机的任务规划。无人机首先的轰炸目标应为部署在 A09 周边的远程雷达,而后再攻击各目标群的防御雷达,最后对剩余目标进行攻击。可在问题 3 求解的基础之上进行修正与叠加。

对于问题 5,主要需解决两个问题:一是对本文求解算法的复杂度进行分析,并且分析得出能有效提高算法效率以增强任务规划时效性的方法。二是基于构建的数学模型和对模型解算的结果,分析不同技术参数的提高对无人机作战能力提升的影响。首先就算法的复杂度进行分析,分别对文中使用的 2-opt 算法和粒子群算法的复杂度进行计算。同时针对粒子群算法,可通过在不同迭代过程中逐代减少粒子数的方法提高算法的时效性。然后根据本文建模时考虑的因素以及本文模型求解出的结果,可从无人机的最大巡航速度、巡航高度、最长巡航时间、机动性能、机载设备(机载传感器的精度、可挂载的武器数量等)、隐身能力(反雷达侦察能力的大小)以及与基地的通信能力这几个方面入手进行分析。

\section{三、基本假设}

根据题意,进行如下假设:

1. 在雷达有效探测距离之内,无人机的飞行处于固定的巡航速度以及巡航高度下。

2. 不考虑无人机在飞行过程中爬升、俯冲机动动作的影响,且由于最小转弯半径相对于其它距离量度来说较小,忽略转弯过程的影响。

3. 忽略成像传感器和光学传感器收集信息时间的影响,认为其可瞬时完成任务,且假设本题中成像传感器统一采用广域搜索模式对目标进行成像,即目标落入传感器成像带宽内即可。

4. 假设通信中继无人机正常工作状态下可随时保持与地面控制中心的通信。

\section{四、符号说明与名词解释}

\subsection{4.1 符号说明}
\begin{tabular}{ll}
$A_{i}$ & 目标群 \\
$P_{i}$ & 无人机基地 \\
$\text { num }_{i}$ & 无人机途径目标点数量 \\
$B_{h}$ & 飞行路径起点、终点 \\
$S_{n}^{1}$ & 第 $n$ 个随机点到第一条飞行路线上某点的距离 \\
$S_{n}^{2}$ & 第 $n$ 个随机点到第二条飞行路线上某点的距离 \\
$d_{i}$ & 所有基地到第 $i$ 个目标群中心点的距离之和 \\
$p_{i}$ & 在第 $i$ 个目标群轰炸的目标个数 \\
$x(1) \sim x(10)$ & 基地向各目标群派遣的携带 D-1 型炸弹的 FY-3 型无人机架次 \\
$x(11) \sim x(20)$ & 基地向各目标群派遣的携带 D-2 型炸弹的 FY-3 型无人机架次 \\
$U_{q}$ & 远程搜索雷达 \\
$S_{U_{q}, T_{j}}$ & 搜索雷达和攻击目标之间的距离 \\
$\omega$ & 惯性权重 \\
$c_{i}$ & 学习因子 \\
$r_{i}$ & 伪随机数 \\
$T$ & 无人机滞留时间 \\
\end{tabular}

\subsection{4.2 名词解释}
\begin{enumerate}
    \item 无人机:(Unmanned Aerial Vehicle, UAV)是一种具备自主飞行和独立执行任务能力的新型作战平台,不仅能够执行军事侦察、监视、搜索、目标指向等非攻击性任务,而且还能够执行对地攻击和目标轰炸等作战任务。
    \item 巡航速度:飞机所装发动机每 100 公里消耗燃油最小情况下的飞行速度称为巡航速度。
    \item 巡航高度:飞机完成起飞阶段进入预定航线后的飞行状态下处在的高度。
    \item 巡航时间:飞机在不进行空中加油的情况下,耗尽其本身携带的可用燃料时,所能持续飞行的时间。
    \item 机动动作:飞行状态(速度、高度和飞行方向)随时间变化的飞行动作,又称机动。
    \item 抛物运动:物体以一定的初速度水平方向抛出,如果物体仅受重力作用,这样的运动叫做平抛运动。平抛运动可看作水平方向的匀速直线运动以及竖直方向的自由落体运动的合运动。
    \item 激光制导:利用激光获得制导信息或传输制导指令使导弹按一定导引规律飞向目标的制导方法。
    \item 广域搜索:采用广域搜索模式对目标进行成像,即目标落入传感器成像带宽内即可
\end{enumerate}

9、无人机任务规划:多无人机协同作战中的任务规划从功能上可大致划分为系统资源分配、任务分配、航线规划、轨迹优化、武器投放规划等。

10、蒙特卡罗方法:也称统计模拟方法,是一种以概率统计理论为指导的数值计算方法,主要使用随机数(或更常见的伪随机数)来解决计算问题。

11、三样条插值:是通过求解三弯矩方程组得出一系列形值点光滑曲线的方法。

12、粒子群算法:也称粒子群优化算法,属于进化算法的一种,从随机解出发,通过追随当前搜索到的最优值来寻找全局最优,通过适应度来评价解的品质,具有实现容易、精度高、收敛快等优点。

\section{五、模型的建立与求解}

\subsection{5.1 问题 1 的求解}

\subsubsection{5.1.1 问题 1 的分析}

一旦有侦察无人机进入防御方某一目标群配属雷达探测范围,防御方 10 个目标群的配属雷达均开机对空警戒和搜索目标,并会采取相应对策,包括发射导弹对无人机进行摧毁等,因此侦察无人机滞留防御方雷达探测范围内时间越长,被其摧毁的可能性就越大。攻击目标群、防御雷达覆盖范围、无人机基地位置示意如图 1 所示。

\begin{figure}[h]
\centering
\includegraphics[width=0.8\textwidth]{image.png}
\caption{攻击目标群、防御雷达覆盖范围、无人机基地位置示意图}
\end{figure}

上图中,红色点表示攻击目标群,绿色圆表示目标群配属雷达最大有效探测范围,蓝色方块表示无人机基地。

为 FY-1 型无人机完成 10 个目标群(共 68 个目标)的侦察任务拟制最佳的路线和无人机调度策略(包括每架无人机起飞基地、加载的载荷、起飞时间、航迹和侦察的目标),要保证侦察无人机滞留防御方雷达有效探测范围内的总时间最小,即要保证无人机进入雷达有效探测范围并侦察到第一个目标、侦察完第一个目标并按任务安排侦察完所有目标、侦察完最后一个目标并离开雷达有效探测范围这三个阶段用时的总和最小。

下面首先验证忽略转弯过程影响的合理性。

\begin{figure}[h]
    \centering
    \includegraphics[width=\textwidth]{image.png}
    \caption{考虑/不考虑转弯过程影响的侦察路径示意图}
    \label{fig:2}
\end{figure}

以目标群 A01 为例,雷达站 A0101 坐标为 $(264\,\text{km}, 715\,\text{km})$,目标 A0103 的坐标为 $(274\,\text{km}, 728\,\text{km})$。若考虑无人机的转弯过程,如图 \ref{fig:2}(a) 所示,通过相关计算可得转弯圆弧所对应的圆心角为 $179.66^\circ$,可得出 BC 段圆弧的长度为 $219.5\,\text{m}$,那么无人机在雷达有效探测范围内的飞行距离 $S_{\text{转}}=92.42\,\text{km}$。若不考虑无人机的转弯过程,如图 \ref{fig:2}(b) 所示,侦察机朝雷达有效探测范围半径方向进入并到达目标 A0103 识别范围后离开,那么无人机在该范围内飞行的距离 $S_{\text{不转}}=92.2\,\text{km}$。

通过上述计算可知,$\frac{S_{\text{转}}-S_{\text{不转}}}{S_{\text{转}}}=\frac{92.42-92.2}{92.42}\approx 0.24\%$,说明转弯过程对无人机飞行距离的影响非常细微,在拟制无人机最佳飞行路线和调度策略时可忽略该过程的影响。

根据“零整互化”的思想,首先化零为整,找到经过 10 个目标群中心点的最短路线,并制定对应的最佳侦察路线与调度策略。再化整为零,筛选出经过 68 个目标点的侦察路线。基于这一思想,拟采用 2-opt 算法进行实现。

\subsubsection{5.1.2 基于 2-opt 算法的模型求解}

2-opt 算法(2-optimization algorithm)是一种解决组合优化问题的局部搜索算法,它的基本思想是:假设 $(i, i+1)$ 和 $(j, j+1)$ 是当前可行解的两条边,则 2-opt 运算后得到两条新边 $(i, j+1)$ 和 $(j, i+1)$,并对原来 $i+1$ 和 $j$ 之间的路径进行翻转,就得到了新的解 \cite{ref1}。如果在遍历过程中找到了更好的可行解,则把该可行解置为当前解,重新进行遍历,直到遍历过程结束。

在利用 2-opt 算法求解无人机侦察路线和调度问题前,首先对每个目标群里面所有目标点的坐标进行均值处理,找到各目标群的中心点,如图 \ref{fig:centers} 所示。

\begin{figure}[h]
    \centering
    \includegraphics[width=\textwidth]{centers.png}
    \caption{各攻击目标群中心点示意图}
    \label{fig:centers}
\end{figure}

目标函数为:
\begin{equation}
\min \sum_{i=1}^{m} d_{i}
\tag{1}
\end{equation}

其中,$m$ 表示加载同一载荷的无人机数量,$d_{i}$ 表示第 $i$ 架无人机侦察完所有指定攻击目标群在防御方雷达有效探测范围内的总飞行距离。

以加载 S-2 型载荷为例,约束条件为:
\begin{equation}
\begin{cases}
\forall Y_{j}, \exists S_{P_{l},Y_{j}} \leq 7.5 \quad (j,l=1,2,\cdots,68) \\
S_{B_{h},A_{k}} \geq 70 \quad (h=1,2,\cdots,2m \quad k=1,2,\cdots,10) \\
\sum_{i=1}^{m} num_{i} = 68
\end{cases}
\tag{2}
\end{equation}

其中,$P_{l}$ 表示路径点,$Y_{j}$ 表示攻击目标点,$B_{h}$ 表示无人机侦察路径上的起始点与终了点,$A_{k}$ 表示防御方雷达所在点,$num_{i}$ 表示第 $i$ 架无人机侦察路线经过的目标点数量。

首先利用最近邻算法在这 10 个目标群中心点之间随机生成一条初始路径,再利用 2-opt 算法对不同架次无人机的路线情况进行规划,从中筛选出最优分配方案,不同架次下走遍 10 个目标中心点无人机最优路径距离如表 \ref{tab:optimal_paths} 所示。

\begin{table}[h]
    \centering
    \caption{不同架次下无人机最优路径距离}
    \label{tab:optimal_paths}
    \begin{tabular}{c c}
        \hline
        架次 & 最优路径距离 \\
        \hline
        1 &  \\
        2 &  \\
        3 &  \\
        4 &  \\
        5 &  \\
        6 &  \\
        7 &  \\
        8 &  \\
        9 &  \\
        10 &  \\
        \hline
    \end{tabular}
\end{table}

\begin{table}
\centering
\caption{无人机不同架次安排下最优路径距离表}
\begin{tabular}{|c|c|c|}
\hline
FY1-S1型无人机架次 & FY1-S1/S3型无人机总最优路径距离(km) & FY1-S1/S3型无人机总最优滞留时间(h) \\
\hline
1架 & 2642.60 & 13.21 \\
\hline
2架 & 1997.09 & 9.99 \\
\hline
3架 & 2053.54 & 10.27 \\
\hline
4架 & 2056.59 & 10.28 \\
\hline
5架 & 2200.86 & 11.00 \\
\hline
6架 & 2287.02 & 11.44 \\
\hline
7架 & 2027.00 & 10.14 \\
\hline
8架 & 2240.00 & 11.2 \\
\hline
\end{tabular}
\end{table}

易知,2架次FY1-S1型无人机下,FY1-S1/S3型无人机总最优路径距离最短、总最优滞留时间最小,至少提高10.7\%的安全性。

10个目标群中心点之间的最优规划路线如图4所示。

\begin{figure}[h]
\centering
\includegraphics[width=0.8\textwidth]{image.png}
\caption{10个目标群中心点之间的最优规划路线示意图}
\end{figure}

从图中可知,如果两条线路的侦察任务同时进行,则最优情况下应一次性调用2架无人机执行侦查任务,共需4架次FY-1型无人机完成对68个目标的侦察。

再利用同样的方法对每个目标群内部所有目标点的飞行路线进行二次最优规划,得出最终的路线规划结果如表2和图5所示,其中图5(a)为整体最优侦察路线示意,图5(b)为考虑FY-1型无人机加载成像传感器时的路线修正,即相对于原路线单侧偏移2km。

\begin{table}
\centering
\caption{无人机最优侦察路线点}
\begin{tabular}{|c|l|}
\hline
机型/载荷 & 侦察路线坐标点 \\
\hline
\multirow{12}{*}{无人机1、2 FY-1 S-1/S-2载荷} & (18.77,381.60)(16.00,450.00)(13.00,450.00)(11.00,449.00)(10.00,451.00) \\
 & (12.00,453.00)(15.00,455.00)(110.00,561.00)(110.00,563.00)(110.00,565.00) \\
 & (109.00,567.00)(112.00,568.00)(160.00,657.00)(159.00,659.00)(161.00,659.00) \\
 & (162.00,660.00)(164.00,658.00)(205.00,618.00)(209.00,615.00)(210.00,605.00) \\
 & (220.00,610.00)(223.00,615.00)(220.00,622.00)(230.00,620.00)(225.00,605.00) \\
 & (223.00,598.00)(174.00,544.00)(168.00,545.00)(164.00,544.00)(168.00,542.00) \\
 & (168.00,538.00)(175.00,472.00)(180.00,476.00)(183.00,473.00)(190.00,470.00) \\
 & (210.00,455.00)(185.00,460.00)(178.00,460.00)(180.00,455.00)(175.00,452.00) \\
 & (170.00,453.00)(107.00,475.00)(104.00,477.00)(103.00,473.00)(105.00,473.00) \\
 & (106.00,471.00)(113.00,416.00)(116.00,410.00)(114.00,405.00)(125.00,410.00) \\
 & (120.00,400.00)(112.00,394.00)(119.00,388.00)(94.00,315.00)(93.00,311.00) \\
 & (100.00,312.00)(96.00,304.00)(88.00,305.00)(86.00,310.00)(30.92,329.80) \\
\hline
无人机3、4 FY-1 S-1/S-2载荷 & (297.70,653.70)(264.00,715.00)(258.00,719.00)(254.00,728.00)(257.00,733.00) \\
 & (260.00,731.00)(262.00,733.00)(264.00,728.00)(268.00,733.00)(270.00,739.00) \\
 & (274.00,728.00)(315.00,762.90) \\
\hline
\end{tabular}
\end{table}

\begin{figure}[h]
    \centering
    \includegraphics[width=\textwidth]{image.png} % 替换为实际图像文件名
    \caption{无人机最优侦察路线规划示意图}
    \label{fig:drone_route}
\end{figure}

\subsubsection{5.1.3 结论与分析}

由上文基于 2-opt 算法优化模型的求解结果可知,无人机的调度策略如表 3 所示。

\begin{table}
\centering
\caption{FY-1型无人机调度策略}
\begin{tabular}{|c|c|c|c|c|}
\hline
无人机编号 & 起飞基地 & 加载载荷 & 起飞时间 & 侦察目标群 \\
\hline
FY-1-01 & P01 & S-1 & $t_1$ & A02, A03, A04 \\
 & & & & A05, A06, A07 \\
 & & & & A08, A09, A10 \\
\hline
FY-1-02 & P01 & S-2 & $t_1 + 3\text{min}$ & A02, A03, A02 \\
 & & & & A05, A06, A07 \\
 & & & & A08, A09, A10 \\
\hline
FY-1-03 & P03 & S-1 & $t_2$ & A01 \\
\hline
FY-1-04 & P03 & S-2 & $t_2 + 3\text{min}$ & A01 \\
\hline
\end{tabular}
\end{table}

在该种调度策略下,4 架次 FY-1 型无人机在防御方雷达有效探测范围内的总飞行距离为 2520.52km,用时 12.60h,为所有调度方案中的最小值。在本文基本假设的前提之下,该种调度策略能保证侦察无人机滞留防御方雷达有效探测范围内的时间总和最小,此时对应的航迹即为最佳侦察路线。

\subsection{5.2 问题 2 的求解}

\subsubsection{5.2.1 问题 2 的分析}

FY-1 型无人机对目标进行侦察时,须将侦察信息实时通过 FY-2 型无人机传回地面控制中心。鉴于 50km 通信距离的限制,需安排多架 FY-2 型无人机升空,以保证空中飞行的侦察无人机随时与 FY-2 型无人机的通信。FY-2 型无人机可同时与多架在其有效通信范围的侦察无人机通信并转发信息。

为了尽可能地安排较少架次的 FY-2 型通信中继无人机配合 FY-1 型无人机执行任务,只要求 FY-2 型通信中继无人机的飞行路线尽可能地经过多个 FY-1 型无人机周围 50km 的区域之中。问题即转化为寻找 FY-1 型无人机各条飞行路线周围 50km 的重叠区域。

根据目标建模的思想,利用概率优化的方法取出符合约束条件的若干点,即可得出 FY-2 型通信中继无人机在公共区域内的飞行路线,然后再合理地分配 FY-2 型通信中继无人机的架次。

\subsubsection{5.2.2 基于蒙特卡罗法的模型求解}

随机模拟方法,即蒙特卡罗(Monte Carlo)方法,它的基本思想是:为了求解该问题,首先在两条飞行路线之间的区域内产生随机数列,然后逐个验证这些元素的值是否符合约束条件,即该点到这两条飞行路线的距离是否都在 50km 之内,若都小于 50km 则保留该点,否则弃置并继续验证下一数值。如此,即能得出在该公共区域内的最佳飞行路线。

首先利用三样条插值(Cubic Spline Interpolation)算法对问题 1 结果中两条相邻的飞行路线进行重构,在每条路线上提取出 1000 个点。实现该目标的算法步骤为 \cite{ref2}:

(1) 输入 $n$ 个插值结点,$a = x_1 < x_2 < \cdots < x_n = b$,对应函数值为 $y_1, y_2, \cdots, y_n$,边界条件为 $y_1', y_2'$,待求插值点 $x_0$;

(2) 计算 \( h_{j} = x_{j+1} - x_{j}, (j = 1, 2, \cdots, n-1) \);

(3) 计算 \( \mu_{j} = \frac{h_{j-1}}{h_{j-1} + h_{j}}, d_{j} = 6 \left( \frac{y_{j+1} - y_{j}}{h_{j}} - \frac{y_{j} - y_{j-1}}{h_{j-1}} \right) \frac{1}{h_{j-1} + h_{j}} \ (j = 2, \cdots, n-1) \);

(4) 计算 \( \beta_{1} = \frac{6}{h_{1}} \left( \frac{y_{2} - y_{1}}{h_{1}} - y_{1}^{'} \right), \beta_{n} = \frac{6}{h_{n-1}} \left( y_{n}^{'} - \frac{y_{n} - y_{n-1}}{h_{n-1}} \right) \);

(5) 构造 \([a, b]\) 上的插值函数;

(6) 输出各区间中样条函数 \( S(x) \) 的表达式;

(7) 判断 \( x_{0} \) 所在区间 \([x_{j}, x_{j+1}]\),并计算插值 \( y_{0} \),绘制各区间的三次样条插值曲线,结束。

其次,利用蒙特卡罗方法,将搜索区域限定在这些点所构成的矩形平面内,并将产生的 100000 个随机点在此区域内进行遍历,筛选出满足下式的点:

\[
\begin{cases}
S_{n}^{1} < 50 \text{km} & n = 1, 2, \cdots, 10000 \\
S_{n}^{2} < 50 \text{km}
\end{cases}
\tag{3}
\]

其中,\( S_{n}^{1} \) 表示第 \( n \) 个随机点到第一条飞行路线上某点的距离,\( S_{n}^{2} \) 表示第 \( n \) 个随机点到第二条飞行路线上某点的距离。存在某随机点到这两条飞行路线上某两点的距离同时小于 50km 时,该点方可被保留在待确定的路线轨迹中。

通过蒙特卡罗法得出距离两条飞行路线 50km 之内的区域示意如图 6 所示,其中蓝色与黄色覆盖范围之内的部分即表示该段路线的有效通信范围。显然,两条飞行路线的有效通信范围没有任何重叠的部分,这说明在两段侦察任务同时进行的前提下,同一架 FY-2 型通信中继无人机无法同时对两架 FY-1 型无人机提供通信服务,应为两架 FY-1 型无人机各安排一架 FY-2 型无人机负责与基地之间的通信。

\begin{figure}[h]
    \centering
    \includegraphics[width=0.8\textwidth]{image.png}
    \caption{飞行路线有效通信范围示意图}
    \label{fig:6}
\end{figure}

\subsubsection{结论与分析}

由上文蒙特卡罗法模型求解的结果可知,为保证空中飞行的侦察无人机随时与 FY-2 型无人机的通信,至少安排 2 架次的 FY-2 型通信中继无人机与多架在其有效通信范围的侦察无人机通信并转发信息,方可完成问题 1 中的侦察任务。

\subsection{5.3 问题 3 的求解}

\subsubsection{5.3.1 问题 3 的分析}

问题要求在 7 个小时内(从第一架攻击无人机进入防御方雷达探测范围内起,到轰炸完最后一个目标止)完成对 10 个目标群所有 68 个地面目标的火力打击任务,并要保证攻击方的无人机滞留防御方雷达有效探测范围内的时间总和最小。根据实际作战规律判断,要保证无人机暴露在敌方雷达探测范围内的时间最小,在已知敌方雷达坐标的前提下,由于本题不考虑敌方防御武器的反击,进攻方应优先对敌方雷达设备重点进行破坏,而后再攻击其余非雷达设备。根据分步优化的思想,将该问题分解为两个步骤进行:

(1)分派一定数量的 FY-3 型无人机集中火力优先攻击 10 个目标群中的雷达,成功毁伤后防御方雷达随即失效,此时无人机暴露在防御方雷达有效探测范围内的时间总和应为最短;

(2)处在待命状态下的 FY-3 型无人机对各个目标群中的剩余目标进行攻击。

\subsubsection{5.3.2 无人机任务规划问题的求解}

鉴于问题以求取无人机滞留防御方雷达有效探测范围内的时间总和最小为任务目标,根据上述两个步骤进行问题的具体分析与求解。

第1步,分派一定数量的FY-3型无人机集中火力优先攻击10个目标群中的雷达,成功毁伤后防御方雷达随即失效,求解这种情况下无人机暴露在防御方雷达有效探测范围内的时间总和。

针对防御方雷达分布情况,统计防御方雷达分布距离,如表4所示:

\textbf{表4 防御方雷达分布距离统计表}

\begin{table}[h]
\centering
\begin{tabular}{|c|c|c|c|c|c|c|c|c|c|c|}
\hline
 & A01 & A02 & A03 & A04 & A05 & A06 & A07 & A08 & A09 & A10 \\
\hline
A01 & \diagbox{}{} & & & & & & & & & \\
\hline
A02 & 116.7 & \diagbox{}{} & & & & & & & & \\
\hline
A03 & 201.4 & 88.0 & \diagbox{}{} & & & & & & & \\
\hline
A04 & 265.5 & 150.7 & 93.0 & \diagbox{}{} & & & & & & \\
\hline
A05 & 346.4 & 230.3 & 146.1 & 105.5 & \diagbox{}{} & & & & & \\
\hline
A06 & 444.0 & 327.5 & 244.8 & 189.2 & 99.0 & \diagbox{}{} & & & & \\
\hline
A07 & 366.3 & 264.5 & 180.4 & 200.0 & 121.2 & 170.3 & \diagbox{}{} & & & \\
\hline
A08 & 115.9 & 83.6 & 122.1 & 210.5 & 263.4 & 362.1 & 258.4 & \diagbox{}{} & & \\
\hline
A09 & 217.8 & 123.1 & \textcolor{red}{62.4} & 145.7 & 161.3 & 257.4 & 148.7 & 111.8 & \diagbox{}{} & \\
\hline
A10 & 289.6 & 178.4 & 90.5 & 106.5 & \textcolor{red}{74.5} & 169.2 & 97.5 & 195.5 & 88.1 & \diagbox{}{} \\
\hline
\end{tabular}
\end{table}

易知,其中A03雷达与A09雷达、A05雷达与A10雷达之间的距离最短。仅将无人机滞留防御方雷达有效探测范围内的时间视为有效,拟定所有FY-1型无人机、FY-3型无人机均已在各目标群雷达探测边缘待命。本题下考虑两种方案进行分析对比,即:仅使用携带D-2型炸弹的FY-3型无人机;仅使用携带D-1型炸弹的FY-3型无人机。

由于携带D-2型炸弹的FY-3型无人机需FY-1型无人机载荷S-3的引导,而受FY-1型无人机架数的限制,因此选定8架携带D-2型炸弹的FY-3型无人机和8架加载载荷S-3的FY-1型无人机分别对A01、A02、A03、A04、A05、A06、A07、A08目标群雷达进行火力攻击。

在这种安排下,无人机滞留防御方雷达有效探测范围内的时间最小总和为:
\[
T = T_1 + T_2 + T_3 + T_4 = 5.66h
\]

其中,$T_1$为无人机被A01、A02、A04、A06、A07、A08目标群雷达覆盖时间;

$T_2$为加载载荷S-3的FY-1型无人机从轰炸A03目标群雷达至飞向A09目标群雷达轰炸点时间(由于FY-1型无人机巡航速度比FY-3型无人机速度慢,因此以FY-1型无人机耗时计算);

$T_3$为加载载荷S-3的FY-1型无人机从轰炸A05目标群雷达至飞向A10目标群雷达轰炸点时间(由于FY-1型无人机巡航速度比FY-3型无人机速度慢,因此以FY-1型无人机耗时计算);

$T_4$为D-2型炸弹轰炸A05目标群雷达与A10目标群雷达过程中所耗时间(两种型号无人机、两项轰炸任务叠加计算)。

\begin{align*}
T_1 &= \left( \frac{55}{200} + \frac{40}{300} + \frac{30}{200 \times 3.6} + \frac{30 - \frac{30}{200 \times 3.6} \times 30}{200 \times 3.6} \right) \times 6 = 2.94h \\
T_2 &= \frac{70 + 62.3939 - 15}{200} \times 2 = 1.17h \\
T_3 &= \frac{70 + 74.5252 - 15}{200} \times 2 = 1.29h \\
T_4 &= \frac{10}{300} \times 8 = 0.26h
\end{align*}

由于D-1型炸弹最终做抛物线运动,因此计算可得D-1型炸弹从$5000m$高空飞向目标共需$31.98s$,在此时间内D-1型炸弹水平方向飞行$2.67km$。因此如果仅用D-1型炸弹轰炸10个目标雷达,共需20架携带D-1型炸弹的FY-3型无人机,无人机滞留防御方雷达有效探测范围内的最小时间总和为:
\[
T = \left( \frac{70 - 2.67}{300} + \frac{31.98}{3600} \right) \times 2 \times 10 = 4.67h
\]

因此,经过对比可以发现,仅用20架携带D-1型炸弹的FY-3型无人机轰炸10个目标群雷达,无人机滞留防御方雷达有效探测范围内的时间最小,为$4.67h$。

第2步,处在待命状态下的FY-3型无人机对各个目标群中的剩余目标进行攻击。

假设在防御方雷达全被炸毁之前无人机不对其余非雷达目标进行轰炸。由于所有基地共拥有97架FY-3型无人机,上述过程已使用其中的20架装载D-1型号炸弹执行任务,现将剩余的FY-3型无人机装载上D-2型号或D-1型号炸弹,进行轰炸任务的优化分配。

单个基地一次性只能提供13架或者15架无人机,且为了保证此次任务的用时最短,不考虑无人机返回基地并再次起飞执行任务的情况。无人机基地与目标群之间的位置关系,如图7所示。

\begin{figure}[h]
    \centering
    \includegraphics[width=0.8\textwidth]{image.png}
    \caption{无人机基地与目标群之间的位置关系}
\end{figure}

为了节省资源与时间,各个基地分配的飞向目标群的无人机的总距离需为最短,同时,为了简化计算,根据原目标群中所有目标点的位置信息找到各目标群的位置中心点,然后对从 P01 和 P06 基地向这 10 个目标群中心点派出无人机的架次进行规划。

该问题的目标函数为:
\begin{equation}
\min = \sum_{i=1}^{10} d_i
\tag{4}
\end{equation}

约束条件为:
\begin{equation}
\begin{cases}
d_i = (x_1(i) + x_1(i+10)) \cdot a(i) + (x_2(i) + x_2(i+10)) \cdot b(i) \\
\quad + (x_3(i) + x_3(i+10)) \cdot c(i) + (x_4(i) + x_4(i+10)) \cdot d(i) \\
\quad + (x_5(i) + x_5(i+10)) \cdot e(i) + (x_6(i) + x_6(i+10)) \cdot f(i) \\
\quad + (x_7(i) + x_7(i+10)) \cdot g(i) \\
p_i = \frac{6}{10} \times (x_1(i) + x_2(i) + x_3(i) + x_4(i) + x_5(i) + x_6(i) + x_7(i)) \\
\quad + \frac{6}{2} \times (x_1(i+10) + x_2(i+10) + x_3(i+10) + x_4(i+10) \\
\quad \quad + x_5(i+10) + x_6(i+10) + x_7(i+10)) \\
10 \leq p_1, p_4 \leq 11 \\
9 \leq p_2 \leq 10 \\
5 \leq p_3, p_8, p_9, p_{10} \leq 6 \\
7 \leq p_5 \leq 8 \\
6 \leq p_6, p_7 \leq 7 \\
\sum_{j=1}^{20} x_i(j) \leq 13 \quad i \text{为奇数时} \\
\sum_{j=1}^{20} x_i(j) \leq 15 \quad i \text{为偶数时} \\
x_k(l) \in \mathbb{N} \quad k = 1, 2, \ldots, 7 \quad l = 1, 2, \ldots, 20
\end{cases}
\tag{5}
\end{equation}

其中,$d_i$ 为所有基地到第 $i$ 个目标群中心点的距离之和,$p_i$ 为能在第 $i$ 个目标群轰炸的目标个数,$x(1)$~$x(10)$ 为基地向各目标群派遣的携带 D-1 型炸弹的 FY-3 型无人机架次,$x(11)$~$x(20)$ 为基地向各目标群派遣的携带 D-2 型炸弹的 FY-3 型无人机架次,$a$、$b$、$c$、$d$、$e$、$f$、$g$ 为基地到各目标群中心点之间的距离。

利用 Lingo 软件对上述模型进行求解,得出最短路径和为 $17966.9 \, \text{km}$,FY-3 型无人机架次分配结果如表 5 所示。

\begin{table}
\centering
\caption{各基地派出 FY-3 型无人机架次的分配结果}
\begin{tabular}{|c|c|c|c|c|c|c|c|c|c|c|}
\hline
\diagbox{目标群}{基地} & A01 & A02 & A03 & A04 & A05 & A06 & A07 & A08 & A09 & A10 \\
\hline
\multirow{2}{*}{P01} & D-1 & 0 & 5 & 0 & 0 & 0 & 0 & 0 & 4 & 0 \\
\cline{2-11}
 & D-2 & 0 & 2 & 1 & 0 & 0 & 0 & 0 & 1 & 0 \\
\hline
\multirow{2}{*}{P05} & D-1 & 2 & 0 & 4 & 0 & 0 & 0 & 0 & 0 & 3 \\
\cline{2-11}
 & D-2 & 3 & 0 & 0 & 0 & 0 & 0 & 0 & 1 & 0 \\
\hline
\multirow{2}{*}{P06} & D-1 & 0 & 0 & 0 & 2 & 2 & 0 & 0 & 0 & 1 \\
\cline{2-11}
 & D-2 & 0 & 0 & 0 & 3 & 2 & 0 & 0 & 0 & 1 \\
\hline
\multirow{2}{*}{P07} & D-1 & 0 & 0 & 0 & 0 & 0 & 5 & 5 & 0 & 0 \\
\cline{2-11}
 & D-2 & 0 & 0 & 0 & 0 & 0 & 1 & 1 & 0 & 0 \\
\hline
\multirow{2}{*}{总计} & D-1 & 2 & 5 & 4 & 2 & 2 & 5 & 5 & 4 & 4 \\
\cline{2-11}
 & D-2 & 3 & 2 & 1 & 3 & 2 & 1 & 1 & 1 & 1 \\
\hline
\end{tabular}
\end{table}

根据 FY-3 型无人机架次的分配结果,可作出如图 8 所示的所有参战无人机的飞行路线规划图。

\begin{figure}[h]
\centering
\includegraphics[width=0.8\textwidth]{image.png}
\caption{所有参战无人机的飞行路线规划图}
\end{figure}

下面分析验证在上述任务规划下,对 10 个目标群所有 68 个地面目标的火力打击任务能在 7 小时之内完成。

先求第 1 步轰炸完所有防御方雷达所需的任务时间为:
\[
t_{step1} = \frac{4.27}{20} = 0.2335h
\]

其次,求从第 1 步完成到第 2 步中最后一个目标点被轰炸完毕所需的任务时间:

(1) 原待命 FY-3 型无人机各自飞行至对应攻击点,这段过程中的任务耗时以最长飞行距离耗时计,即从点 $(279, 4.1km, 446.2.4km)$ 飞行到点 $(161.2km, 658.6km)$,用时为 $0.810h$;

(2) 由于攻击同一目标的 D-2 型炸弹在整个飞行过程中需一架 FY-1 型无人机加载载荷 S-3 进行全程引导,直到命中目标,即发射另一枚炸弹需等待前一枚炸弹击中目标,以最大总等待时间计,用时 $0.417h$。

上述两部分总需要的任务时间为:

\[
t_{step2} = 0.810 + 0.417 = 1.227h
\]

综上,此任务规划下无人机完成对所有地面目标火力打击任务的时间总和 $t_{sum} = t_{step1} + t_{step2} = 1.4605h < 7h$,满足题设条件的要求。

\subsubsection{5.3.3 结论与分析}

根据以上分析求解过程,可得出如表 6 所示的无人机任务规划结果清单。

\textbf{表 6 无人机任务规划结果清单}

\begin{table}[h]
\centering
\begin{tabular}{|c|c|c|c|}
\hline
基地 & 无人机机型 & 炸弹类型与数量 & 攻击目标 \\
\hline
P01 & FY-3 型 (D-1) 9 架次 \\
& FY-3 型 (D-2) 4 架次 \\
& FY-1 型 2 架次 & D-1 型 6 枚/架次 & A02 中 7 个 \\
& & D-2 型 6 枚/架次 & A03 中 1 个 \\
& & & A08 中 5 个 \\
\hline
P03 & FY-1 型 2 架次 & & \\
\hline
P05 & FY-3 型 (D-1) 9 架次 \\
& FY-3 型 (D-2) 4 架次 \\
& FY-1 型 2 架次 & D-1 型 6 枚/架次 & A01 中 5 个 \\
& & D-2 型 6 枚/架次 & A03 中 4 个 \\
& & & A09 中 4 个 \\
\hline
P06 & FY-3 型 (D-1) 9 架次 \\
& FY-3 型 (D-2) 6 架次 & D-1 型 6 枚/架次 & A04 中 5 个 \\
& & D-2 型 6 枚/架次 & A05 中 4 个 \\
& & & A09 中 1 个 \\
& & & A10 中 5 个 \\
\hline
P07 & FY-3 型 (D-1) 9 架次 \\
& FY-3 型 (D-2) 6 架次 \\
& FY-1 型 2 架次 & D-1 型 6 枚/架次 & A06 中 6 个 \\
& & D-2 型 6 枚/架次 & A07 中 6 个 \\
\hline
\end{tabular}
\end{table}

\subsection{5.4 问题 4 的求解}

\subsubsection{5.4.1 问题 4 的分析}

由于在 A02、A05、A09 周边可能还配置有三部远程搜索雷达,该雷达对 FY 型无人机的有效作用距离是 $200km$,工作模式是相继开机工作,即只有首先开机的雷达遭到攻击后才开启第二部雷达,同样只有第二部雷达被攻击后才开启第三部雷达。首先要站在防御方的角度部署远程搜索雷达,每个远程搜索雷达部署的原则是,尽可能多地将己方目标覆盖在 $200km$ 的范围之内,且距离最近的己方目标须分别隶属于 A02、A05、A09 目标群。根据该原则可规划出这三个目标群周边远程搜索雷达的部署位置,且远程雷达的开启顺序应根据覆盖范围内目标个数的多少决定。

远程雷达分布及开启情况确定之后,再进行无人机的任务规划。针对这个问题,可在问题 3 求解的基础之上进行修正与叠加。首先集中火力优先攻击 A09 周边的远程雷达,再攻击各目标群的防御雷达,最后对剩余目标进行攻击。

\subsubsection{5.4.2 基于粒子群算法的远程搜索雷达部署模型求解}

站在防御方的角度部署远程搜索雷达,每个远程搜索雷达的部署都需要尽可能多地将己方目标覆盖在 200km 的范围之内,且距离最近的己方目标须分别隶属于 A02、A05、A09 目标群。将这一原则抽象成目标函数和约束条件分别如式(6)和式(7)所示。

\begin{equation}
\max \sum_{j=1}^{68} \left\lceil 1 - \frac{S_{U_q, T_j}}{200} \right\rceil \quad q = 1, 2, 3 \quad j = 1, 2, \cdots, 68
\tag{6}
\end{equation}

\begin{equation}
\min S_{U_q, T_j}, \quad T_j \in \{A02, A05, A09\}
\tag{7}
\end{equation}

其中,$U_q$ 表示远程搜索雷达,$T_j$ 表示攻击目标,$S_{U_q, T_j}$ 表示远程搜索雷达和攻击目标之间的距离,“$\lceil \rceil$”表示向上取整,集合 $\{A02, A05, A09\}$ 表示 A02、A05、A09 目标点群。

现采用粒子群算法实现这一最优规划。粒子群算法(Particle Swarm Optimization,PSO)是一种基于群体的随机优化技术 [3],它将每个可能产生的解表述为群中的每一个微粒,每个微粒都具有自己的位置向量和速度向量,以及一个由目标函数决定的适应度,所有微粒在搜索空间中以一定速度飞行,在初始化一组随机解后通过迭代搜寻最优解。

粒子群算法解决该问题的过程为 [4]:

(1)初始化一个群体规模 $m=40$ 的二维粒子群,随机初始化粒子的位置 $x_{is}$ 和速度 $v_{is}$;

(2)计算每个粒子的适应值;

(3)对每个粒子将其适应值和其经历过的最好位置 $p_{is}$ 的适应值进行比较,若较好,则将其作为当前的最好位置;

(4)对每个粒子将其适应值和全局经历过的最好位置 $p_{gs}$ 的适应值进行比较,若较好,则将其作为当前的全局最好位置;

(5)根据以下两个式子对粒子的速度和位置进行更新:

\begin{equation}
v_{is}(t+1) = \omega \cdot v_{is}(t) + c_1 r_1 [p_{is}(t) - x_{is}(t)] + c_2 r_2 [p_{gs}(t) - x_{gs}(t)]
\tag{8}
\end{equation}

\begin{equation}
x_{is}(t+1) = x_{is}(t) + v_{is}(t+1)
\tag{9}
\end{equation}

其中,惯性权重 $\omega$ 取 0.7298,学习因子 $c_1=2$、$c_2=2$,$r_1$、$r_2$ 为服从 $[0, 1]$ 上均匀分布的相互独立的伪随机数。

(6)如果达到设定的最大迭代次数 1000,则输出解,否则返回步骤(2)。经粒子群算法模型规划出的远程搜索雷达分布情况如图 9 所示。

\begin{figure}[h]
    \centering
    \includegraphics[width=\textwidth]{image.png}
    \caption{远程搜索雷达分布情况示意图}
    \label{fig:radar_distribution}
\end{figure}

图 9 中,A02 周边远程雷达坐标 $(220.97\,km, 551.54\,km)$,可覆盖 56 个目标;A05 周边远程雷达坐标 $(123.03\,km, 472.73\,km)$,可覆盖 58 个目标;A09 周边远程雷达坐标 $(146.81\,km, 583.35\,km)$,可覆盖 62 个目标。根据远程搜索雷达覆盖范围内的目标总数越多越优先开启的原则,这三部雷达开启的顺序为:A09 周边远程雷达 $\rightarrow$ A05 周边远程雷达 $\rightarrow$ A02 周边远程雷达。

\subsubsection{5.4.3 无人机任务规划问题的求解}

无人机首先的轰炸目标应为部署在 A09 周边的那个远程雷达,而后再攻击目标群以及随后开启的远程雷达。求解的过程中,在问题 3 结果的基础上叠加攻击 A09 周边远程雷达的时间、攻击随后依次开启的两个远程雷达的时间、在各远程雷达覆盖范围之内无人机的滞留时间。

按进攻过程分析,第一个远程雷达开启时的进攻示意如图 10 所示,除了问题 3 中提到的无人机之外,还有 3 架 FY-3 型无人机专门负责进攻三部远程雷达,开始阶段其中两架处于无任务状态,直至后续远程雷达被激活。

\begin{figure}[h]
    \centering
    \includegraphics[width=\textwidth]{image.png}
    \caption{第一个远程雷达开启时的进攻示意图}
    \label{fig:radar_attack}
\end{figure}

首先,P01 基地起飞的 3 架 FY-3 型无人机进入远程雷达的覆盖范围外界,直飞抵达 A02 目标群雷达覆盖范围外界,即从点 $(303.0\,km, 458.2\,km)$ 到点 $(262.5\,km, 545.9\,km)$,用时 $T_{ex1}=0.322 \times 7 = 2.25\,h$。

此时,所有进攻无人机都已在各自目标的包络面上集结完毕,开始进攻远程雷达和各目标群的雷达,用时:
\begin{equation}
T_{1} = \left( \frac{70 - 2.67}{300} + \frac{31.98}{3600} + \frac{30 - \frac{30}{200 \times 3.6} \times 30}{200 \times 3.6} \right) \times 23 = 6.28\,h
\end{equation}

第二个远程雷达开启时,有 35 架无人机在其覆盖范围之内,至该远程雷达被轰炸完毕,用时 $T_{ex2}=0.0277 \times 35 = 0.97\,h$。

第三个远程雷达开启时,有 52 架无人机在其覆盖范围之内,至该远程雷达被轰炸完毕,用时 $T_{ex3}=0.0277 \times 52 = 1.44\,h$。

本问题后续的 $T_{2}$、$T_{3}$、$T_{4}$ 与问题 3 中的相同。

在这种安排下,无人机滞留防御方雷达有效探测范围内的时间最小总和为:
\begin{equation}
T = T_{1} + T_{ex1} + T_{ex2} + T_{ex3} = 10.94\,h
\end{equation}

下面同样要分析验证在上述任务规划下,对 10 个目标群所有 68 个地面目标的火力打击任务能在 7 小时之内完成。

由于此题中无人机总滞留时间相对于问题 3 平均增加了 $2.09\,h$,而问题 3 中计算所得的完成任务总时间为 $1.4605\,h$,则完成对 10 个目标群所有 68 个地面目标的火力打击任务的总时间 $t_{sum}$ 必然满足:

\subsubsection{5.4.4 结论与分析}

本问题先基于防御方的角度根据粒子群算法求解出了三个远程雷达的部署位置,且这三部雷达开启的顺序为 A09 周边远程雷达 \(\rightarrow\) A05 周边远程雷达 \(\rightarrow\) A02 周边远程雷达。而后根据与问题 3 不同的限制条件,得出无人机任务规划结果清单如表 7 所示。

\begin{table}[h]
\centering
\caption{无人机任务规划结果清单}
\begin{tabular}{|c|c|c|c|}
\hline
基地 & 无人机机型 & 炸弹类型与数量 & 攻击目标 \\
\hline
P01 & FY-3型(D-1)9架次 \\
& FY-3型(D-2)4架次 \\
& FY-1型2架次 & D-1型6枚/架次 \\
& & D-2型6枚/架次 & A02中7个 \\
& & & A03中1个 \\
& & & A08中5个 \\
\hline
P03 & FY-1型2架次 & & \\
\hline
P05 & FY-3型(D-1)9架次 \\
& FY-3型(D-2)4架次 \\
& FY-1型2架次 & D-1型6枚/架次 \\
& & D-2型6枚/架次 & A01中5个 \\
& & & A03中4个 \\
& & & A09中4个 \\
\hline
P06 & FY-3型(D-1)9架次 \\
& FY-3型(D-2)6架次 & D-1型6枚/架次 \\
& & D-2型6枚/架次 & A04中5个 \\
& & & A05中4个 \\
& & & A09中1个 \\
& & & A10中5个 \\
\hline
P07 & FY-3型(D-1)9架次 \\
& FY-3型(D-2)6架次 \\
& FY-1型2架次 & D-1型6枚/架次 \\
& & D-2型6枚/架次 & A06中6个 \\
& & & A07中6个 \\
\hline
任意 & FY-3型(D-2)3架次 & D-2型6枚/架次 & A09远程雷达 \\
& & & A05远程雷达 \\
& & & A02远程雷达 \\
\hline
\end{tabular}
\end{table}

\subsection{5.5 问题 5 的求解}

\subsubsection{5.5.1 问题 5 的分析}

本问题主要有两个方面需要考虑,一是对本文求解算法的复杂度进行分析,并且分析得出能有效提高算法效率以增强任务规划时效性的方法。二是基于构建的数学模型和对模型解算的结果,分析不同技术参数的提高对无人机作战能力提升的影响。

首先就算法的复杂度进行分析,分别对文中使用的 2-opt 算法和粒子群算法的复杂度进行计算。同时针对粒子群算法,可通过在不同迭代过程中逐代减少粒子数的方法提高算法的时效性。

根据本文建模时考虑的因素以及本文模型求解出的结果,可从无人机的最大巡航速度、巡航高度、最长巡航时间、机动性能、机载设备(机载传感器的精度、可挂载的武器数量等因、隐身能力(反雷达侦察能力的大小)以及与基地的通信能力这几个方面入手进行分析。

\subsubsection{5.5.2 对模型求解算法的分析}

算法的复杂度主要包括时间复杂度和空间复杂度。

\paragraph{(1) 算法的时间复杂度}

算法的时间复杂度是指执行算法所需要的计算工作量。

为了能够比较客观地反映出一个算法的效率,在度量一个算法的工作量时,不仅应该与所使用的计算机、程序设计语言以及程序编制者无关,而且还应该与算法实现过程中的许多细节无关。为此,可以用算法在执行过程中所需基本运算的执行次数来度量算法的工作量。基本运算反映了算法运算的主要特征,因此,用基本运算的次数来度量算法工作量是客观的也是实际可行的,有利于比较同一问题的几种算法的优劣。

算法所执行的基本运算次数还与问题的规模有关,因此在分析算法的工作量时,还必须对问题的规模进行度量。

综上所述,算法的工作量用算法所执行的基本运算次数来度量,而算法所执行的基本运算次数是问题规模 \( n \) 的函数,即:

\[
\text{算法的工作量} = f(n)
\]

在具体分析一个算法的工作量时,还会存在这样的问题:对于一个固定的规模,算法所执行的基本运算次数还可能与特定的输入有关,而实际上又不可能将所有可能情况都列举出来。在同一问题规模下,如果算法执行所需的基本运算次数取决于某一特定输入时,可以用以下两种方法来分析算法的工作量。

\begin{enumerate}
    \item \textbf{平均性态(Average Behavior)}

    所谓平均性态分析,是指用各种特定输入下的基本运算次数的带权平均值来度量算法的工作量。

    \item \textbf{最坏情况复杂性(Worst-Case Complexity)}

    所谓最坏情况分析,是指在规模为 \( n \) 时算法所执行基本运算的最大次数。它定义为:

    \begin{equation}
        W(n) = \max_{x \in D_n} \{ t(x) \}
        \tag{10}
    \end{equation}

    显然 \( W(n) \) 的计算要比 \( A(n) \) 的计算方便得多。由于 \( W(n) \) 实际上是给出了算法工作量的一个上界,因此,它比 \( A(n) \) 更具有实用价值。
\end{enumerate}

\paragraph{(2) 算法的空间复杂度}

一个算法的空间复杂度,一般是指执行这个算法所需要的内存空间。

一个算法所占用的存储空间包括算法程序所占的空间、输入的初始数据所占的存储空间以及算法执行过程中所需要的额外空间。其中额外空间包括算法程序执行过程中的工作单元以及某种数据结构所需要的附加存储空间。如果额外空间量相对于问题规模来说是常数,则称该算法是原地工作的。在许多实际问题中,为了减少算法所占的存储空间,通常采用压缩存储技术,以便尽量减少不必要的额外空间 \cite{ref5}。

以下主要对进行飞行路线最优规划的 2-opt 算法和进行雷达位置最优分配的粒子群算法的复杂性进行具体分析:

\paragraph{(1) 2-opt 算法的复杂性}

2-opt 算法的时间复杂度为 $O(n^2)$,每一次搜索均是插入新点以新边代替旧边,相当于领域变换中的 2 变换,该局部搜索算法相对其余智能算法能在较短的时间内获得最优解。由于本题以两两目标的距离作最优规划,2-opt 算法的空间复杂度受限于目标点的选取,当问题 1 规划最终 60 个点时,空间复杂度最高。

\paragraph{(2) 粒子群算法的复杂性}

首先针对上文采用的传统 PSO 算法的平均计算时间复杂度进行理论分析。根据传统 PSO 算法的描述可以看出,对于传统的 PSO 算法,其每一次迭代中的粒子数量不变。假设第 $i$ 步迭代中粒子的数量为 $N_i$,其中 $i=1,2,3,\cdots,m$,$m$ 表示最大迭代次数,因此有 $N_1=N_2=\cdots=N_m=N$。假设每个粒子每一次迭代需要的运算时间为 $T_{iter}$,则传统 PSO 算法优化所需要的总运行时间为 $N_i \times m \times T_{iter}$。

问题 4 中采用的 PSO 算法中相应的参数分别设置为:一次 PSO 算法每一步迭代中的粒子数为 $N=40$,共需迭代的次数 $m=1000$,每一次迭代所需的时间为 $T_{iter}=\Delta t$。根据这些参数,可以计算得出问题 4 中采用的 PSO 算法的时间复杂度为:

\begin{equation}
N \times m \times T_{iter} = 40000 \Delta t
\tag{11}
\end{equation}

为了进一步改进远程雷达位置规划问题的求解,提高算法的效率,降低算法的复杂性,主要有以下三种思路:一是减少每次迭代的粒子个数,二是减少一次算法的总迭代次数,三是降低每次迭代所需的时间。其中算法的迭代次数影响算法最优解的收敛,一般情况下不宜选取过小;而每次迭代所需的时间又与每次迭代的粒子个数密切相关,所以应当从减少每次迭代的粒子个数这一角度出发,降低 PSO 算法的时间复杂度。但是如果将每次迭代的粒子个数均减小,势必在一定程度上影响到算法的有效性。考虑到这一点,可以在初始迭代过程中采用较多的粒子个数,随着算法迭代的深入,子代粒子较上一代粒子更优,从而可以适当减少粒子的个数以降低算法的时间复杂度,提高算法的效率。

改进粒子群算法(DPSO)[6]就是采用这样的思路,粒子的数量随迭代逐渐减少,即:$N_1 \geq N_2 \geq \cdots \geq N_m$。假设每个粒子每一次迭代需要的运算时间为 $T_D$,则 DPSO 算法进行优化后所需要的总的运时间为 $\sum_{i=1}^{m} N_i \times T_D$。

由上述分析可知,两种算法复杂度的差别主要体现在每一次迭代中的粒子数量和每个粒子一次迭代需要的运行时间上[7]。所以,为了进一步提高远程雷达布置位置的最优规划问题的求解效率,可以采用比传统 PSO 算法更优的 DPSO 算法。

\subsubsection{5.5.3 对无人机作战能力的分析}

无人机的作战能力可定义为:在给定条件下,运用无人机系统的作战兵力来执行作战任务,所能达到预期目标的程度。作战能力是一个相对的、定量的值,需要考虑特定的使用环境和特定的任务目标。而作战能力评估则是对系统作战能力的分析和评价[8]。

基于本文构建的数学模型和对模型解算的结果,可以看出,无人机的最大巡航速度、巡航高度、最长巡航时间、机动性能、机载设备(机载传感器的精度、可挂载的武器数量等因素)、隐身能力(即反雷达侦察能力的大小)以及与基地的通信能力都是影响无人机作战能力的重要因素。下面具体分析一下不同技术参数的提升对于无人机作战能力的提升。

(1)巡航速度是指飞机所装发动机每 100 公里消耗燃油最小情况下的飞行速度。

无人机的巡航速度越快,其所能达到的最远作战半径就越远。同时挂载在无人机上利用无人机的惯性速度进行目标攻击的炸弹的威力也就越大。因此,提高无人机的最大巡航速度能够有效地提高无人机的作战能力。

(2)巡航高度是指飞机完成起飞阶段进入预定航线后的飞行状态下处在的高度。

无人机的巡航高度越高,其躲避敌方雷达侦察的能力就越强,同时敌方对其进行毁灭的概率就越小,而且挂载在无人机上作平抛运动进行目标攻击的炸弹的威力也就越大。所以,提高无人机的最大巡航速度能够有效地提高无人机的作战能力。

(3)巡航时间是指飞机在不进行空中加油的情况下,耗尽其本身携带的可用燃料时,所能持续飞行的时间。

无人机的巡航时间越长,在巡航速度一定的情况下,无人机的作战半径就越大。所以,无人机巡航时间的提高有助于提升无人机的作战能力。

(4)无人机的机动动作是指飞行状态(速度、高度和飞行方向)随时间变化的飞行动作。

无人机的机动性能越好,其在躲避敌方攻击时改变自身运动轨迹的速度就越快。例如无人机的机动性能越好,其转弯半径就越小,无人机就能够迅速地完成转弯,达到快速返回或者改变飞行路径以规避敌方攻击的目的。因此,提高无人机的机动能力可以有效提高无人机的作战能力。

(5)无人机的机载设备(机载传感器的精度、可挂载的武器数量等因素)对于提高无人机的作战性能同样意义重大。机载传感器的精度越高,无人机就能在离目标越远的范围内完成对目标的侦察、打击,减少无人机暴露在敌方攻击范围内的时间。无人机挂载武器的数量越多,无人机对于敌方的破坏能力就越大。

(6)无人机的隐身性能越好,无人机对于敌方雷达的反侦察能力就越强,无人机在敌方攻击范围内的安全性就越有保障。换言之,削弱了敌方的攻击能力,也就从另一个角度提高了无人机自身的作战能力。

(7)无人机与基地的通信能力对于提升无人机的作战能力具有举足轻重的作用。在信息化战争中,尤其是在缺失了飞行员的现场判断之后,无人机与控制端的通信能力就是无人机战斗能力的集中体现。只有控制端的指令能够实时传输给无人机、无人机侦察到的实时战况反馈给控制端,才能在最大程度上发挥无人机的作战能力。

\section{六、模型的评价与改进}

问题 1 的研究中,构建了基于 2-opt 算法的单目标优化模型,通过遍历比较在 10 个目标群中心点之间随机生成的路径组合,筛选出完成侦察任务的最短路径,以此为据分配出无人机的调度方案。由于 2-opt 算法每一次搜索均是插入新点以新边代替旧边,相对其余智能算法能在较短的时间内获得最优解,使得该问题的解决模型具有较好的时间优势。而由于本题是对两两目标间的距离作最优规划,2-opt 算法的空间复杂度受限于目标点的选取,空间复杂度最高,占用计算机内存较大。本题是在找到 10 个目标群中心点之后进行最优路线选择,再进一步细化为经过 68 个目标点的路线,因而本题模型的改进可结合 2-opt 算法和其它智能算法各自的优势,实现对所有目标点的一次性最优规划。

问题 2 的研究中,利用基于蒙特卡罗法的求解模型,根据目标建模的思想,通过随机模拟的方法寻找可能出现在飞行路线有效通信范围内的点,即寻找是否存在公共区域内的飞行路线可供 FY-2 型通信中继无人机一次性服务多架 FY-1 型无人机,从而合理地分配 FY-2 型无人机的架次。这种方法避免了求解真实曲线的方程表达式,是一种从目标倒推的思想,有效地节省了解算时间,增强了任务规划的时效性。而由于该模型求解的首步是在两条飞行路线之间的区域内产生随机数列,然后逐个验证这些元素的值是否符合约束条件,解算结果一定程度上依赖于该随机数列的规模。

问题 3 和问题 4 的研究中,构建了无人机任务分配的线性规划模型,分别在防御方有无部署远程搜索雷达的前提下,为进攻方火力打击任务规划出了无人机飞行路线、携带炸弹清单、攻击目标清单等具体方案。在利用粒子群算法模型解决部署远程搜索雷达的问题时,粒子群充分发挥了能在全局范围内寻找最优点的优势,提高了确定最佳部署位置点的效率。但粒子群算法本身存在不确定性,每次寻优的结果可能不一样,需要通过多次解算来验证。该模型以消耗无人机资源最小为目标,后续可根据实际情况,在考虑到无人机基地自身防御能力地前提下,协调各基地派出无人机架次的安排。

问题 5 的研究中,对本文算法的复杂度进行了分析。在求解无人机最优航线时,采用分层思想,先对 10 个目标群的中心点寻找最佳侦察路线与调度策略,只需比较 $2^9$ 次组合,避免了一次性处理 $2^{67}$ 次组合,增强了任务规划的时效性。粒子群算法的时间复杂度与迭代粒子数、迭代次数、迭代时间有关,在算法执行的过程中动态改变迭代粒子数可有效提高任务规划的效率。

\section{七、参考文献}

[1] 祝崇隽, 刘民, 吴澄, 等. 针对 CVRP 的 2-OPT 算法的时间复杂度均值分析 [J]. 清华大学学报(自然科学版), 2002, 42(9):1218-1221.

[2] 许小勇, 钟太勇. 三次样条插值函数的构造与 Matlab 实现[J]. 兵工自动化, 2006, 25(11):76-78.

[3] 赵明, 李涛, 苏小红, 等. 三维多无人机系统协同任务规划关键问题综述[J]. 智能计算机与应用, 2016, 6(1).

[4] 高飞. MATLAB 智能算法超级学习手册[M]. 人民邮电出版社, 2014:226.

[5] 高巍. 计算机软件技术基础[M]. 清华大学出版社, 2007:13-15.

[6] 王文韬. 基于改进粒子群算法的三维无人机航迹规划[D]. 沈阳建筑大学, 2013.

[7] 王沁, 李磊, 陆成勇, 等. 平均计算时间复杂度优化的动态粒子群优化算法 [J]. 计算机科学, 2010, 37(3):191-194.

[8] 王晓光. 无人机作战效能评估方法研究[D]. 西北工业大学, 2015.

\section{八、附录}

1、问题 1 中对 10 个中心点的两架次最优调度规划 Matlab 源程序

\begin{verbatim}
clc;clear all;close all;
position=load('坐标点.txt');
[m,n]=size(position);
j=1;
sumx=0;
sumy=0;
for i=1:10
    sumx=sumx+position(i,1);
    sumy=sumy+position(i,2);
end
ave(j,1)=sumx/10;ave(j,2)=sumy/10;
j=j+1;
sumx=0;
sumy=0;
for i=11:19
    sumx=sumx+position(i,1);
    sumy=sumy+position(i,2);
end
ave(j,1)=sumx/9;ave(j,2)=sumy/9;
j=j+1;
sumx=0;
sumy=0;
for i=20:24
    sumx=sumx+position(i,1);
    sumy=sumy+position(i,2);
end
ave(j,1)=sumx/5;ave(j,2)=sumy/5;
j=j+1;
sumx=0;
sumy=0;
for i=25:34
    sumx=sumx+position(i,1);
    sumy=sumy+position(i,2);
end
ave(j,1)=sumx/10;ave(j,2)=sumy/10;
j=j+1;
sumx=0;
sumy=0;
for i=35:41
    sumx=sumx+position(i,1);
    sumy=sumy+position(i,2);
\end{verbatim}

\begin{verbatim}
end
ave(j,1)=sumx/7; ave(j,2)=sumy/7;
j=j+1;
sumx=0;
sumy=0;
for i=42:47
    sumx=sumx+position(i,1);
    sumy=sumy+position(i,2);
end
ave(j,1)=sumx/6; ave(j,2)=sumy/6;
j=j+1;
sumx=0;
sumy=0;
for i=48:53
    sumx=sumx+position(i,1);
    sumy=sumy+position(i,2);
end
ave(j,1)=sumx/6; ave(j,2)=sumy/6;
j=j+1;
sumx=0;
sumy=0;
for i=54:58
    sumx=sumx+position(i,1);
    sumy=sumy+position(i,2);
end
ave(j,1)=sumx/5; ave(j,2)=sumy/5;
j=j+1;
sumx=0;
sumy=0;
for i=59:63
    sumx=sumx+position(i,1);
    sumy=sumy+position(i,2);
end
ave(j,1)=sumx/5; ave(j,2)=sumy/5;
j=j+1;
sumx=0;
sumy=0;
for i=64:68
    sumx=sumx+position(i,1);
    sumy=sumy+position(i,2);
end
ave(j,1)=sumx/5; ave(j,2)=sumy/5;
j=j+1;
% ave=position;
\end{verbatim}

\begin{verbatim}
Routebest=10000000000;
for one=1:2^10/2
    A=[];B=[];index=dec2bin(one,10);
    for two=1:10
        if index(two)=='0'
            A=[A;ave(two,:)];
        else
            B=[B;ave(two,:)];
        end
    end
    [m,n]=size(A);
    fit1=[];p1=[];
    if m==1
        fit1=A;L1=0;
    else
        [d1]=dj(A);
        [p1,L1] = tspsearch2(d1,2);

        for i=1:m
            fit1(i,1)=A(p1(i),1);
            fit1(i,2)=A(p1(i),2);
        end
    end
    [m,n]=size(B);
    fit2=[];p2=[];
    if m==1
        fit2=B;L2=0;
    else
        [d2]=dj(B);
        [p2,L2] = tspsearch2(d2,2);

        for i=1:m
            fit2(i,1)=B(p2(i),1);
            fit2(i,2)=B(p2(i),2);
        end
    end
    if (L1+L2)<Routebest
        Routebest=L1+L2;
        fit1best=[];fit2best=[];p1best=[];p2best=[];
        fit1best=fit1;fit2best=fit2;p1best=p1;p2best=p2;
    end
end
a=0:pi/50:2*pi;R=70;x=R*cos(a)+264;y=R*sin(a)+715;plot(x,y,'g-.');axis equal ;hold on
\end{verbatim}

\begin{verbatim}
a=0:pi/50:2*pi;R=70;x=R*cos(a)+225;y=R*sin(a)+605;plot(x,y,'g-.');axis equal;hold on
a=0:pi/50:2*pi;R=70;x=R*cos(a)+168;y=R*sin(a)+538;plot(x,y,'g-.');axis equal;hold on
a=0:pi/50:2*pi;R=70;x=R*cos(a)+210;y=R*sin(a)+455;plot(x,y,'g-.');axis equal;hold on
a=0:pi/50:2*pi;R=70;x=R*cos(a)+120;y=R*sin(a)+400;plot(x,y,'g-.');axis equal;hold on
a=0:pi/50:2*pi;R=70;x=R*cos(a)+96;y=R*sin(a)+304;plot(x,y,'g-.');axis equal;hold on
a=0:pi/50:2*pi;R=70;x=R*cos(a)+10;y=R*sin(a)+451;plot(x,y,'g-.');axis equal;hold on
a=0:pi/50:2*pi;R=70;x=R*cos(a)+162;y=R*sin(a)+660;plot(x,y,'g-.');axis equal;hold on
a=0:pi/50:2*pi;R=70;x=R*cos(a)+110;y=R*sin(a)+561;plot(x,y,'g-.');axis equal;hold on
a=0:pi/50:2*pi;R=70;x=R*cos(a)+105;y=R*sin(a)+473;plot(x,y,'g-.');axis equal;hold on
for i=1:10
    plot(ave(i,1),ave(i,2),'r*');hold on
end
sum=0;
[m1,n1]=size(fit1best);
if m1==1
    sum=0;
else
    for i=1:m1-1
        sum=sum+sqrt((fit1best(i,1)-fit1best(i+1,1))^2+(fit1best(i,2)-fit1best(i+1,2))^2);
    end
end
[m2,n2]=size(fit2best);
if m2==0
    sum=sum;
else
    for i=1:m2-1
        sum=sum+sqrt((fit2best(i,1)-fit2best(i+1,1))^2+(fit2best(i,2)-fit2best(i+1,2))^2);
    end
end
plot( fit1best(:,1), fit1best(:,2));
plot( fit2best(:,1), fit2best(:,2));
\end{verbatim}

\begin{verbatim}
2、问题 1 中对 68 个目标点的两架次最优调度规划 Matlab 源程序
clc;clear all;close all;
position=load('坐标点.txt');
po1=position(1:10,:);%A1
po2=position(11:19,:);%A2
po8=position(54:58,:);%A8
po9=position(59:63,:);%A9
po3=position(20:24,:); %A3
po4=position(25:34,:);%A4
po10=position(64:68,:);%A10
po7=position(48:53,:);%A7
po5=position(35:41,:);%A5
po6=position(42:47,:); %A6
A=[po2;po3;po4;po5;po6;po7;po8;po9;po10];
B=po1;
[d1]=dj(A);
[p1,L1] = tspsearch2(d1,2);
[m,n]=size(A) ;
for i=1:m
    fit1(i,1)=A(p1(i),1);
    fit1(i,2)=A(p1(i),2);
end
[d2]=dj(B);
[p2,L2] = tspsearch2(d2,2);
[m,n]=size(B) ;
for i=1:m
    fit2(i,1)=B(p2(i),1);
    fit2(i,2)=B(p2(i),2);
end
a=0:pi/50:2*pi;R=70;x=R*cos(a)+264;y=R*sin(a)+715;plot(x,y,'g-.');axis equal ;hold on
a=0:pi/50:2*pi;R=70;x=R*cos(a)+225;y=R*sin(a)+605;plot(x,y,'g-.');axis equal;hold on
a=0:pi/50:2*pi;R=70;x=R*cos(a)+168;y=R*sin(a)+538;plot(x,y,'g-.');axis equal;hold on
a=0:pi/50:2*pi;R=70;x=R*cos(a)+210;y=R*sin(a)+455;plot(x,y,'g-.');axis equal;hold on
a=0:pi/50:2*pi;R=70;x=R*cos(a)+120;y=R*sin(a)+400;plot(x,y,'g-.');axis equal;hold on
a=0:pi/50:2*pi;R=70;x=R*cos(a)+96;y=R*sin(a)+304;plot(x,y,'g-.');axis equal;hold on
a=0:pi/50:2*pi;R=70;x=R*cos(a)+10;y=R*sin(a)+451;plot(x,y,'g-.');axis equal;hold on
a=0:pi/50:2*pi;R=70;x=R*cos(a)+162;y=R*sin(a)+660;plot(x,y,'g-.');axis
\end{verbatim}

\begin{verbatim}
equal; hold on
a=0:pi/50:2*pi; R=70; x=R*cos(a)+110; y=R*sin(a)+561; plot(x,y,'g-.'); axis equal; hold on
a=0:pi/50:2*pi; R=70; x=R*cos(a)+105; y=R*sin(a)+473; plot(x,y,'g-.'); axis equal; hold on
for i=1:68
    plot(position(i,1), position(i,2), 'r.');
end
plot(fit1(:,1), fit1(:,2));
plot(fit2(:,1), fit2(:,2));
axis([-100 350 200 800])
Route = L1 + L2;
\end{verbatim}

3、问题3中无人机架次最优调度规划Lingo源程序

\begin{verbatim}
sets:
num/1..20/: x1, x2, x3, x4, x5, x6, x7; !前10为D-1型炸弹; 后10为D-2型炸弹
row/1..10/: a, b, c, d, e, f, g, p, q, r;
endsets

data:
a = 422.916185549808 329.012326685659 299.728410398481 234.508251453973
     264.773642254385 275.330609348914 379.019092160569 397.610865042695
     356.200056148227 305.175424960792;
b = 684.700591499672 569.832821487535 507.682302232410 426.440992401059
     388.194828004954 315.892826410759 478.544813877331 623.138026443580
     543.035063324644 458.267432838076;
c = 524.777000258205 428.747141227917 392.514611193010 320.473275016810
     330.482454997576 312.267424564978 444.164915568781 495.619410435064
     445.306950316296 383.123269979781;
d = 626.242205540317 521.606599310672 473.131398239432 394.711945600839
     380.874664691788 333.434068112090 486.860492224301 583.509554334803
     518.886384481227 444.269557813722;
e = 471.655488677912 379.116223750859 348.721550810959 281.144304583963
     303.450530547890 301.149372312885 418.163610989234 447.680243030670
     404.221820291780 349.128114021200;
f = 487.810721489391 378.063633679717 326.561050953723 248.040158038976
     240.944146292565 214.086301393724 352.141740717507 437.865961225579
     372.453325934942 300.353525033418;
g = 607.741474641973 492.442664457272 430.604598210470 349.396794490161
     314.652545659170 249.310270769399 410.183055341013 545.894495301061
     467.136040142484 383.756224705216;
enddata
\end{verbatim}

\begin{verbatim}
@for(row(i):
q(i)=(x1(i)+x1(i+10))*a(i)+(x2(i)+x2(i+10))*b(i)+(x3(i)+x3(i+10))*c(i)+(x4(i)+x4(i
+10))*d(i)+(x5(i)+x5(i+10))*e(i)+(x6(i)+x6(i+10))*f(i)+(x7(i)+x7(i+10))*g(i));
min=q(1)+q(2)+q(3)+q(4)+q(5)+q(6)+q(7)+q(8)+q(9)+q(10);

@for(row(i):
r(i)=x1(i)+x2(i)+x3(i)+x4(i)+x5(i)+x6(i)+x7(i));

@for(row(i):
p(i)=6*(x1(i)+x2(i)+x3(i)+x4(i)+x5(i)+x6(i)+x7(i))/10+6*(x1(i+10)+x2(i+10)+x3(i
+10)+x4(i+10)+x5(i+10)+x6(i+10)+x7(i+10))/2;

11>p(1);p(1)>10;10>p(2);p(2)>9;6>p(3);p(3)>5;11>p(4);p(4)>10;8>p(5);p(5)>7;7>
p(6);p(6)>6;7>p(7);p(7)>6;6>p(8);p(8)>5;6>p(9);p(9)>5;6>p(10);p(10)>5;

@for(row(i):r(i)>2);
@for(row(i):@gin(r(i)));

@for(num(i):@gin(x1(i)));@for(num(i):@gin(x2(i)));@for(num(i):@gin(x3(i)));@f
or(num(i):@gin(x4(i)));@for(num(i):@gin(x5(i)));@for(num(i):@gin(x6(i)));@for(
num(i):@gin(x7(i)));

@sum(num(j):x1(j)) < 13;
@sum(num(j):x2(j)) < 15;
@sum(num(j):x3(j)) < 13;
@sum(num(j):x4(j)) < 15;
@sum(num(j):x5(j)) < 13;
@sum(num(j):x6(j)) < 15;
@sum(num(j):x7(j)) < 13;
end
\end{verbatim}

\end{document}