\begin{center}
\textbf{“华为杯”第十三届全国研究生数学建模竞赛}
\end{center}

\begin{center}
上海理工大学
\end{center}

\begin{tabular}{l l}
学校 & \\
\hline
参赛队号 & 10252152 \\
\hline
队员姓名 & 1. 张浩 \\
 & 2. 贾振 \\
 & 3. 张会 \\
\hline
\end{tabular}

\begin{flushright}
参赛密码 \_\_\_\_\_\_\_\_\_\_\_\_\_\_\_\_\_\_\_\_\_\_\_\_\_\_\_\_\_\_\_\_\_\_\_\_\_\_\_\_\_\_\_\_\_\_\_\_\_\_\_\_\_\_\_\_\_\_\_\_\_\_\_\_\_\_\_\_\_\_\_\_\_\_\_\_\_\_\_\_\_\_\_\_\_\_\_\_\_\_\_\_\_\_\_\_\_\_\_\_\_\_\_\_\_\_\_\_\_\_\_\_\_\_\_\_\_\_\_\_\_\_\_\_\_\_\_\_\_\_\_\_\_\_\_\_\_\_\_\_\_\_\_\_\_\_\_\_\_\_\_\_\_\_\_\_\_\_\_\_\_\_\_\_\_\_\_\_\_\_\_\_\_\_\_\_\_\_\_\_\_\_\_\_\_\_\_\_\_\_\_\_\_\_\_\_\_\_\_\_\_\_\_\_\_\_\_\_\_\_\_\_\_\_\_\_\_\_\_\_\_\_\_\_\_\_\_\_\_\_\_\_\_\_\_\_\_\_\_\_\_\_\_\_\_\_\_\_\_\_\_\_\_\_\_\_\_\_\_\_\_\_\_\_\_\_\_\_\_\_\_\_\_\_\_\_\_\_\_\_\_\_\_\_\_\_\_\_\_\_\_\_\_\_\_\_\_\_\_\_\_\_\_\_\_\_\_\_\_\_\_\_\_\_\_\_\_\_\_\_\_\_\_\_\_\_\_\_\_\_\_\_\_\_\_\_\_\_\_\_\_\_\_\_\_\_\_\_\_\_\_\_\_\_\_\_\_\_\_\_\_\_\_\_\_\_\_\_\_\_\_\_\_\_\_\_\_\_\_\_\_\_\_\_\_\_\_\_\_\_\_\_\_\_\_\_\_\_\_\_\_\_\_\_\_\_\_\_\_\_\_\_\_\_\_\_\_\_\_\_\_\_\_\_\_\_\_\_\_\_\_\_\_\_\_\_\_\_\_\_\_\_\_\_\_\_\_\_\_\_\_\_\_\_\_\_\_\_\_\_\_\_\_\_\_\_\_\_\_\_\_\_\_\_\_\_\_\_\_\_\_\_\_\_\_\_\_\_\_\_\_\_\_\_\_\_\_\_\_\_\_\_\_\_\_\_\_\_\_\_\_\_\_\_\_\_\_\_\_\_\_\_\_\_\_\_\_\_\_\_\_\_\_\_\_\_\_\_\_\_\_\_\_\_\_\_\_\_\_\_\_\_\_\_\_\_\_\_\_\_\_\_\_\_\_\_\_\_\_\_\_\_\_\_\_\_\_\_\_\_\_\_\_\_\_\_\_\_\_\_\_\_\_\_\_\_\_\_\_\_\_\_\_\_\_\_\_\_\_\_\_\_\_\_\_\_\_\_\_\_\_\_\_\_\_\_\_\_\_\_\_\_\_\_\_\_\_\_\_\_\_\_\_\_\_\_\_\_\_\_\_\_\_\_\_\_\_\_\_\_\_\_\_\_\_\_\_\_\_\_\_\_\_\_\_\_\_\_\_\_\_\_\_\_\_\_\_\_\_\_\_\_\_\_\_\_\_\_\_\_\_\_\_\_\_\_\_\_\_\_\_\_\_\_\_\_\_\_\_\_\_\_\_\_\_\_\_\_\_\_\_\_\_\_\_\_\_\_\_\_\_\_\_\_\_\_\_\_\_\_\_\_\_\_\_\_\_\_\_\_\_\_\_\_\_\_\_\_\_\_\_\_\_\_\_\_\_\_\_\_\_\_\_\_\_\_\_\_\_\_\_\_\_\_\_\_\_\_\_\_\_\_\_\_\_\_\_\_\_\_\_\_\_\_\_\_\_\_\_\_\_\_\_\_\_\_\_\_\_\_\_\_\_\_\_\_\_\_\_\_\_\_\_\_\_\_\_\_\_\_\_\_\_\_\_\_\_\_\_\_\_\_\_\_\_\_\_\_\_\_\_\_\_\_\_\_\_\_\_\_\_\_\_\_\_\_\_\_\_\_\_\_\_\_\_\_\_\_\_\_\_\_\_\_\_\_\_\_\_\_\_\_\_\_\_\_\_\_\_\_\_\_\_\_\_\_\_\_\_\_\_\_\_\_\_\_\_\_\_\_\_\_\_\_\_\_\_\_\_\_\_\_\_\_\_\_\_\_\_\_\_\_\_\_\_\_\_\_\_\_\_\_\_\_\_\_\_\_\_\_\_\_\_\_\_\_\_\_\_\_\_\_\_\_\_\_\_\_\_\_\_\_\_\_\_\_\_\_\_\_\_\_\_\_\_\_\_\_\_\_\_\_\_\_\_\_\_\_\_\_\_\_\_\_\_\_\_\_\_\_\_\_\_\_\_\_\_\_\_\_\_\_\_\_\_\_\_\_\_\_\_\_\_\_\_\_\_\_\_\_\_\_\_\_\_\_\_\_\_\_\_\_\_\_\_\_\_\_\_\_\_\_\_\_\_\_\_\_\_\_\_\_\_\_\_\_\_\_\_\_\_\_\_\_\_\_\_\_\_\_\_\_\_\_\_\_\_\_\_\_\_\_\_\_\_\_\_\_\_\_\_\_\_\_\_\_\_\_\_\_\_\_\_\_\_\_\_\_\_\_\_\_\_\_\_\_\_\_\_\_\_\_\_\_\_\_\_\_\_\_\_\_\_\_\_\_\_\_\_\_\_\_\_\_\_\_\_\_\_\_\_\_\_\_\_\_\_\_\_\_\_\_\_\_\_\_\_\_\_\_\_\_\_\_\_\_\_\_\_\_\_\_\_\_\_\_\_\_\_\_\_\_\_\_\_\_\_\_\_\_\_\_\_\_\_\_\_\_\_\_\_\_\_\_\_\_\_\_\_\_\_\_\_\_\_\_\_\_\_\_\_\_\_\_\_\_\_\_\_\_\_\_\_\_\_\_\_\_\_\_\_\_\_\_\_\_\_\_\_\_\_\_\_\_\_\_\_\_\_\_\_\_\_\_\_\_\_\_\_\_\_\_\_\_\_\_\_\_\_\_\_\_\_\_\_\_\_\_\_\_\_\_\_\_\_\_\_\_\_\_\_\_\_\_\_\_\_\_\_\_\_\_\_\_\_\_\_\_\_\_\_\_\_\_\_\_\_\_\_\_\_\_\_\_\_\_\_\_\_\_\_\_\_\_\_\_\_\_\_\_\_\_\_\_\_\_\_\_\_\_\_\_\_\_\_\_\_\_\_\_\_\_\_\_\_\_\_\_\_\_\_\_\_\_\_\_\_\_\_\_\_\_\_\_\_\_\_\_\_\_\_\_\_\_\_\_\_\_\_\_\_\_\_\_\_\_\_\_\_\_\_\_\_\_\_\_\_\_\_\_\_\_\_\_\_\_\_\_\_\_\_\_\_\_\_\_\_\_\_\_\_\_\_\_\_\_\_\_\_\_\_\_\_\_\_\_\_\_\_\_\_\_\_\_\_\_\_\_\_\_\_\_\_\_\_\_\_\_\_\_\_\_\_\_\_\_\_\_\_\_\_\_\_\_\_\_\_\_\_\_\_\_\_\_\_\_\_\_\_\_\_\_\_\_\_\_\_\_\_\_\_\_\_\_\_\_\_\_\_\_\_\_\_\_\_\_\_\_\_\_\_\_\_\_\_\_\_\_\_\_\_\_\_\_\_\_\_\_\_\_\_\_\_\_\_\_\_\_\_\_\_\_\_\_\_\_\_\_\_\_\_\_\_\_\_\_\_\_\_\_\_\_\_\_\_\_\_\_\_\_\_\_\_\_\_\_\_\_\_\_\_\_\_\_\_\_\_\_\_\_\_\_\_\_\_\_\_\_\_\_\_\_\_\_\_\_\_\_\_\_\_\_\_\_\_\_\_\_\_\_\_\_\_\_\_\_\_\_\_\_\_\_\_\_\_\_\_\_\_\_\_\_\_\_\_\_\_\_\_\_\_\_\_\_\_\_\_\_\_\_\_\_\_\_\_\_\_\_\_\_\_\_\_\_\_\_\_\_\_\_\_\_\_\_\_\_\_\_\_\_\_\_\_\_\_\_\_\_\_\_\_\_\_\_\_\_\_\_\_\_\_\_\_\_\_\_\_\_\_\_\_\_\_\_\_\_\_\_\_\_\_\_\_\_\_\_\_\_\_\_\_\_\_\_\_\_\_\_\_\_\_\_\_\_\_\_\_\_\_\_\_\_\_\_\_\_\_\_\_\_\_\_\_\_\_\_\_\_\_\_\_\_\_\_\_\_\_\_\_\_\_\_\_\_\_\_\_\_\_\_\_\_\_\_\_\_\_\_\_\_\_\_\_\_\_\_\_\_\_\_\_\_\_\_\_\_\_\_\_\_\_\_\_\_\_\_\_\_\_\_\_\_\_\_\_\_\_\_\_\_\_\_\_\_\_\_\_\_\_\_\_\_\_\_\_\_\_\_\_\_\_\_\_\_\_\_\_\_\_\_\_\_\_\_\_\_\_\_\_\_\_\_\_\_\_\_\_\_\_\_\_\_\_\_\_\_\_\_\_\_\_\_\_\_\_\_\_\_\_\_\_\_\_\_\_\_\_\_\_\_\_\_\_\_\_\_\_\_\_\_\_\_\_\_\_\_\_\_\_\_\_\_\_\_\_\_\_\_\_\_\_\_\_\_\_\_\_\_\_\_\_\_\_\_\_\_\_\_\_\_\_\_\_\_\_\_\_\_\_\_\_\_\_\_\_\_\_\_\_\_\_\_\_\_\_\_\_\_\_\_\_\_\_\_\_\_\_\_\_\_\_\_\_\_\_\_\_\_\_\_\_\_\_\_\_\_\_\_\_\_\_\_\_\_\_\_\_\_\_\_\_\_\_\_\_\_\_\_\_\_\_\_\_\_\_\_\_\_\_\_\_\_\_\_\_\_\_\_\_\_\_\_\_\_\_\_\_\_\_\_\_\_\_\_\_\_\_\_\_\_\_\_\_\_\_\_\_\_\_\_\_\_\_\_\_\_\_\_\_\_\_\_\_\_\_\_\_\_\_\_\_\_\_\_\_\_\_\_\_\_\_\_\_\_\_\_\_\_\_\_\_\_\_\_\_\_\_\_\_\_\_\_\_\_\_\_\_\_\_\_\_\_\_\_\_\_\_\_\_\_\_\_\_\_\_\_\_\_\_\_\_\_\_\_\_\_\_\_\_\_\_\_\_\_\_\_\_\_\_\_\_\_\_\_\_\_\_\_\_\_\_\_\_\_\_\_\_\_\_\_\_\_\_\_\_\_\_\_\_\_\_\_\_\_\_\_\_\_\_\_\_\_\_\_\_\_\_\_\_\_\_\_\_\_\_\_\_\_\_\_\_\_\_\_\_\_\_\_\_\_\_\_\_\_\_\_\_\_\_\_\_\_\_\_\_\_\_\_\_\_\_\_\_\_\_\_\_\_\_\_\_\_\_\_\_\_\_\_\_\_\_\_\_\_\_\_\_\_\_\_\_\_\_\_\_\_\_\_\_\_\_\_\_\_\_\_\_\_\_\_\_\_\_\_\_\_\_\_\_\_\_\_\_\_\_\_\_\_\_\_\_\_\_\_\_\_\_\_\_\_\_\_\_\_\_\_\_\_\_\_\_\_\_\_\_\_\_\_\_\_\_\_\_\_\_\_\_\_\_\_\_\_\_\_\_\_\_\_\_\_\_\_\_\_\_\_\_\_\_\_\_\_\_\_\_\_\_\_\_\_\_\_\_\_\_\_\_\_\_\_\_\_\_\_\_\_\_\_\_\_\_\_\_\_\_\_\_\_\_\_\_\_\_\_\_\_\_\_\_\_\_\_\_\_\_\_\_\_\_\_\_\_\_\_\_\_\_\_\_\_\_\_\_\_\_\_\_\_\_\_\_\_\_\_\_\_\_\_\_\_\_\_\_\_\_\_\_\_\_\_\_\_\_\_\_\_\_\_\_\_\_\_\_\_\_\_\_\_\_\_\_\_\_\_\_\_\_\_\_\_\_\_\_\_\_\_\_\_\_\_\_\_\_\_\_\_\_\_\_\_\_\_\_\_\_\_\_\_\_\_\_\_\_\_\_\_\_\_\_\_\_\_\_\_\_\_\_\_\_\_\_\_\_\_\_\_\_\_\_\_\_\_\_\_\_\_\_\_\_\_\_\_\_\_\_\_\_\_\_\_\_\_\_\_\_\_\_\_\_\_\_\_\_\_\_\_\_\_\_\_\_\_\_\_\_\_\_\_\_\_\_\_\_\_\_\_\_\_\_\_\_\_\_\_\_\_\_\_\_\_\_\_\_\_\_\_\_\_\_\_\_\_\_\_\_\_\_\_\_\_\_\_\_\_\_\_\_\_\_\_\_\_\_\_\_\_\_\_\_\_\_\_\_\_\_\_\_\_\_\_\_\_\_\_\_\_\_\_\_\_\_\_\_\_\_\_\_\_\_\_\_\_\_\_\_\_\_\_\_\_\_\_\_\_\_\_\_\_\_\_\_\_\_\_\_\_\_\_\_\_\_\_\_\_\_\_\_\_\_\_\_\_\_\_\_\_\_\_\_\_\_\_\_\_\_\_\_\_\_\_\_\_\_\_\_\_\_\_\_\_\_\_\_\_\_\_\_\_\_\_\_\_\_\_\_\_\_\_\_\_\_\_\_\_\_\_\_\_\_\_\_\_\_\_\_\_\_\_\_\_\_\_\_\_\_\_\_\_\_\_\_\_\_\_\_\_\_\_\_\_\_\_\_\_\_\_\_\_\_\_\_\_\_\_\_\_\_\_\_\_\_\_\_\_\_\_\_\_\_\_\_\_\_\_\_\_\_\_\_\_\_\_\_\_\_\_\_\_\_\_\_\_\_\_\_\_\_\_\_\_\_\_\_\_\_\_\_\_\_\_\_\_\_\_\_\_\_\_\_\_\_\_\_\_\_\_\_\_\_\_\_\_\_\_\_\_\_\_\_\_\_\_\_\_\_\_\_\_\_\_\_\_\_\_\_\_\_\_\_\_\_\_\_\_\_\_\_\_\_\_\_\_\_\_\_\_\_\_\_\_\_\_\_\_\_\_\_\_\_\_\_\_\_\_\_\_\_\_\_\_\_\_\_\_\_\_\_\_\_\_\_\_\_\_\_\_\_\_\_\_\_\_\_\_\_\_\_\_\_\_\_\_\_\_\_\_\_\_\_\_\_\_\_\_\_\_\_\_\_\_\_\_\_\_\_\_\_\_\_\_\_\_\_\_\_\_\_\_\_\_\_\_\_\_\_\_\_\_\_\_\_\_\_\_\_\_\_\_\_\_\_\_\_\_\_\_\_\_\_\_\_\_\_\_\_\_\_\_\_\_\_\_\_\_\_\_\_\_\_\_\_\_\_\_\_\_\_\_\_\_\_\_\_\_\_\_\_\_\_\_\_\_\_\_\_\_\_\_\_\_\_\_\_\_\_\_\_\_\_\_\_\_\_\_\_\_\_\_\_\_\_\_\_\_\_\_\_\_\_\_\_\_\_\_\_\_\_\_\_\_\_\_\_\_\_\_\_\_\_\_\_\_\_\_\_\_\_\_\_\_\_\_\_\_\_\_\_\_\_\_\_\_\_\_\_\_\_\_\_\_\_\_\_\_\_\_\_\_\_\_\_\_\_\_\_\_\_\_\_\_\_\_\_\_\_\_\_\_\_\_\_\_\_\_\_\_\_\_\_\_\_\_\_\_\_\_\_\_\_\_\_\_\_\_\_\_\_\_\_\_\_\_\_\_\_\_\_\_\_\_\_\_\_\_\_\_\_\_\_\_\_\_\_\_\_\_\_\_\_\_\_\_\_\_\_\_\_\_\_\_\_\_\_\_\_\_\_\_\_\_\_\_\_\_\_\_\_\_\_\_\_\_\_\_\_\_\_\_\_\_\_\_\_\_\_\_\_\_\_\_\_\_\_\_\_\_\_\_\_\_\_\_\_\_\_\_\_\_\_\_\_\_\_\_\_\_\_\_\_\_\_\_\_\_\_\_\_\_\_\_\_\_\_\_\_\_\_\_\_\_\_\_\_\_\_\_\_\_\_\_\_\_\_\_\_\_\_\_\_\_\_\_\_\_\_\_\_\_\_\_\_\_\_\_\_\_\_\_\_\_\_\_\_\_\_\_\_\_\_\_\_\_\_\_\_\_\_\_\_\_\_\_\_\_\_\_\_\_\_\_\_\_\_\_\_\_\_\_\_\_\_\_\_\_\_\_\_\_\_\_\_\_\_\_\_\_\_\_\_\_\_\_\_\_\_\_\_\_\_\_\_\_\_\_\_\_\_\_\_\_\_\_\_\_\_\_\_\_\_\_\_\_\_\_\_\_\_\_\_\_\_\_\_\_\_\_\_\_\_\_\_\_\_\_\_\_\_\_\_\_\_\_\_\_\_\_\_\_\_\_\_\_\_\_\_\_\_\_\_\_\_\_\_\_\_\_\_\_\_\_\_\_\_\_\_\_\_\_\_\_\_\_\_\_\_\_\_\_\_\_\_\_\_\_\_\_\_\_\_\_\_\_\_\_\_\_\_\_\_\_\_\_\_\_\_\_\_\_\_\_\_\_\_\_\_\_\_\_\_\_\_\_\_\_\_\_\_\_\_\_\_\_\_\_\_\_\_\_\_\_\_\_\_\_\_\_\_\_\_\_\_\_\_\_\_\_\_\_\_\_\_\_\_\_\_\_\_\_\_\_\_\_\_\_\_\_\_\_\_\_\_\_\_\_\_\_\_\_\_\_\_\_\_\_\_\_\_\_\_\_\_\_\_\_\_\_\_\_\_\_\_\_\_\_\_\_\_\_\_\_\_\_\_\_\_\_\_\_\_\_\_\_\_\_\_\_\_\_\_\_\_\_\_\_\_\_\_\_\_\_\_\_\_\_\_\_\_\_\_\_\_\_\_\_\_\_\_\_\_\_\_\_\_\_\_\_\_\_\_\_\_\_\_\_\_\_\_\_\_\_\_\_\_\_\_\_\_\_\_\_\_\_\_\_\_\_\_\_\_\_\_\_\_\_\_\_\_\_\_\_\_\_\_\_\_\_\_\_\_\_\_\_\_\_\_\_\_\_\_\_\_\_\_\_\_\_\_\_\_\_\_\_\_\_\_\_\_\_\_\_\_\_\_\_\_\_\_\_\_\_\_\_\_\_\_\_\_\_\_\_\_\_\_\_\_\_\_\_\_\_\_\_\_\_\_\_\_\_\_\_\_\_\_\_\_\_\_\_\_\_\_\_\_\_\_\_\_\_\_\_\_\_\_\_\_\_\_\_\_\_\_\_\_\_\_\_\_\_\_\_\_\_\_\_\_\_\_\_\_\_\_\_\_\_\_\_\_\_\_\_\_\_\_\_\_\_\_\_\_\_\_\_\_\_\_\_\_\_\_\_\_\_\_\_\_\_\_\_\_\_\_\_\_\_\_\_\_\_\_\_\_\_\_\_\_\_\_\_\_\_\_\_\_\_\_\_\_\_\_\_\_\_\_\_\_\_\_\_\_\_\_\_\_\_\_\_\_\_\_\_\_\_\_\_\_\_\_\_\_\_\_\_\_\_\_\_\_\_\_\_\_\_\_\_\_\_\_\_\_\_\_\_\_\_\_\_\_\_\_\_\_\_\_\_\_\_\_\_\_\_\_\_\_\_\_\_\_\_\_\_\_\_\_\_\_\_\_\_\_\_\_\_\_\_\_\_\_\_\_\_\_\_\_\_\_\_\_\_\_\_\_\_\_\_\_\_\_\_\_\_\_\_\_\_\_\_\_\_\_\_\_\_\_\_\_\_\_\_\_\_\_\_\_\_\_\_\_\_\_\_\_\_\_\_\_\_\_\_\_\_\_\_\_\_\_\_\_\_\_\_\_\_\_\_\_\_\_\_\_\_\_\_\_\_\_\_\_\_\_\_\_\_\_\_\_\_\_\_\_\_\_\_\_\_\_\_\_\_\_\_\_\_\_\_\_\_\_\_\_\_\_\_\_\_\_\_\_\_\_\_\_\_\_\_\_\_\_\_\_\_\_\_\_\_\_\_\_\_\_\_\_\_\_\_\_\_\_\_\_\_\_\_\_\_\_\_\_\_\_\_\_\_\_\_\_\_\_\_\_\_\_\_\_\_\_\_\_\_\_\_\_\_\_\_\_\_\_\_\_\_\_\_\_\_\_\_\_\_\_\_\_\_\_\_\_\_\_\_\_\_\_\_\_\_\_\_\_\_\_\_\_\_\_\_\_\_\_\_\_\_\_\_\_\_\_\_\_\_\_\_\_\_\_\_\_\_\_\_\_\_\_\_\_\_\_\_\_\_\_\_\_\_\_\_\_\_\_\_\_\_\_\_\_\_\_\_\_\_\_\_\_\_\_\_\_\_\_\_\_\_\_\_\_\_\_\_\_\_\_\_\_\_\_\_\_\_\_\_\_\_\_\_\_\_\_\_\_\_\_\_\_\_\_\_\_\_\_\_\_\_\_\_\_\_\_\_\_\_\_\_\_\_\_\_\_\_\_\_\_\_\_\_\_\_\_\_\_\_\_\_\_\_\_\_\_\_\_\_\_\_\_\_\_\_\_\_\_\_\_\_\_\_\_\_\_\_\_\_\_\_\_\_\_\_\_\_\_\_\_\_\_\_\_\_\_\_\_\_\_\_\_\_\_\_\_\_\_\_\_\_\_\_\_\_\_\_\_\_\_\_\_\_\_\_\_\_\_\_\_\_\_\_\_\_\_\_\_\_\_\_\_\_\_\_\_\_\_\_\_\_\_\_\_\_\_\_\_\_\_\_\_\_\_\_\_\_\_\_\_\_\_\_\_\_\_\_\_\_\_\_\_\_\_\_\_\_\_\_\_\_\_\_\_\_\_\_\_\_\_\_\_\_\_\_\_\_\_\_\_\_\_\_\_\_\_\_\_\_\_\_\_\_\_\_\_\_\_\_\_\_\_\_\_\_\_\_\_\_\_\_\_\_\_\_\_\_\_\_\_\_\_\_\_\_\_\_\_\_\_\_\_\_\_\_\_\_\_\_\_\_\_\_\_\_\_\_\_\_\_\_\_\_\_\_\_\_\_\_\_\_\_\_\_\_\_\_\_\_\_\_\_\_\_\_\_\_\_\_\_\_\_\_\_\_\_\_\_\_\_\_\_\_\_\_\_\_\_\_\_\_\_\_\_\_\_\_\_\_\_\_\_\_\_\_\_\_\_\_\_\_\_\_\_\_\_\_\_\_\_\_\_\_\_\_\_\_\_\_\_\_\_\_\_\_\_\_\_\_\_\_\_\_\_\_\_\_\_\_\_\_\_\_\_\_\_\_\_\_\_\_\_\_\_\_\_\_\_\_\_\_\_\_\_\_\_\_\_\_\_\_\_\_\_\_\_\_\_\_\_\_\_\_\_\_\_\_\_\_\_\_\_\_\_\_\_\_\_\_\_\_\_\_\_\_\_\_\_\_\_\_\_\_\_\_\_\_\_\_\_\_\_\_\_\_\_\_\_\_\_\_\_\_\_\_\_\_\_\_\_\_\_\_\_\_\_\_\_\_\_\_\_\_\_\_\_\_\_\_\_\_\_\_\_\_\_\_\_\_\_\_\_\_\_\_\_\_\_\_\_\_\_\_\_\_\_\_\_\_\_\_\_\_\_\_\_\_\_\_\_\_\_\_\_\_\_\_\_\_\_\_\_\_\_\_\_\_\_\_\_\_\_\_\_\_\_\_\_\_\_\_\_\_\_\_\_\_\_\_\_\_\_\_\_\_\_\_\_\_\_\_\_\_\_\_\_\_\_\_\_\_\_\_\_\_\_\_\_\_\_\_\_\_\_\_\_\_\_\_\_\_\_\_\_\_\_\_\_\_\_\_\_\_\_\_\_\_\_\_\_\_\_\_\_\_\_\_\_\_\_\_\_\_\_\_\_\_\_\_\_\_\_\_\_\_\_\_\_\_\_\_\_\_\_\_\_\_\_\_\_\_\_\_\_\_\_\_\_\_\_\_\_\_\_\_\_\_\_\_\_\_\_\_\_\_\_\_\_\_\_\_\_\_\_\_\_\_\_\_\_\_\_\_\_\_\_\_\_\_\_\_\_\_\_\_\_\_\_\_\_\_\_\_\_\_\_\_\_\_\_\_\_\_\_\_\_\_\_\_\_\_\_\_\_\_\_\_\_\_\_\_\_\_\_\_\_\_\_\_\_\_\_\_\_\_\_\_\_\_\_\_\_\_\_\_\_\_\_\_\_\_\_\_\_\_\_\_\_\_\_\_\_\_\_\_\_\_\_\_\_\_\_\_\_\_\_\_\_\_\_\_\_\_\_\_\_\_\_\_\_\_\_\_\_\_\_\_\_\_\_\_\_\_\_\_\_\_\_\_\_\_\_\_\_\_\_\_\_\_\_\_\_\_\_\_\_\_\_\_\_\_\_\_\_\_\_\_\_\_\_\_\_\_\_\_\_\_\_\_\_\_\_\_\_\_\_\_\_\_\_\_\_\_\_\_\_\_\_\_\_\_\_\_\_\_\_\_\_\_\_\_\_\_\_\_\_\_\_\_\_\_\_\_\_\_\_\_\_\_\_\_\_\_\_\_\_\_\_\_\_\_\_\_\_\_\_\_\_\_\_\_\_\_\_\_\_\_\_\_\_\_\_\_\_\_\_\_\_\_\_\_\_\_\_\_\_\_\_\_\_\_\_\_\_\_\_\_\_\_\_\_\_\_\_\_\_\_\_\_\_\_\_\_\_\_\_\_\_\_\_\_\_\_\_\_\_\_\_\_\_\_\_\_\_\_\_\_\_\_\_\_\_\_\_\_\_\_\_\_\_\_\_\_\_\_\_\_\_\_\_\_\_\_\_\_\_\_\_\_\_\_\_\_\_\_\_\_\_\_\_\_\_\_\_\_\_\_\_\_\_\_\_\_\_\_\_\_\_\_\_\_\_\_\_\_\_\_\_\_\_\_\_\_\_\_\_\_\_\_\_\_\_\_\_\_\_\_\_\_\_\_\_\_\_\_\_\_\_\_\_\_\_\_\_\_\_\_\_\_\_\_\_\_\_\_\_\_\_\_\_\_\_\_\_\_\_\_\_\_\_\_\_\_\_\_\_\_\_\_\_\_\_\_\_\_\_\_\_\_\_\_\_\_\_\_\_\_\_\_\_\_\_\_\_\_\_\_\_\_\_\_\_\_\_\_\_\_\_\_\_\_\_\_\_\_\_\_\_\_\_\_\_\_\_\_\_\_\_\_\_\_\_\_\_\_\_\_\_\_\_\_\_\_\_\_\_\_\_\_\_\_\_\_\_\_\_\_\_\_\_\_\_\_\_\_\_\_\_\_\_\_\_\_\_\_\_\_\_\_\_\_\_\_\_\_\_\_\_\_\_\_\_\_\_\_\_\_\_\_\_\_\_\_\_\_\_\_\_\_\_\_\_\_\_\_\_\_\_\_\_\_\_\_\_\_\_\_\_\_\_\_\_\_\_\_\_\_\_\_\_\_\_\_\_\_\_\_\_\_\_\_\_\_\_\_\_\_\_\_\_\_\_\_\_\_\_\_\_\_\_\_\_\_\_\_\_\_\_\_\_\_\_\_\_\_\_\_\_\_\_\_\_\_\_\_\_\_\_\_\_\_\_\_\_\_\_\_\_\_\_\_\_\_\_\_\_\_\_\_\_\_\_\_\_\_\_\_\_\_\_\_\_\_\_\_\_\_\_\_\_\_\_\_\_\_\_\_\_\_\_\_\_\_\_\_\_\_\_\_\_\_\_\_\_\_\_\_\_\_\_\_\_\_\_\_\_\_\_\_\_\_\_\_\_\_\_\_\_\_\_\_\_\_\_\_\_\_\_\_\_\_\_\_\_\_\_\_\_\_\_\_\_\_\_\_\_\_\_\_\_\_\_\_\_\_\_\_\_\_\_\_\_\_\_\_\_\_\_\_\_\_\_\_\_\_\_\_\_\_\_\_\_\_\_\_\_\_\_\_\_\_\_\_\_\_\_\_\_\_\_\_\_\_\_\_\_\_\_\_\_\_\_\_\_\_\_\_\_\_\_\_\_\_\_\_\_\_\_\_\_\_\_\_\_\_\_\_\_\_\_\_\_\_\_\_\_\_\_\_\_\_\_\_\_\_\_\_\_\_\_\_\_\_\_\_\_\_\_\_\_\_\_\_\_\_\_\_\_\_\_\_\_\_\_\_\_\_\_\_\_\_\_\_\_\_\_\_\_\_\_\_\_\_\_\_\_\_\_\_\_\_\_\_\_\_\_\_\_\_\_\_\_\_\_\_\_\_\_\_\_\_\_\_\_\_\_\_\_\_\_\_\_\_\_\_\_\_\_\_\_\_\_\_\_\_\_\_\_\_\_\_\_\_\_\_\_\_\_\_\_\_\_\_\_\_\_\_\_\_\_\_\_\_\_\_\_\_\_\_\_\_\_\_\_\_\_\_\_\_\_\_\_\_\_\_\_\_\_\_\_\_\_\_\_\_\_\_\_\_\_\_\_\_\_\_\_\_\_\_\_\_\_\_\_\_\_\_\_\_\_\_\_\_\_\_\_\_\_\_\_\_\_\_\_\_\_\_\_\_\_\_\_\_\_\_\_\_\_\_\_\_\_\_\_\_\_\_\_\_\_\_\_\_\_\_\_\_\_\_\_\_\_\_\_\_\_\_\_\_\_\_\_\_\_\_\_\_\_\_\_\_\_\_\_\_\_\_\_\_\_\_\_\_\_\_\_\_\_\_\_\_\_\_\_\_\_\_\_\_\_\_\_\_\_\_\_\_\_\_\_\_\_\_\_\_\_\_\_\_\_\_\_\_\_\_\_\_\_\_\_\_\_\_\_\_\_\_\_\_\_\_\_\_\_\_\_\_\_\_\_\_\_\_\_\_\_\_\_\_\_\_\_\_\_\_\_\_\_\_\_\_\_\_\_\_\_\_\_\_\_\_\_\_\_\_\_\_\_\_\_\_\_\_\_\_\_\_\_\_\_\_\_\_\_\_\_\_\_\_\_\_\_\_\_\_\_\_\_\_\_\_\_\_\_\_\_\_\_\_\_\_\_\_\_\_\_\_\_\_\_\_\_\_\_\_\_\_\_\_\_\_\_\_\_\_\_\_\_\_\_\_\_\_\_\_\_\_\_\_\_\_\_\_\_\_\_\_\_\_\_\_\_\_\_\_\_\_\_\_\_\_\_\_\_\_\_\_\_\_\_\_\_\_\_\_\_\_\_\_\_\_\_\_\_\_\_\_\_\_\_\_\_\_\_\_\_\_\_\_\_\_\_\_\_\_\_\_\_\_\_\_\_\_\_\_\_\_\_\_\_\_\_\_\_\_\_\_\_\_\_\_\_\_\_\_\_\_\_\_\_\_\_\_\_\_\_\_\_\_\_\_\_\_\_\_\_\_\_\_\_\_\_\_\_\_\_\_\_\_\_\_\_\_\_\_\_\_\_\_\_\_\_\_\_\_\_\_\_\_\_\_\_\_\_\_\_\_\_\_\_\_\_\_\_\_\_\_\_\_\_\_\_\_\_\_\_\_\_\_\_\_\_\_\_\_\_\_\_\_\_\_\_\_\_\_\_\_\_\_\_\_\_\_\_\_\_\_\_\_\_\_\_\_\_\_\_\_\_\_\_\_\_\_\_\_\_\_\_\_\_\_\_\_\_\_\_\_\_\_\_\_\_\_\_\_\_\_\_\_\_\_\_\_\_\_\_\_\_\_\_\_\_\_\_\_\_\_\_\_\_\_\_\_\_\_\_\_\_\_\_\_\_\_\_\_\_\_\_\_\_\_\_\_\_\_\_\_\_\_\_\_\_\_\_\_\_\_\_\_\_\_\_\_\_\_\_\_\_\_\_\_\_\_\_\_\_\_\_\_\_\_\_\_\_\_\_\_\_\_\_\_\_\_\_\_\_\_\_\_\_\_\_\_\_\_\_\_\_\_\_\_\_\_\_\_\_\_\_\_\_\_\_\_\_\_\_\_\_\_\_\_\_\_\_\_\_\_\_\_\_\_\_\_\_\_\_\_\_\_\_\_\_\_\_\_\_\_\_\_\_\_\_\_\_\_\_\_\_\_\_\_\_\_\_\_\_\_\_\_\_\_\_\_\_\_\_\_\_\_\_\_\_\_\_\_\_\_\_\_\_\_\_\_\_\_\_\_\_\_\_\_\_\_\_\_\_\_\_\_\_\_\_\_\_\_\_\_\_\_\_\_\_\_\_\_\_\_\_\_\_\_\_\_\_\_\_\_\_\_\_\_\_\_\_\_\_\_\_\_\_\_\_\_\_\_\_\_\_\_\_\_\_\_\_\_\_\_\_\_\_\_\_\_\_\_\_\_\_\_\_\_\_\_\_\_\_\_\_\_\_\_\_\_\_\_\_\_\_\_\_\_\_\_\_\_\_\_\_\_\_\_\_\_\_\_\_\_\_\_\_\_\_\_\_\_\_\_\_\_\_\_\_\_\_\_\_\_\_\_\_\_\_\_\_\_\_\_\_\_\_\_\_\_\_\_\_\_\_\_\_\_\_\_\_\_\_\_\_\_\_\_\_\_\_\_\_\_\_\_\_\_\_\_\_\_\_\_\_\_\_\_\_\_\_\_\_\_\_\_\_\_\_\_\_\_\_\_\_\_\_\_\_\_\_\_\_\_\_\_\_\_\_\_\_\_\_\_\_\_\_\_\_\_\_\_\_\_\_\_\_\_\_\_\_\_\_\_\_\_\_\_\_\_\_\_\_\_\_\_\_\_\_\_\_\_\_\_\_\_\_\_\_\_\_\_\_\_\_\_\_\_\_\_\_\_\_\_\_\_\_\_\_\_\_\_\_\_\_\_\_\_\_\_\_\_\_\_\_\_\_\_\_\_\_\_\_\_\_\_\_\_\_\_\_\_\_\_\_\_\_\_\_\_\_\_\_\_\_\_\_\_\_\_\_\_\_\_\_\_\_\_\_\_\_\_\_\_\_\_\_\_\_\_\_\_\_\_\_\_\_\_\_\_\_\_\_\_\_\_\_\_\_\_\_\_\_\_\_\_\_\_\_\_\_\_\_\_\_\_\_\_\_\_\_\_\_\_\_\_\_\_\_\_\_\_\_\_\_\_\_\_\_\_\_\_\_\_\_\_\_\_\_\_\_\_\_\_\_\_\_\_\_\_\_\_\_\_\_\_\_\_\_\_\_\_\_\_\_\_\_\_\_\_\_\_\_\_\_\_\_\_\_\_\_\_\_\_\_\_\_\_\_\_\_\_\_\_\_\_\_\_\_\_\_\_\_\_\_\_\_\_\_\_\_\_\_\_\_\_\_\_\_\_\_\_\_\_\_\_\_\_\_\_\_\_\_\_\_\_\_\_\_\_\_\_\_\_\_\_\_\_\_\_\_\_\_\_\_\_\_\_\_\_\_\_\_\_\_\_\_\_\_\_\_\_\_\_\_\_\_\_\_\_\_\_\_\_\_\_\_\_\_\_\_\_\_\_\_\_\_\_\_\_\_\_\_\_\_\_\_\_\_\_\_\_\_\_\_\_\_\_\_\_\_\_\_\_\_\_\_\_\_\_\_\_\_\_\_\_\_\_\_\_\_\_\_\_\_\_\_\_\_\_\_\_\_\_\_\_\_\_\_\_\_\_\_\_\_\_\_\_\_\_\_\_\_\_\_\_\_\_\_\_\_\_\_\_\_\_\_\_\_\_\_\_\_\_\_\_\_\_\_\_\_\_\_\_\_\_\_\_\_\_\_\_\_\_\_\_\_\_\_\_\_\_\_\_\_\_\_\_\_\_\_\_\_\_\_\_\_\_\_\_\_\_\_\_\_\_\_\_\_\_\_\_\_\_\_\_\_\_\_\_\_\_\_\_\_\_\_\_\_\_\_\_\_\_\_\_\_\_\_\_\_\_\_\_\_\_\_\_\_\_\_\_\_\_\_\_\_\_\_\_\_\_\_\_\_\_\_\_\_\_\_\_\_\_\_\_\_\_\_\_\_\_\_\_\_\_\_\_\_\_\_\_\_\_\_\_\_\_\_\_\_\_\_\_\_\_\_\_\_\_\_\_\_\_\_\_\_\_\_\_\_\_\_\_\_\_\_\_\_\_\

\begin{center}
\textbf{参赛密码} \underline{\hspace{5cm}} \\
(由组委会填写)
\end{center}

\begin{center}
\includegraphics[width=0.3\textwidth]{image1.png} \quad
\includegraphics[width=0.3\textwidth]{image2.png} \quad
\includegraphics[width=0.3\textwidth]{image3.png}
\end{center}

\begin{center}
\textbf{“华为杯”第十三届全国研究生数学建模竞赛}
\end{center}

\begin{center}
\textbf{题目} \quad \textbf{多无人机协同任务规划}
\end{center}

\begin{center}
\textbf{摘 \qquad 要:}
\end{center}

随着科学技术的发展,无人机已逐步成为不可或缺的主战装备,执行的任务由传统的侦察、预警、监视等战斗支援,已经开始向侦察、干扰、打击、评估一体化作战方向发展。本文主要研究了多无人机协同任务路线规划和无人机调度的任务规划问题。首先运用模糊聚类分析对初始目标群进行分类,运用重心法模型对目标群精细划分,建立基于多目标侦察路径选择的多目标优化模型,得到无人机的最佳侦察路线和最优调度方案,在此基础上增加 FY-2 型无人机通信范围约束条件,求解 FY-2 最优航迹和最少架次。然后建立目标分配模型,为无人机完成轰炸任务制定具体的最优任务分配方案,保证在完成轰炸任务的同时使攻击方的无人机滞留防御方雷达有效探测范围内的时间总和最小。最后对模型中相关参数进行优化说明,以增强任务时效性,提高无人机作战能力。

对于问题一,确保侦察无人机滞留防御方雷达有效探测范围内的时间总和最小,为 FY-1 型无人机完成 10 个目标群(共 68 个目标)的侦察任务拟制最佳的路线和无人机调度策略。首先运用模糊聚类分析对 10 个目标群进行分区,基于重心法对 10 个目标群在分区的基础上精细划分为 14 个;然后建立基于多目标路径选择的多目标优化模型,运用 matlab 编程求解,得到侦察机在敌方雷达有效探测范围的最小时间为 11.47 小时,最佳路线和最优调度图见图 4-9、图 4-10。

\begin{table}
\centering
\begin{tabular}{|c|c|c|c|c|c|}
\hline
基地 & 侦察目标路线 & \multicolumn{2}{c|}{航行里程数(km)} & \multicolumn{2}{c|}{雷达区范围航行时间(h)} \\ \cline{3-6} 
 & FY-1-S1,FY-1-S2 & FY-1-S1 & FY-1-S2 & FY-1-S1 & FY-1-S2 \\ \hline
P01 & 5-8-6-7-14-13 & 443.123 & 442.353 & 2.216 & 2.212 \\ \hline
P05 & 1-2-3-4-12 & 354.249 & 353.963 & 1.771 & 1.77 \\ \hline
P07 & 10-9-11 & 350.275 & 350.0196 & 1.751 & 1.75 \\ \hline
合计 & & 1147.647 & 1146.336 & 5.738 & 5.732 \\ \hline
总计 & & 2293.983 & & 11.47 & \\ \hline
\end{tabular}
\end{table}

对于问题二,在完成问题一的侦察任务上,分析FY-2型通信中继无人机完成通信任务至少需要的架次数。首先在问题一得出的最佳航线和最优调度的基础上,增加对FY-2型通信范围约束,运用matlab编程求解,得到完成问题一的任务至少需要3架FY-2型通信中继无人机,在防御方雷达区检测范围的航行时间总和为6.876小时。最佳航迹和最优调度如图5-1、图5-2所示。

对于问题三,要求在7个小时内完成对10个目标群的火力打击,并保证攻击方的无人机滞留防御方雷达有效探测范围内的时间总和最小。基于问题一中模糊聚类分析所得3个目标群分区,针对多目标协同攻击规划任务运用竞标算法,建立多无人机多目标协同攻击任务分配模型,求解得到共需要派出91架FY-3、8架FY-1型无人机完成任务,从第一架飞机进入雷达区到轰炸完最后一个目标,共用1.673小时,无人机在防御方雷达监测范围的时间总和为6.924小时。具体飞行路线、FY-3型无人机携带炸弹清单、攻击目标清单如表6-2、表6-3、表6-4、图6-2、图6-3所示。

\begin{table}
\centering
\begin{tabular}{|c|c|c|}
\hline
分区 & 携带D-1型炸弹FY-3数量 & 携带D-2型炸弹FY-3数量 \\ \hline
一区 & 29架 & 3架 \\ \hline
二区 & 32架 & 2架 \\ \hline
三区 & 22架 & 3架 \\ \hline
\end{tabular}
\end{table}

对于问题四,A02、A05、A09周边可能还配置有三部远程搜索雷达,相继开机工作,分析确定3个远程搜索雷达相继开机工作的6种情况,在对问题三无人机任务规划的基础上,以安排距离远程搜索雷达最近的无人机将其摧毁的思路,分别对三部远程搜索雷达6种开机情况下未被炸毁的雷达目标点进行无人机任务的规划。具体任务规划如表7-1到表7-6所示。

对于问题五,首先基于上述问题中所构建的模型和模型运算的结果,对目标遗传算法、目标分配模型优点分析和评价。然后提出提高算法的效率的相应措施。最后改变和调整对无人机作战能力影响较大的参数,提出对无人机的转弯角参数优化,该参数的优化将有效提高无人机作战的能力。

关键字:模糊聚类分析 遗传算法 多目标路径选择 竞标算法 目标分配

\section*{目录}

\section{问题重述}
\subsection{问题背景}
\subsection{需要解决的问题}

\section{模型假设}

\section{符号说明}

\section{问题一:模型的建立和求解}
\subsection{问题一的分析}
\subsection{数据处理}
\subsubsection{基础数据呈现}
\subsubsection{成像传感器的工作原理及性能参数}
\subsubsection{数据整理和分类}
\subsection{模型的建立}
\subsubsection{模型的优化目标及流程}
\subsubsection{确定目标函数}
\subsubsection{确定约束条件}
\subsection{模型求解}
\subsubsection{算法设计流程}
\subsubsection{算子的设计}
\subsubsection{遗传操作}
\subsubsection{模型求解结果}

\section{问题二:模型的建立和求解}
\subsection{问题二的分析}
\subsection{模型分析}
\subsubsection{确定约束条件}
\subsubsection{建立模型}
\subsection{模型求解}

\section{问题三:模型的建立和求解}
\subsection{问题三的分析}
\subsection{目标分配模型建立}

\begin{itemize}
    \item 6.2.1 确定目标函数 \dotfill 26
    \item 6.2.2 确定约束条件 \dotfill 26
    \item 6.2.3 竞标算法 \dotfill 27
    \item 6.3 模型的求解 \dotfill 28
    \item 七 问题四:模型的建立和求解 \dotfill 33
    \item 7.1 问题四的分析 \dotfill 33
    \item 7.2 部署远程搜索雷达的目的分析 \dotfill 33
    \item 7.3 三部远程搜索雷达相继开机情况分析 \dotfill 33
    \item 7.4 求解 \dotfill 34
    \item 八 问题五:模型的改进和评价 \dotfill 37
    \item 8.1 问题五的分析 \dotfill 37
    \item 8.2 模型的算法和复杂度分析 \dotfill 37
    \item 8.3 算法效率的提高 \dotfill 38
    \item 8.4 影响作战能力参数分析 \dotfill 38
    \item 参考文献 \dotfill 39
    \item 附录:遗传算法求最优路径 \dotfill 40
\end{itemize}

\section{问题重述}

\subsection{问题背景}

无人机(Unmanned Aerial Vehicle, UAV)是一种具备自主飞行和独立执行任务能力的新型作战平台,不仅能够执行军事侦察、监视、搜索、目标指向等非攻击性任务,而且还能够执行对地攻击和目标轰炸等作战任务。随着无人机技术的快速发展,越来越多的无人机将应用在未来战场。

某无人机作战部队现配属有 P01~P07 等 7 个无人机基地,各基地均配备一定数量的 FY 系列无人机。其中 FY-1 型无人机主要担任目标侦察和目标指示,FY-2 型无人机主要担任通信中继,FY-3 型无人机用于对地攻击。FY-1 型无人机的巡航飞行速度为 $200\,\text{km/h}$,最长巡航时间为 $10\,\text{h}$,巡航飞行高度为 $1500\,\text{m}$;FY-2 型、FY-3 型无人机的巡航飞行速度为 $300\,\text{km/h}$,最长巡航时间为 $8\,\text{h}$,巡航飞行高度为 $5000\,\text{m}$。受燃料限制,无人机在飞行过程中尽可能减少转弯、爬升、俯冲等机动动作,一般来说,机动时消耗的燃料是巡航的 $2\sim4$ 倍,最小转弯半径 $70\,\text{m}$。

FY-1 型无人机可加载 S-1、S-2、S-3 三种载荷。其中载荷 S-1 系成像传感器,采用广域搜索模式对目标进行成像,传感器的成像带宽为 $2\,\text{km}$;载荷 S-2 系光学传感器,为达到一定的目标识别精度,对地面目标拍照时要求距目标的距离不超过 $7.5\,\text{km}$,可瞬时完成拍照任务;载荷 S-3 系目标指示器,为制导炸弹提供目标指示时要求距被攻击目标的距离不超过 $15\,\text{km}$。由于各种技术条件的限制,该系列无人机每次只能加载 S-1、S-2、S-3 三种载荷中的一种。为保证侦察效果,对每一个目标需安排 S-1、S-2 两种不同载荷各自至少侦察一次,两种不同载荷对同一目标的侦察间隔时间不超过 4 小时。

为保证执行侦察任务的无人机与地面控制中心的联系,需安排专门的 FY-2 型无人机担任通信中继任务,通信中继无人机与执行侦察任务的无人机的通信距离限定在 $50\,\text{km}$ 范围内。通信中继无人机正常工作状态下可随时保持与地面控制中心的通信。

FY-3 型无人机可携带 6 枚 D-1 或 D-2 两种型号的炸弹。其中 D-1 炸弹系某种类型的“灵巧”炸弹,采用抛投方式对地攻击,即投放后炸弹以飞机投弹时的速度作抛物运动,当炸弹接近目标后,可主动寻的攻击待打击的目标,因此炸弹落点位于目标中心 $100\,\text{m}$ 范围内可视为有效击中目标。D-2 型炸弹在激光制导模式下对地面目标进行攻击,其飞行速度为 $200\,\text{m/s}$,飞行方向总是指向目标。攻击同一目标的 D-2 型炸弹在整个飞行过程中需一架 FY-1 型无人机加载载荷 S-3 进行全程引导,直到命中目标。由于某些技术上的限制,携带 D-2 型炸弹的无人机在投掷炸弹时要求距目标 $10\,\text{km}\sim30\,\text{km}$,并且要求各制导炸弹的发射点到目标点连线的大地投影不交叉(以保证弹道不交叉)。为达到一定的毁伤效果,对每个目标(包括雷达站和远程搜索雷达)需成功投掷 10 枚 D-1 型炸弹,而对同一目标投掷 2 枚 D-2 型炸弹即可达到相同的毁伤效果。

多架该型无人机在同时执行任务时可按照一定的编队飞行,但空中飞行时两机相距要求 $200\,\text{m}$ 以上。由于基地后勤技术保障的限制,同一基地的两架无人机起飞时间间隔和降落回收的时间间隔要求在 3 分钟以上。无人机执行完任务后需

返回原基地。

根据任务要求,需完成侦察和打击的目标有 A01~A10 等 10 个目标群,每个目标群包含数量不等的地面目标,每个目标群均配属有雷达站,各目标群配属雷达对 FY 型无人机的有效探测距离为 70km。

\section*{1.2 需要解决的问题}

问题一:需为 FY-1 型无人机完成 10 个目标群(共 68 个目标)的侦察任务拟制最佳的路线和无人机调度策略(包括每架无人机起飞基地、加载的载荷、起飞时间、航迹和侦察的目标),以保证侦察无人机滞留防御方雷达有效探测范围内的时间总和最小。

问题二:FY-1 型无人机对目标进行侦察时,须将侦察信息实时通过 FY-2 型无人机传回地面控制中心。FY-2 型无人机可同时与多架在其有效通信范围的侦察无人机通信并转发信息。为保证侦察无人机滞留防御方雷达有效探测范围内的时间总和最小,并完成 10 个目标群(共 68 个目标)侦察任务,至少安排多少架次的 FY-2 型通信中继无人机。

问题三:所有 FY-1 型无人机现已完成侦察任务并返回基地,均可加载载荷 S-3 用于为制导炸弹提供目标指示。现要求在 7 个小时内(从第一架攻击无人机进入防御方雷达探测范围内起,到轰炸完最后一个目标止)完成对 10 个目标群所有 68 个地面目标的火力打击任务,请给出具体的无人机任务规划结果(包括每架无人机飞行路线、FY-3 型无人机携带炸弹的具体清单和攻击的目标清单),对任务进行规划以保证攻击方的无人机滞留防御方雷达有效探测范围内的时间总和最小。

问题四:由相关信息渠道获知在 A02、A05、A09 周边可能还配置有三部远程搜索雷达,该雷达对 FY 型无人机的有效作用距离是 200km。这三部雷达的工作模式是相继开机工作,即只有首先开机的雷达遭到攻击后才开启第二部雷达,同样只有第二部雷达被攻击后才开启第三部雷达。远程搜索雷达一旦开机工作,攻击方无人机群即可获知信号并锁定目标,而后安排距其最近的无人机对其摧毁。基于防御方部署远程搜索雷达的情形,对在 7 个小时内无人机轰炸任务进行重新规划,以保证攻击方的无人机滞留防御方雷达有效探测范围内的时间总和最小。

问题五:对求解模型的算法的复杂度进行分析;并讨论如何有效地提高算法的效率,以增强任务规划的时效性。基于前面构建的数学模型和对模型解算的结果,讨论哪些技术参数的提高将显著提升无人机的作战能力。

\section*{二 模型假设}

1. 假设无人机在执行任务时不存在台风、暴雨等天气问题影响。

2. 假设无人机在执行任务时不存在故障原因。

3. 假设无人机基地对无人机的控制不存在异常状态。

4. 假设防御方不对无人机进行攻击。

5. 假设被炸毁的雷达区将不在重建,失去监测作用。

\section*{三 符号说明}

\begin{tabular}{|c|c|}
\hline 符号 & 符号说明 \\
\hline $P$ & 基地 \\
\hline $N$ & 目标群个数 \\
\hline $i, j$ & 目标点 \\
\hline $\theta$ & 转角度数 \\
\hline $q$ & FY-1型无人机 \\
\hline $k$ & 荷载种类 \\
\hline $G$ & FY-2型无人机的个数 \\
\hline $x_{ijq}^{p,k}$ & 基地$p$中的第$k$种荷载的无人机$q$从目标点$i$到$j$的侦察情况 \\
\hline $t_{ijq}^{p,k}$ & 基地$p$中的第$k$种荷载的无人机$q$从目标点$i$到$j$侦察时间 \\
\hline $t_{0}$ & 从基地起飞到雷达监测区边界的航行时间 \\
\hline $d_{ij}$ & 目标点$i$到$j$之间的距离 \\
\hline $x_{iq}^{p,k}$ & 基地$p$中装载第$k$种荷载的无人机$q$对目标点$i$的侦察情况 \\
\hline
\end{tabular}

\section*{四 问题一:模型的建立和求解}

\subsection*{4.1 问题一的分析}

首先对无人机基地和10个目标群(68个目标)的地理位置坐标分析,考虑无人机所加载的载荷S-1系成像传感器的成像范围,为进一步优化目标识别精度,运用模糊聚类分析法将10个目标群进行分类,运用matlab编程求解,目标群得到进一步分类结果。考虑到FY-1型无人机加载的S-1系成像传感器对目标侦察时的与目标成像范围为[2,8],为了保证所有的目标点都能成像,因此在聚类分析的基础上运用重心法模型对所分大类中的目标点进行再次优化,对目标群进行精细划分。

为了保证无人机实现最优调度,需要保证侦察机从基地起飞到进入防御方雷

达检测范围边界的时间最短, 因此以侦察无人机从起飞到滞留防御方雷达有效探测范围内的时间总和最小为目标函数, 以每一个目标需安排 $S$-1、$S$-2 两种不同载荷各自至少侦察一次为约束条件一, 每个基地加载不同载荷执行侦察任务的无人机数目不超过该基地所拥有侦察型无人机的最大数目为约束条件二, 每架侦察机最大航行时间不超过 10 小时为约束条件三, 每个目标被两种载荷侦察的时间间隔不超过 4 小时为约束条件四, 同一基地的两架无人机起飞时间间隔和降落回收的时间间隔要求在 3 分钟以上为约束条件五, 每架侦察机从基地起飞到完成任务后飞回原起飞基地为约束条件六。建立基于多目标侦察路径选择的多目标优化模型, 运用 matlab 编程求解, 可以得到无人机的最佳侦察路线和最优调度方案。

\section*{4.2 数据处理}

\subsection*{4.2.1 基础数据呈现}

通过分析附件中无人机基地相关信息和目标群目标的相关信息, 提取出有 FY-1 型无人机基地, 运用 matlab 处理, 可呈现出目标群和无人机基地位置示意图, 如图 4-1 所示。

\begin{figure}[h]
    \centering
    \includegraphics[width=0.8\textwidth]{image.png}
    \caption{目标群和无人机基地位置示意图}
    \label{fig:4-1}
\end{figure}

\subsection*{4.2.2 成像传感器的工作原理及性能参数}

UAV 载荷成像传感器对目标进行侦察时会根据目标的不同特点采用不同的扫描方式。为简化问题, 本文中成像传感器统一采用广域搜索模式对目标进行成像, 即目标落入传感器成像带宽内即可。在二维平面上看, 传感器的成像带宽限定为 $2 \mathrm{~km}$ 是指 $A B$ 两点的距离为 $2 \mathrm{~km}$。一般限定成像传感器在无人机的一侧成像, 图 4-2 呈现的是在无人机右侧成像 (也可在左侧成像)。当无人机加载 $S$-1 型载荷后, 起飞前已完成设备调试, 即固定在无人机的某一侧成像, 飞行中不再调

一般来说,成像传感器对目标进行侦察需要一定的时间来收集需要的信息,所以要求侧向距离 \(OA\) 需大于一定的阈值,同时也有一个最大作用距离的限制,即示意图中的 \(OB\) 需小于一定的阈值。为简化问题,统一限定要求 \(OA > 2\,\text{km}\)、\(OB < 8\,\text{km}\)。当成像传感器采用广域搜索模式对目标进行成像时,为保证成像效果,一般要求载机作匀速直线运动。

\begin{figure}[h]
    \centering
    \includegraphics[width=0.8\textwidth]{image.png}
    \caption{成像传感器工作原理及相关性能参数示意图}
    \label{fig:4-2}
\end{figure}

\subsection{4.2.3 数据整理和分类}

\subsubsection{1. 模糊聚类分析}

\paragraph{(1) 模糊聚类分析概述}

模糊聚类得到了样本属于各个类别的不确定程度,很好地表述了样本类属的中介性,更客观地反映了实际情况,从而成为一种重要的聚类方法。[1]

运用模糊聚类的分析思想,把被分类的对象定义为集合 \(U\) 中的样本 \(u_i\),以一定的隶属度隶属于某一类,即所有样本都分别以不同的隶属度隶属于某一类,每一类是样本集合 \(U\) 上的一个模糊子集,每一个分类结果对应的分类矩阵即为一个模糊矩阵 \(R\)。

分类矩阵满足以下三个条件:

a) \(x_{ij} \in [0, 1]\),即分类矩阵元素在 0 到 1 之间取值;

b) \(\sum_{i=1}^{n} r_{ij} = 1\),即每一列中分别属于各类的隶属度之和为 1,对于一个样本而言,它对各类的隶属度之和为 1;

c) \(\sum_{i=1}^{c} r_{ij} > 0\),即每一行的元素之和大于 0,保证每一类都必须有样本,即总有一些样本不同程度的隶属于各类。

采用最大最小法标定被分类对象的模糊相似关系矩阵,

\begin{equation}
r_{ij} = \frac{\sum\limits_{k=1}^{m} \left( x_{ik} \wedge x_{jk} \right)}{\sum\limits_{k=1}^{m} \left( x_{ik} \vee x_{jk} \right)}
\end{equation}

其中,$\wedge$ 为取小运算,$\vee$ 为取大运算。

求相似关系矩阵 $R$ 的传递闭包 $R^*$ 取 $R$ 的乘幂,$R^2$, $R^4$, $R^8$……,若在某一步有 $R^k = R^{2k} = R^*$,则 $R^*$ 便是一个模糊等价关系,已具有传递性。模糊方阵幂的定义:若 $R$ 为 $n$ 阶方阵,定义 $R^2 = R^0 R$, $R^3 = R^{20} R$, ……,$R^k = R^{k-10} R$。根据模糊矩阵的合成运算对模糊方阵幂进行定义:

\begin{equation}
A = \left( a_{ik} \right)_{m \times s}, \quad B = \left( b_{kj} \right)_{s \times n}
\end{equation}

对于模糊方阵来说,$A$ 和 $B$ 是同一个矩阵,定义模糊矩阵 $A$ 与 $B$ 的合成:

\begin{equation}
A^0 B = \left( c_{ij} \right)_{m \times n}, \quad C_{ij} = \bigvee \left\{ \left( a_{ik} \wedge a_{kj} \right) \middle| 1 \leq k \leq s \right\}
\end{equation}

其中 $i, j \in (0, n-1)$, $k \in (1, n-1)$。

在已建立的模糊等价关系矩阵上,给定不同的 $\lambda$ 水平进行截取,从而得到不同的分类,$\lambda$ 越小,分类越少;$\lambda$ 越大,分类越多越细,因此需取最优的 $\lambda$ 值,得到合理分类。

### (2)聚类分析结果

对 10 个目标集按上述方法进行模糊聚类分析,用最大最小法求得相似关系矩阵 $R$,最后运用 Matlab 编程计算,即可得出模糊矩阵,选用 $\lambda$ 截割传递闭包,对 10 个目标群得出相关分类矩阵如图 4-3 所示,分类示意图如 4-4 所示。

\begin{equation}
\begin{bmatrix}
1 & 1 & 0 & 0 & 0 & 0 & 1 & 0 & 0 \\
1 & 1 & 0 & 0 & 0 & 0 & 1 & 0 & 0 \\
0 & 0 & 1 & 1 & 0 & 0 & 0 & 1 & 1 \\
0 & 0 & 1 & 1 & 0 & 0 & 0 & 1 & 1 \\
0 & 0 & 0 & 0 & 1 & 1 & 1 & 0 & 0 \\
0 & 0 & 0 & 0 & 1 & 1 & 1 & 0 & 0 \\
0 & 0 & 0 & 0 & 1 & 1 & 1 & 0 & 0 \\
1 & 1 & 0 & 0 & 0 & 0 & 1 & 0 & 0 \\
0 & 0 & 1 & 1 & 0 & 0 & 0 & 1 & 1 \\
0 & 0 & 1 & 1 & 0 & 0 & 0 & 1 & 1
\end{bmatrix}
\end{equation}

图 4-3 相关矩阵

根据分类矩阵可知将 10 个目标群分成 3 类,具体分类如表 4-1 所示。

\begin{table}[h]
\centering
\caption{目标群分类表}
\begin{tabular}{|c|c|c|c|}
\hline
类别 & 第一类 & 第二类 & 第三类 \\
\hline
目标群编号 & A01、A02、A08 & A03、A04、A09、A10 & A05、A06、A07 \\
\hline
\end{tabular}
\end{table}

\begin{figure}[h]
    \centering
    \includegraphics[width=\textwidth]{image.png}
    \caption{目标群分三类示意图}
    \label{fig:4-4}
\end{figure}

\section{重心法概述}

\subsection{重心法概述}

重心法首先要在坐标系中标出各个地点的位置,目的在于确定各点的相对距离。坐标系可以随便建立,然后根据各目标点在坐标系中的横纵坐标值求出目标群中重心点位置坐标 \(X\) 和 \(Y\),重心法使用的公式:

\begin{equation}
C_x = \frac{\sum_{i=1}^{n} D_{ix} V_i}{\sum_{i=1}^{n} V_i}, \quad C_y = \frac{\sum_{i=1}^{n} D_{iy} V_i}{\sum_{i=1}^{n} V_i}
\end{equation}

其中 \(C_x\) 表示重心的 \(x\) 重心法坐标;\(C_y\) 表示重心的 \(y\) 坐标;\(D_{ix}\) 表示第 \(i\) 个目标点 \(x\) 坐标;\(D_{iy}\) 表示第 \(i\) 个目标点的 \(y\) 坐标;\(V_i\) 表示第 \(i\) 个目标点权值。

最后,以所求出的重心点坐标值对应的目标点为基准点,对目标群进行重新分类,在此基础上产生新的目标群。

\subsection{重心法对目标群的优化处理结果}

考虑到 FY-1 型无人机加载的 S-1 系成像传感器对目标侦察时的与目标成像范围为 \([2, 8]\),为了保证无人机在侦察时所有的目标点都能成像,需要对已分类的目标群中的特殊目标点进一步优化,精细分群。本文采用基于重心法的优化算法,对 10 个目标群中的目标点进行进一步优化分类。

由图 4-1 可以看出,A04 目标群中的目标点较分散,以 A04 目标群为例,运用 matlab 进行编程求解分组,结果如图 4-5 所示。

\begin{figure}[h]
    \centering
    \includegraphics[width=\textwidth]{image1.png}
    \caption{A04 目标群优化分类}
    \label{fig:4-5}
\end{figure}

运用 matlab 对 10 个目标群进行编程求解,采用基于重心法的优化算法,在原有目标群的基础上可将目标群精确分 14 个目标群,如图 \ref{fig:4-6} 所示。

\begin{figure}[h]
    \centering
    \includegraphics[width=\textwidth]{image2.png}
    \caption{新分类 14 个目标群}
    \label{fig:4-6}
\end{figure}

\begin{table}
\centering
\caption{原目标群与新目标群对应表}
\begin{tabular}{|c|c|}
\hline
原目标群编号 & 新目标群编号 \\
\hline
\multirow{2}{*}{A01} & B01 \\
\cline{2-2}
 & B02 \\
\hline
\multirow{2}{*}{A02} & B03 \\
\cline{2-2}
 & B04 \\
\hline
A03 & B05 \\
\hline
\multirow{3}{*}{A04} & B06 \\
\cline{2-2}
 & B07 \\
\cline{2-2}
 & B08 \\
\hline
A05 & B09 \\
\hline
A06 & B10 \\
\hline
A07 & B11 \\
\hline
A08 & B12 \\
\hline
A09 & B13 \\
\hline
A10 & B14 \\
\hline
\end{tabular}
\end{table}

\begin{table}
\centering
\caption{新目标群分类表}
\begin{tabular}{|c|c|c|c|}
\hline
类别 & 第一类 & 第二类 & 第三类 \\
\hline
目标群编号 & B01、B02、B03、B04、B12 & B05、B06、B07、B08、B13、B14 & B09、B10、B11 \\
\hline
\end{tabular}
\end{table}

\section{模型的建立}

\subsection{模型的优化目标及流程}

\subsubsection{模型的优化目标}

本问题研究是基于多基地控制中心协同调度无人机完成对多个目标的侦察。通过合理的路径选择保证在68个目标被完全侦测到时,使各个侦察机从起飞到在敌方雷达监测范围内滞留的时间最短,即在最优调度方案下在雷达监测范围内行驶的距离最短,路径最优问题。

\subsubsection{模型的优化流程}

本文在对执行侦察任务时的整个工作流程的充分理解的基础上建立了问题模型,具体的流程图如图4-7所示。

\begin{figure}[h]
    \centering
    \includegraphics[width=0.8\textwidth]{model_optimization_flowchart.png}
    \caption{模型优化流程图}
    \label{fig:model_optimization_flowchart}
\end{figure}

详细了解侦察过程中的作业操作流程,可以安全、高效的完成对敌后目标的侦察,同时侦察路径的选择也是指导模型的重要步骤,它详细地说明了在路径选优的过程中具体操作的先后顺序,对建立数学模型起到极为重要的作用。

\subsection{确定目标函数}

本文主要研究的是多基地控制中心协同调度无人机完成对多个目标的侦察路径选择的多目标优化问题。M 个基地,每个基地拥有无人机数量 Q 架协同侦察 N 个敌方目标,这 Q 架无人机由 k 种类型的侦察技术。假设目标 i 的被侦察量为 xi(i=1,2,3……N),xi 定义为 0 或者 1(0 代表该目标不侦察,1 代表该目标侦察)。

本文以进行侦察活动的无人机以最佳侦察路线在敌方雷达监测范围内时间 T 为最短为目标,为了确保同一基地飞机起飞和降落的安全时间,同时也虑无人机从起飞到进入雷达监测区完成任务时间消耗最短为目标进行建模。目标函数为:

\begin{equation}
\min T = \sum_{p} \sum_{k} \sum_{i=1}^{N} \sum_{j=1}^{N} \sum_{q=1}^{Q} \left( t_{ijq}^{p,k} x_{ijq}^{p,k} + t_{jq}^{p,k} x_{ijq}^{p,k} \right) + t_{0}
\end{equation}

其中:
\begin{itemize}
    \item P:表示基地,P 取 P_{1}, P_{2}, ..., P_{m}
    \item N:表示目标群, N 取 1, 2, ..., n;
    \item i, j:表示目标群中的目标点
    \item $x_{ijq}^{p,k}$:表示第 Pm 基地中第 q 架无人机装载 k 种载荷从目标点 i 到目标点 j 的侦察情况;$x_{ijq}^{p,k}$ 取 1 或 0,$x_{ijq}^{p,k}=1$,表示去侦察;$x_{ijq}^{p,k}=0$,表示没去侦察;
\end{itemize}

$t_{ijq}^{p,k}$:表示第 Pm 基地中第 q 架无人机装载 k 种载荷从目标点 i 到目标点 j 的侦察时间;

$t_{jq}^{p,k}$:表示第 Pm 基地中第 q 架无人机装载 k 种载荷,到达目标 j 时转弯行驶的时间;

$t_{0}$:表示各个无人机起飞到达雷达监测区域边界的航行时间。即从基地到第一个目标点的行程时间与雷达监测半径距离的航行时间的差值。

\[ t_{ijq}^{p,k} = \frac{d_{ij}}{\nu} \]

其中,$d_{ij}$ 表示目标点 i 与目标点 j 之间的距离,$\nu$ 表示目标 i 与目标 j 之间的巡航速度。

\[ t_{jq}^{p,k} = \frac{r_{\min}}{\nu} \theta \]

其中,转弯最小半径 $r_{\min} = 70m$,$\theta$ 表示转角度数,其为前一个目标点 a(X1, Y1) 与到达当前目标点 b(X2, Y2) 时方向向量 A 与当前目标点与即将到达下一个目标点 c(X3, Y3) 时的方向向量 B 夹角。可表示为:

\[ \theta = \arctan \frac{Y2 - Y1}{X2 - X1} - \arctan \frac{Y3 - Y2}{X3 - X2} \]

### 4.3.3 确定约束条件

#### 约束条件一:为保证侦察效果,对每一个目标需安排 S-1、S-2 两种不同载荷各自至少侦察一次。

\[ \sum_{k} \sum_{p} \sum_{i} x_{iq}^{p,k} = 1 \quad \forall k = 1, 2; p = 1, 2, 3, 4; \]

$x_{iq}^{p,k}$ 表示 Pm 基地中第 q 架无人机装载 k 种载荷对目标点 i 的侦察情况;$x_{iq}^{p,k}$ 取 0 或 1,$x_{iq}^{p,k} = 1$:去侦察;$x_{iq}^{p,k} = 0$:不去侦察;

#### 约束条件二:每个基地加载不同载荷执行侦察任务的无人机数目不超过该基地所拥有侦察型无人机的最大数目。

\[ \sum_{k}^{2} q^{p,k} \leq 2 \]

#### 约束条件三:由于每架无人机受到燃油消耗的限制,其航行时间不超过 10 个小时。

\[
\sum_{i}^{N} \sum_{j}^{N} \left( t_{ij} x_{ij} + t_{j} x_{ij} \right) \leq 10h
\]

约束条件四:在每一个目标安排 S-1、S-2 两种不同载荷进行侦察时,两种不同载荷对同一目标的侦察间隔时间不超过 4 小时。

\[
\left| t_{0i}^{k_1} - t_{0i}^{k_2} \right| \leq 4h
\]

约束条件五:同一基地的两架无人机起飞时间间隔和降落的时间间隔要求在 3 分钟以上。

\[
\left| t_{0q1}^p - t_{0q2}^p \right| \geq 3 \, \text{min}
\]

约束条件六:某一基地的无人机在执行完目标侦察任务后仍返回原基地。

\[
\sum_{l=1}^{N_i} x_l^{p,k} = 1
\]

综上所述,建立基于多目标侦察路径选择的多目标优化模型为:

\[
\min T = \sum_{p} \sum_{k} \sum_{i=1}^{N} \sum_{j=1}^{N} \sum_{q=1}^{Q} \left( t_{ijq}^{p,k} x_{ijq}^{p,k} + t_{jq}^{p,k} x_{ijq}^{p,k} \right) + t_0
\]

\[
\begin{aligned}
s.t. \quad & \sum_{k} \sum_{q} \sum_{i} \sum_{j} x_{ij}^{p,k} = 1 \quad \forall k=1,2; \, p=1,2,3,4; \\
& \sum_{k} q^{p,k} \leq 2 \\
& \sum_{i}^{N} \sum_{j}^{N} \left( t_{ij} x_{ij} + t_{j} x_{ij} \right) \leq 10h \\
& \left| t_{0i}^{k_1} - t_{0i}^{k_2} \right| \leq 4h \\
& \left| t_{0q1}^p - t_{0q2}^p \right| \geq 3 \, \text{min} \\
& \sum_{l=1}^{N_i} x_l^{p,k} = 1
\end{aligned}
\]

\section*{4.4 模型求解}

本文选用多目标遗传算法对上述模型问题求解。为了满足多目标遗传算法的收敛性和分布性,保持解群体的多样性和分布性,运用基于分区域的进化算法对上述多目标问题进行求解。该算法的主要思想是将目标区域空间分成若干个子区域,在更新种群前,把全部当代个体及其后代分到各个区域的子种群,如果一个区域内子种群的个体数不超过上界 \( U \) 时,则保留全部个体;如果一个区域内子种群的个体数超过 \( U \),则采用极大极小策略选择 \( U \) 个较好的个体到该区域的子种群中。算法中,杂交的两个父代需在同一区域的个体中选取。

\subsection{4.4.1 算法设计流程}

第一步:初始化参数。初始化种群规模 $N$,子区域的数目 $S$、最大迭代次数 $T$、杂交概率 $P_c$ 和变异概率 $P_m$,生成 $S$ 个子区域中心向量 $W^1, W^2, W^3, \ldots, W^S$,生成 $N$ 个权重向量 $V^i = (v_1^i, v_2^i), (1, 2, \ldots, N)$,将 $N$ 个权重向量划分到 $S$ 个子区域中。

第二步:初始化种群。采用本文的设计方案随机产生 $2N$ 个体,用上述区域划分法将种群划分到 $S$ 个区域的子种群中。如果一个区域中的个体数目不超过 $S$,则将这个区域内的个体全部赋给该区域的子种群,如果一个区域中的个体数目超过 $S$,则用极大极小选择算子选择 $S$ 个较好的个体进入子种群。

第三步:分别对各个区域的个体进行杂交变异操作。杂交的两个父代均从属于同一个区域的个体中选。

第四步:更新种群。把杂交和变异后产生的后代和当前代所有子区域中的个体划分到各个区域中。如果一个区域中的个体数目不超过 $S$,则将这个区域内的个体全部赋给该区域的子种群,如果一个区域中的个体数目超过 $S$,则用极大极小选择算子选择 $S$ 个较好的个体进入子种群。

第五步:如果 $t \leq T, t = t + 1$,返回步骤 3,否则输出非劣解。

\begin{figure}[h]
    \centering
    \includegraphics[width=0.8\textwidth]{algorithm_flowchart.png}
    \caption{基于分区域极大极小策略的多目标遗传算法流程图}
    \label{fig:algorithm_flowchart}
\end{figure}

\subsection{4.4.2 算子的设计}

对于本问题模型的多目标遗传算法设计,不仅体现在交叉,变异,自然选择等遗传操作算法的设计上,重要的是在主要算子上的设计,例如染色体的表示,

编码、解码等,因为其采用的结构和主要算子的设计对遗传算法的性能有很大的影响。

1. 染色体编码及其数据结构

编码的恰当与否对问题求解的质量和速度是至关重要的。对于多目标遗传算法求解模型来说,标准遗传算法由二进制表述的个体编码已经不适用于这一类包含复杂信息的问题,因此找到一个合适的编码方式,成为实现算法的关键。本文根据求解问题的特点,采用了较为直观的整数编码方式,通过构造 $M$ 个基地、 $N$ 个目标来确定各基地侦察路线方案的编码,即用一个 $2$ 行 $N$ 列的矩阵来表示一条染色体。本文将染色体数据结构定义为一个 $2 \times N$ 的矩阵,其中,矩阵的列序号对应客户编号,矩阵的第一行数字表示物流中心编号,第二行数字表示客户由统一物流中心配送的顺序号。有相关如下定义:

(1) 一个染色体为一个 $2 \times N$ 的矩阵。

(2) 一个染色体包含了若干个基因,其中表示基地中心的基因取值范围是 $1$ 到整数 $M$ 的任意整数(包括整数 $1$ 和整数 $M$);表示客户顺序号的基因的取值范围是 $1$ 到整数 $N$ 之间的任意整数(包括整数 $1$ 和整数 $N$)。

(3) 染色体内允许基因重复取值。

(4) 一个种群里面同时存在多条染色体。

例如,基地 $M=4$,$N=14$,即 $4$ 个基地中心协同完成对 $14$ 个目标进行侦察任务。基地编号为 $P01=1$、$P03=2$、$P05=3$、$P07=4$,下面个体是由一个 $2$ 行 $14$ 列的矩阵表示,矩阵的第一行取值为 $1, 2, 3, 4$。形式如下所示:

\[
\begin{array}{cccccccccccccc}
3 & 3 & 3 & 3 & 1 & 1 & \mathbf{1} & 1 & 4 & 4 & 1 & 1 & 3 \\
5 & 4 & 1 & 2 & 3 & 1 & \mathbf{2} & 6 & 3 & 1 & 2 & 5 & 4 & 3
\end{array}
\]

其中,第 $7$ 列对应的是目标 $7$ 号点,该列的第一行的数值为 $1$ 表示 $7$ 号点的目标由基地 $1$ 派无人机来完成侦察,其对应的第二行的数字 $2$ 表示由基地 $1$ 无人机在进行调度侦察时,$7$ 号点是第 $2$ 个侦察目标。

2. 初始化种群

根据以上染色体的编码,对染色体进行初始化时,只需对染色体中表示基地的基因随机产生 $1$ 到 $M$ 的整数进行填充;对表示目标顺序号的基因随机产生 $1$ 到 $N$ 的整数进行填充。如此反复对种群内所有染色体进行填充即可完成种群的初始化。

在产生初始种群时,为了得到可行个体并加快算法收敛速度,我们采用上述编码方式产生可行个体。由上述编码方式我们知道个体用一个 $2$ 行 $N$ 列的矩阵表示,由矩阵列序号确定目标编号,由矩阵第一行数字确定目标被对应的基地进行侦察,由矩阵第二行数字确定目标被同一基地的无人机侦察时的先后顺序的编号。因为不是所有的基地都可以为派遣无人机对各个目标随意进行侦察,所以在初始化个体时,先根据目标信息初始化个体第一行,再根据第一行信息初始化第二行。例如,假设 $4$ 个基地中除了基地 $4$ 由于距离目标较远,在进行目标侦察过程中,其余三个基地也刚好可以完成任务,那么所有目标都由基地 $1, 2, 3$ 来侦察,同时矩阵的第一行只能取 $1, 2$ 或 $3$,不能取 $4$。

3. 染色体解码

在解码过程中,根据编码方式确定的目标的侦察顺序号来确定各基地中对目标的侦察顺序。

4. 染色体适应值的计算

将目标函数直接作为适应值函数,将目标函数式及各个无人机的最大飞行时间和各类侦察技术的间隔时间约束条件加上一个惩罚函数来处理获得适应值函数为:

\[
\begin{aligned}
\min T = & \sum_{p} \sum_{k} \sum_{i=1}^{N} \sum_{j=1}^{N} \sum_{q=1}^{Q} \left( t_{ij}^{p,k} x_{ijq}^{p,k} + t_{jq}^{p,k} x_{ijq}^{p,k} \right) + t_0 + A_1 \sum_{k} \sum_{q} \left( \sum_{i}^{N} \sum_{j}^{N} x_{ij}^{q,k} - 1, 0 \right) + A_2 \sum_{k} \sum_{q} \max \left( \sum_{i}^{N} \sum_{j}^{N} \left( t_{ij} x_{ij} + t_{j} x_{ij} \right) - 10h, 0 \right) \\
& + A_3 \sum_{k} \sum_{q} \max \left( \left| t_{0i}^{k_1} - t_{0i}^{k_2} \right| - 4h, 0 \right) + A_4 \sum_{k} \sum_{q} \min \left( \left| t_{0q_1}^{p} - t_{0q_2}^{p} \right| - 3\min, 0 \right),
\end{aligned}
\]

设 $A_i$(i=1,2,3,4)为惩罚因子,可以取一个足够大的数,使得不可行解被赋予极大的适应值,从而在迭代中被淘汰。

### 4.4.3 遗传操作

杂交之后进行变异,变异首先要进行变异个体的选择,这种选择要依照变异概率随机进行,在要进行变异的个体中随机选取同一基地的不同的两个目标,交换其顺序号,以此来完成变异。它实际相当于重新排列了某基地的侦察目标顺序列,因此最后得到的仍将是可行个体。

### 4.4.4 模型求解结果

本问题基于上述模型与算法,进行仿真实验,有四个基地 P01=1、P03=2、P05=3、P07=4,14 个侦察区域目标点,最终的侦察结果为:基地 P01 负责 5、6、7、8、13、14 目标点的侦察,基地 P05 负责 1、2、3、4、12 目标点的侦察,基地 P07 负责 9、10、11 目标点的侦察。

表 4-4 基地对目标侦察分配结果

\begin{tabular}{|c|c|c|}
\hline 负责侦察的基地 & 目标群编号 & (X,Y) 坐标(单位:km) \\
\hline P05 & B01 & (268.2,730.9) \\
\hline P05 & B02 & (259.4,730.9) \\
\hline P05 & B03 & (222.1,609.9) \\
\hline P05 & B04 & (209,615) \\
\hline P01 & B05 & (168,542.10) \\
\hline P01 & B06 & (210,455) \\
\hline P01 & B07 & (178.5,455.8) \\
\hline P01 & B08 & (182.9,473) \\
\hline P07 & B09 & (118.4,400.1) \\
\hline P07 & B10 & (93.14,310.5) \\
\hline P07 & B11 & (12.75,450.2) \\
\hline P05 & B12 & (161,659) \\
\hline P01 & B13 & (110,565) \\
\hline P01 & B14 & (105,473) \\
\hline
\end{tabular}

运用matlab编程求解上述的多目标遗传算法模型,根据本算法的设计方案,算法的运行参数如下:设定初始种群规模 \( N=200 \)、最大迭代次数 \( T=200 \)、杂交概率 \( P_c=0.9 \) 和变异概率 \( P_m=0.1 \)。求得最优调度方案的最佳路线和在该路线的航行下无人机在雷达区域的最短滞留时间。调度方案的解码为:

\[
\begin{array}{cccccccccccccc}
3 & 3 & 3 & 3 & 1 & 1 & 1 & 4 & 4 & 4 & 3 & 1 & 1 \\
1 & 2 & 3 & 4 & 1 & 3 & 4 & 2 & 2 & 1 & 3 & 5 & 6 & 5
\end{array}
\]

运用matlab通过仿真得到解码转换成问题对应的的无人机的侦察调度方案以及各个无人机侦察的航行路线,具体的方案如表4-5所示。

\textbf{表4-5 无人机最佳路线和调度方案结果}

\begin{table}[h]
\centering
\begin{tabular}{|c|c|c|c|c|c|}
\hline
\multirow{2}{*}{基地} & \multirow{2}{*}{侦察目标线路} & \multicolumn{2}{c|}{航行里程数(km)} & \multicolumn{2}{c|}{雷达区范围航行时间(h)} \\
\cline{3-6}
 &  & FY-1-S1 & FY-1-S2 & FY-1-S1 & FY-1-S2 \\
\hline
P01 & FY-1-S1,FY-1-S2 & 443.123 & 442.353 & 2.216 & 2.212 \\
\hline
P05 & 5-8-6-7-14-13 & 354.249 & 353.963 & 1.771 & 1.77 \\
\hline
P07 & 1-2-3-4-12 & 350.275 & 350.0196 & 1.751 & 1.75 \\
\hline
合计 & 10-9-11 & 1147.647 & 1146.336 & 5.738 & 5.732 \\
\hline
总计 & & 2293.983 & & 11.47 & \\
\hline
\end{tabular}
\end{table}

\textbf{FY-1无人机侦察任务的最佳路线}

问题要求侦察无人机滞留防御方雷达有效探测范围内的时间总和最小。同时为保证侦察效果,对每一个目标需安排S-1、S-2两种不同载荷各自至少侦察一次,且两种不同载荷对同一目标的侦察间隔时间不超过4小时。因此本题假设加载两种载荷的侦察无人机在雷达区范围外保持一定的队形飞行,前后依次进入防御方雷达区范围。在通过上述模型,运用matlab软件编程求解,得到FY-1无人机加载S-1、S-2两种载荷的最佳运行路线和最优调度方案。

\begin{figure}[h]
\centering
\includegraphics[width=0.8\textwidth]{image.png}
\caption{图4-9 FY-1无人机加载S-1、S-2两种载荷的最佳路线图}
\end{figure}

由图 4-6 新分类的 14 个目标群可知,B06、B07、B08 三个目标群距离相对较近,因为载荷 S-1 成像传感器的成像带宽有 \(2\,\text{km}\) 限制,因此以这三个目标群为例,放大加载 S-1 载荷的 FY-1 对三个目标群的侦察线路,如图 4-10 所示。

\begin{figure}[h]
    \centering
    \includegraphics[width=\textwidth]{image.png}
    \caption{B06、B07、B08 三个目标群 S-1 载荷的侦察线路图}
    \label{fig:4-10}
\end{figure}

\section*{结果分析:}

由图 4-9 可以看出,FY-1 装载 S-1 最佳运行路线不通过各个目标群的重心点,主要是因为载荷 S-1 成像传感器的成像带宽的范围为 \(2 \sim 8\,\text{km}\),为了保证各个目标点都能完整成像,所以其运行路线与重心点之间有一定偏移,符合条件要求。装载 S-2 载荷的最佳路线都通过各个目标群的重心点,经过计算可得每个目标群的重心点到其他目标点之间的距离均小于 \(7.5\,\text{km}\),符合 S-2 光学传感器对地面目标拍照时距目标的距离不超过 \(7.5\,\text{km}\) 的要求。由图 4-10 可知,装载 S-1 传感器的线路沿雷达站所辐射雷达范围的切线进入,符合最佳路线的要求。

\section*{五 问题二:模型的建立和求解}

\subsection{5.1 问题二的分析}

首先,以问题一中的无人机最佳运行路线和最优调度方案为基础,在此基础上分析完成问题一侦察任务所需的 FY-2 型无人机的最少架数。

然后分析 FY-2 型无人机的航行特点,考虑到 FY-2 型无人机的 \(50\,\text{km}\) 通信距离的限制,并保证与 FY-1 随时保持通信联系,为保证 FY-2 型无人机能与多架在有效范围内的侦察无人机通信,使尽可能多的 FY-1 型无人机在同时执行任务时

与 FY-2 型无人机保持 50km 以内的距离。最后基于问题一已得出的最佳运行路线和最优调度方案,以协同策略为理论指导,以 FY-2 型无人机与 FY-1 型无人机随时满足在 50km 的距离范围内为约束,建立基于蜂群算法的无人机协同航迹规划模型,用 matlab 编程确定 FY-2 无人机航迹,确定最优航迹和最少架次。

\section{模型分析}

在 FY-1 型无人机最佳运行路线和最优调度方案上,为了保证 FY-1 型无人机能将侦察信息实时传回地面控制中心,则需要通过 FY-2 型无人机作为信息转发的依据。但这一过程会由于节点的移动产生信息中断现象。为了避免这种信息中断问题,提高信息数据传回地面控制中心的成功率,下面将分析 FY-2 型无人机的行驶轨迹及数量来保证信息的传输。

\subsection{确定约束条件}

约束条件一:设 FY-2 型无人机的通信范围为 \( R \),其中 \( R = 50 \) km,并在一架 FY-1 侦察过程中,保证在 \( t \) 时刻时 FY-2 型无人机所处位置到下一个轨迹点的时刻为 \( t+1 \),移动距离为 \( r \)。由于 FY-2 以速度 \( V_{FY-2} \) 向前跟随 FY-1 型无人机移动,那么在移动期间的时间差为 \( \Delta t \),其中 \( \Delta t = (R - r) / V_{FY-2} \)。如果要在 FY-1 侦察期间,保证 FY-1 时刻在 FY-2 的通信范围。则:
\[
\begin{cases}
r < R \\
\Delta t > d_{ij} / (v_{FY-2} - v_{FY-1})
\end{cases}
\]

约束条件二:为了保证所有的 FY-1 型无人机在侦察过程中都能接受到来自 FY-2 型无人机的通信信号,则需要配置 \( G \) 架 FY-2 型无人机。并且它们在航行过程中扫描的区域面积一定覆盖所有的雷达覆盖的目标点的面积 \( S \)。则:
\[
G \cdot \pi R^2 \cdot V_{FY-2} \geq S
\]

其中,式子中的雷达的覆盖面积 \( S = 10 \cdot \pi \cdot R_0^2 \),\( R_0 = 70 \) km。

\subsection{建立模型}

由于问题二中 FY-2 是随时对 FY-1 无人机进行通信联系,并随时保持在 50km 的距离内才能实现通信任务。因此,在问题一得出的最佳航线和最优调度的基础上,增加对 FY-2 型通信范围约束,建立模型如下:
\[
\min T = \sum_{p} \sum_{k} \sum_{i=1}^{N} \sum_{j=1}^{N} \sum_{q=1}^{Q} \left( t_{ijq}^{p,k} x_{ijq}^{p,k} + t_{jq}^{p,k} x_{ijq}^{p,k} \right) + t_0
\]

\begin{equation}
\begin{aligned}
s.t. \quad \left\{
\begin{aligned}
& \sum_{k} \sum_{q} \sum_{i}^{N} \sum_{j}^{N} x_{ij}^{p,k} = 1 \quad \forall k=1,2; p=1,2,3,4; \\
& \sum_{k}^{2} q^{p,k} \leq 2 \\
& \sum_{i}^{N} \sum_{j}^{N} \left( t_{ij} x_{ij} + t_{ji} x_{ij} \right) \leq 10h \\
& \left| t_{0i}^{k_1} - t_{0i}^{k_2} \right| \leq 4h \\
& \left| t_{0q1}^{p} - t_{0q2}^{p} \right| \geq 3 \text{min} \\
& \sum_{l=1}^{N_{t_i}} x_{l}^{p,k} = 1 \\
& \Delta t > d_{ij} / \left( \nu_{FY-2} - \nu_{FY-1} \right) \\
& G \cdot \pi R^2 \cdot V_{FY-2} \geq S
\end{aligned}
\right.
\end{aligned}
\end{equation}

\subsection{5.3 模型求解}

在问题一的最佳线路和最优调配 FY-1 无人机的基础上,基于问题一的多目标遗传算法的多目标侦察路径选择模型,运用 matlab 软件编程求解可得到 FY-2 无人机在完成信息传递任务过程最少架次的最佳航线和最优调度。

\begin{table}[h]
\centering
\caption{表5-1 FY-2型无人机调度信息}
\begin{tabular}{|c|c|c|}
\hline
FY-2调度基地 & FY-2型无人机架次(架) & FY-2型无人机航行时间(h) \\
\hline
P06 & 1 & 2.686 \\
\hline
P03 & 1 & 2.003 \\
\hline
P02 & 1 & 2.187 \\
\hline
总和 & & 6.876 \\
\hline
\end{tabular}
\end{table}

\begin{figure}[h]
    \centering
    \includegraphics[width=\textwidth]{image1.png}
    \caption{图 5-1 FY-1 加载 S-1、S-2 和 FY-2 无人机最佳航线图}
\end{figure}

\begin{figure}[h]
    \centering
    \includegraphics[width=\textwidth]{image2.png}
    \caption{图 5-2 3 架 FY-2 无人机最佳航线图}
\end{figure}

结果分析:由图 5-1、图 5-2 可以看出,为了保证 FY-1 加载 S-1、S-2 两种载荷完成侦察任务时在敌方雷达区停留的时间总和最短,同时保证 FY-2 型无人机对 FY-1 型无人机的随时的实时信息通信,至少派出 3 架 FY-2 型无人机来完成信息通信的任务。

\section{六 问题三:模型的建立和求解}

\subsection{6.1 问题三的分析}

问题三要求在 7 个小时内(从第一架攻击无人机进入防御方雷达探测范围内起,到轰炸完最后一个目标止)完成对 10 个目标群(68 个目标)的火力打击,并保证攻击方的无人机滞留防御方雷达有效探测范围内的总时间最小,为无人机作出具体的任务规划。

首先,以问题一中的模糊聚类分析所得 3 个目标群大区为分区,对分区内 10 个目标群中的目标数量和位置进行定位统计。然后确定 7 个无人机基地 FY-3 型、FY-1 型无人机数量和具体分布,确定好基础数据。最后针对多目标攻击协同规划任务运用竞标算法,以无人机在防御方雷达监测区域范围内的总时间最小为目标,建立多无人机多目标攻击协同任务分配模型,对 3 个目标分区分别进行任务规划,得到 3 个分区具体的无人机任务规划方案。

\subsection{6.2 目标分配模型建立}

\subsubsection{6.2.1 确定目标函数}

本文根据无人机对目标的优势进行分配,所以建立目标分配模型如下:

\[
\min T = \sum_{q=1}^{Q} \sum_{j=1}^{M} x_{qj}^{k} t_{qj}^{k}
\]

其中:

- $Q$:表示无人机的架次;
- $M$:表示目标点数目;
- $k$:表示无人机装载的炸弹类型。$k1$ 为装载 D-1 型炸弹,$k2$ 为装载 D-2 型炸弹;
- $x_{qj}$:表示无人机 $q$ 是否攻击目标 $j$ 的情况,$x_{qj}$ 取 0 或 1,当 $x_{qj} = 1$ 时表示无人机 $q$ 发射一枚导弹攻击目标 $j$,当 $x_{qj} = 0$ 时表示无人机 $q$ 不攻击目标 $j$;
- $t_{qj}$:表示无人机 $q$ 攻击目标 $j$ 时所用的时间。

\subsubsection{6.2.2 确定约束条件}

约束条件一:为达到一定的毁伤效果,对每个目标需成功投掷 10 枚 D-1 型炸弹,而对同一目标投掷 2 枚 D-2 型炸弹即可达到相同的毁伤效果。并且,一枚导弹只能攻击一个目标。

\begin{equation}
\sum_{q=1}^{Q} x_{qj}^{k} \leq z \qquad \forall z=2,10
\end{equation}

约束条件二:为保证攻击方无人机的安全,不应使无人机长时间滞留于防御方雷达探测范围,要求每架无人机仅能攻击有限个目标。假设无人机可攻击目标的最大个数为 \( L \),其中一架装载 D-1 型炸弹的无人机可以攻击 0.6 个目标,一架装载 D-2 型炸弹的无人机可以攻击 3 个目标。

\begin{equation}
\sum_{j=1}^{M} x_{qj}^{k} \leq L \qquad \forall L=3,0.6
\end{equation}

约束条件三:保证在 7 个小时内完成对地面目标的火力打击任务,即从第一架攻击无人机进入防御方雷达探测范围内起,到轰炸完最后一个目标止的总时间不超过 7 个小时。其中参与的两种作战类型的攻击无人机,分别是装载 D-1 型导弹的无人机 \( k_1 \)、装载 D-2 型导弹的无人机 \( k_2 \)。

\begin{equation}
\sum_{q=1}^{Q} \sum_{j=1}^{M} x_{qj}^{k_1} t_{D-1} + \sum_{q=1}^{Q} \sum_{j=1}^{M} \left( x_{qj}^{k_2} \frac{d_{qj}^{k_2}}{v_{D-2}} \right) \leq 7
\end{equation}

其中:

\( t_{D-1} \) 为一枚 D-1 型炸弹从发射到炸毁目标过程中所用的时间,由于 D-1 型导弹发射以抛物线方式对目标攻击,即投放后炸弹水平速度为飞机投弹时的速度,垂直方向上以初始速度为 0 作自由落体运动,所以炸弹落地时间为固定值。

\( d_{qj}^{k_2} \) 表示 D-2 型炸弹发射时距离目标的距离,\( 10km \leq d_{qj}^{k_2} \leq 30km \)。

\( v_{D-2} \):表示一枚 D-2 型炸弹的运行速度。

综上所述,建立如下模型:

\begin{equation}
\min T = \sum_{q=1}^{Q} \sum_{j=1}^{M} x_{qj}^{k} t_{qj}
\end{equation}

\begin{equation}
\begin{cases}
\sum_{q=1}^{Q} x_{qj}^{k} \leq z & \forall z=2,10 \\
\sum_{j=1}^{M} x_{qj}^{k} \leq L & \forall L=3,0.6 \\
\sum_{q=1}^{Q} \sum_{j=1}^{M} x_{qj}^{k_1} t_{D-1} + \sum_{q=1}^{Q} \sum_{j=1}^{M} \left( x_{qj}^{k_2} \frac{d_{qj}^{k_2}}{v_{D-2}} \right) \leq 7
\end{cases}
\end{equation}

\subsection{6.2.3 竞标算法}

当某个目标 \( j \) 位于多个无人机的攻击范围时,需要对这些无人机分析竞标该

目标,决定哪些无人机来攻击该目标。具体的竞标算法如下:

1. 建立竞标信息:
   \begin{align*}
   A_{ij} &= i \\
   B_{ij} &= c_{ij} \\
   S_{ij} &= \tau_{ij}
   \end{align*}
   其中:
   \begin{itemize}
       \item $A_{ij} = i$:表示无人机 $q_i$ 竞标目标 $j$;
       \item $B_{ij} = c_{ij}$:表示无人机竞标目标 $j$ 的攻击优势,令 $c_{ij} = w_1 d_{ij} + w_2 p^k$,即综合考虑目标 $j$ 周围的无人机分别距离目标 $j$ 的距离以及各个无人机发射一枚导弹所造成的威胁度 $p$。威胁度 $p$ 为在 $k_1$ 型导弹即 D-1 导弹攻击时为 $p$ 为 0.1,在 $k_2$ 型导弹即 D-2 导弹攻击为 $p$ 为 0.5;
       \item $S_{ij} = \tau_{ij}$:表示无人机 $q_i$ 对目标 $j$ 竞标信息创立的时间点。
   \end{itemize}

2. 竞标目标
   假定目标 $j$ 位于多个无人机的攻击范围中,则这些无人机竞标该目标 $j$,通过多架无人机之间可以构成通讯网络并能相互传递消息,达成信息一致。

   设无人机 $q_i$ 和无人机 $q_j$ 为攻击范围内的任意两架,无人机 $q_i$ 为信息发送者,无人机 $q_j$ 为信息接受者,无人机 $q_i$ 接收到无人机 $q_j$ 的信息后,比较各自的攻击优势 $c_{ij}$ 与 $c_{jj}$,如果 $c_{ij} < c_{jj}$,则无人机 $q_j$ 更新信息,即 $A_{ij} = q_j, B_{ij} = c_{jj}, S_{ij} = \tau_{jj}$。当 $c_{ij} = c_{jj}$ 时,再比较时间点 $S_{ij}$,如果 $\tau_{ij} < \tau_{jj}$,则无人机 $q_i$ 更新信息。

3. 中标目标
   假设在步骤(2)中,产生的竞标获胜无人机为 $q_i$,则令 $x_{ij} = 1$,即把目标 $j$ 分配给无人机 $q_i$。

   假设对于目标 $j$ 周围区域的无人机,均不能满足 $x_{ij}^k \leq z$,则想相邻距离近的区域的无人机发出邀请,进行攻击。

\section*{6.3 模型的求解}

通过对模型的分析和计算,以及上述问题一中所求出的最佳路线,为完成对 10 个目标群 68 个地面目标的火力打击任务,以保证攻击方的无人机滞留防御方雷达有效探测范围内的总时间最小,本文的无人机任务规划方案为优先攻击雷达点所在的目标点,然后对其他目标点进行攻击。无人机 7 个基地共有 97 架 FY-

\section*{3型无人机,本题求解得到共需要派出91架FY-3型无人机,从第一架飞机进入雷达区到轰炸完最后一个目标,共用1.673小时,无人机在防御方雷达监测范围的时间总和为6.924小时。对无人机具体任务规划方案如下所示。}

根据问题一中运用模糊聚类分析方法将10个目标群划分为3个区域,分别对3个区域开展无人机攻击任务计划。

\begin{figure}[h]
    \centering
    \includegraphics[width=\textwidth]{image.png}
    \caption{目标群分区图}
    \label{fig:target_zones}
\end{figure}

对于一区,包含A01、A02、A08三个目标群,共24个目标;二区有A03、A04、A09、A10四个目标群,共25个目标;三区有A05、A06、A07三个目标群,共19个目标。具体目标如表\ref{tab:radar_list}所示。

\begin{table}[h]
    \centering
    \caption{三个区目标群雷达清单}
    \label{tab:radar_list}
    \begin{tabular}{|c|c|c|c|c|c|c|c|c|c|}
        \hline
        分区 & \multicolumn{3}{c|}{一区} & \multicolumn{3}{c|}{二区} & \multicolumn{3}{c|}{三区} \\
        \hline
        目标群 & A01 & A02 & A08 & A03 & A04 & A09 & A10 & A05 & A06 & A07 \\
        \hline
        雷达站 & A0101 & A0201 & A0801 & A0301 & A0401 & A0901 & A1001 & A0501 & A0601 & A0701 \\
        \hline
    \end{tabular}
\end{table}

\section*{1. 一区、二区、三区的携带D-2炸弹的FY-3型无人机任务分配}

为了保证攻击方的无人机滞留防御方雷达有效探测范围内的总时间最小,无人机任务规划要求优先轰炸一区、二区、三区的所有雷达站,然后对其他目标点就行攻击,三个区具体携带D-2炸弹的FY-3型无人机的攻击目标和炸弹清单如表\ref{tab:attack_plan}所示。

\begin{table}[h]
    \centering
    \caption{三个区具体携带D-2炸弹的FY-3型无人机的攻击目标和炸弹清单}
    \label{tab:attack_plan}
    \begin{tabular}{|c|c|c|c|}
        \hline
        分区 & 目标群 & 雷达站 & 炸弹清单 \\
        \hline
        一区 & A01 & A0101 & D-2 \\
        & A02 & A0201 & D-2 \\
        & A08 & A0801 & D-2 \\
        \hline
        二区 & A03 & A0301 & D-2 \\
        & A04 & A0401 & D-2 \\
        & A09 & A0901 & D-2 \\
        & A10 & A1001 & D-2 \\
        \hline
        三区 & A05 & A0501 & D-2 \\
        & A06 & A0601 & D-2 \\
        & A07 & A0701 & D-2 \\
        \hline
    \end{tabular}
\end{table}

\begin{table}
\centering
\caption{三个区携带D-2炸弹的FY-3型无人机任务分配规划}
\begin{tabular}{|c|c|c|c|c|c|}
\hline
分区 & 出发 & FY-3型无人 & FY-3型无人机携带 & 攻击目标 & 攻击目标 \\
 & 基地 & 机架数 & 炸弹的具体清单 & 顺序 & 个数 \\
\hline
\multirow{3}{*}{一区} & P01 & 1 & 6个D-2 & A0101-A0207 & 2 \\
\cline{2-6}
 & \multirow{2}{*}{P06} & \multirow{2}{*}{2} & \multirow{2}{*}{12个D-2} & A0201-A0801-A0209 & 3 \\
\cline{5-6}
 & & & & A0202-A0805 & 2 \\
\hline
\multirow{2}{*}{二区} & \multirow{2}{*}{P02} & 1 & 6个D-2 & A0301-A0901-A1005 & 3 \\
\cline{3-6}
 & & 1 & 6个D-2 & A0401-A1001-A0902 & 3 \\
\hline
\multirow{3}{*}{三区} & P02 & 1 & \multirow{3}{*}{18个D-2} & A0601-A0503 & 2 \\
\cline{2-3} \cline{5-6}
 & P02 & 1 & & A0502-A0704 & 2 \\
\cline{2-3} \cline{5-6}
 & P07 & 1 & & A0501-A0704 & 2 \\
\hline
\end{tabular}
\end{table}

2. 一区、二区、三区的携带D-1炸弹的FY-3型无人机任务分配

经过携带D-2炸弹的FY-3型无人机对上述包括所有的雷达点的19个目标点的攻击后,防御方的所有雷达点均遭到攻击,不再具有监测功能。然后根据每个目标群所剩下的目标点个数,结合每个基地所剩下的无人机架数,对剩余的目标点派携带D-1炸弹的FY-3型无人机对其进行攻击,具体调度如表6-3所示,具体攻击方案如表6-4所示。

\begin{table}
\centering
\caption{携带D-1炸弹的FY-3型无人机调度}
\begin{tabular}{|c|c|c|c|}
\hline
分区 & 出发基地 & FY-3型无人机架数 & FY-3型无人机携带炸弹的具体清单 \\
\hline
\multirow{3}{*}{一区} & P01 & 9 & 54 \\
\cline{2-4}
 & P05 & 10 & 60 \\
\cline{2-4}
 & P03 & 10 & 60 \\
\hline
\multirow{4}{*}{二区} & P03 & 3 & 18 \\
\cline{2-4}
 & P06 & 13 & 78 \\
\cline{2-4}
 & P04 & 13 & 78 \\
\cline{2-4}
 & P07 & 3 & 18 \\
\hline
\multirow{2}{*}{三区} & P07 & 9 & 54 \\
\cline{2-4}
 & P02 & 13 & 78 \\
\hline
\end{tabular}
\end{table}

\begin{table}
\centering
\caption{携带D-1炸弹FY-3无人机对三个区的攻击方案}
\begin{tabular}{|c|c|c|c|c|}
\hline
分区 & 目标群 & 所需携带D-1炸弹FY-3架数 & 消耗炸弹数 & 剩余炸弹数 \\
\hline
\multirow{3}{*}{一区} & A01 & 15 & 90 & 0 \\
\cline{2-5}
 & A02 & 9 & 50 & 4 \\
\cline{2-5}
 & A08 & 5 & 30 & 0 \\
\hline
\multirow{4}{*}{二区} & A03 & 7 & 40 & 2 \\
\cline{2-5}
 & A04 & 15 & 90 & 0 \\
\cline{2-5}
 & A09 & 5 & 30 & 0 \\
\cline{2-5}
 & A10 & 5 & 30 & 0 \\
\hline
\multirow{3}{*}{三区} & A05 & 6 & 36 & -4 \\
\cline{2-5}
 & A06 & 9 & 50 & 4 \\
\cline{2-5}
 & A07 & 7 & 40 & 2 \\
\hline
\end{tabular}
\end{table}

\section*{3. 以三区为例的无人机线路图分析}

\begin{figure}[h]
\centering
\includegraphics[width=\textwidth]{image.png}
\caption{某时刻三区所有无人机线路图}
\end{figure}

由图6-2可知,某时刻由P03出发的FY-1与由P07出发的FY-3-D-2在C1点汇合,同时由P07出发的FY-1与由P02出发的FY-3-D-2在C2点汇合。由P07出发的FY-1与由P02出发的FY-3-D-2、13架的FY-3-D-1在C3点汇合,C3与C1的距离为70km,即为雷达点的监测半径。P07出发的9架FY-3-D-1在以

C2 点与 A0601 点连接线的延长线上,且以 C2 为圆心的圆上。

在下一个时刻,C1 和 C2 两点的无人机分别进入 A0601、A0501 雷达点区域,并对分别对两个目标雷达点进行轰炸,当轰炸结束那一刻,C3 点无人机刚好进入雷达目标点的范围边界处,此时,C3 点与 C1 点的距离仍为 70km。这时,雷达点已被轰炸,不具有监测能力,因此,按照上述方法依次对目标进行轰炸,可以保证无人机在防御方雷达监测范围内滞留的时间总和最小。

\begin{figure}[h]
    \centering
    \includegraphics[width=\textwidth]{image.png}
    \caption{三区中 A06 目标群携带 D-1 炸弹 FY-3 路线图}
    \label{fig:6-3}
\end{figure}

图 6-3 是以三区中的 A06 目标群为例,A06 中雷达点 A0601 已被 D-2 型炸弹轰炸,所以对携带 D-1 炸弹 FY-3 对 A06 中的不是雷达点的目标点进行具体的任务分配,具体路线如图 6-3 中黑色线所示。A06 目标群中未被轰炸的目标个数为 5 个,分别是 A0602、A0603、A0604、A0605、A0606,共需要 50 枚 D-1 炸弹,每架 FY-3 可携带 6 枚 D-1,因此至少需要 9 台携带 D-1 炸弹的 FY-3 进行攻击。具体攻击顺序安排如表 6-5 所示。

\begin{table}[h]
    \centering
    \caption{三区中 A06 目标群携带 D-1 炸弹 FY-3 任务表}
    \label{tab:6-5}
    \begin{tabular}{|c|c|c|c|}
        \hline
        攻击目标顺序 & 所需携带 D-1 炸弹的 FY-1 无人机次数和投弹枚数 & & \\
        \hline
        A0602 & 第一架投弹 6 枚 & 第二架 4 枚 & \\
        \hline
        A0605 & 第二架 2 枚 & 第三架 6 枚 & 第四架 2 枚 \\
        \hline
        A0604 & 第四架 4 枚 & 第五架 6 枚 & \\
        \hline
        A0606 & 第六架 6 枚 & 第七架 4 枚 & \\
        \hline
        A0603 & 第七架 2 枚 & 第八架 6 枚 & 第九架 2 枚 \\
        \hline
    \end{tabular}
\end{table}

结果分析:本题通过划分 3 个分区,实现了对 68 个目标的全部轰炸。一,求解得到共需要派出 91 架 FY-3 型无人机,7 个基地 FY-3 无人机总架数为 97 架,符合无人机数量要求;二,从第一架飞机进入雷达区到轰炸完最后一个目标,共用 1.673 小时,题中要求在 7 个小时内(从第一架攻击无人机进入防御方雷达探测范围内起,到轰炸完最后一个目标止)完成对 10 个目标群所有 68 个地面目标的火力打击任务,符合第三题中完成任务的时间要求;三,无人机在防御方雷达监测范围的时间总和为 6.924 小时,是所求得的最短时间。

\section*{七 问题四:模型的建立和求解}

\subsection*{7.1 问题四的分析}

已知在 A02、A05、A09 周边可能还配置有三部远程搜索雷达,该雷达对 FY 型无人机的有效作用距离是 200km,三部雷达的工作模式是相继开机工作,位置不确定,开机时间不确定,且开机顺序不确定。

首先基于防御方部署远程搜索雷达的情形,分析防御方部署远程雷达的原因和目的。然后基于概率论的思想讨论三部远程搜索雷达相继开机情况,同时考虑到在完成任务的过程中保证攻击方的无人机滞留防御方雷达有效探测范围内的时间总和最小。最后,针对每一种情况安排不同的任务对所有目标和雷达进行攻击,保证从第一架攻击无人机进入防御方雷达探测范围内起,到轰炸完最后一个目标止的时间不超过 7 个小时,完成问题三的任务。

\subsection*{7.2 部署远程搜索雷达的目的分析}

防御方部署的远程搜索雷达与目标群的 10 个雷达组网形成雷达系统,该系统表现出明显的抗击导弹的性能。通过部署的远程搜索雷达将使导弹定位精度大为降低,可以起到干扰和欺骗的作用。另外,三部雷达之间采用交替开关机工作措施,对导弹行程闪烁的电磁辐射源,使其目标跟踪错乱,大大提高了雷达组网的生存能力,提高了目标的安全度。

\subsection*{7.3 三部远程搜索雷达相继开机情况分析}

问题三在进行无人机任务分配后,P01、P05 两个基地各有 3 架 FY-3,一共 6 架。问题四是在对问题三完成任务规划的基础上对 3 个远程搜索雷达进行轰炸。为了方便对问题的描述,对 A02、A05、A09 三个雷达点附近的远程搜索雷达分别定义为 a、b、c 三个远程监控雷达点。考虑随机变量的排列组合问题,对 3 个远程搜索监控雷达点进行排序,一共有 6 个排序方式。即为:

模式一:远程雷达开机顺序 b-a-c

模式二:远程雷达开机顺序 b-c-a

模式三:远程雷达开机顺序 a-b-c

模式四:远程雷达开机顺序 a-c-b

\section*{模式五:远程雷达开机顺序 c-a-b}
模式六:远程雷达开机顺序 c-b-a

\subsection*{7.4 求解}

\begin{figure}[h]
    \centering
    \includegraphics[width=\textwidth]{image.png}
    \caption{某 t 时刻 FY-1 与 FY-3 机组位置图}
    \label{fig:7-1}
\end{figure}

假定在某一时刻 \( t \),携带 D-2 型炸弹的 FY-3 与携带 S-3 的 FY-1 目标指导无人机的位置如图 7-1 中 T1-T8 所示,此时携带 D-1 型炸弹的 FY-3 型无人机位置位于目标雷达点与携带 D-2 型炸弹的 FY-3 位置连线的延长线上,即在蓝色圆上,这样即可保证携带 D-1 的 FY-3 刚进入雷达区域边界时,雷达点就被携带 D-2 的 FY-3 无人机所炸毁,即不再具有雷达监测作用。

由于该问题中三部远程搜索雷达的工作模式是相继开机工作,并根据题意理解三部远程搜索雷达在无人机进入雷达监测区前有一个是开启的。因此,可以对这三部远程搜索雷达的开机情况做六种模式分析。

\textbf{模式一:远程雷达开机顺序 b-a-c}

1. 考虑距离因素,假设从 P06 基地的出发的携带 D-2 的 FY-3 无人机第一个检测到 b 远程雷达,检测到 b 雷达后,立即飞向 b 雷达进行轰炸,此时位于 b 雷达点附近的 A0501 雷达点由 P01 或 P05 发出两架 D-1 来炸毁。

2. a 雷达开启,此时轰炸 A0201 雷达点的无人机已飞过 a 雷达点,因此从 P01 或 P05 派出两架携带 D-1 型炸弹的 FY-3 型无人机对 a 雷达点进行轰炸。

\section*{3.c 雷达点开启,此时轰炸 A0901 雷达点的无人机已飞过 c 雷达点,因此从 P01 或 P05 基地派出两架携带 D-1 型炸弹的 FY-3 型无人机对 c 雷达点进行轰炸。}

由上述分析可知,A0501、a、c 三个雷达点都需要由 P01、P05 两个基地派出无人机炸毁,每个目标点需要 10 个 D-1 炸弹,因此同时派出 P01、P05 两个基地的 6 架 D-1 炸弹的 FY-3 无人机对其进行轰炸。由此可知,除去 A0501、a、c 的其他目标点都由问题三中原始任务方案执行轰炸。具体任务规划如表 7-1 所示。

\begin{table}[h]
\centering
\caption{模式一下未被炸毁点无人机任务规划}
\begin{tabular}{|c|c|c|c|c|}
\hline
未被炸毁 & 出发基地 & FY-3 型无人机 & FY-3 型无人机携带炸弹的 & 攻击目标 \\
目标点 & & 架数 & 具体清单 & 个数 \\
\hline
A0501 & P01、P05 & 2 & 12 个 D-1 & 1 \\
\hline
a & & 2 & 12 个 D-1 & 1 \\
\hline
c & & 2 & 12 个 D-1 & 1 \\
\hline
\end{tabular}
\end{table}

\section*{模式二:远程雷达开机顺序 b-c-a}

1. b 远程雷达第一个开启,检测到 b 雷达后,立即飞向 b 雷达进行轰炸,此时位于 b 雷达点附近的 A0501 雷达点由 P01 或 P05 发出两架 D-1 来炸毁。

2. c 远程雷达点开启,此时轰炸完 A0301 雷达点的无人机继续向 c 远程雷达点飞行,并炸毁 c 远程雷达点。A0901 雷达点由 P01 或 P05 基地派出两架携带 D-1 型炸弹的 FY-3 型无人机对 A0901 雷达点进行轰炸。

3. a 远程雷达开启,有刚摧毁过 A0801 雷达点的携带 D-2 型炸弹的返航机组炸毁。

模式二的分析与模式一类似,在模式二的情形下,具体的任务规划如表 7-2 所示。

\begin{table}[h]
\centering
\caption{模式二下未被炸毁点无人机任务规划}
\begin{tabular}{|c|c|c|c|c|}
\hline
未被炸毁 & 出发基地 & FY-3 型无人机 & FY-3 型无人机携带炸弹的 & 攻击目标 \\
目标点 & & 架数 & 具体清单 & 个数 \\
\hline
A0501 & P01、P05 & 2 & 12 个 D-1 & 1 \\
\hline
A0901 & & 2 & 12 个 D-1 & 1 \\
\hline
无目标点 & & 2 & 12 个 D-1 & 0 \\
\hline
\end{tabular}
\end{table}

\section*{模式三:远程雷达开机顺序 a-b-c}

1. 第一个开启 a 远程监测雷达点,此时由 P01 基地出发的携带 S-3 的 FY-1 无人机首先检测到,配合原线路由 P06 出发的携带 D-2 炸弹的 FY-3 无人机炸毁 a 远程监测雷达点。A0201 雷达点目标由 P01 或 P05 发出两架 D-1 来炸毁。

2. 开启 b 远程监测雷达点,直接由 P01 或 P05 发出两架 D-1 来炸毁。

3. 开启 c 远程检测雷达点,直接由 P01 或 P05 发出两架 D-1 来炸毁。

\begin{table}
\centering
\caption{模式三下未被炸毁点无人机任务规划}
\begin{tabular}{|c|c|c|c|c|}
\hline
未被炸毁 & 出发基地 & FY-3型无人机 & FY-3型无人机携带炸弹的 & 攻击目标 \\
目标点 & & 架数 & 具体清单 & 个数 \\
\hline
A0201 & \multirow{3}{*}{P01、P05} & 2 & 12个D-1 & 1 \\
\hline
b & & 2 & 12个D-1 & 1 \\
\hline
c & & 2 & 12个D-1 & 1 \\
\hline
\end{tabular}
\end{table}

模式四:远程雷达开机顺序a-c-b

1. 第一个开启a远程监测雷达点,此时由P01基地出发的携带S-3的FY-1无人机首先检测到,配合原线路由P06出发的携带D-2炸弹的FY-3无人机炸毁a远程监测雷达点。A0201雷达点目标由P01或P05发出两架D-1来炸毁。

2. 开启c远程监测雷达点,由刚炸毁A0301雷达目标点的无人机组炸毁c远程监测雷达点,A0901雷达目标点由P01或P05发出两架D-1来炸毁。

3. 开启b远程检测雷达点,由刚炸毁过A0501的无人机组返航摧毁,A0701雷达目标点由P01或P05发出两架D-1来炸毁。

\begin{table}
\centering
\caption{模式四下未被炸毁点无人机任务规划}
\begin{tabular}{|c|c|c|c|c|}
\hline
未被炸毁 & 出发基地 & FY-3型无人机 & FY-3型无人机携带炸弹的 & 攻击目标 \\
目标点 & & 架数 & 具体清单 & 个数 \\
\hline
A0201 & \multirow{3}{*}{P01、P05} & 2 & 12个D-1 & 1 \\
\hline
A0901 & & 2 & 12个D-1 & 1 \\
\hline
A0701 & & 2 & 12个D-1 & 1 \\
\hline
\end{tabular}
\end{table}

模式五:远程雷达开机顺序c-a-b

1. 第一个开启c远程监测雷达点,此时由P01基地出发的携带S-3的FY-1无人机首先检测到,配合原线路由P06出发的携带D-2炸弹的FY-3无人机炸毁c远程监测雷达点。A0901雷达点目标由P01或P05发出两架D-1来炸毁。

2. 开启a远程监测雷达点,直接由P01或P05发出两架D-1来炸毁。

3. 开启b远程检测雷达点,由刚炸毁过A1001雷达目标点的无人机组返航摧毁,A0701雷达目标点由P01或P05发出两架D-1来炸毁。

\begin{table}
\centering
\caption{模式五下未被炸毁点无人机任务规划}
\begin{tabular}{|c|c|c|c|c|}
\hline
未被炸毁 & 出发基地 & FY-3型无人机 & FY-3型无人机携带炸弹的 & 攻击目标 \\
目标点 & & 架数 & 具体清单 & 个数 \\
\hline
A0901 & \multirow{3}{*}{P01、P05} & 2 & 12个D-1 & 1 \\
\hline
a & & 2 & 12个D-1 & 1 \\
\hline
A0902 & & 2 & 12个D-1 & 1 \\
\hline
\end{tabular}
\end{table}

\section*{模式六:远程雷达开机顺序 c-b-a}

1. 第一个开启 c 远程监测雷达点,此时由 P01 基地出发的携带 S-3 的 FY-1 无人机首先检测到,配合原线路由 P06 出发的携带 D-2 炸弹的 FY-3 无人机炸毁 c 远程监测雷达点。A0901 雷达点目标由 P01 或 P05 发出两架 D-1 来炸毁。

2. 开启 b 远程监测雷达点,直接由 P01 或 P05 发出两架 D-1 来炸毁。

3. 开启 a 远程检测雷达点,由刚炸毁过 A0801 的无人机组返航摧毁。

\begin{table}[h]
\centering
\caption{模式六下未被炸毁点无人机任务规划}
\begin{tabular}{|c|c|c|c|c|}
\hline
未被炸毁 & 出发基地 & FY-3型无人机 & FY-3型无人机携带炸弹的 & 攻击目标 \\
目标点 & & 架数 & 具体清单 & 个数 \\
\hline
A0901 & \multirow{3}{*}{P01、P05} & 2 & 12个D-1 & 1 \\
\hline
b & & 2 & 12个D-1 & 1 \\
\hline
无目标点 & & 2 & 12个D-1 & 0 \\
\hline
\end{tabular}
\end{table}

\section*{八 问题五:模型的改进和评价}

\subsection*{8.1 问题五的分析}

首先,基于上述问题中所构建的模型和模型运算的结果,通过对模型求解的算法进行分析和回顾,找出改进办法提高算法的效率。然后结合所建模型的参数,通过改变和调整参数,选取对无人机作战能力影响较大的参数,对参数进行调整,从而有效提高无人机作战的能力。

\subsection*{8.2 模型的算法和复杂度分析}

无人机以其低可探测性、高消费比、无人员伤亡、操作灵活等多方面的卓越优点在作战领域的很多方面都有成功的运用。无人侦察机是现今发展最为完善,门类最为齐全的一类无人机,且在实战中有大量运用。无人机自身目标小,不易被敌方截获,能进入高威胁区并根据不同任务调整飞行高度进行侦察监视。因此无人机已成为主要战略和战术侦察装备。

本文采用目标遗传算法求解具有显著的优势。该算法可以求解多目标优化问题。为了使该算法具有很强的收敛性和分布性,同时保证求解的种群的多样性,在运用多目标遗传算法的过程中,考虑了运用分区域策略对模型问题进行求解。该算法通过构造一种新的编码方式来有效地产生初始可行解,从而在构造初始种群和进行个体变异中避免产生无效个体,加快了算法的收敛速度。通过仿真验证所建模型能够有效地解决协同多架无人机侦察多个目标的路径选择问题。

对于目标分配模型,根据目标周围区域内的无人机对目标的攻击优势进行分配。对于目标模型运用竞标算法,通过目标周围多架无人机根据各自的攻击优势选择目标,建立目标的竞标信息,无人机之间通过彼此之间的信息传递完成目标

任务的分配问题。该算法不仅能够实现目标的无冲突分配,同样还能保证无人机执行任务的效率。

\section{算法效率的提高}

采用的多目标遗传算法在运用分区域策略过程中,需要进行合理的区域划分,它是保证算法速度收敛快慢的关键。因此,在进行该算法的过程中,需对目标区域进行最佳的区域划分。

针对目标分配模型采用的竞标算法,需要保证目标周边区域的无人机之间竞标信息有效传递,它是各个无人机高效的完成任务的关键部分。该算法的效率常常依赖于无人机之间的通讯拓扑结构。当无人机之间存在通讯时延,以及通讯拓扑不稳定时,则该算法不能保证有效的实现多无人机对多目标的协同分配问题。因此,需要建立有效的信息传递模型。

\section{影响作战能力参数分析}

在求解无人机侦察目标的最佳航线中,本文采用了基于多目标侦查路径选择的多目标优化模型。该模型考虑了无人机在侦察过程中,在各个目标点之间航行时的转弯角度 $\theta$。但是该角度在求解过程中,假定无人机是在目标点正上空上转弯。实际在航行过程中,由于加载 S1 载荷 FY-1 无人机成像技术的限制,无人机在距离目标点 $2 \mathrm{~km} \sim 8 \mathrm{~km}$ 范围内转弯。因此转弯的角度将大于假定的条件所求得的角度。因此,无人机航行的实际距离将大于所求解的值。通过对转弯角度 $\theta$ 的优化求解,将有助于无人机在侦察及作战中的作战能力的提高。

\section*{参考文献}

[1] 高新波. 模糊聚类分析及应用 [M]. 西安: 西安电子科技大学出版社, 1-5, 2004.

[2] 顾丽娟, 司守奎, 孙慧静. 模糊聚类分析应用于炮兵精确打击效能评估 [J]. 兵工自动化, 34 (12): 1-3, 2015.

[3] 黄捷, 陈谋, 姜长生. 无人机空对地多目标攻击的满意分配决策技术 [J]. 电光与控制, 07 (10): 10-13, 2014.

[4] 陈侠, 魏晓明, 徐光延. 基于分布式邀请拍卖算法的多机协同空战决策 [J]. 系统仿真学报, 06 (12): 1261-1266, 2014.

[5] 赵辉, 李牧东, 韩统, 黄汉桥. 基于多目标 MQABC 算法的无人机协同任务分配 [J]. 华中科技大学学报 (自然科学版), 03:121-126, 2016.

[6] 鲁晓倩. 组网雷达航迹干扰研究 [M]. 电子科技大学, 2007.

[7] 李晶. 协同配送车辆路径问题的多目标优化研究 [D]. 兰州交通大学, 2013.

[8] 吴蔚楠. 巡飞弹协同攻击任务规划问题建模与方法研究 [D]. 哈尔滨工业大学, 2013.

[9] 田菁. 多无人机协同侦察任务规划问题建模与优化技术研究 [D]. 国防科学技术大学, 2007.

[10] 林林. 基于协同机制的多无人机任务规划研究 [D], 北京邮电大学, 2013.

[11] 王健. 多架无人机攻击多目标的协同航迹规划算法研究 [D]. 西北工业大学, 2004.

\section*{附录:遗传算法求最优路径}

\begin{verbatim}
function [path,lmin]=ga(data,d) %data 为点集,d 为距离矩阵,即赋权图
tic
sj0=data;
x=sj0(:,1);y=sj0(:,2);
N=length(x);
L=N;  %sj0 的长度
w=800;dai=1000;
%通过改良圈算法选取优良父代 A
for k=1:w
    c=randperm(L-2);
    c1=[1,c+1,L];
    flag=1;
    while flag>0
        flag=0;
        for m=1:L-3
            for n=m+2:L-1
                if d(c1(m),c1(n))+d(c1(m+1),c1(n+1))<d(c1(m),c1(m+1))+d(c1(n),c1(n+1))
                    flag=1;
                    c1(m+1:n)=c1(n:-1:m+1);
                end
            end
        end
    end
    J(k,c1)=1:L;
end
J=J/L;
J(:,1)=0;J(:,L)=1;
rand('state',sum(clock));
%遗传算法实现过程
A=J;
for k=1:dai %产生 0~1 间随机数列进行编码
    B=A;
    c=randperm(w);
    %交配产生子代 B
    for i=1:2:w
        F=2+floor(100*rand(1));
        temp=B(c(i),F:L);
        B(c(i),F:L)=B(c(i+1),F:L);
        B(c(i+1),F:L)=temp;
    end
\end{verbatim}

\begin{verbatim}
end;
%变异产生子代 C
by=find(rand(1,w)<0.1);
if length(by)==0
    by=floor(w*rand(1))+1;
end
C=A(by,:);
L3=length(by);
for j=1:L3
    bw=floor(1+fix(rand(1,3)*N));     %产生 1-N 的 3 个随机数
    bw=sort(bw);
    C(j,:) = C(j,[1:bw(1)-1, bw(2)+1:bw(3), bw(1):bw(2), bw(3)+1:L]);
end
G=[A;B;C];
TL=size(G,1);
%在父代和子代中选择优良品种作为新的父代
[dd,IX]=sort(G,2);
temp=[];
temp(1:TL)=0;
for j=1:TL
    for i=1:L-1
        temp(j)=temp(j)+d(IX(j,i),IX(j,i+1));
    end
end
[DZ,IZ]=sort(temp);
A=G(IZ(1:w),:);
end
path=IX(IZ(1),:)
% for i=1:length(path)
%     path(i)=path(i)-1;
% end
% path=path(2:end-1);
lmin=0;l=0;
for j=1:(length(path)-1)
    t1=path(j);t2=path(j+1);
    l=d(t1,t2);
    lmin=lmin+l;
end
xx=sj0(path,1);yy=sj0(path,2);
plot(xx,yy,'r-o');
axis equal
toc
\end{verbatim}