\begin{center}
\textbf{第十届华为杯全国研究生数学建模竞赛}
\end{center}

\begin{tabular}{l l}
学校 & 南京航空航天大学 \\
参赛队号 & 10287030 \\
队员姓名 & \begin{tabular}{l l}
1. 程浩 \\
2. 陈军 \\
3. 张顺根
\end{tabular}
\end{tabular}

\begin{center}
参赛密码 \_\_\_\_\_\_\_\_\_\_\_\_\_\_\_\_\_\_\_\_\_\_\_\_\_\_\_\_\_\_\_\_\_\_\_\_\_\_\_\_\_\_\_\_\_\_\_\_\_\_\_\_\_\_\_\_\_\_\_\_\_\_\_\_\_\_\_\_\_\_\_\_\_\_\_\_\_\_\_\_\_\_\_\_\_\_\_\_\_\_\_\_\_\_\_\_\_\_\_\_\_\_\_\_\_\_\_\_\_\_\_\_\_\_\_\_\_\_\_\_\_\_\_\_\_\_\_\_\_\_\_\_\_\_\_\_\_\_\_\_\_\_\_\_\_\_\_\_\_\_\_\_\_\_\_\_\_\_\_\_\_\_\_\_\_\_\_\_\_\_\_\_\_\_\_\_\_\_\_\_\_\_\_\_\_\_\_\_\_\_\_\_\_\_\_\_\_\_\_\_\_\_\_\_\_\_\_\_\_\_\_\_\_\_\_\_\_\_\_\_\_\_\_\_\_\_\_\_\_\_\_\_\_\_\_\_\_\_\_\_\_\_\_\_\_\_\_\_\_\_\_\_\_\_\_\_\_\_\_\_\_\_\_\_\_\_\_\_\_\_\_\_\_\_\_\_\_\_\_\_\_\_\_\_\_\_\_\_\_\_\_\_\_\_\_\_\_\_\_\_\_\_\_\_\_\_\_\_\_\_\_\_\_\_\_\_\_\_\_\_\_\_\_\_\_\_\_\_\_\_\_\_\_\_\_\_\_\_\_\_\_\_\_\_\_\_\_\_\_\_\_\_\_\_\_\_\_\_\_\_\_\_\_\_\_\_\_\_\_\_\_\_\_\_\_\_\_\_\_\_\_\_\_\_\_\_\_\_\_\_\_\_\_\_\_\_\_\_\_\_\_\_\_\_\_\_\_\_\_\_\_\_\_\_\_\_\_\_\_\_\_\_\_\_\_\_\_\_\_\_\_\_\_\_\_\_\_\_\_\_\_\_\_\_\_\_\_\_\_\_\_\_\_\_\_\_\_\_\_\_\_\_\_\_\_\_\_\_\_\_\_\_\_\_\_\_\_\_\_\_\_\_\_\_\_\_\_\_\_\_\_\_\_\_\_\_\_\_\_\_\_\_\_\_\_\_\_\_\_\_\_\_\_\_\_\_\_\_\_\_\_\_\_\_\_\_\_\_\_\_\_\_\_\_\_\_\_\_\_\_\_\_\_\_\_\_\_\_\_\_\_\_\_\_\_\_\_\_\_\_\_\_\_\_\_\_\_\_\_\_\_\_\_\_\_\_\_\_\_\_\_\_\_\_\_\_\_\_\_\_\_\_\_\_\_\_\_\_\_\_\_\_\_\_\_\_\_\_\_\_\_\_\_\_\_\_\_\_\_\_\_\_\_\_\_\_\_\_\_\_\_\_\_\_\_\_\_\_\_\_\_\_\_\_\_\_\_\_\_\_\_\_\_\_\_\_\_\_\_\_\_\_\_\_\_\_\_\_\_\_\_\_\_\_\_\_\_\_\_\_\_\_\_\_\_\_\_\_\_\_\_\_\_\_\_\_\_\_\_\_\_\_\_\_\_\_\_\_\_\_\_\_\_\_\_\_\_\_\_\_\_\_\_\_\_\_\_\_\_\_\_\_\_\_\_\_\_\_\_\_\_\_\_\_\_\_\_\_\_\_\_\_\_\_\_\_\_\_\_\_\_\_\_\_\_\_\_\_\_\_\_\_\_\_\_\_\_\_\_\_\_\_\_\_\_\_\_\_\_\_\_\_\_\_\_\_\_\_\_\_\_\_\_\_\_\_\_\_\_\_\_\_\_\_\_\_\_\_\_\_\_\_\_\_\_\_\_\_\_\_\_\_\_\_\_\_\_\_\_\_\_\_\_\_\_\_\_\_\_\_\_\_\_\_\_\_\_\_\_\_\_\_\_\_\_\_\_\_\_\_\_\_\_\_\_\_\_\_\_\_\_\_\_\_\_\_\_\_\_\_\_\_\_\_\_\_\_\_\_\_\_\_\_\_\_\_\_\_\_\_\_\_\_\_\_\_\_\_\_\_\_\_\_\_\_\_\_\_\_\_\_\_\_\_\_\_\_\_\_\_\_\_\_\_\_\_\_\_\_\_\_\_\_\_\_\_\_\_\_\_\_\_\_\_\_\_\_\_\_\_\_\_\_\_\_\_\_\_\_\_\_\_\_\_\_\_\_\_\_\_\_\_\_\_\_\_\_\_\_\_\_\_\_\_\_\_\_\_\_\_\_\_\_\_\_\_\_\_\_\_\_\_\_\_\_\_\_\_\_\_\_\_\_\_\_\_\_\_\_\_\_\_\_\_\_\_\_\_\_\_\_\_\_\_\_\_\_\_\_\_\_\_\_\_\_\_\_\_\_\_\_\_\_\_\_\_\_\_\_\_\_\_\_\_\_\_\_\_\_\_\_\_\_\_\_\_\_\_\_\_\_\_\_\_\_\_\_\_\_\_\_\_\_\_\_\_\_\_\_\_\_\_\_\_\_\_\_\_\_\_\_\_\_\_\_\_\_\_\_\_\_\_\_\_\_\_\_\_\_\_\_\_\_\_\_\_\_\_\_\_\_\_\_\_\_\_\_\_\_\_\_\_\_\_\_\_\_\_\_\_\_\_\_\_\_\_\_\_\_\_\_\_\_\_\_\_\_\_\_\_\_\_\_\_\_\_\_\_\_\_\_\_\_\_\_\_\_\_\_\_\_\_\_\_\_\_\_\_\_\_\_\_\_\_\_\_\_\_\_\_\_\_\_\_\_\_\_\_\_\_\_\_\_\_\_\_\_\_\_\_\_\_\_\_\_\_\_\_\_\_\_\_\_\_\_\_\_\_\_\_\_\_\_\_\_\_\_\_\_\_\_\_\_\_\_\_\_\_\_\_\_\_\_\_\_\_\_\_\_\_\_\_\_\_\_\_\_\_\_\_\_\_\_\_\_\_\_\_\_\_\_\_\_\_\_\_\_\_\_\_\_\_\_\_\_\_\_\_\_\_\_\_\_\_\_\_\_\_\_\_\_\_\_\_\_\_\_\_\_\_\_\_\_\_\_\_\_\_\_\_\_\_\_\_\_\_\_\_\_\_\_\_\_\_\_\_\_\_\_\_\_\_\_\_\_\_\_\_\_\_\_\_\_\_\_\_\_\_\_\_\_\_\_\_\_\_\_\_\_\_\_\_\_\_\_\_\_\_\_\_\_\_\_\_\_\_\_\_\_\_\_\_\_\_\_\_\_\_\_\_\_\_\_\_\_\_\_\_\_\_\_\_\_\_\_\_\_\_\_\_\_\_\_\_\_\_\_\_\_\_\_\_\_\_\_\_\_\_\_\_\_\_\_\_\_\_\_\_\_\_\_\_\_\_\_\_\_\_\_\_\_\_\_\_\_\_\_\_\_\_\_\_\_\_\_\_\_\_\_\_\_\_\_\_\_\_\_\_\_\_\_\_\_\_\_\_\_\_\_\_\_\_\_\_\_\_\_\_\_\_\_\_\_\_\_\_\_\_\_\_\_\_\_\_\_\_\_\_\_\_\_\_\_\_\_\_\_\_\_\_\_\_\_\_\_\_\_\_\_\_\_\_\_\_\_\_\_\_\_\_\_\_\_\_\_\_\_\_\_\_\_\_\_\_\_\_\_\_\_\_\_\_\_\_\_\_\_\_\_\_\_\_\_\_\_\_\_\_\_\_\_\_\_\_\_\_\_\_\_\_\_\_\_\_\_\_\_\_\_\_\_\_\_\_\_\_\_\_\_\_\_\_\_\_\_\_\_\_\_\_\_\_\_\_\_\_\_\_\_\_\_\_\_\_\_\_\_\_\_\_\_\_\_\_\_\_\_\_\_\_\_\_\_\_\_\_\_\_\_\_\_\_\_\_\_\_\_\_\_\_\_\_\_\_\_\_\_\_\_\_\_\_\_\_\_\_\_\_\_\_\_\_\_\_\_\_\_\_\_\_\_\_\_\_\_\_\_\_\_\_\_\_\_\_\_\_\_\_\_\_\_\_\_\_\_\_\_\_\_\_\_\_\_\_\_\_\_\_\_\_\_\_\_\_\_\_\_\_\_\_\_\_\_\_\_\_\_\_\_\_\_\_\_\_\_\_\_\_\_\_\_\_\_\_\_\_\_\_\_\_\_\_\_\_\_\_\_\_\_\_\_\_\_\_\_\_\_\_\_\_\_\_\_\_\_\_\_\_\_\_\_\_\_\_\_\_\_\_\_\_\_\_\_\_\_\_\_\_\_\_\_\_\_\_\_\_\_\_\_\_\_\_\_\_\_\_\_\_\_\_\_\_\_\_\_\_\_\_\_\_\_\_\_\_\_\_\_\_\_\_\_\_\_\_\_\_\_\_\_\_\_\_\_\_\_\_\_\_\_\_\_\_\_\_\_\_\_\_\_\_\_\_\_\_\_\_\_\_\_\_\_\_\_\_\_\_\_\_\_\_\_\_\_\_\_\_\_\_\_\_\_\_\_\_\_\_\_\_\_\_\_\_\_\_\_\_\_\_\_\_\_\_\_\_\_\_\_\_\_\_\_\_\_\_\_\_\_\_\_\_\_\_\_\_\_\_\_\_\_\_\_\_\_\_\_\_\_\_\_\_\_\_\_\_\_\_\_\_\_\_\_\_\_\_\_\_\_\_\_\_\_\_\_\_\_\_\_\_\_\_\_\_\_\_\_\_\_\_\_\_\_\_\_\_\_\_\_\_\_\_\_\_\_\_\_\_\_\_\_\_\_\_\_\_\_\_\_\_\_\_\_\_\_\_\_\_\_\_\_\_\_\_\_\_\_\_\_\_\_\_\_\_\_\_\_\_\_\_\_\_\_\_\_\_\_\_\_\_\_\_\_\_\_\_\_\_\_\_\_\_\_\_\_\_\_\_\_\_\_\_\_\_\_\_\_\_\_\_\_\_\_\_\_\_\_\_\_\_\_\_\_\_\_\_\_\_\_\_\_\_\_\_\_\_\_\_\_\_\_\_\_\_\_\_\_\_\_\_\_\_\_\_\_\_\_\_\_\_\_\_\_\_\_\_\_\_\_\_\_\_\_\_\_\_\_\_\_\_\_\_\_\_\_\_\_\_\_\_\_\_\_\_\_\_\_\_\_\_\_\_\_\_\_\_\_\_\_\_\_\_\_\_\_\_\_\_\_\_\_\_\_\_\_\_\_\_\_\_\_\_\_\_\_\_\_\_\_\_\_\_\_\_\_\_\_\_\_\_\_\_\_\_\_\_\_\_\_\_\_\_\_\_\_\_\_\_\_\_\_\_\_\_\_\_\_\_\_\_\_\_\_\_\_\_\_\_\_\_\_\_\_\_\_\_\_\_\_\_\_\_\_\_\_\_\_\_\_\_\_\_\_\_\_\_\_\_\_\_\_\_\_\_\_\_\_\_\_\_\_\_\_\_\_\_\_\_\_\_\_\_\_\_\_\_\_\_\_\_\_\_\_\_\_\_\_\_\_\_\_\_\_\_\_\_\_\_\_\_\_\_\_\_\_\_\_\_\_\_\_\_\_\_\_\_\_\_\_\_\_\_\_\_\_\_\_\_\_\_\_\_\_\_\_\_\_\_\_\_\_\_\_\_\_\_\_\_\_\_\_\_\_\_\_\_\_\_\_\_\_\_\_\_\_\_\_\_\_\_\_\_\_\_\_\_\_\_\_\_\_\_\_\_\_\_\_\_\_\_\_\_\_\_\_\_\_\_\_\_\_\_\_\_\_\_\_\_\_\_\_\_\_\_\_\_\_\_\_\_\_\_\_\_\_\_\_\_\_\_\_\_\_\_\_\_\_\_\_\_\_\_\_\_\_\_\_\_\_\_\_\_\_\_\_\_\_\_\_\_\_\_\_\_\_\_\_\_\_\_\_\_\_\_\_\_\_\_\_\_\_\_\_\_\_\_\_\_\_\_\_\_\_\_\_\_\_\_\_\_\_\_\_\_\_\_\_\_\_\_\_\_\_\_\_\_\_\_\_\_\_\_\_\_\_\_\_\_\_\_\_\_\_\_\_\_\_\_\_\_\_\_\_\_\_\_\_\_\_\_\_\_\_\_\_\_\_\_\_\_\_\_\_\_\_\_\_\_\_\_\_\_\_\_\_\_\_\_\_\_\_\_\_\_\_\_\_\_\_\_\_\_\_\_\_\_\_\_\_\_\_\_\_\_\_\_\_\_\_\_\_\_\_\_\_\_\_\_\_\_\_\_\_\_\_\_\_\_\_\_\_\_\_\_\_\_\_\_\_\_\_\_\_\_\_\_\_\_\_\_\_\_\_\_\_\_\_\_\_\_\_\_\_\_\_\_\_\_\_\_\_\_\_\_\_\_\_\_\_\_\_\_\_\_\_\_\_\_\_\_\_\_\_\_\_\_\_\_\_\_\_\_\_\_\_\_\_\_\_\_\_\_\_\_\_\_\_\_\_\_\_\_\_\_\_\_\_\_\_\_\_\_\_\_\_\_\_\_\_\_\_\_\_\_\_\_\_\_\_\_\_\_\_\_\_\_\_\_\_\_\_\_\_\_\_\_\_\_\_\_\_\_\_\_\_\_\_\_\_\_\_\_\_\_\_\_\_\_\_\_\_\_\_\_\_\_\_\_\_\_\_\_\_\_\_\_\_\_\_\_\_\_\_\_\_\_\_\_\_\_\_\_\_\_\_\_\_\_\_\_\_\_\_\_\_\_\_\_\_\_\_\_\_\_\_\_\_\_\_\_\_\_\_\_\_\_\_\_\_\_\_\_\_\_\_\_\_\_\_\_\_\_\_\_\_\_\_\_\_\_\_\_\_\_\_\_\_\_\_\_\_\_\_\_\_\_\_\_\_\_\_\_\_\_\_\_\_\_\_\_\_\_\_\_\_\_\_\_\_\_\_\_\_\_\_\_\_\_\_\_\_\_\_\_\_\_\_\_\_\_\_\_\_\_\_\_\_\_\_\_\_\_\_\_\_\_\_\_\_\_\_\_\_\_\_\_\_\_\_\_\_\_\_\_\_\_\_\_\_\_\_\_\_\_\_\_\_\_\_\_\_\_\_\_\_\_\_\_\_\_\_\_\_\_\_\_\_\_\_\_\_\_\_\_\_\_\_\_\_\_\_\_\_\_\_\_\_\_\_\_\_\_\_\_\_\_\_\_\_\_\_\_\_\_\_\_\_\_\_\_\_\_\_\_\_\_\_\_\_\_\_\_\_\_\_\_\_\_\_\_\_\_\_\_\_\_\_\_\_\_\_\_\_\_\_\_\_\_\_\_\_\_\_\_\_\_\_\_\_\_\_\_\_\_\_\_\_\_\_\_\_\_\_\_\_\_\_\_\_\_\_\_\_\_\_\_\_\_\_\_\_\_\_\_\_\_\_\_\_\_\_\_\_\_\_\_\_\_\_\_\_\_\_\_\_\_\_\_\_\_\_\_\_\_\_\_\_\_\_\_\_\_\_\_\_\_\_\_\_\_\_\_\_\_\_\_\_\_\_\_\_\_\_\_\_\_\_\_\_\_\_\_\_\_\_\_\_\_\_\_\_\_\_\_\_\_\_\_\_\_\_\_\_\_\_\_\_\_\_\_\_\_\_\_\_\_\_\_\_\_\_\_\_\_\_\_\_\_\_\_\_\_\_\_\_\_\_\_\_\_\_\_\_\_\_\_\_\_\_\_\_\_\_\_\_\_\_\_\_\_\_\_\_\_\_\_\_\_\_\_\_\_\_\_\_\_\_\_\_\_\_\_\_\_\_\_\_\_\_\_\_\_\_\_\_\_\_\_\_\_\_\_\_\_\_\_\_\_\_\_\_\_\_\_\_\_\_\_\_\_\_\_\_\_\_\_\_\_\_\_\_\_\_\_\_\_\_\_\_\_\_\_\_\_\_\_\_\_\_\_\_\_\_\_\_\_\_\_\_\_\_\_\_\_\_\_\_\_\_\_\_\_\_\_\_\_\_\_\_\_\_\_\_\_\_\_\_\_\_\_\_\_\_\_\_\_\_\_\_\_\_\_\_\_\_\_\_\_\_\_\_\_\_\_\_\_\_\_\_\_\_\_\_\_\_\_\_\_\_\_\_\_\_\_\_\_\_\_\_\_\_\_\_\_\_\_\_\_\_\_\_\_\_\_\_\_\_\_\_\_\_\_\_\_\_\_\_\_\_\_\_\_\_\_\_\_\_\_\_\_\_\_\_\_\_\_\_\_\_\_\_\_\_\_\_\_\_\_\_\_\_\_\_\_\_\_\_\_\_\_\_\_\_\_\_\_\_\_\_\_\_\_\_\_\_\_\_\_\_\_\_\_\_\_\_\_\_\_\_\_\_\_\_\_\_\_\_\_\_\_\_\_\_\_\_\_\_\_\_\_\_\_\_\_\_\_\_\_\_\_\_\_\_\_\_\_\_\_\_\_\_\_\_\_\_\_\_\_\_\_\_\_\_\_\_\_\_\_\_\_\_\_\_\_\_\_\_\_\_\_\_\_\_\_\_\_\_\_\_\_\_\_\_\_\_\_\_\_\_\_\_\_\_\_\_\_\_\_\_\_\_\_\_\_\_\_\_\_\_\_\_\_\_\_\_\_\_\_\_\_\_\_\_\_\_\_\_\_\_\_\_\_\_\_\_\_\_\_\_\_\_\_\_\_\_\_\_\_\_\_\_\_\_\_\_\_\_\_\_\_\_\_\_\_\_\_\_\_\_\_\_\_\_\_\_\_\_\_\_\_\_\_\_\_\_\_\_\_\_\_\_\_\_\_\_\_\_\_\_\_\_\_\_\_\_\_\_\_\_\_\_\_\_\_\_\_\_\_\_\_\_\_\_\_\_\_\_\_\_\_\_\_\_\_\_\_\_\_\_\_\_\_\_\_\_\_\_\_\_\_\_\_\_\_\_\_\_\_\_\_\_\_\_\_\_\_\_\_\_\_\_\_\_\_\_\_\_\_\_\_\_\_\_\_\_\_\_\_\_\_\_\_\_\_\_\_\_\_\_\_\_\_\_\_\_\_\_\_\_\_\_\_\_\_\_\_\_\_\_\_\_\_\_\_\_\_\_\_\_\_\_\_\_\_\_\_\_\_\_\_\_\_\_\_\_\_\_\_\_\_\_\_\_\_\_\_\_\_\_\_\_\_\_\_\_\_\_\_\_\_\_\_\_\_\_\_\_\_\_\_\_\_\_\_\_\_\_\_\_\_\_\_\_\_\_\_\_\_\_\_\_\_\_\_\_\_\_\_\_\_\_\_\_\_\_\_\_\_\_\_\_\_\_\_\_\_\_\_\_\_\_\_\_\_\_\_\_\_\_\_\_\_\_\_\_\_\_\_\_\_\_\_\_\_\_\_\_\_\_\_\_\_\_\_\_\_\_\_\_\_\_\_\_\_\_\_\_\_\_\_\_\_\_\_\_\_\_\_\_\_\_\_\_\_\_\_\_\_\_\_\_\_\_\_\_\_\_\_\_\_\_\_\_\_\_\_\_\_\_\_\_\_\_\_\_\_\_\_\_\_\_\_\_\_\_\_\_\_\_\_\_\_\_\_\_\_\_\_\_\_\_\_\_\_\_\_\_\_\_\_\_\_\_\_\_\_\_\_\_\_\_\_\_\_\_\_\_\_\_\_\_\_\_\_\_\_\_\_\_\_\_\_\_\_\_\_\_\_\_\_\_\_\_\_\_\_\_\_\_\_\_\_\_\_\_\_\_\_\_\_\_\_\_\_\_\_\_\_\_\_\_\_\_\_\_\_\_\_\_\_\_\_\_\_\_\_\_\_\_\_\_\_\_\_\_\_\_\_\_\_\_\_\_\_\_\_\_\_\_\_\_\_\_\_\_\_\_\_\_\_\_\_\_\_\_\_\_\_\_\_\_\_\_\_\_\_\_\_\_\_\_\_\_\_\_\_\_\_\_\_\_\_\_\_\_\_\_\_\_\_\_\_\_\_\_\_\_\_\_\_\_\_\_\_\_\_\_\_\_\_\_\_\_\_\_\_\_\_\_\_\_\_\_\_\_\_\_\_\_\_\_\_\_\_\_\_\_\_\_\_\_\_\_\_\_\_\_\_\_\_\_\_\_\_\_\_\_\_\_\_\_\_\_\_\_\_\_\_\_\_\_\_\_\_\_\_\_\_\_\_\_\_\_\_\_\_\_\_\_\_\_\_\_\_\_\_\_\_\_\_\_\_\_\_\_\_\_\_\_\_\_\_\_\_\_\_\_\_\_\_\_\_\_\_\_\_\_\_\_\_\_\_\_\_\_\_\_\_\_\_\_\_\_\_\_\_\_\_\_\_\_\_\_\_\_\_\_\_\_\_\_\_\_\_\_\_\_\_\_\_\_\_\_\_\_\_\_\_\_\_\_\_\_\_\_\_\_\_\_\_\_\_\_\_\_\_\_\_\_\_\_\_\_\_\_\_\_\_\_\_\_\_\_\_\_\_\_\_\_\_\_\_\_\_\_\_\_\_\_\_\_\_\_\_\_\_\_\_\_\_\_\_\_\_\_\_\_\_\_\_\_\_\_\_\_\_\_\_\_\_\_\_\_\_\_\_\_\_\_\_\_\_\_\_\_\_\_\_\_\_\_\_\_\_\_\_\_\_\_\_\_\_\_\_\_\_\_\_\_\_\_\_\_\_\_\_\_\_\_\_\_\_\_\_\_\_\_\_\_\_\_\_\_\_\_\_\_\_\_\_\_\_\_\_\_\_\_\_\_\_\_\_\_\_\_\_\_\_\_\_\_\_\_\_\_\_\_\_\_\_\_\_\_\_\_\_\_\_\_\_\_\_\_\_\_\_\_\_\_\_\_\_\_\_\_\_\_\_\_\_\_\_\_\_\_\_\_\_\_\_\_\_\_\_\_\_\_\_\_\_\_\_\_\_\_\_\_\_\_\_\_\_\_\_\_\_\_\_\_\_\_\_\_\_\_\_\_\_\_\_\_\_\_\_\_\_\_\_\_\_\_\_\_\_\_\_\_\_\_\_\_\_\_\_\_\_\_\_\_\_\_\_\_\_\_\_\_\_\_\_\_\_\_\_\_\_\_\_\_\_\_\_\_\_\_\_\_\_\_\_\_\_\_\_\_\_\_\_\_\_\_\_\_\_\_\_\_\_\_\_\_\_\_\_\_\_\_\_\_\_\_\_\_\_\_\_\_\_\_\_\_\_\_\_\_\_\_\_\_\_\_\_\_\_\_\_\_\_\_\_\_\_\_\_\_\_\_\_\_\_\_\_\_\_\_\_\_\_\_\_\_\_\_\_\_\_\_\_\_\_\_\_\_\_\_\_\_\_\_\_\_\_\_\_\_\_\_\_\_\_\_\_\_\_\_\_\_\_\_\_\_\_\_\_\_\_\_\_\_\_\_\_\_\_\_\_\_\_\_\_\_\_\_\_\_\_\_\_\_\_\_\_\_\_\_\_\_\_\_\_\_\_\_\_\_\_\_\_\_\_\_\_\_\_\_\_\_\_\_\_\_\_\_\_\_\_\_\_\_\_\_\_\_\_\_\_\_\_\_\_\_\_\_\_\_\_\_\_\_\_\_\_\_\_\_\_\_\_\_\_\_\_\_\_\_\_\_\_\_\_\_\_\_\_\_\_\_\_\_\_\_\_\_\_\_\_\_\_\_\_\_\_\_\_\_\_\_\_\_\_\_\_\_\_\_\_\_\_\_\_\_\_\_\_\_\_\_\_\_\_\_\_\_\_\_\_\_\_\_\_\_\_\_\_\_\_\_\_\_\_\_\_\_\_\_\_\_\_\_\_\_\_\_\_\_\_\_\_\_\_\_\_\_\_\_\_\_\_\_\_\_\_\_\_\_\_\_\_\_\_\_\_\_\_\_\_\_\_\_\_\_\_\_\_\_\_\_\_\_\_\_\_\_\_\_\_\_\_\_\_\_\_\_\_\_\_\_\_\_\_\_\_\_\_\_\_\_\_\_\_\_\_\_\_\_\_\_\_\_\_\_\_\_\_\_\_\_\_\_\_\_\_\_\_\_\_\_\_\_\_\_\_\_\_\_\_\_\_\_\_\_\_\_\_\_\_\_\_\_\_\_\_\_\_\_\_\_\_\_\_\_\_\_\_\_\_\_\_\_\_\_\_\_\_\_\_\_\_\_\_\_\_\_\_\_\_\_\_\_\_\_\_\_\_\_\_\_\_\_\_\_\_\_\_\_\_\_\_\_\_\_\_\_\_\_\_\_\_\_\_\_\_\_\_\_\_\_\_\_\_\_\_\_\_\_\_\_\_\_\_\_\_\_\_\_\_\_\_\_\_\_\_\_\_\_\_\_\_\_\_\_\_\_\_\_\_\_\_\_\_\_\_\_\_\_\_\_\_\_\_\_\_\_\_\_\_\_\_\_\_\_\_\_\_\_\_\_\_\_\_\_\_\_\_\_\_\_\_\_\_\_\_\_\_\_\_\_\_\_\_\_\_\_\_\_\_\_\_\_\_\_\_\_\_\_\_\_\_\_\_\_\_\_\_\_\_\_\_\_\_\_\_\_\_\_\_\_\_\_\_\_\_\_\_\_\_\_\_\_\_\_\_\_\_\_\_\_\_\_\_\_\_\_\_\_\_\_\_\_\_\_\_\_\_\_\_\_\_\_\_\_\_\_\_\_\_\_\_\_\_\_\_\_\_\_\_\_\_\_\_\_\_\_\_\_\_\_\_\_\_\_\_\_\_\_\_\_\_\_\_\_\_\_\_\_\_\_\_\_\_\_\_\_\_\_\_\_\_\_\_\_\_\_\_\_\_\_\_\_\_\_\_\_\_\_\_\_\_\_\_\_\_\_\_\_\_\_\_\_\_\_\_\_\_\_\_\_\_\_\_\_\_\_\_\_\_\_\_\_\_\_\_\_\_\_\_\_\_\_\_\_\_\_\_\_\_\_\_\_\_\_\_\_\_\_\_\_\_\_\_\_\_\_\_\_\_\_\_\_\_\_\_\_\_\_\_\_\_\_\_\_\_\_\_\_\_\_\_\_\_\_\_\_\_\_\_\_\_\_\_\_\_\_\_\_\_\_\_\_\_\_\_\_\_\_\_\_\_\_\_\_\_\_\_\_\_\_\_\_\_\_\_\_\_\_\_\_\_\_\_\_\_\_\_\_\_\_\_\_\_\_\_\_\_\_\_\_\_\_\_\_\_\_\_\_\_\_\_\_\_\_\_\_\_\_\_\_\_\_\_\_\_\_\_\_\_\_\_\_\_\_\_\_\_\_\_\_\_\_\_\_\_\_\_\_\_\_\_\_\_\_\_\_\_\_\_\_\_\_\_\_\_\_\_\_\_\_\_\_\_\_\_\_\_\_\_\_\_\_\_\_\_\_\_\_\_\_\_\_\_\_\_\_\_\_\_\_\_\_\_\_\_\_\_\_\_\_\_\_\_\_\_\_\_\_\_\_\_\_\_\_\_\_\_\_\_\_\_\_\_\_\_\_\_\_\_\_\_\_\_\_\_\_\_\_\_\_\_\_\_\_\_\_\_\_\_\_\_\_\_\_\_\_\_\_\_\_\_\_\_\_\_\_\_\_\_\_\_\_\_\_\_\_\_\_\_\_\_\_\_\_\_\_\_\_\_\_\_\_\_\_\_\_\_\_\_\_\_\_\_\_\_\_\_\_\_\_\_\_\_\_\_\_\_\_\_\_\_\_\_\_\_\_\_\_\_\_\_\_\_\_\_\_\_\_\_\_\_\_\_\_\_\_\_\_\_\_\_\_\_\_\_\_\_\_\_\_\_\_\_\_\_\_\_\_\_\_\_\_\_\_\_\_\_\_\_\_\_\_\_\_\_\_\_\_\_\_\_\_\_\_\_\_\_\_\_\_\_\_\_\_\_\_\_\_\_\_\_\_\_\_\_\_\_\_\_\_\_\_\_\_\_\_\_\_\_\_\_\_\_\_\_\_\_\_\_\_\_\_\_\_\_\_\_\_\_\_\_\_\_\_\_\_\_\_\_\_\_\_\_\_\_\_\_\_\_\_\_\_\_\_\_\_\_\_\_\_\_\_\_\_\_\_\_\_\_\_\_\_\_\_\_\_\_\_\_\_\_\_\_\_\_\_\_\_\_\_\_\_\_\_\_\_\_\_\_\_\_\_\_\_\_\_\_\_\_\_\_\_\_\_\_\_\_\_\_\_\_\_\_\_\_\_\_\_\_\_\_\_\_\_\_\_\_\_\_\_\_\_\_\_\_\_\_\_\_\_\_\_\_\_\_\_\_\_\_\_\_\_\_\_\_\_\_\_\_\_\_\_\_\_\_\_\_\_\_\_\_\_\_\_\_\_\_\_\_\_\_\_\_\_\_\_\_\_\_\_\_\_\_\_\_\_\_\_\_\_\_\_\_\_\_\_\_\_\_\_\_\_\_\_\_\_\_\_\_\_\_\_\_\_\_\_\_\_\_\_\_\_\_\_\_\_\_\_\_\_\_\_\_\_\_\_\_\_\_\_\_\_\_\_\_\_\_\_\_\_\_\_\_\_\_\_\_\_\_\_\_\_\_\_\_\_\_\_\_\_\_\_\_\_\_\_\_\_\_\_\_\_\_\_\_\_\_\_\_\_\_\_\_\_\_\_\_\_\_\_\_\_\_\_\_\_\_\_\_\_\_\_\_\_\_\_\_\_\_\_\_\_\_\_\_\_\_\_\_\_\_\_\_\_\_\_\_\_\_\_\_\_\_\_\_\_\_\_\_\_\_\_\_\_\_\_\_\_\_\_\_\_\_\_\_\_\_\_\_\_\_\_\_\_\_\_\_\_\_\_\_\_\_\_\_\_\_\_\_\_\_\_\_\_\_\_\_\_\_\_\_\_\_\_\_\_\_\_\_\_\_\_\_\_\_\_\_\_\_\_\_\_\_\_\_\_\_\_\_\_\_\_\_\_\_\_\_\_\_\_\_\_\_\_\_\_\_\_\_\_\_\_\_\_\_\_\_\_\_\_\_\_\_\_\_\_\_\_\_\_\_\_\_\_\_\_\_\_\_\_\_\_\_\_\_\_\_\_\_\_\_\_\_\_\_\_\_\_\_\_\_\_\_\_\_\_\_\_\_\_\_\_\_\_\_\_\_\_\_\_\_\_\_\_\_\_\_\_\_\_\_\_\_\_\_\_\_\_\_\_\_\_\_\_\_\_\_\_\_\_\_\_\_\_\_\_\_\_\_\_\_\_\_\_\_\_\_\_\_\_\_\_\_\_\_\_\_\_\_\_\_\_\_\_\_\_\_\_\_\_\_\_\_\_\_\_\_\_\_\_\_\_\_\_\_\_\_\_\_\_\_\_\_\_\_\_\_\_\_\_\_\_\_\_\_\_\_\_\_\_\_\_\_\_\_\_\_\_\_\_\_\_\_\_\_\_\_\_\_\_\_\_\_\_\_\_\_\_\_\_\_\_\_\_\_\_\_\_\_\_\_\_\_\_\_\_\_\_\_\_\_\_\_\_\_\_\_\_\_\_\_\_\_\_\_\_\_\_\_\_\_\_\_\_\_\_\_\_\_\_\_\_\_\_\_\_\_\_\_\_\_\_\_\_\_\_\_\_\_\_\_\_\_\_\_\_\_\_\_\_\_\_\_\_\_\_\_\_\_\_\_\_\_\_\_\_\_\_\_\_\_\_\_\_\_\_\_\_\_\_\_\_\_\_\_\_\_\_\_\_\_\_\_\_\_\_\_\_\_\_\_\_\_\_\_\_\_\_\_\_\_\_\_\_\_\_\_\_\_\_\_\_\_\_\_\_\_\_\_\_\_\_\_\_\_\_\_\_\_\_\_\_\_\_\_\_\_\_\_\_\_\_\_\_\_\_\_\_\_\_\_\_\_\_\_\_\_\_\_\_\_\_\_\_\_\_\_\_\_\_\_\_\_\_\_\_\_\_\_\_\_\_\_\_\_\_\_\_\_\_\_\_\_\_\_\_\_\_\_\_\_\_\_\_\_\_\_\_\_\_\_\_\_\_\_\_\_\_\_\_\_\_\_\_\_\_\_\_\_\_\_\_\_\_\_\_\_\_\_\_\_\_\_\_\_\_\_\_\_\_\_\_\_\_\_\_\_\_\_\_\_\_\_\_\_\_\_\_\_\_\_\_\_\_\_\_\_\_\_\_\_\_\_\_\_\_\_\_\_\_\_\_\_\_\_\_\_\_\_\_\_\_\_\_\_\_\_\_\_\_\_\_\_\_\_\_\_\_\_\_\_\_\_\_\_\_\_\_\_\_\_\_\_\_\_\_\_\_\_\_\_\_\_\_\_\_\_\_\_\_\_\_\_\_\_\_\_\_\_\_\_\_\_\_\_\_\_\_\_\_\_\_\_\_\_\_\_\_\_\_\_\_\_\_\_\_\_\_\_\_\_\_\_\_\_\_\_\_\_\_\_\_\_\_\_\_\_\_\_\_\_\_\_\_\_\_\_\_\_\_\_\_\_\_\_\_\_\_\_\_\_\_\_\_\_\_\_\_\_\_\_\_\_\_\_\_\_\_\_\_\_\_\_\_\_\_\_\_\_\_\_\_\_\_\_\_\_\_\_\_\_\_\_\_\_\_\_\_\_\_\_\_\_\_\_\_\_\_\_\_\_\_\_\_\_\_\_\_\_\_\_\_\_\_\_\_\_\_\_\_\_\_\_\_\_\_\_\_\_\_\_\_\_\_\_\_\_\_\_\_\_\_\_\_\_\_\_\_\_\_\_\_\_\_\_\_\_\_\_\_\_\_\_\_\_\_\_\_\_\_\_\_\_\_\_\_\_\_\_\_\_\_\_\_\_\_\_\_\_\_\_\_\_\_\_\_\_\_\_\_\_\_\_\_\_\_\_\_\_\_\_\_\_\_\_\_\_\_\_\_\_\_\_\_\_\_\_\_\_\_\_\_\_\_\_\_\_\_\_\_\_\_\_\_\_\_\_\_\_\_\_\_\_\_\_\_\_\_\_\_\_\_\_\_\_\_\_\_\_\_\_\_\_\_\_\_\_\_\_\_\_\_\_\_\_\_\_\_\_\_\_\_\_\_\_\_\_\_\_\_\_\_\_\_\_\_\_\_\_\_\_\_\_\_\_\_\_\_\_\_\_\_\_\_\_\_\_\_\_\_\_\_\_\_\_\_\_\_\_\_\_\_\_\_\_\_\_\_\_\_\_\_\_\_\_\_\_\_\_\_\_\_\_\_\_\_\_\_\_\_\_\_\_\_\_\_\_\_\_\_\_\_\_\_\_\_\_\_\_\_\_\_\_\_\_\_\_\_\_\_\_\_\_\_\_\_\_\_\_\_\_\_\_\_\_\_\_\_\_\_\_\_\_\_\_\_\_\_\_\_\_\_\_\_\_\_\_\_\_\_\_\_\_\_\_\_\_\_\_\_\_\_\_\_\_\_\_\_\_\_\_\_\_\_\_\_\_\_\_\_\_\_\_\_\_\_\_\_\_\_\_\_\_\_\_\_\_\_\_\_\_\_\_\_\_\_\_\_\_\_\_\_\_\_\_\_\_\_\_\_\_\_\_\_\_\_\_\_\_\_\_\_\_\_\_\_\_\_\_\_\_\_\_\_\_\_\_\_\_\_\_\_\_\_\_\_\_\_\_\_\_\_\_\_\_\_\_\_\_\_\_\_\_\_\_\_\_\_\_\_\_\_\_\_\_\_\_\_\_\_\_\_\_\_\_\_\_\_\_\_\_\_\_\_\_\_\_\_\_\_\_\_\_\_\_\_\_\_\_\_\_\_\_\_\_\_\_\_\_\_\_\_\_\_\_\_\_\_\_\_\_\_\_\_\_\_\_\_\_\_\_\_\_\_\_\_\_\_\_\_\_\_\_\_\_\_\_\_\_\_\_\_\_\_\_\_\_\_\_\_\_\_\_\_\_\_\_\_\_\_\_\_\_\_\_\_\_\_\_\_\_\_\_\_\_\_\_\_\_\_\_\_\_\_\_\_\_\_\_\_\_\_\_\_\_\_\_\_\_\_\_\_\_\_\_\_\_\_\_\_\_\_\_\_\_\_\_\_\_\_\_\_\_\_\_\_\_\_\_\_\_\_\_\_\_\_\_\_\_\

\begin{center}
\includegraphics[width=0.3\textwidth]{image1.png} \quad
\includegraphics[width=0.3\textwidth]{image2.png} \quad
\includegraphics[width=0.3\textwidth]{image3.png}
\end{center}

\section*{第十届华为杯全国研究生数学建模竞赛}

\section*{题目:功率放大器非线性特性及预失真建模}

\section*{摘要:}

本文针对功率放大器非线性特性及预失真建模问题进行研究与分析,主要完成以下几部分内容:

针对问题 1-A:分别建立了信号幅值多项式模型、复系数多项式模型、极坐标 Saleh 模型和基于正交三角函数的模型来拟合功放的非线性特性曲线,通过 MATLAB 求解系数的最小二乘解,并求出评价指标参数 NMSE。结果对比表明,四阶复系数多项式模型在实现复杂度和拟合精度方面均具有优越性,其 NMSE 为 $-46.6621 \, \text{dB}$。

针对问题 1-B:首先在一定的约束条件下根据输入信号的幅值范围优化线性放大倍数 $g$。然后根据预失真处理系统的线性约束,将 $1/g$ 倍的功放输出数据 $z(n)/g$ 和功放输入数据 $x(n)$ 分别作为预失真处理器的输入和输出样本。最后根据设定的预失真处理器多项式特性函数,估计出其最小二乘系数,实现预失真模型的建立与求解。对于是否考虑信号的相位信息,我们分别建立了两种多项式特性函数模型。两者的优化结果均给出了不同输入信号幅值范围下的最大幅度放大倍数 $g$,例如给定的测试输入数据的幅值范围 $[0, 1.05]$ 的最大幅度放大倍数均为 $g = 1.8664$;对于四阶多项式预失真模型拟合后信号预失真补偿也均为 $NMSE = -31.6421 \, \text{dB}$。在此基础上,本文还提出了基于输入信号幅值范围的自适应无记忆功放预失真模型,此模型能最大限度地发挥功率放大器的性能。

针对问题 2-A:建立有记忆多项式模型来拟合有记忆功率放大器的非线性特性。为了选取适当的工程实现复杂度和满足一定拟合精度要求的记忆深度 $M$ 以及多项式阶数 $K$,令记忆深度 $M$ 为某定值,通过对比不同阶数多项式的拟合精度得出合适的记忆多项式阶数 $K = 4$;同理得出了合适的记忆多项式记忆深度 $M = 3$。此时该模型的评价指标参数 NMSE 为 $-47.2707 \, \text{dB}$。

针对问题 2-B:首先在一定的约束条件下根据输入信号的幅值范围优化线性放大倍数 $g$。然后根据预失真处理系统的线性约束,将 $1/g$ 倍的功放输出数据 $z(n)/g$ 和功放输入数据 $x(n)$ 分别作为预失真处理器的输入和输出样本。最后根

\begin{center}
- 2 -
\end{center}

据设定的预失真处理器记忆多项式特性函数,估计出其最小二乘系数,实现预失真模型的求解。优化结果给出了不同输入信号幅值范围下的最大幅度放大倍数 \( g \),其中给定的测试输入数据的幅值范围 \([0, 0.7]\) 的最大幅度放大倍数 \( g = 9.6108 \),且对于多项式系数为四阶,记忆深度为三的多项式预失真模型拟合后信号预失真补偿的评价指标参数 NMSE 为 \(-38.4071 \, \text{dB}\)。同时以框图的形式给出了预失真处理模型的实现示意图。在此基础上,本文又提出了基于功率放大器特性的有记忆功放自适应预失真模型以及联合功率放大器特性和输入信号幅值范围的有记忆功放自适应预失真模型。该自适应过程综合考虑了功率放大器特性和最大化利用当前功率放大器特性下的性能,实现实时最大化线性放大倍数 \( g \) 的目的。

针对问题 3:先分别求出所提供的输入信号、输出信号以及问题 2 中求出的预失真处理系统输出信号的自相关函数,从而求出此三种信号的功率谱密度。然后计算出信道的带外失真参数 ACPR 分别为 \(-78.3594 \, \text{dB}\),\(-54.6101 \, \text{dB}\),\(-38.5069 \, \text{dB}\)。可以看出,预失真处理系统的输出带外扩展降低了很多,其相邻信道功率比减小了约 \(16.1 \, \text{dB}\),整个系统性能有很大提升,线性特性也明显加强。

关键词:多项式模型 MATLAB 自适应模型 最小二乘 优化模型 线性特性

\section{问题重述}

功率放大器简称功放,可以用来对信号的功率进行放大,是电子通信系统中的最重要的系统之一,我们希望功放可以进行线性放大。功放的输出信号相对于输入信号可能产生非线性变形,这将带来无益的干扰信号,影响信信息的正确传递和接收,此现象称为非线性失真,这是我们不希望得到的。功放非线性属于有源电子器件的固有特性,研究其机理并采取措施改善,具有重要意义。目前已提出了各种技术来克服改善功放的非线性失真,其中预失真技术是被研究和应用较多的一项新技术,根据功放输出是否与历史输入有关,可以把功放分为无记忆功放与有记忆功放。而本次建模就是要建立在没有预失真的时候,功放的输入与输出模型,以及为了使整个系统达到线性,在功放的前端加上预失真系统,研究预失真系统的输入输出模型。

在题目给定的数据下,求解以下 3 个问题:

1. 根据数据文件 1 给出的复输入-输出测试数据,建立与求解无记忆功放的输入输出数据模型,然后用 NMSE 评价所建模型的准确度,以及加上预失真系统后,根据线性化原则以及“输出幅度限制”和“功率最大化”约束,建立与求解预失真系统的输入输出模型,并且用指标参数 NMSE/EVM 评价预失真补偿的结果。

2. 根据数据文件 2 给出的复输入-输出测试数据,建立与求解有记忆功放的输入输出数据模型,然后用 NMSE 评价所建模型的准确度,以及加上预失真系统后,根据线性化原则以及“输出幅度限制”和“功率最大化”约束,建立与求解预失真系统的输入输出模型,并且用指标参数 NMSE/EVM 评价预失真补偿的结果。

3. 重新考虑题 2,假设所附的数据采样频率 $F_{s}=30.72 \times 12 \, \text{MHz}$,传输信道按照 $20 \, \text{MHz}$,邻信道也是 $20 \, \text{MHz}$。根据给出的数据,请计算功放预失真补偿前后的功率谱密度,并用图形的方式表示三类信号的功率谱密度(输入信号、无预失真补偿的功率放大器输出信号、采用预失真补偿的功率放大器输出信号),最后用 ACPR 对结果进行分析评价。

\section{模型的假设}

1) 假设所提供的功率放大器的输入输出数据符合实际;

2) 假设拟合精度的测量标准 NMSE 值小于 $-30 \, \text{dB}$ 时已经满足要求;

3) 假设功率放大器的特性不随温度、供电以及其他的外界因素变化而变化;

4) 假设功率放大器没有输入时,输出为零;

5) 假设预失真系统部分的输入阻抗为 $0$,输出阻抗为 $\infty$。

\section{符号说明}

\begin{tabular}{ll}
$x(n)$ & 功率放大器的输入信号 \\
$z(n)$ & 功率放大器的输出信号 \\
$r(n)$ & 预失真处理系统的输入信号 \\
$y(n)$ & 预失真处理器的输出信号 \\
$s(n)$ & 预失真处理系统的输出信号 \\
$K$ & 非线性阶数
\end{tabular}

\begin{tabular}{ll} 
$N$ & 序列长度 \\
$M$ & 记忆深度 \\
$g$ & 功放的理想幅度放大倍数 \\
$D$ & 预失真处理系统输入信号的最大幅度 \\
$F(.)$ & 预失真处理器特性函数 \\
$G(.)$ & 功率放大器特性函数 \\
$h_k$ & 功率放大器特性的参数 \\
$F_s$ & 数据采样频率 \\
$p_k$ & 三角函数 k 次谐波正弦系数 \\
$q_k$ & 三角函数 k 次谐波余弦系数 \\
$c_1$ & Saleh 模型的分子参数 \\
$d_1$ & Saleh 模型的分母参数 \\
$\omega$ & 三角函数基波频率 \\
$\Phi_{xx}(m)$ & 信号自相关函数 \\
$p_{xx}(w)$ & 信号功率谱密度 \\
$S(f)$ & 信号频谱函数 \\
\end{tabular}

\section{问题一模型的建立与求解}

\subsection{问题分析}

问题 A 要求我们根据给定的功率放大器的输入输出数据建立无记忆功率放大器的模型。无记忆模型中,其非线性特性主要考虑 AM/AM 变换失真和 AM/PM 变换失真。最常使用的有多项式模型和极坐标 Saleh 模型。我们首先建立不考虑相位变换的多项式模型,然后建立了具有复系数的多项式模型和极坐标 Saleh 模型,最后建立了基于正交三角函数的模型 \cite{ref1}。在模型的求解过程中,我们使用最小二乘算法得出各项的系数,并计算出了对应的 NMSA 来评价各种模型。

问题 B 要求我们根据线性化原则以及“输出幅度限制”和“功率最大化”约束,建立预失真模型。写出目标误差函数,计算线性化后最大可能的幅度放大倍数。建立预失真模型即要求获得预失真处理器特性函数,我们可以首先根据预失真处理器特性设定该特性函数,然后利用所给定的某功放无记忆效应的复输入-输出测试数据进行该函数对应参数的估计。显然该模型的求解过程会涉及到线性化原则,“输出幅度限制”以及“功率最大化”的约束。我们可以在这些约束下建立最大幅度放大倍数的优化模型,获得最大可能的幅度放大倍数,并在此基础上获得预失真模型。最后通过评价指标参数 NMSE/EVM 评价预失真补偿的结果。

\subsection{问题 A 的模型建立与求解}

\subsubsection{模型 1:信号幅值多项式模型}

\begin{equation}
G\big(|x(n)|\big) = \sum_{k=1}^{K} h_k |x(n)|^k
\tag{1}
\end{equation}

根据函数逼近 Weierstrass 定理,对解析函数 $G(x)$ 总可以用一个次数充分大的多项式逼近到任意程度,而且在输入为 0 时输出也应该为 0,故采用没有常数项的多项式模型描述功率放大器的非线性关系。

我们需要找出系数 $h_k$ 使得拟合后的输出与非拟合的输出的均方误差达到最小值,即使得下式最小:
\begin{equation}
\min \left( \sum_{n=0}^{N-1} \left( \sum_{k=1}^{K} h_k |x(n)|^k - z(n) \right)^2 \right)
\tag{2}
\end{equation}

表 1 计算了不同阶数多项式对 $G(x)$ 的逼近程度,并用 NMSE 描述了其逼近程度:

\begin{table}[h]
\centering
\caption{不同阶数不考虑相位的多项式拟合的 NMSE}
\begin{tabular}{c c c c c c c c}
\hline \hline
K 的取值 & 2 & 3 & 4 & 5 & 6 & 7 & 8 \\
\hline
NMSE(dB) & -28.7 & -37.8 & -46.7 & -53.4 & -63.5 & -68.2 & -80.9 \\
\hline \hline
\end{tabular}
\end{table}

从表 1 可以看出多项式的阶数 $K$ 越大,拟合精度越好。但是 $K$ 越大,工程实现的复杂度越大,系统不稳定性也提高了。我们选取 $K=4$,此时 NMSE 值为 -46.6621dB,已经足够小并且阶数越小越易于工程实现。

对于 $K=4$,求得其最小二乘解为:
\begin{equation}
\mathbf{H} = \begin{bmatrix}
\mathbf{h}_1 \\
\mathbf{h}_2 \\
\mathbf{h}_3 \\
\mathbf{h}_4
\end{bmatrix} =
\begin{bmatrix}
2.866 \\
1.4949 \\
-4.1528 \\
1.7
\end{bmatrix}
\end{equation}

写成多项式的形式为:
\begin{equation}
G\big(|x(n)|\big) = 2.866 |x(n)| + 1.4949 |x(n)|^2 - 4.1528 |x(n)|^3 + 1.7 |x(n)|^4
\tag{3}
\end{equation}

非拟合测得的数据与拟合的结果比较如图 1。

\begin{figure}[h]
\centering
\includegraphics[width=0.8\textwidth]{image.png}
\caption{非拟合测得的数据与拟合的结果直观比较图}
\end{figure}

\section*{模型 2:复系数多项式模型}
\begin{equation}
G\big(x(n)\big) = \sum_{k=1}^{K} h_k x(n) |x(n)|^{k-1}
\tag{4}
\end{equation}

此多项式模型对应复输入复输出,根据 Wiener 模型可以得到该模型,该模型与模型 1 具有相同的求解结果,我们只列出 $K=4$ 的情况。

求得参数 $h_k$ 使之满足最小均方误差准则,即根据

\begin{equation}
\min \left( \sum_{n=0}^{N-1} \left( \sum_{k=1}^{K} h_k \, x(n) |x(n)|^{k-1} - y(n) \right)^2 \right)
\end{equation}

求得参数 \( h_k \)。

采用最小二乘求解结果与模型 1 中 \( K=4 \) 时相同。图 2 给出拟合与非拟合的结果数据相位的比较直观示意图:

\begin{figure}[h]
\centering
\includegraphics[width=0.8\textwidth]{image.png}
\caption{拟合与非拟合的结果数据相位的比较}
\end{figure}

\section*{模型 3:极坐标 Saleh 模型}

\begin{equation}
G\big(|x(n)|\big) = \frac{c_1 |x(n)|}{1 + d_1 |x(n)|^2}
\tag{5}
\end{equation}

该模型最初是针对行波管放大器提出来的,因此该模型能够很好地描述行波管的非线性特性,并且形式非常简单。我们依然使用最小均方误差准则求得其未知参数,即求得解满足下式的 \( c_1 \) 和 \( d_1 \):

\begin{equation}
\min \left( \sum_{n=0}^{N-1} \left( \frac{c_1 |x(n)|}{1 + d_1 |x(n)|^2} - y(n) \right)^2 \right)
\tag{6}
\end{equation}

采用最小二乘法求得参数结果为

\begin{equation}
\begin{bmatrix}
c_1 \\
d_1
\end{bmatrix}
=
\begin{bmatrix}
3.1316 \\
0.5927
\end{bmatrix}
\end{equation}

写成表达式的形式如下

\begin{equation}
G\big(|x(n)|\big) = \frac{3.1316 |x(n)|}{1 + 0.5927 |x(n)|^2}
\tag{7}
\end{equation}

利用评价函数,得到 \( NMSE = -33.3057 \, dB \),图 3 是利用该模型得出的拟合输出结果与实际结果数据的直观示意图:

\begin{figure}[h]
\centering
\includegraphics[width=0.8\textwidth]{image2.png}
\caption{拟合输出结果与实际结果数据的直观示意图}
\end{figure}

\begin{figure}[h]
    \centering
    \includegraphics[width=\textwidth]{image.png}
    \caption{Saleh 模型拟合输出结果与实际结果数据的直观示意图}
    \label{fig:3}
\end{figure}

图 3 明显地显示出了很大的误差,与 4 阶多项式模型相比其 NMSE 高 13.36dB,但是此模型由于其复杂度低,易于实现等优点,在早期得到了广泛的应用。

\textbf{模型 4:基于正交三角函数的模型}

\begin{equation}
G\big(|x(n)|\big) = \frac{p_0}{2} + \sum_{k=1}^{M} \big(p_k \sin(k\omega |x(n)|) + q_k \cos(k\omega |x(n)|)\big)
\tag{8}
\end{equation}

该模型基于正交三角函数,由于其正交性,所以常被用来分解连续函数,要求寻找满足最小均方误差准则的参数解,即寻找满足下式的参数解

\begin{equation}
\min \left( \sum_{n=0}^{N-1} \left( \frac{p_0}{2} + \sum_{k=1}^{M} \big(p_k \sin(k\omega |x(n)|) + q_k \cos(k\omega |x(n)|)\big) - y(n) \right)^2 \right)
\tag{9}
\end{equation}

使用最小二乘求得不同阶数的正交三角函数的解,并求得对应的 NMSE 值如表 2 所示。

\begin{table}[h]
\centering
\caption{不同阶数的正交三角函数拟合的 NMSE}
\begin{tabular}{c c c c}
\hline \hline
K 的取值 & 1 & 2 & 3 \\
\hline
NMSE (dB) & -40.1223 & -47.1223 & -73.2658 \\
\hline \hline
\end{tabular}
\end{table}

在 $K=1$ 时,要求解 4 个参数,每当 $K$ 的取值增加 1 就增加 2 个参数 $a_k$ 和 $b_k$,相同的参数个数相对于多项式模型性能略差。如 $K=1$ 时,4 个参数的 NMSE 为 -40.1223dB,而 4 个参数对应的多项式的阶数为 4,其对应的 NMSE 为 -46.6621dB,所以在工程中不建议使用此模型。

当 $K=1$ 时,求得其对应的未知参数如下:

\begin{equation}
\begin{bmatrix}
p_0 \\
p_1 \\
q_1 \\
\omega
\end{bmatrix}
=
\begin{bmatrix}
0.3876 \\
-0.1944 \\
1.6918 \\
1.7796
\end{bmatrix}
\end{equation}

写成表达式为:

\begin{equation}
G\big(|x(n)|\big) = 0.1938 + 1.6918 \sin(1.7796 |x(n)|) - 0.1944 \cos(1.7796 |x(n)|)
\tag{10}
\end{equation}

图 4 是利用该模型得出的拟合输出结果与实际结果数据的直观示意图:

\begin{figure}[h]
    \centering
    \includegraphics[width=\textwidth]{image2.png}
    \caption{利用正交三角函数模型得出的拟合输出结果与实际结果数据的直观示意图}
    \label{fig:4}
\end{figure}

\begin{figure}[h]
    \centering
    \includegraphics[width=0.8\textwidth]{image.png}
    \caption{正交三角函数模型拟合输出结果与实际结果数据的直观示意图}
    \label{fig:4}
\end{figure}

将以上四种模型的参数个数和 NMSA 值列出如表 3:

\begin{table}[h]
    \centering
    \caption{各种模型的 NMSE 与参数个数对比}
    \label{tab:3}
    \begin{tabular}{c c c c c}
        \hline
        模型 & 四阶多项式 & 4 阶 Wiener 多项式 & Saleh 型 & 一阶三角函数 \\
        \hline
        NMSE (dB) & -46.6621 & -46.6621 & -33.3057 & -40.1223 \\
        参数个数 & 4 & 4 & 2 & 4 \\
        \hline
    \end{tabular}
\end{table}

从表 3 可以看出四阶多项式模型和四阶 Wiener 多项式模型相比于其余两个模型具有较好的拟合精度,但四阶 Wiener 多项式模型还考虑了相位拟合,且其工程时间复杂度低,系统较稳定,所以我们确定此模型是以上四种模型中的最优选择。

\subsection{问题 B 模型的建立与求解}

\subsubsection{考虑信号相位信息的无记忆功放预失真模型}

\subsubsection{考虑信号相位信息的无记忆功放预失真模型的建立}

假设预失真技术的原理框图如图 5 所示:

\begin{figure}[h]
    \centering
    \includegraphics[width=0.8\textwidth]{image2.png}
    \caption{预失真技术原理框图}
    \label{fig:5}
\end{figure}

其中功率放大器 (PA) 特性可用给定的复输入-输出测试数据 $x(n)$ 和 $y(n)$ 进行描述,如图 6 所示:

\begin{figure}[h]
    \centering
    \includegraphics[width=0.5\textwidth]{image3.png}
    \caption{功率放大器输入-输出传输框图}
    \label{fig:6}
\end{figure}

假设预失真处理系统呈现线性,且理想的幅度放大倍数为 $g$,则有:

\begin{equation}
    r(n) = \frac{s(n)}{g}
    \tag{11}
\end{equation}

假设功率放大器特性函数 $G()$ 为可逆的,且逆函数用 $G^{-1}()$ 表示,则有:
\begin{equation}
y(n) = G^{-1}(s(n))
\tag{12}
\end{equation}
那么,此时的预失真处理器的输入-输出传输如图 7 所示:

\begin{figure}[h]
\centering
\includegraphics[width=0.8\textwidth]{image7.png}
\caption{预失真处理器的输入-输出传输框图 A}
\end{figure}

如果 $s(n)$ 用功率放大器的输出 $z(n)$ 表示,即 $s(n) = z(n)$,则:
\begin{equation}
G^{-1}(s(n)) = G^{-1}(z(n)) = x(n)
\tag{13}
\end{equation}
那么,此时的预失真处理器的输入-输出传输图如图 8 所示:

\begin{figure}[h]
\centering
\includegraphics[width=0.8\textwidth]{image8.png}
\caption{预失真处理器的输入-输出传输框图 B}
\end{figure}

假设预失真处理器特性函数为 $F()$,则根据图 8 所示有如下函数关系:
\begin{equation}
x(n) = F\left(\frac{z(n)}{g}\right)
\tag{14}
\end{equation}

由于题目中已经给定了 $x(n)$ 和 $z(n)$ 的测试数据,因此可以根据这两组测试数据对预失真处理器特性函数 $F()$ 进行拟合。

现给出无记忆功放预失真模型的计算框图如图 9 所示:

\begin{figure}[h]
\centering
\includegraphics[width=0.8\textwidth]{image9.png}
\caption{无记忆功放预失真模型计算框图}
\end{figure}

考虑无记忆多项式预失真处理器的数学描述如下:
\begin{equation}
y(n) = F(r(n)) = \sum_{k=1}^{K} a_k r(n) |r(n)|^{k-1} \quad n = 0, 1, 2, \dots, N-1
\tag{15}
\end{equation}
根据式 (14),有如下的函数关系:
\begin{equation}
x(n) = F\left(\frac{z(n)}{g}\right) = \sum_{k=1}^{K} a_k \frac{z(n)}{g} \left|\frac{z(n)}{g}\right|^{k-1} \quad n = 0, 1, 2, \dots, N-1
\tag{16}
\end{equation}

此处,令 \( u_k(n) = \frac{z(n)}{g} \left| \frac{z(n)}{g} \right|^{k-1} \),则有:
\begin{equation}
x(n) = \sum_{k=1}^{K} a_k u_k(n) \quad n = 0, 1, 2, \ldots, N-1
\tag{17}
\end{equation}

将式(17)式写成矩阵形式,
\begin{equation}
\mathbf{x} = \mathbf{U} \mathbf{a}
\tag{18}
\end{equation}
其中,\(\mathbf{x} = \left[ x(0), \ldots, x(N-1) \right]^T\),\(\mathbf{U} = \left[ \mathbf{u}_1, \mathbf{u}_2, \ldots, \mathbf{u}_K \right]\),\(\mathbf{u}_k = \left[ u_k(0), \ldots, u_k(N-1) \right]^T\),\(\mathbf{a} = \left[ a_1, \ldots, a_K \right]^T\)。

建立目标误差函数:
\begin{equation}
\min \sum_{n=0}^{N-1} \left| e_n \right|^2 = \sum_{n=0}^{N-1} \left| x(n) - \hat{x}(n) \right|^2 = \sum_{n=0}^{N-1} \left| x(n) - \sum_{k=1}^{K} \hat{a}_k \frac{z(n)}{g} \left| \frac{z(n)}{g} \right|^{k-1} \right|^2
\tag{19}
\end{equation}
在此目标函数下,式(18)的最小二乘解为:
\begin{equation}
\hat{\mathbf{a}} = (\mathbf{U}^H \mathbf{U})^{-1} \mathbf{U}^H \mathbf{x}
\tag{20}
\end{equation}

### 4.3.1.2 考虑信号相位信息的无记忆功放预失真模型的求解

#### 步骤一:幅度放大倍数 \( g \) 的确定

考虑到预失真处理输出幅度限制和功率最大化约束,建立如下关于 \( g \) 的优化模型:

\begin{equation}
\begin{aligned}
& \max \quad g \quad \quad (0 \leq |r(n)| \leq D \leq D_{\max}) \\
\text{s.t.} \quad & \left| F(r(n)) \right| = \left| G^{-1}(s(n)) \right| = \left| G^{-1}(g \bullet r(n)) \right| \leq \max \left\{ \left| x(0) \right|, \ldots, \left| x(N-1) \right| \right\} \\
& \left| s(n) \right| = \left| g \bullet r(n) \right| \leq \max \left\{ \left| z(n) \right|, \ldots, \left| z(N-1) \right| \right\} \\
& g \geq \frac{\max \left( \left| z(n) \right| \right)}{\max \left( \left| x(n) \right| \right)}
\end{aligned}
\tag{21}
\end{equation}

其中,\( D \) 为预失真处理系统输入信号的最大幅值。\( D_{\max} \) 为 \( D \) 能取到的最大值。

上述优化模型中的第一个约束条件是对预失真处理输出幅度的限制,其输出幅度不能超过功放输入的最大值。第二个约束是对预失真处理整个系统的输出幅度的限制。受到功率放大器自身性能的约束,功放输出的最大值限定了整个系统的输出最大值。第三个约束考虑到预失真处理后系统的功率放大性能应该优于功放的功率放大性能,即线性幅度放大倍数 \( g \) 应该不小于功放的最低放大倍数,即 \( \frac{\max \left( \left| z'(n) \right| \right)}{\max \left( \left| x'(n) \right| \right)} \),此约束决定了 \( D_{\max} \) 的取值。

要实现对该优化模型的求解,首先要获得功放特性函数 $G(\bullet)$ 的逆函数 $G^{-1}(\bullet)$。假设这个逆是一个 $K$ 阶的多项式。

\begin{equation}
x(n) = G^{-1}(z(n)) = \sum_{k=1}^{K} b_k z(n) \left| z(n) \right|^{k-1} \quad n = 0, 1, 2, \cdots, N-1
\tag{22}
\end{equation}

利用给定的功放输入信号样本值 $\{x(n), \, n = 0, 1, 2, \cdots, N-1\}$ 和对应的功放输出样本值 $\{z(n), \, n = 0, 1, 2, \cdots, N-1\}$,对 (22) 式进行最小二乘拟合,此处取 $K = 4$ 进行逆函数 $G^{-1}(\bullet)$ 的拟合,得到逆多项式的系数为:

\begin{equation}
B = [0.2677 \, 0.3082 \, -0.4027 \, 0.1598]
\tag{23}
\end{equation}

因此,$G$ 函数的逆多项式为:

\begin{equation}
x(n) = 0.2677 z(n) + 0.3082 z(n) \left| z(n) \right| - 0.4027 z(n) \left| z(n) \right|^2 + 0.1598 z(n) \left| z(n) \right|^3
\tag{24}
\end{equation}

此时信号幅值的 $NMSE = -30.6464 \, dB$,已经满足相应的精度要求。

至此,我们可以获得对于不同的预失真处理系统输入信号区间所对应的最大幅度放大倍数 $g$。

\begin{table}[h]
\centering
\caption{给定信号输入范围下的最大幅度放大倍数}
\begin{tabular}{c c c c c c}
\hline
输入信号取值范围 & $[0 \, 0.3]$ & $[0 \, 0.5]$ & $[0 \, 0.7]$ & $[0 \, 0.9]$ & $[0 \, 1.05]$ \\
\hline
最大幅度放大倍数 $g$ & 6.4248 & 3.8549 & 2.7535 & 2.1416 & 1.8664 \\
\hline
\end{tabular}
\end{table}

由于 $\frac{\max(|z'(n)|)}{\max(|x'(n)|)} \approx 1.8265$,所以信号输入范围的最大值取值为:$D_{\max} = \max(x(n)) = 1.05$。

\textbf{步骤二:给出预失真处理器特性函数}

假设预失真处理器特性函数如式 (15) 所示的 $K$ 阶多项式函数。

\textbf{步骤三:}

此处取预失真特性函数的阶数为 $K = 4$,$g = 1.8694$,利用给定的 $\frac{z(n)}{g}$ 和 $x(n)$ 作为预失真处理器输入和输出样本,利用最小二乘法估计预失真函数特性参数。$F$ 函数的拟合结果如式 (25) 所示:

\begin{equation}
y(n) = 0.4997 r(n) + 1.0734 r(n) \left| r(n) \right| - 2.6181 r(n) \left| r(n) \right|^2 + 1.9386 r(n) \left| r(n) \right|^3
\tag{25}
\end{equation}

分别考虑信号幅度和相位,仿真预失真函数 $F(\bullet)$ 对信号幅度的输入输出图和对信号相位的输入输出图。此处取信号的输入范围为 $[0 \, 1.05]$,$g = 1.8664$,预失真函数为四阶多项式。

\begin{figure}[h]
    \centering
    \includegraphics[width=\textwidth]{image1.png}
    \caption{F函数分别对信号幅值和相位的作用图}
    \label{fig:10}
\end{figure}

分别考虑信号幅度和相位,仿真经过预失真处理后的功率放大器的输入输出间的关系图。此处取信号的输入范围为 $[0 \quad 1.05]$,$g=1.8664$。

\begin{figure}[h]
    \centering
    \includegraphics[width=\textwidth]{image2.png}
    \caption{预失真处理后系统对信号幅值和相位的作用图}
    \label{fig:11}
\end{figure}

此时的信号幅值的预失真补偿 $NMSE = -31.6421dB$,信号相位的预失真补偿 $NMSE = -157.2874dB$,信号预失真补偿 $NMSE = -31.6421dB$。

由表4可知,对于不同的信号输入范围,具有不同的最大幅值放大倍数 $g$,因此也就对应着不同的F函数,图12显示了不同信号输入范围的 $F$ 函数的设定。从图12中可以直观的发现不同信号输入区间的 $F$ 函数作用后的输出值均不大于功率放大器所允许的最大输入值 1.05。

\begin{figure}[h]
    \centering
    \includegraphics[width=\textwidth]{image3.png}
    \caption{不同信号输入范围的 $F$ 函数设定}
    \label{fig:12}
\end{figure}

\begin{figure}[h]
    \centering
    \includegraphics[width=\textwidth]{image1.png}
    \caption{F函数对不同输入范围信号幅值的作用图}
    \label{fig:12}
\end{figure}

\begin{figure}[h]
    \centering
    \includegraphics[width=\textwidth]{image2.png}
    \caption{预失真处理后系统对不同输入范围信号幅值的作用图}
    \label{fig:13}
\end{figure}

图 \ref{fig:13} 显示的是对于不同的输入信号范围的预失真处理系统的特性函数图。从图四可以直观的看出各输入区间所对应的预失真处理系统均呈现出线性特性,且输入信号的区间越小,则最大线性幅值放大倍数就越大。由此,4.3.2 给出了基于输入信号幅值范围的自适应无记忆功放预失真模型。

\subsection{4.3.2 基于输入信号幅值范围的自适应无记忆功放预失真模型}

根据 4.3.1 节的分析可知,不同的信号输入范围总能找到当前信号输入范围下的最优预失真函数 $F(\bullet)$,从而最大化当前时刻的线性放大系数 $g$。因此我们可以根据前 $N$ 时刻的输入信号幅值自适应的调整预失真处理器的参数,实时调整线性放大系数 $g$。具体的实现框图如图 14 所示。

\begin{figure}[h]
    \centering
    \includegraphics[width=\textwidth]{image3.png}
    \caption{实现框图}
    \label{fig:14}
\end{figure}

\begin{figure}[h]
    \centering
    \includegraphics[width=\textwidth]{image.png}
    \caption{基于输入信号幅值范围的自适应无记忆功放预失真模型框图}
    \label{fig:14}
\end{figure}

假设缓存器缓存了前 $N$ 时刻的输入信号 $\{r(n), n=0,\dots,N-1\}$,预失真处理器参数训练模块根据前 $N$ 时刻的输入信号最大值 $\mathbf{D}=\max(r(n))$ 确定当前时刻的信号输入范围 $[0, \max(r(n))]$,根据 4.3.1 节介绍的 $F$ 函数计算方法获得当前时刻下的最优 $F$ 函数参数估计值,自适应的调整预失真处理器中的参数,使得当前时刻为最大线性幅值放大倍数 $g$。

运用此自适应过程能在功率放大器性能限定下最大化输出信号幅值。使得功放输出信号幅值能达到功率放大器最大的允许输出最大值,极大限度的使用功率放大器的性能。

\subsection{4.3.3 不考虑信号相位信息的无记忆功放预失真模型}

\subsubsection{4.3.3.1 不考虑信号相位信息的无记忆功放预失真模型的建立}

假设预失真处理系统呈现线性,且理想的幅度放大倍数为 $g$,则有:

\begin{equation}
|r(n)| = \left| \frac{s(n)}{g} \right|
\tag{26}
\end{equation}

假设功率放大器特性函数 $G()$ 为可逆的,且逆函数用 $G^{-1}()$ 表示,则有:

\begin{equation}
|y(n)| = G^{-1}(|s(n)|)
\tag{27}
\end{equation}

那么,此时的预失真处理器的输入-输出传输如图 \ref{fig:15} 所示:

\begin{figure}[h]
    \centering
    \includegraphics[width=\textwidth]{image2.png}
    \caption{预失真处理器的输入-输出传输框图 A}
    \label{fig:15}
\end{figure}

如果 $s(n)$ 用功率放大器的输出 $z(n)$ 表示,即 $s(n)=z(n)$,则:

\begin{equation}
G^{-1}(|s(n)|) = G^{-1}(|z(n)|) = |x(n)|
\tag{28}
\end{equation}

那么,此时的预失真处理器的输入-输出传输图如图 16 所示:

\begin{figure}[h]
    \centering
    \includegraphics[width=\textwidth]{image.png}
    \caption{预失真处理器的输入-输出传输框图B}
    \label{fig:16}
\end{figure}

假设预失真处理器特性函数为 $F()$,则根据图 \ref{fig:16} 所示有如下函数关系:

\begin{equation}
|x(n)| = F\left(\left|\frac{z(n)}{g}\right|\right)
\tag{29}
\end{equation}

由于题目中已经给定了 $x(n)$ 和 $z(n)$ 的测试数据,因此可以根据这两组测试数据对预失真处理器特性函数 $F()$ 进行拟合。

现给出无记忆功放预失真模型的计算框图如图 \ref{fig:17} 所示:

\begin{figure}[h]
    \centering
    \includegraphics[width=\textwidth]{image2.png}
    \caption{无记忆功放预失真模型计算框图}
    \label{fig:17}
\end{figure}

考虑无记忆多项式预失真处理器的数学描述如下:

\begin{equation}
|y(n)| = F(|r(n)|) = \sum_{k=1}^{K} a_k |r(n)|^k \quad n=0,1,2,\dots,N-1
\tag{30}
\end{equation}

根据式 (29),有如下的函数关系:

\begin{equation}
|x(n)| = F\left(\left|\frac{z(n)}{g}\right|\right) = \sum_{k=1}^{K} a_k \left|\frac{z(n)}{g}\right|^k \quad n=0,1,2,\dots,N-1
\tag{31}
\end{equation}

此处,令 $u_k(n) = \left|\frac{z(n)}{g}\right|^k$,则有:

\begin{equation}
|x(n)| = \sum_{k=1}^{K} a_k u_k(n) \quad n=0,1,2,\dots,N-1
\tag{32}
\end{equation}

将式 (32) 式写成矩阵形式,

\begin{equation}
\mathbf{x} = \mathbf{Ua}
\tag{33}
\end{equation}

其中,$\mathbf{x} = \left[|x(0)|, \dots, |x(N-1)|\right]^T$,$\mathbf{U} = \left[\mathbf{u}_1, \mathbf{u}_2, \dots, \mathbf{u}_K\right]$,

$\mathbf{u}_k = \left[u_k(0), \dots, u_k(N-1)\right]^T$,$\mathbf{a} = \left[a_1, \dots, a_K\right]^T$。

建立目标误差函数:

\begin{equation}
\min \sum_{n=0}^{N-1} |e_n|^2 = \sum_{n=0}^{N-1} \left||x(n)| - |\hat{x}(n)|\right|^2 = \sum_{n=0}^{N-1} \left||x(n)| - \sum_{k=1}^{K} \hat{a}_k \left|\frac{z(n)}{g}\right|^k\right|^2
\tag{34}
\end{equation}

在此目标函数下,式(33)的最小二乘解为:
\begin{equation}
\hat{\mathbf{a}} = (\mathbf{U}^H \mathbf{U})^{-1} \mathbf{U}^H \mathbf{x}
\tag{35}
\end{equation}

### 4.3.3.2 不考虑信号相位信息的无记忆功放预失真模型的求解

#### 步骤一:幅度放大倍数 \( g \) 的确定

考虑到预失真处理输出幅度限制和功率最大化约束,建立如下关于 \( g \) 的优化模型:

\begin{equation}
\begin{aligned}
& \max \quad g \quad \quad (0 \leq |r(n)| \leq D \leq D_{\max}) \\
& \text{s.t.} \left\{
\begin{aligned}
& |F(|r(n)|)| = |G^{-1}(|s(n)|)| = |G^{-1}(|g \cdot r(n)|)| \leq \max \left\{|x(0)|, \ldots, |x(N-1)|\right\} \\
& |s(n)| = |g \cdot r(n)| \leq \max \left\{|z(n)|, \ldots, |z(N-1)|\right\} \\
& g \geq \frac{\max (|z(n)|)}{\max (|x(n)|)}
\end{aligned}
\right.
\tag{36}
\end{aligned}
\end{equation}

其中,\( D \) 为预失真处理系统输入信号的最大幅值。\( D_{\max} \) 为 \( D \) 能取到的最大值。

式(36)中各约束条件的物理解释和 4.3.1.2 中的优化模型的解释一致,此处不再赘述。

同样要实现对该优化模型的求解,首先要获得功放特性函数 \( G(\bullet) \) 的逆函数 \( G^{-1}(\bullet) \)。假设这个逆是一个 \( K \) 阶的多项式。

\begin{equation}
|x(n)| = G^{-1}(|z(n)|) = \sum_{k=1}^{K} b_k |z(n)|^k \quad \quad n = 0, 1, 2, \cdots, N-1
\tag{37}
\end{equation}

利用给定的功放输入信号样本值 \( \{x(n), \, n = 0, 1, 2, \cdots, N-1\} \) 和对应的功放输出样本值 \( \{z(n), \, n = 0, 1, 2, \cdots, N-1\} \),对(37)式进行最小二乘拟合,此处取 \( K = 4 \) 进行逆函数 \( G^{-1}(\bullet) \) 的拟合,得到逆多项式的系数为:

\begin{equation}
B = [0.2677 \, 0.3082 \, -0.4027 \, 0.1598]
\tag{38}
\end{equation}

因此,\( G \) 函数的逆多项式为:

\begin{equation}
|x(n)| = 0.2677 |z(n)| + 0.3082 |z(n)|^2 - 0.4027 |z(n)|^3 + 0.1598 |z(n)|^4
\tag{39}
\end{equation}

此时信号幅值的 \( NMSE = -30.6464 \, dB \) 已经满足相应的精度要求。

至此,我们可以获得对于不同的预失真处理系统输入信号区间所对应的最大幅度放大倍数 \( g \)。

\begin{table}
\centering
\caption{给定信号输入范围下的最大幅度放大倍数}
\begin{tabular}{c c c c c c}
\hline
输入信号取值范围 & $[0$ & $0.3]$ & $[0$ & $0.5]$ & $[0$ & $0.7]$ & $[0$ & $0.9]$ & $[0$ & $1.05]$ \\
\hline
最大幅度放大倍数 & 6.4248 & & 3.8549 & & 2.7535 & & 2.1416 & & 1.8664 & \\
$g$ & & & & & & & & & & \\
\hline
\end{tabular}
\end{table}

由于 $\frac{\max(|z'(n)|)}{\max(|x'(n)|)} \approx 1.8265$,所以信号输入范围的最大值取值为: $D_{\max} = \max(x(n)) = 1.05$。

\textbf{步骤二:给出预失真处理器特性函数}

假设预失真处理器特性函数如式(30)所示的 $K$ 阶多项式函数。

\textbf{步骤三:}

此处取预失真特性函数的阶数为 $K=4$,$g=1.8694$,利用给定的 $\left|\frac{z(n)}{g}\right|$ 和 $|x(n)|$ 作为预失真处理器输入和输出样本,利用最小二乘法估计预失真函数特性参数。$F$ 函数的拟合结果如式(40)所示:

\begin{equation}
|y(n)| = 0.4997|r(n)| + 1.0734|r(n)|^2 - 2.6181|r(n)|^3 + 1.9386|r(n)|^4
\tag{40}
\end{equation}

仿真预失真函数 $F(\cdot)$ 对信号幅度的输入输出图以及经过预失真处理后的功率放大器的输入输出间的关系图。此处取信号的输入范围为 $[0 \quad 1.05]$,$g=1.8664$,预失真函数为四阶多项式。

\begin{figure}[h]
\centering
\includegraphics[width=\textwidth]{image.png}
\caption{$F$ 函数以及预失真处理后系统对信号幅值的作用}
\end{figure}

此时的信号幅值的预失真补偿 $NMSE = -31.6421 \, dB$。

由第 4.2 节可知,由于所给定的功放输入信号样本值 $\{x(n), n=0, \ldots, N-1\}$ 和对应的功放输出样本值 $\{z(n), n=0, 1, 2, \ldots, N-1\}$ 只在幅值上存在较大的非线性失真,而相位上的非线性失真很小可以忽略,所以由这两组数据进行仿真的考虑信号相位的模型和不考虑信号相位的模型的结果是一致的。

由于基于输入信号幅值范围的自适应无记忆功放预失真模型包含了考虑信号相位信息和不考虑信号相位信息两种情况,所以此处不再赘述。

\section{问题二模型的建立与求解}

\subsection{问题分析}

问题 A 要求我们根据给出的复输入和复输出数据建立功率放大器的特性模型,并使用 NMSA 评价标准分析。我们建立一种记忆多项式的模型 \cite{ref2}\cite{ref3},首先使得记忆深度 \(M\) 确定,根据不同阶数记忆多项式对应的 NMSA 确定多项式的阶数;然后多项式的阶数确定,根据不同记忆深度多项式对应的 NMSA 确定多项式的记忆深度,由此选择合适的多项式的阶数 \(K\) 和记忆深度 \(M\)。

问题 B 要求我们根据线性化原则以及“输出幅度限制”和“功率最大化”约束,以框图的方式建立预失真处理的模型实现示意图,计算预失真模型相关参数。建立预失真模型即要求获得预失真处理器特性函数,我们可以首先根据有记忆预失真处理器特性 \cite{ref4}\cite{ref5} 设定该特性函数,然后利用所给定的某功放有记忆效应的复输入-输出测试数据进行该函数对应参数的估计。显然该模型的求解过程会涉及到线性化原则,“输出幅度限制”以及“功率最大化”的约束。我们可以在这些约束下建立最大幅度放大倍数的优化模型,获得最大可能的幅度放大倍数,并在此基础上获得预失真模型。最后通过评价指标参数 NMSE/EVM 评价预失真补偿的结果。并以框图的方式给出预失真处理的模型实现示意图。

\subsection{问题 A 模型的建立与求解}

\subsubsection{有记忆功率放大器模型的建立}

对于有记忆的功率放大器,此时的输出不仅与当前的输入有关,而且与之前的 \(M\) 个输入有关,常用 Kim 等人提出的具有记忆效应的“和记忆多项式”来表示功率放大器的动态的非线性,其表达式如下:

\begin{equation}
z(n) = \sum_{k=1}^{K} \sum_{m=0}^{M} h_{km} x(n-m) |x(n-m)|^{k-1}
\tag{41}
\end{equation}

其中,\(K\) 为多项式的阶数,\(M\) 表示记忆深度。由上式可知,如果输入为复信号,那么输出必是复信号。将 \(z(n)\) 的系数 \(h_{km}\) 写成矩阵 \(\mathbf{H}\) 的形式如下

\begin{equation}
\mathbf{H} =
\begin{pmatrix}
h_{10} & h_{11} & \cdots & h_{1M} \\
h_{20} & h_{21} & \cdots & h_{2M} \\
\vdots & \vdots & \ddots & \vdots \\
h_{K0} & h_{K1} & \cdots & h_{KM}
\end{pmatrix}
\tag{42}
\end{equation}

(41) 式所述的记忆多项式模型可用图 19 描述:

\begin{figure}[h]
    \centering
    \includegraphics[width=0.8\textwidth]{image.png}
    \caption{记忆多项式模型的结构}
    \label{fig:memory_polynomial_structure}
\end{figure}

令 $f_{m}(|x(n-m)|)=\sum_{k=1}^{K}h_{km}|x(n-j)|^{k-1}$,则:

\begin{equation}
z(n)=\sum_{m=0}^{M}x(n-m)f_{m}(|x(n-m)|)
\tag{43}
\end{equation}

本题中给出了随机的复输入信号 $x(n)$ 和实际测得的输出信号 $y(n)$。我们建立的模型需要尽量使得拟合的功率放大器特性与实际测得的特性符合,选用满足最小均方误差准则的 $h_{km}$,即选择 $h_{k,m}$ 使得下式最小

\begin{equation}
\min\left(\sum_{i=1}^{N}\left(y(n)-\hat{z}(n)\right)^{2}\right)
\tag{44}
\end{equation}

其中,$\hat{z}(n)$ 是拟合的功率放大器的同样的输入信号 $x(n)$ 对应的输出信号,$N$ 为输入信号的个数。

注意:对于式 (43) 中的 $x(l)$,当 $l<0$ 时,此时无输入,即 $x(l)=0$。

\subsection{有记忆模型放大器记忆多项式模型的最优解}

首先我们选取 $M=3$,$K=2,3,4,5$ 建立模型,通过曲线的拟合,并分别求得其对应的幅度和相位的均方归一化误差 NMSE,列出表 6 如下

\begin{table}[h]
\centering
\caption{$M=3$ 时,不同取值 $K$ 下幅度和相位的 NMSE}
\label{tab:NMSE_M3}
\begin{tabular}{c c c c c}
\hline \hline
$K$ 的取值 & 2 & 3 & 4 & 5 \\
\hline
幅度的 NMSE & $-35.1844$dB & $-45.5887$dB & $-47.2707$dB & $-47.3891$dB \\
\hline
相位的 NMSE & $-12.0435$dB & $-12.8775$dB & $-13.0244$dB & $-12.9701$dB \\
\hline \hline
\end{tabular}
\end{table}

由表 6 可以看出,相位的 NMSE 基本上保持不变,所以我们只讨论幅度的 NMSE 来选择 $K$ 的值。当 $K$ 从 2 增大到 3 时,拟合的精度有显著的提高,NMSE 降低了 8.5dB;继续增大 $K$ 到 4 时,拟合的精度并没有太多的提高,NMSE 只降低了 1.8dB;可以看到 $K=5$ 时相对于 $K=4$ 的 NMSA 已经相差无几了。综上所述,我们选择多项式的阶数 $K=4$,不仅仅有很好的拟合度,而且满足阶数越小工程中越易实现的要求。

关于记忆深度 $M$ 的选择,我们选取 $K=4$,$M=1,2,3,4$ 进行有记忆功率放大器的拟合,分别求得的对应的 NMSE 如表 7

\begin{table}[h]
\centering
\caption{$K=4$ 时,不同取值 $M$ 下幅度和相位的 NMSE}
\label{tab:NMSE_K4}
\begin{tabular}{c c c c c}
\hline \hline
$M$ 的取值 & 1 & 2 & 3 & 4 \\
\hline \hline
\end{tabular}
\end{table}

\begin{tabular}{l c c c c}
\hline
幅度的 NMSE & $-34.3830\mathrm{dB}$ & $-43.6702\mathrm{dB}$ & $-47.2707\mathrm{dB}$ & $-47.8013\mathrm{dB}$ \\
\hline
相位的 NMSE & $-10.0950\mathrm{dB}$ & $-12.1803\mathrm{dB}$ & $-13.0244\mathrm{dB}$ & $-13.1138\mathrm{dB}$ \\
\hline
\end{tabular}

与分析 $K$ 的取值类似,可知当 $M=3$ 时,可以达到很好的拟合精度以及工程实现的复杂度。

综上所述,我们选取 $M=3$,$K=4$ 确定最终的记忆深度和记忆多项式的阶数,通过前面的最小二乘法和给定的输入输出复信号数据求出记忆多项式的系数矩阵如下:
\[
\mathbf{H} =
\begin{pmatrix}
h_{10} & h_{11} & h_{12} & h_{13} \\
h_{20} & h_{21} & h_{22} & h_{23} \\
h_{30} & h_{31} & h_{32} & h_{33} \\
h_{40} & h_{41} & h_{42} & h_{43}
\end{pmatrix}
=
\begin{pmatrix}
14.3361 - 3.1871i & -22.7150 + 6.2538i & 28.2923 - 8.2182i & -12.0489 + 4.2305i \\
9.7660 + 5.2878i & -1.5719 - 0.5001i & 4.9911 + 1.6773i & -2.6790 - 0.3026i \\
-17.3279 - 12.3948i & 4.6545 + 3.3606i & -9.9700 - 6.0537i & 5.6105 + 2.1431i \\
7.6750 + 8.2164i & -2.5860 - 3.5818i & 6.2855 + 5.5396i & -3.7007 - 2.3106i
\end{pmatrix}
\]

由此系数矩阵我们确定了一个记忆深度 3 阶数为 4 的记忆多项式,其拟合后的幅度的 NMSE 为 $-47.2707\mathrm{dB}$,相位的 NMSE 为 $-13.0244\mathrm{dB}$。其幅度和相位拟合度可由图 20 和图 21 直观地描述:

\begin{figure}[h]
\centering
\includegraphics[width=0.8\textwidth]{image1.png}
\caption{拟合与非拟合的功放输出幅度拟合度示意图}
\end{figure}

\begin{figure}[h]
\centering
\includegraphics[width=0.8\textwidth]{image2.png}
\caption{拟合与非拟合的功放输出相位拟合度示意图}
\end{figure}

\section*{5.3 问题 B 模型的建立与求解}
\subsection{5.3.1 有记忆功放预失真模型}
\subsubsection{5.3.1.1 有记忆功放预失真模型的建立}

假设预失真技术的原理框图如图 22 所示:

\begin{figure}[h]
    \centering
    \includegraphics[width=0.8\textwidth]{image1.png}
    \caption{预失真技术原理框图}
    \label{fig:22}
\end{figure}

其中功率放大器 (PA) 的输入-输出传输过程如图 23 所示:

\begin{figure}[h]
    \centering
    \includegraphics[width=0.5\textwidth]{image2.png}
    \caption{功率放大器输入-输出传输框图}
    \label{fig:23}
\end{figure}

假设预失真处理系统呈现线性,且理想的幅度放大倍数为 $g$,则有:

\begin{equation}
r(n) = \frac{s(n)}{g}
\tag{45}
\end{equation}

假设功率放大器特性函数 $G()$ 为可逆的,且逆函数用 $G^{-1}()$ 表示,则有:

\begin{equation}
y(n) = G^{-1}(s(n))
\tag{46}
\end{equation}

那么,此时的预失真处理器的输入-输出传输如图 24 所示:

\begin{figure}[h]
    \centering
    \includegraphics[width=0.8\textwidth]{image3.png}
    \caption{预失真处理器的输入-输出传输框图 A}
    \label{fig:24}
\end{figure}

如果 $s(n)$ 用功率放大器的输出 $z(n)$ 表示,即 $s(n) = z(n)$,则:

\begin{equation}
G^{-1}(s(n)) = G^{-1}(z(n)) = x(n)
\tag{47}
\end{equation}

那么,此时的预失真处理器的输入-输出传输图如图 25 所示:

\begin{figure}[h]
    \centering
    \includegraphics[width=0.8\textwidth]{image4.png}
    \caption{预失真处理器的输入-输出传输框图 B}
    \label{fig:25}
\end{figure}

假设预失真处理器特性函数为 $F()$,则根据图 25 所示有如下函数关系:

\begin{equation}
x(n) = F\left(\frac{z(n)}{g}\right)
\tag{48}
\end{equation}

由于题目中已经给定了有记忆效应的 \(x(n)\) 和 \(z(n)\) 的测试数据,因此可以根据这两组测试数据对预失真处理器特性函数 \(F()\) 进行拟合。

现给出有记忆功放预失真模型的计算框图如图 26 所示:

\begin{figure}[h]
\centering
\includegraphics[width=\textwidth]{image.png}
\caption{无记忆功放预失真模型计算框图}
\end{figure}

考虑无记忆多项式预失真处理器的数学描述如下:

\begin{equation}
y(n) = F(r(n)) = \sum_{k=1}^{K} \sum_{m=0}^{M} a_{km} r(n-m) \left| r(n-m) \right|^{k-1} \quad n=0,1,2,\dots,N-1
\tag{49}
\end{equation}

根据式 (48),有如下的函数关系:

\begin{equation}
x(n) = F\left( \frac{z(n)}{g} \right) = \sum_{k=1}^{K} \sum_{m=0}^{M} a_{km} \frac{z(n-m)}{g} \left| \frac{z(n-m)}{g} \right|^{k-1} \quad n=0,1,2,\dots,N-1
\tag{50}
\end{equation}

此处,令 \(u_{km}(n) = \frac{z(n-m)}{g} \left| \frac{z(n-m)}{g} \right|^{k-1}\),则有:

\begin{equation}
x(n) = \sum_{k=1}^{K} \sum_{m=0}^{M} a_{km} u_{km}(n) \quad n=0,1,2,\dots,N-1
\tag{51}
\end{equation}

将式 (51) 式写成矩阵形式,

\begin{equation}
\mathbf{x} = \mathbf{U} \mathbf{a}
\tag{52}
\end{equation}

其中,\(\mathbf{x} = \left[ x(0), \dots, x(N-1) \right]^T\),\(\mathbf{U} = \left[ \mathbf{u}_{10}, \dots, \mathbf{u}_{K0}, \dots, \mathbf{u}_{1M}, \dots, \mathbf{u}_{KM} \right]\),\(\mathbf{U}\) 中元素 \(\mathbf{u}_{kq}\) 为 \(\mathbf{u}_{kq} = \left[ u_{kq}(0), \dots, u_{kq}(N-1) \right]^T\),\(\mathbf{a} = \left[ a_{10}, \dots, a_{K0}, \dots, a_{1M}, \dots, a_{KM} \right]^T\)。

建立目标误差函数:

\begin{equation}
\min \sum_{n=0}^{N-1} \left| e_n \right|^2 = \sum_{n=0}^{N-1} \left| x(n) - \hat{x}(n) \right|^2 = \sum_{n=0}^{N-1} \left| x(n) - \sum_{k=1}^{K} \sum_{m=0}^{M} \hat{a}_{km} u_{km}(n) \right|^2
\tag{53}
\end{equation}

在此目标函数下,式 (52) 的最小二乘解为:

\begin{equation}
\hat{\mathbf{a}} = (\mathbf{U}^H \mathbf{U})^{-1} \mathbf{U}^H \mathbf{x}
\tag{54}
\end{equation}

\subsection*{5.3.1.2 有记忆功放预失真模型的求解}

\textbf{步骤一:幅度放大倍数 \(g\) 的确定}

考虑到预失真处理输出幅度限制和功率最大化约束,建立如下关于 \(g\) 的优

化模型:
\begin{equation}
\begin{aligned}
& \max \quad g \quad \quad (0 \leq |r(n)| \leq D \leq D_{\max}) \\
& \text{s.t.} \left\{
\begin{aligned}
& |F(r(n))| = |G^{-1}(s(n))| = |G^{-1}(g \bullet r(n))| \leq \max \left\{|x(0)|, \ldots, |x(N-1)|\right\} \\
& |s(n)| = |g \bullet r(n)| \leq \max \left\{|z(n)|, \ldots, |z(N-1)|\right\} \\
& g \geq \frac{\max (|z(n)|)}{\max (|x(n)|)}
\end{aligned}
\right.
\end{aligned}
\tag{55}
\end{equation}

其中,$D$ 为预失真处理系统输入信号的最大幅值。$D_{\max}$ 为 $D$ 能取到的最大值。

上述优化模型中的第一个约束条件是对预失真处理输出幅度的限制,其输出幅度不能超过功放输入的最大值。第二个约束是对预失真处理整个系统的输出幅度的限制。受到功率放大器自身性能的约束,功放输出的最大值限定了整个系统的输出最大值。第三个约束考虑到预失真处理后系统的功率放大性能应该优于功放的功率放大性能,即线性幅度放大倍数 $g$ 应该不小于功放的最低放大倍数,即 $\frac{\max (|z'(n)|)}{\max (|x'(n)|)}$,此约束决定了 $D_{\max}$ 的取值。

要实现对该优化模型的求解,首先要获得功放特性函数 $G(\bullet)$ 的逆函数 $G^{-1}(\bullet)$。假设这个逆是一个 $K$ 阶的多项式 [6]。

\begin{equation}
x(n) = G^{-1}(z(n)) = \sum_{k=1}^{K} \sum_{m=0}^{M} b_{km} z(n-m) |z(n-m)|^{k-1} \quad n = 0, 1, 2, \ldots, N-1
\tag{56}
\end{equation}

利用给定的功放输入信号样本值 $\{x(n), \, n = 0, 1, 2, \ldots, N-1\}$ 和对应的功放输出样本值 $\{z(n), \, n = 0, 1, 2, \ldots, N-1\}$,对 (56) 式进行最小二乘拟合,此处取 $K = 4$,$M = 3$ 进行逆函数 $G^{-1}(\bullet)$ 的拟合,得到逆多项式的系数矩阵为:

\begin{equation}
B = \begin{bmatrix}
0.0787 + 0.0545i & -0.0092 - 0.0067i & 0.0019 + 0.0012i & -0.0001 - 0.0001i \\
0.1589 - 0.1229i & -0.0012 + 0.0010i & -0.0011 + 0.0003i & 0.0001 - 0.0001i \\
-0.2135 + 0.1415i & -0.0009 - 0.0016i & 0.0017 - 0.0002i & -0.0002 + 0.0000i \\
0.0959 - 0.0623i & 0.0020 + 0.0004i & -0.0011 + 0.0001i & 0.0001 - 0.0000i
\end{bmatrix}
\tag{57}
\end{equation}

此时信号幅值的 $NMSE = -42.6823 \, dB$ 已经满足相应的精度要求。

至此,我们可以获得对于不同的预失真处理系统输入信号区间所对应的最大幅度放大倍数 $g$。

表 8:给定信号输入范围下的最大幅度放大倍数

\begin{tabular}{l c c c c c}
\hline
输入信号取值范围 & $[0 \ 0.3]$ & $[0 \ 0.4]$ & $[0 \ 0.5]$ & $[0 \ 0.6]$ & $[0 \ 0.7]$ \\
\hline
最大幅度放大倍数 & 22.5518 & 16.9139 & 13.5311 & 11.2759 & 9.6108 \\
$g$ & & & & & \\
\hline
\end{tabular}

由于 $\frac{\max (|z'(n)|)}{\max (|x'(n)|)} \approx 10.1491$,所以信号输入范围的最大值取值为:
\[ D_{\max} = \max (x(n)) = 0.7 \]

步骤二:给出预失真处理器特性函数

假设预失真处理器特性函数如式(49)所示的记忆深度为 $M$ 的 $K$ 阶多项式函数。

步骤三:

此处取预失真特性函数的阶数为 $K=4$,记忆深度 $M=3$,$g=9.6108$,利用给定的 $\frac{z(n)}{g}$ 和 $x(n)$ 作为预失真处理器输入和输出样本,利用最小二乘法估计预失真函数特性参数。$F$ 函数的拟合后的参数矩阵如式(58)所示:
\[
A = \begin{bmatrix}
0.7559 + 0.5242i & 1.5270 - 1.1808i & -2.0521 + 1.3596i & 0.9212 - 0.5992i \\
-0.8495 - 0.6203i & -0.1113 + 0.0902i & -0.0839 - 0.1517i & 0.1876 + 0.0344i \\
1.7207 + 1.0639i & -0.9713 + 0.2816i & 1.5044 - 0.1445i & -1.0202 + 0.0756i \\
-0.8839 - 0.4642i & 1.1099 - 0.4784i & -1.7062 + 0.3258i & 1.0416 - 0.1123i
\end{bmatrix}
\tag{58}
\]

分别考虑信号幅度和相位,仿真预失真函数 $F(\bullet)$ 对信号幅度的输入输出图和对信号相位的输入输出图。此处取信号的输入范围为 $[0 \ 0.7]$,$g=9.6108$,预失真函数为四阶多项式,记忆深度为三。

\begin{figure}[h]
\centering
\includegraphics[width=0.8\textwidth]{image.png}
\caption{F 函数对信号幅值和相位作用图}
\end{figure}

分别考虑信号幅度和相位,仿真经过预失真处理后的功率放大器的输入输出间的关系图。此处取信号的输入范围为 $[0 \ 0.7]$,$g=9.6108$。

\begin{figure}[h]
    \centering
    \includegraphics[width=\textwidth]{image1.png}
    \caption{预失真处理后系统对信号幅值和相位作用图}
    \label{fig:28}
\end{figure}

此时的信号幅值的预失真补偿 $NMSE = -41.0032$,信号相位的预失真补偿 $NMSE = -16.1892$,信号的预失真补偿 $NMSE = -38.4071$。

根据上述模型的求解过程,可以建立预失真处理的模型实现示意图,如下图所示:

\begin{figure}[h]
    \centering
    \includegraphics[width=\textwidth]{image2.png}
    \caption{预失真处理的模型实现示意图}
    \label{fig:29}
\end{figure}

\subsection{基于功率放大器特性的有记忆功放自适应预失真模型}

当功率放大器特性发生变化时,如由于温度、供电、器件的老化以及信道间的切换等导致 \cite{ref7,ref8},此时则需要自适应的调整预失真器的特性参数,降低非线性失真。具体的自适应预失真模型结构图如下图所示:

\begin{figure}[h]
    \centering
    \includegraphics[width=\textwidth]{image3.png}
    \caption{基于功率放大器特性的有记忆功放自适应预失真模型框图}
    \label{fig:30}
\end{figure}

跟据图 \ref{fig:30} 所示的自适应预失真框图可知,自适应的过程是通过最小化误差能量 $\|e(n)\|^2$ 来实现的。

假设预失真处理器的特性函数为 $F(\bullet)$,功率放大器的 $G(\bullet)$,则有:
\begin{equation}
\left\{
\begin{aligned}
y(n) &= F(r(n)) \\
\hat{y}(n) &= F\left(\frac{s(n)}{g}\right)
\end{aligned}
\right.
\tag{59}
\end{equation}

根据前 $N$ 个时刻的系统输入值 $\{r(n), \, n=0,1,\ldots,N-1\}$,获得前 $N$ 个时刻的系统输出值 $\{y(n), \, n=0,1,\ldots,N-1\}$ 和 $\{s(n), \, n=0,1,\ldots,N-1\}$。

自适应调整的目标函数为:
\begin{equation}
\min \|e(n)\|^2 = \|y(n) - \hat{y}(n)\|^2
\tag{60}
\end{equation}

利用最小二乘法,使用前 $N$ 个时刻的 $y(n)$ 和 $s(n)$ 的样本值对预失真处理器的特性函数 $F(\bullet)$ 中的参数进行估计,将估计得到的特性参数传输到预失真器中,自适应的调整系统的非线性失真。

在 5.3.1.1 模型的基础之上,采用自适应的方法可以自适应的调整因功率放大器特性发生变化时导致的非线性失真。

### 5.3.3 联合功率放大器特性和输入信号幅值范围的有记忆功放自适应预失真模型

为了最大利用功率放大器的放大性能,可以根据前 $N$ 个时刻输入信号的幅值确定当前时刻信号输入的幅值范围,自适应的调整预失真处理器的特性参数。然而这种自适应的调整是建立在功放特性不发生改变的前提之下。当功率放大器特性发生改变时,需要运用 5.3.1.3 节的自适应算法对预失真处理器的参数进行调整。本节提出了联合功率放大特性和输入信号幅值范围的有记忆功放自适应预失真模型框图,如图 62 所示。

\begin{figure}[h]
\centering
\includegraphics[width=0.8\textwidth]{image.png}
\caption{联合功率放大器特性和输入信号幅值范围的有记忆功放自适应预失真模型框图}
\end{figure}

根据图 31 可知,自适应的关键在于预失真特性参数训练的模块,根据缓存器中缓存的前 $N$ 时刻的数据 $\{r(0),\ldots,r(N-1)\}$,获得当前时刻信号输入的幅值范围 $[0 \, D]$,将此范围传给预失真特性参数训练的模块,跟据前 $N$ 时刻的输出数据 $\{y(0),\ldots,y(N-1)\}$ 和 $\{s(0),\ldots,s(N-1)\}$,利用最小二乘算法对预失真器的特性参数进行估计,用估计后的参数自适应的调整预失真器参数。

这种联合功率放大器特性和输入信号幅值范围的有记忆功放自适应预失真模型能够综合考虑功率放大器的特性和最大化利用功率放大器当前特性。

\section{问题三的求解与分析}

本节首先计算输入输出信号的功率谱密度函数,并通过功率谱密度计算相邻用来衡量带外失真的程度的信道功率比 (ACPR),最后通过对三类信号的功率谱密度(输入信号、无预失真补偿的功率放大器输出信号、采用预失真补偿的功率放大器输出信号)的比较,给出分析评价。

\subsection{功率谱密度的定义}

\begin{equation}
P_{xx}(w) = \sum_{m=-\infty}^{\infty} \Phi_{xx}(m)e^{-jmw}
\tag{61}
\end{equation}

关于功率谱密度的几点说明:

1) 随机信号的自相关函数与它的功率谱密度构成一对离散 Fourier 变换;

2) 随机信号的自相关函数具有离散时间的形式,称为自相关序列,但它的功率谱密度却是变量 $\omega$ 的连续函数;

3) 对于非周期离散时间自相关序列,其功率谱密度为周期信号。功率谱密度在一个周期内的积分与随机序列的平均功率成正比;

4) 随机信号的功率谱是周期信号,因此可以展开为 Fourier 级数,正是 Fourier 级数的各谐波分量的系数;

5) 随机信号的自相关函数与它的功率谱密度从时域、频域反映了随机序列的二阶统计特性。

自相关函数的定义:

\begin{equation}
\Phi_{xx}(m) = \sum_{n=-\infty}^{\infty} x(n)x^*(n+m)
\tag{62}
\end{equation}

相邻信道功率比的定义:

\begin{equation}
ACPR = 10\log_{10} \frac{\int_{f_2}^{f_3} s(f)df}{\int_{f_1}^{f_2} s(f)df}
\tag{63}
\end{equation}

\subsection{输入与输出信号的功率谱密度}

\begin{figure}[h]
    \centering
    \includegraphics[width=0.45\textwidth]{image1.png}
    \caption{输入信号的功率谱密度}
    \label{fig:input_power_spectral_density}
\end{figure}
\begin{figure}[h]
    \centering
    \includegraphics[width=0.45\textwidth]{image2.png}
    \caption{无预失真输出信号的功率谱密度}
    \label{fig:output_power_spectral_density}
\end{figure}

图 32 是经过归一化之后的功率谱密度,可以粗略地看出带宽内的幅度维持在 \(-20\,\text{dB} \sim 0\,\text{dB}\) 之间,而带外扩展达到了 \(-80\,\text{dB}\)。如果采用 ACPR 评价标准,得到 \(\text{ACPR} = -78.3594\,\text{dB}\)。

图 33 是经过归一化之后的功率谱密度,可以粗略地看出带宽内的幅度维持在 \(-30\,\text{dB} \sim 0\,\text{dB}\) 之间,带外扩展达到了 \(-40\,\text{dB}\)。如果采用 ACPR 评价标准,得到 \(\text{ACPR} = -38.5069\,\text{dB}\);很显然,功率放大器使得频谱在带外扩展了,对相邻信道产生了很大的影响;带内信号也产生了衰减。

\begin{figure}[h]
    \centering
    \includegraphics[width=0.8\textwidth]{image.png}
    \caption{有预失真输出信号的功率谱密度}
    \label{fig:34}
\end{figure}

图 34 是经过归一化之后的有预失真输出信号功率谱密度,可以粗略地看出带宽内的幅度维持在 \(-30\,\text{dB} \sim 0\,\text{dB}\) 之间,带外扩展达到了 \(-50\,\text{dB}\)。如果采用 ACPR 评价标准,得到 \(\text{ACPR} = -54.6101\,\text{dB}\);相比于没有采用预失真的系统输出其 ACPR 降低了约 \(16\,\text{dB}\)。结果显示采用预失真后不仅仅使得整个系统逼近线性特性,而且使得带外扩展有效地衰减了。

\section{模型总结}

对于问题一,用四种方法建立了无记忆功率放大器的非线性特性模型,并得出多项式模型的优越性;然后在一定的约束条件下,求出了预失真处理系统的放大倍数 \(g\),从而得出了预失真处理器的输入输出样值,建立了预失真处理器的模型。经过分析,预失真处理系统具有很好的线性特性。

对于问题二,选用有记忆多项式建立有记忆功率放大器的特性模型,并通过分析选择了合适的多项式记忆深度和阶数;然后在一定的约束条件下,得到了预失真处理系统的放大倍数 \(g\),从而得到了预失真处理器的输入输出样值,建立了预失真处理器的有记忆多项式模型。分析得到,预失真系统具有很好的线性特性。

对于问题三,我们计算出了三种信号的频谱密度,比较得出预失真处理系统的频谱带外扩展得到了很好的控制,达到了预期效果。

\section{参考文献}

[1] 胡哲彬,何松柏,马岳林,等.基于正交多项式的功放行为模型研究

\begin{enumerate}
    \item [C]//2011 年全国微波毫米波会议论文集 (下册). 2011.
    \item 吴樟强, 刘太君, 任坤胜, 等. 基于记忆多项式的射频功放模型研究[J]. 宁波大学学报 (理工版), 2008, 21(3).
    \item 金哲. 宽带通信中有记忆射频功率放大器的建模与预失真方法 [D]. 杭州: 浙江大学, 2007.
    \item Ding L, Zhou G T, Morgan D R, et al. A robust digital baseband predistorter constructed using memory polynomials[J]. IEEE Transactions on Communications, 2004, 52(1): 159-165.
    \item Raich R, Qian H, Zhou G T. Digital baseband predistortion of nonlinear power amplifiers using orthogonal polynomials[C]//Acoustics, Speech, and Signal Processing, 2003. Proceedings.(ICASSP'03). 2003 IEEE International Conference on. IEEE, 2003, VI-689-92 vol. 6.
    \item 马岳林. 基于多项式求逆的数字预失真技术研究和测试验证[D]. 电子科技大学, 2011.
    \item 王胜水, 鲍景富, 李红宝. 自适应数字预失真放大器的算法研究[J]. 电子设计工程 ISTIC, 2009, 17(4).
    \item 张小梅, 胡方明, 任爱锋. 基于数字预失真技术的功放线性化研究[J]. 计算机仿真, 2012, 29(7): 393-397.
\end{enumerate}

\section*{附录 程序文件的功能说明}

\begin{tabular}{l l}
T1AFourier.m & 用傅里叶级数表示功放模型 \\
T1ANMSE.m & 将多项式不同阶数的 NMSE 图形化 \\
T1APolynomial.m & 用多项式表示功放模型 \\
T1Asaleh.m & 用 saleh 模型表示功放模型 \\
T1Awiener.m & 用 wiener 模型表示功放模型 \\
T1B1.m & 考虑信号相位信息的无记忆预失真模型中 $G$ 函数的逆函数参数估计程序 \\
T1B2.m~T1B6.m & 考虑信号相位信息的无记忆预失真模型中针对 5 个信号输入范围的最大幅度放大倍数 $g$ 的优化程序 \\
T1B7.m & 考虑信号相位信息的无记忆预失真模型中 $F$ 函数参数估计, 以及 $F$ 函数分别对信号幅值和相位的作用图, 以及预失真处理后系统对信号幅值和相位的作用图 \\
T1B8 文件夹中 tu.m 为主程序 & $F$ 函数对不同输入范围信号幅值的作用图以及预失真处理后系统对不同输入范围信号幅值的作用图 \\
T1B9.m & 不考虑信号相位信息的无记忆预失真模型中 $G$ 函数的逆函数参数估计程序 \\
T1B10.m, T1B12.m & 不考虑信号相位信息的无记忆预失真模型中分别以信号输入范围 $[0 \ 1.05]$ 和 $[0 \ 0.5]$ 为例的最大幅
\end{tabular}

\begin{tabular}{l l}
T1B11.m & 不考虑信号相位信息的无记忆预失真模型中 $F$ 函数参数估计,以及 $F$ 函数分别对信号幅值的作用图,以及预失真处理后系统对信号幅值的作用图 \\
T2A1.m & 有记忆多项式的曲线拟合 $M=3K=2$ \\
T2A2.m & 有记忆多项式的曲线拟合 $M=3K=5$ \\
T2A3.m & 有记忆多项式的曲线拟合 $M=4K=4$ \\
T2A4.m & 有记忆多项式的曲线拟合 $M=3K=4$ \\
T2A5.m & 有记忆多项式的曲线拟合 $M=2K=4$ \\
T2A6.m & 有记忆多项式的曲线拟合 $M=1K=4$ \\
T2A7.m & 有记忆多项式的曲线拟合 $M=3K=3$ \\
T2B1.m & 有记忆预失真模型中 $G$ 函数的逆函数参数估计程序 \\
T2B2.m & 有记忆预失真模型中以信号输入范围 $[0, 0.7]$ 为例的最大幅度放大倍数 $g$ 的优化程序(信号输入范围可在该程序中改变) \\
T2B3.m & 有记忆预失真模型中 $F$ 函数参数估计,以及 $F$ 函数分别对信号幅值和相位的作用图,以及预失真处理后系统对信号幅值和相位的作用图 \\
T3.m & 画出各种信号的功率谱密度及求 ACPR \\
\end{tabular}