\documentclass{article}
\usepackage{amsmath}
\usepackage{amssymb}

\title{构建地下物流系统网络}
\author{}
\date{}

\begin{document}


\begin{abstract}
地下物流系统(Underground Logistics System,ULS)通过地下管道或隧道运输货物,对缓解城市地面交通压力有重大意义,本题要求为南京市仙林地区设计 ULS 系统,达到缓解交通拥堵和降低物流成本 2 个目标。本文采用分步设计方法,在第一问中分别以降低 ULS 节点建设成本和运输成本为目标,采用双层 ULS 物流节点选址规划方法,先后确定二级节点、一级节点选址;在第二问将总成本最低的优化问题转化为区域外、区域间和区域外纯路径优化问题,采用混合整数规划、整数规划等方法设计 ULS 网络构成;在第三问中对第二问涉及的网络架构进行优化,区域间在环网基础上增加中继以缩短货物运输里程并增加备用路径,区域内建设环形辐射状网络增加备用路径以提高抗风险能力;在第四问中,通过建立 ULS 线网建设时序的指标体系,绘制了考虑每年 5\%货物增长率情况下的仙林地区未来 8 年 ULS 线网建设时序演化图。

对于问题 1,本文采用双层 ULS 物流节点选址规划方法,底层为二级节点选址规划,以二级节点数量最少为目标,将 110 个区域中心点作为备选二级节点,采用集合覆盖方法安排尽量少的二级节点满足系统缓解交通拥堵和货物运输需求。上层为一级节点选址规划,采用 CFLP 法,以 28 个二级节点作为备选一级节点,从中安排固定数量的一级节点,使各二级节点到对应隶属一级节点以及各

一级节点到物流园区的距离之和最小,以降低 ULS 管道距离。底层和上层优化分别为混合整数规划和整数规划,决策变量和约束条件数量庞大,本文借助 LINGO 软件编程求解,大幅度降低了仿真难度。

对于问题 2,本文首先根据区域划分关系,将网络架构设计问题划分为区域外、区域间、区域外路径优化问题。区域外为优化物流园区与一级节点间通道建设方案,使进出物流园区的货物尽可能放入地下运输;区域间采用混合整数规划法,搭建各一级节点间货物运输模型,并采用 LINGO 软件求解最优路径及实际流量,使区域间运输成本和通道建设成本最小;区域内通过分析运输成本和通道折旧成本对通道距离的灵敏度,发现路径是影响总成本的关键因素,从而将总成本最低问题转换为最短路径优化的最小生成树问题,并搭建了整数规划模型,并借助 LINGO 软件求解得到区域内最小生成树型网络拓扑。

对于问题 3,基于对所设计的区域间环网和区域内最小生成树型网络拓扑的脆弱性分析,选择在区域间增加一级中继节点在一级节点间建立中继网络实现环网不相邻节点的直线转运,缩短货物运输总里程,且中继网络还可在最极端环境下(环网各段均故障)延缓区域间 ULS 物流系统瘫痪时间;选择在区域内最小生成树的最下游节点和一级节点间建立通路,保证各节点与一级节点间至少有一条备用路径,提高区域内网络抗风险能力。

对于问题 4,本文建立了影响 ULS 线网建设时序的五个效益性指标体系,通过分析计算各项指标值,构建线网建设时序的决策矩阵。通过主观(AHP 法)与客观(熵权法)相结合的方法,确立各项指标的混合权重系数;运用极差变换对各项指标进行归一化处理,最后利用 TOPSIS 法确立了 ULS 线网规划建设的理论时序,并绘制了考虑每年 5\%货物增长率情况下,南京市仙林地区未来 8 年的 ULS 线网建设时序演化图。
\end{abstract}

\tableofcontents

\section{问题重述}

地下物流系统(Underground Logistics System, ULS)利用城市内部及城市之间的地下管道或隧道运输货物,不仅可以有效缓解地面道路的交通压力,通过优化节点选址和地下通道网络,还能提高物流运输效率,对解决城市交通拥堵、能源节约、噪音污染、环境问题等有着重大意义,陆续有越来越多的学者开始关注这一研究方向。目前关于地下物流系统的研究多数集中在物流节点选址问题,对于综合考虑总成本、交通拥堵程度的地下物流系统的整体设计还不多见,国内尚未有成功案例。

题目要求在满足地下物流系统的基本特征和约束的前提下,根据南京市仙林地区的交通货运特征,完成该区域地下物流系统的物流节点选址和线路规划,以满足缓解该区域交通拥堵现状和降低物流成本 2 个直接目标。

具体任务如下:

1) 根据仙林地区的交通货运实际情况,建立 ULS 节点选择模型,完成 ULS 初步物流节点选址,确定各级物流节点的数量、位置、服务范围、各节点实际货运量和各一级节点的转运率。

2) 在 1) 的基础上,以总成本为 ULS 网络优化目标,在转运率相差不大的情况下,合理选择地下隧道形式并适当调整各节点位置,获得最优 ULS 线网架构、各节点实际货运量、各级通道的位置和实际流量。

3) 从全局出发,通过增减物流节点、调整节点位置或级别,增减或调整路径的方法缩短货物运输总里程,降低运输成本。在此基础上,考虑 ULS 的抗风险能力,进一步完善 ULS 网络。

4) 考虑满足该市未来考虑满足该市近 30 年内的交通需求(可以认为需求量每年呈 5\% 增长)并根据建设进度分八年完成 ULS 的建设(每年可建设道路长度大致相等),确定该市 ULS 网络各线路的建设时序及演进过程,并与 3) 所得 ULS 网络进行优劣比较。

\section{问题分析}

根据本题主要任务可见,本题是按照分步设计的思路,首先完成地下物流系统的物流节点选址以基本满足该区域的物流需求,在此基础上以总成本最低为目标合理规划地下物流线网,然后从全局出发调整物流节点选址和物流线网,进一步降低运输成本并增强抗风险能力。最后根据未来交通需求推演地下物流系统的建设路径。这一思路及流程与北京城建设计研究院提出的“面点线多模块网络层次分析法” [1] 的核心思想十分相似。所谓“面”是指拟定地下物流系统的基本构架;“点”是指物流节点的数量及位置规划;“线”是指地下物流线网设计。本文基于此思路对各问题做深入分析。

\subsection{问题一的分析}

基于“面点线多模块网络层次分析法”,该问是对 ULS 的“面—点”联合规划,目的在于确定满足该地区物流需求和 ULS 物流节点约束的地下物流系统基本构架,该构架包括各级节点数量和位置、各节点的服务范围等信息。本文采用由下至上的双层节点选址规划方法,首先采用集合覆盖法确定二级节点的位置及相应的二级服务范围,即底层规划;将所有二级节点作为备选一级节点,采用 CFLP 法确定固定数量的一级节点选址信息,即上层优化。各层优化均包含大量的决策变量和约束条件,采用 LINGO 编程求解可大幅度降低求解难度。

\subsection{问题二的分析}

该问是在“面—点”联合规划的基础上,在一级服务范围和物流节点数量不变的前提下,设计 ULS 网络构成。ULS 网络由区域外(物流园区与一级节点)、区域间(一级节点间)、区域内(一级区域内)三部分构成。

区域外网络架构与物流园区货运总量相关,根据物流园区货运总量确定其与相应一级节点间管道建设方案和管道流量,以保证进出 4 个物流园区的货物尽可能放入地下运输。

区域间网络架构与一级节点的货物分流情况息息相关,在建立环形网架结构的基础上,将节点收发货物分为左向路径和右向路径货运,建立包含左向货运量矩阵、右向货运量矩阵,及各管道建设方案 0-1 决策变量的混合整数规划模型,以运输成本和通道折旧成本之和最小为目标,并借助 LINGO 软件编程求解,确定区域间网络构成和各管道实际流量。

区域内网架总成本虽然包含运输成本、通道折旧成本和节点折旧成本。但考虑到节点数量不变,则节点折旧成本为固定成本。通道折旧成本和运输成本均是与通道距离密切相关量,可通过计算两成本对通道距离变化的灵敏度判断相应成本对总成本的影响力。从而将复杂的经济优化问题转化为纯最短路径优化问题,当连通路径最短时,区域通道折旧成本最低,相应的总成本就最小。本文采用 0-1 整数规划模型求解这一典型最小生成树问题,并借助 LINGO 软件求解,确定各区域内网络构成和各管道实际流量。

\subsection{问题三的分析}

问题三是修正问题二所建网络拓扑,以缩短货物运输总里程、提高 ULS 抗风险能力。分析问题二所建三层网络拓扑(区域外、区域间、区域内),区域外网络拓扑结构简单,其唯一功能需求就是尽可能提高物流园区进入 ULS 的货运量,若调整一级节点位置势必影响另外两层网络拓扑,改建工作量巨大且对缩短货运总里程没有显著成效。区域间网络拓扑在不改变已有一级节点的前提下修改路径或增加一级节点数量,都不会影响另外两层网络拓扑,具有较高的灵活性。分析区域间网络拓扑可见环形网架在运输总里程、鲁棒性方面都表现较差,通过增加中继一级节点的方式便可实现环网中不相邻节点的直接转运,且大大提高了物流系统的鲁棒性。区域内网络拓扑由最小生成树法确定,区域间未形成闭合回路,主要呈辐射状网架,抗干扰能力差、货运流量激增裕度低,建立环形辐射状网络可增加路径备选方案,提高区域内网络的抗风险能力。

\subsection{问题四的分析}

考虑到地下基建设施的高投资、改建困难特点,本问题需要考虑该区域货运量近 30 年每年增长的情况下对 ULS 系统进行顶层设计,分阶段完成建造。该问题需要从空间尺度、时间尺度等多维角度来考虑 ULS 系统的演化过程和系统建设时序。对比前三个问题可以发现,上述问题均是基于货运量给定的前提下进行线网规划的,系统正常运行时只需要满足现有的货运量即可。而考虑到线网货运量逐年激增的情况,则需要对管道类型、容量、造价等方面进行全面的考虑和分析,需要建立详实的量化关系,从而对 ULS 的演化过程进行研判。另一方面,问题四还需要对比分析所提方法与问题三规划方法的优势,并给出问题三网络在时间尺度上与逐年增长货运量的临界点,并给出相应的对策。

\section{符号定义与说明}

符号 | 说明
--- | ---
$p_{ij}$ | 各区域中心点间的距离(/km)
$x_{ij}^{down}$ | 区域i是否被二级节点j覆盖,取1为覆盖,取0不覆盖
$s_{j}^{down}$ | 二级节点j从地面收发货物总量上限(/t)
$r_{j}^{down}$ | 二级节点判断变量,取1表征区域中心点j为二级节点,反之不是
$q_{ij}$ | 区域i对二级节点j的地下货运需求量
$d_{i}$ | 区域i一日货运总量
$k_{i}$ | 区域i的交通拥堵指数
$s_{i}$ | 二级节点货运量(/t)
$p_{ij}^{up}$ | 各二级节点间的距离(/km)
$x_{ij}^{up}$ | 二级节点i是否由一级节点j服务,取1为服务,取0不服务
$w_{jk}$ | 园区k与二级节点j是否相连,取1相连,取0不相连
$V_{k}$ | 园区k的货运量(/t)
$D_{jk}$ | 园区k与二级节点j间的距离(/km)
$c$ | 货物平均运输成本(元/t·km)
$r_{j}^{up}$ | 一级节点判断变量,取1表征二级节点j被选定为二级节点,反之不是
$\varphi_{j}$ | 一级节点j的转运率
$\alpha$ | 年综合折旧率,取1%
$y$ | 设计年限,取100
$Q_{ij}$ | 由i发送至j的管道流量
$F_{1}^{2}$ | 物流园区至各一级节点通道总成本(/元)
$L$ | 一级节点货运量左向分流矩阵
$R$ | 一级节点货运量右向分流矩阵
$D^{L}$ | 一级节点左向运输距离矩阵
$D^{R}$ | 一级节点右向运输距离矩阵
$A$ | 一级节点建设方案决策向量
$F_{2}^{2}$ | 一级节点间通道总成本(/元)
$X_{ij}$ | 一级节点i对目的节点j的送货量(/吨)
$B_{i}$ | 经由该通道顺时针运输的货物流量
$C_{i}$ | 经由该通道逆时针运输的货物流量

\section{模型假设}

根据实际物流情况和文中所给条件,本文对地下物流系统做如下假设:

1) 假设进出 4 个物流园区的货物全部投入地下运输,经地下通道传送至各一级节点。

2) 假设地下物流系统的各节点间的转运均通过地下通道运输。

3) 由假设 2) 可知,二级节点从地面收发货物总量上限是指二级节点和地面之间采用人力或小型车辆在节点服务区域内进行运输的最大收发货物总量。各节点之间的地下通道的货物转运总量不受地面收发货物总量的约束。

4) 假设服务半径是指节点的地面服务半径,在该服务区域内,货物从节点至地面采用的运输方式不影响交通。

5) 由假设 4) 可知,地下物流系统中节点间货物经地下通道转运,因此节点间的距离不受服务半径限制;一级节点经地面交通运输至货运服务区域时,由于影响交通,也不受服务半径限制。

6) 假设每个区域仅能被一个二级节点覆盖,采用多式联运的方式满足货运需求,分别为 ULS 运输和地面交通运输。

\section{模型的建立与求解}

\subsection{问题一的求解}

根据题意可知,本问要求设计满足仙林地区地面交通通畅需求和货运需求的 ULS 的物流节点群,包括一、二级节点数及位置、各节点的服务范围、各节点实际货运量和各一级节点的转运率。

二级节点是 ULS 系统中直接服务于用户(仙林地区各交通货运区域)的终端节点,二级节点的选取直接关系到该地区货运需求能否满足。因此,本文采用由下向上的规划思路,提出了一种双层 ULS 物流节点选址规划策略。

该策略底层为二级节点选址规划,以二级节点建设成本最低为目标,先采用集合覆盖法求取满足该地区物流需求的最少二级节点数量、位置及相应的二级服务范围,再校验结果能否实现区域面积覆盖;上层为一级节点选址规划,采用 CFLP 法,从二级节点中遴选固定数量的一级节点并确定其位置和相应一级服务范围,降低 ULS 的运输成本。

\subsubsection{底层规划——二级节点选址模型}

为简化计算,本文首先假设所有区域均满足若二级节点覆盖了该区域中心点即可视为对该区域进行了覆盖,并将各区域中心点作为二级节点的备选地址。底层规划可描述为在 110 个备选地点中选择数量尽量少的的区域中心点作为二级节点,满足 110 个用户的地下物流需求量,使各区域交通基本通畅、ULS 的节点建设成本最低。其中用户是指各货运区域,地下物流需求量是指各区域通过 ULS 的货运量。

\paragraph{集合覆盖模型建立}

集合覆盖问题是运筹学中典型的整数规划问题,在这一类问题中,会有许多服务装置为一些设备提供相互重叠的服务,目标就是确定安装数目最少的装置来满足覆盖(满足需求)每一个设备[2]。本问底层二级节点选址规划直接影响 ULS 的节点建设成本,为使 ULS 的总成本最低,需选择尽量少的二级节点满足整个仙林地区的货物运输,这与集合覆盖问题的思路十分相似。因此,本文采用集合覆盖法搭建二级节点选址规划模型。

底层规划目标函数为二级节点数量最少,如式(1-1)所示:

\begin{align}
L_{ij} + R_{ij} &= X_{ij} \tag{2-5} \\
\max(B_i, C_i) &\leq (A_i + 1) \times 3600 \tag{2-6}
\end{align}

满足如下约束:

\begin{equation}
p_{ij} x_{ij}^{down} \leq 3
\tag{1-2}
\end{equation}

\begin{equation}
\sum_{i=1}^{110} q_{ij} x_{ij}^{down} \leq s_{j}^{down} r_{j}^{down}
\tag{1-3}
\end{equation}

\paragraph{5.1.1.2 模型求解方法与流程}
从式(1-1)~式(1-5)可知,该优化问题决策变量包括 110个“0-1”变量$r_j^{down}$、110$^2$个“0-1”变量$x_{ij}^{down}$、110$^2$ 个连续型变量$q_{ij}$,可见该优化问题的决策变量数目和约束条件数目庞大,若采用 CEPLX 或 MATLAB 编程求解,编程任务繁重,运行速度极慢。据此,本文采用专业的通用优化求解器——LNGO 软件对该模型编程求解,大大简化了编程难度,可在 4min 内求得优化结果,源程

序见附录二。$$\sum_{j=1}^{110}x_{ij}^{down}=1$$

(1-5)


其中,式(1-2)为二级节点的服务半径约束,表示二级节点$j$的二级服务区域内的用户到该二级节点的距离不超过 3 公里;式 (1-3) 为二级节点从地面收发货物总量上限约束,代表二级节点$j$ 的二级服务区域内所有用户的地下物流需求量之和不超过该二级节点从地面收发货物总量上限;式(1-4)为各用户区域交通通畅程度约束,表示用户区域$i$的地面交通货物总量造成的地面交通拥堵指数不得超过 4,即保证地面交通至少基本通畅;式(1-5)表示一个用户只能被一个二级节点服务。

\paragraph{5.1.1.3 二级节点选址结果及校验}
通过 LINGO 程序的算法实现,得到仙林地区的 28 个二级节点及其位置、
相应的二级服务范围、货运量,结果如表1-1 所示。

\begin{table}[h]
\centering
\caption{表1-1 二级节点选址信息}
\begin{tabular}{|c|c|c|c|}
\hline
节点编号 & 位置(区域中心点) & 二级服务范围(服务区域编号) & 货运量(吨) \\
\hline
1 & 799 & 791、794、799、805、808 & 2958.291 \\
\hline
2 & 800 & 795、802、806、816、818、819 & 2799.748 \\
\hline
3 & 808 & 792、812、822 & 2943.397 \\
\hline
4 & 810 & 793、801、817、827 & 2773.918 \\
\hline
5 & 815 & 797、798、800、807、809 & 2869.726 \\
\hline
6 & 816 & 796、803、804、821、824 & 2905.861 \\
\hline
7 & 823 & 810、814、842、843、851 & 2992.767 \\
\hline
8 & 826 & 825、826、829 & 2802.600 \\
\hline
9 & 829 & 831、837、838、840 & 2927.764 \\
\hline
10 & 833 & 811、820、834、854 & 2982.928 \\
\hline
11 & 834 & 828、835、836、855、861 & 2923.009 \\
\hline
12 & 841 & 815、841、846 & 2808.763 \\
\hline
13 & 842 & 813、823、850、853 & 2990.265 \\
\hline
14 & 849 & 844、847、848、849 & 2962.904 \\
\hline
15 & 851 & 830、845、857、859 & 2999.970 \\
\hline
16 & 855 & 833、856、863、864、870 & 2728.650 \\
\hline
17 & 858 & 852、858、862、868 & 2345.286 \\
\hline
18 & 866 & 860、865、866、867、869 & 2762.791 \\
\hline
19 & 873 & 832、871、872、873、874、875、878、879、 & 2889.448 \\
\hline
\end{tabular}
\end{table}

\begin{table}
\centering
\begin{tabular}{c c c c}
\hline
 & & 882 & \\
\hline
20 & 877 & 839、884 & 2899.610 \\
\hline
21 & 878 & 880、881、883 & 2988.637 \\
\hline
22 & 879 & 876、877、890 & 2997.720 \\
\hline
23 & 891 & 886、888、895 & 2981.074 \\
\hline
24 & 892 & 892、899 & 2915.756 \\
\hline
25 & 893 & 885、887、889、891、894 & 2976.652 \\
\hline
26 & 894 & 893、898 & 2952.035 \\
\hline
27 & 896 & 896 & 406.7080 \\
\hline
28 & 899 & 897、900 & 1161.403 \\
\hline
\end{tabular}
\end{table}

\textbf{注:}二级节点仅在区域中心点选址,不代表该区域一定被位于该区域中心点的二级节点服务,例如二级节点6位于区域816中心点,但区域816不在其服务范围内。

\bigskip

建模时为了简化计算而假设所有用户区域只需覆盖其区域中心点就视为完全覆盖该区域,所得优化结果是对区域中心点的集合全覆盖。现需对优化结果进行校验,对于货运量与面积之比较大区域判断二级节点能否对其实现面积全覆盖。

各区域货运量与面积之比(Freight Transport Divide Area,FTDA)的数量分布如图1-1所示,可见大部分区域的FTDA集中在2000以下,本文将这些区域认定为货运量与面积之比过小,节点覆盖了这些区域中心点即可视为对该区域进行了完全覆盖。

\begin{figure}[h]
\centering
\includegraphics[width=0.8\textwidth]{image.png}
\caption{各区域FTDA数量分布图}
\end{figure}

剩余需要实现面积全覆盖的区域有791,810,815,819,820,839,845,846,848,851,874,876,877,888,889。二级节点对这些区域的面积覆盖情况如图1-2所示,其中红色节点代表二级节点所在位置。可见,所选二级节点对这些区域实现了面积完全覆盖。因此,本文所设28个节点可在基本不影响交通畅的前提下,满足所有区域的地下物流需求,是符合实际的有效选址方案。

\begin{figure}[h]
    \centering
    \includegraphics[width=\textwidth]{image.png}
    \caption{二级节点服务范围覆盖图}
    \label{fig:coverage_map}
\end{figure}

\subsubsection{上层规划——一级节点选址模型}

一级节点从二级节点中产生。考虑到一级节点的建设成本高昂,而且物流园区的货物必须经由最近的一级节点转运至其他所有一级节点、或本一级区域内的二级服务区域,本文认为一级节点数量至少有 4 个。则上层规划就表述为对于 28 个二级节点服务区域、28 个备选一级节点、建设 4 个一级节点的问题。可采用简化的 CFLP(Capacitated Facility Location Problem)法[3]建模和求解。

简化的 CFLP 法是用现行规划中的运输规划法确定个配送中心的市场占有率,求出配送分担地区的重心,再用整数规划法来确定配送中心的建设位置。

\paragraph{简化 CFLP 模型建立}

一级节点的位置影响 ULS 的地下管道建设成本和运输成本,从选址角度而言,应选择距离物流园区最近的、同时与该一级节点服务范围内各二级节点距离之和最小的备选位置作为一级节点。

在本问中,我们假定一级节点与服务范围内的二级节点成辐射状网络,二级节点间必须通过一级节点相连。则优化目标函数由各一级服务范围内 ULS 运输成本和物流园区运输成本构成,如式(1-6)所示。

\begin{equation}
F_{2} = \sum_{i=1}^{28} \sum_{j=1}^{28} c x_{ij}^{up} S_{i} p_{ij}^{up} + \sum_{j=1}^{28} \sum_{k=1}^{4} w_{jk} V_{k} D_{jk}
\tag{1-6}
\end{equation}

约束条件包括一级节点服务范围约束和物流园区约束。其中一级节点服务范围约束如式(1-7)~式(1-9)所示,物流园区约束如式(1-10)~式(1-11)所示。

\begin{equation}
\sum_{j=1}^{28} r_{j}^{up} = 4 \tag{1-7}
\end{equation}

\begin{equation}
x_{ij}^{up} - r_{j}^{up} \leq 0 \tag{1-8}
\end{equation}

\begin{equation}
\sum_{j=1}^{28} x_{ij}^{up} = 1 \tag{1-9}
\end{equation}

\begin{equation}
\sum_{j=1}^{28} w_{jk} = 1 \tag{1-10}
\end{equation}

\begin{equation}
\sum_{k=1}^{4} w_{ik} - r_{j}^{up} \leq 0 \tag{1-11}
\end{equation}

其中,式(1-7)为一级节点数量约束,表示一级节点数量固定为4;式(1-8)为辐射状网络约束,保证二级节点间不相连;式(1-9)为二级节点隶属唯一性约束,表示一个二级节点只属于一个一级节点服务范围;式(1-10)表示一级节点只能与一个物流园区相连;式式(1-11)式表示物流园区只能与一级节点相连。

所建模型为整数规划模型,变量和线性约束的数量分别为924和873,为了提高运算速度,本文采用LINGO软件编程,利用软件自带的B-N-P算法求解,源程序见附录一。

\paragraph{一级节点选址结果}

通过LINGO程序的算法实现,得到仙林地区的4个一级节点及其位置、相应的一级服务范围、货运量,结果如表1-2所示。

\begin{table}[h]
\centering
\caption{一级节点选址信息}
\begin{tabular}{|c|c|c|c|}
\hline
节点编号 & 位置 & 服务范围 & 货运量(吨) \\ \hline
1 & 808 & 799、808、816、826、829、873 & 17427 \\ \hline
2 & 810 & 800、810、815、823、833、834、841、842、849、855、878 & 31821 \\ \hline
3 & 858 & 851、858、866 & 8108 \\ \hline
4 & 891 & 877、879、891、892、893、894、896、899 & 19291 \\ \hline
\end{tabular}
\end{table}

一级节点的位置分布如图1-3所示,图中绿色节点代表一级节点。

\begin{figure}[h]
    \centering
    \includegraphics[width=\textwidth]{image.png}
    \caption{一级节点位置分布及服务范围}
    \label{fig:node_distribution}
\end{figure}

\paragraph{一级节点转运率}

根据转运率的定义:从物流园区经由最近的一级节点转运至其他所有一级节点的货物量占该物流园区总出货量的百分比,可知计算一级节点转运率的关键在于计算该一级节点服务范围内所有区域和与该一级节点相连物流园区的货运总量。一级节点转运率计算公式如式 (1-12) 所示。

\begin{equation}
\varphi_j = 1 - \frac{\text{一级节点服务范围与该物流园区货运量}}{\text{与一级节点相连物流园区的进出货物总量}}
\tag{1-12}
\end{equation}

物流园区只和与它距离最近的一级节点相连,且一个一级节点仅与一个物流园区相连,根据一级节点位置可以确定其与物流园区的关联关系,进而计算获得一级节点服务范围内所有区域与该物流园区之间的货运量和转运率,数据如表 1-3 所示。

\begin{table}[h]
\centering
\caption{一级节点转运率计算}
\label{tab:transit_rate}
\begin{tabular}{|c|c|c|c|c|}
\hline
节点编号 & 相连物流园区 & 服务范围与物流园区间货运量 (t) & 物流园区货运量 (t) & 转运率 \\ \hline
1 & 2 & 17986.08 & 37149.64 & 51.58\% \\ \hline
2 & 1 & 19639.07 & 37740.73 & 47.96\% \\ \hline
3 & 4 & 3478.30 & 17377.47 & 79.98\% \\ \hline
4 & 3 & 16437.32 & 36597.70 & 55.09\% \\ \hline
\end{tabular}
\end{table}

\subsection{问题二的求解}

本问是在第一问所得节点群的基础上设计各节点间连通路径,使 ULS 总成本最小。本问要求各一级节点转运率变化不大,即一级节点服务范围不变。在第一问中选择一级节点时,是以一级节点服务范围采用辐射状网络为前提,以各二级节点到一级节点、一级节点到对应园区的距离之和最短为优化目标,所得结果已经保证了二级节点到隶属一级节点的距离最短,因此在本问中一、二级节点位置都不调整。在此基础上,第二问简化为纯路径规划问题,分为三个部分:

1) 区域外路径规划:一级节点与物流园区的路径规划;

2) 区域间路径规划:一级节点间的路径规划;

3) 区域内路径规划:各一级节点服务范围内的路径规划。

规划思路如图 2-1 所示。

\begin{figure}[h]
\centering
\includegraphics[width=\textwidth]{image.png}
\caption{ULS 网络设计思路}
\end{figure}

其中,区域外路径规划可确定物流园区与一级节点间的网络构成、通道位置和流量,以及各一、二级节点的实际货运量。区域间路径规划采用整数规划方法,可确定一级节点间的网络构成、通道位置和流量。各区域内路径规划互不影响,分别采用最小生成树方法确定各一级节点服务范围内网络构成、各通道位置和实际流量。

\subsubsection{区域外路径规划}

物流园区和一级节点间的关联关系和货运量已知,如表 1-3 所示,为使通道长度最短,采用直线通道连接。物流园区与一级节点间的管道需满足流量约束。已知物流园区与一级节点间的收发货物量,如表 2-1 所示。

\begin{table}
\centering
\caption{物流园区与一级节点间的收发货物量}
\begin{tabular}{|c|c|c|c|}
\hline
物流园区 & 一级节点 & 园区发出货物量(吨) & 园区接收货物量(吨) \\
\hline
1 & 2 & 18441.37 & 19299.37 \\
\hline
2 & 1 & 18168.52 & 18981.11 \\
\hline
3 & 4 & 17912.88 & 18684.82 \\
\hline
4 & 3 & 8453.80 & 8923.68 \\
\hline
\end{tabular}
\end{table}

\begin{table}
\centering
\caption{物流园区与一级节点间通道建设方案及流量上限}
\begin{tabular}{|c|c|}
\hline
建设方案 & 单向流量上限(吨) \\
\hline
双向双轨(5吨) & 3600 \\
\hline
双向双轨(10吨) & 7200 \\
\hline
双向四轨(5吨) & 7200 \\
\hline
双向四轨(10吨) & 14400 \\
\hline
\end{tabular}
\end{table}

\begin{table}
\centering
\caption{物流园区网络构成及通道流量}
\begin{tabular}{|c|c|c|c|c|}
\hline
通道 & 建设方案 & 通道长度(/km) & 送出流量(吨) & 接收流量(吨) \\
\hline
园区1-一级节点2 & 双向四轨(10吨) & 6.543 & 14400 & 14400 \\
\hline
园区2-一级节点1 & 双向四轨(10吨) & 8.983 & 14400 & 14400 \\
\hline
园区3-一级节点4 & 双向四轨(10吨) & 7.606 & 14400 & 14400 \\
\hline
园区4-一级节点3 & 双向四轨(10吨) & 6.680 & 8453.80 & 8923.68 \\
\hline
\end{tabular}
\end{table}

根据区域外管道流量限制,确定各一级节点间的实际货运量如表2-4 所示,
一级节点间通道内流通的货物不仅包括物流园区的转运货物,还包括各一级节点
服务范围间产生的货物收发,例如一级节点 1 服务范围内各区域与一级节点 2
或3 服务范围内各区域间存在货物收发关系,这些跨一级区域的货物收发也将通
过一级节点间通道,表中行表示由该一级节点发出,列表示由该一级节点接收。
各一、二级节点的实际货运量如表2-5 所示。

\begin{table}
\centering
\caption{一级节点间货运量(/吨)}
\begin{tabular}{|c|c|c|c|c|}
\hline
 & 节点1 & 节点2 & 节点3 & 节点4 \\
\hline
节点1 & 0 & 6186.071 & 1174.048 & 5165.268 \\
\hline
节点2 & 4939.536 & 0 & 4987.547 & 4034.207 \\
\hline
节点3 & 2297.817 & 5871.356 & 0 & 2155.937 \\
\hline
节点4 & 6740.467 & 6514.993 & 906.4861 & 0 \\
\hline
\end{tabular}
\end{table}

\begin{table}
\centering
\begin{tabular}{|c|c|c|c|}
\hline
节点 & 实际货运量(吨) & 节点 & 实际货运量(吨) \\
\hline
二级节点1 & 2958.29 & 二级节点15 & 2999.97 \\
\hline
二级节点2 & 2799.75 & 二级节点16 & 2728.65 \\
\hline
二级节点3 & 2943.40 & 二级节点17 & 2345.29 \\
\hline
二级节点4 & 2773.92 & 二级节点18 & 2762.79 \\
\hline
二级节点5 & 2869.73 & 二级节点19 & 2889.45 \\
\hline
二级节点6 & 2905.86 & 二级节点20 & 2899.61 \\
\hline
二级节点7 & 2992.77 & 二级节点21 & 2988.64 \\
\hline
二级节点8 & 2802.60 & 二级节点22 & 2997.72 \\
\hline
二级节点9 & 2927.76 & 二级节点23 & 2981.07 \\
\hline
二级节点10 & 2982.93 & 二级节点24 & 2915.76 \\
\hline
二级节点11 & 2923.01 & 二级节点25 & 2976.65 \\
\hline
二级节点12 & 2808.76 & 二级节点26 & 2952.04 \\
\hline
二级节点13 & 2990.27 & 二级节点27 & 406.71 \\
\hline
二级节点14 & 2962.90 & 二级节点28 & 1161.40 \\
\hline
一级节点1(园区及一级节点间) & 53196.1 & 一级节点1(实际总货运量) & 56139.49 \\
\hline
一级节点2(园区及一级节点间) & 71956.82 & 一级节点2(实际总货运量) & 74730.74 \\
\hline
一级节点3(园区及一级节点间) & 21298.44 & 一级节点3(实际总货运量) & 23643.73 \\
\hline
一级节点4(园区及一级节点间) & 51496.02 & 一级节点4(实际总货运量) & 54477.1 \\
\hline
\end{tabular}
\caption{一、二级节点实际货运量}
\end{table}

物流园区至一级节点ULS通道总成本包括货物运输成本和地下物流隧道折旧成本,如式(2-1)所示。

\begin{equation}
F_{1}^{2}=c\sum_{k=1}^{4}\sum_{j=1}^{4}(Q_{jk}+Q_{kj})D_{jk}+\frac{5\times10^{8}}{365}\sum_{k=1}^{4}\sum_{j=1}^{4}D_{jk}\frac{\alpha(1+\alpha)^{y}}{(1+\alpha)^{y}-1}
\tag{2-1}
\end{equation}

将表2-3的数据带入式(2-1),计算得到物流园区至一级节点通道总成本为7261591.85元。

\subsubsection{区域间路径规划}

\paragraph{网络构成}

观察一级节点的地理分布可见,当一级节点顺次相连时地下同道总距离最短,因此在本问中一级节点间路径采用环形连通方式,实现各一级节点服务范围间的货物收发,如图2-2所示。

\begin{figure}[h]
    \centering
    \includegraphics[width=0.8\textwidth]{image.png} % 替换为实际图片路径
    \caption{一级节点路径图}
    \label{fig:primary_node_path}
\end{figure}

\paragraph{混合整数规划模型}

由表 2-4 可知,节点间的收发货物量很大,为满足通道流量限制,可能会存在货物分流,从不同路径到达目的节点。节点收发货物分流将影响一级节点间通道的流量需求,进而影响通道建设方案和折旧成本,因此合理安排货物分流,是降低通道折旧成本和货物运输成本的关键。

考虑到一级节点间采用环形流通路径,每个节点的货物分流存在两个方向选择,本文将其定义为左向路径和右向路径,左向路径是指按顺时针路径到达目的节点,右向路径是指按逆时针路径达到目的节点。设定决策变量矩阵 \( L \) 和 \( R \) 表征一级节点的货物分流,\( L \) 矩阵的第 \( i \) 行代表节点 \( i \) 采用左向路径到达目的节点 \( j \) 的货运量,\( R \) 矩阵的第 \( i \) 行代表节点 \( i \) 采用右向路径到达目的节点 \( j \) 的货运量。设定 \( D^L \) 和 \( D^R \) 矩阵分别表征节点 \( i \) 至目的节点 \( j \) 的左向路径距离和右向路径距离,一级节点间距离和流通路径已确定,\( D^L \) 和 \( D^R \) 矩阵已知,如式 (2-2) 和式 (2-3) 所示(单位:km):

\begin{equation}
D^L = 
\begin{bmatrix}
0 & 8.46 & 21.27 & 40.77 \\
38.91 & 0 & 12.81 & 32.31 \\
26.09 & 34.55 & 0 & 19.50 \\
6.59 & 15.05 & 27.87 & 0
\end{bmatrix}
\tag{2-2}
\end{equation}

\begin{equation}
D^R = 
\begin{bmatrix}
0 & 38.91 & 26.09 & 6.59 \\
8.46 & 0 & 34.55 & 15.05 \\
21.27 & 12.81 & 0 & 27.87 \\
40.77 & 32.31 & 19.50 & 0
\end{bmatrix}
\tag{2-3}
\end{equation}

设定决策变量向量 \( A \),代表与一级节点 \( i \) 顺时针相连通道的建设方案,\( A_i = 1 \) 代表采用双向四轨(5 吨);\( A_i = 0 \) 代表采用双向双轨(5 吨)。

一级节点间通道总成本包括货物运输成本和通道折旧成本,如式 (2-4) 所示。

\begin{equation}
F_2^2 = \min \left[ c \sum_{i=1}^4 \sum_{j=1}^4 (L_{ij} D_{ij}^L + R_{ij} D_{ij}^R) + \sum_{i=1}^4 \frac{(3 + 0.5 A_i) \times 10^8}{365} D_i \frac{\alpha (1 + \alpha)^y}{(1 + \alpha)^y - 1} \right]
\tag{2-4}
\end{equation}

式中,\( D_i \) 代表通道 \( i \) 的长度。

约束条件如下:

\begin{align}
L_{ij} + R_{ij} &= X_{ij} \tag{2-5} \\
\max(B_i, C_i) &\leq (A_i + 1) \times 3600 \tag{2-6}
\end{align}

其中,式(2-5)为货物总量平衡约束,节点 \(i\) 采用左向路径和右向路径发送至目的节点 \(j\) 的货物量之和等于节点 \(i\) 送给节点 \(j\) 的货物总量,\(X_{ij}\) 为一级节点 \(i\) 对目的节点 \(j\) 的送货量,参数见表 2-4。

式(2-6)为通道流量约束,\(B_i\) 为经由该通道顺时针运输的货物流量,\(C_i\) 为经由该通道逆时针运输的货物流量,二者最大值应小于通道流量上限。

\paragraph{区域间网络构成}

采用 LINGO 程序对上述混合整数规划模型求解,得到区域间通道位置、流量和建设方案,和区域间通道总成本,如表 2-6 所示。

\begin{table}[h]
\centering
\caption{区域间网络构成信息表}
\begin{tabular}{|c|c|c|c|}
\hline
编号 & 通道位置 & 建设方案 & 实际流量(顺时针/逆时针)(吨) \\
\hline
1 & 一级节点 1——一级节点 2 & 双向四轨(10 吨) & 13981.96/13976.83 \\
\hline
2 & 一级节点 2——一级节点 3 & 双向四轨(10 吨) & 13857.23/12822.17 \\
\hline
3 & 一级节点 3——一级节点 4 & 双向四轨(10 吨) & 12795.66/12874.12 \\
\hline
4 & 一级节点 4——一级节点 1 & 双向四轨(10 吨) & 13915.83/13945.26 \\
\hline
\multicolumn{3}{|c|}{区域间运输成本(/元)} & 108210.3 \\
\hline
\end{tabular}
\end{table}

\subsubsection{区域内路径规划}

已知二级节点与非本区域一级节点仅通过本区域一级节点连通,因此各一级区域内网络构成互不影响,本文分别设计各区域内网络构成。

\paragraph{目标函数分析与简化}

一级区域内网络总成本由三部分构成:节点折旧成本、通道折旧成本和货物运输成本 3 部分构成。一级区域内的一、二级节点数量位置固定,因此节点折旧成本视为固定成本。对一级区域总成本的优化关键在于对运输成本和通道折旧成本的优化。

接着我们再分析运输成本和通道折旧成本的构成。根据题意,货物运输成本由通道内货物流量和通道长度两个因素决定,通道折旧成本由通道建设方案即通道建设成本和通道长度两个因素决定。可见通道长度是二者的共同影响因素。通过对比两个成本对通道长度变化的敏感度,即可判断两种成本对总成本中的影响力。

假设区域内所有通道均采用双向四轨(5 吨),则通道长度变化 1 公里时,单位长度的双向运输成本最大变动幅度为 14400 元/公里,即运输成本对通道长度的最大敏感度为 14400 元/公里。假设区域内所有通道均采用建设成本最低的双向双轨(5 吨),则通道长度变化 1 公里时,单位长度的双向通道折旧成本变化 217340 元,即通道折旧成本对通道长度的最小敏感度为 217340 元/公里。可见通道折旧成本对通道长度的灵敏度远大于运输成本,当通道长度变化时,通道折旧成本则是网络总成本变化的主要因素,而运输成本的变化对总成本的影响力十分薄弱。因此,只要找到总距离最短的区域通道构成就能大幅度降低网络总成本。

据此,本文将由 3 部分成本构成的复杂的网络总成本优化问题,转化为区域内最短连通路径优化问题。

求固定节点间的最小连通路径就是典型的最小生成树[4-5]问题。求最小生成树的方法有破圈法、避圈法和 Dijkstra 算法,这些算法都是在图上手工操作,当图的规模较大时手工操作很费时间,例如本文中一级区域内节点数量最多有 11 个,程序设计很复杂。对此,我们把求最短连通路径(最小生成树)问题转化为整数规划,然后用 LINGO 软件求解,大大简化了编程难度,可迅速得到最优路径。

\paragraph{最小生成树的线性整数规划模型}

用节点 1 表示树根,区域内节点 \(i\) 到 \(j\) 的距离为 \(p_{ij}^{up}\),当两个节点之间没有线路相通时,两点之间距离用 M(很大的实数)表示。

设定 0-1 整数变量 \(z_{ij}\) 表示从 \(i\) 到 \(j\) 的边是否在树中,\(z_{ij}=1\) 表示在树中,\(z_{ij}=0\) 表示该边不在树中。

以树的路径最短为优化目标函数,如式(2-7)所示。

\[
\min \sum_{i=1}^{n} \sum_{j=1}^{n} p_{ij}^{up} z_{ij}
\tag{2-7}
\]

式中 \(n\) 为该一级区域内一、二级节点总数。

满足约束条件[5]:

(1)除树根外的每个节点都有线路通到(且只需要一次),表示为

\[
\sum_{i=1}^{n} z_{ij} = 1, j=2,3,\ldots,n, i \neq j
\]

(2)至少有一条线路从节点 1 出来

\[
\sum_{j=2}^{n} z_{1j} \geq 1
\]

(3)添加附加条件 \(u_{j}\),满足 \(u_{1}=0, 1 \leq u_{j} \leq n-1, j=2,3,\ldots,n\)

(4)为避免形成圈以及重复路线,应满足

\[
u_{j} \geq u_{k} + z_{kj} - (n-2)(1-z_{kj}) + (n-3)z_{jk}, 1 \leq k \leq n, 2 \leq j \leq n, j \neq k
\]

综上所述,区域内最短连通路径问题转化成如下整数规划:

\[
\begin{aligned}
& \min \sum_{i=1}^{n} \sum_{j=1}^{n} p_{ij}^{up} z_{ij} \\
& \text{s.t.} \left\{
\begin{aligned}
& \sum_{i=1}^{n} z_{ij} = 1, j=2,3,\ldots,n, i \neq j \\
& \sum_{j=2}^{n} z_{1j} \geq 1 \\
& u_{1} = 0, 1 \leq u_{j} \leq n-1, j=2,3,\ldots,n \\
& u_{j} \geq u_{k} + z_{kj} - (n-2)(1-z_{kj}) + (n-3)z_{jk}, 1 \leq k \leq n, 2 \leq j \leq n, j \neq k
\end{aligned}
\right.
\end{aligned}
\tag{2-8}
\]

其中,决策变量 \(z\) 是 0-1 型,约束变量 \(u\) 是整数型。

\paragraph{各区域网络构成}

假设区域内所有通道均采用建造成本最低的双向双轨(5 吨)建设方案,采用 LINGO 软件对式(2-8)所示模型求解,分别得到 4 个一级区域网络构成,如

\begin{table}
\centering
\begin{tabular}{|c|c|c|c|c|}
\hline
编号 & 通道位置 & 建设方案 & 实际流量(吨) & 通道长度 \\
\hline
1 & 799-808 & 双向双轨(5吨) & 2958.29 & 1553.40 \\
\hline
2 & 808-816 & 双向双轨(5吨) & 2905.86 & 1298.07 \\
\hline
3 & 808-826 & 双向四轨(5吨) & 8619.81 & 7662.23 \\
\hline
4 & 826-829 & 双向双轨(5吨) & 5817.21 & 7760.68 \\
\hline
5 & 829-873 & 双向双轨(5吨) & 2889.45 & 3740.47 \\
\hline
\multicolumn{3}{|c|}{运输成本(/元)} & 23190.62 & \\
\hline
\multicolumn{3}{|c|}{通道折旧成本(/元)} & 3037285.531 & \\
\hline
\multicolumn{3}{|c|}{总成本(/元)} & 3060476.151 & \\
\hline
\end{tabular}
\caption{一级节点1覆盖区域网络构成}
\end{table}

\begin{table}
\centering
\begin{tabular}{|c|c|c|c|c|}
\hline
编号 & 通道位置 & 建设方案 & 实际流量(吨) & 通道长度 \\
\hline
1 & 800-815 & 双向双轨(5吨) & 2799.75 & 2681.72 \\
\hline
2 & 815-810 & 双向四轨(5吨) & 14528.62 & 803.86 \\
\hline
3 & 810-823 & 双向四轨(5吨) & 11754.70 & 1154.72 \\
\hline
4 & 823-841 & 双向四轨(5吨) & 8761.93 & 1599.63 \\
\hline
5 & 841-842 & 双向双轨(5吨) & 5953.17 & 1223.11 \\
\hline
6 & 842-849 & 双向双轨(5吨) & 2962.90 & 2113.02 \\
\hline
7 & 815-833 & 双向四轨(5吨) & 11623.22 & 3042.72 \\
\hline
8 & 833-834 & 双向四轨(5吨) & 8640.30 & 1512.03 \\
\hline
9 & 834-855 & 双向双轨(5吨) & 5717.29 & 1715.22 \\
\hline
10 & 855-878 & 双向双轨(5吨) & 2988.64 & 3061.81 \\
\hline
\multicolumn{3}{|c|}{运输成本(/元)} & 75730.52 & \\
\hline
\multicolumn{3}{|c|}{通道折旧成本(/元)} & 2700211.768 & \\
\hline
\multicolumn{3}{|c|}{总成本(/元)} & 2775942.288 & \\
\hline
\end{tabular}
\caption{一级节点2覆盖区域网络构成}
\end{table}

\begin{table}
\centering
\begin{tabular}{|c|c|c|c|c|}
\hline
编号 & 通道位置 & 建设方案 & 实际流量(吨) & 通道长度 \\
\hline
1 & 851-858 & 双向双轨(5吨) & 5762.76 & 3497.93 \\
\hline
2 & 851-866 & 双向双轨(5吨) & 2762.79 & 4331.69 \\
\hline
\multicolumn{3}{|c|}{运输成本(/元)} & 8525.55 & \\
\hline
\multicolumn{3}{|c|}{通道折旧成本(/元)} & 1020990.278 & \\
\hline
\multicolumn{3}{|c|}{总成本(/元)} & 1029515.828 & \\
\hline
\end{tabular}
\caption{一级节点3覆盖区域网络构成}
\end{table}

\begin{table}
\centering
\begin{tabular}{|c|c|c|c|c|}
\hline
编号 & 通道位置 & 建设方案 & 实际流量(吨) & 通道长度 \\
\hline
1 & 877-879 & 双向双轨(5吨) & 2899.61 & 1318.19 \\
\hline
2 & 879-891 & 双向双轨(5吨) & 5879.33 & 6037.09 \\
\hline
3 & 891-894 & 双向四轨(5吨) & 10412.55 & 3487.90 \\
\hline
4 & 894-893 & 双向四轨(5吨) & 7460.52 & 2091.73 \\
\hline
5 & 893-899 & 双向双轨(5吨) & 4483.87 & 6382.49 \\
\hline
6 & 899-892 & 双向双轨(5吨) & 3322.46 & 2977.02 \\
\hline
7 & 892-896 & 双向双轨(5吨) & 406.71 & 5161.51 \\
\hline
\multicolumn{3}{|c|}{运输成本(/元)} & 34865.05 & \\
\hline
\multicolumn{3}{|c|}{通道折旧成本(/元)} & 4055729.294 & \\
\hline
\multicolumn{3}{|c|}{总成本(/元)} & 4090594.344 & \\
\hline
\end{tabular}
\caption{一级节点4覆盖区域网络构成}
\end{table}

\begin{figure}[h]
\centering
\includegraphics[width=0.8\textwidth]{image.png}
\caption{仙林地区ULS网络架构图}
\end{figure}

\subsection{问题三的求解}

本问要求通过增加、减少节点的个数,调整节点的位置或级别,增加、减少、改变路径的方法缩短货物运输总里程(同时节省运输时间),降低运输成本。

\subsubsection{区域间建设中继网络拓扑}

\paragraph{1) 缩短货物运输总里程,降低运输成本}

在第二问确定的网络架构中,拓展能力和调节能力最强的是区域间网络结构。对区域间网络结构的调整不会影响物流园区相连通道的建设和流量,也不会影响各一级区域内部网络构架。因此,本文优先分析区域间网络构成,通过调整一级节点数量和连通路径,在不影响区域内网络构成的前提下,提高整体线网的经济性和运输效率。

在第二问中,4 个一级节点间采用环形网架,不相邻节点间的传输距离增大,提高了货物运输成本。本文选择一级节点 13 连线和一级节点 24 连线的交叉点处设置一个一级中继节点,并与其他 4 个一级节点直线连接,网络拓扑如图 3-1 所示。

\begin{figure}[h]
    \centering
    \includegraphics[width=0.8\textwidth]{network_topology.png}
    \caption{区域间中继网络拓扑}
    \label{fig:network_topology}
\end{figure}

一级中继节点仅承担中继任务,不与任何物流园区或二级节点相连,也不服务任何区域货运需求。在该拓扑下,一级节点 1 和 3、一级节点 2 和 4 间优先选择通过中继一级节点实现直线运输,可大大降低一级节点间运输成本。以一级节点 1 为例,一级节点货物运输路径优先顺序如表 3-1 所示,根据运输距离排列,数值越小表示优先级越高。

\begin{table}
\centering
\caption{一级节点1货物运输路径优先级}
\begin{tabular}{|c|c|c|}
\hline
货物运输起讫点 & 运输路径 & 优先级排序 \\
\hline
\multirow{5}{*}{$1 \leftrightarrow 2$} & 通道1 & 1 \\
\cline{2-3}
 & 通道1-5$\leftrightarrow$通道2-5 & 2 \\
\cline{2-3}
 & 通道1-5$\leftrightarrow$通道3-5$\leftrightarrow$通道2 & 3 \\
\cline{2-3}
 & 通道4$\leftrightarrow$通道3$\leftrightarrow$通道2 & 4 \\
\cline{2-3}
 & 通道1-5$\leftrightarrow$通道4-5$\leftrightarrow$通道3$\leftrightarrow$通道2 & 5 \\
\hline
\multirow{5}{*}{$1 \leftrightarrow 3$} & 通道1-5$\leftrightarrow$通道3-5 & 1 \\
\cline{2-3}
 & 通道1$\leftrightarrow$通道2 & 2 \\
\cline{2-3}
 & 通道1-5$\leftrightarrow$通道2-5$\leftrightarrow$通道2 & 3 \\
\cline{2-3}
 & 通道4$\leftrightarrow$通道3 & 4 \\
\cline{2-3}
 & 通道1-5$\leftrightarrow$通道4-5$\leftrightarrow$通道3 & 5 \\
\hline
\multirow{4}{*}{$1 \leftrightarrow 4$} & 通道4 & 1 \\
\cline{2-3}
 & 通道1-5$\leftrightarrow$通道4-5 & 2 \\
\cline{2-3}
 & 通道1-5$\leftrightarrow$通道3-5$\leftrightarrow$通道3 & 3 \\
\cline{2-3}
 & 通道1$\leftrightarrow$通道2$\leftrightarrow$通道3 & 4 \\
\hline
\end{tabular}
\end{table}

由表3-1可见,采用原4节点环网中,不相邻节点间的环网路径优先级低于经中继一级节点直线转运。采用本问改进的区域间网络构成可缩短环网中不相邻节点间的货物运输距离,降低运输成本。

\paragraph{提高区域间网络抗风险能力}

在环形网络中,若某一条通道故障,该通道上流通货物必须通过相反路径(如将左向路径货物总量叠加到右向路径),导致剩余正常通道出现“拥堵”现象,即通道货物流量超过流量上限,严重情况下将导致区域间物流网络瘫痪,继而扩散到每个区域内物流网络,最终导致整个ULS系统瘫痪。

当增加一个中继一级节点时,若节点间任一条线路故障,都有至少3个备选路径完成货物运输,避免正常通道出现流量剧增而瘫痪。这种中继节点拓扑结构还能应对最极端情况,即所有节点间的环形网络均故障时,各节点仍可通过中继节点建立关联路径,延缓物流系统瘫痪,为道路施工创造时间。

当某方向货运量激增时,中继节点拓扑结构为各节点间的货运运输提供了至少4种备选路径,当某方向货运量激增时,节点可按优先级顺序一次安排最短路径,保证货物运输的同时降低运输成本。

\subsubsection{区域内建设环形辐射状网络拓扑}

第二问采用最小生成树法确定区域内节点连通路径,该路径为保证路径距离之和最短而舍弃了闭网结构,节点只与一个上级节点相连,若该通道故障,下游节点将处于孤岛状态,无法实现货物收发。可见最小生成树型网络结构十分脆弱,无法抵抗通道故障风险;当某通道货物量激增时,该节点没有备选路径进行分流,导致货物积压,运输效率低。

为了解决上述问题,本文在最小生成数型网络结构的基础上建立环网,保证两个节点间至少有两条路径选择。为了避免环网冗余而增加建设成本和折旧成本,本文选择将区域内所有没有下游节点连接起来实现通路,即可实现环网的作用,保证任一节点间至少有两条路径选择。

货运量越大的二级节点,其瘫痪损失就越大。为了提高货运量大的二级节点的安全裕度,本文要求此类节点与一级节点之间至少有 3 种备选路径,一级节点必须位于环线上。

根据上述建网要求,绘制环形辐射网,如图 3-2 所示。

\begin{figure}[h]
    \centering
    \includegraphics[width=0.8\textwidth]{image.png}
    \caption{区域内环形辐射状网络框架}
    \label{fig:3-2}
\end{figure}

\subsection{问题四的求解}

\subsubsection{ULS 线网评价与建设时序指标的选取}

根据题目的内容可知,确定影响因素对于地下物流系统的评价和建设时序指标是有重要意义的,必然会涉及到技术、经济效益和城市规划等方方面面的因素。

\paragraph{ULS 线网规划流程}

本研究对象是南京仙林地区 ULS 系统建设方案,重点是针对其线网布局以及线网建设时序的合理性进行研究。首先是确立评价指标体系,通过总结归纳,依据主成分分析法,选取轨道交通规划线网的评价指标,另外,通过专家问卷调查的形式确定轨道建设时序的评价指标。其次,通过对每项指标的量化分析,得出各项指标值,在通过恰当的数学评价方法,确定目标值。最后,通过对目标值的分析和研究,得出规划中的不足之处,并针对这些不足之处,给出合理有效的建议。主要流程如图所示:

\begin{figure}[h]
\centering
\includegraphics[width=0.8\textwidth]{ULS流程图}
\caption{ULS 线网规划流程图}
\end{figure}

\paragraph{ULS 线网评价指标选取}

针对 ULS 网络规划的现有研究内容,对已有评价指标进行了选取和梳理,构建了本文 ULS 线网综合评价指标体系,并根据相关分析和主成分分析,筛选其中的有效指标,结果如下表所示。

\begin{table}
\centering
\caption{表 4-1 ULS 线网评价指标选取表}
\begin{tabular}{|c|c|c|c|c|}
\hline
准则层 & 指标层 & 指标类型 & 指标说明 & 筛选结果 \\
\hline
\multirow{5}{*}{线路结构} & 线路长度系数 & 效益(定量) & 仙林地区所能容纳的 ULS 轨道长度与规划长度之间的关系系数 & 保留,C1 \\
\cline{2-5}
 & 节点设立系数 & 效益(定量) & 节点设立的合理系数 & 保留,C2 \\
\cline{2-5}
 & 非直线系数 & 成本(定量) & 影响线路弯曲路段的决定因素 & 保留,C3 \\
\cline{2-5}
 & 线路覆盖 & 效益(定量) & ULS 线网在区域内的覆盖率 & 不可观测删除 \\
\cline{2-5}
 & 位置系数 & 效益(定量) & ULS 线路在城市空间上的定量表示 & 相关分析删除 \\
\hline
\multirow{7}{*}{运营效果} & 日运货量 & 效益(定量) & 平均每天的运货量 & 相关分析删除 \\
\cline{2-5}
 & 日运货周转率 & 效益(定量) & 货运量与运货距离的乘积 & 相关分析删除 \\
\cline{2-5}
 & 平均运距 & 效益(定量) & 平均每天的货运距离 & 主成分分析删除 \\
\cline{2-5}
 & 线路负荷强度 & 效益(定量) & 单位线路长度的货运量 & 保留,C4 \\
\cline{2-5}
 & 货流断面不均匀系数 & 成本(定量) & 货流分布的均衡性 & 相关分析删除 \\
\cline{2-5}
 & 高峰最大断面货流 & 效益(定量) & 高峰货流时段,单位时间内的吨数 & 主成分分析删除 \\
\cline{2-5}
 & 高峰小时系数 & 效益(定量) & 高峰时间货运量与高峰时间货流流动率之比 & 主成分分析删除 \\
\hline
\multirow{2}{*}{社会效益} & ULS 占货运量比重 & 效益(定量) & 轨道运货量所占比重 & 保留,C5 \\
\cline{2-5}
 & ULS 运货节省时间 & 效益(定量) & 货物通过 ULS 所节约的时间 & 不可观测删除 \\
\hline
\multirow{2}{*}{发展战略} & 与物流的协调系数 & 效益(定量) & 单位面积上轨道的长度与轨道流量的相关系数 & 保留,C6 \\
\cline{2-5}
 & 沿线土地资源开发价值 & 效益(定性) & 线路土地附近的价值 & 相关分析删除 \\
\hline
\multirow{3}{*}{可实施性} & 与城市布局结构的协调 & 效益(定性) & 线路走向与城市规划的协调性 & 主成分分析删除 \\
\cline{2-5}
 & 工程难易程度 & 成本(定性) & 线路建造的难易程度 & 不可观测删除 \\
\cline{2-5}
 & 工程技术可行性 & 效益(定性) & 实施技术的可行性 & 不可观测删除 \\
\hline
\end{tabular}
\end{table}

下面针对上表中保留的指标进行详细解释:

线路长度系数(C1):实际线路总长与预测的轨道线路总长之比。该系数越大,ULS 线网的建设越好。

节点设立系数(C2):结合一级、二级节点的分类和节点货流两部分得到。节点系数越大,说明 ULS 节点设立越合理,即 ULS 的建设越合理。

非直线系数(C3):决定弯曲的轨道线路安全与成本的最终因子。非直线系数越大,弯曲线路的安全越高,成本越低,即轨道线网设计越合理。

线路负荷强度(C4):该城市 ULS 每天的货流量与城市轨道实际长度之比。即单位长度上每天的货流量,负荷强度越大,ULS 线网建设的越好。

ULS 占货运量比重(C5):ULS 货运量在城市总货运量当中的比重,该比重越大,说明 ULS 设计越合理就越多。

与物流的协调系数(C6):单位面积上轨道的长度与轨道流量的相关系数,即线路密度与货流密度两者的相关性。该值越大,则 ULS 线网设计的越合理。

\paragraph{ULS 线网规划建设时序指标选取}

确定 ULS 规划建设时序指标的关键问题在于确定规划建设时序的影响因素,而 ULS 建设时序问题在国内外的文献中都比较少提及,尚无一套完整的体系。因此,本文借鉴了轨道交通建设领域的相关研究,将以下 5 个指标作为 ULS 规划建设时序的指标,具体如下表所示:

\textbf{表 4-2 ULS 规划建设时序指标}

\documentclass{article}
\usepackage{array}

\begin{document}

\begin{table}[htbp]
\centering
\caption{ULS 规划建设时序指标}
\begin{tabular}{|c|l|p{10cm}|}
\hline
\textbf{编号} & \textbf{名称} & \textbf{含义} \\ \hline
S1 & 线路日物流负荷强度 & 线路日物流负荷强度是指该条线路承担的日物流量与线路长度之比。单位线路长度承担的货物流量越大,线路越繁忙。物流负荷强度越大,线路建设次序越高。 \\ \hline
S2 & 线路位置系数 & 线路位置系数是ULS线路在仙林地区所处位置定性分析的量化结果。将区域分为核心区域、中间区域以及外围区域,原则上,位置系数越大,线路的建设次序越高。 \\ \hline
S3 & 线路布局重要程度指数 & 线路布局重要程度指数是该条ULS线路在整体规划线网中所起作用的量化表示。ULS线网可以划分为延伸线路、辅助线路以及骨架线路,线路重要程度指数越大,则线路建设次序越高。 \\ \hline
S4 & 线路节点位置系数 & 线路节点位置系数是指线路节点所在位置的合理性,通过分析节点种类来说明节点的合理性。位置系数越大,说明该条线路建设次序越高。 \\ \hline
S5 & 线路与城市发展方向吻合系数 & 线路与城市发展方向吻合指数指城市总体规划中城市的发展方向ULS规划线路走向的吻合程度表示。吻合指数越大,则线路的建设次序越高。 \\ \hline
\end{tabular}
\end{table}

\end{document}

\subsubsection{仙林地区 ULS 线网评价与建设时序的指标分析}

针对上一小节的 ULS 线网评价和建设时序选取结果,本小节针对各个指标进行量化,并研究、推导其量化计算公式。

\paragraph{对指标的分析准备}

通过对前面指标的分析,本小节得出主要的指标准备值,包括货物需求量、仙林地区面积、物流强度、货物转运分布。

\paragraph{货物需求量}

货物需求量是指整个仙林地区的货物需求总量,本文以单位“吨”表示,文中已经给出货物需求量每年的增长为 \(5\%\),因此货物需求量的预测模型为:

\[
Q(t) = Q(t_0) \cdot (1 + 5\%)^{t - t_0}
\tag{4-1}
\]

式中,$Q$ 表示当年货物需求量;$t_0$ 表示初始年份。

\paragraph{仙林地区面积}

根据题中所给 OD 表中数据,将所有区域面积相加,得到整个南京市仙林地区面积为 293.76 平方公里,本文认为仙林地区的行政区划面积在未来 30 年不变。

\paragraph{物流强度}

物流强度是指一定时间内通过两物流点间的物料的数量,也叫做物流量,定义为总货物需求量乘以平均运距。如何对给定的物流系统中的物流量进行量化,是物流系统分析、规划、设计和评价的前提。可以看出,伴随着城市规模的扩大以及居民生活水平的日益提高,物流强度必然呈现增长趋势。但是考虑到环境承载能力、土地资源限制,其增长不会是无限度的,而是呈现 S 型曲线,如以下公式所示:

\begin{equation}
D(t) = D(t_0) \left[ 1 + r_2 \left( 1 - \frac{D(t)}{D_m} \right) (t - t_0) \right]
\tag{4-2}
\end{equation}

式中,$D(t)$ 表示 $t$ 年度的日均物流强度,$D_m$ 表示物流强度最大值,$r_2$ 表示增长率,对该式取极限后可得物流强度的预测模型为:

\begin{equation}
\begin{cases}
\frac{dD(t)}{dt} = r_2 \left( 1 - \frac{D(t)}{D_m} \right) D(t) \\
D(t_0) = D_0
\end{cases}
\tag{4-3}
\end{equation}

对上式进行求解可得微分方程的解为:

\begin{equation}
D(t) = \frac{D_m}{1 + \left( \frac{D_m}{D_0} - 1 \right) e^{-r_2 (t - t_0)}}
\tag{4-4}
\end{equation}

根据远期规划,本文以园区 4 到区域中心点 896 的最远距离和最大地面交换量 3000 吨作为 $D_m$ 的计算标准,计算可得:$D_m = 1.13 \times 10^8$。

\paragraph{仙林地区 ULS 线网评价指标分析}

本小节对线网评价指标进行分析。

\paragraph{线路长度系数}

ULS 规划线路总长度 $L$ 可以通过土地面积、远期需求量进行预测,其数学模型为:

\begin{equation}
L = a_0 \cdot Q^{a_1} \cdot S^{a_2}
\tag{4-5}
\end{equation}

式中,$a_0$、$a_1$、$a_2$ 均为待定系数,本文依据轨道交通规划的有关研究,将其取值定为 1.839、0.640、0.1996;$Q$ 为未来期望货物需求量;$S$ 为区域面积,本文认为不变。结合规划线网总长度 $L_0$,可以确定线路长度指标系数 $C_1$,公式为:

\begin{equation}
C_1 = \frac{L_0}{L}, \, L_0 \leq L
\tag{4-6}
\end{equation}

可以看出,$C_1$ 取值越接近 1,越能够说明 ULS 线网总长切合实际、合理。

\paragraph{节点设立系数}

ULS 的节点可以分为两类,一类是园区节点,用于货物集散;另一类则是中间的一级、二级节点,即货物转运收发点,是物流吸引点。因此,节点的合理选择推动城市交通的优化以及城市经济的发展。在文中,物流吸引点被分为一级节点 $q_1$ 和二级节点 $q_2$ 两类。考虑到物流园区已建成无法改建,故本文的节点设立系数也是针对物流吸引点而言的。

节点设立系数表示为:

\begin{equation}
C_2 = \overline{Q}(1 + \lambda)
\tag{4-7}
\end{equation}

式中,$\overline{Q}$ 表示每个节点货物流量的平均值,$\lambda$ 表示每个站点的权重系数,本文一级节点取 0.57,二级节点取 0.43。

\paragraph{线路负荷强度}

ULS 在地区物流运输中的分担,表明 ULS 线路规划设计的合理。但是,说明 ULS 性能优劣,还需要考虑到轨道的线网程度。进而,线路负荷强度可表示为单位线路长度承载的货物数量,同时在一定程度上,ULS 线路负荷强度反映了被评价线网的运营效率和经济性。从上可知,ULS 线路负荷强度 $C_3$ 可表示为:

\begin{equation}
C_3 = \frac{M}{L_0}
\tag{4-8}
\end{equation}

式中:$M$ 为 ULS 系统在规划线网程度下每天的运货量(单位:吨),$L_0$ 为 ULS 规划线路总长(单位:千米)。从宏观上讲,每个较为完善 ULS 系统的平均负荷强度的大小可以说明其合理与否。因此,ULS 线路负荷强度 $C_3$ 越大,则表明相应的 ULS 线网设计越合理。

\paragraph{ULS 占货运量比重}

ULS 占货运量的比重指标 $C_4$ 表示为:

\begin{equation}
C_4 = \frac{M}{M_{total}}
\tag{4-9}
\end{equation}

式中,$M_{total}$ 表示全天仙林地区总的货运量。其值越大则表明 ULS 承担的物流任务越多,发挥的效益也就越大。

\paragraph{与物流的协调系数}

根据前三问的模型不难发现,各区域 OD 图对 ULS 线网规划起到很大的影响。基于此,本小节通过对线路密度与各区域中心点密度进行规划研究,提出并分析与物流的协调系数这一评价指标的概念。

首先,定义轨道线路密度为一个二级节点服务范围内 ULS 轨道线路的长度,其表达式为:

\begin{equation}
a_i = \frac{L_i}{S_i}
\tag{4-10}
\end{equation}

式中,$S_i$ 为第 $i$ 个二级节点服务面积,本文中可以近似认为是二级节点为圆心半径为 3km 的圆,$L_i$ 为范围内轨道线路长度,$a_i$ 则表示为轨道线路的密度。

其次,定义二级节点服务密度为:

\begin{equation}
b_i = \frac{n_i}{S_i}
\tag{4-11}
\end{equation}

式中,$n_i$ 为一个二级节点所服务的区域中心点个数,$b_i$ 则表示二级节点服务密度。

\begin{figure}[h]
    \centering
    \includegraphics[width=0.8\textwidth]{image.png}
    \caption{区域示意图}
    \label{fig:region_diagram}
\end{figure}

将所有的轨道线路密度、二级节点服务密度整合为向量形式,向量的表达式以及与物流的协调系数定义为:

\begin{equation}
\begin{cases}
\overrightarrow{A}=(a_{1},a_{1},\ldots,a_{1}) \\
\overrightarrow{B}=(b_{1},b_{1},\ldots,b_{1})
\end{cases}
\tag{4-12}
\end{equation}

\begin{equation}
C_{5}=\frac{\overrightarrow{A}\cdot\overrightarrow{B}}{|\overrightarrow{A}|\cdot|\overrightarrow{B}|}
\tag{4-13}
\end{equation}

由上式可以看出,$C_{5}$越大,ULS线路与物流需求点的相关性越好,则规划更具有合理性。

\paragraph{仙林地区 ULS 线网建设时序指标分析}

本小节对线网建设时序指标进行分析。

\paragraph{线路日物流负荷强度}

线路日物流负荷强度即该条线路上单位长度承担的物流量。线路日物流负荷强度在一定程度上反映了被研究线路在总体线网中的运营效率和经济性。通过上述分析可知,线路日物流负荷强度 $S_{1}$ 可以表示为:

\begin{equation}
S_{1}=\frac{m}{l}
\tag{4-14}
\end{equation}

式中,$m$ 表示该条 ULS 线路日货运量,$l$ 表示该条 ULS 线路的长度。根据相关性,已知该条 ULS 日货运量,与该条线路的位置、线路布局重要程度以及与城市发展吻合成正比关系。本文假定一个数量级系数 $a$,而三个系数存在同等数量级上,那么该条线路的日货运量 $m$ 可表示为:

\begin{equation}
m=a(S_{2}+S_{3}+S_{5})
\tag{4-15}
\end{equation}

因此,对于单条线路而言,该条 ULS 线路日物流负荷强度在数值上能说明线路建设的优先次序,即该条线路日物流负荷强度越高,则该条线路的建设次序越高。

\paragraph{线路位置系数}

线路位置系数是该条 ULS 线路在城市中所处位置定性分析之后的量化表示。本文根据区域拥堵系数将南京市仙林地区分为 5 个层次共 11 级数值指标,具体如下表所示。若设计的 ULS 线路所连接的两个区域为不同层级,则取其平均值,从原则上讲,位置系数越大,线路的建设次序越高。

\begin{table}
\centering
\caption{线路位置系数}
\begin{tabular}{|c|c|}
\hline
位置 & 拥堵系数 \\
\hline
畅通区域 & 0-2 \\
\hline
基本畅通区域 & 2-4 \\
\hline
轻度拥堵区域 & 4-6 \\
\hline
中度拥堵区域 & 6-8 \\
\hline
严重拥堵区域 & 8-10 \\
\hline
\end{tabular}
\end{table}

\paragraph{线路布局重要程度指数}

线路布局重要程度指数是将 ULS 线路按其作用进行划分之后,对该线路作用的定量表示。一般按作用功能将轨道交通线路分为骨架线路、辅助线路和延伸线路,用十级数值指标对线路进行标定。线路布局重要程度指数越大,线路的建设次序越高。

\paragraph{线路节点位置系数}

线路节点位置系数是指线路上节点所在位置的合理性,主要是通过分析节点种类来说明线路节点的合理性。本文线路节点位置系数 $S_4$ 表示为该条线路上节点权重的平均取值,具体表现形式为:

\begin{equation}
S_4 = \frac{1}{n} \sum \lambda
\tag{4-16}
\end{equation}

可以看出,该条规划线路站点位置系数越大,说明该条线路建设次序越高。

\paragraph{线路与城市发展方向吻合系数}

线路与城市发展方向吻合指数是指城市总体规划中城市的发展方向与该条 ULS 规划线路走向的吻合程度表示。吻合指数越大,线路的建设次序越高,可以通过专家评估的方法确定线路的吻合指数。

\subsubsection{仙林地区 ULS 规划建设时序研究方法}

ULS 规划建设时序研究方法是综合评价方法的应用,目前综合评价方法多种多样,如模糊评价法、加权平均法、层次分析法、主成分分析法、熵权评价法、灰色关联评价法、TOPSIS 方法和数据包络法等。本文结合层次分析法、熵权评价法、加权评价法和 TOPSIS 方法,分析仙林地区 ULS 规划建设时序问题。

\paragraph{时序指标评价方法介绍}

\paragraph{层次分析法}

层次分析法(以下简称 AHP 法)早在 21 世纪 70 年代中期由美国运筹学家托马斯·塞蒂就正式提出。该方法的思路是:将一个多目标的决策问题看作是一个系统,并将目标的问题分解,使其转化为多个准则,并把它分解为多个指标的各个层次,通过对定性指标的大致量化,计算出不同层次的权重,用来为多方案的优化与决策。AHP 法不仅存在不定性主观信息的情况下适用,还允许通过符合逻辑的方式,但它的评价结果很大程度上由人的主观意志决定,具有较强的主观性。

\paragraph{熵权评价法}

ULS 建设时序是一项非常复杂的系统工程,在参考相关专家们主观丰富经验的同时,也不能忽视指标本身所蕴含的信息。因此,研究在采用 AHP 法的同时,也采用了熵权分析法来处理指标权重。按照信息论最基本的原理,信息作为系统有序程度的度量,熵则是系统无序程度的度量,有可以理解为在客观上,原始指标信息数据的本身也存在一定的数学关系,它不依赖与人们的主观经验判断,具有的客观性较强。运用熵权法就是将指标本身的数据进行处理,得到它客观的权重。

\paragraph{加权评价法}

加权评价法主要是考虑指标体系中不同的指标所处的位置和作用的不同,给每个指标加上不同的权重来予以区别,从而构建指标的决策矩阵。具体形式可表示为:

\[
U = \sum_{i}^{n} w_i s_i, \sum_{i}^{n} w_i = 1
\]

式中,$U$ 为加权后的指标值,$w_i$ 为第 $i$ 个指标的权重,$s_i$ 为第 $i$ 个指标的指标值。

\paragraph{TOPSIS 方法}

TOPSIS 法是用来处理指标决策问题的多方案排序和选择的方法,它的基本思想是:依据理想点的理论原理,找寻距离理想点最近的方案。并通过计算对象与最优解、最劣解的距离大小,确定顺序。即先设定一个虚拟的最优解(又称正理想解)和一个最劣解(又称负理想解),将各备选方案与正负理想解相互比较,若方案最靠近最优解即又距最劣解最远,为最好。另外,TOPSIS 方法需要的评价指标决策矩阵和指标权重,都由上面三种方法给出。

\paragraph{时序指标研究方法}

\paragraph{建设时序指标权重确定方法}

本文采取主观与客观相结合的赋权法。传统确定指标权重的方法主要包括主观赋权法和客观赋权法这两种。主观赋权法主要是根据专家自己的主观经验确定,这种方法的缺点就是客观性差;而客观赋权法则是根据指标的原始数据处理得到,它不靠主观判断,客观性上较强。在城市轨道交通规划建设时序中,采用主观与客观赋权法相结合,既考虑专家们的丰富主观经验,又包含指标本身所蕴含的实际客观信息。文中采用 AHP 法(主观赋权法)和熵权法(客观赋权法)来确定建设时序指标的权重。具体内容如下:

AHP 法:其基本原理是采用(9,7,5,3,1)分层次法构造判断矩阵,经层次单排序及一致性检验和层次总排序及一致性检验,求得指标权重。记 AHP 法求得指标 $j$ 的权重为 $w_j'$。

熵权法:若有 $n$ 个对象,$m$ 个评价指标,初始化决策矩阵 $A = (x_{ij})_{n \times m}$,$(i = 1, 2, \ldots, n; j = 1, 2, \ldots, m)$。由于熵的变量一般取值范围在 0-1 之间,为了确保数据符合要求,必须要对初始评价数据进行预先的处理,采用处理的方法是归一化:

\[
\lambda_{ij} = \frac{x_{ij}}{\sum_{i=1}^{n} x_{ij}}
\]

从而得到标准化决策矩阵 $B$,其中元素为 $\lambda_{ij}$,而某指标 $x_j$ 的信息熵为:

\[
e_j = -k \sum_{i=1}^{n} \lambda_{ij} \ln(\lambda_{ij})
\]

其中 $k = 1 / \ln n$,则第 $j$ 个的指标熵权 $w_j''$ 可以定义为:

\[
w_j'' = \frac{1 - e_j}{\sum_{j=1}^{m} (1 - e_j)}
\]

因此,结合 AHP 法和熵权法最终得到第 $j$ 个指标的综合权重为:

\begin{equation}
w_{j}=\frac{w_{j}^{\prime} w_{j}^{\prime \prime}}{\sum_{j=1}^{m} w_{j}^{\prime} w_{j}^{\prime \prime}}
\tag{4-21}
\end{equation}

\paragraph{构建标准化决策矩阵}

对于 $n$ 条需要规划排序的线路

\begin{equation}
A=\left\{A_{1}, A_{2}, \ldots, A_{n}\right\}, \quad N=\{1,2, \ldots, n\}
\tag{4-22}
\end{equation}

建设时序指标为

\begin{equation}
S=\left\{S_{1}, S_{2}, \ldots, S_{m}\right\}, \quad M=\{1,2, \ldots, m\}
\tag{4-23}
\end{equation}

则构建的初始化决策矩阵为:

\begin{equation}
A=\left(a_{i j}\right)_{n \times m},(i \in N, j \in M)
\tag{4-24}
\end{equation}

本文采用极差变换法将初始矩阵标准化,消除不同量纲指标对决策的影响,对于效益型指标,其规范化公式为:

\begin{equation}
b_{i j}=\frac{a_{i j}-\min _{i} a_{i j}}{\max _{i} a_{i j}-\min _{i} a_{i j}},(i \in N, j \in M)
\tag{4-25}
\end{equation}

对于成本型指标,其规范化公式为:

\begin{equation}
b_{i j}=\frac{\max _{i} a_{i j}-a_{i j}}{\max _{i} a_{i j}-\min _{i} a_{i j}},(i \in N, j \in M)
\tag{4-26}
\end{equation}

经过极差变换后,得到的规范化标准决策矩阵为:

\begin{equation}
A=\left(a_{i j}\right)_{n \times m},(i \in N, j \in M)
\tag{4-27}
\end{equation}

\paragraph{构建加权标准决策化矩阵}

加权标准决策化矩阵为:

\begin{equation}
C=\left(c_{i j}\right)_{n \times m}=\left(b_{i j} w_{j}\right)_{n \times m},(i \in N, j \in M)
\tag{4-28}
\end{equation}

\paragraph{采用 TOPSIS 模型判定建设时序}

首先,确定最优劣解的规划建设时序方案。依据 TOPSIS 法的基本原理,假定最优解为所有的线路规划建设时序的指标都达到了最优解的建设时序方案,而最劣解则恰恰相反,具体表现形式为:

\begin{equation}
\begin{cases}
C^{+}=\left\{\left(\max _{j} c_{i j} \mid S_{j} \in Q_{1}\right),\left(\min _{j} c_{i j} \mid S_{j} \in Q_{2}\right)\right\},(i \in N, j \in M) \\
C^{-}=\left\{\left(\min _{j} c_{i j} \mid S_{j} \in Q_{1}\right),\left(\max _{j} c_{i j} \mid S_{j} \in Q_{2}\right)\right\}
\end{cases}
\tag{4-29}
\end{equation}

式中,$C^{+}$ 和 $C^{-}$ 分别表示最优劣的建设时序方案,$Q_{1}$ 是效益型指标集,$Q_{2}$ 是成本型指标集。

其次,计算规划线路的指标值与理想方案之间的差距。若线路 $A_{i}$ 距最优解方案为 $D_{i}^{+}$,距最劣解方案为 $D_{i}^{-}$,则欧几里得距离为:

\begin{equation}
\begin{cases}
D_{i}^{+}=\sqrt{\sum_{j=1}^{m}\left(c_{ij}-c_{j}^{+}\right)^{2}} \\
D_{i}^{-}=\sqrt{\sum_{j=1}^{m}\left(c_{ij}-c_{j}^{-}\right)^{2}}
\end{cases}
\tag{4-30}
\end{equation}

最后,计算研究规划建设时序与理想建设时序方案的贴近度。

\begin{equation}
E_{i}=\frac{D_{i}^{+}}{D_{i}^{+}+D_{i}^{-}},(i\in N)
\tag{4-31}
\end{equation}

式中:$E_{i}$ 为 ULS 规划线路 $A_{i}$ 与理想解的方案贴近度。贴近度 $E_{i}$ 则反映了 ULS 规划线路 $A_{i}$ 贴近最优解(正理想解)建设时序方案,或者也远离最劣(负理想)建设时序方案的程度。可以看出 $E_{i}$ 的取值在 0 到 1 之间,当贴近度 $E_{i}$ 接近 0 时,说明 ULS 规划线路戒靠近最优解(正理想解)建设时序方案同时远离最劣解(负理想解)建设时序方案,即线路 $A_{i}$ 需求度越高,建设次序越高;同时,当 $E_{i}$ 接近 1 时,说明城市轨道交通规划线路 $A_{i}$ 远离最优解(正理想解)建设时序方案同时靠近最劣解(负理想解)建设时序方案,即城市轨道交通规划线路 $A_{i}$ 需求度越低,建设次序越低。

\subsubsection{仙林地区 ULS 规划建设时序与演进结果}

考虑货运量需求每年 5\% 增长的原则,通过顶层设计,本文对 30 年后仙林地区的地下物流系统设计如下图所示。相关节点和线路规格如图例所示。

\begin{figure}[h]
\centering
\includegraphics[width=0.8\textwidth]{image.png}
\caption{30 年后仙林地区 ULS 拓扑图}
\end{figure}

下面根据 5.4.1-5.4.4 部分相关参数的计算结果,绘制如下仙林地区 8 年的增长结果:

(1) 第 1 年

\begin{figure}[h]
    \centering
    \includegraphics[width=\textwidth]{image1.png}
    \caption{第 1 年增长情况}
    \label{fig:year1}
\end{figure}

(2) 第 2 年

\begin{figure}[h]
    \centering
    \includegraphics[width=\textwidth]{image2.png}
    \caption{第 2 年增长情况}
    \label{fig:year2}
\end{figure}

(3) 第 3 年

\begin{figure}[h]
    \centering
    \includegraphics[width=\textwidth]{image3.png}
    \caption{第 3 年增长情况}
    \label{fig:year3}
\end{figure}

\begin{figure}[h]
    \centering
    \includegraphics[width=\textwidth]{image1.png}
    \caption{第 3 年增长情况}
    \label{fig:year3}
\end{figure}

(4) 第 4 年

\begin{figure}[h]
    \centering
    \includegraphics[width=\textwidth]{image2.png}
    \caption{第 4 年增长情况}
    \label{fig:year4}
\end{figure}

(5) 第 5 年

\begin{figure}[h]
    \centering
    \includegraphics[width=\textwidth]{image3.png}
    \caption{第 5 年增长情况}
    \label{fig:year5}
\end{figure}

\begin{figure}[h]
    \centering
    \includegraphics[width=\textwidth]{image1.png}
    \caption{第 5 年增长情况}
    \label{fig:year5}
\end{figure}

(6) 第 6 年

\begin{figure}[h]
    \centering
    \includegraphics[width=\textwidth]{image2.png}
    \caption{第 6 年增长情况}
    \label{fig:year6}
\end{figure}

(7) 第 7 年

\begin{figure}[h]
    \centering
    \includegraphics[width=\textwidth]{image3.png}
    \caption{第 7 年增长情况}
    \label{fig:year7}
\end{figure}

\begin{figure}[h]
    \centering
    \includegraphics[width=\textwidth]{image1.png}
    \caption{第 7 年增长情况}
    \label{fig:year7}
\end{figure}

(8) 第 8 年

\begin{figure}[h]
    \centering
    \includegraphics[width=\textwidth]{image2.png}
    \caption{第 8 年增长情况}
    \label{fig:year8}
\end{figure}

\section{模型总结与评价}

1. 本文采用分层 ULS 物理节点选址规划策略,底层规划采用集合覆盖法确定数量最少的二级节点选址,上层规划采用 CFLP 法在二级节点中按辐射状网络下运输距离之和最短为目标选择 4 个一级节点,分别为混合整数规划和整数规划问题。考虑到变量和约束条件数量庞大,本文采用 LINGO 软件编程求解,大大降低了编程难度,节省了运算时间。

2. 采用分层设计原则由外向内,依此设计了区域外(物流园区与一级节点)、区域间(一级节点之间)和区域内(一级节点与所覆盖二级节点间)的网络构成。区域外网络以保证进出物流园区的货物尽最大可能放入地下运输为原则选择双向四轨(10 吨)直线通道建设方案。区域间网络考虑节点货运分流,采用整数规划法设计了环形通道网络,充分利用管道流量上限,降低通道折旧成本。区域内根据距离敏感度分析确定通道折旧成本是总成本的决定因素,以通道距离之和最短为目标,建立最小生成树模型,采用线性整数规划法分别确定了 4 个区域的最小生成树型网络构成。该网络构成保证了节点间连通路径之和最小,即通道折旧成本最低,等效于总成本最低。

3. 针对区域间环形网络鲁棒性差和不相邻节点间运输成本高的问题,本文提出了增加一级中继节点在环网不相邻节点间建立直线转运的网路优化方案,有效缩短了货物运输总里程,增加了节点间运输备用路径,提高区域间网络鲁棒性;针对区域内最小生成树型网络两节点建仅有一条连通路径问题,提出了建立环形放射状网络的优化思路,保证每个节点与一级节点间至少有一个备用路径,提高区域内网络抗干扰能力。但由于时间仓促,该部分以理论验证为主,后续我们将建立普遍试用的网络优化体系和模型。

4. 在考虑仙林地区未来 30 年货物需求量以每年 5\% 增长的情况下对 ULS 线网规划进行顶层设计的问题,本文通过分析决策 ULS 线网建设时序的五个影响指标,对各项指标进行了量化,建立了相应的数学模型;根据建设时序指标值的计算结果,通过运用层次分析法、熵权评价法、TOPSIS 方法等综合评价方法计算了未来 8 年时间里南京市仙林地区 ULS 线网规划的理论建设时序。该部分通过对建设时序指标的分析和量化,较好地模拟了 ULS 线网规划时序,但是该部分的数据处理部分由于时间仓促未能在正文中完整呈现,与问题三中线网规划模型的对比和比较工作尚未能展开。

[REFERENCES:1]

\end{document}