\documentclass{article}
\usepackage{ctex}
\usepackage{geometry}
\geometry{a4paper,scale=0.8}

\title{构建地下物流系统网络}
\author{}
\date{}

\begin{document}

\maketitle

\begin{abstract}
随着城市货运物流需求的快速增长,构建地下物流系统对于缓解城市交通拥堵、减轻城市污染具有重要的意义。本文针对南京市仙林地区的地下物流运输问题,通过运用圆覆盖算法、基于最近邻的改进算法寻找二级节点,通过 K-means 算法确定一级节点,通过最优化模型构建出地下物流系统网络模型。该模型以缓解交通拥堵、降低物流成本为目标,较为科学合理地设计出网络构成,并综合考虑南京市未来的交通需求增长与系统抗风险能力,设计出地下物流系统的建设时序及动态演进过程。

\textbf{问题一:} 根据题意需要将区域交通情况改善为至少基本畅通,我们认为拥堵指数降为 4 即满足条件。根据交通拥堵指数与区域货运总量成正比这一假设,计算出各区域需要转入地下物流系统运输的货运量。根据区域中心点的离散程度及区域整体形状,以 890 区域的左边线为界,将整个区域划分为右侧区域与左侧区域:在右侧区域运用圆覆盖模型,在左侧区域运用基于最近邻的改进算法,最终找出所有节点共 29 个,分别给出位置坐标;接着运用 K-means 聚类算法确定其中一级节点共有 4 个,剩余 25 个为二级节点;最终确定各级节点的服务区域及实际货运量,以及各一级节点的转运率:最小为 58.96\%,最大为 68.12\%。

\textbf{问题二:} 在已确定地下物流节点的基础上,选择合适的地下路线以建立该区域的网络拓扑结构。我们先对各区域全天 OD 流量矩阵进行预处理,得到每天需通过地下运输的 OD 流量矩阵。以总成本为目标函数,在满足运输要求的约束下最小化目标函数,通过求解最优解,可以找到最合适的边,进而确定网络的拓扑结构。根据网络拓扑结构可以进一步确定各节点的实际货运量和网络中管道的实际流量,计算出该地下物流网络的建设成本为 597.04 亿元,每天的运输成本为 112.49 万元,折旧费用为 165.84 万元,最后得到每天的总成本为 278.33 万元。

\textbf{问题三:} 在问题一和问题二已经确定整个网络结构的情况下,仿真运输过程,定义管道中实际流量与管道最大负载流量的比值为管道利用率;定义各节点实际货物流量与所有同级节点货物流量均值的比值与 1 的差为节点的负荷度。通过分析数据,发现在现有网络下主网(园区-一级节点共同构成的网络)中管道 $e_{z_3a_3}$ 利用率偏高,管道 $e_{a_3a_4}$ 利用率偏低;子网(一级节点与其所包含二级节点构成的网络)中管道 $e_{a_{25}a_{26}}$ 利用率偏高,管道 $e_{a_{34}a_{35}}$ 利用率偏低。节点方面 $a_{24}, a_{26}$ 负荷度偏高,$a_{43}$ 偏低。针对利用率高管道和负荷度高的节点,在其周围适当改设服务节点级别,对利用率较低的管道和负荷度较低的节点不作改变,用作抗风险的备用。经过调整以后网络的运输成本有所下降,并且运输速度显著提高。

\textbf{问题四:} 对 ULS 做好顶层设计,使其能够满足该区域未来 30 年交通需求的目标。交通需求每年呈 5\% 增长,我们求得 30 年后的货运需求为目前货运量的 4.32 倍,为满足需求量,我们提出三项扩容方案,包括将双向双轨通道改建为双向四轨,增设一级节点以及延长班车运营时间。由于每年可建设道路长度大致相等,根据园区与一级节点之间的通道、一级节点之间的通道、一级节点与二级节点之间以及二级节点之间的通道长度之间的比例确定权重,按照权重分配建设年限,计算出三类通道需要的建设年限分别为 2 年,3 年,3 年。

关键词:物流网络 圆覆盖 最近邻改进算法 最优化 负荷度 扩容
\end{abstract}

\section*{“华为杯”第十四届中国研究生数学建模竞赛}

\section*{题目}
构建地下物流系统网络

\section{目录}

\section{一、问题重述}
\subsection{1.1 研究背景}
\subsection{1.2 研究问题}

\section{二、问题分析}
\subsection{2.1 问题一的分析}
\subsection{2.2 问题二的分析}
\subsection{2.3 问题三的分析}
\subsection{2.4 问题四的分析}

\section{三、基本假设}

\section{四、符号说明}

\section{五、模型的建立与求解}
\subsection{5.1 问题一的模型建立与求解}
\subsubsection{5.1.1 数据的预处理}
\subsubsection{5.1.2 圆覆盖算法及最近邻规则改进算法确定节点位置}
\subsubsection{5.1.3 K-means 聚类算法确定节点构成}

\subsection{5.2 问题二的模型建立与求解}
\subsubsection{5.2.1 数据的预处理}
\subsubsection{5.2.2 最优化模型}
\subsubsection{5.2.3 网络的构成}
\subsubsection{5.2.4 每日总成本的计算}

\subsection{5.3 问题三的模型建立}
\subsubsection{5.3.1 实际运行时轨道状况的分析}
\subsubsection{5.3.2 网络优化部分的选择}

\subsection{5.4 问题四的模型建立}
\subsubsection{5.4.1 扩容的相关方案建议}
\subsubsection{5.4.2 地下物流系统的建设时序及演进过程}

\section{六、总结}

\section{七、模型评价及推广}
\subsection{7.1 模型的优缺点评价}
\subsection{7.2 模型的推广}

\begin{table}[h]
\centering
\begin{tabular}{l l}
学校 & 河海大学 \\
\hline
参赛队号 & 10294003 \\
\hline
队员姓名 & 1. 王旭 \\
 & 2. 尹紫依 \\
 & 3. 丁毓蓝 \\
\end{tabular}
\end{table}

[REFERENCES:1]

\end{document}