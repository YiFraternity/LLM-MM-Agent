\begin{center}
\textbf{第十一届华为杯全国研究生数学建模竞赛}
\end{center}



\begin{center}
\textbf{题目} \quad 机动目标的跟踪与反跟踪
\end{center}

\begin{abstract}  

本文针对机动目标的跟踪与反跟踪问题进行了建模、求解与分析,主要完成了以下几部分内容:

针对问题一:首先,对三部雷达关于目标运动状态的量测数据进行预处理,包括将目标运动状态量测值由空间极坐标系转换到地球直角坐标系中,通过线性插值方法对测量数据进行时间对准,以及对同一时刻雷达 2、雷达 3 获得的目标状态量测值进行加权融合。其中,对于融合中的权重选择问题,本文将雷达 2、雷达 3 的测距误差、方位角误差及俯仰角误差转换到直角坐标系中各坐标轴方向上的误差,并按照各时刻雷达在各坐标轴上的标准差比例计算权重,经加权得到融合后的量测数据。然后,根据数据预处理后目标航迹的特性,建立了协同转换模型和变加速 Singer 模型的目标三维机动模型,并采用基于扩展卡尔曼滤波的交互式多模型算法 (IMM-EKF) 对目标进行在线跟踪,进而通过滤波得到目标的估计航迹、速度和加速度。最后,根据滤波得到的加速度大小、加速度方位角及加速度俯仰角随时间的变化,对目标加速度的大小和方向进行统计,并由此对目标的机动运动状态进行分析。通过研究分析可以得到,目标在 36600.0s 到 36850.0s 期间做机动转弯运动,并从 36900.0s 直至跟踪结束做连续的机动转弯运动。

针对问题二:为了对 Data2.txt 中各目标的量测数据进行快速、可靠地航迹关联,本文提出了基于启发式动态规划的数据关联算法。该算法动态地处理雷达连续扫描期间接收的顺序观测值,采用启发式规则的方法对目标进行速度域判决与角度(包含方位角、俯仰角)域判决,尽可能地减少错误的航迹关联,从而稳定、可靠地获得各目标相应的航迹。通过分析可以得到,所提算法在没有目标先验信息的情况下,计算量小,能够达到快速航迹关联的目的,具有序贯实时的特点。

针对问题三:首先,针对 Data3.txt 中关于目标运动状态的量测数据,将其由空间极坐标系转换到雷达质心直角坐标系中。然后,根据坐标系转换后所得目标点迹的特性,建立了协同转换模型和变加速 Singer 模型的目标三维机动模型,并采用 IMM-EKF 对目标进行在线跟踪,进而通过滤波得到目标的估计航迹、速度和加速度。最后,根据滤波得到的加速度大小、加速度方位角及加速度俯仰角随时间的变化,研究目标的机动变化规律。通过分析可以得出,从 14520.0s 开始至 14900.0s 期间,目标做爬升运动;从 14900.0s 开始直至跟踪结束,目标做俯冲运动。同时,可以看出本文所采用的目标跟踪模型,综合考虑了目标的转弯、变加速等机动运动方式,对于机动目标的跟踪具有很好地鲁棒性。

针对问题四:为了对目标进行实时预测并且考虑到算法的复杂度,本文采用 IMM-EKF 算法对目标将来时刻的状态值进行无量测更新的纯预测。通过预测目标在地球直角坐标系下的状态值,判断其径向半径是否小于等于地球平均半径,从而估计目标的着落点坐标。通过仿真分析,Data3.txt 中数据对应的目标着落点坐标为:东经 116.4761 度,北纬 39.8145 度。在此基础上,本文将此算法与基于多项式拟合的目标轨迹预测方法进行了比较。由于多项式拟合存在阶数选择问题,且阶数与拟合精度并没有明确的关系,从而在实际的轨迹预测中很难选择合适的阶数。此外,基于多项式拟合的目标轨迹预测方法首先需要对 $M$ 个量测数据进行拟合,算法复杂度为 $O(M^3)$。然而,本文采用 IMM-EKF 算法直接对当前时刻的目标状态和各模型概率的滤波结果进行预测,减少了数据拟合给算法带来的复杂度,更容易达到算法的实时性。通过计算分析得出,本文所采用的 IMM-EKF 目标轨迹预测算法和基于多项式拟合的目标轨迹预测算法在预测阶段的算法复杂度均为 $O(N)$,其中 $N$ 为所需预测的点数。

针对问题五:为摆脱雷达的锁定跟踪,在雷达前后两次扫描时间间隔取最小值 0.5s 的情况下,本文建立了基于波束覆盖范围限制的最小化逃逸距离数学优化模型。该模型将波束分别投影到直角坐标系的 XOY 面、XOZ 面以及 YOZ 面上,考虑到目标要在下一时刻逃逸并摆脱雷达跟踪,目标必须在下一时刻处于雷达波束覆盖范围以外,由此利用线性规划理论,建立了目标逃逸的机动优化模型。由于目标要选择有利的逃逸策略,所以本文将目标当前位置和下一时刻逃逸位置的距离最小化作为优化的目标函数。本文对 Data2.txt 中数据的一个目标量测值进行仿真分析,给出了该目标运动轨迹在本文所采用的 IMM-EKF 算法下的逃逸轨迹。针对目标可能采取的逃逸等反跟踪策略,又提出了雷达对目标的稳定跟踪方案,包括增大波束覆盖面积、减小雷达扫描时间间隔以及边扫描边跟踪的策略。对于 Data2.txt 中目标的逃逸轨迹,本文运用 IMM-EKF 算法进行在线跟踪。

关键词:扩展卡尔曼滤波,交互式多模型,启发式动态规划,线性规划,逃逸机动优化模型

\end{abstract}

\tableofcontents

\section{问题重述}

目标跟踪是指根据雷达等传感器所获得的对目标的测量信息,连续地对目标的运动状态进行估计,进而获取目标的运动态势及意图。目标机动则是指目标的速度大小和方向在短时间内发生变化,通常采用加速度作为衡量指标。目标跟踪与目标机动是“矛”与“盾”的关系。因此,引入了目标机动时雷达如何准确跟踪的问题。

机动目标跟踪的难点在于以下几个方面:(1) 描述目标运动的模型,即目标的状态方程难于准确建立。通常情况下跟踪的目标都是非合作目标,目标的速度大小和方向如何变化难于准确描述;(2) 传感器自身测量精度有限加之外界干扰,传感器获得的测量信息,如距离、角度等包含一定的随机误差,用于描述传感器获得测量信息能力的测量方程难于完全准确反映真实目标的运动特征;(3) 当存在多个机动目标时,除了要解决(1)、(2)两个问题外,还需要解决测量信息属于哪个目标的问题,即数据关联。由于以上多个挑战因素以及目标机动在战术上主动的优势,机动目标跟踪已成为近年来跟踪理论研究的热点和难点\cite{ref1}。

目标跟踪处理流程通常可分为航迹起始、点迹航迹关联(数据关联)、航迹滤波等步骤。另外,不同类型目标的机动能力不同。因此,在对机动目标跟踪时,必须根据不同的目标类型选择相应的跟踪模型\cite{ref2}。

根据题目提供的 3 组机动目标测量数据,本文拟解决以下问题:

问题一 根据附件中的 Data1.txt 数据,分析目标机动发生的时间范围,并统计目标加速度的大小和方向。建立对该目标的跟踪模型,并利用多个雷达的测量数据估计出目标的航迹。鼓励在线跟踪。

问题二 附件中的 Data2.txt 数据对应两个目标的实际检飞考核的飞行包线(检飞:军队根据国家军标规则设定特定的飞行路线用于考核雷达的各项性能指标,因此包线是有实战意义的)。请完成各目标的数据关联,形成相应的航迹,并阐明你们所采用或制定的准则(鼓励创新)。如果用序贯实时的方法实现更具有意义。若出现雷达一段时间只有一个回波点迹的状况,怎样使得航迹不丢失?请给出处理结果。

问题三 根据附件中 Data3.txt 的数据,分析空间目标的机动变化规律(目标加速度随时间变化)。若采用第 1 问的跟踪模型进行处理,结果会有哪些变化?

问题四 请对第 3 问的目标轨迹进行实时预测,估计该目标的着落点的坐标,给出详细结果,并分析算法复杂度。

问题五 Data2.txt 数据中的两个目标已被雷达锁定跟踪。在目标能够及时了解是否被跟踪,并已知雷达的测量精度为雷达波束宽度为 $3^\circ$,即在以雷达为锥顶,雷达与目标连线为轴,半顶角为 $1.5^\circ$ 的圆锥内的目标均能被探测到;雷达前后两次扫描时间间隔最小为 $0.5\,\text{s}$。为应对你们的跟踪模型,目标应该采用怎样的有利于逃逸的策略与方案?反之为了保持对目标的跟踪,跟踪策略又应该如何相应地变换?

\section{问题假设}

1) 估计目标着落点坐标时假设地球为正球体;

2) 假设 singer 模型中加速度的时间相关函数为零均值的指数衰减形式;

3) 假设 singer 模型中目标机动加速度服从近似均匀分布;

\section{三 符号说明}

\begin{tabular}{ll}
$r$ & 目标径向距离 \\
$\theta$ & 目标方位角 \\
$\varphi$ & 目标俯仰角 \\
$\sigma_{r}$ & 雷达极坐标下测距误差 \\
$\sigma_{\theta}$ & 雷达极坐标下方位角误差 \\
$\sigma_{\varphi}$ & 雷达极坐标下俯仰角误差 \\
$\sigma_{x}$ & 雷达在地球直角坐标下 $x$ 轴上的标准差 \\
$\sigma_{y}$ & 雷达在地球直角坐标下 $y$ 轴上的标准差 \\
$\sigma_{z}$ & 雷达在地球直角坐标下 $z$ 轴上的标准差 \\
$X_{k}$ & 目标的运动状态 \\
$F_{k}$ & $k$ 时刻状态转移矩阵 \\
$W_{k}$ & $k$ 时刻目标状态预测误差 \\
$P_{k}$ & 目标状态变量的估计误差的均方值 \\
$S_{k}^{j}$ & 伴随协方差矩阵 \\
$\beta$ & 标准差系数 \\
\end{tabular}

\section{四 问题一的建模与求解}

\subsection{4.1 问题分析}

问题一要求我们根据附件中的 Data1.txt 数据,利用多部雷达的测量数据对目标航迹进行估计。建立目标的跟踪模型,研究分析目标机动运动发生的时间范围,并统计目标的加速度大小及方向。我们首先对目标的量测数据进行预处理,然后,根据得到的目标点迹特性,建立目标的机动运动模型,并采用基于扩展卡尔曼滤波的交互式多模型算法 (IMM-EKF) $^{[3][4]}$ 对目标在线跟踪,最后,通过滤波得到的目标信息对其运动状态进行分析。

\subsection{问题建模}

图 1 给出了问题一的建模及求解流程图。本文采用的方法是:(1) 将三部雷达关于目标运动的量测数据进行预处理 \cite{ref5}, 包括:坐标系的转换,通过线性插值方法对量测数据进行时间对准,以及对同一时刻多部雷达获得的目标量测点迹进行加权融合;(2) 得到目标量测点迹;(3) 根据目标量测点迹,建立目标三维机动模型;(4) 采用 IMM-EKF 进行目标跟踪;(5) 通过 (4) 中的目标跟踪算法,估计出目标轨迹;(6) 根据目标轨迹,对目标的运动状态进行估计,分析出目标机动的时间范围,并统计加速度的大小及方向。

\begin{figure}[h]
\centering
\includegraphics[width=\textwidth]{image1.png}
\caption{问题一建模及求解流程图}
\end{figure}

\subsection{问题求解}

{步骤一:量测数据预处理}

{(1) 坐标系转换}

对于题目中的多雷达系统,为了充分利用各雷达的探测信息,必须将各雷达的探测数据转换到统一的坐标系中 \cite{ref2}。此处采用的方法是:(a) 首先,将三部雷达进行测量的空间极坐标系转换到各自的笛卡尔直角坐标系中;(b) 之后,将各雷达的笛卡尔直角坐标系转换到雷达质心坐标系下;(c) 最后,利用质心坐标系与地球直角坐标系的关系完成两者之间的坐标转换。坐标转换步骤如图 2 所示。

\begin{figure}[h]
\centering
\includegraphics[width=\textwidth]{image2.png}
\caption{坐标转换步骤}
\end{figure}

{(a) 空间极坐标系与笛卡尔直角坐标系的转换}

由 Data1.txt 中数据可以看出,三部雷达关于目标的测量值是在空间极坐标系中获得的。在问题 (1) 中,分别取各雷达质心作为原点 \(O\),雷达质心与当地纬度切线方向指向东为 \(x\) 轴,雷达质心与当地经度切线方向指北为 \(y\) 轴,地心与雷达质心连线指向天向的为 \(z\) 轴,目标方位角指北向顺时针夹角,目标俯仰角指天向顺时针夹角。

角指雷达质心与目标连线和地平面的夹角。

\begin{figure}[h]
    \centering
    \includegraphics[width=0.7\textwidth]{space_polar_coordinates.png}
    \caption{空间极坐标系}
    \label{fig:space_polar_coordinates}
\end{figure}

如图 \ref{fig:space_polar_coordinates} 所示,设目标 $P$ 为空间中已知一点,由 $P$ 向 $XOY$ 面作垂线,垂足为 $L$,用 $r$ 表示径向距离,$\theta$、$\varphi$ 分别表示目标的方位角、俯仰角。由图中的几何关系,可以得到目标 $P$ 空间极坐标的笛卡尔直角坐标表示为:
\begin{align*}
\text{XOY 面: } & y > \frac{x_1' \sin(-\theta) + y_1' \cos(-\theta)}{x_1' \cos(-\theta) - y_1' \sin(-\theta)} x \text{ 或 } y < \frac{x_1' \sin(\theta) + y_1' \cos(\theta)}{x_1' \cos(\theta) - y_1' \sin(\theta)} x \\
\text{XOZ 面: } & z > \frac{x_1' \sin(-\theta) + z_1' \cos(-\theta)}{x_1' \cos(-\theta) - z_1' \sin(-\theta)} x \text{ 或 } z < \frac{x_1' \sin(\theta) + z_1' \cos(\theta)}{x_1' \cos(\theta) - z_1' \sin(\theta)} x \\
\text{YOZ 面: } & z > \frac{y_1' \sin(\theta) + z_1' \cos(\theta)}{y_1' \cos(\theta) - z_1' \sin(\theta)} y \text{ 或 } z < \frac{y_1' \sin(-\theta) + z_1' \cos(-\theta)}{y_1' \cos(-\theta) - z_1' \sin(-\theta)} y
\end{align*}

(b) 笛卡尔直角坐标系与雷达质心坐标系的转换:

由于本题目中雷达位于地面,因此,可将笛卡尔直角坐标分别转换为各部雷达对应的质心坐标系,其原点设在雷达质心上。如图 \ref{fig:radar_centroid_coordinates} 所示。

\begin{figure}[h]
    \centering
    \includegraphics[width=0.7\textwidth]{radar_centroid_coordinates.png}
    \caption{雷达质心坐标系}
    \label{fig:radar_centroid_coordinates}
\end{figure}

其中,$x_{i}$ 为地理指南方向,$y_{i}$ 为地理自转切线方向,$z_{i}$ 为地心指向雷达质心的方向。则目标 $P$ 笛卡尔直角坐标的雷达质心坐标可以表示为:
\begin{cases}
x_l=-y_g \\
y_l=x_g \\
z_l=z_g & 
\end{cases}


(c) 雷达质心坐标系与地球坐标系的转换:

地球坐标系的原点选在地球球心。$Z$ 轴为地球的自旋轴,从地球球心指向北极;$X$ 轴被定义为在赤道平面上,从地球球心指向子午线的轴线;$Y$ 轴是 $X$ 轴和 $Z$ 轴正交的结果,如图 5 所示。通常,该坐标系中的目标坐标用经度、纬度和高度表示。

\begin{figure}[h]
    \centering
    \includegraphics[width=0.8\textwidth]{image.png}
    \caption{雷达质心坐标系与地球直角坐标系的坐标轴线之间的旋转关系}
\end{figure}

设雷达的经度、纬度和高度分别为 $L$、$B$、$H$,则其在地球直角坐标系中的坐标 $\mathbf{X}_o = [x_o, y_o, z_o]$ 为:

\begin{equation}
\begin{cases}
x_o = (N_R + H) \cos B \cos L \\
y_o = (N_R + H) \cos B \sin L \\
z_o = \left[ N_R \left( 1 - e_1^2 \right) + H \right] \sin B
\end{cases}
\tag{3}
\end{equation}

式中,$e_1^2 = \frac{\left( a^2 - b^2 \right)}{a^2}$ 为第一偏心率,$N_R = \frac{a}{\sqrt{1 - e_1^2 \sin^2 B}}$,$a$ 为半长轴,$b$ 为短半轴,$a = 6378137 \, \text{m}$,$b = 6356752 \, \text{m}$。

假设目标在雷达质心坐标系与地球直角坐标系下的坐标参数分别为 $\mathbf{X}_l = [x_l, y_l, z_l]$,$\mathbf{X} = [x, y, z]$。根据图 5,可得地球直角坐标系与雷达质心坐标系的转换关系为:

\begin{equation}
\mathbf{X} = \mathbf{T}_x \mathbf{T}_y \mathbf{T}_z \mathbf{X}_l + \mathbf{X}_o
\tag{4}
\end{equation}

式中,

\begin{equation}
\mathbf{T}_x =
\begin{bmatrix}
1 & 0 & 0 \\
0 & \cos A_x & \sin A_x \\
0 & -\sin A_x & \cos A_x
\end{bmatrix}
\tag{5}
\end{equation}

\begin{equation}
\mathbf{T}_{y}=\left[\begin{array}{ccc}
\cos A_{y} & 0 & -\sin A_{y} \\
0 & 1 & 0 \\
\sin A_{y} & 0 & \cos A_{y}
\end{array}\right]
\tag{6}
\end{equation}

\begin{equation}
\mathbf{T}_{z}=\left[\begin{array}{ccc}
\cos A_{z} & \sin A_{z} & 0 \\
-\sin A_{z} & \cos A_{z} & 0 \\
0 & 0 & 1
\end{array}\right]
\tag{7}
\end{equation}

其中,$A_{x}=0^{\circ}$,$A_{y}=B+270^{\circ}$,$A_{z}=360^{\circ}-L$。

图 6 给出了经坐标转换后在地球直角坐标系下三部雷达对目标的测量点迹。

\begin{figure}[h]
\centering
\includegraphics[width=\textwidth]{image.png}
\caption{地球直角坐标系下各雷达对目标的测量点迹}
\end{figure}

{通过线性插值进行时间对准}

根据 Data1.txt 中所给数据,绘制三部雷达的工作序列。从图 7 中可以看到,三台雷达对同一目标的观测时刻不完全一样。其中雷达 1 最先单独工作,雷达 1 结束工作后一段时间雷达 2 才开始观测,随后雷达 3 开始观测且与雷达 2 共同工作一段时间,在雷达 2 结束观测后雷达 3 又继续观测一段时间。另外,三部雷达在工作的过程中均出现不同程度的数据丢失现象。

\begin{figure}[h]
    \centering
    \includegraphics[width=\textwidth]{image1.png}
    \caption{三部雷达工作时序}
    \label{fig:radar_sequence}
\end{figure}

为了后续对雷达观测数据进行滤波,从而得到目标航迹信息,需要对丢失的数据进行插值。在此,采用线性插值法将丢失时刻的目标量测值补齐,保证相邻两个量测数据间的时间间隔均为 1.0 s。

设定 $t_1$、$t_2$ 分别为两个连续的目标观测时刻,$t_1$ 时刻目标在极坐标系下的观测值为 $(t_1, r_1, \theta_1, \varphi_1)$,其对应的地球直角坐标系下的观测值为 $(t_1, x_1, y_1, z_1)$,$t_2$ 时刻目标在极坐标系下的观测值为 $(t_2, r_2, \theta_2, \varphi_2)$,其对应的地球直角坐标系下的观测值为 $(t_2, x_2, y_2, z_2)$。$t_1$、$t_2$ 两个时刻之间存在数据缺失,为了得到其间某个时刻 $t$ 的直角坐标值,利用前后两个坐标值进行线性插值,以 $y$ 轴的线性插值为例:

\begin{figure}[h]
    \centering
    \includegraphics[width=0.8\textwidth]{image2.png}
    \caption{线性插值示意图}
    \label{fig:linear_interpolation}
\end{figure}

如图 \ref{fig:linear_interpolation} 所示,已知坐标 $(t_1, y_1)$ 与 $(t_2, y_2)$,要得到 $(t_1, t_2)$ 区间内某一时刻 $t$ 的值。利用式 (8) 进行计算:

\begin{equation}
    \frac{y - y_1}{t - t_1} = \frac{y_2 - y_1}{t_2 - t_1}
    \tag{8}
\end{equation}

由于 $t$ 值已知,可得:

\begin{equation}
y = y_1 + (t - t_1) \frac{y_2 - y_1}{t_2 - t_1}
\tag{9}
\end{equation}

(3) 同一时刻多量测点迹的加权融合

由图7可知,雷达2、雷达3在第37119.40秒至37219.40秒同时对目标进行观测。为了得到更加可靠的目标量测点迹,将这段时间内雷达2和雷达3关于目标的量测数据进行加权融合。

采用的加权融合算法步骤如下:

(a) 利用极坐标下雷达2、雷达3的测量误差计算地球直角坐标系下的标准差:

已知雷达在空间极坐标系下的误差,如表1所示:

\begin{table}[h]
\centering
\caption{雷达测量误差}
\begin{tabular}{c c c c}
\hline
雷达标号 & 测距误差 $r$ (m) & 方位角误差 $\theta$ ($^\circ$) & 俯仰角误差 $\varphi$ ($^\circ$) \\
\hline
2 & 40 & 0.3 & 0.3 \\
3 & 60 & 0.5 & 0.5 \\
\hline
\end{tabular}
\end{table}

由于雷达极坐标下的测距误差与方位角、俯仰角误差量纲不统一,首先,将极坐标下雷达2、雷达3的测距误差、方位角误差和俯仰角误差转化为地球直角坐标系 $x$ 轴、$y$ 轴、$z$ 轴上的测距标准差。由此,可得雷达在地球直角坐标系 $x$ 轴、$y$ 轴、$z$ 轴上的测距标准差分别为:

\begin{equation}
\sigma_x = \sqrt{\cos^2 \varphi \sin^2 \theta \sigma_r^2 + r^2 \sin^2 \varphi \sin^2 \theta \sigma_\varphi^2 + r^2 \cos^2 \varphi \cos^2 \theta \sigma_\theta^2}
\tag{10}
\end{equation}

\begin{equation}
\sigma_y = \sqrt{\cos^2 \theta \cos^2 \theta \sigma_r^2 + r^2 \cos^2 \theta \sin^2 \varphi \sigma_\varphi^2 + r^2 \cos^2 \varphi \sin^2 \theta \sigma_\theta^2}
\tag{11}
\end{equation}

\begin{equation}
\sigma_z = \sqrt{\sin^2 \varphi \sigma_r^2 + r^2 \cos^2 \varphi \sigma_\varphi^2}
\tag{12}
\end{equation}

式中,$\sigma_r$ 为极坐标下的测距误差,$\sigma_\theta$ 为极坐标下的方位角误差,$\sigma_\varphi$ 为极坐标下的俯仰角误差,$\sigma_x$ 为地球直角坐标下 $x$ 轴上的标准差,$\sigma_y$ 为 $y$ 轴上的标准差,$\sigma_z$ 为 $z$ 轴上的标准差。

(b) 计算并对比两部雷达同时工作时目标量测数据的标准差:

图9给出了雷达2、雷达3同时工作时关于目标量测数据的标准差对比。从图中可以看出,在地球直角坐标系下,雷达3的测量标准差比雷达2低。这主要是因为虽然雷达2的测量误差比雷达3小,但是目标相对于雷达2的距离大于目标相对于雷达3的距离,由公式(9)-(12)可以看出,这是雷达2的测距标准差大于雷达3的主要原因。

(c) 对雷达2、雷达3的量测数据进行加权融合:

由于在地球直角坐标系下,雷达3的测量标准差比雷达2低,因此,采用加权算法对两部雷达的量测数据赋予不同的权值,进行融合。利用雷达2、雷

\begin{figure}[h]
    \centering
    \includegraphics[width=\textwidth]{image1.png}
    \caption{雷达2、雷达3的测距标准差}
    \label{fig:radar_std_dev}
\end{figure}

雷达3的测距标准差计算第37119.40秒至37219.40秒间每一时刻两部雷达各自的权重。假设某一时刻,直角坐标某方向上雷达2、雷达3的测距标准差分别为 $\sigma_2$ 和 $\sigma_3$,目标量测数据分别为 $X_2$、$X_3$,则有加权融合公式:

\begin{equation}
\begin{cases}
w_2 = \frac{1/\sigma_2}{1/\sigma_2 + 1/\sigma_3} \\
w_3 = \frac{1/\sigma_3}{1/\sigma_2 + 1/\sigma_3}
\end{cases}
\tag{13}
\end{equation}

其中,$w_2$ 为当前时刻雷达2权值,$w_3$ 为雷达3权值。因此,融合后的目标量测值为:

\begin{equation}
X = w_2 \cdot X_2 + w_3 \cdot X_3
\tag{14}
\end{equation}

{步骤二 获得目标量测点迹}

图10给出了经线性插值后雷达1对目标的测量点迹以及雷达2、雷达3经线性插值和加权融合后对目标的测量点迹。

{步骤三 建立目标三维机动模型}

{(1) 协同转弯模型}

雷达目标的运动状态转移方程可以表示为:

\begin{equation}
X_{k+1} = F_k X_k + G_k W_k
\tag{15}
\end{equation}

其中,$X_k$ 表示目标的运动状态 [6],通常它是由目标当前时刻的各坐标轴方向上

\begin{figure}[h]
    \centering
    \includegraphics[width=\textwidth]{image.png}
    \caption{经插值、融合后各雷达对目标的测量点迹}
    \label{fig:10}
\end{figure}

图 \ref{fig:10} 经插值、融合后各雷达对目标的测量点迹

的位置、速度以及加速度构成,此处仅以 $x$ 坐标轴为例,可以表示成 $X_{k}=[x_{k},v_{xk},a_{xk}]^{T}$。$F_{k}$ 表示为 $k$ 时刻状态转移矩阵,它决定了不同的运动模型。$W_{k}$ 表示为 $k$ 时刻目标状态预测误差,它是均值为 0,方差为 $\sigma^{2}$ 的高斯白噪声。

题目中的雷达目标具有很大的机动性,不仅做直线运动,还做转弯、变加速等运动形式。对于三维空间中做匀速圆周运动的目标,其 $k$ 时刻的运动状态可表示为:$X_{k}=[x_{k},v_{xk},a_{xk},y_{k},v_{yk},a_{yk},z_{k},v_{zk},a_{zk}]^{T}$。于是可以获得转弯模型的状态方程为:

\begin{equation}
\dot{X}_{k}=diag\left\{A(\omega),A(\omega),A(\omega)\right\}X_{k}+diag\left\{B,B,B\right\}w_{k}
\tag{16}
\end{equation}

其中,$\omega$ 代表目标运动的角速度,$A(\omega)=\begin{pmatrix}0 & 1 & 0 \\ 0 & 0 & 1 \\ 0 & -\omega^{2} & 0\end{pmatrix}$,$B=\begin{pmatrix}0 \\ 0 \\ 1\end{pmatrix}$。

根据式(16),其状态转移矩阵为:

\begin{equation}
F_{k}=\text{diag}\left\{F(\omega),F(\omega),F(\omega)\right\}
\tag{17}
\end{equation}

式中,

\begin{equation}
F(\omega)=\begin{bmatrix}1 & \frac{\sin\omega T_{s}}{\omega} & 0 & -\frac{1-\cos\omega T_{s}}{\omega} \\ 0 & \cos\omega T_{s} & 0 & -\sin\omega T_{s} \\ 0 & \frac{1-\cos\omega T_{s}}{\omega} & 1 & \frac{\sin\omega T_{s}}{\omega} \\ 0 & \sin\omega T_{s} & 0 & \cos\omega T_{s}\end{bmatrix}
\tag{18}
\end{equation}

其中,$T_{s}$ 表示雷达目标跟踪的采样间隔。

(2) Singer 模型

另外,在跟踪过程中,目标表现出变加速度的运动特性。此时的目标运动状态方程可表示为:
\begin{equation}
\dot{X}(t)=
\begin{cases}
A(t)X(t)+B(t)a(t); & X(t)=[x,\dot{x}]^{T} \\
A(t)X(t)+B(t)\ddot{a}(t); & X(t)=[x,\dot{x},\ddot{x}]^{T}
\end{cases}
\tag{19}
\end{equation}

此处的 $a(t)$ 为机动目标运动的加速度。在此,对机动加速度做两点假设:

假设 1:根据平稳随机过程时间相关函数的对称性和衰减性,假设机动加速度的时间相关函数为零均值的指数衰减形式,其相关函数如下:
\begin{equation}
R_{a}(\tau)=E\{a(t)a(t+\tau)\}=\sigma_{a}^{2}e^{-\alpha|\tau|} \ (\alpha\geq0)
\tag{20}
\end{equation}
其中,$\sigma_{a}^{2}$ 为目标机动加速方差,$\alpha$ 为机动频率,是决定目标机动特性的特定参数。

假设 2:假设雷达目标的机动加速度 $a(t)$ 服从近似均匀分布,其概率密度函数表示如下:
\begin{equation}
f(a)=\frac{1-(P_{0}+2P_{\text{max}})}{2A_{\text{max}}} \ (-A_{\text{max}}<a<A_{\text{max}})
\tag{21}
\end{equation}
其中,$A_{\text{max}}$ 为雷达目标最大机动加速度,$P_{0}$ 为非机动概率,$P_{\text{max}}$ 最大机动加速度时发生的概率。

利用 Wiener-Kolmogorov 方法对机动加速度 $a(t)$ 的相关函数 $R_{a}(\tau)$ 进行白化,可得其一阶时间相关模型:
\begin{equation}
\dot{a}(t)=-\alpha a(t)+\omega(t)
\tag{22}
\end{equation}
其中,$\omega(t)$ 为零均值,$2\alpha\sigma_{a}^{2}$ 方差的高斯白噪声。Singer 模型的运动状态方程为:
\begin{equation}
\begin{bmatrix}
\dot{x} \\
\ddot{x} \\
\dddot{x}
\end{bmatrix}=
\begin{bmatrix}
0 & 1 & 0 \\
0 & 0 & 1 \\
0 & 0 & -\alpha
\end{bmatrix}
\begin{bmatrix}
x \\
\dot{x} \\
\ddot{x}
\end{bmatrix}+
\begin{bmatrix}
0 \\
0 \\
1
\end{bmatrix}
\omega(t)
\tag{23}
\end{equation}
由此获得其状态转移矩阵为:
\begin{equation}
F_{k}=
\begin{bmatrix}
1 & T_{s} & (\alpha T_{s}-1+e^{-\alpha T_{s}})/\alpha^{2} \\
0 & 1 & (1-e^{-\alpha T_{s}})/\alpha \\
0 & 0 & e^{-\alpha T_{s}}
\end{bmatrix}
\tag{24}
\end{equation}

步骤四 采用基于扩展卡尔曼滤波的交互式多模型算法对目标在线跟踪

在此,采用基于扩展卡尔曼滤波的交互式多模型算法 (IMM-EKF) 对目标进行在线跟踪。首先,先建立扩展卡尔曼滤波算法模型,再基于此建立交互式多模型。

(1) 扩展卡尔曼滤波算法

扩展卡尔曼滤波的关键在于根据运动目标先前时刻运动状态的预测值和当前时刻的观测值预测雷达目标当前时刻的运动状态,从而形成目标运动轨迹。

假设雷达目标在 $k$ 时刻的状态信息为向量 $x_{k}$,$k$ 时刻雷达对目标的量测信息用向量 $z_{k}$ 表示。从而可将雷达目标的状态方程和量测方程表示如下:

\begin{equation}
X_{k} = f(X_{k-1}) + V_{k-1}
\tag{25}
\end{equation}

\begin{equation}
Z_{k} = h(X_{k}) + N_{k}
\tag{26}
\end{equation}

其中,$f(\cdot)$、$h(\cdot)$ 为非线性函数,分别表示目标的状态转移函数和量测函数,$V_{k-1}$ 为状态转移误差,$N_{k}$ 为量测误差。$V_{k-1}$ 和 $N_{k}$ 也均为高斯白噪声,其均值为 0,方差分别为 $Q_{k}$ 和 $R_{k}$。

扩展卡尔曼滤波算法的关键在于使得目标状态变量的估计误差的均方值 $P_{k}$ 最小。假设目标 $k-1$ 时刻的状态变量预测值为 $\hat{X}_{k-1}$,则修正后的状态变量预测值 $\hat{X}_{k}$ 为:

\begin{equation}
\hat{X}_{k} = f\left(\hat{X}_{k-1}\right) + H_{k} \tilde{Z}_{k}
\tag{27}
\end{equation}

其中,$H_{k}$ 为加权矩阵。此时的目标状态变量的估计误差的均方值 $P_{k}$ 为:

\begin{equation}
P_{k} = E[(X_{k} - \hat{X}_{k})(X_{k} - \hat{X}_{k})^{T}]
\tag{28}
\end{equation}

扩展卡尔曼滤波算法的递推过程首先要对非线性函数进行处理,即计算这两个方程的 Jacobi 矩阵:

\begin{equation}
F_{k} = \left. \frac{\partial f(X_{k-1})}{\partial X_{k-1}} \right|_{X_{k-1} = \hat{X}_{k-1|k-1}}
\tag{29}
\end{equation}

\begin{equation}
H_{k} = \left. \frac{\partial h(X_{k})}{\partial X_{k}} \right|_{X_{k} = \hat{X}_{k|k-1}}
\tag{30}
\end{equation}

扩展卡尔曼滤波算法的具体递推流程如图 11 所示。

(2) 交互式多模型算法

由于目标的高机动性,单一模型的目标跟踪算法已经很难实现对机动目标的高精度跟踪 [7][8]。因此,此处需在扩展卡尔曼滤波的基础上,采用交互式多模型算法。该算法使用了多个扩展卡尔曼滤波器,在设定的模型转换概率下,通

\begin{figure}[h]
    \centering
    \includegraphics[width=\textwidth]{image.png}
    \caption{扩展卡尔曼滤波算法递归流程图}
    \label{fig:ekf_flowchart}
\end{figure}

过模型的交互概率和模型的更新概率交互计算获得各滤波器的输入和输出,对各滤波器的输出结果加权获得该时刻目标运动状态的估计。

假设目标运动的状态方程和观测方程可分别表示如下:

\begin{equation}
X_{k+1}^{j} = f^{j}(X_{k}^{j}) + V_{k}^{j}
\tag{31}
\end{equation}

\begin{equation}
Z_{k}^{j} = h^{j}(X_{k}^{j}) + W_{k}^{j}
\tag{32}
\end{equation}

其中,$X_{k}^{j}$ 表示第 $j$ 个模型在 $k$ 时刻目标的状态向量;$Z_{k}^{j}$ 为观测向量;$f^{j}(\cdot)$ 和 $h^{j}(\cdot)$ 分别表示第 $j$ 个模型的状态转移函数和量测函数;$V_{k}^{j}$ 和 $W_{k}^{j}$ 分别表示第 $j$ 个模型的系统噪声和量测噪声,它们的均值为零,对应的协方差矩阵分别是 $Q_{k}^{j}$ 和 $R_{k}^{j}$。

同时也假设模型转换概率矩阵为 $\pi = [\pi_{ij}]_{r \times r}$,其中,$r$ 表示目标运动所有可能的模型的个数,此处假设该系统运动目标的运动模型集为:$M = \{M^{1}, M^{2}, \ldots, M^{r}\}$;$\pi_{ij}$ 为第 $i$ 个模型转换到第 $j$ 个模型的转换概率,具体数学表示形式如下:

\begin{equation}
P(M_{k+1}^{j} \mid M_{k}^{i}) = \pi_{ij}
\tag{33}
\end{equation}

其中,$M_{k}^{i}$ 表示 $k$ 时刻的运动模型为 $M^{i}$,且 $\sum_{j=1}^{r} \pi_{ij} = 1$,$i = 1, 2, \ldots, r$。

而 $k$ 时刻每个模型的概率用 $\mu_{k}^{j}$ 表示,$\mu_{k}^{j} = P(M_{k}^{j} \mid Z_{k}^{j})$。

交互式多模型算法的递归过程如下:

\textbf{(a) 模型混合概率的计算:}

\begin{equation}
\mu_{i|j}(k-1) = \frac{1}{c_{j}(k-1)} \cdot \pi_{ij} \cdot \mu_{k-1}^{i}
\tag{34}
\end{equation}

其中,$c_{j}(k-1)$ 为归一化因子,$c_{j}(k-1) = \sum_{i=1}^{M} \pi_{ij} \cdot \mu_{k-1}^{i}$。

\textbf{(b) 各滤波器输入的混合状态估计和协方差矩阵估计:}

\begin{equation}
\bar{X}_{j}(k-1|k-1)=\sum_{i=1}^{r} \hat{X}_{i}(k-1|k-1) \cdot \mu_{i|j}(k-1), j=1,2, \cdots, r
\tag{35}
\end{equation}

\begin{equation}
\begin{aligned}
\bar{P}_{j}(k-1|k-1) &= \sum_{i=1}^{M} \{[\hat{X}_{i}(k-1|k-1)-\bar{X}_{j}(k-1|k-1)] \cdot \\
& [\hat{X}_{i}(k-1|k-1)-\bar{X}_{j}(k-1|k-1)]^{T} + \hat{P}_{i}(k-1|k-1)\} \cdot \mu_{i|j}(k-1), j=1,2, \cdots, r
\end{aligned}
\tag{36}
\end{equation}

(c) 各模型的滤波:

利用扩展卡尔曼滤波算法,获得各模型的状态估计值 $\hat{X}_{j}(k|k)$ 和 $P_{j}(k|k)$。

(d) 似然函数的计算:

\begin{equation}
\Lambda_{j}(k)=N\left(d_{j}(k); 0, S_{k}^{j}\right)=\frac{\exp \left[-\frac{d_{j}^{2}(k)}{2}\right]}{\sqrt{\left(2 \pi\right)^{r} \cdot\left|S_{k}^{j}\right|}}
\tag{37}
\end{equation}

其中,

\begin{equation}
d_{j}^{2}(k)=\left[Z_{k}^{j}-h^{j}\left(X_{k}\right)\right]^{T} \cdot S_{k}^{j} \cdot\left[Z_{k}^{j}-h^{j}\left(X_{k}\right)\right]
\tag{38}
\end{equation}

$S_{k}^{j}$ 为伴随协方差矩阵,$S_{k}^{j}=H_{k} P_{k|k-1} H_{k}^{T}+R_{k}$。

(e) 模型概率的更新计算:

\begin{equation}
\mu_{k}^{j}=\frac{1}{\sum_{j=1}^{r} \Lambda_{j}(k) \cdot c_{j}(k-1)} \cdot \Lambda_{j}(k) \cdot c_{j}(k-1), j=1,2, \cdots, r
\tag{39}
\end{equation}

(f) 交互输出计算:

\begin{equation}
\hat{X}(k|k)=\sum_{j=1}^{r} \mu_{k}^{j} \hat{X}_{j}(k|k)
\tag{40}
\end{equation}

\begin{equation}
P(k|k)=\sum_{j} \left\{P_{j}(k|k)+\left[\hat{X}_{j}(k|k)-\hat{X}(k|k)\right] \cdot\left[\hat{X}_{j}(k|k)-\hat{X}(k|k)\right]^{T}\right\} \mu_{k}^{j}
\tag{41}
\end{equation}

基于扩展卡尔曼滤波的交互式多模型算法框图如图 12 所示。

{步骤五 估计目标轨迹}

图 13 示出了三部雷达对目标的测量点迹以及采用基于扩展卡尔曼滤波的交互式多模型算法对目标的估计轨迹。从图 13 中可以看出,采用本文提出的建模与求解流程图以及文中的目标跟踪模型、在线跟踪算法,可以从大量的目标量测点中提取出目标的航迹信息,从而建立对目标航迹可靠、精确地估计。同时,本文采用的目标运动模型和跟踪算法可以精确地描述目标的运动特性,这对下一步分析目标的机动特性、加速度的大小与方向变化具有重要意义。另

\begin{figure}[h]
    \centering
    \includegraphics[width=\textwidth]{image.png}
    \caption{基于扩展卡尔曼滤波的交互式多模型算法框图}
    \label{fig:12}
\end{figure}

外,值得一提的是,本文采用的算法计算量小、复杂度低,对机动目标的航迹估计与跟踪具有良好的稳健性与实时性。

\begin{figure}[h]
    \centering
    \includegraphics[width=\textwidth]{image2.png}
    \caption{采用基于扩展卡尔曼滤波的交互式多模型算法对目标的估计航迹}
    \label{fig:13}
\end{figure}

{步骤六 分析目标运动特性}

(1) 分析目标的机动运动及发生的时间范围

由于建立的 Singer 目标跟踪模型受到最大加速度参数 $a_{m}$ 设置的影响,而且目标类型未知,因此,根据题目中所给定的不同类型目标的机动能力进行参数

设置。选取 $a_{m}=10g$ 和 $a_{m}=100g$ 分别代表通常情况下战斗机和导弹目标的最大加速度。

(a) $a_{m}=10g$ 情况:

图 14 给出了 $a_{m}=10g$ 情况下目标的加速度随时间的变化曲线。在 36600.0s 到 36850.0s 期间,目标加速度呈现先减小后增大的变化趋势,这说明,目标在做机动运动。从 36900.0s 雷达 2 与雷达 3 对目标稳定跟踪开始,直至跟踪结束,目标的加速度周期性地增大而后减小,这说明目标在做连续的机动转弯运动。

\begin{figure}[h]
    \centering
    \includegraphics[width=\textwidth]{image1.png}
    \caption{$a_{m}=10g$ 情况下加速度随时间的变化曲线}
    \label{fig:14}
\end{figure}

\begin{figure}[h]
    \centering
    \includegraphics[width=\textwidth]{image2.png}
    \caption{$a_{m}=10g$ 情况下速度随时间的变化曲线}
    \label{fig:15}
\end{figure}

图 15 给出了 $a_{m}=10g$ 情况下目标速度随时间的变化曲线。从图 14 和图 15 可以看出,在 36620.0s 到 36630.0s 期间,目标的加速度和速度均出现一个大的尖峰,这主要是因为在 36620.0s 雷达刚建立目标航迹,目标的加速度、速度由

初始值 \(0 \, \text{m/s}^2\) 和 \(0 \, \text{m/s}\) 骤然上升,而后又恢复正常值。从图 15 可以看出,目标速度的变化与图 14 中目标加速度的变化相吻合。这也验证了图 14 中目标机动运动及分析的正确性。

(b) \(a_{m}=100g\) 情况:

\begin{figure}[h]
    \centering
    \includegraphics[width=\textwidth]{image1.png}
    \caption{\(a_{m}=100g\) 情况下加速度随时间的变化曲线}
    \label{fig:16}
\end{figure}

类似地,图 16 给出了 \(a_{m}=100g\) 情况下目标的加速度随时间的变化曲线。在 36600.0s 到 36850.0s 期间,目标加速度呈现先减小后增大的变化趋势,这说明,目标在做机动运动。从 36900.0s 雷达 2 与雷达 3 对目标稳定跟踪开始,直至跟踪结束,目标的加速度周期性地增大而后减小,这说明目标在做连续的机动转弯运动。

\begin{figure}[h]
    \centering
    \includegraphics[width=\textwidth]{image2.png}
    \caption{\(a_{m}=100g\) 情况下速度随时间的变化曲线}
    \label{fig:17}
\end{figure}

图 17 给出了 \(a_{m}=10g\) 情况下目标速度随时间的变化曲线。如前所述,从图

16 和图 17 可以看出,在 36620.0s 到 36630.0s 期间,目标的加速度和速度均出现一个大的尖峰,这主要是因为在 36620.0s 雷达刚建立目标航迹,目标的加速度、速度由初始值 \(0 \, \text{m/s}^2\) 和 \(0 \, \text{m/s}\) 骤然上升,而后又恢复正常值。从图 17 可以看出,目标速度的变化与图 16 中目标加速度的变化相吻合。这也验证了图 16 中目标机动运动及分析的正确性。

(2) 统计目标跟踪过程中目标的加速度方向变化

(a) \(a_{m}=10g\) 情况:

图 18 给出了 \(a_{m}=10g\) 情况下目标的加速度方位角和俯仰角随时间的变化曲线。从图 18 可以看出,在 36600.0s 到 36850.0s 期间,目标加速度方位角先增大后减小,加速度俯仰角在呈现出先减小后增大的变化趋势。这说明,在此期间,目标做机动运动,即先俯冲而后爬升。从 36900.0s 雷达 2 与雷达 3 对目标稳定跟踪开始,直至跟踪结束,目标加速度的方位角与俯仰角周期性地增大而后减小,呈现近似正弦函数曲线,这说明目标在做连续的机动转弯运动。

\begin{figure}[h]
    \centering
    \includegraphics[width=\textwidth]{image.png}
    \caption{\(a_{m}=10g\) 情况下加速度方位角与俯仰角随时间的变化曲线}
    \label{fig:18}
\end{figure}

(b) \(a_{m}=100g\) 情况:

类似地,图 19 给出了 \(a_{m}=100g\) 情况下目标加速度方位角和俯仰角随时间的变化曲线。与图 18 类似,在 36600.0s 到 36850.0s 期间,目标加速度方位角先增大后减小,加速度俯仰角在呈现出先减小后增大的变化趋势。这说明,在此期间,目标做机动运动。从 36900.0s 雷达 2 与雷达 3 对目标稳定跟踪开始,直至跟踪结束,目标加速度的方位角与俯仰角周期性地增大而后减小,这说明目标在做连续的机动转弯运动。

综上所述,图 14 至图 19 中目标的加速度、速度、加速度方位角与俯仰角

\begin{figure}[h]
    \centering
    \includegraphics[width=\textwidth]{image.png}
    \caption{$a_{m}=100g$ 情况下加速度方位角与俯仰角随时间的变化曲线}
\end{figure}

图 19. $a_{m}=100g$ 情况下加速度方位角与俯仰角随时间的变化曲线的变化时间段及变化趋势基本一致,从而进一步验证了目标的机动运动及其分析的正确性。

\section{五 问题二的建模与求解}

\subsection{5.1 问题分析}

问题二要求我们根据 Data2.txt 中两个目标的量测数据,设计相应的航迹关联准则,尽可能地采用序贯实时地方法,对各目标进行数据关联\cite{ref9}。我们提出一种基于启发式动态规划的数据关联算法,可对雷达连续扫描期间接收的顺序观测值,采用启发式规则在目标的速度域及角度域上对其进行数据关联。

\subsection{5.2 问题建模}

针对问题二,提出了基于启发式动态规划的数据关联算法,从而序贯实时地获得各目标相应的航迹。该算法动态处理雷达连续扫描期间接收的顺序观测值,采用启发式规则的方法对目标进行速度域判决、角度(包含方位角、俯仰角)域判决,减少错误的航迹关联,从而得到各目标相应的航迹。所提方法在没有目标先验信息的情况下,计算量小,可以达到快速航迹关联的目的,具有序贯实时的特点。

问题二的建模及求解流程如图 20 所示。

\begin{figure}[h]
    \centering
    \includegraphics[width=\textwidth]{image2.png}
    \caption{问题二建模及求解流程图}
\end{figure}

\subsection{5.3 问题求解}

{步骤一 坐标系转换}

由 Data2.txt 中数据可以看出,雷达关于目标的测量值是在空间极坐标系中获得的。在此,采用前文所述的坐标系转换方法,将目标测量值由空间极坐标系转换到雷达质心直角坐标系中。

{步骤二 基于启发式动态规划的数据关联算法}

{(1)速度域判决准则}

设定 \( t_1 \)、\( t_2 \) 分别为两个连续的目标观测时刻,\( t_1 \) 时刻目标在极坐标系下的观测值为 \( (t_1, r_1, \theta_1, \varphi_1) \),其对应的地球直角坐标系下的观测值为 \( (t_1, x_1, y_1, z_1) \),\( t_2 \) 时刻目标在极坐标系下的观测值为 \( (t_2, r_2, \theta_2, \varphi_2) \),其对应的地球直角坐标系下的观测值为 \( (t_2, x_2, y_2, z_2) \),如图 20 所示。于是,\( t_2 \) 时刻目标在 \( x \) 轴、\( y \) 轴、\( z \) 轴上的平均速度可以表示为:

\begin{equation}
\left\{
\begin{aligned}
v_{x2} &= \frac{x_2 - x_1}{t_2 - t_1} \\
v_{y2} &= \frac{y_2 - y_1}{t_2 - t_1} \\
v_{z2} &= \frac{z_2 - z_1}{t_2 - t_1}
\end{aligned}
\right.
\tag{42}
\end{equation}

\( x \) 轴、\( y \) 轴、\( z \) 轴上目标平均速度的标准差分别为:

\begin{equation}
\left\{
\begin{aligned}
\Delta \sigma_x &= \frac{\sqrt{\cos^2 \varphi_2 \sin^2 \theta_2 \sigma_{r_2}^2 + r_2^2 \sin^2 \varphi_2 \sin^2 \theta_2 \sigma_{\varphi_2}^2 + r_2^2 \cos^2 \varphi_2 \cos^2 \theta_2 \sigma_{\theta_2}^2}}{t_2 - t_1} \\
&\quad - \frac{\sqrt{\cos^2 \varphi_1 \sin^2 \theta_1 \sigma_{r_1}^2 + r_1^2 \sin^2 \varphi_1 \sin^2 \theta_1 \sigma_{\varphi_1}^2 + r_1^2 \cos^2 \varphi_1 \cos^2 \theta_1 \sigma_{\theta_1}^2}}{t_2 - t_1} \\
\Delta \sigma_y &= \frac{\sqrt{\cos^2 \theta_2 \cos^2 \theta_2 \sigma_{r_2}^2 + r_2^2 \cos^2 \theta_2 \sin^2 \varphi_2 \sigma_{\varphi_2}^2 + r_2^2 \cos^2 \varphi_2 \sin^2 \theta_2 \sigma_{\theta_2}^2}}{t_2 - t_1} \\
&\quad - \frac{\sqrt{\cos^2 \theta_1 \cos^2 \theta_1 \sigma_{r_1}^2 + r_1^2 \cos^2 \theta_1 \sin^2 \varphi_1 \sigma_{\varphi_1}^2 + r_1^2 \cos^2 \varphi_1 \sin^2 \theta_1 \sigma_{\theta_1}^2}}{t_2 - t_1} \\
\Delta \sigma_z &= \frac{\sqrt{\sin^2 \varphi_2 \sigma_{r_2}^2 + r_2^2 \cos^2 \varphi_2 \sigma_{\varphi_2}^2}}{t_2 - t_1} - \frac{\sqrt{\sin^2 \varphi_1 \sigma_{r_1}^2 + r_1^2 \cos^2 \varphi_1 \sigma_{\varphi_1}^2}}{t_2 - t_1}
\end{aligned}
\right.
\tag{43}
\end{equation}

当测得的目标平均速度满足:

\begin{equation}
\left[ \left( |v_{x2}| - \beta \cdot \Delta \sigma_x \right)^+ \right]^2 + \left[ \left( |v_{y2}| - \beta \cdot \Delta \sigma_y \right)^+ \right]^2 + \left[ \left( |v_{z2}| - \beta \cdot \Delta \sigma_z \right)^+ \right]^2 < v_{\text{max}}^2
\tag{44}
\end{equation}

则判定为目标 1 航迹,否则判定为目标 2 航迹。式中,\( [x]^+ = \max(0, x) \),\( \beta \) 为预先设定的标准差系数。

\begin{figure}[h]
    \centering
    \includegraphics[width=0.8\textwidth]{image.png}
    \caption{$t_{1}$与$t_{2}$时刻目标与雷达的位置关系}
    \label{fig:21}
\end{figure}

{角度域判决准则}

测得的两个时刻目标方位角变化的绝对值小于某最大值$\delta_{\theta}$,同时,目标俯仰角变化的绝对值也小于某最大值$\delta_{\varphi}$,则判定为目标1航迹,否则判定为目标2航迹。从数学角度讲,该判决可以表述为:
\begin{equation}
\begin{cases}
|\theta_{2}-\theta_{1}|<\delta_{\theta} \\
|\varphi_{2}-\varphi_{1}|<\delta_{\varphi}
\end{cases}
\tag{45}
\end{equation}

综上所述,基于启发式动态规划的数据关联算法的本质就是目标在速度域满足判决准则:
\begin{equation}
\left[\left(|v_{x_{2}}|-\beta\cdot\Delta\sigma_{x}\right)^{+}\right]^{2}+\left[\left(|v_{y_{2}}|-\beta\cdot\Delta\sigma_{y}\right)^{+}\right]^{2}+\left[\left(|v_{z_{2}}|-\beta\cdot\Delta\sigma_{z}\right)^{+}\right]^{2}<v_{\text{max}}^{2}
\tag{46}
\end{equation}
或在角度域满足判决准则:
\begin{equation}
\begin{cases}
|\theta_{2}-\theta_{1}|<\delta_{\theta} \\
|\varphi_{2}-\varphi_{1}|<\delta_{\varphi}
\end{cases}
\tag{47}
\end{equation}

的条件下,判定为目标1航迹,否则判定为目标2航迹。

基于启发式动态规划的数据关联算法的流程图如图22所示。

{步骤三 目标航迹关联处理}

图23给出了采用所提算法得到的目标1与目标2的航迹关联轨迹。从图中可以分析得出,本文提出的基于启发式动态规划的数据关联算法通过动态地处理雷达连续扫描期间接收的顺序观测值,对目标进行速度域判决和角度(包含方位角、俯仰角)域判决,能够可靠地对Data2.txt中的两个目标量测数据进行

\begin{figure}[h]
    \centering
    \includegraphics[width=\textwidth]{image1.png}
    \caption{基于启发式动态规划的数据关联算法流程图}
    \label{fig:flowchart}
\end{figure}

关联,从而区分出各个目标相应的航迹。另外,所提算法计算复杂度低,在没有目标先验信息的情况下,可以达到快速航迹关联的目的,具有序贯实时的特点。

\begin{figure}[h]
    \centering
    \includegraphics[width=\textwidth]{image2.png}
    \caption{目标 1 与目标 2 的航迹关联轨迹}
    \label{fig:trajectories}
\end{figure}

\section{问题三的建模与求解}

\subsection{问题分析}

问题三要求我们根据 Data3.txt 中的数据,研究目标的机动变化规律,并分析问题一中所采用的目标跟踪模型对结果的影响。我们由坐标系转换后的目标点迹,建立目标跟踪模型,并采用 IMM-EKF 对目标进行序贯跟踪,进而获得滤波后的目标航迹信息,并在此基础上分析目标的机动运动规律。

\subsection{问题建模}

图 24 给出了问题三的建模及求解流程图。本文采用的方法是:(1) 将目标测量值由空间极坐标系转换到雷达质心直角坐标系中;(2) 根据目标量测点迹,建立目标三维机动模型;(3) 采用基于扩展卡尔曼滤波的交互式多模型算法进行目标跟踪;(4) 通过 (3) 中的目标跟踪算法,估计出目标航迹;(5) 分析目标的机动规律,并讨论问题一中的目标跟踪模型对问题三的影响。

\begin{figure}[h]
    \centering
    \includegraphics[width=0.8\textwidth]{image.png} % 替换为实际图片路径
    \caption{问题三建模及求解流程图}
    \label{fig:flowchart}
\end{figure}

\subsection{问题求解}

{步骤一 坐标系转换}

由 Data3.txt 中数据可以看出,雷达关于目标的测量值是在空间极坐标系中获得的。因此,采用前文所述的坐标系转换方法,将目标测量值由空间极坐标系转换到雷达质心直角坐标系中。

{步骤二 建立目标三维机动模型}

此处,为了适应目标的机动运动,仍然采用问题一中所建立的协同转弯模型与变加速 Singer 模型,具体模型如前所述。

{步骤三 采用基于扩展卡尔曼滤波的交互式多模型算法对目标在线跟踪}

具体算法参见前文。

{步骤四 估计目标航迹}

图 25 给出了雷达对目标的测量点迹以及采用基于扩展卡尔曼滤波的交互式多模型算法对目标的估计航迹。从图 25 中可以看出,采用问题一中的目标跟踪模型与在线跟踪算法,对 Data3.txt 中的数据进行分析,仍可以很好地从大量

的目标量测点中提取出目标的航迹信息,从而建立对目标航迹可靠、精确地估计,并保证了对机动目标的航迹估计与跟踪的稳健性与实时性。

\begin{figure}[h]
    \centering
    \includegraphics[width=\textwidth]{image.png}
    \caption{雷达质心坐标系下目标的运动轨迹}
    \label{fig:radar_trajectory}
\end{figure}

{步骤五 分析目标机动运动规律}

由于建立的变加速 Singer 目标跟踪模型受到最大加速度参数 $a_m$ 设置的影响,而且目标类型未知,因此,根据题目中所给定的不同类型目标的机动能力进行参数设置。选取 $a_m=10g$ 和 $a_m=100g$ 分别代表通常情况下战斗机和导弹目标的最大加速度。

{$a_m=10g$ 情况}

如前所述,在 14470.0s 到 14520.0s 期间,目标的加速度和速度均出现一个大的尖峰,这主要是因为在 14470.0s 雷达刚建立目标航迹,目标的加速度、速度由初始值 $0 \, \text{m/s}^2$ 和 $0 \, \text{m/s}$ 骤然上升,而后又恢复正常值。

由图 26-图 28 可以看出,从 14520.0s 开始,目标做爬升运动。其间,目标加速度由 $30 \, \text{m/s}^2$ 左右增至 $75 \, \text{m/s}^2$ 左右,而后减小至 $20 \, \text{m/s}^2$ 左右并维持到目标跟踪结束;目标速度维持在 $2000$-$2500 \, \text{m/s}$;目标的加速度方位角和加速度俯仰角持续增大。从 14900.0s 至 15000.0s,目标开始做俯冲运动。其间,目标加速度稳定在 $20 \, \text{m/s}^2$ 左右,目标速度由 $2600 \, \text{m/s}$ 左右增至 $3100 \, \text{m/s}$ 左右。

\begin{figure}[h]
    \centering
    \includegraphics[width=\textwidth]{image1.png}
    \caption{$a_{m}=10g$ 情况下目标加速度随时间的变化曲线}
    \label{fig:acceleration}
\end{figure}

\begin{figure}[h]
    \centering
    \includegraphics[width=\textwidth]{image2.png}
    \caption{$a_{m}=10g$ 情况下目标速度随时间的变化曲线}
    \label{fig:velocity}
\end{figure}

\begin{figure}[h]
    \centering
    \includegraphics[width=\textwidth]{image1.png}
    \caption{$a_{m}=10g$ 情况下目标加速度方位角与俯仰角随时间的变化曲线}
    \label{fig:28}
\end{figure}

(2) $a_{m}=100g$ 情况:

\begin{figure}[h]
    \centering
    \includegraphics[width=\textwidth]{image2.png}
    \caption{$a_{m}=100g$ 情况下目标加速度随时间的变化曲线}
    \label{fig:29}
\end{figure}

如前所述,在 14470.0s 到 14520.0s 期间,目标的加速度和速度均出现一个大的尖峰,这主要是因为在 14470.0s 雷达刚建立目标航迹,目标的加速度、速度由初始值 $0\,\text{m/s}^2$ 和 $0\,\text{m/s}$ 骤然上升,而后又恢复正常值。

类似地,由图 29-图 31 可以看出,从 14520.0s 开始,目标做爬升运动。其间,目标加速度由 $80\,\text{m/s}^2$ 左右增至 $400\,\text{m/s}^2$ 左右,而后减小至 $100\,\text{m/s}^2$ 左右并维持

到目标跟踪结束;目标速度维持在 \(2500 \, \text{m/s}\) 左右;目标的加速度方位角和加速度俯仰角持续增大。从 \(14900.0 \, \text{s}\) 至 \(15000.0 \, \text{s}\),目标开始做俯冲运动。其间,目标加速度稳定在 \(100 \, \text{m/s}^2\) 左右,目标速度由 \(2500 \, \text{m/s}\) 左右增至 \(3200 \, \text{m/s}\) 左右。

\begin{figure}[h]
    \centering
    \includegraphics[width=\textwidth]{image1.png}
    \caption{\(a_m = 100g\) 情况下目标速度随时间的变化曲线}
    \label{fig:velocity}
\end{figure}

\begin{figure}[h]
    \centering
    \includegraphics[width=\textwidth]{image2.png}
    \caption{\(a_m = 100g\) 情况下目标加速度方位角与俯仰角随时间的变化曲线}
    \label{fig:angles}
\end{figure}

本文所采用的目标跟踪模型,综合考虑了目标的转弯、变加速等机动运动方式,对于机动目标的跟踪具有很好地适应性和稳健性。

\section{七 问题四的建模与求解}

\subsection{7.1 问题分析}

问题四要求我们对 Data3.txt 中的目标轨迹进行实时预测,估计目标的着落点,并分析所采用算法的复杂度。为此,我们采用 IMM-EKF 算法对目标的着落点进行预测。在此基础上,我们将对 IMM-EKF 算法与基于多项式拟合的目标轨迹预测方法进行比较,并分析了两者的算法复杂度。

\subsection{7.2 问题建模}

\begin{figure}[h]
    \centering
    \includegraphics[width=0.8\textwidth]{image.png}
    \caption{问题四建模及求解流程图}
    \label{fig:flowchart}
\end{figure}

图 32 给出了问题四的建模及求解流程图。通过采用基于扩展卡尔曼滤波的交互式多模型算法对目标将来时刻的状态值进行无量测更新的纯预测。在此基础上,通过计算算法的复杂度将此算法与基于多项式拟合的目标轨迹预测方法进行了比较与分析。

\subsection{7.3 问题求解}

{步骤一 坐标系转换}

由 Data3.txt 中数据可以看出,雷达关于目标的测量值是在空间极坐标系中获得的。因此,采用前文所述的坐标系转换方法,将目标测量值由空间极坐标系转换到地球直角坐标系中。

{步骤二 采用基于扩展卡尔曼滤波的交互式多模型算法对目标进行跟踪}

基于扩展卡尔曼滤波的交互式多模型算法已在问题一中进行了详细的介绍,此处仅对预测方法展开介绍。由于没有新的量测数据,扩展卡尔曼滤波中基于新息的目标状态预测值的更新过程无法实现。所以只能根据状态转移方程对目标状态进行纯预测。

假设目标在第 $k$ 时刻的状态值为 $x_k$,第 $j$ 个模型的状态转移函数为 $f^j(\cdot)$,则第 $k+1$ 时刻第 $j$ 个模型目标状态预测值 $\hat{x}_{k+1}^j$ 为:

\begin{equation}
\hat{X}_{k+1}^{j} = f^{j}(X_{k})
\tag{48}
\end{equation}

利用最后一时刻滤波后各模型的概率 $P_{r}^{j}$ 计算 $k+1$ 时刻目标状态的预测值 $X_{k+1}$ 为:

\begin{equation}
X_{k} = \sum_{j=1}^{M} P_{r}^{j} \cdot \hat{X}_{k+1}^{j}
\tag{49}
\end{equation}

其中, $M$ 为模型的个数。

运用式 (48) 和式 (49) 即可预测 $N$ 时刻目标的状态值。

{步骤三 由跟踪算法预测目标着落点}

图 33 给出了基于交互式多模型跟踪算法的目标着落点预测流程图。通过计算每一个预测时刻的目标径向距离, 判断其与地球平均半径的大小, 如果在某一时刻小于等于地球平均半径, 则认为当前时刻目标的坐标即为目标的着落点。

\begin{figure}[h]
\centering
\includegraphics[width=\textwidth]{image1.png}
\caption{基于 IMM-EKF 算法目标着落点预测流程图}
\end{figure}

\begin{figure}[h]
\centering
\includegraphics[width=\textwidth]{image2.png}
\caption{地球直角坐标系下目标运动轨迹及预测}
\end{figure}

对于经过坐标转换的 Data3.txt 中数据, 利用图 34 给出的着落点预测流程图, 估计目标在 42 秒后着落, 目标在地球直角坐标系中的着落点坐标为 $(-2180.0, 4377.1, 4076.5)$, 所在的经纬度为: 东经 116.4761 度, 北纬 39.8148 度。

图 34 给出了 Data3.txt 中目标轨迹的滤波值以及着落前的目标预测轨迹。

{步骤四 采用最小二乘算法拟合目标运动轨迹}

为了与基于扩展卡尔曼滤波的交互式多模型算法对目标着落点的预测算法进行对比,此处采用基于最小二乘拟合对目标着落点进行预测。

在地球直角坐标系下,假设目标在 \(x\) 轴、\(y\) 轴及 \(z\) 轴上的坐标随时间的变化函数可以用 \(K\) 阶多项式进行拟合,拟合多项式如下:

\begin{equation}
\left\{
\begin{aligned}
x(t) &= \sum_{k=1}^{K} a_k t^k + a_0 \\
y(t) &= \sum_{k=1}^{K} b_k t^k + b_0 \\
z(t) &= \sum_{k=1}^{K} c_k t^k + c_0
\end{aligned}
\right.
\tag{50}
\end{equation}

式中,\(a_0, \ldots, a_K\),\(b_0, \ldots, b_K\),\(c_0, \ldots, c_K\) 为待求系数。

通过最小均方误差准则,即式 (51),计算出待求系数。

\begin{equation}
\left\{
\begin{aligned}
\min_{a_k} \left( \sum_{n=1}^{N} \left( \sum_{k=1}^{K} a_k t^k + a_0 - x(n) \right) \right) \\
\min_{b_k} \left( \sum_{n=1}^{N} \left( \sum_{k=1}^{K} b_k t^k + b_0 - y(n) \right) \right) \\
\min_{c_k} \left( \sum_{n=1}^{N} \left( \sum_{k=1}^{K} c_k t^k + c_0 - z(n) \right) \right)
\end{aligned}
\right.
\tag{51}
\end{equation}

此处给出了 Data3.txt 中数据在地球坐标系下各坐标轴上的坐标关于时间的多项式拟合函数,此处采用三阶多项式进行拟合。图 35 给出了三阶多项式拟合的结果。

\begin{figure}[h]
\centering
\includegraphics[width=\textwidth]{image.png}
\caption{目标在各坐标轴上坐标关于时间的三阶多项式拟合图}
\end{figure}

{步骤五 由拟合算法预测目标着落点}

表 2 给出了各多项式拟合阶数下目标的着落点在地球坐标系下的坐标,着落时间,以及与交互式多模型扩展卡尔曼滤波预测下目标着落点的距离。

\begin{table}
\centering
\caption{基于多项式拟合的目标着落点预测}
\begin{tabular}{l c c c c}
\hline
多项式拟合 & 2 & 3 & 4 & 5 \\
阶数 & & & & \\
\hline
着落点坐标 & (-2167.9, & (-2186.1, & (-2169.5, & (-2158.2, \\
 & 4382.9, & 4376.6, & 4375.4, & 4390.9, \\
 & 4077.9) & 4074.1) & 4083.0) & 4074.8) \\
\hline
着落时间 & 35 & 42 & 39 & 30 \\
\hline
与 IMM-EKF 预测 & 13.44km & 6.52km & 12.44km & 25.83km \\
着落点的距离 & & & & \\
\hline
\end{tabular}
\end{table}

{步骤六 对比两种算法所得目标着落点结果}

由上表可见当拟合多项式的阶数为 3 时,与 IMM-EKF 预测的着落点最接近,且着落预测时间和 IMM-EKF 方法相同。具体算法复杂度上的两种算法比较与分析将在步骤九中详细介绍。

{步骤七 分析目标跟踪算法复杂度}

预测 $N$ 秒后目标的运动状态 $x_{N}$,首先需要迭代状态转移方程,即式 (48),此处考虑状态转移方程为线性方程,则可以将式 (49) 改写为:

\begin{equation}
\hat{X}_{k+1}^{j} = F^{j} X_{k}
\tag{52}
\end{equation}

其中,$F^{j}$ 为第 $j$ 个目标的状态转移矩阵,此处考虑目标状态 $X_{k}$ 为包含目标在 $xyz$ 轴上坐标的三维向量 $X_{k} = (x_{k}, y_{k}, z_{k})$。

由于 $X_{k}$ 和 $\hat{X}_{k+1}^{j}$ 是三维向量,$F^{j}$ 为 $3 \times 3$ 矩阵,则每次迭代进行 9 次乘法,6 次加法。本文采用的模型数为 3,计算 $N$ 秒后的目标运动状态,共需进行 $27 \cdot N$ 次乘法和 $18N$ 次加法,则算法复杂度为 $O(N)$;

将式 (49) 写成矩阵形式,如下:

\begin{equation}
X_{k} = P \cdot \hat{X}_{k+1}^{j}
\tag{53}
\end{equation}

式中,$P$ 是 $3 \times 3$ 的模型概率矩阵,需要计算 $9 \cdot N$ 次乘法和 $6 \cdot N$ 加法,此运算复杂度为 $O(N)$。

计算直角坐标 $X_{N} = (x_{N}, y_{N}, z_{N})$ 下目标距离地心的距离 $r$。如果 $r$ 小于地球半径,则认为目标已经着落。

\begin{equation}
R_{N} = \sqrt{x_{N}^{2} + y_{N}^{2} + z_{N}^{2}}
\tag{54}
\end{equation}

式 (54) 需进行 $3N$ 次乘法和 $3N$ 次加法,此运算复杂度为 $O(N)$。

综合上述,可知采用基于扩展卡尔曼滤波的交互式多模型预测算法需要 $39\cdot N$ 次乘法,$27\cdot N$ 次加法,算法复杂度为 $O(N)$。

{步骤八 分析拟合算法复杂度}

对已知轨迹进行基于最小二乘法的多项式拟合,利用拟合后的多项式计算 $N$ 秒后目标的坐标并判断目标是否着落。

分别对目标每个坐标轴方向上的曲线进行多项式拟合,假设多项式的阶数为 $k$,待拟合的曲线的数据点数为 $M$,则拟合运算复杂度为 $O(M^3)$。

用 $k$ 阶拟合多项式计算目标在 $N$ 秒后的预测坐标 $(x_N, y_N, z_N)$,假设 $k$ 阶多项式为:
\begin{equation}
y = \sum_{i=1}^{k} a_i x^i
\tag{55}
\end{equation}
则计算 $N$ 个点的坐标需进行 $3\frac{k(k+1)}{2}N$ 次乘法和 $3kN$ 次加法,时间复杂度为 $O(N)$。

{步骤九 两种预测算法的复杂度比较与分析}

基于交互式多模型扩展卡尔曼滤波预测算法和基于多项式拟合预测算法的复杂度都是 $O(N)$,但多项式拟合预测的算法首先要利用已有的 $M$ 个数据点进行拟合,此过程中算法复杂度为 $O(M^3)$,而且没有利用卡尔曼滤波得到的结果。相比较而言,基于交互式多模型扩展卡尔曼滤波预测算法在先前滤波结果的基础上直接进行预测,避免数据拟合所需要的计算,更容易做到实时更新。由于多项式拟合的阶数与拟合精度并没有完全确定的规律,在实际的轨迹预测中只能选择一些经验数据。所以基于交互式多模型扩展卡尔曼滤波预测算法更适合对目标进行实时预测。

\section{八 问题五的建模与求解}

\subsection{8.1 问题分析}

问题五要求我们针对本文所采用的目标跟踪模型,设计相应的目标逃逸策略以摆脱雷达的锁定跟踪,并针对目标逃逸等反跟踪策略,设计相应的雷达稳定跟踪方案。对此,我们建立了基于波束覆盖范围限制的最小化逃逸距离数学优化模型。在此基础上,又提出了雷达对目标的稳定跟踪策略。

\subsection{8.2 问题建模}

图 36 给出了问题五的建模及求解流程图。本文采用的方法步骤是:(1) 取 Data2.txt 中一个目标的测量数据,并对目标的测量值进行坐标转换,将目标测

量值由空间极坐标系转换到地球直角坐标系中;(2)根据目标量测点迹,建立基于协同转换模型和变加速 Singer 模型的目标三维机动模型;(3)采用 IMM-EKF 算法进行目标跟踪;(4)通过(3)中的目标跟踪算法,估计出目标航迹;(5)在雷达前后两次扫描时间间隔取最小值 0.5s 的情况下,通过研究分析雷达照射锥角与目标机动运动模型之间的关系,利用线性规划理论,建立适用于目标逃逸并摆脱雷达锁定跟踪的机动优化模型;(6)针对目标可能采取的逃逸等反跟踪策略,又提出雷达对目标的稳定跟踪方案,从而保持对目标稳定、可靠的跟踪。

\begin{figure}[h]
    \centering
    \includegraphics[width=\textwidth]{image.png}
    \caption{问题五建模及求解流程图}
    \label{fig:flowchart}
\end{figure}

\subsection{问题求解}

{步骤一 坐标系转换}

由 Data2.txt 中数据可以看出,雷达关于目标的测量值是在空间极坐标系中获得的。因此,采用前文所述的坐标系转换方法,将目标测量值由空间极坐标系转换到雷达质心直角坐标系中。

{步骤二 建立目标三维机动模型}

此处,为了适应目标的机动运动,仍然采用问题一中所建立的协同转弯模型与 Singer 模型,具体模型如前所述。

{步骤三 采用 IMM-EKF 算法对目标在线跟踪}

具体算法参见前文。

{步骤四 估计目标航迹}

图 37 给出了雷达对目标的测量点迹以及采用基于扩展卡尔曼滤波的交互式多模型算法对目标的估计航迹。从图 37 中可以看出,采用问题一中的目标跟踪模型与在线跟踪算法,仍可以对目标航迹进行可靠、精确地估计,同时保证了对机动目标的航迹估计与跟踪的稳健性与实时性。

\begin{figure}[h]
    \centering
    \includegraphics[width=\textwidth]{image1.png}
    \caption{雷达质心坐标系下的目标轨迹}
    \label{fig:37}
\end{figure}

{步骤五 分析雷达照射锥角与目标机动运动关系,建立目标逃逸的机动优化模型}

由题目五可知,Data2.txt 数据中的目标已被雷达锁定跟踪,且目标能够了解是否被跟踪。雷达的测量精度为雷达波束宽度为 $3^\circ$,即在以雷达为锥顶,雷达与目标连线为轴,半顶角为 $1.5^\circ$ 的圆锥内的目标均能被探测到;而且,雷达前后两次扫描时间间隔最小为 $0.5\,\text{s}$。在此,我们考虑假设雷达为锁定目标,始终保持前后两次对目标的扫描时间间隔为最小值 $0.5\,\text{s}$。

因此,可以得到目标逃逸雷达锁定跟踪的策略:只要目标在雷达下一次扫描时,能够逃逸出雷达的照射波束,即雷达的照射锥角范围,就可以摆脱雷达的锁定跟踪。

\begin{figure}[h]
    \centering
    \includegraphics[width=\textwidth]{image2.png}
    \caption{雷达质心坐标系下的预测目标运动轨迹}
    \label{fig:38}
\end{figure}

图 \ref{fig:38} 给出了雷达质心坐标系下的预测目标运动轨迹。雷达根据已获得的目标的测量点迹,采用基于扩展卡尔曼滤波的交互式多模型算法对目标未来的运动轨迹进行预测,预测的目标运动轨迹如图38 中红色实线所示。 
图39-图41 分别给出了预测的目标运动轨迹与雷达照射波束在XOY、XOZ、
YOZ 平面上投影的示意图。如图所示,目标的当前位置处于绿色“*”处,雷达预测目标0.5s 后的位置处于红色“*”处,0.5s 后雷达相应的照射波束位置为红色实线所示意的范围。

\begin{figure}[h]
    \centering
    \includegraphics[width=\textwidth]{image1.png}
    \caption{雷达照射波束XOY平面示意图}
    \label{fig:radar_xoy}
\end{figure}

\begin{figure}[h]
    \centering
    \includegraphics[width=\textwidth]{image2.png}
    \caption{雷达照射波束XOZ平面示意图}
    \label{fig:radar_xoz}
\end{figure}

\begin{figure}[h]
    \centering
    \includegraphics[width=\textwidth]{image.png}
    \caption{雷达照射波束 YOZ 平面示意图}
    \label{fig:radar_beam}
\end{figure}

根据我们提出的目标逃逸策略,可以知道,只要目标在下一次雷达扫描过程中能尽快地逃逸出其照射波束范围,即目标在 XOY、XOZ、YOZ 平面上投影均处于红色雷达波束范围之外,目标就能够摆脱雷达的锁定跟踪。

由此,我们建立了基于雷达照射波束位置限制的目标逃逸优化模型。假设目标当前位置为 \((x_1, y_1, z_1)\),而根据采用的目标跟踪算法预测 0.5 秒后目标的位置为 \(\left(x_1', y_1', z_1'\right)\)。由图 39-图 41,可得,目标为摆脱雷达锁定跟踪,在雷达下一扫描时刻目标的位置应满足:

\begin{align*}
\text{XOY 面: } & y > \frac{x_1' \sin(-\theta) + y_1' \cos(-\theta)}{x_1' \cos(-\theta) - y_1' \sin(-\theta)} x \text{ 或 } y < \frac{x_1' \sin(\theta) + y_1' \cos(\theta)}{x_1' \cos(\theta) - y_1' \sin(\theta)} x \\
\text{XOZ 面: } & z > \frac{x_1' \sin(-\theta) + z_1' \cos(-\theta)}{x_1' \cos(-\theta) - z_1' \sin(-\theta)} x \text{ 或 } z < \frac{x_1' \sin(\theta) + z_1' \cos(\theta)}{x_1' \cos(\theta) - z_1' \sin(\theta)} x \\
\text{YOZ 面: } & z > \frac{y_1' \sin(\theta) + z_1' \cos(\theta)}{y_1' \cos(\theta) - z_1' \sin(\theta)} y \text{ 或 } z < \frac{y_1' \sin(-\theta) + z_1' \cos(-\theta)}{y_1' \cos(-\theta) - z_1' \sin(-\theta)} y
\end{align*}

式中,\(\theta = 1.5^\circ\) 为雷达波束的半顶角。

另外,考虑到目标发生机动时最有利的逃逸手段是向前偏转机动,因此,目标逃逸的机动优化模型可表示为:

\begin{equation}
\begin{aligned}
& \min _{\left\{x_{0}, y_{0}, z_{0}\right\}} d^{2}\left(X_{1}, X_{0}\right)=\left(x_{1}-x_{0}\right)^{2}+\left(y_{1}-y_{0}\right)^{2}+\left(z_{1}-z_{0}\right)^{2} \\
& \text { s.t.: } y_{0}<\frac{x_{1}^{\prime} \sin (\theta)+y_{1}^{\prime} \cos (\theta)}{x_{1}^{\prime} \cos (\theta)-y_{1}^{\prime} \sin (\theta)} x_{0} \\
& z_{0}<\frac{x_{1}^{\prime} \sin (\theta)+z_{1}^{\prime} \cos (\theta)}{x_{1}^{\prime} \cos (\theta)-z_{1}^{\prime} \sin (\theta)} x_{0} \\
& z_{0}>\frac{y_{1}^{\prime} \sin (\theta)+z_{1}^{\prime} \cos (\theta)}{y_{1}^{\prime} \cos (\theta)-z_{1}^{\prime} \sin (\theta)} y_{0}
\end{aligned}
\end{equation}

为验证所提出的目标逃逸优化模型,在图 38 的基础上进行仿真。图 42 给出了雷达质心坐标系下的预测目标运动轨迹与目标实际逃逸轨迹。从图中可以看出,采用本文提出的目标逃逸优化模型,目标可以有效地摆脱雷达的锁定跟踪。

\begin{figure}[h]
    \centering
    \includegraphics[width=\textwidth]{image.png} % 替换为实际图像文件名
    \caption{雷达质心坐标系下的预测目标运动轨迹与目标实际逃逸轨迹}
    \label{fig:42}
\end{figure}

{步骤六 针对目标可能的机动逃逸策略,建立相应的雷达跟踪模型}

雷达对目标跟踪丢失的主要原因是雷达波束覆盖面积有限,不能在较大范围内照射到目标,从而丢失了下一时刻目标的测量数据,使得所采用的目标跟踪模型与跟踪算法失效。

因此,为了保持对目标的稳定跟踪,可以采用以下方法:

\begin{enumerate}
    \item \textbf{增大波束覆盖面积:}

    这样,可使目标在雷达下一扫描时刻难以逃出其波束覆盖范围,从而保证雷达能够对目标获得连续、稳定的测量值。

    \item \textbf{缩短雷达扫描时间间隔:}

    尽可能地缩短雷达前后两次扫描时间间隔,从而增大测得的目标数据率,
\end{enumerate}

保持雷达对目标的稳定跟踪。

(3) 采用边扫描边跟踪策略:

雷达在对目标进行跟踪的同时,还可以对其他的空域范围或者目标附近的空域进行扫描,以避免目标通过机动摆脱雷达的锁定跟踪。

图 43 给出了雷达质心坐标系下,雷达针对目标逃逸采取稳定跟踪策略所得的跟踪轨迹。从图中可以看出,在目标采取逃逸等反跟踪策略时,采用本文提出的雷达稳定跟踪模型,能够增大雷达获得的关于目标的量测数据,从而有效地避免目标通过机动摆脱雷达的锁定跟踪。

\begin{figure}[h]
    \centering
    \includegraphics[width=\textwidth]{image.png}
    \caption{雷达质心坐标系下的雷达对目标逃逸轨迹的跟踪}
    \label{fig:43}
\end{figure}

\section{九 模型总结}

对于问题一,首先,通过对三部雷达关于目标的量测数据进行预处理,获得目标点迹。之后,建立目标运动模型,并采用 IMM-EKF 算法对目标进行在线跟踪,进而通过滤波得到目标的估计轨迹、速度和加速度。最后,统计目标加速度的大小和方向,并分析目标的机动运动状态。

对于问题二,提出了基于启发式动态规划的数据关联算法。该算法采用启发式规则的方法对目标进行速度域判决与角度域判决,从而获得各目标相应的航迹。通过分析得到,所提算法具有序贯实时的特点,能够达到快速航迹关联的目的。

对于问题三,根据坐标系转换后的目标点迹,建立目标运动模型,并采用 IMM-EKF 对目标进行序贯跟踪,进而获得滤波后目标的估计轨迹、速度和加速度。最后,研究分析目标的机动变化规律。可以得出,本文所采用的目标跟踪模型对机动目标的跟踪具有很好地适应性和稳健性。

对于问题四,本文采用 IMM-EKF 算法对目标的着落点进行预测。在此基础上,本文将 IMM-EKF 算法与基于多项式拟合的目标轨迹预测方法进行了比较,并分析了两者的算法复杂度。
对于问题五,建立了基于波束覆盖范围限制的最小化逃逸距离数学优化模
型。分析得到,该逃逸优化模型对于本文采用的目标跟踪模型,可快速、有效地
摆脱雷达的锁定跟踪。在此基础上,又提出了雷达对目标的稳定跟踪策略。
\section{参考文献}

[1] 何友, 信息融合理论及应用[M], 第 2 版. 北京:电子工业出版社, 2010.

[2] 何友, 修建娟, 张晶炜等, 雷达数据处理及应用[M], 第 2 版. 北京: 电子工业出版社, 2009.

[3] 李世忠, 王国宏, 吴巍等, IMM-EKF 雷达与红外序贯滤波跟踪机动目标[J], 火力与指挥控制, 37(1): 39-42, 2012.

[4] 尹瑞, 王荫槐, 王峰, 交互式多模型机动目标跟踪方法的仿真[J], 现代雷达, 29(7): 52-54, 2007.

[5] Li X R, Jilkov V P, A Survey of Maneuvering Target Tracking—Part III: Measurement Models[C], In: Proceedings of SPIE - Signal and Data Processing of Small Targets 2001, 31 July–2 August 2001, San Diego, CA, USA, pp. 423–446.

[6] Lan J, Li X R, Jilkov V P, et al, Second-order Markov chain based multiple-model algorithm for maneuvering target tracking[J], IEEE Transactions on Aerospace and Electronic Systems, 49(1): 3-19, 2013.

[7] Sobhani B, Paolini E, Giorgetti A, et al, Target tracking for UWB multistatic radar sensor networks[J], IEEE Journal of Selected Topics in Signal Processing, 8(1): 125-136, 2014.

[8] Lan J, Li X R, Jilkov V P, et al, Second-order Markov chain based multiple-model algorithm for maneuvering target tracking[J]. IEEE Transactions on Aerospace and Electronic Systems, 49(1): 3-19, 2013.

[9] Ikram M Z, Ali M, A new data association method for 3-D object tracking in automotive applications[C], 2014 IEEE Radar Conference, 1187-1191, 2014.

\section{附录 程序文件的功能说明}

\begin{tabular}{ll}
T1_1.m & 地球直角坐标系下 Data1 中各雷达对目标的测量点迹 \\
T1_2.m & Data1 中雷达 1 数据差值后转换到地球直角坐标系下 \\
T1_3.m & 首先运行 T1_2.m 文件,获 T1_radar1.mat 数据文件,在 \\
 & 运行该 m 文件,获得雷达 1 对目标的跟踪轨迹 \\
T1_4.m & 首先运行 T1_2.m 文件,获 T1_radar1.mat 数据文件,在 \\
 & 运行该 m 文件,获得雷达 1 目标在最大加速度 10g 限制下的加速度、速度、加速度方位角和加速度俯仰角随时间变化的曲线 \\
T1_5.m & 首先运行 T1_2.m 文件,获 T1_radar1.mat 数据文件,在 \\
 & 运行该 m 文件,获得雷达 1 目标在最大加速度 100g 限制下的加速度、速度、加速度方位角和加速度俯仰角随时间变化的曲线 \\
T1_6.m & 获得雷达 2 和雷达 3 目标的跟踪轨迹 \\
\end{tabular}

\begin{itemize}
    \item T1\_7.m
    \begin{itemize}
        \item 获得雷达2和雷达3目标在最大加速度10g限制下的加速度、速度、加速度方位角和加速度俯仰角随时间变化的曲线
    \end{itemize}
    \item T1\_8.m
    \begin{itemize}
        \item 获得雷达2和雷达3目标的在最大加速度100g限制下的加速度、速度、加速度方位角和加速度俯仰角随时间变化的曲线
    \end{itemize}
    \item T2.m
    \begin{itemize}
        \item 给出了Data2中目标1与目标2的航迹关联轨迹
    \end{itemize}
    \item T3\_1.m
    \begin{itemize}
        \item 获得Data3中目标的跟踪轨迹
    \end{itemize}
    \item T3\_2.m
    \begin{itemize}
        \item 获得Data3中雷达目标在最大加速度10g限制下的加速度、速度、加速度方位角和加速度俯仰角随时间变化的曲线
    \end{itemize}
    \item T3\_3.m
    \begin{itemize}
        \item 获得Data3中雷达目标在最大加速度100g限制下的加速度、速度、加速度方位角和加速度俯仰角随时间变化的曲线
    \end{itemize}
    \item T4\_1.m
    \begin{itemize}
        \item 将Data3中数据转换成地球坐标系下
    \end{itemize}
    \item T4\_2.m
    \begin{itemize}
        \item 运行T4\_1.m文件获得T4\_radarl.mat数据,获得Data3中目标轨迹以及着落前的目标预测轨迹
    \end{itemize}
    \item T4\_3.m
    \begin{itemize}
        \item 运行T4\_2.m文件获得T4.mat数据,运行该m文件获得n阶多项式对Data3数据拟合的结果图,着落点的坐标,以及与IMM-EKF估计的着落点的距离
    \end{itemize}
    \item T5\_1.m
    \begin{itemize}
        \item 对Data2中目标1在0.5s跟踪间隔下对目标的跟踪轨迹及预测轨迹
    \end{itemize}
    \item T5\_2.m
    \begin{itemize}
        \item 求解优化模型获得目标逃逸轨迹
    \end{itemize}
    \item T5\_3.m
    \begin{itemize}
        \item 获得逃逸轨迹的跟踪轨迹
    \end{itemize}
    \item shujuchuli1.m
    \begin{itemize}
        \item 对缺失的数据进行线性插值,对同一时刻两部雷达数据进行融合
    \end{itemize}
    \item interpradar.m
    \begin{itemize}
        \item 线性差值函数
    \end{itemize}
    \item radarsigma.m
    \begin{itemize}
        \item 已知雷达测量误差,计算某时刻直角坐标下的各坐标轴方向误差
    \end{itemize}
    \item radarfusion.m
    \begin{itemize}
        \item 计算两部雷达的权值并进行数据融合
    \end{itemize}
\end{itemize}