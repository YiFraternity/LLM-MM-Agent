\documentclass{article}
\usepackage{amsmath}
\usepackage{amssymb}

\title{线性流量阀的内筒孔设计}
\author{}
\date{}

\begin{document}

\maketitle

\begin{abstract}
本文讨论了维修线性流量阀时的内筒设计问题。首先我们建立了一个理论模型,设存在某曲线 \( f(x) \) 满足过流面积与内筒旋转角度成严格线性关系,计算得此曲线不存在。然后建立了一个简单的模型,假设内筒孔为正方形,并考察了其过流面积与内筒旋转角度的关系。我们发现其过流面积虽不与内筒旋转角度成严格线性关系,但较接近线性。在此基础上,考虑到尽量减少摩擦,我们从圆锥曲线角度来改进模型。通过适当地变动圆锥曲线参数,寻找使得过流面积与内筒旋转角度的渐近线性效果最好的曲线,结果显示椭圆方案最好。通过对椭圆参数取不同值得到过流面积与旋转角度的数据,然后进行线性回归。以残差平方和均值最小为目标函数,在一个合理的范围内得到:大小按外筒孔为单位圆的比例,内筒孔右半部分是 \( a=4 \), \( b=1.2 \) 的椭圆弧,左半部分是单位圆弧的光滑曲线,作为最好渐近线性的内筒孔设计方案。

对于问题二,为了满足主要工作区占“最大范围”区间长度的 75\% 以上,其间最大过流面积达到外筒孔的面积 85\% 以上的条件,在问题一的基础上对模型进行改进,设计内筒孔曲线为:右半部分 \( a=4 \), \( b=1.3 \) 的椭圆弧,左半部分是单位圆弧的光滑曲线。在考虑到外筒孔存在较小磨损时,我们要求过流面积尽量由内筒孔来控制,即内筒孔在满足上述条件下面积尽量小。采取同样的方法求得了最佳内筒孔曲线设计方案:右半部分 \( a=3 \), \( b=1.1 \) 的椭圆弧,左半部分是单位圆弧的光滑曲线。

\textbf{关键词} \quad 过流面积 \quad 渐进线性 \quad 线性拟合 \quad 残差平方和
\end{abstract}

\section{问题重述}

油田采油钻井,需要通过固井机喷出砂浆形成水泥井管。为了保证水泥井管质量,必须合理的控制砂浆的流量。固井机上用来控制砂浆流量的阀是影响水泥管质量的关键部件。由于内孔损耗严重,维修时只能根据固井机工作原理来加工内筒的曲线形状,使得流量阀要尽量使得砂浆流量与阀体旋转角度成线性。若不能成严格的线性,都应该要让实际情况与线性关系的误差最小。为了使喷射砂浆的效果更好,设计出来内筒孔的形状,在保持渐近线性的同时,应该能使主要工作区间所对应的旋转角度的线性区间至少达“最大范围”的 $75\%$,其间最大过流面积至少达外筒孔的 $85\%$ 以上。

\section{假设说明}

\begin{enumerate}
    \item 外筒孔近似看作圆孔,并假定这个圆的半径为单位长度;
    \item 假定内外筒旋转速度均匀。把圆筒展开成平面,筒转动转化为长方形平动;
    \item 从实际加工的角度考虑,我们设计出的内筒孔曲线形状都定为不太复杂的光滑曲线;
    \item 假设外筒的直径是它上面圆孔的直径的四倍;即外孔筒的直径等于 8;
    \item 内筒孔的最大宽度不超过内筒横切面圆周长的 $25\%$;
    \item 内筒孔的旋转角度与其展成平面后的平动距离本质一样;
\end{enumerate}

\section{符号说明}

\begin{itemize}
    \item $S$: 过流面积;
    \item $t$: 内筒曲线平动的距离;
    \item $a$: 椭圆的长轴长度;
    \item $b$: 椭圆的短轴长度;
\end{itemize}

\section{问题一}

\subsection{模型分析和建立}

要使得过流面积与内筒旋转角度成线性关系,必须设计出合适的内筒孔曲线形状。考虑到实际加工的情况,我们先从简单的对称曲线入手。

为了简化模型我们把两个圆柱筒展成平面,即为长方形,筒转动转化为两个长方形的平动。把外筒假定为圆心在原点的单位圆。由于内外筒轴向没有相对运动,可以把外筒平面固定,只让内筒曲线平动。因为我们只研究过流面积,考虑内筒曲线从刚进入外简单位圆(此时过流面积为零),平动到过流面积最大为止(此时过流面积为单位圆面积)。我们可以把模型简化为只考虑内筒孔半边曲线边界切割外简单位圆的情形。

\subsubsection{严格线性的分析}

要设计内筒形状来实现过流面积与其旋转角度成严格的线性关系,我们先作如下分析。假设内筒孔形状是一个光滑的轴对称曲线,其方程为 $f(x)$,且 $f(x)$ 能实现过流面积与内筒的旋转角度成严格的线性关系。由于我们考虑的曲线和圆都是对称的,鉴于讨论方便,下面我们都只考虑曲线所得过流面积在 $X$ 轴上方那部分的面积,讨论其是否与平动距离 $t$ 成线性关系。

见下图:

\textbf{图 4.1}

\begin{figure}[h]
    \centering
    \includegraphics[width=0.5\textwidth]{image.png}
    \caption{}
\end{figure}

图中的单位圆表示外筒孔,曲线表示内筒孔的形状,阴影部分表示过流面积 $S$ 的一部分。设内筒的旋转速度为 $t$,则由假设知

\[
S = kt
\]

又由上图知 $S$ 是由两部分的积分之和。有

\begin{align*}
x_1 &= \frac{b^2(t - a - 1) - \sqrt{b^4(a + 1 - t)^2 - (b^2 - a^2)(t^2b^2 - 2atb^2 - 2tb^2 + 2ab^2 + b^2 + a^2)}}{b^2 - a^2} \\
x_2 &= \frac{b^2(t - a - 1) + \sqrt{b^4(a + 1 - t)^2 - (b^2 - a^2)(t^2b^2 - 2atb^2 - 2tb^2 + 2ab^2 + b^2 + a^2)}}{b^2 - a^2}
\end{align*}

又有

\begin{align*}
S &= \int_{-1}^{x_2} \sqrt{1-x^2} \, dx + \int_{x_2}^{t-1} b \sqrt{1-\frac{(x+a+1-t)^2}{a^2}} \, dx \\
&= \frac{1}{2} \left( x_2 \sqrt{1-x_2^2} + \arcsin x_2 \right) + \frac{\pi}{4} + \frac{\pi}{4} ab \\
&\quad - \frac{b}{2a} \left[ (x_2+a+1-t) \sqrt{a^2-(x_2+a+1-t)^2} + a^2 \arcsin \frac{x_2+a+1-t}{a} \right]
\end{align*}

对所有的 \( t \leq 2 \) 都成立。其中 \( x_0 \) 表示 \( f(x) \) 与圆的交点的横坐标。

若分别取 \( t = \frac{1}{4}, \frac{1}{2}, \frac{3}{4}, 1 \),

则可以得到下面的方程组:

\begin{align*}
x_1 &= \frac{b^2(t - a - 1) - \sqrt{b^4(a + 1 - t)^2 - (b^2 - a^2)(t^2b^2 - 2atb^2 - 2tb^2 + 2ab^2 + b^2 + a^2)}}{b^2 - a^2} \\
x_2 &= \frac{b^2(t - a - 1) + \sqrt{b^4(a + 1 - t)^2 - (b^2 - a^2)(t^2b^2 - 2atb^2 - 2tb^2 + 2ab^2 + b^2 + a^2)}}{b^2 - a^2}
\end{align*}

求解可知上述的 \( k \) 不存在,也就是说过流面积与内筒的旋转角度不能成严格的线性关系。

\subsubsection{近似线性的设计}

据上述讨论,我们知道过流面积与内筒旋转角度不能成严格线性关系,那么我们下面的问题就是设计出合理的内筒孔形状,使得过流面积与内筒旋转角度近似成线性关系,使其与严格线性关系的误差最小,从而优化维修内筒线性流量阀问题。

\paragraph{模型 1:} 我们首先考虑内筒形状为边长为 2 的正方形的情形

单位圆方程: \( x^2 + y^2 = 1 \)

直线AB方程:$x=-1$(正方形右边在直线AB上)

图4.2

\begin{figure}[h]
    \centering
    \includegraphics[width=0.5\textwidth]{image.png}
    \caption{图4.2}
\end{figure}

直线AB向右平动$t$个单位后,方程变为:$x=-1+t$,通过积分可得阴影部分面积为:

\[
S=\int_{-1}^{t-1} \sqrt{1-x^2} \, dx = \frac{1}{2} \left[ (t-1) \sqrt{1-(t-1)^2} + \arcsin(t-1) \right] + \frac{\pi}{4}
\]

通过Matlab求解,得到正方形对应的过流面积关于平动距离$t$的函数图像,见下图

图4.3 过流面积$s$与平动距离$t$的模拟图

\begin{figure}[h]
    \centering
    \includegraphics[width=\textwidth]{image2.png}
    \caption{图4.3}
\end{figure}

注:曲线为实际函数图像

从图中可知函数与直线比较近似。

\begin{figure}[h]
    \centering
    \includegraphics[width=\textwidth]{image.png}
    \caption{过流面积s与平动距离t的线性回归图}
    \label{fig:linear_regression}
\end{figure}

\begin{table}[h]
    \centering
    \caption{过流面积s与平动距离t的线性回归分析表}
    \label{tab:linear_regression_analysis}
    \begin{tabular}{l c c c c}
        \hline
        Variable & Coefficient & Std. Error & t-Statistic & Prob. \\
        \hline
        T & 0.404470 & 0.003980 & 101.6320 & 0.0000 \\
        \hline
        R-squared & 0.987329 & Mean dependent var & 0.785398 & \\
        Adjusted R-squared & 0.987329 & S.D. dependent var & 0.526058 & \\
        S.E. of regression & 0.059217 & Akaike info criterion & -2.791139 & \\
        Sum squared resid & 0.140265 & Schwarz criterion & -2.749345 & \\
        Log likelihood & 58.21835 & Durbin-Watson stat & 0.035589 & \\
        \hline
    \end{tabular}
\end{table}

注:拟合优度0.987329,t统计量101.6320,残差平方和0.140265,样本容量41。

通过分析t统计量,拟合优度和残差平方和这几个重要的指标,结果显示直线所得过流面积线性较好,可以接受。但是出于对实际情况的考虑,我们选择的曲线应该能尽量的减少摩擦,于是我们考虑一些接近流线形的曲线。基于这种考虑我们在直线上进行改进,考虑圆锥曲线的情形。

\paragraph{模型 2:} 内筒孔曲线形状为椭圆

在圆锥曲线中,我们先考虑最普遍的椭圆曲线,由于我们不能确定具体的椭圆方程,我们尝试对一系列椭圆族作分析,考查它们的过流面积与平动距离的线性关系。为了使过流面积最大时能达到外简单位圆面积,先可定椭圆短轴 $b \geq 1$。

单位圆方程: $x^2 + y^2 = 1$

椭圆族方程: $\frac{(x + a + 1 - t)^2}{a^2} + \frac{y^2}{b^2} = 1$

其中 $t$ 是椭圆曲线平动的距离,$t > 0$,

图 4.5

\begin{figure}[h]
    \centering
    \includegraphics[width=0.8\textwidth]{image1.png}
\end{figure}

其交点坐标为:

\begin{align*}
x_1 &= \frac{b^2(t - a - 1) - \sqrt{b^4(a + 1 - t)^2 - (b^2 - a^2)(t^2b^2 - 2atb^2 - 2tb^2 + 2ab^2 + b^2 + a^2)}}{b^2 - a^2} \\
x_2 &= \frac{b^2(t - a - 1) + \sqrt{b^4(a + 1 - t)^2 - (b^2 - a^2)(t^2b^2 - 2atb^2 - 2tb^2 + 2ab^2 + b^2 + a^2)}}{b^2 - a^2}
\end{align*}

其中阴影部分的面积为过流面积的一部分,其面积积分由于平动距离 $t$ 的变动而需分情况讨论:

\begin{enumerate}
    \item 当内筒旋转角度很小,两筒孔刚相交时,其交点的 $X$ 坐标记为:$x_1 \leq -1 \leq x_2 \leq 1$
    \begin{figure}[h]
    \centering
    \includegraphics[width=0.8\textwidth]{image2.png}
\end{figure}

    其阴影面积积分为

    \begin{align*}
S &= \int_{-1}^{x_2} \sqrt{1-x^2} \, dx + \int_{x_2}^{t-1} b \sqrt{1-\frac{(x+a+1-t)^2}{a^2}} \, dx \\
&= \frac{1}{2} \left( x_2 \sqrt{1-x_2^2} + \arcsin x_2 \right) + \frac{\pi}{4} + \frac{\pi}{4} ab \\
&\quad - \frac{b}{2a} \left[ (x_2+a+1-t) \sqrt{a^2-(x_2+a+1-t)^2} + a^2 \arcsin \frac{x_2+a+1-t}{a} \right]
\end{align*}

    当平动距离 $t=2$ 时,相交情况如下图,其面积积分与情况(1)相同,图 4.7

    \begin{figure}[h]
    \centering
    \includegraphics[width=0.8\textwidth]{image1.png}
    \caption{}
\end{figure}

    \item 当内筒继续旋转,过流面积较大时,此时交点记为 $-1 \leq x_1 \leq x_2 \leq 1$,图 4.8

    \begin{figure}[h]
    \centering
    \includegraphics[width=0.8\textwidth]{image2.png}
    \caption{}
\end{figure}

    其阴影面积积分得:

    \begin{align*}
S &= \int_{-1}^{x_1} \sqrt{1-x^2} \, dx + \int_{x_1}^{x_2} b \sqrt{1-\frac{(x+a+1-t)^2}{a^2}} \, dx + \int_{x_2}^1 \sqrt{1-x^2} \, dx \\
&= \frac{1}{2} \left( x_1 \sqrt{1-x_1^2} + \arcsin x_1 \right) - \frac{1}{2} \left( x_2 \sqrt{1-x_2^2} + \arcsin x_2 \right) + \frac{\pi}{2} \\
&\quad + \frac{b}{2a} \left[ (x_2+a+1-t) \sqrt{a^2-(x_2+a+1-t)^2} + a^2 \arcsin \frac{x_2+a+1-t}{a} \right] \\
&\quad - \frac{b}{2a} \left[ (x_1+a+1-t) \sqrt{a^2-(x_1+a+1-t)^2} + a^2 \arcsin \frac{x_1+a+1-t}{a} \right]
\end{align*}

    \item 当内筒旋转到过流面积最大时,其面积就是单位圆孔的面积,阴影部分面积 $S = \frac{\pi}{2}$

    \begin{figure}[h]
    \centering
    \includegraphics[width=0.6\textwidth]{image.png}
    \caption{图 4.9}
\end{figure}
\end{enumerate}

按照此思路,我们在 Matlab 上实现其过程,计算出不同椭圆曲线的过流面积 \( S \) 随平动距离 \( t \) 变化的结果(见附录 1),然后再在 EViews 中作线性拟合,选取与线性的误差平方和均值最小的椭圆曲线。

为了求线性拟合较好的椭圆曲线具体方程,我们需要确定它的参数,即求出它的长轴 \( a \) 和短轴 \( b \)(\( b \geq 1 \))的具体值。由假设知外筒的直径是它上面圆孔的直径的四倍,那么椭圆长轴就必须满足 \( a < 4\pi \),并且从内筒的坚固的角度考虑,那么内筒孔就必须比较小。如果要使得内筒孔的最大宽度不超过内筒横切面圆周长的 \( 25\% \),按此比例分析,长轴最大长度不能超过 6。从减少磨损的角度考虑,应该让内筒孔与外筒孔重复的部分尽量的多。也就是说长轴的取值应该尽量的小,而短轴的取值应该比外孔圆的半径要大一些。根据上述分析,我们尝试取长轴 \( a = 2, 3, 4, 5 \),定下长轴,再变动短轴 \( b \) 分别讨论 \( b = 1, 1.1, 1.2, 1.3 \) 时的情况,利用搜索法,穷举出这些取值下的椭圆,其过流面积与平动距离的渐近线性关系,在 EViews 线性拟合得到当 \( a = 4, b = 1.2 \) 时,椭圆的过流面积与平动距离的渐近线性效果最好,其与严格线性的误差平方和最小(见残差图 1.10 和回归分析表 1.2)。其它椭圆情形的比较分析见附录。

\begin{figure}[h]
    \centering
    \includegraphics[width=0.8\textwidth]{image.png} % 替换为实际图像文件名
    \caption{过流面积s与平动距离t的线性回归图}
    \label{fig:4.10}
\end{figure}

\begin{table}[h]
    \centering
    \caption{过流面积s与平动距离t的线性回归分析表}
    \label{tab:4.2}
    \begin{tabular}{l c c c c}
        \hline
        Variable & Coefficient & Std. Error & t-Statistic & Prob. \\
        \hline
        T & 0.588274 & 0.010274 & 57.26028 & 0.0000 \\
        \hline
        R-squared & 0.973846 & Mean dependent var & 0.792354 & \\
        Adjusted R-squared & 0.973846 & S.D. dependent var & 0.557950 & \\
        S.E. of regression & 0.090233 & Akaike info criterion & -1.938965 & \\
        Sum squared resid & 0.227977 & Schwarz criterion & -1.891817 & \\
        Log likelihood & 29.11499 & Durbin-Watson stat & 0.072049 & \\
        \hline
    \end{tabular}
\end{table}

注:拟合优度0.973846,t统计量57.26028,残差平方和0.227977,样本容量29。

无论是从拟合优度,t统计量,还是残差平方和,我们都能够接受s与t的线性关系。

我们对抛物线作了类似的讨论,对得到的数据作线性拟合。我们得到的结果显示椭圆拟合效果最佳。事实上,从图形的角度来说,椭圆的弧度能比抛物线和双曲线更好的包含圆,也就是说,这样使得内筒孔能与外筒孔更好地磨合。

按照上述分析,我们设计的内筒孔曲线形状为椭圆曲线与圆的组合体,即它是由左边的一段圆弧和右边的椭圆弧所组成的一个光滑的封闭曲线(如下图)。

\begin{figure}[h]
    \centering
    \includegraphics[width=0.8\textwidth]{image.png}
    \caption{内筒孔曲线形状示意图}
\end{figure}

实线为内筒孔曲线形状。其大小按照外筒孔为单位圆的比例,设计内筒孔曲线为右半曲线是 $a=4, b=1.2$ 的椭圆曲线,左半曲线是单位圆的曲线形状。此曲线形状与严格线性关系的误差最小,能够很好的使过流面积与内筒旋转角成近似线性关系。

\section{问题二}

\subsection{问题分析}

考虑实际情况,固井机向孔壁喷射水泥砂浆时经常采用的“过流面积”是在一个稍小的范围内,被称为主要工作区。为了使线性流量阀的工作效率更高,有效性更好,我们希望设计出的内筒孔形状应该使主要工作区中所对应的旋转角度的线性区间尽量长(至少达“最大范围”区间长度的 75\% 以上),而且主要工作区的最大“过流面积”尽量大(至少要达到外筒孔面积的 85\% 以上),并且仍然要尽量好地保持“过流面积”和内筒的旋转角度之间的“线性关系”。事实上,这就只要求过流面积在从零增加到最大时要满足如下条件:在其随平动距离 $t$ 变化的时间至少要达到总变化时间的 75\% 的这一段过程中,经过的最大过流面积至少达到外筒孔面积的 85\% 的同时,过流面积随 $t$ 的变化能很好地保持线性关系。

在求解问题一时,我们从得到的数据和图表显示:过流面积在内筒旋转角度很小和很大时其变化比较缓慢,但其他情况下变化稍快些,而且变化比较平稳,趋于线性。受此启发,我们在问题一的基础上,对其结果进行改进,在满足问题的两个条件下,使过流面积和内筒旋转角度之间的线性关系更好;如果固井机的外筒孔也发生了轻度的磨损,则在设计内筒孔时,我们只要使得内筒孔的面积在满足线性条件下尽量的小,就能够在一定的程度上弥补此问题。

\subsection{模型的建立及求解}

通过对问题一的分析,可以确定比较理想的内筒曲线图形应该同问题一的结果类似,如下:

\begin{figure}[htbp]
    \centering
    \includegraphics[width=0.6\textwidth]{image1.png}
    \caption{内筒孔的大致形状}
    \label{fig:inner-hole}
\end{figure}

但还需对曲线形状进行调整使其满足问题分析中的(1)(2)两条件。

设定椭圆族方程为:

\[
\frac{(x+a+1-t)^2}{a^2} + \frac{y^2}{b^2} = 1
\]

其中 $t$ 是椭圆曲线的平动距离,$t > 0$。

\begin{figure}[htbp]
    \centering
    \includegraphics[width=0.6\textwidth]{image2.png}
    \caption{图 5.2}
    \label{fig:inner-hole}
\end{figure}

其中阴影部分的面积为过流面积,其面积积分由于平动距离 $t$ 的变动而需分情况讨论:

\begin{enumerate}
    \item 当内筒旋转角度很小,两筒孔刚相交时,如下示意图

    \begin{figure}[htbp]
    \centering
    \includegraphics[width=0.6\textwidth]{image3.png}
    \caption{图 5.3}
    \label{fig:inner-hole}
    \end{figure}

    \includegraphics[width=0.6\textwidth]{image3.png}

    其阴影面积随平动距离变化缓慢;

    \item 当内筒继续旋转,过流面积较大时,示意图如下

    \begin{figure}[htbp]
    \centering
    \includegraphics[width=0.6\textwidth]{image1.png}
    \caption{图 5.4}
    \label{fig:inner-hole}
    \end{figure}

    其阴影面积随平动距离的变化趋于平稳,且线性性更佳。

    \item 当内筒继续旋转,过流面积达到很大时,即过流面积接近外孔的面积。示意图如下

    \begin{figure}[htbp]
    \centering
    \includegraphics[width=0.6\textwidth]{image2.png}
    \caption{图 5.5}
    \label{fig:inner-hole}
    \end{figure}

    其阴影面积随平动距离的变化而变化缓慢。
\end{enumerate}

改变参数 \(a, b\),通过 Matlab 处理得到一系列 \(s\) 的值(数据见附表)。再利用 EViews 对 \(s\) 与 \(t\) 作线性回归。挑选出符合如下条件的一系列 \(s\) 作处理:

去掉过流面积 \(s\) 的前一部分值和最后一部分值,即寻找主要工作区间,并且保证:在其随平动距离 \(t\) 变化的时间至少要达到总变化时间的 \(75\%\) 的这一段过程中,经过的最大过流面积至少达到外筒孔面积的 \(85\%\) 的同时,过流面积随 \(t\) 的变化能很好地保持线性关系。

通过这次筛选后,再对剩下的一系列 \(s\) 值用 EViews 作 \(s\) 与 \(t\) 的线性回归。以线性回归的残差平方和的均值最小作为判断拟合线性效果好坏的标准。得到最优的结果如下:

\begin{figure}[h]
    \centering
    \includegraphics[width=\textwidth]{image.png}
    \caption{截去首尾数据后的s与t拟合图}
    \label{fig:5.7}
\end{figure}

\begin{table}[h]
    \centering
    \caption{截去首尾数据后的s与t回归分析表}
    \label{tab:5.1}
    \begin{tabular}{l l l l l}
        \hline
        Variable & Coefficient & Std. Error & t-Statistic & Prob. \\
        \hline
        T & 0.796908 & 0.002102 & 379.1940 & 0.0000 \\
        \hline
        R-squared & 0.994431 & Mean dependent var & & 0.778864 \\
        Adjusted R-squared & 0.994431 & S.D. dependent var & & 0.441041 \\
        S.E. of regression & 0.032913 & Akaike info criterion & & -3.984799 \\
        Sum squared resid & 0.210150 & Schwarz criterion & & -3.968014 \\
        Log likelihood & 389.5179 & Durbin-Watson stat & & 0.002885 \\
        \hline
    \end{tabular}
\end{table}

注:拟合优度0.994431,t统计量379.1940,残差平方和0.210015

样本容量195。

无论是从拟合优度,t 统计量,还是残差平方和,我们都能够接受截去首尾数据后的 s 与 t 成线性关系。

其中 \( a=4, b=1.3, t=2.58 \)

最优的内筒孔椭圆曲线方程为:

\[
\frac{(x+2.42)^2}{4^2} + \frac{y^2}{1.3^2} = 1
\]

则最优内筒曲线图像如下:

\begin{figure}[h]
\centering
\includegraphics[width=0.8\textwidth]{image.png}
\caption{最优内筒孔曲线图}
\end{figure}

注:最优内筒孔曲线为上图实线部分

如果考虑固井机的外筒孔也发生了程度较轻的磨损,我们认为,为了很好地控制砂浆流量,此时过流面积应尽量由内筒孔来控制,在满足上述条件的同时,其内筒孔面积应尽量小。采取同样的方法,我们求得了最佳内筒孔曲线设计方案:右半部分为 \( a=3, b=1.1 \) 的椭圆弧,左半部分为单位圆弧的光滑曲线。

\section{模型的评价与改进}

\begin{enumerate}
    \item 我们模型设计的弧线形内筒孔能够很好地减小水泥砂浆与内筒孔的摩擦,符合实际情况。
    \item 模型能很好地满足过流面积与内筒孔旋转角度近似成线性这一要求。
    \item 本文以线性回归后的残差平方和的均值作为渐近线性好坏的判断标准,具有一定实际意义。
    \item 寻找内筒孔曲线形状时模型全面考虑到了各种因素,较精确地确定出了参数的范围,再变动参数搜索出最优解,提高了模型的准确性。
    \item 本模型设计出的内筒孔形状不是很复杂,便于加工。但过流面积与内孔旋转角度近似成线性这一要求被弱化。
    \item 本模型主要考虑的是利用圆锥曲线的性质对直线的情况进行改进。可以通过考虑更加复杂的曲线的性质来对模型进行推广。比如我们可以考虑一些类似流线形的光滑曲线来做相应的改进。这种曲线有较好的性质,其弯曲度能调节过流面积相对于旋转角度的变化;也能更好地减小工作时水泥砂浆对阀体的摩擦。
\end{enumerate}

\section{参考文献}

[1] 苏金明, 邱沈勇. MATLAB 6.1 实用指南. 北京: 电子工业出版社, 2002, 1

[2] 姜启源. 数学模型 (第二版). 北京: 高等教育出版社, 2002, 6

[3] 丘维声. 解析几何. 北京: 北京大学出版社, 1988

[4] 宋来忠, 王志明. 数学建模与实验. 北京: 科学出版社, 2005

[5] 刘建军, 章宝华. 流体力学. 北京: 北京大学出版社, 2006

\begin{table}
\centering
\begin{tabular}{cccccccc}
a=2,b=1 & a=2.5,b=1 & a=3,b=1 & a=3.5,b=1 & a=4,b=1 & a=4.5,b=1 & a=5,b=1 & t \\
\hline
0 & 0 & 0 & 0 & 0 & 0 & 0 & 0 \\
0.017127 & 0.015868 & 0.014851 & 0.014008 & 0.013293 & 0.012678 & 0.012141 & 0.1 \\
0.048197 & 0.044687 & 0.041846 & 0.039486 & 0.037485 & 0.035761 & 0.034254 & 0.2 \\
0.088089 & 0.081735 & 0.076582 & 0.072295 & 0.068655 & 0.065515 & 0.062769 & 0.3 \\
0.134918 & 0.125283 & 0.117452 & 0.110925 & 0.105377 & 0.100586 & 0.096393 & 0.4 \\
0.187565 & 0.174306 & 0.163506 & 0.154489 & 0.146815 & 0.14018 & 0.134369 & 0.5 \\
0.245252 & 0.228098 & 0.214091 & 0.202377 & 0.192393 & 0.183752 & 0.176178 & 0.6 \\
0.307393 & 0.286127 & 0.268719 & 0.254134 & 0.241684 & 0.230898 & 0.221434 & 0.7 \\
0.37352 & 0.347972 & 0.327002 & 0.309399 & 0.294351 & 0.281299 & 0.269837 & 0.8 \\
0.443248 & 0.413284 & 0.388623 & 0.367877 & 0.350118 & 0.334694 & 0.321137 & 0.9 \\
0.516247 & 0.481771 & 0.453312 & 0.429322 & 0.408753 & 0.390866 & 0.37513 & 1 \\
0.59223 & 0.553176 & 0.520838 & 0.493519 & 0.470056 & 0.449628 & 0.431636 & 1.1 \\
0.670943 & 0.627274 & 0.590998 & 0.56028 & 0.533854 & 0.510815 & 0.490503 & 1.2 \\
0.752157 & 0.703865 & 0.663609 & 0.62944 & 0.599992 & 0.574284 & 0.551594 & 1.3 \\
0.835666 & 0.782767 & 0.738509 & 0.700849 & 0.668332 & 0.639904 & 0.614785 & 1.4 \\
0.921277 & 0.863812 & 0.815549 & 0.774371 & 0.738747 & 0.707558 & 0.679968 & 1.5 \\
1.008814 & 0.946848 & 0.89459 & 0.84988 & 0.811124 & 0.777139 & 0.747041 & 1.6 \\
1.09811 & 1.031731 & 0.975508 & 0.927264 & 0.885356 & 0.84855 & 0.815912 & 1.7 \\
1.18901 & 1.118328 & 1.058183 & 1.006414 & 0.961345 & 0.921698 & 0.886497 & 1.8 \\
1.281365 & 1.206514 & 1.142505 & 1.087232 & 1.039001 & 0.996501 & 0.958715 & 1.9 \\
1.375035 & 1.296168 & 1.22837 & 1.169623 & 1.118238 & 1.072878 & 1.032494 & 2 \\
1.44052 & 1.36308 & 1.294755 & 1.234756 & 1.181848 & 1.134887 & 1.092913 & 2.1 \\
1.484029 & 1.411977 & 1.34561 & 1.286088 & 1.23294 & 1.185375 & 1.142614 & 2.2 \\
1.51485 & 1.450211 & 1.387209 & 1.329179 & 1.276561 & 1.228998 & 1.185937 & 2.3 \\
1.536571 & 1.480521 & 1.421846 & 1.366045 & 1.314529 & 1.267422 & 1.224434 & 2.4 \\
1.551427 & 1.504547 & 1.450875 & 1.397862 & 1.347897 & 1.301612 & 1.258996 & 2.5 \\
1.561046 & 1.523441 & 1.475227 & 1.425432 & 1.37738 & 1.332217 & 1.290223 & 2.6 \\
1.566734 & 1.538082 & 1.4956 & 1.449349 & 1.403504 & 1.359712 & 1.318555 & 2.7 \\
1.569603 & 1.549182 & 1.512549 & 1.470079 & 1.426676 & 1.384468 & 1.34433 & 2.8 \\
1.570648 & 1.55734 & 1.526527 & 1.488001 & 1.447229 & 1.406781 & 1.367818 & 2.9 \\
1.570796 & 1.563082 & 1.537924 & 1.503433 & 1.465437 & 1.426895 & 1.389243 & 3 \\
1.566876 & 1.547077 & 1.516648 & 1.481533 & 1.445018 & 1.408793 &  & 3.1 \\
1.569152 & 1.554285 & 1.527884 & 1.495721 & 1.46133 & 1.42663 &  & 3.2 \\
1.570311 & 1.559819 & 1.537354 & 1.508179 & 1.475987 & 1.442895 &  & 3.3 \\
1.570736 & 1.563928 & 1.545249 & 1.519067 & 1.489126 & 1.457711 &  & 3.4 \\
1.570796 & 1.566843 & 1.551742 & 1.528528 & 1.500873 & 1.47119 &  & 3.5 \\
1.568781 & 1.556996 & 1.536692 & 1.51134 & 1.48343 &  &  & 3.6 \\
1.569949 & 1.561159 & 1.543679 & 1.520628 & 1.494522 &  &  & 3.7 \\
1.570546 & 1.564372 & 1.549602 & 1.528832 & 1.504547 &  &  & 3.8 \\
\end{tabular}
\end{table}

\begin{table}
\centering
\begin{tabular}{c c c c c c}
1.570765 & 1.566768 & 1.554564 & 1.536039 & 1.513583 & 3.9 \\
1.568473 & 1.558663 & 1.54233 & 1.5217 &  & 4 \\
1.56961 & 1.561992 & 1.54778 & 1.528961 &  & 4.1 \\
1.570297 & 1.564638 & 1.552462 & 1.535429 &  & 4.2 \\
1.570649 & 1.566685 & 1.556443 & 1.541162 &  & 4.3 \\
1.570778 & 1.568215 & 1.559787 & 1.546213 &  & 4.4 \\
1.569306 & 1.562557 & 1.550633 &  &  & 4.5 \\
1.570035 & 1.56481 & 1.554473 &  &  & 4.6 \\
1.570476 & 1.566605 & 1.557778 &  &  & 4.7 \\
1.570701 & 1.567996 & 1.560594 &  &  & 4.8 \\
1.570784 & 1.569037 & 1.562964 &  &  & 4.9 \\
1.56978 & 1.564929 &  &  &  & 5 \\
1.570277 & 1.56653 &  &  &  & 5.1 \\
1.570577 & 1.567807 &  &  &  & 5.2 \\
1.570732 & 1.568798 &  &  &  & 5.3 \\
1.570788 & 1.56954 &  &  &  & 5.4 \\
1.57007 &  &  &  &  & 5.5 \\
1.570425 &  &  &  &  & 5.6 \\
1.57064 &  &  &  &  & 5.7 \\
1.57075 &  &  &  &  & 5.8 \\
1.570791 &  &  &  &  & 5.9 \\
\end{tabular}
\end{table}

\textbf{附表2(椭圆长轴为2或3,短轴取1.1,1.2,1.3时过流面积s随t变化数据)}

\begin{table}
\centering
\begin{tabular}{c c c c c c c}
a=2,b=1.1 & a=2,b=1.2 & a=2,b=1.3 & a=3,b=1.1 & a=3,b=1.2 & a=3,b=1.3 & t \\
0 & 0 & 0 & 0 & 0 & 0 & 0 \\
0.018212 & 0.01919 & 0.020068 & 0.015923 & 0.016913 & 0.017825 & 0.1 \\
0.051248 & 0.053992 & 0.056452 & 0.044865 & 0.047651 & 0.050213 & 0.2 \\
0.093661 & 0.098661 & 0.103133 & 0.082105 & 0.087195 & 0.091868 & 0.3 \\
0.143445 & 0.151077 & 0.157887 & 0.125918 & 0.133709 & 0.140853 & 0.4 \\
0.199406 & 0.20998 & 0.219384 & 0.175285 & 0.186111 & 0.196018 & 0.5 \\
0.260718 & 0.27449 & 0.286697 & 0.229506 & 0.243652 & 0.256572 & 0.6 \\
0.326753 & 0.343941 & 0.359118 & 0.288055 & 0.305771 & 0.321917 & 0.7 \\
0.397014 & 0.4178 & 0.436075 & 0.350516 & 0.372023 & 0.391578 & 0.8 \\
0.471086 & 0.495621 & 0.517087 & 0.416547 & 0.44204 & 0.465159 & 0.9 \\
0.548616 & 0.57702 & 0.601731 & 0.485859 & 0.515512 & 0.542325 & 1 \\
0.629293 & 0.661656 & 0.689626 & 0.558203 & 0.592167 & 0.622779 & 1.1 \\
0.712844 & 0.749222 & 0.780419 & 0.633358 & 0.671765 & 0.706257 & 1.2 \\
0.799018 & 0.839431 & 0.873767 & 0.711127 & 0.754092 & 0.792516 & 1.3 \\
0.887586 & 0.932011 & 0.969325 & 0.791334 & 0.838948 & 0.881329 & 1.4 \\
0.978334 & 1.026691 & 1.066727 & 0.873816 & 0.926152 & 0.972477 & 1.5 \\
1.071059 & 1.123197 & 1.165558 & 0.958422 & 1.015529 & 1.065747 & 1.6 \\
\end{tabular}
\end{table}

\begin{table}
\centering
\begin{tabular}{c c c c c c c}
\hline
1.165563 & 1.221226 & 1.265298 & 1.045014 & 1.106915 & 1.16092 & 1.7 \\
1.26165 & 1.320422 & 1.365208 & 1.13346 & 1.200147 & 1.257762 & 1.8 \\
1.359118 & 1.420302 & 1.463993 & 1.223635 & 1.295062 & 1.356009 & 1.9 \\
1.457742 & 1.520052 & 1.558242 & 1.315421 & 1.39149 & 1.455328 & 2 \\
1.520811 & 1.568888 & 1.570788 & 1.384366 & 1.460759 & 1.521387 & 2.1 \\
1.555352 & 1.570765 & & 1.435032 & 1.50795 & 1.558763 & 2.2 \\
1.570691 & & & 1.474682 & 1.541173 & & 2.3 \\
& & & 1.505937 & 1.562706 & & 2.4 \\
& & & 1.530301 & 1.570734 & & 2.5 \\
& & & 1.548738 & & & 2.6 \\
& & & 1.56183 & & & 2.7 \\
& & & 1.569625 & & & 2.8 \\
& & & 1.570695 & & & 2.9 \\
\hline
\end{tabular}
\end{table}

\textbf{附表3(椭圆长轴为4或5,短轴取1.1,1.2,1.3时过流面积s随t变化数据)}

\begin{table}
\centering
\begin{tabular}{c c c c c c c}
\hline
a=4,b=1.1 & a=4,b=1.2 & a=4,b=1.3 & a=5,b=1.1 & a=5,b=1.2 & a=5,b=1.3 & t \\
0 & 0 & 0 & 0 & 0 & 0 & 0 \\
0.014325 & 0.015292 & 0.016197 & 0.014325 & 0.014062 & 0.016197 & 0.1 \\
0.040393 & 0.043118 & 0.045664 & 0.040393 & 0.039671 & 0.045664 & 0.2 \\
0.073978 & 0.078964 & 0.083617 & 0.073978 & 0.072691 & 0.083617 & 0.3 \\
0.113545 & 0.121189 & 0.128315 & 0.113545 & 0.111622 & 0.128315 & 0.4 \\
0.158191 & 0.168827 & 0.178732 & 0.158191 & 0.155586 & 0.178732 & 0.5 \\
0.207295 & 0.221216 & 0.234163 & 0.207295 & 0.203982 & 0.234163 & 0.6 \\
0.260398 & 0.277862 & 0.294082 & 0.260398 & 0.256361 & 0.294082 & 0.7 \\
0.317134 & 0.338372 & 0.35807 & 0.317134 & 0.312372 & 0.35807 & 0.8 \\
0.377205 & 0.402429 & 0.425785 & 0.377205 & 0.371728 & 0.425785 & 0.9 \\
0.440362 & 0.469762 & 0.496939 & 0.440362 & 0.434187 & 0.496939 & 1 \\
0.50639 & 0.54014 & 0.571278 & 0.50639 & 0.499542 & 0.571278 & 1.1 \\
0.575099 & 0.613357 & 0.648583 & 0.575099 & 0.567613 & 0.648583 & 1.2 \\
0.646322 & 0.689233 & 0.728653 & 0.646322 & 0.63824 & 0.728653 & 1.3 \\
0.71991 & 0.767604 & 0.811307 & 0.71991 & 0.711278 & 0.811307 & 1.4 \\
0.795725 & 0.848319 & 0.896378 & 0.795725 & 0.786597 & 0.896378 & 1.5 \\
0.873644 & 0.931242 & 0.983706 & 0.873644 & 0.864079 & 0.983706 & 1.6 \\
0.95355 & 1.016243 & 1.073143 & 0.95355 & 0.943613 & 1.073143 & 1.7 \\
1.035338 & 1.103201 & 1.16454 & 1.035338 & 1.025097 & 1.16454 & 1.8 \\
1.118907 & 1.192001 & 1.257752 & 1.118907 & 1.108436 & 1.257752 & 1.9 \\
1.204162 & 1.282533 & 1.352631 & 1.204162 & 1.193539 & 1.352631 & 2 \\
1.271478 & 1.352425 & 1.423594 & 1.271478 & 1.261424 & 1.423594 & 2.1 \\
1.324394 & 1.405665 & 1.475026 & 1.324394 & 1.315436 & 1.475026 & 2.2 \\
1.368664 & 1.448758 & 1.514134 & 1.368664 & 1.361084 & 1.514134 & 2.3 \\
\hline
\end{tabular}
\end{table}

\begin{table}
\centering
\begin{tabular}{c c c c c c c}
1.406355 & 1.483998 & 1.543177 & 1.406355 & 1.400324 & 1.543177 & 2.4 \\
1.438664 & 1.512661 & 1.562861 & 1.438664 & 1.434271 & 1.562861 & 2.5 \\
1.466395 & 1.535531 & & 1.466395 & 1.463664 & & 2.6 \\
1.490134 & 1.553053 & & 1.490134 & 1.489029 & & 2.7 \\
1.510329 & 1.565275 & & 1.510329 & 1.510749 & & 2.8 \\
1.527332 & & & 1.527332 & & & 2.9 \\
1.541416 & & & 1.541416 & & & 3 \\
1.552786 & & & 1.552786 & & & 3.1 \\
1.56157 & & & 1.56157 & & & 3.2 \\
1.567761 & & & 1.567761 & & & 3.3 \\
\end{tabular}
\end{table}

\textbf{附表4(截取首尾数据后的过流面积s随t变化数据)}

\begin{table}
\centering
\begin{tabular}{c c c c c c c c}
a=1.5,b=1 & a=3,b=1.1 & a=4,b=1.3 & t & a=1.5,b=1 & a=3,b=1.1A=4,b=1.3t & & \\
0.057791 & 0.082105 & 0.105197 & 0 & 0.728374 & 0.695373 & 0.753186 & 0.98 \\
0.061928 & 0.086211 & 0.109702 & 0.01 & 0.737022 & 0.703238 & 0.761414 & 0.99 \\
0.066156 & 0.090381 & 0.114267 & 0.02 & 0.745694 & 0.711127 & 0.769668 & 1 \\
0.070471 & 0.094614 & 0.118891 & 0.03 & 0.75439 & 0.719041 & 0.777946 & 1.01 \\
0.074873 & 0.098908 & 0.123574 & 0.04 & 0.76311 & 0.726979 & 0.78625 & 1.02 \\
0.079358 & 0.103263 & 0.128315 & 0.05 & 0.771852 & 0.734941 & 0.794578 & 1.03 \\
0.083926 & 0.107678 & 0.133113 & 0.06 & 0.780618 & 0.742927 & 0.80293 & 1.04 \\
0.088574 & 0.112152 & 0.137967 & 0.07 & 0.789407 & 0.750936 & 0.811307 & 1.05 \\
0.0933 & 0.116684 & 0.142876 & 0.08 & 0.798218 & 0.75897 & 0.819708 & 1.06 \\
0.098104 & 0.121273 & 0.14784 & 0.09 & 0.807051 & 0.767026 & 0.828133 & 1.07 \\
0.102982 & 0.125918 & 0.152858 & 0.1 & 0.815906 & 0.775106 & 0.836582 & 1.08 \\
0.107935 & 0.130618 & 0.157929 & 0.11 & 0.824784 & 0.783208 & 0.845055 & 1.09 \\
0.112959 & 0.135372 & 0.163053 & 0.12 & 0.833682 & 0.791334 & 0.853551 & 1.1 \\
0.118055 & 0.140181 & 0.168228 & 0.13 & 0.842602 & 0.799482 & 0.86207 & 1.11 \\
0.123221 & 0.145042 & 0.173455 & 0.14 & 0.851543 & 0.807653 & 0.870612 & 1.12 \\
0.128455 & 0.149956 & 0.178732 & 0.15 & 0.860504 & 0.815847 & 0.879178 & 1.13 \\
0.133756 & 0.154921 & 0.184059 & 0.16 & 0.869486 & 0.824062 & 0.887767 & 1.14 \\
0.139123 & 0.159937 & 0.189436 & 0.17 & 0.878488 & 0.8323 & 0.896378 & 1.15 \\
0.144556 & 0.165004 & 0.194862 & 0.18 & 0.887511 & 0.84056 & 0.905012 & 1.16 \\
0.150052 & 0.17012 & 0.200336 & 0.19 & 0.896553 & 0.848841 & 0.913668 & 1.17 \\
0.155611 & 0.175285 & 0.205857 & 0.2 & 0.905614 & 0.857145 & 0.922346 & 1.18 \\
0.161232 & 0.180498 & 0.211426 & 0.21 & 0.914695 & 0.865469 & 0.931047 & 1.19 \\
0.166913 & 0.18576 & 0.217042 & 0.22 & 0.923796 & 0.873816 & 0.939769 & 1.2 \\
0.172655 & 0.191068 & 0.222703 & 0.23 & 0.932914 & 0.882183 & 0.948514 & 1.21 \\
0.178455 & 0.196423 & 0.228411 & 0.24 & 0.942052 & 0.890572 & 0.95728 & 1.22 \\
0.184314 & 0.201825 & 0.234163 & 0.25 & 0.951208 & 0.898981 & 0.966067 & 1.23 \\
0.190229 & 0.207272 & 0.239961 & 0.26 & 0.960382 & 0.907411 & 0.974876 & 1.24 \\
0.196201 & 0.212764 & 0.245803 & 0.27 & 0.969574 & 0.915862 & 0.983706 & 1.25 \\
\end{tabular}
\end{table}

\begin{tabular}{r r r r r r r}
0.202229 & 0.2183 & 0.251688 & 0.28 & 0.978784 & 0.924334 & 0.992558 & 1.26 \\
0.208311 & 0.223881 & 0.257618 & 0.29 & 0.988012 & 0.932826 & 1.00143 & 1.27 \\
0.214448 & 0.229506 & 0.26359 & 0.3 & 0.997257 & 0.941338 & 1.010323 & 1.28 \\
0.220637 & 0.235173 & 0.269605 & 0.31 & 1.006518 & 0.94987 & 1.019236 & 1.29 \\
0.22688 & 0.240884 & 0.275661 & 0.32 & 1.015797 & 0.958422 & 1.02817 & 1.3 \\
0.233174 & 0.246637 & 0.28176 & 0.33 & 1.025092 & 0.966994 & 1.037125 & 1.31 \\
0.239519 & 0.252431 & 0.2879 & 0.34 & 1.034404 & 0.975586 & 1.046099 & 1.32 \\
0.245914 & 0.258267 & 0.294082 & 0.35 & 1.043732 & 0.984197 & 1.055094 & 1.33 \\
0.252359 & 0.264144 & 0.300303 & 0.36 & 1.053076 & 0.992828 & 1.064108 & 1.34 \\
0.258854 & 0.270062 & 0.306565 & 0.37 & 1.062435 & 1.001478 & 1.073143 & 1.35 \\
0.265396 & 0.27602 & 0.312867 & 0.38 & 1.071811 & 1.010148 & 1.082196 & 1.36 \\
0.271987 & 0.282018 & 0.319209 & 0.39 & 1.081201 & 1.018836 & 1.09127 & 1.37 \\
0.278625 & 0.288055 & 0.32559 & 0.4 & 1.090607 & 1.027543 & 1.100362 & 1.38 \\
0.28531 & 0.294131 & 0.332009 & 0.41 & 1.100028 & 1.036269 & 1.109474 & 1.39 \\
0.29204 & 0.300246 & 0.338467 & 0.42 & 1.109463 & 1.045014 & 1.118605 & 1.4 \\
0.298816 & 0.306399 & 0.344964 & 0.43 & 1.118913 & 1.053777 & 1.127755 & 1.41 \\
0.305637 & 0.31259 & 0.351498 & 0.44 & 1.128377 & 1.062559 & 1.136923 & 1.42 \\
0.312503 & 0.318819 & 0.35807 & 0.45 & 1.137856 & 1.071359 & 1.14611 & 1.43 \\
0.319412 & 0.325085 & 0.364679 & 0.46 & 1.147348 & 1.080177 & 1.155316 & 1.44 \\
0.326365 & 0.331388 & 0.371325 & 0.47 & 1.156854 & 1.089013 & 1.16454 & 1.45 \\
0.33336 & 0.337728 & 0.378007 & 0.48 & 1.166373 & 1.097867 & 1.173782 & 1.46 \\
0.340398 & 0.344104 & 0.384726 & 0.49 & 1.175906 & 1.106739 & 1.183042 & 1.47 \\
0.347478 & 0.350516 & 0.391481 & 0.5 & 1.185452 & 1.115628 & 1.192319 & 1.48 \\
0.354599 & 0.356963 & 0.398271 & 0.51 & 1.195011 & 1.124535 & 1.201615 & 1.49 \\
0.36176 & 0.363446 & 0.405097 & 0.52 & 1.204583 & 1.13346 & 1.210928 & 1.5 \\
0.368963 & 0.369964 & 0.411959 & 0.53 & 1.214167 & 1.142402 & 1.220259 & 1.51 \\
0.376205 & 0.376517 & 0.418855 & 0.54 & 1.223763 & 1.151361 & 1.229607 & 1.52 \\
0.383486 & 0.383104 & 0.425785 & 0.55 & 1.233371 & 1.160336 & 1.238971 & 1.53 \\
0.390807 & 0.389725 & 0.43275 & 0.56 & 1.242992 & 1.169329 & 1.248353 & 1.54 \\
0.398166 & 0.396381 & 0.439749 & 0.57 & 1.252624 & 1.178339 & 1.257752 & 1.55 \\
0.405563 & 0.40307 & 0.446782 & 0.58 & 1.262267 & 1.187365 & 1.267167 & 1.56 \\
0.412998 & 0.409792 & 0.453849 & 0.59 & 1.271922 & 1.196408 & 1.276599 & 1.57 \\
0.42047 & 0.416547 & 0.460949 & 0.6 & 1.281588 & 1.205468 & 1.286048 & 1.58 \\
0.427979 & 0.423335 & 0.468082 & 0.61 & 1.291266 & 1.214543 & 1.295512 & 1.59 \\
0.435524 & 0.430155 & 0.475247 & 0.62 & 1.300953 & 1.223635 & 1.304993 & 1.6 \\
0.443106 & 0.437008 & 0.482445 & 0.63 & 1.310652 & 1.232743 & 1.314489 & 1.61 \\
0.450723 & 0.443893 & 0.489676 & 0.64 & 1.32036 & 1.241867 & 1.324001 & 1.62 \\
0.458375 & 0.450809 & 0.496939 & 0.65 & 1.330079 & 1.251007 & 1.333529 & 1.63 \\
0.466062 & 0.457757 & 0.504233 & 0.66 & 1.339808 & 1.260163 & 1.343072 & 1.64 \\
0.473784 & 0.464737 & 0.511559 & 0.67 & 1.349547 & 1.269334 & 1.352631 & 1.65 \\
0.48154 & 0.471747 & 0.518916 & 0.68 & 1.359295 & 1.278521 & 1.361398 & 1.66 \\
0.489329 & 0.478788 & 0.526305 & 0.69 & 1.369053 & 1.287723 & 1.369513 & 1.67 \\
\end{tabular}

\begin{tabular}{r r r r r r r}
0.497152 & 0.485859 & 0.533725 & 0.7 & 1.37882 & 1.296941 & 1.37721 & 1.68 \\
0.505008 & 0.492961 & 0.541175 & 0.71 & 1.388596 & 1.306173 & 1.38457 & 1.69 \\
0.512896 & 0.500093 & 0.548655 & 0.72 & 1.398381 & 1.315421 & 1.391644 & 1.7 \\
0.520817 & 0.507255 & 0.556166 & 0.73 & 1.408175 & 1.323909 & 1.398463 & 1.71 \\
0.52877 & 0.514446 & 0.563707 & 0.74 & 1.417977 & 1.331772 & 1.40505 & 1.72 \\
0.536754 & 0.521667 & 0.571278 & 0.75 & 1.427788 & 1.339235 & 1.411425 & 1.73 \\
0.54477 & 0.528916 & 0.578879 & 0.76 & 1.437606 & 1.346377 & 1.417603 & 1.74 \\
0.552816 & 0.536195 & 0.586509 & 0.77 & 1.447433 & 1.353247 & 1.423594 & 1.75 \\
0.560893 & 0.543503 & 0.594168 & 0.78 & 1.457267 & 1.359874 & 1.429411 & 1.76 \\
0.569001 & 0.550839 & 0.601856 & 0.79 & 1.467109 & 1.366284 & 1.435061 & 1.77 \\
0.577138 & 0.558203 & 0.609573 & 0.8 & 1.476959 & 1.372493 & 1.440552 & 1.78 \\
0.585305 & 0.565595 & 0.617318 & 0.81 &  & 1.378516 & 1.445891 & 1.79 \\
0.593501 & 0.573016 & 0.625092 & 0.82 &  &  & 1.451084 & 1.8 \\
0.601726 & 0.580464 & 0.632894 & 0.83 &  &  & 1.456136 & 1.81 \\
0.60998 & 0.587939 & 0.640725 & 0.84 &  &  & 1.461052 & 1.82 \\
0.618262 & 0.595442 & 0.648583 & 0.85 &  &  & 1.465837 & 1.83 \\
0.626572 & 0.602971 & 0.656468 & 0.86 &  &  & 1.470494 & 1.84 \\
0.63491 & 0.610528 & 0.664382 & 0.87 &  &  & 1.475026 & 1.85 \\
0.643275 & 0.618112 & 0.672322 & 0.88 &  &  & 1.479438 & 1.86 \\
0.651668 & 0.625721 & 0.680289 & 0.89 &  &  & 1.483733 & 1.87 \\
0.660087 & 0.633358 & 0.688284 & 0.9 &  &  & 1.487912 & 1.88 \\
0.668533 & 0.64102 & 0.696305 & 0.91 &  &  & 1.491978 & 1.89 \\
0.677005 & 0.648708 & 0.704353 & 0.92 &  &  & 1.495934 & 1.9 \\
0.685503 & 0.656422 & 0.712427 & 0.93 &  &  & 1.499783 & 1.91 \\
0.694027 & 0.664162 & 0.720527 & 0.94 &  &  & 1.503525 & 1.92 \\
0.702576 & 0.671927 & 0.728653 & 0.95 &  &  & 1.507163 & 1.93 \\
0.711151 & 0.679717 & 0.736805 & 0.96 &  &  & 1.510699 & 1.94 \\
0.71975 & 0.687533 & 0.744983 & 0.97 &  &  &  &
\end{tabular}

注:后四列数据时前四列数据的续

\end{document}