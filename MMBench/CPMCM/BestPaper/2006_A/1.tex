\section*{Ad Hoc 网络中的区域划分和资源分配问题}

\section{问题假设}

本文根据 Ad Hoc 网络中的区域划分和资源分配问题的一些要求,为了达到简化模型的目的,除问题中已给出的假设外,仅以可不作出以下假设

(1) 本文考虑在指定区域 $1000 \times 1000$ 的范围内,假设均放置等大小的圆,并且使其在区域范围内对称(仅对无湖的情况考虑对称)。

(2) 假设两圆相切不相交时不属于相邻的情况

(3) 假设在一个节点有多个信道的情况下,多信道可以同时工作。

(4) 假设转发也可以看作是一次接收和一次发射。

注:部分问题涉及到的假设,在相应问题中给出。

\section{参数说明}

\begin{itemize}
    \item $R$: 一跳覆盖区圆的半径;
    \item $\theta$: 相邻两圆公共面积所确定弦长对应的圆心角;
    \item $W$: 单个节点的发射功率;
    \item $h$: 单个节点的最大传输距离;
    \item $n$: 正方形区域内的节点总数;
    \item $M$: 整个网络通信中,数据包丢失的几率
\end{itemize}

\section{问题分析}

\subsection{对于问题一的分析}

(1) 对覆盖整个网络所用最少圆的问题分析,要考虑一个实际问题,即对称性。通过分析,必须保证网络的覆盖对称,这样才能保证所用的圆的数量最少。还要满足任意两个相交的圆的面积要大于 $5\%$,通过后面的定量计算,使得任意一个圆都和周围的六个圆相交,才能符合。

(2) 对信道选择的理解为,任意两个相邻的圆均要拥有不同的频率,并且要选择最少的频率来填充。此问题类似于用最少的颜色来填充地图问题,属于离散数学方面的问题。

(3) 对于网络抗毁性的理解

随机抽掉一定比例的节点后,只考虑剩余的节点是否任意两个都可以通信,即为连通即可。故当任意一个节点周围均匀分布的所有节点都被抽掉时,网络会有一个节点无法与外界通信,这也就是网络抗毁性研究的临界情况。

\subsection{对问题二的分析}

由于半径是可变化的,可先考虑无湖情况下,采用什么样的一跳覆盖区可以使得全部圆半径之和最小。在此基础上,考虑有湖情况下,改进现有方案。

\subsection{对问题三的分析}

采用基于节点的划分方式,可以将实际问题抽象为,在 $1000 \times 1000$ 的正方形区域内,用半径为 $75-100$ 的圆将节点全部覆盖(允许在没有节点的区域内不进行覆盖)。这样的模型要满足这样的条件,任意相交的两个圆的重叠部分的面积要大于其中的大圆面积的 $5\%$,并且使得这样的所有大小不等的圆的半径之和最小。

\subsection{对问题四的分析}

基于问题三的模型,利用节点的划分方式以及概率的计算方法,进行定量的计算并比较。考虑节点移动的时候,其所覆盖的区域的半径也在变化(参考问题三的模型与算法),这样当其走出最大可能半径范围而又未进入其他覆盖区,则此时其发生通信中断。计算这个最大半径的值与问题三的模型进行比较,得出结论。

果。

\section*{3.5 对问题五的分析}

本问题所要解决的目标是节能,并使第一个退出网络的节点的时间尽量长。则根据题设,半径越小,则越能节省能源,使得网络的周期延长。基于此思想,可以列出能量相等的公式,得出所要求的模型中圆的半径与题设中给的半径 100 的比值关系,这样就能确定最优答案。

电池的总能量是可看作固定常量,题中假设覆盖半径为 100 发送状态下的工作总时间是 400 个时间单位,可看作半径 100 时正常工作 $400t$ 电池能量耗尽。现要求工作 $1200t$,则可以求出满足条件的临界覆盖半径。

\section*{3.6 对问题六的分析}

对问题五中的网络通信质量进行定量的分析,就是讨论数据包在申请接收的过程中,丢失的几率。可以将网络划分为两种状态,即“忙”(接收和发射中)与“闲”(等待中)。这样,让数据包去询问节点,是否可以接收,若“忙”则等待十个时间单位,若“闲”则进行接收,数据包最多可以进行 4 次访问,此时若还是没有被接收,则数据包丢失。然后进行概率计算,从而求得,在整个网络通信中,数据包丢失的几率。

\section*{4 模型建立}

\subsection*{4.1 问题一}

所有一跳覆盖区均为半径 100 的圆,按相邻两圆的公共面积分以下两种情况分析。

(1)相邻两个圆的公共面积不小于 5\%

1)所需最少圆个数的模型建立

由已知条件则可得到如下方程:

\[
\frac{2\left(S_{\text {扇 }}-S_{\text {三角形}}\right)}{S_{\text {圆 }}}=5\%
\]

即

\[
2\left(\frac{\pi R^{2} \theta}{360}-R^{2} \cos \frac{\theta}{2} \sin \frac{\theta}{2}\right)=5\%
\]

其中 $\theta$ 为公共面积所确定的弦长对应的圆心角,示意图见图 1。

化简得到:

\[
\frac{\theta}{180}-\frac{\sin \theta}{\pi}=5\%
\]

使用牛顿迭代法计算:

令

\[
F(\theta)=\frac{\theta}{180}-\frac{\sin \theta}{\pi}-0.05
\]

\[
\theta_{k+1}=\theta_{k}-\frac{F(\theta)}{F^{\prime}(\theta)}=\theta_{k}-\frac{\pi \theta_{k}-180 \sin \theta_{k}-0.05 * 180 \pi}{\pi-180 \cos \theta_{k}}
\]

令 $\theta_{0}=57.1^{\circ}$ 可求得 $\theta_{1}$,代入编程公式求得较为精确的 $\theta_{k}$ 值。经过 15 次迭代,使 $\theta_{k}$ 趋近 $\theta$ 的真值:

\begin{table}[h]
\centering
\caption{牛顿迭代法计算机模拟(一维)}
\begin{tabular}{ccc}
\hline $\theta_{k}$ & $\theta_{k}$ 加减项 & $\theta_{k+1}$ \\
\hline 57.1 & -0.378344475 & 57.47834447 \\
57.47834447 & -0.076949773 & 57.55529425 \\
\hline
\end{tabular}
\end{table}

\begin{tabular}{ccc}
57.55529425 & -0.004672622 & 57.55996687 \\
57.55996687 & -1.79893E-05 & 57.55998486 \\
57.55998486 & -2.67173E-10 & 57.55998486 \\
57.55998486 & -1.5381E-15 & 57.55998486 \\
57.55998486 & -1.5381E-15 & 57.55998486 \\
57.55998486 & -1.5381E-15 & 57.55998486 \\
57.55998486 & -1.5381E-15 & 57.55998486 \\
57.55998486 & -1.5381E-15 & 57.55998486 \\
57.55998486 & -1.5381E-15 & 57.55998486 \\
57.55998486 & -1.5381E-15 & 57.55998486 \\
57.55998486 & -1.5381E-15 & 57.55998486 \\
57.55998486 & -1.5381E-15 & 57.55998486 \\
\end{tabular}

取 $\theta = 57.56^\circ$,则以其中任意一个圆为中心时满足条件的与其相邻的圆的个数 $n = 360 / 57.56 = 6.254$。故 $n = 6$ 即为足以满足 $5\%$ 条件并且浪费一跳区面积最少的最优解。

若将此相邻的 6 圆的圆心做连线,则可得到的图形是一个规则的六边形。要覆盖整个 $1000 \times 1000$ 的正方形区域,可看作是用若干规则的六边形依次相接直至覆盖整个区域。

下面计算覆盖整个区域所需圆的个数:

\textbf{纵向考虑:}

六边形两横边距离的一半即为圆形一跳覆盖区的一层排列距离,其值为相邻两圆的圆心距乘以 $\cos 30^\circ$,即

\[
d = 2R \cos \frac{\theta}{2} \cos 30^\circ = 2 \times 100 \times \cos 30^\circ \times \cos 30^\circ = 150
\]

此外在边缘区还可利用的纵向长度为 $a = R \sin \frac{\theta}{2} = 100 \times \sin 30^\circ = 50$。纵向需覆盖的长度 $1000 = 6 \times 150 + 50 + 50$,即 6 层圆加上两个边缘相交圆可利用的长度刚好可以覆盖整个区域,所以满足条件时的纵向所需的圆为 7 排。

\textbf{横向考虑:}

任取左边缘一个圆形一跳区,它与其上排圆的左交点设为 $B$,与其下排圆的左交点设为 $C$,则 $B$ 与 $C$ 的连线即为可利用的面积的左起始点,如图 2 所示,经计算可证明此直线刚好经过此边缘圆的圆心 $D$。简单证明如下:

由 $\theta = 60^\circ$,则 $AB = AC = BC = d + a = 150 + 50 = 200$。$ABC$ 为等边三角形,故过 $A$ 点向 $BC$ 做垂线交 $BC$ 于 $D'$,由对称特性,圆心也 $D$ 必在此垂线上。

$AD$ 为两圆的圆心距,则 $AD = 2R \times \cos 30^\circ = 2 \times 100 \times \sqrt{3} / 2 = 100\sqrt{3}$。而

\[
AD' = AB \cdot \cos \frac{\theta}{2} = 200 \times \cos 30^\circ = 100\sqrt{3}。
\]

$AD = AD'$ 且 $D$ 与 $D'$ 都在 $\theta$ 角的平分线上,故 $D$ 与 $D'$ 重合。此问题即可简化为直接采用圆心距来计算求解。

\begin{figure}[h]
    \centering
    \includegraphics[width=0.45\textwidth]{image1.png}
    \caption{相邻两圆相交的情况分析}
    \label{fig:1}
\end{figure}
\begin{figure}[h]
    \centering
    \includegraphics[width=0.45\textwidth]{image2.png}
    \caption{六边形表示法的横向边界情况}
    \label{fig:2}
\end{figure}

圆心距 $b = 2R \cos \frac{\theta}{2} = 2 * 100 * \cos 30^\circ = 100\sqrt{3}$

横向所需的圆的层数为 $\frac{1000}{100\sqrt{3}} \approx 5.77$,故为能覆盖所有区域横向也为 6 层。

因此,可得到用六边形方式划分结果如图 \ref{fig:3} 所示,每个交点代表一个圆的圆心,用圆的形式画出如图 \ref{fig:4} 所示。

\begin{figure}[h]
    \centering
    \includegraphics[width=0.45\textwidth]{image3.png}
    \caption{以六边形表示法覆盖所有区域}
    \label{fig:3}
\end{figure}
\begin{figure}[h]
    \centering
    \includegraphics[width=0.45\textwidth]{image4.png}
    \caption{满足条件的区域划分}
    \label{fig:4}
\end{figure}

当一个圆只有部分在正方形区域内,也按一个计算,故满足条件所需的圆最少为 $7 \times 3 + 6 \times 4 = 45$(个)。

\section*{2) 所需圆的信道分配}

由于处于方形一定范围内的圆同时与六个圆相邻,加上本身,最少需要三种信道,如图 \ref{fig:5} 所示。

\begin{figure}[h]
    \centering
    \includegraphics[width=0.45\textwidth]{image5.png}
    \caption{}
    \label{fig:5}
\end{figure}

\begin{figure}[h]
    \centering
    \includegraphics[width=0.6\textwidth]{image1.png}
    \caption{所需圆的信道分配}
    \label{fig:5}
\end{figure}

用改进的 Welch Powell 法对上图进行信道分配:
\begin{enumerate}
    \item 将第一个信道分配给第一个圆,并以离第一个圆最近的不相邻的圆的圆心,到第一个圆的圆心的距离为半径画圆,对圆心位于所画圆周上的圆分配同样的信道。
    \item 用第二个信道对尚未分配信道的圆重复 A),用第三个、第四个……信道继续这种做法,直到所有的点全部分配信道为止。
\end{enumerate}

从上面的结果可以看到,使用三个信道即能满足要求。如图 6:

\begin{figure}[h]
    \centering
    \includegraphics[width=0.8\textwidth]{image2.png}
    \caption{满足条件的圆的信道分配}
    \label{fig:6}
\end{figure}

\section*{3) 网络的抗毁性}
\begin{enumerate}
    \item 对此问题的注释与说明
\end{enumerate}
首先,在此网络中任意一个圆内都相同的分布着 7 个节点,以每一个节点为中心,周围相邻最近的等距离的六个点,全部抽掉的时候,则圆心附近区域没有被覆盖,即网络瘫痪。

其次,边界的情况应特殊考虑,有些边界点周围相邻最近的等距离的点不是 6 个。如下图,存在周围有 3 个,4 个以及 5 个的情况。这样,在作此问题的时

候应全面考虑,方才周到。

再次,题目中,抽取的点数为总数的 $2\%$, $5\%$, $10\%$, $15\%$。通过计算,这样的比例抽取,不是整数,那么我们假定取小于此数的最大整数来计算。由于,问题的目的是要比较抽取不同比例后的抗毁性的问题的比较,基于此做出假设,并不影响问题的解决。

(2) 具体的做法如下:

公式: $\sum_{I=3}^{6} \frac{C_{N-I}^{M_{J}-I}}{C_{N}^{M_{J}}} \times K_{I} \quad (J=1, 2, 3, 4)$

说明: 第一, $J$ 取 $1, 2, 3, 4$ 分别代表抽取节点为 $2\%$, $5\%$, $10\%$, $15\%$ 的情况

第二, $I \leq M_{J} \quad N=152$

\[
M_{1}=N \times 2\% \approx 3 \quad M_{2}=N \times 5\% \approx 7
\]

\[
M_{3}=N \times 10\% \approx 15 \quad M_{4}=N \times 15\% \approx 22
\]

( $I$ 是某点瘫痪时周边抽掉的点数, $M_{J}$ 是随机抽取的点数)

第三, $K_{3}=13 \quad K_{4}=24 \quad K_{5}=3 \quad K_{6}=112$

( $K_{J}$ 分别代表抽取 $3, 4, 5, 6$ 个点后网络瘫痪, 不同类型的点的个数)

具体计算:

a. 当抽取的比例为 $2\%$ 时, 即 $M_{1}=3$

\[
\frac{C_{152-3}^{3-3}}{C_{152}^{3}} \times 13=2.2656 \times 10^{-5}=2.2656 \times 10^{-3} \%
\]

b. 当抽取的比例为 $5\%$ 时, 即 $M_{2}=7$

\[
\frac{C_{152-3}^{7-3}}{C_{152}^{7}} \times 13+\frac{C_{152-4}^{7-4}}{C_{152}^{7}} \times 24+\frac{C_{152-5}^{7-5}}{C_{152}^{7}} \times 3+\frac{C_{152-6}^{7-6}}{C_{152}^{7}}=0.004723=0.4723 \%
\]

c. 当抽取的比例为 $10\%$ 时, 即 $M_{3}=15$

\[
\frac{C_{152-3}^{15-3}}{C_{152}^{15}} \times 13+\frac{C_{152-4}^{15-4}}{C_{152}^{15}} \times 24+\frac{C_{152-5}^{15-5}}{C_{152}^{15}} \times 3+\frac{C_{152-6}^{15-6}}{C_{152}^{15}}=0.01188=1.188 \%
\]

d. 当抽取的比例为 $15\%$ 时, 即 $M_{4}=22$

\[
\frac{C_{152-3}^{22-3}}{C_{152}^{22}} \times 13+\frac{C_{152-4}^{22-4}}{C_{152}^{22}} \times 24+\frac{C_{152-5}^{22-5}}{C_{152}^{22}} \times 3+\frac{C_{152-6}^{22-6}}{C_{152}^{22}}=0.04378=4.378 \%
\]

\section*{Ad Hoc 网络中的区域划分和资源分配问题}

(3) 对比分析如下(见图 7):
1. 当取的点越少,抗毁性越好。
2. 如下图,抗毁性的分布与抽点的数量的关系为指数关系。

\begin{figure}[h]
\centering
\includegraphics[width=0.8\textwidth]{image1.png}
\caption{取点数量与网络瘫痪的关系}
\end{figure}

综上所述:由于抽点数量与网络瘫痪几率的比值近似可以看成指数分布,应尽量少抽取节点才能尽可能的保证网络不被摧毁

(2) 相邻两个圆的公共面积不小于 18\%

1) 所需最少圆的个数

假设两圆相切不相交时不属于相邻的情况。则采用如上方法计算得到 \( n=4 \),则考虑用四边形覆盖整个区域,横向纵向情况相同,只考虑一种即可。圆心距经计算为 \( 100\sqrt{2} \),边缘可利用长度刚好为圆心距的一半 \( 50\sqrt{2} \),而 \( \frac{1000}{50\sqrt{2}} \approx 14.14 \),故横向纵向需要的圆形一跳区均为 7 排,则所需的最少圆的个数为 \( 8 \times 8 = 64 \)(个)。设计结果如图 8 所示。

2) 上述覆盖区域方法所需圆的信道分配

分配方法同上,可得到所需信道数为 2,具体分配如图 9 所示。

\begin{figure}[h]
\centering
\includegraphics[width=0.45\textwidth]{image2.png}
\caption{满足条件的区域划分}
\end{figure}
\begin{figure}[h]
\centering
\includegraphics[width=0.45\textwidth]{image3.png}
\caption{所需圆的信道分配}
\end{figure}

\section*{3) 网络的抗毁性}

同(1)题的抗毁性求法,应用上述公式,可求得抽掉节点百分比不同各种情况的抗毁性结果(见图 10)。

\textbf{说明:}

(1) 在上面的图中不难得到如下结果

A. 抽掉周围 2 个节点则网络瘫痪,这样的节点共有 4 个

B. 抽掉周围 3 个节点则网络瘫痪,这样的节点共有 32 个

C. 抽掉周围 3 个节点则网络瘫痪,这样的节点共有 20 个

D. 抽掉周围 3 个节点则网络瘫痪,这样的节点共有 24 个

E. 抽掉周围 3 个节点则网络瘫痪,这样的节点共有 96 个

这样,共有节点 176 个。(值得注意的是,与 1 问不同的是,这里有 56 个点在 $1000 \times 1000$ 正方形外面,考虑到所有正方形内的连通,所以算抗毁性是,将在外面的节点考虑到公式里面)

(2) 当分别抽掉 2\%, 5\%, 10\%, 15\%节点时,得出的结果并非整数。同上一问假设相同,假定取小于此数的最大整数来计算,则分别为 3, 8, 17, 26。

\textbf{计算结果:}

A. 当抽掉 2\%时,网络瘫痪的几率为 0.0815\%

B. 当抽掉 5\%时,网络瘫痪的几率为 0.932\%

C. 当抽掉 10\%时,网络瘫痪的几率为 6.11\%

D. 当抽掉 15\%时,网络瘫痪的几率为 18.7\%

\begin{figure}[h]
\centering
\includegraphics[width=0.8\textwidth]{image.png}
\caption{取点数量与网络瘫痪的关系}
\end{figure}

\subsection*{4.2 问题二}

(1) 若先不考虑湖泊问题,一跳覆盖区圆形的半径在 75~100 间任意选择。则任意两个相邻的圆可分为以下几种情况:两圆半径相同且为最大值(R1=R2=100);两圆半径相同且为最小值(R1=R2=75);两圆半径不同取值为 75~100。按照问题一的排列方式(以正六边形为单位),每个圆与六个圆相邻,弧被等分为六份,每份对应一个公共弦,一个公共面积和一个 $60^\circ$ 的圆心角。其中设公共弦两端

点与圆心组成的三角形面积为单位有效面积(整个面积近似由单位有效面积组成,如下图 11)。

\begin{figure}[h]
    \centering
    \includegraphics[width=0.8\textwidth]{image1.png}
    \caption{问题 2 示意图}
    \label{fig:11}
\end{figure}

当满足所求条件“全部圆半径之和为最小”时,两圆相交公共面积为 5%,两个单位有效面积与两半径和的比值为最大,即用单位面积填满正方形时所需的半径和最小。设相交两圆的两个单位面积之和为 \( S \)(当 \( R \) 不相等时两单位面积不等),左圆单位有效面积为 \( S_1 \),右圆单位有效面积为 \( S_2 \),公共弦长之半为 \( a \),单位有效面积与两半径和的比值为 \( k \),\( R_1 \) 对应圆心角 \( \theta \),\( R_2 \) 对应圆心角 \( \alpha \)。则

\[
k = \frac{S}{R_1 + R_2} = \frac{S_1 + S_2}{R_1 + R_2}
\]

假设变化过程为右圆半径由 100 减小到 75(过程 1),而后左圆半径同样由 100 减小到 75(过程 2),试推导 \( k \) 与半径减小的关系,得出 \( k \) 的最大值。由上式得出在过程一中 \( R_2 \) 从 100 减小到 75,假设弦长 \( a \) 与 \( R_2 \) 同级减小,则 \( S_1 \) 与 \( R_2 \) 同级减小,\( S_2 \) 与 \( R_2^2 \) 同级减小,\( R_1 \) 不变,则 \( k \) 随 \( R_2 \) 的减小而减小。同理,\( k \) 随 \( R_1 \) 的减小而减小。通过以下三种情况讨论加以验证:

\begin{figure}[h]
    \centering
    \includegraphics[width=0.8\textwidth]{image2.png}
    \caption{单位有效面积与两半径和的比值证明示意图}
    \label{fig:12}
\end{figure}

\begin{center}
Ad Hoc 网络中的区域划分和资源分配问题
\end{center}

\begin{enumerate}
    \item 当 R1=R2=100 时,$\theta=\alpha=57.56^\circ$,此时
\end{enumerate}

\begin{figure}[h]
    \centering
    \includegraphics[width=0.8\textwidth]{image1.png}
    \caption{R1=R2=100 图示}
    \label{fig:13}
\end{figure}

\begin{equation}
S = \frac{\theta \pi R_1^2}{360} + \frac{\alpha \pi R_2^2}{360} - 5\% \pi R_1^2 = 2697378 \pi = 8475.319
\end{equation}

\begin{equation}
k = \frac{S}{R_1 + R_2} = \frac{S}{2R_1} = 42.3766
\end{equation}

\begin{enumerate}
    \setcounter{enumi}{1}
    \item 当 R1=100,R2=75 时,此时 $\theta < 60^\circ$,$\alpha > 60^\circ$,得方程组:
\end{enumerate}

\begin{figure}[h]
    \centering
    \includegraphics[width=0.8\textwidth]{image2.png}
    \caption{R1=100,R2=75 图示}
    \label{fig:14}
\end{figure}

\begin{equation}
\begin{cases}
R_1 \sin \frac{\theta}{2} = R_2 \sin \frac{\alpha}{2} \\
5\% \pi R_1^2 = \left( \frac{\theta \pi R_1^2}{360} - \frac{1}{2} * 2R_1 \sin \frac{\theta}{2} R_1 \cos \frac{\theta}{2} \right) + \left( \frac{\alpha \pi R_2^2}{360} - \frac{1}{2} * 2R_2 \sin \frac{\alpha}{2} R_2 \cos \frac{\alpha}{2} \right)
\end{cases}
\end{equation}

令

\begin{equation}
F_1(\theta, \alpha) = R_1 \sin \frac{\theta}{2} - R_2 \sin \frac{\alpha}{2}
\end{equation}

\begin{equation}
F_2(\theta, \alpha) = \left( \frac{\theta \pi R_1^2}{360} - R_1 \sin \frac{\theta}{2} R_1 \cos \frac{\theta}{2} \right) + \left( \frac{\alpha \pi R_2^2}{360} - R_2 \sin \frac{\alpha}{2} R_2 \cos \frac{\alpha}{2} \right) - 0.05 \pi R_1^2
\end{equation}

得非线性方程组
\[
\begin{cases}
F_1(\theta, \alpha) = 0 \\
F_2(\theta, \alpha) = 0
\end{cases}
\]

求解上述方程组的牛顿迭代格式为
\[
\begin{pmatrix}
\theta_{k+1} \\
\alpha_{k+1}
\end{pmatrix}
=
\begin{pmatrix}
\theta_k \\
\alpha_k
\end{pmatrix}
-
\begin{pmatrix}
\frac{\partial F_1}{\partial \theta} & \frac{\partial F_1}{\partial \alpha} \\
\frac{\partial F_2}{\partial \theta} & \frac{\partial F_2}{\partial \alpha}
\end{pmatrix}^{-1}
\begin{pmatrix}
F_1(\theta_k, \alpha_k) \\
F_2(\theta_k, \alpha_k)
\end{pmatrix}
\Bigg|_{(\theta_k, \alpha_k)}
\]

估计真值所在区间为 $\theta_k$ (50,53),$\alpha_k$ (62,71),取初值 $\theta_k = 51$,$\alpha_k = 66$,代入方程组通过十五次迭代编程计算。

\begin{table}[h]
\centering
\caption{牛顿迭代法计算机模拟(二维)}
\begin{tabular}{c c c c c c c c}
$\theta_k$ & $\alpha_k$ & $\theta_k/2$ & $\alpha_k/2$ & $\theta_k$ 加减项 & $\alpha_k$ 加减项 & $\theta_{k+1}$ & $\alpha_{k+1}$ \\
\hline
51 & 66 & 25.5 & 33 & -0.8369 & -0.0662177 & 51.8369 & 66.0662 \\
51.8369 & 66.0662 & 25.9185 & 33.0331 & -0.143823 & 0.51566736 & 51.980724 & 65.55056 \\
51.9807 & 65.5506 & 25.9904 & 32.7753 & 0.0723577 & 0.00860988 & 51.908366 & 65.54194 \\
51.9084 & 65.5419 & 25.9542 & 32.771 & -0.000229 & 0.00504569 & 51.908595 & 65.53689 \\
51.9086 & 65.5369 & 25.9543 & 32.7684 & 7.897E-06 & 4.4716E-06 & 51.908587 & 65.53689 \\
51.9086 & 65.5369 & 25.9543 & 32.7684 & 3.821E-12 & 6.3774E-11 & 51.908587 & 65.53689 \\
51.9086 & 65.5369 & 25.9543 & 32.7684 & -1.34E-15 & -4.233E-15 & 51.908587 & 65.53689 \\
51.9086 & 65.5369 & 25.9543 & 32.7684 & -1.34E-15 & -4.233E-15 & 51.908587 & 65.53689 \\
51.9086 & 65.5369 & 25.9543 & 32.7684 & -1.34E-15 & -4.233E-15 & 51.908587 & 65.53689 \\
51.9086 & 65.5369 & 25.9543 & 32.7684 & -1.34E-15 & -4.233E-15 & 51.908587 & 65.53689 \\
51.9086 & 65.5369 & 25.9543 & 32.7684 & -1.34E-15 & -4.233E-15 & 51.908587 & 65.53689 \\
51.9086 & 65.5369 & 25.9543 & 32.7684 & -1.34E-15 & -4.233E-15 & 51.908587 & 65.53689 \\
51.9086 & 65.5369 & 25.9543 & 32.7684 & -1.34E-15 & -4.233E-15 & 51.908587 & 65.53689 \\
51.9086 & 65.5369 & 25.9543 & 32.7684 & -1.34E-15 & -4.233E-15 & 51.908587 & 65.53689 \\
51.9086 & 65.5369 & 25.9543 & 32.7684 & -1.34E-15 & -4.233E-15 & 51.908587 & 65.53689 \\
\end{tabular}
\end{table}

由表 2 解得的趋近真值为 $\theta_k = 51.908587$,$\alpha_k = 65.53689$。

\[
S = \frac{1}{2} R_1^2 \sin \theta + \frac{1}{2} R_2^2 \sin \alpha = 6495.15393
\]

\[
k = \frac{S}{R_1 + R_2} = \frac{S}{2R_1} = 37.1152
\]

(3) 当 R1=R2=75 时,$\theta = \alpha = 57.56^\circ$,此时

\begin{figure}[h]
    \centering
    \includegraphics[width=0.8\textwidth]{image1.png}
    \caption{R1=R2=75 图示}
    \label{fig:15}
\end{figure}

\begin{equation}
S = \frac{\theta \pi R_{1}^{2}}{360} + \frac{\alpha \pi R_{2}^{2}}{360} - 5\% \pi R_{1}^{2} = 1517.5 \pi = 4767.367
\end{equation}

\begin{equation}
k = \frac{S}{R_{1} + R_{2}} = \frac{S}{2R_{1}} = 31.782
\end{equation}

由上面证明得出,在不考虑边缘的情况下,当满足所求条件 “全部圆半径之和为最小” 时,两圆相交公共面积为 5%,两个单位有效面积与两半径和的比值为最大的 “大圆相交” 的分划方式成为首选。即 R=100 的大圆两两相交所的半径和最小,一大一小两圆相交次之,两小圆相交最大。

\subsubsection{考虑湖泊存在时满足条件的区域分划}

\begin{figure}[h]
    \centering
    \includegraphics[width=0.8\textwidth]{image2.png}
    \caption{有湖情况下的区域分划}
    \label{fig:16}
\end{figure}

如上图,通过在邻湖区域内的面积的半径和计算(图中黑色框内的圆的个数 × 半径 100。其中部分圆的个数按照其框内部分与整个面积的比值计算)。当半径在 75~100 之间时,计算的半径为 100 时半径和最小。

\section*{Ad Hoc 网络中的区域划分和资源分配问题}

(3) 信道分配方案

当如上图分布时,每个圆最多与 7 个圆相邻,则该单位最少需要分配 4 个信道。如下图

\begin{figure}[h]
    \centering
    \includegraphics[width=0.8\textwidth]{image.png}
    \caption{有湖情况下的最少信道分配}
    \label{fig:17}
\end{figure}

\subsection{4.3 问题三}

(1) 当无湖存在条件下,在问题二中已证得,在不考虑边缘的情况下,$R=100$ 的大圆两两相交所的半径和最小,一大一小两圆相交。本题首选以较大的半径划分区域。采用以下算法编程:

假设每个一跳覆盖区依然是以节点为圆心,计算覆盖后如还有没被覆盖的点,再取消圆心为节点的限制。

假设处于一条边缘上的点看成已被覆盖。

1) 选择任意节点为起点(图中点 $A_1$),以 175.29497 为半径画圆(当两大圆相交面积为 5\% 时圆心距)。如果在圆的内部与圆的边缘距离 $d_1$ 最小的点(图中点 B)和起点 $A_1$ 的距离在 $100 \sim 175.29497$ 之间,那么

2) 以这个点 $B_1$ 为新的圆心,175.29497 为半径再画圆

3) 通过上一步的 $A_1$ 点为圆心,100 为半径画圆,通过子程序 1 求得圆 $B_1$ 半径 $r_1$(已知圆 $A_1$ 半径和圆心距,相交面积为大圆面积的 5\%),考察以 $B_1$ 为圆心,半径为 100 的区域内若没有点与 $B_1$ 点的距离在 $(r_1, 100)$ 之间,则以 $r_1$ 为半径画圆。否则连接 $B_1$ 点与圆内离 $B_1$ 最远的点,并以此为半径画圆。

\section*{Ad Hoc 网络中的区域划分和资源分配问题}

4) 每个点设置一个二进制符号位和一个数据位(初值为 0),当节点在步骤 3 所画的圆内,第一位位置 1。当节点 $(B_{1})$ 为圆心,数据位记录所在的较小的同心圆的半径。

5) 重复此操作,直到第 $n$ 次以 175.29497 为半径画圆后,没有节点在 $100 \sim 175.29497$ 之间。

6) 此时查找是否有节点的符号位为 0。如果有,通过查找第 $n$ 次以 $100 \sim 175.29497$ 为半径画圆时,是否还有点在圆内且符号位为 0。如没有,查找第 $n-1$ 次直到第一次以 $100 \sim 175.29497$ 为半径画圆的情况;如果有,查看圆的内部与圆的边缘距离 $d_{2}$ 第二小的点 $B_{n_{2}}$ 和第 $n$ 个起点 $A_{n}$ 的距离是否在 $100 \sim 175.29497$ 之间。如果在,通过子程序 1 求得圆 $B_{n_{2}}$ 半径 $r_{n_{2}}$。以 $B_{n_{2}}$ 为圆心,半径为 100 的区域内若没有点与 $B_{n_{2}}$ 点的距离在 $(r_{n_{2}}, 100)$ 之间,则以 $r_{n_{2}}$ 为半径画圆。否则连接 $B_{n_{2}}$ 点与圆内离 $B_{n_{2}}$ 最远的点,并以此为半径画圆。期间若出现下图情况,即当通过 A 点确定点 C 及 $r_{C}$,并以点 C 为圆心做圆时,若在半径 200 范围内有数据位不为零的点(即圆心),通过子程序 2(比较两圆半径大小,以大圆面积确定 5\% 面积,已知两圆半径,解得满足条件的圆心距 $H$),当圆心距 $BC$ 在 $(H, r_{C} + r_{B})$ 之间,圆 $C$ 与圆 $B$ 相交但不满足相交面积为大圆面积的 5\%,则分别以 $A$、$B$ 为圆心,$AC$、$H$ 为半径作弧。两弧相交于点 $D$。此时以点 $D$ 为新的圆心,$r_{C}$ 为半径做圆,则圆 $D$ 与圆 $A$ 和圆 $B$ 相交的公共面积满足条件,且点 $C$ 到点 $D$ 的移动距离小于圆 $C$ 半径的 $1/3$。新增节点 $D$,将 $r_{C}$ 的值计入点 $D$ 的数据位。

\begin{figure}[h]
    \centering
    \includegraphics[width=0.8\textwidth]{image.png}
    \caption{三圆公共面积优化}
\end{figure}

7) 重复 6,直到没有节点的符号位为 0 或直到查看第一个起点 $A_{1}$ 后仍有节点

的符号位为 0。当直到查看第一个起点 $A_{1}$ 后仍有节点 A 的符号位为 0 时,以该点为圆心,$(300-24.705=275.295)$ 为半径作圆(如图 19)。若无点在圆内,则该点不连通;若有点在圆内,找出与 A 点距离最近的节点 C,连接 AC,并以 C 点为圆心,100 为半径做圆,交 AC 于点 D。求 AC 上圆 E 半径 $=[AC-(100-24.705)]/2$。以点 A 为起点在 AC 上找到点 E,并以 E 为圆心,AE 为半径做圆,则点 A 可连通。新增节点 E,将 AE 计入点节点 E 的数据位。

\begin{figure}[h]
    \centering
    \includegraphics[width=0.8\textwidth]{image.png} % 替换为实际图片路径
    \caption{最远距离时的算法}
    \label{fig:19}
\end{figure}

注:$(100-24.705)$ 为公共部分面积中两段弧间的最大距离。

8) 将所有节点的数据位相加,即为所有圆的半径和。

9) 依次以 926 个点为起点,重复步骤 $1 \sim 8$,找出最小半径和及其区域分布方式。

子程序 1:已知圆 $A_{1}$ 半径和圆心距,相交面积为大圆面积的 $5\%$

子程序 2:比较两圆半径大小,以大圆面积确定 $5\%$ 面积,已知两圆半径,解得满足条件的圆心距 H

(2)当有湖存在条件下,在运行上述程序之前,先将代入椭圆方程左端:

\[
\frac{(x-550)^{2}}{205^{2}}+\frac{(y-550)^{2}}{105^{2}}=1
\]

若结果小于 1,则该点在湖内,不予以考虑。剔除湖内点之后再运行 $1 \sim 9$ 步程序。

\section*{区域连通的充分条件:运行程序后无未覆盖的节点}

\section*{区域连通的必要条件:该节点与其距离最近的临节点之间的距离不大于 $400-24.705=375.295$}

\subsection{问题四}

运动的方向角、速度是分别服从在 $[0, 2\pi]$、$[0, 2]$ 上均匀分布的随机变量。

则

\[
E(\theta)=\pi, \quad \sigma^{2}(\theta)=\frac{\pi^{2}}{3}; \quad E(v)=1, \quad \sigma^{2}(\theta)=\frac{1}{3}
\]

\[
\varphi(\theta)=\frac{1}{2\pi}, \quad 0<\theta<2\pi; \quad \varphi(v)=\frac{1}{2}, \quad 0<v<2;
\]

$400t/30t \approx 13.3$,故 400 单位时间后前十个用户改变方向角和速度各 13 次。

前十个用户作折线运动,由于网络采用基于节点的划分方式,故随着用户的移动其周围的覆盖区也在发生变化。当其走出最大可能半径范围而又未进入其他覆盖区,则此时其发生通信中断。

其按平均值移动时,$400t$ 后偏移量为 20。考虑改变方向角和速度最大的情

况,400t 后最大偏移量为

\[
\begin{aligned}
d & = \sqrt{\left[(30 \times 6 + 10) \times \frac{3 + \sqrt{3}}{3} \times \sin \frac{\pi}{\sqrt{3}}\right]^2 + \left[30 \times \frac{3 + \sqrt{3}}{3} (7 - 6 \cos \frac{\pi}{\sqrt{3}}) - 10 \times \frac{3 + \sqrt{3}}{3} \times \cos \frac{\pi}{\sqrt{3}}\right]^2} \\
& = 33.09
\end{aligned}
\]

即任意节点的偏移量的范围为 \(20 \sim 33.09\),则 400 个时间单位后任一节点的可能位于以原初始位置为圆心,分别以 20、33.09 为半径的圆环内。

圆环面积可求:\(S_{\text{环}} = \pi (R^2 - r^2) = \pi \times (33.09^2 - 20^2) \approx 2183.24\)

以该节点初始位置为圆心,100 为半径画圆,找出在其范围内包含的所有问题三中确定的圆心点,圆心距可求,环形的内外环半径已知,则可求圆环与该圆心点所确定的圆的公共面积,计算圆内所有满足此条件的面积和 \(S_0\),注意多圆相交面积不要重复计算。判断 \(S_{\text{环}} - S_0\) 是否为 0,若为零则网络可正常工作,若大于零则部分网络将中断。

### 4.5 问题五

假设最大传输距离取极限情况即 \(h = 2R\),则单个节点的发射功率 \(W = K_1 h^3 = K_1 (2R)^3 = 8K_1 R^3\),由于发射、接收、备用三种状态的功率之比为 11:10:1,故三种情况对应的功率分别为 \(W\)、\(10W/11\)、\(W/11\)。由题目所给数据可得节点总数为 \(n = 926\)。

假设转发也可以看作是一次接收和一次发射,每个节点平均发射 25 次,则每个节点平均接收也是 25 次,以单个节点的能量情况为例计算:发射能量消耗 \(E_{\text{发}} = W \times 4 \times 25\),接收能量消耗 \(E_{\text{收}} = \frac{10W}{11} \times 4 \times 25\),备用时能量消耗 \(E_{\text{备}} = (1200 - 4 \times 25 \times 2) \frac{W}{11}\)。

假设在一个节点有多个信道的情况下,多信道可以同时工作。则实际消耗的总能量为 \(E_1 = (E_{\text{发}} + E_{\text{收}} + E_{\text{备}}) \times 926 = \frac{2870600}{11} \times 8K_1 \times R^3\)。

由题设条件可知,当覆盖半径为 100 时,电池可工作的时间为 400t。且平均 1200t 时间内每个节点的发射次数为 25,则 400t 时间内每个节点的平均发射次数为 \(25/3\)。则按照上述方法可得:电池的总能量 \(E_{\text{总}} = \frac{2870600}{33} \times 8K_1 \times 100^3\)。

实际工作中为使网络可以满足条件运行 1200t,则需要使 \(E_1 \leq E_{\text{总}}\)。

带入即得:\(\frac{2870600}{11} \times 8K_1 \times R^3 \leq \frac{2870600}{33} \times 8K_1 \times 100^3\)

解得:\(R \leq 69.34\)

故为使其工作时间满足 1200t,则最大的覆盖半径应为 69。

区域划分方法可参照问题三中基于节点划分方法的编程思想。对组网方式可

\section*{Ad Hoc 网络中的区域划分和资源分配问题}

以结合路由监测,动态监测每个节点的即时位置,随着节点的运动动态的改变组网方式,以提高网络的抗毁性。

\subsection*{4.6 问题六}

(1) 问题分析

对问题五中的网络通信质量进行定量的分析,就是讨论数据包在申请接收的过程中,丢失的几率。选取区域内任意一点,给出剩下的 \(N-1\) 个点对该点申请接受几率的数学公式,看其余点与所选点连线的几率。并将网络划分为两种状态,即“忙”(接收和发射中)与“闲”(等待中)。这样,让数据包去询问节点,是否可以接收,若“忙”则等待十个时间单位,若“闲”则进行接收,数据包最多可以进行 4 次访问,此时若还是没有被接收,则数据包丢失。然后进行概率计算,从而求得,在整个网络通信中,数据包丢失的几率。

(2) 问题计算

公式如下:

\[
M = \left[ 1 - \left( 1 - \frac{1}{n-1} \right)^{n-1} \times \frac{1}{6} \right]^4
\]

\[
n = 926 \quad \text{(网络区域内所有节点的数量)}
\]

\(M\):整个网络通信中,数据包丢失的几率

应用泰勒展开公式:

\[
\left( 1 - \frac{1}{n-1} \right)^{n-1} \approx C_{n-1}^0 - C_{n-1}^1 \frac{1}{n-1} + C_{n-1}^2 \left( \frac{1}{n-1} \right)^2
\]

近似计算 \(M = 4.8 \times 10^{-5}\)

(3) 通信质量评价

可以看出 \(M\) 的值相当小,百万分之四十八,数据包丢失几率很小,因此网络的通信质量相对较好。

\section*{5 模型结果的分析}

通过上述的计算结果,可以看出基于圆覆盖全部面积方式的模型假设(即问题 1,2)可以较好的节省成本,但是网络抗毁性差,相邻两圆相交部分越大(即节点设置的越多),抗毁性越差。基于节点划分方式的模型假设(即问题三),可以较好的解决抗毁性差的问题。在后面的问题中,计算出了比较节能的区域划分方式,满足题设条件的圆半径为 69,并且该网络的建立,可以很好的解决通信质量问题。

\section*{6 模型的优缺点与改进方向}

(1) 基于圆覆盖全部面积的方式(即问题 1,2)

优点:技术要求不高,实际操作简单,易于实现,耗能相对较少,相对成本较低。

缺点:但是不能较高的抗毁性,网络寿命较短,出现信息不能传输到指定位置的几率大。

\section*{Ad Hoc 网络中的区域划分和资源分配问题}

改进方法:可以在通信技术上进行改革,在能够保证较高的抗毁性的前提下,使得相邻圆重叠的面积尽量小,甚至可以相切,并保证所有圆的半径之和相对较小。这样可以最大限度的节省能量,减少服务的成本。但是有可能会增加维护的费用,具体操作还要在今后继续探索研究。

(2) 基于节点划分方式(即问题 3,4,5,6)

优点:能保证较高的抗毁性,网络寿命较长,信息的传输接受能力强,较小的掉包率,通信质量高。

缺点:技术要求高,实际操作困难,不易于实现,耗能相对较多,相对成本较高。

改进方法:通过技术改革,简化实际操作,降低成本。

\section*{参考文献}

[1] 左孝凌,李为鑑,刘永才《离散数学》上海科学技术文献出版社 2003.6 P319

[2] 唐焕文,秦学志《实用最优化方法》大连理工大学出版社 2004.1

[3] 李慶揚 《数值分析(第 4 版)》 清華大學出版社 2001.8.1