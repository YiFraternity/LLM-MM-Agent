\section{全国第三届研究生数学建模竞赛}



\textbf{题目} \hfill Ad Hoc 网络中的区域划分和资源分配问题

\begin{abstract}
针对问题 1 中的约束条件,以要求覆盖圆最少为目标函数,建立了区域覆盖模型 1。联系移动通信中蜂窝网结构特点,得到在满足公共面积不小于一圆的 5\% 的情况下,最少需要 45 个圆覆盖,在 18\% 的情况下,最少需要 61 个圆覆盖。两种情况下最少需要 3 个信道分配。在有湖的条件下,建立了区域覆盖模型 2,最终得到半径之和为 4333.7,最少需要 3 个信道分配。

在对网络的抗毁性研究时,首先考虑节点的全局连通性判断,定义了连通矩阵和连通归类矩阵,得到了两矩阵的许多重要性质,并且得到网络区域节点连通的充分必要条件为存在一个 \( m < N \) 使得 \( L^m = (1)_{N \times N} \)。在此基础上,定义了抗毁概率来衡量网络的抗毁性,此抗毁概率即为网络连通性概率。通过蒙特卡罗实验得到节点在不同抽取率下的抗毁概率。

对于问题 3,我们建立了节点划分模型 3,结合约束条件提出了一种实现的方案——自适应 K-中心聚类方法,最终分别得到了有湖和无湖两种情况下的节点划分方案,无湖时需要 44 个圆,半径之和为 3989.5,有湖时需要 42 个圆,半径之和为 3782.5,并通过在不同抽取率下的抗毁实验得到了网络抗毁性。在上面得到的节点划分方案上,针对问题 4 进行蒙特卡罗实验,得到网络的连通性概率约为 98.12\%。对于问题 5,我们从理论上分析了以任意一个网络节点工作时间尽量长为目标函数,建立区域划分节能模型 4。在问题 6 中通过定义信息丢包度量函数和通信度量函数,我们给出了对网络通信质量的较准确的定量评价方法。
\end{abstract}

\tableofcontents

\section{问题重述}
Ad  Hoc 网络是当前网络和通信技术研究的热点之一,对于诸如军队和在野
外作业的大型公司和集团来说,Ad  Hoc 网络有着无需基站、无需特定交换和路
由节点、随机组建、灵活接入、移动方便等特点,因而具有极大的吸引力。 
现在在一个1000 1000(面积单位)的区域内构建一个Ad  Hoc 网络,需要完
成以下工作: 
(1)      将此正方形区域用若干个半径都是100 的圆完全覆盖,要求相邻两
个圆的公共面积不小于一个圆面积的5%,最少需要多少个圆?若给每个圆分配
一个信道,使得有公共部分的圆拥有不同的信道,最少需要几个信道?怎样分
配?如果将上面的5%改为18%,其它不变,结果又如何?对以上两种划分,若
每个公共部分中心和相应圆心各恰有一个节点,讨论网络的抗毁性。 
(2) 设正方形区域中有一中心在(550,550)、长轴与正方形水平的一条边成
30 度角、长度为410、短轴为210 的椭圆形湖泊。节点仅能设置在地面上,假设
一跳覆盖区圆的半径可以在75~100 间随意选择,两个面积不等的圆相交,它们
之间的公共面积应不小于大圆面积的5%,其他假设同(1),研究使全部圆半径
之和为最小的区域分划和信道分配方案。 
(3)  将正方形区域内的节点(用户)分成若干个簇,以完全覆盖某一簇内
所有节点、且半径不大于100 的圆作为一个一跳覆盖区。在满足有转发任务的相
邻一跳覆盖区的公共面积不小于较大一跳覆盖区面积的5%、且正方形区域内所
有节点连通的条件下,以附件1 给出的数据作为静止(节点不移动)状态,针对
正方形中无湖和有湖两种情况,研究使全部一跳覆盖区半径之和为最小的一跳覆
盖区划分和信道分配方案。找出区域连通的充分、必要条件。类似于(1),讨论
你们建立的Ad Hoc 网络的抗毁性? 
(4)  进一步假设数据文件中的前10 个用户只作折线运动,每30 个单位时
间可能改变一次运动的方向和速度,运动的方向角、速度是分别服从在[0,2]  、
[0,2]上均匀分布的随机变量,其他节点不移动。节点到达正方形区域边界后只
可能向区域内运动。请考虑400 单位时间后Ad Hoc 网络的连通性。 
  4
(5)  以附件 1 给出的数据为初始状态,设想网络需要运行 1200 个时间单
位,在整个运行时间内,每个节点平均产生 25 次呼出(节点入网后,可处于发
射、接收和备用三种状态,相应的能耗比约为 11:10:1。当需要多跳转发时随机
选择一条通路进行。),每次通信持续时间服从指数分布,平均为4 个单位时间。
假设电池在覆盖半径为100 发送状态下的工作总时间是400 个时间单位,一旦电
池用尽节点即退出网络。请按照(3)中给出的办法(无湖的情况),找到比较节
能的区域分划方式,使出现第一个退出网络的节点的时间尽量长。通过对该网络
的运行状况进行分析,提出你们对组网方式的改进意见。 
(6)  Ad  Hoc 网络中还有一个重要的问题就是如何保证通信的质量。Ad 
Hoc 网络中通信实行先到先服务。如果当其他节点对某节点有通信要求时,该节
点却处于忙状态,则会产生一次重发,所产生的时间差称为延时,将一次通信看
成一个分组,粗略地认为重发3 次(包括3 次)或延时30 个时间单位就可能丢
包。显然信息丢包(包括网络不通)是严重影响网络通信质量的大问题,对(5)
中这方面的通信质量进行定量评价。

\section{符号说明}
D:正方形区域 
F:圆覆盖后图形区域 
B:椭圆形湖区域 
E:D B  
ri:圆i的半径; 
dij :圆i与圆 j的圆心距; 
sij :圆i与圆 j的重叠面积; 
T:目标区域内所有节点集合 
C:公共节点集合 
O:圆心节点集合 
M :覆盖区域圆个数 
N:区域内节点总数 
L1:一阶连通矩阵 
Ln :n 阶连通矩阵 
\mathrm{I}:连通归类矩阵 
Pn:节点全局不连通概率 
Pc:抗毁概率,也表示网络连通性概率

\section{模型假设}
1.  此正方形区域为平面区域。 
2.  区域中各节点处于同等重要地位,且各节点的性能参数相同。 
3.  问题 2 中,因为椭圆形湖面上相交的圆之间公共面积内并没有承担转发
任务,可以假设此公共面积不要求满足5%条件。 
4.  问题3 中,节点运动缓慢,我们假设其静止。 
5.  节点发射和接收不能同时,且一个节点同一时间只能接收一路信号。

\section{问题分析}
\subsection{1.  问题1}
(1) 将此正方形区域用若干个半径都是 100 的圆完全覆盖,要求相邻两个圆
的公共面积不小于一个圆面积的5%,最少需要多少个圆(如果一个圆只有部分
在正方形区域中,也按一个计算)?若给每个圆分配一个信道,使得有公共部分
的圆拥有不同的信道,最少需要几个信道?怎样分配(用示意图标出)? 
(2)  如果将上面的5%改为18%,其它不变,结果又如何? 
(3)  对以上两种划分,若每个公共部分中心和相应圆心各恰有一个节点,讨
论网络的抗毁性。(即从节点集合中随机地抽掉2%、5%、10%、15%等数量的节
点后网络是否仍然连通)。 
难点1:如何确定满足区域全覆盖 
难点2:如何满足相交圆的公共面积不小于一个圆面积的5%(18%) 
难点3:如何进行区域连通性判断及抗毁性度量 
难点 1、2 解决思路:考虑此网络结构与移动通信中蜂窝网络的相似性,尝
试使用蜂窝网中的有关方法和结论解决问题。 
难点3 解决思路:建立区域连通性判断准则,定义适当的抗毁概率 cP,并通
过蒙特卡罗实验估计 cP来衡量网络的抗毁性。

\subsection{问题 2}
全部覆盖含有椭圆形湖的正方形区域的全部圆半径之和为最小的区域分划
方法和信道分配方案 
难点1:有湖,导致最优全覆盖困难 
难点2:覆盖圆的半径(75~100)可变,且要求全部圆半径之和为最小 
难点1、2 解决思路:在问题1 覆盖的基础上进行局部优化,先用半径为100
的圆进行覆盖,再进行边缘区域的圆半径、位置优化。

\subsection{问题3}
(1)  针对正方形中无湖和有湖(有湖时认为湖中节点不存在)两种情况,将
节点分簇全覆盖,研究使全部一跳覆盖区半径之和为最小的一跳覆盖区划分和信
道分配方案 
(2)  区域连通的充分、必要条件 
(3)  所建立的Ad Hoc 网络的抗毁性 
难点1:如何进行有效合理的节点分簇 
解决思路:利用数据挖掘技术进行节点聚类处理 
难点2:如何保证区域连通 
解决思路:合理定义有转发任务的公共覆盖区域 
难点3:如何确立区域连通的条件 
解决思路:简化区域连通性判断准则,建立适当的判断函数

\subsection{问题4}
前 10 个用户作满足题意的折线运动,400 单位时间后,检验问题 3 中建立
网络的连通性 
难点1:蒙特卡罗试验仿真实现

\subsection{问题5}
(1)找到比较节能的区域分划方式,使出现第一个退出网络的节点的时间尽量
长。 
(2)通过对该网络的运行状况进行分析,提出你们对组网方式的改进意见。 
难点1:如何建立满足条件的区域划分节能模型 
难点2:衡量网络运行状况的标准建立

\subsection{问题6}
对问题5 中划分的节能区域模型的通信质量进行定量评价 
难点1:信息丢包如何度量 
难点2:通信质量如何度量 
解决思路:建立适当的度量函数

\section{模型建立与问题求解 }
相邻圆的公共面积计算公式:
\begin{aligned}
s_{\bar{y}} & =r_{i}^{2}\arccos\frac{r_{i}^{2}+d_{ij}^{2}-r_{j}^{2}}{2r_{i}d_{ij}^{2}}-\frac{1}{2}r_{i}^{2}\sin\left(2\arccos\frac{r_{i}^{2}+d_{ij}^{2}-r_{j}^{2}}{2r_{i}d_{ij}^{2}}\right) \\
 & +r_{j}^{2}\arccos\frac{r_{j}^{2}+d_{ij}^{2}-r_{i}^{2}}{2r_{j}d_{ij}^{2}}-\frac{1}{2}r_{j}^{2}\sin\left(2\arccos\frac{r_{j}^{2}+d_{ij}^{2}-r_{i}^{2}}{2r_{j}d_{ij}^{2}}\right) \\
\text{当}r_{i}=r_{j}\text{时},\quad s_{ij}=2r^{2}|\mathrm{arccos}\frac{d_{ij}}{2r}-\sin\left(2\mathrm{arccos}\frac{d_{ij}}{2r}\right)|
\end{aligned}

\subsection{.1模型1 区域覆盖模型建立 }
目标函数: min M,其中 M 表示目标区域圆个数。
约束条件:
1) 用半径为 100 的圆覆盖正方形区域,即 ∀(x, y) ∈ D,D 表示正方形区域都有 (x, y) ∈ F,F 表示覆盖圆的区域。第 i 个圆覆盖的区域记为 F_i,则 F = ∪F_i。
2) 相邻两圆的公共面积不小于一个圆面积的 5%(18%),即
$$s_{ij} \geq 5\% \pi r^2 \quad (s_{ij} \geq 18\% \pi r^2)$$
数学模型:
目标函数: min M
约束条件:
$$
\begin{cases}
\forall (x, y) \in D, (x, y) \in F = \bigcup_{i=1}^{M} F_i; \\
s_{ij} \geq 5\% \pi r^2 \quad (s_{ij} \geq 18\% \pi r^2); \quad (i, j = 1, \cdots, M) \\
r_i = 100;
\end{cases}
$$

\subsection{1.2  区域覆盖模型1求解}

(1) 5%覆盖问题 
考虑到此问题是一个区域覆盖问题,因此可以借鉴移动通信中蜂窝网的构建
方法。要求完全覆盖整个区域,用圆的内接正多边形来近似地代替圆,是比较方
便符合实际情况的。可以证明,由正多边形彼此邻接构成平面时,只能是正三角
形、正方形和正六边形。通过计算,采用正六边形结构时,重叠部分的面积为
5.77%,满足题中“相邻两个圆的公共面积不小于一个圆面积的5%”的要求。 
求解结果:通过仿真实验可以得到最少需要45 个半径为100 的圆来覆盖; 
同样,类比于蜂窝网的信道分配方案,最少需要3 个信道分配,分配
方法如图1。(信道分别用☆,+,*表示) 
(2) 18%覆盖问题 
当要求相邻两个圆的公共面积不小于一个圆面积的18%时,同上,我们尝试
用圆的内接正方形来代替圆进行覆盖,但此时有多个圆处于相切状态,但是通常
认为相切即是相邻,则不满足“相邻两个圆的公共面积不小于一个圆面积的18%”
的要求,因此,我们考虑同样用正六边形来近似地代替圆,通过缩短圆心距来达
到覆盖面积不小于18%的要求。 
求解结果:通过仿真实验可以得到最少需要61 个半径为100 的圆来覆盖; 
最少需要3 个信道分配,分配方法如图2。(信道分别用☆,+,*表示) 
 
\begin{figure}[h]
    \centering
    \includegraphics[width=\textwidth]{image.png}
    \caption{5%覆盖问题信道分配图}
    \label{fig:6}
\end{figure}

\begin{figure}[h]
    \centering
    \includegraphics[width=\textwidth]{image.png}
    \caption{18%覆盖问题信道分配图}
\end{figure}

\subsection{1.3  网络抗毁性研究 }
对于Ad Hoc 网络通信,在受到攻击时,其抗毁性是衡量网络好坏的一项重
要标准。所谓的抗毁性就是此网络部分节点在遇到破坏时的通信连通的能力。

\subsubsection{1.3.1 相关定义及其性质}
我们定义一个抗毁概率来衡量网络的连通性,首先定义不连通概率。 
定义1 不连通概率 nP定义 
考虑如果在以一定概率破坏节点时,整个网络节点不能保持连通的概率,此
概率我们称为网络的不连通概率,记为 nP。 
定义2 抗毁概率 $P_c$ 定义

抗毁概率定义为 $P_c = 1 - P_n$,表示网络节点连通的概率。此概率也可称为网络连通概率。

为了判断网络的连通性,我们构造了一系列的矩阵表示,下面说明相关定义:

定义3 一阶连通矩阵定义

一阶连通矩阵 $L^1 = \left( l_{ij}^{(1)} \right)_{N \times N}$,其中

$l_{ij}^{(1)} = \begin{cases} 1, & \text{若节点 } i \text{ 与节点 } j \text{ 可以直接连通}; \\ 0, & \text{若节点 } i \text{ 与节点 } j \text{ 不可以直接连通}; \end{cases} (i, j \in \{1, 2, \cdots, N\})$

一阶连通矩阵 $L^1$ 表示两节点之间可以直接连通的情况,即两节点位于同一覆盖区域内,因此一阶连通矩阵可以直接由覆盖区域内的节点分布情况直接建立。

定义4 二阶连通矩阵以及 $n$ 阶连通矩阵定义

二阶连通矩阵 $L^2 = \left( l_{ij}^{(2)} \right)_{N \times N}$,其中 $l_{ij}^{(2)}$ 是按照矩阵乘法计算的逻辑和,即:

$l_{ij}^{(2)} = \begin{cases} 1, & \text{若 } \sum_{k=1}^{N} l_{ik} l_{kj} \neq 0 \\ 0, & \text{若 } \sum_{k=1}^{N} l_{ik} l_{kj} = 0 \end{cases}$

从上面定义的 $L^2$ 计算式的实际意义表示的是:

$l_{ij}^{(2)} = \begin{cases} 1, & \text{若节点 } i \text{ 与节点 } j \text{ 通过 } 1 \text{ 次转发可以连通}; \\ 0, & \text{若节点 } i \text{ 与节点 } j \text{ 通过 } 1 \text{ 次转发不可以连通}; \end{cases} (i, j \in \{1, 2, \cdots, N\})$

\subsection{模型 2 区域覆盖模型 2 建立}

目标函数: $\min R \triangleq \min \sum_{i=1}^{M} r_{i}$

其中 $M$ 表示覆盖的圆个数,$r_{i}$ 表示覆盖的圆的半径。

约束条件:

1) 用半径为 $75 \sim 100$ 的圆覆盖有椭圆形湖 $B$ 的正方形区域 $D$,即 $\forall (x, y) \in D - B \triangleq E$ 都有 $(x, y) \in F$,$F$ 表示覆盖圆的区域。第 $i$ 个圆覆盖的区域记为 $F_{i}$,则 $F = \bigcup_{i=1}^{M} F_{i}$;

2) 相邻两圆的公共面积不小于大圆面积的 $5\%$,即 $s_{ij} \geq 5\% \pi \left[ \max \left( r_{i}, r_{j} \right) \right]^{2}$;

依次可类推定义n阶连通矩阵$L^n=(l^{(n)}_{ij})_{N\times N}$,$1\leq n<N$:

其中$l^{(n)}_{ij}=\begin{cases}1, & 若\sum_{k=1}^{N}l^{(n-1)}_{jk}l_{kj}\neq0 \\0, & 若\sum_{k=1}^{N}l^{(n-1)}_{jk}l_{kj}=0\end{cases}$

$l^{(n)}_{ij}=\begin{cases}1, & 若节点i与节点j通过n-1次转发可以连通;(i,j\in\{1,2,\cdots,N\}) \\0, & 若节点i与节点j通过n-1次转发不可以连通;\end{cases}$

$n$阶连通矩阵$L^n$表示节点之间通过$n-1$次转发连通的情况。可以证明含有$N$个节点的连通网络,最多只需要$N-2$次转发就可以达到全部节点连通,因此我们只需要定义$N-1$阶连通矩阵。

定义5 连通归类矩阵定义

连通归类矩阵$\Gamma=f(L^n)$,其中$f(\bullet)$表示对$L^n$的运算,运算方法规定如下。

对于$n$阶连通矩阵$L^n$,将其行与列进行变换至

$\Gamma=\begin{bmatrix}1_{k_1\times k_1} & 0 & \cdots & 0 \\0 & 1_{k_2\times k_2} & \cdots & 0 \\0 & 0 & \ddots & 0 \\0 & 0 & \cdots & 1_{k_s\times k_s}\end{bmatrix}$

其中$1_{k\times k}=\begin{bmatrix}1 & \cdots & 1 \\ \vdots & \ddots & \vdots \\ 1 & \cdots & 1\end{bmatrix}$。由连通性矩阵的定义可以知道$n$阶连通矩阵$L^n$总是可以化为连通归类矩阵形式。

例如:$L^n=\begin{bmatrix}1 & 0 & 0 & 1 \\0 & 1 & 1 & 0 \\0 & 1 & 1 & 0 \\1 & 0 & 0 & 1\end{bmatrix}\Rightarrow\Gamma=\begin{bmatrix}1 & 1 & 0 & 0 \\1 & 1 & 0 & 0 \\0 & 0 & 1 & 1 \\0 & 0 & 1 & 1\end{bmatrix}$(将第4列和第4行分别与第2列和第2行分别对换)

相关性质:

性质1:$n$阶连通性矩阵$L^n$($n=1,\cdots,N$)为主对角线元素为1的对称阵。

性质 2: 网络节点连通的充分必要条件为存在一个$m<N$使得
$$L^m=\left(l_{ij}^{(m)}\right)_{N\times N}=\left(1\right)_{N\times N}\:。$$

推论$\mathbf{1}:$如果$\forall n\in \left \{ 1, 2, \cdots , N- 1\right \}$, $L^n= \left ( l_{\ddot{y} }^{( n) }\right ) _{N\times N}\neq \left ( 1\right ) _{N\times N}$,则网络节点不连通。

推论 2: 网络连通的充要条件为$L^{N-1}=\left(1\right)_{N\times N}$

性质 3: 在连通归类矩阵$\Gamma$ 中每个$1_{k_i\times k_i}$内对应的节点之问连通
推论 3: 在连通归类矩阵$\Gamma\text{中任意两个1}_{k_i\times k_i}$中的任意两个节点都不连通。

\subsubsection{1.3.2 抗毁性研究}
基本思路: 
1通过蒙特卡罗实验来求解抗毁概率 cP。 
2 全局连通性判断: 
方法1:通过n阶连通矩阵的性质判断 
方法2:通过遍历搜索方法判断 
现在首先考虑全局连通的判断方法实现过程。由上面定义以及推论2 可以得
到使用n阶连通矩阵判断方法。具体方法实现如
下: 
1.  由节点及区域覆盖数据建立 1 阶连通矩阵
1L; 
2.  按照定义的计算方法计算n阶连通矩阵 nL  
3.  判断n阶连通矩阵 nL 是否为全 1 矩阵
 1 N N ,若是则退出判断;否则判断n是否小于
1N ,小于时,按照计算方法计算 1n 阶连通矩
阵 1nL ,大于则退出判断。 
具体判断方法流程如图3。 
建立的n 阶连通矩阵为N N 矩阵,在节点数据较大的时候,其计算量较庞
大,下面尝试减少计算量。我们可以将节点分为两类:同属多个一跳覆盖区的公
 1n
N NL 1n n 
n N
图4  全局连通性判断流程图 
  13
共节点集合C和只属于单个一跳覆盖区的圆心节点集合O。由于圆心节点只能使
用此圆内分配的单一信道,因此对于区域的连通不具有贡献,对于公共节点,可
以使用各相邻一跳覆盖区的各个信道。所以可以将节点分为两部分分别考虑,首
先判断圆心节点,如果在该圆心节点所在圆内存在公共节点,则我们称该圆心节
点是连通的,只有所有圆心节点都连通的时候,才进行公共节点的连通判断,否
则称网络不连通。 
在进行公共节点判断时,我们可以采用两种方法来实现。 
方法1:简化的n阶连通矩阵判断方法 
此时节点数据仅为公共节点数据,采用上面相同的方法判断公共节点连通
性。 
方法2:遍历搜索的方法 
建立一个记录库,任意选取一个公共节点记入记录库,搜索其它公共节点,
如果与记录库中节点连通,则加入记录库中,继续搜索直到没有公共节点与记录
库中节点连通。如果记录库包含所有存在的公共节点,说明网络节点连通,否则
不连通。 
求解结果: 
按题意从节点集合中随机地抽掉2%、5%、10%、15%等数量的节点后,各
做 50000 次蒙特卡罗实验,5%覆盖网络与 18%覆盖网络节点不连通概率分别列
于表1、表2。

\begin{table}[htbp]
\centering
\caption{5\% 覆盖网络抗毁实验}
\begin{tabular}{|c|c|c|c|c|c|c|}
\hline
\textbf{抽取率} & 2\% & 5\% & 10\% & 15\% \\ \hline
不连通次数 & 3 & 31 & 374 & 1186 \\
抗毁概率 $P_c$ & 99.994\% & 99.938\% & 97.252\% \\
\hline
\end{tabular}
\caption{表 1}
\end{table}

\begin{table}[htbp]
\centering
\caption{18\%覆盖网络抗毁实验}
\begin{tabular}{c|c|c|c|c|c|c|}
\hline
\textbf{抽取率} & 2\% & 5\% & 10\% & 15\% \\ \hline
不连通次数 & 8 & 71 & 672 & 1952 \\
抗毁概率 $P_c$ & 99.984\% & 99.858\% & 96.096\% \\
\hline
\end{tabular}
\caption{表 2}
\end{table}

\section{问题2}
模型2 区域覆盖模型2 建立

目标函数:$\min R \doteq \min \sum_{i=1}^{M} r_{i}$

其中 $M$ 表示覆盖的圆个数,$r_{i}$ 表示覆盖的圆的半径。

约束条件:

1) 用半径为 75~100 的圆覆盖有椭圆形湖 B 的正方形区域 D,即 $\forall (x,y) \in D-B \doteq E$ 都有 $(x,y) \in F$,$F$ 表示覆盖圆的区域。第 i 个圆覆盖的区域记为 $F_{i}$,则 $F=\bigcup_{i=1}^{M} F_{i}$;

2) 相邻两圆的公共面积不小于大圆面积的 5%,即 $s_{ij} \geq 5\% \pi \left[\max\left(r_{i}, r_{j}\right)\right]^{2}$;
数学模型

目标函数:$\min R \triangleq \min \sum_{i=1}^{M} r_{i}$

约束条件:

\begin{equation}
\begin{cases}
\forall (x, y) \in D, (x, y) \in F = \bigcup_{i=1}^{M} F_{i}; \\
s_{ij} \geq 5\% \pi r^{2} \left( s_{ij} \geq 18\% \pi r^{2} \right); \, (i, j = 1, \cdots, M) \\
r_{i} = 100;
\end{cases}
\end{equation}

模型 2 求解

覆盖区域 $E$ 的圆半径和与覆盖它的圆的个数 $M$ 以及每个圆的半径有关,圆的个数越少,半径越小,则结果越好,但是由于当圆的个数减少,则必然通过增大半径来覆盖区域 $E$,反之亦然。考虑到问题 1 的结果,发现当要求相邻两圆的

公共面积不小于圆面积的 $5\%$ 时,正六边形分布需要的圆的个数最少,在此基础上我们设想在此基础上对网络结构加以改进,得到满足条件的一种较优解。

改进的方法为:对于不与椭圆形湖相交的圆,我们仍然按照原来的方式排列;对于与椭圆形湖相交的圆,通过缩小圆半径和改变圆心位置,找到满足约束条件的圆替代原来的覆盖圆。

求解结果

区域划分方案:

通过计算机仿真得到满足约束条件的一种区域划分的结构,如图 4。最终得到的半径之和为:$R = 4333.7$

信道分配方案:

仍然可以采用前文中的分配方式,采用 3 信道即满足条件。

[FIGUREENV:1]

\section{问题 3}

模型 3 节点划分模型 3 建立

目标函数: $\min \sum_{i=1}^{M} r_{i}$

其中 $M$ 表示覆盖的圆个数, $r_{i}$ 表示覆盖的圆的半径。

\textbf{约束条件:}

1) 半径可变 $0 \sim 100$;

2) 圆要覆盖目标区域内的所有节点, 即 $\forall(x, y) \in T$, $T$ 表示所有节点的集合, 都有 $(x, y) \in F$, $F$ 表示覆盖圆的区域。第 $i$ 个圆覆盖的区域记为 $F_{i}$, 则

\[
F = \bigcup_{i=1}^{M} F_{i};
\]

3) 满足转发任务的相邻圆的公共面积不小于大圆的 $5\%$, 即

\[
s_{ij} \geq 5\% \pi \left[\max \left(r_{i}, r_{j}\right)\right]^{2},
\]

其中 $r_{i}, r_{j}$ 为满足转发任务的相邻圆半径;

说明: 这里我们理解有转发任务的公共区域为: 满足区域内所有一跳覆盖区域的圆都能连通的必要的公共区域;

4) 存在 $n \leq N$, 使得 $n$ 阶连通矩阵 $L^{n} = (1)_{N \times N}$;

\textbf{数学模型:}

目标函数: $\min \sum_{i=1}^{M} r_{i}$

\begin{equation}
\begin{cases}
\forall (x, y) \in E, (x, y) \in F = \bigcup_{i=1}^{M} F_{i}; \\
s_{ij} \geq 5\% \pi r^{2}; & (i, j = 1, \cdots, M) \\
75 \leq r_{i} \leq 100;
\end{cases}
\end{equation}

模型求解

(1) 节点划分(自适应 K-中心聚类算法)和信道分配方案

理论实现过程:

由于此问题较复杂, 我们考虑先分析满足条件的一跳覆盖区划分理论上实现的可行性方法:

① 使用节点分簇方法将节点分成集合 $C_{i}, i=1,2, \cdots, G$, 对应集合内的节点记为 $c_{i_{k}}, k=1, \cdots, G_{i}$, 其中 $G_{i}$ 为节点分簇集合内节点个数, 对应覆盖圆区域为 $\Phi_{i}, i=1,2, \cdots, G$

② 第 $i$ 簇节点的覆盖圆表达式为:
\[
\left(x-x_{i}\right)^{2}+\left(y-y_{i}\right)^{2}=r_{i}^{2}
\]
其中 $\left(x_{i}, y_{i}\right)$ 表示 $\Phi_{i}$ 的圆心, $0<r_{i} \leq 100$ 为圆的半径。

③ 判断两圆是否相邻:
\[
\begin{cases}
\text { 若 }\left(x_{i}-x_{j}\right)^{2}+\left(y_{i}-y_{j}\right)^{2}>\left(r_{i}+r_{j}\right)^{2}, \text { 相离 } \\
\text { 若 }\left(x_{i}-x_{j}\right)^{2}+\left(y_{i}-y_{j}\right)^{2} \leq\left(r_{i}+r_{j}\right)^{2}, \text { 相邻 }
\end{cases}
\]

④ 判断两圆需要满足转发任务, 这里我们理解转发任务即为在两圆的公共部分是否存在公共节点

⑤ 判断相邻两圆 $\Phi_{i}$ 与 $\Phi_{j}$ 的公共面积 $s_{ij} \geq 5 \% \pi \max \left(r_{i}, r_{j}\right)^{2}$, 其中
\[
\begin{aligned}
s_{ij}= & r_{i}^{2} \arccos \frac{r_{i}^{2}+d_{ij}^{2}-r_{j}^{2}}{2 r_{i} d_{ij}^{2}}-\frac{1}{2} r_{i}^{2} \sin \left(2 \arccos \frac{r_{i}^{2}+d_{ij}^{2}-r_{j}^{2}}{2 r_{i} d_{ij}^{2}}\right) \\
& +r_{j}^{2} \arccos \frac{r_{j}^{2}+d_{ij}^{2}-r_{i}^{2}}{2 r_{j} d_{ij}^{2}}-\frac{1}{2} r_{j}^{2} \sin \left(2 \arccos \frac{r_{j}^{2}+d_{ij}^{2}-r_{i}^{2}}{2 r_{j} d_{ij}^{2}}\right)
\end{aligned}
\]

⑥ 判断是否存在 $n \leq N$, 使得 $n$ 阶连通矩阵 $L^{n}=\left(1\right)_{N \times N}$, 存在则该解即为满足约束条件的一个可行解

⑦ 在可行解中找到半径之和最小的解, 则此时的一跳覆盖区划分即为我们所要求的最优解

实际可行实现方案——自适应 K-中心聚类方法

从上面的分析, 节点分簇方法是实现该一跳覆盖区划分的关键, 下面尝试使用数据挖掘中的 K-中心聚类方法结合约束条件解决上面问题, 我们称该方法为

自适应 K-中心聚类方法。

首先解释数据挖掘中的 K-中心聚类方法,它是选取 K 个初始中心,计算各数据点与 K 个中心的距离,到同一中心点距离最近的作为一类,则形成 K 类。

按照题目要求,最终的一跳覆盖区能够包含所有的节点,而不要求一跳覆盖区要完全覆盖目标区域,显然问题 1 中的正六边形一跳覆盖区方案是符合条件的一个可行解。

自适应 K-中心聚类方法基本思想:使用问题 1 中圆心数据作为 K-中心聚类的初始中心,开始聚类,然后用最小的圆覆盖同类中的节点,判断是否满足约束条件,不满足则调整圆,再将此时的圆心作为下一次聚类的中心,重新聚类,重复上面过程,最终得到满足题目条件的一跳覆盖区划分的结果。

自适应 K-中心聚类方法流程如下:

\begin{enumerate}
    \item 设置迭代次数 Num,使用问题 1 中 5\% 正六边形一跳覆盖区方案的圆心位置作为聚类的初始中心。采用这种初始位置选择的方法,使得我们的初始中心在目标区域内的分布较平均,使得聚类后的同类范围相差不大;
    \item K-中心聚类方法聚成 K 类;
    \item 用最小的圆来完全覆盖 K 类中的节点,此时的覆盖圆集满足约束条件 2);
    \item 考虑约束条件 1),当上面的覆盖圆集中某圆的半径 \( r_i > 100 \) 时,我们选择圆内节点与其它圆心的距离最远的点,将圆心朝着该点移动,直到可以用半径为 100 的圆覆盖它为止。此时该半径为 100 的圆肯定不覆盖原来同类中的所有节点,漏覆盖的点将其加入最接近的一类中。由于我们初始中心的为半径为 100 的正六边形一跳覆盖区方案,所以一定可以用半径小于 100 的圆覆盖住露出的节点。从而得到新的覆盖圆集;
    \item 考虑约束条件 3) 中满足转发任务的条件,这里我们理解转发任务即为在两圆的公共部分是否存在公共节点,存在则满足转发任务,反之则不满足。检查上面得到的覆盖圆集中圆的公共部分中是否存在公共节点,如果不存在,则找到离此圆距离最近的外部节点,扩大圆的半径并移动圆心,直到覆盖距离最近外部节点。圆心移动和扩大半径示意图见图 5。满足此条件的覆盖圆集中的节点为连通的,也就解决了约束条件 4) 的问题;
\end{enumerate}

\begin{itemize}
    \item[(6)] 考虑约束条件 3) 中相邻圆的公共面积不小于大圆的 $5\%$ 情况。如果覆盖圆集中某圆的公共面积小于大圆的 $5\%$,则扩大此圆的半径直到公共面积等于大圆的 $5\%$ 时停止;
    \item[(7)] 取上面结果的覆盖圆集的圆心集合作为下一次聚类的中心,重复上述步骤。
\end{itemize}

\textbf{求解结果:}

\textbf{节点划分模型实现结果:}

通过计算机仿真找到最优的一跳覆盖区划分方案见图 6、图 7

\begin{figure}[h]
    \centering
    \includegraphics[width=\textwidth]{image.png}
    \caption{无湖一跳覆盖区域划分}
    \label{fig:6}
\end{figure}

\begin{figure}[h]
    \centering
    \includegraphics[width=\textwidth]{image.png}
    \caption{图示}
\end{figure}

通过仿真优化计算,无湖时所有圆半径总和为 3989.5,共有 44 个圆;有湖时所有圆半径总和为 3782.5,共有 42 个圆。

\textbf{信道分配方案:}

类比于问题 1,只需要 3 个信道就可以满足题中要求。

区域连通的充分、必要条件

从前面连通性矩阵的定义和性质,可以得到:

网络节点连通的充分必要条件为:

\begin{equation}
\text{存在一个 } m < N \text{ 使得 } L^m = \left(l_{ij}^{(m)}\right)_{N \times N} = \left(1\right)_{N \times N}
\end{equation}

所建立的 Ad Hoc 网络的抗毁性

采用与问题 1 相同的方法进行判断,在不同的抽取率下(2\%、5\%、10\%、

\begin{table}[h]
\centering
\caption{无湖网络抗毁实验}
\begin{tabular}{|c|c|c|c|c|}
\hline
抽取率 & 2\% & 5\% & 10\% & 15\% \\ \hline
不连通次数 & 4 & 26 & 180 & 812 \\ \hline
抗毁概率$P_{c}$ & 99.92\% & 99.48\% & 96.40\% & 83.76\% \\ \hline
\end{tabular}
\end{table}

\begin{table}[h]
\centering
\caption{有湖网络抗毁实验}
\begin{tabular}{|c|c|c|c|c|}
\hline
抽取率 & 2\% & 5\% & 10\% & 15\% \\ \hline
不连通次数 & 9 & 45 & 392 & 1598 \\ \hline
抗毁概率$P_{c}$ & 99.82\% & 99.10\% & 92.16\% & 68.04\% \\ \hline
\end{tabular}
\end{table}

\section{问题 4}

求解思路

按照问题 3 一跳覆盖区域划分方案,发现前 10 个用户都不在公共区域内(见图 6、图 7 中菱形点表示),也就意味着这 10 个用户对于网络中其他节点的连通性不会产生影响,因此讨论 400 单位时间后 Ad Hoc 网络的连通性就只需要考虑这 10 个用户在 400 单位时间后不处于覆盖圆内区域的概率。

求解方法

使用蒙特卡罗方法,通过计算机仿真,试验次数为 5000 次。

求解结果

网络的连通性概率 $P_{c} = 98.12\%$

\section{问题 5}

{节能区域的划分}

{已知条件的假设转换}

(1) “发射、接收和备用三种状态,相应的能耗比约为 11:10:1”,“假设电池在覆盖半径为 100 发送状态下的工作总时间是 400 个时间单位”,我们设定比例系数为 1,那么可以认为每个节点的初始能量为 $11 \times 400 = 4400$ 能量单位,发射、接收和备用三种状态,相应的能耗为 11、10、1 能量单位/时间单位;

(2) “在整个运行时间内,每个节点平均产生 25 次呼出,两节点之间原始的平均通信次数大致与它们之间的距离的平方成反比”,由附件 1 给出的数据可以得出整个区域内有 926 个节点,因而在整个运行时间内共计产生 23150 次呼出;

{模型 4 建立}
{1. 节点能耗的计算}

1) 节点 $J_i$ 的发射能耗:

\[ 11 \cdot T_{ij}^1, \, j = 1, 2, \ldots, n_i, \] 其中 $T_{ij}^1$ 为第 $j$ 次发射延续的时间,服从均值为 4 的指数分布,$n_i$ 为 $[0, t]$ 时间内的发射次数,$t$ 为节点退出网络的时间;

指数分布:\[ p_e(x) = \begin{cases} \frac{1}{\lambda} e^{-\frac{1}{\lambda} x}, & x \geq 0 \\ 0, & x < 0 \end{cases}, \lambda = 4 \]

2) 节点 $J_i$ 的接收能耗:

\[ 10 \cdot T_{ij}^2, \, j = 1, 2, \ldots, m_i, \] 其中 $T_{ij}^2$ 为第 $j$ 次接收延续的时间,服从均值为 4 的指数分布,$m_i$ 为 $[0, t]$ 时间内的接收次数,$t$ 为节点退出网络的时间;

3) 节点 $J_i$ 的备用能耗:

备用时间 $T_{ij}^3 = 1200 - T_{ij}^1 - T_{ij}^2$,备用能耗为 $T_{ij}^3$

4) 节点 $J_i$ 的总能耗:

\[
1_{k \times k}=
\begin{bmatrix}
1 & \cdots & 1 \\
\vdots & \ddots & \vdots \\
1 & \cdots & 1
\end{bmatrix}_{k \times k}
\]

\subsection{2. 节点退出网络的时间的计算}

[MATHENV:4]

\section{问题 6}

模型建立
信息丢包的度量分析

(1) 一次通信时间为 \( T \),\( T \) 服从均值为 4 的指数分布:

\[
p_e(x) =
\begin{cases}
\frac{1}{\lambda} e^{-\frac{1}{\lambda} x}, & x \geq 0, \lambda = 4 \\
0, & x < 0
\end{cases}
\]

(2) \([0, T]\) 内产生重发射的概率为 \( P_0 = E \left\{ \frac{25T}{1200} + \frac{25T}{1200} \right\} \),则节点重发次数的期望值为 \( n = 25P_0 \);其中重发情况发生在目标节点处于发射或接收状态。

(3) 一次通信中,信息丢包的概率:

\[
P_1 = P \big[ \{ T \geq 30, n \geq 1 \} \cup \{ 0 < T < 30, n \geq 3 \} \big] = P \{ T \geq 30, n \geq 1 \} + P \{ 0 < T < 30, n \geq 3 \}
\]

(4) 定义与目标节点不连通的概率为 \( P_2 \);

从而建立通信质量度量函数:\( P = P_1 + P_2 \)

[REFERENCES:1]