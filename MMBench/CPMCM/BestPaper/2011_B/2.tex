\begin{center}
\textbf{全国第八届研究生数学建模竞赛}
\end{center}

\begin{tabular}{l l}
学校 & 中国人民解放军信息工程大学 \\
\hline
参赛队号 & 90005018 \\
\hline
队员姓名 & \begin{tabular}{l l}
1. & 于合理 \\
2. & 黄海 \\
3. & 田英国 \\
\end{tabular} \\
\hline
\end{tabular}

\begin{flushright}
参赛密码 \_\_\_\_\_\_\_\_\_\_\_\_\_\_\_\_\_\_\_\_\_\_\_\_\_\_\_\_\_\_\_\_\_\_\_\_\_\_\_\_\_\_\_\_\_\_\_\_\_\_\_\_\_\_\_\_\_\_\_\_\_\_\_\_\_\_\_\_\_\_\_\_\_\_\_\_\_\_\_\_\_\_\_\_\_\_\_\_\_\_\_\_\_\_\_\_\_\_\_\_\_\_\_\_\_\_\_\_\_\_\_\_\_\_\_\_\_\_\_\_\_\_\_\_\_\_\_\_\_\_\_\_\_\_\_\_\_\_\_\_\_\_\_\_\_\_\_\_\_\_\_\_\_\_\_\_\_\_\_\_\_\_\_\_\_\_\_\_\_\_\_\_\_\_\_\_\_\_\_\_\_\_\_\_\_\_\_\_\_\_\_\_\_\_\_\_\_\_\_\_\_\_\_\_\_\_\_\_\_\_\_\_\_\_\_\_\_\_\_\_\_\_\_\_\_\_\_\_\_\_\_\_\_\_\_\_\_\_\_\_\_\_\_\_\_\_\_\_\_\_\_\_\_\_\_\_\_\_\_\_\_\_\_\_\_\_\_\_\_\_\_\_\_\_\_\_\_\_\_\_\_\_\_\_\_\_\_\_\_\_\_\_\_\_\_\_\_\_\_\_\_\_\_\_\_\_\_\_\_\_\_\_\_\_\_\_\_\_\_\_\_\_\_\_\_\_\_\_\_\_\_\_\_\_\_\_\_\_\_\_\_\_\_\_\_\_\_\_\_\_\_\_\_\_\_\_\_\_\_\_\_\_\_\_\_\_\_\_\_\_\_\_\_\_\_\_\_\_\_\_\_\_\_\_\_\_\_\_\_\_\_\_\_\_\_\_\_\_\_\_\_\_\_\_\_\_\_\_\_\_\_\_\_\_\_\_\_\_\_\_\_\_\_\_\_\_\_\_\_\_\_\_\_\_\_\_\_\_\_\_\_\_\_\_\_\_\_\_\_\_\_\_\_\_\_\_\_\_\_\_\_\_\_\_\_\_\_\_\_\_\_\_\_\_\_\_\_\_\_\_\_\_\_\_\_\_\_\_\_\_\_\_\_\_\_\_\_\_\_\_\_\_\_\_\_\_\_\_\_\_\_\_\_\_\_\_\_\_\_\_\_\_\_\_\_\_\_\_\_\_\_\_\_\_\_\_\_\_\_\_\_\_\_\_\_\_\_\_\_\_\_\_\_\_\_\_\_\_\_\_\_\_\_\_\_\_\_\_\_\_\_\_\_\_\_\_\_\_\_\_\_\_\_\_\_\_\_\_\_\_\_\_\_\_\_\_\_\_\_\_\_\_\_\_\_\_\_\_\_\_\_\_\_\_\_\_\_\_\_\_\_\_\_\_\_\_\_\_\_\_\_\_\_\_\_\_\_\_\_\_\_\_\_\_\_\_\_\_\_\_\_\_\_\_\_\_\_\_\_\_\_\_\_\_\_\_\_\_\_\_\_\_\_\_\_\_\_\_\_\_\_\_\_\_\_\_\_\_\_\_\_\_\_\_\_\_\_\_\_\_\_\_\_\_\_\_\_\_\_\_\_\_\_\_\_\_\_\_\_\_\_\_\_\_\_\_\_\_\_\_\_\_\_\_\_\_\_\_\_\_\_\_\_\_\_\_\_\_\_\_\_\_\_\_\_\_\_\_\_\_\_\_\_\_\_\_\_\_\_\_\_\_\_\_\_\_\_\_\_\_\_\_\_\_\_\_\_\_\_\_\_\_\_\_\_\_\_\_\_\_\_\_\_\_\_\_\_\_\_\_\_\_\_\_\_\_\_\_\_\_\_\_\_\_\_\_\_\_\_\_\_\_\_\_\_\_\_\_\_\_\_\_\_\_\_\_\_\_\_\_\_\_\_\_\_\_\_\_\_\_\_\_\_\_\_\_\_\_\_\_\_\_\_\_\_\_\_\_\_\_\_\_\_\_\_\_\_\_\_\_\_\_\_\_\_\_\_\_\_\_\_\_\_\_\_\_\_\_\_\_\_\_\_\_\_\_\_\_\_\_\_\_\_\_\_\_\_\_\_\_\_\_\_\_\_\_\_\_\_\_\_\_\_\_\_\_\_\_\_\_\_\_\_\_\_\_\_\_\_\_\_\_\_\_\_\_\_\_\_\_\_\_\_\_\_\_\_\_\_\_\_\_\_\_\_\_\_\_\_\_\_\_\_\_\_\_\_\_\_\_\_\_\_\_\_\_\_\_\_\_\_\_\_\_\_\_\_\_\_\_\_\_\_\_\_\_\_\_\_\_\_\_\_\_\_\_\_\_\_\_\_\_\_\_\_\_\_\_\_\_\_\_\_\_\_\_\_\_\_\_\_\_\_\_\_\_\_\_\_\_\_\_\_\_\_\_\_\_\_\_\_\_\_\_\_\_\_\_\_\_\_\_\_\_\_\_\_\_\_\_\_\_\_\_\_\_\_\_\_\_\_\_\_\_\_\_\_\_\_\_\_\_\_\_\_\_\_\_\_\_\_\_\_\_\_\_\_\_\_\_\_\_\_\_\_\_\_\_\_\_\_\_\_\_\_\_\_\_\_\_\_\_\_\_\_\_\_\_\_\_\_\_\_\_\_\_\_\_\_\_\_\_\_\_\_\_\_\_\_\_\_\_\_\_\_\_\_\_\_\_\_\_\_\_\_\_\_\_\_\_\_\_\_\_\_\_\_\_\_\_\_\_\_\_\_\_\_\_\_\_\_\_\_\_\_\_\_\_\_\_\_\_\_\_\_\_\_\_\_\_\_\_\_\_\_\_\_\_\_\_\_\_\_\_\_\_\_\_\_\_\_\_\_\_\_\_\_\_\_\_\_\_\_\_\_\_\_\_\_\_\_\_\_\_\_\_\_\_\_\_\_\_\_\_\_\_\_\_\_\_\_\_\_\_\_\_\_\_\_\_\_\_\_\_\_\_\_\_\_\_\_\_\_\_\_\_\_\_\_\_\_\_\_\_\_\_\_\_\_\_\_\_\_\_\_\_\_\_\_\_\_\_\_\_\_\_\_\_\_\_\_\_\_\_\_\_\_\_\_\_\_\_\_\_\_\_\_\_\_\_\_\_\_\_\_\_\_\_\_\_\_\_\_\_\_\_\_\_\_\_\_\_\_\_\_\_\_\_\_\_\_\_\_\_\_\_\_\_\_\_\_\_\_\_\_\_\_\_\_\_\_\_\_\_\_\_\_\_\_\_\_\_\_\_\_\_\_\_\_\_\_\_\_\_\_\_\_\_\_\_\_\_\_\_\_\_\_\_\_\_\_\_\_\_\_\_\_\_\_\_\_\_\_\_\_\_\_\_\_\_\_\_\_\_\_\_\_\_\_\_\_\_\_\_\_\_\_\_\_\_\_\_\_\_\_\_\_\_\_\_\_\_\_\_\_\_\_\_\_\_\_\_\_\_\_\_\_\_\_\_\_\_\_\_\_\_\_\_\_\_\_\_\_\_\_\_\_\_\_\_\_\_\_\_\_\_\_\_\_\_\_\_\_\_\_\_\_\_\_\_\_\_\_\_\_\_\_\_\_\_\_\_\_\_\_\_\_\_\_\_\_\_\_\_\_\_\_\_\_\_\_\_\_\_\_\_\_\_\_\_\_\_\_\_\_\_\_\_\_\_\_\_\_\_\_\_\_\_\_\_\_\_\_\_\_\_\_\_\_\_\_\_\_\_\_\_\_\_\_\_\_\_\_\_\_\_\_\_\_\_\_\_\_\_\_\_\_\_\_\_\_\_\_\_\_\_\_\_\_\_\_\_\_\_\_\_\_\_\_\_\_\_\_\_\_\_\_\_\_\_\_\_\_\_\_\_\_\_\_\_\_\_\_\_\_\_\_\_\_\_\_\_\_\_\_\_\_\_\_\_\_\_\_\_\_\_\_\_\_\_\_\_\_\_\_\_\_\_\_\_\_\_\_\_\_\_\_\_\_\_\_\_\_\_\_\_\_\_\_\_\_\_\_\_\_\_\_\_\_\_\_\_\_\_\_\_\_\_\_\_\_\_\_\_\_\_\_\_\_\_\_\_\_\_\_\_\_\_\_\_\_\_\_\_\_\_\_\_\_\_\_\_\_\_\_\_\_\_\_\_\_\_\_\_\_\_\_\_\_\_\_\_\_\_\_\_\_\_\_\_\_\_\_\_\_\_\_\_\_\_\_\_\_\_\_\_\_\_\_\_\_\_\_\_\_\_\_\_\_\_\_\_\_\_\_\_\_\_\_\_\_\_\_\_\_\_\_\_\_\_\_\_\_\_\_\_\_\_\_\_\_\_\_\_\_\_\_\_\_\_\_\_\_\_\_\_\_\_\_\_\_\_\_\_\_\_\_\_\_\_\_\_\_\_\_\_\_\_\_\_\_\_\_\_\_\_\_\_\_\_\_\_\_\_\_\_\_\_\_\_\_\_\_\_\_\_\_\_\_\_\_\_\_\_\_\_\_\_\_\_\_\_\_\_\_\_\_\_\_\_\_\_\_\_\_\_\_\_\_\_\_\_\_\_\_\_\_\_\_\_\_\_\_\_\_\_\_\_\_\_\_\_\_\_\_\_\_\_\_\_\_\_\_\_\_\_\_\_\_\_\_\_\_\_\_\_\_\_\_\_\_\_\_\_\_\_\_\_\_\_\_\_\_\_\_\_\_\_\_\_\_\_\_\_\_\_\_\_\_\_\_\_\_\_\_\_\_\_\_\_\_\_\_\_\_\_\_\_\_\_\_\_\_\_\_\_\_\_\_\_\_\_\_\_\_\_\_\_\_\_\_\_\_\_\_\_\_\_\_\_\_\_\_\_\_\_\_\_\_\_\_\_\_\_\_\_\_\_\_\_\_\_\_\_\_\_\_\_\_\_\_\_\_\_\_\_\_\_\_\_\_\_\_\_\_\_\_\_\_\_\_\_\_\_\_\_\_\_\_\_\_\_\_\_\_\_\_\_\_\_\_\_\_\_\_\_\_\_\_\_\_\_\_\_\_\_\_\_\_\_\_\_\_\_\_\_\_\_\_\_\_\_\_\_\_\_\_\_\_\_\_\_\_\_\_\_\_\_\_\_\_\_\_\_\_\_\_\_\_\_\_\_\_\_\_\_\_\_\_\_\_\_\_\_\_\_\_\_\_\_\_\_\_\_\_\_\_\_\_\_\_\_\_\_\_\_\_\_\_\_\_\_\_\_\_\_\_\_\_\_\_\_\_\_\_\_\_\_\_\_\_\_\_\_\_\_\_\_\_\_\_\_\_\_\_\_\_\_\_\_\_\_\_\_\_\_\_\_\_\_\_\_\_\_\_\_\_\_\_\_\_\_\_\_\_\_\_\_\_\_\_\_\_\_\_\_\_\_\_\_\_\_\_\_\_\_\_\_\_\_\_\_\_\_\_\_\_\_\_\_\_\_\_\_\_\_\_\_\_\_\_\_\_\_\_\_\_\_\_\_\_\_\_\_\_\_\_\_\_\_\_\_\_\_\_\_\_\_\_\_\_\_\_\_\_\_\_\_\_\_\_\_\_\_\_\_\_\_\_\_\_\_\_\_\_\_\_\_\_\_\_\_\_\_\_\_\_\_\_\_\_\_\_\_\_\_\_\_\_\_\_\_\_\_\_\_\_\_\_\_\_\_\_\_\_\_\_\_\_\_\_\_\_\_\_\_\_\_\_\_\_\_\_\_\_\_\_\_\_\_\_\_\_\_\_\_\_\_\_\_\_\_\_\_\_\_\_\_\_\_\_\_\_\_\_\_\_\_\_\_\_\_\_\_\_\_\_\_\_\_\_\_\_\_\_\_\_\_\_\_\_\_\_\_\_\_\_\_\_\_\_\_\_\_\_\_\_\_\_\_\_\_\_\_\_\_\_\_\_\_\_\_\_\_\_\_\_\_\_\_\_\_\_\_\_\_\_\_\_\_\_\_\_\_\_\_\_\_\_\_\_\_\_\_\_\_\_\_\_\_\_\_\_\_\_\_\_\_\_\_\_\_\_\_\_\_\_\_\_\_\_\_\_\_\_\_\_\_\_\_\_\_\_\_\_\_\_\_\_\_\_\_\_\_\_\_\_\_\_\_\_\_\_\_\_\_\_\_\_\_\_\_\_\_\_\_\_\_\_\_\_\_\_\_\_\_\_\_\_\_\_\_\_\_\_\_\_\_\_\_\_\_\_\_\_\_\_\_\_\_\_\_\_\_\_\_\_\_\_\_\_\_\_\_\_\_\_\_\_\_\_\_\_\_\_\_\_\_\_\_\_\_\_\_\_\_\_\_\_\_\_\_\_\_\_\_\_\_\_\_\_\_\_\_\_\_\_\_\_\_\_\_\_\_\_\_\_\_\_\_\_\_\_\_\_\_\_\_\_\_\_\_\_\_\_\_\_\_\_\_\_\_\_\_\_\_\_\_\_\_\_\_\_\_\_\_\_\_\_\_\_\_\_\_\_\_\_\_\_\_\_\_\_\_\_\_\_\_\_\_\_\_\_\_\_\_\_\_\_\_\_\_\_\_\_\_\_\_\_\_\_\_\_\_\_\_\_\_\_\_\_\_\_\_\_\_\_\_\_\_\_\_\_\_\_\_\_\_\_\_\_\_\_\_\_\_\_\_\_\_\_\_\_\_\_\_\_\_\_\_\_\_\_\_\_\_\_\_\_\_\_\_\_\_\_\_\_\_\_\_\_\_\_\_\_\_\_\_\_\_\_\_\_\_\_\_\_\_\_\_\_\_\_\_\_\_\_\_\_\_\_\_\_\_\_\_\_\_\_\_\_\_\_\_\_\_\_\_\_\_\_\_\_\_\_\_\_\_\_\_\_\_\_\_\_\_\_\_\_\_\_\_\_\_\_\_\_\_\_\_\_\_\_\_\_\_\_\_\_\_\_\_\_\_\_\_\_\_\_\_\_\_\_\_\_\_\_\_\_\_\_\_\_\_\_\_\_\_\_\_\_\_\_\_\_\_\_\_\_\_\_\_\_\_\_\_\_\_\_\_\_\_\_\_\_\_\_\_\_\_\_\_\_\_\_\_\_\_\_\_\_\_\_\_\_\_\_\_\_\_\_\_\_\_\_\_\_\_\_\_\_\_\_\_\_\_\_\_\_\_\_\_\_\_\_\_\_\_\_\_\_\_\_\_\_\_\_\_\_\_\_\_\_\_\_\_\_\_\_\_\_\_\_\_\_\_\_\_\_\_\_\_\_\_\_\_\_\_\_\_\_\_\_\_\_\_\_\_\_\_\_\_\_\_\_\_\_\_\_\_\_\_\_\_\_\_\_\_\_\_\_\_\_\_\_\_\_\_\_\_\_\_\_\_\_\_\_\_\_\_\_\_\_\_\_\_\_\_\_\_\_\_\_\_\_\_\_\_\_\_\_\_\_\_\_\_\_\_\_\_\_\_\_\_\_\_\_\_\_\_\_\_\_\_\_\_\_\_\_\_\_\_\_\_\_\_\_\_\_\_\_\_\_\_\_\_\_\_\_\_\_\_\_\_\_\_\_\_\_\_\_\_\_\_\_\_\_\_\_\_\_\_\_\_\_\_\_\_\_\_\_\_\_\_\_\_\_\_\_\_\_\_\_\_\_\_\_\_\_\_\_\_\_\_\_\_\_\_\_\_\_\_\_\_\_\_\_\_\_\_\_\_\_\_\_\_\_\_\_\_\_\_\_\_\_\_\_\_\_\_\_\_\_\_\_\_\_\_\_\_\_\_\_\_\_\_\_\_\_\_\_\_\_\_\_\_\_\_\_\_\_\_\_\_\_\_\_\_\_\_\_\_\_\_\_\_\_\_\_\_\_\_\_\_\_\_\_\_\_\_\_\_\_\_\_\_\_\_\_\_\_\_\_\_\_\_\_\_\_\_\_\_\_\_\_\_\_\_\_\_\_\_\_\_\_\_\_\_\_\_\_\_\_\_\_\_\_\_\_\_\_\_\_\_\_\_\_\_\_\_\_\_\_\_\_\_\_\_\_\_\_\_\_\_\_\_\_\_\_\_\_\_\_\_\_\_\_\_\_\_\_\_\_\_\_\_\_\_\_\_\_\_\_\_\_\_\_\_\_\_\_\_\_\_\_\_\_\_\_\_\_\_\_\_\_\_\_\_\_\_\_\_\_\_\_\_\_\_\_\_\_\_\_\_\_\_\_\_\_\_\_\_\_\_\_\_\_\_\_\_\_\_\_\_\_\_\_\_\_\_\_\_\_\_\_\_\_\_\_\_\_\_\_\_\_\_\_\_\_\_\_\_\_\_\_\_\_\_\_\_\_\_\_\_\_\_\_\_\_\_\_\_\_\_\_\_\_\_\_\_\_\_\_\_\_\_\_\_\_\_\_\_\_\_\_\_\_\_\_\_\_\_\_\_\_\_\_\_\_\_\_\_\_\_\_\_\_\_\_\_\_\_\_\_\_\_\_\_\_\_\_\_\_\_\_\_\_\_\_\_\_\_\_\_\_\_\_\_\_\_\_\_\_\_\_\_\_\_\_\_\_\_\_\_\_\_\_\_\_\_\_\_\_\_\_\_\_\_\_\_\_\_\_\_\_\_\_\_\_\_\_\_\_\_\_\_\_\_\_\_\_\_\_\_\_\_\_\_\_\_\_\_\_\_\_\_\_\_\_\_\_\_\_\_\_\_\_\_\_\_\_\_\_\_\_\_\_\_\_\_\_\_\_\_\_\_\_\_\_\_\_\_\_\_\_\_\_\_\_\_\_\_\_\_\_\_\_\_\_\_\_\_\_\_\_\_\_\_\_\_\_\_\_\_\_\_\_\_\_\_\_\_\_\_\_\_\_\_\_\_\_\_\_\_\_\_\_\_\_\_\_\_\_\_\_\_\_\_\_\_\_\_\_\_\_\_\_\_\_\_\_\_\_\_\_\_\_\_\_\_\_\_\_\_\_\_\_\_\_\_\_\_\_\_\_\_\_\_\_\_\_\_\_\_\_\_\_\_\_\_\_\_\_\_\_\_\_\_\_\_\_\_\_\_\_\_\_\_\_\_\_\_\_\_\_\_\_\_\_\_\_\_\_\_\_\_\_\_\_\_\_\_\_\_\_\_\_\_\_\_\_\_\_\_\_\_\_\_\_\_\_\_\_\_\_\_\_\_\_\_\_\_\_\_\_\_\_\_\_\_\_\_\_\_\_\_\_\_\_\_\_\_\_\_\_\_\_\_\_\_\_\_\_\_\_\_\_\_\_\_\_\_\_\_\_\_\_\_\_\_\_\_\_\_\_\_\_\_\_\_\_\_\_\_\_\_\_\_\_\_\_\_\_\_\_\_\_\_\_\_\_\_\_\_\_\_\_\_\_\_\_\_\_\_\_\_\_\_\_\_\_\_\_\_\_\_\_\_\_\_\_\_\_\_\_\_\_\_\_\_\_\_\_\_\_\_\_\_\_\_\_\_\_\_\_\_\_\_\_\_\_\_\_\_\_\_\_\_\_\_\_\_\_\_\_\_\_\_\_\_\_\_\_\_\_\_\_\_\_\_\_\_\_\_\_\_\_\_\_\_\_\_\_\_\_\_\_\_\_\_\_\_\_\_\_\_\_\_\_\_\_\_\_\_\_\_\_\_\_\_\_\_\_\_\_\_\_\_\_\_\_\_\_\_\_\_\_\_\_\_\_\_\_\_\_\_\_\_\_\_\_\_\_\_\_\_\_\_\_\_\_\_\_\_\_\_\_\_\_\_\_\_\_\_\_\_\_\_\_\_\_\_\_\_\_\_\_\_\_\_\_\_\_\_\_\_\_\_\_\_\_\_\_\_\_\_\_\_\_\_\_\_\_\_\_\_\_\_\_\_\_\_\_\_\_\_\_\_\_\_\_\_\_\_\_\_\_\_\_\_\_\_\_\_\_\_\_\_\_\_\_\_\_\_\_\_\_\_\_\_\_\_\_\_\_\_\_\_\_\_\_\_\_\_\_\_\_\_\_\_\_\_\_\_\_\_\_\_\_\_\_\_\_\_\_\_\_\_\_\_\_\_\_\_\_\_\_\_\_\_\_\_\_\_\_\_\_\_\_\_\_\_\_\_\_\_\_\_\_\_\_\_\_\_\_\_\_\_\_\_\_\_\_\_\_\_\_\_\_\_\_\_\_\_\_\_\_\_\_\_\_\_\_\_\_\_\_\_\_\_\_\_\_\_\_\_\_\_\_\_\_\_\_\_\_\_\_\_\_\_\_\_\_\_\_\_\_\_\_\_\_\_\_\_\_\_\_\_\_\_\_\_\_\_\_\_\_\_\_\_\_\_\_\_\_\_\_\_\_\_\_\_\_\_\_\_\_\_\_\_\_\_\_\_\_\_\_\_\_\_\_\_\_\_\_\_\_\_\_\_\_\_\_\_\_\_\_\_\_\_\_\_\_\_\_\_\_\_\_\_\_\_\_\_\_\_\_\_\_\_\_\_\_\_\_\_\_\_\_\_\_\_\_\_\_\_\_\_\_\_\_\_\_\_\_\_\_\_\_\_\_\_\_\_\_\_\_\_\_\_\_\_\_\_\_\_\_\_\_\_\_\_\_\_\_\_\_\_\_\_\_\_\_\_\_\_\_\_\_\_\_\_\_\_\_\_\_\_\_\_\_\_\_\_\_\_\_\_\_\_\_\_\_\_\_\_\_\_\_\_\_\_\_\_\_\_\_\_\_\_\_\_\_\_\_\_\_\_\_\_\_\_\_\_\_\_\_\_\_\_\_\_\_\_\_\_\_\_\_\_\_\_\_\_\_\_\_\_\_\_\_\_\_\_\_\_\_\_\_\_\_\_\_\_\_\_\_\_\_\_\_\_\_\_\_\_\_\_\_\_\_\_\_\_\_\_\_\_\_\_\_\_\_\_\_\_\_\_\_\_\_\_\_\_\_\_\_\_\_\_\_\_\_\_\_\_\_\_\_\_\_\_\_\_\_\_\_\_\_\_\_\_\_\_\_\_\_\_\_\_\_\_\_\_\_\_\_\_\_\_\_\_\_\_\_\_\_\_\_\_\_\_\_\_\_\_\_\_\_\_\_\_\_\_\_\_\_\_\_\_\_\_\_\_\_\_\_\_\_\_\_\_\_\_\_\_\_\_\_\_\_\_\_\_\_\_\_\_\_\_\_\_\_\_\_\_\_\_\_\_\_\_\_\_\_\_\_\_\_\_\_\_\_\_\_\_\_\_\_\_\_\_\_\_\_\_\_\_\_\_\_\_\_\_\_\_\_\_\_\_\_\_\_\_\_\_\_\_\_\_\_\_\_\_\_\_\_\_\_\_\_\_\_\_\_\_\_\_\_\_\_\_\_\_\_\_\_\_\_\_\_\_\_\_\_\_\_\_\_\_\_\_\_\_\_\_\_\_\_\_\_\_\_\_\_\_\_\_\_\_\_\_\_\_\_\_\_\_\_\_\_\_\_\_\_\_\_\_\_\_\_\_\_\_\_\_\_\_\_\_\_\_\_\_\_\_\_\_\_\_\_\_\_\_\_\_\_\_\_\_\_\_\_\_\_\_\_\_\_\_\_\_\_\_\_\_\_\_\_\_\_\_\_\_\_\_\_\_\_\_\_\_\_\_\_\_\_\_\_\_\_\_\_\_\_\_\_\_\_\_\_\_\_\_\_\_\_\_\_\_\_\_\_\_\_\_\_\_\_\_\_\_\_\_\_\_\_\_\_\_\_\_\_\_\_\_\_\_\_\_\_\_\_\_\_\_\_\_\_\_\_\_\_\_\_\_\_\_\_\_\_\_\_\_\_\_\_\_\_\_\_\_\_\_\_\_\_\_\_\_\_\_\_\_\_\_\_\_\_\_\_\_\_\_\_\_\_\_\_\_\_\_\_\_\_\_\_\_\_\_\_\_\_\_\_\_\_\_\_\_\_\_\_\_\_\_\_\_\_\_\_\_\_\_\_\_\_\_\_\_\_\_\_\_\_\_\_\_\_\_\_\_\_\_\_\_\_\_\_\_\_\_\_\_\_\_\_\_\_\_\_\_\_\_\_\_\_\_\_\_\_\_\_\_\_\_\_\_\_\_\_\_\_\_\_\_\_\_\_\_\_\_\_\_\_\_\_\_\_\_\_\_\_\_\_\_\_\_\_\_\_\_\_\_\_\_\_\_\_\_\_\_\_\_\_\_\_\_\_\_\_\_\_\_\_\_\_\_\_\_\_\_\_\_\_\_\_\_\_\_\_\_\_\_\_\_\_\_\_\_\_\_\_\_\_\_\_\_\_\_\_\_\_\_\_\_\_\_\_\_\_\_\_\_\_\_\_\_\_\_\_\_\_\_\_\_\_\_\_\_\_\_\_\_\_\_\_\_\_\_\_\_\_\_\_\_\_\_\_\_\_\_\_\_\_\_\_\_\_\_\_\_\_\_\_\_\_\_\_\_\_\_\_\_\_\_\_\_\_\_\_\_\_\_\_\_\_\_\_\_\_\_\_\_\_\_\_\_\_\_\_\_\_\_\_\_\_\_\_\_\_\_\_\_\_\_\_\_\_\_\_\_\_\_\_\_\_\_\_\_\_\_\_\_\_\_\_\_\_\_\_\_\_\_\_\_\_\_\_\_\_\_\_\_\_\_\_\_\_\_\_\_\_\_\_\_\_\_\_\_\_\_\_\_\_\_\_\_\_\_\_\_\_\_\_\_\_\_\_\_\_\_\_\_\_\_\_\_\_\_\_\_\_\_\_\_\_\_\_\_\_\_\_\_\_\_\_\_\_\_\_\_\_\_\_\_\_\_\_\_\_\_\_\_\_\_\_\_\_\_\_\_\_\_\_\_\_\_\_\_\_\_\_\_\_\_\_\_\_\_\_\_\_\_\_\_\_\_\_\_\_\_\_\_\_\_\_\_\_\_\_\_\_\_\_\_\_\_\_\_\_\_\_\_\_\_\_\_\_\_\_\_\_\_\_\_\_\_\_\_\_\_\_\_\_\_\_\_\_\_\_\_\_\_\_\_\_\_\_\_\_\_\_\_\_\_\_\_\_\_\_\_\_\_\_\_\_\_\_\_\_\_\_\_\_\_\_\_\_\_\_\_\_\_\_\_\_\_\_\_\_\_\_\_\_\_\_\_\_\_\_\_\_\_\_\_\_\_\_\_\_\_\_\_\_\_\_\_\_\_\_\_\_\_\_\_\_\_\_\_\_\_\_\_\_\_\_\_\_\_\_\_\_\_\_\_\_\_\_\_\_\_\_\_\_\_\_\_\_\_\_\_\_\_\_\_\_\_\_\_\_\_\_\_\_\_\_\_\_\_\_\_\_\_\_\_\_\_\_\_\_\_\_\_\_\_\_\_\_\_\_\_\_\_\_\_\_\_\_\_\_\_\_\_\_\_\_\_\_\_\_\_\_\_\_\_\_\_\_\_\_\_\_\_\_\_\_\_\_\_\_\_\_\_\_\_\_\_\_\_\_\_\_\_\_\_\_\_\_\_\_\_\_\_\_\_\_\_\_\_\_\_\_\_\_\_\_\_\_\_\_\_\_\_\_\_\_\_\_\_\_\_\_\_\_\_\_\_\_\_\_\_\_\_\_\_\_\_\_\_\_\_\_\_\_\_\_\_\_\_\_\_\_\_\_\_\_\_\_\_\_\_\_\_\_\_\_\_\_\_\_\_\_\_\_\_\_\_\_\_\_\_\_\_\_\_\_\_\_\_\_\_\_\_\_\_\_\_\_\_\_\_\_\_\_\_\_\_\_\_\_\_\_\_\_\_\_\_\_\_\_\_\_\_\_\_\_\_\_\_\_\_\_\_\_\_\_\_\_\_\_\_\_\_\_\_\_\_\_\_\_\_\_\_\_\_\_\_\_\_\_\_\_\_\_\_\_\_\_\_\_\_\_\_\_\_\_\_\_\_\_\_\_\_\_\_\_\_\_\_\_\_\_\_\_\_\_\_\_\_\_\_\_\_\_\_\_\_\_\_\_\_\_\_\_\_\_\_\_\_\_\_\_\_\_\_\_\_\_\_\_\_\_\_\_\_\_\_\_\_\_\_\_\_\_\_\_\_\_\_\_\_\_\_\_\_\_\_\_\_\_\_\_\_\_\_\_\_\_\_\_\_\_\_\_\_\_\_\_\_\_\_\_\_\_\_\_\_\_\_\_\_\_\_\_\_\_\_\_\_\_\_\_\_\_\_\_\_\_\_\_\_\_\_\_\_\_\_\_\_\_\_\_\_\_\_\_\_\_\_\_\_\_\_\_\_\_\_\_\_\_\_\_\_\_\_\_\_\_\_\_\_\_\_\_\_\_\_\_\_\_\_\_\_\_\_\_\_\_\_\_\_\_\_\_\_\_\_\_\_\_\_\_\_\_\_\_\_\_\_\_\_\_\_\_\_\_\_\_\_\_\_\_\_\_\_\_\_\_\_\_\_\_\_\_\_\_\_\_\_\_\_\_\_\_\_\_\_\_\_\_\_\_\_\_\_\_\_\_\_\_\_\_\_\_\_\_\_\_\_\_\_\_\_\_\_\_\_\_\_\_\_\_\_\_\_\_\_\_\_\_\_\_\_\_\_\_\_\_\_\_\_\_\_\_\_\_\_\_\_\_\_\_\_\_\_\_\_\_\_\_\_\_\_\_\_\_\_\_\_\_\_\_\_\_\_\_\_\_\_\_\_\_\_\_\_\_\_\_\_\_\_\_\_\_\_\_\_\_\_\_\_\_\_\_\_\_\_\_\_\_\_\_\_\_\_\_\_\_\_\_\_\_\_\_\_\_\_\_\_\_\_\_\_\_\_\_\_\_\_\_\_\_\_\_\_\_\_\_\_\_\_\_\_\_\_\_\_\_\_\_\_\_\_\_\_\_\_\_\_\_\_\_\_\_\_\_\_\_\_\_\_\_\_\_\_\_\_\_\_\_\_\_\_\_\_\_\_\_\_\_\_\_\_\_\_\_\_\_\_\_\_\_\_\_\_\_\_\_\_\_\_\_\_\_\_\_\_\_\_\_\_\_\_\_\_\_\_\_\_\_\_\_\_\_\_\_\_\_\_\_\_\_\_\_\_\_\_\_\_\_\_\_\_\_\_\_\_\_\_\_\_\_\_\_\_\_\_\_\_\_\_\_\_\_\_\_\_\_\_\_\_\_\_\_\_\_\_\_\_\_\_\_\_\_\_\_\_\_\_\_\_\_\_\_\_\_\_\_\_\_\_\_\_\_\_\_\_\_\_\_\_\_\_\_\_\_\_\_\_\_\_\_\_\_\_\_\_\_\_\_\_\_\_\_\_\_\_\_\_\_\_\_\_\_\_\_\_\_\_\_\_\_\_\_\_\_\_\_\_\_\_\_\_\_\_\_\_\_\_\_\_\_\_\_\_\_\_\_\_\_\_\_\_\_\_\_\_\_\_\_\_\_\_\_\_\_\_\_\_\_\_\_\_\_\_\_\_\_\_\_\_\_\_\_\_\_\_\_\_\_\_\_\_\_\_\_\_\_\_\_\_\_\_\_\_\_\_\_\_\_\_\_\_\_\_\_\_\_\_\_\_\_\_\_\_\_\_\_\_\_\_\_\_\_\_\_\_\_\_\_\_\_\_\_\_\_\_\_\_\_\_\_\_\_\_\_\_\_\_\_\_\_\_\_\_\_\_\_\_\_\_\_\_\_\_\_\_\_\_\_\_\_\_\_\_\_\_\_\_\_\_\_\_\_\_\_\_\_\_\_\_\_\_\_\_\_\_\_\_\_\_\_\_\_\_\_\_\_\_\_\_\_\_\_\_\_\_\_\_\_\_\_\_\_\_\_\_\_\_\_\_\_\_\_\_\_\_\_\_\_\_\_\_\_\_\_\_\_\_\_\_\_\_\_\_\_\_\_\_\_\_\_\_\_\_\_\_\_\_\_\_\_\_\_\_\_\_\_\_\_\_\_\_\_\_\_\_\_\_\_\_\_\_\_\_\_\_\_\_\_\_\_\_\_\_\_\_\_\_\_\_\_\_\_\_\_\_\_\_\_\_\_\_\_\_\_\_\_\_\_\_\_\_\_\_\_\_\_\_\_\_\_\_\_\_\_\_\_\_\_\_\_\_\_\_\_\_\_\_\_\_\_\_\_\_\_\_\_\_\_\_\_\_\_\_\_\_\_\_\_\_\_\_\_\_\_\_\_\_\_\_\_\_\_\_\_\_\_\_\_\_\_\_\_\_\_\_\_\_\_\_\_\_\_\_\_\_\_\_\_\_\_\_\_\_\_\_\_\_\_\_\_\_\_\_\_\_\_\_\_\_\_\_\_\_\_\_\_\_\_\_\_\_\_\_\_\_\_\_\_\_\_\_\_\_\_\_\_\_\_\_\_\_\_\_\_\_\_\_\_\_\_\_\_\_\_\_\_\_\_\_\_\_\_\_\_\_\_\_\_\_\_\_\_\_\_\_\_\_\_\_\_\_\_\_\_\_\_\_\_\_\_\_\_\_\_\_\_\_\_\_\_\_\_\_\_\_\_\_\_\_\_\_\_\_\_\_\_\_\_\_\_\_\_\_\_\_\_\_\_\_\_\_\_\_\_\_\_\_\_\_\_\_\_\_\_\_\_\_\_\_\_\_\_\_\_\_\_\_\_\_\_\_\_\_\_\_\_\_\_\_\_\_\_\_\_\_\_\_\_\_\_\_\_\_\_\_\_\_\_\_\_\_\_\_\_\_\_\_\_\_\_\_\_\_\_\_\_\_\_\_\_\_\_\_\_\_\_\_\_\_\_\_\_\_\_\_\_\_\_\_\_\_\_\_\_\_\_\_\_\_\_\_\_\_\_\_\_\_\_\_\_\_\_\_\_\_\_\_\_\_\_\_\_\_\_\_\_\_\_\_\_\_\_\_\_\_\_\_\_\_\_\_\_\_\_\_\_\_\_\_\_\_\_\_\_\_\_\_\_\_\_\_\_\_\_\_\_\_\_\_\_\_\_\_\_\_\_\_\_\_\_\_\_\_\_\_\_\_\_\_\_\_\_\_\_\_\_\_\_\_\_\_\_\_\_\_\_\_\_\_\_\_\_\_\_\_\_\_\_\_\_\_\_\_\_\_\_\_\_\_\_\_\_\_\_\_\_\_\_\_\_\_\_\_\_\_\_\_\_\_\_\_\_\_\_\_\_\_\_\_\_\_\_\_\_\_\_\_\_\_\_\_\_\_\_\_\_\_\_\_\_\_\_\_\_\_\_\_\_\_\_\_\_\_\_\_\_\_\_\_\_\_\_\_\_\_\_\_\_\_\_\_\_\_\_\_\_\_\_\_\_\_\_\_\_\_\_\_\_\_\_\_\_\_\_\_\_\_\_\_\_\_\_\_\_\_\_\_\_\_\_\_\_\_\_\_\_\_\_\_\_\_\_\_\_\_\_\_\_\_\_\_\_\_\_\_\_\_\_\_\_\_\_\_\_\_\_\_\_\_\_\_\_\_\_\_\_\_\_\_\_\_\_\_\_\_\_\_\_\_\_\_\_\_\_\_\_\_\_\_\_\_\_\_\_\_\_\_\_\_\_\_\_\_\_\_\_\_\_\_\_\_\_\_\_\_\_\_\_\_\_\_\_\_\_\_\_\_\_\_\_\_\_\_\_\_\_\_\_\_\_\_\_\_\_\_\_\_\_\_\_\_\_\_\_\_\_\_\_\_\_\_\_\_\_\_\_\_\_\_\_\_\_\_\_\_\_\_\_\_\_\_\_\_\_\_\_\_\_\_\_\_\_\_\_\_\_\_\_\_\_\_\_\_\_\_\_\_\_\_\_\_\_\_\_\_\_\_\_\_\_\_\_\_\_\_\_\_\_\_\_\_\_\_\_\_\_\_\_\_\_\_\_\_\_\_\_\_\_\_\_\_\_\_\_\_\_\_\_\_\_\_\_\_\_\_\_\_\_\_\_\_\_\_\_\_\_\_\_\_\_\_\_\_\_\_\_\_\_\_\_\_\_\_\_\_\_\_\_\_\_\_\_\_\_\_\_\_\_\_\_\_\_\_\_\_\_\_\_\_\_\_\_\_\_\_\_\_\_\_\_\_\_\_\_\_\_\_\

\section*{全国第八届研究生数学建模竞赛}

\begin{center}
\includegraphics[width=0.5\textwidth]{logo.png}
\end{center}

\section*{题目}

吸波材料与微波暗室吸波性能研究

\section*{摘要:}

对于问题一,本文首先建立了二维情况下入射波线在尖劈几何空缺间反射过程的数学模型,提出了入射波在尖劈吸波体内反射的临界状态的概念。根据光学原理,三角几何原理推出了辐射波在各个反射点处的入射角公式及反射点距尖劈底面高度的公式,求出了最终反射波的方向、辐射强度和反射点的位置、反射次数、与已知反射率、诸几何参数之间的定量关系。最后根据坐标分解法将其拓展到三维空间。

对于问题二,本文综合考虑辐射波在六个面间的相互影响,运用微元法的思想和微积分的相关知识,推导出静区直接接收到微波信号的公式以及微波信号经整个微波暗室内壁一次反射后被静区接收到的微波功率公式,用数值积分的方法求得结果,并用 Matlab 软件画图分析。对于部分反射光难以计算的问题提出了数学模型修正的建议和方法。该模型初步解决了暗室能否满足仿真技术要求的问题,并运用图表直观地表达了微波暗室的性能。最终得出了静区从诸墙面得到的反射信号的功率与之静区从信号源直接得到的微波功率之比 \(\gamma\),在 \(\rho=0.05\) 时微波暗室不能满足仿真技术要求,且当 \(t=2.2s\) 时 \(\gamma\) 取得最小值 0.039。在 \(\rho=0.05\) 时微波暗室能够满足仿真要求,且当 \(t=2.2s\) 时 \(\gamma\) 取得最小值 0.004。

\section*{关键词:临界状态 三角几何原理 微元法 数值积分}

\section{问题重述}

新型隐身歼击机歼-20 最近试飞成功,标志着我国在隐身技术领域取得了重大进展。隐身技术的基础研究包括探索不同频段上吸波的机理,研制高效吸波的特殊材料,将吸波材料设计成合理的形状使之发挥最大效能等。近代各种干扰已无法全部避免,所以其测试与仿真,放置在被称为“无回波暗室”的实验室中进行。

无回波暗室能够屏蔽外界干扰信号,通过内墙敷设的吸波体,吸收各类反射信号,使室内反射大为减弱,被测设备接收到的“似乎”只有测试信号源发出的实验所需信号。这样,它为测试设备提供了一个几乎没有反射信号的“自由空间”。

自主寻的式导弹的制导系统的核心设备之一是安置在头部、能自动寻找和跟踪目标的导引头。在导弹的研制过程中需要在地面条件下模拟导引头跟踪目标的性能。设导引头的工作波段在微波段。一种已经研究成功的仿真系统主要由目标模拟器系统,作为导引头支架的三轴转台和微波暗室组成。

目标模拟器用来模拟目标运动,它由天线阵列子系统及其控制子系统组成。天线阵列是安置在微波暗室靠近一面墙、有规律排列在同一球面的若干个微波天线,各天线的中心轴线对准球心,按某种规律依次发射模拟目标回波的微波信号,模拟自由空间中目标相对于导弹的运动。需要测试的导引头安装在三轴转台上,转台根据导引头跟踪目标时发出的制导指令作三自由度角度的转动,带动导引头模拟导弹在空间的三自由度运动。微波暗室提供一个微波“自由空间”。

为了提高无回波暗室的吸波性能,可采用尖劈形状的吸波体,大量尖劈有规律地排列组成的整体粘贴在墙上构成吸波体。它们能使得辐射波射入尖劈吸波体后在尖形的几何空缺间形成多次反射和透射-反射,降低反射出去的能量,实现高效率吸波。

微波暗室所有墙面铺设同一规格的吸波体。目标模拟器是圆弧形线阵列,而非球面阵列,它安装在靠近一面墙的中心水平面内,圆弧线对两边的墙处于对称位置。暗室右端中心的小方块面积处是安置导引头的部位,称为“静区”。静区小方块的中心点与目标模拟阵列圆弧的圆心重合。静区接收到的电磁能量直接对导弹的导引仿真有重要影响,根据导引仿真要求,静区从诸墙面得到的反射信号的功率之和与从信号源直接得到的微波功率之比 \(\gamma\),始终满足 \(\gamma \leq 0.03\)。

本文拟解决如下问题

1. 试建立入射辐射波在一个尖劈几何空缺间反射过程的数学模型,即分别刻画最终反射辐射波的方向,反射次数,反射波的辐射强度与已知反射率、诸几何参数之间的定量关系。

2. 若暗室铺设平板形吸波材料,其垂直反射率 \(\rho = 0.50\)。试建立合适的数学模型,在上述假设下,根据提供的数据,通过对模型的分析与数值计算,判断这样的微波暗室能否能满足仿真技术要求?在此弹目相对运动过程中,何时的 \(\gamma\) 值最小?并进一步分析,若暗室改为铺设尖劈形吸波材料,尖劈形吸波体各平面处的平均反射率为 \(\rho = 0.05\)(相当于尖劈形吸波体被换成另一种吸波性能更好材料的平板形吸波体的垂直反射率),再次用模型进行计算,根据结果判断,这样的暗室是否能满足仿真技术要求?何时的 \(\gamma\) 值最小?

\section{问题分析}

本文主要解决入射波在尖劈吸波材料几何空缺间反射过程的问题以及微波暗室性能是否满足导引仿真要求的问题。

问题一是入射波在尖劈几何空缺间的传播反射问题,因波在三维空间的传播过程比较复杂,我们作如下处理,首先将复杂问题简单化,研究入射波在二维空间的传播过程,建立入射波在二维空间的传播反射过程的数学模型。通过分析入射波在传播过程中的特性,知波线最终一定会射出尖劈吸波体。利用几何光学原理,三角形边角关系,求出每一次反射打在尖劈上的反射角及反射点到尖劈底面的距离,观察规律,并用数学归纳法论证了第 $k$ 次反射点处的反射角及其距尖劈底面的距离与反射次数、尖劈角、尖劈高、尖劈底面宽度等参数之间的关系。波在尖劈几何空缺间的反射过程,分为上行线路和下行线路,本文定义当反射波线从向水平面下方反射转变到向水平面上方反射的转变状态称为反射过程的上下行线路转变的临界状态,由几何光学知识知,若第 $k$ 次反射时为临界状态,则波最终射出尖劈体时所经历的反射次数一定为 $2k-1$ 或 $2k$ 次。如何判断是 $2k-1$ 次还是 $2k$ 次是本文要重点考虑的问题,根据第 $2k-1$ 和 $2k$ 反射时的反射角与劈尖两斜边与水平面夹角的大小关系,以及第 $2k-1$ 次反射线与其所对尖劈边或尖劈边延长线的交点到劈尖底部的高度与劈尖高的大小关系来判断是 $2k-1$ 次还是 $2k$ 次,同时利用三角形的相似性原理求出各个反射点在直角坐标系中的坐标值,进而可得到最终反射辐射波的方向,反射点位置,反射次数,反射波的辐射强度与已知反射率、诸几何参数之间的定量关系。最后运用坐标分解法把二维空间条件下的数学模型拓展到三维空间。

问题二是求静区直接从信号源接收到的微波功率和从诸墙面反射后接收到的微波功率之比的问题,静区接收到的微波功率就是指接收到的辐射通量,导引头视在目标为余弦辐射体,根据余弦辐射体的特性,知其向各个方向的辐射强度各不相同,而静区为一小方块,故求辐射源直接辐射到静区的辐射通量之时,应利用微元法进行处理,求出辐射源辐射到静区一微小面积元的辐射通量,然后用数值积分法对静区的整个区域进行积分,即可求得辐射源直接辐射到静区内的总通量。静区从诸墙面反射后接收到的微波功率的计算比较复杂,因为在微波暗室中,辐射源即视在目标向四周辐射电磁波,发出的辐射波到暗室六个面后每点又变成新的余弦辐射体,因此被辐射到的每个面上都有无穷多个新的余弦发射体。封闭微波暗室的六个面间反射微波相互作用,相互影响,辐射波要在静区内经历无数次的反射。因此我们不能单纯的考虑每一条辐射波经过了多少次反射并最终被静区吸收,需综合考虑信号源发射的微波信号经六个面反射后到达静区量。鉴于此,采用分割法和微元法进行求解,首先计算微波暗室墙面上一微小面积元接收到视在目标辐射的通量,求得此微小面积元表面的辐射照度,进而求出被墙壁吸收后的辐射出射度,在算此面积元辐射到静区内的辐射通量时又可把此微小面积元看作一个辐射点来处理,最后通过积分即可求得经反射后辐射到静区的总通量,在求解过程中根据对称性原理可简化相关计算。

\section*{三、模型假设}

1、假设尖劈材料的电性能参数各处均匀

2、假设本题反射率公式可简化为 $\rho(\alpha)=\rho\cos\alpha$

3、假设经两次以上投射进入相邻空间的反射能量可以忽略不计

4、假设只考虑波在两种不同介质处的反射,不考虑边缘处的绕射

5、假设视在天线发射功率强度分布满足余弦辐射体

6、假设只考虑所有墙面对辐射的反射,不计入墙面的散射

7、假设不计入模拟器的天线及其安装支架,以及导引头本身对辐射的影响

\section*{四、符号说明}

\begin{itemize}
    \item $\lambda_{i}$: 第 $i$ 次反射时的反射角
    \item $P_{i}$: 第 $i$ 次反射点
    \item $H_{i}$: 第 $i$ 次反射点距尖劈底面的高度
    \item $R$: 视在目标到 A 点的距离
    \item $\Phi$: 辐射通量
    \item $I$: 辐射强度
    \item $\omega$: 立体角
    \item $\gamma$: 静区从诸墙面得到的反射信号的功率之和与直接得到的微波功率之比
    \item $i$: 辐射方向与其法线方向的夹角
    \item $\theta$: 微元面积 $dS$ 的法线方向与 $d\omega$ 的轴线的夹角
\end{itemize}

\section{模型建立与求解}

\subsection{问题一模型的建立与求解}

\subsubsection{反射角计算}

辐射波是在三维空间内传播的,但三维空间情况比较复杂,本文采取从易到难,逐步深入的方法,先从二维空间着手分析波在尖劈空缺间传播过程,建立相应的数学模型,最后再将二维空间拓展到三维空间。

\begin{figure}[h]
    \centering
    \includegraphics[width=0.8\textwidth]{image.png}
    \caption{尖劈形吸波体吸收波功能示意图}
    \label{fig:1}
\end{figure}

如图 \ref{fig:1} 所示,一束在 $YOZ$ 面强度为 $I$ 辐射波以与 $Z$ 轴反向成 $\theta$ 角的方向向反射率为 $\rho$ 的尖劈射去,经过多次反射并最终射出尖劈吸波体。

波在尖劈吸波体中的传播是先向下反射后向上反射上,中间有一临界状态。为便于分析,本文逐一分析下行反射波路、上行反射波路及中间的临界状态。首先分析辐射波反射方向向下时的情况,为便于分析图 2 只画出了下行的反射波路。

\begin{figure}[h]
    \centering
    \includegraphics[width=0.8\textwidth]{image.png}
    \caption{辐射波在尖劈几何空缺间的下行传播过程}
\end{figure}

根据光学原理,可知辐射波的第一次反射角 $\lambda_{1}$ 满足关系式:
\begin{equation}
\angle AEB = \angle BAE + \lambda_{1} = 180^\circ - \angle EDB - \angle EBD
\tag{1}
\end{equation}
解得:
\begin{equation}
\lambda_{1} = 90^\circ - 2\alpha - \angle BAE = 90^\circ - \alpha - \theta
\tag{2}
\end{equation}
同理得第二次反射的反射角 $\lambda_{2}$
\begin{equation}
\lambda_{2} = \angle BCF = 90^\circ - \angle ACB = 90^\circ - 3\alpha - \theta
\tag{3}
\end{equation}
由式 (2) 和 (3) 知,$\lambda_{1}$、$\lambda_{2}$ 的关系如下:
\begin{equation}
\lambda_{1} - \lambda_{2} = 2\alpha
\tag{4}
\end{equation}
同理可求得第三次反射时的反射角 $\lambda_{3}$
\begin{equation}
\lambda_{3} = 90^\circ - 5\alpha - \theta
\tag{5}
\end{equation}
易得 $\lambda_{2}$、$\lambda_{3}$ 的关系式如下
\begin{equation}
\lambda_{2} - \lambda_{3} = 2\alpha
\tag{6}
\end{equation}
通过观察 $\lambda_{1}$、$\lambda_{2}$、$\lambda_{3}$ 关系,猜想 $\lambda_{i-1}$、$\lambda_{i} (i \geq 2)$ 之间满足关系式
\begin{equation}
\lambda_{i} - \lambda_{i-1} = 2\alpha \, (i \leq k)
\tag{7}
\end{equation}
即有
\begin{equation}
\lambda_{i} = 90^\circ - (2i-1)\alpha - \theta
\tag{8}
\end{equation}
利用数学归纳法可以证明上述结论的正确性。从公式知,当辐射波下行时,$\lambda_{i}$ 是呈

递减之势,由于 \(\lambda_{i} \geq 0\),所以存在一个临界角 \(\lambda_{k} > 0\)、\(\lambda_{k+1} < 0\),即从 \(k\) 次反射后辐射波的反射方向从向下方传播变为向上方传播,则第 \(k\) 次的反射角 \(\lambda_{k+1}\) 不应再根据式 (8) 来求。

\begin{figure}[h]
    \centering
    \includegraphics[width=0.8\textwidth]{image1.png}
    \caption{辐射波在尖劈几何空缺间的传播上下行转化的临界状态}
    \label{fig:3}
\end{figure}

由图 3 可知,当辐射波到达临界状态时,临界时的反射角 \(\lambda_{k}\) 和下一个点的反射角 \(\lambda_{k+1}\) 满足如下关系:

\begin{equation}
\lambda_{k} + \lambda_{k+1} = 2\alpha
\tag{9}
\end{equation}

当辐射波开始向上反射时反射角为 \(\lambda_{i} \, (i \geq k+1)\),用分析下行辐射波的方法来分析上行辐射波,各边角关系如图 4(为便于分析只画出了辐射波上行反射路径)。

\begin{figure}[h]
    \centering
    \includegraphics[width=0.8\textwidth]{image2.png}
    \caption{辐射波在尖劈几何空缺间的上行传播过程}
    \label{fig:4}
\end{figure}

可得到如下关系

\begin{equation}
\angle DBC = 180^\circ - \angle BDC - \angle DCB = 90^\circ - 2\alpha - \lambda_{i}
\tag{10}
\end{equation}

同理可得

\begin{equation}
\lambda_{i+1} = \lambda_{i} + 2\alpha
\tag{11}
\end{equation}

结合上述临界条件和辐射波向下方反射时的反射角的表达式可以得到辐射波向上

反射时的反射角的表达式
\begin{equation}
\lambda_{i}=2\left(k-1\right)\alpha+\theta-90^{\circ}\left(i\geq k+1\right)
\tag{12}
\end{equation}
综上所述,辐射波在尖劈内的反射角满足如下关系。
\begin{equation}
\lambda_{i}=
\begin{cases}
90^{\circ}-\left(2i-1\right)\alpha-\theta\left(i\leq k\right) \\
2\left(i-1\right)\alpha+\theta-90^{\circ}\left(i\geq k+1\right)
\end{cases}
\tag{13}
\end{equation}
\textbf{4.1.2 反射点距离底面的高度计算}

设反射点为 $p_{i}$,反射点距离尖劈底端的高度为 $H_{i}$,过 $p_{i}$ 作与尖劈底边平行的直线与尖劈的另一条边的交点为 $p_{i}^{'}$。

\begin{figure}[h]
\centering
\includegraphics[width=0.8\textwidth]{image.png}
\caption{辐射波在尖劈几何空缺间的传播过程}
\end{figure}

第一个反射点时,三角形 $\Delta ABP_{1}$ 的三个内角为:
\begin{equation}
\angle ABP_{1}=90^{\circ}-\alpha
\tag{14}
\end{equation}
\begin{equation}
\angle P_{1}AB=90^{\circ}-\theta
\tag{15}
\end{equation}
\begin{equation}
\angle AP_{1}B=90^{\circ}-\lambda_{1}
\tag{16}
\end{equation}

由正弦定理可求得 $AP_{1}$
\begin{equation}
AP_{1}=\frac{d}{\sin\left(\angle AP_{1}B\right)}\cdot\sin\left(\angle ABP_{1}\right)=\frac{d\cos\alpha}{\cos\lambda_{1}}
\tag{17}
\end{equation}

在三角形 $\Delta APP_{1}^{'}$ 中 $\angle P_{1}^{'}AP_{1}=\theta-\alpha,\angle AP_{1}^{'}P_{1}=90^{\circ}+\alpha$,根据正弦定理则有
\begin{equation}
\frac{PP_{1}^{'}}{\sin\left(\angle P_{1}^{'}AP_{1}\right)}=\frac{AP_{1}}{\sin\left(\angle AP_{1}^{'}P_{1}\right)}=\frac{P_{1}P_{1}^{'}}{\sin\left(\theta-\alpha\right)}=\frac{AP_{1}}{\cos\alpha}
\tag{18}
\end{equation}

将式(17)带入式(18)化简得:
\begin{equation}
P_{1}P_{1}^{\prime}=\frac{d\sin(\theta-\alpha)}{\cos\lambda_{1}}
\tag{19}
\end{equation}
在大三角形 $\Delta ABC$ 由正弦定理可得:
\begin{equation}
H_{1}=\frac{h\sin(\theta-\alpha)}{\cos\lambda_{1}}
\tag{20}
\end{equation}

在第二个反射点时,在三角形 $\Delta P_{1}P_{1}^{\prime}P_{2}$ 有 $\angle P_{1}P_{1}^{\prime}P_{2}=90^{\circ}-\alpha$, $\angle P_{1}^{\prime}P_{2}P_{1}=90^{\circ}-\lambda_{2}$ 根据正弦定理有:
\begin{equation}
\frac{P_{1}P_{2}}{\sin(\angle P_{1}P_{1}^{\prime}P_{2})}=\frac{P_{1}P_{1}^{\prime}}{\sin\left(\angle P_{1}^{\prime}P_{2}P_{1}\right)}=\frac{P_{1}P_{2}}{\cos\alpha}=\frac{P_{1}P_{1}^{\prime}}{\cos\lambda_{2}}
\tag{21}
\end{equation}

在三角形 $\Delta P_{2}P_{2}^{\prime}P_{1}$ 中有 $\angle P_{2}P_{1}P_{2}^{\prime}=90^{\circ}-\lambda_{1}$, $\angle P_{2}P_{2}^{\prime}P_{1}=90^{\circ}+\alpha$ 根据正弦定理有:
\begin{equation}
\frac{P_{1}P_{2}}{\sin\left(\angle P_{2}P_{2}^{\prime}P_{1}\right)}=\frac{P_{2}P_{2}^{\prime}}{\sin(\angle P_{2}P_{1}P_{2}^{\prime})}=\frac{P_{1}P_{2}}{\cos(\alpha)}=\frac{P_{2}P_{2}^{\prime}}{\cos(\lambda_{1})}
\tag{22}
\end{equation}

将 $P_{1}P_{1}^{\prime}$、$P_{1}P_{2}$ 带入式(22)可得:
\begin{equation}
P_{2}P_{2}^{\prime}=\frac{d\sin(\theta-\alpha)}{\cos\lambda_{2}}
\tag{23}
\end{equation}
\begin{equation}
H_{2}=\frac{h\sin(\theta-\alpha)}{\cos\lambda_{2}}
\tag{24}
\end{equation}

同理可求得第三个反射点处 $P_{2}P_{2}^{\prime}$、$P_{2}P_{3}$
\begin{equation}
P_{3}P_{3}^{\prime}=\frac{d\sin(\theta-\alpha)}{\cos\lambda_{3}}
\tag{25}
\end{equation}
\begin{equation}
H_{3}=\frac{h\sin(\theta-\alpha)}{\cos\lambda_{3}}
\end{equation}

通过观察猜想有如下结论
\begin{equation}
P_{i}P_{i}^{\prime}=\frac{d\sin(\theta-\alpha)}{\cos\lambda_{i}}
\tag{26}
\end{equation}
\begin{equation}
H_{i}=\frac{h\sin(\theta-\alpha)}{\cos\lambda_{i}}
\end{equation}

经过数学归纳法证明,上面结论对于任一尖劈边上的反射点都成立。但单纯运用 $H_{i}$ 并不能判断在 $i$ 次反射后,辐射波是否能直接射出劈尖。为此可利用三角形的边角关系求出在第 $i$ 次反射后辐射波与尖劈另一边或另一边延长线交点距尖劈吸液体底面的高

度,通过比较此高度与劈尖高的大小来判断 \(i\) 次反射后辐射波是否能射出劈尖。

\begin{figure}[h]
    \centering
    \includegraphics[width=0.8\textwidth]{image.png} % 替换为实际图片路径
    \caption{第 \(i\) 次反射后的波线与尖劈边或其延长线的交点}
    \label{fig:6}
\end{figure}

当 \(i=2j-1\) 有

\begin{equation}
P_{2j-1}P_{2j-1}' = \frac{d\sin(\theta-a)}{\cos(\lambda_{2j-1})}
\tag{27}
\end{equation}

图 6 中 \(A'\) 为 \(P_{2j-1}\) 的反射辐射波与 \(OA\) 边的交点。则在三角形 \(\Delta A'P_{2j-1}P_{2j-1}'\) 中

\[
\angle A'P_{2j-1}P_{2j-1}' = 90^\circ + \alpha - \lambda_{2j-1}, \quad \angle P_{2j-1}A'P_{2j-1}' = 90^\circ - \lambda_{2j-1} - 2\alpha,
\]

根据正弦定理可以计算得到

\begin{equation}
A'P_{2j-1}' = \frac{P_{2j-1}P_{2j-1}'\cos(\alpha - \lambda_{2j-1})}{\cos(\lambda_{2j-1} + 2\alpha)} = \frac{d\sin(\theta-a)\cos(\alpha - \lambda_{2j-1})}{\cos(\lambda_{2j-1})\cos(\lambda_{2j-1} + 2\alpha)}
\tag{28}
\end{equation}

则 \(A'\) 点距尖劈底面的高度 \(H_{A'}\)

\begin{equation}
H_{A'} = A'P_{2j-1}' \times \sin(90 - \alpha) + H_{2j-1}
\tag{29}
\end{equation}

当 \(H_{A'} > h\) 时,辐射波射出尖劈几何空缺,当 \(H_{A'} < h\) 辐射波打在尖劈边上继续在尖劈几何空缺中传播。

\subsection{4.1.3 最终反射方向、反射次数、辐射强度的求解}

辐射波进入尖劈就一定会从尖劈中射出来,辐射波何时从尖劈中射出直接影响着辐射波在尖劈几何空缺中的反射次数。当反射角大于等于尖劈边与水平面的夹角时,辐射波即可从尖劈当中射出,此时反射角应满足

\begin{equation}
2(i-1)\alpha + \theta - 90^\circ \geq 90^\circ - 2\alpha
\tag{30}
\end{equation}

即:

\begin{equation}
i \geq \frac{180^\circ - \theta}{2\alpha}
\tag{31}
\end{equation}

当辐射波向下反射时,反射角是递减的且反射角是大于零,当达到临界状态时的反

射次数 \( k \) 为满足式 (32) 的最大整数 \( i \)

\begin{equation}
90^\circ - (2i-1)\alpha - \theta \geq 0
\tag{32}
\end{equation}

即到达临界状态时的反射次数 \( k \) 为小于 \( \frac{90^\circ - \theta - \alpha}{2\alpha} \) 的最大整数。

由光学几何原理知,当临界反射次数为 \( k \) 时,则辐射波最终射出尖劈时的反射次数只能为 \( 2k-1 \) 或 \( 2k \) 次。

在图 1 所示的 \( YOZ \) 直角坐标系中,当辐射波最终射出时,\( Z \) 轴方向上值为 \( H_i \),利用三角形相似性原理,可求得反射点的 \( Y \) 轴方向上坐标值与反射次数之间的关系为

\begin{equation}
y =
\begin{cases}
\frac{d}{2} - \frac{dH_i}{2h} & (i = 2k) \\
\frac{d}{2} + \frac{dH_i}{2h} & (i = 2k-1)
\end{cases}
\tag{33}
\end{equation}

故当辐射波最终反射出尖劈吸波体,反射点坐标与反射次数有关。当反射次数 \( i = 2k \) 时反射点坐标为 \( (H_i, \frac{d}{2} - \frac{dH_i}{2h}) \),当反射次数 \( i = 2k-1 \) 时反射点坐标为 \( (H_i, \frac{d}{2} + \frac{dH_i}{2h}) \)。

由相邻两次反射波辐射强度间的关系式 \( I_{i+1} = I_i \rho \cos(\lambda_{i+1}) \),可知最终反射波线的辐射强度仅与反射次数和各次入射角有关。可求得当临界状态的反射次数为 \( k \) 次时,最终微波反射出尖劈几何空缺时的辐射强度为

\begin{equation}
\begin{cases}
I_{2k-1} = I \prod_{i=1}^k \rho \sin((2i-1)\alpha + \theta) \prod_{i=k+1}^{2k-1} \rho \sin(2(i-1)\alpha + \theta) & \text{(反射总次数为 \( 2k-1 \))} \\
I_{2k} = I \prod_{i=1}^k \rho \sin((2i-1)\alpha + \theta) \prod_{i=k+1}^{2k} \rho \sin(2(i-1)\alpha + \theta) & \text{(反射总次数为 \( 2k \))}
\end{cases}
\tag{34}
\end{equation}

当临界反射次数为 \( k \) 时通过讨论可得到最终反射波的方向、反射点的位置、反射次数、反射波的辐射强度与已知反射率、诸几何参数之间的定量关系。结论如下:

当 \( 90^\circ - \lambda_{2k-1} \leq 2a \) 时,即 \( 180^\circ - 2(2k-1)\alpha \leq \theta \) 时,辐射波最终从尖劈空缺右侧边上点 \( (H_{2k-1}, \frac{d}{2} + \frac{dH_{2k-1}}{2h}) \) 处射出,总的反射次数为 \( 2k-1 \) 次,反射波的辐射强度为

\[
I_{\text{反}} = I \prod_{i=1}^k \rho \sin((2i-1)\alpha + \theta) \prod_{i=k+1}^{2k-1} \rho \sin(2(i-1)\alpha + \theta) .
\]

当 \( 90^\circ - \lambda_{2k-1} > 2a \),\( 90^\circ - \lambda_{2k} \leq 2a \) 时,即 \( 180^\circ - 4k\alpha \leq \theta < 180^\circ - 2(2k-1)a \) 时,辐射波最终从尖劈空缺左侧边上点 \( (H_{2k}, \frac{d}{2} + \frac{dH_{2k}}{2h}) \) 处射出,总的反射次数为 \( 2k \),反射波的辐射强度为

\[
I_{\text{反}} = I \prod_{i=1}^k \rho \sin((2i-1)\alpha + \theta) \prod_{i=k+1}^{2k} \rho \sin(2(i-1)\alpha + \theta) .
\]

当 \( 90 - \lambda_{2k-1} > 2a \),\( 90 - \lambda_{2k} > 2a \),\( H_A > h \) 时,即 \( 180^\circ - 4k\alpha > \theta \) 且 \( H_A > h \) 时,辐射

波最终从尖劈空缺右侧边上点 \((H_{2k-1}, \frac{d}{2} + \frac{dH_{2k-1}}{2h})\) 处射出,总的反射次数为 \(2k-1\),反射波的辐射强度为 \(I_{\text{反}} = I \prod_{i=1}^{k} \rho \sin((2i-1)\alpha + \theta) \prod_{i=k+1}^{2k-1} \rho \sin(2(i-1)\alpha + \theta)\)。

当 \(90 - \lambda_{2k-1} > 2a\),\(90 - \lambda_{2k} > 2a\),\(H_{A} < h\) 时,即 \(180^\circ - 4k\alpha > \theta\) 且 \(H_{A} < h\) 时,辐射波最终从尖劈空缺左侧边上点 \((H_{2k}, \frac{d}{2} + \frac{dH_{2k}}{2h})\) 处射出,总的反射次数为 \(2k\),反射波的辐射强度为 \(I_{\text{反}} = I \prod_{i=1}^{k} \rho \sin((2i-1)\alpha + \theta) \prod_{i=k+1}^{2k} \rho \sin(2(i-1)\alpha + \theta)\)。

\subsection*{4.1.4 三维空间的情况分析}

辐射波在三维空间中的传播时可将辐射波强度进行分解,分解到三坐标轴方向上

\begin{equation}
\begin{aligned}
I_x &= \sin(\theta) \cos(\varphi) \\
I_y &= \sin(\theta) \sin(\varphi) \\
I_z &= \cos(\theta)
\end{aligned}
\tag{35}
\end{equation}

由于尖劈沿 \(X\) 轴方向无限长,故波在入射到尖劈吸波材料上后,\(X\) 方向上的辐射能量会随 \(X\) 轴方向传递下去,而不会被吸波材料吸收和反射。反射后的辐射波在空间内的方位角 \(\varphi\) 应为 \(90^\circ\) 或 \(-90^\circ\)。

\textbf{YOZ 面上的初始辐射强度}

\begin{equation}
I_{yoz} = \sqrt{(\sin(\theta) \sin(\varphi))^2 + \cos(\theta)^2}
\tag{36}
\end{equation}

得到最终辐射波反射出尖劈吸波液体是几种情况如下

当 \(90^\circ - \lambda_{2k-1} \leq 2a\) 时,即 \(180^\circ - 2(2k-1)\alpha \leq \theta\) 时,辐射波最终从尖劈空缺右侧边上射出,总得反射次数为 \(2k-1\) 次,最终反射波线的方位角为 \(\varphi = -90^\circ\),射向角为 \(\lambda_{2k-1} + \alpha\),反射波的辐射强度为

\[
I_{\text{反}} = \sqrt{(\sin(\theta) \sin(\varphi))^2 + \cos(\theta)^2} \prod_{i=1}^{k} \rho \sin((2i-1)\alpha + \theta) \prod_{i=k+1}^{2k-1} \rho \sin(2(i-1)\alpha + \theta)。
\]

当 \(90^\circ - \lambda_{2k-1} > 2a\),\(90^\circ - \lambda_{2k} \leq 2a\) 时,即 \(180^\circ - 4k\alpha \leq \theta < 180^\circ - 2(2k-1)a\) 时,辐射波最终从尖劈空缺左侧边上射出,总的反射次数为 \(2k\),最终反射波线的方位角为 \(\varphi = 90^\circ\),射向角为 \(\lambda_{2k} + \alpha\),反射波的辐射强度为

\[
I_{\text{反}} = \sqrt{(\sin(\theta) \sin(\varphi))^2 + \cos(\theta)^2} \prod_{i=1}^{k} \rho \sin((2i-1)\alpha + \theta) \prod_{i=k+1}^{2k} \rho \sin(2(i-1)\alpha + \theta)。
\]

当 \(90 - \lambda_{2k-1} > 2a\),\(90 - \lambda_{2k} > 2a\),\(H_{A} > h\) 时,即 \(180^\circ - 4k\alpha > \theta\) 且 \(H_{A} > h\) 时,辐射波最终从尖劈空缺右侧边上射出,总的反射次数为 \(2k-1\),最终反射波线的方位角为 \(\varphi = -90^\circ\),射向角为 \(\lambda_{2k-1} + \alpha\),反射波的辐射强度为

\begin{equation}
I_{\text{反}} = \sqrt{(\sin(\theta)\sin(\varphi))^2 + \cos(\theta)^2} \prod_{i=1}^{k} \rho \sin((2i-1)\alpha + \theta) \prod_{i=k+1}^{2k-1} \rho \sin(2(i-1)\alpha + \theta).
\end{equation}

当 $90 - \lambda_{2k-1} > 2a$,$90 - \lambda_{2k} > 2a$,$H_{A} < h$ 时,即 $180^\circ - 4k\alpha > \theta$ 且 $H_{A} < h$ 时,辐射波最终从尖劈空缺左侧边上射出,总的反射次数为 $2k$,最终反射波线的方位角为 $\varphi = 90^\circ$,射向角为 $\lambda_{2k} + \alpha$,反射波的辐射强度为

\begin{equation}
I_{\text{反}} = \sqrt{(\sin(\theta)\sin(\varphi))^2 + \cos(\theta)^2} \prod_{i=1}^{k} \rho \sin((2i-1)\alpha + \theta) \prod_{i=k+1}^{2k} \rho \sin(2(i-1)\alpha + \theta).
\end{equation}

\subsection{4.2 问题二模型的建立与求解}

\subsubsection{4.2.1 静区从信号源直接得到的微波功率计算}

本文所建立坐标系及相关诸参数如下图所示,暗室的宽 $B$,高 $H$,长 $L$,线阵列的圆弧半径 $R$,单位均为米。所有墙面铺设同一规格的吸波体(上述数据均从吸波体的顶端平面算起)。图 7 所示暗室右端中心的 $s \times s$ 的小方块面积处是安置导引头的部位,称为“静区”。静区小方块的中心点与目标模拟阵列圆弧的圆心重合。目标模拟器对导引头的视在目标运动从左端开始,以匀角速运动到右端,前后共 4 秒,视在天线中心轴线对准静区中心,中心轴线处的发射功率强度随时间线性增大,结束时比初始时增大了一倍。

\begin{figure}[h]
    \centering
    \includegraphics[width=\textwidth]{image.png}
    \caption{微波暗室坐标系及诸参数示意图}
\end{figure}

根据条件可得视在目标 $P$ 点的轨迹方程

\begin{equation}
\begin{cases}
x_0 = R \sin\left(\frac{\pi}{8} - \frac{\pi}{16} t\right), & 0 < t < 4 \\
y_0 = L - R \cos\left(\frac{\pi}{8} - \frac{\pi}{16} t\right)
\end{cases}
\tag{37}
\end{equation}

导引头视在目标 $P$ 直接照射到静区内的辐射信号示意图如下

\begin{figure}[h]
    \centering
    \includegraphics[width=\textwidth]{image2.png}
\end{figure}

\begin{figure}[h]
    \centering
    \includegraphics[width=0.8\textwidth]{image.png}
    \caption{信号源直射到静区示意图}
    \label{fig:signal_source}
\end{figure}

采用微元法考虑,静区内任一点 $A$ 处以微小方块内接收到的辐射通量为
\begin{equation}
d\Phi = I d\omega
\tag{38}
\end{equation}
任一方向波的辐射强度($I$ 为与法线成 $i$ 角方向的辐射强度)
\begin{equation}
I = I_0 \cos i
\tag{39}
\end{equation}
静区内任一点 $A$ 处微小方块所对应的立体角 $d\omega$
\begin{equation}
d\omega = \frac{\cos\theta dS}{R^2}
\tag{40}
\end{equation}
微小面积元 $dS$
\begin{equation}
dS = dx dz
\tag{41}
\end{equation}
联立以上几式可得
\begin{equation}
d\Phi = \frac{I_0 \cos i \cos\theta}{R^2} dx dz
\tag{42}
\end{equation}
由三角形余弦定理可得
\begin{equation}
\cos i = \frac{|AP|^2 + |BP|^2 - |AB|^2}{2|AP|\cdot|BP|}
\tag{43}
\end{equation}
微元面积 $dS$ 的法线方向与 $d\omega$ 的轴线的夹角为 $\theta$
\begin{equation}
\cos\theta = \frac{L - y_0}{|AP|}
\tag{44}
\end{equation}
$R$ 为视在目标到点 $A$ 的距离
\begin{equation}
R = |AP| = \sqrt{(x - x_0)^2 + (L - y_0)^2 + z^2}
\tag{45}
\end{equation}
点 $A$ 与导引头之间的距离
\begin{equation}
|AB| = \sqrt{x^2 + z^2}
\tag{46}
\end{equation}

导引头与视在目标之间的距离

\begin{equation}
|BP| = \sqrt{x_0^2 + (y_0 - L)^2}
\tag{47}
\end{equation}

在 $XOZ$ 面内对微小通量进行积分即可得到静区直接接收的总辐射通量

\begin{equation}
\Phi = \int_{-s/2}^{s/2} \int_{-s/2}^{-s/2} \frac{I_0 \cos i \cos \theta}{R^2} dz dx
\tag{48}
\end{equation}

将以上各式联立可得静区直接接收的总辐射通量

\begin{equation}
\begin{aligned}
\Phi &= \int_{-s/2}^{s/2} \int_{-s/2}^{-s/2} \frac{I_0 (L - y_0) \left[ (x - x_0)^2 + (L - y_0)^2 + z^2 + x_0^2 + (y_0 - L)^2 - z^2 - x^2 \right]}{2 \left[ (x - x_0)^2 + (L - y_0)^2 + z^2 \right]^2 \sqrt{x_0^2 + (y_0 - L)^2}} dz dx \\
&= \frac{I_0 (L - y_0)}{\sqrt{x_0^2 + (y_0 - L)^2}} \int_{-s/2}^{s/2} \int_{-s/2}^{-s/2} \frac{-x x_0 + (L - y_0)^2 + x_0^2}{\left[ (x - x_0)^2 + (L - y_0)^2 + z^2 \right]^2} dz dx
\end{aligned}
\tag{49}
\end{equation}

直接计算上式比较困难,根据已知条件由于静区面积远远小于整个区域的面积,故可将静区看作一个点近似处理,得到简化后公式 (50)

\begin{equation}
\begin{aligned}
\Phi &= \frac{I_0 (L - y_0)}{\sqrt{x_0^2 + (y_0 - L)^2}} \int_{-s/2}^{s/2} \int_{-s/2}^{-s/2} \frac{-x x_0 + (L - y_0)^2 + x_0^2}{\left[ (x - x_0)^2 + (L - y_0)^2 + z^2 \right]^2} dz dx \\
&\approx \frac{I_0 (L - y_0)}{\sqrt{x_0^2 + (y_0 - L)^2}} \frac{(L - y_0)^2 + x_0^2}{\left[ x_0^2 + (L - y_0)^2 \right]^2} \cdot s^2 \\
&= \frac{I_0 (L - y_0) \cdot s^2}{\left[ (L - y_0)^2 + x_0^2 \right] \sqrt{x_0^2 + (y_0 - L)^2}}
\end{aligned}
\tag{50}
\end{equation}

利用 Matlab 画图得 $\Phi$ 与时间的关系图如下

\begin{figure}[h]
\centering
\includegraphics[width=\textwidth]{image.png}
\caption{信号源直射到静区功率与时间关系图}
\end{figure}

\subsection{静区从诸墙面得到的反射信号的功率计算}

导引头视在目标为余弦辐射体,在微波暗室中,辐射源即视在目标像向四周(前半球面)辐射电磁波,发出的电磁波辐射到暗室六个面后每点又变成新的余弦辐射体,因

此每个面上都有无穷多个新的余弦发射体,不是每点只有一条反射线。微波暗室是封闭的,暗室的六个面间相互作用,相互影响,辐射波要在静区内经历无数次的反射。因此我们不能单纯的考虑每一条辐射波经过了多少次反射并最终被静区吸收,应综合考虑六个面的相互作用,运用微元法思想和数值积分知识进行求解,并利用对称性简化计算过程。

以上表面分析为例,微小面积元 \( dS \) 接收到辐射源(视在目标)的辐射波,接收到的辐射波能量一部分被吸收,剩余辐射能又在接收点成为余弦辐射体向四周发出辐射能。

\begin{figure}[h]
    \centering
    \includegraphics[width=0.8\textwidth]{image.png}
    \caption{信号源射到暗室顶面示意图}
\end{figure}

上表面微小面积元接收到视在目标辐射的通量
\begin{equation}
d\Phi = \frac{I_0 \cos i \cos \theta}{R^2} dxdy
\tag{51}
\end{equation}
由余弦定理得
\begin{equation}
\cos i = \frac{|A_1P|^2 + |BP|^2 - |A_1P|^2}{2|A_1P| \cdot |BP|}
\tag{52}
\end{equation}
微元面积 \( dS \) 的法线方向与 \( d\omega \) 的轴线的夹角为 \( \theta \)
\begin{equation}
\cos \theta = \frac{H/2}{|A_1P|}
\tag{53}
\end{equation}
由两点间距离公式可得信号源到 \( A_1 \) 点的距离
\begin{equation}
R = |A_1P| = \sqrt{(x - x_0)^2 + (y - y_0)^2 + (H/2)^2}
\tag{54}
\end{equation}
点 \( A_1 \) 与导引头之间的距离
\begin{equation}
|A_1B| = \sqrt{x^2 + (y - L)^2 + (H/2)^2}
\tag{55}
\end{equation}

导引头与视在目标之间的距离

\begin{equation}
|BP| = \sqrt{x_0^2 + (y_0 - L)^2}
\tag{56}
\end{equation}

\begin{figure}[h]
    \centering
    \includegraphics[width=0.8\textwidth]{image.png} % 替换为实际图片路径
    \caption{第一次反射后信号静区示意图}
    \label{fig:11}
\end{figure}

经反射后微波暗室顶面 A 点的辐射照度

\begin{equation}
E_1 = \frac{d\Phi}{dS} = \frac{d\Phi}{dxdy} = \frac{I_0 \cos i \cos \theta}{R^2}
\tag{57}
\end{equation}

故该点的辐射输出射度

\begin{equation}
M_1 = \rho \cos \alpha E_1, \text{ 其中 } \alpha = \theta
\tag{58}
\end{equation}

该点经反射输出后输入到静区的辐射照度

\begin{equation}
I_1 = \frac{d\Phi_1}{d\omega_1} = \frac{d\Phi_1}{dS_1 / R_1^2} = R_1^2 \frac{d\Phi_1}{dS_1} = R_1^2 M_1 = R_1^2 \rho \cos \theta \frac{I_0 \cos i \cos \theta}{R^2}
\tag{59}
\end{equation}

故经上暗室内壁反射到达静区的辐射通量

\begin{equation}
\Phi_{\text{静}} = \int_{-s/2}^{s/2} \int_{-s/2}^{s/2} I_1 \cdot \frac{\cos i_1 \cos \theta_1}{R_1^2} dx' dz' = \int_{-s/2}^{s/2} \int_{-s/2}^{s/2} R_1^2 \rho \cos \theta \frac{I_0 \cos i \cos \theta}{R^2} \cdot \frac{\cos i_1 \cos \theta_1}{R_1^2} dx' dz'
\tag{60}
\end{equation}

\begin{equation}
= \int_{-s/2}^{s/2} \int_{-s/2}^{s/2} \frac{\rho I_0 \cos i \cos^2 \theta}{R^2} \cdot \cos i_1 \cos \theta_1 dx' dz'
\end{equation}

\begin{equation}
\cos i_1 = \frac{|A_1 C_1|^2 + |A_1 B|^2 - |C_1 B|^2}{2 |A_1 C_1| \cdot |A_1 B|}
\tag{61}
\end{equation}

\begin{equation}
\cos \theta_1 = \frac{(H/2) - z'}{A_1 C_1}
\tag{62}
\end{equation}

\begin{equation}
R_{1} = |A_{1}C_{1}| = \sqrt{(x-x')^{2} + (y-y')^{2} + (z'-H/2)^{2}}
\tag{63}
\end{equation}

\begin{equation}
|A_{1}B| = \sqrt{x^{2} + (y-L)^{2} + (H/2)^{2}}
\tag{64}
\end{equation}

\begin{equation}
|BC_{1}| = \sqrt{x'^{2} + (y'-L)^{2}}
\tag{65}
\end{equation}

将上式代入得

\begin{equation}
\Phi_{1\text{静}} = \int_{-s/2}^{s/2} \int_{-s/2}^{s/2} I_{1} \cdot \frac{\cos i_{1} \cos \theta_{1}}{R_{1}^{2}} dx' dz' = \int_{-s/2}^{s/2} \int_{-s/2}^{s/2} \frac{\rho I_{0} \cos i \cos^{2} \theta}{R^{2}} \cdot \cos i_{1} \cos \theta_{1} dx dy
\tag{66}
\end{equation}

\[
\frac{\rho I_{0} \cos i \cos^{2} \theta}{2R^{2} \sqrt{x^{2} + (y-L)^{2} + (H/2)^{2}}} \cdot \int_{-s/2}^{s/2} \int_{-s/2}^{s/2} \frac{(H/2-z')[-2xx' - z'H + [2x^{2} + 2(y-L)^{2} + H^{2}/2]}{(x-x')^{2} + (y-L)^{2} + (z'-H/2)^{2}} dx' dz'
\]

直接计算上式比较困难,根据已知条件由于静区面积远远小于整个区域的面积,故可将静区看作一个点近似处理得

\begin{equation}
\Phi_{1\text{静}} = \int_{-s/2}^{s/2} \int_{-s/2}^{s/2} I_{1} \cdot \frac{\cos i_{1} \cos \theta_{1}}{R_{1}^{2}} dx' dz' \approx \frac{\rho I_{0} Hs^{2} \cos i \cos^{2} \theta}{2R^{2} \sqrt{x^{2} + (y-L)^{2} + (H/2)^{2}}}
\tag{67}
\end{equation}

\begin{equation}
\Phi_{1\text{上}} = \int_{-B/2}^{B/2} \left( \int_{y_{0}}^{L} \Phi_{1\text{静}} dy \right) dx = \int_{y_{0}}^{L} \left( \int_{-B/2}^{B/2} \frac{\rho I_{0} Hs^{2} \cos i \cos^{2} \theta}{2R^{2} \sqrt{x^{2} + (y-L)^{2} + (H/2)^{2}}} dx \right) dy
\tag{68}
\end{equation}

\[
= \frac{I_{0} \rho H^{3} s^{2}}{8} \int_{y_{0}}^{L} \left( \int_{-B/2}^{B/2} \frac{-x_{0}x + (L-y_{0})y + (y_{0}^{2} + x_{0}^{2} - y_{0}L)}{[(x-x_{0})^{2} + (y-y_{0})^{2} + (H/2)^{2}]^{5/2}} \cdot \frac{1}{x^{2} + (y-L)^{2} + (H/2)^{2}} dx \right) dy
\]

由对称性知

\begin{equation}
\Phi_{1\text{底}} = \Phi_{1\text{上}}
\tag{69}
\end{equation}

同理可得经前暗室内壁反射到达静区的辐射通量

\begin{equation}
\Phi_{1\text{前}} = \frac{I_{0} \rho (B+2x_{0})^{2} s^{2}}{8[x_{0}^{2} + (y_{0}-L)^{2}]}
\tag{70}
\end{equation}

\[
\int_{-H/2}^{H/2} \left( \int_{y_{0}}^{L} \frac{L-y}{[(B/2)^{2} + (y-L)^{2} + z^{2}]^{3/2}} \cdot \frac{2y(L-y_{0}) + [2x_{0}^{2} + 2y_{0}^{2} + Bx_{0} + 2y_{0}L]}{(x_{0}+B/2)^{2} + (y-y_{0})^{2} + z^{2}} dy \right) dz
\]

同理可得经后暗室内壁反射到达静区的辐射通量

\begin{equation}
\Phi_{1\text{后}} = \frac{I_{0} \rho (B-2x_{0})^{2} s^{2}}{8[x_{0}^{2} + (y_{0}-L)^{2}]}
\tag{71}
\end{equation}

\[
\int_{-H/2}^{H/2} \left( \int_{y_{0}}^{L} \frac{L-y}{[(B/2)^{2} + (y-L)^{2} + z^{2}]^{3/2}} \cdot \frac{2y(L-y_{0}) + [2x_{0}^{2} + 2y_{0}^{2} - Bx_{0} - 2y_{0}L]}{(x_{0}-B/2)^{2} + (y-y_{0})^{2} + z^{2}} dy \right) dz
\]

运用 Matlab 计算最终暗室静区接收到反射微波的结果图如下图所示

\begin{figure}[h]
    \centering
    \includegraphics[width=0.45\textwidth]{placeholder1.png}
    \caption{$\rho=0.5$时信号源经反射后到静区功率图}
    \label{fig:12}
\end{figure}
\begin{figure}[h]
    \centering
    \includegraphics[width=0.45\textwidth]{placeholder2.png}
    \caption{$\rho=0.05$时信号源经反射后到静区功率图}
    \label{fig:13}
\end{figure}

\subsection{微波暗室是否满足仿真技术要求的判断}

根据 4.2.2 和 4.2.3 可得静区从诸墙面得到的反射信号的功率与之静区从信号源直接得到的微波功率之比 $\gamma$

\begin{equation}
\gamma = (\Phi_{\text{底}} + \Phi_{\text{上}} + \Phi_{\text{前}} + \Phi_{\text{后}}) / \Phi
\end{equation}

\begin{figure}[h]
    \centering
    \includegraphics[width=0.45\textwidth]{placeholder3.png}
    \caption{$\rho=0.5$时$\gamma$值随时间变化图}
    \label{fig:14}
\end{figure}
\begin{figure}[h]
    \centering
    \includegraphics[width=0.45\textwidth]{placeholder4.png}
    \caption{$\rho=0.05$时$\gamma$值随时间变化图}
    \label{fig:15}
\end{figure}

根据所建模型求得静区从诸墙面得到的反射信号的功率与之静区从信号源直接得到的微波功率之比 $\gamma$,易知,当 $\rho=0.5$ 时微波暗室不能满足仿真技术要求,且当 $t=2.2s$ 时 $\gamma$ 取得最小值 0.039。当 $\rho=0.05$ 时微波暗室能够满足仿真要求,且当 $t=2.2s$ 时 $\gamma$ 取得最小值 0.004。

\section*{六、模型评价及改进}

本文先从二维情况入手,探讨波在尖劈中的传播反射过程,得到波在尖劈空缺间传播的数学模型,并最终拓展到三维之中,本文提出了临界状态这一概念,并利用临界状态的反射次数来判定最终反射波线的所经历的反射次数,使得问题变的清晰明了。

问题二是一个较复杂的问题,本文利用微元法、数值积分法、对称性等方法对其进行有效的分析和求解,并得出了一般性的结论,有效地解决了该问题。

由于经暗室内壁多次反射被静区接收到的微波功率的计算问题比较复杂,该模型未能作出计算,但是由于微波经过多次反射其损耗较大,故可以通过将其设为一个未知小量修正该模型来达到更加良好的效果。

由于时间仓促,本文在建模过程中也存在一些问题,如未考虑光的透射等情况。

\section*{参考文献}

[1] 邱琴,张晏清,张雄,电磁吸波材料研究进程[J],电子元件与材料,8:78-81,2009。

[2] 杨恩耀,周朝栋,电磁场与微波技术[M],北京:北京理工大学出版社,1998。

[3] 蒋谱成,室内建材与电磁波传输特性[J],电子对抗实验,2:45-48,1999。

[4] 林吕禄,天线测量技术,成都:国防科技大学出版社,1989。

[5] 徐宏伟,王天顺,大型屏蔽暗室静区静度测试初探[J],飞机设计,6(2):64-66,2003。

[6] 康颖,大学物理,北京:科学出版社,2006。