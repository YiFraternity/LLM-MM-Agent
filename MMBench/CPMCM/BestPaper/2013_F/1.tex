\begin{center}
\textbf{第十届华为杯全国研究生数学建模竞赛}
\end{center}

\begin{table}[h]
\centering
\begin{tabular}{c c}
\hline
学校 & 江苏大学 \\
\hline
参赛队号 & 10299001 \\
\hline
队员姓名 & 1. 万根顺 \\
 & 2. 王维 \\
 & 3. 陈宇 \\
\hline
\end{tabular}
\end{table}

\begin{flushright}
参赛密码 \underline{\hspace{2cm}} \\
(由组委会填写)
\end{flushright}

\begin{center}
\includegraphics[width=0.3\textwidth]{image1.png} \quad
\includegraphics[width=0.3\textwidth]{image2.png} \quad
\includegraphics[width=0.3\textwidth]{image3.png}
\end{center}

\section*{第十届华为杯全国研究生数学建模竞赛}

\textbf{题目} 可持续的中国城乡居民养老保险体系的数学模型研究

\section*{摘 要}

本文在可持续的中国城乡居民养老保险体系问题上,通过建立养老金收入、支出的宏观数学模型对养老金的缺口进行合理估计,进而寻找矛盾最尖锐的时间。而后通过分析国内外养老保险的不同模式,寻找保证我国养老保险体系可持续性的替代率和缴费率的合理区间,并提出改进措施。

针对问题一,从城镇职工社会养老保险、新型农村社会养老保险和城镇居民社会养老保险三个方面构建养老金收入、支出的宏观数学模型,并通过建立养老金替代率和缴费率模型、改进的离散人口发展模型、分段加权死亡率模型、双变量 ARIMA 经济增速模型、曲线拟合工资预测模型,结合财政补贴、投资效益、城市化率、就业率、金融危机、鼓励政策等因素加以强化完善。

针对问题二,利用问题一中模型的收支差预估 2013 年到 2035 年的养老金缺口,结果显示在 2028 年结余达到峰值;维持情况不变,我国城乡居民养老保险收支矛盾最尖锐的情况发生在 2046 年,养老金将出现收不抵支;结合十八大提出的收入倍增计划,对模型中的工资水平以及就业率进行调整。

针对问题三,通过分析各国养老保险的不同模式和我国的实际情况,利用养老金缺口模型,从 ELES 基本消费支出法和人均可支配收入法两方面估计保证我国养老保险体系可持续运行的针对不同收入群体的合意替代率下限和上限,并给出合理区间 [35\%-53\%];考虑替代率和缴费率的函数关系,利用模型仿真求出缴费率的合理范围 [25\%-33\%];提出设定替代率标准、倾斜特定群体等建议,仿真结果显示其能有效缓解亏空压力,实现尖锐矛盾的平稳过渡。

针对问题四,在考虑养老保险模式和养老保险制度的基础上,选择以企业年金基金投资组合及收益率作为可调节变量,通过收益率和风险指数的约束建立投

资组合目标规划模型确定最优化投资方案。

关键词:收支模型 养老金缺口 合意替代率 可持续性 基金投资组合

\section*{目录}

\begin{itemize}
    \item[1.] 问题重述 \dotfill 1
        \begin{itemize}
            \item[1.1] 问题背景 \dotfill 1
            \item[1.2] 问题提出 \dotfill 1
        \end{itemize}
    \item[2.] 模型假设 \dotfill 2
    \item[3.] 符号说明 \dotfill 2
    \item[4.] 模型的建立与求解 \dotfill 3
        \begin{itemize}
            \item[4.1] 问题一 \dotfill 3
                \begin{itemize}
                    \item[4.1.1] 城镇职工社会养老保险的收入和支出模型 \dotfill 4
                    \item[4.1.2] 新型农村社会养老保险的收入和支出模型 \dotfill 5
                    \item[4.1.3] 城镇居民社会养老保险的收入和支出模型 \dotfill 6
                    \item[4.1.4] 改进的离散化人口结构预测模型 \dotfill 7
                    \item[4.1.5] 替代率模型 \dotfill 11
                    \item[4.1.6] 经济预测模型 \dotfill 18
                \end{itemize}
            \item[4.2] 问题二 \dotfill 23
                \begin{itemize}
                    \item[4.2.1] 对养老金缺口的理解 \dotfill 23
                    \item[4.2.2] 对未来有关情况的合理估计 \dotfill 24
                    \item[4.2.3] 从今年至2035年我国的养老金缺口 \dotfill 25
                    \item[4.2.4] 缺口预测合理性分析 \dotfill 26
                    \item[4.2.5] 最尖锐的情况及严重程度 \dotfill 26
                    \item[4.2.6] 变量调整 \dotfill 26
                \end{itemize}
            \item[4.3] 问题三 \dotfill 28
                \begin{itemize}
                    \item[4.3.1] 养老保险国内外对比 \dotfill 28
                    \item[4.3.2] 替代率和缴费率的区间确定 \dotfill 30
                    \item[4.3.3] 建议与仿真 \dotfill 38
                \end{itemize}
            \item[4.4] 问题四 \dotfill 39
                \begin{itemize}
                    \item[4.4.1] 变量分析 \dotfill 39
                    \item[4.4.2] 投资组合目标规划模型优化设计 \dotfill 39
                \end{itemize}
        \end{itemize}
    \item[5.] 结束语 \dotfill 41
\end{itemize}

\section{问题重述}

\subsection{问题背景}

中国共产党第十八次全国代表大会政治报告中提出了“统筹推进城乡社会保障体系建设”的任务:“社会保障是保障人民生活、调节社会分配的一项基本制度。要坚持全覆盖、保基本、多层次、可持续方针,以增强公平性、适应流动性、保证可持续性为重点,全面建成覆盖城乡居民的社会保障体系”。

自20世纪90年代以来,中国加快建立以社会统筹与个人账户相结合的城镇职工基本养老保险为主体的社会养老保障体系,覆盖范围不断扩大,社会保险基金和财政投入规模持续增长,社会化养老保障体系不断完善,但是,面对日益严重的老龄化问题,我国的社会养老保障体系还不健全,制度运行还需要进一步完善。

我国当前实行社会统筹与个人账户相结合的养老保险制度,但是在实际的实施中却形成了一种在资金流程上与现收现付制没有什么本质区别的“空账”运行机制。1997年26号文件《国务院关于建立统一的企业职工养老保险制度的决定》指出已经退休职工的养老金继续按照过去的标准,这意味着企业要同时承担退休职工养老之需和为在职职工积累养老金的双重任务,造成企业负担过重,缴费困难、逃费、欠费现象严重。

中国保监会副主席陈文辉在论坛上承认,中国养老金缺口“确实非常大”,而且近10年来基本养老保险的财政补贴已经超过了1万亿,老年人口的抚养比到去年末已经上升到122.23\%。陈文辉称,基本养老保险存在着比较大的养老金缺口。最新一期的财经杂志上面讲到国家资产负债表的时候,包括中国银行的一个团队,还有一个团队,两个不同团队研究都研究了养老金缺口问题,包括其他的世行的一些机构都在研究。大家一个共同的看法,就是这个缺口很大,当然有的说大的没谱,有的说没有那些大。但是不管怎么说,确实非常大,而且近十年来基本养老保险的财政补贴已经超过了1万亿,大家都知道我们国家是世界上唯一一个老年人口超过一亿的国家,而且老年人口的抚养比到去年末已经上升到122.23\%。

至于我国养老金的缺口究竟有多大?戴相龙表示,这需要国务院有关部门进行研究。缺口算出来后怎么办?戴相龙表示:“不是被动的准备钱,而是调整这个制度。”对于中长期养老金收支平衡的压力,戴相龙强调,只要完善养老制度,社会养老收支平衡是能够做到的。

\subsection{问题提出}

1. 分别建立合乎国情、适应国力的中国城乡居民(含新农保)养老金收入、支出的宏观数学模型,至少包括替代率(基本养老保险人均养老金占城镇单位在岗职工平均工资比率)、缴费率(基本养老保险人均缴费占城镇单位在岗职工平均工资比率)、人口结构、分年龄段死亡率、经济增速、财政补贴、工资水平或物价指数、投资效益等主要因素,要做到模型结合现实,分多个层次(含企业基金等),体现“多缴多得,长缴多得”(不考虑分省、分地区模型)。

2. 根据你们的数学模型、对养老金缺口的理解和对未来有关情况的合理估

计,估计从今年至 2035 年我国养老金缺口,并说明你们对养老金缺口分析的合理性。如果全部情况维持不变,按照你们的数学模型我国城乡居民养老保险收支矛盾最尖锐的情况发生在什么时间,严重程度如何?考虑到党的十八大提出的收入倍增计划,你们的数学模型哪些部分需要调整?

3. 养老保险制度也是调节社会分配,请你们分析各国养老保险的不同模式,取其精华,去其糟粕,根据你们建立的数学模型和中国的实际情况,利用仿真手段寻找替代率和缴费率的合理区间以保证我国养老保险体系的可持续性(因为人口结构、分年龄段死亡率、经济增速、投资效益等主要因素几乎无法人为较大幅度改动);在步入良性循环之前,在矛盾最尖锐到来前的过渡期内应该采取哪些政策措施实现平稳过渡并仿真预测相关政策的效果。

4. 尝试建立第三问增加可调节变量的数学模型。

\section*{2. 模型假设}

1. 本文不考虑居民重复参保的情况;
2. 参保人每年年初按不同的缴费标准向个人的账户中供款,缴费标准档次不发生变化;
3. 在集体补助和政府补贴标准不变的情况下,同时记入参保农民的个人账户中;
4. 假设参保人符合领取标准时在每年年初领取一年的养老金;
5. 假设国家根据经济发展情况和物价变化情况,对全国新型农村社会养老保险基础养老金的最低标准保持不变;
6. 一定时期内,消费者对各种商品和劳务的需求量取决于消费者的收入和商品的价格;
7. 外部事件发生前后其他的影响因素都保持不变;
8. 在金融危机恢复期 \( n \) 年后同比增长率与预期相同;
9. 每一项可供选择的投资在一定持有期内都存在预期收益率的概率分布。

\section*{3. 符号说明}

1. \( I_t \) 表示 \( t \) 年的社会养老保险的收入

2. \( I_t^{zb} \) 表示 \( t \) 年的城镇职工社会养老保险的收入

3. \( I_t^{xnb} \) 表示 \( t \) 年的新型农村社会养老保险的收入

4. \( I_t^{cjb} \) 表示 \( t \) 年的城镇居民社会养老保险的收入

5. \( T_t \) 表示 \( t \) 年的投资收益额

6. \( B_t \) 表示 \( t \) 年的政府补贴

7. $E_{t}$ 表示 $t$ 年的社会养老保险的支出

8. $E_{t}^{z b}$ 表示 $t$ 年的城镇职工社会养老保险的支出

9. $E_{t}^{x n b}$ 表示 $t$ 年的新型农村社会养老保险的支出

10. $E_{t}^{c j b}$ 表示 $t$ 年的城镇居民社会养老保险的支出

\section*{4. 模型的建立与求解}

\subsection*{4.1 问题一}

基本养老保险亦称国家基本养老保险,它是按国家统一政策规定强制实施的为保障广大离退休人员基本生活需要的一种养老保险制度。在我国,90 年代之前,企业职工实行的是单一的养老保险制度。1991 年,《国务院关于企业职工养老保险制度改革的决定》中明确提出:“随着经济的发展,逐步建立起基本养老保险与企业补充养老保险和职工个人储蓄性养老保险相结合的制度”。从此,我国逐步建立起多层次的养老保险体系。在这种多层次养老保险体系中,基本养老保险可称为第一层次,也是最高层次。

我国的基本养老保险分三类,即城镇职工社会养老保险(职保)、新型农村社会养老保险(新农保)和城镇居民社会养老保险(城居保)。《社会保险法》第十条规定,职工应当参加基本养老保险,由用人单位和职工共同缴纳基本养老保险费。无雇工的个体工商户、未在用人单位参加基本养老保险的非全日制从业人员以及其他灵活就业人员可以参加基本养老保险,由个人缴纳基本养老保险费。第二十条规定,国家建立和完善新型农村社会养老保险制度。新型农村社会养老保险实行个人缴费、集体补助和政府补贴相结合。第二十二条规定,国家建立和完善城镇居民社会养老保险制度。省、自治区、直辖市人民政府根据实际情况,可以将城镇居民社会养老保险和新型农村社会养老保险合并实施。第二十一条规定,新型农村社会养老保险待遇由基础养老金和个人账户养老金组成。参加新型农村社会养老保险的农村居民,符合国家规定条件的,按月领取新型农村社会养老保险待遇。第十一条规定,基本养老保险实行社会统筹与个人账户相结合。基本养老保险基金由用人单位和个人缴费以及政府补贴等组成。

因此,中国城乡居民(含新农保)养老金收入、支出的宏观数学模型也将从城镇职工社会养老保险、新农保、城居保三个方面展开。

社会养老保险的收入 = 城镇职工社会养老保险的收入 + 新型农村社会养老保险的收入 + 城镇居民社会养老保险的收入 + 投资收益额 + 政府补贴,即

\begin{equation}
I_{t} = I_{t}^{z b} + I_{t}^{x n b} + I_{t}^{c j b} + T_{t} +
\tag{1-1}
\end{equation}

社会养老保险的支出 = 城镇职工社会养老保险的支出 + 新型农村社会养老保险的支出 + 城镇居民社会养老保险的支出,即

\begin{equation}
E_{t} = E_{t}^{z b} + E_{t}^{x n b} + E_{t}
\tag{1-2}
\end{equation}

\subsection*{4.1.1 城镇职工社会养老保险的收入和支出模型}

城镇职工社会养老保险的收入 = 缴费人数 * 在岗职工平均工资 * 缴费比率 * 收缴率,即

\begin{equation}
I_{t}^{z b} = N_{t}^{z b} \times S_{t}^{z b} \times J_{t}^{z b} \times T_{t}^{z b}
\tag{1-3}
\end{equation}

城镇职工社会养老保险的支出 = 赡养人数 * 在岗职工平均工资 * 职保养老金替代率,即

\begin{equation}
E_{t}^{z b} = W_{t}^{z b} \times S_{t}^{z b} \times R_{t}^{z b}
\tag{1-4}
\end{equation}

其中:

(1) 职保的缴费人数 $N_{t}^{z b}$:本项预测中缴费人数由全国城镇企业及个体从业人数与养老保险覆盖率加以确定,具体的计算方法为:缴费人数 = 基期缴费人数 + 新参保人数 - 当年退休人数;即 $N_{t}^{z b} = N_{t-1}^{z b} + N_{t}^{z b-n} - N_{t}^{z b-r}$,其中:新参保人数 = 新增就业人数 * 参保率;$N_{t}^{z b-n} = N_{t}^{z b-j} \times r_{t}^{1}$;

(2) 职保的缴费比率 $J_{t}^{z b}$:按照我国现行企业基本养老保险制度的规定,企业的缴费比率设定在工资总额的 20%;个人缴费比率为 8%,因此,总缴费比率为 28%;

(3) 职保的收缴率 $T_{t}^{z b}$:收缴率依据目前的水平确定为 90%(2012 年),假定以后每年提高 1%,2020 年提高至 98% 后保持稳定;

(4) 在岗职工平均工资 $S_{t}^{z b}$:在岗职工平均工资我们根据 2000-2012 年的在岗职工平均工资和预测的经济增长速度进行曲线拟合的最小二乘法进行预测;

(5) 经济增长速度 $G_{t}$:以稳定减速的经济模型与外在事件对因变量所产生的影响的双变量 ARIMA 模型,根据往年的经济增长速度、经融危机和整体的经济发展趋势对未来的经济增长速度进行预测;

(6) 职保的赡养人数 $W_{t}^{z b}$:领取养老金的人数,即赡养人数 = 上年退休人员总数 + 当年退休人员数 - 当年死亡人员数,即,$W_{t}^{z b} = W_{t-1}^{z b} + N_{t}^{z b-r} - N_{t}^{z b-d}$,

其中,当年死亡人数根据相关人口统计资料和全国领取养老金的离退休人员实际死亡水平,离退休人员的死亡率确定为 $r_{t}^{2}=0.033273$,即当年死亡人数=上年离退休人员总数 $\times r_{t}^{2}$,$N_{t}^{zb-d}=W_{t-1}^{zb} \times r_{t}^{2}$;

(7) 职保的养老金替代率 $R_{t}^{zb}$:基本养老保险人均养老金占城镇单位在岗职工平均工资比率,表示养老金的支付水平。

4.1.2 新型农村社会养老保险的收入和支出模型

新农保的收入=缴费人数*缴费金额,即

\begin{equation}
I_{t}^{xnb}=\sum_{i=1}^{5} N_{ti}^{xnb} \times M_{ti}^{xnb}
\tag{1-5}
\end{equation}

新农保的收入=赡养人数*(基础养老金*12+目标替代率*前一年农民人均收入),即

\begin{equation}
E_{t}^{xnb}=W_{t}^{xnb} \times\left(B_{t}^{xnb} \times 12+R_{t}^{xnb} \times S_{t}^{xnb}\right)
\tag{1-6}
\end{equation}

其中:

(1) 新农保的缴费人数 $N_{ti}^{xnb}$:本项预测中缴费人数由农村居民人数与养老保险覆盖率、及各档次覆盖率加以确定,具体的计算方法为:缴费人数=基期缴费人数+新参保人数-当年满 60 岁的人数,即

\begin{equation}
N_{ti}^{xnb}=N_{(t-1) i}^{xnb}+N_{ti}^{xnb-n}-N_{ti}^{xnb-r}
\end{equation}

其中:新参保人数=新增总人数(以 20 开始)*参保率*档次覆盖率(五个等级),即 $N_{ti}^{xnb-n}=N_{t}^{xnb-n} \times r_{t}^{3} \times r_{ti}^{xnb}$;当年满 60 岁的人数 $N_{ti}^{xnb-r}=N_{t}^{xnb-r} \times r_{ti}^{xnb}$;

(2) 新农保的缴费金额 $M_{ti}^{xnb}$:按照不同档次分为 100,200,300,400,500 五种,即 $M_{ti}^{xnb}$;

(3) 新农保的赡养人数 $W_{t}^{xnb}$:领取养老金的人数,即赡养人数=上年领取人员总数+当年满 60 周岁的人总数-当年死亡人员数,即

\begin{equation}
W_{t}^{xnb}=W_{t-1}^{xnb}+N_{t}^{xnb-r}-N_{t}^{xnb-d}
\end{equation}

其中,当年死亡人数根据相关人口统计资料和全国领取养老金的农村居民的实际死亡水平,农村养老居民的死亡率

确定为 $r_{t}^{4}$,即当年死亡人数=领取人员总数$\times r_{t}^{4}$,即 $N_{t}^{xnb-d}=W_{t-1}^{xnb}\times r_{t}^{4}$;

(4) 新农保的基础养老金 $B_{t}^{xnb}$:基础养老金为每月 55 元;

(5) 农民人均收入 $S_{t}^{xnb}$:农民人均收入我们根据 2000-2012 年的农民人均收入和预测的经济增长速度进行曲线拟合的最小二乘法进行预测;

(6) 新农保养老金替代率 $R_{t}^{xnb}$:基本养老保险人均养老金占农民人均收入比率,表示养老金的支付水平。

### 4.1.3 城镇居民社会养老保险的收入和支出模型

城居保的收入=缴费人数$\times$缴费金额,即

\[
I_{t}^{cjb}=\sum_{i=1}^{10} N_{ti}^{cjb}\times M_{ti}^{cjb}
\tag{1-7}
\]

城居保的收入=赡养人数$\times$(基础养老金$\times$12+目标替代率$\times$前一年城镇人均收入),即

\[
E_{t}^{cjb}=W_{t}^{cjb}\times(B_{t}^{cjb}\times12+R_{t}^{cjb}\times S_{t}^{cjb})
\tag{1-8}
\]

其中:

(1) 城居保的缴费人数 $N_{ti}^{cjb}$:本项预测中缴费人数由城镇居民人数与养老保险覆盖率、及各档次覆盖率加以确定,具体的计算方法为:缴费人数=基期缴费人数+新参保人数-当年满 60 岁的人数;即

\[
N_{ti}^{cjb}=N_{(t-1)i}^{cjb}+N_{ti}^{cjb-n}-N_{ti}^{cjb-r}
\]

其中:新参保人数=新增总人数(以 20 开始)$\times$参保率$\times$档次覆盖率(五个等级),即 $N_{ti}^{cjb-n}=N_{t}^{cjb-n}\times r_{t}^{5}\times r_{ti}^{cjb}$;当年满 60 岁的人数 $N_{ti}^{xnb}=r_{t}^{-}\times j_{t}^{r}$;

(2) 城居保的缴费金额 $M_{ti}^{cjb}$:按照不同档次分为 100,200,300,400,500,600,700,800,900,1000 十种,即 $M_{ti}^{cjb}$;

(3) 城居保的赡养人数 $W_{t}^{cjb}$:领取养老金的人数,即赡养人数=上年领取人员总数+当年满 60 周岁的人总数-当年死亡人员数,即

\begin{equation}
W_{t}^{cjb} = W_{t-1}^{cjb} + N_{t}^{cjb-r} - N_{t}^{cjb-d},
\end{equation}
其中,当年死亡人数根据相关人口统计资料和全国领取养老金的城镇居民的实际死亡水平,城镇养老居民的死亡率确定为 \( r_{t}^{6} \),即当年死亡人数 = 领取人员总数 \(* r_{t}^{6}\),即 \( N_{t}^{cjb-d} = W_{t-1}^{cjb} \times r_{t}^{6} \)。

(4) 城居保的基础养老金:基础养老金为每月 55 元;

(5) 城镇人均收入 \( S_{t}^{cjb} \):城镇人均收入我们根据 2000-2012 年的城镇人均收入和预测的经济增长速度进行曲线拟合的最小二乘法进行预测;

(6) 城居保养老金替代率 \( R_{t}^{cjb} \):基本养老保险人均养老金占城镇人均收入比率,表示养老金的支付水平。

\subsection*{4.1.4 改进的离散化人口结构预测模型}

近年来我国的人口发展出现了一些新的特点,例如,老龄化进程加速、出生人口性别比持续升高,以及乡村人口不断城镇化等,这些都影响着中国人口的增长。然而,在传统的人口预测模型中,大多忽略了这些因素的影响,难以对人口的发展做出准确的预测 \({}^{[1]}\)。鉴于此,本文引入了女婴比、乡村入迁城镇人口数目等参数,改进了传统的离散人口发展模型 \({}^{[2]}\)。

离散化人口发展方程组为:

\begin{equation}
\begin{cases}
\eta(t) = \beta(t) \sum_{i=15}^{49} y_{i}^{w}(t) h_{i}(t) \\
y_{0}^{w}(t) = \left(1 - \delta_{0}^{w}(t)\right) k(t) \eta(t) \\
y_{0}^{m}(t) = \left(1 - \delta_{0}^{m}(t)\right) \left(1 - k(t)\right) \eta(t) & i = 0, 1, 2, \dots, n \\
y_{i+1}^{w}(t+1) = \left(1 - \delta_{t}^{w}(t)\right) y_{i}^{w}(t) \\
y_{i+1}^{m}(t+1) = \left(1 - \delta_{t}^{m}(t)\right) y_{i}^{m}(t) \\
y_{i+1}(t+1) = y_{i+1}^{w}(t+1) + y_{i+1}^{m}(t+1)
\end{cases}
\tag{1-9}
\end{equation}

其中:\( y_{i}(t) \) 表示 \( t \) 年时满 \( i \) 周岁但不到 \( i+1 \) 周岁的人口总数;\( y_{i}^{m}(t) \) 表示 \( t \) 年时满 \( i \) 周岁但不到 \( i+1 \) 周岁的男性人口总数;\( y_{i}^{w}(t) \) 表示 \( t \) 年时满 \( i \) 周岁但不到 \( i+1 \) 周岁的女性人口总数;\( \beta(t) \) 表示 \( t \) 年时育龄妇女总和生育率;\( \delta_{i}^{m}(t) \) 表示 \( t \) 年时 \( i \) 周岁的男性的死亡率,其中,\( \delta_{0}^{m}(t) \) 表示 \( t \) 年时出生的男婴的死亡率;\( \delta_{t}^{w}(t) \) 表示 \( t \) 年时 \( i \) 周岁的女性的死亡率,其中,\( \delta_{0}^{w}(t) \) 表示 \( t \) 年时出生的女婴的死亡率;\( k(t) \) 表示 \( t \) 年时出生婴儿女性占当年出生总婴儿数量的比例;\( h_{i}(t) \) 表示 \( t \) 年时 \( i \) 周岁女性占育龄妇女的比例。

示生育模式函数; $n$ 是人口年龄上限。

为了进行人口预测即求解人口发展方程,首先要有人口初始状况下的年龄构成 $\left\{y_{1}^{m}(0), y_{2}^{m}(0), \ldots, y_{n}^{m}(0)\right\}$ 和 $\left\{y_{1}^{w}(0), y_{2}^{w}(0), \ldots, y_{n}^{w}(0)\right\}$,其次要估计未来人口参数的可能变化趋势。

### 4.1.4.1 育龄妇女生育模式 $h_{i}(t)$

生育模式函数表示在 $t$ 年时满 $i$ 周岁妇女的生育概率,分别取 2011 年,2009 年,2008 年(2010 年数据未找到)[3] 的生育概率,用高斯密度函数来拟合它,即

\[
h_{i}(t)=\sum_{j=1}^{7} a_{j} \times e^{-\left(\frac{x-b_{j}}{c_{j}}\right)^{2}}
\tag{1-10}
\]

拟合结果如下:

#### 2011

General model Gauss7:

\[
f(x)=\sum_{j=1}^{7} a_{j} \times e^{-\left(\frac{x-b_{j}}{c_{j}}\right)^{2}}
\tag{1-11}
\]

Coefficients (with 95% confidence bounds):

\[
\begin{aligned}
& \text{a1} = 0.03224; \quad \text{b1} = 22.35; \quad \text{c1} = 10.06 \\
& \text{a2} = 0.009128; \quad \text{b2} = 37.15; \quad \text{c2} = 2.369 \\
& \text{a3} = 0.02252; \quad \text{b3} = 48.12; \quad \text{c3} = 2.317 \\
& \text{a4} = 0; \quad \text{b4} = 38.4; \quad \text{c4} = 9.978e-006 \\
& \text{a5} = 0.006119; \quad \text{b5} = 29.13; \quad \text{c5} = 0.822 \\
& \text{a6} = 0.03302; \quad \text{b6} = 42.11; \quad \text{c6} = 5.482 \\
& \text{a7} = 0.01232; \quad \text{b7} = 33.11; \quad \text{c7} = 3.161 \, (-6.471, 12.79)
\end{aligned}
\]

Goodness of fit:

\[
\begin{aligned}
& \text{SSE: } 8.937e-005; \quad \text{R-square: } 0.8321; \quad \text{Adjusted R-square: } 0.5923 \\
& \text{RMSE: } 0.002527
\end{aligned}
\]

并用作出三条曲线,如图 1.1,发现生育概率的主要变化是生育时间向后推迟了,这主要是受女性教育程度及女性就业状况的影响,所以预测妇女生育概率在以后相当长一段时间内不会有大的变化,所以在求解过程中假定其不随时间 $t$ 改变而改变。并在预测其生育率时,以 2011 年为标准,选取高斯模型。

\begin{figure}[h]
    \centering
    \includegraphics[width=\textwidth]{image.png}
    \caption{生育模式拟合图}
    \label{fig:1.1}
\end{figure}

\subsection{死亡率 \(\delta_i(t)\) 的确定}

随着经济的发展,人民生活水平的提高,医疗卫生状况的改善,未来人口死亡率将呈下降趋势。青壮年人口的死亡主要由意外事故或严重病情等因素导致,因此一般认为,其死亡率和经济状况以及医疗设施关系不大,下降较为明显的主要 是 0-5 岁婴幼儿和 50 岁以上的老年人。根据中国人口年龄死亡率现状,应用分段加权法设定未来较长一段时间内的人口死亡率函数的变化趋势:

\begin{equation}
\delta_i(t) =
\begin{cases}
\delta_i(t-1)(1-e_1) & i \leq 5 \\
\delta_i(t-1)(1-e_2) & 6 < i < 50 \\
\delta_i(t-1)(1-e_3) & 50 \leq i
\end{cases}
\quad i=0,1,2,\dots,n
\tag{1-12}
\end{equation}

根据 2003 年到 2011 年的数据确定 \(e_1=0.04\),\(e_2=0.03\),\(e_3=0.01\),其中 \(t\) 表示年代,男性和女性死亡率的变化规律都可以用 \(\delta_i(t)\) 表示,取不同初始值即可。

\subsection{女婴比 \(k(t)\) 的确定}

目前我国人口出生性别比在 110~120 之间波动,此种状况要维持相当长的一段时间,而自然条件下性别比应该在 103~107 之间,并且每一年的具体值有一定随机性,据此建立如下模型来模拟性别比:

\begin{equation}
Sr(t) =
\begin{cases}
-\frac{1}{4}t + 115 + \varepsilon_1 & (2012 \leq t \leq 2045) \\
105 + \varepsilon_2 & (t > 2045)
\end{cases}
\tag{1-13}
\end{equation}

$Sr(t)$ 为 $t$ 年出生性别比,$\varepsilon_1$,$\varepsilon_2$ 分别服从 $(0 \sim 5)$ 和 $(-2 \sim 2)$ 上均匀分布的随机变量,由此可以得到女婴比 $k(t)$ 表达式:

\begin{equation}
k(t) = \frac{100}{Sr(t) + 100}
\tag{1-14}
\end{equation}

### 4.1.4.4 总和生育率 $\beta(t)$ 的估计

总和生育率指假设妇女按照某一年的年龄别生育率度过育龄期,平均每个妇女在育龄期生育的孩子数。本文假设我国的总和生育率一直维持在 1.8。

### 4.1.4.5 城市化率 [4]

每年农村向城市的增长人口为:

\begin{equation}
Z(t) = r_c(t) \times p(t)
\tag{1-15}
\end{equation}

其中,$r_c(t)$ 代表城镇化水平增长量(城市化水平每年增长的百分点);$p(t)$ 代表全国男性或者女性的人口总量;

#### 表1.1 城镇化趋势表

\begin{table}[h]
\centering
\begin{tabular}{|c|c|c|c|c|c|c|c|c|c|c|}
\hline
 & 2003 & 2004 & 2005 & 2006 & 2007 & 2008 & 2009 & 2010 & 2011 & 2012 \\ \hline
城镇居民所占百分比 & 40.5 & 41.8 & 43 & 44.3 & 45.9 & 47 & 48.3 & 50 & 51.3 & 52.6 \\ \hline
城镇化水平增长量 $r_c(t)$ (\%) & --- & 1.3 & 1.2 & 1.3 & 1.6 & 1.1 & 1.3 & 1.7 & 1.3 & 1.3 \\ \hline
\end{tabular}
\end{table}

由国家统计局公布的数据,可以得到表1.1,以近几年的城镇化水平增量 $r_c(t)$ 的平均值作为计算参数,即 $r_c(t) = 0.01344$。据估计,20年内城市化速率不会发生太大的变化,但随着城市人口比例的增加,速率也会随之下降。参照其他国家城市化的发展历程,假设2100年城市化速率变为0,于是可以建立下面数学表达式:

\begin{equation}
R_c(t) =
\begin{cases}
0.01344, & 2012 \leq t \leq 2025 \\
0.01344 - \frac{0.01344}{75}(t - 2025), & t > 2025
\end{cases}
\tag{1-16}
\end{equation}

由图和表可知,城镇人口的失业率,剔除2008和2009年(由于经济危机),会发现基本稳定4.1%,并且会保持长期稳定。

\begin{table}[h]
\centering
\caption{城乡就业趋势表}
\begin{tabular}{|c|c|c|c|c|c|c|c|c|c|}
\hline
 & 2003 & 2004 & 2005 & 2006 & 2007 & 2008 & 2009 & 2010 & 2011 \\
\hline
就业人口(城镇) & 26230 & 27293 & 28389 & 29630 & 30953 & 32103 & 33322 & 34687 & 35914 \\
\hline
就业人口(乡村) & 47506 & 46971 & 46258 & 45348 & 44368 & 43461 & 42506 & 44418 & 40506 \\
\hline
总就业人口 & 73736 & 74264 & 74647 & 74978 & 75321 & 75564 & 75828 & 76105 & 76420 \\
\hline
城镇登记失业人口 & 800 & 827 & 839 & 847 & 830 & 886 & 921 & 908 & 922 \\
\hline
城镇登记失业率sy(t)(\%) & 4.3 & 4.2 & 4.2 & 4.1 & 4.0 & 4.2 & 4.3 & 4.1 & 4.1 \\
\hline
城镇人口 & 52376 & 54283 & 56212 & 58288 & 60633 & 62403 & 64512 & 66978 & 71182 \\
\hline
乡村人口 & 76851 & 75705 & 74544 & 73160 & 71496 & 70399 & 68938 & 67113 & 64222 \\
\hline
总人口 & 129227 & 129988 & 130756 & 131448 & 132129 & 132802 & 133450 & 134091 & 134735 \\
\hline
城镇人口就业比率jyc(t) & 0.50080 & 0.50279 & 0.50503 & 0.50833 & 0.51049 & 0.51444 & 0.51652 & 0.51788 & 0.51989 \\
 & 2 & 1 & 5 & 8 & 8 & 6 & 4 & 6 & 8 \\
\hline
城镇人口就业比率 & --- & 0.00198 & 0.00224 & 0.00303 & 0.00216 & 0.00394 & 0.00207 & 0.00136 & 0.00201 \\
 & & & 4 & & 9 & 8 & 8 & 2 & 1 \\
\hline
\end{tabular}
\end{table}

\begin{table}
\centering
\begin{tabular}{|c|c|c|c|c|c|c|c|c|c|}
\hline
增长量 & & & & & & & & & \\
jycz(t) & & & & & & & & & \\
\hline
乡村就业 & 0.61815 & 0.62044 & 0.62054 & 0.61984 & 0.62056 & 0.61735 & 0.61658 & 0.61713 & 0.61694 \\
率jyx(t) & 7 & 8 & 6 & 7 & 6 & 3 & 3 & 8 & 3 \\
\hline
乡村就业 & & 0.00229 & 0.00009 & -0.0006 & 0.00071 & -0.0032 & -0.0007 & 0.00055 & -0.0001 \\
比率增长 & & 1 & 8 & 9934 & 9275 & 13666 & 69503 & 5243 & 95406 \\
量jyxz(t) & & & & & & & & & \\
\hline
总就业率 & 0.57059 & 0.57131 & 0.57088 & 0.5704 & 0.57005 & 0.56899 & 0.56821 & 0.56756 & 0.56718 \\
jyz(t) & 3 & 4 & 8 & 7 & 7 & 3 & 2 & 7 & \\
\hline
总就业比 & & 0.00072 & -0.0004 & -0.0004 & -0.0003 & -0.0010 & -0.0007 & -0.0005 & -0.0003 \\
率增加量 & & 1 & 3 & 9 & 4 & 6 & 8 & 6 & 7 \\
jyzz(t) & & & & & & & & & \\
\hline
\end{tabular}
\end{table}

我国企业职工基本养老保险实行“社会统筹”与“个人账户”相结合的模式,即企业把职工工资总额按一定比例(20%)缴纳到社会统筹基金账户,再把职工个人工资按一定比例(8%)缴纳到个人账户。这两个账户我们合称为养老保险基金。退休后,按职工在职期间每月(或年)的缴费工资与社会平均工资之比(缴费指数),再考虑到退休前一年的社会平均工资等因素,从社会统筹账户中拨出资金(基础养老金),加上个人工资账户中一定比例的资金(个人账户养老金),作为退休后每个月的养老金。养老金会随着社会平均工资的调整而调整。如果职工死亡,社会统筹账户中的资金不退给职工,个人账户中的余额可继承。个人账户储存额以银行当时公布的一年期存款利率计息,为简单起见,利率统一设定为3%。

参加市城镇企业职工基本养老保险社会统筹的人员,达到国家规定的退休年龄,实际缴费年限满15年以上的,按月计发基本养老金。按照2005年颁布的《国务院关于完善企业职工基本养老保险制度的决定》和《百度百科:养老金》[5]等材料,可以得到养老金的如下计算方法[6]:

职工退休时的养老金由两部分组成:养老金=基础养老金+个人账户养老金;个人账户养老金=个人账户储存额÷计发月数,计发月数可以查表得到,因而只需计算个人账户储存额。

下面先来计算个人账户储存额,职工个人每年将工资按照一定缴费率J,缴纳到个人账户作为个人账户储存额,个人账户储存额以每年3.25%计息。

设$C_{i}$表示职工第i年的年平均工资,m表示职工退休年龄养老金缴纳的年限,则第i年的账户储存额$C_{i}\times J_{r}$将每年以年利率$r=3.25\%$增长,到达退休年龄时的第i年缴纳本息和应为:$C_{i}\times J_{r}\times(1+r\%)^{m-i}$,则该职工m年总的个人账户储存额

\[
=\sum_{i=1}^{m}C_{i}\times J_{r}\times(1+r)^{m-i}
\]

基础养老金=(全国上年度在岗职工月平均工资+本人指数化月平均缴费工12

资)÷2×缴费年限×1%;

本人指数化月平均缴费工资=全国上年度在岗职工月平均工资×本人平均缴费指数;

能够正确反映“本人指数化月平均缴费工资 (Average Indexed Monthly Earnings)”指标的计算公式为:

\[
S = \frac{x_1 \times \frac{c_1}{c_1} + x_2 \times \frac{c_1}{c_2} + \cdots + x_m \times \frac{c_1}{c_m}}{n}
\tag{1-18}
\]

公式 (1-18) 中,$x_1, x_2, \cdots, x_m$ 为参保人员退休前 1 年、2 年、……、m 年本人缴费工资额;$c_1, c_2, \cdots, c_m$ 为参保人员退休前 1 年、2 年、……、m 年全国“职工平均工资”或称“社会平均工资”;n 为企业和职工实际缴纳基本养老保险费的月数合计(可以简单认为等于 12m,m 为企业和职工实际缴纳基本养老保险费的年限);$\frac{x_i}{c_i}$ 称为退休前第 i 年的缴费指数,$i=1, \cdots, m$。

参保人员 i 年度的本人缴费工资 $x_i$ 通过工资指数 $\frac{c_1}{c_i}$ 得到指数化缴费工资 $x_i \times \frac{c_1}{c_i}$,从而使各年度不可比的 $x_i$ 换算为相当于参保人员退休前 1 年社会平均工资 $c_1$ 水平的、可比的各年度指数化缴费工资 $x_i \times \frac{c_1}{c_i}$,各年度指数化缴费工资 $x_i \times \frac{c_1}{c_i}$ 加总再除以参保人员实际缴费月数和 n,进而得到本人指数化月平均缴费工资 S。由此,该指标能够反映参保人员在整个缴费年限的缴费工资平均水平。为了简化模型,我们将个人账户养老金计发月数 J 选择为男、女两类,即男 60 岁,计发月数为 139 个月,女 55 岁,计发月数为 170 个月。所以,职保的替代率模型为:

\[
R = \frac{\left(c_1 + \frac{x_1 \times \frac{c_1}{c_1} + x_2 \times \frac{c_1}{c_2} + \cdots + x_m \times \frac{c_1}{c_m}}{12m}\right) / 2 \times m \times 1\% + \sum_{i=1}^m C_i \times J_r \times (1 + r)^{m-i} / J}{c_1 / 12}
\tag{1-19}
\]

由此可以得到男女两类的任意缴费年限、任意年份的替代率,如表 1.3 和表 1.4 所示。

表 1.3 不同缴费年限、年份的男职工替代率

\begin{table}
\centering
\begin{tabular}{c c c c c c}
\hline
 & 2013年 & 2020年 & 2030年 & 2040年 & 2050年 \\
\hline
30 & 27.86\% & 30.94\% & 34.57\% & 37.07\% & 39.97\% \\
35 & 32.47\% & 36.06\% & 40.36\% & 43.50\% & 47.28\% \\
40 & 37.41\% & 41.58\% & 46.61\% & 50.37\% & 55.09\% \\
45 & 42.75\% & 47.60\% & 53.46\% & 57.87\% & 63.53\% \\
50 & 48.57\% & 54.21\% & 61.02\% & 66.17\% & 72.84\% \\
\hline
\end{tabular}
\caption{不同缴费年限、年份的女职工替代率}
\end{table}

\begin{table}
\centering
\begin{tabular}{c c c c c c}
\hline
 & 2013年 & 2020年 & 2030年 & 2040年 & 2050年 \\
\hline
30 & 25.73\% & 28.29\% & 31.34\% & 33.44\% & 35.87\% \\
35 & 29.95\% & 32.94\% & 36.53\% & 39.16\% & 42.33\% \\
40 & 34.44\% & 37.91\% & 42.10\% & 45.24\% & 49.19\% \\
45 & 39.27\% & 43.29\% & 48.16\% & 51.84\% & 56.56\% \\
50 & 44.48\% & 49.16\% & 54.80\% & 59.09\% & 64.63\% \\
\hline
\end{tabular}
\end{table}

4.1.5.2 新农保养个人账户养老金的替代率 \(R_{x}^{nb}\) 的计算 \({}^{[7]}\)

2009年9月,国务院发布《关于开展新型农村社会养老保险试点的指导意见》,正式提出建立个人缴费、集体补助和政府补贴相结合的新型农村社会养老保险制度。新农保的基金筹集采取个人缴费、集体补助和政府补贴的方式。个人缴费采取固定分档费额机制,参保人可自主选择每年100元、200元、300元、400元和500元等五个档次缴费,地方政府可根据实际情况增设缴费档次,同时要求有条件的村集体应当对参保人缴费给予补助。同时,政府对符合领取条件的参保人全额支付新农保基础养老金,且地方政府承担补贴参保人缴费的责任,补贴标准不低于每人每年30元。新农保的基金管理采取完全积累的个人账户制,个人缴费、集体补助和地方政府的财政补贴,全部记入个人账户。个人账户基金参考当年的中国人民银行公布的金融机构人民币一年期存款利率计息。新农保养老金待遇分为基础养老金和个人账户养老金两部分组成。参保人年满60周岁,并累计缴费15年,即可开始申领养老金。个人账户养老金的月计发标准为个人账户全部储存额除以139。

在我国,养老金替代率常用以下定义:一是目标替代率。目标替代率是指单个参保人给付期第一年的养老金与给付期前一年收入的比率,这一指标是以个人为对象的;二是平均替代率。平均替代率是指社会平均养老金与社会平均劳动收入的比率;三是退休人口养老金总额与当年劳动人口收入总额之比。它能够反映出整个社会的负担程度。本方案是以新型农村社会养老保险的参保个人作为主要调研对象的,所以采用的是目标替代率的定义,即新型农村社会养老保险个人账户替代率定义为参保人给付期的平均养老金收入和给付期前一年收入的比值。

根据保险精算的平衡原则,即要求参保人缴费期基金的积累总额等于给付期基金领取总额现值。给付期第一年参保农民开始领取养老金时个人账户的基金积累总额 \(M\) 为:

\[
M = (C_1 + C_2) \sum_{i=1}^{60-n} (1+r)^{60-n+1-i}
\tag{1-20}
\]

其中:$C_{1}$ 为个人年缴费金额,分为 100, 200, 300, 400, 500 元五个档次;$C_{2}$ 为政府补贴个人账户年缴费金额,补贴标准不低于每人每年 30 元,对选择较高档次标准缴费的,可给予适当鼓励,按 30 元/年计算;$n$ 为参保农民开始缴费年龄;$r$ 为个人账户养老金积累基金的年计息利率,按 2013 年 9 月的利率计算,即为 3.25%。

$n$ 岁参保农民在 59 岁终止缴费后各年的养老金给付额在 60 岁时的领取总额现值为:

\begin{equation}
N = 12P \sum_{j=0}^{m-1} \left( \frac{1}{1+l} \right)^j
\tag{1-21}
\end{equation}

其中:$P$ 为个人账户养老金月给付额;$m$ 为参保农民个人账户养老金的领取期限,新农保制度个人账户养老金的月计发标准为个人账户全部储存额除以 139 个月,因此领取年限 $m=139/12=11.58$,取整按 $m=12$ 年计算;$l$ 为养老金给付期间计息利率,按 2013 年 9 月的利率计算,即为 3.25%;$\frac{1}{1+l}$ 为给付期的年贴现因子。

根据精算原理,个人账户的纵向平衡要求基金积累总额($M$)等于未来给付精算现值($N$),即 $M=N$,由此可以得出养老金月给付额为:

\begin{equation}
p = \frac{M}{12 \sum_{j=0}^{m-1} \left( \frac{1}{1+l} \right)^j} = \frac{(C_{1} + C_{2}) \sum_{i=1}^{60-n} (1+r)^{60-n+1-i}}{12 \sum_{j=0}^{m-1} \left( \frac{1}{1+l} \right)^j}
\tag{1-22}
\end{equation}

则个人账户养老金的替代率为:

\begin{equation}
R_{t}^{xnb} = \frac{12P}{W} = \frac{12P}{W_{0} \times (1+k)^{60-n}}
\tag{1-23}
\end{equation}

其中:$W$ 为给付期前一年的农村居民人均纯收入;$W_{0}$ 为缴费期开始前一年的农村居民人均纯收入;$k$ 为农村居民人均纯收入年增长率;$n$ 为参保农民开始缴费年龄。

对于公式 (1-23),缴费期开始前一年的农村居民人均纯收入,即 $(W_{0})$ 在 2013 年取 2012 年农村居民人均纯收入,即 $W_{0}=7917$ 元,根据农民纯收入的不断变化不断不更新。至于精算期内农村居民人均纯收人年增长率 $K$,拟以预测的农民人均收入为依据,即到 2050 年农村居民人均纯收人将达到 57913 元的。根据这个目标测算,则农村人均纯收人将保持平均 5.38% 的速度增长。据此,本文假设 $k=5.38\%$。我们可以得到任意档次、任意年龄、任意年份的替代率。

\begin{table}
\centering
\caption{2013年不同年龄、不同档次的替代率}
\begin{tabular}{c c c c c c}
\hline
 & 100元 & 200元 & 300元 & 400元 & 500元 \\
\hline
16岁 & 1.48\% & 2.63\% & 3.77\% & 4.91\% & 6.05\% \\
20岁 & 1.54\% & 2.72\% & 3.91\% & 5.09\% & 6.28\% \\
25岁 & 1.59\% & 2.81\% & 4.04\% & 5.26\% & 6.49\% \\
30岁 & 1.61\% & 2.86\% & 4.10\% & 5.34\% & 6.58\% \\
35岁 & 1.59\% & 2.82\% & 4.05\% & 5.28\% & 6.50\% \\
40岁 & 1.52\% & 2.68\% & 3.85\% & 5.02\% & 6.18\% \\
45岁 & 1.35\% & 2.40\% & 3.44\% & 4.48\% & 5.52\% \\
50岁 & 1.08\% & 1.91\% & 2.73\% & 3.56\% & 4.39\% \\
55岁 & 0.64\% & 1.14\% & 1.64\% & 2.13\% & 2.63\% \\
59岁 & 0.15\% & 0.26\% & 0.38\% & 0.49\% & 0.61\% \\
\hline
\end{tabular}
\end{table}

\begin{table}
\centering
\caption{2020年不同年龄、不同档次的替代率}
\begin{tabular}{c c c c c c}
\hline
 & 100元 & 200元 & 300元 & 400元 & 500元 \\
\hline
16岁 & 0.87\% & 1.54\% & 2.21\% & 2.88\% & 3.56\% \\
20岁 & 0.90\% & 1.60\% & 2.30\% & 2.99\% & 3.69\% \\
25岁 & 0.93\% & 1.65\% & 2.37\% & 3.09\% & 3.81\% \\
30岁 & 0.95\% & 1.68\% & 2.41\% & 3.14\% & 3.87\% \\
35岁 & 0.94\% & 1.66\% & 2.38\% & 3.10\% & 3.82\% \\
40岁 & 0.89\% & 1.58\% & 2.26\% & 2.95\% & 3.63\% \\
45岁 & 0.80\% & 1.41\% & 2.02\% & 2.63\% & 3.24\% \\
50岁 & 0.63\% & 1.12\% & 1.61\% & 2.09\% & 2.58\% \\
55岁 & 0.38\% & 0.67\% & 0.96\% & 1.25\% & 1.54\% \\
59岁 & 0.09\% & 0.15\% & 0.22\% & 0.29\% & 0.36\% \\
\hline
\end{tabular}
\end{table}

4.1.5.3 城居保个人账户养老金的替代率 \(R_{i}^{jb}\) 的计算

2011年7月1日,城镇居民养老保险在全国层面试点推行,这是继2009年新型农村社会养老保险试点后党中央、国务院为加快建设覆盖城乡居民的社会保障体系作出的又一重大战略部署,是促进和谐社会建设的一项重大民心工程,意味着我国人人都“老有所养”的千年夙愿将基本得以实现。指导意见称,城镇居民养老保险基金主要由个人缴费和政府补贴构成。缴费标准目前设为每年100元、200元、300元、400元、500元、600元、700元、800元、900元、1000元10个档次,地方人民政府可以根据实际情况增设缴费档次。参保人自主选择档次缴费,多缴多得。地方人民政府应对参保人员缴费给予补贴,补贴标准不低于每人每年30元;对选择较高档次标准缴费的,可给予适当鼓励,具体标准和办法由省(区、市)人民政府确定。对城镇重度残疾人等缴费困难群体,地方人民政府为其代缴部分或全部最低标准的养老保险费。国家为每个参保人员建立终身记录的养老保险个人账户。养老金待遇由基础养老金和个人账户养老金构成,支付终身。中央确定的基础养老金标准为每人每月55元。参加城镇居民养老保险的城镇居民,年满60周岁,可按月领取养老金。城镇居民养老保险制度实施时,已年满60周岁,未享受职工基本养老保险待遇以及国家规定的其他养老待遇。

遇的,不用缴费,可按月领取基础养老金;距领取年龄不足 15 年的,应按年缴费,也允许补缴,累计缴费不超过 15 年;距领取年龄超过 15 年的,应按年缴费,累计缴费不少于 15 年。

本方案同样采用目标替代率作为研究对象。

根据保险精算的平衡原则,即要求参保人缴费期基金的积累总额等于给付期基金领取总额现值。给付期第一年城镇居民开始领取养老金时个人账户的基金积累总额 \( M \) 为:

\[
M = (C_1 + C_2) \sum_{i=1}^{60-n} (1+r)^{60-n+1-i}
\tag{1-24}
\]

其中: \( C_1 \) 为个人年缴费金额,分为 \( 100, 200, 300, 400, 500 \), \( 600, 700, 800, 900, 1000 \) 元十个档次; \( C_2 \) 为政府补贴个人账户年缴费金额,本着参保人自主选择档次缴费,多缴多得的原则,我们将其体现在政府补贴个人账户年缴费金额上面,即参保人如果选择缴费档次 100 元,政府每人每年补贴 30 元,选择缴费档次 200 元,每人每年补贴 35 元,选择缴费档次 300 元,每人每年补贴 40 元,选择缴费档次 400 元,每人每年补贴 45 元,选择缴费档次 500 元及其以上,每人每年补贴 50 元; \( n \) 为参保城镇居民开始缴费年龄; \( r \) 为个人账户养老金积累基金的年计息利率,按 2013 年 9 月的利率计算,即为 \( 3.25\% \); \( n \) 岁参保城镇居民在 59 岁终止缴费后各年的养老金给付额在 60 岁时的领取总额现值为:

\[
N = 12P \sum_{j=0}^{m-1} \left( \frac{1}{1+l} \right)^j
\tag{1-25}
\]

其中: \( P \) 为个人账户养老金月给付额; \( m \) 为参保城镇居民个人账户养老金的领取期限,城居保制度个人账户养老金的月计发标准为个人账户全部储存额除以 139 个月,因此领取年限 \( m = 139/12 = 11.58 \),取整按 \( m = 12 \) 年计算; \( l \) 为养老金给付期间计息利率,按 2013 年 9 月的利率计算,即为 \( 3.25\% \); \( \frac{1}{1+l} \) 为给付期的年贴现因子。

根据精算原理,个人账户的纵向平衡要求基金积累总额 \( (M) \) 等于未来给付精算现值 \( (N) \),即 \( M = N \),由此可以得出养老金月给付额为:

\[
p = \frac{M}{12 \sum_{j=0}^{m-1} \left( \frac{1}{1+l} \right)^j} = \frac{(C_1 + C_2) \sum_{i=1}^{60-n} (1+r)^{60-n+1-i}}{12 \sum_{j=0}^{m-1} \left( \frac{1}{1+l} \right)^j}
\tag{1-26}
\]

则个人账户养老金的替代率为:

\begin{equation}
R_{t}^{jb} = \frac{12P}{W} = \frac{12P}{W_{0} \times (1+k)^{60-n}}
\tag{1-27}
\end{equation}

其中:W为给付期前一年的城镇居民人均纯收入;$W_{0}$为缴费期开始前一年的城镇居民人均纯收入;k为城镇居民人均纯收入年增长率;n为参保城镇居民开始缴费年龄。对于公式(1-27),缴费期开始前一年的城镇居民人均纯收入,即($W_{0}$)在2012年取2011年农村居民人均纯收人,即$W_{0}$=23979元,根据农民纯收入的不断变化不断不更新。至于精算期内农村居民人均纯收人年增长率K,拟以预测的农民人均收入为依据,即到2050年农村居民人均纯收人将达到197210元的。根据这个目标测算,则农村人均纯收人将保持平均5.55%的速度增长。据此,本文假设k=5.55%。可以得到任意档次、任意年龄、任意年份的替代率。

\textbf{表1.7 2013年不同年龄、不同档次的替代率}

\begin{table}[h]
\centering
\begin{tabular}{|c|c|c|c|c|c|c|c|c|c|c|}
\hline
 & 100元 & 200元 & 300元 & 400元 & 500元 & 600元 & 700元 & 800元 & 900元 & 1000元 \\
\hline
16岁 & 0.41\% & 0.73\% & 1.06\% & 1.39\% & 1.72\% & 2.05\% & 2.37\% & 2.70\% & 3.03\% & 3.36\% \\
\hline
25岁 & 0.44\% & 0.80\% & 1.15\% & 1.51\% & 1.87\% & 2.22\% & 2.58\% & 2.94\% & 3.29\% & 3.65\% \\
\hline
35岁 & 0.45\% & 0.81\% & 1.18\% & 1.54\% & 1.90\% & 2.27\% & 2.63\% & 2.99\% & 3.36\% & 3.72\% \\
\hline
45岁 & 0.39\% & 0.70\% & 1.02\% & 1.33\% & 1.64\% & 1.96\% & 2.27\% & 2.58\% & 2.90\% & 3.21\% \\
\hline
55岁 & 0.19\% & 0.34\% & 0.49\% & 0.64\% & 0.79\% & 0.95\% & 1.10\% & 1.25\% & 1.40\% & 1.55\% \\
\hline
\end{tabular}
\end{table}

\textbf{表1.8 2020年不同年龄、不同档次的替代率}

\begin{table}[h]
\centering
\begin{tabular}{|c|c|c|c|c|c|c|c|c|c|c|}
\hline
 & 100元 & 200元 & 300元 & 400元 & 500元 & 600元 & 700元 & 800元 & 900元 & 1000元 \\
\hline
16岁 & 0.81\% & 1.47\% & 2.12\% & 2.78\% & 3.44\% & 4.09\% & 4.75\% & 5.41\% & 6.06\% & 6.72\% \\
\hline
25岁 & 0.88\% & 1.60\% & 2.31\% & 3.02\% & 3.74\% & 4.45\% & 5.17\% & 5.88\% & 6.59\% & 7.31\% \\
\hline
35岁 & 0.90\% & 1.63\% & 2.35\% & 3.08\% & 3.81\% & 4.54\% & 5.26\% & 5.99\% & 6.72\% & 7.44\% \\
\hline
45岁 & 0.78\% & 1.40\% & 2.03\% & 2.66\% & 3.29\% & 3.91\% & 4.54\% & 5.17\% & 5.79\% & 6.42\% \\
\hline
55岁 & 0.38\% & 0.68\% & 0.98\% & 1.29\% & 1.59\% & 1.89\% & 2.19\% & 2.50\% & 2.80\% & 3.10\% \\
\hline
\end{tabular}
\end{table}

4.1.6 经济预测模型

4.1.6.1 经济增速预测模型

经济运行有其内在的周期,学界一般将其分为短周期、中周期与长周期。由于金融危机对经济衰退产生加速作用,经济运行从衰退到触底(衰退—萧条)的持续时间大为缩短。政府扩张性政策的实施,一定程度上破坏了经济体内在的自组织机能,提前动用了经济体复苏和繁荣阶段的增长潜能,使后两个阶段的发展相对“贫血”,复苏和达到繁荣阶段所需的时间被延长。

由于中国改革开放以来只有 30 年时间,因此,至多只能进行 10 年左右的中周期分析。2002-2007 年,中国经济增速逐年攀升,2007 年 GDP 增速达到本轮中周期的峰值 11.9%,已处于经济运行趋热区间。而从 2008 年开始,中国经济运行进入第三个中周期的下行调整通道,经济增速将从峰值逐步向低谷回落。而对于更长远的经济增长,各路专家均表示,中国经济的高速增长时代已经结束,未来中国经济增速将回归至正常区间。

因此本文采用 2013 年-2015 年的中周期预测模型加上未来长时间内的经济增长趋于减速稳定的模型建立经济增速模型 \({}^{[8]}\)。其中,减速稳定模型结合经济周期理论、经济增长 S 曲线轨迹、中国未来人口自然增长率趋于稳定的实际情况、未来中国经济增长的主要因素贡献度可能递减和许宪春等对 GDP 的预测等综合考虑,未来 50 年中国经济整体分为三个阶段:第一阶段,2000~2020 年,为全面建设小康社会阶段,中国经济增长率平均将为 \(6.5\% \pm 1\%\);第二阶段,2020~2040 年,为初步实现现代化社会阶段,中国经济增长率平均将为 \(4.5\% \pm 1\%\);第三阶段,2040~2050,为基本实现社会现代化阶段,中国经济增长率平均将为 \(2.5\% \pm 1\%\)。

对经济增长率的预测主要使用减速稳定的经济模型与外在事件对因变量所产生的影响的双变量 ARIMA 模型:

\begin{equation}
\ln Z_{t} = N_{t} + \sum W_{it} \times I_{it}
\tag{1-28}
\end{equation}

其中,\(Z_{t}\) 为 \(t\) 时刻经济增长率;\(W_{it}\) 为第 \(i\) 个外部事件在 \(t\) 时刻的波动参数;\(I_{it}\) 为第 \(i\) 个外部事件在 \(t\) 时刻的波动函数;\(N_{t}\) 为减速稳定的经济模型。

一般地,影响经济增长的因素很多,但是在 2008-2016 年中,造成经济增速减缓的最主要的原因仍然是金融危机。

因此,在探讨经济增长的预测模型时,我们假设外部事件发生前后其他的影响因素都保持不变,因此金融危机对入境旅游人数的影响可以表示为:

\begin{equation}
\ln Z_{t} = W_{t} \times I_{t} + N_{t}
\tag{1-29}
\end{equation}

其中,金融危机的波动函数 \(I_{t}\) 可以用脉冲函数来表示即:

\begin{equation}
I_{t} =
\begin{cases}
1, & t \text{ 为有金融危机影响的时期} \\
0, & t \text{ 为无金融危机影响的时期}
\end{cases}
\tag{1-30}
\end{equation}

\(N_{t}\) 为减速稳定的经济模型,即

\begin{equation}
e^{N_{t}}=\begin{cases}
6.5\%+(2020-t)\times 0.59\%, & 2008\leq t<2020 \\
4.5\%+(2040-t)\times 0.1\%, & 2020\leq t<2040 \\
2.5\%+(2050-t)\times 0.2\%, & 2040\leq t<2050
\end{cases}
\tag{1-31}
\end{equation}

代入得

\begin{equation}
\ln Z_{t}=\begin{cases}
W_{t}+N_{t}, & t\text{为有金融危机影响的时期} \\
N_{t}, & t\text{为无金融危机影响的时期}
\end{cases}
\tag{1-32}
\end{equation}

我们把2008年看成是金融危机影响的初期,根据式(1-32)计算得

\textbf{表 1.9 2008-2012年金融危机影响波动因子}

\begin{tabular}{|c|c|c|c|c|c|}
\hline 年份 & 2008 & 2009 & 2010 & 2011 & 2012 \\
\hline 经济增长率$Z_{t}$ & 9.6\% & 9.2\% & 10.4\% & 9.3\% & 7.7\% \\
\hline 预期经济增长率 & 13.61\% & 13.02\% & 12.43\% & 11.84\% & 11.25\% \\
\hline 波动参数$W_{t}$ & -0.3490 & -0.3473 & -0.1783 & -0.2415 & -0.3791 \\
\hline
\end{tabular}

从表1.9可以看到,外在因素在2010年对经济增长的影响是最小的,而同时,世界经济在2010年同样表现出强劲增长态势,但2011年仍然下降了2.54\%,说明金融危机仍然是罪魁祸首,这也就间接证明了以上的论述是有根据的。

假设在金融危机恢复期$n$年后同比增长率与预期相同,经济增长率也将随时间等比例增大到正常水平。假设波动参数$W_{t}$的参数值等比例增大到恢复期的参数值($W_{p}$为2007年的同期增长率的波动参数,$W_{p}=0$),设等比例因子为$\alpha$,则由:

\begin{equation}
\begin{cases}
W_{t-n}=W_{t_{2013}}+n\alpha \\
W_{t}=W_{t_{2013}}+\alpha(t-2012) \\
W_{t-n}=W_{p}
\end{cases}
\tag{1-33}
\end{equation}

得:

\begin{equation}
W_{t}=W_{t_{2012}}+(W_{t-n}-W_{t_{2012}})\times(t-2012)/n=-0.379+(0+0.379)\times(t-2012)/n
\tag{1-34}
\end{equation}

假设根据2011年12月份后,再过5年恢复到预期经济增长速度的条件来估算金融危机对经济增长速度的影响:

根据方程$W_{t}=W_{t_{2012}}+(W_{t-n}-W_{t_{2012}})\times(t-2012)/n$分别计算金融危机影响恢复

\begin{table}[h]
\centering
\begin{tabular}{|c|c|c|c|c|}
\hline
年份 & 2012 & 2013 & 2014 & 2015 \\
\hline
预期经济增长率 & 7.7\% & 7.86\% & 8.00\% & 8.12\% \\
\hline
波动参数 & -0.3791 & -0.3033 & -0.228 & -0.1527 \\
\hline
年份 & 2016 & 2017 & 2018 & 2019 \\
\hline
预期经济增长率 & 8.21\% & 8.28\% & 7.68\% & 7.09\% \\
\hline
波动参数 & -0.0774 & 0 & 0 & 0 \\
\hline
\end{tabular}
\caption{金融危机恢复期预测经济增长率}
\end{table}

\begin{equation}
Z_t =
\begin{cases}
4.5\% + (2040 - t) \times 0.1\%, & 2020 \leq t < 2040 \\
2.5\% + (2050 - t) \times 0.2\%, & 2040 \leq t \leq 2050
\end{cases}
\tag{1-34}
\end{equation}

\section*{4.1.6.2 投资收益模型}

由于我国现在人口老龄化的逐步加剧,养老保险基金的支付压力也越来越大。加强基金管理水平、切实做到保值增值,对缓解养老保险基金支付压力,保证我国基本养老保险制度的平稳运行有着重要的作用。养老基金的养老保障功能决定了其投资原则的顺序是:安全性、收益性、流动性,即必须在保证基金安全的基础上提高基金的收益率,同时确保其流动性需要。

\begin{table}[h]
\centering
\begin{tabular}{|c|c|c|c|c|c|c|c|}
\hline
 & 投资收益率 & 投资收益额 & 通货膨胀率 & 投资额 & GDP增长速度 & 投资比例 & 结余总额 \\
\hline
2001 & 0.0173 & 7.42 & 0.7 & 428.9 & 0.083 & 0.4069 & 1054.1 \\
\hline
2002 & 0.0259 & 19.77 & 0.8 & 763.32 & 0.0908 & 0.4747 & 1608 \\
\hline
2003 & 0.0356 & 44.71 & 1.2 & 1255.9 & 0.1003 & 0.5692 & 2206.5 \\
\hline
2004 & 0.0261 & 36.72 & 3.9 & 1406.9 & 0.1009 & 0.4729 & 2975 \\
\hline
2005 & 0.0416 & 71.22 & 1.8 & 1712.02 & 0.1131 & 0.4237 & 4041 \\
\hline
2006 & 0.2901 & 619.79 & 1.5 & 2136.47 & 0.1268 & 0.3892 & 5488.9 \\
\hline
2007 & 0.4319 & 1453.5 & 4.8 & 3365.36 & 0.1416 & 0.4553 & 7391.4 \\
\hline
2008 & -0.0679 & -393.72 & 5.9 & 5798.53 & 0.0963 & 0.5839 & 9931 \\
\hline
2009 & 0.1612 & 850.43 & -0.7 & 5275.62 & 0.0921 & 0.4212 & 12526.1 \\
\hline
2010 & 0.0423 & 321.22 & 3.3 & 7593.85 & 0.1045 & 0.4942 & 15365.3 \\
\hline
2011 & 0.0084 & 73.37 & 5.4 & 8734.52 & 0.093 & 0.448 & 19497 \\
\hline
平均 & 0.092045455 & & 2.6 & & & 0.4672 & \\
\hline
\end{tabular}
\caption{投资收益参数表}
\end{table}

根据2001年公布的《全国社会保障基金投资管理暂行办法》的规定,社保基金投资的范围限于银行存款、买卖国债和其他具有良好流动性的金融工具,包括上市流通的证券投资基金、股票、信用等级在投资级以上的企业债、金融债等有价证券。其中,银行存款和国债投资的比例不得低于50\%,企业债、金融债投资的比例不得高于10\%,证券投资基金、股票投资的比例不得高于40\%,且不允许投人高风险高收益的项目。

\begin{table}
\centering
\caption{投资收益率与GDP的关系模拟}
\begin{tabular}{c c c c c c c c c}
\hline
\multirow{2}{*}{Equation} & \multicolumn{5}{c}{Model Summary} & \multicolumn{3}{c}{Parameter Estimates} \\
\cline{2-9}
 & R Square & F & df1 & df2 & Sig. & Constant & b1 & b2 \\
\hline
Linear & .690 & 20.030 & 1 & 9 & .002 & -640 & 7.053 & \\
Logarithmic & .637 & 15.792 & 1 & 9 & .003 & 1.794 & .748 & \\
Inverse & .577 & 12.300 & 1 & 9 & .007 & .848 & -.077 & \\
Quadratic & .835 & 20.233 & 2 & 8 & .001 & 1.708 & -35.850 & 190.542 \\
Compound$^a$ & & & & & & .000 & .000 & \\
Exponential$^a$ & & & & & & .000 & .000 & \\
\hline
\end{tabular}
\end{table}

\begin{table}
\centering
\caption{投资收益率与通货膨胀率的关系模拟}
\begin{tabular}{c c c c c c c c c}
\hline
\multirow{2}{*}{Equation} & \multicolumn{5}{c}{Model Summary} & \multicolumn{3}{c}{Parameter Estimates} \\
\cline{2-9}
 & R Square & F & df1 & df2 & Sig. & Constant & b1 & b2 \\
\hline
Linear & .004 & .036 & 1 & 9 & .854 & .103 & -.004 & \\
Logarithmic$^a$ & & & & & & & & \\
Inverse & .046 & .436 & 1 & 9 & .526 & .109 & -.042 & \\
Quadratic & .010 & .042 & 2 & 8 & .959 & .092 & .013 & -.003 \\
Compound$^b$ & & & & & & .000 & .000 & \\
Exponential$^b$ & & & & & & .000 & .000 & \\
\hline
\end{tabular}
\end{table}

与GDP的增长率、通货膨胀率的关系较弱,所以我们选取投资比例为以往每年投资比例的平均值$R_{i}$为46.72\%。

同时,全国基本养老基金投资运营管理办法正制定,所以我们相信基金投资能够以稳定的趋势收益,如2012年的投资收益率为7\%,结合2001-2011年的投资收益率并剔除特殊的数据,我们选取投资收益率$R_{s}$为5.8\%。

每年的社会养老保险的投资收益额$T_{i}$为

\begin{equation}
T_{i}=M_{i-1}\times R_{i}\times R_{s} \tag{1-35}
\end{equation}

其中,$M_{i-1}$为上一年的社会养老保险结余总额,$R_{i}$为投资比例,$R_{s}$为投资收益率。

\subsection*{4.1.6.3 政府补贴模型}

我们选取的政府补贴比重$R_{b}$为0.424\%。每年的政府补贴为$B_{i}$为

\begin{equation}
B_{i}=S_{i}\times R_{b} \tag{1-36}
\end{equation}

其中,$S_{i}$为当年的国民生产总值,$R_{b}$为补贴比例。

\begin{equation}
S_{t} = S_{2012} \times \prod_{i=1}^{t-2012} (1 + G_{t})
\tag{1-37}
\end{equation}

其中,$G_{t}$ 为 $t$ 年的 GDP 增长速度。

\begin{table}[h]
\centering
\caption{政府补贴参数表}
\begin{tabular}{|c|c|c|c|}
\hline
 & GDP 增长速度 & 补贴金额 & 补贴比重 \\
\hline
2001 & 0.083 & 1054.1 & 0.37 \\
\hline
2002 & 0.0908 & 1608 & 0.38 \\
\hline
2003 & 0.1003 & 2206.5 & 0.39 \\
\hline
2004 & 0.1009 & 2975 & 0.38 \\
\hline
2005 & 0.1131 & 4041 & 0.35 \\
\hline
2006 & 0.1268 & 5488.9 & 0.45 \\
\hline
2007 & 0.1416 & 7391.4 & 0.44 \\
\hline
2008 & 0.0963 & 9931 & 0.46 \\
\hline
2009 & 0.0921 & 12526.1 & 0.48 \\
\hline
2010 & 0.1045 & 15365.3 & 0.49 \\
\hline
2011 & 0.093 & 19497 & 0.48 \\
\hline
\end{tabular}
\end{table}

\section{问题二}

\subsection{对养老金缺口的理解}

在我国,社会养老保险一直以来被看作是退休后收入的最重要支柱。随着人口老龄化日益加剧,我国养老保障账户的缺口成了人们普遍关注的问题。

目前,对养老金缺口的定义主要有三种:

一、即所谓的“空账”,是指尽管你的账户里名义上有钱,但实际上却只是个无法兑现的空头数字。

社会科学院世界社保研究中心主任郑秉文近日在接受《经济参考报》记者采访时透露,2011年城镇基本养老保险个人账户记账额为2.5万亿左右,而实账部分仅为2703亿元左右,“空账”达到2.25万亿。

实际上,这是我国社会保险制度所必须承担的转轨成本的一部分。目前,我国社会正面临一个严峻的现实——老龄化加速逼近,而且社会保障不够完善、相关产业未成规模、失能老人生存堪忧。在这个现实面前,如何让我们的老人颐养天年?养老之忧,正在成为人们的普遍忧虑,也成为我国社会转型期一个无法回避和必须解答的重要课题。

在我国,社会养老保险一直以来被看作是退休后收入的最重要支柱。随着人口老龄化日益加剧,我国养老保障账户的缺口成了人们普遍关注的问题。

目前我国城镇职工基本养老保险施行的是上世纪90年代确立的“统账结合”制度,基本养老保险由社会统筹和个人账户两部分组成。社会统筹部分由单位负担缴费,为单位职工工资总额的20%,个人账户则由职工个人缴费,为个人工资

的 8\%。前者“现收现付”,用于支付已退休人员的养老金,后者实行的是长期封闭积累、产权个人所有的“完全积累”制,原则上不能调剂借用。

然而,由于在现行养老保险制度确立之前,企业员工基本无需缴纳养老保险费用,所以现有的养老保险基金中没有这部分职工的个人账户部分。但在养老保险制度设立之后,这部分职工退休后却从养老保险基金中领取养老金。因此,仅靠统筹账户不足以应对当期发放,加之各地财政实力不同,多数地区不得不在实际上采用了“现收现付制”的方法,即挪用个人账户的资金、用正在工作的一代人合计缴纳的 28\% 的月工资来支付现有退休人员的退休金,个人账户仅仅记账,所谓记账额由此形成。

为解决这一问题,我国从 2000 年开始了“做实”个人账户试点。截至 2011 年底,参与试点的辽宁、江苏、山东等 13 个省份共积累个人账户基金 2703 亿元,但其与记账额之间的差额,仍达到 2.23 万亿元,此“空账”被舆论定义成了养老保险“缺口”概念而广为传播。

二、是指现金流,也就是说是用当前各种的收入总和与支出的总和的差值;

“空账”的说法尽管言之有据,但上述判断却一直未能获得官方层面的认可。人力资源和社会保障部副部长胡晓义年初对这一问题有过明确表示:社保制度建立之初大部分省市养老金是收不抵支的,但之后缺口逐渐缩小,去年中国养老金略有结余。因此从全国层面看,不存在养老金缺口的问题。实际上,胡晓义此处否认的“缺口”并非指“空账”,而是现金流的概念,简单说就是当期发放没有问题。

为了便于理解和计算,本文所采用的缺口计算方法即为此种方法。

三、在维持现在的养老金给付水平的情况下,除现在已有的养老金(即上文中所指“结存”)以外,为了保证未来 70 年退休金的发放,在 2013 年这个时点,还需要的资金即为“缺口”。

中国银行首席经济学家曹远征在此前一份研究报告中称,到 2013 年,中国养老金的缺口将达到 18.3 万亿元。在目前养老制度不变的情况下,往后的年份缺口逐年放大,假设 GDP 年增长率为 6\%,到 2033 年时养老金缺口将达到 68.2 万亿元,占当年 GDP 的 38.7\%。

也许这一概念听上去有些杞人忧天的意味,但专家指出,与仍在可控范围内的“空账”相比,这一缺口才是真正值得关注的长期问题。随着人口老龄化和缴费的人数趋于下降,流量和存量的问题会相互转化并可能威胁到养老金支付。

参与上述 18.3 万亿缺口报告撰写的中国银行研究员廖淑萍指出,“如此大的存量缺口意味着,未来某一个年份后,中国年度的养老金流量缺口会是负的,而且还会不断扩大。

\subsection*{4.2.2 对未来有关情况的合理估计}

根据《2011 年度人力资源和社会保障事业发展统计公报》,2011 年全年城镇基本养老保险基金总收入 16895 亿元,比上年增长 25.9\%。全年基金总支出 12765 亿元,比上年增长 20.9\%。年末基本养老保险基金累计结存 19497 亿元。但是,养老保险收入增速之所以快于支出增速,主要得益于中国正处于社保普及

\begin{table}
\centering
\begin{tabular}{c c c c c c c}
\hline
 & & & & & & \\
\hline
地区 & 2012年 & 2013年 & 2014年 & 2015年 & 2016年 & 结余状况变化 \\
\hline
 & & & & & & \\
\hline
 & & & & & & 老工业基地省份(7个和兵团) \\
\hline
黑龙江 & -249.19 & -323.89 & -408.01 & -501.52 & -604.44 & 亏损趋势加重 \\
\hline
辽宁 & -173.65 & -196.85 & -221.57 & -247.81 & -275.57 & 亏损趋势加重 \\
\hline
吉林 & -61.74 & -79.09 & -98.72 & -120.62 & -144.8 & 亏损趋势加重 \\
\hline
天津 & -80.1 & -86.45 & -92.65 & -98.71 & -104.61 & 亏损趋势加重 \\
\hline
新疆兵团 & -69.91 & -81.37 & -93.67 & -106.84 & -120.86 & 亏损趋势加重 \\
\hline
上海 & -104 & -72.39 & -33.99 & 11.18 & 63.13 & 实现收支平衡 \\
\hline
重庆 & -6.86 & 0.34 & 9.16 & 19.6 & 31.67 & 实现收支平衡 \\
\hline
陕西 & -30.43 & -24.03 & -16.03 & -6.41 & 4.81 & 实现收支平衡 \\
\hline
 & & & & & & 中部流动人口大省(6个) \\
\hline
广西 & -35.48 & -50.94 & -68.43 & -87.96 & -109.52 & 亏损趋势加重 \\
\hline
河南 & -45.01 & -48.08 & -51.31 & -54.68 & -58.2 & 保持收不抵支 \\
\hline
江西 & -38.85 & -42.55 & -46.55 & -50.87 & -55.5 & 保持收不抵支 \\
\hline
湖南 & -19.25 & -11.8 & -3.37 & 6.02 & 16.4 & 收支基本平衡 \\
\hline
湖北 & 34.03 & 40.6 & 46.06 & 50.41 & 53.66 & 维持收支结余 \\
\hline
河北 & 34.77 & 58.08 & 83.54 & 111.15 & 140.91 & 维持收支结余 \\
\hline
 & & & & & & 边远地区省份(5个) \\
\hline
海南 & -18.44 & -17.62 & -16.28 & -14.43 & -12.07 & 保持收不抵支 \\
\hline
内蒙古 & 12.92 & 9.82 & 5.35 & -0.47 & -7.67 & 收支基本平衡 \\
\hline
西藏 & 2.77 & 4.18 & 5.75 & 7.48 & 9.37 & 收支基本平衡 \\
\hline
青海 & 7.46 & 7.55 & 7.51 & 7.34 & 7.05 & 收支基本平衡 \\
\hline
云南 & 61.82 & 75.27 & 89.77 & 105.32 & 121.92 & 维持收支结余 \\
\hline
\end{tabular}
\caption{18个省份(以及兵团)未来5年收支结余变化情况}
\end{table}

\begin{table}
\centering
\begin{tabular}{|c|c|c|c|c|c|}
\hline
年份 & 缺口(亿) & 年份 & 缺口(亿) & 年份 & 缺口(亿) \\
\hline
2012 & --- & 2017 & 85337.90 & 2022 & 110093.88 \\
\hline
2013 & 61230.36 & 2018 & 92970.21 & 2023 & 113900.01 \\
\hline
2014 & 65878.00 & 2019 & 98856.95 & 2024 & 116342.85 \\
\hline
2015 & 71950.58 & 2020 & 104336.31 & 2025 & 117313.54 \\
\hline
2016 & 78544.34 & 2021 & 109818.72 & 2026 & 117829.929 \\
\hline
2027 & 118607.28 & 2035 & 71696.14 & 2043 & 16270.07 \\
\hline
2028 & 120022.63 & 2036 & 64095.30 & 2044 & 8500.71 \\
\hline
2029 & 113559.20 & 2037 & 56733.09 & 2045 & 612.24 \\
\hline
2030 & 104113.71 & 2038 & 50023.91 & 2046 & -9158.15 \\
\hline
2031 & 95673.17 & 2039 & 43298.7 & 2047 & -18150.82 \\
\hline
2032 & 87216.01 & 2040 & 36300.84 & 2048 & -27663.50 \\
\hline
2033 & 79126.06 & 2041 & 29670.28 & 2049 & -36735.00 \\
\hline
2034 & 64095.30 & 2042 & 22962.22 & 2050 & -46612.44 \\
\hline
\end{tabular}
\caption{未来38年我国养老金缺口}
\end{table}

4.2.4 缺口预测合理性分析

(1) 建立了改进的离散人口分析模型,该模型针对我国城市化步伐加快、性别比持续偏高等具体现状,引入了女婴比、死亡率降低、生育率变化、乡村入迁城镇人口数目等参数,改进了以往的离散人口发展模型,得出了我国人口总量、城市化程度等在中长期内的发展变化规律,合理预测了各个年龄段的人口。便于后期精确分析每年的退休人数(男60岁,女55岁)等;

(2) 养老金替代率从职保、新农保和城居保三个方面展开,符合现行政策;双变量ARIMA经济增速模型则将金融危机等外部因素考虑在内,符合世界经济的波动趋势;曲线拟合工资预测模型从不同的工作群体入手,覆盖面广,预测结果更加精确;

(3) 收支模型考虑养老金替代率、缴费率、人口发展模型、死亡率、经济增速、工资预测、财政补贴、投资效益、城市化率、就业率、金融危机、鼓励政策等因素加以强化完善,并且分企业基金等多个层次,在方案中体现“多缴多得,长缴多得”的政策,与实际的养老保险体系接轨。

4.2.5 最尖锐的情况及严重程度

如果全部情况维持不变,按照本文数学模型仿真的结果,我国城乡居民养老保险收支矛盾最尖锐的情况发生在到2046年左右,此时将会出现‘入不敷出’的情况,此时养老保险的收入已经不足以支付当年的支出,必须加大财政支等。

4.2.6 变量调整

党的十八大提出的到2020年实现国内生产总值和城乡居民人均收入比2010年翻一番的目标。所谓国民收入倍增计划,是指在一个相对确定的较短时期内,通过提高国民经济各部门生产效率和效益、显著提高居民实际收入水平、建立健

全收入分配和社会保障制度等方式,实现居民收入翻番目标的一种经济社会发展方案。这里的国民收入,并非统计学意义上的GDP或GNP,而是指居民收入。

根据2010年的实际情况和“收入倍增计划”,到2020年我国GDP总量、城镇居民人均可支配收入和农村居民人均纯收入按2010年的价格将分别达到80.3万亿元、38218.8元和11838元。由于2011年和2012年经济增长速度以及居民收入的增长速度已成定局,要实现“收入倍增计划”,2013-2020年的8年间,我国GDP、城镇居民人均可支配收入和农村居民人均纯收入的年平均增长速度需要分别达到6.86%、6.86%和6.3%的水平。

1960年,日本宣布启动为期10年的国民收入倍增计划,采取了包括充实社会资本、实行最低工资制、推行社会保障、增加农业劳动者收入、推动中小企业发展、削减个人收入调节税和企业税等一系列措施,仅用7年就使国民收入翻倍。

有人担心财政收入会因此而减少。事实上,国民收入增长是一种高效的经济价值创造机制,也是财政收入增长的源泉。仍以日本为例,其国民收入倍增计划结束时,国民生产总值年均增长率达到11.6%,国民收入年均增长率达到11.5%,实现了国富与民富同步。还有一种担心是,工资上涨会导致成本推动型的通货膨胀。从理论上看,在一个收入预期相对稳定的市场中,产品、服务的需求和供给不会出现剧烈变动。还应看到,虽然近几年我国职工工资提高较快,但劳动力价格仍较低廉,企业利润侵蚀劳动工资的现象还普遍存在。如果能依靠市场的力量实现工资收入增长并稳定社会预期,通胀在很大程度上就是可控的。

收入倍增计划中收入倍增者指的是中低收入者的收入翻一番。实现首付倍增的途径主要有三个:一是转变经济增长方式,二是实现高质量的就业,三是抓好收入分配。在路径选择上应注重以下几个方面:把推进经济转型升级、促进跨越发展作为实现倍增计划的有力支撑。把促进充分就业、鼓励自主创业作为实现倍增计划的必要前提。把增加职工薪酬、优化收入结构作为实现倍增计划的基础工程。把完善保障体系、提高保障水平作为实现倍增计划的重要内容。把强化行政推动、强化部门责任作为实现倍增计划的重要保证。

对于人口预测模型而言,居民收入增加,首先影响的是就业率,收入增长必定会促进GDP增长,尤其是消费GDP,相应的必定会增加更多的岗位,所以就业率会在接下去十年持续上涨。其次,就是城市化比率,收入增加会促使更多的人进城,城市化率必定会以比现有增长率更快的速度增长。

通过调节居民平均工资水平、就业率以及城市化比率,我们可以得到表(2.3):

\begin{table}[h]
\centering
\caption{未来38年养老金结余表}
\begin{tabular}{|c|c|c|c|c|c|}
\hline
年份 & 结余(亿) & 年份 & 结余(亿) & 年份 & 结余(亿) \\
\hline
2012 & ---- & 2025 & 119858.04 & 2038 & 51490.17 \\
\hline
2013 & 58367.75 & 2026 & 120702.68 & 2039 & 49951.57 \\
\hline
2014 & 63883.84 & 2027 & 121854.30 & 2040 & 38786.5 \\
\hline
2015 & 70545.72 & 2028 & 123687.61 & 2041 & 32540.69 \\
\hline
2016 & 77679.93 & 2029 & 117659.98 & 2042 & 26317.44 \\
\hline
2017 & 84947.83 & 2030 & 108617.52 & 2043 & 19008.9 \\
\hline
\end{tabular}
\end{table}

\begin{table}
\centering
\begin{tabular}{|c|c|c|c|c|c|}
\hline
2018 & 93061.05 & 2031 & 100622.77 & 2044 & 10353.43 \\
\hline
2019 & 99267.03 & 2032 & 92612.10 & 2045 & 2292.11 \\
\hline
2020 & 105066.5 & 2033 & 84979.35 & 2046 & -6207.47 \\
\hline
2021 & 110899.9 & 2034 & 78020.27 & 2047 & -15217.7 \\
\hline
2022 & 111563.7 & 2035 & 70800.62 & 2048 & -23765.5 \\
\hline
2023 & 115792.6 & 2036 & 63982.97 & 2049 & -33114.8 \\
\hline
2024 & 118583 & 2037 & 57747.76 & 2050 & --- \\
\hline
\end{tabular}
\end{table}

与表(2.2)进行比较得出,收入倍增计划并不能有效的缓解养老金缺口的压力。

\section*{4.3 问题三}

\subsection*{4.3.1 养老保险国内外对比}

纵观世界,目前世界上各个国家和地区建立了不同类型的养老保险制度,按照其覆盖范围保障水平和基金模式,大致可分为传统型养老保险(美、德、法等发达市场经济国家)、福利型养老保险(英、澳、加、日等发达市场经济国家)、混合型养老保险(原福福利型养老保险国家正逐渐转变为混合型养老保险模式)、国家型养老保险(苏联、东欧国家)、储蓄金型养老保险(新加坡、智利等新兴市场经济国家)几种类型。

\subsubsection{一、国外养老保险的主要做法}

\textbf{德国:}德国养老保险制度的三大支柱是法定养老保险、企业补充养老保险和私人养老保险。德国的公务员不参加养老保险,实行退休制度,养老金由财政预算安排。养老金根据退休者退休时的工资和工龄长短计算,但最高不超过退休前最后一个月工资的75\%。

\textbf{美国:}凡交纳社会保险税的年满65岁的公民都可享受养老退休金,62-65岁退休者只能享受部分养老退休金。该基金是一种非累积型的,其良好的运转需要依靠两个条件,一个是实际工资的不断增长,另一个是劳动人口的增长。然而,由于美国经济增长速度比较缓慢(从而实际工资增长也变得非常缓慢),现收现付制的运转基础发生了变化。

\textbf{英国:}英国的养老金包括基本养老金和附加养老金两部分。凡是超过法定退休年龄的公民都可得到基本养老金,而附加养老金只有平时按规定金额交纳社会保险金的公民退休后方可得到。附加养老金的多寡由公民交纳的社会保险金时间的长短决定。对于延期退休者雇主代他交纳。

\textbf{日本:}日本的养老金制度具有多层次、多部门、多基础的特点,比较复杂。日本的养老保险属于公共年金范畴,是一种通过国家立法强制实行的社会保险。日本养老金的经费由雇主、雇员、国家三方负担。除了养老金制度外,日本还有老年福利年金,领取这种年金者不需要交纳保险费。

\textbf{瑞士:}瑞士的养老保险由“三个支柱”构成,包括强制性的基本养老保险、补充性的职业养老保险和个人自愿的商业养老保险。

\textbf{北欧:}北欧国家的社会保障原则是“社会保险,人人受益”。在瑞典,养老退休金包括基本年金和附加年金,约为退休前工资的70\%。

\subsubsection{二、国内外养老保险制度的对比分析}

1、国内外养老保险制度的环境差异分析:比较国内外的养老保险制度环境,我国养老保险的运营环境的困难主要体现在保险基金无积累、养老保险制度供养人数增长迅速、法制体系不健全、特有的二元化社会造成养老保险覆盖面窄、社

会养老保险基金足额收缴困难、养老金给付标准面临上涨、个人支出成本上升等几个方面。

2、国内外养老保险模式的差异分析:从国外的发展来看,各国都根据各自的国情发展出来相对适应自身的较稳定养老保险制度模式,而我国的养老保险模式还处于适应阶段。

3、国内外养老保险基金运作机制的差异分析:与美国、智力等国家的养老保险基金管理模式相比,我国的养老保险基金管理制度还存在体系僵化、基础建设不足、收益率偏低等不足,容易导致养老保险基金被挤占、挪用,造成基金流失,养老保险基金投资渠道狭窄,保值增值无保证的局面。

4、养老保险的管理制度的差异分析:综观西方国家养老保险管理制度,它们都有一些共同之处。比如养老保险开支分别由中央和地方政府管理,而不是单纯由中央政府一个层次管理,因此,在管理的负担和风险上具有一定的分散性。

三、各国养老保险制度中的借鉴点

1、养老保险筹资模式的选择:从上述各国政府的做法中我们可以看到,尽管不同的国家筹集养老保险资金的渠道各有千秋,但不外乎有社会保险税、单位交费、个人交费、政府补贴等形式。问题的关键是如何确立适合本国的养老保险需要的筹资模式,这种筹资模式是否能够体现公平与效率的标准,是各国政府在构建本国养老保险制度过程中所追求的一种目标,这也正是我们从别国做法中需要借鉴的内容之一,因为养老保险筹资模式的选择往往是一国养老保险制度构建的核心,养老保险筹资模式选择错误将会导致受益人不能得到有效保障(如现收现付模式)、或者会导致代际间不公平(如社会保险税率过高)、或者会导致政府负担过重(如财政补贴过大),等等问题。

第一,提高保险费。将交纳保险费的比例在原来占工资总额 8% 的基础上,每年以一定的幅度提高;

第二,逐年降低向被保险者支付保险金的数额。计划将平均养老金支付额逐年降低。

第三,调整养老金发放方法。根据物价跌幅,减少养老金的支付额。(评注:这样的养老保险改革易使国民对保险制度失去信任,加剧了人们对失去劳动能力之后的不安心理。)或将受益人的福利与个人的贡献挂钩。

第四,努力实现政府、企业和个人三者间的平衡,社会统筹部分通过现收现付制来提供最低养老保障,而个人帐户方面通过企业和个人缴纳的基金积累则既有利于提高社会储蓄率,又有利于激励个人的工作积极性。

第五,提高退休年龄和工龄,控制提前退休。

2、养老保险管理方式的恰当选择与运用,是一国养老保险制度实现的保障。欧美国家的做法,尤其是日本政府的做法非常值得我国借鉴。遵循养老保险基金投资的“高收益性、高安全性、低风险性、高流动性”原则,结合国外经验,对养老保险基金在传统投资渠道、指数化基金、股市、抵钾货款、不动产、海外投资等相关投资渠道的对比,在这一背景下,我国养老保险基金的投资方式要根据社会经济发展的状况不断进行调整,并充分运用投资组合理论、套利定价理论等投资理念,有效整合各种金融工具,实现投资回报的“帕累托最优”,确保养老保险基金在保证低风险的情况下实现保值增值的预期效果。

第一,建立统一的养老保险基金运营监管机构,要提高我国养老保险基金的运营效益,也必须在监管机构上统一,不能政出多门。

第二,投资渠道多样化,一方面可以提高养老保险基金收益率,另一方面也

可以降低养老保险基金运营风险。

第三,对养老保险基金的监管是以市场化为导向的,各养老保险基金运作在政府有效的监管下,充分竞争,服务水平和效率均比较高。

第四,注重培育和发挥审计、精算等中介机构的监督作用,提高了基金运作的透明度。

\section*{四、我国养老保险制度重点问题的研究}

1、农村城市化进程中的养老保险制度改革

伴随产业结构的调整,城市化将成为我国经济发展的重要阶段。城市化进程对农业生产方式变革的影响,伴随着农村人口老龄化,会使我国农村人口老年风险积聚。分析我国农村养老保险制度的历史沿革和特征,借鉴国外相关做法,加快行之有效的农村养老保险制度的研讨迫在眉睫,我们应该尝试在改革中注意不同地区与不同行业的互助;精简管理部门,统一管理;加强资金运用,做到保值增值;积极引导,消除强制手段的反作用。

2、基本养老保险个人账户基金入市运营及对策研究

国外养老金运营的经验、我国证券市场的规范发展以及企业年金和全国社保基金的成功运营表明,我国基本养老保险个人账户基金进入证券市场是一个可行的选择。根据国外基本养老保险个人账户基金入市模式的利弊分析,认为我国应采用全部信托型投资方式。但是,基本养老保险个人账户基金入市运营还面临着若干问题,如法律缺位,转制成本需要化解,统筹层次太低等,因此我国应及时修订相关的法律法规;多方筹集转制成本资金;提高统筹层次,在实现省级统筹的基础上实现全国统筹。

\section*{五、总结}

结合我国的现实情况,我们应该加强养老立法。用法律保障养老制度的权威性与规范性;逐步推行养老金的个人账户基金积累制,改变现有的现收现付制;多方进行开源节流,缓解社会养老金支付的困难以继续维持现有的现收现付制;个人账户建立在个人自愿和个人自由选择私人管理公司管理账户基金的基础上。当然,在现有的经济发展状况,人口增长趋势下,我们应该在寻找替代率和缴费率的合理区间以保证我国养老保险体系的可持续性。

\section*{4.3.2 替代率和缴费率的区间确定}

中国社会保障制度改革的目标模式可持续发展,是当今各国政府十分重视的发展模式。所谓可持续发展,按世界环境与发展委员会的定义,是“在不牺牲未来几代人需要的情况下,满足我们这代人的需要。”可持续发展战略体现了公平性、持续性、共同性和协调性的原则,是一种以自然持续发展为基础,以经济持续发展为任务,以社会发展为目标的新发展观。作为可持续发展战略的内在构成要素,社会保障可持续发展既属于经济可持续发展,但更主要的是属于社会可持续发展的范畴。国际经验表明,没有社会保障可持续发展,也就没有经济和社会可持续发展。因此,要实现社会保障可持续发展,就必须选择可持续发展的社会保障模式。因为人口结构、分年龄段死亡率、经济增速、投资效益等主要因素几乎无法人为较大幅度改动,所以我们要选择寻找替代率和缴费率的合理区间以保证我国养老保险体系的可持续性。

本节的可持续性我们将从合意替代率的角度出发,即从需求角度确定合理的保障水平区间。因为养老金保障水平即养老金替代率必须有一个合理的范围,过高的替代率必然对应着过高的缴费率,而缴费率过高会增加企业的劳动力成本,

增加参保人及企业的负担,影响企业的发展;但过低的替代率又无法保障老年人的基本生活。养老保险制度的两个基本目标是为制度覆盖范围内的老年人提供定期的基本生活保障和一定的收入替代。因此,研究替代率必须确定替代率的合理区间。

本节主要使用两种方法即支出法和收入法对合意替代率的上限和下限进行测算 \({}^{[9]}\) 。先从消费需求角度,即从满足退休人员基本生活需要角度,构建替代率下限测算模型,对城镇居民的消费支出结构进行实证分析,确定退休者应该领取的最低替代率水平;再从可支配收入角度,即从在职者及其抚养或供养的人口数量与可支配收入之间的关系入手,设计合意替代率上限的测算模型。替代率水平的下限,是满足退休者衣、食、住、行等基本消费的养老金标准;替代率水平的上限,是能够达到退休前社会平均生活水平的养老金标准。

\subsection*{4.3.2.1 替代率的区间确定}

\subsubsection{一、替代率下限的模型设计}

基本消费支出法需要使用扩展性线性支出 ELES 模型,通过测算老年人的基本生活消费水平来测算基本养老金合意替代率的下限 \(R_{L}\) 。ELES 模型把消费者对各类商品或服务的消费支出看作收入和价格的函数,在某个时期一定的价格和收入水平下,消费者获得的可支配收入,首先满足与收入水平无关的基本需求,然后在扣除基本需求消费以后,剩余的收入按一定的比例在各类商品或服务之间按一定的边际消费倾向进行分配。我们要测算替代率水平的下限,需要用该模型计算出与居民收入水平无关的基本消费支出,然后用基本支出去除同期居民的可支配收入。

所有商品和劳务的消费支出模型为:

\[
\sum_{i=1}^{n} C_{i} = \sum_{i=1}^{n} P_{i} X_{i} + \sum_{i=1}^{n} \beta_{i} \left( Y - \sum_{i=1}^{n} P_{i} X_{i} \right) \quad \left( 0 \leq \beta_{i} \leq 1, \sum_{i=1}^{n} \beta_{i} \leq 1 \right) \tag{4-1}
\]

其中,\(C_{i}\) 为消费者对第 \(i\) 类商品或服务的消费支出;\(P_{i}\) 为第 \(i\) 类商品或服务的价格;\(X_{i}\) 为消费者对第 \(i\) 类商品或服务的基本需求量;\(P_{i} X_{i}\) 为消费者对第 \(i\) 类商品的基本消费支出;\(\sum_{i=1}^{n} P_{i} X_{i}\) 为消费者对所有商品或服务的基本消费支出;\(\beta_{i}\) 为第 \(i\) 类商品或服务的边际消费倾向;\(Y\) 为可支配收入;\(\beta_{i} \left( Y - \sum_{i=1}^{n} P_{i} X_{i} \right)\) 为消费者对第 \(i\) 类商品的超额消费支出。

即

\[
\sum_{i=1}^{n} C_{i} = (1 - \beta_{i}) \sum_{i=1}^{n} P_{i} X_{i} + \sum_{i=1}^{n} \beta_{i} Y \tag{4-2}
\]

按照合意替代率下限的基本思想,如果假定养老金是退休职工的唯一收入来源,而且养老保险收入只用于满足老年人的生活需要,那么养老保险收入=总消费支出,即 \( Y_{\text{养老}} = \sum_{i=1}^{n} C_i \),也就是退休者所得到的养老金完全用于老年的生活消费,不能有剩余也不能有储蓄,这符合“基本养老”的含义。另外,根据养老保险制度的第一目标,老年人的退休金只用于满足基本生活需要,那么,总消费支出=总基本消费支出,即 \(\sum_{i=1}^{n} C_i = \sum_{i=1}^{n} P_i X_i\),所以

\[
R_L = \frac{\sum_{i=1}^{n} P_i X_i}{Y}
\tag{4-3}
\]

表4.1 不同收入群体2002-2011年收入及支出情况(单位:元)

\begin{table}[h]
\centering
\begin{tabular}{c|c|c|c|c|c|c|c|c}
\hline
\multirow{2}{*}{年份} & \multicolumn{2}{c|}{最低收入户} & \multicolumn{2}{c|}{低收入户} & \multicolumn{2}{c|}{中等偏下户} & \multicolumn{2}{c}{中等收入户} \\
\cline{2-9}
 & 消费性支出 & 可支配收入 & 消费性支出 & 可支配收入 & 消费性支出 & 可支配收入 & 消费性支出 & 可支配收入 \\
\hline
2002 & 2,387.51 & 2,408.60 & 3,259.60 & 3,649.20 & 4,206.00 & 4,932.00 & 5,452.90 & 6,656.80 \\
\hline
2003 & 2,562.40 & 2,590.20 & 3,549.30 & 3,970.00 & 4,557.80 & 5,377.30 & 5,848.00 & 7,278.80 \\
\hline
2004 & 2,855.20 & 2,862.40 & 3,942.20 & 4,429.10 & 5,096.20 & 6,024.10 & 6,498.40 & 8,166.50 \\
\hline
2005 & 3,111.50 & 3,134.90 & 4,295.40 & 4,885.30 & 5,574.30 & 6,710.60 & 7,308.10 & 9,190.10 \\
\hline
2006 & 3,423.00 & 3,568.70 & 4,765.60 & 5,540.70 & 6,108.30 & 7,554.20 & 7,905.40 & 10,269.7 \\
\hline
2007 & 4,036.30 & 4,210.10 & 5,634.20 & 6,504.60 & 7,123.70 & 8,900.50 & 9,097.40 & 12,042.3 \\
\hline
2008 & 4,532.90 & 4,753.60 & 6,195.30 & 7,363.30 & 7,993.70 & 10,195.6 & 10,344.7 & 13,984.2 \\
\hline
2009 & 4,900.60 & 5,253.20 & 6,743.10 & 8,162.10 & 8,738.80 & 11,243.6 & 11,309.7 & 15,399.9 \\
\hline
2010 & 5,471.80 & 5,948.10 & 7,360.20 & 9,285.30 & 9,649.20 & 12,702.1 & 12,609.4 & 17,224.0 \\
\hline
2011 & 6,431.90 & 6,876.10 & 8,509.30 & 10,672.00 & 10,872.8 & 14,498.3 & 14,028.2 & 19,544.9 \\
\hline
\multirow{2}{*}{年份} & \multicolumn{2}{c|}{中等偏上户} & \multicolumn{2}{c|}{高收入户} & \multicolumn{2}{c|}{最高收入户} & \multicolumn{2}{c}{全国平均} \\
\cline{2-9}
 & 消费性支出 & 可支配收入 & 消费性支出 & 可支配收入 & 消费性支出 & 可支配收入 & 消费性支出 & 可支配收入 \\
\hline
2002 & 6,940.0 & 8,869.50 & 8,919.90 & 11,772.8 & 13,040.7 & 18,995.9 & 6,029.90 & 7,702.80 \\
\hline
2003 & 7,547.3 & 9,763.40 & 9,627.60 & 13,123.1 & 14,515.7 & 21,837.3 & 6,510.90 & 8,472.20 \\
\hline
2004 & 8,345.7 & 11,050.9 & 10,749.4 & 14,970.9 & 16,841.8 & 25,377.2 & 7,182.10 & 9,421.60 \\
\hline
2005 & 9,410.8 & 12,603.4 & 12,102.5 & 17,202.9 & 19,153.7 & 28,773.1 & 7,942.90 & 10,493.0 \\
\hline
2006 & 10,218.3 & 14,049.2 & 13,169.8 & 19,069.0 & 21,061.7 & 31,967.3 & 8,696.60 & 11,759.50 \\
\hline
2007 & 11,570.4 & 16,385.8 & 15,297.7 & 22,233.6 & 23,337.3 & 36,784.5 & 9,997.50 & 13,785.8 \\
\hline
2008 & 13,316.6 & 19,254.1 & 17,888.2 & 26,250.1 & 26,982.1 & 43,613.8 & 11,242.9 & 15,780.8 \\
\hline
2009 & 14,964.4 & 21,018.0 & 19,263.9 & 28,386.5 & 29,004.4 & 46,826.1 & 12,264.6 & 17,174.7 \\
\hline
2010 & 16,140.4 & 23,188.9 & 21,000.4 & 31,044.0 & 31,761.6 & 51,431.6 & 13,471.5 & 19,109.4 \\
\hline
2011 & 18,160.9 & 26,420. & 23,906.2 & 35,579.2 & 35,183.6 & 58,841.9 & 15,160.9 & 21,809.8 \\
\hline
\end{tabular}
\end{table}

附:收入等级为所有城镇调查户按人均可支配收入由低到高排队,按10%、10%、20%、20%、

\begin{table}
\centering
\caption{2002年收入水平}
\begin{tabular}{|c|c|c|c|c|c|c|c|}
\hline
2002年 & 最低收入户 & 低收入户 & 中等偏下户 & 中等收入户 & 中等偏上户 & 高收入户 & 最高收入户 \\
\hline
消费性支出 & 2387.51 & 3259.6 & 4206 & 5452.9 & 6940 & 8919.9 & 13040.7 \\
\hline
可支配收入 & 2,408.60 & 3,649.20 & 4,932.00 & 6,656.80 & 8,869.50 & 11,772.80 & 18,995.90 \\
\hline
\end{tabular}
\end{table}

\begin{table}
\centering
\caption{Model Summary (b)}
\begin{tabular}{|c|c|c|c|c|}
\hline
Model & R & R Square & Adjusted R Square & Std. Error of the Estimate \\
\hline
1 & .998(a) & .997 & .996 & 224.61702 \\
\hline
\end{tabular}
\end{table}

\begin{table}
\centering
\caption{ANOVA (b)}
\begin{tabular}{|c|c|c|c|c|c|}
\hline
Model & & Sum of Squares & df & Mean Square & F & Sig. \\
\hline
\multirow{3}{*}{1} & Regression & 82110649.37 & 1 & 82110649.37 & 1627.474 & .000(a) \\
\cline{2-7}
 & Residual & 252264.025 & 5 & 50452.805 & & \\
\cline{2-7}
 & Total & 82362913.40 & 6 & & & \\
\hline
\end{tabular}
\end{table}

\begin{table}
\centering
\begin{tabular}{c c c c c c}
\hline
\multirow{2}{*}{Model} & \multicolumn{2}{c}{Unstandardized Coefficients} & \multirow{2}{*}{Standardized Coefficients} & \multirow{2}{*}{t} & \multirow{2}{*}{Sig.} \\
\cline{2-3}
 & B & Std. Error & Beta & & \\
\hline
1 & 1032.890 & 156.053 & & 6.619 & .001 \\
(Constant) & .645 & .016 & .998 & 40.342 & .000 \\
可支配收入 & & & & & \\
2002 & & & & & \\
\hline
\end{tabular}
\caption{Coefficients(a)}
\end{table}

回归模型的常数项为1032.890,自变量可支配收入的回归模型系数0.645。因此我们可以得到线性回归方程为:
\[
\sum_{i=1}^{n} C_i = 1032.890 + 0.645Y
\]

回归系数的显著性水平为0.001,0.000,同样说明因变量消费性支出和自变量可支配收入的线性关系是显著的,可建立线性模型。

同理,2002-2011年的线性回归结果如表4.6所示:

\begin{table}
\centering
\begin{tabular}{c c c c c c}
\hline
年份 & $(1-\beta_i)\sum_{i=1}^{n} P_i X_i$ & $\sum_{i=1}^{n} \beta_i$ & 年份 & $(1-\beta_i)\sum_{i=1}^{n} P_i X_i$ & $\sum_{i=1}^{n} \beta_i$ \\
\hline
2002 & 1032.89 & 0.645 & 2002 & 1829.656 & 0.591 \\
2003 & 1147.374 & 0.626 & 2003 & 2075.464 & 0.58 \\
2004 & 1314.161 & 0.619 & 2004 & 2225.192 & 0.582 \\
2005 & 1370.532 & 0.623 & 2005 & 2310.754 & 0.582 \\
2006 & 1402.832 & 0.618 & 2006 & 2918.156 & 0.561 \\
\hline
\end{tabular}
\caption{2002-2011年的线性回归结果}
\end{table}

则不同群体的替代率下限如表4.7所示:

\begin{table}
\centering
\begin{tabular}{c c c c c c c c c}
\hline
年份 & 基本消费支出(元) & 全国平均替代率(\%) & 最低收入户替代率(\%) & 低收入户替代率(\%) & 中等偏下户替代率(\%) & 中等收入户替代率(\%) & 中等偏上户替代率(\%) & 高收入户替代率(\%) & 最高收入户替代率(\%) \\
\hline
2002 & 2909.549 & 37.773 & 120.798 & 79.731 & 58.993 & 43.708 & 32.804 & 24.714 & 15.317 \\
2003 & 3067.845 & 36.211 & 118.440 & 77.276 & 57.052 & 42.148 & 31.422 & 23.377 & 14.049 \\
2004 & 3449.241 & 36.610 & 120.502 & 77.877 & 57.257 & 42.236 & 31.212 & 23.040 & 13.592 \\
2005 & 3635.363 & 34.646 & 115.964 & 74.414 & 54.173 & 39.557 & 28.844 & 21.132 & 12.635 \\
2006 & 3672.335 & 31.229 & 102.904 & 66.279 & 48.613 & 35.759 & 26.139 & 19.258 & 11.488 \\
2007 & 4473.487 & 32.450 & 106.256 & 68.774 & 50.261 & 37.148 & 27.301 & 20.120 & 12.161 \\
2008 & 4941.581 & 31.314 & 103.954 & 67.111 & 48.468 & 35.337 & 25.665 & 18.825 & 11.330 \\
2009 & 5323.426 & 30.996 & 101.337 & 65.221 & 47.346 & 34.568 & 25.328 & 18.753 & 11.369 \\
\hline
\end{tabular}
\caption{不同群体的替代率下限}
\end{table}

\begin{tabular}{|c|c|c|c|c|c|c|c|c|}
\hline
2010 & 5528.120 & 28.929 & 92.939 & 59.536 & 43.521 & 32.095 & 23.840 & 17.807 & 10.748 \\
\hline
2011 & 6647.280 & 30.478 & 96.672 & 62.287 & 45.849 & 34.010 & 25.160 & 18.683 & 11.297 \\
\hline
历年平均替代率 & 33.063 & 107.977 & 69.851 & 51.153 & 37.657 & 27.772 & 20.571 & 12.399 \\
(\%) & & & & & & & & & \\
\hline
\end{tabular}

从表中可以直观地得到以下结论:

(1) 以基本支出法表示的替代率,历年平均为 $33.063\%$,按时间序列看,以满足老年人基本生活需要为目标的养老保障水平呈现不断下降的趋势。

(2) 随着生活水平和物价的提高,居民的年基本生活消费支出稳定增长:从2002年的2909.549元上升到2011年的6647.280元,年均增长率为 $9.61\%$,与GDP的增速基本持平。

(3) 根据替代率的下限的发展趋势,我们考虑不同的群体的替代率以及历年平均替代率定义替代率的下限为 $35\%$。

\section*{二、替代率上限的模型设计}

人均可支配收入法主要是从家庭人均收入和家庭人均消费角度测算退休人员的养老金替代率上限,这个替代标准能够保障退休人员领取的养老金水平基本维持在社会平均生活水平上,也就是说,这个上限所实现的替代率 $R_U$ 可以确保退休人员的生活水平和社会平均生活水平相当。不过,需要指出的是,水平相当的含义并不是说基本养老金的保障水平要与个人退休前工资水平持平,而是达到同期社会平均生活水平。

城镇的每一个就业职工所获取的可支配收入,除了供自己消费以外,还要抚养或赡养其他人。我们把可支配收入总额设为 $S$,用于就业者本人消费的那部分可支配收入设为 $B$,用于抚养或赡养其他人的那部分收入假定为 $S-B$,如果退休者得到的养老金仅用于自身消费,那么他只需要领取数额为 $B$ 的养老金,其生活水平就与在职时相同或相对在职期间没有下降。换个角度讲,只要退休人员领取的养老金数额能达到 $B$,就可以确保退休者的生活水平和退休前的社会平均生活水平相当。假定每一就业者负担人数(包括本人)为 $N$,在职者本人和他所负担的人的人均消费相同,那么退休者的养老金等于人均可支配收入,则

\begin{equation}
R_U = \frac{B}{S} = \left(\frac{S}{N}\right) / S = \frac{1}{N}
\tag{4-4}
\end{equation}

\begin{equation}
\sum_{i=1}^{n} C_i = \sum_{i=1}^{n} P_i X_i + \sum_{i=1}^{n} \beta_i \left(Y - \sum_{i=1}^{n} P_i X_i\right) \quad (0 \leq \beta_i \leq 1, \sum_{i=1}^{n} \beta_i \leq 1)
\tag{4-5}
\end{equation}

其中,$C_i$ 为消费者对第 $i$ 类商品或服务的消费支出;$P_i$ 为第 $i$ 类商品或服务的价格;$X_i$ 为消费者对第 $i$ 类商品或服务的基本需求量;$P_i X_i$ 为消费者对第 $i$ 类

商品的基本消费支出;$\sum_{i=1}^{n}P_{i}X_{i}$ 为消费者对所有商品或服务的基本消费支出;$\beta_{i}$ 为第 $i$ 类商品或服务的边际消费倾向;$Y$ 为可支配收入;$\beta_{i}(Y-\sum_{i=1}^{n}P_{i}X_{i})$ 为消费者对第 $i$ 类商品的超额消费支出。

即
\[
\sum_{i=1}^{n}C_{i}=(1-\beta_{i})\sum_{i=1}^{n}P_{i}X_{i}+\sum_{i=1}^{n}\beta_{i}Y \tag{4-6}
\]

按照合意替代率下限的基本思想,如果假定养老金是退休职工的唯一收入来源,而且养老保险收入只用于满足老年人的生活需要,那么养老保险收入 = 总消费支出,即 $Y_{\text{养老}}=\sum_{i=1}^{n}C_{i}$,也就是退休者所得到的养老金完全用于老年的生活消费,不能有剩余也不能有储蓄,这符合“基本养老”的含义。另外,根据养老保险制度的第一目标,老年人的退休金只用于满足基本生活需要,那么,总消费支出 = 总基本消费支出,即 $\sum_{i=1}^{n}C_{i}=\sum_{i=1}^{n}P_{i}X_{i}$,所以

\[
R_{L}=\frac{\sum_{i=1}^{n}P_{i}X_{i}}{Y} \tag{4-7}
\]

表4.8 以人均可支配收入法测算的合意替代率上线

\begin{tabular}{|c|c|c|c|c|c|}
\hline
 & 平均每户就业 & 每户工资总 & 平均每户家庭 & 人均工资 & 替代率 \\
 & 人口(人) & 收入(元) & 人口(人) & 收入(元) & (\%) \\
\hline
1985 & 2.2 & 2.2A & 3.9 & 0.5641A & 56.41 \\
\hline
1990 & 2.0 & 2.0B & 3.5 & 0.5714B & 57.14 \\
\hline
1995 & 1.9 & 1.9C & 3.2 & 0.5938C & 59.38 \\
\hline
1998 & 1.8 & 1.8D & 3.2 & 0.5625D & 56.25 \\
\hline
1999 & 1.8 & 1.8E & 3.1 & 0.5806E & 58.06 \\
\hline
2001 & 1.65 & 1.65F & 3.09 & 0.5340F & 53.40 \\
\hline
2002 & 1.57 & 1.57G & 3.03 & 0.5182G & 51.82 \\
\hline
2003 & 1.57 & 1.57H & 3.00 & 0.5233H & 52.33 \\
\hline
2004 & 1.56 & 1.56I & 2.97 & 0.5253I & 52.53 \\
\hline
2005 & 1.51 & 1.51J & 2.95 & 0.5119J & 51.19 \\
\hline
2006 & 1.53 & 1.53K & 2.93 & 0.5222K & 52.22 \\
\hline
2007 & 1.54 & 1.54L & 2.91 & 0.5292L & 52.92 \\
\hline
2008 & 1.5 & 1.5M & 2.9 & 0.5172M & 51.72 \\
\hline
2009 & 1.49 & 1.49N & 2.9 & 0.5138N & 51.38 \\
\hline
\end{tabular}

\begin{table}
\begin{tabular}{c|c|c|c|c|c}
2010 & 1.49 & 1.490 & 2.88 & 0.51740 & 51.74 \\
2011 & 1.48 & 1.48P & 2.87 & 0.5157P & 51.57 \\
\end{tabular}
\end{table}

(注:把每年的社会平均工资设为A-Q类)

\begin{figure}[h]
\centering
\includegraphics[width=0.8\textwidth]{image.png}
\caption{2001-2011年替代率上限变化图}
\end{figure}

通过上图我们可以看到二十一世纪以来的替代率上限基本在51\%-53\%的范围内稳定波动。因此,在当前的人口结构条件下,我们定义替代率合意水平的上限为53\%。

\subsubsection{缴费率}

从静态角度来看,缴费率是现收现付制养老保险体系的直接影响因素,职工退休人员养老保障与福利水平的高低与整个养老保险与福利体系的缴费率正相关,为了不降低原有的替代率水平,在抚养比不断上升的情况下,政府将不得不提高缴费率水平,从而避免增加企业和大量在职人员的负担。

在职保的替代率模型中,我们已经分析了缴费率与替代率的关系,即职工退休时的养老金由两部分组成:

\begin{itemize}
    \item 养老金 = 基础养老金 + 个人账户养老金
    \item 个人账户养老金 = 个人账户储存额 ÷ 计发月数
    \item 基础养老金 = (全国上年度在岗职工月平均工资 + 本人指数化月平均缴费工资)÷ 2 × 缴费年限 × 1\%
    \item 本人指数化月平均缴费工资 = 全国上年度在岗职工月平均工资 × 本人平均缴费指数
\end{itemize}

所以缴费率与替代率的关系如图所示,

\begin{equation}
R = \frac{x_1 \times \frac{C_1}{c_1} + x_2 \times \frac{C_1}{c_1} + L + x_m \times \frac{C_1}{c_m}}{\left(c_1 + \frac{C_1}{c_2} + \frac{C_2}{12m}\right) / 2 \times m \times 1\% + \sum_{i=1}^{m} C_i \times J_r \times (1 + r)^{m-i} / J}
\end{equation}

\begin{equation}
\frac{c_1}{12}
\end{equation}

(4-8)

\begin{table}[h]
\centering
\caption{替代率合理区间}
\begin{tabular}{|c|c|c|c|c|c|c|c|}
\hline
缴费率 & 0.22 & 0.23 & 0.24 & 0.25 & 0.26 & 0.27 & 0.28 \\
\hline
2020 & 0.26 & 0.27 & 0.29 & 0.34 & 0.36 & 0.38 & 0.40 \\
\hline
2035 & 0.27 & 0.30 & 0.33 & 0.35 & 0.38 & 0.41 & 0.44 \\
\hline
缴费率 & 0.29 & 0.3 & 0.31 & 0.32 & 0.33 & 0.34 & 0.35 \\
\hline
2020 & 0.42 & 0.44 & 0.46 & 0.48 & 0.50 & 0.53 & 0.55 \\
\hline
2035 & 0.47 & 0.50 & 0.52 & 0.54 & 0.57 & 0.60 & 0.63 \\
\hline
\end{tabular}
\end{table}

所以我们选择的缴费区间定位25%-33%。

\subsection*{4.3.3 建议与仿真}

\subsubsection{4.3.3.1 建议}

1、设定养老金替代率的最低标准和最高标准,以维护退休人员的正当利益

如果养老保险制度的目标只是为了确保退休人员最基本的生活消费支出,那么,替代率水平可以逐年降低。可以预期,在以后的年度,如果不采取措施,随着经济的快速增长,以生活消费支出衡量的平均替代率水平必然会降低。这样会引发一个问题,即这样做会把退休人员完全排斥在社会经济增长之外,从而激化社会矛盾,不利于社会的和谐稳定。由此,替代率不断降低的做法是不足取的,必须设定替代率的最低标准,这个标准应能维护退休人员的正当经济利益,保证他们可以不同程度地分享社会经济发展的成果。

2、贫富差距扩大,基本养老保险政策应该向低收入群体倾斜

低收入群体较高的替代率水平说明该群体的全部消费都是必要消费,而高收入群体的情况恰好与此相反。由于不同收人群体的必要消费支出都相同,因此低收入群体的收入增长幅度远远低于高收入群体,社会的贫富差距越来越大。为实现基本养老保险制度的社会公正性,基本养老政策应向低收入群体倾斜。

3、按物价指数调整养老金替代率有利于保持养老金的长期收支平衡

替代率调整时,可以按工资增长率调整,也可以按物价指数调整,而按物价指数调整具有较强的合理性,因为这种调整方式一方面可以维持退休人员的基本生活水平不至于因面临物价水平攀升和社会平均工资提高等双重压力而下降过多,另一方面不会对养老保险基金造成过大的支付压力。不仅如此,这种调整方案制度化以后,可以使基金在新的制度参数下形成规范的财务平衡机制,有利于保持养老基金的长期平衡。

\begin{table}
\centering
\begin{tabular}{|c|c|c|c|c|c|}
\hline
年份 & 缺口 & 年份 & 缺口 & 年份 & 缺口 \\
\hline
2012 &  & 2013 & 59139.72 & 2014 & 66141.99 \\
\hline
2015 & 72486.74 & 2016 & 79358.41 & 2017 & 86433.73 \\
\hline
2018 & 94350.78 & 2019 & 100527.22 & 2020 & 106305.98 \\
\hline
2021 & 112099.32 & 2022 & 112692.71 & 2023 & 116823.03 \\
\hline
2024 & 119594.16 & 2025 & 120900.03 & 2026 & 121761.59 \\
\hline
2027 & 122894.93 & 2028 & 124676.35 & 2029 & 118587.46 \\
\hline
2030 & 109524.04 & 2031 & 64095.30 & 2032 & 101475.38 \\
\hline
2033 & 93417.52 & 2034 & 85734.11 & 2035 & 78716.98 \\
\hline
2036 & 71533.61 & 2037 & 64593.77 & 2038 & 58311.21 \\
\hline
2039 & 52015.44 & 2040 & 45449.57 & 2041 & 39254.27 \\
\hline
2042 & 32983.75 & 2043 & 26736.72 & 2044 & 19419.96 \\
\hline
2045 & 10769.26 & 2046 & 2697.97 & 2047 & -5808.95 \\
\hline
2048 & -14823.92 & 2049 & -23384.75 & 2050 & -32739.53 \\
\hline
\end{tabular}
\caption{仿真结果}
\end{table}

和替代率限制前相比,缺口相应的有所减小,到2047年出现‘入不敷出’的情况,相比较于原来推迟了一年,支出与收入之间的矛盾有所缓解。

\subsection*{4.4 问题四}

\subsubsection{4.4.1 变量分析}

基本养老保险基金收支的周期平衡受到很多因素的影响。具体来说,影响养老保险基金收支平衡的关键因素有替代率、缴费率和抚养比;其他影响因素还包括工资增长率、养老金增值率、通货膨胀、人口迁移、经济景气变动、财政等,这些因素本身虽然并不是养老保险制度所规定的,但却是建立养老保险制度时不能不考虑的外部因素。本模型将对收益率、企业年金和退休年龄等因素的自身特征及其对替代率的影响分析,将其转化为可调节的变量。

我们依旧从变量替代率着手,考虑到养老保险制度对养老保险替代率的影响因素:缴费率、收益率、职工的退休年龄、企业年金方面,我们选择在企业年金基金投资组合及收益率方面进行建模改进作为选择。

企业年金基金投资组合的资产配置状况,是影响投资收益率及影响年金替代率水平的一个重要因素。投资于高风险资产的比例越高,企业年金积累额和替代率也越高,但同时风险也越大;投资低风险的银行存款和债券比例越高,年金资产的安全性越高,但相应的投资收益率也越低。为保证年金基金的投资安全,必须要权衡年金资产配置的适当比例,协调年金基金投资组合风险和收益之间的关系,使年金资产在承担较小风险的情况下收益最大化。

\subsubsection{4.4.2 投资组合目标规划模型优化设计}

对资产组合收益与风险的研究,需要参考马科维茨资产组合模型,其基本假

设如下:每一项可供选择的投资在一定持有期内都存在预期收益率的概率分布;投资者都追求某一时期内预期效用最大化,且其效用曲线表明财富的边际效用呈递减趋势;投资者根据预期收益率的波动性,估计资产组合的风险;投资者完全根据预期收益率和风险作出决策,其效用曲线只是预期收益率和预期收益率方差 (或标准差) 的函数;在给定风险水平下,投资者偏好较高的收益率[10]。

基于上述假设,该理论认为可以用收益率方差来衡量组合资产的风险,并把反映收益波动离散程度的统计测度——方差和标准差作为风险测量标准。资产组合的标准差是单项投资的标准差与该项资产组合中各项投资之间收益率协方差的函数,其数学本质是一个带有约束的最优化问题。

假定我们用 \( r_{i} \), (i=1, 2, …, N) 代表投资组合中Ⅳ类资产的收益率向量,\( \mu_{i} \) 表示资产 i 的收益率的期望值,且假设该种资产的收益率 \( r_{i} \) 呈正态分布,则第 i 类资产的风险可用其收益率的方差 \( \sigma_{i}^{2} \) 来度量,\( \sigma_{i}^{2} = \frac{1}{n} \sum_{i=1}^{n} \left( r_{i}^{t} - \mu_{i} \right) \),其中 \( r_{i}^{t} \) 是第 i 类资产在第 t 期的投资收益率。那么影响投资组合的主要因素为:各单个资产的收益率的方差、组合资产中两种资产间收益率变化的协方差、单个资产的权重。马柯维茨用组合资产方差模型来描述组合资产的风险,即

\[
\sigma_{p}^{2} = \sum_{i=1}^{n} \sum_{j=1}^{n} w_{i} w_{j} \operatorname{cov}\left(r_{i}, r_{j}\right)
\tag{4-1}
\]

其中 i, j=(1, 2, …, N),\( w_{i} \) 为第 i 类资产所占的权重,\( \operatorname{cov}\left(r_{i}, r_{j}\right) \) 为资产 i 和资产 j 的协方差:\( \operatorname{cov}\left(r_{i}, r_{j}\right) = \sigma_{ij} = E\left(r_{i} - \mu_{i}\right)\left(r_{j} - \mu_{j}\right) \)

理性的投资者寻求使得投资收益率达到一定程度的前提下组合投资的风险最小。每类资产的风险及资产间的关系是不以人的意志为转移的,因此需要寻求投资组合的最优配置。即合理的投资权重 \( W_{i} \) 在条件约束下达到风险最小。

\[
\begin{cases}
\min: \sigma_{p}^{2} = \sum_{i=1}^{n} \sum_{j=1}^{n} w_{i} w_{j} \operatorname{cov}\left(r_{i}, r_{j}\right) \\
s.t.: \mu_{p} = \sum_{i=1}^{n} w_{i} \mu_{i} \geq \mu \\
w_{1} + w_{2} + w_{3} + \ldots \ldots + w_{n} = 1 \\
l_{i} \leq w_{i} \leq k_{i} \left(i = 1, 2, \ldots, N\right)
\end{cases}
\tag{4-2}
\]

其中为 \( l_{i} \),\( k_{i} \) 为权重的上下限,\( \mu \) 为企业年金基金投资组合最低投资收益率。

依据《企业年金基金管理试行办法》第四十七条,企业年金基金财产的投资,按市场价计算应当符合下列规定:投资银行活期存款、中央银行票据、短期债券回购等流动性产品及货币市场基金的比例,不低于基于净资产的 20%。企业债、

金融债投资的比例不得高于 $10\%$。投资银行定期存款、协议存款、国债、金融债、企业债等固定收益类产品及可转换债、债券基金的比例,不高于基金净资产的 $50\%$。其中,投资国债的比例不低于基金净资产的 $20\%$。投资股票等权益类产品及投资性保险产品、股票基金的比例,不高于基金净资产的 $30\%$。其中,投资股票的比例不高于基金净资产的 $20\%$。

根据上述讨论,我们限定投资组合包括银行存款、国债、股票三类资产,用向量 $W = (w_1, w_2, w_3)$ 分别表示相应的投资比例,则 $Z = w_1 R_1 + w_2 R_2 + w_3 R_3$ 代表整个投资组合,$R = (R_1, R_2, R_3)$ 表示相应的投资收益率。因为企业年金基金必须保值增值才能保障员工退休后的生活,其组合的最小投资收益率至少要比消费价格指数高,根据上述历年通货膨胀率平均值约为 $5\%$,该方程组的约束条件为 $5\%$。建立的模型如下:

\begin{equation}
\begin{cases}
\min \delta_p^2 = \text{VAR}(Z) \\
s.t. \mu_p = E(Z) > 5 \\
w_1 + w_2 + w_3 = 1 \\
20\% \leq w_1 \leq 80\% \\
20\% \leq w_2 \leq 50\% \\
0 \leq w_3 \leq 30\%
\end{cases}
\tag{4-3}
\end{equation}

然后,我们利用 SV—T 模型预测将来的收益率。SV—T 模型中涉及的是经过标准化后的日收益率,因此不管是利用已知数据对模型参数进行模拟还是利用 SV—T 模型对收益率进行预测,对应的也都是标准化后的日收益率,而我们最终需要的是没有经过标准化处理的真实的日收益率。这需要利用标准化的收益率预测值对非标准化的收益率数据进行重构。方法是通过给定初值 $\overline{R}^{(0)}, S^{(0)}$,借助不动点迭代原理,利用下式得到。

\begin{equation}
\begin{cases}
\hat{R}_t^{(i)} = y_i S^{(i)} + \overline{R}^{(i)} \left(t = 1, 2, \ldots, D\right) \\
\overline{R}^{(i+1)} = \frac{1}{D} \sum_{i=1}^D \hat{R}_t^{(i)}, S^{(i+1)} = \sqrt{\frac{1}{D-1} \sum_{i=1}^D \left(\hat{R}_t^{(i)} - \overline{R}^{(i+1)}\right)^2}
\end{cases}
\tag{4-4}
\end{equation}

我们通过在投资效益和风险指数的约束下,通过对投资组合目标规划模型,找到最优化的投资模型,寻求最合理的收益率,从而影响整个模型的求解。

\section*{5. 结束语}

由于时间仓促,本文在模型建立上还存在一些问题,比如样本分析中选取的历史数据较少所以不够精确。另外,对于问题四的最后一问,我们也仅仅粗略的考虑了投资风险和收益率对投资方式组合的优化,建立了简单的模型,而没有具体的实验仿真,这也为我们未来的研究提供了一个发展方向。

\section*{参考文献:}
[1] 宋健. 人口预测和人口控制 [M]. 北京: 人民出版社, 1980.
[2] 杨峰, 范海菊. 改进的离散人口发展模型在人口预测中的应用研究 [J]. 中国科技论文, 2004.
[3] 国家统计局. http://www.stats.gov.cn. 2013.09.
[4] 郭永斌. 我国养老保险资金缺口的评估和可持续性分析 [J]. 保险市场, 2013.04.
[5] http://baike.baidu.com/view/407916.html.
[6] 刘玉堂, 李新芳. 养老金替代率的数学模型 [J]. 河南机电高等专科学院学报, 2012.5.
[7] 丁煜. 新农保个人账户设计的改进: 基于精算模型的分析 [J]. 社会保障研究, 2011.5.
[8] 马崇明. 中国现代化预测文献综述与理论依据 [J]. 中国现代化进程, 2003.
[9] 徐颖. 中国社会养老保险保障水平分析与评价 [M]. 社会科学文献出版社: 54-60
[10] 徐颖. 中国社会养老保险保障水平分析与评价 [M]. 社会科学文献出版社: 120-134

\section*{附录}

\section*{主函数}
\begin{verbatim}
function [Qk] = main()
    [I_cjb, E_cjb] = CJB();
    [I_xnb, E_xnb] = YNB();
    [I_zgb, E_zgb] = ZC_zgb();
    for i = 1:1:37
        Qk1(i) = I_cjb(i) + I_xnb(i) + I_zgb(i) - E_cjb(i) - E_xnb(i) - E_zgb(i);
        if Qk1(i) < 0
            Qk(i) = Qk1(i)
        else
            Qk(i) = Qk1(i) + Mbt(i+2012) * 100000000;
        end
    end
End
\end{verbatim}

\section*{计算城居保}
\begin{verbatim}
function [I_cjb, E_cjb] = CJB()
    [Num_cz, Num_jy, Num_xz, Num_tx, Num_jq] = ZGB();
    [number, M, W] = yuce1(50);
    for i = 1:1:37
        R(i, 1) = Rcjb(100, 40, 2012+i);
    end
\end{verbatim}

\begin{verbatim}
R(i,2)=Rcjb(200,40,2012+i);
R(i,3)=Rcjb(300,40,2012+i);
R(i,4)=Rcjb(400,40,2012+i);
R(i,5)=Rcjb(500,40,2012+i);
R(i,6)=Rcjb(600,40,2012+i);
R(i,7)=Rcjb(700,40,2012+i);
R(i,8)=Rcjb(800,40,2012+i);
R(i,9)=Rcjb(900,40,2012+i);
R(i,10)=Rcjb(1000,40,2012+i);
Z(i)=cz(i+2012);
end
for i=1:1:15
Num_cz60(i)=(M(60,i)+W(55,i))*(0.513+0.00344*i)*(1-0.519898-0.002387*i);
Num_cz20(i)=(M(20,i)+W(20,i))*(0.513+0.00344*i)*(1-0.519898-0.002387*i);
end
for i=16:1:40
Num_cz60(i)=(M(60,i)+W(55,i))*(0.5646+(0.00344-(0.00344/75)*(i-15)))*(1-0.555703-(0.002387-(0.002387/75)*(i-15)));
Num_cz20(i)=(M(20,i)+W(20,i))*(0.5646+(0.00344-(0.00344/75)*(i-15)))*(1-0.555703-(0.002387-(0.002387/75)*(i-15)));
end
Num_czjf(1)=80090000;
for i=1:1:38
Num_czjf(i+1)=Num_czjf(i)+Num_cz20(i)*0.9-Num_cz60(i)*0.7;
Num_czsy(1)=17750000;
Num_czsy(i+1)=Num_czsy(i)+Num_cz60(i)-Num_czsy(i)*0.311273;
end
for i=1:1:37
E_cjb(i)=Num_czsy(i)*(55*12+(0.04*R(i,1)+0.03*R(i,2)+0.08*R(i,3)+0.09*R(i,4)+0.17*R(i,5)+0.11*R(i,6)+0.09*R(i,7)+0.19*R(i,8)+0.06*R(i,9)+0.12*R(i,10))*Z(i));
I_cjb(i)=Num_czjf(i)*(0.04*100+0.03*200+0.08*300+0.09*400+0.17*500+0.11*600+0.09*700+0.19*800+0.06*900+0.12*1000);
End
\end{verbatim}

\textbf{城居保替换率}

\begin{verbatim}
function [w]=Rcjb(d,n,t)
r=0.0325;
pl=0;
for i=1:60-n
\end{verbatim}

\begin{verbatim}
p1 = p1 + (1 + r)^(60 - n + 1 - i)
end
p2 = 0;
for j = 0:12
    p2 = p2 + (1 / (1 + r))^j
end
for u = 2013:2050
    x(u) = G_GDP(u);
    if u == 2013
        y(u) = 7917 * (1 + x(u));
    else
        y(u) = y(u - 1) * (1 + x(u));
    end
end
y(2012) = 26959;
p = (d + 30 + (d / 100 - 1) * 5) * p1 / 12 / p2;
w = 12 * p / y(t - 1) / (1 + 0.0555)^(60 - n)
end

新农保
function [I_xnb, E_xnb] = YNB()
[number, M, W] = yucel(50);
[Num_cz, Num_jy, Num_xz, Num_tx, Num_jq] = ZGB();
for i = 1:1:40
    Num_nc(i) = number(i) - Num_cz(i);
end

for i = 1:1:38
    z(i) = nm(2012 + i);
    R(i, 1) = Rxnb(100, 40, 2012 + i);
    R(i, 2) = Rxnb(200, 40, 2012 + i);
    R(i, 3) = Rxnb(300, 40, 2012 + i);
    R(i, 4) = Rxnb(400, 40, 2012 + i);
    R(i, 5) = Rxnb(500, 40, 2012 + i);
end

for i = 1:1:15
    Num_nc60(i) = (M(60, i) + W(60, i)) * (1 - 0.513 - 0.00344 * i);
    Num_nc20(i) = (M(20, i) + W(20, i)) * (0.513 + 0.00344 * i);
end
for i = 16:1:38
    Num_nc60(i) = (M(60, i) + W(60, i)) * (0.5646 + (0.00344 - (0.00344 / 75) * (i - 15)));
    Num_nc20(i) = (M(20, i) + W(20, i)) * (0.5646 + (0.00344 - (0.00344 / 75) * (i - 15)));
end
\end{verbatim}

\begin{verbatim}
Num_ncjf(1)=326430000;
Num_ncsy(1)=85250000;
for i=1:1:38
    Num_ncjf(i+1)=Num_ncjf(i)+Num_nc20(i)*0.98-Num_nc60(i)*0.9;
    Num_ncsy(i+1)=Num_ncsy(i)+Num_nc60(i)*0.9-Num_ncsy(i)*0.311273;
    I_xnb(i)=Num_ncjf(i)*(0.88*100+0.063*200+0.023*300+0.006*400+0.028*500);
    E_xnb(i)=Num_ncsy(i)*(55*12+(0.88*R(i,1)+0.063*R(i,2)+0.023*R(i,3)+0.006*R(i,4)
    +0.028*R(i,5)));
End

新农保替换率
function [w]=Rxnb(d,n,t)
r=0.0325;
p1=0;
for i=1:60-n
    p1=p1+(1+r)^(60-n+1-i)
end
p2=0;
for j=0:12
    p2=p2+(1/(1+r))^(j)
end
for u=2013:2050
    x(u)=G_GDP(u);
    if(u==2013)
        y(u)=7917*(1+x(u));
    else
        y(u)=y(u-1)*(1+x(u));
    end
end
y(2012)=7917;
p=(d+30)*p1/12/p2;
w=12*p/y(t-1)/(1+0.0538)^(60-n)
end

职工保
function [I_zgb,E_zgb,Num_sy]=ZC_zgb()
[Num_cz,Num_jy,Num_xz,Num_tx,Num_jq] = ZGB();

for i=1:1:38
    w(i)=Rzb(0,40,2012+i);
    y(i)=zg(2012+i);
end

for i=1:1:9
\end{verbatim}

\begin{verbatim}
I_zgb(i) = Num_jq(i) * 0.28 * (0.9 + 0.01 * i) * y(i); % 
Num_sy(i+1) = Num_sy(i) + Num_tx(i) - Num_sy(i) * 0.023273;
E_zgb(i) = Num_sy(i) * y(i) * w(i);
end
for i = 10:1:38
    I_zgb(i) = Num_jq(i) * 0.28 * 0.98 * y(i);
    Num_sy(i+1) = Num_sy(i) + Num_tx(i) - Num_sy(i) * 0.023273;
    E_zgb(i) = Num_sy(i) * y(i) * w(i);
end
\end{verbatim}

\textbf{职工保替代率}
\begin{verbatim}
function [w] = Rzb(d, n, t)
for u = 2013:2050
    x(u) = G_GDP(u);
    if (u == 2013)
        y(u) = 47593 * (1 + x(u));
    else
        y(u) = y(u-1) * (1 + x(u));
    end
end
for u = 1:2012
    y(u) = 47593 / (1.1364)^(2012-u);
end
r = 0.0325;
p1 = 0;
p2 = 0;
for i = 1:n
    p1 = p1 + (1 + r)^(n-i) * 0.08 * y(t-i);
    p2 = p2 + y(t-i);
end
if d == 1
    J = 139;
else
    J = 170;
end
p1 = p1 / J;
p2 = (y(t-1) / 12 + p2 * 0.28 / 12 / n) / 2 * n * 0.01;
w = (p1 + p2) * 12 / y(t-1);
if w > 0.53
    w = 0.53;
else if w < 0.35
    w = 0.35
end
\end{verbatim}

\begin{verbatim}
end

人口预测
function [number, Man, Woman] = yucel(n)
format long g;
M0 = xlsread('男出生率.xlsx');
M(:, 1) = M0(:, 2);
W0 = xlsread('女出生率.xlsx');
W(:, 1) = W0(:, 4);
T0 = xlsread('男死亡率.xls');
T1 = xlsread('女死亡率.xls');
Y = xlsread('出生率.xls');
for t = 1:n
    k(t) = 100 / ((5/n)*t + 100 - 0.25*(2011 + t));
    for c = 1:5
        M(1, t+1) = 0;
        W(1, t+1) = 0;
        M(c+1, t+1) = (1 - (T0(c+1, c)/1000)) * M(c, t);
        W(c+1, t+1) = (1 - (T1(c)/1000)) * W(c, t);
        T0(c+1, c) = T0(c+1, c) * (1 - 0.04);
        T0(c+1, c) = T0(c+1, c) * (1 - 0.04);
    end
    for c = 6:50
        M(c+1, t+1) = (1 - (T0(c+1, c)/1000)) * M(c, t);
        W(c+1, t+1) = (1 - (T1(c)/1000)) * W(c, t);
        T0(c+1, c) = T0(c+1, c) * (1 - 0.03);
        T0(c+1, c) = T0(c+1, c) * (1 - 0.03);
    end
    for c = 51:90
        M(c+1, t+1) = (1 - (T0(c+1, c)/1000)) * M(c, t);
        W(c+1, t+1) = (1 - (T1(c)/1000)) * W(c, t);
        T0(c+1, c) = T0(c+1, c) * (1 - 0.01);
        T0(c+1, c) = T0(c+1, c) * (1 - 0.01);
    end
    for s = 15:49
        M(1, t+1) = M(1, t) + W(s, 1) * (2*Y(s-14)/1000) * 1.8 * (1 - k(t));
        W(1, t+1) = W(1, t) + W(s, 1) * (2*Y(s-14)/1000) * 1.8 * k(t);
    end
    number(t) = sum(W(:, t+1) + M(:, t+1));
end
Man = M;
Woman = W;
\end{verbatim}

\begin{verbatim}
GDP预测
function [zt] = G_GDP(t)
if (t >= 2040 && t <= 2050)
    zt = 0.025 + 0.002 * (2050 - t);
elseif (t >= 2020 && t < 2040)
    zt = 0.045 + 0.001 * (2040 - t);
elseif (t >= 2013 && t < 2020)
    zt1 = 0.065 + (2020 - t) * (0.142 - 0.065) / 13;
    wt = [-0.3033 -0.2280 -0.1527 -0.0774 0 0 0];
    wt1 = wt(t - 2012);
    zt = exp(log(zt1) + wt1);
end

政府补贴
function [e] = Mbt(t)
j = 1;
for i = 2013:t
    j = j * (1 + G_GDP(t));
end
s = j * 519322.07;
e = s * 0.00424;

城镇就业人口结构
function [Num_cz, Num_jy, Num_xz, Num_tx, Num_jq] = ZGB()
[number, M, W] = yucel(50);
Num_cz(1) = number(1) * 0.513;
Num_jy(1) = 359140000;
Num_jq(1) = 215650000;
for i = 1:1:15
    Num_cz(i + 1) = number(i) * 0.01344 + Num_cz(i) * (1 - 0.00714);
    Num_jy(i + 1) = Num_cz(i + 1) * (0.519898 + 0.002387 * i);
end
for i = 16:1:40
    Rc(i) = 0.001344 - (0.001344 / 75) * (i - 15);
    Num_cz(i + 1) = number(i) * Rc(i) + Num_cz(i) * (1 - 0.00714);
    Num_jy(i + 1) = Num_cz(i + 1) * (0.555703 + (0.002387 - (0.002387 / 75) * (i - 15)));
end

for i = 1:1:15
    Num_tx(i) = (M(60, i) + W(55, i)) * (0.513 + 0.00344 * i) * (0.519898 + 0.002387 * i);
    Num_xz(i) = Num_jy(i + 1) + Num_tx(i) - Num_jy(i);
    Num_jq(i + 1) = Num_jq(i) + Num_xz(i) * 0.9 - Num_tx(i) * (0.6004622 + ((0.9 - 0.6004622) / 40));
\end{verbatim}

\begin{verbatim}
*i);
end
for i=16:1:40
    Num_tx(i)=(M(60,i)+W(55,i))*(0.5646+(0.00344-(0.00344/75)*(i-15)))*(0.555703+(0.002387-(0.002387/75)*(i-15)));
    Num_xz(i)=Num_jy(i+1)+Num_tx(i)-Num_jy(i);
    Num_jq(i+1)=Num_jq(i)+Num_xz(i)*0.9-Num_tx(i)*(0.6004622+((0.9-0.6004622)/40)*i);
end

平均工资
function [z]=zg(i)
for i=2013:2050
    x(i)=G_GDP(i);
    if(i==2013)
        y(i)=47593*(1+x(i));
    else
        y(i)=y(i-1)*(1+x(i));
    end
end
z=y(i);
function [z]=nm(i)
for i=2013:2050
    x(i)=G_GDP(i);
    if(i==2013)
        y(i)=7917*(1+x(i));
    else
        y(i)=y(i-1)*(1+x(i));
    end
end
z=y(i);
function [z]=cz(i)
for i=2013:2050
    x(i)=G_GDP(i);
    if(i==2013)
        y(i)=26959*(1+x(i));
    else
        y(i)=y(i-1)*(1+x(i));
    end
end
z=y(i);
\end{verbatim}