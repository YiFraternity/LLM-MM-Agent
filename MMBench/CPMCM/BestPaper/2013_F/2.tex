\begin{center}
\textbf{第十届华为杯全国研究生数学建模竞赛}
\end{center}

\begin{table}[h]
\centering
\begin{tabular}{c c}
\hline
学 校 & 浙江工商大学 \\
\hline
\end{tabular}
\end{table}

\begin{table}[h]
\centering
\begin{tabular}{c c}
\hline
参赛队号 & 10353003 \\
\hline
\end{tabular}
\end{table}

\begin{table}[h]
\centering
\begin{tabular}{c|c}
\hline
队员姓名 & 1. 孙鹏 \\
\hline
 & 2. 贺洁琼 \\
\hline
 & 3. 孙恒宇 \\
\hline
\end{tabular}
\end{table}

\begin{flushright}
参赛密码 \_\_\_\_\_\_\_\_\_\_\_\_\_\_\_\_\_\_\_\_\_\_\_\_\_\_\_\_\_\_\_\_\_\_\_\_\_\_\_\_\_\_\_\_\_\_\_\_\_\_\_\_\_\_\_\_\_\_\_\_\_\_\_\_\_\_\_\_\_\_\_\_\_\_\_\_\_\_\_\_\_\_\_\_\_\_\_\_\_\_\_\_\_\_\_\_\_\_\_\_\_\_\_\_\_\_\_\_\_\_\_\_\_\_\_\_\_\_\_\_\_\_\_\_\_\_\_\_\_\_\_\_\_\_\_\_\_\_\_\_\_\_\_\_\_\_\_\_\_\_\_\_\_\_\_\_\_\_\_\_\_\_\_\_\_\_\_\_\_\_\_\_\_\_\_\_\_\_\_\_\_\_\_\_\_\_\_\_\_\_\_\_\_\_\_\_\_\_\_\_\_\_\_\_\_\_\_\_\_\_\_\_\_\_\_\_\_\_\_\_\_\_\_\_\_\_\_\_\_\_\_\_\_\_\_\_\_\_\_\_\_\_\_\_\_\_\_\_\_\_\_\_\_\_\_\_\_\_\_\_\_\_\_\_\_\_\_\_\_\_\_\_\_\_\_\_\_\_\_\_\_\_\_\_\_\_\_\_\_\_\_\_\_\_\_\_\_\_\_\_\_\_\_\_\_\_\_\_\_\_\_\_\_\_\_\_\_\_\_\_\_\_\_\_\_\_\_\_\_\_\_\_\_\_\_\_\_\_\_\_\_\_\_\_\_\_\_\_\_\_\_\_\_\_\_\_\_\_\_\_\_\_\_\_\_\_\_\_\_\_\_\_\_\_\_\_\_\_\_\_\_\_\_\_\_\_\_\_\_\_\_\_\_\_\_\_\_\_\_\_\_\_\_\_\_\_\_\_\_\_\_\_\_\_\_\_\_\_\_\_\_\_\_\_\_\_\_\_\_\_\_\_\_\_\_\_\_\_\_\_\_\_\_\_\_\_\_\_\_\_\_\_\_\_\_\_\_\_\_\_\_\_\_\_\_\_\_\_\_\_\_\_\_\_\_\_\_\_\_\_\_\_\_\_\_\_\_\_\_\_\_\_\_\_\_\_\_\_\_\_\_\_\_\_\_\_\_\_\_\_\_\_\_\_\_\_\_\_\_\_\_\_\_\_\_\_\_\_\_\_\_\_\_\_\_\_\_\_\_\_\_\_\_\_\_\_\_\_\_\_\_\_\_\_\_\_\_\_\_\_\_\_\_\_\_\_\_\_\_\_\_\_\_\_\_\_\_\_\_\_\_\_\_\_\_\_\_\_\_\_\_\_\_\_\_\_\_\_\_\_\_\_\_\_\_\_\_\_\_\_\_\_\_\_\_\_\_\_\_\_\_\_\_\_\_\_\_\_\_\_\_\_\_\_\_\_\_\_\_\_\_\_\_\_\_\_\_\_\_\_\_\_\_\_\_\_\_\_\_\_\_\_\_\_\_\_\_\_\_\_\_\_\_\_\_\_\_\_\_\_\_\_\_\_\_\_\_\_\_\_\_\_\_\_\_\_\_\_\_\_\_\_\_\_\_\_\_\_\_\_\_\_\_\_\_\_\_\_\_\_\_\_\_\_\_\_\_\_\_\_\_\_\_\_\_\_\_\_\_\_\_\_\_\_\_\_\_\_\_\_\_\_\_\_\_\_\_\_\_\_\_\_\_\_\_\_\_\_\_\_\_\_\_\_\_\_\_\_\_\_\_\_\_\_\_\_\_\_\_\_\_\_\_\_\_\_\_\_\_\_\_\_\_\_\_\_\_\_\_\_\_\_\_\_\_\_\_\_\_\_\_\_\_\_\_\_\_\_\_\_\_\_\_\_\_\_\_\_\_\_\_\_\_\_\_\_\_\_\_\_\_\_\_\_\_\_\_\_\_\_\_\_\_\_\_\_\_\_\_\_\_\_\_\_\_\_\_\_\_\_\_\_\_\_\_\_\_\_\_\_\_\_\_\_\_\_\_\_\_\_\_\_\_\_\_\_\_\_\_\_\_\_\_\_\_\_\_\_\_\_\_\_\_\_\_\_\_\_\_\_\_\_\_\_\_\_\_\_\_\_\_\_\_\_\_\_\_\_\_\_\_\_\_\_\_\_\_\_\_\_\_\_\_\_\_\_\_\_\_\_\_\_\_\_\_\_\_\_\_\_\_\_\_\_\_\_\_\_\_\_\_\_\_\_\_\_\_\_\_\_\_\_\_\_\_\_\_\_\_\_\_\_\_\_\_\_\_\_\_\_\_\_\_\_\_\_\_\_\_\_\_\_\_\_\_\_\_\_\_\_\_\_\_\_\_\_\_\_\_\_\_\_\_\_\_\_\_\_\_\_\_\_\_\_\_\_\_\_\_\_\_\_\_\_\_\_\_\_\_\_\_\_\_\_\_\_\_\_\_\_\_\_\_\_\_\_\_\_\_\_\_\_\_\_\_\_\_\_\_\_\_\_\_\_\_\_\_\_\_\_\_\_\_\_\_\_\_\_\_\_\_\_\_\_\_\_\_\_\_\_\_\_\_\_\_\_\_\_\_\_\_\_\_\_\_\_\_\_\_\_\_\_\_\_\_\_\_\_\_\_\_\_\_\_\_\_\_\_\_\_\_\_\_\_\_\_\_\_\_\_\_\_\_\_\_\_\_\_\_\_\_\_\_\_\_\_\_\_\_\_\_\_\_\_\_\_\_\_\_\_\_\_\_\_\_\_\_\_\_\_\_\_\_\_\_\_\_\_\_\_\_\_\_\_\_\_\_\_\_\_\_\_\_\_\_\_\_\_\_\_\_\_\_\_\_\_\_\_\_\_\_\_\_\_\_\_\_\_\_\_\_\_\_\_\_\_\_\_\_\_\_\_\_\_\_\_\_\_\_\_\_\_\_\_\_\_\_\_\_\_\_\_\_\_\_\_\_\_\_\_\_\_\_\_\_\_\_\_\_\_\_\_\_\_\_\_\_\_\_\_\_\_\_\_\_\_\_\_\_\_\_\_\_\_\_\_\_\_\_\_\_\_\_\_\_\_\_\_\_\_\_\_\_\_\_\_\_\_\_\_\_\_\_\_\_\_\_\_\_\_\_\_\_\_\_\_\_\_\_\_\_\_\_\_\_\_\_\_\_\_\_\_\_\_\_\_\_\_\_\_\_\_\_\_\_\_\_\_\_\_\_\_\_\_\_\_\_\_\_\_\_\_\_\_\_\_\_\_\_\_\_\_\_\_\_\_\_\_\_\_\_\_\_\_\_\_\_\_\_\_\_\_\_\_\_\_\_\_\_\_\_\_\_\_\_\_\_\_\_\_\_\_\_\_\_\_\_\_\_\_\_\_\_\_\_\_\_\_\_\_\_\_\_\_\_\_\_\_\_\_\_\_\_\_\_\_\_\_\_\_\_\_\_\_\_\_\_\_\_\_\_\_\_\_\_\_\_\_\_\_\_\_\_\_\_\_\_\_\_\_\_\_\_\_\_\_\_\_\_\_\_\_\_\_\_\_\_\_\_\_\_\_\_\_\_\_\_\_\_\_\_\_\_\_\_\_\_\_\_\_\_\_\_\_\_\_\_\_\_\_\_\_\_\_\_\_\_\_\_\_\_\_\_\_\_\_\_\_\_\_\_\_\_\_\_\_\_\_\_\_\_\_\_\_\_\_\_\_\_\_\_\_\_\_\_\_\_\_\_\_\_\_\_\_\_\_\_\_\_\_\_\_\_\_\_\_\_\_\_\_\_\_\_\_\_\_\_\_\_\_\_\_\_\_\_\_\_\_\_\_\_\_\_\_\_\_\_\_\_\_\_\_\_\_\_\_\_\_\_\_\_\_\_\_\_\_\_\_\_\_\_\_\_\_\_\_\_\_\_\_\_\_\_\_\_\_\_\_\_\_\_\_\_\_\_\_\_\_\_\_\_\_\_\_\_\_\_\_\_\_\_\_\_\_\_\_\_\_\_\_\_\_\_\_\_\_\_\_\_\_\_\_\_\_\_\_\_\_\_\_\_\_\_\_\_\_\_\_\_\_\_\_\_\_\_\_\_\_\_\_\_\_\_\_\_\_\_\_\_\_\_\_\_\_\_\_\_\_\_\_\_\_\_\_\_\_\_\_\_\_\_\_\_\_\_\_\_\_\_\_\_\_\_\_\_\_\_\_\_\_\_\_\_\_\_\_\_\_\_\_\_\_\_\_\_\_\_\_\_\_\_\_\_\_\_\_\_\_\_\_\_\_\_\_\_\_\_\_\_\_\_\_\_\_\_\_\_\_\_\_\_\_\_\_\_\_\_\_\_\_\_\_\_\_\_\_\_\_\_\_\_\_\_\_\_\_\_\_\_\_\_\_\_\_\_\_\_\_\_\_\_\_\_\_\_\_\_\_\_\_\_\_\_\_\_\_\_\_\_\_\_\_\_\_\_\_\_\_\_\_\_\_\_\_\_\_\_\_\_\_\_\_\_\_\_\_\_\_\_\_\_\_\_\_\_\_\_\_\_\_\_\_\_\_\_\_\_\_\_\_\_\_\_\_\_\_\_\_\_\_\_\_\_\_\_\_\_\_\_\_\_\_\_\_\_\_\_\_\_\_\_\_\_\_\_\_\_\_\_\_\_\_\_\_\_\_\_\_\_\_\_\_\_\_\_\_\_\_\_\_\_\_\_\_\_\_\_\_\_\_\_\_\_\_\_\_\_\_\_\_\_\_\_\_\_\_\_\_\_\_\_\_\_\_\_\_\_\_\_\_\_\_\_\_\_\_\_\_\_\_\_\_\_\_\_\_\_\_\_\_\_\_\_\_\_\_\_\_\_\_\_\_\_\_\_\_\_\_\_\_\_\_\_\_\_\_\_\_\_\_\_\_\_\_\_\_\_\_\_\_\_\_\_\_\_\_\_\_\_\_\_\_\_\_\_\_\_\_\_\_\_\_\_\_\_\_\_\_\_\_\_\_\_\_\_\_\_\_\_\_\_\_\_\_\_\_\_\_\_\_\_\_\_\_\_\_\_\_\_\_\_\_\_\_\_\_\_\_\_\_\_\_\_\_\_\_\_\_\_\_\_\_\_\_\_\_\_\_\_\_\_\_\_\_\_\_\_\_\_\_\_\_\_\_\_\_\_\_\_\_\_\_\_\_\_\_\_\_\_\_\_\_\_\_\_\_\_\_\_\_\_\_\_\_\_\_\_\_\_\_\_\_\_\_\_\_\_\_\_\_\_\_\_\_\_\_\_\_\_\_\_\_\_\_\_\_\_\_\_\_\_\_\_\_\_\_\_\_\_\_\_\_\_\_\_\_\_\_\_\_\_\_\_\_\_\_\_\_\_\_\_\_\_\_\_\_\_\_\_\_\_\_\_\_\_\_\_\_\_\_\_\_\_\_\_\_\_\_\_\_\_\_\_\_\_\_\_\_\_\_\_\_\_\_\_\_\_\_\_\_\_\_\_\_\_\_\_\_\_\_\_\_\_\_\_\_\_\_\_\_\_\_\_\_\_\_\_\_\_\_\_\_\_\_\_\_\_\_\_\_\_\_\_\_\_\_\_\_\_\_\_\_\_\_\_\_\_\_\_\_\_\_\_\_\_\_\_\_\_\_\_\_\_\_\_\_\_\_\_\_\_\_\_\_\_\_\_\_\_\_\_\_\_\_\_\_\_\_\_\_\_\_\_\_\_\_\_\_\_\_\_\_\_\_\_\_\_\_\_\_\_\_\_\_\_\_\_\_\_\_\_\_\_\_\_\_\_\_\_\_\_\_\_\_\_\_\_\_\_\_\_\_\_\_\_\_\_\_\_\_\_\_\_\_\_\_\_\_\_\_\_\_\_\_\_\_\_\_\_\_\_\_\_\_\_\_\_\_\_\_\_\_\_\_\_\_\_\_\_\_\_\_\_\_\_\_\_\_\_\_\_\_\_\_\_\_\_\_\_\_\_\_\_\_\_\_\_\_\_\_\_\_\_\_\_\_\_\_\_\_\_\_\_\_\_\_\_\_\_\_\_\_\_\_\_\_\_\_\_\_\_\_\_\_\_\_\_\_\_\_\_\_\_\_\_\_\_\_\_\_\_\_\_\_\_\_\_\_\_\_\_\_\_\_\_\_\_\_\_\_\_\_\_\_\_\_\_\_\_\_\_\_\_\_\_\_\_\_\_\_\_\_\_\_\_\_\_\_\_\_\_\_\_\_\_\_\_\_\_\_\_\_\_\_\_\_\_\_\_\_\_\_\_\_\_\_\_\_\_\_\_\_\_\_\_\_\_\_\_\_\_\_\_\_\_\_\_\_\_\_\_\_\_\_\_\_\_\_\_\_\_\_\_\_\_\_\_\_\_\_\_\_\_\_\_\_\_\_\_\_\_\_\_\_\_\_\_\_\_\_\_\_\_\_\_\_\_\_\_\_\_\_\_\_\_\_\_\_\_\_\_\_\_\_\_\_\_\_\_\_\_\_\_\_\_\_\_\_\_\_\_\_\_\_\_\_\_\_\_\_\_\_\_\_\_\_\_\_\_\_\_\_\_\_\_\_\_\_\_\_\_\_\_\_\_\_\_\_\_\_\_\_\_\_\_\_\_\_\_\_\_\_\_\_\_\_\_\_\_\_\_\_\_\_\_\_\_\_\_\_\_\_\_\_\_\_\_\_\_\_\_\_\_\_\_\_\_\_\_\_\_\_\_\_\_\_\_\_\_\_\_\_\_\_\_\_\_\_\_\_\_\_\_\_\_\_\_\_\_\_\_\_\_\_\_\_\_\_\_\_\_\_\_\_\_\_\_\_\_\_\_\_\_\_\_\_\_\_\_\_\_\_\_\_\_\_\_\_\_\_\_\_\_\_\_\_\_\_\_\_\_\_\_\_\_\_\_\_\_\_\_\_\_\_\_\_\_\_\_\_\_\_\_\_\_\_\_\_\_\_\_\_\_\_\_\_\_\_\_\_\_\_\_\_\_\_\_\_\_\_\_\_\_\_\_\_\_\_\_\_\_\_\_\_\_\_\_\_\_\_\_\_\_\_\_\_\_\_\_\_\_\_\_\_\_\_\_\_\_\_\_\_\_\_\_\_\_\_\_\_\_\_\_\_\_\_\_\_\_\_\_\_\_\_\_\_\_\_\_\_\_\_\_\_\_\_\_\_\_\_\_\_\_\_\_\_\_\_\_\_\_\_\_\_\_\_\_\_\_\_\_\_\_\_\_\_\_\_\_\_\_\_\_\_\_\_\_\_\_\_\_\_\_\_\_\_\_\_\_\_\_\_\_\_\_\_\_\_\_\_\_\_\_\_\_\_\_\_\_\_\_\_\_\_\_\_\_\_\_\_\_\_\_\_\_\_\_\_\_\_\_\_\_\_\_\_\_\_\_\_\_\_\_\_\_\_\_\_\_\_\_\_\_\_\_\_\_\_\_\_\_\_\_\_\_\_\_\_\_\_\_\_\_\_\_\_\_\_\_\_\_\_\_\_\_\_\_\_\_\_\_\_\_\_\_\_\_\_\_\_\_\_\_\_\_\_\_\_\_\_\_\_\_\_\_\_\_\_\_\_\_\_\_\_\_\_\_\_\_\_\_\_\_\_\_\_\_\_\_\_\_\_\_\_\_\_\_\_\_\_\_\_\_\_\_\_\_\_\_\_\_\_\_\_\_\_\_\_\_\_\_\_\_\_\_\_\_\_\_\_\_\_\_\_\_\_\_\_\_\_\_\_\_\_\_\_\_\_\_\_\_\_\_\_\_\_\_\_\_\_\_\_\_\_\_\_\_\_\_\_\_\_\_\_\_\_\_\_\_\_\_\_\_\_\_\_\_\_\_\_\_\_\_\_\_\_\_\_\_\_\_\_\_\_\_\_\_\_\_\_\_\_\_\_\_\_\_\_\_\_\_\_\_\_\_\_\_\_\_\_\_\_\_\_\_\_\_\_\_\_\_\_\_\_\_\_\_\_\_\_\_\_\_\_\_\_\_\_\_\_\_\_\_\_\_\_\_\_\_\_\_\_\_\_\_\_\_\_\_\_\_\_\_\_\_\_\_\_\_\_\_\_\_\_\_\_\_\_\_\_\_\_\_\_\_\_\_\_\_\_\_\_\_\_\_\_\_\_\_\_\_\_\_\_\_\_\_\_\_\_\_\_\_\_\_\_\_\_\_\_\_\_\_\_\_\_\_\_\_\_\_\_\_\_\_\_\_\_\_\_\_\_\_\_\_\_\_\_\_\_\_\_\_\_\_\_\_\_\_\_\_\_\_\_\_\_\_\_\_\_\_\_\_\_\_\_\_\_\_\_\_\_\_\_\_\_\_\_\_\_\_\_\_\_\_\_\_\_\_\_\_\_\_\_\_\_\_\_\_\_\_\_\_\_\_\_\_\_\_\_\_\_\_\_\_\_\_\_\_\_\_\_\_\_\_\_\_\_\_\_\_\_\_\_\_\_\_\_\_\_\_\_\_\_\_\_\_\_\_\_\_\_\_\_\_\_\_\_\_\_\_\_\_\_\_\_\_\_\_\_\_\_\_\_\_\_\_\_\_\_\_\_\_\_\_\_\_\_\_\_\_\_\_\_\_\_\_\_\_\_\_\_\_\_\_\_\_\_\_\_\_\_\_\_\_\_\_\_\_\_\_\_\_\_\_\_\_\_\_\_\_\_\_\_\_\_\_\_\_\_\_\_\_\_\_\_\_\_\_\_\_\_\_\_\_\_\_\_\_\_\_\_\_\_\_\_\_\_\_\_\_\_\_\_\_\_\_\_\_\_\_\_\_\_\_\_\_\_\_\_\_\_\_\_\_\_\_\_\_\_\_\_\_\_\_\_\_\_\_\_\_\_\_\_\_\_\_\_\_\_\_\_\_\_\_\_\_\_\_\_\_\_\_\_\_\_\_\_\_\_\_\_\_\_\_\_\_\_\_\_\_\_\_\_\_\_\_\_\_\_\_\_\_\_\_\_\_\_\_\_\_\_\_\_\_\_\_\_\_\_\_\_\_\_\_\_\_\_\_\_\_\_\_\_\_\_\_\_\_\_\_\_\_\_\_\_\_\_\_\_\_\_\_\_\_\_\_\_\_\_\_\_\_\_\_\_\_\_\_\_\_\_\_\_\_\_\_\_\_\_\_\_\_\_\_\_\_\_\_\_\_\_\_\_\_\_\_\_\_\_\_\_\_\_\_\_\_\_\_\_\_\_\_\_\_\_\_\_\_\_\_\_\_\_\_\_\_\_\_\_\_\_\_\_\_\_\_\_\_\_\_\_\_\_\_\_\_\_\_\_\_\_\_\_\_\_\_\_\_\_\_\_\_\_\_\_\_\_\_\_\_\_\_\_\_\_\_\_\_\_\_\_\_\_\_\_\_\_\_\_\_\_\_\_\_\_\_\_\_\_\_\_\_\_\_\_\_\_\_\_\_\_\_\_\_\_\_\_\_\_\_\_\_\_\_\_\_\_\_\_\_\_\_\_\_\_\_\_\_\_\_\_\_\_\_\_\_\_\_\_\_\_\_\_\_\_\_\_\_\_\_\_\_\_\_\_\_\_\_\_\_\_\_\_\_\_\_\_\_\_\_\_\_\_\_\_\_\_\_\_\_\_\_\_\_\_\_\_\_\_\_\_\_\_\_\_\_\_\_\_\_\_\_\_\_\_\_\_\_\_\_\_\_\_\_\_\_\_\_\_\_\_\_\_\_\_\_\_\_\_\_\_\_\_\_\_\_\_\_\_\_\_\_\_\_\_\_\_\_\_\_\_\_\_\_\_\_\_\_\_\_\_\_\_\_\_\_\_\_\_\_\_\_\_\_\_\_\_\_\_\_\_\_\_\_\_\_\_\_\_\_\_\_\_\_\_\_\_\_\_\_\_\_\_\_\_\_\_\_\_\_\_\_\_\_\_\_\_\_\_\_\_\_\_\_\_\_\_\_\_\_\_\_\_\_\_\_\_\_\_\_\_\_\_\_\_\_\_\_\_\_\_\_\_\_\_\_\_\_\_\_\_\_\_\_\_\_\_\_\_\_\_\_\_\_\_\_\_\_\_\_\_\_\_\_\_\_\_\_\_\_\_\_\_\_\_\_\_\_\_\_\_\_\_\_\_\_\_\_\_\_\_\_\_\_\_\_\_\_\_\_\_\_\_\_\_\_\_\_\_\_\_\_\_\_\_\_\_\_\_\_\_\_\_\_\_\_\_\_\_\_\_\_\_\_\_\_\_\_\_\_\_\_\_\_\_\_\_\_\_\_\_\_\_\_\_\_\_\_\_\_\_\_\_\_\_\_\_\_\_\_\_\_\_\_\_\_\_\_\_\_\_\_\_\_\_\_\_\_\_\_\_\_\_\_\_\_\_\_\_\_\_\_\_\_\_\_\_\_\_\_\_\_\_\_\_\_\_\_\_\_\_\_\_\_\_\_\_\_\_\_\_\_\_\_\_\_\_\_\_\_\_\_\_\_\_\_\_\_\_\_\_\_\_\_\_\_\_\_\_\_\_\_\_\_\_\_\_\_\_\_\_\_\_\_\_\_\_\_\_\_\_\_\_\_\_\_\_\_\_\_\_\_\_\_\_\_\_\_\_\_\_\_\_\_\_\_\_\_\_\_\_\_\_\_\_\_\_\_\_\_\_\_\_\_\_\_\_\_\_\_\_\_\_\_\_\_\_\_\_\_\_\_\_\_\_\_\_\_\_\_\_\_\_\_\_\_\_\_\_\_\_\_\_\_\_\_\_\_\_\_\_\_\_\_\_\_\_\_\_\_\_\_\_\_\_\_\_\_\_\_\_\_\_\_\_\_\_\_\_\_\_\_\_\_\_\_\_\_\_\_\_\_\_\_\_\_\_\_\_\_\_\_\_\_\_\_\_\_\_\_\_\_\_\_\_\_\_\_\_\_\_\_\_\_\_\_\_\_\_\_\_\_\_\_\_\_\_\_\_\_\_\_\_\_\_\_\_\_\_\_\_\_\_\_\_\_\_\_\_\_\_\_\_\_\_\_\_\_\_\_\_\_\_\_\_\_\_\_\_\_\_\_\_\_\_\_\_\_\_\_\_\_\_\_\_\_\_\_\_\_\_\_\_\_\_\_\_\_\_\_\_\_\_\_\_\_\_\_\_\_\_\_\_\_\_\_\_\_\_\_\_\_\_\_\_\_\_\_\_\_\_\_\_\_\_\_\_\_\_\_\_\_\_\_\_\_\_\_\_\_\_\_\_\_\_\_\_\_\_\_\_\_\_\_\_\_\_\_\_\_\_\_\_\_\_\_\_\_\_\_\_\_\_\_\_\_\_\_\_\_\_\_\_\_\_\_\_\_\_\_\_\_\_\_\_\_\_\_\_\_\_\_\_\_\_\_\_\_\_\_\_\_\_\_\_\_\_\_\_\_\_\_\_\_\_\_\_\_\_\_\_\_\_\_\_\_\_\_\_\_\_\_\_\_\_\_\_\_\_\_\_\_\_\_\_\_\_\_\_\_\_\_\_\_\_\_\_\_\_\_\_\_\_\_\_\_\_\_\_\_\_\_\_\_\_\_\_\_\_\_\_\_\_\_\_\_\_\_\_\_\_\_\_\_\_\_\_\_\_\_\_\_\_\_\_\_\_\_\_\_\_\_\_\_\_\_\_\_\_\_\_\_\_\_\_\_\_\_\_\_\_\_\_\_\_\_\_\_\_\_\_\_\_\_\_\_\_\_\_\_\_\_\_\_\_\_\_\_\_\_\_\_\_\_\_\_\_\_\_\_\_\_\_\_\_\_\_\_\_\_\_\_\_\_\_\_\_\_\_\_\_\_\_\_\_\_\_\_\_\_\_\_\_\_\_\_\_\_\_\_\_\_\_\_\_\_\_\_\_\_\_\_\_\_\_\_\_\_\_\_\_\_\_\_\_\_\_\_\_\_\_\_\_\_\_\_\_\_\_\_\_\_\_\_\_\_\_\_\_\_\_\_\_\_\_\_\_\_\_\_\_\_\_\_\_\_\_\_\_\_\_\_\_\_\_\_\_\_\_\_\_\_\_\_\_\_\_\_\_\_\_\_\_\_\_\_\_\_\_\_\_\_\_\_\_\_\_\_\_\_\_\_\_\_\_\_\_\_\_\_\_\_\_\_\_\_\_\_\_\_\_\_\_\_\_\_\_\_\_\_\_\_\_\_\_\_\_\_\_\_\_\_\_\_\_\_\_\_\_\_\_\_\_\_\_\_\_\_\_\_\_\_\_\_\_\_\_\_\_\_\_\_\_\_\_\_\_\_\_\_\_\_\_\_\_\_\_\_\_\_\_\_\_\_\_\_\_\_\_\_\_\_\_\_\_\_\_\_\_\_\_\_\_\_\_\_\_\_\_\_\_\_\_\_\_\_\_\_\_\_\_\_\_\_\_\_\_\_\_\_\_\_\_\_\_\_\_\_\_\_\_\_\_\_\_\_\_\_\_\_\_\_\_\_\_\_\_\_\_\_\_\_\_\_\_\_\_\_\_\_\_\_\_\_\_\_\_\_\_\_\_\_\_\_\_\_\_\_\_\_\_\_\_\_\_\_\_\_\_\_\_\_\_\_\_\_\_\_\_\_\_\_\_\_\_\_\_\_\_\_\_\_\_\_\_\_\_\_\_\_\_\_\_\_\_\_\_\_\_\_\_\_\_\_\_\_\_\_\_\_\_\_\_\_\_\_\_\_\_\_\_\_\_\_\_\_\_\_\_\_\_\_\_\_\_\_\_\_\_\_\_\_\_\_\_\_\_\_\_\_\_\_\_\_\_\_\_\_\_\_\_\_\_\_\_\_\_\_\_\_\_\_\_\_\_\_\_\_\_\_\_\_\_\_\_\_\_\_\_\_\_\_\_\_\_\_\_\_\_\_\_\_\_\_\_\_\_\_\_\_\_\_\_\_\_\_\_\_\_\_\_\_\_\_\_\_\_\_\_\_\_\_\_\_\_\_\_\_\_\_\_\_\_\_\_\_\_\_\_\_\_\_\_\_\_\_\_\_\_\_\_\_\_\_\_\_\_\_\_\_\_\_\_\_\_\_\_\_\_\_\_\_\_\_\_\_\_\_\_\_\_\_\_\_\_\_\_\_\_\_\_\_\_\_\_\_\_\_\_\_\_\_\_\_\_\_\_\_\_\_\_\_\_\_\_\_\_\_\_\_\_\_\_\_\_\_\_\_\_\_\_\_\_\_\_\_\_\_\_\_\_\_\_\_\_\_\_\_\_\_\_\_\_\_\_\_\_\_\_\_\_\_\_\_\_\_\_\_\_\_\_\_\_\_\_\_\_\_\_\_\_\_\_\_\_\_\_\_\_\_\_\_\_\_\_\_\_\_\_\_\_\_\_\_\_\_\_\_\_\_\_\_\_\_\_\_\_\_\_\_\_\_\_\_\_\_\_\_\_\_\_\_\_\_\_\_\_\_\_\_\_\_\_\_\_\_\_\_\_\_\_\_\_\_\_\_\_\_\_\_\_\_\_\_\_\_\_\_\_\_\_\_\_\_\_\_\_\_\_\_\_\_\_\_\_\_\_\_\_\_\_\_\_\_\_\_\_\_\_\_\_\_\_\_\_\_\_\_\_\_\_\_\_\_\_\_\_\_\_\_\_\_\_\_\_\_\_\_\_\_\_\_\_\_\_\_\_\_\_\_\_\_\_\_\_\_\_\_\_\_\_\_\_\_\_\_\_\_\_\_\_\_\_\_\_\_\_\_\_\_\_\_\_\_\_\_\_\_\_\_\_\_\_\_\_\_\_\_\_\_\_\_\_\_\_\_\_\_\_\_\_\_\_\_\_\_\_\_\_\_\_\_\_\_\_\_\_\_\_\_\_\_\_\_\_\_\_\_\_\_\_\_\_\_\_\_\_\_\_\_\_\_\_\_\_\_\_\_\_\_\_\_\_\_\_\_\_\_\_\_\_\_\_\_\_\_\_\_\_\_\_\_\_\_\_\_\_\_\_\_\_\_\_\_\_\_\_\_\_\_\_\_\_\_\_\_\_\_\_\_\_\_\_\_\_\_\_\_\_\_\_\_\_\_\_\_\_\_\_\_\_\_\_\_\_\_\_\_\_\_\_\_\_\_\_\_\_\_\_\_\_\_\_\_\_\_\_\_\_\_\_\_\_\_\_\_\_\_\_\_\_\_\_\_\_\_\_\_\_\_\_\_\_\_\_\_\_\_\_\_\_\_\_\_\_\_\_\_\_\_\_\_\_\_\_\_\_\_\_\_\_\_\_\_\_\_\_\_\_\_\_\_\_\_\_\_\_\_\_\_\_\_\_\_\_\_\_\_\_\_\_\_\_\_\_\_\_\_\_\_\_\_\_\_\_\_\_\_\_\_\_\_\_\_\_\_\_\_\_\_\_\_\_\_\_\_\_\_\_\_\_\_\_\_\_\_\_\_\_\_\_\_\_\_\_\_\_\_\_\_\_\_\_\_\_\_\_\_\_\_\_\_\_\_\_\_\_\_\_\_\_\_\_\_\_\_\_\_\_\_\_\_\_\_\_\_\_\_\_\_\_\_\_\_\_\_\_\_\_\_\_\_\_\_\_\_\_\_\_\_\_\_\_\_\_\_\_\_\_\_\_\_\_\_\_\_\_\_\_\_\_\_\_\_\_\_\_\_\_\_\_\_\_\_\_\_\_\_\_\_\_\_\_\_\_\_\_\_\_\_\_\_\_\_\_\_\_\_\_\_\_\_\_\_\_\_\_\_\_\_\_\_\_\_\_\_\_\_\_\_\_\_\_\_\_\_\_\_\_\_\_\_\_\_\_\_\_\_\_\_\_\_\_\_\_\_\_\_\_\_\_\_\_\_\_\_\_\_\_\_\_\_\_\_\_\_\_\_\_\_\_\_\_\_\_\_\_\_\_\_\_\_\_\_\_\_\_\_\_\_\_\_\_\_\_\_\_\_\_\_\_\_\_\_\_\_\_\_\_\_\_\_\_\_\_\_\_\_\_\_\_\_\_\_\_\_\_\_\_\_\_\_\_\_\_\_\_\_\_\_\_\_\_\_\_\_\_\_\_\_\_\_\_\_\_\_\_\_\_\_\_\_\_\_\_\_\_\_\_\_\_\_\_\_\_\_\_\_\_\_\_\_\_\_\_\_\_\_\_\_\_\_\_\_\_\_\_\_\_\_\_\_\_\_\_\_\_\_\_\_\_\_\_\_\_\_\_\_\_\_\_\_\_\_\_\_\_\_\_\_\_\_\_\_\_\_\_\_\_\_\_\_\_\_\_\_\_\_\_\_\_\_\_\_\_\_\_\_\_\_\_\_\_\_\_\_\_\_\_\_\_\_\_\_\_\_\_\_\_\_\_\_\_\_\_\_\_\_\_\_\_\_\_\_\_\_\_\_\_\_\_\_\_\_\_\_\_\_\_\_\_\_\_\_\_\_\_\_\_\_\_\_\_\_\_\_\_\_\_\_\_\_\_\_\_\_\_\_\_\_\_\_\_\_\_\_\_\_\_\_\_\_\_\_\_\_\_\_\_\_\_\_\_\_\_\_\_\_\_\_\_\_\_\_\_\_\_\_\_\_\_\_\_\_\_\_\_\_\_\_\_\_\_\_\_\_\_\_\_\_\_\_\_\_\_\_\_\_\_\_\_\_\_\_\_\_\_\_\_\_\_\_\_\_\_\_\_\_\_\_\_\_\_\_\_\_\_\_\_\_\_\_\_\_\_\_\_\_\_\_\_\_\_\_\_\_\_\_\_\_\_\_\_\_\_\_\_\_\_\_\_\_\_\_\_\_\_\_\_\_\_\_\_\_\_\_\_\_\_\_\_\_\_\_\_\_\_\_\_\_\_\_\_\_\_\_\_\_\_\_\_\_\_\_\_\_\_\_\_\_\_\_\_\_\_\_\_\_\_\_\_\_\_\_\_\_\_\_\_\_\_\_\_\_\_\_\_\_\_\_\_\_\_\_\_\_\_\_\_\_\_\_\_\_\_\_\_\_\_\_\_\_\_\_\_\_\_\_\_\_\_\_\_\_\_\_\_\_\_\_\_\_\_\_\_\_\_\_\_\_\_\_\_\_\_\_\_\_\_\_\_\_\_\_\_\_\_\_\_\_\_\_\_\_\_\_\_\_\_\_\_\_\_\_\_\_\_\_\_\_\_\_\_\_\_\_\_\_\_\_\_\_\_\_\_\_\_\_\_\_\_\_\_\_\_\_\_\_\_\_\_\_\_\_\_\_\_\_\_\_\_\_\_\_\_\_\_\_\_\_\_\_\_\_\_\_\_\_\_\_\_\_\_\_\_\_\_\_\_\_\_\_\_\_\_\_\_\_\_\_\_\_\_\_\_\_\_\_\_\_\_\_\_\_\_\_\_\_\_\_\_\_\_\_\_\_\_\_\_\_\_\_\_\_\_\_\_\_\_\_\_\_\_\_\_\_\_\_\_\_\_\_\_\_\_\_\_\_\_\_\_\_\_\_\_\_\_\_\_\_\_\_\_\_\_\_\_\_\_\_\_\_\_\_\_\_\_\_\_\_\_\_\_\_\_\_\_\_\_\_\_\_\_\_\_\_\_\_\_\_\_\_\_\_\_\_\_\_\_\_\_\_\_\_\_\_\_\_\_\_\_\_\_\_\_\_\_\_\_\_\_\_\_\_\_\_\_\_\_\_\_\_\_\_\_\_\_\_\_\_\_\_\_\_\_\_\_\_\_\_\_\_\_\_\_\_\_\_\_\_\_\_\_\_\_\_\_\_\_\_\_\_\_\_\_\_\_\_\_\_\_\_\_\_\_\_\_\_\_\_\_\_\_\_\_\_\_\_\_\_\_\_\_\_\_\_\_\_\_\_\_\_\_\_\_\_\_\_\_\_\_\_\_\_\_\_\_\_\_\_\_\_\_\_\_\_\_\_\_\_\_\_\_\_\_\_\_\_\_\_\_\_\_\_\_\_\_\_\_\_\_\_\_\_\_\_\_\_\_\_\_\_\_\_\_\_\_\_\_\_\_\_\_\_\_\_\_\_\_\_\_\_\_\_\_\_\_\_\_\_\_\_\_\_\_\_\_\_\_\_\_\_\_\_\_\_\_\_\_\_\_\_\_\_\_\_\_\_\_\_\_\_\_\_\_\_\_\_\_\_\_\_\_\_\_\_\_\_\_\_\_\_\_\_\_\_\_\_\_\_\_\_\_\_\_\_\_\_\_\_\_\_\_\_\_\_\_\_\_\_\_\_\_\_\_\_\_\_\_\_\_\_\_\_\_\_\_\_\_\_\_\_\_\_\_\_\_\_\_\_\_\_\_\_\_\_\_\_\_\_\_\_\_\_\_\_\_\_\_\_\_\_\_\_\_\_\_\_\_\_\_\_\_\_\_\_\_\_\_\_\_\_\_\_\_\_\_\_\_\_\_\_\_\_\_\_\_\_\_\_\_\_\_\_\_\_\_\_\_\_\_\_\_\_\_\_\_\_\_\_\_\_\_\_\_\_\_\_\_\_\_\_\_\_\_\_\_\_\_\_\_\_\_\_\_\_\_\_\_\_\_\_\_\_\_\_\_\_\_\_\_\_\_\_\_\_\_\_\_\_\_\_\_\_\_\_\_\_\_\_\_\_\_\_\_\_\_\_\_\_\_\_\_\_\_\_\_\_\_\_\_\_\_\_\_\_\_\_\_\_\_\_\_\_\_\_\_\_\_\_\_\_\_\_\_\_\_\_\_\_\_\_\_\_\_\_\_\_\_\_\_\_\_\_\_\_\_\_\_\_\_\_\_\_\_\_\_\_\_\_\_\_\_\_\_\_\_\_\_\_\_\_\_\_\_\_\_\_\_\_\_\_\_\_\_\_\_\_\_\_\_\_\_\_\_\_\_\_\

\begin{center}
\includegraphics[width=0.3\textwidth]{image1.png} \quad
\includegraphics[width=0.3\textwidth]{image2.png} \quad
\includegraphics[width=0.3\textwidth]{image3.png}
\end{center}

\begin{center}
\textbf{\Large 第十届华为杯全国研究生数学建模竞赛}
\end{center}

\begin{center}
\textbf{题目} \quad 中国城乡居民养老保险可持续发展体系研究
\end{center}

\section*{摘要:}

本文以我国现阶段的城乡社会养老保障体系为依托,建立了相关数学模型,并对养老金的缺口问题进行了深入研究,提出了一套适合我国国情的养老金制度调整方案,以保证其未来发展的可持续性。

首先,基于保险精算原理,测算我国未来几年各个年龄段的男女人口数量,并针对城镇和农村,分别建立了养老金的收入和支出模型。在城镇方面,从国家财政补贴层次、基本养老保险层次和企业年金层次全面分析我国现行的养老金制度,并对基本养老保险层次进行了重点探索,不仅从“老人”和“中人、新人”的角度分别提出了相应的收支模型,还预测了 2012—2035 年我国城镇养老保险基金收支总额,明确了其未来的发展趋势。而在企业年金给付模型中,建立了待遇确定型模型和缴费确定型模型。在农村方面,对不同年龄段及不同缴费水平下的养老金替代率进行了详尽的测算,并从两个维度体现了国家“多缴多得,长缴多得”的政策。

其次,以本文测算的养老金收支情况为依据,对养老金缺口进行预测。在城镇方面,除了考虑收支差异,还引入了转轨成本,使模型更符合中国国情。最终测算得出:城镇的养老金缺口从 2012—2018 年呈现递减趋势,2019—2035 年以年均 40.67% 的速度递增,到 2035 年缺口已经达到 772553 亿元。而在农村方面,到 2035 年,养老金缺口将达到 639842.55 亿元。

此外,本文以城镇的养老金制度为例,运用 R 软件对不同模型参数下的养老金缺口进行了模拟,得出了能够保证我国养老保险体系可持续性的男性替代率区间为 (49.48\%, 56.48\%),女性替代率区间为 (40.02\%, 45.02\%),缴费率区间为 (41.84\%, 44.34\%)。当国家偿付危机即将来临之前适当地降低男性替代率,女性替代率和工资调整系数,适时地提高一定比例的缴费率,可使得国家平稳度过危险期。

\section*{关键词:养老金 收支模型 缺口 转轨成本 仿真模拟}

\section*{中国城乡居民养老保险可持续发展体系研究}

\section*{1. 问题重述}

近几年来,社会对于养老问题的关注力度大幅提升,尤其是在两会上,“养老问题”成为一大热点,引发了全民的热议。在党的十八次我国代表大会报告中,更是将社会保障纳入其中,强调“要坚持全覆盖、保基本、多层次、可持续方针,以增强公平性、适应流动性、保证可持续性为重点,全面建成覆盖城乡居民的社会保障体系”。

然而,由于我国人口基础大,地区差异以及城乡差异明显,从而使得养老保险体系的建设面临着巨大的挑战。随着养老金新制度的实施,养老基金缺口逐年增大,由此引发了各方学者的全面研究。

针对此现状,我们需要解决以下几个问题。

首先,建立适合我国国情的城乡居民(含新农保)养老金收入、支出的宏观数学模型,至少包含替代率、缴费率、人口结构、分年龄段死亡率、经济增速、财政补贴、工资水平或物价指数、投资效益等主要因素,并且分多个层次体现“多缴多得,长缴多得”。

其次,估计今年至 2035 年我国的养老金缺口,并说明养老金缺口分析的合理性,同时根据模型,指出我国城乡居民养老金保险收支矛盾最尖锐的时间及其严重程度。在此基础上,考虑十八大的收入倍增计划,并对模型需要调整的部分给予说明。

第三,分析各国养老保险的不同模式,并利用仿真手段,寻找替代率和缴费率的合理区间,从而保证我国养老保险体系的可持续性。与此同时,指出在步入良性循环之前,应采取哪些政策措施以平稳度过矛盾最尖锐的时期,并用仿真方法预测相关措施的效果。

第四,在第三问的基础上增加可调节的变量,尝试建立相关模型。

\section*{2. 基本假设}

(1)假设城镇地区,男性职工 60 岁退休,女性职工 55 岁退休,在未来的几年内,此退休年龄保持不变;农村地区,无论男女,都从 60 周岁开始领取养老金。

(2)假设新农村地区的参保农民在确定自己的养老金缴费标准后,未来几年均保持不变。

(3)假设我国人口死亡率以中国人寿保险业的经验生命表(2000—2003)为依据,并在未来的几年内不发生变动。

(4)假设妇女的适龄生育年龄为 15 岁至 49 周岁,以第六次人口普查的数据所计算的 15 至 49 周岁之间各个年龄孕龄妇女的平均生育率作为标准,并且在未来 25 年间保持不变。

\section*{3. 符号说明}

\begin{tabular}{|c|c|}
\hline $\mathbf{W}_{t, y}^{f}$ & $t$ 年 $y$ 岁城镇女性人口的数量 \\
\hline $\mathbf{W}_{t, y}^{\mathrm{m}}$ & $t$ 年 $y$ 岁城镇男性人口的数量 \\
\hline $(L P)_{t}$ & $t$ 年的城镇适龄劳动人口 \\
\hline
\end{tabular}

\begin{table}
\centering
\begin{tabular}{|c|c|}
\hline
$(TI)_{t}$ & $t$年的城镇养老保险总收入 \\
\hline
$(AW)_{t}$ & $t$年的城镇在岗职工的平均工资 \\
\hline
$Z_{t}$ & $t$农村养老金缺口 \\
\hline
$I_{t}$ & $t$年的农村养老金总收入 \\
\hline
$Q_{t}$ & $t$年的农村养老金总支出 \\
\hline
\end{tabular}
\end{table}

备注:其他符号详见文中具体说明

\section*{4. 问题分析}

四大问题层层相扣,其目的是建立我国城乡社会养老保险的数学模型,并以此为依据进行合理的计算与分析,最终实现可持续发展的目标。

第一题,要求建立中国城乡居民(含新农保)养老金收入、支出的宏观数学模型,并体现“多缴多得,长缴多得”的政策。对于此题,我们将城镇和农村分开来考虑,建立各自的收支模型。其中,城镇部分,根据题目,建立以“基本养老保险”为第一支柱,“企业年金基金”为第二支柱,“自愿的个人储蓄和商业保险”为第三支柱的“三支柱养老保障体系”,并着重对前两部分进行详细分析。新农村部分,结合国家政策,对不同缴费档次的情况进行计算,比较不同缴费情况下所能获得的养老金数额的差别,并对每年度国家整体的养老金收支情况进行分析。

第二题,要求计算养老金缺口,并对缺口的合理性进行分析,同时估计我国城乡居民养老保险收支最矛盾的时刻以及其严重程度。其次,考虑党的十八大收入倍增计划,提出模型需要调整的方面。针对此题,我们以第一题中的城乡收支模型为依据,首先对“缺口”进行定义,其次计算缺口值。在城镇方面,除了考虑收支差距外,引入“转轨成本”,使得模型更符合现实,而对于乡村而言,主要考虑收支之差。在此基础上,考虑收入倍增计划,对模型中的参数进行调整。

第三题,要求比较各国养老保险模式的不同,并取其精华,去其糟粕,并用仿真技术寻找替代率和缴费率的合理区间,目的是保证我国养老保险的可持续发展。题中,可持续发展可以理解为将第二题中的养老金缺口控制在一个稳定的范围内。通过模拟缴费率和替代率的变动情况,分析其对养老金缺口产生的影响,并对此缺口进行限定,则可完成题中设定的目标。同时,结合现实情况,对模型系数进行调整,实现矛盾最尖锐期的安全度过。

第四题,要求尝试增加可调节的变量,并建立相应的数学模型。此题主要在第三题的基础上进行分析,由于第三问中只考虑了替代率和缴费率的变动,故在第四问中可以增加对工资调整系数等可变因素的考虑,从而完善模型。

\section*{5. 模型的建立及求解}

\subsection*{5.1 我国城镇居民养老金收支模型}

现阶段,我国已经进入典型的老龄化社会。根据 2010 年我国第六次人口普查数据,我国 60 岁以上老年人口占我国总人口的 13.3\%,65 岁以上人口占我国总人口的 8.9\%,

这两个比例均超过国际公认的老龄化社会所界定标准。在诸多社会老龄化所带来的问题当中,首当其中的当属老年人的养老问题。因此,建立一套完善的社会养老保障体系,让城乡居民平等的享受国家经发展的成果成为一个亟需解决的问题。党的十七大报告指出,国家需要加快建立“广覆盖、保基础、多层次、可持续”的覆盖城乡居民的社会保障体系。其中,如何建立一个“多层次”的社会保障体系成为重点以及难点。

\subsection*{5.1.1 我国现行的城镇居民养老保险体系}

关于多层次的养老保险问题,世界银行曾于 1991 年提出三支柱养老保险体系。并且经过多年的实践,2005 年底世界银行扩展了三支柱的思想,提出了五支柱的概念和建议,即:提供最低水平保障的非缴费型“零支柱”;与本人收入水平挂钩的缴费型“第一支柱”;不同形式的个人储蓄账户性质的强制性“第二支柱”;灵活多样的雇主发起的自愿性“第三支柱”;建立家庭成员之间或代际之间非正规保障形式的所谓“第四支柱”。对其原有的三支柱进行了补充和扩展。

就我国而言,目前我国建立了三支柱的社会养老保险制度:

(一)第一支柱是社会统筹和个人账户相结合的基本养老保险

从 20 世纪 80 年代开始,我国政府在进行试点和总结实践经验的基础上,建立了社会统筹和个人账户相结合的基本养老保险。基本养老保险是国家强制执行的,适用于所有企业和劳动者,由企业和个人共同缴费形成基本养老保险基金。其中社会统筹基金由企业缴费的一部分形成,职工个人账户基金由个人缴费的全部和企业缴费的一部分形成。

(二)第二支柱是近年来逐步明确的企业年金基金

部分企业可以根据自身能力,为本企业职工建立企业补充养老保险,职工个人自愿建立储蓄性养老保险。企业为每个参保职工建立个人账户,基金实行完全积累,实账运行。

(三)我国城镇养老的第三支柱主要包括自愿的个人储蓄和个人购买商业养老保险,一般属于个人行为,通常不列在社会养老保险计算范围内。

以上三个支柱在理论上将一个养老保险制度应有的再分配功能、储蓄功能与保险功能有机地结合在一个共同的养老保险制度之中。

在第一支柱的社会基本养老保险之前,存在一个所谓的“零支柱”,即国家财政补贴。它是政府所提供的普惠式国民年金,目的是向所有老年人提供最基本收入保障,经费来源与税收,老年人无需缴费,待遇与工资无关,但与物价挂钩,并随着社会经济发展水平变化而变化,确保老年人享受社会进步文明成果,体现社会保障制度的公平性。

\subsection*{5.1.2 我国城镇居民养老金收入——国家财政补贴层次}

上文中已经说到,国家对于养老基金的补贴目的是为了保障老年人最基本的生活要求,国家每年会从财政收入中拿出一部分发放给老人。我们注意到这部分补贴与国内生产总值具有一个较为稳定的比例关系。因此,为了测算这一部分的养老保险收入 \( IN_{g} \),本文将会利用时间序列方法预测未来各年 \( GDP_{t} \),并根据历史数据,测算财政补贴所占 GDP 比例 \( (r)_{t} \),则测算模型为:

\begin{equation}
IN_{g} = GDP_{t} \times r(t)
\tag{1}
\end{equation}

利用 1978 年至今的 GDP 数据,本文建立了时间序列回归模型:

\begin{table}[h]
\centering
\caption{财政补贴预测值}
\begin{tabular}{|c|c|c|c|}
\hline
年份 & 财政补贴 & 年份 & 财政补贴 \\
\hline
2012 & 2328.75 & 2024 & 7187.29 \\
\hline
2013 & 2600.27 & 2025 & 7771.35 \\
\hline
2014 & 2893.28 & 2026 & 8386.86 \\
\hline
2015 & 3208.60 & 2027 & 9034.63 \\
\hline
2016 & 3547.05 & 2028 & 9715.50 \\
\hline
2017 & 3909.47 & 2029 & 10430.31 \\
\hline
2018 & 4296.69 & 2030 & 11179.88 \\
\hline
2019 & 4709.54 & 2031 & 11965.04 \\
\hline
2020 & 5148.85 & 2032 & 12786.62 \\
\hline
2021 & 5615.45 & 2033 & 13645.46 \\
\hline
2022 & 6110.17 & 2034 & 14542.37 \\
\hline
2023 & 6633.84 & 2035 & 15478.21 \\
\hline
\end{tabular}
\end{table}

由表1中结果可以看到,国家对于养老保险基金的财政补贴逐年增加。到2035年,这一数额将达到15478亿元,接近2012年水平的7倍之多,年均增长约为571亿元。

国家对于社会养老保险基金补贴的增长对于提高全社会的养老保险水平无疑是有利的,但是需要知道,国家的财政补贴在我国社会养老保险基金的收入中只占到一个很小的比例,因此主要依靠国家的财政补贴来提高养老保险水平是不可能的。因此,国家需要建立“多层次”的社会养老保障体系。

\subsection*{5.1.3 我国城镇居民养老金收入——基本养老保险层次}

\subsubsection{5.1.3.1 我国城镇养老保险基础人口测算}

(1) 我国城镇人口变动的影响因素

一个国家和地区的人口总数始终处于变动之中,这种变化受很多因素的影响,包括自然因素和人为因素,其中国家的政策对于人口变动影响巨大。在新中国建立初期,我国人口在4亿左右,但经历60年代末人口的剧烈膨胀之后。我国人口超过10亿。受着资源的制约以及出于可持续发展的战略的考虑,我国实行了计划生育政策。随着这一政策的持续效用,人口膨胀这一趋势得到遏制。因此,可以看到一个国家和地区的人口始终处于变化之中,如何准确测算未来的人口数目具有相当大的难度,但这也是测算养老保险收入的基础。

在影响人口的诸多因素中,出生率、死亡率以及人口迁移率的作用尤为突出。出生5

率是指一定时期内(通常为一年)出生人数与同期平均人数之比。影响出生率的包括生育政策、风俗习惯和经济状况等因素。在出生率保持稳定不变的情况下,该地区的新生人口数量主要和该地区的育龄妇女的数量以及年龄结构有关系。死亡率是指一定时期内(通常为一年)死亡人数和同期平均人数之比。影响死亡率的主要因素包括年龄、性别/医疗水平和生活质量。迁移率是指一定时期内(通常是一年)迁移人数和同期平均人数之比,分为迁出率和迁入率,二者之差为净迁入率。目前我国处在城镇化的快速发展时期,因此人口迁移对于城镇人口的影响十分巨大。同时,不同年龄段人口的迁移率差别较大,需要区别对待。

(2) 我国城镇人口测算模型

我国城镇养老金收入测算需要以准确的人口测算作为基础,如果人口测算的偏差过大,那么养老金收支模型准确性会大打折扣。因此如何测算未来中国的城镇人口是模型建立的重点以及难点所在。考虑到新生人口主要是由当年适龄孕妇的结构和生育率影响,而 0 岁以上人口数主要是由该年龄段的死亡率所决定,因此本文将建立不同的模型来测算 0 岁和 0 岁以上的人口数目。

(a) 我国城镇人口新生人口测算模型 \footnote{1}

一个地区每年的新生人口数目主要受该地区育龄妇女的数量和年龄结构影响。如果孕龄妇女在人口比重大,那么相对来说该地区的新生人口会多。同时,如果一个地区的孕龄妇女中,生育率较高的年龄层次的妇女数量多,那么该地区的新生人口数目也会较多。$\mathbf{W}_{t,y}^{f}$ 表示在 $t$ 年 $y$ 岁妇女的数量,$f_{t,y}$ 表示 $t$ 年 $y$ 岁妇女的生育率,则在 $t$ 年 $y$ 岁妇女生育的新生婴儿的人数 $\mathbf{B}_{t,x}$ 为:

\begin{equation}
\mathbf{B}_{t,x} = \mathbf{W}_{t,y}^{f} \times f_{t,y}
\tag{3}
\end{equation}

由于育龄妇女的生育年龄在 15 至 49 岁之间,因此在 $t$ 年新生人口总和为各年龄育龄妇女所生的婴儿数目总和,即 $\mathbf{B}_{t}$ 为:

\begin{equation}
\mathbf{W}_{t,0} = \mathbf{B}_{t} = \sum_{x=15}^{49} B_{t,x} = \sum_{X=15}^{49} W_{t,y}^{f} \times f_{t,y}
\tag{4}
\end{equation}

同时为了考虑未来人口中男女比例,我们需要测算新生婴儿中的男女新生儿的数量。因此,设 $t$ 年男孩在新生儿中的比例为 $\mathbf{r}_{t}^{m}$,则男性和女性新生儿的数量 $\mathbf{W}_{t,0}^{\mathrm{m}}$ 和 $\mathbf{W}_{t,0}^{f}$ 分别为:

\begin{equation}
\mathbf{W}_{t,0}^{\mathrm{m}} = \mathbf{W}_{t,0} \times \mathbf{r}_{t}^{m}
\tag{5}
\end{equation}

\begin{equation}
\mathbf{W}_{t,0}^{f} = \mathbf{W}_{t,0} \times (1 - \mathbf{r}_{t}^{m})
\tag{6}
\end{equation}

由上述模型即可测算出 $t$ 年新生男性和女性的人口数量。

(b) 我国城镇非 0 岁人口测算模型 \footnote{1}

0 岁以上的人口变动主要受死亡和迁移这两方面因素的影响。假设 $t$ 年 $y$ 岁人口的净迁入率是 $(\mathrm{NIR})_{t,y}$,$t$ 年 $y$ 岁男性和女性的死亡率分别为 $q_{t,y}^{m}$ 和 $q_{t,y}^{f}$,则 $t+1$ 年 $y+1$ 岁

的男性和女性人口数 $\mathbf{W}_{t+1, y+1}^{\mathrm{m}}$ 和 $\mathbf{W}_{t+1, y+1}^{f}$ 是 $t$ 年 $y$ 岁人口存活一年后的人数加上净迁入人数,该数学模型为:
\begin{equation}
\mathbf{W}_{t+1, y+1}^{\mathrm{m}}=\mathbf{W}_{t, y}^{\mathrm{m}} \times\left(1-q_{t, y}^{m}\right)+\mathbf{W}_{t, y}^{\mathrm{m}} \times\left(1-q_{t, y}^{m}\right) \times(\mathrm{NIR})_{t, y}=\mathbf{W}_{t, y}^{\mathrm{m}} \times\left(1-q_{t, y}^{m}\right) \times\left[1+(\mathrm{NIR})_{t, y}\right]
\tag{7}
\end{equation}
\begin{equation}
\mathbf{W}_{t+1, y+1}^{f}=\mathbf{W}_{t, y}^{f} \times\left(1-q_{t, y}^{f}\right)+\mathbf{W}_{t, y}^{f} \times\left(1-q_{t, y}^{f}\right) \times(\mathrm{NIR})_{t, y}=\mathbf{W}_{t, y}^{f} \times\left(1-q_{t, y}^{f}\right) \times\left[1+(\mathrm{NIR})_{t, y}\right]
\tag{8}
\end{equation}
由此模型可以测算出每年我国城镇 0 岁以上男性和女性的人口数量。

(3)我国城镇人口测算基础数据及精算假设

根据上述人口测算模型,所需要的指标包括期初人口、生育率、死亡率、迁移率等基础数据,由于一些数据难以精确获得,需要对其中某些数据进行合适的假设。

(a)我国城镇人口测算基础数据

测算现在到 2035 年的人口数据需要起始人口数据,本文选取了国家在 2010 年开展的第六次人口普查方案资料作为基础数据。其中包含了各年龄段人口数目、新生人口、生育等相关重要的基础数据。

(b)我国城镇人口生育率假设

妇女的适龄生育年龄为 15 岁至 49 周岁,2010 年我国第六次人口普查的数据列出了在 15 至 49 周岁之间各个年龄孕龄妇女的平均生育率。

\begin{figure}[h]
\centering
\includegraphics[width=\textwidth]{image.png}
\caption{生育率分布}
\end{figure}

由生育率的分布来看,当前的高生育率人口主要集中于 20 至 31 岁。其中 22 至 26 岁生育率最高,基本维持在 $50\%$ 以上。

(c)我国城镇人口死亡率假设

不同年龄的人群具有不同的死亡率。死亡率主要受一个国家的物质生活水平和社会医疗条件的影响,因此本文假设在未来 25 年间我国的物质生活水平不发生根本的变以及社会医疗条件没有取得重大突破,即假设未来 25 年间我国城镇人口死亡率保持不变。

根据 2005 中国保监会所公布的最新经验生命表,可以得到我国处于不同年龄和性别的城镇人口死亡率。

\begin{figure}[h]
    \centering
    \includegraphics[width=\textwidth]{image1.png}
    \caption{死亡率分布}
    \label{fig:death_rate}
\end{figure}

可以看到,在70岁之前,男性和女性的死亡率都很低,并且上升及其缓慢,但超过70岁之后,死亡率上升加快,同时,男性略高于女性。

\subsubsection{(d) 我国城镇人口迁移率假设}

城镇人口的变动不仅仅受生育和死亡的影响,人口迁移同样影响城镇人口。随着我国近几十年来经济的快速发展,我国的城镇化水平也不断提高,目前我国的城镇化水平已经达到52\%。在过去五年,我国的城镇化率平均每年提高1.3个百分点。这表明,我国城镇居民始终具有一个正的净迁移率。根据普查资料,我们可以推算出各年龄的净迁入率如下图所示:

\begin{figure}[h]
    \centering
    \includegraphics[width=\textwidth]{image2.png}
    \caption{迁入率分布}
    \label{fig:migration_rate}
\end{figure}

由上图可以看到,不同年龄的净迁入率差距较大,其中10岁以下和16至42岁人口的净迁入率较高。16岁至42岁人口净迁入率较高是因为该年龄段的农村人口更愿意去城市里求学工作。而10岁以下的孩子由于需要父母照顾,可能随同父母一起外出。

\subsubsection{(4) 我国城镇人口测算结果}

根据文中所建立的测算模型以及基本假设和数据,可以得到从2011年到2035年间我国城镇人口数量,结果如下表:

\begin{table}[h]
    \centering
    \begin{tabular}{c|c}
        \hline
        年份 & 城镇人口数量(万人) \\
        \hline
        2011 &  \\
        2012 &  \\
        2013 &  \\
        2014 &  \\
        2015 &  \\
        2016 &  \\
        2017 &  \\
        2018 &  \\
        2019 &  \\
        2020 &  \\
        2021 &  \\
        2022 &  \\
        2023 &  \\
        2024 &  \\
        2025 &  \\
        2026 &  \\
        2027 &  \\
        2028 &  \\
        2029 &  \\
        2030 &  \\
        2031 &  \\
        2032 &  \\
        2033 &  \\
        2034 &  \\
        2035 &  \\
        \hline
    \end{tabular}
    \caption{我国城镇人口数量预测}
    \label{tab:urban_population}
\end{table}

\begin{table}
\centering
\begin{tabular}{|c|c|c|c|c|c|}
\hline
年份 & 男性(万人) & 女性(万人) & 年份 & 男性(万人) & 女性(万人) \\
\hline
2011 & 35423 & 33763 & 2024 & 50520 & 49614 \\
\hline
2012 & 36589 & 34874 & 2025 & 51752 & 50802 \\
\hline
2013 & 37798 & 36025 & 2026 & 52969 & 51972 \\
\hline
2014 & 39046 & 37211 & 2027 & 54169 & 53123 \\
\hline
2015 & 39046 & 38428 & 2028 & 55352 & 54255 \\
\hline
2016 & 40307 & 39666 & 2029 & 56520 & 55368 \\
\hline
2017 & 41587 & 40921 & 2030 & 57671 & 56462 \\
\hline
2018 & 42877 & 42184 & 2031 & 58810 & 57542 \\
\hline
2019 & 44170 & 43447 & 2032 & 59936 & 58608 \\
\hline
2020 & 45461 & 44706 & 2033 & 61053 & 59663 \\
\hline
2021 & 46742 & 45953 & 2034 & 62163 & 60709 \\
\hline
2022 & 48012 & 47186 & 2035 & 63263 & 61745 \\
\hline
2023 & 49273 & 48408 & & & \\
\hline
\end{tabular}
\caption{城镇人口预测值}
\end{table}

\subsection*{5.1.3.2 我国城镇养老保险收入测算模型}

目前,我国城镇养老保险基金实行社会统筹账户和和个人账户相结合的部分积累制模式。一部分由城镇职工所在企业按照缴费工资总额20\%缴纳,所交金额计入个人累计账户;二是由职工按照个人缴费工资的8\%缴纳的养老金,计入个人累积账户。那么我国城镇养老保险收入总额为企业和个人缴纳养老金的总和。那么,要想测算我国城镇养老经济收入,需要确定企业职工的平均工资以及缴费人数。

(1) 我国城镇养老基金缴费工资测算模型

由于企业和个人缴纳养老保险基金都是以职工的个人工资为基础,所以需要根据已有资料来预测未来我国城镇职工平均工资。这一部分将利用时间序列分析方法建立回归模型,预测2011年至2035年间我国城镇职工平均工资。

\begin{figure}[h]
\centering
\includegraphics[width=0.8\textwidth]{历年城镇职工工资趋势图}
\caption{历年城镇职工工资趋势图}
\end{figure}

\begin{table}[h]
\centering
\caption{城镇职工平均工资预测值}
\begin{tabular}{|c|c|c|c|}
\hline
年份 & 平均工资 & 年份 & 平均工资 \\
\hline
2011 & 42452.00 & 2024 & 137517.26 \\
\hline
2012 & 45953.37 & 2025 & 148433.23 \\
\hline
2013 & 51130.63 & 2026 & 159924.07 \\
\hline
2014 & 56704.34 & 2027 & 172004.64 \\
\hline
2015 & 62689.37 & 2028 & 184689.82 \\
\hline
2016 & 69100.59 & 2029 & 197994.47 \\
\hline
2017 & 75952.86 & 2030 & 211933.46 \\
\hline
2018 & 83261.06 & 2031 & 226521.66 \\
\hline
2019 & 91040.05 & 2032 & 241773.93 \\
\hline
2020 & 99304.70 & 2033 & 257705.15 \\
\hline
2021 & 108069.88 & 2034 & 274330.18 \\
\hline
2022 & 117350.45 & 2035 & 291663.89 \\
\hline
2023 & 127161.29 & & \\
\hline
\end{tabular}
\end{table}

(2) 我国城镇养老基金缴费人数测算模型

养老保险基金的缴纳人群是在岗职工。受劳动参与率以及参保率等影响,缴费人员是适龄劳动人口的一部分。适龄劳动人口是指年龄处于适合参加劳动的阶段。按照目前我国的规定,处在16至59周岁的男性以及16至54周岁的女性为适龄劳动人口,以 $(LP)_{t}$ 表示 $t$ 年的适龄劳动人口,$W_{t,y}^{m}$ 和 $W_{t,y}^{f}$ 分别表示 $t$ 年 $y$ 岁男性和女性人口数,则:

\begin{equation}
(LP)_{t} = \sum_{y=16}^{59} W_{t,y}^{m} + \sum_{y=16}^{54} W_{t,y}^{f}
\tag{10}
\end{equation}

然而,在适龄劳动人群中,并不是所有的人都去参加工作,一些人因为各种原因并没有参加工作,适龄劳动人口中参与工作的人群比例称之为劳动参与率,用

(RLFP)_{t} 表示;在参加工作的劳动人口中,有一部分因为某种原因而失业,用 (RUE)_{t} 表示失业率,则 \( t \) 年的在岗人口 \( (EP)_{t} \) 为:

\begin{equation}
(EP)_{t} = (LP)_{t} \times (RLFP)_{t} \times [1 - (RUE)_{t}]
\tag{11}
\end{equation}

\begin{equation}
= \left[ (RLFP)_{t}^{m} \sum_{y=16}^{59} W_{t,y}^{m} + (RLFP)_{t}^{f} \sum_{y=16}^{54} W_{t,y}^{f} \right] \times [1 - (RUE)_{t}]
\tag{12}
\end{equation}

目前,我国的就业人员中,包括个体户以及机关事业单位人员不需要参加养老保险,只有企业单位职工需要缴纳养老基金,令 \( (RCLP)_{t} \) 表示 \( t \) 年企业单位工作人口占就业总人口的比率,即企业劳动人口率;由于各种原因,存在部分企业职工没有参加养老保险情况,使得参保率小于 1,令 \( (RC)_{t} \) 表示 \( t \) 年企业劳动人口参保率,则 \( t \) 年参加养老保险的缴费人口 \( (TCL)_{t} \) 为:

\begin{equation}
(TCL)_{t} = (EP)_{t} \times (RCLP)_{t} \times (RC)_{t}
\tag{13}
\end{equation}

在人口测算、平均工资预期以及精算假设的基础上,利用该模型即可测算出每年我国城镇养老保险缴费人数。

(3) 我国城镇养老保险基金收入测算模型 \({}^{[2]}\)

养老保险的年收入是缴费人数、平均缴费工资、缴费率和收缴率的乘积。\( (RI)_{t} \) 表示 \( t \) 年养老保险的缴费率,\( (AW)_{t} \) 表示 \( t \) 年的城镇在岗职工的平均工资,\( (REI)_{t} \) 表示 \( t \) 年的收缴率,则 \( t \) 年的养老保险总收入 \( (TI)_{t} \) 为:

\begin{equation}
(TI)_{t} = (TCL)_{t} \times (AW)_{t-1} \times (RI)_{t} \times (REI)_{t}
\tag{14}
\end{equation}

利用该模型可测算出我国城镇养老保险收入

(4) 我国城镇养老基金收入测算基础数据及精算假设

在收入测算模型中,缴费人数 \( (TCL)_{t} \) 和缴费工资数 \( (AW)_{t-1} \) 可以采用上面模型的预测值;另外假设未来几年养老保险的缴费率不发生变动,企业和个人的合计缴费率 \( (RI)_{t} \) 为 28%。在养老保险的实际收缴过程中,部分企业由于经济困难,存在拖欠缴费的现象,另外还有部分企业存在少报缴费基数、漏报缴费人数等情况,根据实际情况,假设养老保险的收缴率 \( (REI)_{t} \) 为 65%。

在所有适龄劳动人口中,男性的劳动参与率高于女性,根据现有资料假设男性和女性的劳动参与率分别为 65% 和 60%。假设未来几年参加工作人员的失业率维持在 4% 的水平。在参加工作人口中,在政府等部门工作的行政人员和个体户并不缴纳养老金,只有

\begin{table}
\centering
\caption{养老保险测算结果}
\begin{tabular}{|c|c|c|c|}
\hline
年份 & 养老保险收入(亿元) & 年份 & 养老保险收入(亿元) \\
\hline
2011 & 16962.63 & 2024 & 81864.68 \\
\hline
2012 & 20017.09 & 2025 & 90365.11 \\
\hline
2013 & 22376.48 & 2026 & 99458.79 \\
\hline
2014 & 25731.44 & 2027 & 109037.96 \\
\hline
2015 & 28882.07 & 2028 & 118999.65 \\
\hline
2016 & 33003.49 & 2029 & 129771.88 \\
\hline
2017 & 37410.23 & 2030 & 141136.94 \\
\hline
2018 & 42136.67 & 2031 & 153403.18 \\
\hline
2019 & 47454.97 & 2032 & 166586.70 \\
\hline
2020 & 53317.21 & 2033 & 180443.33 \\
\hline
2021 & 59826.59 & 2034 & 195085.73 \\
\hline
2022 & 66771.94 & 2035 & 210682.63 \\
\hline
2023 & 73920.07 & & \\
\hline
\end{tabular}
\end{table}

不同模型分别进行养老金的支出测算。

(1) 我国城镇“老人”养老金支出测算模型

由于“老人”退休年份都在1998年以前,本文在养老金支出测算的起始年份是2012年,那时“老人”男性年龄都在75岁以上,女性都在70岁以上。因为养老保险覆盖率小于1,所以每年领取养老金的“老人”仅仅为适龄退休人口中的一部分,那么令 $RTP_t$ 为领取养老金“老人”占适龄人口的比重,则每年领取养老金“老人”的数量 $TOP_t$ 为:

\begin{equation}
TOP_t = TOP_t^m + TOP_t^f = \left( \sum_{x=75+t-2012}^{100} P_{t,x}^m + \sum_{x=70+t-2012}^{100} P_{t,x}^f \right) \times RTP_t \quad (t \geq 2012)
\tag{15}
\end{equation}

式中 $TOP_t^m$ 表示领取养老金男性“老人”数量,$TOP_t^f$ 表示领取养老金女性“老人”数量,$P_{t,x}^m$ 表示在 $t$ 年 $x$ 岁的适龄男性人数,$P_{t,x}^f$ 表示在 $t$ 年 $x$ 岁的适龄女性人数。

为维持“老人”的生活水平,“老人”养老金水平在1997年的基础上,政府每年做适当比例的调整。假设每年的发放标准按照上一年工资增长率 $G$ 的一定比例进行调整,调整系数为 $\delta$,则 $t$ 年的发放标准 $ORS_t$ 为:

\begin{equation}
ORS_t = ORS_{t-1} \times (1 + \delta G_{t-1}) \times \prod_{t=1997}^{t-1} (1 + \delta G_t)
\tag{16}
\end{equation}

“老人”养老金的支出金额为各年领取养老金“老人”人数和发放标准的乘积,在每年“老人”人数和发放标准一定的前提下,每年“老人”的养老金发放金额 $TOC_t$ 为:

\begin{equation}
TOC_t = TOP_t \times ORS_t
\tag{17}
\end{equation}

(2) 我国城镇“中人”和“新人”养老金支出测算模型

按照国家政策规定,参保人员只有在缴费年限和视同缴费年限累积满15年后,退休后才发给基本养老金,基本养老金由基础养老金和个人账户养老金组成。基础养老金月标准按上年度在岗职工的平均工资和本人的指数化平均缴费工资为基数,缴费每满一年发给1%。个人账户养老金月标准为个人账户储存额除以计发月数,计发月数根据职工退休时城镇人口的平均预期寿命、本人退休年龄等因素确定。“新人”只发放基础户养老金和个人账户养老金;“中人”在发放基础养老金和个人账户养老金的基础上再发放过渡养老金。而过渡养老金的发放原则为“待遇水平合理衔接、新老政策平稳过渡”。由于缴费年限至少为15年,因此“新人”在2013年才会出现,在之前退休的皆为“中人”。根据待遇水平合理衔接、平稳过渡的原则,“中人”发放过渡养老金的原则主要就是弥补视同缴费年限少积累的养老金部分。本文假设“中人”和“新人”养老金的标准基本相同,均为上年平均工资的一定替代率。在计算养老金支出时,可以不用区分“中人”和“新人”,合并计算支出金额即可。

由于各年度平均工资有所不同,各年龄“中人”和“新人”养老金发放标准也有所不同,那么计算养老金支出时要分别计算各年龄“中人”和“新人”的数量。与“老人”

数量计算方法类似,领取养老金的“中人”和“新人”也为适龄人口中的一定比例,$RTP_{t}$ 为领取养老金“老人”占适龄人口的比重,则 $t$ 年 $x$ 岁的领取养老金的“中人”和“新人”数量 $RP_{t,x}$ 为:

\begin{equation}
RP_{t,x} = (P_{t,x}^{m} + P_{t,x}^{f}) \times RTP_{t}
\tag{18}
\end{equation}

根据规定,退休者首年按照上一年社会平均工资的一定比例领取养老金,设 $t$ 年退休者养老金的替代率为 $RS_{t}$,则 $t$ 年新退休的“中人”和“新人”不同性别的养老金的发放标准 $FP_{t}$ 为:

\begin{align}
FP_{t}^{m} &= AW_{t-1} \times RS_{t}^{m} \tag{19} \\
FP_{t}^{f} &= AW_{t-1} \times RS_{t}^{f} \tag{20}
\end{align}

式中 $FP_{t}^{m}$ 表示 $t$ 年新退休的男性“中人”和“新人”养老金发放标准,$FP_{t}^{f}$ 表示 $t$ 年新退休的女性“中人”和“新人”养老金发放标准,$RS_{t}^{m}$ 表示男性“中人”和“新人”替代率,$RS_{t}^{f}$ 表示女性“中人”和“新人”替代率。

退休者首年按照上年工资的一定比例领取养老金,以后各年的标准则根据维持一定生活水平的需要做适当地调整。假设每年的发放标准按照上一年工资增长率的一定比例 $\delta$ 进行调整,则 $t-n$ 年退休年者在 $t$ 年度养老金发放标准 $NRS_{t}$ 为:

\begin{align}
NRS_{t,x}^{m} &= FP_{t-n}^{m} \times \prod_{t-n}^{t} (1 + \delta G_{t-1}) \tag{21} \\
NRS_{t,x}^{f} &= FP_{t-n}^{m} \times \prod_{t-n}^{t} (1 + \delta G_{t-1}) \tag{22}
\end{align}

测算出养老金领取人数和发放标准,则每年的养老金支出金额为二者的乘积,$t$ 年 $x$ 岁男性和女性退休“中人”和“新人”的养老金发放金额为:

\begin{align}
ND_{t,x}^{m} &= RP_{t,x}^{m} \times NRS_{t,x}^{m} \tag{23} \\
ND_{t,x}^{f} &= RP_{t,x}^{f} \times NRS_{t,x}^{f} \tag{24}
\end{align}

由于新的发放政策从 1998 年开始实施,在 1998 年的男性和女性“中人”“新人”的年龄分别为 61 岁和 56 岁,以后每年随着“中人”“新人”的陆续退休,年龄逐步上移,则 $t$ 年领取养老金男性的年龄范围为 60 岁到 $60 + (t-1997)$ 岁,女性的年龄范围为 55 岁到 $55 + (t-1997)$ 岁。每年“中人”“新人”的养老金支出总额为各个年龄男性和女性退休人员的养老金合计支出,令 $NZC_{t}$ 为 $t$ 年“中人”“新人”的养老金支出总额:

\begin{equation}
NZC_{t} = \sum_{x=60}^{60+(t-1997)} ND_{t,x}^{m} + \sum_{x=55}^{55+(t-1997)} ND_{t,x}^{f} (t \geq 2012)
\tag{25}
\end{equation}

那么每年的养老金支出总额为“老人”、“中人”和“新人”的支出总额之和,令 \(ZP_{t}\) 为 \(t\) 年的养老金支出总额,那么支出总额测算模型为:

\begin{equation}
ZP_{t} = TOC_{t} + NZC_{t}
\tag{26}
\end{equation}

(3) 我国城镇养老金支出测算模型基本数据和参数假设

以1998年为分界点,分别计算“老人”、“中人”“新人”所在适龄人群的人口数,由于个体就业者和机关工作人员并不参加养老保险,在前面养老金收入测算时,假设参加养老保险职工在全体就业人员中的覆盖率为80%,那么在计算养老金支出时,假设养老金领取人员在适龄人群中的覆盖率 \(RTP_{t}\) 为80%。

根据2009年中国劳动和社会保障年鉴数据,1997年末企业离退休老人数为2533万人,养老金支出为1251.3亿元,人均养老金支出为4939.992元。因此“老人”的养老金发放标准用1997年4939.992元作为基数,以后各年发放标准根据调整办法进行相应地调整。“中人”和“新人”退休年份的发放标准根据上一年社会平均工资和替代率进行计算。在适龄期间,假设男性和女性的平均工作年限为35年和30年,那么根据养老金的计法标准,60岁退休的男性和55岁退休女性养老金的发放时间分别为139个月和179个月。根据徐颖和王建梅(2009)测算男性和女性的替代率分别为49.48%和40.02%。年度平均工资则利用时间序列预测模型所得的预测工资,各年工资增长率根据统计局公布的在岗职工平均工资和预测数据计算而得,调整系数本文假定0.7。

(4) 我国城镇养老金支出测算结果

根据以上模型和参数假定利用Excel测算出2012—2035年“老人”养老金支出、“中人”“新人”养老金支出和总的养老金支出金额,结果如下表。

表5 2012—2035年我国城镇养老金支出预测值 单位:亿元

\begin{table}[h]
\centering
\begin{tabular}{|c|c|c|c|c|c|}
\hline
年份 & “老人” & \multicolumn{3}{c|}{“中人”、“新人”支出} & 支出合计 \\
\cline{3-5}
 & 支出 & 男 & 女 & 合计 & \\
\hline
2012 & 4169.418 & 6066.898504 & 6877.714768 & 12944.6133 & 17114.03105 \\
\hline
2013 & 4275.591 & 7212.179706 & 8086.467197 & 15298.6469 & 19574.23773 \\
\hline
2014 & 4368.639 & 8786.314424 & 9488.279582 & 18274.594 & 22643.23277 \\
\hline
2015 & 4360.543 & 10539.10467 & 10450.76133 & 20989.866 & 25350.40915 \\
\hline
2016 & 4423.168 & 13116.59932 & 13068.00073 & 26184.6 & 30607.76824 \\
\hline
2017 & 4466.272 & 15072.95483 & 15611.23108 & 30684.1859 & 35150.45765 \\
\hline
2018 & 4487.729 & 17787.35505 & 19034.21266 & 36821.5677 & 41309.29694 \\
\hline
2019 & 4485.774 & 20481.51311 & 22630.65929 & 43112.1724 & 47597.94624 \\
\hline
2020 & 4458.869 & 23593.08507 & 26687.50485 & 50280.5899 & 54739.45892 \\
\hline
2021 & 4405.288 & 26632.71542 & 31205.61544 & 57838.3309 & 62243.61848 \\
\hline
2022 & 4324.25 & 31024.63024 & 35982.67839 & 67007.3086 & 71331.55877 \\
\hline
2023 & 4215.7 & 36981.48903 & 42024.18691 & 79005.6759 & 83221.37546 \\
\hline
2024 & 4079.316 & 42980.71906 & 48556.88425 & 91537.6033 & 95616.91956 \\
\hline
2025 & 3915.605 & 49621.50693 & 56178.1671 & 105799.674 & 109715.2791 \\
\hline
2026 & 3725.925 & 56796.00051 & 64296.49262 & 121092.493 & 124818.4182 \\
\hline
\end{tabular}
\end{table}

\begin{table}
\begin{tabular}{|c|c|c|c|c|c|}
\hline
2027 & 3283.074 & 64115.36826 & 73273.80858 & 137389.177 & 140672.2511 \\
\hline
2028 & 3278.074 & 73489.93455 & 83038.40371 & 156528.338 & 159806.4124 \\
\hline
2029 & 3024.119 & 83327.86221 & 93517.70528 & 176845.567 & 179869.6864 \\
\hline
2030 & 2757.548 & 94775.05284 & 104787.049 & 199562.102 & 202319.6498 \\
\hline
2031 & 2477.634 & 106711.4755 & 116987.1534 & 223698.629 & 226176.2634 \\
\hline
2032 & 2196.061 & 119578.5149 & 129653.1093 & 249231.624 & 251427.6856 \\
\hline
2033 & 1912.772 & 133334.5409 & 143912.8478 & 277247.389 & 279160.1609 \\
\hline
2034 & 1630.728 & 147755.7402 & 159617.8436 & 307373.584 & 309004.3121 \\
\hline
2035 & 1361.453 & 162812.7574 & 176251.772 & 339064.529 & 340425.9824 \\
\hline
\end{tabular}
\end{table}

根据表4得出如下2012—2035年养老金支出总额趋势图:

\begin{figure}[h]
\centering
\includegraphics[width=0.8\textwidth]{image1.png}
\caption{2012—2035年我国城镇养老金支出总额趋势图}
\end{figure}

根据上图可得中国城镇养老金支出总额总体呈现上升趋势,而且增长速度不断加快,平均每年增长速度达到13.27\%。

\begin{figure}[h]
\centering
\includegraphics[width=0.8\textwidth]{image2.png}
\caption{2012—2035年我国城镇老人养老金支出总额}
\end{figure}

根据图6可知,“老人”养老金支出呈现出先上升后下降的趋势,在2018年达到最高点为4487.729亿,从2009年开始到2035年一直递减,平均每年递减速度为6.85\%。

\subsection*{5.1.4 我国城镇居民养老保险收入——企业年金层次}

企业年金是企业根据自身经济能力,为本企业职工建立的旨在提高职工养老保险待遇水平的收入保障计划。它是我国多层次养老保险体系的第二层次,对经济社会都有

16

着至关重要的影响。我国企业年金主要有待遇确定型 (DB) 和缴费确定型 (DC) 两种企业年金制度,由此可以得出有两种给付模型,分别为待遇确定型给付模型和缴费确定型给付模型 \cite{ref11}。

\subsubsection{待遇确定型模型}

待遇确定型企业年金养老保险缴费水平的确定是建立在缴费现值与退休金给付现值平衡的基础上。目前我国企业年金退休金给付规定主要采用以平均工资的一定比例给付。参加养老保险的职工在职期间可能由于死亡、伤残、调离、和退休等减因概率而减少,退休后因死亡而减少。本文只考虑在职职工的减少概率由死亡和退休两种因素构成,假定:

\begin{enumerate}[(a)]
    \item $a, r$ 为分别为职工加入保险的年龄和职工退休年龄;
    \item $w$ 为职工退休后的最高存活年龄;
    \item $v$ 为利率贴现因子,$i$ 为银行利率,则 $v = \frac{1}{1+i}$;
    \item ${}_{t}p_{x}$ 为 $x$ 岁职工 $t$ 年后仍存活概率;
    \item $S_{y}$ 为 $a$ 岁加入养老保险,在 $y$ 岁时的实际工资额;
    \item $SS_{y}$ 为企业的工资尺度函数,影响 $SS_{y}$ 的因素有:工作业绩、工作年限、技术水平、企业效益等;
    \item $ES_{y+t}$ 为 $a$ 岁加入养老保险,在 $y+t$ 岁时的预期工资额;则现龄 $y$ 岁职工在 $y+t$ 岁时的预期工资额:
\end{enumerate}

\begin{equation}
ES_{y+t} = S_{y} \times \frac{SS_{y+t}}{SS_{y}}
\tag{27}
\end{equation}

其中,$\frac{SS_{y+t}}{SS_{y}}$ 表示由工作业绩、工作年限、技术水平、企业效益等因素使工资增长的比例。职工从 $a$ 岁到 $r-1$ 岁的预期累积工资总额:

\begin{equation}
S = \sum_{t=a}^{r-1} S_{a} \times \frac{SS_{t}}{SS_{y}}
\tag{28}
\end{equation}

我国的退休给付规定通常采用平均工资给付,平均工资是以职工加入养老保险到退休前平均工资的一定比例(设调整系数为 $g$)乘以工作年数规定给付。因此平均工资给付公式:

\begin{equation}
B_{r} = S \times g
\tag{29}
\end{equation}

退休金给付方式通常采用年金形式,保证退休职工得到定期收入来源,对每年一单位元的生存年金,在职工退休当年 $r$ 岁时的精算现值:

\begin{equation}
\ddot{x} = \sum_{t=0}^{w} {}_{t}P_{a}v^{t}
\tag{30}
\end{equation}

年龄 \(x=a\) 岁职工的未来给付精算现值:

\begin{equation}
PVFB_{a} = B_{r}{}_{r-a}P_{a}v^{r-a}\ddot{x}_{r}
\tag{31}
\end{equation}

年龄 \(x\) 岁 \((a < x < r)\) 职工的未来给付精算现值:

\begin{equation}
PVFB_{x} = B_{r}{}_{r-x}P_{x}v^{r-x}\ddot{x}_{r}
\tag{32}
\end{equation}

年龄 \(x=a\) 岁,即刚加入计划时职工的缴费精算现值:

\begin{equation}
APVC_{a} = \sum_{t=a}^{r-1} c(a) {}_{t-a}P_{a}v^{r-a}
\tag{33}
\end{equation}

年龄 \(x\) 岁年龄 \(x\) 岁 \((a < x < r)\) 职工的缴费精算现值:

\begin{equation}
APVC_{x} = \sum_{k=a}^{x-1} c(k)(1+i)^{x-k} + \sum_{t=x}^{r-1} c(t) {}_{t-x}P_{x}v^{r-x}
\tag{34}
\end{equation}

其中,\(\sum_{k=a}^{x-1} c(k)(1+i)^{x-k}\) 为 \(x\) 岁职工从年龄 \(a\) 到 \(x-1\) 岁的缴费收入与投资收益的累和。

由职工的缴费精算现值 = 退休金给付精算现值,即可确定不同年龄职工的缴费水平。根据上述分析我们得到计算待遇确定型企业年金中不同年龄职工缴费水平的一种算法:

当 \(x=a\),则由 \(APVC_{a} = PVFB_{a}\),可得

\begin{equation}
c(a) = \frac{B_{r}{}_{r-a}P_{a}v^{r-a}\ddot{x}_{r}}{\sum_{t=a}^{r-1} {}_{t-a}P_{a}v^{r-a}}
\tag{35}
\end{equation}

当 \(x < r-1\),则由 \(APVC_{x} = PVFB_{x}\),得

\begin{equation}
c(a) = \frac{B_{r}{}_{r-x}P_{x}v^{r-x}\ddot{x}_{r} - \sum_{t=a}^{x-1} c(t)(1+i)^{x-t}}{\sum_{t=x}^{r-1} {}_{t-x}P_{x}v^{r-x}}
\tag{36}
\end{equation}

\section*{(2) 缴费确定型给付模式模型}

缴费确定型企业年金为每个参加企业年金计划的雇员建立了个人账户,采取以收定支原则。每个职工的未来权益就是其个人账户的缴费积累值和投资收益之和,在计算中无需考虑总体收支平衡以及职工死亡、伤残、解约等因素。设立 \(h\) 为职工缴纳保费占工资比重;\(A\) 为职工退休时企业每年支付的退休金。则 \(x\) 岁职工缴费的精算现值:

当 $(a<x<r-1)$ 时,
\begin{equation}
APVC_{x}=\sum_{t=a}^{x-1} h S_{a} \frac{S S_{t}}{S S_{a}}(1+i)^{x-t}+\sum_{k=x}^{r-1} h S_{a} \frac{S S_{k}}{S S_{x}} P_{x} v^{k-x}
\tag{37}
\end{equation}

当 $x=a$ 时,
\begin{equation}
APVC_{a}=\sum_{k=a}^{r-1} h S_{a} \frac{S S_{k}}{S S_{a}} P_{a} v^{k-a}
\tag{38}
\end{equation}

退休金的未来给付精算现值:

当 $(a<x<r)$ 时,
\begin{equation}
PVFB_{x}=\left(\sum_{k=0}^{w} A v_{k}^{k} p_{r}\right) v^{r-x}
\tag{39}
\end{equation}

当 $x=a$ 时,
\begin{equation}
PVFB_{a}=\left(\sum_{k=0}^{w} A v_{k}^{k} p_{r}\right) v^{r-a}
\tag{40}
\end{equation}

由缴费确定计划的缴费精算现值 = 退休金给付精算现值,$APVC_{x}=PVFB_{x}$,得

当 $x=a$ 时,
\begin{equation}
\frac{APVC_{a}}{PVFB_{a}}=\frac{\sum_{k=x}^{r-1} h S_{x} \frac{S S_{k}}{S S_{x}} P_{a} v^{k-a}+\sum_{t=a}^{x-1} h S_{a} \frac{S S_{t}}{S S_{a}}(1+i)^{x-t}}{\left(\sum_{k=0}^{w} A v_{k}^{k} p_{r}\right) v^{r-x}}
\tag{41}
\end{equation}

当 $(a<x<r)$ 时,
\begin{equation}
A=\frac{APVC_{x}}{PVFB_{x}}=\frac{\sum_{k=a}^{r-1} h S_{a} \frac{S S_{k}}{S S_{a}} P_{a} v^{k-a}}{\left(\sum_{k=0}^{w} A v_{k}^{k} p_{r}\right) v^{r-a}}
\tag{42}
\end{equation}

\subsection{5.2 新型农村社会养老金收支模型}

\subsubsection{5.2.1 新型农村社会养老保险政策背景分析}

2009 年,国务院就新型农村社会养老保险问题颁布了《国务院关于开展新型农村社会养老保险试点的指导意见》(国发[2009]32 号),拉开了新型农村社会保险的试点性工程。该意见主要包括以下几个要点。

首先,将参保的范围划定为:年满 16 周岁(不含在校学生)、未参加城镇职工基本养老保险的农村居民。

其次,将领取养老金的人群年龄定为 60 周岁。领取的人员分为三类,第一类,“老人”,即已满 60 周岁,未享受城镇职工基本养老保险待遇的,不用缴费,也可以领取基础养老金;第二类为“中人”,即距领取年龄不足 15 年的,应按年缴费,也允许补缴,累计缴费不超过 15 年;第三类为“新人”,即距领取年龄超过 15 年的人,按年缴费,并且累计缴费不少于 15 年。

第三,新农保的基金由三部分构成,即个人缴费、集体补助以及政府补贴。在个人缴费部分,设定缴费标准为 100 元、200 元、300 元、400 元、500 元 5 个档次,由参保人自行选择,实行多缴多得。政府对参保人的的最低补贴为每人每年 30 元。

第四,养老金的待遇由基础养老金和个人账户养老金组成,中央规定,基础养老金标准为每人每月 55 元,地方可做适当调整。个人账户的养老金计发标准为个人账户全部储存额除以 139。

本文结合国发[2009]32 号文件,建立新型农村社会养老保险基金的收支模型。

\subsection*{5.2.2 新型农村社会养老金收入模型}

农村社会养老保险金的来源主要由三部分构成,即个人缴费、集体补助以及政府补贴。由于集体补助的标准由村委会决定,主要来源为社会公益和其他经济组织等机构,存在不稳定性,故在此处不考虑集体补助的数额。因此,农村社会养老保险基金的来源主要为个人缴费和政府补贴。

由于此处,“老人”指的是新型农村社会养老保险实施当年已年满 60 岁的人,该部分人群不用缴费即可领取基本养老金,故对于基金而言,没有收入可言。而对于“中人”和“新人”,其对基金收入的贡献主要体现在缴费上,另外,国家对于参保人也有每人 30 元的补贴。

由此可以建立新型农村社会养老保险基金收入模型为

\begin{equation}
I_t = \sum_{x=16}^{59} Q_{t,x,c} \times f \times l \times C \times (1 + \delta g)^{t-2010} + T \times (1 + \delta g)^{t-2010} \times \sum_{x=16}^{59} Q_{x,t}
\tag{43}
\end{equation}

其中,$Q_{t,x,c}$ 表示 $t$ 年 $x$ 岁以 $c$ 标准(从 100 到 500 元共五档)缴纳养老金的人数,$f$ 表示覆盖率,$l$ 表示收缴率,$g$ 表示农村经济发展的平均增长率,$\delta$ 表示调整系数,$T$ 表示政府补助,即依据政策,$T = 30$。

(1)模型参数估计及假设

覆盖率 $f$,在十一届我国人大四次会议上,国家发展和改革委员会副主任指出,在今后的五年中,新型农村养老保险要全覆盖,故覆盖率 $f$ 假定为 100%。

假定收缴率 $l = 70\%$。

假定调整系数 $\delta$ 与城镇的情况相同,即 $\delta = 0.7$。

$Q_{t,x,c}$ 的预测方法与上文中城镇社会养老保险基金收支模型的预测方法相同。

同时,由《中国统计年鉴》数据可知,2001 年农村居民家庭平均每人纯收入为 2366.4 元,2011 年农村居民家庭平均每人纯收入为 6977.3 元,2001 年和 2011 年的定基物价指数分别为 437 和 565,故在这 10 年间我国农村居民家庭平均每人纯收入的增长率为

\[
\sqrt[10]{\frac{6977.3 / 565}{2366.4 / 437}} - 1 = 8.59\%,
\]

并以此数据作为 $g$ 的估计。

由于100至500档次每档的人数不知,故在此假定每档的缴费人数相同。

(2) 模型的测算

由以上参数及模型可以预测2012年至2035年新型农村社会养老保险基金的总收入情况。

\textbf{表6 2012年-2035年新型农村社会养老保险基金总收入}

\begin{tabular}{|c|c|c|c|}
\hline 年份 & 养老金总收入 & 年份 & 养老金总收入 \\
 & (亿元) & & (亿元) \\
\hline 2012 & 1088.5 & 2024 & 1888.1 \\
\hline 2013 & 1140.6 & 2025 & 1967.2 \\
\hline 2014 & 1193.7 & 2026 & 2044 \\
\hline 2015 & 1246.8 & 2027 & 2123.5 \\
\hline 2016 & 1304.6 & 2028 & 2194.4 \\
\hline 2017 & 1362.9 & 2029 & 2274.6 \\
\hline 2018 & 1426.6 & 2030 & 2351.9 \\
\hline 2019 & 1500.6 & 2031 & 2438.7 \\
\hline 2020 & 1579.3 & 2032 & 2530.2 \\
\hline 2021 & 1667.1 & 2033 & 2627.7 \\
\hline 2022 & 1743.3 & 2034 & 2731.5 \\
\hline 2023 & 1810.2 & 2035 & 2845.8 \\
\hline
\end{tabular}

由以上预测结果可以发现,每年养老金的总收入呈现出较为明显的上升趋势,到2035年,我国农村的社会养老保险基金总收入将达到2845.8亿元,是2012年的近3倍。

\subsection*{5.2.3 新型农村社会养老金支出模型}

在此假设,参保的农民按照自己的意愿选择缴费标准后,之后的缴费档次不再变动。国家依据一定的经济发展情况每年适当增加各档次的缴费标准,并且政府的缴费补贴也计入参保农民的个人账户,该补贴每年的增长幅度相同。

由于“老人”并未缴费,直接可领取养老金。故将其与“中人”和“新人”分开计算。国家养老保险总支出即农村村民养老保险总收入,由此可得新型农村社会养老保险基金支出的模型。

\begin{equation}
O_{t}=\sum_{x=60+(t-2010)}^{100} L_{t,x} \times P_{0} \times(1+\delta g)^{t-2010}+\sum_{x=60(t>2010)}^{60+(t-2010-1)} Q_{t,x,c} \times W_{t-1} \times R R_{\text {中新 }} \times(1+\delta g)^{t-2010}
\tag{44}
\end{equation}

其中,$O_{t}$表示$t$年新型农村社会养老保险基金的总支出,$L_{t,x}$表示$t$年$x$岁“老人”的总人数,$P_{0}$表示新型农村社会养老保险开始实施年份(2009年)的基础养老金标准,$W_{t-1}$表示$t-1$年的农民人均纯收入水平,$R R_{\text {中新 }}$表示“中人”和“新人”的养老金替代率。

\subsubsection{5.2.3.1 模型参数估计及假设}

(1) 养老金替代率的测算$^{[8]}$

在此计算“中人”和“新人”的养老替代率。由于“中人”和“新人”的养老金待遇由基础养老金和个人账户养老金两部分组成。假定每年我国新农保基础养老金的调整比例与我国农民人均纯收入的增长率相等,由此可以得到年满60岁的参保农民领取的

基础养老金为:
\begin{equation}
P_{1}=P_{0} \times(1+g)^{b-a}
\tag{45}
\end{equation}

其中,$a$ 表示参保农民开始缴费的年龄,其范围为政策中规定的 16-59;$b$ 表示参保农民开始领取养老金的年龄,即 60 岁。

同时,可以得到年满 60 岁的参保农民上一年度的我国农民人均纯收入 $Y$ 为:
\begin{equation}
Y=P_{0} \times(1+g)^{b-a}
\tag{46}
\end{equation}

由公式(45)和公式(46)可得到“中人”和“新人”的基础养老金替代率为
\begin{equation}
R R_{1}=\frac{P_{1}}{Y}=\frac{P_{0}}{Y_{0}}
\tag{47}
\end{equation}

以下计算个人账户的养老金替代率。参保农民年满 60 岁时的个人账户累计总额 $M$ 为
\begin{equation}
M=C \sum_{i=1}^{b-a}(1+g_{1})^{b-a-i}(1+r)^{i}+T \sum_{i=1}^{b-a}(1+g_{2})^{b-a-i}(1+r)^{i}
\tag{48}
\end{equation}

其中,$C$ 表示开始缴费的档次,即分为 100 元,200 元,300 元,400 元和 500 元五个档次;$T$ 表示政府补助,即依据政策,设定为 30 元;$r$ 表示个人账户基金名义收益率;$g_{1}$ 表示为每年缴费档次增长率;$g_{2}$ 表示每年的补贴增长率。

则参保农民各年的个人账户养老金在开始领取时的总额现值 $N$ 为
\begin{equation}
N=P_{2} \sum_{j=0}^{m-1} \frac{1}{[1+(r-h)]^{j}}
\tag{49}
\end{equation}

$m$ 表示养老金计发年数,$h$ 表示通货膨胀率。

根据保险精算的平衡原理,基金的个人账户累计总额 $M$ 与领取总额的现值 $N$ 相等,故由(48)式和(49)式得到
\begin{equation}
P_{2}=\frac{C \sum_{i=1}^{b-a}(1+g_{1})^{b-a-i}(1+r)^{i}+T \sum_{i=1}^{b-a}(1+g_{2})^{b-a-i}(1+r)^{i}}{\sum_{j=0}^{m-1} \frac{1}{[1+(r-h)]^{j}}}
\end{equation}

即
\begin{equation}
R R_{2}=\frac{P_{2}}{Y}=\frac{C \sum_{i=1}^{b-a}(1+g_{1})^{b-a-i}(1+r)^{i}+T \sum_{i=1}^{b-a}(1+g_{2})^{b-a-i}(1+r)^{i}}{Y_{0} \times(1+g)^{b-a} \times \sum_{j=0}^{m-1} \frac{1}{[1+(r-h)]^{j}}}
\tag{50}
\end{equation}

则由式(47)和式(50)得到
\begin{equation}
R R_{\text {中新 }}=R R_{1}+R R_{2}=\frac{P_{0}}{Y_{0}}+\frac{C \sum_{i=1}^{b-a}(1+g_{1})^{b-a-i}(1+r)^{i}+T \sum_{i=1}^{b-a}(1+g_{2})^{b-a-i}(1+r)^{i}}{Y_{0} \times(1+g)^{b-a} \times \sum_{j=0}^{m-1} \frac{1}{[1+(r-h)]^{j}}}
\tag{51}
\end{equation}

以下计算此模型中的各项参数。

首先,假定每年缴费档次增长率与每年的补贴增长率相等,并且都与经济发展情况有关,等于人均纯收入的增长率,即 \( g_1 = g_2 = g \)。而 \( g \) 的数值在新型农村社会养老保险收入模型中已求,即 \( g = 8.59\% \)。

其次,国发[2009]32号文件规定“个人账户储存额目前每年参考中国人民银行公布的金融机构人民币一年期存款利率计息。”由于我国的人民币存款利率有频繁调整的情况,故在此选用 2001 年到 2011 年的一年期存款利率的平均值来代替,由此计算得 \( r = 3.03\% \)。

表7 2001年-2011年金融机构人民币一年期存款利率

\begin{table}[h]
\centering
\begin{tabular}{|c|c|c|c|}
\hline
调整时间 & 一年期存款率 (\%) & 调整时间 & 一年期存款率 (\%) \\
\hline
2001.01.01 & 2.25 & 2008.10.09 & 3.87 \\
\hline
2002.02.21 & 1.98 & 2008.10.30 & 3.60 \\
\hline
2004.10.29 & 2.25 & 2008.11.27 & 2.52 \\
\hline
2006.08.19 & 2.52 & 2008.12.23 & 2.25 \\
\hline
2007.03.18 & 2.79 & 2010.10.20 & 2.50 \\
\hline
2007.05.19 & 3.06 & 2010.12.26 & 2.75 \\
\hline
2007.07.21 & 3.33 & 2011.02.09 & 3.00 \\
\hline
2007.08.22 & 3.60 & 2011.04.06 & 3.25 \\
\hline
2007.09.15 & 3.87 & 2011.07.07 & 3.50 \\
\hline
2007.12.21 & 4.14 & 2011.12.31 & 3.50 \\
\hline
\end{tabular}
\end{table}

数据来源:中国人民银行网站

\( h \) 为通货膨胀率,由题目所给的 2001 年到 2011 年的通货膨胀率可得到平均值,即 \( h = 2.43\% \)。

由于 2009 年的农村居民家庭人均纯收入为 5153.2 元,即在此处 \( Y_0 = 5153.2 \)。

此外,国发[2009]32号文件规定,基础养老金标准为每人每月 55 元,即每年 660 元,并且个人账户养老金的月计发标准为个人账户全部的存储额除以 139,由此计算参保农民个人账户的平均计发年限为 \( m = 139/12 \approx 12 \) 年。

由以上数据代入 (51) 式,可得不同参保年龄、不同缴费档次下“中人”和“新人”的养老金替代率,由于文件中规定,年满 16 周岁即可参保,故在此计算 16 岁到 59 岁的替代率。

表8 不同年龄、不同缴费档次下“中新人”的养老金替代率 (\%)

\begin{table}[h]
\centering
\begin{tabular}{|c|c|c|c|c|c|c|c|c|c|c|}
\hline
年龄 & \( C=100 \) & \( C=200 \) & \( C=300 \) & \( C=400 \) & \( C=500 \) & 年龄 & \( C=100 \) & \( C=200 \) & \( C=300 \) & \( C=400 \) & \( C=500 \) \\
\hline
16 & 20.66 & 25.89 & 31.12 & 36.35 & 41.58 & 38 & 17.39 & 20.10 & 22.81 & 25.53 & 28.24 \\
\hline
17 & 20.46 & 25.54 & 30.62 & 35.69 & 40.77 & 39 & 17.29 & 19.92 & 22.55 & 25.18 & 27.82 \\
\hline
18 & 20.27 & 25.20 & 30.12 & 35.05 & 39.98 & 40 & 17.19 & 19.74 & 22.30 & 24.85 & 27.41 \\
\hline
\end{tabular}
\end{table}

\begin{table}
\centering
\begin{tabular}{|c|c|c|c|c|c|c|c|c|c|c|c|}
\hline
19 & 20.08 & 24.86 & 29.64 & 34.43 & 39.21 & 41 & 17.09 & 19.57 & 22.05 & 24.53 & 27.01 \\
\hline
20 & 19.90 & 24.54 & 29.18 & 33.82 & 38.46 & 42 & 16.99 & 19.40 & 21.81 & 24.21 & 26.62 \\
\hline
21 & 19.72 & 24.23 & 28.73 & 33.24 & 37.74 & 43 & 16.90 & 19.24 & 21.57 & 23.91 & 26.25 \\
\hline
22 & 19.55 & 23.92 & 28.29 & 32.67 & 37.04 & 44 & 16.81 & 19.08 & 21.35 & 23.61 & 25.88 \\
\hline
23 & 19.38 & 23.62 & 27.87 & 32.11 & 36.36 & 45 & 16.72 & 18.93 & 21.13 & 23.33 & 25.53 \\
\hline
24 & 19.22 & 23.34 & 27.46 & 31.58 & 35.70 & 46 & 16.64 & 18.78 & 20.91 & 23.05 & 25.18 \\
\hline
25 & 19.06 & 23.06 & 27.06 & 31.06 & 35.05 & 47 & 16.56 & 18.63 & 20.71 & 22.78 & 24.85 \\
\hline
26 & 18.91 & 22.79 & 26.67 & 30.55 & 34.43 & 48 & 16.48 & 18.49 & 20.50 & 22.52 & 24.53 \\
\hline
27 & 18.76 & 22.53 & 26.29 & 30.06 & 33.83 & 49 & 16.40 & 18.36 & 20.31 & 22.26 & 24.21 \\
\hline
28 & 18.62 & 22.27 & 25.93 & 29.58 & 33.24 & 50 & 16.33 & 18.22 & 20.12 & 22.01 & 23.91 \\
\hline
29 & 18.48 & 22.02 & 25.57 & 29.12 & 32.67 & 51 & 16.26 & 18.10 & 19.94 & 21.78 & 23.62 \\
\hline
30 & 18.34 & 21.78 & 25.23 & 28.67 & 32.12 & 52 & 16.19 & 17.97 & 19.76 & 21.54 & 23.33 \\
\hline
31 & 18.21 & 21.55 & 24.89 & 28.24 & 31.58 & 53 & 16.12 & 17.85 & 19.58 & 21.32 & 23.05 \\
\hline
32 & 18.08 & 21.33 & 24.57 & 27.81 & 31.06 & 54 & 16.05 & 17.73 & 19.42 & 21.10 & 22.78 \\
\hline
33 & 17.96 & 21.11 & 24.25 & 27.40 & 30.55 & 55 & 15.99 & 17.62 & 19.25 & 20.88 & 22.52 \\
\hline
34 & 17.84 & 20.89 & 23.95 & 27.01 & 30.06 & 56 & 15.92 & 17.51 & 19.09 & 20.68 & 22.26 \\
\hline
35 & 17.72 & 20.69 & 23.65 & 26.62 & 29.58 & 57 & 15.86 & 17.40 & 18.94 & 20.48 & 22.02 \\
\hline
36 & 17.61 & 20.49 & 23.36 & 26.24 & 29.12 & 58 & 15.80 & 17.30 & 18.79 & 20.28 & 21.78 \\
\hline
37 & 17.50 & 20.29 & 23.09 & 25.88 & 28.67 & 59 & 15.75 & 17.20 & 18.65 & 20.09 & 21.54 \\
\hline
\end{tabular}
\end{table}

由表7可得到相应的不同年龄、不同缴费档次下“中人”和“新人”的养老金替代率图,如下所示:

\begin{figure}[h]
    \centering
    \includegraphics[width=\textwidth]{image1.png}
    \caption{不同年龄不同缴费标准下“中人”和“新人”的养老金替代率(\%)}
    \label{fig:7}
\end{figure}

图 7 不同年龄、不同缴费档次下“中人”和“新人”的养老金替代率图

事实上,由于每个人的基本养老金标准由国家或地方统一划定,故对于不同的“中人”和“新人”,其个体领取的养老金不同主要是因为个人账户的储存额不同,更深层次的原因则是养老金的替代率不同。

图 7 中,横向看,对于每个缴费档次,从 16 岁到 59 岁,其替代率的变化趋势完全一致,16 岁时替代率最高,而 59 岁时替代率最低。事实上,年龄不同意味着缴费的时间长度不同,16 岁的参保人员,到其 60 岁退休,缴费 44 年,而 59 岁的参保人员,到其 60 岁退休,缴费仅为 1 年。由模型可知,对于两个不同年龄的人,其养老金领取额的不同主要跟替代率有关,而替代率越高意味着退休后获得的养老金越多,故模型中,16 岁的人缴纳的时间比 59 岁的人长,其最终获得的养老金也会越多,由此体现了“长缴多得”。

从纵向看,每一个年龄上,500 元的缴费档次对应的替代率均高于相应年龄 100 元缴费档次的替代率。而相同年龄的人,其到 60 岁领取养老金时,前一年的农村家庭人均纯收入水平是一致的,故按此替代率计算,选择交 500 元档的人领取的养老金也将比选择交 100 元档人领取的养老金多,由此体现了“多缴多得”。

\subsubsection{(2) 农民人均纯收入水平预测}

利用国家统计局网站上 1997 年至 2011 年的农村居民家庭人均纯收入数据预测后几年的农村居民家庭人均纯收入水平,即计算模型中的 $W_{t-1}$。

将 1997 年定为基数 1,以此类推,则 2011 年为 15,绘制农村居民家庭人均纯收入情况,得到下图。

\begin{figure}[h]
    \centering
    \includegraphics[width=\textwidth]{image2.png}
    \caption{1997 年至 2011 年农村居民家庭人均纯收入情况}
    \label{fig:8}
\end{figure}

由以上趋势图可知,年份和居民家庭人均纯收入之间的拟合模型可以设为
\begin{equation}
W_{t}=C_{1}+\alpha n^{3}
\tag{52}
\end{equation}
其中,$C_{1}$为常数,$n$即以1997年为1,1998年为2,此后几年以此类推的数值。用EViews软件,得到拟合模型为

\begin{tabular}{l l l l l}
\hline
Dependent Variable: $W$ & & & & \\
Method: Least Squares & & & & \\
Date: 09/22/13 Time: 19:49 & & & & \\
Sample: 1997 2011 & & & & \\
Included observations: 15 & & & & \\
\hline
Variable & Coefficient & Std. Error & t-Statistic & Prob. \\
\hline
$C$ & 2174.584 & 26.69022 & 81.47493 & 0.0000 \\
$N^{3}$ & 1.409031 & 0.018723 & 75.25770 & 0.0000 \\
\hline
R-squared & 0.997710 & Mean dependent var & 3527.253 & \\
Adjusted R-squared & 0.997534 & S.D. dependent var & 1538.784 & \\
S.E. of regression & 76.41752 & Akaike info criterion & 11.63387 & \\
Sum squared resid & 75915.28 & Schwarz criterion & 11.72827 & \\
Log likelihood & -85.25400 & F-statistic & 5663.722 & \\
Durbin-Watson stat & 1.670691 & Prob(F-statistic) & 0.000000 & \\
\hline
\end{tabular}

图9 EViews输出结果图

此处,调整后的$R^{2}=0.9975$,说明模型的拟合情况很好,并且$t$统计量均大于2,说明变量显著,即得到模型为$W_{t}=2174.58+1.41n^{3}$。从而可以预测2012年到2035年的农村居民家庭人均纯收入,见下表。

表9 2012年至2035年的农村居民家庭人均纯收入预测值

\begin{tabular}{|c|c|c|c|c|c|}
\hline
年份 & 农村居民家庭人均纯收入(元) & 年份 & 农村居民家庭人均纯收入(元) & 年份 & 农村居民家庭人均纯收入(元) \\
\hline
2012 & 7949.94 & 2020 & 21666.42 & 2028 & 48377.46 \\
\hline
2013 & 9101.91 & 2021 & 24205.83 & 2029 & 52845.75 \\
\hline
2014 & 10397.7 & 2022 & 26956.74 & 2030 & 57593.22 \\
\hline
2015 & 11845.77 & 2023 & 29927.61 & 2031 & 62628.33 \\
\hline
2016 & 13454.58 & 2024 & 33126.9 & 2032 & 67959.54 \\
\hline
2017 & 15232.59 & 2025 & 36563.07 & 2033 & 73595.31 \\
\hline
2018 & 17188.26 & 2026 & 40244.58 & 2034 & 79544.1 \\
\hline
2019 & 19330.05 & 2027 & 44179.89 & 2035 & 85814.37 \\
\hline
\end{tabular}

(3) 其他相关参数假定

假定模型中的$g$与2001年至2011年农村家庭人均纯收入的平均增长率相等,即$g=8.59\%$;假定调整系数$\delta$与城镇的情况相同,即$\delta=0.7$。此外,由于中央规定,基础养老金标准为每人每月55元,故,计算每年基础养老金发放标准为每人每年660元,

\begin{table}
\centering
\begin{tabular}{|c|c|c|c|c|c|}
\hline
缴费 & C=100 & C=200 & C=300 & C=400 & C=500 \\
\hline
年档次 & & & & & \\
份 & & & & & \\
\hline
2012 & 39.14 & 42.80 & 46.45 & 50.10 & 53.76 \\
\hline
2013 & 71.96 & 78.76 & 85.57 & 92.38 & 99.19 \\
\hline
2014 & 118.47 & 129.84 & 141.21 & 152.58 & 163.95 \\
\hline
2015 & 179.72 & 197.21 & 214.70 & 232.18 & 249.67 \\
\hline
2016 & 256.93 & 282.26 & 307.58 & 332.91 & 358.23 \\
\hline
2017 & 359.98 & 395.97 & 431.96 & 467.94 & 503.93 \\
\hline
2018 & 482.84 & 531.71 & 580.59 & 629.47 & 678.35 \\
\hline
2019 & 619.50 & 682.85 & 746.20 & 809.55 & 872.90 \\
\hline
2020 & 791.61 & 873.49 & 955.38 & 1037.27 & 1119.15 \\
\hline
2021 & 986.37 & 1089.42 & 1192.48 & 1295.54 & 1398.59 \\
\hline
2022 & 1283.00 & 1419.52 & 1556.05 & 1692.57 & 1829.09 \\
\hline
2023 & 1692.36 & 1876.43 & 2060.51 & 2244.58 & 2428.65 \\
\hline
2024 & 2164.33 & 2404.04 & 2643.76 & 2883.47 & 3123.19 \\
\hline
2025 & 2746.38 & 3056.18 & 3365.98 & 3675.78 & 3985.58 \\
\hline
2026 & 3453.52 & 3850.23 & 4246.94 & 4643.65 & 5040.36 \\
\hline
2027 & 4258.44 & 4755.50 & 5252.56 & 5749.62 & 6246.68 \\
\hline
2028 & 5287.77 & 5916.73 & 6545.70 & 7174.66 & 7803.63 \\
\hline
2029 & 6456.74 & 7237.83 & 8018.92 & 8800.00 & 9581.09 \\
\hline
2030 & 7871.06 & 8840.53 & 9810.00 & 10779.48 & 11748.95 \\
\hline
2031 & 9462.22 & 10646.92 & 11831.61 & 13016.31 & 14201.01 \\
\hline
2032 & 11286.31 & 12722.30 & 14158.30 & 15594.30 & 17030.29 \\
\hline
2033 & 13355.40 & 15081.56 & 16807.71 & 18533.87 & 20260.03 \\
\hline
2034 & 15688.91 & 17748.19 & 19807.48 & 21866.77 & 23926.05 \\
\hline
2035 & 18260.86 & 20692.99 & 23125.12 & 25557.25 & 27989.37 \\
\hline
\end{tabular}
\caption{2012年-2035年各缴费档次下新型农村社会养老保险基金支出额(亿元)}
\end{table}

\begin{figure}[h]
    \centering
    \includegraphics[width=\textwidth]{image.png}
    \caption{2012年-2035年各缴费档次下新型农村社会养老保险基金支出额}
    \label{fig:10}
\end{figure}

结合图表,可以得到,每一年度中,对缴费500元的人的养老金支出额都大于对其他缴费档次的人的养老金支出额,换而言之,从总额上看,缴费100元的人获得的养老金收入的总数最低,缴费200元的人获得的养老金收入比缴费100元的人获得的略高,缴费500元的人则获得最高额度的养老金收入。此结果与国家政策相符,体现了“多缴多得”的原则。

此外,在同一年度下,对各个缴费档次的养老金支出额进行加总,可以得到各年度的养老金支出总额。

\begin{table}[h]
    \centering
    \caption{2012年-2035年新型农村社会养老保险基金总支出}
    \label{tab:11}
    \begin{tabular}{|c|c|c|c|}
        \hline
        年份 & 养老金总支出 & 年份 & 养老金总支出 \\
             & (亿元) & & (亿元) \\
        \hline
        2012 & 919.852 & 2024 & 13966.131 \\
        \hline
        2013 & 1130.191 & 2025 & 17566.258 \\
        \hline
        2014 & 1421.967 & 2026 & 21957.018 \\
        \hline
        2015 & 1801.626 & 2027 & 26968.005 \\
        \hline
        2016 & 2276.768 & 2028 & 33413.458 \\
        \hline
        2017 & 2907.629 & 2029 & 40756.193 \\
        \hline
        2018 & 3657.888 & 2030 & 49685.393 \\
        \hline
        2019 & 4490.909 & 2031 & 59764.196 \\
        \hline
        2020 & 5539.520 & 2032 & 71365.842 \\
        \hline
        2021 & 6725.238 & 2033 & 84578.663 \\
        \hline
        2022 & 8540.688 & 2034 & 99541.005 \\
        \hline
        2023 & 11057.876 & 2035 & 116091.075 \\
        \hline
    \end{tabular}
\end{table}

观察以上数据可以发现,就整体趋势而言,养老金总支出呈上升趋势。就增长率而言,从2012年到2035年,平均每年增长23.46%,以2023年的增长率最为显著,达到29.47%。具体而言,在开始的几年间,新农保基金的总支出增长幅度较小,而到后几年,尤其是从2020年附近开始,养老金的总支出增长幅度开始增大,并且逐年增加,其原因在于60年代人口较多,而到2020年时,60年代的人正值60岁,达到领取新农村养

老保险金的资格,因此,随着人口基础的增大,国家负担的养老保险金的压力也开始加大。2012 年时农村养老金总支出为 919.852 亿元,而到 2035 年增加到 116091.075 亿元,是 2012 年的 126 倍,其增长之快着实让人惊叹。

\subsection*{5.3 我国社会养老金的缺口分析}

\subsubsection{5.3.1 我国城镇养老金的缺口分析}

\subsubsection*{5.3.1.1 养老基金缺口理解}

本文认为我国城镇养老金缺口是指退休职工对养老金的需求与在职职工养老金的供给之间的差额。2005 年政府颁布了《关于完善企业职工基本养老保险制度的决定》,标志着我国养老金制度从原来的现收现付制转为社会统筹和个人账户相结合的部分积累制。那么我国养老金缺口可能出现在三个部分,前两个部分是实行现收现付制的社会统筹和实行完全积累制的个人账户的缺口,这两个部分可以表示为养老金收支赤字;第三部分为制度转变之初,由于 “老人” 和 “中人” 在之前并没有缴纳过养老金个人账户部分导致带来了养老金制度转变的转轨成本。因此我国城镇养老金缺口就等于养老金收支赤字与转轨成本之和。养老金收入分为财政补贴与征缴收入,转轨成本分为 “老人” 转轨成本与 “中人” 转轨成本。由于前文已经测算出养老金收入与支出,因此只需要测算转轨成本。

\subsubsection*{5.3.1.2 转轨成本测算模型}

中国的养老制度从现收现付制过渡到统账结合制,意味着就业人口养老金交纳的一部分用于建立个人账户,这部分养老金不再作代际转移,因此出现了一块 “缺口”,这块 “缺口” 就是转轨成本。转轨成本实际上涉及两种人的养老金缴纳和养老金支付问题。第一种是已经退休的 “老人”,他们以前没有个人账户,他们面临未来需求金额养老金时个人账户的空白;第二种是 “中人”,他们还没有退休,但是已经缴纳了相当年份的现收现付制养老金,如果根据统账结合制,他们将面临过去年份中应该有的那部分个人账户的缺失。因此,转轨成本就是在制度转变之初($t=0$ 时刻), “老人” 和 “中人” 两部分职工在之前所缺失的个人账户缴纳额。

\textbf{(1) “老人” 转轨成本测算模型}

在 $t=0$ 时刻,$k$ 岁职工 ($b+1 \leq k < c$) 缺失的个人账户养老金就是在其在职期间应该缴纳的金额累计到 0 时刻的值:

\begin{equation}
TC_{1,k} = n_0 \int_{k-1}^k g_0(x) dx \sum_{t=a}^b \tau_2 W_t^{t-k} \exp \left( \int_{t-k}^0 r(s) ds \right)
\tag{53}
\end{equation}

$\mathbf{n}_t$ 表示第 $t$ 年在职职工和退休职工的总人数,$\mathbf{g}_t(\mathbf{x})$ 为 $t$ 时刻人口密度函数,$a$ 为职工参保年龄,$b$ 为退休年龄,$\tau_2$ 表示个人账户部分养老金缴费率,$r(s)$ 表示利率关于时间的函数,$W_t^{t-k+1}$ 表示第 $t$ 年 $k$ 岁退休职工在其工作时的工资流,$c$ 表示极限年龄。

因此 $\mathbf{TC}_1$ 表示初始时刻所有 “老人” 个人账户缺失的加总:

\begin{equation}
TC_1 = \sum_{k=b+1}^c TC_{1,k}
\tag{54}
\end{equation}

(2) “中人”转轨成本测算模型

在 \( t=0 \) 时刻,k 岁职工 \((a+1 \leq k < b)\) 缺失的个人账户养老金就是转轨之前应该缴纳的金额累积到 0 时刻的值:

\[
TC_{2,k} = n_0 \int_{k-1}^k g_0(x) dx \sum_{t=a}^k \tau_2 W_t^{t-k} \exp \left( \int_{t-k}^0 r(s) ds \right)
\]

\( TC_2 \) 表示初始时刻所有“中人”个人账户缺失的加总:

\[
TC_2 = \sum_{k=a+1}^k TC_{2,k}
\]

那么制度转变之初产生的总的转轨成本为:

\[
TC = TC_1 + TC_2
\]

上式表示的总转轨成本在得不到弥补的情况下,随着时间的推移逐年增加,就形成了转轨成本养老金缺口。在第 \( y \) 年,这个缺口值为

\[
L_3^y = TC \times \exp \left( \int_0^y r(s) ds \right)
\]

所以,第 \( y \) 年我国城镇养老金体系中累积缺口为累积收支赤字与转轨成本之和。

(3) 参数假设

本文假设所有职工都为 60 岁退休,25 岁开始参加养老金个人账户缴纳保费,极限寿命为 105 岁即:\( a=25, b=60, c=105 \)。然后假设 2005 年 25 岁职工总工资为 30000 元,即 \( W_{25}^0 = 30000 \),利率为 2.5%。

(4) 转轨成本模型测算结果

根据以上参数通过模拟得到 2005—2035 年转轨成本预测结果如下图:

\begin{figure}[h]
\centering
\includegraphics[width=\textwidth]{image.png}
\caption{2005—2035 年转轨成本预测值}
\end{figure}

\begin{table}
\centering
\begin{tabular}{|c|c|c|c|c|c|c|}
\hline
年份 & 养老金收入(亿元) & 养老金支出(亿元) & 收支差额(亿元) & 累计结余(亿元) & 转轨成本(亿元) & 养老金缺口(亿元) \\
\hline
2012 & 20806.06 & 17114.03 & 3692.027 & 23189.03 & 39583.26 & -16394.2 \\
\hline
2013 & 23255.49 & 19574.24 & 3681.253 & 26870.28 & 40674.44 & -13804.2 \\
\hline
2014 & 26645.38 & 22643.23 & 4002.146 & 30872.43 & 41765.62 & -10893.2 \\
\hline
2015 & 29868.97 & 25350.41 & 4518.56 & 35390.99 & 42856.8 & -7465.81 \\
\hline
2016 & 34011.81 & 30607.77 & 3404.037 & 38795.02 & 43947.98 & -5152.96 \\
\hline
2017 & 38441.99 & 35150.46 & 3291.535 & 42086.56 & 45039.16 & -2952.6 \\
\hline
2018 & 43192.08 & 41309.3 & 1882.787 & 43969.34 & 46130.34 & -2161 \\
\hline
2019 & 48514.13 & 47597.95 & 916.1814 & 44885.53 & 47221.52 & -2335.99 \\
\hline
2020 & 54364.74 & 54739.46 & -374.718 & 44510.81 & 48312.7 & -3801.89 \\
\hline
2021 & 60840.00 & 62243.62 & -1403.62 & 43107.19 & 49403.88 & -6296.69 \\
\hline
2022 & 67745.81 & 71331.56 & -3585.75 & 39521.44 & 50495.06 & -10973.6 \\
\hline
2023 & 74867.75 & 83221.38 & -8353.62 & 31167.82 & 51586.24 & -20418.4 \\
\hline
2024 & 82754.69 & 95616.92 & -12862.2 & 18305.58 & 52677.42 & -34371.8 \\
\hline
2025 & 91185.30 & 109715.3 & -18530 & -224.396 & 53768.6 & -53993 \\
\hline
2026 & 100194.97 & 124818.4 & -24623.4 & -24847.8 & 54859.78 & -79707.6 \\
\hline
2027 & 109685.05 & 140672.3 & -30987.2 & -55835 & 55950.96 & -111786 \\
\hline
2028 & 119561.34 & 159806.4 & -40245.1 & -96080.1 & 57042.14 & -153122 \\
\hline
2029 & 130219.73 & 179869.7 & -49650 & -145730 & 58133.32 & -203863 \\
\hline
2030 & 141460.13 & 202319.6 & -60859.5 & -206590 & 59224.5 & -265814 \\
\hline
2031 & 153567.97 & 226176.3 & -72608.3 & -279198 & 60315.68 & -339514 \\
\hline
2032 & 166558.96 & 251427.7 & -84868.7 & -364067 & 61406.86 & -425473 \\
\hline
2033 & 180208.53 & 279160.2 & -98951.6 & -463018 & 62498.04 & -525516 \\
\hline
2034 & 194621.51 & 309004.3 & -114383 & -577401 & 63589.22 & -640990 \\
\hline
2035 & 209954.48 & 340426 & -130472 & -707873 & 64680.4 & -772553 \\
\hline
\end{tabular}
\caption{2012-2035年城镇养老金缺口预测值}
\end{table}

\begin{figure}[h]
\centering
\includegraphics[width=0.8\textwidth]{image.png}
\caption{2012-2035年养老金缺口的趋势图}
\end{figure}

根据上图和上表可以得出,从2012—2018年养老金缺口是逐年减少,从16394.2亿元下降到2018年的2161年,平均每年降幅为33.57%,但是从2019—2035年缺口值以很快的速度上升,从2335.99亿元增加到772553元,平均每年增长速度为40.67%。因此可以分析出2018年是拐点,是2012—2035年间缺口最小的一年,那是因为从2018年之前养老金有盈余,而且收入大于支出,因此累积的盈余也是逐年增长。其实从累积盈余来看,它的拐点出现在2020年,即在这一年开始养老金收入小于支出,但是从养老金缺口来看却提前了2年,那是因为有转轨成本的存在,每年的盈余需要弥补转轨成本的增长。

从养老金缺口的绝对数来看,在2012—2035年间最严重的发生在2035年,它的缺口额达到772553亿元。因为从2019年随着累积的盈余无法弥补转轨成本,而且累积的盈余也被入不敷出所要弥补的缺口慢慢消耗掉,所以2019年之后养老金缺口额由开始逐年增长,而且速度很快,一直持续到2035年。

考虑党的十八大明确提出“确保到2020年实现全面建成小康社会目标”,需要对模型进行局部调整。由于具体目标为:到2020年,实现国内生产总值和城乡居民人均收入比2010年翻一番。从退休人员角度来讲,领取的养老金也会相应的按照一定水平提高,而不能再假定是上一年平均工资增长率的70%来调整,因此本文支出模型中的养老金调整系数应该要变化;为了让更多的人可以享受到经济增长的成果,应该让更多的人纳入到养老保险体系中,那么对于本文来说,模型中的参保覆盖率应该要调整。从职工角度来讲,为了提高自己的收入水平,必须提高所在企业的增加值或者说效益,进而为我国增加值的提高做出自己贡献,这样国家分配的基数就会增大,职工报酬才有可能提高,那么对于本文来说,模型的劳动参与率就需要调整,而不是前文所假设的数值。

\subsection*{5.3.2 新型农村社会养老金缺口模型}

新型农村的社会养老保险基金缺口可以理解为基金收入与基金支出之差,即可表示为

\begin{equation}
Z_{t}=I_{t}-Q_{t}
\tag{59}
\end{equation}

其中 \(Z_{t}\) 表示新农村的养老金缺口。

由已预测的2012年至2035年的养老保险基金收支数据可计算得到下表。

表13 2012年-2035年新型农村社会养老保险基金缺口

\begin{tabular}{|c|c|c|c|}
\hline 年份 & 养老金缺口 & 年份 & 养老金缺口 \\
 & (亿元) & & (亿元) \\
\hline 2012 & 1367.62 & 2024 & -44284.93 \\
\hline 2013 & 1378.01 & 2025 & -59884.03 \\
\hline 2014 & 1149.71 & 2026 & -79797.04 \\
\hline 2015 & 594.93 & 2027 & -104641.55 \\
\hline 2016 & -377.26 & 2028 & -135860.59 \\
\hline 2017 & -1921.96 & 2029 & -174342.20 \\
\hline 2018 & -4153.24 & 2030 & -221675.66 \\
\hline 2019 & -7143.57 & 2031 & -279001.19 \\
\hline 2020 & -11103.74 & 2032 & -347836.87 \\
\hline 2021 & -16161.83 & 2033 & -429787.82 \\
\hline 2022 & -22959.21 & 2034 & -526597.29 \\
\hline 2023 & -32206.90 & 2035 & -639842.55 \\
\hline
\end{tabular}

对以上数据进行作图,得到缺口的整体变化趋势图,如下所示。

\begin{figure}[h]
    \centering
    \includegraphics[width=\textwidth]{image.png}
    \caption{2012年-2035年新型农村社会养老保险金缺口情况}
\end{figure}

结合图表分析可以发现,2012年至2015年的新农村养老保险不存在负缺口,即收入大于支出,而从2016年开始,入不敷出,开始出现缺口,并逐年增大。而这一时期刚好是60年代的人开始领取养老金的时刻,模型与实际相等,说明可信度较高,模型的缺口分析是合理的。此后,缺口逐年增大,直到2035年,缺口增大到639842.55亿元。

在新型农村养老金缺口模型中,缺口自2016年为负后逐年增大,到我们预测的2035年还未见底,若国家不进行调整,结果将更加严重。

考虑到十八大提出的收入倍增计划,对于农村而言,意味着到2020年,农村居民家庭人均可支配收入要实现翻一番,由国家统计局的数据可知,2010年农村家庭平均每人纯收入为5919.0元,即到2020年人均要达到11838元,因此,在我们的模型中,$W_{t}$的数值需要调整,相应的农村居民人均收入的平均增长率也需要调整,可以以2010年到2020年的平均增长率作为$g$的估计。此外,与城镇养老金缺口模型相同,调整系数不能再设定为0.7,而应根据相应的经济发展水平进行更改。

\subsection{5.4 我国社会养老保险体系可持续性分析}

\subsubsection{5.4.1 中外各国养老金制度比较分析}

养老保险作为社会发展的重要体现,在各国都受到了高度重视。现选取几个具有代表性的发达国家和发展中国家,对其养老金制度进行比较分析,从而为中国的养老金制度发展提供借鉴。

\subsubsection{5.4.1.1 发达国家养老金制度}

(1)美国的养老金制度\footnote{7}

美国的养老金保障体系主要由基本社会保障计划、雇主补充养老计划和私人储蓄养老构成。按具体实施部门来分,可分为公共部门和私人部门,公共部门养老保障的对象主要包括联邦政府、学校等机构,私人部门养老保障对象主要包括企业雇员、自营劳动者等。其中,私人部门较为活跃。

私人部门雇主补充养老金计划主要包括 DB 计划、组合计划和 DC 计划。DB 计划即确定给付计划,由雇主向雇员给付一定的金额。DC 计划即确定缴费计划,由雇主为每一位雇员建立一个账户,并定期存入一定的缴费。公共部门雇主养老金补充计划中比较有名的有 401(K)计划和 403(b)计划等。

总之,美国的养老金保障体系较为完善,从私人和公共部门,两方面全面保证了国民的养老水平,值得我国参考。

(2) 加拿大的养老金制度 \footnote{6}

加拿大的养老金制度最大的特点是多层次性,实行国家、企业和个人三联动的养老金制度。

加拿大联邦政府的养老金计划主要由两老年金制度和退休金计划构成。

老年金主要从国家财政收入中支出,无需享受者付出任何代价,满足条件即可领取,其对象为:在加拿大合法居住 10 年以上,并且年满 65 岁的老人。退休金计划则是为了使退休的人能够获得一定的养老金收入,其主要来源实雇员和雇主缴纳的养老保险费,缴纳比例统一由联邦税务局收取,一次性确定 25 年的缴费比例,之后每隔 5 年进行一次调整。退休金的发放主要以退休人员的工龄、工资水平、缴费情况和物价水平确定,退休后所获老年金大约占工资收入水平的 15%,退休金的比例大约达到 25%。

除此之外,加拿大将公司养老金计划作为养老金体系的补充,以提高公司员工的积极性。同时,加拿大政府鼓励个人进行储蓄,推出个人注册退休储蓄计划,由个人自愿参加,并且存入的钱只能到退休时才能领取。对于这笔资金,政府给予一定的政策优惠,从而推动个人养老金体系的建设。

总之,加拿大的养老金制度中多层次性的特点值得我国借鉴。

(3) 瑞典的养老金制度 \footnote{7}

一直以来,在社会保障方面,瑞典一直具有很强的代表性。瑞典的养老模式可以概括为 “名义缴费确定型” 模式,该模式兼具了 “给付确定型现收现付制” 的特点,又有 “缴费确定型累积制” 的特点,提出了养老金的给付指数化问题,即根据通货膨胀率的情况,对养老金的给付指数进行调整,并依据具体国情,对该账户附加了一些其他功能,如设立自愿性个人账户,以及 “最低养老金” 制度等,保证了整体养老金制度的完整性。

“名义缴费确定型” 模式有很多的优点,但也存在欠缺之处,比如它的再分配功能比较弱,从而使得在贫困人群中推行的难度较大,对弱势群体无利。此外,对于短期内财政的自动调节问题,该账户也难以起到很大的作用。

而对于中国而言,相比于其他制度,“名义缴费确定型” 模式对中国具有很强的适应性,主要原因在于该模式不仅可以很好地解决转轨成本问题,而且可以实现兼顾公平的问题,也可以克服在养老金给付问题上存在的缴费 “搭便车” 问题等。所以,对于转型期的中国养老保险体制而言,可以适当借鉴瑞典的该 “名义缴费确定型” 模式。

\subsection*{5.4.1.2 发展中国家养老金制度}

与发达国家相比,发展中国家的养老金制度相对较为薄弱,故在此,仅以智利 \footnote{7} 为例进行分析。

20 世纪 70 年代,智利实行 “现收现付” 的社会保障体制,由此出现了较多的问题,如将以非独立工人为保障对象,将独立工人以及劳动力排除在外,并且分配不公。此后,智利进行了养老金制度的改革。从 1981 年开始实行新的养老保险制度,此后,智利养老金的缴费人数在参保人数中所占的比重一直在 50% 以上。

\section*{表14 中外各国社会养老保险缴费率(单位:\%)}

\begin{tabular}{|c|c|c|c|c|}
\hline
\multirow{2}{*}{ 国家 } & \multicolumn{4}{c|}{ 社会养老保险缴费率 } \\
\cline{2-5}
& 总缴费率 & 雇主缴费率 & 雇员缴费率 & 雇员:雇主 \\
\hline
德国 & 18.6 & 9.3 & 9.3 & 1:1 \\
\hline
意大利 & 29.64 & 21.3 & 8.34 & 2.6:1 \\
\hline
挪威 & 22 & 14.2 & 7.8 & 1.8:1 \\
\hline
西班牙 & 28.3 & 23.6 & 4.7 & 5:1 \\
\hline
日本 & 15.4 & 7.7 & 7.7 & 1:1 \\
\hline
英国 & 22.2 & 10.2 & 12 & 0.9:1 \\
\hline
美国 & 12.4 & 6.2 & 6.2 & 1:1 \\
\hline
阿根廷 & 23.7 & 12.7 & 11.0 & 1.2:1 \\
\hline
巴西 & 31.0 & 20.0 & 11.0 & 1.8:1 \\
\hline
中国 & 28.0 & 20.0 & 8.0 & 2.5:1 \\
\hline
印度 & 24.0 & 12.0 & 12.0 & 1:1 \\
\hline
印尼 & 6.0 & 4.0 & 2.0 & 2:1 \\
\hline
俄罗斯 & 26.0 & 26.0 & 0 & \\
\hline
\end{tabular}

数据来源:张士斌、杨黎源、张天龙《养老金替代率的国际比较与中国改革路径》

\begin{table}
\centering
\caption{中外各国社会养老保险替代率水平(单位:\%)}
\begin{tabular}{c|c|c|c|c|c|c}
\hline
 & & & & & & \\
\hline
\multirow{6}{*}{发达国家} & 澳大利亚 & 47.3 & 意大利 & 64.5 & 希腊 & 95.7 \\
\hline
 & 奥地利 & 76.6 & 日本 & 34.5 & 意大利 & 64.5 \\
\hline
 & 比利时 & 57.6 & 韩国 & 42.1 & 美国 & 78.2 \\
\hline
 & 加拿大 & 69.7 & 挪威 & 53.1 & 瑞典 & 53.8 \\
\hline
 & 法国 & 49.1 & 葡萄牙 & 53.9 & 英国 & 68.6 \\
\hline
 & 德国 & 59.0 & 西班牙 & 81.2 & & \\
\hline
\multirow{3}{*}{发展中国家} & 阿根廷 & 78.1 & 俄罗斯 & 52.3 & 南非 & 33.1 \\
\hline
 & 巴西 & 85.9 & 印度 & 65.2 & 智利 & 44.9 \\
\hline
 & 中国 & 49.3 & 印尼 & 14.1 & 墨西哥 & 30.9 \\
\hline
\end{tabular}
\end{table}

数据来源:张士斌、杨黎源、张天龙《养老金替代率的国际比较与中国改革路径》

由上表中各国的社会养老保险替代率水平可以发现,发达国家中,希腊的替代率水平远远高于其他发达国家和发展中国家,此外,西班牙的替代率水平也较高,与发展中国家中的巴西相近。

与发达国家相比,中国的替代率明显落后,仅高于澳大利亚、法国、日本和韩国,并且高的幅度很低。事实上,2002年以前,我国的养老金替代率一直处于较高的水平,高达72.5%,之后开始下降,2009年为49.3%。与发展中国家相比,我国的替代率处于中等水平,高于印尼、南非、智利和墨西哥。

总体而言,我国的替代率水平较为适中,但可参照其他国家的水平给予一定的调整。

\subsection*{5.4.2 我国养老金缺口影响因素分析}

以下以城镇为例,对养老金缺口的影响因素进行具体分析。根据徐颖,王建梅(2009)的研究,如果按照60岁退休,我国目前实际的城镇养老金替代率为49.48%,而在现行养老保险制度下,充分挖掘制度的内在潜力可以达到的最高替代率为68%;如果按照55岁退休,“标准人”的城镇养老金替代率为40.02%,最高可达到56.17%。本文假设男性60岁退休,女性55岁退休,因此男性的养老金替代率为49.48\%—68\%,女性的养老金替代率为40.02\%—56.17\%。

根据孙亚娜和安曼(2010)的研究,他们按照不同的个人工资与社会平均工资的比重测算了不同的养老金最优缴费,最小的为23.44\%,最大的为44.36\%,因此本文设定的养老保险缴费率的粗略范围是(23.44\%,44.36\%)

根据题意,本文需要在男性替代率(49.48\%,68\%),女性替代率(40.02\%—56.17\%),

缴费率(23.44%—44.36%)这三个大的区间内,寻找更小的范围使得我国城镇养老保险能够可持续发展。

本文先选取的范围 2012—2035 年,根据前文预测得出我国在这一范围内的养老金缺口一直存在,虽然在 2018 年的时候降到最低,但是仍然有 2100 多亿,而从 2019—2035 年养老金缺口持续上升,在 2035 年达到 77 多万亿,由此可见需要调整缴费率和替代率使得缺口缩小。本文的理解为寻找替代率和缴费率的合理区间使得我国养老金缺口维持在一定水平,最好能有有盈余,提高偿付能力。那么在 2012—2035 年这 24 年间最好有盈余,因此本文假定至少有 22 年养老金收支差额累积大于转轨成本。

通过计算机对上文建立的养老金缺口模型变动男性替代率,女性替代率,缴费率进行模拟,此时令 \( x1 \) 为男性替代率,\( x2 \) 为女性替代率,\( x3 \) 为缴费率,计算机以 0.005 个单位进行模拟一共有 52668 个点,在 2012—2035 年间符合养老金 22 年有盈余的点 \((x, y, z)\) 有 240 个,具体结果如下图:

\begin{figure}[h]
    \centering
    \includegraphics[width=0.8\textwidth]{image.png}
    \caption{替代率和缴费率变动图}
    \label{fig:14}
\end{figure}

由图可见,这 240 个点聚集在一个平面里,方程如下:
\begin{equation}
25x1 + 35x2 - 7x3 \leq 25
\tag{60}
\end{equation}

而此时男性替代率 \( x1 \) 区间为:(49.48%, 56.48%);女性替代率 \( x2 \) 区间为:(40.02%, 45.02%);缴费率 \( x3 \) 区间为:(41.84%, 44.34%)。

本文中替代率,缴费率,工资调整系数是可以调整的变量,也是影响养老金缺口走势的变量。为了更好地研究替代率和缴费率如何影响,因此在研究男性替代率对养老金缺口走势影响时控制女性替代率,缴费率和工资调整系数不变,研究女性替代率和缴费率也进行同样的操作。

首先考虑变动男性替代率,通过计算机对养老金缺口模型进行模拟得到下图:

\begin{figure}[h]
    \centering
    \includegraphics[width=0.8\textwidth]{image2.png}
    \caption{变动男性替代率的模拟结果}
    \label{fig:15}
\end{figure}

\begin{figure}[h]
    \centering
    \includegraphics[width=\textwidth]{image1.png}
    \caption{男性替代率对养老金缺口走势的影响}
    \label{fig:15}
\end{figure}

上图男性替代率 \( x1 \) 顺序依次为 \((0.49, 0.54, 0.59, 0.64)\),从图中可以看出,随着 \( x1 \) 的减少,波峰值在不断地右移,比如说当 \( x1=0.64 \) 时,养老金缺口在 2035 年非常大,国家具有养老金偿付危机,但是降低 \( x1 \) 值后,从 0.64 依次降到 0.49 时,2035 年的养老金缺口越来越小,这样就会使得国家在 2035 年的养老金支付压力减少,提高偿付能力,使得国家的养老金支付危机延后,因此当国家在支付危机来临之际适当降低男性替代率。

接着考虑变动女性替代率,利用计算机模拟可得下图:

\begin{figure}[h]
    \centering
    \includegraphics[width=\textwidth]{image2.png}
    \caption{女性替代率对养老金缺口走势的影响}
    \label{fig:16}
\end{figure}

由上图可得,随着女性替代率降低,养老金缺口的波峰值逐渐右移,与男性替代率对养老金缺口走势有着同方向的影响,即:在国家养老金偿付危机来临之前适当的降低女性替代率可以使得国家可以顺利渡过危险期,或者使危险期延后。

然后考虑变动缴费率,利用 R 软件模拟得到下图:

\begin{figure}[h]
    \centering
    \includegraphics[width=\textwidth]{image3.png}
    \caption{缴费率对养老金缺口走势的影响}
    \label{fig:17}
\end{figure}

\begin{figure}[h]
    \centering
    \includegraphics[width=\textwidth]{image1.png}
    \caption{缴费率对养老金缺口走势的影响}
    \label{fig:17}
\end{figure}

由上图可得,随着缴费率的提高,养老金缺口的波峰值逐渐右移,即在国家养老金偿付危机来临之前适当的提高缴费率,可以帮助国家将危险期延后。

最后考虑工资调整系数,模拟得到下图:

\begin{figure}[h]
    \centering
    \includegraphics[width=\textwidth]{image2.png}
    \caption{工资调整系数对养老金缺口走势的影响}
    \label{fig:18}
\end{figure}

上图的工资调整系数分别为 \(0.95, 0.65, 0.45, 0\),由图可以看出随着工资调整系数的下降,养老金缺口的峰值往上移动,使得本来即将来临的危机顺利延后了。比如当工资调整系数为 \(0.95\) 时,国家在 2035 年的缺口很大,偿付危机应该将要来临,当工资调整系数下降时,随着峰值上移,2035 年的养老金缺口值也会不断缩小,这样就会不断缓解偿付压力,帮助国家度过危机。

综上所述,当国家即将出现养老金偿付危机时,在其他条件不变的情况下,适当地降低男女性替代率和工资调整系数,适时提高一定缴费率。

\section{模型评价}

\subsection{模型的优点}

1. 本文的全国养老基金收入测算模型以人口普查数据为基础。由于未来各期内影响养老基金收入的最重要因素是各期的人口数目,而本文利用 2010 全国第六次人口普查数据,以保险精算理论为基础,测算出未来各年我国各个年龄的男女人口数量,这些数据为养老金收入模型测算的准确性提供了坚实的保障。

2. 本文的城镇养老金支出模型既考虑了“老人”支出,又考虑“中人和新人”支出层次比较清晰,符合现实情况。并且,养老金缺口模型考虑了收支赤字累积与转轨成本两个因素,对中国社会而言,具有很强的适用性。

3. 本文的新农村的养老保险支出模型在参数估计上,对替代率进行了详细建模,并计算了不同年龄和不同缴费档次下的替代率水平,估计结果符合现实,并体现了国家“多缴多得,长缴多得”的政策。

4. 本文运用 R 软件,对城镇未来几年养老金的缺口进行了预测,并通过模拟的方法,调整了替代率和缴费率等因素,并指出了其他可变因素对于缺口模型的影响,对我国未来的养老金保险体系的可持续性发展具有重大的现实意义。

\subsection{模型的缺点}

1. 无论是城镇的养老金支出模型还是农村的养老金支出模型,都未考虑某些参保人员可能还未到达养老金领取年龄而死亡的情况,即对于这部分人员,应按照其已交的费用给予一定额度的一次性补贴。故在模型中,应增加这一块养老金的支出。

2. 本文的模型测算基于多个精算假设。其中,在死亡率、出生率、劳动参与率等指标上,我们都假设其固定不变。所以,在此后的研究应对每年的这几项指标进行调整以提高模型的精确度。

\section{参考文献}

[1] 庞洪涛, 北京市城镇养老金收支测算及灵敏度分析[D], 首都经济贸易大学硕士论文, 2009。

[2] 王晓军, 胡劲松等, 养老保险精算理论与实务[M], 北京: 中国劳动社会保险出版社, 2008。

[3] 钱冰冰, 企业职工养老金的缺口测算及改革措施[A], 大庆师范学院学报, 2012。

[4] 张士斌, 杨黎源, 张天龙, 养老金替代率的国际比较与中国改革路径[A], 浙江学刊, 2012。

[5] 张慧, 城乡一体化的多层次养老保险体系结构比例研究[D], 上海工程技术大学硕士论文, 2012。

[6] 张秋月, 借鉴国际经验建立中国多层次养老保险制度[D], 对外经济贸易大学硕士论文, 2002。

[7] 中国保监会, 养老保险国别研究及对中国的启示[M], 中国财政经济出版社, 北京, 2007。

[8] 邓大松, 薛惠元, 新型农社会养老保险替代率的测算与分析[A], 国民经济管理, 第 4 期, 2010。

[9] 徐颖, 王建梅, 对城镇基本养老保险制度设计替代率的评估分析[A], 人口与经济, 第 4 期, 2009。

\begin{enumerate}
    \item [10] 王晓军,对我国城镇职工基本养老保险制度收入替代率的定量模拟分析 [A],第 3 期,统计研究,2002。
    \item [11] 高建伟,邱苑华,企业补充养老保险计划精算模型 [A],系统工程理论与实践,第 5 期,2003。
    \item [12] 张宁,樊毅,企业年金替代率精算模型与实证测算 [A],企业管理,2010。
    \item [13] 刘晓颖,随机利率下缴费预定企业年金支付模型研究 [A],理论探究,2012。
    \item [14] 万英豪,基本养老保险基金收支缺口与财政对策研究——基于 H 地市的实证分析 [D],中南大学硕士论文,2010。
    \item [15] 夏心雄,中国现行养老金制度下的缺口分析及对策 [D],中南京财经大学硕士论文,2010。
    \item [16] 孙雅娜,边恕,穆怀中,中国养老保险最优缴费率的实证分析——基于贴现银子和劳动增长率差异的分析 [A],当代经济管理,第 7 期,2009。
    \item [17] 孙雅娜,安曼,中国养老保险最优缴费率研究——基于行业收入差异的分析 [A],河社会科学辑刊,第 2 期,2010。
    \item [18] 党耀国,刘思峰等,灰色预测与决策模型研究 [M],北京:科学出版社,2009。
    \item [19] 王泓,随社会保险基金的良性运营:系统动力学模型、方法、运用 [M],北京:北京大学出版社,2008。
    \item [20] 高铁梅等,计量经济分析方法与建模 [M],北京:清华大学出版社,2009。
    \item [21] 梁平,胡以涛,付小鹏,新型农村社会养老保险替代率测算方法与预测研究 [A],安徽农业科学,第 6 期,2012。
    \item [22] 郎茂祥等,预测理论方法与实务 [M],北京:北京交通大学大学出版社,2011。
    \item [23] 魏迎宁,养老保险国别研究及对中国的启示 [M],北京:中国财政经济出版社,2007。
    \item [24] 康晶,美国私人养老金制度对中国的启示 [D],武汉科技大学硕士论文,2009。
    \item [25] 杨勇刚,姜泽许,中国城镇养老保险支出水平测量模型分析 [A],河南大学学报,第 4 期,2010。
    \item [26] 江芹,通货膨胀对个人养老基金缺口的影响与实证分析 [D],山东大学硕士论文,2008。
\end{enumerate}

\section*{附录}

\section*{附录 1: R 软件运行程序}

\subsection*{程序 1 人口计算}
\begin{verbatim}
population <- function(data, rate, birth, i) {
  datanew <- data.frame()
  data_tmp <- data[16:50, 2]

  birth_s <- c(sum(data_tmp * birth / 1000) / 2.2 * 1.2, sum(data_tmp * birth / 1000) / 2.2)
  birth_e <- data * rate[, 1:2] * rate[, 3]
  datanew <- rbind(birth_s, birth_e)
  datanew <- datanew[-102, ]
  name <- paste(i, c('male', 'female'), sep='')
  names(datanew) <- name
  return(datanew)
}

dat <- read.table('tmp.txt', header=T)
rat <- read.table('rate.txt', header=T)
birt <- read.table('birth.txt', header=T)
datnew <- dat
for (i in 2011:2035) {
  dat <- population(dat, rat, birt, i)
  datnew <- cbind(datnew, dat)
}
write.csv(datnew, 'urban.csv')
\end{verbatim}

\subsection*{程序 2 养老金缺口计算}
\begin{verbatim}
outputld <- function(tmp2, r1, r2, gz, r3, qq) {
  # r3 是缴费率
  tim <- seq(2012, 2035)
  num1 <- tim - 2011
  mdata <- c()
  for (i in num1) {
    j <- i * 2 - 1
    data_tmp <- tmp2[, j:(j + 1)]
    data_m <- data_tmp[61:(74 + i), 1]
    data_f <- data_tmp[56:(69 + i), 2]
    output_m <- data_m * gz[i, ] * r1 * 0.8 / 1000000000
    output_f <- data_f * gz[i, ] * r2 * 0.8 / 1000000000
    mdata <- c(mdata, sum(output_m) + sum(output_f))
  }
  output_all <- mdata + qq[, 5]
  sumincome <- apply(qq[, 1:2], 1, function(x) prod(x) * r3 * 0.6 / 1000000000)
  a1 <- c(19497, sumincome - output_all)
  a2 <- cumsum(a1)
}
\end{verbatim}

\begin{verbatim}
a2 <- a2[-1]
a3 <- a2 - qq[, 4]
return(a3)
}
rr1 <- seq(0.4948, .68, by=.005)
rr2 <- seq(.4002, .5617, by=.005)
rr3 <- seq(.2334, .4436, by=.005)
num = 1:25
data7 <- data.frame()
for (r1 in rr1) {
    for (r2 in rr2) {
        for (r3 in rr3) {
            aa <- output(tmp2, r1, r2, gz, r3, qq)
            if (plu(aa) > 22) {
                data7 <- rbind(data7, c(r1, r2, r3))
            }
        }
    }
}
data7
aa <- output(tmp2, .68, .56, gz, .3, qq)
plot(aa)

des <- function(data) {
    datnew1 <- data[2] - data[1]
    datnew2 <- data[3] - data[2]
    if (datnew1 > 0 & datnew2 > 0) {
        output = TRUE
    } else {
        output = FALSE
    }
    return(output)
}

plu <- function(data) {
    aab <- 0
    for (i in 1:length(data)) {
        if (data[i] > 0) {
            aab <- aab + 1
        }
    }
    return(aab)
}

程序 3 乡村人口养老金支出
country <- function(data, tdl, gz2) {
    ot <- c()
    for (i in 1:26) {
        j <- i * 2 - 1
        data_tmp <- data[, j:(j + 1)]
        data_new <- apply(data_tmp, 1, sum)[61:nrow(data_tmp)]
    }
}
\end{verbatim}

\begin{verbatim}
data_b <- data_new[1:(i-1)]
data_a <- sum(data_new[i:41])
td <- apply(tdl, 1, sum)

if (i == 1) {
    ot <- c(ot, data_a * 660 * (1 + .7 * .0859)^(i - 1))
} else {
    td <- td[(46 - i):44]
    ot <- c(ot, data_a * 660 * (1 + .7 * .0859)^(i - 1) + sum(data_b * gz2[i, ] * td / 500 * (1 + .7 * .0859)^(i - 1)))
}
return(ot)
\end{verbatim}

\textbf{程序 4 画四维度对养老金缺口的影响}

\begin{verbatim}
par(mfrow = c(2, 2))
gz <- read.table('gz.txt', header = T)
x <- c(0.49, .45, .43)
y <- paste('x', 1:3, ' = ', x, sep = '')
x1 <- 2012:2035
plot(x1, outputold(tmpp, x[1], x[2], gz, x[3], qq), main = y, xlab = '年份', ylab = '养老金缺口', ylim = c(-200000, 200000))
\end{verbatim}

\begin{table}
\centering
\begin{tabular}{|c|c|c|c|c|}
\hline
\multirow{2}{*}{年龄} & \multicolumn{2}{c|}{非养老金业务表} & \multicolumn{2}{c|}{养老金业务表} \\
\cline{2-5}
 & 男(CL1) & 女(CL2) & 男(CL3) & 女(CL4) \\
\hline
0 & 0.000722 & 0.000661 & 0.000627 & 0.000575 \\
\hline
1 & 0.000603 & 0.000536 & 0.000525 & 0.000466 \\
\hline
2 & 0.000499 & 0.000424 & 0.000434 & 0.000369 \\
\hline
3 & 0.000416 & 0.000333 & 0.000362 & 0.000290 \\
\hline
4 & 0.000358 & 0.000267 & 0.000311 & 0.000232 \\
\hline
5 & 0.000323 & 0.000224 & 0.000281 & 0.000195 \\
\hline
6 & 0.000309 & 0.000201 & 0.000269 & 0.000175 \\
\hline
7 & 0.000308 & 0.000189 & 0.000268 & 0.000164 \\
\hline
8 & 0.000311 & 0.000181 & 0.000270 & 0.000158 \\
\hline
9 & 0.000312 & 0.000175 & 0.000271 & 0.000152 \\
\hline
10 & 0.000312 & 0.000169 & 0.000272 & 0.000147 \\
\hline
11 & 0.000312 & 0.000165 & 0.000271 & 0.000143 \\
\hline
12 & 0.000313 & 0.000165 & 0.000272 & 0.000143 \\
\hline
13 & 0.000320 & 0.000169 & 0.000278 & 0.000147 \\
\hline
14 & 0.000336 & 0.000179 & 0.000292 & 0.000156 \\
\hline
15 & 0.000364 & 0.000192 & 0.000316 & 0.000167 \\
\hline
16 & 0.000404 & 0.000208 & 0.000351 & 0.000181 \\
\hline
17 & 0.000455 & 0.000226 & 0.000396 & 0.000196 \\
\hline
18 & 0.000513 & 0.000245 & 0.000446 & 0.000213 \\
\hline
19 & 0.000572 & 0.000264 & 0.000497 & 0.000230 \\
\hline
20 & 0.000621 & 0.000283 & 0.000540 & 0.000246 \\
\hline
21 & 0.000661 & 0.000300 & 0.000575 & 0.000261 \\
\hline
22 & 0.000692 & 0.000315 & 0.000601 & 0.000274 \\
\hline
23 & 0.000716 & 0.000328 & 0.000623 & 0.000285 \\
\hline
24 & 0.000738 & 0.000338 & 0.000643 & 0.000293 \\
\hline
25 & 0.000759 & 0.000347 & 0.000660 & 0.000301 \\
\hline
26 & 0.000779 & 0.000355 & 0.000676 & 0.000308 \\
\hline
27 & 0.000795 & 0.000362 & 0.000693 & 0.000316 \\
\hline
28 & 0.000815 & 0.000372 & 0.000712 & 0.000325 \\
\hline
29 & 0.000842 & 0.000386 & 0.000734 & 0.000337 \\
\hline
30 & 0.000881 & 0.000406 & 0.000759 & 0.000351 \\
\hline
31 & 0.000932 & 0.000432 & 0.000788 & 0.000366 \\
\hline
32 & 0.000994 & 0.000465 & 0.000820 & 0.000384 \\
\hline
33 & 0.001055 & 0.000496 & 0.000855 & 0.000402 \\
\hline
\end{tabular}
\end{table}

\begin{table}
\centering
\begin{tabular}{|c|c|c|c|c|}
\hline
\multirow{2}{*}{年龄} & \multicolumn{2}{c|}{非养老金业务表} & \multicolumn{2}{c|}{养老金业务表} \\
\cline{2-5}
 & 男(CL1) & 女(CL2) & 男(CL3) & 女(CL4) \\
\hline
34 & 0.001121 & 0.000528 & 0.000893 & 0.000421 \\
\hline
35 & 0.001194 & 0.000563 & 0.000936 & 0.000441 \\
\hline
36 & 0.001275 & 0.000601 & 0.000985 & 0.000464 \\
\hline
37 & 0.001367 & 0.000646 & 0.001043 & 0.000493 \\
\hline
38 & 0.001472 & 0.000699 & 0.001111 & 0.000528 \\
\hline
39 & 0.001589 & 0.000761 & 0.001189 & 0.000569 \\
\hline
40 & 0.001715 & 0.000828 & 0.001275 & 0.000615 \\
\hline
41 & 0.001845 & 0.000897 & 0.001366 & 0.000664 \\
\hline
42 & 0.001978 & 0.000966 & 0.001461 & 0.000714 \\
\hline
43 & 0.002113 & 0.001033 & 0.001560 & 0.000763 \\
\hline
44 & 0.002255 & 0.001103 & 0.001665 & 0.000815 \\
\hline
45 & 0.002413 & 0.001181 & 0.001783 & 0.000873 \\
\hline
46 & 0.002595 & 0.001274 & 0.001918 & 0.000942 \\
\hline
47 & 0.002805 & 0.001389 & 0.002055 & 0.001014 \\
\hline
48 & 0.003042 & 0.001527 & 0.002238 & 0.001123 \\
\hline
49 & 0.003299 & 0.001690 & 0.002446 & 0.001251 \\
\hline
50 & 0.003570 & 0.001873 & 0.002666 & 0.001393 \\
\hline
51 & 0.003847 & 0.002074 & 0.002880 & 0.001548 \\
\hline
52 & 0.004132 & 0.002295 & 0.003085 & 0.001714 \\
\hline
53 & 0.004434 & 0.002546 & 0.003300 & 0.001893 \\
\hline
54 & 0.004778 & 0.002836 & 0.003545 & 0.002093 \\
\hline
55 & 0.005203 & 0.003178 & 0.003838 & 0.002318 \\
\hline
56 & 0.005744 & 0.003577 & 0.004207 & 0.002607 \\
\hline
57 & 0.006427 & 0.004036 & 0.004676 & 0.002979 \\
\hline
58 & 0.007260 & 0.004556 & 0.005275 & 0.003410 \\
\hline
59 & 0.008229 & 0.005133 & 0.006039 & 0.003816 \\
\hline
60 & 0.009313 & 0.005768 & 0.006989 & 0.004272 \\
\hline
61 & 0.010490 & 0.006465 & 0.007867 & 0.004781 \\
\hline
62 & 0.011747 & 0.007235 & 0.008725 & 0.005351 \\
\hline
63 & 0.013091 & 0.008094 & 0.009677 & 0.005988 \\
\hline
64 & 0.014542 & 0.009059 & 0.010731 & 0.006701 \\
\hline
65 & 0.016134 & 0.010148 & 0.011900 & 0.007499 \\
\hline
66 & 0.017905 & 0.011376 & 0.013229 & 0.008408 \\
\hline
67 & 0.019886 & 0.012760 & 0.014705 & 0.009438 \\
\hline
68 & 0.022103 & 0.014316 & 0.016344 & 0.010592 \\
\hline
69 & 0.024571 & 0.016066 & 0.018164 & 0.011886 \\
\hline
70 & 0.027309 & 0.018033 & 0.020184 & 0.013337 \\
\hline
71 & 0.030340 & 0.020241 & 0.022425 & 0.014964 \\
\hline
72 & 0.033684 & 0.022715 & 0.024911 & 0.016787 \\
\hline
\end{tabular}
\end{table}

\begin{table}
\centering
\begin{tabular}{|c|c|c|c|c|}
\hline
\multirow{2}{*}{年龄} & \multicolumn{2}{c|}{非养老金业务表} & \multicolumn{2}{c|}{养老金业务表} \\
\cline{2-5}
 & 男(CL1) & 女(CL2) & 男(CL3) & 女(CL4) \\
\hline
73 & 0.037371 & 0.025479 & 0.027668 & 0.018829 \\
\hline
74 & 0.041430 & 0.028561 & 0.030647 & 0.021117 \\
\hline
75 & 0.045902 & 0.031989 & 0.033939 & 0.023702 \\
\hline
76 & 0.050829 & 0.035796 & 0.037577 & 0.026491 \\
\hline
77 & 0.056262 & 0.040026 & 0.041594 & 0.029602 \\
\hline
78 & 0.062257 & 0.044726 & 0.046028 & 0.033070 \\
\hline
79 & 0.068871 & 0.049954 & 0.050920 & 0.036935 \\
\hline
80 & 0.076187 & 0.055774 & 0.056312 & 0.041241 \\
\hline
81 & 0.084224 & 0.062253 & 0.062253 & 0.046033 \\
\hline
82 & 0.093071 & 0.069494 & 0.068791 & 0.051365 \\
\hline
83 & 0.102800 & 0.077511 & 0.075983 & 0.057291 \\
\hline
84 & 0.113489 & 0.086415 & 0.083883 & 0.063872 \\
\hline
85 & 0.125221 & 0.096294 & 0.092554 & 0.071174 \\
\hline
86 & 0.138080 & 0.107243 & 0.102059 & 0.079267 \\
\hline
87 & 0.152157 & 0.119364 & 0.112464 & 0.088225 \\
\hline
88 & 0.167543 & 0.132763 & 0.123836 & 0.098129 \\
\hline
89 & 0.184333 & 0.147553 & 0.136246 & 0.109061 \\
\hline
90 & 0.202621 & 0.163850 & 0.149763 & 0.121107 \\
\hline
91 & 0.222500 & 0.181775 & 0.164456 & 0.134355 \\
\hline
92 & 0.244059 & 0.201447 & 0.180392 & 0.148896 \\
\hline
93 & 0.267383 & 0.222987 & 0.197631 & 0.164816 \\
\hline
94 & 0.292544 & 0.246507 & 0.216228 & 0.182201 \\
\hline
95 & 0.319604 & 0.272115 & 0.236229 & 0.201129 \\
\hline
96 & 0.348606 & 0.299903 & 0.257666 & 0.221667 \\
\hline
97 & 0.379572 & 0.329942 & 0.280553 & 0.243870 \\
\hline
98 & 0.412495 & 0.362281 & 0.304887 & 0.267773 \\
\hline
99 & 0.447334 & 0.396933 & 0.330638 & 0.293385 \\
\hline
100 & 0.484010 & 0.433869 & 0.357746 & 0.320685 \\
\hline
101 & 0.522397 & 0.473008 & 0.386119 & 0.349615 \\
\hline
102 & 0.562317 & 0.514211 & 0.415626 & 0.380069 \\
\hline
103 & 0.603539 & 0.557269 & 0.446094 & 0.411894 \\
\hline
104 & 0.645770 & 0.601896 & 0.477308 & 0.444879 \\
\hline
105 & 1.000000 & 1.000000 & 1.000000 & 1.000000 \\
\hline
\end{tabular}
\end{table}