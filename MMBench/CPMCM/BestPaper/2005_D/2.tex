\title{仓库容量有限条件下的随机存贮管理问题}
\author{}
\date{}

\documentclass{article}
\usepackage{amsmath}
\usepackage{tikz}

\begin{document}

\maketitle

\begin{abstract}
本文讨论了仓库容量有限条件下的随机存贮管理问题。首先建立了一个理论模型,根据题目要求写出平均费用的函数,该函数是关于订货点 \( L \) 和缺货天数 \( X \) 的函数,因为缺货天数 \( X \) 是一随机变量,这里给出了 \( X \) 为离散型和连续型两种模型,分三种情况讨论了各自的损失费用,然后得出期望平均费用函数 \( E[f(L, X)] \),经过 Maple 软件的辅助,对 \( L \) 进行求导,令 \( \frac{dE[f(L, X)]}{dL} = 0 \) 从而得出求解最优订货点 \( L^* \) 的方程。由于计算量太过庞大,所以在建立理论模型之后,本文还给出了一种比较实用的求解全局最优的遍历搜索算法,应用于问题 2 中求解出了三种商品各自的最佳订货点。经 SPSS 验证可知,问题 2 中给出的 3 种商品的缺货天数 \( X \) 不符合一些常见连续型的概率分布,故在求解时将 \( X \) 看做离散型的随机变量,采用频率替换的方法,计算出各自的 \( P(X) \),经遍历搜索找到最佳订货点。在问题 3 中,类似问题 1 的解法,求出计算最佳订货点 \( L_i^* \) 的理论模型,同时为便于问题 4 的求解,依然采用解决问题 2 的遍历算法,经 Matlab 编程求解得出问题 4 的最佳订货点 \( L^* \)、\( Q_{0i} \) 和 \( Q_i \)。问题 5 中假定商品的销售量是离散变化的,采用马氏链模型,在问题 2 的基础上补充假设条件,给出了单品种商品随机存贮模型,这一模型结合实际可以加以推广。

本文分为三个部分:第一部分是问题重述;第二部分是给出随机存贮问题的数学模型和一个搜索算法,通过计算机编程,应用于问题 2 和 4 中,从而给出了实际问题的最优解。第三部分为模型的评价和推广。

关键字:随机存贮,K-S 检验,频率替换,遍历搜索,马氏链模型
\end{abstract}


\section{问题重述}

工厂生产需定期地定购各种原料,商家销售要成批地购进各种商品。无论是原料或商品,都有一个怎样存贮的问题。存得少了无法满足需求,影响利润;存得太多,存贮费用就高。因此说存贮管理是降低成本、提高经济效益的有效途径和方法。本问题主要考虑的是在销售速率固定,交货时间为一随机变量的情况下,商场如何确定最优的订货点 $L^*$ 来使的其总的损失费用达到最低。

\section{数学建模和问题求解}

\subsection{问题 1}

\subsubsection{模型假设}

\begin{enumerate}
    \item 供货量充足
    \item 每次订货的时候存贮量 $q$ 刚好是 $L$。
\end{enumerate}

\subsubsection{符号说明}

\begin{itemize}
    \item $c_{1}$: 每次进货的订货费
    \item $c_{2}$: 使用自己的仓库存贮商品时,单位商品每天的存贮费用
    \item $c_{3}$: 使用租用的仓库存贮商品时,单位商品每天的存贮费用
    \item $c_{4}$: 缺货情况下,单位商品每天的损失费用
    \item $X$: 交货时间,是一个随机变量
    \item $Q_{0}$: 自己的仓库用于存贮商品的最大容量
    \item $Q$: 到货时商品的存贮量
    \item $q$: 商品的存贮量
    \item $L$: 订货点,即订货时商品所剩下的存贮量
    \item $L^*$: 最优订货点,即使总损失费用达到最低的订货点
    \item $f(L, X)$:一个订货周期内的平均总损失费用,是关于 $L$ 和 $X$ 的函数
    \item $p(X)$:交货时间的概率密度函数
\end{itemize}

\subsubsection{解题思路}

假设随机变量 $X$(交货时间)的密度函数已知为 $p(x)$。考虑某个周期内平均每天的损失费用设为 $g(L)$,求其最小时 $L$ 的取值 $L^*$。因为 $L \leq Q_0$ 和 $L > Q_0$ 时 $g(L)$ 的表达式是不同的,所以分三步来求 $L^*$。

\begin{enumerate}
    \item 当 $L \leq Q_0$ 时,求出 $L_1^* = L_1(p(x))$,使得 $g(L_1^*) = \min(g(L), \forall L \leq Q_0)$。
    \item 当 $L > Q_0$ 时,求出 $L_2^* = L_2(p(x))$,使得 $g(L_2^*) = \min(g(L), \forall L > Q_0)$ 最小。
    \item $L^* = \begin{cases} L_1^* & g(L_1^*) \leq g(L_2^*) \\ L_2^* & g(L_1^*) > g(L_2^*) \end{cases}$
\end{enumerate}

\subsubsection{具体步骤}

第一步:考虑 $L \leq Q_0$ 时一个周期内平均每天的损失费用 $f(L, X)$。因为在一个周期内可能会发生 $0 \leq X \leq \frac{L}{r}$(会发生缺货)或 $X > \frac{L}{r}$(不会缺货)的情况。分别考虑得到费用函数为:

\[
f(L, X) =
\begin{cases}
\frac{A - \frac{c_2 r}{2} \left(X - \frac{L}{r}\right)^2}{X + \frac{Q - L}{r}} & 0 \leq X \leq \frac{L}{r} \\
\frac{A + c_4 r \left(X - \frac{L}{r}\right)}{X + \frac{Q - L}{r}} & X > \frac{L}{r}
\end{cases}
\]

其中 $A = c_1 + \frac{c_2 Q^2}{2r} + \frac{(c_3 - c_2)(Q - Q_0)^2}{2r}$

现在来具体解释 $f(L, X)$:

\begin{figure}[h]
    \centering
    \begin{tikzpicture}[scale=0.8]
        \draw[->] (0,0) -- (12,0) node[right] {时间};
        \draw[->] (0,0) -- (0,8) node[above] {存贮量};

        \draw[dashed] (0,7) -- (8,7) node[above] {$Q$};
        \draw[dashed] (0,6) -- (6,6) node[above] {$Q_0$};
        \draw[dashed] (0,5) -- (4,5) node[above] {$L$};

        \draw (0,7) -- (4,5) -- (6,6) -- (8,7);

        \draw[dashed] (4,0) -- (4,5) node[below] {$a = \frac{Q-L}{r}$};
        \draw[dashed] (6,0) -- (6,6) node[below] {$a+X$};
        \draw[dashed] (8,0) -- (8,7) node[below] {$\frac{Q}{r}$};

        \node at (2,6.5) {C};
        \node at (5,5.5) {d};

        \node at (11,7.5) {$L \leq Q_0 \quad 0 \leq X \leq \frac{L}{r}$};
        \node at (11,0.5) {$2a+X$};

        \node at (1,0.5) {$0$};

        \node at (6,8) {图 1.1};
    \end{tikzpicture}
    \caption{当 $0 \leq X \leq \frac{L}{r}$ 时}
\end{figure}

见图 1.1:设在 0 时刻,存贮量为 $Q$;在 $a$ 时刻存贮量降为 $L$,开始订货(易知 $a = \frac{Q-L}{r}$)。在 $X+a$ 时刻货到,在 $X+2a$ 时刻存贮量又降为 $L$。以 $[a \quad X+2a]$ 为一个周期,求其费用。由图易知,$[a \quad X+2a]$ 周期内的费用等于 $[0 \quad a+X]$ 的费用,所以只需求 $[0 \quad a+X]$ 的费用,其值为 $c_1 + c_3 \times S(Q, Q_0, c) + c_2 \times S(0, Q_0, c, a+X)$(注:$S(Q, Q_0, c)$ 表示 $(Q, Q_0, c)$ 围成的面积)。易证

\begin{align*}
c_1 + c_3 \times S(Q, Q_0, c) + c_2 \times S(0, Q_0, c, a+X) &= \\
c_1 + c_2 \times S(0, Q, \frac{Q}{r}) + (c_3 - c_2) \times S(Q, Q_0, c) - c_2 \times S(d, a+X, \frac{Q}{r}) &= \\
c_1 + c_2 \int_{0}^{\frac{Q}{r}} r t dt + (c_3 - c_2) \int_{0}^{\frac{Q-Q_0}{r}} r t dt - c_2 \int_{0}^{\frac{L-X}{r}} r t dt &= \\
A - \frac{c_2 r}{2} \left( X - \frac{L}{r} \right)^2
\end{align*}

所以这个周期内的平均每天的费用为 $\frac{A - \frac{c_2 r}{2} \left( X - \frac{L}{r} \right)^2}{X + \frac{Q-L}{r}}$。

(2) 当 $X > \frac{L}{r}$ 时

\begin{figure}[h]
    \centering
    \includegraphics[width=\textwidth]{image.png}
    \caption{图 1.2}
    \label{fig:1.2}
\end{figure}

见图 \ref{fig:1.2},考虑 $[0, a+X]$ 之间的费用,其值为

\begin{align*}
& c_1 + c_3 \times S(Q, Q_0, c) + c_2 \times S(0, Q_0, c, \frac{Q}{r}) + c_4 r (X - \frac{L}{r}) \text{。易证:} \\
& c_1 + c_3 \times S(Q, Q_0, c) + c_2 \times S(0, Q_0, c, \frac{Q}{r}) + c_4 r (X - \frac{L}{r}) = \\
& c_1 + c_2 \times S(0, Q, \frac{Q}{r}) + (c_3 - c_2) \times S(Q, Q_0, c) + c_4 r (X - \frac{L}{r}) = A + c_4 r (X - \frac{L}{r})
\end{align*}

所以这个周期内的平均每天的费用为

\[
\frac{A + c_4 r (X - \frac{L}{r})}{X + \frac{Q - L}{r}}
\]

要求 $L^*$ 使得平均每天费用的期望值最小,即满足:

\[
E[f(L_1^*, X)] = \min E[f(L, X)] \quad (\forall L \leq Q_0)
\]

(1) 若 $X$ 为离散型:

\[
E[f(L, X)] = \sum_i f(L, x_i) P(x_i), \text{这是一个关于 $L$ 的函数,在 $P(x_i)$ 已知的情况下,}
\]

令 $\frac{dE[f(L, X)]}{dL} = 0$,可求得 $L_1^*$

(2) 若 $X$ 为连续型则

\[
E[f(L, X)] = \int_0^\infty f(L, x) p(x) dx = \int_0^{\frac{L}{r}} f(L, x) p(x) dx + \int_{\frac{L}{r}}^\infty f(L, x) p(x) dx
\]

\begin{align*}
&= \int_{0}^{\frac{L}{r}} \left[ \frac{A - \frac{c_2 r}{2} \left( x - \frac{L}{r} \right)^2}{x + \frac{Q - L}{r}} \right] p(x) dx + \int_{\frac{L}{r}}^{\infty} \left[ \frac{A + c_4 r \left( x - \frac{L}{r} \right)}{x + \frac{Q - L}{r}} \right] p(x) dx
\end{align*}

这是一个关于 \( L \) 的函数,令 \(\frac{dE[f(L, X)]}{dL} = 0\),即可求得 \( L_1^* \)。

通过 Maple 求 \(\frac{dE[f(L, X)]}{dL}\) 有

\begin{align*}
\frac{dE[f(L, X)]}{dL} &= \int_{0}^{\frac{L}{r}} \frac{c_2 \left( x - \frac{L}{r} \right) p(x)}{x + \frac{Q - L}{r}} + \frac{\left[ c_1 + \frac{c_2 Q^2}{2r} + \frac{(c_3 - c_2)(Q - Q_0)^2}{2r} - \frac{1}{2} c_2 r \left( x - \frac{L}{r} \right)^2 \right] p(x)}{\left[ x + \frac{Q - L}{r} \right]^2 r} dx \\
&+ \int_{\frac{L}{r}}^{\infty} \frac{- \frac{c_4 p(x)}{x + \frac{Q - L}{r}} + \frac{\left[ c_1 + \frac{c_2 Q^2}{2r} + \frac{(c_3 - c_2)(Q - Q_0)^2}{2r} + c_4 r \left( x - \frac{L}{r} \right) \right] p(x)}{(x + \frac{Q - L}{r})^2 r}}{x + \frac{Q - L}{r}} dx
\tag{1.1}
\end{align*}

则最优订货点 \( L_1^* \) 就是代入 (1.1) 中使之值为 0 的 \( L \) 值。

第二步:考虑 \( L > Q_0 \) 时一个周期内平均每天的费用 \( f(L, X) \)。

\begin{align*}
c_1 + c_3 \times S(Q, Q_0, c) + c_2 \times S(0, Q_0, c, a+X) &= \\
c_1 + c_2 \times S(0, Q, \frac{Q}{r}) + (c_3 - c_2) \times S(Q, Q_0, c) - c_2 \times S(d, a+X, \frac{Q}{r}) &= \\
c_1 + c_2 \int_{0}^{\frac{Q}{r}} r t dt + (c_3 - c_2) \int_{0}^{\frac{Q-Q_0}{r}} r t dt - c_2 \int_{0}^{\frac{L-X}{r}} r t dt &= \\
A - \frac{c_2 r}{2} \left( X - \frac{L}{r} \right)^2
\end{align*}

其中 \( A = c_1 + \frac{c_2 Q^2}{2r} + \frac{(c_3 - c_2)(Q - Q_0)^2}{2r} \)

现在来具体解释 \( f(L, X) \):

\begin{figure}[h]
    \centering
    \begin{tikzpicture}[scale=0.8]
        \draw[->] (0,0) -- (12,0) node[right] {时间};
        \draw[->] (0,0) -- (0,8) node[above] {存贮量};
        \draw[dashed] (0,6) -- (10,6) node[right] {$Q$};
        \draw[dashed] (0,4) -- (10,4) node[right] {$L$};
        \draw[dashed] (0,2) -- (10,2) node[right] {$Q_0$};
        \draw[dashed] (2,0) -- (2,8);
        \draw[dashed] (6,0) -- (6,8);
        \draw[dashed] (8,0) -- (8,8);
        \draw (2,6) -- (6,2) -- (8,4) -- (10,6);
        \node at (1.8,0) [below] {$0$};
        \node at (2.2,0) [below] {$a=\frac{Q-L}{r}$};
        \node at (6,0) [below] {$a+X$};
        \node at (8,0) [below] {$2a+X$};
        \node at (10,0) [below] {$\frac{Q}{r}$};
        \node at (0,6) [left] {$Q$};
        \node at (0,4) [left] {$L$};
        \node at (0,2) [left] {$Q_0$};
        \node at (5,5) [above] {e};
        \node at (7,3) [above] {d};
        \node at (9,5) [above] {c};
        \node at (11.5,7) {$L>Q_0 \quad 0\leq X<\frac{L-Q_0}{r}$};
        \node at (6,7) {图 1.3};
    \end{tikzpicture}
    \caption{存贮量与时间关系图}
\end{figure}

见图 1.3,同理考虑 $[0 \quad a+X]$ 的费用,其值为

\begin{align*}
& c_1 + c_3 \times S(Q, Q_0, c) + c_2 \times S(0, Q_0, c, \frac{Q}{r}) + c_4 r (X - \frac{L}{r}) \text{。易证:} \\
& c_1 + c_3 \times S(Q, Q_0, c) + c_2 \times S(0, Q_0, c, \frac{Q}{r}) + c_4 r (X - \frac{L}{r}) = \\
& c_1 + c_2 \times S(0, Q, \frac{Q}{r}) + (c_3 - c_2) \times S(Q, Q_0, c) + c_4 r (X - \frac{L}{r}) = A + c_4 r (X - \frac{L}{r})
\end{align*}

易证:

\begin{align*}
c_1 + c_3 \times S(Q, Q_0, c) + c_2 \times S(0, Q_0, c, a+X) &= \\
c_1 + c_2 \times S(0, Q, \frac{Q}{r}) + (c_3 - c_2) \times S(Q, Q_0, c) - c_2 \times S(d, a+X, \frac{Q}{r}) &= \\
c_1 + c_2 \int_{0}^{\frac{Q}{r}} r t dt + (c_3 - c_2) \int_{0}^{\frac{Q-Q_0}{r}} r t dt - c_2 \int_{0}^{\frac{L-X}{r}} r t dt &= \\
A - \frac{c_2 r}{2} \left( X - \frac{L}{r} \right)^2
\end{align*}

所以这个周期内的平均每天的费用为

\begin{align*}
&= \int_{0}^{\frac{L}{r}} \left[ \frac{A - \frac{c_2 r}{2} \left( x - \frac{L}{r} \right)^2}{x + \frac{Q - L}{r}} \right] p(x) dx + \int_{\frac{L}{r}}^{\infty} \left[ \frac{A + c_4 r \left( x - \frac{L}{r} \right)}{x + \frac{Q - L}{r}} \right] p(x) dx
\end{align*}

(2) 当 $\frac{L-Q_0}{r} < X \leq \frac{L}{r}$ 和 $X > \frac{L}{r}$ 时候的情况,参看图 1.4 和 1.5,类似于第一步里的情形,即可得到其平均每天的费用函数。

\begin{tikzpicture}[scale=0.8]
    % 图 1.4
    \draw[->] (0,0) -- (12,0) node[right] {时间};
    \draw[->] (0,0) -- (0,8) node[above] {存贮量};
    \draw[dashed] (0,7) -- (10,7) node[above] {Q};
    \draw[dashed] (0,6) -- (10,6) node[above] {L};
    \draw[dashed] (0,5) -- (10,5) node[above] {Q$_{0}$};
    \draw (2,7) -- (2,6) -- (8,6) -- (8,7) -- cycle;
    \draw[dashed] (2,7) -- (8,5);
    \node at (5,6.5) {C};
    \node at (8,5.5) {d};
    \node at (1,0) {0};
    \node at (4,0) {$a=\frac{Q-L}{r}$};
    \node at (6,0) {a+X};
    \node at (8,0) {2a+X};
    \node at (10,0) {$\frac{Q}{r}$};
    \node at (11,7) {$L>Q_{0}$};
    \node at (14,7) {$\frac{L-Q_{0}}{r}\leq X\leq\frac{L}{r}$};
    \node at (6,8) {图 1.4};

    % 图 1.5
    \draw[->] (0,-10) -- (12,-10) node[right] {时间};
    \draw[->] (0,-10) -- (0,-2) node[above] {存贮量};
    \draw[dashed] (0,-1) -- (10,-1) node[above] {Q};
    \draw[dashed] (0,-2) -- (10,-2) node[above] {L};
    \draw[dashed] (0,-3) -- (10,-3) node[above] {Q$_{0}$};
    \draw (2,-1) -- (2,-2) -- (8,-2) -- (8,-1) -- cycle;
    \draw[dashed] (2,-1) -- (8,-3);
    \node at (5,-1.5) {C};
    \node at (1, -12) {0};
    \node at (4, -12) {$a=\frac{Q-L}{r}$};
    \node at (6, -12) {$\frac{Q}{r}$};
    \node at (8, -12) {a+X};
    \node at (10, -12) {2a+X};
    \node at (11, -1) {$L\leq Q_{0}$};
    \node at (14, -1) {$X>\frac{L}{r}$};
    \node at (6, -0.5) {图 1.5};
\end{tikzpicture}

要求 $L_{2}^{*}$ 使得平均每天费用的期望值最小,即要满足:
\begin{align*}
\frac{dE[f(L, X)]}{dL} &= \int_{0}^{\frac{L}{r}} \frac{c_2 \left( x - \frac{L}{r} \right) p(x)}{x + \frac{Q - L}{r}} + \frac{\left[ c_1 + \frac{c_2 Q^2}{2r} + \frac{(c_3 - c_2)(Q - Q_0)^2}{2r} - \frac{1}{2} c_2 r \left( x - \frac{L}{r} \right)^2 \right] p(x)}{\left[ x + \frac{Q - L}{r} \right]^2 r} dx \\
&+ \int_{\frac{L}{r}}^{\infty} \frac{- \frac{c_4 p(x)}{x + \frac{Q - L}{r}} + \frac{\left[ c_1 + \frac{c_2 Q^2}{2r} + \frac{(c_3 - c_2)(Q - Q_0)^2}{2r} + c_4 r \left( x - \frac{L}{r} \right) \right] p(x)}{(x + \frac{Q - L}{r})^2 r}}{x + \frac{Q - L}{r}} dx
\tag{1.1}
\end{align*}

(1) 若 $X$ 为离散型:
\begin{align*}
E[f(L, X)] &= \int_{0}^{\infty} f(L, x) p(x) dx \\
&= \int_{0}^{\frac{L - Q_0}{r}} f(L, x) p(x) dx + \int_{\frac{L - Q_0}{r}}^{\frac{L}{r}} f(L, x) p(x) dx + \int_{\frac{L}{r}}^{\infty} f(L, x) p(x) dx \\
&= \int_{0}^{\frac{L - Q_0}{r}} \frac{A - \frac{c_2 r}{2} \left( x - \frac{L}{r} \right)^2 - \frac{(c_3 - c_2) r}{2} \left( \frac{L - Q_0}{r} - x \right)^2}{x + \frac{Q - L}{r}} p(x) dx \\
&\quad + \int_{\frac{L - Q_0}{r}}^{\frac{L}{r}} \frac{A - \frac{c_2 r}{2} \left( x - \frac{L}{r} \right)^2}{x + \frac{Q - L}{r}} p(x) dx + \int_{\frac{L}{r}}^{\infty} \frac{A + c_4 \left( x - \frac{L}{r} \right)}{x + \frac{Q - L}{r}} p(x) dx
\end{align*}
这是一个关于 $L$ 的函数,在 $P(x_{i})$ 已知的情况下,令
\begin{align*}
\frac{dE[f(L, X)]}{dL} &= \int_{0}^{\frac{L - Q_0}{r}} \frac{\left[ c_2 \left( x - \frac{L}{r} \right) - (c_3 - c_2) \left( \frac{L - Q_0}{r} - x \right) \right] p(x)}{x + \frac{Q - L}{r}} dx \\
&\quad + \int_{0}^{\frac{L - Q_0}{r}} \frac{\left[ c_1 + \frac{c_2 Q^2}{2r} + \frac{(c_3 - c_2)(Q - Q_0)^2}{2r} - \frac{1}{2} c_2 r \left( x - \frac{L}{r} \right)^2 - \frac{1}{2} (c_3 - c_2) r \left( \frac{L - Q_0}{r} - x \right)^2 \right] p(x)}{\left[ x + \frac{Q - L}{r} \right]^2 r} dx \\
&\quad + \int_{\frac{L - Q_0}{r}}^{\frac{L}{r}} \frac{c_2 \left( x - \frac{L}{r} \right) p(x)}{x + \frac{Q - L}{r}} dx + \int_{\frac{L - Q_0}{r}}^{\frac{L}{r}} \frac{\left[ c_1 + \frac{c_2 Q^2}{2r} + \frac{(c_3 - c_2)(Q - Q_0)^2}{2r} - \frac{1}{2} c_2 r \left( x - \frac{L}{r} \right)^2 \right] p(x)}{\left[ x + \frac{Q - L}{r} \right]^2 r} dx \\
&\quad + \int_{\frac{L}{r}}^{\infty} -\frac{c_4 p(x)}{x + \frac{Q - L}{r}} dx + \int_{\frac{L}{r}}^{\infty} \frac{\left[ c_1 + \frac{c_2 Q^2}{2r} + \frac{(c_3 - c_2)(Q - Q_0)^2}{2r} + c_4 r \left( x - \frac{L}{r} \right) \right] p(x)}{\left( x + \frac{Q - L}{r} \right)^2 r} dx \tag{1.2}
\end{align*}
可求得 $L_{2}^{*}$。

(2) 若 $X$ 为连续型:

\begin{align*}
E[f(L, X)] &= \int_{0}^{\infty} f(L, x) p(x) dx \\
&= \int_{0}^{\frac{L - Q_0}{r}} f(L, x) p(x) dx + \int_{\frac{L - Q_0}{r}}^{\frac{L}{r}} f(L, x) p(x) dx + \int_{\frac{L}{r}}^{\infty} f(L, x) p(x) dx \\
&= \int_{0}^{\frac{L - Q_0}{r}} \frac{A - \frac{c_2 r}{2} \left( x - \frac{L}{r} \right)^2 - \frac{(c_3 - c_2) r}{2} \left( \frac{L - Q_0}{r} - x \right)^2}{x + \frac{Q - L}{r}} p(x) dx \\
&\quad + \int_{\frac{L - Q_0}{r}}^{\frac{L}{r}} \frac{A - \frac{c_2 r}{2} \left( x - \frac{L}{r} \right)^2}{x + \frac{Q - L}{r}} p(x) dx + \int_{\frac{L}{r}}^{\infty} \frac{A + c_4 \left( x - \frac{L}{r} \right)}{x + \frac{Q - L}{r}} p(x) dx
\end{align*}

这是一个关于 \( L \) 的函数。

令 \(\frac{dE[f(L, X)]}{dL} = 0\),即可求得 \( L_2^* \)。

通过 Maple 求 \(\frac{dE[f(L, X)]}{dL}\) 有

\begin{align*}
\frac{dE[f(L, X)]}{dL} &= \int_{0}^{\frac{L - Q_0}{r}} \frac{\left[ c_2 \left( x - \frac{L}{r} \right) - (c_3 - c_2) \left( \frac{L - Q_0}{r} - x \right) \right] p(x)}{x + \frac{Q - L}{r}} dx \\
&\quad + \int_{0}^{\frac{L - Q_0}{r}} \frac{\left[ c_1 + \frac{c_2 Q^2}{2r} + \frac{(c_3 - c_2)(Q - Q_0)^2}{2r} - \frac{1}{2} c_2 r \left( x - \frac{L}{r} \right)^2 - \frac{1}{2} (c_3 - c_2) r \left( \frac{L - Q_0}{r} - x \right)^2 \right] p(x)}{\left[ x + \frac{Q - L}{r} \right]^2 r} dx \\
&\quad + \int_{\frac{L - Q_0}{r}}^{\frac{L}{r}} \frac{c_2 \left( x - \frac{L}{r} \right) p(x)}{x + \frac{Q - L}{r}} dx + \int_{\frac{L - Q_0}{r}}^{\frac{L}{r}} \frac{\left[ c_1 + \frac{c_2 Q^2}{2r} + \frac{(c_3 - c_2)(Q - Q_0)^2}{2r} - \frac{1}{2} c_2 r \left( x - \frac{L}{r} \right)^2 \right] p(x)}{\left[ x + \frac{Q - L}{r} \right]^2 r} dx \\
&\quad + \int_{\frac{L}{r}}^{\infty} -\frac{c_4 p(x)}{x + \frac{Q - L}{r}} dx + \int_{\frac{L}{r}}^{\infty} \frac{\left[ c_1 + \frac{c_2 Q^2}{2r} + \frac{(c_3 - c_2)(Q - Q_0)^2}{2r} + c_4 r \left( x - \frac{L}{r} \right) \right] p(x)}{\left( x + \frac{Q - L}{r} \right)^2 r} dx \tag{1.2}
\end{align*}

则最优订货点 \( L_2^* \) 就是代入 (1.2) 中使之值为 0 的 \( L \) 值。

第三步:
\[
L^{*} =
\begin{cases}
L_{1}^{*} & E[f(L_{1}^{*}, X)] \leq E[f(L_{2}^{*}, X)] \\
L_{2}^{*} & E[f(L_{1}^{*}, X)] > E[f(L_{2}^{*}, X)]
\end{cases}
\]

\subsection{问题 2}

\subsubsection{缺货天数的数据分析}

这里首先要找出缺货天数 \( X \) 服从何种分布,下面根据题目给出的到货天数的记录判断 \( X \) 服从的分布。

我们采用单样本的 K-S 检验,它是一种拟合优度的非参数检验方法,利用样本的数据推断总体是否服从某一理论分布,它适用于探索连续型随机变量的分布形态。它可以将一个变量的实际频数分布与正态分布、均匀分布和指数分布进行比较。

这里以商品三的缺货天数为例来检验 \( X \) 是否服从上述三种连续型分布。

\begin{table}
\begin{tabular}{|l|l|}
\hline
 & VAR00002 \\
\hline
N & 61 \\
Normal Parameters\footnotemark[1] & Mean \\
 & 1.9508 \\
 & Std. Deviation \\
 & 1.16084 \\
Most Extreme & Absolute \\
Differences & .254 \\
 & Positive \\
 & .254 \\
 & Negative \\
 & -.206 \\
Kolmogorov-Smirnov Z & 1.981 \\
Asymp. Sig. (2-tailed) & .001 \\
\hline
\end{tabular}
\caption{One-Sample Kolmogorov-Smirnov Test}
\end{table}

\footnotetext[1]{a. Test distribution is Normal.}
\footnotetext[1]{b. Calculated from data.}

\begin{table}
\begin{tabular}{|l|l|}
\hline
 & VAR00002 \\
\hline
N & 61 \\
Uniform Parameters a, b & Minimum 1.00 \\
 & Maximum 6.00 \\
Most Extreme & Absolute .570 \\
Differences & Positive .570 \\
 & Negative -.016 \\
Kolmogorov-Smirnov Z & 4.456 \\
Asymp. Sig. (2-tailed) & .000 \\
\hline
\end{tabular}
\caption{One-Sample Kolmogorov-Smirnov Test 2}
\end{table}
\footnotetext[1]{a. Test distribution is Uniform.}
\footnotetext[1]{b. Calculated from data.}

\begin{table}
\begin{tabular}{|l|l|}
\hline
 & VAR00002 \\
\hline
N & 61 \\
Exponential parameter. a, b Mean & 1.9508 \\
Most Extreme & Absolute .401 \\
Differences & Positive .129 \\
 & Negative -.401 \\
Kolmogorov-Smirnov Z & 3.132 \\
Asymp. Sig. (2-tailed) & .000 \\
\hline
\end{tabular}
\caption{One-Sample Kolmogorov-Smirnov Test 3}
\end{table}
\footnotetext[1]{a. Test Distribution is Exponential.}
\footnotetext[1]{b. Calculated from data.}

上面三张表分别体现的是检验缺货天数 \( X \) 是否服从正态分布、均匀分布和指数分布,上述三表中的相伴概率值分别为 0.001,0.000 和 0.000,均小于 0.05,说明 \( X \) 不服从上述三种连续型分布。

因此在这里我们就假定 \( X \) 不是连续型的随机变量,另一方面,当我们假定 \( X \) 是离散型的随机变量时,可以大大简化实际问题的求解难度。

\subsubsection{程序实现}

考虑到问题 2 中 \( L \) 是正整数,当 \( 0 < L \leq Q_0 \),则 \( L \) 可能的取值最多为 \( Q_0 \) 个,则对每个 \( L \) 的可能取值求其 \( E[f(L, X)] \),计算 \( Q_0 \) 次就可得到 \( L_1^* \)。这比 \( \frac{dE[f(L, X)]}{dL} = 0 \),来求得 \( L_1^* \) 更方便。

具体流程如下图:

\begin{figure}[h]
    \centering
    \includegraphics[width=\textwidth]{flowchart.png}
    \caption{流程图}
\end{figure}

其中 Lmin1 用来存储 $L_{1}^{*}$ 的值,Emin1 用来存储 $E[f(L_{1}^{*},X)]$ 的值;Lmin2 用来存储 $L_{2}^{*}$ 的值,Emin2 用来存储 $E[f(L_{2}^{*},X)]$ 的值。

对于商品一,我们通过连续的 36 次订货到达时间天数的数据得到:

\begin{tabular}{|l|l|}
\hline
 & VAR00002 \\
\hline
N & 61 \\
Normal Parameters\footnotemark[1] & Mean \\
 & 1.9508 \\
 & Std. Deviation \\
 & 1.16084 \\
Most Extreme & Absolute \\
Differences & .254 \\
 & Positive \\
 & .254 \\
 & Negative \\
 & -.206 \\
Kolmogorov-Smirnov Z & 1.981 \\
Asymp. Sig. (2-tailed) & .001 \\
\hline
\end{tabular}

通过上述算法我们得出 $\min E[f(L,X)]=3.3878$,$L^{*}=36$。

对于商品二,我们通过连续的 43 次订货到达时间天数的数据得到:

\begin{tabular}{|l|l|}
\hline
 & VAR00002 \\
\hline
N & 61 \\
Uniform Parameters a, b & Minimum 1.00 \\
 & Maximum 6.00 \\
Most Extreme & Absolute .570 \\
Differences & Positive .570 \\
 & Negative -.016 \\
Kolmogorov-Smirnov Z & 4.456 \\
Asymp. Sig. (2-tailed) & .000 \\
\hline
\end{tabular}

通过上述算法我们得出 $\min E[f(L, X)] = 4.7353$,$L^* = 45$。

对于商品三,我们通过连续的 61 次订货到达时间天数的数据得到:

\begin{tabular}{|c|c|c|c|c|c|c|}
\hline
到达天数 $X$ & 1 & 2 & 3 & 4 & 5 & 6 \\
\hline
出现频率 $P(X)$ & $\frac{27}{61}$ & $\frac{20}{61}$ & $\frac{8}{61}$ & $\frac{3}{61}$ & $\frac{2}{61}$ & $\frac{1}{61}$ \\
\hline
\end{tabular}

通过上述算法我们得出 $\min E[f(L, X)] = 10.55$,$L^* = 34$。

\subsection{问题 3}

\subsubsection{模型假设}

\begin{enumerate}
    \item 各种商品的订货时间是相同的,而且到货时间也是相同的;
    \item 供货量充足,保证到货后商品存贮体积能够补充到固定值 $Q$,同时各种商品的存贮体积都能补充到各自的固定值 $Q_i$;
    \item 商品的每天的销售是一个连续的状态,而不考虑成离散的状态。
\end{enumerate}

\subsubsection{符号说明}

\begin{itemize}
    \item $m$:商品的种类数
    \item $r_i$:第 $i$ 种商品的销售速率
    \item $c_1$:每次进货的订货费,与商品的数量和品种无关
    \item $c_{2i}$:第 $i$ 种商品在使用自己的仓库存贮商品时,单位体积商品每天的存贮费用
    \item $c_{3i}$:第 $i$ 种商品在使用租用的仓库存贮商品时,单位体积商品每天的存贮费用
    \item $c_{4i}$:第 $i$ 种商品在缺货情况下,单位体积的商品每天的损失费用
    \item $X$:到货时间,是一个随机变量,$m$ 种商品的交货时间都相同
    \item $Q_{0i}$:第 $i$ 种商品自己的仓库的体积容量
    \item $Q_i$:第 $i$ 种商品到货时存贮量补充到的固定体积
    \item $Q_0$:自己的仓库用于存贮 $m$ 种商品的总体积容量
    \item $Q$: $m$ 种商品到货时存贮量补充到的总固定体积
    \item $q$: $m$ 种商品的存贮量总体积
    \item $L$: 订货点,即 $m$ 种商品订货时商品所剩下的存贮量总体积
    \item $L^*$: 最优订货点,即使总损失费用达到最低的订货点
    \item $L_i^*$: 总损失费用达到最低时第 $i$ 种商品的订货点,且有 $L^* = \sum_{i=1}^m L_i^*$
    \item $f_i(L_i, X)$: 第 $i$ 种商品在一个订货周期内的平均总损失费用,是关于 $L_i$ 和 $X$ 的函数
    \item $f(L, X)$: $m$ 种商品在一个订货周期内的平均总损失费用,有 $f(L, X) = \sum_{i=1}^m f_i(L, X)$
    \item $p(x)$: 到货时间 $X$ 的概率密度函数
    \item $E[f(L, X)]$: $m$ 种商品在一个订货周期内的平均总损失费用的期望
\end{itemize}

\subsubsection{模型建立}

问题 3 是讨论多种商品需要订货的问题。但由于所有商品每次订货都是在同一时间从同一供应站订货,所以可以以问题 1 中的单一商品存贮模型为基础,建立多种商品的随机存贮模型。但必须注意,问题 1 模型的考虑是以商品的数量为基准的,而问题 3 的模型是以商品的体积为基准考虑。

先说明当 $L$ 固定,$Q_i$ 固定时候,任意次当存贮量 $q$ 降到 $L$ 时即任意次开始订货的时候,此时每种商品的存贮量 $L_i$ 是固定不变的。

解释如下:

\begin{figure}[h]
    \centering
    \includegraphics[width=\textwidth]{image.png}
    \caption{图 3.1}
\end{figure}

见图 3.1:设 $t=0$ 时所有商品的存贮量为 $Q_{i}$,经过 $k$ 时间,总商品存贮量首次降到 $L$,此时各商品的存储量为 $L_{i}=\max(-r_{i}k+Q_{i},0)$,开始订货。过了 $x$ 时间,货到,各商品的存贮量又为 $Q_{i}$,则以 $x$ 时刻为起点,经过 $k$ 时间总商品的存贮量将首次到达 $L$,此时各商品的存储量为 $L_{i}$,由图易知对于同一商品 $L_{i}$ 不变。

因为各种商品的订货时间是相同则可得到:
\begin{equation}
\begin{cases}
\frac{Q_{1}-L_{1}}{v_{1}r_{1}}=\frac{Q_{2}-L_{2}}{v_{2}r_{2}}=\ldots=\frac{Q_{i}-L_{i}}{v_{i}r_{i}}=k & (\text{当 } L_{i}>0 \text{ 时}) \\
\frac{Q_{i}-L_{i}}{v_{i}r_{i}}\leq k & (\text{当 } L_{i}=0 \text{ 时})
\end{cases}
\end{equation}

所以
\begin{equation}
k=\max\left(\frac{Q_{i}-L_{i}}{r_{i}}\right) \quad 1\leq i\leq m
\end{equation}

针对 $L_{i}$ 的不同取值范围,分三情况进行讨论。

情况 1:当 $L_{i}=0$ 时,其费用 $f_{i}(L_{i},X)$ 函数如下:
\begin{equation}
f_{i}(L_{i},X)=\frac{\frac{c_{2i}Q_{i}^{2}}{2v_{i}r_{i}}+\frac{(c_{3i}-c_{2i})(Q_{i}-Q_{0i})^{2}}{2v_{i}r_{i}}+c_{4}v_{i}r_{i}X}{k+X}
\end{equation}

情况 2:当 $L_{i}\leq Q_{0i}$ 时,参照第一个问题,其费用 $f_{i}(L_{i},X)$ 函数如下,

\begin{equation}
f_i(L_i, X) =
\begin{cases}
\displaystyle \frac{A_i + c_{4i}v_ir_i\left(X - \frac{L_i}{v_ir_i}\right)}{X + \frac{Q_i - L_i}{v_ir_i}} & X > \frac{L_i}{v_ir_i} \\
\displaystyle \frac{A_i - \frac{1}{2}c_{2i}v_ir_i\left(X - \frac{L_i}{v_ir_i}\right)^2}{X + \frac{Q_i - L_i}{v_ir_i}} & X \leq \frac{L_i}{v_ir_i}
\end{cases}
\tag{3.1}
\end{equation}

其中
\[
A_i = c_1 + \frac{c_{2i}Q_i^2}{2v_ir_i} + \frac{(c_{3i} - c_{2i})(Q_i - Q_{0i})^2}{2v_ir_i}
\]

情况 3:当 \(L_i > Q_{0i}\) 时:

\begin{equation}
f_i(L_i, X) =
\begin{cases}
\displaystyle \frac{A_i - \frac{c_{2i}v_ir_i}{2}\left(X - \frac{L_i}{v_ir_i}\right)^2 - \frac{(c_{3i} - c_{2i})v_ir_i}{2}\left(\frac{L_i - Q_{0i}}{v_ir_i} - X\right)^2}{X + \frac{Q_i - L_i}{v_ir_i}} & 0 \leq X \leq \frac{L_i - Q_{0i}}{v_ir_i} \\
\displaystyle \frac{A_i - \frac{c_{2i}v_ir_i}{2}\left(X - \frac{L_i}{v_ir_i}\right)^2}{X + \frac{Q_i - L_i}{v_ir_i}} & \frac{L_i - Q_{0i}}{v_ir_i} < X \leq \frac{L_i}{v_ir_i} \\
\displaystyle \frac{A_i + c_{4i}v_ir_i\left(X - \frac{L_i}{v_ir_i}\right)}{X + \frac{Q_i - L_i}{v_ir_i}} & X > \frac{L_i}{v_ir_i}
\end{cases}
\tag{3.2}
\end{equation}

\[
f(L, X) = \sum_{i=1}^m f_i(L_i, X)
\]

则原题化为求 \(L^*\) 满足:

\[
E[f(L^*, X)] = \min E[f(L, X)] = \int_0^\infty \sum_{i=1}^m f_i(L_i, x)p(x)dx
\]

且这个多种商品的随机存贮模型同时满足下列几个约束条件:

\begin{equation}
\text{s.t. } \left\{
\begin{aligned}
\sum_{i=1}^{m} Q_{0i} &= Q_0 \\
\sum_{i=1}^{m} Q_i &= Q \\
\sum_{i=1}^{m} L_i &= L \\
\frac{Q_1 - L_1}{\nu_1 r_1} &= \frac{Q_2 - L_2}{\nu_2 r_2} = \ldots = \frac{Q_i - L_i}{\nu_i r_i} = k \quad (\text{当 } L_i > 0 \text{ 时}) \\
\frac{Q_i - L_i}{\nu_i r_i} &\leq k \quad (\text{当 } L_i = 0 \text{ 时})
\end{aligned}
\right.
\end{equation}

但由于 \(X\) 的分布未知,同时 \(Q_i\) 和 \(Q_{0i}\) 都是未知量;而且在问题 1 的单品种存贮模型中我们知道 \(L^*\) 的求解是十分困难的。所以对于多品种商品的存贮问题,直接解上面的模型是十分困难的。同时 \(L_i = 0\) 情况的存在,增加了其求解难度。因此在实际问题的解决中,我们可以用上面的模型作为基础,并假设 \(L_i > 0\) 对任意的 \(i\) 都成立,即排除 \(L_i = 0\) 的情况,通过计算机编程,建立类似于问题 2 中的合适算法求解上述模型。在问题 4 的解决中我们可以看到这样的解题思路。

\subsection{问题 4}

\subsubsection{问题分析}

由于货物到达商场的时间 \(X\) 是整数,所以可以认为 \(X\) 服从 1 天到 3 天之间的均匀分布实际上是一种离散分布,等价于 \(P(X=1) = P(X=2) = P(X=3) = \frac{1}{3}\)。问题 2 中原来的商品单价是以商品的个数为单位考虑的;而要利用问题 3 中的多品种商品存贮模型,则必须以商品的体积为单位进行考虑,所以我们对问题 2 中损失费用的单价进行转化处理,将它除以每单位商品的体积得到模型所需要的单价,如下表:

\begin{tabular}{|l|l|}
\hline
 & VAR00002 \\
\hline
N & 61 \\
Exponential parameter. a, b Mean & 1.9508 \\
Most Extreme & Absolute .401 \\
Differences & Positive .129 \\
 & Negative -.401 \\
Kolmogorov-Smirnov Z & 3.132 \\
Asymp. Sig. (2-tailed) & .000 \\
\hline
\end{tabular}

由于每种商品的个数是有限的,所以我们可以参照问题 2 中的求解办法,采用遍历搜索算法。但是由于 $Q_{i}$ 和 $Q_{0i}$ 都是未知量,这对于我们利用 (3.1) 以及 (3.2) 进行遍历搜索的求解增加了很大难度。所以我们可以对 $Q_{i}$ 和 $Q_{0i}$ 也进行遍历搜索,对于搜索的每一个 $Q_{i}$ 和 $Q_{0i}$,作为已知参数,然后利用问题 2 的模型求解 $L^{*}$。

\subsubsection{优化的 $Q_{i}$ 和 $Q_{0i}$ 的遍历搜索算法}

在这个问题中,$Q_{i}$ 和 $Q_{0i} \ (i=1,2,3)$ 总共有六个参数,那么在算法中就起码有六次嵌套的循环。$Q_{i}<Q=10$,所以如果以 0.01 为循环的步长,那么光一个 $Q_{i}$ 的搜索就有 1000 次循环,$Q_{i}$ 和 $Q_{0i}$ 总共的循环次数大约 $10$ 的 $36$ 次方,运算的次数和时间是非常惊人的,也让计算机无法承受。因此,我们对遍历搜索算法进行改进。先使步长为 1,则总循环次数不超过 $10$ 的 $6$ 次方,通过对 $Q_{i}$ 和 $Q_{0i}$ 进行遍历,然后再遍历 $L_{i}$,求得较优的 $E[f(L,X)]$ 以及确定对应的 $Q_{i}$ 和 $Q_{0i}$。由于是以步长 1 进行遍历搜索,所以 $Q_{i}$ 和 $Q_{0i}$ 的误差半径不超过 1。然后我们再取步长为 0.1,对 $Q_{i}$ 和 $Q_{0i}$ 在 1 的误差半径的范围内进行遍历搜索,则每一重的循环次数不超过 20,求得更优的 $E[f(L,X)]$ 以及确定对应的 $Q_{i}$ 和 $Q_{0i}$。由于第二次是以步长 0.1 进行遍历搜索,所以 $Q_{i}$ 和 $Q_{0i}$ 的误差半径不超过 0.1。最后又以步长 0.01 对 $Q_{i}$ 和 $Q_{0i}$ 在 0.1 的误差半径的范围内进行遍历搜索,同样每次循环次数不超过 20,这样我们最后求得了 $\min E[f(L,X)]$ 以及确定对应的 $Q_{i}$,$Q_{0i}$ 以及 $L^{*}$。

\subsubsection{$L^{*}$ 的遍历搜索求解}

在进行 $Q_{i}$ 和 $Q_{0i}$ 的每一次遍历取值以后,我们可以对每一个 $L_{i}$ 进行遍历求解。因为我们有等式 $\frac{Q_{i}-L_{i}}{v_{i}r_{i}}=k \quad i=1,2,..m$,所以我们实际上只需要对 $L_{1}$ 进行遍历,而 $L_{2}$ 和 $L_{3}$ 都可以通过这个等式确定。通过循环比较可以求得最优解 $\min E[f(L,X)]$ 以及 $L^{*}$。

大致的算法流程如下图:

\begin{figure}[h]
    \centering
    \includegraphics[width=\textwidth]{image.png}
    \caption{流程图}
\end{figure}

通过 matlab 编程求解,最后解得 $E[f(L^*,X)] = \min E[f(L,X)] = 17.494$,$L^* = 7.3333$,$L_1^* = 1.55$,$L_2^* = 1.8$,$L_3^* = 3.9833$,$Q_1 = 2.05$,$Q_2 = 2.30$,$Q_3 = 5.65$,$Q_{01} = 1.27$,$Q_{02} = 1.68$,$Q_{03} = 3.05$。

\subsection{问题 5}

\subsubsection{问题假设}

为了方便讨论,我们只考虑一种商品的随机存贮模型。并在问题 1 的基础上补充了以下假定:

\begin{enumerate}
    \item 订货情况 $c_1 = c(t)$,即 $c_1$ 是一个与 $t$ 有关的函数,而且具有周期性,周期为 $T$。
    \item 一般来说,商品的销售速率 $\{R_t\}$ 是一个随机的马氏链过程,并且同样具有周期性,周期也为 $T$;因为 $\{R_t\}$ 具有马氏性,所以是一个在周期 $T$ 内时间、状态均为离散的随机转移过程。
    \item $X$ 的概率分布 $P_X(x)$ 是离散的。
\end{enumerate}

\subsubsection{模型建立}

因为 $\{R_t\}$ 是马氏过程,所以
\[
P(R_{t+1} = r_{t+1} \mid R_1 = r_1, \ldots, R_t = r_t) = P(R_{t+1} = r_{t+1} \mid R_t = r_t).
\]

即 $R_t$ 的分布只与上一时刻的 $R_t$ 取值有关。

因为 $R_t$ 的取值状态是离散的,所以 $R_t$ 的分布是离散型的,并且取值的集合有限,设为 $R$。则对于每一个 $t$,$R_t$ 的分布为 $P_{R_t}(r)$。

由于总损失费用与 $c_1$ 有关,而 $c_1 = c(t)$ 是跟时间 $t$ 有关的;同时总损失费用也和 $R_t$ 有关,所以可以设总损失费用函数为 $f_t(L, R_t, X)$。

首先对 $f_t(L, R_t, X)$ 关于 $X$ 求期望
\[
E_X[f_t(L, R_t, X)] = \sum_{x=0}^\infty f_t(L, R_t, x)P_X(x),
\]
再对 $E[f(L_t, R_t, X)]$ 关于 $R_t$ 求期望
\[
E_{R_t}(E_X[f_t(L, R_t, X)]),
\]
再令
\[
g(L) = \min_{t \in T} E_{R_t}(E_X[f_t(L, R_t, X)]).
\]
则模型的求解就化为求 $L^*$ 使得 $g(L^*) = \min g(L)$。

当然,这一模型只是建立了一个基本的理论框架基础,具体模型的建立必须结合实际的问题,同时在模型成功建立的例子上能否求解出 $L^*$ 还必须结合问题的复杂性以及有关参数的分布和取值。

\section{模型的评价和推广}

本文建立了一个关于随机存贮的理论模型,为便于实际问题的求解,还另外给出了一个遍历搜索算法。前者具有一定的理论基础,后者采用前者的思想,全局寻优,给出了实际问题的最优解。在求解问题 2 时,由于遍历搜索的可行性,获得的是全局最优解。而在采用该算法在解决问题 4 时,由于商品品种较多,采取全局遍历搜索的算法会十分耗时而且难以实现,所以采用了修正步长的优化的遍历搜索算法,所以最终求得的结果是一接近全局最优的近似解,其结果的误差小于 0.01。这一方法可以推广到少量品种的随机存贮管理的情况。但是这一算法在随机存贮的商品品种较多的情况下运算量将非常大,要根据实际情况进行修改。

\section{参考文献:}

\begin{enumerate}
    \item 陈有禄, 罗秋兰. 仓库容量有限条件下的一类存贮管理模型. 数学的实践与认识, 2004. 6
    \item 周宏. 订货点决策模拟研究. 系统仿真学报, 2004. 1
    \item 杜世田. 多品种随机存贮模型的存贮策略. 山东工业大学学报, 2001. 2
    \item 陈同英. 原木随机存贮策略的最优化问题. 运筹与管理, 2002. 1
    \item 姜启源著. 数学模型(第二版). 北京: 高等教育出版社, 2002. 6
    \item 钱敏平, 龚光鲁著. 随机过程论. 北京: 北京大学出版社, 1992. 10
\end{enumerate}

\section{附录:}

\begin{verbatim}
%求解问题 2 商品一
prob=[2/36 4/36 5/36 15/36 5/36 3/36 1/36 1/36];
r=12;
c1=10;
c2=0.01;
c3=0.02;
c4=0.95;
Q0=40;
Q=60;
A=c1+c2*Q^2/(2*r)+(c3-c2)*(Q-Q0)^2/(2*r);
Emin1=1000000;
Lmin1=0;
for L=1:Q0
    sum=0;
    for X=0:length(prob)-1
        if (X>L/r)
            temp=(A+c4*r*(X-L/r))/(X+(Q-L)/r)*prob(X+1);
        else
            temp=(A-c2*r/2*(X-L/r)^2)/(X+(Q-L)/r)*prob(X+1);
        end;
        sum=sum+temp;
    end;
    if (sum<Emin1)
        Lmin1=L;
        Emin1=sum;
    end;
end;

Emin2=1000000;
Lmin2=0;
for L=Q0+1:Q
    sum=0;
    for X=0:length(prob)-1
        if (X>L/r)
            temp=(A+c4*r*(X-L/r))/(X+(Q-L)/r)*prob(X+1);
        else
            temp=(A-c2*r/2*(X-L/r)^2)/(X+(Q-L)/r)*prob(X+1);
        end;
        sum=sum+temp;
    end;
end;
\end{verbatim}

\begin{verbatim}
if (sum < Emin2)
    Lmin2 = L;
    Emin2 = sum;
end;
end;

if (Emin1 < Emin2)
    Emin = Emin1;
    Lmin = Lmin1;
else
    Emin = Emin2;
    Lmin = Lmin2;
end;
\end{verbatim}

\begin{verbatim}
%求解问题 2 商品二
prob=[2/43 23/43 12/43 5/43 1/43];
r=15;
c1=10;
c2=0.03;
c3=0.04;
c4=1.5;
Q0=40;
Q=60;
A=c1+c2*Q^2/(2*r)+(c3-c2)*(Q-Q0)^2/(2*r);
Emin1=1000000;
Lmin1=0;
for L=1:Q0
    sum=0;
    for X=1:length(prob)
        if (X>L/r)
            temp=(A+c4*r*(X-L/r))/(X+(Q-L)/r)*prob(X);
        else
            temp=(A-c2*r/2*(X-L/r)^2)/(X+(Q-L)/r)*prob(X);
        end;
        sum=sum+temp;
    end;
    if (sum<Emin1)
        Lmin1=L;
        Emin1=sum;
    end;
end;

Emin2=1000000;
Lmin2=0;
for L=Q0+1:Q
    sum=0;
    for X=1:5
        if (X>L/r)
            temp=(A+c4*r*(X-L/r))/(X+(Q-L)/r)*prob(X);
        else if (X<=L/r & X>(L-Q0)/r)
            temp=(A-c2*r/2*(X-L/r)^2)/(X+(Q-L)/r)*prob(X);
        else
            temp=(A-c2*r/2*(X-L/r)^2-(c3-c2)*r/2*((L-Q0)/r-X)^2)/(X+(Q-L)/r)*prob(X);
        end;
        sum=sum+temp;
    end;
    if (sum<Emin2)
\end{verbatim}

\begin{verbatim}
Lmin2=L;
Emin2=sum;
end;
end;

if (Emin1<Emin2)
    Emin=Emin1;
    Lmin=Lmin1;
else
    Emin=Emin2;
    Lmin=Lmin2;
end;
\end{verbatim}

\begin{verbatim}
%求解问题 2 商品三
prob=[27/61 20/61 8/61 3/61 2/61 1/61];
r=20;
c1=10;
c2=0.06;
c3=0.08;
c4=1.25;
Q0=20;
Q=40;
A=c1+c2*Q^2/(2*r)+(c3-c2)*(Q-Q0)^2/(2*r);

Emin1=1000000;
Lmin1=0;
for L=1:Q0
    sum=0;
    for X=1:length(prob)
        if (X>L/r)
            temp=(A+c4*r*(X-L/r))/(X+(Q-L)/r)*prob(X);
        else
            temp=(A-c2*r/2*(X-L/r)^2)/(X+(Q-L)/r)*prob(X);
        end;
        sum=sum+temp;
    end;
    if (sum<Emin1)
        Lmin1=L;
        Emin1=sum;
    end;
end;

Emin2=1000000;
Lmin2=0;
for L=Q0+1:Q
    sum=0;
    for X=1:6
        if (X>L/r)
            temp=(A+c4*r*(X-L/r))/(X+(Q-L)/r)*prob(X);
        elseif (X<=L/r && X>(L-Q0)/r)
            temp=(A-c2*r/2*(X-L/r)^2)/(X+(Q-L)/r)*prob(X);
        else
            temp=(A-c2*r/2*(X-L/r)^2-(c3-c2)*r/2*((L-Q0)/r-X)^2)/(X+(Q-L)/r)*prob(X);
        end;
        sum=sum+temp;
    end;
end;
\end{verbatim}

\begin{verbatim}
if (sum < Emin2)
    Lmin2 = L;
    Emin2 = sum;
end;
end;

if (Emin1 < Emin2)
    Emin = Emin1;
    Lmin = Lmin1;
else
    Emin = Emin2;
    Lmin = Lmin2;
end;
\end{verbatim}

\begin{verbatim}
%求解问题4
%参数初始化
r=[12 15 20];
%c 矩阵  行:c1,c2,c3,c4;列:商品 1,2,3
c=[10 10 10;
   0.01 0.03 0.06;
   0.02 0.04 0.08;
   0.95 1.50 1.25];
v=[0.05 0.04 0.1];
Q=zeros(1,3);
Q0=zeros(1,3);
L=zeros(1,3);
EL=zeros(1,3);
EQ=zeros(1,3);
EQ0=zeros(1,3);
EK=zeros(1,3);
LL=0;
TQ0=6;
TQ=10;

%X 的分布列
prob=[1/3 1/3 1/3];

%程序开始
Emin=1000000000;
Lmin=0;
for Q1=1:1:10
    Q(1)=Q1; %遍历 Q1
    for Q2=1:1:TQ-Q(1)
        Q(2)=Q2; %遍历 Q2
        for Q3=1:1:TQ-Q(1)-Q(2)
            Q(3)=Q3; %遍历 Q3
            for Q01=1:1:min(6,Q(1))
                Q0(1)=Q01; %遍历 Q0_1
                for Q02=1:1:min([6,Q(2),TQ0-Q0(1)])
                    Q0(2)=Q02; %遍历 Q0_2
                    for Q03=1:1:min([6,Q(3),TQ0-Q0(1)-Q0(2)])
                        Q0(3)=Q03; %遍历 Q0_3
                        for K1=1:10:min(200,fix(Q(1)/v(1)))
                            L(1)=K1*v(1); %遍历商品一的数目
                            for K2=1:10:min(fix((Q-L(1))/v(2)),fix(Q(2)/v(2)))
                                L(2)=K2*v(2); %遍历商品二的数目
                                for K3=1:10:min(fix((Q-L(1)-L(2))/v(3)),fix(Q(3)/v(3)))
                                    L(3)=K3*v(3); %遍历商品三的数目
                                end
                            end
                        end
                    end
                end
            end
        end
    end
end
\end{verbatim}

\begin{verbatim}
LL = L(1) + L(2) + L(3);
total = 0;  % 初始化期望
for i = 1:3
    sum = 0;
    A = c(1,i) + c(2,i)*Q(i)^2/(2*v(i)^2*r(i)) + (c(3,i)-c(2,i))*(Q(i)-Q0(i))^2/(2*v(i)^2*r(i));
    for X = 1:3
        if (X > L(i)/(v(i)*r(i)))
            temp = (A + c(4,i)*r(i)*(X-L(i)/(v(i)*r(i)))/(X+(Q(i)-L(i))/(v(i)*r(i))))*prob(X);
        else
            temp = (A - c(2,i)*r(i)/2*(X-L(i)/(v(i)*r(i)))^2)/(X+(Q(i)-L(i))/(v(i)*r(i)))*prob(X);
        end;
        sum = sum + temp;
    end;
    total = total + sum;
end;
if total < Emin
    EQ0 = Q0;
    EQ = Q;
    EL = L;
    Emin = total;
    EK(1) = K1;
    EK(2) = K2;
    EK(3) = K3;
    Lmin = LL;
end;
end;
end;
end;
end;
end;
end;
\end{verbatim}

\end{document}