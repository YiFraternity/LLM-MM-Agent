\section*{仓库容量有限条件下的随机存贮管理}

\section*{摘要}

本文解决仓库容量一定、销售速率不变、到货时间不定的随机性存贮管理问题。在商场,货物存贮过少,无法满足需求,会发生损失费用(缺货损失费用);存得过多,存贮费用就高,也会发生损失费用(余货损失费用)。有效的存贮管理可以降低成本、提高经济效益。

我们首先分单一商品存贮管理和多种商品存贮管理两种情况建立数学模型。对于单一商品存贮管理问题,取缺货损失费用和余货损失费用之和最小为目标建立了“单一商品最优订货点决策模型”,根据问题 2 的数据,通过统计分析推断出交货时间的概率分布,由此计算各种损失费用对应的概率,运用 lingo 编程求解得到最优订货时间。根据订货时间与订货点的关系,确定三种商品最优订货点分别为:34 盒、38 盒和 39 袋;对于多种商品存贮管理问题,除了考虑上述两种损失费用以外,还考虑了各种商品不同的库存安排可能导致的损失费用。首先建立以多种商品总存贮费用最小为目标函数的非线性“最佳库存安排模型”,运用 Lingo 软件编程求解,得到问题 4 中自己的仓库用于存贮 3 种商品的最佳体积容量分别为:2.523 立方米,0.673 立方米,2.804 立方米;两种仓库用于存贮 3 种商品的最佳体积容量分别为:3.828 立方米;1.195 立方米,4.978 立方米。然后,以上述最佳的库存安排为约束,再建立“多种商品最优订货点决策模型”来确定 3 种商品最佳同时订货时间和相应的最优订货点。运用 Lingo 编程求解,得到问题 4 的 3 种商品最佳同时订货时间为 1.175 天,最优订货点为 6.24 立方米及此时对应三种商品的体积容量分别为 3.123 立方米、0.49 立方米、2.628 立方米。

最后,建立了需求和订货时间均随机时的最佳订货点安排模型并对模型进行了讨论。

\textbf{关键词:随机存贮管理、单一商品存贮管理、多种商品存贮管理、Lingo 软件编程}

\section*{1. 问题的提出}

日常生活中,人们往往将所需的物资、用品和食物暂时的储备起来,以备将来的使用或消费。在商业中,储存物品主要为了解决供给与需求间不协调:商店里若存储商品数量不足,容易发生缺货现象,不仅无法满足消费者需求,而且还会因为失去销售机会影响利润;但如果存量过多,一时售不出去,便会造成商品积压,产生不必要的存贮费用,从而造成经济损失。因此,商店存贮管理就是寻找一种合理的订货策略和存贮策略,使得总费用损失最小化。

考虑如下问题:

(1) 市场上某种商品的销售速率一定;每次订货费为常数,自用仓库存贮费、租借仓库存贮费用、缺货单位商品损失、自用仓库容量、每次到货后存贮补充量均已知,每当存贮量降到一定时即开始订货。建立在交货时间是随机的条件下,使总损失费用达到最小的订货点的数学模型。

(2) 通过(1)中模型,并根据给定三种产品的相关数据,分别计算出三种商品各自的最优订货点。

(3) 在多种产品存贮管理中,多种商品需要同时订货。要求在交货时间随机变化的情况下,考虑使总损失费用最小时的最优订货点的计算、自用仓库存货空间的合理划分,以及在订货到达时使各商品补充到的体积数量。

(4) 把问题 2 中的三种商品按问题 3 的方法同时订货,要求得到这 3 种商品的最优订货点、自用仓库存货空间的合理划分,以及在订货到达时使各商品补充到的体积数量。

(5) 商品的销售经常是随机的、订货情况在一段时间后是会发生变化的,相应地商家就应该调整订货和存贮策略。对此建立数学模型并加以讨论。

上述问题的解决,将会使企业在实际运作中充分考虑需求与订货期的随机性的基础上,制定出更加符合自身实际的订货策略和存贮策略,具有重要的现实意义。

\section*{2. 问题的假设}

问题涉及如下假设:

(1) 库存量两次到达 $Q$ 的时间,为一个存货周期;

(2) 库存量随时间呈连续变化;

(3) 进货时先将货物存入自己仓库,装满自己仓库后方能装入租借仓库;销售商品时先从租借仓库中取货,取完后再从自己仓库中取货;

(4) 订货时不发生除已有损失外的其他损失,存货入库过程无时间消耗,存入仓库的存货不会发生意外损失;

(5) 多种商品的市场需求相互独立。

\section*{3. 符号说明}

$t_{d}$:订货时间;

$L$:订货时的存贮量;

\begin{itemize}
    \item $r$: 销售速率;
    \item $x$: 交货时间;
    \item $t_{d}+x$: 到货时间;
    \item $Q_{0}$ 表示商品存放在自己仓库里的数量;
    \item $Q$: 补充到的固定值;
    \item $Q-Q_{0}$: 表示商品存放在租界仓库里的数量;
    \item $P_{1}$: 缺货发生的概率
    \item $P_{21}$: 到货时的库存量在 $[0, Q-Q_{0})$ 范围内发生的概率;
    \item $P_{22}$: 到货时的库存量在 $(Q-Q_{0}, Q_{0})$ 范围内发生的概率;
    \item $P_{23}$: 到货时的库存量在 $(Q_{0}, Q)$ 范围内发生的概率;
    \item $C_{p}$: 缺货损失费用;
    \item $C_{pi}$: 第 $i$ 种商品缺货损失费用;
    \item $C_{w1}$: 到货时时的库存量在 $[0, Q-Q_{0})$ 范围内的余货损失费用;
    \item $C_{w2}$: 到货时库存量在 $(Q-Q_{0}, Q_{0})$ 范围内的余货损失费用;
    \item $C_{w3}$: 到货时库存量在 $(Q_{0}, Q)$ 范围内的余货损失费用;
    \item $C$: 单一商品的总期望损失费用;
    \item $C_{m}$: $m$ 种商品总存贮费用;
    \item $C_{w1i}$: 第 $i$ 种商品到货时的库存量在 $[0, Q_{i}^{*}-Q_{0i}^{*})$ 范围内的余货损失费用;
    \item $C_{w2i}$: 第 $i$ 种商品到货时的库存量在 $(Q_{i}^{*}-Q_{0i}^{*}, Q_{0i}^{*})$ 范围内的余货损失费用;
    \item $C_{w3i}$: 第 $i$ 种商品到货时的库存量在 $(Q_{0i}^{*}, Q_{i}^{*})$ 范围内的余货损失费用;
    \item $C_{i}$: 第 $i$ 种商品的总期望损失费用。
\end{itemize}

\section{数学模型}
\subsection{单一商品最优订货点决策模型}
\subsubsection{模型的建立}

这是一个仓库容量一定、销售速率不变、需要区分不同仓库存贮费用、到货时间不定的随机性存贮管理问题。我们的目标寻求最优订货点,来使得总损失费用最小。在单一商品情况下,总损失费用包括两个部分:缺货损失费用和余货损失费用。

\section*{(1) 缺货损失费用}

由于订货时间过长,使得到货时间超出了库存商品可以销售的时间,造成了存贮的供不应求。这时因缺货而造成了利润损失,产生了缺货损失费用。

\begin{figure}[h]
\centering
\includegraphics[width=0.8\textwidth]{image.png}
\end{figure}

上图中 \( t_d + x \) 为到货时间,\( \frac{Q}{r} \) 为以销售速率 \( r \) 销售完 \( Q \) 的时间;

存贮量是时间的连续函数:\( F(t) = -rt + Q \)

当 \( t = t_d \) 时,\( L = F(t_d) = -rt_d + Q \),得 \( t_d = \frac{Q - L}{r} \)

缺货时间为:\( (t_d + x) - \frac{Q}{r} \),即 \( x - \frac{L}{r} \);

到货时发生缺货:\( t_d + x \geq \frac{Q}{r} \),即 \( x \geq \frac{Q}{r} - t_d \);

随机变量 \( X \) 在 \([0, +\infty)\) 上服从概率密度函数为 \( f(x) \) 的分布,设此时发生的概率为 \( P_1 \),缺货损失费用为 \( C_p \)。

\section*{(2) 余货损失费用}

由于订货时间较短,使得在没有完全售完存货的情况下,新到的货物又将库存弥补至最大库存,这时弥补的量产生了多余的存贮费用造成了相应损失。

\begin{tikzpicture}[scale=0.8]
    \draw[->] (0,0) -- (10,0) node[right] {时间 $t$};
    \draw[->] (0,0) -- (0,7) node[above] {库存量};
    \draw (0,6) -- (8,0);
    \draw[dashed] (0,5) -- (8,5) node[right] {到货时间};
    \draw[dashed] (0,4) -- (8,4);
    \draw[dashed] (6,0) -- (6,6);
    \draw[dashed] (6,5) -- (8,5);
    \draw[dashed] (6,4) -- (8,4);
    \node at (-0.5,6) {$Q$};
    \node at (-0.5,5) {$L^*$};
    \node at (-0.5,4) {$L$};
    \node at (6,-0.5) {定货点 $t_d$};
    \node at (8,-0.5) {$\left(\frac{Q}{r}\right)$};
\end{tikzpicture}

出现余货的情况下,由于自己仓库和租用仓库的存贮费用不一样,在计算存贮费用时要区分对待。根据商品到达时的库存量的多少,又分为库存量在 $(0, Q-Q_0)$、$(Q-Q_0, Q_0)$、$(Q_0, Q)$ 三种情况。

其中,
- $Q-Q_0$ 表示商品存放在租借仓库里的数量;
- $Q_0$ 表示商品存放在自己仓库里的数量;

(1) 商品到达时的库存量在 $(0, Q-Q_0)$ 范围内

设此时发生的概率为 $P_{21}$,相应的费用为 $C_{w1}$

(2) 商品到达时的库存量在 $(Q-Q_0, Q_0)$ 范围内

设此时发生的概率为 $P_{22}$,相应的费用为 $C_{w2}$

(3) 商品到达时的库存量在 $(Q_0, Q)$ 范围内

设此时发生的概率为 $P_{23}$,相应的费用为 $C_{w3}$;

商品的期望损失费用:
\[
C = C_p \times P_1 + C_{w1} \times P_{21} + C_{w2} \times P_{22} + C_{w3} \times P_{23}
\]

要使得期望损失费用最小,即:
\[
\min(C)
\]

各种情况下发生的概率应该满足:
\[
0 \leq P_1 \leq 1
\]

\begin{align*}
0 \leq P_{2j} \leq 0, j = 1, 2, 3
\end{align*}

总概率之和应为 1:
\begin{align*}
P_1 + P_{21} + P_{22} + P_{23} = 1
\end{align*}

各种情况下损失费用非负:
\begin{align*}
C_p &\geq 0 \\
C_{wj} &\geq 0, j = 1, 2, 3
\end{align*}

模型可表示为:
\begin{align*}
\min(C_p \times P_1 + C_{w1} \times P_{21} + C_{w2} \times P_{22} + C_{w3} \times P_{23})
\end{align*}
\begin{equation}
s.t.
\begin{cases}
0 \leq P_1 \leq 1 \\
0 \leq P_{2j} \leq 1 \\
P_1 + P_{21} + P_{22} + P_{23} = 1 \\
C_p \geq 0 \\
C_{wj} \geq 0, j = 1, 2, 3
\end{cases}
\end{equation}

其中 $C_p$, $P_1$, $C_{w1}$, $P_{2j}$, $C_{w2}$, $P_{22}$, $C_{w3}$, $P_{23}$ 都可以用 $t_d$ 来表示,请参考附录 1。

\subsection{4.1.2 模型求解与结果分析——问题 2 求解}

问题 2 是在问题 1 的基础上,通过给定具体的数据,利用模型 1 来求解最优订货时间 $t_d^*$,和最优订货点 $L^*$。求解步骤:

(1) 运用 SPSS 软件对各组数据进行统计性质的分析,推断交货时间的概率分布,其目的是为了求解各种损失费用发生的概率;

(2) 期望损失费用 $C(t_d, X)$ 是订货时间和交货时间的函数,由于在现实生活中,交货时间往往是比较稳定的,并且基本上稳定于交货时间的均值上,所以,在计算期望损失费用时,更多的考察当 $X = EX$ 情况下产生的最小期望费用,此时,期望损失费用函数变为 $C(t_d, EX)$,仅是订货时间的函数;

(3) 根据总期望损失费用最小的非线性优化模型 $\min C$,通过计算各种损失费用和发生概率,运用 lingo 程序求得最优订货时间 $t_d^*$ 和最优订货点 $L^*$;

\subsubsection{4.1.2.1 商品一最优订货点决策模型的求解}

(1) 运用 SPSS 软件对商品一数据进行统计分析,推断出交货时间 $X$ 服从正态分布,均值为 $\mu_1$,方差为 $\delta_1$,$X \sim N(\mu_1, \sigma_1)$,其概率分布密度函数为 $f_1(x)$,将 $X$ 标准化后,

\begin{align*}
\frac{X - \mu_1}{\sigma_1} &\sim N(0, 1), \text{ 分布函数为 } \Phi_1\left(\frac{X - \mu_1}{\sigma_1}\right), \text{ 库存量 } F_1(t) \text{ 与时间 } t \text{ 满足关系: } \\
F_1(t) &= -r_1 \times t + Q_1;
\end{align*}

(2) 将相关数据带入模型,得到 $C_{p1}$, $C_{w11}$, $C_{w21}$, $C_{w31}$, $P_1^1$, $P_{21}^1$, $P_{22}^1$, $P_{23}^1$;

$r_1 = 12 \text{ 盒/天}, c_{21} = 0.01 \text{ 元/盒·天}, c_{31} = 0.02 \text{ 元/盒·天}, c_{41} = 0.95 \text{ 元/盒·天}, Q_{01} = 40 \text{ 盒}, Q_1 = 60 \text{ 盒}, \mu_1 = 2.9722 \text{ 天}, \sigma_1 = 1.521$;

运用 lingo 程序编程求解(具体求解过程参见附录 2),得到:

$t_d^* = 2.215$;

根据 $L_1 = r_1 \times t_d + Q_1 = -12 \times t_d + 60$,有:

$L_1^* = 34$;

① 结果说明:当商品一在已知连续的 36 次订货后到货时间天数的情况下,商品一的最优订货时间是 2.215 天,最优订货点是 34 盒,即在开始销售后的 2.215 天,库存还有 34 盒康师傅精装巧碗香菇遁炖鸡面时,订货所造成的总损失费用最小;

② 结果分析:运用 SPSS 对商品一的交货时间进行统计特性的分析(见附录 3),可以看到交货时间的均值为 2.972 天,交货时间为 3 天的概率为 0.417,交货时间在 3 天内(包括 3 天)的概率为 0.722,如果在库存为 34 盒时订货,此时决大多数情况都不发生缺货,就算有缺货,缺货的数量也相对较小,由于缺货发生的单位损失费用远大于余货的单位损失费用,所以,应该考虑在交货时间等于均值附近时,尽量少产生缺货的情况,我们所求的结果基本上满足此原理。

\subsection*{4.1.2.2 商品二最优定货点的计算}

(1) 运用 SPSS 软件对商品二数据进行统计分析,推断交货时间近似服从正态分布,均值为 $\mu_2$,方差为 $\delta_2$,$X \sim N(\mu_2, \sigma_2)$,其概率分布密度函数为 $f_2(x)$,将 $X$ 标准化后,

\begin{align*}
\frac{X - \mu_2}{\sigma_2} &\sim N(0, 1), \text{ 分布函数为 } \Phi_2\left(\frac{X - \mu_2}{\sigma_2}\right), \text{ 库存量 } F_2(t) \text{ 与时间 } t \text{ 满足关系: } \\
F_2(t) &= -r_2 \times t + Q_2;
\end{align*}

(2) 将相关数据带入模型,得到 $C_{p2}$, $C_{w12}$, $C_{w22}$, $C_{w32}$, $P_1^2$, $P_{21}^2$, $P_{22}^2$, $P_{23}^2$;

$r_2 = 15 \text{ 盒/天}, c_{22} = 0.03 \text{ 元/盒·天}, c_{32} = 0.04 \text{ 元/盒·天}, c_{42} = 1.5 \text{ 元/盒·天}, Q_{02} = 40 \text{ 盒}, Q_2 = 60 \text{ 盒}, \mu_2 = 2.535 \text{ 天}, \sigma_2 = 0.855$;

运用 lingo 程序编程求解(具体求解过程参见附录 2),得到:

\[
t_{d}^{*}=1.465;
\]

根据 \(L_{2}=r_{2} \times t_{d}+Q_{2}=-15 \times t_{d}+60\),有:

\[
L_{2}^{*}=38;
\]

① 结果说明:当商品二在已知连续的 43 次订货后到货时间天数的情况下,商品二的最优订货时间是 1.465 天,最优订货点是 38 盒,即在开始销售后的 1.465 天,库存还有 38 盒心相印手帕纸时,订货所造成的总损失费用最小;

② 结果分析:运用 SPSS 对商品二的交货时间进行统计特性的分析(见附录 3),可以看到交货时间的均值为 2.535 天,交货时间在 2 天内(包括 2 天)的概率为 0.58,交货时间在 3 天内(包括 3 天)的概率为 0.86,如果在库存为 38 盒时订货,此时有 58% 以上的概率都不发生缺货,就算有缺货,缺货的数量也相对较小,由于缺货发生的单位损失费用远大于余货的单位损失费用,所以,应该考虑在交货时间等于均值附近时,尽量少产生缺货的情况,我们所求的结果基本上满足此原理。

### 4.1.2.3 商品三最优定货点的计算

(1)运用 SPSS 软件对商品三数据进行统计分析,推断出交货时间服从正态分布,均值为 \(\mu_{3}\),方差为 \(\sigma_{3}\),\(X \sim N\left(\mu_{3}, \sigma_{3}\right)\),其概率分布密度函数为 \(f_{3}(x)\),将 \(X\) 标准化后,

\[
\frac{X-\mu_{3}}{\sigma_{3}} \sim N(0,1), \text{ 分布函数为 } \Phi_{3}\left(\frac{X-\mu_{3}}{\sigma_{3}}\right),
\]

库存量 \(F_{3}(t)\) 与时间 \(t\) 满足关系:

\[
F_{3}(t)=-r_{3} \times t+Q_{3};
\]

(2)将相关数据带入模型,得到 \(C_{p3}\),\(C_{w13}\),\(C_{w23}\),\(C_{w33}\),\(P_{1}^{3}\),\(P_{21}^{3}\),\(P_{22}^{3}\),\(P_{23}^{3}\);

\(r_{3}=20\) 袋/天,\(c_{23}=0.06\) 元/袋·天,\(c_{33}=0.08\) 元/袋·天,\(c_{43}=1.25\) 元/袋·天,\(Q_{03}=20\) 袋,\(Q_{3}=40\) 袋,\(\mu_{3}=1.951\) 天,\(\sigma_{3}=1.161\);

运用 lingo 程序编程求解(具体求解过程参见附录 2),得到:

\[
t_{d}^{*}=0.0544;
\]

根据 \(L_{3}=r_{3} \times t_{d}+Q_{3}=-20 \times t_{d}+20\),有:

\[
L_{3}^{*}=39;
\]

① 结果说明:当商品三在已知连续的 61 次订货后到货时间天数的情况下,商品三的最优订货时间是 0.05 天,最优订货点是 39 袋,即在一开始销售就要准备订货,这是由于商品三是一种需求程度很高的商品,然而商品三的库存量相对较小的特殊情况决定的。

② 结果分析:商品三的总库存量为 40 袋,而每天销售 20 袋,如果没有进货,库存的商品三 2 天就售完。根据已知的 61 次订货后到货时间天数的数据(见附录 3),我们可以看

到 1 天到货的概率为 44.3%,超过 1 天到达的概率为 55.7%。如果我们按照连续订货模式,即第一批货物到达时,立即开始订下一批货物,这样的订货模式有 44.3% 的可能会带来货物提早到达而产生的额外库存费;有 32.8% 的可能货物刚好卖完新货刚好到达,这时的损失费用为 0;有 22.9% 的可能会由于缺货造成利润损失。由于缺货造成的单位商品损失费用远远大于多存储单位商品的存储费,所以,要使总损失费用最小,应该尽量不要发生缺货损失费,即最好在商品到达时立即开始订购下一批货物,也就是最佳订货点应该接近于 40。

\subsection*{4.2 多种商品最优订货点决策模型}

对于多产品产生总损失费用,除了考虑单一产品中的缺货损失费用与余货损失费用外,还考虑了各种商品不同的库存安排可能导致的损失费用。

各种商品初始库存的不合理安排会带来损失。比如一种商品库存很多,而另一种商品库存很少。在下次到货时会出现有的商品仓库内剩余很多,而另一些商品仓库内早就没有了。这种不合理的安排导致有些商品因进货过多而发生不必要的存贮费用,而有些商品因进货过少而发生供不应求的利润损失。因此,对多种商品的初始库存量进行一个合理安排是很有必要的。

\textbf{存贮量(立方米)}

\begin{figure}[h]
    \centering
    \begin{tikzpicture}[scale=0.8]
        \draw[->, thick] (0,0) -- (10,0) node[right] {时间 $t$};
        \draw[->, thick] (0,0) -- (0,8) node[above] {存贮量};
        
        \draw[dashed] (0,7) -- (7,0) node[below] {T};
        \draw[dashed] (0,6) -- (6,0);
        \draw[dashed] (0,5) -- (5,0);
        \draw[dashed] (0,4) -- (4,0);
        
        \draw[thick] (0,7) -- (7,0);
        \draw[thick] (0,6) -- (6,0);
        \draw[thick] (0,5) -- (5,0);
        \draw[thick] (0,4) -- (4,0);
        
        \node[left] at (0,7) {A2};
        \node[left] at (0,6) {A1};
        \node[left] at (0,5) {B1};
        \node[left] at (0,4) {B2};
        
        \node[above] at (3.5,4) {到货时间};
        
        \node[below] at (0,0) {0};
    \end{tikzpicture}
    \caption{库存变化示意图}
\end{figure}

如上图,假设有两种商品,第一种状态下 1、2 两种商品初始库存量分别为 A1、B1,第二种状态下 1、2 两种商品初始库存量分别为 A2、B2,两种状态下总库存量是不变的,即 A1+B1=A2+B2。当到货时间发生在如上图时,第一种状态下,刚好没有缺货损失费用和余货损失费用;而在第二种状态下,商品 2 发生了缺货损失费用,商品 1 发生了余货损失费用。显然,由于初始库存量的不同,在第二种状态下的总损失费用要大于第一种状态下的总损失费用。这就说明,商品初始库存量的不合理安排会带来额外的损失费用。

\subsubsection{4.2.1 最佳库存安排模型}

$m$ 种商品的销售速率分别为 $r_{i}$(袋或盒/天)$(i=1,2,\ldots,m)$,每袋(或盒)的体积分

别为 $v_{i} \, (i=1,2,\ldots,m)$, 则 $m$ 种商品以体积表示的销售速率分别为 $r_{i} \times v_{i}$ (立方米/天)。第 $i$ 种商品存贮量补充到的固定体积为 $Q_{i}$, 其中存放在自己仓库内 $Q_{0i}$ 立方米, 存放在租界仓库内的 $(Q_{i}-Q_{i0})$。很明显有 $Q_{i} \geq Q_{0i}$ 即 $(Q_{i}-Q_{i0}) \geq 0$, 仍然假设存贮体积随时间的变化呈连续变化。

第 $i$ 种商品存贮体积与时间 $t$ 之间存在如下关系:
\[
V_{i}(t) = (-r_{i}v_{i})t + Q_{i}
\]

\begin{tikzpicture}[scale=0.8]
    \draw[->, thick] (0,0) -- (10,0) node[right] {时间 $t$};
    \draw[->, thick] (0,0) -- (0,6) node[above] {存贮量(立方米)};
    
    \draw[thick] (0,5) -- (8,0);
    
    \draw[thick] (0,4) -- (2,4) -- (2,0);
    
    \node at (0,5.2) {$Q_{i}$};
    \node at (0,4.2) {$Q_{i0}$};
    \node at (-0.5,0) {$0$};
    \node at (2,-0.5) {$\frac{Q_{i}-Q_{i0}}{r_{i}v_{i}}$};
    \node at (8,-0.5) {$(\frac{Q_{i}}{r_{i}v_{i}})$};
\end{tikzpicture}

设第 $i$ 种商品的所有库存费用为 $C_{i}$, 则 $m$ 种商品总存贮费用 $(C_{m})$ 为:
\[
C_{m} = \sum_{i=1}^{m} \left[ c_{3i} \times \frac{1}{2} \times \frac{(Q_{i}-Q_{i0})^{2}}{r_{i}v_{i}} + c_{2i} \times \frac{1}{2} \times \frac{Q_{i0}^{2}}{r_{i}v_{i}} \right]
\]

要使存贮费用 $(C_{m})$ 最小, 即:
\[
\min \left\{ \sum_{i=1}^{m} \left[ c_{3i} \times \frac{1}{2} \times \frac{(Q_{i}-Q_{i0})^{2}}{r_{i}v_{i}} + c_{2i} \times \frac{1}{2} \times \frac{Q_{i0}^{2}}{r_{i}v_{i}} \right] \right\}
\]

其中各种商品补充到的体积为 $Q$, 即 $\sum_{i=1}^{m} Q_{i} = Q$;

各种商品存放在自己仓库内的体积总和等于自己仓库的总体积, 即:
\[
\sum_{i=1}^{m} Q_{i0} = Q_{0}
\]

由于租界仓库存贮费用大于自己仓库存贮费用, 存放在自己仓库内的商品量应大于存放在租界仓库内的商品量, 即:

\[
\mathcal{Q}_i \geq \mathcal{Q}_{0i}
\]

由于各种商品都要销售,故: \(\mathcal{Q}_{0i} > 0\)

模型为:
\[
\min \left\{ \sum_{i=1}^{m} \left[ c_{3i} \times \frac{1}{2} \times \frac{(\mathcal{Q}_i - \mathcal{Q}_{i0})^2}{r_i \nu_i} + c_{2i} \times \frac{1}{2} \times \frac{\mathcal{Q}_{i0}^2}{r_i \nu_i} \right] \right\}
\]
\[
\text{s.t.} \left\{
\begin{aligned}
\sum_{i=1}^{m} \mathcal{Q}_i &= \mathcal{Q} \\
\sum_{i=1}^{m} \mathcal{Q}_{i0} &= \mathcal{Q}_0 \\
\mathcal{Q}_i &\geq \mathcal{Q}_{0i} \\
\mathcal{Q}_{0i} &> 0
\end{aligned}
\right.
\]

这是一个以 \(\mathcal{Q}_i\) 和 \(\mathcal{Q}_{0i}\) 为决策变量有约束的非线性规划问题。运用 Lingo 软件,可以求解出相应的最优解,使得目标函数最小。通过最优化求解,得到的最优决策变量表示为 \(\mathcal{Q}_i^*\) 和 \(\mathcal{Q}_{0i}^*\)。

### 4.2.2 多种商品最优订货点决策模型

将上述解得的 \(\mathcal{Q}_i^*\) 和 \(\mathcal{Q}_{0i}^*\) 作为已知数据,进行多种商品最优订货点决策模型。先考察具有代表性的第 \(i\) 种单独商品情况下,因订货点和定货时间的不确定,导致的损失费用。建立 “多种商品最优订货点决策模型” 来确定多种商品最佳同时订货时间和相应的最优订货点同样第 \(i\) 种商品的损失费用仍然包括库存过多的损失费用和库存不足的损失费用。

\begin{figure}[h]
\centering
\begin{tikzpicture}[scale=0.8]
    \draw[->, thick] (0,0) -- (10,0) node[right] {时间 \(t\)};
    \draw[->, thick] (0,0) -- (0,6) node[above] {库存量(立方米)};
    \draw[thick] (0,5) -- (8,0);
    \draw[dashed] (0,4) -- (2,4) -- (2,0);
    \node at (0,5) [left] {\(\mathcal{Q}_i^*\)};
    \node at (0,4) [left] {\(\mathcal{Q}_{0i}^*\)};
    \node at (2,0) [below] {\(t_d\)};
    \node at (8,0) [below] {\(\left(\frac{\mathcal{Q}_i^*}{r_i \nu_i}\right)\)};
    \node at (0,0) [below left] {0};
\end{tikzpicture}
\end{figure}

同样设定货时间为 \(t_d\) ,\(t_d \in (0, +\infty)\)。交货时间 \(X\) 是随机的,其概率密度函数为 \(f(x)\),

概率分布函数为 $F(x)$,设其为连续型概率密度函数。当交货时间为 $x$ 时,货物到达的时间为 $t_{d}+x$。当货物送达时,存在两种情况:一种仓库没有剩余,另一种为仓库仍有剩余。

(1) 当 $t_{d}+x \geq \frac{Q_{i}^{*}}{r_{i}v_{i}}$,即 $t_{d} \geq \frac{Q_{i}^{*}}{r_{i}v_{i}}-x$ 时,仓库没有剩余

设这种情况的概率为 $P_{1}$,用 $C_{pi}$ 表示这部分的利润损失,

(2) 当 $t_{d}+x < \frac{Q_{i}^{*}}{r_{i}v_{i}}$,即 $t_{d} < \frac{Q_{i}^{*}}{r_{i}v_{i}}-x$ 时,仓库有剩余

在商品到达时,仓库有剩余商品的情况下,由于自己仓库和租用仓库的存贮费用不一样,在计算存贮费用要区分对待。根据商品到达时的库存量的多少,又分为库存量在 $(0, Q_{i}^{*}-Q_{0i}^{*})$、$(Q_{i}^{*}-Q_{0i}^{*}, Q_{0i}^{*})$、$(Q_{0i}^{*}, Q_{i}^{*})$ 三种情况:

其中,

- $Q_{i}^{*}-Q_{0i}^{*}$ 表示第 $i$ 种商品存放在租界仓库里的最优数量;
- $Q_{0i}^{*}$ 表示第 $i$ 种商品存放在自己仓库里的最优数量;
- $Q_{i}^{*}$ 第 $i$ 种商品最优总存贮量;

① 商品到达时的库存量在 $(0, Q_{i}^{*}-Q_{0i}^{*})$ 范围内

设此种情况发生的概率为 $P_{21}$,此时发生的损失费用 $C_{w1i}$

② 商品到达时的库存量在 $(Q_{i}^{*}-Q_{0i}^{*}, Q_{0i}^{*})$ 范围内

设此种情况发生的概率为 $P_{22}$,相应损失费用为 $C_{w2i}$

③ 商品到达时的库存量在 $(Q_{0i}^{*}, Q_{i}^{*})$ 范围内

设此种情况发生的概率为 $P_{23}$,损失费用 $C_{w3i}$

如果用 $C_{i}$ 表示第 $i$ 种商品的期望费用,有:

\[
C_{i} = C_{pi} \times P_{1} + C_{w1i} \times P_{21} + C_{w2i} \times P_{22} + C_{w3i} \times P_{23}
\]

$m$ 种商品的总期望损失费用为:

\[
\sum_{i=1}^{m} C_{pi} \times P_{1} + C_{w1i} \times P_{21} + C_{w2i} \times P_{22} + C_{w3i} \times P_{23}
\]

取 $m$ 种商品的总期望损失费用最小。

\begin{equation}
\min \sum_{i=1}^{m} C_{p_{i}} \times P_{1} + C_{w1i} \times P_{21} + C_{w2i} \times P_{22} + C_{w3i} \times P_{23}
\end{equation}

各种情况下发生的概率应该满足:
\begin{align}
0 \leq P_{1} \leq 1 \\
0 \leq P_{2j} \leq 0, j = 1, 2, 3
\end{align}

总概率之和应为 1:
\begin{equation}
P_{1} + P_{21} + P_{22} + P_{23} = 1
\end{equation}

各种情况下损失费用为非负:
\begin{align}
C_{p_{i}} &\geq 0 \\
C_{wji} &\geq 0, j = 1, 2, 3
\end{align}

模型可表示为:

目标函数:
\begin{equation}
\min \sum_{i=1}^{m} C_{p_{i}} \times P_{1} + C_{w1i} \times P_{21} + C_{w2i} \times P_{22} + C_{w3i} \times P_{23}
\end{equation}

约束条件:
\begin{equation}
s.t.
\begin{cases}
0 \leq P_{1} \leq 1; \\
0 \leq P_{2j} \leq 1; \\
P_{1} + P_{21} + P_{22} + P_{23} = 1; \\
C_{p_{i}} \geq 0; \\
C_{wji} \geq 0, j = 1, 2, 3;
\end{cases}
\end{equation}

其中 $C_{p_{i}}$、$P_{1}$、$C_{w1i}$、$P_{2j}$、$C_{w2i}$、$P_{22}$、$C_{w3i}$、$P_{23}$ 都可以用 $t_{d}$ 来表示,请参考附录 4。

在 $X$ 遵循一定概率分布 $f(x)$ 的条件下,给定交货时间为 $x$,目标函数仅是以 $t_{d}$ 为变量的函数。这个使得 $m$ 种商品的总期望损失费用最小的模型求解,是一个以 $t_{d}$ 为决策变量的非线性规划问题。根据 Lingo 软件,可以求解出相应的最优解,使得目标函数最小,即 $m$ 种商品的总期望损失费用最小。

通过最优化求解,得到的最优决策变量可表示为 $t_{d}^{*}$;即在 $t_{d}^{*}$ 时刻开始订货,可使得 $m$ 种商品的总期望损失费用最小。

由于第 $i$ 种商品库存量与时间 $t$ 之间存在如下关系:
\begin{equation}
V_{i}(t) = (-r_{i}v_{i})t + Q_{i}^{*}
\end{equation}

则 $t_{d}$ 时刻第 $i$ 种商品的库存量为:
\begin{equation}
Q_{i}^{*} - r_{i}v_{i}t_{d}
\end{equation}

$t_{d}$ 时刻 $m$ 种商品总库存量为:
\[
\sum_{i=1}^{m} V_{i}(t_{d}) = \sum_{i=1}^{m} (Q_{i}^{*} - r_{i} v_{i} t_{d}) = L
\]

由于 $t_{d}^{*}$ 是使得 $m$ 种商品的总期望损失费用最小的订货时间, 那么相应的 $L^{*}$ 就是使得 $m$ 种商品的总期望损失费用最小的订货点, 有:
\begin{align*}
L^{*} &= \sum_{i=1}^{m} (Q_{i}^{*} - r_{i} v_{i} t_{d}^{*}) \\
&= Q - t_{d}^{*} \sum_{i=1}^{m} r_{i} v_{i}
\end{align*}

### 4.2.3 模型的求解与运用——问题 4 求解

这是一个以 $Q_{i}$ 和 $Q_{0i}$ 为决策变量有约束的非线性规划问题。运用 Lingo 软件, 可以求解出相应的最优解, 使得目标函数最小。通过最优化求解, 得到的最优决策变量表示为 $Q_{i}^{*}$ 和 $Q_{0i}^{*}$。

\[
\min \left\{ \left[ \frac{(Q_{1} - Q_{01})^{2}}{3} + \frac{Q_{01}^{2}}{6} \right] + \left[ \frac{(Q_{2} - Q_{02})^{2}}{1.2} + \frac{Q_{02}^{2}}{1.6} \right] + \left[ \frac{(Q_{3} - Q_{03})^{2}}{5} + \frac{3Q_{03}^{2}}{20} \right] \right\}
\]

\[
s.t.
\begin{cases}
Q_{1} + Q_{2} + Q_{3} = 10 \\
Q_{01} + Q_{02} + Q_{03} = 6 \\
Q_{1} \geq Q_{01} \\
Q_{2} \geq Q_{02} \\
Q_{3} \geq Q_{03} \\
Q_{01} > 0 \\
Q_{02} > 0 \\
Q_{03} > 0
\end{cases}
\]

运用 Lingo 软件编程, 求得如下结果(程序见附件 5, 精确到小数点后 3 位):
\[
Q_{1}^{*} = 3.828 \text{ 立方米}, \, Q_{2}^{*} = 1.195 \text{ 立方米}, \, Q_{3}^{*} = 4.978 \text{ 立方米}
\]
\[
Q_{01}^{*} = 2.523 \text{ 立方米}, \, Q_{02}^{*} = 0.673 \text{ 立方米}, \, Q_{03}^{*} = 2.804 \text{ 立方米}
\]

结果说明: 3 种商品各自体积容量为 2.523 立方米、0.673 立方米、2.804 立方米以及在订货到达时使这 3 种商品各自存贮量补充到的固定体积分别为 3.828 立方米、1.195 立方米、4.978 立方米时, 各种商品总的库存量和各种商品在两个仓库内的安排, 会使得不会因不合理的安排而发生损失费用。

将上述结论作为已知条件,求解 “多种商品最佳订货点决策模型”。

到货时间 \( X \) 是服从在 1 天到 3 天之间的均匀分布,则在 \([0, +\infty)\) 内概率分布函数:

\[
F(x) =
\begin{cases}
0, & [0, 1) \\
\frac{x-1}{2}, & [1, 3] \\
1, & (3, +\infty)
\end{cases}
\]

当 \( t_d + x \geq \frac{Q_i^*}{r_i v_i} \),即 \( t_d + x \geq \frac{Q_i^*}{r_i v_i} \) \( t_d \geq \frac{Q_i^*}{r_i v_i} - x \) 时,仓库没有剩余库存量为零。

由不等式 \( t_d + x \geq \frac{Q_i^*}{r_i v_i} \) 得

\[
x \geq \frac{Q_i^*}{r_i v_i} - t_d
\]

因为随机变量 \( X \) 概率密度函数为 \( f(x) \),故发生仓库没有剩余这种情况的概率为:

\[
P_1 = \int_{\frac{Q_i^*}{r_i v_i} - t_d}^{+\infty} f(x) dx = 1 - F\left(\frac{Q_i^*}{r_i v_i} - t_d\right)
\]

当 \( \frac{Q_i^*}{r_i v_i} - t_d \geq 3 \),即 \( t_d \leq \frac{Q_i^*}{r_i v_i} - 3 \) 时,\( P_1 = 0 \);

当 \( \frac{Q_i^*}{r_i v_i} - t_d < 1 \),即 \( t_d > \frac{Q_i^*}{r_i v_i} - 1 \) 时,\( P_1 = 1 \);

当 \( 1 \leq \frac{Q_i^*}{r_i v_i} - t_d \leq 3 \),即 \( \frac{Q_i^*}{r_i v_i} - 3 \leq t_d \leq \frac{Q_i^*}{r_i v_i} - 1 \) 时,\( P_1 = 1 - \frac{\frac{Q_i^*}{r_i v_i} - t_d - 1}{2} \);

在这种情况下,会因库存不足导致利润损失。如果用 \( C_{pi} \) 表示这部分的利润损失,可表示为:

\[
C_{pi} = r_i \times v_i \times c_{4i} \times \left(t_d + x - \frac{Q_i^*}{r_i v_i}\right)
\]

同理可以计算,库存量在 \((0, Q_i^* - Q_{0i}^*)\)、\((Q_i^* - Q_{0i}^*, Q_{0i}^*)\)、\((Q_{0i}^*, Q_i^*)\) 三种情况下,相应的存贮损失费用 \( C_{w1i} \)、\( C_{w2i} \)、\( C_{w3i} \) 和发生的概率 \( P_{21} \)、\( P_{22} \)、\( P_{23} \)。这是以订货时间为决策变量的非线性规划问题。代入问题 2 及问题 4 中的数据,运用里 lingo 软件编程(程序见附录 6),求得 \( L^* = 6.240 \)(立方米)及此时对应三种商品的体积容量分别为 3.123 立方米、0.49 立方米、2.628 立方米。

\section{销售数量、订货时间均随机时,最佳订货模型的建立与讨论}

上述模型的重要特点在于,商品订货的提前期是随机的。然而,在实际工作中,商品的销售(即需求)通常也是随机的,这导致了在建立总损失费用函数时必须考虑各种更为复杂的存货情况以及相应的概率。这就要求我们建立并讨论销售数量、订货时间均随机时的最佳订货模型。

\subsection{模型的建立}

首先对模型进行如下假设:

(1) 假定在一个批量 $Q$ 到达仓库后,现有库存水平总是高于再订货水平 $r$,因此在一个提前期内,只会有一个订货,即不会发生合同交叉的问题。这个假设也保证了连续订货,产生随机提前期问题。

(2) 连续的随机提前期 $X$ 是相互独立的,其概率密度函数 $p(x)$、概率分布函数 $F(x)$ 已知。

(3) 单位时间需求服从是期望值为 $D$,标准方差为 $\delta$ 的分布,提前期需求密度函数就是 $f(x, DL, \delta, L)$,期望值为 $DL$,方差为 $\delta L$。

(4) 再订货点水平是非负的,$r \geq 0$;

(5) $h$ 为单位库存维持成本(单位时间单位库存量),$C_0$ 为固定订货成本(单位时间单位库存量),$C_s$ 为单位缺货成本(单位时间单位库存量)。

\begin{figure}[h]
\centering
\includegraphics[width=\textwidth]{image.png}
\caption{库存量与时间关系图}
\end{figure}

\subsection{模型分析}

对于随机型库存系统常见的订货和存贮策略有 3 种:$t$ 循环策略(定期订货法)、$(R, Q)$ 策略(定点订货法)、$(t, R, S)$ 策略(综合订货法)。

分析问题后我们发现,此问题的核心,就是让我们确定订货点以及订货量。由于商品的销售经常是随机的、订货情况在一段时间后也是会发生变化,比较适合使 $(R, Q)$ 策略,故我们将运用 $(R, Q)$ 策略,通过建立预期总费用函数,并对其进行最小化约束,进而确定订货点,实现模型求解。

\section*{5.3 模型建立}

假定提前期需求密度函数为 $f(x)$,分布函数为 $F(x)$。根据图2,从订货点 $t$ 到到货点 $t+L$ 是一个库存补充周期,我们可以计算出这个补充周期 $[t+L]$ 内的期望库存维持成本和缺货成本之和为:

\begin{align*}
E(c_a + c_b) &= h \int_{x=0}^{r} \frac{r^2 - (r-x)^2}{2} f(x) dx + \int_{x=r}^{+\infty} \left[ \frac{hr^2}{2} + \frac{C_s (x-r)^2}{2} \right] f(x) dx \\
&\quad + h \int_{x=0}^{+\infty} \left[ \frac{(r-x+Q)^2}{2} \right] f(x) dx - \frac{hr^2}{2} \\
&= -h \int_{x=0}^{r} \frac{(r-x)^2}{2} f(x) dx + C_s \int_{x=r}^{+\infty} \frac{(x-r)^2}{2} f(x) dx + h \int_{x=0}^{+\infty} \frac{(r-x+Q)^2}{2} f(x) dx
\end{align*}

根据单位时间需求 $M$ 是随机的,其期望值为 $D$,标准差为 $\delta$,其密度函数为 $f(x, DL, \delta, \sqrt{L})$,故我们得出总费用期望函数:

\[
E[TC(Q, r)] = \frac{C_0 D}{Q} + \left( h + C_s \frac{D}{Q} \right) \int_{x=r}^{+\infty} \frac{(x-r)^2}{2Q} f(x, DL, \delta \sqrt{L}) dx + h \int_{x=0}^{+\infty} \left[ r - x + \frac{Q}{2} \right] f(x, DL, \delta \sqrt{L}) dx
\]

目标为:

\[
\text{Min}(E[TC(Q, r)])
\]

约束条件:

(1) 满足假设一不会发生订货合同的交叉

\[
L < \frac{Q}{D}
\]

(2) 各数据满足非负性

\[
Q, L, D, \delta, h, C_0, C_s, r \text{ 非负}
\]

(3) 需求服从是期望值为 $D$,标准方差为 $\delta$ 的分布;连续的随机提前期 $X$ 是相互独立的,其概率密度函数为 $p(x)$、概率分布函数为 $F(x)$。

需求与提前期均随机时,最佳定货点的模型建立:

\[
\text{Min}(E[TC(Q, r)])
\]

\[
s.t.
\begin{cases}
L < \frac{Q}{D} \\
Q, L, D, \delta, h, C_0, C_s, r \in R^+
\end{cases}
\]

\section*{5.4 模型讨论}

(1) 可以通过平方根公式直接求得经济订货批量(EOQ),为

\begin{equation}
\text{E.0. } Q = \sqrt{2D \frac{C_0 + (h + C_s) \int_{x=r}^{+\infty} (x-r)^2 f(x, DL, \delta \sqrt{L}) dx}{h}}
\end{equation}

(2) 可以求得相应订货水平 \( r \) 应该满足关系:
\begin{equation}
\int_{x=r}^{+\infty} f(x, DL, \delta \sqrt{L}) dx = \frac{Qh}{h + DC_s / Q}
\end{equation}

(3) 相关讨论:

对于总费用期望函数而言, 固定 \( Q \), 即可以求解最优订货点 \( r \), \( r \) 满足:
\begin{equation}
\int_{x=r}^{+\infty} f(x, DL, \delta \sqrt{L}) dx = \frac{Qh}{h + DC_s / Q}
\end{equation}
该等式左边可以看作一个周期内的期望缺货量, 两边同除以 \( Q \), 就为单位时间缺货率的期望值 \( \frac{h}{h + DC_s / Q} \)。由于 \( D/Q \) 确定, 即期望值大小取决于 \( h \) 与 \( C_s \), 如果存货保管较大时, 在需求一定的情况下, 厂商可以通过增加缺货次数以降低总费用; 若单位缺货成本较大, 则厂商会损失一部分存贮费用, 尽量保证不出现缺货损失。

同理, 固定 \( r \), 我们可以求解优化经济订货量 \( Q \), E.0. \( Q \) 满足:
\begin{equation}
\text{E.0. } Q = \sqrt{2D \frac{C_0 + (h + C_s) \int_{x=r}^{+\infty} (x-r)^2 f(x, DL, \delta \sqrt{L}) dx}{h}}
\end{equation}
如果 \( h/C_s \) 较小, 则 E.0. \( Q \) 相应较大, 即厂商必须以较高的订货量以避免缺货情况的发生。反之, 则 E.0. \( Q \) 相应较小, 即厂商不愿意支付较高的维持费用, 而以较低的 E.0. \( Q \) 进行订货、存贮决策。

该模型考虑了设置并运用了较多的参数, 能够对实际中出现的销售数量、订货时间均随机的情况进行较好的模拟, 不足之处在于较多的参数造成模型运算是较大, 给模型求解增加了一定的麻烦。

\section*{6. 模型的评价与推广}

\subsection*{6.1 模型的优点}

(1) 模型能够有效的解决仓库容量一定、销售速率不变、到货时间不定的随机性存贮管理问题。模型的建立与求解依据概率统计的原理与方法, 结果合理可靠。

(2) 模型对企业在库存过程中产生各类费用的环节进行了透彻的阐释; 在订货时间随机的情况下, 详尽的分析了各类费用发生的概率; 同时在扩展模型中, 引入需求随机这一普遍现象, 进一步拓展了模型的适用范围。

(3) 模型运用时,通过对真实的到货数据进行详细统计分析,使得决策过程更为科学。  
(4) 模型求解多种商品存贮管理问题时,建立了以多种商品总存贮费用最小为目标函数的非线性“最佳库存安排模型”,并在此基础上建立“多种商品最优订货点决策模型”,采用分步规划的方法,构思精妙。

\subsection*{6.2 模型的不足}

(1) 模型运用了较多的参数,同时涉及较多的变量,因此在建模尤其是模型求解过程中,造成了较大的工作量。  
(2) 由于时间原因,我们没有对模型深入讨论,因此没有设计出一套手工算法,这为其在存货日常管理中的推广造成了一定困难。

\subsection*{6.3 模型的推广}

该模型广泛适用于随机性存贮问题,针对需求随机、(t,s)型存贮策略、多维随机变量混合影响等情况的最佳定货量均可作出较好的规划求解,能为企业进行订货决策、存贮决策提供有益参考。

\section*{参考文献}

1. 甘应爱、田丰等,《运筹学》[M],北京:清华大学出版社,2002年6月版;  
2. 陈希孺,《概率论与数理统计》[M],合肥:中国科技大学出版社,1993年版;  
3. 贾湖、张世英、赵蓉,《具有随机需求过程和随机供货时间的库存模型》[J],天津:天津大学学报 $P_{508}-P_{510}$,2001年第4期。