\begin{center}
\textbf{“华为杯”第十三届全国研究生数学建模竞赛}
\end{center}


\begin{center}
\textbf{题目} \quad 军事行动避空侦察的时机和路线选择
\end{center}

\begin{abstract}  
针对当前军用目标机动过程中易被卫星侦察的问题,对处于不同位置的机动目标进行侦察卫星覆盖特性分析,本文研究的是地面军用目标对于太空侦察卫星的反侦察行为。通过侦察手段,既可以确定卫星经过作战区的时间和星下点覆盖区域,为部队实施机动、规避、佯动提供最佳时机和最佳地域;又可以确定在作战过程中哪些作战对象被发现,从而判断出对方对我方部队行动的掌握情况,为作战指挥和作战行动筹划提供重要支持。在此前提条件下,分别研究题中所给的三个问题。

针对问题一,本文使用的是“单站短弧定轨”、坐标转换和卫星轨道运动学的建模思路。以附件所给的数据为基础,先分别算出 Q 型卫星,L 型卫星(L1 和 L2)以及 K 型卫星的“六根数”。接着能够得出,Q 型卫星在 D3 和 D5、L 型卫星在 8 月 23 日、K 型卫星未来三次被观测的情况(具体结果见正文)。还可以得到 Q 型卫星和 L 型卫星通过某施工区域的情况:在 D3 和 D5 两天中,Q 星只能在 D5 某一时间段侦察到该区域;在 8 月 23 日,L1 和 L2 均不能侦察到该区域。在本题中还讨论了 L 星的双星协同侦察的相对位置和薄弱环节;以及观测次数对于预报情况的精度影响,随着观测次数提升,精度显著增加。

针对问题二,本文使用的是“模拟退火算法”来选取全局最优解,将城市离散成点,取两者之间的连线通过百度地图修正来获得路程长度。首先简化模型,考虑普通公路和高速公路,对可供选择的 16 种路线运用模拟退火算法进行求

解。在以时间为目标函数的情况下,得出若全程走高速公路则时间最短。求得最优路线后,考虑需要经过的城市,把最优路线离散为 46 段小路线的组合。对这些段路线进行研究,选择合适的地点休息,以及和田执行任务的 24 小时,能够得到途径的地点及其时间。最后按分钟离散路线,确定车队所在位置,并和同时刻 L1 型卫星星下点比较。当车队在卫星侦察范围内下,调整车队的行车路线,计算 Q 型卫星过顶时间方法同上。经计算,当 L1 星可以侦察到车队时,车队正在奎屯休息,无须改变行车路线;当 Q 星可以侦察到时,车队正从阿克苏驶往巴楚,需就地停车隐藏。

针对问题三,本文在问题一的基础之上继续进行研究。与前两问不同,在该问题中,导弹发射装置更具有主动规避能力,根据预测得到的卫星轨道数据,实时地进行移动,做好军事反侦察行动。本文首先研究的是在单一型号卫星侦察时,导弹发射装置的安全时间和区域,提出了能够形象规划路线、展示数据的侦察时序图;接着讨论在多型号卫星联合侦察下,发射装置的移动方案。针对更多路线、区域等约束条件,采用增加时序图维数的解决方法。还提出了“侦察盲区”概念,在该区域内,战斗目标只要按照“侦察盲区”进行周期移动,就可以达到规避侦察卫星的效果。

关键词:单站短弧定轨,卫星轨道运动学,模拟退火算法,离散,侦察时序图,侦察盲区
\end{abstract}

\tableofcontents

\section{问题重述}

\subsection{问题背景}

大型国防工程施工、武器装备实验或部队大规模移动的隐蔽性关系到国家安全以及战争胜败,通常采用“避、变、骗、反”四种手段对付卫星侦察。“避”,就是掌握卫星运行规律,避开卫星过顶的时间段组织行动;“变”,就是针对侦察卫星的特点,相应地改变地面部队的活动规律,减弱卫星侦察的效果;“骗”,就是将军事目标伪装成非军事目标;“反”,就是利用各种武器摧毁卫星上的设备或卫星载体。无论哪种方式,都必须准确掌握卫星的运行规律。

未来高技术战争中,卫星是收集战略情报的主要来源,世界主要军事大国 $70\%$ 以上的战略情报均由卫星获得。通过航空、航天侦察手段可以获取对方兵力、兵器部署、变更、机动和作战等方面的情报。卫星侦察过程仿真,是现代作战仿真,通过仿真,一方面可以确定卫星经过作战区的时间和星下点覆盖区域,为部队实施机动、规避、佯动提供最佳时机和最佳地域;另一方面,通过仿真可以确定在作战过程中哪些作战对象被发现,从而判断出对方对我方部队行动的掌握情况,为作战指挥和作战行动筹划提供重要支持。

作为被侦察的一方,对侦察卫星的预警也就显得十分的重要,因此在这个背景下提出了一下几个待解决的问题。

\subsection{问题提出}

{问题一:}

某地域(地图坐标:北纬 $31.90 \sim 32.25$ 度;东经 $118.02 \sim 118.91$ 度)内拟建设一大型国防工程,计划利用境外卫星过顶的间隙组织施工。该地域长期受 Q 型、L 型卫星(有关数据见附件 1)监视。附件 2-1、附件 2-2、附件 2-3 是 Q 型、L 型、K 型卫星被配置在该区域内某观察站(北纬:$32.0209$ 度;东经:$118.7681$ 度)观测到的情况,请你们据此完成以下任务(注:附件中数据不是附件 4 中定义的“过顶时间”,而是观察站本次最早观察到卫星的时刻、卫星与观察站距离最近的时刻和本次观察结束的时刻,但它们之间可以换算):

1. 附件 2-1 给出了 D0、D1、D2 日 Q 型卫星被该观察站观测到的情况,请预测此后一天(D3)、此后三天(D5)的卫星被观测到的情况及过顶情况,并结合 Q 型卫星的侦察范围给出 D3、D5 两天内确保安全施工的时段。

2. L 型卫星是双星(L-1、L-2)协作工作。附件 2-2 给出了 L-1、L-2 卫星在 8 月 16 日-21 日被该观察站观测到的情况,请你们研究两星之间的相对位置的变化情况,由于 L 型卫星是雷达成像照相侦察卫星,能全天候、全天时进行侦察,并有一定的穿透能力,因而威胁比较大,请给出 8 月 23 日 L-1、L-2 卫星被观测到的情况及过顶情况和确保安全施工的时段(不考虑 Q 型卫星),并进一步寻找它们在侦察方面的薄弱环节。

3. 附件 2-3 是某卫星(记为 K 型)十次被该观察站观测到的情况,除此对其一无所知。请你们预测其未来三次的被观测到的情况,并说明该卫星已经被连续观察最少 $n$ 次才能够确定下次被观测到的情况所需要的 $n$,以及观察次数对预报精度的影响。

{问题二:}

{某部需要从新疆的阿勒泰隐蔽地经喀什运动到和田并在和田执行某任务,24小时后再隐蔽地返回阿勒泰(不必经喀什),部队可以按需要选择在高速公路(最大速度 100 公里/小时)或普通公路(除高速之外的其他公路,最大速度 50 公里)上行进,假设部队出发时(2016 年 11 月 1 日凌晨 5 时整)Q 型卫星、L-1 卫星(它们的轨道要素见附件 1,其他 L 型卫星都不考虑)均位于各自轨道的近地点。行车时车队最大长度 2 千米,部队每开进 10-12 小时可选择途经的县级以上(含县级)城市休息 10 小时以上(即连续开进时间不少于 10 小时,不多于 12 小时),请你们根据附件 3 给出的地图(必要时可借助因特网获取有关地理信息),设计合理的行军时机、路线和宿营地,避开 L-1 卫星侦察,并预测 Q 型卫星的过顶时刻,以便及时做好隐蔽工作,尽可能快地安全到达目的地。}

{问题三:}

研究导弹发射装置的战时隐蔽问题。有专家提出,运动方式可能是移动发射装置规避卫星侦察的有效方案。请你们研究以下问题:假设某移动发射装置可在某一指定区域内自由运动,分别研究针对 Q 型、L 型(包括 L-1、L-2)、K 型卫星的侦察能够规避的可行性、条件(区域大小、形状、路网状况及其他你们认为需要的条件)和方式。欢迎进一步针对两种或三种卫星的组合侦察能够规避的可行性、条件和方式。并考虑卫星参数变化对方案的影响。

\section{问题假设与符号说明}

\subsection{问题假设}

针对问题一:

(1) 设地球为球形,不考虑其扁率,星下点即为卫星与地心连线在地球表面的交点;

(2) 设卫星轨道为圆形,因为其偏心率较小,预报时默认半长轴、轨道倾角不变;

(3) 设太阳同步轨道卫星 Q 在一个轨道周期内升交点赤经不变,而每天变化 $0.986^\circ$,其他卫星则根据多次测量数据拟合升交点赤经的变化;

(4) 仅考虑 $J_2$ 摄动项的影响,即对轨道升交点赤经的影响;

(5) 半长轴和轨道倾角使用前几次所测数据的平均值;

(6) 在计算轨道六根数时采用双矢量定轨法。

针对问题二:

(1) 将连续的路程离散成多个点(县级市及以上)之间的直线距离,再通过网络地图数据进行修正;

(2) 在高速公路和普速公路上一般都按照最快速度行驶;

(3) 在公路上可以随时停车以便对侦察卫星进行隐蔽;

针对问题三:

(1) 三种卫星的初始位置分别使用的是附件 2-1、附件 2-2 和附件 2-3 所给的数据;

(3) 仅考虑经度规避,不考虑纬度规避,因为卫星轨道倾角都较大,会沿着纬度变化方向扫过较大面积;

(2) 其余假设类似于问题一的假设。

\subsection{符号说明}

在此,对本文所使用的符号进行定义。

针对问题一:

\begin{tabular}{|c|c|}
\hline 符号 & 定义 \\
\hline $r_i$ & 单站定轨观测的第 $i$ 次位置矢量数据 \\
$a$ & 半长轴 \\
$e$ & 偏心率 \\
$E$ & 偏近点角 \\
\hline
\end{tabular}

\begin{table}
\centering
\begin{tabular}{|c|c|}
\hline
$i$ & 轨道倾角 \\
\hline
$\Omega$ & 升交点赤经 \\
\hline
$u$ & 纬度幅角 \\
\hline
$f$ & 真近点角 \\
\hline
$\mu$ & 地球引力常数 \\
\hline
\end{tabular}
\end{table}

针对问题二:

\begin{table}
\centering
\begin{tabular}{|c|c|}
\hline
符号 & 定义 \\
\hline
$t$ & 来回总时间 \\
\hline
$L$ & 总路程 \\
\hline
$L_G$ & 高速路程 \\
\hline
$L_P$ & 普速路程 \\
\hline
$v_G$ & 高速公路行驶速度 \\
\hline
$v_P$ & 普速公路行驶速度 \\
\hline
$t_G$ & 高速公路行驶时间 \\
\hline
$t_P$ & 普速公路行驶时间 \\
\hline
$t_R$ & 行进途中休息时间 \\
\hline
$t_H$ & 和田执勤时间,为24小时 \\
\hline
$t_Y$ & 隐蔽侦察卫星时间 \\
\hline
\end{tabular}
\end{table}

针对问题三:

\begin{table}
\centering
\begin{tabular}{|c|c|}
\hline
符号 & 定义 \\
\hline
$T$ & 卫星运行一圈的周期 \\
\hline
$\dot{\Omega}$ & 升交点赤经变化率 \\
\hline
\end{tabular}
\end{table}

\begin{tabular}{c l}
$\tilde{\omega}$ & 地球自转角速度 \\
$m$ & 每一圈卫星星下点经度西移角度 \\
$W$ & 卫星辐照宽度 \\
\end{tabular}

\section{问题一的分析与建模}

\subsection{问题分析}

问题一虽然分为三小问,但是本质是相同的,都是采用“单站定轨”的方法。下面首先介绍一下“单站定轨”,也可以认为是“单站短弧定轨”的基本原理。

卫星轨道确定通常有两个概念: 初轨计算和轨道改进(现称精密定轨)。前者是在没有任何初始信息的前提下利用短弧资料来确定轨道, 它和无摄运动的二体问题紧密相联。而后者则对应“完整”力学系统(符合问题精度要求的力学模型)的受摄二体问题, 通常需要较长的测量弧段, 这是定轨精度要求所需要的。就定轨的基本原理而言, 两者并无实质性差别。但精密定轨涉及多变元迭代, 必须具有一定精度的初始信息, 这正是轨道“改进”之意, 改进的对象就是精度未达到要求的初始信息; 而初轨确定往往是在缺乏初始信息情况下进行的, 这是精密定轨方法无法替代的。在本题的分析过程中, 遵循初轨的确定方法 \cite{ref1}。

\subsection{模型建立}

{单站定轨原理 \cite{ref1,ref2,ref3}}

如已知某一时刻卫星在赤道惯性坐标的位置矢量 $\boldsymbol{r}$ 和速度矢量 $\boldsymbol{v}$, 则可直接得出卫星轨道的六要素。通常, 单次观测仅获得卫星在测站坐标系下位置矢量 $\boldsymbol{\rho}$。初轨确定的任务是由卫星的多个位置矢量计算轨道的六要素。最常用的方法是高斯方法。

假定在时刻 $t_{1}$, $t_{2}$ 分别测得卫星的位置矢量 $\boldsymbol{\rho}_{1}$, $\boldsymbol{\rho}_{2}$, 经测站坐标转换且考虑地球的自转, 得出该两时刻卫星在赤道惯性坐标的位置矢量 $\boldsymbol{r}_{1}$, $\boldsymbol{r}_{2}$ 两位置的时刻差 $\Delta t$ 和真近点角距 $\Delta f$:

\begin{equation}
\Delta t = t_{1} - t_{2}
\end{equation}

\begin{equation}
\Delta f = \arccos \left( \frac{\boldsymbol{r}_{1} \cdot \boldsymbol{r}_{2}}{r_{1} r_{2}} \right)
\end{equation}

通过两位置的轨道可有无限多, 但时间差为 $\Delta t$ 的轨道是唯一的。通常两位置的测轨时间差小于轨道周期, 对于二体轨道, 在位置 $\boldsymbol{r}_{1}$ 的卫星速度矢量 $\boldsymbol{v}_{1}$ 位于 $\boldsymbol{r}_{1}$ 和 $\boldsymbol{r}_{2}$ 的平面内(即轨道平面),见图 3-1,有矢量的线性组合式

\begin{equation}
\boldsymbol{r}_{2} = F \boldsymbol{r}_{1} + G \boldsymbol{v}_{1}
\tag{3-1}
\end{equation}

\begin{figure}[h]
    \centering
    \includegraphics[width=0.8\textwidth]{satellite_orbit_diagram.png}
    \caption{卫星轨道示意图}
    \label{fig:3-1}
\end{figure}

式中,$F$ 和 $G$ 为标量。以 $\boldsymbol{v}_{1}$ 和 $\boldsymbol{r}_{1}$ 分别叉乘上式,即可分离出标量 $F$、$G$,有
\begin{equation}
\left\{
\begin{aligned}
\left(\boldsymbol{r}_{1} \times \boldsymbol{v}_{1}\right) F &= \left(\boldsymbol{r}_{2} \times \boldsymbol{v}_{1}\right) \\
\left(\boldsymbol{r}_{1} \times \boldsymbol{v}_{1}\right) G &= \left(\boldsymbol{r}_{1} \times \boldsymbol{r}_{2}\right)
\end{aligned}
\right.
\tag{3-2}
\end{equation}
上式两端的矢量都沿轨道面的法线,应用动量矩等式 $h = \left|\boldsymbol{r}_{1} \times \boldsymbol{v}_{1}\right|$,如两位置的角距小于 $90^{\circ}$,则有
\begin{align*}
F &= \frac{1}{h} \left|\boldsymbol{r}_{2} \times \boldsymbol{v}_{1}\right| \\
G &= \frac{1}{h} \left|\boldsymbol{r}_{1} \times \boldsymbol{r}_{2}\right|
\end{align*}
引用近地点坐标系 $OPQW$,单位矢量 $\boldsymbol{P}$ 在轨道面内指向近地点,单位矢量 $\boldsymbol{Q}$ 在轨道面内超前 $\boldsymbol{P} \, 90^{\circ}$,单位矢量 $\boldsymbol{W}$ 沿轨道法线且与动量矩 $\boldsymbol{h}$ 方向一致。相对此坐标系,卫星的位置 $\boldsymbol{r}$ 和速度 $\boldsymbol{v}$ 可列为
\begin{align*}
\boldsymbol{r} &= r \cos f \boldsymbol{P} + r \sin f \boldsymbol{Q} \\
\boldsymbol{v} &= \sqrt{\frac{\mu}{p}} \left[-\sin f \boldsymbol{P} + (e + \cos f) \boldsymbol{Q}\right] \\
\boldsymbol{h} &= \sqrt{\mu p} \boldsymbol{W}
\end{align*}
可将第一、第二代入式 (3-2),并使用轨道椭圆方程,它的等效公式 $r = r / (1 + e \cos f)$,可得标量 $F$、$G$ 为
\begin{equation}
\begin{aligned}
F &= \frac{1}{p} \left[r_{2} \cos f_{2} (e + \cos f_{1}) + r_{2} \sin f_{2} \sin f_{1}\right] = 1 - \frac{r_{3}}{p} (1 - \cos \Delta f) \\
G &= \frac{r_{1} r_{2} \sin \Delta f}{\sqrt{\mu p}}
\end{aligned}
\tag{3-3}
\end{equation}
上式中 $f_{1}$、$f_{2}$ 为位置 $\boldsymbol{r}_{1}$、$\boldsymbol{r}_{2}$ 的真近点角,两位置的真近点角,两位置的真近点角 $\Delta f = f_{2} - f_{1}$ 可由测轨直接获得。

因此,如求得轨道的半通径 $p$,可直接得出标量值 $F$、$G$,代入式 (3-1),得出在位置 $\boldsymbol{r}_{1}$ 的轨道速度 $\boldsymbol{v}_{1}$:
\begin{equation}
\boldsymbol{v}_{1} = \frac{1}{G} \left(\boldsymbol{r}_{2} - F \boldsymbol{r}_{1}\right)
\tag{3-4}
\end{equation}
从而可以完全确定轨道。

通常采用迭代法求半通径 $p$。设定初值 $p^{(0)}$:
\begin{equation}
p^{(0)} = \frac{1}{2} (r_{1} + r_{2})
\end{equation}
并计算 $e \cos f_{i}$ 和 $e \sin f_{i} \, (i = 1, 2)$ 的值,并计算式为:

\begin{align*}
e \cos f_i &= \frac{p^{(0)}}{r_i} - 1 \quad i = 1, 2 \\
e \sin f_1 &= \frac{1}{\sin \Delta f} \left[ e \cos f_1 \cos \Delta f - e \cos f_2 \right] \\
e \sin f_2 &= \frac{1}{\sin \Delta f} \left[ -e \cos f_2 \cos \Delta f + e \cos f_1 \right]
\end{align*}

由此可得轨道要素的估值为

\begin{align*}
e &= \left[ \left( e \cos f_1 \right)^2 + \left( e \sin f_1 \right)^2 \right]^{\frac{1}{2}} \\
a &= p^{(0)} / \left( 1 - e^2 \right) \\
n &= \left( \mu / a^3 \right)^{\frac{1}{2}}
\end{align*}

以及两位置的偏近点角参数:

\begin{align*}
\sin E_i &= \frac{r_i}{p} \left( 1 - e^2 \right)^{\frac{1}{2}} \sin f_i, i = 1, 2 \\
\cos E_i &= \frac{r_i}{p} \left( e + \cos f_i \right), i = 1, 2
\end{align*}

根据平近点角 \( M \) 的定义最后得出两位置的飞行时间 \( \Delta t^{(0)} \)

\[
\Delta t^{(0)} = \frac{1}{n} \left( M_2 - M_1 \right) = \frac{1}{n} \left[ \left( E_2 - E_1 \right) + e \left( \sin E_1 - \sin E_2 \right) \right]
\]

比较飞行时间的计算值和实测值之差 \( t_B \):

\[
t_B = \Delta t - \Delta t^{(0)} \tag{3-5}
\]

选择更新值 \( p^{(1)} \),迭代结果,直至 \( |t_B| << \varepsilon \)。这样就可以根据“单站短弧定轨”测出卫星的轨道,也就可以得出其星下点轨迹。

{2. 参考坐标系确定}

地球惯性坐标系:本解答使用的地球惯性坐标系以 J2000 坐标系为基础进行更改,坐标原点位于地球质心,Z 轴与地球自转轴重合且不考虑其进动章动,X 轴在 D0 天 00:00:00 时刻从地心指向 0 经度线,Y 轴与 X 轴 Z 轴构成右手直角坐标系。

地球固连坐标系:该坐标系统自转轴随地球转动而转动,在 D0 天 00:00:00 时刻与地球惯性坐标系重合。

{3. 卫星轨道动力学}

卫星轨道动力学主要是研究卫星在无主动控制时,地球、太阳、月球等重力场和其他摄动力作用下的质点运动学问题。本文主要考虑 \( J_2 \) 摄动项的作用。

{(1) 二体问题的轨道特性}

在最简单的研究卫星相对地球运动(或者航天器相对中心体运动)的情况时,卫星的尺寸远远小于它和地球之间的距离,因此可以视为质点。地球又可以近似地视作理想球体(即质量分布均匀的完美球体或其质量等密度层呈一定厚度的同心球壳),就引力效果而言,卫星和地球近似地看成一个二体模型。根据二体问题运动方程,卫星在前文中规定的地球惯性坐标系中的运动方程为

\[
\begin{cases}
\dot{x} = v_x & \dot{v}_x = -\frac{\mu_e x}{r^3} \\
\dot{y} = v_y & \dot{v}_y = -\frac{\mu_e y}{r^3} \\
\dot{z} = v_z & \dot{v}_z = -\frac{\mu_e z}{r^3}
\end{cases}
\]

式中 $\boldsymbol{X} = \begin{bmatrix} x & y & z & v_x & v_y & v_z \end{bmatrix}^{\mathrm{T}}$ ——为卫星的状态变量

\[
r = \sqrt{x^2 + y^2 + z^2} \quad \text{——为卫星到地心得距离;}
\]

\[
\mu_e = GM_e \quad \text{——} G \text{为万有引力常数,} M_e \text{为地球质量。}
\]

这是一个非线性微分方程,若给定某时刻 $t$ 的卫星位置和卫星速度,即该时刻卫星状态,则该方程完全可解。这些初始条件也可以由另外 6 个积分常数代替表示,即轨道六要素:轨道半长轴 $a$,偏心率 $e$,轨道倾角 $i$,升交点赤经 $\Omega$,近升角距 $\omega$,偏近点角 $E$(或真近点角 $f$ 和平近点角 $M$,取决于过近地点时刻 $\tau$)。

{(2) 卫星轨道摄动力分析}

卫星在空间中运动时除了受到地球中心引力的作用外,还受到来自宇宙空间的各种摄动力的作用,这些摄动力会使卫星轨道发生变化,包括轨道形状的变化及轨道法向方向的变化等。考虑到卫星可能受到的各种摄动力,在上文基础上建立卫星轨道动力学方程:

\[
\begin{cases}
\dot{\boldsymbol{r}} = \boldsymbol{v} \\
\dot{\boldsymbol{v}} = -\mu_e \frac{\boldsymbol{r}}{r^3} + \boldsymbol{P}(\boldsymbol{r}, \boldsymbol{v}, t)
\end{cases}
\]

式中 $\boldsymbol{P} = \boldsymbol{P}_G + \boldsymbol{P}_D + \boldsymbol{P}_R + \boldsymbol{P}_S + \boldsymbol{P}_M$ ——为卫星受到的摄动加速度组成 \footnote{[i]}

\[
\boldsymbol{v} = \begin{bmatrix} v_x & v_y & v_z \end{bmatrix}^{\mathrm{T}} \quad \text{——为卫星速度}
\]

\[
\boldsymbol{r} = \begin{bmatrix} x & y & z \end{bmatrix}^{\mathrm{T}} \quad \text{——为卫星位置}
\]

\[
\boldsymbol{P}_G \quad \text{——为地球非球形引力摄动加速度} \quad \boldsymbol{P}_D \quad \text{——为大气阻力摄动加速度}
\]

\[
\boldsymbol{P}_R \quad \text{——为太阳光压摄动加速度} \quad \boldsymbol{P}_S \quad \text{——为太阳对卫星引力摄动加速度}
\]

\[
\boldsymbol{P}_M \quad \text{——为月球对卫星引力摄动加速度}
\]

根据本研究的问题与情况,忽略除地球非球形引力 $J_2$ 摄动以外的其他摄动,而 $J_2$ 摄动引起的升交点赤经的变化也由多次测量数据拟合得出,而非使用轨道

动力学积分计算。

{坐标系变换 \cite{ref4}}

在讨论、分析和计算物体的运动特性时,要将涉及的物理量统一到一个坐标系内,这常需要进行坐标系间矢量坐标的转换。两个坐标系的关系有多种描述方法,如方向余弦矩阵、欧拉角、四元数等,方向余弦矩阵可以方便地实现矢量坐标的转换。本文的求解也是主要基于方向余弦阵进行的。

单位矢量在坐标系三轴上的投影称为方向余弦。方向余弦阵是指以某个坐标系三轴的单位矢量在另一个坐标系中的方向余弦为列组成的矩阵。例如,设有两个坐标系 \( A \) 和 \( B \),它们间的方向余弦阵为

\begin{equation}
\mathbf{C}_{A}^{B} =
\begin{pmatrix}
i_{B} \cdot i_{A} & j_{B} \cdot i_{A} & k_{B} \cdot i_{A} \\
i_{B} \cdot j_{A} & j_{B} \cdot j_{A} & k_{B} \cdot j_{A} \\
i_{B} \cdot k_{A} & j_{B} \cdot k_{A} & k_{B} \cdot k_{A}
\end{pmatrix}
\tag{3-6}
\end{equation}

若把两个坐标系三轴的单位矢量看成三维矢量空间的两组基,则由线性空间的理论可知,方向余弦阵实际上是两组基之间的过渡矩阵。矢量 \( P \) 在两个坐标系中坐标的关系满足

\begin{equation}
\begin{bmatrix}
x_{A} \\
y_{A} \\
z_{A}
\end{bmatrix}
=
\mathbf{C}_{A}^{B} \cdot
\begin{bmatrix}
x_{B} \\
y_{B} \\
z_{B}
\end{bmatrix}
\tag{3-7}
\end{equation}

方向余弦阵是正交矩阵,故有

\begin{equation}
\left( \mathbf{C}_{A}^{B} \right)^{-1} =
\left( \mathbf{C}_{A}^{B} \right)^{T} =
\mathbf{C}_{A}^{B}
\end{equation}

方向余弦阵具有传递性,三个坐标系 \( A \)、\( B \)、\( C \) 间的方向余弦阵具有如下关系:

\begin{equation}
\mathbf{C}_{C}^{A} =
\mathbf{C}_{B}^{A} \cdot \mathbf{C}_{C}^{B}
\tag{3-8}
\end{equation}

\begin{figure}[h]
\centering
\includegraphics[width=0.8\textwidth]{coordinate_rotation_diagram.png}
\caption{坐标系旋转示意图}
\end{figure}

\subsection{问题求解}

{第一问}

附件 2-1 给出了 D0、D1、D2 日 Q 型卫星被该观察站观测到的情况,请预测此后一天(D3)、此后三天(D5)的卫星被观测到的情况及过顶情况,并结合 Q 型卫星的侦察范围给出 D3、D5 两天内确保安全施工的时段。

Q 型卫星是太阳同步卫星,这类卫星每天在相同的地方时经过世界各地,每天以大致相同的太阳对地光照条件下观测地面,因此可以用作侦察探测。

题目附件给出了前三天经过地面某一观测站时所能测量的数据,根据之前的单站定轨原理可以计算出该卫星的轨道六根数\cite{ref5}。(Q 型卫星所有观测计算出的数据见附录二表 1)

\textbf{地心距 $r$ 和速度 $v$:}
\begin{equation}
r = \sqrt{x_0^2 + y_0^2 + z_0^2}
\tag{3-9}
\end{equation}
\begin{equation}
v = \sqrt{\dot{x}_0^2 + \dot{y}_0^2 + \dot{z}_0^2}
\end{equation}

\textbf{半长轴 $a$(通过活力公式求取):}
\begin{equation}
a = \frac{\mu r}{2\mu - r v^2}
\tag{3-10}
\end{equation}

\textbf{偏心率 $e$ 和 $t_0$ 时刻偏近点角 $E$:}
\begin{equation}
e \sin E = \sqrt{\frac{1}{\mu a}} \left( x \dot{x} + y \dot{y} + z \dot{z} \right)
\tag{3-11}
\end{equation}
\begin{equation}
e \cos E = 1 - \frac{r}{a}
\end{equation}

\textbf{轨道倾角 $i$:}
\begin{equation}
\cos i = \frac{x \dot{y} - y \dot{x}}{\sqrt{\mu a \left( 1 - e^2 \right) \sin i}}
\tag{3-12}
\end{equation}

\textbf{升交点赤经 $\Omega$:}
\begin{equation}
\sin \Omega = \frac{y \dot{z} - z \dot{y}}{\sqrt{\mu a \left( 1 - e^2 \right) \sin i}}
\tag{3-13}
\end{equation}
\begin{equation}
\cos \Omega = \frac{x \dot{z} - z \dot{x}}{\sqrt{\mu a \left( 1 - e^2 \right) \sin i}}
\end{equation}

\textbf{$t_0$ 时刻卫星纬度幅角 $u$:}
\begin{equation}
\sin u = \frac{z}{r \sin i}
\tag{3-14}
\end{equation}
\begin{equation}
\cos u = \frac{y}{r} \sin \Omega + \frac{x}{r} \cos \Omega
\end{equation}

\begin{table}
\centering
\caption{计算出的Q型卫星最后一次被观测的轨道参数}
\begin{tabular}{|c|c|c|c|c|}
\hline
半长轴$a$ & 偏心率$e$ & 倾角$i$ & 升交点赤经$\Omega$ & 纬度幅角$\mu$ \\
\hline
6872.039088 & 0.0012984 & 97.385854 & 121.933354 & 35.37719849 \\
\hline
\end{tabular}
\end{table}

表3-1所计算出的参数与所给的Q型卫星轨道参数基本吻合,因此该方法正确,可以做进一步计算。且由于单站定轨的原理,每一次观测的三组数据仅能得到前两组的轨道参数。

对于Q型卫星的预报从最后一次观测的时刻开始,半长轴、偏心率和轨道倾角都使用前面观测到的平均值;纬度幅角使用最后一次观测到的值(纬度幅角是真近点角和近地点幅角之和);虽然根据观测数据,Q卫星的升交点赤经并没有呈现一般太阳同步轨道的变化特性,但是考虑题设还是使其每天东移0.9856°。

\begin{figure}[h]
\centering
\includegraphics[width=0.8\textwidth]{image.png} % 替换为实际图片路径
\caption{Q卫星升交点赤经变化}
\end{figure}

预报卫星在之后几天在该观测站上空的过顶情况以及某一区域的安全施工时段。由于只关注D3和D5的情况(Q型卫星的后三天全部过顶情况见附录二表2),可得

\begin{table}
\centering
\caption{Q型卫星在D3和D5的过顶情况}
\begin{tabular}{|c|c|c|c|c|}
\hline
日期 & 时间/24小时制 & Azm/$^\circ$ & Elv/$^\circ$ & Range/Km \\
\hline
\multirow{5}{*}{D3} & 0:31:15 & 209.69363 & 10.043725 & 1675.395818 \\
\cline{2-5}
 & 0:34:09 & 263.43348 & 22.623222 & 1089.653948 \\
\cline{2-5}
 & 0:37:03 & 317.01876 & 10.004562 & 1677.978638 \\
\cline{2-5}
 & & & & \\
\cline{2-5}
 & 12:03:02 & 31.645962 & 10.010298 & 1677.599967 \\
\hline
\end{tabular}
\end{table}

\begin{table}
\centering
\begin{tabular}{c c c c c}
\hline
 & 12:06:21 & 98.386905 & 33.94343 & 822.4140348 \\
\hline
 & 12:09:39 & 164.90886 & 10.078348 & 1673.117304 \\
\hline
 & & & & \\
\hline
 & 13:38:45 & 303.38719 & 10.009459 & 1677.655323 \\
\hline
 & 13:39:31 & 291.02782 & 10.547074 & 1642.718402 \\
\hline
 & 13:40:17 & 278.66221 & 10.018705 & 1677.045206 \\
\hline
 & & & & \\
\hline
\multirow{7}{*}{D5} & 12:51:02 & 1.8838369 & 10.006499 & 1677.850737 \\
\hline
 & 12:54:34 & 284.11606 & 55.772383 & 586.8785921 \\
\hline
 & 12:58:06 & 205.26074 & 10.084953 & 1672.683161 \\
\hline
 & & & & \\
\hline
 & 23:23:25 & 126.66851 & 10.027791 & 1676.445945 \\
\hline
 & 23:26:19 & 72.990168 & 23.242162 & 1070.431508 \\
\hline
 & 23:29:14 & 18.962791 & 10.007138 & 1677.808554 \\
\hline
\end{tabular}
\end{table}

(Azm—天线方位角度,Elv—天线仰角,Range—星与观测者之间距离)

在求得了Q的过顶情况之后,继续求解Q型卫星对该施工区域的监测情况。Q型卫星的监测范围设置为垂直于星下点轨迹方向的正负272Km内。可得如下结果:

表3-3Q型卫星可以侦察到施工区域的时间和位置

\begin{table}
\centering
\begin{tabular}{c c c c}
\hline
天数 & 时间 & 星下点纬度 & 星下点经度 \\
\hline
\multirow{6}{*}{D5} & 12:54:46 & 31.89985 & 115.408306 \\
\hline
 & 12:54:47 & 31.837076 & 115.392779 \\
\hline
 & 12:54:48 & 31.774302 & 115.377268 \\
\hline
 & 12:54:49 & 31.711526 & 115.361772 \\
\hline
 & 12:54:50 & 31.64875 & 115.346291 \\
\hline
 & 12:54:51 & 31.585973 & 115.330825 \\
\hline
\end{tabular}
\end{table}

从计算结果得(后三天所有侦察到施工区域的时间和位置见附录2表3)图3-4和图3-5,D3和D5这两天中只有D5才能侦察到施工区域。

\begin{figure}[h]
    \centering
    \includegraphics[width=\textwidth]{image1.png}
    \caption{Q 星在 D3 天星下点轨迹图与施工区域图(红色代表施工区域)}
    \label{fig:3-4}
\end{figure}

\begin{figure}[h]
    \centering
    \includegraphics[width=\textwidth]{image2.png}
    \caption{Q 星在 D5 天星下点轨迹图与施工区域图(红色代表施工区域)}
    \label{fig:3-5}
\end{figure}

因此,D3 和 D5 可以安全施工的时段是除去 D5 的 12:54:46 到 12:54:51 之外的时间,由于在准备隐蔽和解除隐蔽的过程中也会耗费相应的时间。因此,被刨除的时间可以酌情增加。

{第二问}

第二问与第一问原理相同,首先可以根据求得的 L1 星和 L2 星的轨道参数求出在 8 月 23 日 L1 星和 L2 星被观测到的情况,再求出施工区域在 8 月 23 日安全施工的时间段。

表 3-6、表 3-7 分别给出了 L1、L2 卫星最后一次被观测站观测到后计算的轨道参数,L 型卫星预报的数据使用情况同 Q 星。图 3-6、图 3-7 展示了 L1、L2 卫星升交点赤经变化及拟合情况。

\begin{table}
\centering
\caption{计算出的L1型卫星最后一次被观测的轨道参数}
\begin{tabular}{|c|c|c|c|c|}
\hline
半长轴$a$ & 偏心率$e$ & 倾角$i$ & 升交点赤经$\Omega$ & 纬度幅角$\mu$ \\
\hline
7021.1 & 0.002662 & 68.06021 & 21.19967796 & 33.31295117 \\
\hline
\end{tabular}
\end{table}

\begin{table}
\centering
\caption{计算出的L2型卫星最后一次被观测的轨道参数}
\begin{tabular}{|c|c|c|c|c|}
\hline
半长轴$a$ & 偏心率$e$ & 倾角$i$ & 升交点赤经$\Omega$ & 纬度幅角$\mu$ \\
\hline
7034.93 & 0.000689 & 57.03292 & 143.9515131 & 151.528566 \\
\hline
\end{tabular}
\end{table}

\begin{figure}[h]
\centering
\includegraphics[width=0.8\textwidth]{image1.png}
\caption{L1星升交点赤经随天数变化情况图}
\end{figure}

\begin{figure}[h]
\centering
\includegraphics[width=0.8\textwidth]{image2.png}
\caption{L2星升交点赤经随天数变化情况图}
\end{figure}

\begin{table}
\centering
\caption{L1星在8月23日被观测到的情况}
\begin{tabular}{|c|c|c|c|c|}
\hline
 & 时间 & Azm & Elv & Range \\
\hline
23日 & 3:03:42 & 12.4248523 & 10.0219717 & 2030.14075 \\
\hline
\end{tabular}
\end{table}

\begin{table}
\begin{tabular}{c c c c c}
\hline
 & 3:06:55 & 57.2208116 & 18.4873005 & 1543.85024 \\
\hline
 & 3:10:07 & 101.7291 & 10.0106162 & 2030.94575 \\
\hline
 & & & & \\
\hline
 & 4:42:50 & 317.321984 & 10.0203684 & 2030.25438 \\
\hline
 & 4:47:01 & 251.512364 & 35.763736 & 1018.257 \\
\hline
 & 4:51:12 & 185.445793 & 10.0586526 & 2027.54353 \\
\hline
 & & & & \\
\hline
 & 17:12:27 & 164.092396 & 10.0644406 & 2027.13415 \\
\hline
 & 17:16:17 & 107.026841 & 26.6683509 & 1238.93434 \\
\hline
 & 17:20:07 & 50.1519407 & 10.0204252 & 2030.25036 \\
\hline
 & & & & \\
\hline
 & 18:52:31 & 246.795613 & 10.0250789 & 2029.92055 \\
\hline
 & 18:56:14 & 301.160718 & 23.8571016 & 1329.67008 \\
\hline
 & 18:59:56 & 355.260847 & 10.065507 & 2027.05873 \\
\hline
\end{tabular}
\end{table}

由表可知,L1星在8月23日会在四个时间段被观测到。

表3-7 L2星在8月23日被观测到的情况

\begin{table}
\begin{tabular}{c c c c c}
\hline
 & 时间 & Azm & Elv & Range \\
\hline
\multirow{10}{*}{23日} & 2:32:23 & 196.2881 & 10.06342 & 2054.64 \\
\cline{2-5}
 & 2:36:53 & 123.3181 & 46.39037 & 870.8186 \\
\cline{2-5}
 & 2:41:23 & 50.56916 & 10.04545 & 2055.918 \\
\cline{2-5}
 & & & & \\
\cline{2-5}
 & 4:14:58 & 267.8202 & 10.02949 & 2057.053 \\
\cline{2-5}
 & 4:18:36 & 318.547 & 21.61623 & 1434.834 \\
\cline{2-5}
 & 4:22:14 & 9.172473 & 10.00096 & 2059.086 \\
\cline{2-5}
 & & & & \\
\cline{2-5}
 & 11:08:54 & 336.9267 & 10.03586 & 2056.6 \\
\cline{2-5}
 & 11:13:20 & 47.08755 & 40.02929 & 960.3396 \\
\cline{2-5}
 & 11:17:46 & 117.3375 & 10.03815 & 2056.437 \\
\hline
\end{tabular}
\end{table}

\begin{table}
\begin{tabular}{c c c c c}
\hline
 & & & & \\
\hline
 & 12:51:03 & 290.4237 & 10.02707 & 2057.226 \\
\hline
 & 12:54:37 & 240.9404 & 21.64557 & 1433.683 \\
\hline
 & 12:58:12 & 191.0868 & 10.04017 & 2056.293 \\
\hline
\end{tabular}
\end{table}

由表可知,L2星在8月23日会在四个时间段被观测到。

\begin{figure}[h]
\centering
\includegraphics[width=\textwidth]{image.png}
\caption{L1/L2卫星8月23日星下点轨迹与施工区域}
\end{figure}

经计算得图3-8,在8月23日L1和L2均不能侦察到该区域,而在22或24日可以侦察得到(见附录二表4和表5)。L1和L2型卫星的侦察范围同Q型卫星。

L1和L2相对位置和薄弱环节:

这里使用了问题三的时序图来描述L1和L2卫星之间的相对位置变化情况。基本能够确定L1和L2卫星实现了全球经度的范围的全覆盖。若要对其进行规避则只能选择重返间隔之间的区域。

\begin{figure}[h]
    \centering
    \includegraphics[width=\textwidth]{image1.png}
    \caption{L1/L2 卫星相对位置变化图(蓝-L1,绿-L2)}
    \label{fig:3-9}
\end{figure}

{第三问}

题目给出 K 型卫星 10 次的观测数据,来推测出未来三次的观测情况。与之前给的不同,这次附件 2-3 给出的观测数据不包含观测天数,即不清楚十次观测数据是否为连续的天数内观察。

通过使用前三次观测数据预报,认为这 10 次观测数据是连续产生的,且 K 星的升交点赤经也能较好的拟合出一条一次曲线,拟合度很高,$R^{2}=0.9836$

\begin{figure}[h]
    \centering
    \includegraphics[width=\textwidth]{image2.png}
    \caption{K 星升交点赤经随天数变化情况图}
    \label{fig:3-10}
\end{figure}

接下来的方法和前两问类似,通过多次的观测值,可以求出 K 型卫星的轨道参数,如表 3-8 所示。

\begin{table}
\centering
\begin{tabular}{|c|c|c|c|c|}
\hline
半长轴$a$ & 偏心率$e$ & 倾角$i$ & 升交点赤经$\Omega$ & 纬度幅角$\mu$ \\
\hline
6993.119834 & 1.53E-03 & 95.98529528 & 240.6225424 & 145.6557421 \\
\hline
\end{tabular}
\caption{计算出的K型卫星轨道参数(升交点赤经和纬度幅角为最后一次值)}
\end{table}

通过计算出的K型卫星轨道参数预测未来三次的被观测情况,见表3-9。

\begin{table}
\centering
\begin{tabular}{|c|c|c|c|c|}
\hline
日期 & 时间 & Azm & Elv & Range \\
\hline
\multirow{10}{*}{D3} & 7:50:15 & 171.3688 & 10.05045 & 1963.261 \\
\cline{2-5}
 & 7:54:32 & 259.4261 & 84.09031 & 618.1153 \\
\cline{2-5}
 & 7:58:49 & 348.8681 & 10.08411 & 1960.908 \\
\cline{2-5}
 & & & & \\
\cline{2-5}
 & 19:41:12 & 33.2957 & 10.08072 & 1961.145 \\
\cline{2-5}
 & 19:44:59 & 95.65555 & 30.26189 & 1093.455 \\
\cline{2-5}
 & 19:48:45 & 157.6198 & 10.08468 & 1960.869 \\
\cline{2-5}
 & & & & \\
\cline{2-5}
 & 21:18:06 & 328.579 & 10.01991 & 1965.399 \\
\cline{2-5}
 & 21:20:51 & 288.3769 & 16.90046 & 1560.923 \\
\cline{2-5}
 & 21:23:36 & 248.1505 & 10.02324 & 1965.165 \\
\hline
\end{tabular}
\caption{K型卫星未来三次被观测到的情况}
\end{table}

接下来讨论观测次数$n$对精度的影响。

以附录2-3为基础数据,该附录给出了十次观测数据,选择数量不等的数据(1~8次)中观测数据来预报第九、十次的卫星与观测站距离最近时刻。结果如表3-10所示。

\begin{table}
\centering
\begin{tabular}{|c|c|c|c|c|}
\hline
\multicolumn{2}{|c|}{n次观测预报效果} & & & \\
\hline
 & & 距离最近时间/秒 & 题目给出时间/秒 & 差/秒 \\
\hline
1次观测 & 9 & 249024 & 243007 & 6017 \\
\hline
\end{tabular}
\caption{不同观测次数对预报精度的影响}
\end{table}

\begin{table}
\centering
\begin{tabular}{c c c c c}
\hline
 & 10 & 254769 & 248768 & 6001 \\
\hline
\multirow{2}{*}{2次观测} & 9 & 243227 & 243007 & 220 \\
 & 10 & 248989 & 248768 & 221 \\
\hline
\multirow{2}{*}{3次观测} & 9 & 243126 & 243007 & 119 \\
 & 10 & 248889 & 248768 & 121 \\
\hline
\multirow{2}{*}{4次观测} & 9 & 243106 & 243007 & 99 \\
 & 10 & 248871 & 248768 & 103 \\
\hline
\multirow{2}{*}{5次观测} & 9 & 243109 & 243007 & 102 \\
 & 10 & 248874 & 248768 & 106 \\
\hline
\multirow{2}{*}{6次观测} & 9 & 243067 & 243007 & 60 \\
 & 10 & 248831 & 248768 & 63 \\
\hline
\multirow{2}{*}{7次观测} & 9 & 243055 & 243007 & 48 \\
 & 10 & 248819 & 248768 & 51 \\
\hline
\multirow{2}{*}{8次观测} & 9 & 243033 & 243007 & 26 \\
 & 10 & 248798 & 248768 & 30 \\
\hline
\end{tabular}
\end{table}

从表3-10中可以看出,随着所选观测次数的提升,预报的时间和题目所给的时间之差越来越小。并且使用两次或三次观测数据进行预测就能使预报结果的误差在可接受的范围内。

因此,预报的精度是随着使用的观测数据次数的提升而显著升高。

\section{四. 问题二的分析与建模}

\subsection{4.1. 问题分析}

问题二所需求解的问题是,在侦察卫星的约束条件下,寻找最优的行驶路径的问题。部队车队从新疆的阿勒泰出发,走高速公路或者普速公路,途径喀什,最终到达和田休息24小时返回阿勒泰。车队每行进10小时至12小时之间休息超过10小时,车队总长为2Km。再在问题一的基础上,躲避Q型卫星和L1型卫星,达到反侦察的目的。

对于行驶的路程,将沿线的城市(县级及其以上)离散成一个个点,再将这些点用直线相连,所需行驶的路程即成为这些直接的长度之和。由于公路有高速和普速之分,并且城市之间不止有一条路,需要进行路径的优化选择,因此本文选择模拟退火算法来进行优化计算。

\subsection{4.2. 模型建立}

{1. 城市离散化}

通过网络资源,查找出途径城市的经纬度,并将其列出:

\textbf{表4-1 新疆各需要经过的县级以上地区经纬度}

\begin{table}[h]
\centering
\begin{tabular}{|c|c|c|c|}
\hline
 & 城市 & 纬度 & 经度 \\
\hline
\multirow{11}{*}{新疆} & 阿勒泰市 & 47.825993 & 88.128427 \\
\hline
 & 北屯市 & 47.346667 & 87.794444 \\
\hline
 & 福海县 & 47.110802 & 87.483361 \\
\hline
 & 乌尔禾区 & 46.088356 & 85.691012 \\
\hline
 & 克拉玛依市 & 45.579078 & 84.886804 \\
\hline
 & 奎屯市 & 44.42574 & 84.900912 \\
\hline
 & 石河子市 & 44.304887 & 86.077659 \\
\hline
 & 呼图壁县 & 44.19047 & 86.896402 \\
\hline
 & 乌鲁木齐市 & 43.824459 & 87.614444 \\
\hline
 & 达坂城区 & 43.362064 & 88.308122 \\
\hline
 & 托克逊县 & 42.791228 & 88.651263 \\
\hline
 & 和硕县 & 42.267188 & 86.861522 \\
\hline
\end{tabular}
\end{table}

\begin{table}
\centering
\begin{tabular}{c c c c}
\hline
 & 库尔勒市 & 41.764539 & 86.152665 \\
\hline
 & 库车县 & 41.716741 & 82.959189 \\
\hline
 & 阿克苏市 & 41.166567 & 80.26122 \\
\hline
 & 巴楚县 & 39.784916 & 78.547355 \\
\hline
 & 阿图什市 & 39.715735 & 76.165232 \\
\hline
 & 喀什市 & 39.467467 & 75.990845 \\
\hline
 & 疏勒县 & 39.401024 & 76.045017 \\
\hline
 & 英吉沙县 & 38.930645 & 76.172599 \\
\hline
 & 莎车县 & 38.414446 & 77.243196 \\
\hline
 & 泽普县 & 38.194044 & 77.269023 \\
\hline
 & 叶城县 & 37.882945 & 77.410683 \\
\hline
 & 皮山县 & 37.62145 & 78.28147 \\
\hline
 & 墨玉县 & 37.276457 & 79.735861 \\
\hline
 & 和田市 & 37.112495 & 79.911624 \\
\hline
 & 阿拉尔市 & 40.545724 & 81.277606 \\
\hline
 & 沙雅县 & 41.220606 & 82.778944 \\
\hline
 & 麦盖提县 & 38.896756 & 77.639697 \\
\hline
\end{tabular}
\end{table}

可以通过经纬度计算出两城市之间的距离(见附录二城市距离),进而能够给出不同路线的路程距离。

\begin{enumerate}
    \item 从阿勒泰去和田(途径喀什)单位:Km \\
    路线一:总距离2654.6574,其中高速距离2207.7571,普速距离446.9003 \\
    高速:阿勒泰——北屯——福海——乌尔禾——克拉玛依——奎屯 \\
    (525.7923) \\
    高速:奎屯——石河子——呼图壁——乌鲁木齐——达坂城——托克逊 \\
    ——和硕——库尔勒——库车(958.3399) \\
    高速:库车——阿克苏(233.3113)
\end{enumerate}

\begin{itemize}
    \item 普速:阿克苏——巴楚——阿图什——喀什(446.9003)
    \item 高速:喀什——疏勒——英吉沙——莎车——泽普——叶城——和田(490.3136)
    \item 路线二:总距离 2227.9195,其中高速距离 1249.4192,普速距离 978.5003
    \item 高速:阿勒泰——北屯——福海——乌尔禾——克拉玛依——奎屯(525.7923)
    \item 普速:奎屯——库车(531.6)
    \item 高速:库车——阿克苏(233.3133)
    \item 普速:阿克苏——巴楚——阿图什——喀什(446.9003)
    \item 高速:喀什——疏勒——英吉沙——莎车——泽普——叶城——和田(490.3136)
\end{itemize}

\begin{enumerate}
    \item[(2)] 从和田回阿勒泰(无须经过喀什)单位:Km
    \item 路线一:总距离 2140.372,其中高速距离 1486.1322,普速距离 654.2398
    \item 普速:和田——阿拉尔市——沙雅——库车(654.2398)
    \item 高速:库车——库尔勒——和硕——托克逊——达坂城——乌鲁木齐——呼图壁——石河子——奎屯(958.3399)
    \item 高速:奎屯——克拉玛依——乌尔禾——福海——北屯——阿勒泰(527.7923)
    \item 路线二:总距离 2458.0678,其中高速距 2246.6109,普速距离 211.4569
    \item 高速:和田——叶城——泽普——莎车——麦盖提——巴楚(527.1674)
    \item 普速:巴楚——阿克苏(211.4569)
    \item 高速:阿克苏——库车(233.3113)
    \item 高速:库车——库尔勒——和硕——托克逊——达坂城——乌鲁木齐——呼图壁——石河子——奎屯(958.3399)
    \item 高速:奎屯——克拉玛依——乌尔禾——福海——北屯——阿勒泰(527.7923)
    \item 路线三:总距离 1713.6321,其中高速距离 527.7923,普速距离为 1185.8398
    \item 普速:和田——阿拉尔市——沙雅——库车——奎屯(1185.8398)
    \item 高速:奎屯——克拉玛依——乌尔禾——福海——北屯——阿勒泰(527.7923)
    \item 路线四:总距离 2031.3279,其中高速距离 1288.271,普速距离 743.0569
    \item 高速:和田——叶城——泽普——莎车——麦盖提——巴楚(527.1674)
    \item 普速:巴楚——阿克苏(211.4569)
    \item 高速:阿克苏——库车(233.3113)
\end{enumerate}

普速:库车——奎屯(531.6)

高速:奎屯——克拉玛依——乌尔禾——福海——北屯——阿勒泰(527.7923)

从上可知,阿勒泰——和田有两条路线可以选择,和田——阿勒泰有四条路线可以选择,共有 8 种情况需要考虑。后经百度地图查询,发现阿勒泰与奎屯两个城市之间既有高速公路和普通公路,因此情况增至 16 种。

根据题意,应尽可能快地到达目的地,因此需要得到最优解。可知总时间 \( t \) 由以下公式求得:

\[
t = \frac{L_G}{v_G} + \frac{L_P}{v_P} + t_R + t_H + t_Y = t_G + t_P + t_R + t_H + t_Y
\tag{4-1}
\]

从式 (4-1) 可知,要使 \( t \) 最小,则需令其每一项都可以取最小值。因此,行驶速度需要尽可能快,休息和隐蔽时间需要尽可能地短。若普通公路距离和高速公路相比并没有显著减少,则尽量采用高速优先的原则。

基于以上的几点考量,本题选择模拟退火算法进行优化。

{2. 模拟退火算法简介 \cite{ref6,ref7}}

模拟退火算法来源于固体退火原理,是一种基于概率的算法,将固体加温至充分高,再让其徐徐冷却。加温时,固体内部粒子随温升变为无序状,内能增大,而徐徐冷却时粒子渐趋有序,在每个温度都达到平衡态,最后在常温时达到基态,内能减为最小。

模拟退火算法 (Simulated Annealing, SA) 最早的思想是由 N. Metropolis 等人于 1953 年提出。1983 年,S. Kirkpatrick 等成功地将退火思想引入到组合优化领域。它是基于 Monte-Carlo 迭代求解策略的一种随机寻优算法,其出发点是基于物理中固体物质的退火过程与一般组合优化问题之间的相似性。模拟退火算法从某一较高初温出发,伴随温度参数的不断下降,结合概率突跳特性在解空间中随机寻找目标函数的全局最优解,即在局部最优解能概率性地跳出并最终趋于全局最优。模拟退火算法是一种通用的优化算法,理论上算法具有概率的全局优化性能,目前已在工程中得到了广泛应用,诸如 VLSI、生产调度、控制工程、机器学习、神经网络、信号处理等领域。

模拟退火算法新解的产生和接受可分为如下四个步骤:

第一步是由一个产生函数从当前解产生一个位于解空间的新解;为便于后续的计算和接受,减少算法耗时,通常选择由当前新解经过简单地变换即可产生新解的方法,如对构成新解的全部或部分元素进行置换、互换等,注意到产生新解的变换方法决定了当前新解的邻域结构,因而对冷却进度表的选取有一定的影响。

第二步是计算与新解所对应的目标函数差。因为目标函数差仅由变换部分产生,所以目标函数差的计算最好按增量计算。事实表明,对大多数应用而言,这是计算目标函数差的最快方法。

第三步是判断新解是否被接受,判断的依据是一个接受准则,最常用的接受准则是 Metropolis 准则:若 \( \Delta T < 0 \) 则接受 \( S' \) 作为新的当前解 \( S \),否则以概率 \( e^{-\Delta T / T} \)

接受 $S'$ 作为新的当前解 $S$。

第四步是当新解被确定接受时,用新解代替当前解,这只需将当前解中对应于产生新解时的变换部分予以实现,同时修正目标函数值即可。此时,当前解实现了一次迭代。可在此基础上开始下一轮试验。而当新解被判定为舍弃时,则在原当前解的基础上继续下一轮试验。

模拟退火算法与初始值无关,算法求得的解与初始解状态 $S$(是算法迭代的起点)无关;模拟退火算法具有渐近收敛性,已在理论上被证明是一种以概率 1 收敛于全局最优解的全局优化算法;模拟退火算法具有并行性。

本文使用模拟退火算法是为了求得全局最优解。

\subsection{4.3. 问题求解}

{1. 在不考虑侦察卫星的情况下求解}

\begin{figure}[h]
    \centering
    \includegraphics[width=\textwidth]{image1.png}
    \caption{去程(阿勒泰-喀什-和田)平面路线图}
    \label{fig:4-1}
\end{figure}

\begin{figure}[h]
    \centering
    \includegraphics[width=\textwidth]{image2.png}
    \caption{回程(和田-阿勒泰)平面路线图}
    \label{fig:4-2}
\end{figure}

\begin{table}
\centering
\caption{城市路程节点间道路选择}
\begin{tabular}{|c|c|c|c|c|c|c|c|}
\hline
地名 & 阿-奎 & 奎-库 & 库-喀 & 喀-和 & 和-库 & 库-奎 & 奎-阿 \\
\hline
普速1 & 586.91 & 531.25 & 711.62 & 530.16 & 651.98 & 531.25 & 586.91 \\
\hline
高速2 & 549.16 & 985.22 & 736.90 & 503.51 & 1025.47 & 985.22 & 549.26 \\
\hline
\end{tabular}
\end{table}

(阿—阿勒泰,奎—奎屯,库—库车,喀—喀什,和—和田)

当以目标函数为距离时,高速公路用2表示,普通公路用1表示,运用模拟退火算法计算$^{[8-9]}$。得到的结果为(2112112),$L$为4028.03公里,$t$为64.54小时。

当以行驶时间为目标函数时,运用模拟退火算法,得到行驶路线为(222222),即全程为高速公路行驶,行驶的时间最短,$t$为53.35小时,$L$为5334.75千米。(模拟退火算法程序见附录一)

从以上可以得出,在不考虑休息的情况下,高速路行驶的时间更短,更有利于行军时机,这样,以下进行的讨论,均建立于在高速行驶,并考虑休息时间。

当军队在高速路上行驶时,不考虑要躲避卫星的情况下,考虑行驶10~12个小时休息$t_{R}=10$小时(根据式(4-1)),并且在和田执行任务$t_{H}=24$小时。利用MATLAB编程,可到得到当军队在2016年11月1日五点从阿勒泰出发时,经过各点的时间。结果如下图所示:(具体表格见附录二表6)

\begin{figure}[h]
\centering
\includegraphics[width=\textwidth]{image.png}
\caption{部队经过各城市的时间}
\end{figure}

所选取的休息的点为:和硕,喀什,库车,奎屯。

\section{2. 考虑卫星的情况下求解}

\begin{figure}[h]
    \centering
    \includegraphics[width=\textwidth]{image1.png}
    \caption{L1 星的星下点轨迹图}
    \label{fig:4-4}
\end{figure}

\begin{figure}[h]
    \centering
    \includegraphics[width=\textwidth]{image2.png}
    \caption{Q 星的星下点轨迹图}
    \label{fig:4-5}
\end{figure}

对行车过程进行离散,让军队每隔一分钟都有一个确切的位置。将时间信息引入第三个维度,在此情况下,对军队和卫星在每一分钟的经纬度进行比较,判断车队有没有可能在卫星的辐照下。离散的结果如下图所示。与离散前的图吻合度很高。

\begin{figure}[h]
    \centering
    \includegraphics[width=0.8\textwidth]{image1.png}
    \caption{离散路程图和路程图的比较}
    \label{fig:1}
\end{figure}

利用离散后的路线跟卫星的路线进行比较,从结果中得到,在 \( t=100.4667 \) 这个点,车队在卫星的侦察范围内。因此,需要优化路线,才能让车队避开 L1 星。当卫星辐照宽度是以星下点轨迹为中心,左右各 272Km,卫星和车队的经纬度差的平方和小于 4 时,认为两者相遇,即车队在卫星的侦察范围内。

根据计算结果得到,当行驶时间为 108.447 小时时,即当卫星到经纬度 (44.1180, 84.4190),车队的经纬度为 (45.2670, 84.8906),两者相遇(见图 \ref{fig:2})。

\begin{figure}[h]
    \centering
    \includegraphics[width=\textwidth]{image2.png}
    \caption{L1 星可侦察车队时间点}
    \label{fig:2}
\end{figure}

观察此时的时间-速度图可知,当时部队在奎屯休息。因此不需要对卫星进行规避,只需在驻地分散隐藏即可。

计算 Q 星过顶的时间,所依据的程序跟 L1 星相同。

经计算可得,时间点为 \( t=28.8000 \),其中 \( t \) 在行驶路线过程中,要进行就地伪装。此时 Q 星的经纬度是 (39.7817, 78.4372),卫星的经纬度是 (39.4790, 77.1430),此时卫星正从阿克苏开往巴楚,需要就地停车隐藏。

\begin{figure}[h]
    \centering
    \includegraphics[width=0.8\textwidth]{image1.png}
    \caption{Q星可侦察车队时间点}
    \label{fig:q_sat_time}
\end{figure}

\begin{figure}[h]
    \centering
    \includegraphics[width=\textwidth]{image2.png}
    \caption{L1星/Q星与车队相遇的时间-速度图}
    \label{fig:l1_q_speed}
\end{figure}

\begin{table}[h]
    \centering
    \caption{车队与侦察卫星相遇时的规避方式}
    \label{tab:avoidance_methods}
    \begin{tabular}{|c|c|c|c|c|c|}
        \hline
        时间/H & 卫星 & 卫星经纬度 & 行车经纬度 & 行车位置 & 规避方式 \\
        \hline
        108.447 & L1星 & (44.1180, 84.4190) & (45.2670, 84.8906) & 奎屯休息 & 驻地隐藏 \\
        \hline
        28.8000 & Q星 & (39.7817, 78.4372) & (39.4790, 77.1430) & 阿克苏—巴楚 & 就地停车隐藏 \\
        \hline
    \end{tabular}
\end{table}

由于遭遇L1星,车队正处于休息状态,可以不必因此而改变线路和行车时间。所以车队路上行驶的时间 \( t \) 为 53.35 小时,总路程 \( L \) 为 5334.75 千米。线路为阿勒泰——北屯——福海——乌尔禾——克拉玛依——奎屯——石河子——呼。

图壁——乌鲁木齐——达坂城——托克逊——和硕——库尔勒——库车——阿克苏——巴楚——阿图什——喀什——疏勒——英吉沙——莎车——泽普——叶城——和田——叶城——泽普——莎车——麦盖提——巴楚——阿克苏——库车——库尔勒——和硕——托克逊——达坂城——乌鲁木齐——呼图壁——石河子——奎屯——克拉玛依——乌尔禾——福海——北屯——阿勒泰,全程高速行驶(所给的地图上阿克苏到巴楚为普速公路,经百度地图查询,该地区已通高速公路,所以其按高速公路计算),路线图见 4-3。

\section{问题三的分析与建模}

\subsection{问题分析}

研究导弹发射装置的战时隐蔽问题。有专家提出,运动方式可能是移动发射装置规避卫星侦察的有效方案。

本题与问题一不同,与问题二类似。在问题一中,施工区域是静止的,而卫星是一直在运动,对该区域进行动态观测,施工区域对于侦察卫星需隐藏;在问题二中,行军路线和卫星都是运动的,车队对于侦察卫星需躲避。在问题三中,导弹发射装置则更具有主动权,根据预测得到的卫星轨道数据,实时地进行移动,以防被侦察卫星发现。

题目中假设了三种卫星,Q 型、L 型(包含 L1 和 L2)以及 K 型卫星,它们都具有一定的侦察能力(通过前两问可知)。在本问题中,首先要分别对这三种卫星的隐蔽问题进行分析,接着适当地对两种或者三种类型卫星的叠加进行更深入的分析。分析的主要内容包括隐蔽的路线和时间,以及所能隐蔽区域的大小,寻找其中的薄弱环节。

\subsection{模型建立}

首先建立简化的数学模型。

以问题一为基础,侦察卫星的侦察范围同 Q 型卫星的辐照范围。该模型是以卫星每绕行一圈,经度会西移一个固定的角度,依次往复,以一定的周期进行移动。轨道移动一定的经度,必然会造成侦察盲区,这就给战略导弹的转移留出了空间。这就是建模和编程的理论依据。

\begin{enumerate}
    \item 于 Q 型卫星:
    \begin{itemize}
        \item $ T = 5668.356s = 94.4726min $,也就是等于一天可以绕地球 15.2425 圈。地球在惯性空间自转角速度 $\bar{\omega} = 360^\circ / \text{天}$。$\dot{\Omega}$ 为升交点赤经的变化率,是根据问题一描述得:
        \begin{equation}
            \dot{\Omega} = 0.9856^\circ / d
        \end{equation}
        每一圈卫星相对地面经度西移 $ m = \frac{T}{86400} \times (360 - \dot{\Omega}) = 23.55349^\circ $
        \item Q 型卫星的辐照宽度覆盖的经度(使用平面角近似球面角,以下同理)
        \begin{equation}
            w = \frac{272 \times 2 \times \cos(i - 90)}{6378} \cdot \frac{180}{\pi} = 4.8461^\circ
        \end{equation}
    \end{itemize}
    \item 对于 L1 型卫星:
    \begin{itemize}
        \item $ T = 5855.884s = 97.598min $,也就是等于一天可以绕地球 14.754 圈。地球在惯性空间自转角速度 $\bar{\omega} = 360^\circ / \text{天}$。$\dot{\Omega}$ 为升交点赤经的变化率,是根据问题一所得的数据拟合而成:
        \begin{equation}
            \dot{\Omega} = -3.0109^\circ / d
        \end{equation}
        每一圈卫星相对地面经度西移 $ m = \frac{T}{86400} \times (360 - \dot{\Omega}) = 24.6036^\circ $
    \end{itemize}
\end{enumerate}

L1 型卫星的辐照宽度
\[
w = \frac{272 \times 2 \times \cos(i - 90)}{6378} \cdot \frac{180}{\pi} = 4.532^\circ
\]

③ 对于 L2 型卫星:
\[
T = 5871.298 \, \text{s} = 97.855 \, \text{min} \text{,也就是等于一天可以绕地球 14.716 圈。地球在惯性空间自转角速度 } \tilde{\omega} = 360^\circ / \text{天。} \dot{\Omega} \text{ 为升交点赤经的变化率,是根据问题一所得的数据拟合而成}
\]
\[
\dot{\Omega} = -2.3483^\circ / \text{d}
\]
每一圈卫星相对地面经度西移
\[
m = \frac{T}{86400} \times (360 - \dot{\Omega}) = 24.6233^\circ
\]

L2 型卫星的辐照宽度
\[
w = \frac{272 \times 2 \times \cos(i - 90)}{6378} \cdot \frac{180}{\pi} = 4.102^\circ
\]

④ 对于 K 型卫星:
\[
T = 5819.925 \, \text{s} = 96.99876 \, \text{min} \text{,也就是等于一天可以绕地球 14.84555 圈。地球在惯性空间自转角速度 } \tilde{\omega} = 360^\circ / \text{天。} \dot{\Omega} \text{ 为升交点赤经的变化率,是根据问题一所得的数据拟合而成}
\]
\[
\dot{\Omega} = -0.23699047^\circ / \text{d}
\]
每一圈卫星相对地面经度西移
\[
m = \frac{T}{86400} \times (360 - \dot{\Omega}) = 24.26565^\circ
\]

K 型卫星的辐照宽度
\[
w = \frac{272 \times 2 \times \cos(i - 90)}{6378} \cdot \frac{180}{\pi} = 4.86^\circ
\]

\subsection{5.3. 问题求解}

将导弹发射装置的经度设在东经 \(120^\circ\),北纬 \(30^\circ\) 左右的范围内,优先考虑在侦察卫星下我国的战时隐蔽问题。

依照上述的简化,进行 MATLAB 编程绘制时序图。这里使用地理经度作横坐标,时间(天数)作为纵坐标,将卫星在相应时间经过北纬 \(30^\circ\) 的经度以条状显示。图中某一点被条状图形覆盖则表示在此刻该经度会被此种卫星侦查到。该种图较为新颖地将侦查情况可视化,可在图中直接看出相应的机动规避方案。

{1. Q 型卫星:}

由图 5-1 可知,在 Q 型卫星单一侦察情况下,可以找到一段持续时间最长的安全区域,约有 4 天的安全时间(3.9 天)。

\begin{figure}[h]
    \centering
    \includegraphics[width=\textwidth]{image1.png}
    \caption{Q 型卫星下的战略导弹转移(蓝色竖条为安全区段)}
    \label{fig:q_satellite}
\end{figure}

{L 型卫星:}

由图 \ref{fig:l_satellite} 可知,在 L 型卫星(包含 L1 和 L2)侦察情况下,由题设初始条件可以找到一段持续时间最长的安全区域,约有 3 天的安全时间。可以看出 L1 和 L2 型卫星利用初始入轨经度、辐照宽度和轨道转动基本覆盖了所有经度范围。

\begin{figure}[h]
    \centering
    \includegraphics[width=\textwidth]{image2.png}
    \caption{L 型卫星下的战略导弹转移}
    \label{fig:l_satellite}
\end{figure}

(蓝-L1,绿-L2,蓝色竖条为安全区段)

{K 型卫星:}

由图 \ref{fig:k_satellite} 可知,在 K 型卫星单一侦察情况下,可以找到一段持续时间最长的安全区域,约有 3.5 天的安全时间。K 型卫星情况与 Q 型卫星较为相似。

\begin{figure}[h]
    \centering
    \includegraphics[width=\textwidth]{image3.png}
    \caption{K 型卫星下的战略导弹转移}
    \label{fig:k_satellite}
\end{figure}

\begin{figure}[h]
    \centering
    \includegraphics[width=\textwidth]{image1.png}
    \caption{K 型卫星下的战略导弹转移(蓝色竖条为安全区段)}
    \label{fig:5-3}
\end{figure}

从以上三幅图可以看出,卫星会有一大块侦察盲区,只需将导弹发射装置放在该盲区,并随盲区做周期运动就可以规避卫星侦察。

特别提出侦察盲区的概念,指的是:在一个时间段内不能被侦察到的区域,可以用来躲避和隐藏军事装备。

{联合侦察}

以此为原理,接下来直接考虑三种类型的卫星同时进行侦察。

\begin{figure}[h]
    \centering
    \includegraphics[width=\textwidth]{image2.png}
    \caption{三种类型卫星同时侦察下的观测区域}
    \label{fig:5-4}
\end{figure}

(红-Q,蓝-L1,绿-L2,洋红-K)

观察图 5-4,可以发现有侦察盲区一直处于两条带之间,并且有规律地随着经度做移动。因此,可将导弹发射装置布置于该区域,可以随着时间的推移做周期性地运动。图中有一段折线为规避侦察的示例,直线段代表导弹发射装置可以

不移动而躲避侦查;折线段表示导弹为规避侦查而进行经度方向上的机动,由于横纵坐标的设置,折线斜率可以表示机动的速度,在某些条件下可能会出现要求速度过高而规避失败。

与单一卫星侦察相同,在三种类型卫星同时侦察下,也会有侦察盲区出现,只是与前者相比,盲区的范围减小,所需的运动周期也减短,需要有更快地战时反侦察响应。

{5. 综合机动与联合侦查}

当考虑到多颗卫星联合侦查和区域大小、路网情况、经度纬度联合机动时,方案规划会变得困难许多。由于竞赛时间限制,这里仅提出一种新颖的分析方法:方法的基础是上文使用的时序图。

首先考虑纬度方向上的机动,将导弹在经纬度方向上的可行的路线作为图形的第三维,形成相应的约束。然后考虑导弹在三维空间中的运动,需要规避卫星侦查和满足路线约束,同时路线的斜率还要满足导弹的机动速度。最后使用蚁群算法、粒子群算法等全局智能优化算法进行轨迹的规划,形成较好的方案。

更进一步,如果有更多的约束条件,都可以通过增加维数的方法归纳到这个计算方法中,这样会使得建模的通用性更好,当然也可能使算法的计算时间大大增加,陷入局部极值的可能性也增加。

\section{六. 结论与分析}

本文以军事行动避空侦察为背景,分别建模并讨论了卫星轨道观测和预报、地面静止军事目标和运动军事目标对与侦察卫星的反侦察行为、行军路线的优化。计算结果表明,建立的预报模型精确度高,具有较高的可信度,与所给的卫星轨道参数吻合很好;规划优化算法性能较好。而对于问题三的导弹发射装置机动隐蔽指导,特别提出了“侦察盲区”的概念并使用较为直观的侦察时序图,不仅解决了该题,也为我国的国防反侦察事业给出了全新的解决方案。

综上所述,本文较好的解决了军事行动避空侦察的时机和路线选择,给出了精确的模型和新颖地思路,取得了预期的目标。

\section{参考文献}

[1] 刘林, 张巍. 关于各种类型数据的初轨计算方法 [J]. 飞行器测控学报, Vol28, No.3, June, 2009.

[2] 章仁为. 卫星轨道姿态动力学与控制 [M]. 北京: 北京航空航天大学出版社, 121~123, 1998.

[3] 杨颂华, 向春生, 马佳光. 单站短弧段人造卫星轨道预测 [J]. 光电工程, Vol33, No.7, July, 2006.

[4] 张洪波. 航天器轨道力学理论与方法 [M]. 北京: 国防工业出版社, 46~47, 2015.

[5] 杨嘉墀. 航天器轨道动力学与控制 [M]. 北京: 中国宇航出版社, 55~56, 2009.

[6] 高尚. 模拟退火算法中的退火策略研究 [J]. 航空计算技术. 2002(04).

[7] 向阳, 龚新高. 推广模拟退火方法及其应用 [J]. 物理学进展. 2000(03).

[8] 盛国华, 陈玉金. 改进模拟退火算法求解 TSP 问题 [J]. 电脑知识与技术. 2008(15).

[9] 张建航, 李国. 模拟退火算法及其在求解 TSP 中的应用 [J]. 现代电子技术. 2006(22)

\section{附录一 MATLAB 源程序}

问题二 模拟退火算法以行驶路程为目标函数程序

\begin{verbatim}
clear
clc
a = 0.95;
k = [586.91 531.25 711.62 530.16 651.98 531.25 586.91;
     549.16 985.22 736.90 503.51 1025.47 985.22 549.26];
num = size(k,2);
sol_new = ones(1,num);          % 生成初始解
E_current = inf; E_best = inf;
% E_current 是当前解对应的目标函数值;
% E_new 是新解的目标函数值;
% E_best 是最优解的
sol_current = sol_new; sol_best = sol_new;
t0=97; tf=3; t=t0;
p=1;

while t>=tf
    for r=1:100
        % 产生随机扰动
        tmp=ceil(rand.*num);
        sol_new(1,tmp)=ceil(2*rand);
        % 计算行走的公里数
        sum=0;
        for i=1:num
            sum=sum+k(sol_new(i),i);
        end
        E_new=sum;
        if E_new<E_current
            E_current=E_new;
            sol_current=sol_new;
            if E_new<E_best
                E_best=E_new;
                sol_best=sol_new;
            end
        else
            p=exp(-(E_new-E_current)/t);
            if rand

\begin{verbatim}
% 把冷却过程中最好的解保存下来
E_best = E_new;
sol_best = sol_new;
end
else
    if rand < exp(-(E_new - E_current) / t)
        E_current = E_new;
        sol_current = sol_new;
    else
        sol_new = sol_current;
    end
end
end
t = t * a;
end
format bank
disp('最优解为:')
sol_best
disp('最短距离等于:')
val = E_best
问题二 模拟退火算法以行驶时间为目标函数程序
clear
clc
a = 0.95;
k = [586.91 531.25 711.62 530.16 651.98 531.25 586.91;
     549.16 985.22 736.90 503.51 1025.47 985.22 549.26];
num = size(k, 2);
sol_new = ones(1, num);          % 生成初始解
E_current = inf; E_best = inf;
% E_current 是当前解对应的目标函数值;
% E_new 是新解的目标函数值;
% E_best 是最优解的
\end{verbatim}

\begin{verbatim}
sol_current = sol_new; sol_best = sol_new;
t0=97; tf=3; t=t0;
p=1;

while t>=tf
    for r=1:100
        %产生随机扰动
        tmp=ceil(rand.*num);
        sol_new(1,tmp)=ceil(2*rand);
        % 计算行走的公里数
        time=0;
        for i=1:num
            time=time+k(sol_new(i),i)./(sol_new(i)*50);
        end
        E_new=time;
        if E_new<E_current
            E_current=E_new;
            sol_current=sol_new;
            if E_new<E_best
                % 把冷却过程中最好的解保存下来
                E_best=E_new;
                sol_best=sol_new;
            end
        else
            if rand<exp(-(E_new-E_current)./t)
                E_current=E_new;
                sol_current=sol_new;
            else
                sol_new=sol_current;
            end
        end
    end
end
\end{verbatim}

\begin{verbatim}
end
end
t=t.*a;
end
    dis=0;
    for i=1:num
        dis=dis+k(sol_best(i),i);
    end
format bank
disp('最优解为:')
sol_best
disp('最短时间等于:')
val=E_best
disp('行驶距离等于:')
dis
问题二计算的经过各城市的时间程序
time=0;
sum=0;
tingdundian=[];
tingdunshijian=[];
for i=1:47
    if i==26
        sum=sum+24+time;
        time=0;
    end
    t=A(1,i)/100;
    time=time+t;
    tingdunshijian = [tingdunshijian,sum+time];
    a=time-10;
    if a>0
        time;
        time=time+10;
    end
end
\end{verbatim}

\begin{verbatim}
tingdundian=[tingdundian,i];
sum=sum+time;
time=0;
end
end
tingdundian;
tingdunshijian
riqi=tingdunshijian./24;
jizhun=datenum(2016,11,1,5,0,0);
riqi=riqi+jizhun;
datestr(riqi)
\end{verbatim}

\begin{table}
\centering
\begin{tabular}{|c|c|c|c|c|c|}
\hline
时间 & 半长轴 & 偏心率 & 倾角 & 升交点赤经 & 卫星纬度幅角 \\
\hline
0.00494213 & 6869.3671 & 0.0001397 & 97.383245 & 121.913424 & 18.36310699 \\
\hline
0.00736111 & 6871.3113 & 0.0022333 & 97.427793 & 121.939166 & 31.64738725 \\
\hline
0.48612269 & 6872.711 & 0.0029377 & 97.431223 & 121.88579 & 139.9082181 \\
\hline
0.48782407 & 6870.4481 & 0.001067 & 97.415394 & 121.895868 & 149.2449416 \\
\hline
0.55077546 & 6873.1428 & 0.0012097 & 97.400229 & 121.901257 & 134.8385145 \\
\hline
0.55267361 & 6871.7202 & 0.0010721 & 97.430626 & 121.878913 & 145.2645751 \\
\hline
0.98869213 & 6868.8105 & 4.92E-05 & 97.390965 & 121.927658 & 19.05373668 \\
\hline
0.99116898 & 6871.5245 & 0.0032789 & 97.43474 & 121.954033 & 32.65973877 \\
\hline
1.53415509 & 6873.0949 & 0.0020794 & 97.413955 & 121.892085 & 133.5080182 \\
\hline
1.53655093 & 6871.7847 & 0.000429 & 97.389176 & 121.909422 & 146.6551994 \\
\hline
1.97280093 & 6869.7522 & 0.0010304 & 97.430883 & 121.961163 & 21.72847375 \\
\hline
1.97501157 & 6871.08 & 0.0027919 & 97.412161 & 121.94932 & 33.85958451 \\
\hline
2.03923611 & 6869.8733 & 8.77E-05 & 97.408894 & 121.935199 & 26.45637782 \\
\hline
2.03997685 & 6870.521 & 0.0024094 & 97.406104 & 121.933663 & 30.50984204 \\
\hline
2.5178588 & 6873.3481 & 0.003506 & 97.41462 & 121.902093 & 133.9499928 \\
\hline
2.52038194 & 6870.1167 & 0.0003218 & 97.438573 & 121.886003 & 147.7882324 \\
\hline
2.95741898 & 6870.4332 & 0.0019922 & 97.410068 & 121.949624 & 27.17636619 \\
\hline
2.95891204 & 6872.0391 & 0.0012984 & 97.385854 & 121.933354 & 35.37719849 \\
\hline
\end{tabular}
\end{table}

\begin{table}
\centering
\begin{tabular}{|c|c|c|c|c|c|}
\hline
日期 & 时间/s & 时间/24小时制 & Azm/$^\circ$ & Elv/$^\circ$ & Range/Km \\
\hline
D3 & 261075 & 0:31:15 & 209.69363 & 10.043725 & 1675.395818 \\
\hline
 & 261249 & 0:34:09 & 263.43348 & 22.623222 & 1089.653948 \\
\hline
\end{tabular}
\end{table}

\begin{table}
\centering
\begin{tabular}{|c|c|c|c|c|c|}
\hline
 & 261423 & 0:37:03 & 317.01876 & 10.004562 & 1677.978638 \\
\hline
 &  &  &  &  &  \\
\hline
 & 302582 & 12:03:02 & 31.645962 & 10.010298 & 1677.599967 \\
\hline
 & 302781 & 12:06:21 & 98.386905 & 33.94343 & 822.4140348 \\
\hline
 & 302979 & 12:09:39 & 164.90886 & 10.078348 & 1673.117304 \\
\hline
 &  &  &  &  &  \\
\hline
 & 308325 & 13:38:45 & 303.38719 & 10.009459 & 1677.655323 \\
\hline
 & 308371 & 13:39:31 & 291.02782 & 10.547074 & 1642.718402 \\
\hline
 & 308417 & 13:40:17 & 278.66221 & 10.018705 & 1677.045206 \\
\hline
 &  &  &  &  &  \\
\hline
D4 & 346075 & 0:07:55 & 180.88908 & 10.083454 & 1672.781651 \\
\hline
 & 346287 & 0:11:27 & 259.73266 & 55.304962 & 589.8412262 \\
\hline
 & 346500 & 0:15:00 & 338.98707 & 10.018676 & 1677.047118 \\
\hline
 &  &  &  &  &  \\
\hline
 & 387690 & 11:41:30 & 58.850412 & 10.049511 & 1675.014702 \\
\hline
 & 387817 & 11:43:37 & 95.026489 & 14.885899 & 1397.757047 \\
\hline
 & 387944 & 11:45:44 & 131.15061 & 10.013918 & 1677.361037 \\
\hline
 &  &  &  &  &  \\
\hline
 & 393251 & 13:14:11 & 341.04596 & 10.007351 & 1677.794489 \\
\hline
 & 393426 & 13:17:06 & 287.01793 & 23.249183 & 1070.217263 \\
\hline
 & 393600 & 13:20:00 & 233.32662 & 10.031676 & 1676.189822 \\
\hline
 &  &  &  &  &  \\
\hline
 & 431119 & 23:45:19 & 154.73046 & 10.082216 & 1672.863036 \\
\hline
 & 431331 & 23:48:51 & 75.908651 & 55.751259 & 587.011113 \\
\hline
 & 431543 & 23:52:23 & 358.12419 & 10.007793 & 1677.765322 \\
\hline
 &  &  &  &  &  \\
\hline
D5 & 478262 & 12:51:02 & 1.8838369 & 10.006499 & 1677.850737 \\
\hline
 & 478474 & 12:54:34 & 284.11606 & 55.772383 & 586.8785921 \\
\hline
 & 478686 & 12:58:06 & 205.26074 & 10.084953 & 1672.683161 \\
\hline
\end{tabular}
\end{table}

\begin{table}
\centering
\begin{tabular}{c c c c c c}
\hline
 & & & & & \\
\hline
 & 516205 & 23:23:25 & 126.66851 & 10.027791 & 1676.445945 \\
\hline
 & 516379 & 23:26:19 & 72.990168 & 23.242162 & 1070.431508 \\
\hline
 & 516554 & 23:29:14 & 18.962791 & 10.007138 & 1677.808554 \\
\hline
\end{tabular}
\end{table}

\begin{table}
\centering
\caption{表3Q型卫星可以侦察到施工区域的时间和位置}
\begin{tabular}{c c c c c}
\hline
时间 & & 时间 & 星下点纬度 & 星下点经度 \\
\hline
346289 & D4 & 0:11:29 & 31.606235 & 115.505272 \\
\hline
346290 & & 0:11:30 & 31.669012 & 115.489801 \\
\hline
346291 & & 0:11:31 & 31.731788 & 115.474316 \\
\hline
346292 & & 0:11:32 & 31.794563 & 115.458814 \\
\hline
346293 & & 0:11:33 & 31.857337 & 115.443298 \\
\hline
346294 & & 0:11:34 & 31.92011 & 115.427766 \\
\hline
 & & & & \\
\hline
478486 & D5 & 12:54:46 & 31.89985 & 115.408306 \\
\hline
478487 & & 12:54:47 & 31.837076 & 115.392779 \\
\hline
478488 & & 12:54:48 & 31.774302 & 115.377268 \\
\hline
478489 & & 12:54:49 & 31.711526 & 115.361772 \\
\hline
478490 & & 12:54:50 & 31.64875 & 115.346291 \\
\hline
478491 & & 12:54:51 & 31.585973 & 115.330825 \\
\hline
\end{tabular}
\end{table}

\begin{table}
\centering
\caption{表4L1型卫星可以侦察到施工区域的时间和位置}
\begin{tabular}{c c c c}
\hline
日期 & 时间 & 纬度 & 经度 \\
\hline
8月22日 & 4:22:19 & 32.08404 & 119.2612 \\
\hline
 & 4:22:20 & 32.02887 & 119.289 \\
\hline
 & 4:22:21 & 31.9737 & 119.3168 \\
\hline
 & 4:22:22 & 31.91851 & 119.3445 \\
\hline
\end{tabular}
\end{table}

\begin{table}
\centering
\begin{tabular}{c c c c}
\hline
 & 4:22:23 & 31.86332 & 119.3722 \\
\hline
 & 4:22:24 & 31.80813 & 119.3999 \\
\hline
 & 4:22:25 & 31.75292 & 119.4275 \\
\hline
 & 4:22:26 & 31.69771 & 119.4551 \\
\hline
 & 4:22:27 & 31.64249 & 119.4826 \\
\hline
 & 4:22:28 & 31.58726 & 119.5101 \\
\hline
 & 4:22:29 & 31.53202 & 119.5376 \\
\hline
 & 4:22:30 & 31.47677 & 119.565 \\
\hline
 & 4:22:31 & 31.42152 & 119.5924 \\
\hline
 & 4:22:32 & 31.36626 & 119.6198 \\
\hline
 & & & \\
\hline
8月24日 & 17:40:53 & 31.06911 & 120.6371 \\
\hline
 & 17:40:54 & 31.12441 & 120.6642 \\
\hline
 & 17:40:55 & 31.17971 & 120.6914 \\
\hline
 & 17:40:56 & 31.23499 & 120.7186 \\
\hline
 & 17:40:57 & 31.29027 & 120.7459 \\
\hline
 & 17:40:58 & 31.34554 & 120.7732 \\
\hline
 & 17:40:59 & 31.40081 & 120.8006 \\
\hline
 & 17:41:00 & 31.45606 & 120.8279 \\
\hline
 & 17:41:01 & 31.51131 & 120.8554 \\
\hline
\end{tabular}
\caption{表5L2型卫星可以侦察到施工区域的时间和位置}
\end{table}

\begin{table}
\centering
\begin{tabular}{c c c c}
\hline
日期 & 时间 & 纬度 & 经度 \\
\hline
8月24日 & 3:05:52 & 32.54191 & 116.9659 \\
\hline
 & 3:05:53 & 32.58878 & 117.0086 \\
\hline
 & 3:05:54 & 32.63562 & 117.0514 \\
\hline
 & 3:05:55 & 32.68245 & 117.0942 \\
\hline
\end{tabular}
\end{table}

\begin{table}
\centering
\begin{tabular}{c c c c}
\hline
 & 3:05:56 & 32.72926 & 117.1371 \\
\hline
 & 3:05:57 & 32.77605 & 117.18 \\
\hline
 & 3:05:58 & 32.82283 & 117.223 \\
\hline
 & 3:05:59 & 32.86958 & 117.266 \\
\hline
 & 3:06:00 & 32.91632 & 117.3091 \\
\hline
 & 3:06:01 & 32.96304 & 117.3522 \\
\hline
 & 3:06:02 & 33.00975 & 117.3954 \\
\hline
 & & & \\
\hline
 & 11:42:21 & 32.69157 & 118.0474 \\
\hline
 & 11:42:22 & 32.64475 & 118.0903 \\
\hline
 & 11:42:23 & 32.59791 & 118.1331 \\
\hline
 & 11:42:24 & 32.55105 & 118.1758 \\
\hline
 & 11:42:25 & 32.50417 & 118.2185 \\
\hline
 & 11:42:26 & 32.45728 & 118.2611 \\
\hline
 & 11:42:27 & 32.41037 & 118.3037 \\
\hline
 & 11:42:28 & 32.36344 & 118.3462 \\
\hline
 & 11:42:29 & 32.3165 & 118.3887 \\
\hline
 & 11:42:30 & 32.26953 & 118.4312 \\
\hline
 & 11:42:31 & 32.22255 & 118.4736 \\
\hline
 & 11:42:32 & 32.17556 & 118.5159 \\
\hline
 & 11:42:33 & 32.12854 & 118.5582 \\
\hline
 & 11:42:34 & 32.08151 & 118.6005 \\
\hline
 & 11:42:35 & 32.03446 & 118.6427 \\
\hline
 & 11:42:36 & 31.9874 & 118.6848 \\
\hline
 & 11:42:37 & 31.94032 & 118.7269 \\
\hline
 & 11:42:38 & 31.89322 & 118.769 \\
\hline
\end{tabular}
\end{table}

\begin{table}
\centering
\begin{tabular}{c c c c}
\hline
 & 11:42:39 & 31.8461 & 118.811 \\
\hline
 & 11:42:40 & 31.79897 & 118.8529 \\
\hline
 & 11:42:41 & 31.75183 & 118.8949 \\
\hline
 & 11:42:42 & 31.70466 & 118.9367 \\
\hline
 & 11:42:43 & 31.65748 & 118.9786 \\
\hline
 & 11:42:44 & 31.61028 & 119.0203 \\
\hline
 & 11:42:45 & 31.56307 & 119.0621 \\
\hline
 & 11:42:46 & 31.51584 & 119.1037 \\
\hline
 & 11:42:47 & 31.46859 & 119.1454 \\
\hline
 & 11:42:48 & 31.42133 & 119.187 \\
\hline
 & 11:42:49 & 31.37405 & 119.2285 \\
\hline
 & 11:42:49 & 31.32675 & 119.27 \\
\hline
 & 11:42:50 & 31.27944 & 119.3115 \\
\hline
 & 11:42:51 & 31.23211 & 119.3529 \\
\hline
 & 11:42:52 & 31.18477 & 119.3942 \\
\hline
 & 11:42:53 & 31.13741 & 119.4356 \\
\hline
\end{tabular}
\end{table}

\begin{table}
\centering
\caption{从阿勒泰出发到达各城市的具体时间}
\begin{tabular}{c c c c c}
\hline
城市 & 纬度 & 经度 & 时间 & 日期 \\
\hline
北屯 & 47.346667 & 87.794444 & 0.589569 & 01-Nov-2016 05:35:22 \\
\hline
福海 & 47.110802 & 87.483361 & 0.942045 & 01-Nov-2016 05:56:31 \\
\hline
乌尔禾 & 46.088356 & 85.691012 & 2.970045 & 01-Nov-2016 07:58:12 \\
\hline
克拉玛依 & 45.579078 & 84.886804 & 3.812932 & 01-Nov-2016 08:48:46 \\
\hline
奎屯 & 44.425740 & 84.900912 & 5.257932 & 01-Nov-2016 10:15:28 \\
\hline
\end{tabular}
\end{table}

\begin{table}
\begin{tabular}{|c|c|c|c|c|}
\hline 石河子 & 44.304887 & 86.077659 & 6.204015 & 01-Nov-2016 11:12:14 \\
\hline 呼图壁 & 44.190470 & 86.896402 & 6.869198 & 01-Nov-2016 11:52:09 \\
\hline 乌鲁木齐 & 43.824459 & 87.614444 & 7.573843 & 01-Nov-2016 12:34:25 \\
\hline 达坂城 & 43.362064 & 88.308122 & 8.333925 & 01-Nov-2016 13:20:02 \\
\hline 托克逊 & 42.791228 & 88.651263 & 9.027932 & 01-Nov-2016 14:01:40 \\
\hline 和硕 & 42.267188 & 86.861522 & 11.377932 & 01-Nov-2016 16:22:40 \\
\hline 库尔勒 & 41.764539 & 86.152665 & 22.188358 & 02-Nov-2016 03:11:18 \\
\hline 库车 & 41.716741 & 82.959189 & 24.841331 & 02-Nov-2016 05:50:28 \\
\hline 阿克苏 & 41.166567 & 80.261220 & 27.174444 & 02-Nov-2016 08:10:27 \\
\hline 巴楚 & 39.784916 & 78.547355 & 29.289013 & 02-Nov-2016 10:17:20 \\
\hline 阿图什 & 39.715735 & 76.165232 & 31.329187 & 02-Nov-2016 12:19:45 \\
\hline 喀什 & 39.467467 & 75.990845 & 31.643447 & 02-Nov-2016 12:38:36 \\
\hline 疏勒 & 39.401024 & 76.045017 & 41.730854 & 02-Nov-2016 22:43:51 \\
\hline 英吉沙 & 38.930645 & 76.172599 & 42.26593 & 02-Nov-2016 23:15:57 \\
\hline 莎车 & 38.414446 & 77.243196 & 43.56293 & 03-Nov-2016 00:33:46 \\
\hline 泽普 & 38.194044 & 77.269023 & 43.809316 & 03-Nov-2016 00:48:33 \\
\hline 叶城 & 37.882945 & 77.410683 & 44.177227 & 03-Nov-2016 01:10:38 \\
\hline 皮山 & 37.621450 & 78.281470 & 44.997077 & 03-Nov-2016 01:59:49 \\
\hline 墨玉 & 37.276457 & 79.735861 & 46.338541 & 03-Nov-2016 03:20:18 \\
\hline 和田 & 37.112495 & 79.911624 & 46.578554 & 03-Nov-2016 03:34:42 \\
\hline 墨玉 & 37.276457 & 79.735861 & 70.818567 & 04-Nov-2016 03:49:06 \\
\hline 皮山 & 37.621450 & 78.281470 & 72.160031 & 04-Nov-2016 05:09:36 \\
\hline 叶城 & 37.882945 & 77.410683 & 72.979881 & 04-Nov-2016 05:58:47 \\
\hline 泽普 & 38.194044 & 77.269023 & 73.347792 & 04-Nov-2016 06:20:52 \\
\hline
\end{tabular}
\end{table}

\begin{table}
\centering
\begin{tabular}{|c|c|c|c|c|}
\hline
莎车 & 38.414446 & 77.243196 & 73.594178 & 04-Nov-2016 06:35:39 \\
\hline
麦盖提 & 38.896756 & 77.639697 & 74.527378 & 04-Nov-2016 07:31:38 \\
\hline
巴楚 & 39.784916 & 78.547355 & 76.171978 & 04-Nov-2016 09:10:19 \\
\hline
阿克苏 & 41.166567 & 80.261220 & 78.286547 & 04-Nov-2016 11:17:11 \\
\hline
库车 & 41.716741 & 82.959189 & 80.61966 & 04-Nov-2016 13:37:10 \\
\hline
库尔勒 & 41.764539 & 86.152665 & 93.272633 & 05-Nov-2016 02:16:21 \\
\hline
和硕 & 42.267188 & 86.861522 & 94.083059 & 05-Nov-2016 03:04:59 \\
\hline
托克逊 & 42.791228 & 88.651263 & 96.433059 & 05-Nov-2016 05:25:59 \\
\hline
达坂城 & 43.362064 & 88.308122 & 97.127066 & 05-Nov-2016 06:07:37 \\
\hline
乌鲁木齐 & 43.824459 & 87.614444 & 97.887148 & 05-Nov-2016 06:53:13 \\
\hline
呼图壁 & 44.190470 & 86.896402 & 98.591793 & 05-Nov-2016 07:35:30 \\
\hline
石河子 & 44.304887 & 86.077659 & 99.256976 & 05-Nov-2016 08:15:25 \\
\hline
奎屯 & 44.425740 & 84.900912 & 100.701976 & 05-Nov-2016 09:42:07 \\
\hline
克拉玛依 & 45.579078 & 84.886804 & 112.146976 & 05-Nov-2016 21:08:49 \\
\hline
乌尔禾 & 46.088356 & 85.691012 & 112.989863 & 05-Nov-2016 21:59:23 \\
\hline
福海 & 47.110802 & 87.483361 & 115.017863 & 06-Nov-2016 00:01:04 \\
\hline
北屯 & 47.346667 & 87.794444 & 115.370339 & 06-Nov-2016 00:22:13 \\
\hline
阿勒泰 & 47.346667 & 87.794444 & 115.959908 & 06-Nov-2016 00:57:35 \\
\hline
\end{tabular}
\end{table}