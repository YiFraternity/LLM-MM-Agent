\begin{center}
\includegraphics[width=0.3\textwidth]{image1.png} \\
\includegraphics[width=0.3\textwidth]{image2.png} \\
\includegraphics[width=0.3\textwidth]{image3.png} \\
\end{center}

\begin{center}
\textbf{“华为杯”第十三届全国研究生数学建模竞赛}
\end{center}

\begin{tabular}{l l}
学校 & 火箭军工程大学 \\
\hline
参赛队号 & 90059029 \\
\hline
队员姓名 & 1.胡莉军 \\
 & 2.朱东海 \\
 & 3.周冰 \\
\end{tabular}

\begin{center}
参赛密码 \_\_\_\_\_\_\_\_\_\_\_\_\_\_\_\_\_\_\_\_\_\_\_\_\_\_\_\_\_\_\_\_\_\_\_\_\_\_\_\_\_\_\_\_\_\_\_\_\_\_\_\_\_\_\_\_\_\_\_\_\_\_\_\_\_\_\_\_\_\_\_\_\_\_\_\_\_\_\_\_\_\_\_\_\_\_\_\_\_\_\_\_\_\_\_\_\_\_\_\_\_\_\_\_\_\_\_\_\_\_\_\_\_\_\_\_\_\_\_\_\_\_\_\_\_\_\_\_\_\_\_\_\_\_\_\_\_\_\_\_\_\_\_\_\_\_\_\_\_\_\_\_\_\_\_\_\_\_\_\_\_\_\_\_\_\_\_\_\_\_\_\_\_\_\_\_\_\_\_\_\_\_\_\_\_\_\_\_\_\_\_\_\_\_\_\_\_\_\_\_\_\_\_\_\_\_\_\_\_\_\_\_\_\_\_\_\_\_\_\_\_\_\_\_\_\_\_\_\_\_\_\_\_\_\_\_\_\_\_\_\_\_\_\_\_\_\_\_\_\_\_\_\_\_\_\_\_\_\_\_\_\_\_\_\_\_\_\_\_\_\_\_\_\_\_\_\_\_\_\_\_\_\_\_\_\_\_\_\_\_\_\_\_\_\_\_\_\_\_\_\_\_\_\_\_\_\_\_\_\_\_\_\_\_\_\_\_\_\_\_\_\_\_\_\_\_\_\_\_\_\_\_\_\_\_\_\_\_\_\_\_\_\_\_\_\_\_\_\_\_\_\_\_\_\_\_\_\_\_\_\_\_\_\_\_\_\_\_\_\_\_\_\_\_\_\_\_\_\_\_\_\_\_\_\_\_\_\_\_\_\_\_\_\_\_\_\_\_\_\_\_\_\_\_\_\_\_\_\_\_\_\_\_\_\_\_\_\_\_\_\_\_\_\_\_\_\_\_\_\_\_\_\_\_\_\_\_\_\_\_\_\_\_\_\_\_\_\_\_\_\_\_\_\_\_\_\_\_\_\_\_\_\_\_\_\_\_\_\_\_\_\_\_\_\_\_\_\_\_\_\_\_\_\_\_\_\_\_\_\_\_\_\_\_\_\_\_\_\_\_\_\_\_\_\_\_\_\_\_\_\_\_\_\_\_\_\_\_\_\_\_\_\_\_\_\_\_\_\_\_\_\_\_\_\_\_\_\_\_\_\_\_\_\_\_\_\_\_\_\_\_\_\_\_\_\_\_\_\_\_\_\_\_\_\_\_\_\_\_\_\_\_\_\_\_\_\_\_\_\_\_\_\_\_\_\_\_\_\_\_\_\_\_\_\_\_\_\_\_\_\_\_\_\_\_\_\_\_\_\_\_\_\_\_\_\_\_\_\_\_\_\_\_\_\_\_\_\_\_\_\_\_\_\_\_\_\_\_\_\_\_\_\_\_\_\_\_\_\_\_\_\_\_\_\_\_\_\_\_\_\_\_\_\_\_\_\_\_\_\_\_\_\_\_\_\_\_\_\_\_\_\_\_\_\_\_\_\_\_\_\_\_\_\_\_\_\_\_\_\_\_\_\_\_\_\_\_\_\_\_\_\_\_\_\_\_\_\_\_\_\_\_\_\_\_\_\_\_\_\_\_\_\_\_\_\_\_\_\_\_\_\_\_\_\_\_\_\_\_\_\_\_\_\_\_\_\_\_\_\_\_\_\_\_\_\_\_\_\_\_\_\_\_\_\_\_\_\_\_\_\_\_\_\_\_\_\_\_\_\_\_\_\_\_\_\_\_\_\_\_\_\_\_\_\_\_\_\_\_\_\_\_\_\_\_\_\_\_\_\_\_\_\_\_\_\_\_\_\_\_\_\_\_\_\_\_\_\_\_\_\_\_\_\_\_\_\_\_\_\_\_\_\_\_\_\_\_\_\_\_\_\_\_\_\_\_\_\_\_\_\_\_\_\_\_\_\_\_\_\_\_\_\_\_\_\_\_\_\_\_\_\_\_\_\_\_\_\_\_\_\_\_\_\_\_\_\_\_\_\_\_\_\_\_\_\_\_\_\_\_\_\_\_\_\_\_\_\_\_\_\_\_\_\_\_\_\_\_\_\_\_\_\_\_\_\_\_\_\_\_\_\_\_\_\_\_\_\_\_\_\_\_\_\_\_\_\_\_\_\_\_\_\_\_\_\_\_\_\_\_\_\_\_\_\_\_\_\_\_\_\_\_\_\_\_\_\_\_\_\_\_\_\_\_\_\_\_\_\_\_\_\_\_\_\_\_\_\_\_\_\_\_\_\_\_\_\_\_\_\_\_\_\_\_\_\_\_\_\_\_\_\_\_\_\_\_\_\_\_\_\_\_\_\_\_\_\_\_\_\_\_\_\_\_\_\_\_\_\_\_\_\_\_\_\_\_\_\_\_\_\_\_\_\_\_\_\_\_\_\_\_\_\_\_\_\_\_\_\_\_\_\_\_\_\_\_\_\_\_\_\_\_\_\_\_\_\_\_\_\_\_\_\_\_\_\_\_\_\_\_\_\_\_\_\_\_\_\_\_\_\_\_\_\_\_\_\_\_\_\_\_\_\_\_\_\_\_\_\_\_\_\_\_\_\_\_\_\_\_\_\_\_\_\_\_\_\_\_\_\_\_\_\_\_\_\_\_\_\_\_\_\_\_\_\_\_\_\_\_\_\_\_\_\_\_\_\_\_\_\_\_\_\_\_\_\_\_\_\_\_\_\_\_\_\_\_\_\_\_\_\_\_\_\_\_\_\_\_\_\_\_\_\_\_\_\_\_\_\_\_\_\_\_\_\_\_\_\_\_\_\_\_\_\_\_\_\_\_\_\_\_\_\_\_\_\_\_\_\_\_\_\_\_\_\_\_\_\_\_\_\_\_\_\_\_\_\_\_\_\_\_\_\_\_\_\_\_\_\_\_\_\_\_\_\_\_\_\_\_\_\_\_\_\_\_\_\_\_\_\_\_\_\_\_\_\_\_\_\_\_\_\_\_\_\_\_\_\_\_\_\_\_\_\_\_\_\_\_\_\_\_\_\_\_\_\_\_\_\_\_\_\_\_\_\_\_\_\_\_\_\_\_\_\_\_\_\_\_\_\_\_\_\_\_\_\_\_\_\_\_\_\_\_\_\_\_\_\_\_\_\_\_\_\_\_\_\_\_\_\_\_\_\_\_\_\_\_\_\_\_\_\_\_\_\_\_\_\_\_\_\_\_\_\_\_\_\_\_\_\_\_\_\_\_\_\_\_\_\_\_\_\_\_\_\_\_\_\_\_\_\_\_\_\_\_\_\_\_\_\_\_\_\_\_\_\_\_\_\_\_\_\_\_\_\_\_\_\_\_\_\_\_\_\_\_\_\_\_\_\_\_\_\_\_\_\_\_\_\_\_\_\_\_\_\_\_\_\_\_\_\_\_\_\_\_\_\_\_\_\_\_\_\_\_\_\_\_\_\_\_\_\_\_\_\_\_\_\_\_\_\_\_\_\_\_\_\_\_\_\_\_\_\_\_\_\_\_\_\_\_\_\_\_\_\_\_\_\_\_\_\_\_\_\_\_\_\_\_\_\_\_\_\_\_\_\_\_\_\_\_\_\_\_\_\_\_\_\_\_\_\_\_\_\_\_\_\_\_\_\_\_\_\_\_\_\_\_\_\_\_\_\_\_\_\_\_\_\_\_\_\_\_\_\_\_\_\_\_\_\_\_\_\_\_\_\_\_\_\_\_\_\_\_\_\_\_\_\_\_\_\_\_\_\_\_\_\_\_\_\_\_\_\_\_\_\_\_\_\_\_\_\_\_\_\_\_\_\_\_\_\_\_\_\_\_\_\_\_\_\_\_\_\_\_\_\_\_\_\_\_\_\_\_\_\_\_\_\_\_\_\_\_\_\_\_\_\_\_\_\_\_\_\_\_\_\_\_\_\_\_\_\_\_\_\_\_\_\_\_\_\_\_\_\_\_\_\_\_\_\_\_\_\_\_\_\_\_\_\_\_\_\_\_\_\_\_\_\_\_\_\_\_\_\_\_\_\_\_\_\_\_\_\_\_\_\_\_\_\_\_\_\_\_\_\_\_\_\_\_\_\_\_\_\_\_\_\_\_\_\_\_\_\_\_\_\_\_\_\_\_\_\_\_\_\_\_\_\_\_\_\_\_\_\_\_\_\_\_\_\_\_\_\_\_\_\_\_\_\_\_\_\_\_\_\_\_\_\_\_\_\_\_\_\_\_\_\_\_\_\_\_\_\_\_\_\_\_\_\_\_\_\_\_\_\_\_\_\_\_\_\_\_\_\_\_\_\_\_\_\_\_\_\_\_\_\_\_\_\_\_\_\_\_\_\_\_\_\_\_\_\_\_\_\_\_\_\_\_\_\_\_\_\_\_\_\_\_\_\_\_\_\_\_\_\_\_\_\_\_\_\_\_\_\_\_\_\_\_\_\_\_\_\_\_\_\_\_\_\_\_\_\_\_\_\_\_\_\_\_\_\_\_\_\_\_\_\_\_\_\_\_\_\_\_\_\_\_\_\_\_\_\_\_\_\_\_\_\_\_\_\_\_\_\_\_\_\_\_\_\_\_\_\_\_\_\_\_\_\_\_\_\_\_\_\_\_\_\_\_\_\_\_\_\_\_\_\_\_\_\_\_\_\_\_\_\_\_\_\_\_\_\_\_\_\_\_\_\_\_\_\_\_\_\_\_\_\_\_\_\_\_\_\_\_\_\_\_\_\_\_\_\_\_\_\_\_\_\_\_\_\_\_\_\_\_\_\_\_\_\_\_\_\_\_\_\_\_\_\_\_\_\_\_\_\_\_\_\_\_\_\_\_\_\_\_\_\_\_\_\_\_\_\_\_\_\_\_\_\_\_\_\_\_\_\_\_\_\_\_\_\_\_\_\_\_\_\_\_\_\_\_\_\_\_\_\_\_\_\_\_\_\_\_\_\_\_\_\_\_\_\_\_\_\_\_\_\_\_\_\_\_\_\_\_\_\_\_\_\_\_\_\_\_\_\_\_\_\_\_\_\_\_\_\_\_\_\_\_\_\_\_\_\_\_\_\_\_\_\_\_\_\_\_\_\_\_\_\_\_\_\_\_\_\_\_\_\_\_\_\_\_\_\_\_\_\_\_\_\_\_\_\_\_\_\_\_\_\_\_\_\_\_\_\_\_\_\_\_\_\_\_\_\_\_\_\_\_\_\_\_\_\_\_\_\_\_\_\_\_\_\_\_\_\_\_\_\_\_\_\_\_\_\_\_\_\_\_\_\_\_\_\_\_\_\_\_\_\_\_\_\_\_\_\_\_\_\_\_\_\_\_\_\_\_\_\_\_\_\_\_\_\_\_\_\_\_\_\_\_\_\_\_\_\_\_\_\_\_\_\_\_\_\_\_\_\_\_\_\_\_\_\_\_\_\_\_\_\_\_\_\_\_\_\_\_\_\_\_\_\_\_\_\_\_\_\_\_\_\_\_\_\_\_\_\_\_\_\_\_\_\_\_\_\_\_\_\_\_\_\_\_\_\_\_\_\_\_\_\_\_\_\_\_\_\_\_\_\_\_\_\_\_\_\_\_\_\_\_\_\_\_\_\_\_\_\_\_\_\_\_\_\_\_\_\_\_\_\_\_\_\_\_\_\_\_\_\_\_\_\_\_\_\_\_\_\_\_\_\_\_\_\_\_\_\_\_\_\_\_\_\_\_\_\_\_\_\_\_\_\_\_\_\_\_\_\_\_\_\_\_\_\_\_\_\_\_\_\_\_\_\_\_\_\_\_\_\_\_\_\_\_\_\_\_\_\_\_\_\_\_\_\_\_\_\_\_\_\_\_\_\_\_\_\_\_\_\_\_\_\_\_\_\_\_\_\_\_\_\_\_\_\_\_\_\_\_\_\_\_\_\_\_\_\_\_\_\_\_\_\_\_\_\_\_\_\_\_\_\_\_\_\_\_\_\_\_\_\_\_\_\_\_\_\_\_\_\_\_\_\_\_\_\_\_\_\_\_\_\_\_\_\_\_\_\_\_\_\_\_\_\_\_\_\_\_\_\_\_\_\_\_\_\_\_\_\_\_\_\_\_\_\_\_\_\_\_\_\_\_\_\_\_\_\_\_\_\_\_\_\_\_\_\_\_\_\_\_\_\_\_\_\_\_\_\_\_\_\_\_\_\_\_\_\_\_\_\_\_\_\_\_\_\_\_\_\_\_\_\_\_\_\_\_\_\_\_\_\_\_\_\_\_\_\_\_\_\_\_\_\_\_\_\_\_\_\_\_\_\_\_\_\_\_\_\_\_\_\_\_\_\_\_\_\_\_\_\_\_\_\_\_\_\_\_\_\_\_\_\_\_\_\_\_\_\_\_\_\_\_\_\_\_\_\_\_\_\_\_\_\_\_\_\_\_\_\_\_\_\_\_\_\_\_\_\_\_\_\_\_\_\_\_\_\_\_\_\_\_\_\_\_\_\_\_\_\_\_\_\_\_\_\_\_\_\_\_\_\_\_\_\_\_\_\_\_\_\_\_\_\_\_\_\_\_\_\_\_\_\_\_\_\_\_\_\_\_\_\_\_\_\_\_\_\_\_\_\_\_\_\_\_\_\_\_\_\_\_\_\_\_\_\_\_\_\_\_\_\_\_\_\_\_\_\_\_\_\_\_\_\_\_\_\_\_\_\_\_\_\_\_\_\_\_\_\_\_\_\_\_\_\_\_\_\_\_\_\_\_\_\_\_\_\_\_\_\_\_\_\_\_\_\_\_\_\_\_\_\_\_\_\_\_\_\_\_\_\_\_\_\_\_\_\_\_\_\_\_\_\_\_\_\_\_\_\_\_\_\_\_\_\_\_\_\_\_\_\_\_\_\_\_\_\_\_\_\_\_\_\_\_\_\_\_\_\_\_\_\_\_\_\_\_\_\_\_\_\_\_\_\_\_\_\_\_\_\_\_\_\_\_\_\_\_\_\_\_\_\_\_\_\_\_\_\_\_\_\_\_\_\_\_\_\_\_\_\_\_\_\_\_\_\_\_\_\_\_\_\_\_\_\_\_\_\_\_\_\_\_\_\_\_\_\_\_\_\_\_\_\_\_\_\_\_\_\_\_\_\_\_\_\_\_\_\_\_\_\_\_\_\_\_\_\_\_\_\_\_\_\_\_\_\_\_\_\_\_\_\_\_\_\_\_\_\_\_\_\_\_\_\_\_\_\_\_\_\_\_\_\_\_\_\_\_\_\_\_\_\_\_\_\_\_\_\_\_\_\_\_\_\_\_\_\_\_\_\_\_\_\_\_\_\_\_\_\_\_\_\_\_\_\_\_\_\_\_\_\_\_\_\_\_\_\_\_\_\_\_\_\_\_\_\_\_\_\_\_\_\_\_\_\_\_\_\_\_\_\_\_\_\_\_\_\_\_\_\_\_\_\_\_\_\_\_\_\_\_\_\_\_\_\_\_\_\_\_\_\_\_\_\_\_\_\_\_\_\_\_\_\_\_\_\_\_\_\_\_\_\_\_\_\_\_\_\_\_\_\_\_\_\_\_\_\_\_\_\_\_\_\_\_\_\_\_\_\_\_\_\_\_\_\_\_\_\_\_\_\_\_\_\_\_\_\_\_\_\_\_\_\_\_\_\_\_\_\_\_\_\_\_\_\_\_\_\_\_\_\_\_\_\_\_\_\_\_\_\_\_\_\_\_\_\_\_\_\_\_\_\_\_\_\_\_\_\_\_\_\_\_\_\_\_\_\_\_\_\_\_\_\_\_\_\_\_\_\_\_\_\_\_\_\_\_\_\_\_\_\_\_\_\_\_\_\_\_\_\_\_\_\_\_\_\_\_\_\_\_\_\_\_\_\_\_\_\_\_\_\_\_\_\_\_\_\_\_\_\_\_\_\_\_\_\_\_\_\_\_\_\_\_\_\_\_\_\_\_\_\_\_\_\_\_\_\_\_\_\_\_\_\_\_\_\_\_\_\_\_\_\_\_\_\_\_\_\_\_\_\_\_\_\_\_\_\_\_\_\_\_\_\_\_\_\_\_\_\_\_\_\_\_\_\_\_\_\_\_\_\_\_\_\_\_\_\_\_\_\_\_\_\_\_\_\_\_\_\_\_\_\_\_\_\_\_\_\_\_\_\_\_\_\_\_\_\_\_\_\_\_\_\_\_\_\_\_\_\_\_\_\_\_\_\_\_\_\_\_\_\_\_\_\_\_\_\_\_\_\_\_\_\_\_\_\_\_\_\_\_\_\_\_\_\_\_\_\_\_\_\_\_\_\_\_\_\_\_\_\_\_\_\_\_\_\_\_\_\_\_\_\_\_\_\_\_\_\_\_\_\_\_\_\_\_\_\_\_\_\_\_\_\_\_\_\_\_\_\_\_\_\_\_\_\_\_\_\_\_\_\_\_\_\_\_\_\_\_\_\_\_\_\_\_\_\_\_\_\_\_\_\_\_\_\_\_\_\_\_\_\_\_\_\_\_\_\_\_\_\_\_\_\_\_\_\_\_\_\_\_\_\_\_\_\_\_\_\_\_\_\_\_\_\_\_\_\_\_\_\_\_\_\_\_\_\_\_\_\_\_\_\_\_\_\_\_\_\_\_\_\_\_\_\_\_\_\_\_\_\_\_\_\_\_\_\_\_\_\_\_\_\_\_\_\_\_\_\_\_\_\_\_\_\_\_\_\_\_\_\_\_\_\_\_\_\_\_\_\_\_\_\_\_\_\_\_\_\_\_\_\_\_\_\_\_\_\_\_\_\_\_\_\_\_\_\_\_\_\_\_\_\_\_\_\_\_\_\_\_\_\_\_\_\_\_\_\_\_\_\_\_\_\_\_\_\_\_\_\_\_\_\_\_\_\_\_\_\_\_\_\_\_\_\_\_\_\_\_\_\_\_\_\_\_\_\_\_\_\_\_\_\_\_\_\_\_\_\_\_\_\_\_\_\_\_\_\_\_\_\_\_\_\_\_\_\_\_\_\_\_\_\_\_\_\_\_\_\_\_\_\_\_\_\_\_\_\_\_\_\_\_\_\_\_\_\_\_\_\_\_\_\_\_\_\_\_\_\_\_\_\_\_\_\_\_\_\_\_\_\_\_\_\_\_\_\_\_\_\_\_\_\_\_\_\_\_\_\_\_\_\_\_\_\_\_\_\_\_\_\_\_\_\_\_\_\_\_\_\_\_\_\_\_\_\_\_\_\_\_\_\_\_\_\_\_\_\_\_\_\_\_\_\_\_\_\_\_\_\_\_\_\_\_\_\_\_\_\_\_\_\_\_\_\_\_\_\_\_\_\_\_\_\_\_\_\_\_\_\_\_\_\_\_\_\_\_\_\_\_\_\_\_\_\_\_\_\_\_\_\_\_\_\_\_\_\_\_\_\_\_\_\_\_\_\_\_\_\_\_\_\_\_\_\_\_\_\_\_\_\_\_\_\_\_\_\_\_\_\_\_\_\_\_\_\_\_\_\_\_\_\_\_\_\_\_\_\_\_\_\_\_\_\_\_\_\_\_\_\_\_\_\_\_\_\_\_\_\_\_\_\_\_\_\_\_\_\_\_\_\_\_\_\_\_\_\_\_\_\_\_\_\_\_\_\_\_\_\_\_\_\_\_\_\_\_\_\_\_\_\_\_\_\_\_\_\_\_\_\_\_\_\_\_\_\_\_\_\_\_\_\_\_\_\_\_\_\_\_\_\_\_\_\_\_\_\_\_\_\_\_\_\_\_\_\_\_\_\_\_\_\_\_\_\_\_\_\_\_\_\_\_\_\_\_\_\_\_\_\_\_\_\_\_\_\_\_\_\_\_\_\_\_\_\_\_\_\_\_\_\_\_\_\_\_\_\_\_\_\_\_\_\_\_\_\_\_\_\_\_\_\_\_\_\_\_\_\_\_\_\_\_\_\_\_\_\_\_\_\_\_\_\_\_\_\_\_\_\_\_\_\_\_\_\_\_\_\_\_\_\_\_\_\_\_\_\_\_\_\_\_\_\_\_\_\_\_\_\_\_\_\_\_\_\_\_\_\_\_\_\_\_\_\_\_\_\_\_\_\_\_\_\_\_\_\_\_\_\_\_\_\_\_\_\_\_\_\_\_\_\_\_\_\_\_\_\_\_\_\_\_\_\_\_\_\_\_\_\_\_\_\_\_\_\_\_\_\_\_\_\_\_\_\_\_\_\_\_\_\_\_\_\_\_\_\_\_\_\_\_\_\_\_\_\_\_\_\_\_\_\_\_\_\_\_\_\_\_\_\_\_\_\_\_\_\_\_\_\_\_\_\_\_\_\_\_\_\_\_\_\_\_\_\_\_\_\_\_\_\_\_\_\_\_\_\_\_\_\_\_\_\_\_\_\_\_\_\_\_\_\_\_\_\_\_\_\_\_\_\_\_\_\_\_\_\_\_\_\_\_\_\_\_\_\_\_\_\_\_\_\_\_\_\_\_\_\_\_\_\_\_\_\_\_\_\_\_\_\_\_\_\_\_\_\_\_\_\_\_\_\_\_\_\_\_\_\_\_\_\_\_\_\_\_\_\_\_\_\_\_\_\_\_\_\_\_\_\_\_\_\_\_\_\_\_\_\_\_\_\_\_\_\_\_\_\_\_\_\_\_\_\_\_\_\_\_\_\_\_\_\_\_\_\_\_\_\_\_\_\_\_\_\_\_\_\_\_\_\_\_\_\_\_\_\_\_\_\_\_\_\_\_\_\_\_\_\_\_\_\_\_\_\_\_\_\_\_\_\_\_\_\_\_\_\_\_\_\_\_\_\_\_\_\_\_\_\_\_\_\_\_\_\_\_\_\_\_\_\_\_\_\_\_\_\_\_\_\_\_\_\_\_\_\_\_\_\_\_\_\_\_\_\_\_\_\_\_\_\_\_\_\_\_\_\_\_\_\_\_\_\_\_\_\_\_\_\_\_\_\_\_\_\_\_\_\_\_\_\_\_\_\_\_\_\_\_\_\_\_\_\_\_\_\_\_\_\_\_\_\_\_\_\_\_\_\_\_\_\_\_\_\_\_\_\_\_\_\_\_\_\_\_\_\_\_\_\_\_\_\_\_\_\_\_\_\_\_\_\_\_\_\_\_\_\_\_\_\_\_\_\_\_\_\_\_\_\_\_\_\_\_\_\_\_\_\_\_\_\_\_\_\_\_\_\_\_\_\_\_\_\_\_\_\_\_\_\_\_\_\_\_\_\_\_\_\_\_\_\_\_\_\_\_\_\_\_\_\_\_\_\_\_\_\_\_\_\_\_\_\_\_\_\_\_\_\_\_\_\_\_\_\_\_\_\_\_\_\_\_\_\_\_\_\_\_\_\_\_\_\_\_\_\_\_\_\_\_\_\_\_\_\_\_\_\_\_\_\_\_\_\_\_\_\_\_\_\_\_\_\_\_\_\_\_\_\_\_\_\_\_\_\_\_\_\_\_\_\_\_\_\_\_\_\_\_\_\_\_\_\_\_\_\_\_\_\_\_\_\_\_\_\_\_\_\_\_\_\_\_\_\_\_\_\_\_\_\_\_\_\_\_\_\_\_\_\_\_\_\_\_\_\_\_\_\_\_\_\_\_\_\_\_\_\_\_\_\_\_\_\_\_\_\_\_\_\_\_\_\_\_\_\_\_\_\_\_\_\_\_\_\_\_\_\_\_\_\_\_\_\_\_\_\_\_\_\_\_\_\_\_\_\_\_\_\_\_\_\_\_\_\_\_\_\_\_\_\_\_\_\_\_\_\_\_\_\_\_\_\_\_\_\_\_\_\_\_\_\_\_\_\_\_\_\_\_\_\_\_\_\_\_\_\_\_\_\_\_\_\_\_\_\_\_\_\_\_\_\_\_\_\_\_\_\_\_\_\_\_\_\_\_\_\_\_\_\_\_\_\_\_\_\_\_\_\_\_\_\_\_\_\_\_\_\_\_\_\_\_\_\_\_\_\_\_\_\_\_\_\_\_\_\_\_\_\_\_\_\_\_\_\_\_\_\_\_\_\_\_\_\_\_\_\_\_\_\_\_\_\_\_\_\_\_\_\_\_\_\_\_\_\_\_\_\_\_\_\_\_\_\_\_\_\_\_\_\_\_\_\_\_\_\_\_\_\_\_\_\_\_\_\_\_\_\_\_\_\_\_\_\_\_\_\_\_\_\_\_\_\_\_\_\_\_\_\_\_\_\_\_\_\_\_\_\_\_\_\_\_\_\_\_\_\_\_\_\_\_\_\_\_\_\_\_\_\_\_\_\_\_\_\_\_\_\_\_\_\_\_\_\_\_\_\_\_\_\_\_\_\_\_\_\_\_\_\_\_\_\_\_\_\_\_\_\_\_\_\_\_\_\_\_\_\_\_\_\_\_\_\_\_\_\_\_\_\_\_\_\_\_\_\_\_\_\_\_\_\_\_\_\_\_\_\_\_\_\_\_\_\_\_\_\_\_\_\_\_\_\_\_\_\_\_\_\_\_\_\_\_\_\_\_\_\_\_\_\_\_\_\_\_\_\_\_\_\_\_\_\_\_\_\_\_\_\_\_\_\_\_\_\_\_\_\_\_\_\_\_\_\_\_\_\_\_\_\_\_\_\_\_\_\_\_\_\_\_\_\_\_\_\_\_\_\_\_\_\_\_\_\_\_\_\_\_\_\_\_\_\_\_\_\_\_\_\_\_\_\_\_\_\_\_\_\_\_\_\_\_\_\_\_\_\_\_\_\_\_\_\_\_\_\_\_\_\_\_\_\_\_\_\_\_\_\_\_\_\_\_\_\_\_\_\_\_\_\_\_\_\_\_\_\_\_\_\_\_\_\_\_\_\_\_\_\_\_\_\_\_\_\_\_\_\_\_\_\_\_\_\_\_\_\_\_\_\_\_\_\_\_\_\_\_\_\_\_\_\_\_\_\_\_\_\_\_\_\_\_\_\_\_\_\_\_\_\_\_\_\_\_\_\_\_\_\_\_\_\_\_\_\_\_\_\_\_\_\_\_\_\_\_\_\_\_\_\_\_\_\_\_\_\_\_\_\_\_\_\_\_\_\_\_\_\_\_\_\_\_\_\_\_\_\_\_\_\_\_\_\_\_\_\_\_\_\_\_\_\_\_\_\_\_\_\_\_\_\_\_\_\_\_\_\_\_\_\_\_\_\_\_\_\_\_\_\_\_\_\_\_\_\_\_\_\_\_\_\_\_\_\_\_\_\_\_\_\_\_\_\_\_\_\_\_\_\_\_\_\_\_\_\_\_\_\_\_\_\_\_\_\_\_\_\_\_\_\_\_\_\_\_\_\_\_\_\_\_\_\_\_\_\_\_\_\_\_\_\_\_\_\_\_\_\_\_\_\_\_\_\_\_\_\_\_\_\_\_\_\_\_\_\_\_\_\_\_\_\_\_\_\_\_\_\_\_\_\_\_\_\_\_\_\_\_\_\_\_\_\_\_\_\_\_\_\_\_\_\_\_\_\_\_\_\_\_\_\_\_\_\_\_\_\_\_\_\_\_\_\_\_\_\_\_\_\_\_\_\_\_\_\_\_\_\_\_\_\_\_\_\_\_\_\_\_\_\_\_\_\_\_\_\_\_\_\_\_\_\_\_\_\_\_\_\_\_\_\_\_\_\_\_\_\_\_\_\_\_\_\_\_\_\_\_\_\_\_\_\_\_\_\_\_\_\_\_\_\_\_\_\_\_\_\_\_\_\_\_\_\_\_\_\_\_\_\_\_\_\_\_\_\_\_\_\_\_\_\_\_\_\_\_\_\_\_\_\_\_\_\_\_\_\_\_\_\_\_\_\_\_\_\_\_\_\_\_\_\_\_\_\_\_\_\_\_\_\_\_\_\_\_\_\_\_\_\_\_\_\_\_\_\_\_\_\_\_\_\_\_\_\_\_\_\_\_\_\_\_\_\_\_\_\_\_\_\_\_\_\_\_\_\_\_\_\_\_\_\_\_\_\_\_\_\_\_\_\_\_\_\_\_\_\_\_\_\_\_\_\_\_\_\_\_\_\_\_\_\_\_\_\_\_\_\_\_\_\_\_\_\_\_\_\_\_\_\_\_\_\_\_\_\_\_\_\_\_\_\_\_\_\_\_\_\_\_\_\_\_\_\_\_\_\_\_\_\_\_\_\_\_\_\_\_\_\_\_\_\_\_\_\_\_\_\_\_\_\_\_\_\_\_\_\_\_\_\_\_\_\_\_\_\_\_\_\_\_\_\_\_\_\_\_\_\_\_\_\_\_\_\_\_\_\_\_\_\_\_\_\_\_\_\_\_\_\_\_\_\_\_\_\_\_\_\_\_\_\_\_\_\_\_\_\_\_\_\_\_\_\_\_\_\_\_\_\_\_\_\_\_\_\_\_\_\_\_\_\_\_\_\_\_\_\_\_\_\_\_\_\_\_\_\_\_\_\_\_\_\_\_\_\_\_\_\_\_\_\_\_\_\_\_\_\_\_\_\_\_\_\_\_\_\_\_\_\_\_\_\_\_\_\_\_\_\_\_\_\_\_\_\_\_\_\_\_\_\_\_\_\_\_\_\_\_\_\_\_\_\_\_\_\_\_\_\_\_\_\_\_\_\_\_\_\_\_\_\_\_\_\_\_\_\_\_\_\_\_\_\_\_\_\_\_\_\_\_\_\_\_\_\_\_\_\_\_\_\_\_\_\_\_\_\_\_\_\_\_\_\_\_\_\_\_\_\_\_\_\_\_\_\_\_\_\_\_\_\_\_\_\_\_\_\_\_\_\_\_\_\_\_\_\_\_\_\_\_\_\_\_\_\_\_\_\_\_\_\_\_\_\_\_\_\_\_\_\_\_\_\_\_\_\_\_\_\_\_\_\_\_\_\_\_\_\_\_\_\_\_\_\_\_\_\_\_\_\_\_\_\_\_\_\_\_\_\_\_\_\_\_\_\_\_\_\_\_\_\_\_\_\_\_\_\_\_\_\_\_\_\_\_\_\_\_\_\_\_\_\_\_\_\_\_\_\_\_\_\_\_\_\_\_\_\_\_\_\_\_\_\_\_\_\_\_\_\_\_\_\_\_\_\_\_\_\_\_\_\_\_\_\_\_\_\_\_\_\_\_\_\_\_\_\_\_\_\_\_\_\_\_\_\_\_\_\_\_\_\_\_\_\_\_\_\_\_\_\_\_\_\_\_\_\_\_\_\_\_\_\_\_\_\_\_\_\_\_\_\_\_\_\_\_\_\_\_\_\_\_\_\_\_\_\_\_\_\_\_\_\_\_\_\_\_\_\_\_\_\_\_\_\_\_\_\_\_\_\_\_\_\_\_\_\_\_\_\_\_\_\_\_\_\_\_\_\_\_\_\_\_\_\_\_\_\_\_\_\_\_\_\_\_\_\_\_\_\_\_\_\_\_\_\_\_\_\_\_\_\_\_\_\_\_\_\_\_\_\_\_\_\_\_\_\_\_\_\_\_\_\_\_\_\_\_\_\_\_\_\_\_\_\_\_\_\_\_\_\_\_\_\_\_\_\_\_\_\_\_\_\_\_\_\_\_\_\_\_\_\_\_\_\_\_\_\_\_\_\_\_\_\_\_\_\_\_\_\_\_\_\_\_\_\_\_\_\_\_\_\_\_\_\_\_\_\_\_\_\_\_\_\_\_\_\_\_\_\_\_\_\_\_\_\_\_\_\_\_\_\_\_\_\_\_\_\_\_\_\_\_\_\_\_\_\_\_\_\_\_\_\_\_\_\_\_\_\_\_\_\_\_\_\_\_\_\_\_\_\_\_\_\_\_\_\_\_\_\_\_\_\_\_\_\_\_\_\_\_\_\_\_\_\_\_\_\_\_\_\_\_\_\_\_\_\_\_\_\_\_\_\_\_\_\_\_\_\_\_\_\_\_\_\_\_\_\_\_\_\_\_\_\_\_\_\_\_\_\_\_\_\_\_\_\_\_\_\_\_\_\_\_\_\_\_\_\_\_\_\_\_\_\_\_\_\_\_\_\_\_\_\_\_\_\_\_\_\_\_\_\_\_\_\_\_\_\_\_\_\_\_\_\_\_\_\_\_\_\_\_\_\_\_\_\_\_\_\_\_\_\_\_\_\_\_\_\_\_\_\_\_\_\_\_\_\_\_\_\_\_\_\_\_\_\_\_\_\_\_\_\_\_\_\_\_\_\_\_\_\_\_\_\_\_\_\_\_\_\_\_\_\_\_\_\_\_\_\_\_\_\_\_\_\_\_\_\_\_\_\_\_\_\_\_\_\_\_\_\_\_\_\_\_\_\_\_\_\_\_\_\_\_\_\_\_\_\_\_\_\_\_\_\_\_\_\_\_\_\_\_\_\_\_\_\_\_\_\_\_\_\_\_\_\_\_\_\_\_\_\_\_\_\_\_\_\_\_\_\_\_\_\_\_\_\_\_\_\_\_\_\_\_\_\_\_\_\_\_\_\_\_\_\_\_\_\_\_\_\_\_\_\_\_\_\_\_\_\_\_\_\_\_\_\_\_\_\_\_\_\_\_\_\_\_\_\_\_\_\_\_\_\_\_\_\_\_\_\_\_\_\_\_\_\_\_\_\_\_\_\_\_\_\_\_\_\_\_\_\_\_\_\_\_\_\_\_\_\_\_\_\_\_\_\_\_\_\_\_\_\_\_\_\_\_\_\_\_\_\_\_\_\_\_\_\_\_\_\_\_\_\_\_\_\_\_\_\_\_\_\_\_\_\_\_\_\_\_\_\_\_\_\_\_\_\_\_\_\_\_\_\_\_\_\_\_\_\_\_\_\_\_\_\_\_\_\_\_\_\_\_\_\_\_\_\_\_\_\_\_\_\_\_\_\_\_\_\_\_\_\_\_\_\_\_\_\_\_\_\_\_\_\_\_\_\_\_\_\_\_\_\_\_\_\_\_\_\_\_\_\_\_\_\_\_\_\_\_\_\_\_\_\_\_\_\_\_\_\_\_\_\_\_\_\_\_\_\_\_\_\_\_\_\_\_\_\_\_\_\_\_\_\_\_\_\_\_\_\_\_\_\_\_\_\_\_\_\_\_\_\_\_\_\_\_\_\_\_\_\_\_\_\_\_\_\_\_\_\_\_\_\_\_\_\_\_\_\_\_\_\_\_\_\_\_\_\_\_\_\_\_\_\_\_\_\_\_\_\_\_\_\_\_\_\_\_\_\_\_\_\_\_\_\_\_\_\_\_\_\_\_\_\_\_\_\_\_\_\_\_\_\_\_\_\_\_\_\_\_\_\_\_\_\_\_\_\_\_\_\_\_\_\_\_\_\_\_\_\_\_\_\_\_\_\_\_\_\_\_\_\_\_\_\_\_\_\_\_\_\_\_\_\_\_\_\_\_\_\_\_\_\_\_\_\_\_\_\_\_\_\_\_\_\_\_\_\_\_\_\_\_\_\_\_\_\_\_\_\_\_\_\_\_\_\_\_\_\_\_\_\_\_\_\_\_\_\_\_\_\_\_\_\_\_\_\_\_\_\_\_\_\_\_\_\_\_\_\_\_\_\_\_\_\_\_\_\_\_\_\_\_\_\_\_\_\_\_\_\_\_\_\_\_\_\_\_\_\_\_\_\_\_\_\_\_\_\_\_\_\_\_\_\_\_\_\_\_\_\_\_\_\_\_\_\_\_\_\_\_\_\_\_\_\_\_\_\_\_\_\_\_\_\_\_\_\_\_\_\_\_\_\_\_\_\_\_\_\_\_\_\_\_\_\_\_\_\_\_\_\_\_\_\_\_\_\_\_\_\_\_\_\_\_\_\_\_\_\_\_\_\_\_\_\_\_\_\_\_\_\_\_\_\_\_\_\_\_\_\_\_\_\_\_\_\_\_\_\_\_\_\_\_\_\_\_\_\_\_\_\_\_\_\_\_\_\_\_\_\_\_\_\_\_\_\_\_\_\_\_\_\_\_\_\_\_\_\_\_\_\_\_\_\_\_\_\_\_\_\_\_\_\_\_\_\_\_\_\_\_\_\_\_\_\_\_\_\_\_\_\_\_\_\_\_\_\_\_\_\_\_\_\_\_\_\_\_\_\_\_\_\_\_\_\_\_\_\_\_\_\_\_\_\_\_\_\_\_\_\_\_\_\_\_\_\_\_\_\_\_\_\_\_\_\_\_\_\_\_\_\_\_\_\_\_\_\_\_\_\_\_\_\_\_\_\_\_\_\_\_\_\_\_\_\_\_\_\_\_\_\_\_\_\_\_\_\_\_\_\_\_\_\_\_\_\_\_\_\_\_\_\_\_\_\_\_\_\_\_\_\_\_\_\_\_\_\_\_\_\_\_\_\_\_\_\_\_\_\_\_\_\_\_\_\_\_\_\_\_\_\

\section*{目录}
\begin{itemize}
    \item[1] 问题重述 \dotfill 1
    \item[2] 问题分析 \dotfill 2
    \begin{itemize}
        \item[2.1] 问题1的分析 \dotfill 2
        \item[2.2] 问题2的分析 \dotfill 3
        \item[2.3] 问题3的分析 \dotfill 4
    \end{itemize}
    \item[3] 问题假设 \dotfill 5
    \item[4] 符号说明 \dotfill 5
    \item[5] 问题1模型的建立和求解 \dotfill 6
    \begin{itemize}
        \item[5.1] 问题1.1模型的建立和求解 \dotfill 6
        \begin{itemize}
            \item[5.1.1] Q星数据分析 \dotfill 6
            \item[5.1.2] 问题1.1模型的建立 \dotfill 7
            \item[5.1.3] 问题1.1模型的求解 \dotfill 11
        \end{itemize}
        \item[5.2] 问题1.2模型的建立和求解 \dotfill 14
        \begin{itemize}
            \item[5.2.1] L星数据分析 \dotfill 14
            \item[5.2.2] 问题1.2模型的建立与求解 \dotfill 16
        \end{itemize}
        \item[5.3] 问题1.3模型的建立与求解 \dotfill 18
        \begin{itemize}
            \item[5.3.1] 十次观测数据分析 \dotfill 18
            \item[5.3.2] 问题1.3模型的建立与求解 \dotfill 19
        \end{itemize}
    \end{itemize}
    \item[6] 问题2模型的建立和求解 \dotfill 20
    \begin{itemize}
        \item[6.1] 模型的建立 \dotfill 20
        \item[6.2] 模型的求解 \dotfill 21
        \begin{itemize}
            \item[6.2.1] 避开L-1型卫星的最优机动路径 \dotfill 23
            \item[6.2.2] Q型卫星的过顶情况预测 \dotfill 26
        \end{itemize}
    \end{itemize}
    \item[7] 问题3模型的建立和求解 \dotfill 26
    \begin{itemize}
        \item[7.1] 过顶情况数据分析 \dotfill 26
        \item[7.2] 模型的建立 \dotfill 27
        \item[7.3] 模型的求解 \dotfill 27
    \end{itemize}
\end{itemize}

\begin{itemize}
    \item[8] 模型的评价和改进 \dotfill 28
    \item[8.1] 误差分析 \dotfill 28
    \item[8.2] 优缺点 \dotfill 29
    \item[8.3] 模型的改进 \dotfill 29
    \item[9] 参考文献 \dotfill 30
    \item[10] 附录 \dotfill 30
\end{itemize}

\section{问题重述}

社会的发展离不开国防建设,目前国防建设和部队活动开展频繁,而国外对我部队侦察密集,因为无法抵近侦察,采用卫星侦察方式已经常态化。为了保护军事秘密,维护国家安全,需要防止敌卫星侦察,主要通过伪装和躲避的方式。这就要求掌握卫星的运行规律和过顶数据,本题对卫星运行规律和过顶情况的计算有以下两个方面的要求:

(1) 对于国外军用侦察卫星来说,由于其根数无法获得,无法通过六要素运动方程直接计算出卫星的实时位置数据和侦察范围以及过顶时间。那么此时,对卫星运行规律的探索只能通过观测数据进行预测。

(2) 本题是通过观测站对卫星的过境情况进行观测,得到观测数据,进而利用观测到的数据进行相关推导和计算。

依据观测数据,解决以下三个方面的问题:

\textbf{问题 1:} 在北纬 $31.90 \sim 32.25$ 度,东经 $118.02 \sim 118.91$ 度这一区域范围内准备建设一大型国防工程,但是该地域长期受到 Q 型、L 型卫星监视,为了躲避境外卫星的过顶侦察,在该区域内配置了一个观察站(观察站位置:北纬:$32.0209$ 度;东经:$118.7681$ 度),通过观察站对该区域的过境卫星(Q 型、L 型、K 型卫星)进行观测,得到了一些观测数据,根据观测数据求解以下三个子问题:

\textbf{问题 1.1:} Q 型卫星侦察隐蔽问题。根据题目附件 2-1 中给出的 D0、D1、D2 日 Q 型卫星被该观察站观测到的情况,包括观察站最先观察到卫星的时刻、离观察站最近的时刻、最后观察到的时刻以及三个时刻对应的天线方位角、天线仰角、卫星亮度、卫星与观测站距离、太阳方位角、太阳仰角。

根据观测到的这些数据,建立数学模型,预测 D3、D5 日的卫星被观测到的情况及过顶情况,并结合 Q 型卫星的侦察范围给出 D3、D5 两天内确保安全施工的时段,即除去过顶时间外的时段。

\textbf{问题 1.2:} L 型卫星侦察隐蔽问题。L 型卫星为双星卫星,是 L-1、L-2 两颗卫星协同工作的卫星系统。题目附件 2-2 给出了 L-1、L-2 卫星在 8 月 16 日-21 日被观察站观测到的情况,需要完成以下两个任务:

(1) 根据观测到的这些数据,建立数学模型,预测 8 月 23 日卫星被观测到的情况及过顶情况,并给出确保安全施工的时段。

(2) 由于 L 型卫星是 SAR(雷达成像照相)侦察卫星,其特点是能全天时、全天候侦察,不受天气和光照等因素的影响,并且具有一定的穿透能力,普通的伪装网是不能遮蔽其侦察的,根据卫星的这些特点和预测的卫星过顶数据,对 L 型卫星侦察方面的薄弱环节进行分析。

\textbf{问题 1.3:} K 型卫星侦察隐蔽问题。观测站对 K 型卫星观测到的数据只有十次,根据这连续十次的观测数据,预测其未来三次被观测到的情况,并说明该卫星已经被连续观察最少 $n$ 次才能够确定下次被观测到的情况所需要的 $n$,以及观察次数对预报精度的影响。

\textbf{问题 2:}

某部需要从新疆的阿勒泰隐蔽地经喀什运动到和田并在和田执行某任务,24 小时后再隐蔽地返回阿勒泰(不必经喀什),部队可以按需要选择在高速公路(最大速度 100 公里/小时)或普通公路(除高速之外的其他公路,最大速度 50 公里)上行进,假设部队出发时(2016 年 11 月 1 日凌晨 5 时整)Q 型卫星、L-1 卫星(它们的轨道要素见附

件 1, 其他 L 型卫星都不考虑)均位于各自轨道的近地点。行车时车队最大长度 2 千米, 部队每开进 10~12 小时可选择途经的县级以上(含县级)城市休息 10 小时以上(即连续开进时间不少于 10 小时,不多于 12 小时),请你们根据附件 3 给出的地图(必要时可借助因特网获取有关地理信息),设计合理的行军时机、路线和宿营地,避开 L-1 卫星侦察,并预测 Q 型卫星的过顶时刻,以便及时做好隐蔽工作,尽可能快地安全到达目的地。

\section*{问题 3:}

研究导弹发射装置的战时隐蔽问题。有专家提出,运动方式可能是移动发射装置规避卫星侦察的有效方案。请你们研究以下问题:假设某移动发射装置可在某一指定区域内自由运动,分别研究针对 Q 型、L 型(包括 L-1、L-2)、K 型卫星的侦察能够规避的可行性、条件(区域大小、形状、路网状况及其他你们认为需要的条件)和方式。欢迎进一步针对两种或三种卫星的组合侦察能够规避的可行性、条件和方式。并考虑卫星参数变化对方案的影响。

\section*{2 问题分析}

\subsection*{2.1 问题 1 的分析}

问题 1 是在题设要求的基础上,通过对附件 2-1、附件 2-2、附件 2-3 中的数据进行分析,预测观察站对 Q 型、L 型、K 型卫星的观测情况和卫星的过顶情况,解决目标区域何时不被侦察的问题,给出安全施工的时间段。为了对本题进行解答,需要结合卫星运行的物理背景,在对观测数据分析推导的前提下,从已知条件和天体运动规律(开普勒三大定律)出发,建立预测模型,从而解决以下三个子问题:

问题 1.1 是针对 Q 型卫星的预测问题。首先根据题目附件 2-1 中给出的 D0、D1、D2 日 Q 型卫星被该观察站观测到的情况,对数据进行分析,提取预测基础。附件 2-1 中给出的方位角、仰角和距离,都是针对地面观测站而言的,在空间中来看,数据信息是针对站心地平直角坐标系而言的(右手系,Y 轴指向正北,X 轴指向正东,观测站为 O 点,Z 轴为 O 点法线,指向天顶为正)。但是,根据开普勒定律,卫星在椭圆轨道上绕地球运行,地球在椭圆的一个焦点上,要预测卫星的过顶情况就需要预测卫星在观测时刻点对应的位置信息,那么这个位置信息的计算从空间来看,是基于球心直角坐标系的。那么,为了预测卫星观测情况,需要解决以下几个问题:

(1)确定 D3、D5 日的观测时刻;

(2)对已知卫星位置(已经观测到的数据)建立站心地平直角坐标系、大地坐标系和球心直角坐标系,并根据相对位置推导三个坐标系之间的转换关系;

(3)依据某一时刻卫星观测数据,通过坐标系之间的相互转换,计算下一观测时刻的卫星位置信息,进而得到卫星相对于观测点的方位角和仰角,以及距离和其他信息。

(4)根据预测得到的 D3、D5 日观测数据,计算过顶情况,得到目标区域的安全施工时段。

问题 1.2:L 型卫星侦察隐蔽问题。L 型卫星为双星卫星,是 L-1、L-2 两颗卫星协同工作的卫星系统。题目附件 2-2 给出了 L-1、L-2 卫星在 8 月 16 日-21 日被观察站观测到的情况,需要完成以下两个任务:

(1)根据观测到的这些数据,建立数学模型,预测 8 月 23 日卫星被观测到的情况及过顶情况,并给出确保安全施工的时段。

(2) 由于 L 型卫星是 SAR(雷达成像照相)侦察卫星,其特点是能全天时、全天候侦察,不受天气和光照等因素的影响,并且具有一定的穿透能力,普通的伪装网是不能遮蔽其侦察的,根据卫星的这些特点和预测的卫星过顶数据,对 L 型卫星侦察方面的薄弱环节进行分析。

问题 1.3:K 型卫星侦察隐蔽问题。观测站对 K 型卫星观测到的数据为连续十次,需要据此预测其未来三次被观测到的情况,并说明该卫星已经被连续观察最少 \( n \) 次才能够确定下次被观测到的情况所需要的 \( n \),以及观察次数对预报精度的影响。

对于此题,要根据相邻两次观测到的最短时间,即近似等于卫星绕行一周的时间,算出卫星运行的周期和每日卫星绕行圈数。先预测未来三天的观测时间,结合模型计算其他的卫星被观测情况。

问题 1 求解思想如图 1 所示。

\begin{figure}[h]
\centering
\begin{tikzpicture}[node distance=2cm, auto, >=latex']
    % 节点定义
    \node (start) [ellipse, draw] {开始};
    \node (analysis) [rectangle, draw, below of=start] {已观测数据分析,提取时间规律};
    \node (satellite_period) [rectangle, draw, below left of=analysis, xshift=-3cm] {卫星运行周期};
    \node (earth_rotation) [rectangle, draw, below of=satellite_period] {地球自转};
    \node (time_prediction) [rectangle, draw, below of=analysis] {卫星被观测时间预测};
    \node (model) [rectangle, draw, below of=time_prediction] {建立基于时间参量的预测模型};
    \node (coordinate_conversion) [rectangle, draw, right of=model, xshift=3cm] {空间位置坐标转换:大地坐标、球心直角坐标、站心地平直角坐标};
    \node (Q_satellite) [rectangle, draw, below of=satellite_period, yshift=-2cm] {Q 型卫星};
    \node (L_satellite) [rectangle, draw, below of=model, yshift=-2cm] {L 型卫星};
    \node (K_satellite) [rectangle, draw, below of=coordinate_conversion, yshift=-2cm] {K 型卫星};
    \node (D3_D5_observation) [rectangle, draw, below of=Q_satellite] {D3、D5 日被观测情况求解};
    \node (D3_D5_pass) [rectangle, draw, below of=D3_D5_observation] {D3、D5 日目标区域过顶情况求解};
    \node (safe_construction) [rectangle, draw, below of=D3_D5_pass] {安全施工时段};
    \node (L_satellite_observation) [rectangle, draw, below of=L_satellite] {8 月 22 日被观测情况和过顶情况};
    \node (L_satellite_pass) [rectangle, draw, below of=L_satellite_observation] {8 月 23 日被观测情况和过顶情况};
    \node (L_safe_construction) [rectangle, draw, below of=L_satellite_pass] {8 月 23 日安全施工时段};
    \node (K_future_observation) [rectangle, draw, below of=K_satellite] {未来三次卫星被观测情况预测};
    \node (K_min_observations) [rectangle, draw, below of=K_future_observation] {预测所需最少连续观测次数的确定};
    \node (end) [ellipse, draw, below of=L_safe_construction, xshift=3cm] {结束};

    % 边缘定义
    \draw[->] (start) -- (analysis);
    \draw[->] (analysis) -- (time_prediction);
    \draw[->] (satellite_period) -- (time_prediction);
    \draw[->] (earth_rotation) -- (time_prediction);
    \draw[->] (time_prediction) -- (model);
    \draw[->] (model) -- (Q_satellite);
    \draw[->] (model) -- (L_satellite);
    \draw[->] (model) -- (coordinate_conversion);
    \draw[->] (coordinate_conversion) -- (K_satellite);
    \draw[->] (Q_satellite) -- (D3_D5_observation);
    \draw[->] (D3_D5_observation) -- (D3_D5_pass);
    \draw[->] (D3_D5_pass) -- (safe_construction);
    \draw[->] (L_satellite) -- (L_satellite_observation);
    \draw[->] (L_satellite_observation) -- (L_satellite_pass);
    \draw[->] (L_satellite_pass) -- (L_safe_construction);
    \draw[->] (K_satellite) -- (K_future_observation);
    \draw[->] (K_future_observation) -- (K_min_observations);
    \draw[->] (safe_construction) -- (end);
    \draw[->] (L_safe_construction) -- (end);
    \draw[->] (K_min_observations) -- (end);
\end{tikzpicture}
\caption{问题 1 求解思想流程图}
\label{fig:flowchart}
\end{figure}

\subsection{2.2 问题 2 的分析}

问题 2 是在卫星监视下,部队的机动路线优化问题,从新疆的阿勒泰隐蔽地经喀什运动到和田并在和田执行某任务,24 小时后再隐蔽地返回阿勒泰(不必经喀什)。这是在一个给定的道路交通网中,需要求取从出发地到目的地的最优路径,主要是通过在道路交通网的节点处做出选择与判断。结合卫星的过顶区域,在节点处选择沿那条路机动,

考虑过顶时间决定什么时候开始机动,如果所有可选择的路段在某时间段内都在卫星的固定区域,那么就要在节点处(途经的县级以上(含县级)城市)休息。

问题的目标函数为:一条最优机动路径。

约束条件有两个:

(1)部队以最短的时间到达目的地;

(2)避开卫星过顶侦察。

进一步分析,由于某一路段的机动时间和对应路段长度与行进速度都有关,题目给出部队可以按需要选择在高速公路(最大速度 100 公里/小时)或普速公路(除高速之外的其他公路,最大速度 50 公里)上行进,可见行进速度主要受道路等级的制约,因此可以把道路等级、路段长度都转化为时间参数。那么,问题就转化为求以时间为参数的对应某一出发时刻的最短路径的单目标问题。而约束条件:最短时间和卫星过顶侦察都是跟时间有关系的,那么目标函数是一种权值随时间变化的最短路径问题。

问题 2 求解思想如图 2 所示。

\begin{figure}[h]
\centering
\includegraphics[width=\textwidth]{image.png}
\caption{问题 2 求解思想流程图}
\end{figure}

\subsection{2.3 问题 3 的分析}

问题 3 是在指定区域内,对于移动导弹发射装置规避卫星侦察问题。根据问题 1 和问题 2 的求解,我们已经得到了 Q 型、L 型、K 型卫星的被观测情况和过顶情况,为了方便计算,本题设定区域为问题 2 中的区域,通过对问题 2 得到的该区域三种卫星过顶情况进行分析,得出过顶时间规律,依据时间参数建立模型,求取卫星过顶时的覆盖区域(星下点轨迹),得到区域大小、形状等信息。根据问题 2 建立该区域路网模型,求解移动导弹发射装置的规避路径和条件。

在模型求解过程中,涉及到了三种情况:单星、双星组网、三星组网。分析得到反卫星侦察能够规避的可行性、条件(区域大小、形状、路网状况及其他你们认为需要的)

条件)和方式。

问题3求解思想如图3所示。

\begin{figure}[h]
\centering
\includegraphics[width=0.8\textwidth]{image.png}
\caption{问题2求解思想流程图}
\end{figure}

\section{3 问题假设}

假设1:由于短时间内卫星轨道面在太空中变化很小,一天只随太阳转动0.986度,在实际建模和计算时假设卫星轨道面不变。

假设2:在分析计算过程中,忽略地球扁率、大气摄动等的影响,将地球作为球体,并且卫星轨道无摄动。

假设3:部队机动过程中忽略天气等因素的影响。

\section{4 符号说明}

\begin{table}[h]
\centering
\begin{tabular}{|c|l|}
\hline
\(a, e\) & 卫星椭圆轨道的长半轴、离心率 \\
\hline
\(\theta\) & 卫星与焦点连线的夹角 \\
\hline
\((X, Y, Z)\) & 球心直角坐标 \\
\hline
\((L, B, H)\) & 大地坐标(经度、纬度、高程) \\
\hline
\((x, y, z)\) & 站心地平直角坐标 \\
\hline
\(\alpha\) & 卫星在轨道平面相对于近地点偏转角度 \\
\hline
\(\chi, \Omega, i\) & 近地点幅角、升交点赤径、轨道倾角 \\
\hline
\end{tabular}
\end{table}

\begin{table}
\centering
\begin{tabular}{|c|l|}
\hline
$\gamma$、$\varphi$、$d$ & 观测站的天线仰角、方位角、与卫星距离 \\
\hline
$m$、$n$ & 施工区域长、宽 \\
\hline
$S$ & 以星下点轨迹为中心的卫星左右探测距离 \\
\hline
$D$ & 卫星到观察点的距离 \\
\hline
$V$ & 县级以上(含县级)城市节点 \\
\hline
$A$、$a_{ij}$ & 路段集、连接节点$i$和节点$j$的路段 \\
\hline
$T_{i}$、$t_{ij}$ & 原始通行时间集、$a_{ij}$上的原始通行时间 \\
\hline
$\tau$ & 卫星对所有路段的过顶时间集 \\
\hline
$\left(\tau_{i}, \tau_{j}\right)$ & 卫星对路段$a_{ij}$的过顶时间 \\
\hline
$\phi$、$\delta$ & 为地心纬度、地球上的经度 \\
\hline
$T$、$\omega$ & 卫星运行周期、卫星角速度。 \\
\hline
$t_{0}$、$L$ & 部队机动出发时间、为最优机动路径 \\
\hline
$f(i, j)$ & 节点$i$到节点$j$之间的延迟时间函数 \\
\hline
\end{tabular}
\end{table}

\section{问题1模型的建立和求解}

\subsection{问题1.1模型的建立和求解}

\subsubsection{Q星数据分析}

题目中给出的卫星为,近地轨道对地观测卫星,经过计算可以得到其轨道为近圆轨道,根据附件1中给出的卫星的运行周期为94.37分钟,每天的绕地球的运行圈数为15.26圈。

地球自转一圈的周期为24小时,即1440分钟,而卫星绕地球转动15圈的总时间为1415.55分钟,即卫星转动15圈比地球自转一周的时间少24分37秒,因此,当卫星再次转动15圈时,卫星与观察者最近时刻的时间,理论上应该比前一天的时刻,提前24分37秒。

以D0天的第一次,D0天的第四次和D1天的第二次作为一组数据,以D0天的第三次,D1天的第一次和D2天的第二次作为另一组数据,详细数据见表1,表2,观察到的卫星的时刻为例,进行验证。

\begin{table}
\centering
\caption{第一组分析数据}
\begin{tabular}{|c|l|c|c|}
\hline
次数 & 分析时刻点 & 时间 & 时间差值 \\
\hline
D0-1 & 离观察者最近的时间 & 00:10:36 & 0 \\
\hline
D0-2 & 离观察者最近的时间 & 23:47:17 & 23分19秒 \\
\hline
D1-2 & 离观察者最近的时间 & 23:24:01 & 23分16秒 \\
\hline
D2-3 & 离观察者最近的时间 & 23:00:50 & 23分11秒 \\
\hline
\end{tabular}
\end{table}

\begin{table}
\centering
\caption{第二组分析数据}
\begin{tabular}{|c|l|c|c|}
\hline
次数 & 分析时刻点 & 时间 & 时间差值 \\
\hline
D0-3 & 离观察者最近的时间 & 13:15:51 & 0 \\
\hline
D1-1 & 离观察者最近的时间 & 12:52:38 & 23分13秒 \\
\hline
D2-2 & 离观察者最近的时间 & 12:29:21 & 23分17秒 \\
\hline
\end{tabular}
\end{table}

理论 上, 时间应该提前为 24 分 37 秒, 但是, 经过对实际观察到的时间数据的分析, 得到平均时间差为 23 分 15 秒。其理论计算和实际观察误差为 1 分 22 秒。产生误差的原因有以下三个方面:

(1) 在下一个 15 次时, 观察者与卫星之间的最近距离, 已经发生变化, 所以在 24 分 37 秒时刻, 尚未达到最近距离, 卫星在 1 分 22 秒内, 转过的弧度为 5.15 度, 即在 15 次之后, 再次观察到卫星时, 应比前一次晚一段时间;

(2) 卫星轨道面在太空中, 一天随太阳转动 0.986 度, 即在下一次观察到卫星的时候, 卫星的轨道面已经发生改变, 而本题中将其忽略;

(3) 地球实际为一个椭圆, 而实际计算时按照圆来计算, 忽略了卫星在运行中的摄动影响。

综上分析, 可以根据 D2 天观测时刻点来预测 D3 天观测到卫星的时刻:
\[
T_{D_{3}}=T_{D_{2}}-23 \text { 分 } 16 \text { 秒 }
\]
那么, 可以得到 D3、D5 天观测到卫星的时刻, 如表 3 所示。

表 3 D3、D5 天观测到卫星的时刻

\begin{tabular}{|c|c|c|c|c|}
\hline & \multicolumn{1}{|c|}{\begin{tabular}{c} 
观测到卫星次数 \\
观测到卫星时间
\end{tabular}} & \multicolumn{1}{|c|}{D3-1} & \multicolumn{1}{|c|}{D3-2} & \multicolumn{1}{|c|}{D5-1} & \multicolumn{1}{|c|}{D5-2} \\
\hline 最先观察到卫星的时间 & 00:30:08 & 12:01:26 & 12:45:39 & 23:17:31 \\
\hline 离观察者最近的时间 & 00:33:07 & 12:04:54 & 12:52:38 & 23:24:01 \\
\hline 最后观察到的时间 & 00:36:04 & 12:08:11 & 12:56:09 & 23:27:07 \\
\hline
\end{tabular}

\section*{5.1.2 问题 1.1 模型的建立}

根据对附件 2-1 中 Q 型卫星 D0、D1、D2 三天观测数据的分析, 得到了 D3 天观测站对卫星的观测时刻点, 知道了时间参量, 我们可以建立关于时间参量的方程组对卫星位置进行求解 [1]。

根据开普勒第一定律, 人造卫星轨道是绕地球运行的椭圆轨道, 地球中心位于椭圆的一个焦点上, 建立卫星椭圆轨道的极坐标方程。
\[
r=a\left(1-e^{2}\right) /(1-e \cos \theta)
\]
其中, $a$ 为长半轴, $e$ 为离心率, $\theta$ 为卫星与焦点连线的夹角, 为时间 $t$ 和角速度的函数。

根据开普勒第二定律, 卫星与地心向径在单位时间内扫过的面积相等, 建立同等时间向径扫过面积方程:
\[
\int_{t_{1}}^{t_{2}} r(t) d t=\int_{t_{3}}^{t_{4}} r(t) d t
\]
由于下一观测时刻已知, 根据开普勒方程即可求出该时刻对应的向径, 再者卫星椭圆轨道极坐标系的极点为地心, 与球心直角坐标系原点重合, 可以通过向径和轨道参数求出卫星在球心直角坐标系中的位置坐标 $(X, Y, Z)$ :
\[
\left[\begin{array}{l}
X \\
Y \\
Z
\end{array}\right]=r \cdot\left[\begin{array}{c}
\cos \alpha \cos \chi \\
\sin \alpha \cos (90-(360-\Omega)) \\
\cos \alpha \sin \chi
\end{array}\right]
\]
其中, $\alpha$ 为卫星在轨道平面相对于近地点偏转角度, $\chi$ 为近地点幅角, $\Omega$ 为升交点赤径。

\begin{figure}[h]
    \centering
    \includegraphics[width=0.8\textwidth]{image1.png}
    \caption{球心直角坐标系}
    \label{fig:1}
\end{figure}

\begin{figure}[h]
    \centering
    \includegraphics[width=0.8\textwidth]{image2.png}
    \caption{卫星轨道极坐标系}
    \label{fig:2}
\end{figure}

将球心直角坐标系转换到站心地平直角坐标系即可求出卫星在下一观测时刻相对于观测站的位置。但是,在卫星球心直角坐标到站心地平直角坐标的转换计算过程中涉及到了观测站大地坐标向球心直角坐标的转换计算。下面是转换过程\cite{ref1}:

(1) 观测站大地坐标向球心直角坐标的转换

以下是观测站大地坐标 $(L_0, B_0, H_0)$ 向球心直角坐标 $(X_0, Y_0, Z_0)$ 的转换计算过程:

\begin{figure}[h]
    \centering
    \includegraphics[width=0.8\textwidth]{image3.png}
    \caption{大地坐标系}
    \label{fig:3}
\end{figure}

由于观测站 P 位置在地球表面,所以 $H_0 = 0$,由图 \ref{fig:3} 大地坐标系中的三角关系可以列出方程:

\begin{equation}
\begin{bmatrix}
X_0 \\
Y_0 \\
Z_0
\end{bmatrix}
=
\begin{bmatrix}
x \cos L_0 \\
x \sin L_0 \\
y
\end{bmatrix}
\end{equation}

由图 7 可知,过观测点 P 子午线的切线 TP,与 \(x\) 轴夹角为 \(90^\circ + B\),那么:
\[
\frac{dy}{dx} = -\cot B
\]

\begin{figure}[h]
    \centering
    \includegraphics[width=0.8\textwidth]{image.png}
    \caption{观测点的子午线平面}
    \label{fig:7}
\end{figure}

\(x, y\) 需要计算,将子午圈椭圆方程和第一偏心率计算公式代入上式,同时为了化简形式,将卯酉圈方程代入即可求得 \(x, y\) 简化公式。

子午圈方程:
\[
\frac{x^2}{a^2} + \frac{y^2}{b^2} = 1
\]

第一偏心率公式:
\[
e = \frac{\sqrt{a^2 - b^2}}{a}
\]

卯酉圈方程:
\[
N = \frac{a}{\sqrt{1 - e^2 \sin^2 B}}
\]

得到:
\[
x = N \cos B
\]
\[
y = N \big(1 - e^2\big) \sin B
\]

那么,大地坐标到球心直角坐标的转换计算式为:
\[
\begin{bmatrix}
X_0 \\
Y_0 \\
Z_0
\end{bmatrix}
=
\begin{bmatrix}
(N + H_0) \cos B_0 \cos L_0 \\
(N + H_0) \cos B_0 \sin L_0 \\
\big[N \big(1 - e^2\big) + H_0\big] \sin B_0
\end{bmatrix}
\]

(2)卫星球心直角坐标到站心地平直角坐标的转换

\begin{figure}[h]
    \centering
    \includegraphics[width=0.8\textwidth]{image.png}
    \caption{站心地平直角坐标系}
    \label{fig:8}
\end{figure}

从球心直角坐标 $(X, Y, Z)$ 转换到站心地平直角坐标 $(x, y, z)$,可以通过将 $(X, Y, Z)$ 旋转两次实现,先绕 $Z$ 轴旋转 $\left(90^\circ + L\right)$,再绕 $X$ 轴旋转 $\left(90^\circ - B\right)$,其中 $(L, B, H)$ 为卫星的大地坐标。旋转转换方程为:

\begin{equation}
\begin{bmatrix}
x \\
y \\
z
\end{bmatrix}
= R_x\left(90^\circ - B\right) R_z\left(90^\circ + L\right)
\left(
\begin{bmatrix}
X \\
Y \\
Z
\end{bmatrix}
-
\begin{bmatrix}
X_0 \\
Y_0 \\
Z_0
\end{bmatrix}
\right)
\end{equation}

在上式中,$(X_0, Y_0, Z_0)$ 是站心地平直角坐标系原点(即观测站)对应的球心直角坐标系中的坐标,由题可知观测站的大地坐标 $(L_0, B_0, H_0)$。

在上式中,

\begin{equation}
R_x\left(90^\circ - B\right) =
\begin{bmatrix}
1 & 0 & 0 \\
0 & \cos\left(90^\circ - B\right) & \sin\left(90^\circ - B\right) \\
0 & -\sin\left(90^\circ - B\right) & \cos\left(90^\circ - B\right)
\end{bmatrix}
\end{equation}

\begin{equation}
R_z\left(90^\circ + L\right) =
\begin{bmatrix}
\cos\left(90^\circ + L\right) & \sin\left(90^\circ + L\right) & 0 \\
-\sin\left(90^\circ + L\right) & \cos\left(90^\circ + L\right) & 0 \\
0 & 0 & 1
\end{bmatrix}
\end{equation}

那么,球心直角坐标到站心地平直角坐标的转换计算式为:

\begin{equation}
\begin{bmatrix}
x \\
y \\
z
\end{bmatrix}
=
\begin{bmatrix}
-\sin L & \cos L & H \\
-\sin B \cos L & -\sin B \sin L & \cos B \\
\cos B \cos L & \cos B \sin L & \sin B
\end{bmatrix}
\cdot
\left(
\begin{bmatrix}
X \\
Y \\
Z
\end{bmatrix}
-
\begin{bmatrix}
X_0 \\
Y_0 \\
Z_0
\end{bmatrix}
\right)
\end{equation}

由此即可得到卫星在下一观测时刻相对于观测站的位置坐标 $(x, y, z)$。

(3) 观测结果求取

在站心地平直角坐标系中,观测站为原点 \((0,0,0)\),卫星坐标通过上面的计算已经得到,为 \((x, y, z)\)。据此,即可算出观测站的天线仰角 \(\gamma\)、方位角 \(\varphi\)、与卫星距离 \(d\) 等参数。
\begin{align*}
Elv: \quad & \gamma = \arctan \frac{z}{\sqrt{x^2 + y^2}} \\
Amz: \quad & \varphi = \arccos \frac{y}{\sqrt{x^2 + y^2}} \\
Range: \quad & d = \sqrt{x^2 + y^2 + z^2}
\end{align*}

由于是连续观测,太阳方位角和仰角可根据前面观测的数据进行分析得到。

(4)过顶情况

对区域目标,卫星星下点进、出给定区域的时间区间称为卫星的过顶时间。其中,卫星与地球中心连线在地球表面的交点称为星下点。卫星运动和地球自转使星下点在地面移动所画出的轨迹称为星下点轨迹。

已知条件为观测站观测范围、目标区域范围和观测站位置,那么计算卫星过顶情况需要结合这几个方面的因素,从观测数据计算出过顶数据。

\begin{figure}[h]
\centering
\includegraphics[width=0.8\textwidth]{image.png}
\caption{施工区域示意图}
\end{figure}

根据题目给定的经纬度可以求出施工区域范围为 \(m \times n \, (m \geq n)\),如图 9 所示,卫星的探测距离以星下点轨迹为中心左右各为 \(s\),那么卫星可探测到的星下点的范围是 \(0 \leq S \leq m + s\),所以,卫星到观察点的距离为:
\[
D = \sqrt{S^2 + H^2}
\]

只有卫星到观察点的距离满足 \(D < D_{max}\) 时,卫星才能过顶,\(D_{max} = \sqrt{S_{\text{max}}^2 + H^2}\),H 表示卫星的高程。因此,可以根据卫星距离观察点最近时的距离,对卫星过顶情况进行搜索,卫星距离观察点最近时,\(D < D_{max} - m\),肯定会出现过顶情况。

\subsection*{5.1.3 问题 1.1 模型的求解}

(1)预测卫星 D3、D5 日观测情况

\textbf{Step1:已观测到卫星位置大地坐标求取}

经纬度换算规则:东经取正值,西经取负值;北纬取 90 度减去纬度值,南纬取 90 度加上纬度值。所以观察者的经纬度坐标可以表示为 \((118.7681, 57.9791)\),卫星的经纬度可以表示为 \((L, 90 - B)\)。

假设,两点的经纬度信息为(L1,B1),(L2,B2),根据三角函数推导,可以得到以上两个经纬度之间的距离的计算公式为:
\begin{align*}
Distance &= R \cdot arccos(C) \cdot \frac{\pi}{180} \\
C &= \sin(B1) \cdot \sin(B2) \cdot \cos(L1 - L2) + \cos(L1) \cdot \cos(L2)
\end{align*}
式中,R 表示为地球半径,取值为 $R = 6371.004 \, km$。上式中 Distance 表示卫星在地面的投影点到观察点之间的距离,由天线的方位角和距离可以得到 $Distance = 403.83 \, km$,所以,上式(1)和式(2)中只有 L2 和 B2 两个未知数。又因为,地球的子午线总长度约为 40008Km,所以平均纬度改变 1 度,两点之间距离改变为 111Km,纬度改变 1 分,距离改变 1.85km,纬度改变 1 秒,距离改变 30.9m。所以根据,两点之间的距离差值,以及天线的方位角 259.9 度,可以计算得到两点之间的纬度差为 $D \cdot \cos(270 - 259.9) = 397.57 \, km$,经计算,两点之间的纬度差为 3.03 度,所以卫星的纬度值为 35.0509 度,由此可以计算得到卫星的经度值 L = 116.3371 度。所以,此次观察时,卫星的坐标为(116.3371, 35.0509, 479.60)。

由此,可以计算得到附件 2-1 中,所有卫星与观察者之间距离最近时的卫星的坐标为见表 4

\begin{table}[h]
\centering
\caption{卫星距离观察点最近的 9 次的空间坐标}
\begin{tabular}{c|c|c|c|c}
\hline
天数 & 观察次数 & 卫星经度 & 卫星纬度 & 卫星高程(km) \\
\hline
\multirow{4}{*}{D0} & 1 & 116.3371 & 28.4227 & 479.60 \\
\cline{2-5}
& 2 & 116.8581 & 28.6393 & 379.79 \\
\cline{2-5}
& 3 & 123.4079 & 35.7255 & 406.78 \\
\cline{2-5}
& 4 & 119.0646 & 36.4599 & 488.3 \\
\hline
\multirow{2}{*}{D1} & 1 & 119.9141 & 36.3272 & 473.56 \\
\cline{2-5}
& 2 & 120.1716 & 36.0999 & 448.34 \\
\hline
\multirow{3}{*}{D2} & 1 & 115.9609 & 29.2916 & 307.39 \\
\cline{2-5}
& 2 & 117.6221 & 27.6277 & 492.08 \\
\cline{2-5}
& 3 & 125.3824 & 35.3366 & 363.61 \\
\hline
\end{tabular}
\end{table}

\textbf{Step2:D3、D5 日卫星被观测情况求解}

根据建立的预测计算模型,利用 VC++ 在 MFC 环境下编译卫星被观测情况预测界面如图 10。输入参数为:地球参数 $a = 6378.137 \, km$,$b = 6356.752 \, km$;观测站大地坐标(L0 = 118.7681,B0 = 32.0209,H0 = 0),已测点(D0、D1、D2 日)的大地坐标 $(L, B, H)$ 在 Step1 中已经求得,输入已知观测时刻和对应的天线仰角,以及数据分析得到的预测时刻,便可求出卫星观测到的观测时刻对应的地平直角坐标和天线仰角、方位角等参数。表 5 为 D3 天观察到的卫星数据,表 6 为 D5 天观察到的卫星数据。

\begin{figure}[h]
\centering
\includegraphics[width=0.8\textwidth]{satellite_prediction_interface.png}
\caption{卫星被观测情况预测界面}
\end{figure}

\begin{table}
\centering
\caption{D3天两次观察到的卫星数据}
\begin{tabular}{c c c c c c c c}
\hline
D3 & Time & Azm & Elv & Mag & Range & S.Azm & S.Elv \\
\hline
\multirow{3}{*}{1} & 00:30:08 & 205.6 & 10.1 & ecl & 1676 & 6.8 & -38.6 \\
 & 00:33:07 & 263.5 & 28.7 & ecl & 947 & 7.1 & -38.7 \\
 & 00:36:04 & 312.1 & 10 & ecl & 1680 & 7.3 & -38.6 \\
\multirow{3}{*}{2} & 12:01:26 & 23.6 & 10.1 & ecl & 1684 & 172.5 & 75.8 \\
 & 12:04:54 & 100.1 & 56.7 & ecl & 616 & 174.7 & 75.5 \\
 & 12:08:11 & 177.5 & 10.1 & ecl & 1686 & 175.9 & 75.3 \\
\hline
\end{tabular}
\end{table}

\begin{table}
\centering
\caption{D5天两次观察到的卫星数据}
\begin{tabular}{c c c c c c c c}
\hline
D5 & Time & Azm & Elv & Mag & Range & S.Azm & S.Elv \\
\hline
\multirow{3}{*}{1} & 12:45:39 & 34.5 & 10 & ecl & 1687 & 212.1 & 73.8 \\
 & 12:52:38 & 314.9 & 47.9 & ecl & 670 & 214.7 & 74.3 \\
 & 12:56:09 & 235.7 & 10.1 & ecl & 1674 & 217.3 & 73.9 \\
\multirow{3}{*}{2} & 23:17:31 & 131.2 & 10.1 & ecl & 1670 & 342.7 & -37.7 \\
 & 23:24:01 & 73.5 & 27.8 & ecl & 968 & 343.6 & -37.8 \\
 & 23:27:07 & 16.1 & 10 & ecl & 1683 & 344.67 & -38 \\
\hline
\end{tabular}
\end{table}

(2) 预测卫星D3、D5日过顶情况

根据题目给定的施工区域信息(北纬31.90~32.25度;东经118.02~118.91度),可以计算出施工区域的长度约为84.8km,宽度约为38.85km。即
\[ m=84.8km \quad n=38.85km \]

卫星过顶情况粗略搜索:

依据施工区域的,最长边为84.8km,卫星的探测距离是中心线左右各272km,可探测到的星下点的范围是 \(0 \leq S \leq 356.8km\)。所以,只有 \(D < D_{max}\) 时,其中,D表示卫星到观察点的距离,\(D_{max} = \sqrt{S_{max}^2 + H^2}\),H表示卫星的高程,只有满足以上条件时,卫星才有可能通过施工区域。因此,可以根据卫星距离观察点最近时的距离,对卫星过顶情况进行粗略搜索。

根据附件1给出的数据,可以求得 \(D_{max} = 605.3km\),即只有在 \(D < 605.3km\) 时,才有可能出现过顶情况。当 \(D < 559.6km\) 时,肯定会出现过顶情况,卫星过顶情况如下表所示,表7为D0-D2三天的过顶数据,表8为D3、D5两天的预测过顶情况。

\begin{table}
\centering
\caption{D0~D2天卫星过顶情况}
\begin{tabular}{c c c c c c c c}
\hline
过顶次数 & 观测时刻点 & Time & Azm & Elv & Mag & Range & S.Azm & S.Elv \\
\hline
\multirow{3}{*}{D0天第三次} & 最先观察时间 & 23:43:43 & 160.5 & 10.1 & ecl & 1666 & 351.6 & -38.8 \\
 & 离观察者最近 & 23:47:17 & 76.4 & 69.3 & ecl & 522 & 352.7 & -38.9 \\
 & 最后观察到时间 & 23:50:52 & 353.9 & 10.1 & ecl & 1679 & 353.7 & -39 \\
\multirow{3}{*}{D2天第二次} & 最先观察时间 & 12:25:43 & 13.1 & 10.1 & 8.5 & 1684 & 194.2 & 76 \\
 & 离观察者最近 & 12:29:21 & 104.6 & 82.8 & 7.3 & 496 & 197.6 & 75.8 \\
 & 最后观察到时间 & 12:32:57 & 189.9 & 10 & 13.4 & 1675 & 200.9 & 75.5 \\
\hline
\end{tabular}
\end{table}

\begin{table}
\centering
\caption{D3和D5天预测卫星过顶情况}
\begin{tabular}{c c c c c c c c}
\hline
过顶 & 观测时刻点 & Time & Azm & Elv & Mag & Range & S.Azm & S.El \\
次数 & & & & & & & & v \\
\hline
\multirow{3}{*}{D3天第二次} & 最先观察时间 & 12:01:26 & 23.6 & 10.1 & ecl & 1684 & 172.5 & 75.8 \\
 & 离观察者最近 & 12:04:54 & 100.1 & 56.7 & ecl & 567 & 174.7 & 75.5 \\
 & 最后观察到时间 & 12:08:11 & 177.5 & 10.1 & ecl & 1686 & 175.9 & 75.3 \\
\hline
\end{tabular}
\end{table}

(3) D3、D5两天安全施工时段

为了安全施工的考虑,在D3天和D5天,虽然只有在D3天的第二次观察到卫星的时刻,才会出现过顶情况,但是为了有效“避开”卫星过顶的时间段,充分“改变”施工区域的活动规律,将军事目标伪装成非军事目标,减弱卫星侦察的效果,成功“骗过”卫星侦察,建议在D3天的12:01:26到12:08:11时间段内,不要施工。所以,在D3和D5天内,安全的施工时间段应该将这一时间段除去。

\subsection{5.2 问题1.2模型的建立和求解}

\subsubsection{5.2.1 L星数据分析}

根据附件1,给出的数据,L-1星的周期为97.6分钟,每天绕地球的圈数为14.75圈。卫星运行59圈,5758.4分钟,地球自转4圈,的时间为5760分钟,两者时间差为1.6分钟,即1分钟36秒。所以在8月16号和8月20号,对应观察到卫星时刻的差值应为1分钟36秒,实际观测到的时间差值为在8月16号和8月20号,8月17号和8月21号,四次卫星距离观察者最近的时间的差值分别为7分8秒和6分40秒,详细数据见表9,造成这种情况的主要原因是卫星轨道为一个椭圆轨道,在远地点弧段运行时的时间要比近地点弧段的运行时间多28秒。

\begin{table}
\centering
\caption{四次卫星距离观察者最近的时间差值}
\begin{tabular}{c c c c}
\hline
组数 & 日期 & 距离最近时刻 & 时间差值 \\
\hline
\multirow{2}{*}{1} & 2016-8-16 & 05:22:03 & \multirow{2}{*}{7分8秒} \\
 & 2016-8-20 & 05:14:55 & \\
\hline
\multirow{2}{*}{2} & 2016-8-16 & 19:30:01 & \multirow{2}{*}{6分40秒} \\
 & 2016-8-20 & 19:23:21 & \\
\hline
\multirow{2}{*}{3} & 2016-8-17 & 04:05:10 & \multirow{2}{*}{6分42秒} \\
 & 2016-8-21 & 03:58:28 & \\
\hline
\multirow{2}{*}{4} & 2016-8-17 & 18:13:34 & \multirow{2}{*}{7分8秒} \\
 & 2016-8-21 & 18:06:26 & \\
\hline
\end{tabular}
\end{table}

预测依据:

(1) 前4天的数据,理论误差为1分36秒,实际误差为7分8秒和6分40秒;

(2) 前一天的观察到的卫星情况,理论误差为24分钟,实际误差为23分15秒。其中理论误差由卫星一天转15圈,地球一天为24小时,所以两个时间相减,就能够得到后一天和前一天的观察时刻的理论误差,实际误差由观察数据得到。

理论误差产生的原因由以下四个方面:(1)在下一个15次时,观察者与卫星之间的最近距离,已经发生变化,所以在24分钟时刻,尚未达到最近距离;(2)卫星轨道面在太空中,一天随太阳转动0.986度,即在下一次观察到卫星的时候,卫星的轨道面已经发生改变,而本题中将其忽略;(3)地球实际为一个椭圆,而实际计算时按照圆来计算,忽略了卫星在运行中的摄动影响;(4)卫星的轨道倾角为69.987度,在运行过程中会对运行时间造成一定的误差。

\begin{table}
\centering
\caption{8月17日卫星最新被观察情况}
\begin{tabular}{|c|c|c|}
\hline
日期 & 距离最近时刻 & 时间差值 \\
\hline
\multirow{2}{*}{2016-8-17} & 04:05:10 & \multirow{2}{*}{14:8:24} \\
\cline{2-2}
 & 18:13:34 & \\
\hline
\multirow{2}{*}{2016-8-19} & 03:11:28 & \multirow{2}{*}{14:8:54} \\
\cline{2-2}
 & 17:20:22 & \\
\hline
\end{tabular}
\end{table}

\begin{table}
\centering
\caption{验证数据}
\begin{tabular}{|c|c|c|}
\hline
日期 & 时刻 & 天线仰角 \\
\hline
2016-8-18 & 04:24:13 & 10 \\
\hline
2016-8-18 & 04:28:37 & 43.3 \\
\hline
2016-8-18 & 04:32:59 & 10.1 \\
\hline
2016-8-18 & 06:05:45 & 10 \\
\hline
2016-8-18 & 06:08:07 & 14 \\
\hline
2016-8-18 & 06:10:29 & 10 \\
\hline
\end{tabular}
\end{table}

\begin{table}
\centering
\caption{8月22日观测到L-1型卫星的时刻}
\begin{tabular}{|c|c|c|c|c|}
\hline
观测到L-1星次数 & 1 & 2 & 3 & 4 \\
\hline
8月22日时间 & & & & \\
\hline
最先观察到卫星的时间 & 02:38:44 & 04:17:26 & 16:46:30 & 18:24:53 \\
\hline
离观察者最近的时间 & 02:41:57 & 04:21:58 & 16:50:21 & 18:29:28 \\
\hline
最后观察到的时间 & 02:45:08 & 04:26:30 & 16:54:11 & 18:34:03 \\
\hline
\end{tabular}
\end{table}

\begin{table}
\centering
\caption{8月23日观测到L-1型卫星的时刻}
\begin{tabular}{|c|c|c|c|c|}
\hline
观测到L-1星次数 & 1 & 2 & 3 & 4 \\
\hline
8月23日时间 & & & & \\
\hline
最先观察到卫星的时间 & 03:01:59 & 04:17:26 & 16:46:30 & 18:24:53 \\
\hline
离观察者最近的时间 & 03:05:12 & 04:21:58 & 16:50:21 & 18:29:28 \\
\hline
最后观察到的时间 & 03:08:23 & 04:26:30 & 16:54:11 & 18:34:03 \\
\hline
\end{tabular}
\end{table}

\begin{table}
\centering
\caption{8月23日观测到L-2型卫星的时刻}
\begin{tabular}{|c|c|c|c|c|}
\hline
观测到L-2星次数 & 1 & 2 & 3 & 4 \\
\hline
8月23日时间 & & & & \\
\hline
最先观察到卫星的时间 & 04:11:08 & 12:47:06 & 02:33:20 & 11:08:56 \\
\hline
离观察者最近的时间 & 04:16:27 & 12:52:34 & 02:35:26 & 11:11:34 \\
\hline
最后观察到的时间 & 04:21:34 & 12:58:04 & 02:37:56 & 11:13:52 \\
\hline
\end{tabular}
\end{table}

\subsection*{5.2.2 问题1.2模型的建立与求解}

根据对附件2-2中L型卫星2016年8月16日-21日期间观测数据的分析,得到了8月23日观测站对卫星的观测时刻点,知道了时间参量,我们可以建立关于时间参量的方程组对卫星位置进行求解。

建立和问题1.1同样的模型,对问题进行求解:

(1) 预测卫星8月22日和23日被观测情况

由以上分析可得,L-1型卫星和L-2型卫星2016年8月16日-23日期间被观测到的次数,见表15。

\begin{table}
\centering
\caption{L-1型卫星和L-2型卫星8月16日-23日期间被观测到的次数}
\begin{tabular}{|c|c|c|c|c|c|c|c|c|}
\hline
日期 & 16号 & 17号 & 18号 & 19号 & 20号 & 21号 & 22号 & 23号 \\
\hline
L-1出现次数 & 2 & 4 & 3 & 4 & 4 & 2 & 4 & 4 \\
\hline
L-2出现次数 & 4 & 4 & 4 & 4 & 4 & 4 & 4 & 4 \\
\hline
\end{tabular}
\end{table}

8月22日,可以观察到的L-1型卫星的次数为4次,观察到L-1型卫星的情况的详细数据,见表16。

\begin{table}
\centering
\caption{8月22日L-1型卫星被观测情况}
\begin{tabular}{|c|c|c|c|c|c|c|c|}
\hline
22日L-1星观测次数 & Time & Azm & Elv & Mag & Range & S.Azm & S.El \\
\hline
\multirow{3}{*}{第一次} & 02:38:44 & 19.4 & 10 & 4.9 & 2040 & 42.4 & -37.6 \\
\cline{2-8}
 & 02:41:57 & 47.3 & 17.7 & 4.6 & 1660 & 42.8 & -37.2 \\
\cline{2-8}
 & 02:45:08 & 75.2 & 10 & 4.9 & 2037 & 43.1 & -36.8 \\
\hline
\multirow{3}{*}{第二次} & 04:17:26 & 305.5 & 10.1 & 4.9 & 2037 & 64.6 & -14.7 \\
\cline{2-8}
 & 04:21:58 & 70.8 & 76.2 & 2.4 & 675 & 65.9 & -15.6 \\
\cline{2-8}
 & 04:26:30 & 196.1 & 10 & 4.9 & 2033 & 67.2 & -16.5 \\
\hline
\multirow{3}{*}{第三次} & 16:46:30 & 150.6 & 10 & 5 & 2041 & 277.9 & 12.2 \\
\cline{2-8}
 & 16:50:21 & 105.2 & 20.5 & 4.7 & 1621 & 278.2 & 11.5 \\
\cline{2-8}
 & 16:54:11 & 59.8 & 10.1 & 5.1 & 2037 & 278.5 & 10.8 \\
\hline
\multirow{3}{*}{第四次} & 18:24:53 & 166.5 & 10 & 5.3 & 2023 & 281.8 & 2.6 \\
\cline{2-8}
 & 18:29:28 & 284.2 & 61.2 & 3.2 & 746 & 282.5 & 1.6 \\
\cline{2-8}
 & 18:34:03 & 41.9 & 10 & 5.2 & 2033 & 283.2 & 0.7 \\
\hline
\end{tabular}
\end{table}

预测出8月22号,能够观察到卫星的情况以后,结合前4天,即8月19号和前一天,即8月22号观察数据,可以预测得到L-1型卫星和L-2型卫星8月23号的观测数据,见表17,表18。

\begin{table}
\centering
\caption{8月23日L-1型卫星被观测情况}
\begin{tabular}{c|c|c|c|c|c|c|c}
\hline
23日L-1星 & Time & Azm & Elv & Mag & Range & S.Azm & S.Elz \\
观测次数 & & & & & & & \\
\hline
\multirow{3}{*}{第一次} & 03:01:59 & 343.3 & 10 & 5 & 2038 & 64.1 & -13.4 \\
 & 03:05:12 & 51.7 & 41.6 & 3.2 & 912 & 64.7 & -14.1 \\
 & 03:08:23 & 119.9 & 10.1 & 4.9 & 2037 & 65.2 & -14.8 \\
\hline
\multirow{3}{*}{第二次} & 04:40:41 & 331.2 & 10.1 & 4.9 & 2037 & 68.2 & -9.8 \\
 & 04:45:13 & 247.9 & 79.1 & 2.6 & 680 & 68.8 & -8.9 \\
 & 04:49:45 & 164.9 & 10 & 4.9 & 2033 & 69.5 & -8 \\
\hline
\multirow{3}{*}{第三次} & 17:09:45 & 180.1 & 10 & 5 & 2038 & 279.6 & 4.5 \\
 & 17:13:36 & 109.1 & 45.8 & 3 & 850 & 280.1 & 3.7 \\
 & 17:17:26 & 38.1 & 10.1 & 4.9 & 2037 & 280.6 & 2.9 \\
\hline
\multirow{3}{*}{第四次} & 18:48:08 & 156.4 & 10 & 5.3 & 2023 & 286.1 & 2.9 \\
 & 18:52:43 & 274.4 & 85.3 & 2.0 & 670 & 287.5 & 1.7 \\
 & 18:57:18 & 31.8 & 10 & 5.2 & 2033 & 288.4 & 0.8 \\
\hline
\end{tabular}
\end{table}

\begin{table}
\centering
\caption{8月23日L-2型卫星被观测情况}
\begin{tabular}{c|c|c|c|c|c|c|c}
\hline
23日L-2星 & Time & Azm & Elv & Mag & Range & S.Azm & S.Elz \\
观测次数 & & & & & & & \\
\hline
\multirow{3}{*}{第一次} & 04:11:08 & 344.5 & 10 & 4.8 & 2054 & 62.8 & -13.8 \\
 & 04:16:27 & 134.5 & 77.1 & 3.7 & 690 & 63.6 & -14.9 \\
 & 04:21:34 & 284.5 & 10 & 4.8 & 2046 & 64.3 & -16.1 \\
\hline
\multirow{3}{*}{第二次} & 12:47:06 & 353.5 & 10 & 6.9 & 2065 & 208 & 67.2 \\
 & 12:52:34 & 93.6 & 82 & 3.5 & 672 & 211.2 & 66.4 \\
 & 12:58:04 & 195.8 & 10 & 8.4 & 2064 & 216.5 & 65.6 \\
\hline
\multirow{3}{*}{第三次} & 02:33:20 & 130.7 & 10 & 4.8 & 2054 & 39.4 & -38.6 \\
 & 02:35:26 & 115.2 & 11.3 & 4.6 & 1930 & 40 & -38 \\
 & 02:37:56 & 95.6 & 10 & 4.8 & 2066 & 40.6 & -37.4 \\
\hline
\multirow{3}{*}{第四次} & 11:08:56 & 2.7 & 10 & 7.2 & 2054 & 128.5 & 65.9 \\
 & 11:11:34 & 36.8 & 13.5 & 7.1 & 1809 & 130.2 & 66.2 \\
 & 11:13:52 & 70.8 & 10 & 7.8 & 2061 & 133.1 & 66.5 \\
\hline
\end{tabular}
\end{table}

(2) 预测卫星8月23日过顶情况

由于题目没有给出,L型卫星的辐照宽度没有给出,在模型求解时参照Q型卫星,进行计算,辐照宽度为星下点轨迹左右各272km。依据施工区域的,最长边为84.8km,卫星的探测距离是中心线左右各272km,可探测到的星下点的范围是$0\leq S\leq 356.8km$。

所以,只有$D<D_{max}$时,其中,D表示卫星到观察点的距离,$D_{max}=\sqrt{S_{max}^2+H^2}$,H表示卫星的高程,只有满足以上条件时,卫星才有可能通过施工区域。因此,可以根据卫星距离观察点最近时的距离,对卫星过顶情况进行粗略搜索。

根据附件1给出的数据,可以求得$D_{max}=700.5km$,即只有在$D<700.5km$时,才有可能出现过顶情况。当$D<753.4km$时,肯定会出现过顶情况,L-1型卫星过顶情况见表19,L-2型卫星过顶情况见表20。

\begin{table}
\centering
\caption{8月23日L-1型卫星过顶情况}
\begin{tabular}{c c c c c c c}
\hline
8月23日 & Time & Azm & Elv & Mag & Range & S.Azm & S.El \\
过顶次数 & & & & & & & v \\
\hline
\multirow{3}{*}{第一次} & 04:40:41 & 331.2 & 10.1 & 4.9 & 2037 & 68.2 & -9.8 \\
 & 04:45:13 & 247.9 & 79.1 & 2.6 & 680 & 68.8 & -8.9 \\
 & 04:49:45 & 164.9 & 10 & 4.9 & 2033 & 69.5 & -8 \\
\hline
\multirow{3}{*}{第二次} & 18:48:08 & 156.4 & 10 & 5.3 & 2023 & 286.1 & 2.9 \\
 & 18:52:43 & 274.4 & 85.3 & 2.0 & 670 & 287.5 & 1.7 \\
 & 18:57:18 & 31.8 & 10 & 5.2 & 2033 & 288.4 & 0.8 \\
\hline
\end{tabular}
\end{table}

\begin{table}
\centering
\caption{8月23日L-2型卫星过顶情况}
\begin{tabular}{c c c c c c c}
\hline
8月23日 & Time & Azm & Elv & Mag & Range & S.Azm & S.El \\
过顶次数 & & & & & & & v \\
\hline
\multirow{3}{*}{第一次} & 04:11:08 & 344.5 & 10 & 4.8 & 2054 & 62.8 & -13.8 \\
 & 04:16:27 & 134.5 & 77.1 & 3.7 & 690 & 63.6 & -14.9 \\
 & 04:21:34 & 284.5 & 10 & 4.8 & 2046 & 64.3 & -16.1 \\
\hline
\multirow{3}{*}{第一次} & 12:47:06 & 353.5 & 10 & 6.9 & 2065 & 208 & 67.2 \\
 & 12:52:34 & 93.6 & 82 & 3.5 & 672 & 211.2 & 66.4 \\
 & 12:58:04 & 195.8 & 10 & 8.4 & 2064 & 216.5 & 65.6 \\
\hline
\end{tabular}
\end{table}

(3) 预测卫星8月23日安全施工时段

为了安全施工的考虑,在8月23号,虽然只有在观察者距离卫星最近的一段时间,才会出现过顶情况,但是为了有效“避开”卫星过顶的时间段,充分“改变”施工区域的活动规律,将军事目标伪装成非军事目标,减弱卫星侦察的效果,成功“骗过”,卫星侦察,建议在8月23号的4:11:08和4:21:34,4:40:41到4:49:45,12:47:06和12:58:04,18:48:08和18:57:18,4个时间段内,不要施工。即在8月23号,安全的施工时间段应该将这一时间段除去。

\subsection*{5.3 问题1.3模型的建立与求解}

\subsubsection{5.3.1 十次观测数据分析}

观察附件2-3数据可知,后一天与前一天出现在观察点最近距离的时间差值为14分钟,在观察时,以3次在7时左右的观察数据为例,进行说明,详细数据见表21。

\begin{table}
\centering
\caption{3次在7时左右的被观测数据}
\begin{tabular}{c c c c c c c}
\hline
次数 & Time & Azm & Elv & Mag & Range & S.Azm & S.El \\
 & & & & & & & v \\
\hline
\multirow{3}{*}{1} & 07:08:31 & 130.7 & 10 & 11.2 & 1969 & 80.6 & 21.7 \\
 & 07:12:06 & 74 & 26.2 & 10.1 & 1205 & 81 & 22.5 \\
 & 07:15:42 & 17.2 & 10 & 11.1 & 1978 & 81.5 & 23.2 \\
\hline
\multirow{3}{*}{2} & 07:22:06 & 145 & 10.1 & 11.2 & 1965 & 82.5 & 24.5 \\
 & 07:26:06 & 75.9 & 39.5 & 9.5 & 913 & 82.9 & 25.3 \\
 & 07:30:07 & 7.1 & 10 & 11.1 & 1976 & 83.4 & 26.2 \\
\hline
\multirow{3}{*}{3} & 07:35:54 & 158.3 & 10.1 & 11.1 & 1963 & 84.4 & 27.3 \\
 & 07:40:07 & 78.5 & 62 & 8.9 & 689 & 84.9 & 28.2 \\
 & 07:44:22 & 357.9 & 10.1 & 11.1 & 1974 & 85.4 & 29.1 \\
\hline
\end{tabular}
\end{table}

由表20中数据可得,第2次与第1次,卫星距离观察者最近时的时刻分别为7:26:06和7:12:06,其时间差值为14分钟,第3次与第2次,卫星距离观察者最近时的时刻分别为7:44:22和7:26:06,其时间差值为14分01秒,所以可以根据所有数据进行求均值。

相邻两次观测到的最短时间,即近似等于卫星绕行一周的时间,由十次连续观测数据得到周期为96分钟,结合卫星转动一定次数后与地球自转一圈的时间差为14分,可得每日卫星绕行圈数:
\[
N = \frac{24 \times 60}{96} \approx 15 \text{(圈)}
\]
考虑卫星转动与地球自转的时间差,计算卫星运行的较为精确周期:
\[
\frac{24 \times 60 + 14}{15} = 96.93 \, \text{(min)}
\]
又因为,在上表数据中,天线的仰角的值,是一个上升趋势,说明此时卫星并没有离开观察区域,所以,经过计算可以得出,下一个在7时时刻,出现卫星与观察点最近距离的时刻应该为7:56:07。

\subsection*{5.3.2 问题 1.3 模型的建立与求解}

根据对附件2-3中连续十次观测数据的分析,得到了卫星运行的精确周期,观测站对卫星的观测时刻点,知道了时间参量,我们可以建立关于时间参量的方程组对卫星位置进行求解。

建立和问题1.1同样的模型,对问题进行求解。

(1) 预测未来三次卫星被观测情况

根据上述模型可以预测出此后3次观察到卫星的时刻和相关数据,如表22所示。

\textbf{表22 未来三次被观测情况}

\begin{tabular}{|c|c|c|c|c|c|c|c|}
\hline 未来观测次数 & Time & Azm & Elv & Mag & Range & S.Azm & S.El \\
\hline \multirow{3}{*}{第11次} & 07:52:29 & 173.2 & 10 & 11.2 & 1976 & 86.5 & 30.2 \\
\hline & 07:56:07 & 80.4 & 87.1 & 7.7 & 440 & 87 & 31.1 \\
\hline & 07:59:13 & 352.7 & 10.1 & 11.1 & 1981 & 87.4 & 32.1 \\
\hline \multirow{3}{*}{第12次} & 19:26:58 & 28.2 & 10.1 & 10.2 & 1969 & 297.2 & -8.1 \\
\hline & 19:30:07 & 95.4 & 33.4 & 8.8 & 1078 & 297.8 & -8.9 \\
\hline & 19:33:15 & 162.6 & 10.1 & 10.2 & 1962 & 298.5 & -9.7 \\
\hline \multirow{3}{*}{第13次} & 21:17:15 & 332.2 & 10 & 10.2 & 1966 & 313.5 & -24.9 \\
\hline & 21:20:09 & 288.3 & 18.7 & 9.6 & 1486 & 314.2 & -25.4 \\
\hline & 21:23:26 & 194.4 & 10 & 10.2 & 1958 & 314.8 & -26.9 \\
\hline
\end{tabular}

(2) 预测所需最少连续观察次数 \( n \) 的确定

若要确定卫星已经被连续观察最少多少次才能预测下次被观测到的情况,必须清楚卫星的过顶周期(上面已经求得)。根据对K卫星的观测情况分析可知,在每日都可能出现连续两个周期内被观测到的情况,而且第二天与前一天相比,同一时刻也可以观测到该卫星,那么, \( n \) 的最小值为一日之内可以观测到的最多次数和下一天内的一次相加,即
\[
n_{\text{min}} = 4 + 1
\]
上式中,4为一日之内最多可以观测到的次数,1代表下一天内观测到的一次。所以,K型卫星被连续观察最少5次才能预测下次被观测到的情况。

\section{问题 2 模型的建立和求解}

\subsection{模型的建立}

问题 2 是某部队需要从新疆的阿勒泰隐蔽地经喀什运动到和田并在和田执行某任务,24 小时后再隐蔽地返回阿勒泰,为防止 Q 型卫星、L-1 卫星对其进行侦察,需要设计合理的行进方案,同时保证部队在该路段机动时间最短。问题 2 是在卫星监视下,部队的机动路线优化问题,一方面要求躲避卫星侦察,另一方面要求机动时间最短。

\textbf{建立条件约束优化模型:}

\textbf{目标函数:} 最优机动路径。

\textbf{约束条件:} (1) 部队机动时避开卫星过顶侦察时间;

\quad (2) 机动路径用时最短。

那么,需要建立约束条件的求解模型:

\subsubsection{过顶时间求解模型}

首先要建立目标路段的道路交通网$^{[3]}$。设部队的机动任务从阿勒泰经喀什到和田的道路交通网络图为 $G=(V,A,T,\tau)$,其中,$V$ 代表节点阿勒泰、喀什、和田等县级以上(含县级)城市;$A$ 代表弧集(路段),$a_{ij}$ 代表连接节点 $i$ 和节点 $j$ 的路段,即阿勒泰到喀什,喀什到和田等路段;$T_{i}$ 代表原始通行时间集(即假设卫星不侦察时,部队从阿勒泰机动到和田的时间集),$t_{ij}$ 代表 $a_{ij}$ 上的原始通行时间;$\tau$ 代表卫星对所有路段的过顶时间集,$\left(\tau_{i},\tau_{j}\right)$ 代表卫星对路段 $a_{ij}$ 的过顶时间。建立的道路交通网如图 11 所示(斜线部分代表卫星过顶情况)。

\begin{figure}[h]
    \centering
    \includegraphics[width=0.8\textwidth]{image.png} % 替换为实际图片路径
    \caption{道路交通网及卫星过顶示意图}
    \label{fig:road_network}
\end{figure}

卫星的过顶时间是由卫星星下点轨迹决定的,卫星与地球中心连线在地球表面的交点称为星下点,卫星运动和地球自转使星下点在地面移动所画出的轨迹称为星下点轨迹。星下点轨迹能够直观的反映卫星的过顶情况,即卫星可侦察范围。建立卫星的星下点轨迹方程:

\begin{equation}
\left\{
\begin{aligned}
\phi &= \arcsin \left( \sin i \sin \frac{360}{T} t \right) \\
\delta &= \Omega + \arctan \left( \cos i \tan \frac{360}{T} t \right) - \omega t
\end{aligned}
\right.
\end{equation}

其中,$\phi$ 为地心纬度,$\delta$ 地球上的经度,$\Omega$ 为升交点赤径,$i$ 为轨道倾角,$T$ 为卫星运行周期,$\omega$ 为卫星角速度。从图 11 中可以明显看到卫星过顶时覆盖区域的纬度范围。那么反推,若给出目标区域的经纬度范围就可知道卫星的过顶时间。

### (2) 机动路径优化模型

通过上面分析建立了道路交通网,题目给出了卫星轨道根数,那么路径优化模型至只与部队机动出发时间有关,由于题目给出了出发时间和返回时间,那么可以建立目标函数,目标函数为从出发点到目的地用时最短的路径,用模型反应为对应路径上原始机动时间和延迟时间(卫星过顶时间)积累最少者的值,此时的路径即为最优机动路径 [3]。

目标函数为:

\begin{equation}
F(t_0) = \min \left( \sum_L T_{i,j} + \sum_L f(i,j) \right)
\end{equation}

式中,$t_0$ 为部队机动出发时间,L 为最优机动路径,$f(i,j)$ 为节点 $i$ 到节点 $j$ 之间的延迟时间函数:

\begin{equation}
f(i,j) = g(t_i, t_{ij}, \tau_{ij})
\end{equation}

$f(i,j)$ 是部队到达 $i$ 节点时刻点 $t_i$、路段 $a_{ij}$ 上的原始通行时间 $t_{ij}$ 和卫星对路段 $a_{ij}$ 的过顶时间 $\tau_{ij}$ 的函数。过顶时间 $\tau_{ij}$ 对 $f(i,j)$ 的影响很大,为了确定 $f(i,j)$ 的表达式,需要对卫星的过顶情况进行分析,有三种情况,如图 12 所示。

\begin{figure}[h]
\centering
\includegraphics[width=\textwidth]{satellite_passing_cases.png}
\caption{卫星过顶情况关系示意图}
\end{figure}

(1) 当星下点轨迹为 a 时,部队在到达节点 $i$ 前,卫星已经过境,离开目标路段;$f(i,j) = 0$。

(2) 当星下点轨迹为 b 时,卫星正好过顶,能够对部队在目标路段的机动情况进行监视。这时,部队必须在节点处进行休息等待,$f(i,j) = \tau_j - t_i$。

(3) 当星下点轨迹为 c 时,部队在到达节点 $j$ 后,卫星才开始过境,部队在目标路段的机动已经完成;$f(i,j) = 0$。

## 6.2 模型的求解

分析阿勒泰、喀什、和田的地理位置(如图 13 所示),查取该地区的县级以上(含县级)城市,获取所有城市地理信息(经纬度),见表 23。

\begin{figure}[h]
    \centering
    \includegraphics[width=0.8\textwidth]{image1.png}
    \caption{阿勒泰、喀什、和田的地理方位}
    \label{fig:13}
\end{figure}

\begin{table}[h]
    \centering
    \begin{tabular}{|c|c|c|}
        \hline
        城市名称 & 经度值 & 纬度值 \\
        \hline
        乌鲁木齐 & 87.62 & 43.82 \\
        \hline
        克拉玛依 & 84.87 & 45.60 \\
        \hline
        吐鲁番 & 89.17 & 42.49 \\
        \hline
        哈密 & 93.52 & 42.83 \\
        \hline
        昌吉 & 87.30 & 44.02 \\
        \hline
        博乐 & 82.07 & 44.90 \\
        \hline
        库尔勒 & 86.15 & 41.77 \\
        \hline
        阿克苏 & 80.27 & 41.17 \\
        \hline
        阿图什市 & 76.17 & 39.72 \\
        \hline
        喀什 & 75.98 & 39.47 \\
        \hline
        和田 & 79.92 & 37.12 \\
        \hline
        伊犁哈萨克 & 81.32 & 43.92 \\
        \hline
        塔城 & 82.98 & 46.75 \\
        \hline
        石河子 & 86.03 & 44.30 \\
        \hline
    \end{tabular}
    \caption{县级以上(含县级)城市地理信息表}
    \label{tab:23}
\end{table}

其中一段星下点轨迹,与新疆地区和目标施工区域,在地球平面上的方位示意图如图\ref{fig:14}所示。

\begin{figure}[h]
    \centering
    \includegraphics[width=\textwidth]{image2.png}
    \caption{星下点轨迹与新疆和施工区域方位图}
    \label{fig:14}
\end{figure}

对目标区域进行放大显示,如图\ref{fig:15}。

\begin{figure}[h]
    \centering
    \includegraphics[width=\textwidth]{image3.png}
    \caption{放大显示的目标区域}
    \label{fig:15}
\end{figure}

\begin{figure}[h]
    \centering
    \includegraphics[width=\textwidth]{image.png}
    \caption{星下点轨迹与新疆和施工区域方位放大图}
    \label{fig:15}
\end{figure}

\subsection{避开L-1型卫星的最优机动路径}

规划路径方案是从新疆的阿勒泰经喀什运动到和田并在和田执行某任务,24小时后再返回阿勒泰,所以最优路径求解分为两个部分:去程(从阿勒泰到和田)和返程(从和田到阿勒泰)。

模型计算L-1型卫星所需参数:
\begin{equation}
\Omega = 296.321 \text{度}, \, i = 57.009 \text{度}, \, T = 97.87 \, \text{(min)}, \, \omega = \frac{2\pi}{T}
\end{equation}

\subsubsection{从阿勒泰经喀什到和田}

从图中可以得到,新疆阿勒泰地区的经纬坐标为(88.13,47.85),施工区域的经纬度坐标为(118.7681,32.0209),两者之间的时间差值为2小时,L-1型卫星每转动15圈与地球自转一周的时间差值为24分钟,所以,为了确定在多少周期以后,卫星的星下点轨迹能够经过新疆阿勒泰地区,经过以上数据计算可以得出在5天之后,卫星经过问题1中施工区域的被观察情况,与卫星到达新疆阿勒泰地区的卫星被观察情况应该相近。

又因为,L-1型卫星的运转周期为97.6分钟,所以在61天后,该卫星被观察情况,应该与问题一中施工区域在相近的时刻卫星的被观察情况形同。同理可知,若要求在2016年11月1日凌晨5时的卫星被观察情况,应该与2016年9月1日凌晨5时,该卫星被观察情况相同。

由以上分析可得,在2016年9月1日凌晨5时卫星被观察情况,应与2016年8月27日凌晨5时,卫星的被观察情况相近。由于卫星运行59圈的时间与地球自转时间的时间差值为1.6分钟,所以,2016年8月27号的卫星被观察情况,可以根据2016年8月23号的卫星被观察情况来确定,具体数据见第一问,问题2的模型求解,所以2016年11月1号凌晨5点起的卫星被观察情况,如表24所示。

\begin{table}
\centering
\caption{2016年11月1日L-1卫星在阿勒泰被观察情况}
\begin{tabular}{c c c c c c c c}
\hline
11月1日被 & Time & Azm & Elv & Mag & Range & S.Azm & S.Elv \\
观测到次数 & & & & & & & \\
\hline
\multirow{3}{*}{第一次} & 17:11:21 & 180.1 & 10 & 5 & 2038 & 279.6 & 4.5 \\
 & 17:15:12 & 109.1 & 45.8 & 3 & 850 & 280.1 & 3.7 \\
 & 17:19:02 & 38.1 & 10.1 & 4.9 & 2037 & 280.6 & 2.9 \\
\hline
\multirow{3}{*}{第二次} & 18:49:44 & 156.4 & 10 & 5.3 & 2023 & 286.1 & 2.9 \\
 & 18:54:19 & 274.4 & 85.3 & 2.0 & 670 & 287.5 & 1.7 \\
 & 18:58:54 & 31.8 & 10 & 5.2 & 2033 & 288.4 & 0.8 \\
\hline
\end{tabular}
\end{table}

\section*{第一次休整:}
由于从阿勒泰凌晨5时出发,按照100km/h的速度行进,在下午17时之前,到阿克苏市距离为1358km,数据由附件3得到,超过了每日最长行车时间,所以,必须在此之前选择一个县级市进行休整,结合L-1型卫星过顶时间,综合考虑,可以在巴音郭楞自治州进行休息,巴音郭楞自治州的经纬度坐标为(88.17,39.02),部队行车距离为1056km,行车时间为约12小时。可以得到,2016年11月2日L-1卫星在阿勒泰被观察情况,见表25。

\begin{table}
\centering
\caption{2016年11月2日L-1卫星在阿勒泰被观察情况}
\begin{tabular}{c c c c c c c c}
\hline
11月2日被 & Time & Azm & Elv & Mag & Range & S.Azm & S.Elv \\
观测到次数 & & & & & & & \\
\hline
\multirow{3}{*}{第一次} & 03:31:19 & 3.8 & 10 & 4.9 & 2040 & 56 & -24.2 \\
 & 03:35:05 & 58.9 & 24.4 & 4 & 1312 & 56.7 & -23.5 \\
 & 03:38:51 & 114.2 & 10 & 4.9 & 2033 & 57.4 & -22.9 \\
\hline
\multirow{3}{*}{第二次} & 05:11:06 & 309.3 & 10 & 4.9 & 2040 & 71.9 & -5.2 \\
 & 05:14:55 & 253.1 & 25.9 & 3.9 & 1261 & 72.4 & -4.4 \\
 & 05:18:43 & 196.9 & 10 & 4.9 & 2028 & 72.9 & -3.7 \\
\hline
\multirow{3}{*}{第三次} & 17:39:06 & 176.1 & 10 & 5.3 & 2020 & 276.7 & 12.6 \\
 & 17:43:19 & 108.5 & 37.4 & 3.7 & 983 & 277.2 & 11.7 \\
 & 17:47:32 & 41.6 & 10.1 & 5.2 & 2031 & 277.8 & 10.9 \\
\hline
\multirow{3}{*}{第四次} & 19:20:15 & 260.5 & 10 & 5.3 & 2027 & 290.1 & -8.3 \\
 & 19:23:21 & 303.5 & 17.7 & 4.7 & 1580 & 290.5 & -8.9 \\
 & 19:26:28 & 346.4 & 10 & 5.1 & 2038 & 291 & -9.5 \\
\hline
\end{tabular}
\end{table}

\section*{第二次休整:}
由于,部队需要休整10个小时以上,结合卫星过顶时间,部队可以选择5点30分从巴音郭楞自治州出发,由于从巴音郭楞自治州到喀什市的距离为1006km,数据由附件3得到,在行车时间12小时之内可以到达,结合卫星过顶时刻为下午17时30分之后,所以可以在喀什市安排休息,喀什市的经纬度坐标为(75.9,39.47),部队行车距离为1006km,行车时间为约12小时。可以得到,2016年11月3日L-1卫星在阿勒泰被观察情况,见表26。

\begin{table}
\centering
\caption{2016年11月3日L-1卫星在阿勒泰被观察情况}
\begin{tabular}{c c c c c c c c}
\hline
11月3日被 & Time & Azm & Elv & Mag & Range & S.Azm & S.Elv \\
观测到次数 & & & & & & & \\
\hline
\multirow{3}{*}{第一次} & 03:53:55 & 343.6 & 10.1 & 4.9 & 2037 & 60.3 & -20.3 \\
 & 03:58:28 & 64.5 & 61.3 & 2.6 & 724 & 61.1 & -19.5 \\
 & 04:03:00 & 145.7 & 10 & 4.9 & 2028 & 61.9 & -18.6 \\
\hline
\multirow{3}{*}{第二次} & 18:01:52 & 206.6 & 10 & 5.4 & 2020 & 279.3 & 7.7 \\
 & 18:06:26 & 294.5 & 79.6 & 2.8 & 650 & 279.9 & 6.7 \\
 & 18:11:02 & 21.3 & 10 & 5.2 & 2036 & 280.5 & 5.8 \\
\hline
\end{tabular}
\end{table}

\section*{第三次休整:}
由于,从喀什市到和田市的距离为516km,数据由附件3得到,所以可以在喀什市

\begin{table}
\centering
\caption{2016年11月5日L-1卫星在阿勒泰被观察情况}
\begin{tabular}{c c c c c c c c}
\hline
11月5日被观测到次数 & Time & Azm & Elv & Mag & Range & S.Azm & S.Elvt \\
\hline
\multirow{3}{*}{第一次} & 03:01:59 & 343.3 & 10 & 5 & 2038 & 64.1 & -13.4 \\
 & 03:05:12 & 51.7 & 41.6 & 3.2 & 912 & 64.7 & -14.1 \\
 & 03:08:23 & 119.9 & 10.1 & 4.9 & 2037 & 65.2 & -14.8 \\
\hline
\multirow{3}{*}{第二次} & 04:40:41 & 331.2 & 10.1 & 4.9 & 2037 & 68.2 & -9.8 \\
 & 04:45:13 & 247.9 & 79.1 & 2.6 & 680 & 68.8 & -8.9 \\
 & 04:49:45 & 164.9 & 10 & 4.9 & 2033 & 69.5 & -8 \\
\hline
\multirow{3}{*}{第三次} & 17:09:45 & 180.1 & 10 & 5 & 2038 & 279.6 & 4.5 \\
 & 17:13:36 & 109.1 & 45.8 & 3 & 850 & 280.1 & 3.7 \\
 & 17:17:26 & 38.1 & 10.1 & 4.9 & 2037 & 280.6 & 2.9 \\
\hline
\multirow{3}{*}{第四次} & 18:48:08 & 156.4 & 10 & 5.3 & 2023 & 286.1 & 2.9 \\
 & 18:52:43 & 274.4 & 85.3 & 2.0 & 670 & 287.5 & 1.7 \\
 & 18:57:18 & 31.8 & 10 & 5.2 & 2033 & 288.4 & 0.8 \\
\hline
\end{tabular}
\end{table}

\begin{table}
\centering
\caption{2016年11月6日L-1卫星在阿勒泰被观察情况}
\begin{tabular}{c c c c c c c c}
\hline
11月6日被观测到次数 & Time & Azm & Elv & Mag & Range & S.Azm & S.Elvt \\
\hline
\multirow{3}{*}{第一次} & 05:17:27 & 338 & 10.1 & 5.1 & 2037 & 71.5 & -3.4 \\
 & 05:22:03 & 64.2 & 82.3 & 2.6 & 649 & 72.2 & -2.5 \\
 & 05:26:39 & 154.5 & 10 & 5.2 & 2030 & 72.8 & -1.5 \\
\hline
\multirow{3}{*}{第二次} & 19:25:31 & 215.3 & 10 & 5.1 & 2023 & 291.8 & -8.4 \\
 & 19:30:01 & 295.1 & 59.1 & 2.8 & 735 & 292.4 & -9.3 \\
 & 19:34:33 & 15.7 & 10.1 & 5 & 2032 & 293.1 & -10.2 \\
\hline
\end{tabular}
\end{table}

\begin{figure}[h]
    \centering
    \includegraphics[width=0.8\textwidth]{image1.png}
    \caption{最优机动路径}
    \label{fig:optimal_path}
\end{figure}

\subsection{Q型卫星的过顶情况预测}

由于,在附件中的数据,对Q卫星来说,没有出现具体的日期,所以在行车过程中,很难预测在2016年11月1日到2016年11月5号之间的Q卫星过顶情况。然而,结合附件2-1中Q卫星的观察数据,一共观察到卫星为9次,总体分布在两个时间段,分别时10点半到0点半,和11点半到13点半之间。

在从阿勒泰经喀什到达和田行车过程中,主要的Q型卫星可能的过顶时间,应该出现在这两个时间段之间,在行车过程中可以利用这一段时间,进行部队休整。在返回途中,采用相同的原理,避开Q型卫星的过顶时间段。

\section{问题3模型的建立和求解}

\subsection{过顶情况数据分析}

通过过顶数据分析可知卫星在短暂时段内多次观测同一地区,这是由于侦察卫星经过赤道节点的地方时不变,那么太阳方位角也不变,所以卫星不论在升轨段还是在降轨段,经过同一纬度地区的时间相同。

由以上分析,Q型卫星在前4次被观察到的运动轨迹,假设方框区域为导弹车的自由运动区域,坐标的中心点为观察点,如图\ref{fig:missile_avoidance}所示。

\begin{figure}[h]
    \centering
    \includegraphics[width=0.8\textwidth]{image2.png}
    \caption{导弹发射装置规避Q型卫星侦察示意图}
    \label{fig:missile_avoidance}
\end{figure}

由图中路线,可以判断导弹车,只有在预知卫星过顶时,运用观察数据,得到卫星的运行轨迹之后,根据运行轨迹,判断在某一时刻,应该怎样“避开”卫星的侦察,保证导弹发射系统的安全。

\section{模型的建立}

为了研究移动发射装置规避卫星侦察,设定某一确定区域,对题目中的 Q 型、L 型、K 型三种卫星的侦察情况进行分析。针对设定区域建立侦察卫星对目标机动路段的覆盖模型。

首先对卫星过顶情况进行分析,建立星下点轨迹方程。设地面点的当前纬度为 $\phi_{0}$,经度为 $\delta_{0}$,卫星星下点机动目标的当前纬度为 $\phi_{s}$,经度为 $\delta_{s}$,建立卫星的星下点轨迹方程:
\[
\begin{cases}
\phi_{s} = \arcsin \left( \sin i \sin \frac{360}{T} t \right) \\
\delta_{s} = \Omega + \arctan \left( \cos i \tan \frac{360}{T} t \right) - \omega t
\end{cases}
\]
其中,$\Omega$ 为升交点赤径,$i$ 为轨道倾角,$T$ 为卫星运行周期,$\omega$ 为卫星角速度。如果卫星的覆盖角包含了卫星星下点与地面目标点的弧线,那么该目标点在卫星覆盖范围内,扩展到机动路径,如果覆盖面包含了星下点轨迹与地面路段的弧面,那么该路段就会被卫星侦察覆盖 [4]。

地面点的卫星覆盖情况表示为:
\[
\xi \leq \beta
\]
用地面点和卫星星下点经纬度表示为:
\[
\cos \beta \leq \sin \phi_{0} \sin \phi_{s} + \cos \phi_{0} \cos \phi_{s} \cos (\delta_{0} - \delta_{s})
\]
其中,$\beta$ 为卫星覆盖角,$\xi$ 为星下点与地面点的弧长 $(0 \leq \xi \leq \pi)$。那么可得,卫星的覆盖带宽为:
\[
L = 2 \left[ a \cos \left( \sin \Psi \right) - a \cos \left( \frac{H + R_{e}}{R_{e}} \right) * \sin \Psi \right]
\]
其中 $\Psi$ 为相机(CCD 相机或 SAR 成像探测器)视场的一半。

对目标区域机动路段的覆盖情况,建立模型为:
\[
X =
\begin{cases}
1 & L \cap S = \Phi \\
0 & L \cap S \neq \Phi
\end{cases}
\]
其中,$\Phi$ 代表空集。

当 $X = 1$,机动路段避开了卫星覆盖带宽,则该路段被卫星侦察覆盖,可行;当 $X = 0$,卫星覆盖带宽与机动路段重叠,则该路段没有被卫星侦察覆盖,不可行。

\section{模型的求解}

由于问题 2 中指定区域 Q 型、L 型卫星的过顶情况已经求得,本题也以附件 3 给出的地图为目标区域,分析机动路径对卫星侦察的规避情况,得到针对 Q 型卫星、L 型卫星、Q 与 L 二星组网在新疆地区的机动路径。如图 18、19、20 所示。

\begin{figure}[h]
    \centering
    \includegraphics[width=0.8\textwidth]{image1.png}
    \caption{导弹发射装置规避 Q 型卫星侦察示意图}
    \label{fig:q_satellite}
\end{figure}

\begin{figure}[h]
    \centering
    \includegraphics[width=0.8\textwidth]{image2.png}
    \caption{导弹发射装置规避 L 型卫星侦察示意图}
    \label{fig:l_satellite}
\end{figure}

\begin{figure}[h]
    \centering
    \includegraphics[width=0.8\textwidth]{image3.png}
    \caption{导弹发射装置规避 Q、L 双星组网侦察示意图}
    \label{fig:q_l_satellite}
\end{figure}

\section{模型的评价和改进}

\subsection{误差分析}

由于本题涉及到的是卫星运动和观测数据,从数据分析、模型建立和模型求解中存在一些误差,下面进行分析:

(1)卫星被观测时间的预测误差。三个问题的模型都涉及到一个前提,对卫星被观测时间的预测,但是,经过对实际观察到的时间数据的分析,其理论计算和实际观察有一定误差(在每个问题的数据分析都有说明)。产生误差的原因有以下三个方面:
\begin{itemize}
    \item[a.] 卫星轨道面在太空中,一天随太阳转动 0.986 度,即在下一次观察到卫星的时候,卫星的轨道面已经发生改变,而本题中将其忽略;
    \item[b.] 地球实际为一个椭球,而实际计算时按照球来计算;
    \item[c.] 卫星的轨道倾角在运行过程中会对运行时间造成一定的误差,倾角越小误差越大。
\end{itemize}

(2) 模型建立时的近似模型误差。模型建立时,将地球近似为球体处理,忽略地球扁率的影响,忽略大气摄动、日月引力、太阳光压摄动等的影响,并且假设卫星轨道无摄动。

在实际情况下,由于大气阻力,卫星的轨道高度和运行周期都会衰减,可以通过大气密度变化的插值计算来解决大气摄动的影响。对于日月引力和太阳光压摄动,可以通过数据拟合影响系数,降低其对模型精度的影响程度。

(3) 卫星观测位置误差对观测站天线方位角、仰角计算的影响误差。对于方位角,如果卫星位置误差为 $(\Delta x, \Delta y)$,那么方位角误差为:
\[
\Delta \varphi = \arctan \frac{x + \Delta x}{y + \Delta y} - \arctan \frac{\Delta x}{\Delta y}
\]
对于仰角,如果卫星位置误差为 $\Delta Loc$,误差在 $z$ 轴上的分量为 $\Delta z$,那么仰角误差为:
\[
\Delta \gamma = \arcsin \frac{z + \Delta z}{H + \Delta Loc} - \arcsin \frac{z}{H}
\]
以 L 型卫星为例,其轨道长半轴为 7000km 左右,如果卫星位置偏差 3km,那么可以计算其最大仰角误差大约为 0.2 度。

\section*{8.2 优缺点}

在题目给定卫星参数情况下,通过对卫星的被观测数据进行分析,建立了基于时间的预测模型,对三种卫星的被观测情况和过顶情况进行了预测,为实现新疆地区部队隐蔽机动任务,建立了机动路径优化模型,求解得到了避开 L-1 卫星的最优机动路径,给出了 Q 型卫星过顶预测情况,并在问题三中建立了侦察卫星对目标机动路段的覆盖模型,给出了移动发射装置对不同卫星侦察的规避方案。对问题的讨论由浅入深,并进行了严格的数学分析与推导,使模型建立在坚实的数学基础上。针对三个问题的模型建立和求解过程,具有以下特点:

(1) 在问题分析过程中,针对问题数据信息不明确,分析出了观测时间的变化规律,通过坐标系转换,解决了卫星被观测到的位置求解,使问题求解过程清晰化,有很强的逻辑性。

(2) 在建模过程中,充分考虑各种约束条件,避免了人为主观因素对模型的影响,具有很强的客观性。

(3) 依靠严密的数据运算,模型求解过程中运用 C++ 软件,结合人工处理,使求解过程简化,结果可靠。

(4) 在行军路线规划问题上,建立了最优路径目标函数,结合了对问题求解结果的分析,给出了科学性强、符合实际的方案。

总体来说,整个模型建立和求解过程中,数据分析比较合理,模型建立有针对性,求解过程比较科学严谨。但是依然存在以下两个方面的问题:

(1) 忽略了地球椭球体和卫星摄动的影响,求解结果会有误差产生。

(2) 由于时间紧迫、问题复杂,在考虑问题时难免忽略部分细节。

\section*{8.3 模型的改进}

在规避卫星侦察部队机动路径选择问题上,可以结合已有的地理信息系统软件 GIS,建立基于 GIS 的路径选择模型,使问题的求解更加直观、高效。

建立基于 GIS 的考虑卫星覆盖的路径选择模型并求解,可以从以下几个方面进行:

\section*{9 参考文献}

[1] 付兆萍. 卫星轨道运动方程数值算法研究[D]. 华中科技大学, 2006.

[2] 凌震莹. 大地坐标系与站心地平直角坐标系的坐标转换[J]. 声光与电子工程, 2009, 4: 31-34.

[3] 张建新, 陈浩光. 卫星监视下导弹部队战役机动路线优化模型研究[J]. 中国运筹学会第八届学术交流会论文集, 西安, 2006, 817-822.

[4] 周少钦, 范耀祖, 沈晓蓉. 侦察卫星地面机动目标覆盖特性分析与仿真[J]. 计算机仿真, 2008, 8: 64-68.

\section*{10 附录}

\textbf{卫星被观测情况预测界面程序:}

\begin{verbatim}
double r=6371;//km,地球半径
double u;//方位角
double v;//仰角
double d;//卫星到观测站距离
double D;//卫星到地心距离
double p;//角速度
double T;//卫星周期
double w0;//卫星初始转角
double t,w;//卫星转过的时间和对应角度
double a0,b0,c0,e0;
double t1,t2;//观测到卫星的前一时刻和后一时刻
double m;//近地点幅角
double n;//升焦点赤径
double X1,Y1,Z1;//地心坐标系中卫星坐标
CString str9,str10,str11;
m_text9.GetWindowText(str9);
m_text10.GetWindowText(str10);
m_text11.GetWindowText(str11);
d=1299;
m=36.122;
n=318.063;
v=atof(str9);
t1=atof(str10);
t2=atof(str11);
t=t2-t1;
a0=6866;
c0=2;
b0=sqrt(a0*a0-c0*c0);
e0=c0/a0;
T=94.37;//min
p=2*M_PI/T;
\end{verbatim}

\begin{verbatim}
w = p * t;
D = d * sin(v) + r;
w0 = acos((1 - a0 * (1 - e0 * e0) / D) / e0);
r = a0 * (1 - e0 * e0) / (1 - e0 * cos(w + w0));
X1 = r * cos(w0 + w) * cos(m);
Y1 = r * sin(w0 + w) * cos(90 - (360 - n));
Z1 = r * cos(w0 + w) * sin(m);
double a, b, e, N;
double L0, B0, H0;
double X0, Y0, Z0;
double L, B, H;
double X, Y, Z;
double x, y, z;
double U, V;
CString str1, str2, str3, str4, str5, str6, str7, str8;
m_text1.GetWindowText(str1);
m_text2.GetWindowText(str2);
m_text3.GetWindowText(str3);
m_text4.GetWindowText(str4);
m_text5.GetWindowText(str5);
a = atof(str1);
b = atof(str2);
L0 = atof(str3);
B0 = atof(str4);
H0 = atof(str5);
e = sqrt(a * a - b * b) / a;
N = a / sqrt(1 - e * e * sin(B0) * sin(B0));
X0 = N * cos(B0) * cos(L0);
Y0 = N * cos(B0) * sin(L0);
Z0 = N * (1 - e * e) * sin(B0);
L = atof(str6);
B = atof(str7);
H = atof(str8);
X = X1;
Y = Y1;
Z = Z1;
x = (-sin(L)) * (X - X0) + cos(L) * (Y - Y0) + H * (Z - Z0);
y = (-sin(B)) * cos(L) * (X - X0) + (-sin(B)) * sin(L) * (Y - Y0) + cos(B) * (Z - Z0);
z = (cos(B)) * cos(L) * (X - X0) + (cos(B)) * sin(L) * (Y - Y0) + sin(B) * (Z - Z0);
m_result1.Format("%f", x);
m_result2.Format("%f", y);
m_result3.Format("%f", z);
V = atan(z / (sqrt(x * x + y * y)));
U = acos(y / (sqrt(x * x + y * y)));
m_Mcom.Format("%f", V);
m_Pcom.Format("%f", U);
UpdateData(false);
\end{verbatim}