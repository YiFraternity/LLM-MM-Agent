\begin{center}
\textbf{“华为杯”第十三届全国研究生数学建模竞赛}
\end{center}

\begin{table}[h]
\centering
\begin{tabular}{l l}
学 校 & 天津工业大学 \\
\hline
参赛队号 & 10058003 \\
\hline
队员姓名 & 1. 常虹 \\
 & 2. 万兵涛 \\
 & 3. 王晓红 \\
\hline
\end{tabular}
\end{table}

\begin{flushright}
参赛密码 \underline{\hspace{2cm}} \\
(由组委会填写)
\end{flushright}

\begin{center}
\textbf{参赛密码} \underline{\hspace{2cm}} \\
(由组委会填写)
\end{center}

\begin{center}
\includegraphics[width=0.3\textwidth]{image1.png} \quad
\includegraphics[width=0.3\textwidth]{image2.png} \quad
\includegraphics[width=0.3\textwidth]{image3.png}
\end{center}

\begin{center}
\textbf{“华为杯”第十三届全国研究生数学建模竞赛}
\end{center}

\begin{center}
\textbf{题目} \quad 粮食最低收购价问题研究
\end{center}

\section*{摘要:}

粮食,不仅是人们日常生活的必需食品,而且还是维护国家经济发展和政治稳定的战略物资。粮食最低收购价政策是保护粮食产业的重要举措之一。本文结合历史数据,分析研究粮食最低收购价政策实施有效性以及合理定价问题,对于完善和深化这项政策、促进粮食产业的健康发展具有重要意义。

我国实行小麦最低收购价政策的有6个省份,水稻最低收购价政策有11个主产省。根据经济学理论,本文分析确定了影响种植面积的12个指标,分别为农业劳动力人口、农民受教育程度、城乡收入差距、家庭负担、水稻(或小麦)的市场价格、水稻(或小麦)的生产成本、水稻(或小麦)的相对收益竞争力、农作物播种面积、受灾情况、水稻(或小麦)的净出口、城市化水平和最低收购价政策,并且搜集了1995-2014年的数据。另外还搜集了2006年至2014年小麦和水稻的最低收购价格数据。

种植面积的影响因素分析(问题1)。首先选取了影响区域特征的指标,分别对小麦的6个主产省份和水稻的11个主产省份进行了聚类分析,将小麦的6个主产省份分为了两类,水稻的11个主产省份分为了三类。根据聚类结果,本文选取山东省作为华北地区、选取安徽作为华中地区小麦研究代表。选取江苏作为华中地区、选取吉林作为东北地区、选取湖南代表华南地区的水稻研究代表。分别对山东和安徽小麦代表省份和江苏、吉林、湖南三个水稻代表省份进行了种植面积影响因素的Spearman相关检验,分别建立了影响粮食种植面积的指标体系,并且对该指标体系进行了合理性评价。最后利用所选取的指标分别建立了主成分回归模型,指出了指标体系指标的重要性等次。

粮食最低收购价政策执行效果的评价(问题 2)。根据粮食品种与区域的差异,本文根据问题 1 中建立的指标体系,运用主成分-统计控制的方法分别对安徽省和山东省的小麦和吉林省、湖南省与江苏省的水稻进行粮食最低收购价政策执行效果的评价。为了消除变量间的共线性,提高评价的精确度,除了粮食最低收购价政策外,对指标体系其他变量进行主成分分析,根据累计方差贡献率选取前几个主成分。并作为协变量进入统计控制模型,建立粮食最低收购价政策与粮食种植面积的协方差分析模型。经过计算发现:粮食最低收购价政策在山东和安徽小麦的执行效果显著,其中粮食最低收购价政策促使安徽省平均每年小麦种植面积增加了 12.52%,促使山东省平均小麦种植面积增加了 21.72%。对于水稻所选取的三个省份,吉林省的粮食最低收购价政策促使水稻种植面积平均每年增加了 11.81%,但是湖南省和江苏省水稻的粮食最低收购价政策的执行效果却并不显著。最后选取实行了粮食最低收购价政策的山东省与没有实行粮食最低收购价政策的山西省,对它们的小麦种植面积的变化进行了对比,结果发现山东省相对于山西省小麦种植面积发生了明显提高。综上所述,粮食最低收购价政策在不同地区不同品种上表现出来的执行效果并不一致。

粮食价格体系模型的建立与分析(问题 3)。首先作图定性分析了粮食最低收购价价格与粮食市场价格的趋势对比,从直观上得出两种价格具有很强的趋同性。并计算了两种价格的相关系数,进一步确认了它们之间存在着很强的相关性。接着本文运用市场收购价理论和局部调整模型分别建立了包含粮食最低收购价在内的粮食价格形成模型,包括供应量模型、企业收购量模型和市场收购价格模型。运用历史数据分别对安徽、山东小麦和吉林、江苏、湖南水稻价格形成模型进行了实证分析。结果表明,粮食最低收购价价格对粮食市场收购价格有直接的正的影响,最低收购价价格的出台实际上抬高了粮食市场收购价。

粮食最低收购价合理定价的线性规划模型(问题 4)。以粮食产量为目标模型,以价格波动、财政支出、库存和种植面积为约束条件建立了粮食最低收购价合理定价的线性规划模型。然后我们分别以小麦和水稻为例,实证分析了两个小麦主产省和三个水稻主产省在“十二五”规划期间粮食最低收购价对粮食产量的影响。其中,安徽小麦、山东小麦、吉林水稻最低收购价政策效果显著。截至 2013 年,粮食最低收购价在调高 10% 的情况下,粮食产量涨幅分别为 7.9%、9.9% 和 7.2%;相比之下,江苏水稻的粮食最低收购价政策效果不明显。粮食最低收购价在调高 10% 的情况下,粮食产量涨幅为 2.9%;然而湖南省的最低收购价政策却没有对粮食产量的增长起到促进作用。粮食最低收购价在调高 10% 的情况下,粮食产量涨幅为 -4.9%。最后,我们利用该模型确定出 2017 年的合理最低收购价。结果见下表:

粮食最低收购价的可行性分析(问题 5)。对问题 4 小麦最低收购价合理定价的线性规划模型的约束模型进行了修改,建立了以小麦面积为目标的小麦最低收购价合理定价的线性规划模型。然后对 2015 年小麦最低收购价进行实证分析,最终确定在当年粮食最低收购价政策的基础上,收购价增加为 1.33 元/斤时,在满足约束条件的基础上,可将小麦的种植面积增加至增加至 6003.9 千公顷,实现了较 2000 年增大超过 5% 的目标。

调控粮食种植的优化决策和建议(问题 6)。结合上述研究得出了粮食最低收购价问题的研究结论,并且对我国粮食价格政策提出了政策建议。

关键词: Spearman 主成分回归 统计控制评价 价格形成 线性规划

\section*{第 1 章 前言}

\subsection{1.1 研究的目的与意义}

“民以食为天”作为一个命题,其所揭示的问题在生物学意义上的真实性是毋庸置疑的。粮食不仅是人们日常生活的必需品,还是维护国家经济发展和政治稳定的战略物资,具有不可替代性。由于耕地减少、水资源短缺、气候变化等问题日益凸显,中国粮食产业的弱质性、低效性在国际市场中就表现得尤为明显,使得中国粮农的分散经营更加难以抵御国外大规模的现代农场的冲击(黄家顺,2002;许经勇,2002)。因此,在中国,对粮食产业进行政策保护就显得尤为重要。

中国粮食保护政策体系主要有三大政策组成,即粮食生产支持政策、粮食价格支持政策和收入支持政策(何兵,2005)。粮食最低收购价政策就是粮食价格支持政策的一种。

中国粮食收购价格一般情况下是由市场供求决定,国家在充分发挥市场机制作用的基础上实行宏观调控。为保护农民利益、保障粮食市场供应,必要时由国务院决定对短缺的重点粮食品种,在粮食主产区实行最低收购价格。当市场粮价低于国家确定的最低收购价时,国家委托符合一定资质条件的粮食企业,按照国家确定的最低收购价收购农民的粮食。显然,它不同于2004年之前实行的保护价政策。

自2005年以来,中国陆续在河北、江苏、安徽、山东、河南、湖北等6个小麦主产省份以及辽宁、吉林、黑龙江、江苏、安徽、江西、河南、湖北、湖南、广西、四川等11个稻谷主产省(区)实行了最低收购价政策。

10年来的实践表明,粮食最低收购价政策有利于保护农民种粮利益、促进粮食生产发展,有利于粮食市场的稳定和国家的粮食安全(鲁晓东,2010)。然而这种政策也有其不完善的地方。要想健康、有效、长期的执行粮食最低收购价政策,需要及时了解并解决三个问题:第一,实行最低收购价政策对粮食的种植面积到底有多大影响?第二,粮食最低收购价政策对粮食市场收购价会产生怎样的影响?第三,如何确定合理有效的最低收购价格水平?而本文所需要解决的几个问题,其实质即包含于这三个问题之中。对于这些问题的科学研究和探讨,将为中国政府完善最低收购价政策体系提供可靠的决策参考。

本课题将运用农业经济学、福利经济学、制度经济学等相关理论,在搜集大量数据的基础上,对五个代表省份(安徽、山东、江苏、吉林、湖南)粮食最低收购价的经济效应进行深入分析,探究最低收购价政策对于种植面积如何影响。本课题还将通过建立和运用优化模型,探讨在耕地面积、成本收益、财政支持、储库容量、国外粮价等约束条件下,以提高粮食产量为目标的粮食最低收购价最优决策机制。本课题的研究对于有效构建和完善中国粮食最低收购价政策体系、稳定粮食生产、提高粮农的经济收入、促进国民经济又好又快发展具有非常重要的理论和现实意义。

\subsection{1.2 文献综述}

从2005年到2010年,连续6年的“中央一号文件”都对最低收购价政策提出了明确的要求。粮食最低收购价与国家宏观发展战略、粮食安全、农民增收、农业补贴政策等息息相关,其相关研究文献主要集中在以下几个方面。

\subsection{1.2.1 最低收购价及粮食补贴经济效应的研究}

自中国实施粮食最低收购价以及直接补贴政策以来,国内外学者对此展开了广泛的研究。20世纪60年代的欧盟通过建立最低收购价机制让大量农产品价格显著地高于世界市场价格,以确保农民的收入保持在一个合理的水平上(K.J.munk,1989)。与之类似,中国实施粮食最低收购价政策以来也取得了明显的效果。在稳定粮食生产、增加农民收入、支持市场粮价、完善粮食市场宏观调控机制方面发挥了重要作用(刘全书、刘福保、王明强,2006;杨银海,2006)。此项政策的重要性在于有利于防止“谷贱伤农”、有利于促进农民增产增收、有利于保证国家粮食安全、有利于刺激农村消费拉动内需(冯文丽、苏晓鹏,2009)。但是,一些学者在肯定粮食最低收购价的经济正面效应的同时也指出了隐忧,认为粮食最低收购价不利于粮食的市场化改革,不利于粮食种植结构调整优化和种植品种的换代升级(丁伟,2006)。张向东(2006)、刘梅生(2007)、焦善伟(2009)等也阐述了近年来粮食最低收购价变化情况及市场影响展望,对最低售价的利弊作了阐述,就完善最低收购价政策和操作给出了建议。但这些文献基本上都停留在定性描述上,并没有就最低收购价对粮食单产、种植面积、粮农增收等方面的经济效应大小作定量研究。

但粮食直接补贴的经济效应研究就比较深入了。Fred Gale, Bryan Lohmar, and Francis Tuan(2005)发现,中国的粮食直接补贴政策与单产水平没有必然联系,对提高单产没有激励作用。李鹏、谭向勇(2006)运用敏感度系数法分析和评价了直接补贴政策对农民种粮净收益的影响,结论是,直接补贴政策对提高农民种粮净收益有一定的作用,但作用不大。叶慧、王雅鹏(2006)的研究表明,中国粮食补贴制度效率普遍不高,各省市补贴制度效率也存在较大差距。王姣、肖海峰(2007)利用数学规划模型,对粮食最低收购价、农机补贴和减免农业税政策效果进行了分析。可以预见,中国粮食最低收购价政策的经济效应分析,必将和粮食直接补贴的经济效应一样,成为学术界关注的热点,上述文献对本课题研究具有借鉴作用。

\subsection{1.2.2 最低收购价格优化机制的研究}

粮食的最低收购价水平过低或过高都会带来不利影响。保持合理的粮食最低收购价水平有利于粮食生产的稳定。在国内,一些学者建议中国应实行浮动的粮食最低收购价水平(甘霖、兰录平,2007),并指出:合理的、能反映不同地区不同粮食品种、不同品质的、具有地区差异的最低收购价格水平是有效支持市场粮价、均衡财政支出的一个重要手段和措施(刘福保、王明强,2006;丁伟,2006)。

在中国目前形势下,为保护农民种粮积极性、促进农民增收、进而拉动内需促进经济发展,粮食最低收购价的不断提高是个必然趋势(苏民,2009)。但是,最低收购价格也不可能定得过高。因为粮食作为最基础的原材料之一,其价格的高低不仅直接影响相关下游产品的价格,而且价格的大幅上涨,也容易引发其他产品的价格上涨,并成为引发通货膨胀的诱因(廖西元,方福平,王志刚,2007)。

那么,确定粮食最低收购价要顾及哪些因素呢?粮食最低收购价格水平应足以弥补其生产成本;应充分反映较长时期内粮食的供求关系,形成较长期的均衡价格,以引导农民的粮食生产活动;要保证粮食生产所获得的利润率与从事非农业生产的利润率不能相差太大,否则就会影响农民种粮积极性;而且,

最低收购价的制定要顾及消费者的购买能力以及国家财政负担能力;最后还要考虑国际粮价标准(蔡贤恩,2008)。

目前国内外粮食最低收购价格的制定比较简单。美国、日本、欧盟等政府每年规定主要粮食的支持价格。其支持价格等于生产成本与合理收益之和。方鸿(2009)以生产成本为基准,再加上适当利润,从而得出了粮食最低收购价水平。

赵德余、顾海英(2004)的研究还发现,粮食补贴政策的地区差异,符合各地区农业资源禀赋、经济发展水平以及补贴政策改革的渐进性战略,这就启示我们不同地区、不同品种、不同品质的粮食最低收购价是有所区别的。

但是,这些研究也大都停留在定性分析上,建立定量模型研究确定最低收购价价格水平的文献很少。

目前,关于粮食最低收购价的经济效应的研究几乎都停留在定性分析上,没有涉及到计量问题。对于最低收购价格优化机制的研究目前也只停留在考虑生产成本和国家财政,没有考虑到消费者的承受能力等其他因素。农业补贴的经济效应研究也只停留在良种补贴、要素补贴等,很少涉及到最低收购价。本课题将对粮食主要产区粮食最低收购价政策的经济效应作定量分析,并在此基础上,建立多目标多约束模糊优化模型,合理确定最低收购价水平,为政府确定合理有效的粮食最低收购价政策提供决策参考。

\subsection*{1.3 指标的选取与数据的搜集}

\subsection*{1.3.1 指标的选取}

根据相关研究和经济学理论,选取了 12 个可能影响小麦和水稻种植面积的指标。它们分别是农业劳动力人口、农民受教育程度、城乡收入差距、家庭负担、粮食的市场收购价、粮食的生产成本、粮食的相对收益竞争力、农作物播种面积、受灾情况、粮食的净出口、城市化水平、最低收购价政策。我们先从直观上对这些指标进行分析,搜集相应数据,然后再运用统计方法通过实际验证进行筛选。因为在本文我们研究的粮食就是小麦和水稻,所以以下内容所涉及的粮食均指小麦和水稻,特在此处说明。

\subsubsection{1. 农业劳动力人口}

农业劳动力的人口,是指社会中符合劳动年龄并有劳动能力的人的数量,和不到劳动年龄或已超过劳动年龄但实际参加劳动的人的数量。在当前我国农业现代化还不够发达的情况下,粮食的种植面积与从事农业的人口具有一定的关联性。

在本文中,我们选用的省份都是粮食的主产省,第一产业就业人员主要来自于从事农业人员,所以我们搜集近 20 年来各个省份第一产业就业人员数据,用该指标来体现农业劳动力人口与粮食种植面积的关系。

\subsubsection{2. 农民受教育程度}

农民的受教育程度从另一方面也反映着农业劳动力的质量。主要是指农业劳动力的体力强弱、技术熟练程度和科学、文化水平的高低。农业劳动力的数量和质量因受自然、社会、经济文化教育等各种因素的影响而处于不断变化之中。该指标与粮食种植面积具有一定的关联性。

在本文,我们选用各个省份农民人均受教育时间来作为农民受教育程度的衡量,所以我们搜集近 20 年来农民人均受教育时间,用该指标来体现农民受教育程度与粮食种植面积的关系。

\section*{3. 城乡收入差距}

随着改革开放的发展,我们的生活水平和收入都得到了很大的提高。但是随之而来的还有城乡之间的差距也逐渐显现并不断加大,这就刺激了一部分追求致富的农民努力奋斗,去城市发展。在一定程度上对农业造成了影响。所以我们考虑城乡城乡收入差距与粮食种植面积有一定的关联性,并对此进行研究。

在本文,我们选用各个省份城镇居民人均可支配收入与农村居民人均纯收入的比值来作为衡量城乡收入差距的指标。我们分别搜集了各个省份 20 年来城镇居民人均可支配收入与农村居民人均纯收入的数据,计算出它们的比值,用该比值来作为衡量城乡收入差距与粮食种植面积的关系的指标。

\section*{4. 家庭负担}

家庭负担与粮食种植面积的关系主要体现在经济收入上。当农民仅仅依靠粮食收入无法承担家庭的负担时,也会影响到其对种田的精力投入,对粮食种植面积产生影响。

在本文,我们选用各个省份农户家庭最大支出总和来作为衡量家庭负担的指标。我们分别搜集了各个省份 20 年来农户家庭最大支出总和的数据,用该指标来作为衡量家庭负担与粮食种植面积的关系的指标。

\section*{5. 粮食的市场收购价}

当农民感觉粮食价格高时,作为主要的收入来源,农民就会对种粮报以极大的热情并投入相当大的精力,而且会有更多的人投入到种粮中去,从而会扩大粮食种植面积。所以粮食的市场收购价与粮食种植面积的关系也极其具有研究意义。

在本文,我们选用各个省份农户粮食的市场价格来作为衡量家庭负担的指标。我们分别搜集了各个省份 20 年来粮食的市场价格的数据,用该指标来作为衡量粮食的市场收购价与粮食种植面积的关系的指标。

\section*{6. 粮食的生产成本}

当农民的种粮成本过高时,就会影响到农民的收入,尤其是粮食的价格又不能弥补种类成本的投入时,就会在一定程度上影响农民的种粮积极性,影响粮食的种植面积。

在本文,我们选用各个省份粮食的每亩化肥、农药、种子和机械作业费用实际值来作为衡量粮食的生产成本的指标。我们分别搜集了各个省份 20 年来粮食的每亩化肥、农药、种子和机械作业费用实际值的数据,用该指标来作为衡量粮食的生产成本与粮食种植面积的关系的指标。

\section*{7. 粮食的相对收益竞争力}

在经济活动中,人们会通过不断选择不同的方式来使得自己获得最大收益。当种粮收益高时,农民会选择种田;当在能力范围内有更高的收益的其它挣钱方式时,农民就会放弃种田进而选择其它的挣钱方式。所以我们认为在不同时期的粮食的相对收益竞争力与粮食种植面积也有关系。

在本文,我们选用各个省份粮食的每亩净收益与该省棉花每亩净收益的差值来作为衡量粮食的相对收益竞争力的指标。我们分别搜集了各个省份 20 年来粮食的每亩净收益与该省棉花每亩净收益的数据,然后计算它们的差值,用该指标来作为衡量粮食的相对收益竞争力与粮食种植面积的关系的指标。

\section*{8. 农作物播种面积}

粮食的种植面积是农作物播种面积的一部分,自然地认为,农作物播种面积与粮食种植面积具有一定的关系。

在本文,我们选用各个省份农作物总播种面积来作为衡量农作物播种面积的指标。我们分别搜集了各个省份 20 年来农作物总播种面积的数据,用该指标来作为衡量农作物播种面积的关系的指标。

\section*{9. 受灾情况}

粮食作物的生长对环境是有一定的要求的,所以灾害对粮食种植面积是直接的,尤其是当发生旱涝灾害时,就会导致粮食作物无法生长,直接缩减粮食的种植面积。

在本文,我们选用各个省份旱灾受灾面积与该省旱灾成灾面积的比值来作为衡量受灾情况的指标。我们分别搜集了各个省份 20 年来旱灾受灾面积与该省旱灾成灾面积的数据,然后求出它们的比值,用该指标来作为衡量受灾情况与粮食种植面积的关系的关系的指标。

\section*{10. 粮食的净出口}

粮食的净出口是指粮食进口总量和出口总量的差值。当粮食产量达到一定量时,人们对粮食的需求就会减弱,进而影响到粮食的生产。所以粮食净出口会在一定程度上扩大粮食需求,进而影响到粮食种植面积。

在本文,我们选用我国粮食出口量与进口量的差值来作为衡量粮食的净出口的指标。我们搜集了我国 20 年来粮食出口量与进口量的数据,然后求出它们的差值,用该指标来作为衡量粮食的净出口与粮食种植面积的关系的指标。

\section*{11. 城市化水平}

随着城市化水平地提高,从事各行各业人数就会增多,农村的从事农业的人数就会相应地减少,也会影响到粮食的种植面积。

在本文,我们选用各个省份城镇人口所占比例来作为衡量城市化水平的指标。我们搜集了我国 20 年来各个省份城镇人口所占比例的数据,用该指标来作为衡量城市化水平与粮食种植面积的关系的指标。

\section*{12. 最低收购价政策}

最低收购价政策在稳定粮食价格、促进农民增收、服务宏观调控等方面发挥了一定的作用,根据之前的分析,这些因素与粮食种植面积息息相关,所以我们认为最低收购价政策与粮食种植面积具有一定的关系。

在本文,我们选用各个省份实施粮食最低收购价政策的年份,定义未实施该政策为 0,实施该政策定义为 1,搜集出一系列数据。用该数据来作为衡量最低收购价政策与粮食种植面积的关系的指标。

\section*{13. 粮食种植面积}

在亩产量稳定的情况下,粮食最低收购价政策的执行效果主要体现在粮食的种植面积上。本文将以粮食的种植面积作为因变量进行分析、建模和求解。

\subsection*{1.3.2 数据的搜集}

根据资料显示,我国政府是从 2005 年开始实施粮食最低收购价格政策的,当年实施对象为水稻一个品种,在 7 个主产省实施(吉林、黑龙江、安徽、江西、湖北、湖南、四川)。2006 年度,我国粮食最低收购价格政策实施对象又增加了一个品种即小麦,实施对象增加至两个品种,在 6 个小麦主产省实施(湖北、山东、河南、安徽、江苏、河北)。从执行地区来看,稻谷托市政策的执行范围在 2008 年以后有了明显扩大,早籼稻的执行范围增加了广西,中晚籼稻增加了江苏、河南和广西,粳稻增加了辽宁。小麦最低收购价格执行范围则一直没有发生变化。

所以我们将资料搜集的范围确定为 6 个小麦主产省(湖北、山东、河南、安徽、江苏、河北)和 11 个水稻主产省(吉林、黑龙江、安徽、江西、湖北、湖南、四川、江苏、河南、辽宁和广西)这 17 个省份的范围内。每个省份都包含有以上确定的 12 个影响因素和种植面积在内的 13 个指标。数据的来源包括:各省统计年鉴、中国统计局网站、布瑞克农产品数据库以及各个浏览器网页。数据量非常大,在此不再呈现于论文中,我们将以数据压缩包的形式随论文一起上传。在之后的建模求解中,我们将展示分析处理后的部分数据,具体数据见压缩包。

\section*{1.4 本文的架构设计}

文章的结构如下:

- 第一部分:前言,主要讲述了粮食最低收购价相关问题的历史背景和研究意义,并对问题的本质核心进行了思考和归纳。除此以外,我们在阅读国内外相关问题的研究文献的基础上,还对国内外的研究现状进行了的分析和概括,确立了本文研究问题的方向和目标。然后我们确定了研究本问题所涉及的区域以及指标,搜集并整理了本文所需要的一系列数据。最后对必要的符号进行了说明并且给出了本文的结构框架。

- 第二部分:问题一解答。我们选取了影响区域特征的指标,分别对小麦的 6 个主产省份和水稻的 11 个主产省份进行了聚类分析,在小麦主产区选出了两个有代表性的省份,在水稻主产区选出了三个具有代表性的省份。接着分别对选取的两个小麦代表省份和三个水稻代表省份进行 Spearman 相关系数检验,用统计分析的方法确定了影响这五个省份的相应的指标,进而利用所选取的指标建立了主成分回归模型,得出的相应结果的可信度和可靠性给出检验和分析。

- 第三部分:问题二的解答。根据粮食品种与区域的差异,本文根据问题 1 中建立的指标体系,运用主成分-统计控制的方法分别对安徽省和山东省的小麦和吉林省、湖南省与江苏省的水稻进行粮食最低收购价政策执行效果的评价。

- 第四部分:问题三的解答。首先作图定性分析粮食最低收购价价格与粮食市场价格的趋势对比,接着运用市场收购价理论和局部调整模型分别建立了包含粮食最低收购价在内的粮食价格形成模型,包括供应量模型、企业收购量模型和市场收购价格模型。运用历史数据分别对安徽、山东小麦和吉林、江苏、湖南水稻价格形成模型进行实证分析。

- 第五部分:问题四的解答。以两个小麦代表省份和三个水稻代表省份为整体,分别建立两个品种的粮食最低收购价合理定价线性规划模型。首先建立理论定价模型。以粮食总供给量最大为目标,分别以价格形成平衡模型、财政补贴预算、库存限制、土地面积限制为约束建立模型。并将该模型分别应用在 5 个小麦和水稻上。由于篇幅和实践的限制,小麦还是选用安徽和山东作为全集暂时代替全国。水稻还是选用吉林、湖南、江苏作为全集暂时代替全国进行优化模型的计算。

- 第六部分:问题五的解答。运用第四问的小麦优化模型,假定这两个省份作为全国,不断调整小麦最低收购价,看能否是的小麦种植面积相比于 2000 年提高 5\%。

- 第七部分:问题六的解答。结合前面的研究结论,写出有针对性的对策与建议。

- 第八部分:可能存在的创新和不足。

\begin{figure}[h]
    \centering
    \includegraphics[width=\textwidth]{research_framework.png}
    \caption{研究框架图}
    \label{fig:research_framework}
\end{figure}

具体研究思路见下图。

\section*{第 2 章 问题一解答}

\section*{——粮食种植面积的指标体系和数学模型的建立}

在本章中,我们首先选取了影响区域特征的指标,分别对小麦的 6 个主产省份和水稻的 11 个主产省份进行了聚类分析,在小麦主产区选出了两个有代表性的省份,在水稻主产区选出了三个具有代表性的省份。接着分别对选取的两个小麦代表省份和三个水稻代表省份进行 Spearman 相关系数检验,用统计分析的方法确定了影响这五个省份的相应的指标,进而利用所选取的指标建立了主成分回归模型,得出的相应结果的可靠性的检验和分析。

\subsection*{2.1 模型的符号说明}

以下符号所代表的指标是 1.3 节所描述的数据,是本文所有问题研究的基础指标。随着研究问题的深入,我们又定义了一些新的指标变量,为了方便读者阅读,我们将那些指标的符号说明呈现于相应章节中。

\begin{tabular}{ll}
\hline 符号 & 符号说明 \\
\hline $x_{1}$ & 农业劳动人口 \\
$x_{2}$ & 农民受教育程度 \\
$x_{3}$ & 城乡收入差距 \\
$x_{4}$ & 家庭负担 \\
$x_{5}$ & 小麦(水稻)的市场收购价 \\
$x_{6}$ & 小麦(水稻)的生产成本 \\
$x_{7}$ & 小麦(水稻)的相对收益竞争力 \\
$x_{8}$ & 农作物播种面积 \\
$x_{9}$ & 受灾情况 \\
$x_{10}$ & 小麦(水稻)的净出口 \\
$x_{11}$ & 城市化水平 \\
$x_{12}$ & 最低收购价政策 \\
$y$ & 小麦(水稻)的种植面积 \\
\hline
\end{tabular}

\subsection*{2.2 对小麦和水稻分别进行聚类分析}

我们在本小节对小麦的 6 个主产省份和水稻的 11 个主产省份进行了聚类分析,旨在选出能代表不同区域的省份,使得整个分析过程更加简洁高效。

\subsubsection{2.2.1 能够衡量区域特征的指标选取与计算}

为了能通过聚类分析划分省份所在的不同区域,我们就选取能够体现地域特征的指标进行分析。

我们可以粗略地认为,在一定的区域内,某个省份的经济发展水平、人口结构分布、气候状况在整体上是相近的。基于此分析,在不断尝试之后,选择了以下指标作为体现地域特征的指标:

农业劳动人口、家庭负担、小麦(水稻)的市场收购价、小麦(水稻)的生产成本、受灾情况、城市化水平

\begin{table}
\centering
\caption{表2.1小麦主产省份聚类数据}
\begin{tabular}{l l l l l l l l}
 & x1 & x4 & x5 & x6 & x9 & x11 & x13 \\
\hline
江苏 & 1485.761 & 4722.213 & 77.8945 & 451.787 & 3.586586 & 0.471973 & 0.260758 \\
山东 & 2504.97 & 3420.347 & 1584.045 & 446.426 & 2.963689 & 0.415813 & 0.331485 \\
河北 & 1563.466 & 2872.448 & 77.8945 & 451.787 & 1.897291 & 0.335262 & 0.280366 \\
湖北 & 1653.991 & 4758.935 & 77.8945 & 451.787 & 2.211994 & 0.428287 & 0.128421 \\
河南 & 2988.6 & 2572.821 & 77.8945 & 451.787 & 3.298543 & 0.30465 & 0.36341 \\
安徽 & 1774.965 & 4390.402 & 1488.4 & 337.832 & 7.696478 & 0.34503 & 0.245126 \\
\end{tabular}
\end{table}

\begin{table}
\centering
\caption{表2.2水稻主产省份聚类数据}
\begin{tabular}{l l l l l l l l}
 & x1 & x4 & x5 & x6 & x9 & x11 & x13 \\
\hline
江苏 & 1485.761 & 4722.213 & 1653.67 & 487.778 & 3.586586 & 0.471973 & 0.286549 \\
四川 & 2441.445 & 2623.127 & 1581.625 & 528.2321 & 2.367618 & 0.315902 & 0.219196 \\
广西 & 1556.621 & 2456.372 & 1581.625 & 528.2321 & 1.950559 & 0.314763 & 0.368301 \\
湖南 & 1877.028 & 3066.744 & 1470.389 & 443.7958 & 2.246343 & 0.354118 & 0.481201 \\
江西 & 931.0474 & 2685.125 & 1581.625 & 528.2321 & 2.79006 & 0.342258 & 0.562902 \\
黑龙江 & 832.7316 & 2860.172 & 1581.625 & 528.2321 & 2.178037 & 0.542014 & 0.179201 \\
吉林 & 566.9789 & 2841.387 & 1829.9 & 477.2126 & 1.946288 & 0.465493 & 0.125951 \\
辽宁 & 1062.209 & 3003.414 & 1581.625 & 528.2321 & 1.743607 & 0.556567 & 0.149801 \\
湖北 & 1653.991 & 4758.935 & 1636.044 & 555.734 & 2.211994 & 0.428287 & 0.277464 \\
河南 & 2988.6 & 2572.821 & 1636.044 & 555.734 & 3.298543 & 0.30465 & 0.040204 \\
安徽 & 1774.965 & 4390.402 & 1636.044 & 555.734 & 7.696478 & 0.34503 & 0.24822 \\
\end{tabular}
\end{table}

\begin{figure}[h]
    \centering
    \includegraphics[width=\textwidth]{image1.png}
    \caption{小麦聚类图}
    \label{fig:1}
\end{figure}

从图中可以看出,我们把省份分为两类,其中第一类为 2、3、5,所对应的省份为:山东、河北、河南;第二类为:1、4、6,所对应的省份为:江苏、湖北、安徽。它们在地图上的方位显示如下图:

\begin{figure}[h]
    \centering
    \includegraphics[width=\textwidth]{image2.png}
    \caption{小麦主产省份分布图}
    \label{fig:2}
\end{figure}

从图中各个省份的位置以及聚类结果来看,第一类主要分布在华北区域,我们选择山东作为该地区代表;第二类主要分布在华中区域,我们选取安徽作为该地区代表。

(2)对水稻主产省聚类

同样基于以上数据,我们编写 Matlab 程序,将程序文件 ‘jvleifenxid’ 和数据文件 ‘sdshengfen’ 放在同一目录下,使用 Matlab R2015b 进行运行,所得聚类图如下:

\begin{figure}[h]
    \centering
    \includegraphics[width=\textwidth]{image3.png}
    \caption{水稻主产省聚类图}
    \label{fig:3}
\end{figure}

\begin{figure}[h]
    \centering
    \includegraphics[width=\textwidth]{image1.png}
    \caption{水稻聚类图}
    \label{fig:rice_clustering}
\end{figure}

从图中可以看出,我们把省份分为两类,其中第一类为 1、9、11、10,所对应的省份为:江苏、湖北、河南、安徽;第二类为:6、8、7,所对应的省份为:黑龙江、吉林、辽宁。第三类为:4、5、3、2,所对应的省份为:湖南、江西、四川、广西。它们在地图上的方位显示如下图:

\begin{figure}[h]
    \centering
    \includegraphics[width=\textwidth]{image2.png}
    \caption{水稻主产省份分布图}
    \label{fig:rice_provinces}
\end{figure}

从图中各个省份的位置以及聚类结果来看,第一类主要分布在华中区域,我们选择江苏作为该地区代表;第二类主要分布在东北区域,我们选取吉林作为该地区代表;第三类主要分布在华南区域,我们选取湖南作为该地区代表。

通过以上聚类分析,我们从 6 个小麦主产省中选择了分别代表华北和华中的山东和安徽作为代表;从 11 个水稻主产省中选择了分别代表华中、东北和华南区域的江苏省、吉林省和湖南省。使得我们对粮食最低收购价政策在不同粮食品种,不同区域的分析变得清晰明了,为进一步的建模分析做了铺垫。

\section{斯皮尔曼等级相关系数的检验}

在本小节中,我们对上节所选的五个省份分别进行了斯皮尔曼相关系数检验,从统计学意义上对每个省份的指标进一步进行筛选,确定影响每个省份的粮食种植面积的相关性指标。最后,我们对所选择的指标进行了合理性评价。

### 2.3.1 斯皮尔曼等级相关系数的检验原理

斯皮尔曼(Spearman)相关系数是描述两组变量之间是否存在着相同或相反趋同性的一种指标,由于该检验不需要假定服从正态分布,仅需要确定变量在每个点(时期)上的等级即可获得,因此具有较好的性质。在两组数据都没有重复观测值的情况下,斯皮尔曼等级相关系数的公式为:
\[
r_{SP} = 1 - \frac{6 \sum d_i^2}{n(n^2 - 1)}
\]
其中 $d_i$ 表示两组数据的等级之差,$n$ 为样本量。

### 2.3.2 对小麦进行 Spearman 检验及合理性评价

#### (1)对安徽省小麦的检验

运用 SAS 软件,得到小麦种植面积与各指标间的斯皮尔曼检验结果如表 2.2。

**表 2.3 斯皮尔曼检验结果**

| 变量             | $r_{SP}$   | $t$ 检验 $P$ 值 |
|------------------|------------|----------------|
| 农业劳动力人口   | -0.77143   | $<0.0001$     |
| 农民受教育程度   | 0.65564    | 0.0017        |
| 家庭负担         | 0.74135    | 0.0002        |
| 小麦的市场收购价 | 0.9188     | $<0.0001$     |
| 小麦生产成本     | 0.9218     | $<0.0001$     |
| 农作物播种面积   | 0.81053    | $<0.0001$     |
| 受灾情况         | 0.60474    | 0.0047        |
| 城市化水平       | 0.70827    | 0.0005        |
| 最低收购价政策   | 0.65361    | 0.0018        |

由以上斯皮尔曼检验的结果可以看出,安徽小麦种植面积与农业劳动人口、家庭负担、小麦的市场收购价、小麦生产成本、农作物播种面积、受灾情况、城市化水平和最低收购价政策具有相关性。在之后的问题中,我们将使用这 9 个指标作为影响安徽小麦种植面积的主要因素进行建模分析。

#### (2)对山东省小麦的检验

运用 SAS 软件,得到小麦种植面积与各指标间的斯皮尔曼检验结果如表 2.2。

**表 2.4 斯皮尔曼检验结果**

| 变量             | $r_{SP}$   | $t$ 检验 $P$ 值 |
|------------------|------------|----------------|
| 农业劳动力人口   | 0.46667    | 0.044          |
| 农民受教育程度   | -0.58421   | 0.0086         |
| 城乡收入差距     | -0.53509   | 0.0182         |
| 家庭负担         | -0.55614   | 0.0134         |
| 小麦的净出口     | -0.507     | 0.0267         |
| 城市化水平       | -0.5491    | 0.0149         |
| 最低收购价政策   | -0.5509    | 0.0145         |

由以上斯皮尔曼检验的结果可以看出,山东小麦种植面积与农业劳动人口、农民受教育程度、城乡收入差距、家庭负担、小麦的净出口、城市化水平和最低收购价政策具有相关性,而且关系非常紧密。在之后的问题中,我们将使用这 7 个指标作为影响山东小麦种植面积的主要因素进行建模分析。

### 2.3.3 对水稻进行 Spearman 检验

#### (1)对湖南省水稻的检验

运用 SAS 软件,得到水稻种植面积与各指标间的斯皮尔曼检验结果如表 2.2

**表 2.5 斯皮尔曼检验结果**

| 变量               | $r_{SP}$   | $t$ 检验 P 值 |
|--------------------|------------|---------------|
| 农业劳动人口       | -0.52331   | 0.0179        |
| 城乡收入差距       | -0.4782    | 0.033         |
| 水稻的市场收购价   | 0.66015    | 0.0015        |
| 水稻生产成本       | 0.64511    | 0.0021        |
| 水稻的相对收益竞争力 | 0.48421    | 0.0305        |
| 水稻的净出口       | -0.71278   | 0.0004        |

由以上斯皮尔曼检验的结果可以看出,安徽水稻种植面积与农业劳动人口、城乡收入差距、水稻的市场收购价、水稻生产成本、水稻相对收益竞争力和水稻净出口具有相关性,而且关系非常紧密。在之后的问题中,我们将使用这 6 个指标作为影响安徽水稻种植面积的主要因素进行建模分析。

#### (2)对吉林省水稻的检验

运用 SAS 软件,得到水稻种植面积与各指标间的斯皮尔曼检验结果如表 2.2

**表 2.6 斯皮尔曼检验结果**

| 变量               | $r_{SP}$   | $t$ 检验 P 值 |
|--------------------|------------|---------------|
| 农民受教育程度     | 0.84511    | $<.0001$      |
| 家庭负担           | 0.85865    | $<.0001$      |
| 水稻的市场收购价   | 0.56187    | 0.0099        |
| 水稻生产成本       | 0.63459    | 0.0027        |
| 农作物播种面积     | 0.89774    | $<.0001$      |
| 城市化水平         | 0.85865    | $<.0001$      |
| 最低收购价政策     | 0.69369    | 0.0007        |

由以上斯皮尔曼检验的结果可以看出,吉林水稻种植面积与农民受教育程度、家庭负担、水稻的市场收购价、水稻的生产成本、农作物播种面积、城市化水平和最低收购价政策具有相关性,而且关系非常紧密。在之后的问题中,我们将使用这 7 个指标作为影响吉林水稻种植面积的主要因素进行建模分析。

#### (3)对江苏省水稻的检验

运用 SAS 软件,得到水稻种植面积与各指标间的斯皮尔曼检验结果如表 2.2

\begin{table}
\centering
\caption{斯皮尔曼检验结果}
\begin{tabular}{l c c}
\hline
变量 & $r_{SP}$ & $t$检验P值 \\
\hline
城乡收入差距 & -0.75113 & 0.002855 \\
水稻的市场收购价 & 0.79699 & 0.003907 \\
水稻生产成本 & 0.64211 & 0.001213 \\
水稻的相对收益竞争力 & 0.84662 & 0.005185 \\
农作物播种面积 & 0.61203 & 0.00091 \\
受灾情况 & 0.72546 & 0.002414 \\
最低收购价政策 & 0.75458 & 0.00297 \\
\hline
\end{tabular}
\end{table}

由以上斯皮尔曼检验的结果可以看出, 江苏水稻种植面积与城乡收入差距、水稻的市场收购价、水稻的生产成本、水稻的相对收益竞争力、农作物播种面积、受灾情况和最低收购价政策具有相关性, 而且关系非常紧密。在之后的问题中, 我们将使用这 7 个指标作为影响江苏水稻种植面积的主要因素进行建模分析。

\subsubsection{合理性评价}

从定性分析的角度来看, 这些指标都是建立在经济学理论的基础上进行筛选的, 具有其合理性。

根据统计学的相关知识, 从定量分析的角度来看, 这些指标的选取也是十分合理的。因为一方面, 从 $t$ 检验P值来看, 所选择的指标的P值远远小于 0.05; 另一方面, 从 $r_{SP}$ 的值来看, 所选择的指标与粮食种植面积的相关性大部分在 0.8 以上, 说明这些指标大部分与粮食种植面积的相关性还是很强的, 即指标的选取是合理的。

\subsection{主成分回归模型的建立}

在本节, 我们分别对两个省的小麦和三个省的水稻建立了主成分回归模型。该模型的建立分两部分完成, 第一部分是在上一节 Spearman 相关性检验的基础上, 对这五个省的通过检验的各指标进行主成分分析, 来消除指标间的多重共线性, 提高了指标的可靠性和可信度; 第二部分是对第一部分所得出的主成分选择信息量进行回归模型的建立, 成功地建立影响粮食种植面积的指标体系和关于粮食种植面积的数学模型。

\subsubsection{主成分分析}

利用 SAS 软件编程, 对于所选出来的省份进行主成分分析, 得出了各个省份的主成分方差共线表。具体分析结果如下。

(1) 安徽省小麦的主成分分析结果如下:

对通过 spearman 相关性检验的 8 个变量进行主成分分析。

\begin{table}
\centering
\caption{各主成分的方差贡献表}
\begin{tabular}{c c c c}
\hline
 & 特征值 & 方差贡献率 & 累计贡献率 \\
\hline
1 & 6.45864553 & 0.8073 & 0.8073 \\
2 & 0.89162175 & 0.1115 & 0.9188 \\
3 & 0.31737241 & 0.0397 & 0.9585 \\
4 & 0.18495932 & 0.0231 & 0.9816 \\
5 & 0.06975104 & 0.0087 & 0.9903 \\
6 & 0.05331258 & 0.0067 & 0.997 \\
7 & 0.01489736 & 0.0019 & 0.9988 \\
8 & 0.00943999 & 0.0012 & 1 \\
\hline
\end{tabular}
\end{table}

上表为安徽省小麦主成分分析的方差贡献表,可见前两个主成分的累计贡献率达到了91.88%,即前两个主成分所包含的信息占全部信息的91.88%,因此选择前两个主成分即可。各变量在前两个主成分中的系数见下表。

\begin{table}
\centering
\caption{各变量的系数}
\begin{tabular}{c c c}
\hline
变量 & Prin 1 & Prin 2 \\
\hline
x1 & -0.383555 & 0.010069 \\
x2 & 0.3719 & -0.106931 \\
x4 & 0.362478 & -0.280319 \\
x5 & 0.363001 & -0.118951 \\
x6 & 0.377341 & -0.173674 \\
x8 & 0.360402 & 0.246421 \\
x9 & 0.197525 & 0.896566 \\
x11 & 0.374331 & -0.031929 \\
\hline
\end{tabular}
\end{table}

上表为8个变量在前两个主成分中的系数,可以看出,第一主成分除x9外,受各个指标的影响比均衡;第二主成分主要由x9决定。

(2) 山东省小麦的主成分分析结果如下:

对通过 spearman 相关性检验的6个变量进行主成分分析。

\begin{table}
\centering
\caption{山东省各主成分的方差贡献表}
\begin{tabular}{c c c c}
\hline
变量 & 特征值 & 方差贡献率 & 累计贡献率 \\
\hline
1 & 4.18922225 & 0.6982 & 0.6982 \\
2 & 1.12213828 & 0.187 & 0.8852 \\
3 & 0.49458305 & 0.0824 & 0.9677 \\
4 & 0.13905968 & 0.0232 & 0.9908 \\
5 & 0.03377084 & 0.0056 & 0.9965 \\
6 & 0.02122591 & 0.0035 & 1 \\
\hline
\end{tabular}
\end{table}

上表为山东省小麦主成分分析的方差贡献表,可见前三个主成分的累计贡献率达到了96.77%,即前三个主成分所包含的信息占全部信息的96.77%,因此选择前三个主成分即可。各变量在前三个主成分中的系数见下表。

\begin{table}
\centering
\caption{各变量的系数}
\begin{tabular}{c c c c}
变量 & Prin1 & Prin2 & Prin3 \\
\hline
x1 & -0.4616 & 0.179696 & -0.049248 \\
x2 & 0.475883 & -0.001438 & 0.096718 \\
x3 & 0.372183 & 0.304353 & -0.788684 \\
x4 & 0.42898 & -0.330582 & 0.413963 \\
x10 & 0.101198 & 0.873678 & 0.441366 \\
x11 & 0.477154 & 0.049786 & 0.0053 \\
\end{tabular}
\end{table}

\begin{table}
\centering
\caption{各主成分的方差贡献表}
\begin{tabular}{c c c c}
变量 & 特征值 & 方差贡献率 & 累计贡献率 \\
\hline
1 & 5.63743396 & 0.8053 & 0.8053 \\
2 & 0.84067884 & 0.1201 & 0.9254 \\
3 & 0.28520088 & 0.0407 & 0.9662 \\
4 & 0.10834542 & 0.0155 & 0.9817 \\
5 & 0.08339242 & 0.0119 & 0.9936 \\
6 & 0.0410428 & 0.0059 & 0.9994 \\
7 & 0.00390568 & 0.0006 & 1 \\
\end{tabular}
\end{table}

\begin{table}
\centering
\caption{各变量的系数}
\begin{tabular}{c c c}
变量 & Prin1 & Prin2 \\
\hline
x2 & 0.378209 & 0.357707 \\
x4 & 0.411831 & 0.01919 \\
x5 & 0.396412 & -0.258406 \\
x6 & 0.404728 & -0.028077 \\
x8 & 0.370625 & 0.338526 \\
x10 & -0.259975 & 0.817835 \\
x11 & 0.401816 & 0.143743 \\
\end{tabular}
\end{table}

\begin{table}
\centering
\caption{湖南省各主成分的方差贡献表}
\begin{tabular}{c c c c}
\hline
 & 特征值 & 方差贡献率 & 累计贡献率 \\
\hline
1 & 4.08562521 & 0.5837 & 0.5837 \\
2 & 1.56918126 & 0.2242 & 0.8078 \\
3 & 0.59833429 & 0.0855 & 0.8933 \\
4 & 0.37742346 & 0.0539 & 0.9472 \\
5 & 0.29915106 & 0.0427 & 0.99 \\
6 & 0.05154221 & 0.0074 & 0.9973 \\
7 & 0.01874251 & 0.0027 & 1 \\
\hline
\end{tabular}
\end{table}

\begin{table}
\centering
\caption{各变量的系数}
\begin{tabular}{c c c}
\hline
变量 & Prin1 & Prin2 \\
\hline
x1 & -0.447196 & 0.205596 \\
x3 & -0.082946 & -0.701722 \\
x5 & 0.483703 & -0.032892 \\
x6 & 0.479279 & 0.035098 \\
x7 & 0.169988 & 0.62144 \\
x8 & 0.42314 & 0.052167 \\
x10 & -0.348588 & 0.272203 \\
\hline
\end{tabular}
\end{table}

\begin{table}
\centering
\caption{江苏省各主成分的方差贡献表}
\begin{tabular}{c c c c}
\hline
变量 & 特征值 & 方差贡献率 & 累计贡献率 \\
\hline
1 & 2.78538679 & 0.5571 & 0.5571 \\
2 & 1.23211696 & 0.2464 & 0.8035 \\
3 & 0.60881709 & 0.1218 & 0.9253 \\
4 & 0.3376894 & 0.0675 & 0.9928 \\
5 & 0.03598977 & 0.0072 & 1 \\
\hline
\end{tabular}
\end{table}

\begin{table}
\centering
\begin{tabular}{c c c}
\hline
变量 & Prin1 & Prin2 \\
\hline
x3 & -0.568463 & -0.178623 \\
x6 & -0.474468 & 0.244546 \\
x8 & 0.510118 & 0.146289 \\
x9 & 0.220627 & 0.735323 \\
x10 & 0.377933 & -0.58838 \\
\hline
\end{tabular}
\caption{各变量的系数}
\end{table}

上表为5个变量在前两个主成分中的系数,可以看出,第一主成分主要由x3、x6和x8组成;第二主成分主要由x9和x10决定。

\subsection{2.3.2 指标的可靠性和可信度评价}

从以上主成分分析的结果来看,所选取的两个主成分中的第一个主成分包含了大部分指标的信息,其余的信息则主要在第二个主成分中体现出来。对指标选取不同的主成分后消除了指标之间的多重共线性,使得指标的可靠性和可信度更好。

本节的主成分分析的结果对于后文主成分回归模型的建立以及在问题二中对各省粮食的种植面积进行的统计控制评价分析做了铺垫。

\subsection{2.4 主成分回归模型的建立}

本节利用主成分回归的方法,得出不同的指标对粮食种植面积的弹性值,从而从经济效应的角度得出各项指标对粮食种植面积的影响,用以验证粮食最低收购价的合理性。

\subsubsection{2.4.1 模型的基本假设及模型的建立}

\subsubsection*{2.4.1.1 模型假设}

(1) 假定所选择的影响因素对对粮食种植面积综合影响等于各要素对产出单独作用的总和。未选择的投入要素通常是一些固定的或不可控的生产要素,这些要素的作用可以用生产函数模型中的常数项来体现。

(2) 假定不考虑气候、自然灾害对粮食产量的影响。事实上,粮食生产受自然条件的影响很大。

(3) 假定不考虑制度变迁、政策等的变化。

(4) 假定在WTO规则下,中国政府所能采取的要素补贴手段为粮食最低收购价政策。

\subsubsection*{2.4.1.2 模型的建立}

根据各指标的经济学含义以及数据特征,我们建立线性回归模型形式如下:
\[
Y = \alpha + \beta_1 X_1 + \beta_2 X_2 + \cdots + \beta_n X_n
\]

\subsection{2.4.2 指标的选择与数据诊断}

在本节中,我们首先根据之前的运算结果选择了五个省份的影响粮食种植面积的指标,然后对每个省份的指标进行了数据诊断,得出了共线性表现比较强的一些指标,为下一节的主成分回归模型的求解做了铺垫。

\begin{table}
\centering
\caption{五个省选取的指标}
\begin{tabular}{c c c c c c}
\hline
变量 & 安徽 & 山东 & 江苏 & 吉林 & 湖南 \\
\hline
$x_{1}$ & 农业劳动人口 & 选择 & 选择 & & \\
$x_{2}$ & 农民受教育程度 & 选择 & 选择 & 选择 & 选择 \\
$x_{3}$ & 城乡收入差距 & & 选择 & 选择 & \\
$x_{4}$ & 家庭负担 & 选择 & 选择 & 选择 & 选择 \\
$x_{5}$ & 市场收购价 & 选择 & & 选择 & 选择 \\
$x_{6}$ & 生产成本 & 选择 & & 选择 & 选择 \\
$x_{7}$ & 相对收益竞争力 & & & & \\
$x_{8}$ & 农作物播种面积 & 选择 & 选择 & 选择 & 选择 \\
$x_{9}$ & 受灾情况 & 选择 & 选择 & & \\
$x_{10}$ & 净出口 & & 选择 & 选择 & 选择 \\
$x_{11}$ & 城市化水平 & 选择 & 选择 & 选择 & 选择 \\
\hline
\end{tabular}
\end{table}

\subsubsection{以上各省的数据诊断}

(1) 安徽小麦的数据诊断:

用 SAS 软件对 $x_{1}, x_{2}, x_{4}, x_{5}, x_{6}, x_{8}, x_{9}, x_{11}$ 其作共线性诊断,见表 2.19。

\begin{table}
\centering
\caption{影响安徽小麦种植面积的各指标之间的共线性诊断结果}
\begin{tabular}{c c c c c c c c c c}
\hline
\multirow{2}{*}{ 个数 } & \multirow{2}{*}{ 特征值 } & \multirow{2}{*}{ 条件指数 } & \multicolumn{7}{c}{ 偏差比例 } \\
\cline{4-10}
& & & $x1$ & $x2$ & $x4$ & $x5$ & $x6$ & $x8$ & $x9$ & $x11$ \\
\hline
1 & 6.45 & 1 & 0.00058 & 0.00063 & 0.001 & 0.0007 & 0.0005 & 0.0015 & 0.003 & 0.0005 \\
2 & 0.89 & 2.69 & 0.0000 & 0.00038 & 0.007 & 0.0005 & 0.0007 & 0.005 & 0.47 & 0.00002 \\
3 & 0.32 & 4.51 & 0.002 & 0.018 & 0.013 & 0.04 & 0.005 & 0.00084 & 0.035 & 0.0203 \\
4 & 0.18 & 5.91 & 0.008 & 0.018 & 0.075 & 0.003 & 0.0013 & 0.203 & 0.37 & 0.0002 \\
5 & 0.069 & 9.62 & 0.101 & 0.039 & 0.34 & 0.004 & 0.105 & 0.005 & 0.007 & 0.00043 \\
6 & 0.053 & 11.0 & 0.095 & 0.016 & 0.42 & 0.15 & 0.014 & 0.324 & 0.006 & 0.00084 \\
7 & 0.014 & 20.8 & 0.038 & 0.718 & 0.006 & 0.021 & 0.02 & 0.2 & 0.038 & 0.89 \\
8 & 0.009 & 26.2 & 0.756 & 0.188 & 0.1327 & 0.78 & 0.852 & 0.26 & 0.075 & 0.087 \\
\hline
\end{tabular}
\end{table}

表最后一行的条件指数 (Condition Index) 大于 10,说明自变量间存在较强的共线性,从该行的偏差比例 (Proportion of Variation) 大于 0.5 的数值可知:自变量间的共线性主要表现在 $x_{1}, x_{5}, x_{6}$ 上。

(2) 山东小麦的数据诊断:

用 SAS 软件对 $x_{1}, x_{2}, x_{3}, x_{4}, x_{10}, x_{11}$ 其作共线性诊断,见表 2.20。

\begin{table}
\centering
\caption{影响山东小麦种植面积的各指标之间的共线性诊断结果}
\begin{tabular}{c c c c c c c c c}
\hline
个数 & 特征值 & 条件 & 偏差比例 & & & & & \\
 & & 指数 & x1 & x2 & x3 & x4 & x10 & x11 \\
\hline
1 & 4.20596 & 1.00000 & 0.00378 & 0.00330 & 0.00412 & 0.00377 & 0.00779 & 0.00825 \\
2 & 0.88275 & 2.18280 & 0.00358 & 0.00000581 & 0.00679 & 0.00264 & 0.65757 & 0.00239 \\
3 & 0.59440 & 2.66007 & 0.00001263 & 0.0009321 & 0.11599 & 0.01689 & 0.03423 & 0.08801 \\
4 & 0.19379 & 4.65874 & 0.06355 & 0.01946 & 0.02866 & 0.11490 & 0.14872 & 0.41557 \\
5 & 0.09684 & 6.59043 & 0.27218 & 0.2638 & 0.0141 & 0.10455 & 0.02702 & 0.20026 \\
6 & 0.02627 & 12.65430 & 0.65690 & 0.7125 & 0.83034 & 0.75726 & 0.12465 & 0.28552 \\
\hline
\end{tabular}
\end{table}

表最后一行的条件指数(Condition Index) 大于 10,说明自变量间存在较强的共线性,从该行的偏差比例(Proportion of Variation) 大于 0.5 的数值可知:自变量间的共线性主要表现在 \(x_1, x_2, x_3, x_4\) 上。

(3) 江苏水稻的数据诊断:用 SAS 软件对 \(x_3, x_6, x_8, x_9, x_{10}\) 其作共线性诊断,见表 2.21。

\begin{table}
\centering
\caption{影响江苏水稻种植面积的各指标之间的共线性诊断结果}
\begin{tabular}{c c c c c c c c}
\hline
个数 & 特征值 & 条件 & 偏差比例 & & & & \\
 & & 指数 & x3 & x6 & x8 & x9 & x10 \\
\hline
1 & 2.78539 & 1.00000 & 0.00733 & 0.01903 & 0.00948 & 0.01101 & 0.02500 \\
2 & 1.23212 & 1.50355 & 0.00164 & 0.01143 & 0.00176 & 0.27652 & 0.13698 \\
3 & 0.60882 & 2.13894 & 0.00411 & 0.12447 & 0.06455 & 0.23739 & 0.01744 \\
4 & 0.33769 & 2.87200 & 0.00912 & 0.20608 & 0.00104 & 0.32188 & 0.69261 \\
5 & 0.03599 & 18.79738 & 0.97779 & 0.63898 & 0.92317 & 0.15320 & 0.12797 \\
\hline
\end{tabular}
\end{table}

表最后一行的条件指数(Condition Index) 大于 10,说明自变量间存在共线性,从该行的偏差比例(Proportion of Variation) 大于 0.5 的数值可知:自变量间的共线性主要表现在 \(x_3, x_6, x_8\) 上。

(4) 吉林水稻的数据诊断:用 SAS 软件对 \(x_2, x_4, x_5, x_6, x_8, x_{10}, x_{11}\) 其作共线性诊断,见表 2.22。

\begin{table}
\centering
\caption{影响吉林水稻种植面积的各指标之间的共线性诊断结果}
\begin{tabular}{c c c c c c c c c}
\hline
\multirow{2}{*}{ 个数 } & \multirow{2}{*}{ 特征值 } & \multirow{2}{*}{ 条件指数 } & \multicolumn{6}{c}{ 偏差比例 } \\
 & & & x2 & x4 & x5 & x6 & x8 & x10 & x11 \\
\hline
1 & 5.63743 & 1.00000 & 0.00308 & 0.00021185 & 0.00187 & 0.00025931 & 0.00191 & 0.00368 & 0.00215 \\
2 & 0.84068 & 2.58956 & 0.01848 & 0.00000308 & 0.00533 & 0.00000837 & 0.01070 & 0.24410 & 0.00185 \\
3 & 0.28520 & 4.44596 & 0.05254 & 0.00149 & 0.01225 & 0.00541 & 0.09390 & 0.20035 & 0.01685 \\
4 & 0.10835 & 7.21332 & 0.02672 & 0.01440 & 0.11881 & 0.01769 & 0.02271 & 0.00465 & 0.21866 \\
5 & 0.08339 & 8.22200 & 0.84008 & 0.00014720 & 0.01008 & 0.00200 & 0.31486 & 0.02587 & 0.04373 \\
6 & 0.04104 & 11.71986 & 0.00895 & 0.00012652 & 0.85029 & 0.00061930 & 0.02851 & 0.51230 & 0.71471 \\
7 & 0.00391 & 37.99204 & 0.05015 & 0.98362 & 0.00137 & 0.97401 & 0.52741 & 0.00905 & 0.00206 \\
\hline
\end{tabular}
\end{table}

表2最后一行的条件指数(Condition Index) 远大于10,说明自变量间存在较强的共线性,从该行的偏差比例(Proportion of Variation) 大于0.5的数值可知:自变量间的共线性主要表现在$x_4, x_6, x_{10}$上。

(5) 湖南水稻的数据诊断:

\begin{table}
\centering
\caption{影响湖南水稻种植面积的各指标之间的共线性诊断结果}
\begin{tabular}{c c c c c c c c c}
\hline
\multirow{2}{*}{ 个数 } & \multirow{2}{*}{ 特征值 } & \multirow{2}{*}{ 条件指数 } & \multicolumn{6}{c}{ 偏差比例 } \\
 & & & x2 & x4 & x5 & x6 & x8 & x11 \\
\hline
1 & 5.29088 & 1.00000 & 0.00159 & 0.00045144 & 0.00146 & 0.00045200 & 0.00866 & 0.00131 \\
2 & 0.35125 & 3.88111 & 0.03255 & 0.00005297 & 0.00120 & 0.00002300 & 0.63480 & 0.01352 \\
3 & 0.29134 & 4.26149 & 0.04304 & 0.00186 & 0.03598 & 0.00936 & 0.24746 & 0.01015 \\
4 & 0.03557 & 12.19529 & 0.02229 & 0.10400 & 0.53329 & 0.04956 & 6.057505E-7 & 0.16386 \\
5 & 0.02399 & 14.84941 & 0.77793 & 0.00691 & 0.24705 & 0.00781 & 2.981236E-7 & 0.80967 \\
6 & 0.00695 & 27.58480 & 0.12260 & 0.88673 & 0.18104 & 0.93279 & 0.10908 & 0.00149 \\
\hline
\end{tabular}
\end{table}

表最后一行的条件指数(Condition Index) 大于10,说明自变量间存在较强的共线性,从该行的偏差比例(Proportion of Variation) 大于0.5的数值可知:自变量间的共线性主要表现在$x_4, x_6$上。

诊断结果说明,对于以上五个省的数据诊断,我们确定了每个省自变量间的共线性主要表现的指标,说明了建立主成分回归模型的必要性。

\subsection*{2.4.3 回归模型的求解与模型解释}

在本节,我们在2.3.1节主成分分析运算结果的基础上,我们我们建立了主成分回归模型,并对结果进行了分析。

\subsection*{2.4.3.1 安徽小麦的主成分回归模型的求解与模型解释}

(1) 主成分回归求解

为了得出安徽小麦种植面积对农业劳动人口、农民受教育程度、家庭负担、市场价格、生产成本、农作物播种面积、受灾面积、城市化水平的C-D函数关系,首先必须得出小麦种植面积 \( y \) 对Prin1-Prin8的多元线性回归函数。由于主成分是 \( x_1, x_2, x_4, x_5, x_6, x_8, x_9, x_{11} \) 的线性函数,因此可以得到 \( y \) 与 \( x_1, x_2, x_4, x_5, x_6, x_8, x_9, x_{11} \) 之间的回归方程。运用SAS软件的REG模块以及pcomit=1, 2, 3, 4, 5, 6, 7 outvif 语句进行计算,得到主成分回归、最终回归模型及其检验结果,见下表。

\begin{table}[h]
\centering
\caption{主成分回归系数及模型检验}
\begin{tabular}{l c c c c c c}
\hline
变量 & 自由度 & 参数 & 标准 & t 值 & Pr > |t| & 标准化 \\
 & & 估计 & 误差 & & & 估计 \\
\hline
Intercept & 1 & 2134.82055 & 15.53275 & 137.44 & <.0001 & 0 \\
Prin1 & 1 & 78.71161 & 6.27070 & 12.55 & <.0001 & 0.84919 \\
Prin2 & 1 & 28.94074 & 16.87703 & 1.71 & 0.1144 & 0.11601 \\
Prin3 & 1 & 127.41962 & 28.28797 & 4.50 & 0.0009 & 0.30473 \\
Prin4 & 1 & -125.87565 & 37.05513 & -3.40 & 0.0060 & -0.22981 \\
Prin5 & 1 & 34.63870 & 60.34080 & 0.57 & 0.5775 & 0.03884 \\
Prin6 & 1 & 74.12371 & 69.01946 & 1.07 & 0.3058 & 0.07266 \\
Prin7 & 1 & -348.63400 & 130.56649 & -2.67 & 0.0218 & -0.18064 \\
Prin8 & 1 & -419.83729 & 164.02143 & -2.56 & 0.0265 & -0.17317 \\
\hline
F值 & & & & 25.94 & & \\
Pr > F & & & & <.0001 & & \\
R2 & & & & 0.9497 & & \\
\hline
主成分回归 & & & & \( y = 2134.82055 + 78.71161 \text{Prin1} \) & & \\
\hline
最终回归模型 & & & & \begin{tabular}{l} 
\( y = 900.45 - 0.14592x1 + 33.640x2 + 0.0090x4 + 0.06885x5 \) \\
\( + 0.21751x6 + 0.10559x8 + 1.46570x9 + 310.93x11 \)
\end{tabular} & & \\
\hline
共线诊断 & & & & \begin{tabular}{l} 
\( x1(0.02278) \quad x2(0.021) \quad x4(0.0203) \quad x5(0.0204) \) \\
\( x6(0.02205) \quad x8(0.02011) \quad x9(0.00604) \quad x11(0.02) \)
\end{tabular} & & \\
\hline
\end{tabular}
\end{table}

注:1. 由于Prin1-Prin8是相互正交的,故删除某一个主成分不影响回归方程其他主成分的回归系数。 \\
2. 最后一行共线诊断里的括号为IPCVI。

(2) 模型解释

1) 8个主成分回归方程中1个主成分均保留原有信息的80.73\%,说明最终回归模型切实有效。从最终共线性诊断中看,所有的IPCVIF均小于0.5,说明最终模型均克服了多重共线的问题。

2) 在安徽小麦种植面积的回归模型中,农业劳动力人口的弹性为负值,为

\section*{2.4.3.2 山东小麦的主成分回归模型的求解与模型解释}

(1) 主成分回归求解

为了得出山东小麦种植面积对农业劳动人口、农民受教育程度、城乡收入差距、家庭负担、净出口、城市化水平的 C-D 函数关系,首先必须得出小麦种植面积 \( y \) 分别对 Prin1-Prin5 的多元线性回归函数。由于主成分是 \( x_1, x_2, x_3, x_4, x_{10}, x_{11} \) 的线性函数,因此可以得到 \( y \) 与 \( x_1, x_2, x_3, x_4, x_{10}, x_{11} \) 之间的回归方程。运用 SAS 软件的 REG 模块以及 pcomit=1, 2, 3, 4, 5 outvif 语句进行计算,得到主成分回归、最终回归模型及其检验结果,见下表。

\begin{table}[h]
\centering
\caption{主成分回归系数及模型检验}
\begin{tabular}{l c c c c c c}
\hline
变量 & 自由度 & 参数 & 标准 & t 值 & Pr > |t| & 标准化 \\
 & & 估计 & 误差 & & & 估计 \\
\hline
Intercept & 1 & 3529.13789 & 29.50980 & 119.59 & <.0001 & 0 \\
Prin1 & 1 & -157.71147 & 14.78340 & -10.67 & <.0001 & -0.67685 \\
Prin2 & 1 & -69.56295 & 32.26923 & -2.16 & 0.0521 & -0.13677 \\
Prin3 & 1 & 195.21694 & 39.32484 & 4.96 & 0.0003 & 0.31496 \\
Prin4 & 1 & -621.63885 & 68.87197 & -9.03 & <.0001 & -0.57266 \\
Prin5 & 1 & 220.81891 & 97.42901 & 2.27 & 0.0427 & 0.14380 \\
Prin6 & 1 & 484.87278 & 187.07355 & 2.59 & 0.0236 & 0.16444 \\
\hline
F 值 & & 39.40 & & & & \\
Pr > F & & 0.2864 & & & & \\
R2 & & <.0001 & & & & \\
\hline
主成分回归 & & \( y = 3922.092 + 35.86865 \text{Prin1} \) & & & & \\
\hline
最终回归模型 & & \( y = 4451.93 + 0.20471x_1 - 93.373x_2 - 196.893x_3 - 0.03212x_4 - 0.02859x_{10} - 439.34x_{11} \) & & & & \\
\hline
共线诊断 & & \( x_1(0.1419) \) & \( x_2(0.12936) \) & \( x_3(0.11079) \) & \( x_4(0.45631) \) & \( x_{10}(0.332) \) \\
& & & & & & \( x_{11}(0.33225) \) \\
\hline
\end{tabular}
\end{table}

注:1. 由于 Prin1-Prin6 是相互正交的,故删除某一个主成分不影响回归方程其他主成分的回归系数。 \\
2. 最后一行共线诊断里的括号为 IPCVI。

(2) 模型解释

1) 6 个主成分回归方程中 1 个主成分均保留原有信息的 69.82\%,说明最终回归模型比较有效。从最终共线性诊断中看,所有的 IPCVIF 均小于 0.5,说明最终模型均克服了多重共线的问题。

2) 在单产的回归模型中,农业劳动人口的弹性为正值,为 0.20471。表明

\section*{2.4.3.3 江苏水稻的主成分回归模型的求解与模型解释}

(1) 主成分回归求解

为了得出江苏水稻种植面积对城乡收入差距、生产成本、农作物播种面积、受灾面积、净出口的 C-D 函数关系,首先必须得出水稻种植面积 \(y\) 分别对 Prin1-Prin5 的多元线性回归函数。由于主成分是 \(x_3, x_6, x_8, x_9, x_{10}\) 的线性函数,因此可以得到 \(y\) 与 \(x_3, x_6, x_8, x_9, x_{10}\) 之间的回归方程。运用 SAS 软件的 REG 模块以及 pcomit=1, 2, 3, 4 outvif 语句进行计算,得到主成分回归、最终回归模型及其检验结果,见表。

\begin{table}[h]
\centering
\caption{主成分回归系数及模型检验}
\begin{tabular}{l c c c c c}
\hline
变量 & 自由度 & 参数估计 & 标准误差 & t 值 & Pr > |t| & 标准化估计 \\
\hline
Intercept & 1 & 2213.75700 & 29.15293 & 75.94 & <.0001 & 0 \\
Prin1 & 1 & 1.44359 & 17.92164 & 0.08 & 0.9369 & 0.01760 \\
Prin2 & 1 & 46.86155 & 26.94601 & 1.74 & 0.1039 & 0.37990 \\
Prin3 & 1 & 73.55061 & 38.33337 & 1.92 & 0.0756 & 0.41914 \\
Prin4 & 1 & 24.59889 & 51.47090 & 0.48 & 0.6401 & 0.10440 \\
Prin5 & 1 & 19.28066 & 157.66342 & 0.12 & 0.9044 & 0.02671 \\
\hline
F 值 & & & & 1.39 & & \\
Pr > F & & & & 0.2864 & & \\
R2 & & & & 0.3319 & & \\
\hline
主成分回归 & & & & & & \\
& & & & \(y = 3922.092 + 35.86865 \text{Prin1}\) & & \\
\hline
最终回归模型 & & & & & & \\
& & & & \(y = 1931.57 - 30.7563x_3 + 0.06065x_6 + 0.03925x_8 + 7.92136x_9 - 0.00919x_{10}\) & & \\
\hline
共线诊断 & & \(x_3(0.1419)\) & \(x_6(0.12936)\) & \(x_8(0.11079)\) & \(x_9(0.45631)\) & \(x_{10}(0.33225)\) \\
\hline
\end{tabular}
\end{table}

注:1. 由于 Prin1-Prin5 是相互正交的,故删除某一个主成分不影响回归方程其他主成分的回归系数。 \\
2. 最后一行共线诊断里的括号为 IPCVI。

(2) 模型解释

1) 5 个主成分回归方程中 1 个主成分均保留原有信息的 55.71\%,说明最终回归模型相对有效。从最终共线性诊断中看,所有的 IPCVIF 均小于 0.5,说明最终模型均克服了多重共线的问题。

2) 在单产的回归模型中,城乡收入差距与净出口的弹性为负值,分别为 -30.7563 和 -0.00919。表明城乡收入差距的拉大与净出口的增加减少了江苏水稻的种植面积。

\section*{2.4.3.4 吉林水稻的主成分回归模型的求解与模型解释}

\subsubsection{(1) 主成分回归求解}

为了得出吉林水稻种植面积对农业劳动人口、农民受教育程度、家庭负担、市场价格、生产成本、农作物播种面积、净出口、城市化水平的C-D函数关系,首先必须得出水稻种植面积 $y$ 分别对 Prin1-Prin7 的多元线性回归函数。由于主成分是 $x_2, x_4, x_5, x_6, x_8, x_{10}, x_{11}$ 的线性函数,因此可以得到 $y$ 与 $x_2, x_4, x_5, x_6, x_8, x_{10}, x_{11}$ 之间的回归方程。运用 SAS 软件的 REG 模块以及 pcomit=1, 2, 3, 4, 5, 6 outvif 语句进行计算,得到主成分回归、最终回归模型及其检验结果,见表 5。

\begin{table}[h]
\centering
\caption{主成分回归系数及模型检验}
\begin{tabular}{c c c c c c}
\hline
变量 & 自由度 & 参数估计 & 标准误差 & t 值 & Pr > |t| & 标准化估计 \\
\hline
Intercept & 1 & 607.94100 & 7.38544 & 82.32 & <.0001 & 0 \\
Prin1 & 1 & 33.48521 & 3.19135 & 10.49 & <.0001 & 0.75575 \\
Prin2 & 1 & 44.92149 & 8.26418 & 5.44 & 0.0002 & 0.39152 \\
Prin3 & 1 & -74.71664 & 14.18859 & -5.27 & 0.0002 & -0.37930 \\
Prin4 & 1 & -42.38053 & 23.02022 & -1.84 & 0.0905 & -0.13260 \\
Prin5 & 1 & 82.87071 & 26.23926 & 3.16 & 0.0082 & 0.22748 \\
Prin6 & 1 & -1.85825 & 37.40213 & -0.05 & 0.9612 & -0.00358 \\
Prin7 & 1 & -15.81103 & 121.24581 & -0.13 & 0.8984 & -0.00939 \\
\hline
F 值 & & 25.82 & & & & \\
Pr > F & & <.0001 & & & & \\
R2 & & 0.9377 & & & & \\
\hline
主成分回归 & & $y = 607.941 + 33.48521 \text{Prin1} + 44.92149 \text{Prin2}$ & & & & \\
最终回归模型 & & \begin{tabular}{l}
$y = -379.515 + 53x_2 + 0.00749x_4 + 0.0024x_5 + 0.04988x_6$ \\
$+ 0.05592x_8 + 0.00954x_{10} + 424.38x_{11}$
\end{tabular} & & & & \\
共线诊断 & & \begin{tabular}{l}
$x_2(0.178) \quad x_4(0.03052) \quad x_5(0.1073) \quad x_6(0.02999)$ \\
$x_8(0.16068) \quad x_{10}(0.80760) \quad x_{11}(0.053)$
\end{tabular} & & & & \\
\hline
\end{tabular}
\end{table}

注:1. 由于 Prin1-Prin7 是相互正交的,故删除某一个主成分不影响回归方程其他主成分的回归系数。 \\
2. 最后一行共线诊断里的括号为 IPCVI。

\subsubsection{(2) 模型解释}

1) 7 个主成分回归方程中 2 个主成分均保留原有信息的 92.54\%,说明最终回归模型切实有效。从最终共线性诊断中看,所有的 IPCVIF 均小于 0.5,说

\section*{2.4.3.5 湖南水稻的主成分回归模型的求解与模型解释}

(1) 主成分回归求解

为了得出湖南水稻种植面积对农民受教育程度、家庭负担、市场价格、生产成本、农作物播种面积、城市化水平的 C-D 函数关系,首先必须得出水稻种植面积 \(y\) 对 Prin1-Prin6 的多元线性回归函数。由于主成分是 \(x_2, x_4, x_5, x_6, x_8, x_{11}\) 的线性函数,因此可以得到 \(y\) 与 \(x_2, x_4, x_5, x_6, x_8, x_{11}\) 之间的回归方程。运用 SAS 软件的 REG 模块以及 pcomit=1, 2, 3, 4, 5 outvif 语句进行计算,得到主成分回归、最终回归模型及其检验结果,见表。

\begin{table}[h]
\centering
\caption{主成分回归系数及模型检验}
\begin{tabular}{l c c c c c}
\hline
变量 & 自由度 & 参数估计 & 标准误差 & t 值 & Pr > |t| & 标准化估计 \\
\hline
Intercept & 1 & 3922.09200 & 21.64691 & 181.18 & <.0001 & 0 \\
Prin1 & 1 & 35.86865 & 9.65539 & 3.71 & 0.0026 & 0.41780 \\
Prin2 & 1 & 177.43631 & 37.47365 & 4.73 & 0.0004 & 0.53253 \\
Prin3 & 1 & -209.74655 & 41.14635 & -5.10 & 0.0002 & -0.57331 \\
Prin4 & 1 & 100.30298 & 117.75037 & 0.85 & 0.4097 & 0.09580 \\
Prin5 & 1 & -187.13793 & 143.37696 & -1.31 & 0.2144 & -0.14679 \\
Prin6 & 1 & 317.72472 & 266.34216 & 1.19 & 0.2542 & 0.13416 \\
\hline
F 值 & & 11.01 & & & & \\
Pr > F & & 0.0002 & & & & \\
R2 & & 0.8356 & & & & \\
\hline
主成分回归 & & \(y = 3922.092 + 35.86865 \text{Prin1}\) & & & & \\
\hline
最终回归模型 & & \begin{tabular}{l}
\(y = 3232.43 + 21.599x_2 + 0.00511x_4 + 0.00906x_5 + 0.0247x_5\) \\
\(+ 0.08495x_6 + 0.04239x_8 + 177.76x_{11}\)
\end{tabular} & & & & \\
\hline
共线诊断 & x2(0.030) & x4(0.03478) & x5(0.0325) & x6(0.03333) & x8(0.02534) & x11(0.03) \\
\hline
\end{tabular}
\end{table}

注:1. 由于 Prin1-Prin6 是相互正交的,故删除某一个主成分不影响回归方程其他主成分的回归系数。 \\
2. 最后一行共线诊断里的括号为 IPCVIF。

(2) 模型解释

1) 6 个主成分回归方程中 1 个主成分均保留原有信息的 82.35\%,说明最终回归模型切实有效。从最终共线性诊断中看,所有的 IPCVIF 均小于 0.5,说明最终模型均克服了多重共线的问题。

\begin{enumerate}
    \setcounter{enumi}{1}
    \item 城市化水平、农民受教育程度、生产成本、农作物播种面积与市场价格的弹性大小递减,且均为正数。这说明,要提高湖南水稻的种植面积,增加这些指标值均为有效。
\end{enumerate}

\section*{第3章 问题二解答}

---粮食最低收购价执行的效果评价

\subsection*{3.1 问题分析}

要想衡量粮食最低收购价政策实施的效果,主要是比较政策实施前后粮食种植面积是否发生了显著性的变化。但是,仅仅依据种植面积的变化并不能真实反映粮食最低收购价政策的效果,因为可能影响种植面积的因素有很多。传统的方差分析法并不能排除掉其它因素的干扰,所以在本文中,我们采用协方差分析的方法来判断粮食最低收购价政策的有效性。另一方面,影响种植面积的各个因素之间存在着显著的线性关系,如果直接将这些变量代入协方差分析模型中,可能会使得模型的效果较差。因此,在进行协方差分析之前,首先对各个变量进行主成分分析。

依据问题一中聚类的结果,对于小麦和水稻,分别在各个聚类区域选取一个省份为代表来进行分析。每个省份选取的影响种植面积的变量依据问题一中 spearman 相关性检验的结果。具体选取的省份与各个省份影响种植面积的变量见表3.1。

\begin{table}[h]
\centering
\caption{选取的省份及相关指标}
\begin{tabular}{c c c c c c c c c}
\hline
品种 & 省份 & X1 & X2 & X4 & X5 & X6 & X8 & X9 & X11 \\
\hline
小麦 & 安徽 & X1 & X2 & X4 & X5 & X6 & X8 & X9 & X11 \\
 & 山东 & X1 & X2 & X3 & X4 & X10 & X11 & & \\
 & 湖南 & X1 & X3 & X5 & X6 & X7 & X8 & X10 & \\
水稻 & 吉林 & X2 & X4 & X5 & X6 & X8 & X10 & X11 & \\
 & 江苏 & X3 & X6 & X8 & X9 & X10 & & & \\
\hline
\end{tabular}
\end{table}

经过查找资料获知,每个省开始实行粮食最低收购价政策的年份是不一样的,具体实施年份见表3.2。

\begin{table}[h]
\centering
\caption{实施粮食最低收购价政策开始年份}
\begin{tabular}{c c c}
\hline
粮食品种 & 省份 & 开始年份 \\
\hline
小麦 & 安徽 & 2006 \\
 & 山东 & 2006 \\
 & 吉林 & 2004 \\
水稻 & 湖南 & 2004 \\
 & 江苏 & 2008 \\
\hline
\end{tabular}
\end{table}

在粮食最低收购价政策效果的评价模型中,分别对政策实施前后的粮食种植面积进行比较。如果粮食最低收购价政策实施前后的粮食种植面积发生了显著性变化,则说明政策是有效的;如果在政策实施前后种植面积没有发生显著性变化,则说明政策是无效的。

最后对于粮食种植面积,将实行粮食最低收购价政策的省份与没有实行粮食最低收购价政策的相邻省份进行对比,比较粮食最低收购价政策的有效性。

\subsection*{3.2 主成分分析}

利用 SAS 软件编程,对于所选出来的省份进行主成分分析,具体结果如下。

\begin{table}
\centering
\begin{tabular}{c c c c c}
 & 特征值 & 方差贡献的差 & 方差贡献率 & 累计贡献率 \\
\hline
1 & 6.45864553 & 5.56702377 & 0.8073 & 0.8073 \\
2 & 0.89162175 & 0.57424934 & 0.1115 & 0.9188 \\
3 & 0.31737241 & 0.13241309 & 0.0397 & 0.9585 \\
4 & 0.18495932 & 0.11520828 & 0.0231 & 0.9816 \\
5 & 0.06975104 & 0.01643846 & 0.0087 & 0.9903 \\
6 & 0.05331258 & 0.03841522 & 0.0067 & 0.997 \\
7 & 0.01489736 & 0.00545737 & 0.0019 & 0.9988 \\
8 & 0.00943999 & & 0.0012 & 1 \\
\end{tabular}
\caption{表3.3 各主成分的方差贡献表}
\end{table}

\begin{table}
\centering
\begin{tabular}{c c c}
变量 & Prin 1 & Prin 2 \\
\hline
x1 & -0.383555 & 0.010069 \\
x2 & 0.3719 & -0.106931 \\
x4 & 0.362478 & -0.280319 \\
x5 & 0.363001 & -0.118951 \\
x6 & 0.377341 & -0.173674 \\
x8 & 0.360402 & 0.246421 \\
x9 & 0.197525 & 0.896566 \\
x11 & 0.374331 & -0.031929 \\
\end{tabular}
\caption{表3.4 各变量的系数}
\end{table}

\begin{table}
\centering
\caption{表3.5 主成分得分}
\begin{tabular}{c c c c c c}
\hline
年份 & Prin1 & Prin2 & 年份 & Prin1 & Prin2 \\
\hline
1995 & -2.99205 & -0.0366 & 2005 & -0.64375 & -0.14133 \\
1996 & -2.40855 & 0.18462 & 2006 & -0.08804 & -0.07906 \\
1997 & -2.15986 & -0.17551 & 2007 & 0.60786 & -0.37459 \\
1998 & -2.22598 & 0.07861 & 2008 & 1.75155 & 1.94907 \\
1999 & -2.14598 & -0.041 & 2009 & 2.16507 & 0.72655 \\
2000 & -2.39927 & -0.21451 & 2010 & 2.55888 & 1.76467 \\
2001 & -2.45492 & -0.38387 & 2011 & 3.33022 & 1.61567 \\
2002 & -2.11095 & -0.20957 & 2012 & 3.29364 & -1.21312 \\
2003 & -1.78581 & -0.19986 & 2013 & 3.75842 & -1.09466 \\
2004 & -0.7955 & -0.24749 & 2014 & 4.74503 & -1.90802 \\
\hline
\end{tabular}
\end{table}

表3.5为安徽省小麦的前两个主成分得分,利用此主成分得分代替原始变量进行协方差分析,可以克服变量间具有显著性线性关系的缺点。

\subsection*{3.2.2 吉林省水稻的主成分分析}

对通过 spearman 相关性检验的 7 个变量进行主成分分析。

\begin{table}
\centering
\caption{表3.6 各主成分的方差贡献表}
\begin{tabular}{c c c c c}
\hline
特征值 & 方差贡献的差 & 方差贡献率 & 累计贡献率 \\
\hline
1 & 5.63743396 & 4.79675512 & 0.8053 & 0.8053 \\
2 & 0.84067884 & 0.55547796 & 0.1201 & 0.9254 \\
3 & 0.28520088 & 0.17685546 & 0.0407 & 0.9662 \\
4 & 0.10834542 & 0.02495301 & 0.0155 & 0.9817 \\
5 & 0.08339242 & 0.04234962 & 0.0119 & 0.9936 \\
6 & 0.0410428 & 0.03713712 & 0.0059 & 0.9994 \\
7 & 0.00390568 & & 0.0006 & 1 \\
\hline
\end{tabular}
\end{table}

上表为吉林省水稻主成分分析的方差贡献表,前两个主成分的累计贡献率达到了 92.54%,即前两个主成分所包含的信息占全部信息的 92.54%,因此选择前两个主成分即可。各变量在前两个主成分中的系数见表3.7。

\begin{table}
\centering
\caption{表3.7 各变量的系数}
\begin{tabular}{c c c}
\hline
 & Prin1 & Prin2 \\
\hline
x2 & 0.378209 & 0.357707 \\
x4 & 0.411831 & 0.01919 \\
x5 & 0.396412 & -0.258406 \\
x6 & 0.404728 & -0.028077 \\
x8 & 0.370625 & 0.338526 \\
x10 & -0.259975 & 0.817835 \\
x11 & 0.401816 & 0.143743 \\
\hline
\end{tabular}
\end{table}

表3.7 为各个变量在前两个主成分中的系数,可以看出,第一主成分主要

\begin{table}
\centering
\begin{tabular}{c c c c c}
年份 & Prin1 & Prin2 & 年份 & Prin1 \\
\hline
1995 & -1.72933 & -2.58105 & 2005 & -0.353 \\
1996 & -2.03411 & -1.71904 & 2006 & -0.26175 \\
1997 & -1.96793 & -1.03885 & 2007 & -0.09452 \\
1998 & -2.52534 & 0.38369 & 2008 & 0.36425 \\
1999 & -2.68257 & 0.93723 & 2009 & 1.52664 \\
2000 & -2.55344 & 0.94113 & 2010 & 2.46475 \\
2001 & -1.71201 & 0.68718 & 2011 & 2.8637 \\
2002 & -1.60682 & 0.82512 & 2012 & 3.61709 \\
2003 & -1.43786 & 1.18158 & 2013 & 4.11523 \\
2004 & -0.58963 & 0.45671 & 2014 & 4.59665 \\
\end{tabular}
\end{table}

\begin{table}
\centering
\begin{tabular}{c c c c c}
特征值 & 方差贡献的差 & 方差贡献率 & 累计贡献率 \\
\hline
1 & 4.18922225 & 3.06708397 & 0.6982 & 0.6982 \\
2 & 1.12213828 & 0.62755523 & 0.187 & 0.8852 \\
3 & 0.49458305 & 0.35552337 & 0.0824 & 0.9677 \\
4 & 0.13905968 & 0.10528884 & 0.0232 & 0.9908 \\
5 & 0.03377084 & 0.01254494 & 0.0056 & 0.9965 \\
6 & 0.02122591 & & 0.0035 & 1 \\
\end{tabular}
\caption{山东省各主成分的方差贡献表}
\end{table}

\begin{table}
\centering
\begin{tabular}{c c c c c}
特征值 & 方差贡献的差 & 方差贡献率 & 累计贡献率 \\
\hline
1 & 4.08562521 & 2.51644395 & 0.5837 & 0.5837 \\
2 & 1.56918126 & 0.97084697 & 0.2242 & 0.8078 \\
3 & 0.59833429 & 0.22091083 & 0.0855 & 0.8933 \\
4 & 0.37742346 & 0.0782724 & 0.0539 & 0.9472 \\
5 & 0.29915106 & 0.24760885 & 0.0427 & 0.99 \\
6 & 0.05154221 & 0.0327997 & 0.0074 & 0.9973 \\
7 & 0.01874251 & & 0.0027 & 1 \\
\end{tabular}
\caption{湖南省各主成分的方差贡献表}
\end{table}

\begin{table}
\centering
\caption{江苏省各主成分的方差贡献表}
\begin{tabular}{c c c c c}
\hline
 & 特征值 & 方差贡献的差 & 方差贡献率 & 累计贡献率 \\
\hline
1 & 2.78538679 & 1.55326982 & 0.5571 & 0.5571 \\
2 & 1.23211696 & 0.62329988 & 0.2464 & 0.8035 \\
3 & 0.60881709 & 0.27112769 & 0.1218 & 0.9253 \\
4 & 0.3376894 & 0.30169963 & 0.0675 & 0.9928 \\
5 & 0.03598977 & & 0.0072 & 1 \\
\hline
\end{tabular}
\end{table}

三省各变量在主成分中的系数表与主成分得分表见附录四。

\subsection{3.3 统计控制评价模型的基本思想及数学模型}

本文中的统计控制评价模型的所利用的方法是协方差分析法。协方差分析是以统计控制为目的,综合回归分析与方差分析所得到的统计分析方法。统计控制是指在单因素或多因素试验中,如果有无法控制的因素 \(X\) 影响试验指标 \(Y\) 的观测值,并且 \(X\) 与 \(Y\) 之间有显著的线性关系时,利用线性回归矫正 \(Y\) 的观测值,消除 \(X\) 的差异对 \(Y\) 的影响。因此协方差分析就是先利用实验指标 \(Y\) 与协变量 \(X\) 进行线性回归,把协变量 \(X\) 化为相等后对各组 \(Y\) 的均值进行矫正,然后对各组 \(Y\) 矫正后的均值进行比较的假设检验。原假设 \(H_{0}: \mu_{1} = \mu_{2} = \cdots = \mu_{r}\),备择假设 \(\mu_{1}, \mu_{2}, \cdots, \mu_{r}\) 中至少有两个不相等。

对于只有一个协变量的单因素协方差分析的数学模型可以写成如下形式:
\[
y_{i,j} = \mu_{i} + \beta (x_{i,j} - \bar{x}_{..}) + \varepsilon_{i,j} = \mu + \alpha_{i} + \beta (x_{i,j} - \bar{x}_{..}) + \varepsilon_{i,j},
\]
其中 \(\beta\) 为协变量 \(x\) 回归系数,\(y_{i,j}\) 为实验指标 \(Y\) 的实际观测值,\(x_{i,j}\) 为协变量 \(X\) 的观测值,\(\bar{x}_{..}\) 为 \(x_{i,j}\) 的均值,\(\mu\) 为 \(Y\) 的总体均值,\(\mu_{i}\) 为各水平 \(Y\) 的均值,\(\varepsilon_{i,j}\) 为服从正态分布的随机误差,\(\alpha_{i} = \mu_{i} - \mu\) 为水平 \(A_{i}\) 的效应。

经过具体的计算公式(略)或者 SAS 软件等许多方式可以得出方差分析表,具体见表 3.12。当 \(p\) 值小于显著性水平时,拒绝原假设,认为矫正后 \(Y\) 的观测值有显著性的差异。

\begin{table}
\centering
\caption{方差分析表}
\begin{tabular}{c c c c c}
\hline
方差来源 & 平方和 (type I) & 自由度 & 均方和 & F 值 & P 值 \\
\hline
组间 & \(Q_{A}\) & \(r-1\) & \(MQ_{A}\) & \(F\) \\
误差 & \(Q_{E}\) & \(rs-r-1\) & \(MQ_{A}\) \\
总和 & \(Q_{T}\) & \(rs-2\) \\
\hline
\end{tabular}
\end{table}

\subsection{3.4 建立主成分-统计控制评价模型以及模型求解结果}

分别对于小麦和水稻,以影响种植面积的变量的主成分变量为协变量,建立粮食最低收购价政策与种植面积的协方差分析模型,即主成分-统计控制模型。通过统计控制,消除协变量对水稻种植面积的影响,从而分析粮食最低收购价政策对小麦和水稻种植面积是否具有显著性的影响。

该协方差分析模型为单因素两水平协方差分析模型,其中单因素为粮食种植面积,两个水平为实施粮食最低收购价政策前后,协变量为经过主成分分析之后选取的主成分变量,建立主成分-统计控制模型。

\section*{3.4.1 安徽省小麦的评价}

经过 SAS 软件对所选数据进行协方差分析,首先根据 spearman 检验结果选前两个主成分作为协变量进入协方差分析模型,得到的各主成分回归系数估计与检验表如下。

\textbf{表3.13 回归方程系数估计及检验}

\begin{tabular}{l c c c c}
\hline
变量 & 估计值 & 标准误 & T值 & P值 \\
\hline
常数 & 2266.610339 & 76.6384365 & 29.58 & $<.0001$ \\
Prin1 & 35.513886 & 25.9065235 & 1.37 & 0.1893 \\
Prin2 & 9.329458 & 31.0930363 & 0.3 & 0.768 \\
\hline
\end{tabular}

由上表可知,第二主成分的回归系数不显著,因此去掉第二主成分再进行协方差分析,结果如下。

\textbf{表3.14 回归方程系数估计及检验修正}

\begin{tabular}{l c c c c}
\hline
变量 & 估计值 & 标准误 & T值 & P值 \\
\hline
常数 & 2274.007418 & 70.5962294 & 32.21 & $<.0001$ \\
Prin1 & 33.089295 & 23.9462232 & 1.38 & 0.1849 \\
\hline
\end{tabular}

上表为安徽省小麦主成分回归方程系数估计及检验修正表,可以看出,修正后的主成分回归系数较显著,因此可以以这个主成分作为协变量,分析粮食最低收购价政策对安徽省小麦种植面积的影响。得到矫正后的方差分析表如下。

\textbf{表3.15 方差分析表}

\begin{tabular}{l c c c c c}
\hline
方差来源 & 平方和(type I) & 自由度 & 均方和 & F值 & P值 \\
\hline
Prin1 & 760278.1918 & 1 & 760278.1918 & 55.61 & $<.0001$ \\
因素A & 61595.787 & 1 & 61595.787 & 4.51 & 0.0488 \\
误差 & 232430.295 & 17 & 13672.37 & & \\
总和 & 1054304.274 & 19 & & & \\
\hline
\end{tabular}

对于粮食最低收购价政策对安徽小麦种植面积的影响的显著性检验结果见表3.15,其检验结果为 $F=4.51, P\{F>4.51\}=0.0488$,小于显著性水平 0.05。说明在消除两个主成分协变量的影响后,粮食最低收购价政策之前与粮食最低收购价政策之后安徽小麦种植面积具有显著性的差异,即粮食最低收购价政策对安徽小麦种植面积有显著的影响。

\textbf{表3.16 政策影响的显著性}

\begin{tabular}{l c c c c}
\hline
A & y均值 & 标准误 & p值(y) & p值(A) \\
\hline
0 & 2020.94037 & 59.68475 & $<.0001$ & 0.0488 \\
1 & 2274.00743 & 70.59622 & $<.0001$ & \\
\hline
\end{tabular}

由表3.16可知,粮食最低收购价政策实施前矫正的安徽平均小麦种植面积为 2020.94 千公顷,粮食最低收购价政策实施后矫正的安徽平均小麦种植面积为 2274 千公顷。由此可见,粮食最低收购价政策实行后,安徽省的小麦种植面积有了大幅的增加,增加幅度为 12.52\%。

\section*{3.4.2 山东省小麦的评价}

经过 SAS 软件对所选数据进行协方差分析,首先根据 spearman 检验结果选前三个主成分作为协变量进入协方差分析模型,得到的各主成分回归系数估计与检验表如下。

\begin{table}
\centering
\caption{回归方程系数估计及检验}
\begin{tabular}{l c c c c}
\hline
变量 & 估计值 & 标准误 & T值 & P值 \\
\hline
常数 & 4016.872935 & 56.35232105 & 71.28 & $<.0001$ \\
Prin1 & -227.513764 & 23.24294594 & -9.79 & $<.0001$ \\
Prin2 & -90.630569 & 22.42017565 & -4.04 & 0.0011 \\
Prin3 & 134.581872 & 33.96771171 & 3.96 & 0.0013 \\
\hline
\end{tabular}
\end{table}

\begin{table}
\centering
\caption{方差分析表}
\begin{tabular}{l c c c c}
\hline
方差来源 & 平方和(type I) & 自由度 & 均方和 & F值 & P值 \\
\hline
Prin1 & 423390.6332 & 1 & 423390.6332 & 39.73 & $<.0001$ \\
Prin2 & 228086.1241 & 1 & 228086.1241 & 21.4 & 0.0003 \\
Prin3 & 267044.7905 & 1 & 267044.7905 & 25.06 & 0.0002 \\
因素A & 626739.4447 & 1 & 626739.4447 & 58.81 & $<.0001$ \\
误差 & 159865.757 & 15 & 10657.717 & & \\
总和 & 1705126.749 & 19 & & & \\
\hline
\end{tabular}
\end{table}

\begin{table}
\centering
\caption{政策影响的显著性}
\begin{tabular}{l c c c c}
\hline
A & y均值 & 标准误 & p值(y) & p值(A) \\
\hline
0 & 3300.11478 & 47.97881 & $<.0001$ & $<.0001$ \\
1 & 4016.87305 & 56.35234 & $<.0001$ & \\
\hline
\end{tabular}
\end{table}

\begin{table}
\centering
\caption{回归方程系数估计及检验}
\begin{tabular}{l c c c c}
\hline
变量 & 估计值 & 标准误 & T值 & P值 \\
\hline
常数 & 638.2745249 & 22.31471325 & 28.6 & $<.0001$ \\
Prin1 & 22.0004036 & 8.83778541 & 2.49 & 0.0242 \\
Prin2 & 43.3050661 & 13.96886916 & 3.1 & 0.0069 \\
\hline
\end{tabular}
\end{table}

\begin{table}
\centering
\caption{方差分析表}
\begin{tabular}{l l l l l l}
\hline
方差来源 & 平方和(type I) & 自由度 & 均方和 & F值 & P值 \\
\hline
Prin1 & 120099.5202 & 1 & 120099.5202 & 38.73 & $<$0.0001 \\
Prin2 & 32232.2356 & 1 & 32232.2356 & 10.39 & 0.0053 \\
因素A & 8322.0625 & 1 & 8322.0625 & 2.68 & 0.1209 \\
误差 & 49619.7015 & 16 & 3101.2313 & & \\
总和 & 210273.5198 & 19 & & & \\
\hline
\end{tabular}
\end{table}

对于粮食最低收购价政策对吉林水稻种植面积的影响的显著性检验结果见表3.21,其检验结果为$F=2.68, P\{F>2.68\}=0.1209$,小于显著性水平0.15。说明在消除两个主成分协变量的影响后,粮食最低收购价政策之前与粮食最低收购价政策之后吉林水稻种植面积存在较为显著的差异,即粮食最低收购价政策对吉林水稻种植面积有较为显著的影响。

\begin{table}
\centering
\caption{政策影响的显著性}
\begin{tabular}{l l l l l}
\hline
A & y均值 & 标准误 & p值(y) & p值(A) \\
\hline
0 & 570.866718 & 25.831615 & $<$0.0001 & 0.1209 \\
1 & 638.274503 & 22.314714 & $<$0.0001 & \\
\hline
\end{tabular}
\end{table}

由表3.22可知,粮食最低收购价政策实施前矫正的吉林平均水稻种植面积为570.87千公顷,粮食最低收购价政策实施后矫正的吉林平均水稻种植面积为638.27千公顷。由此可见,粮食最低收购价政策实行后,吉林省的水稻种植面积有了大幅的增加,增加幅度为11.81%。

\subsubsection{湖南省水稻的评价}

通过SAS软件,首先选前两个主成分作为协变量进入协方差分析模型,得到主成分回归系数估计与检验表如下。

\begin{table}
\centering
\caption{回归方程系数估计及检验}
\begin{tabular}{l l l l l}
\hline
变量 & 估计值 & 标准误 & T值 & P值 \\
\hline
常数 & 3845.107114 & 78.087814 & 49.24 & $<$0.0001 \\
Prin1 & 92.271952 & 36.1333043 & 2.55 & 0.0212 \\
Prin2 & 6.230349 & 37.4727278 & 0.17 & 0.87 \\
\hline
\end{tabular}
\end{table}

由上表结果可知,第二主成分回归系数的显著性没有通过检验,因此将第二主成分去掉,再次进行协方差分析,得到修正后的主成分回归系数表如下。

\begin{table}
\centering
\caption{回归方程系数估计及检验修正表}
\begin{tabular}{l l l l l}
\hline
变量 & 估计值 & 标准误 & T值 & P值 \\
\hline
常数 & 3838.088954 & 63.7892918 & 60.17 & $<$0.0001 \\
Prin1 & 95.427498 & 29.8551777 & 3.2 & 0.0053 \\
\hline
\end{tabular}
\end{table}

由上表可知,修正后的主成分回归系数通过了显著性水平0.05的检验,因此该效应是显著的,可以用这个主成分作为协变量来分析粮食最低收购价对湖南省水稻种植面积的影响。

\begin{table}
\centering
\caption{方差分析表}
\begin{tabular}{l l l l l l}
\hline
方差来源 & 平方和(type I) & 自由度 & 均方和 & F值 & P值 \\
\hline
Prin1 & 258061.8455 & 1 & 258061.8455 & 10.42 & 0.0049 \\
因素A & 61752.1938 & 1 & 61752.1938 & 2.49 & 0.1328 \\
误差 & 421097.281 & 17 & 24770.4283 & & \\
总和 & 740911.3203 & 19 & & & \\
\hline
\end{tabular}
\end{table}

\begin{table}
\centering
\caption{政策影响的显著性}
\begin{tabular}{c c c c c}
\hline
A & y均值 & 标准误 & p值(y) & p值(A) \\
\hline
0 & 4024.76239 & 73.93833 & $<.0001$ & 0.1328 \\
1 & 3838.08895 & 63.78929 & $<.0001$ & \\
\hline
\end{tabular}
\end{table}

由表3.26可知,粮食最低收购价政策实施前矫正的湖南平均水稻种植面积为4024.76千公顷,粮食最低收购价政策实施后矫正的湖南平均水稻种植面积为3838.09千公顷,粮食最低收购价政策实行后,湖南省的水稻种植面积反而有了小幅度的下降。

对于粮食最低收购价政策对湖南水稻种植面积的影响的显著性检验结果见表3.25,其检验结果为$F=2.49, P\{F>2.49\}=0.1328$,稍微大于显著性水平0.1。说明在消除两个主成分协变量的影响后,粮食最低收购价政策之前与粮食最低收购价政策之后湖南水稻种植面积的差异不显著,即粮食最低收购价政策对湖南水稻种植面积的效果不显著。

\subsection*{3.4.5 江苏省水稻的评价}

通过SAS软件,首先选前两个主成分进入协方差分析模型,得到主成分回归系数估计与检验表如下。

\begin{table}
\centering
\caption{回归方程系数估计及检验}
\begin{tabular}{c c c c c}
\hline
变量 & 估计值 & 标准误 & T值 & P值 \\
\hline
常数 & 2279.960968 & 68.55068297 & 33.26 & $<.0001$ \\
Prin1 & 23.627665 & 27.63322185 & 0.86 & 0.4051 \\
Prin2 & 41.301118 & 28.03157827 & 1.47 & 0.16 \\
\hline
\end{tabular}
\end{table}

由上述结果可知,第一主成分的回归系数不显著,因此将第一主成分去掉,用剩下的主成分继续进行协方差分析,得到修正后的主成分回归系数估计表如下。

\begin{table}
\centering
\caption{回归方程系数估计及检验修正表}
\begin{tabular}{c c c c c}
\hline
变量 & 估计值 & 标准误 & T值 & P值 \\
\hline
常数 & 2240.44759 & 50.23042703 & 44.6 & $<.0001$ \\
Prin2 & 44.619867 & 27.54112831 & 1.62 & 0.1236 \\
\hline
\end{tabular}
\end{table}

由上表可知,修正后的主成分回归系数的显著性水平提高了,且比较显著,因此该效应是显著的,可以用这个主成分作为协变量来分析粮食最低收购价对江苏省水稻种植面积的影响。

\begin{table}
\centering
\caption{方差分析表}
\begin{tabular}{c c c c c}
\hline
方差来源 & 平方和(type I) & 自由度 & 均方和 & F值 & P值 \\
\hline
Prin2 & 51409.151 & 1 & 51409.151 & 2.94 & 0.1046 \\
因素A & 7554.21811 & 1 & 7554.21811 & 0.43 & 0.5198 \\
误差 & 297239.4373 & 17 & 17484.6728 & & \\
总和 & 356202.8064 & 19 & & & \\
\hline
\end{tabular}
\end{table}

\begin{table}
\centering
\caption{政策影响的显著性}
\begin{tabular}{c c c c c}
\hline
A & y均值 & 标准误 & p值(y) & p值(A) \\
\hline
0 & 2199.38514 & 36.77371 & $<.0001$ & 0.5198 \\
1 & 2240.44759 & 50.23043 & $<.0001$ & \\
\hline
\end{tabular}
\end{table}

对于粮食最低收购价政策对江苏水稻种植面积的影响的显著性检验结果见表 3.29,其检验结果为 \( F = 0.43, P\{F > 0.43\} = 0.5198 \),大于显著性水平 0.1。说明在消除两个主成分协变量的影响后,粮食最低收购价政策之前与粮食最低收购价政策之后江苏水稻种植面积的差异不显著,即粮食最低收购价政策对江苏水稻种植面积的影响不显著。

由表 3.30 可知,粮食最低收购价政策实施前矫正的江苏平均水稻种植面积为 2199.39 千公顷,粮食最低收购价政策实施后矫正的江苏平均水稻种植面积为 2240.45 千公顷。由此可见,粮食最低收购价政策实行后,江苏省的水稻种植面积仅有小幅度的增加,增加幅度为 1.87%。

\section*{3.5 主成分-统计控制评价模型的结果分析}

通过上述主成分-统计控制评价模型的计算,粮食最低收购价政策对选出来的省份的有效性综合结果见表 3.31。

\begin{table}[h]
\centering
\caption{政策有效性综合结果}
\begin{tabular}{l c c c}
\hline
省份(粮食品种) & P值 & 种植面积显著性 & 政策有效性 \\
\hline
安徽省(小麦) & 0.0488 & 显著 & 有效 \\
山东省(小麦) & \(<0.0001\) & 非常显著 & 非常有效 \\
吉林省(水稻) & 0.1209 & 比较显著 & 比较有效 \\
湖南省(水稻) & 0.1328 & 不显著 & 无效 \\
江苏省(水稻) & 0.5198 & 不显著 & 无效 \\
\hline
\end{tabular}
\end{table}

从所选的这几个省来看,粮食最低收购价政策对山东省小麦的效果最为显著,对安徽省小麦的效果次之,对吉林省水稻的效果比较有效,而政策对江苏省和湖南省水稻的种植没有效果。这说明粮食最低收购价政策在不同区域、不同品种的执行效果是有差异的,并不一致。

\section*{3.6 与未实行粮食最低收购价政策省份的对比}

为了研究粮食最低收购价政策的效果,不光要研究同一个省份实行该政策前后粮食种植面积的变化,还要与相邻没有实施该政策的省份的粮食种植面积进行对比。本文以山东省为例,研究与其相邻的没有实施粮食最低收购价政策的山西省的小麦种植面积的变化,将两个省在政策实行前后时间的小麦种植面积的变化进行比较,研究粮食最低收购价政策的效果。

首先,搜集到 1995 年至 2014 年山西省小麦种植面积的数据,采用方差分析的方法研究小麦种植面积在 2005 年前后的变化。

此方差分析模型为单因素两水平方差分析模型,单因素为小麦的种植面积,两个水平分别为粮食最低收购价前后。设 2005 年之前小麦种植面积的均值为 \(\mu_1\),2005 年之后小麦种植面积为 \(\mu_2\),则原假设为 \(H_0: \mu_1 = \mu_2\),备择假设为 \(H_0: \mu_1 \neq \mu_2\)。经过计算可以得出方差分析表,当 p 值小于显著性水平时,拒绝原假设,认为 2005 年前后小麦的种植面积发生了显著性变化。

通过 R 软件编程,得到方差分析表如下。

\begin{table}[h]
\centering
\caption{方差分析表}
\begin{tabular}{l c c c c}
\hline
方差来源 & 平方和 & 自由度 & 均方和 & F值 & P值 \\
\hline
因素 & 124167 & 1 & 124167 & 20.72 & 0.000247 \\
误差 & 297239.4373 & 18 & 5992 & & \\
总和 & 232028 & 19 & & & \\
\hline
\end{tabular}
\end{table}

\begin{table}[h]
\centering
\caption{山东省与山西省小麦种植面积比较}
\begin{tabular}{l l l l l}
\hline
省份 & 方差分析显著性结果 & 2005年前种植面积(千公顷) & 2005年前种植面积(千公顷) & 种植面积变化率 & 种植面积变化趋势 \\
\hline
山东省 & 显著 & 3300.11 & 4016.87 & 12.52\% & 上升 \\
山西省 & 显著 & 857.253 & 699.667 & 18.38\% & 下降 \\
\hline
\end{tabular}
\end{table}

\section*{表4.1 小麦品种的价格数据}

\begin{tabular}{c c c c c}
\hline
\multirow{2}{*}{ 年份 } & \multicolumn{2}{c}{ 山东 } & \multicolumn{2}{c}{ 安徽 } \\
\cline{2-5}
 & 市场价 & 最低价 & 市场价 & 最低价 \\
\hline
2006 & 0.7266 & 0.71 & 0.704 & 0.71 \\
2007 & 0.8031 & 0.71 & 0.7127 & 0.71 \\
2008 & 0.8388 & 0.72 & 0.7636 & 0.72 \\
2009 & 0.9555 & 0.84 & 0.8657 & 0.84 \\
2010 & 1.0232 & 0.9 & 0.9371 & 0.9 \\
2011 & 1.061 & 0.94 & 0.9786 & 0.94 \\
2012 & 1.0443 & 1.02 & 1.0443 & 1.02 \\
2013 & 1.2358 & 1.12 & 1.1293 & 1.12 \\
2014 & 1.2374 & 1.18 & 1.1484 & 1.18 \\
\hline
\end{tabular}

单位:元/斤

\begin{table}
\centering
\caption{水稻品种的价格数据}
\begin{tabular}{c c c c c c}
\hline
 & 吉林(粳稻) & & 湖南(早稻) & & 江苏(中晚稻) \\
\hline
年份 & 市场价 & 最低价 & 市场价 & 最低价 & 市场价 & 最低价 \\
\hline
2006 & 0.9557 & 0.75 & 0.7095 & 0.7 & 0.754 & 0.72 \\
2007 & 0.8271 & 0.75 & 0.7596 & 0.7 & 0.8298 & 0.72 \\
2008 & 0.9385 & 0.82 & 0.8973 & 0.77 & 0.9057 & 0.79 \\
2009 & 1.1044 & 0.95 & 0.8952 & 0.9 & 0.9645 & 0.92 \\
2010 & 1.4226 & 1.05 & 0.9773 & 0.93 & 1.1262 & 0.97 \\
2011 & 1.4586 & 1.28 & 1.1788 & 1.02 & 1.3327 & 1.07 \\
2012 & 1.4967 & 1.4 & 1.2932 & 1.2 & 1.2553 & 1.25 \\
2013 & 1.4392 & 1.5 & 1.2563 & 1.32 & 1.3146 & 1.35 \\
2014 & 1.5218 & 1.55 & 1.3154 & 1.35 & 1.3515 & 1.38 \\
\hline
\end{tabular}
\end{table}

单位:元/斤

\subsection*{4.2 价格规律分析}

在本节我们对已经搜集整理好的数据分析。分别是作图和软件操作分析,具体分析结果见下文。

\subsubsection{4.2.1 价格规律的直观描述}

我们编写了简单的 matlab 程序,使用 matlabR2015b 运行所得结果如下:

(1) 五个省份的粮食价格规律图

\begin{figure}[h]
\centering
\includegraphics[width=\textwidth]{image.png}
\caption{粮食价格规律图}
\end{figure}

(3) 对五个省份的价格规律分析

从以上五个图来看,各省粮食的市场价和最低收购价虽然在某些年份存在着此消彼长的波动趋势,但是整体上两种价格是呈上升趋势的。而且他们之间的上升趋势具有一定的同步性,我们决定使用 Pearson 相关系数来研究它们之间的趋同程度。

\subsection*{4.2.2 价格规律的定量分析}

在本节我们使用 SAS 软件对以上整理数据进行 Pearson 相关系数分析,具体分析结果如下:

\begin{table}[h]
\centering
\caption{Pearson 相关系数分析表}
\begin{tabular}{c c c}
\hline
 & 皮尔森相关系数 & sig 值 \\
\hline
\multirow{2}{*}{小麦} & 安徽 & 0.991 \\
 & 山东 & 0.971 \\
 & 吉林 & 0.918 \\
\multirow{2}{*}{水稻} & 湖南 & 0.956 \\
 & 江苏 & 0.928 \\
\hline
\end{tabular}
\end{table}

从上表可以看出,五个省份的相关系数值都通过了检验,而且 sig 值远远小于 0.05,说明各个省份的粮食市场价和粮食最低收购价之间的相关性很强。具体结果为:安徽小麦的市场价和最低收购价之间的相关系数为 0.991;山东小麦的市场价和最低收购价之间的相关系数为 0.971;吉林水稻的市场价和最低收购价之间的相关系数为 0.918;湖南水稻的市场价和最低收购价之间的相关系数为 0.956;江苏水稻的市场价和最低收购价之间的相关系数为 0.928。

\subsection*{4.3 价格理论模型的建立}

运用预期理论模型和局部调整模型,力图建立供给、需求以及收购价格模型构成的联立方程组模型,统称为粮食市场价格模型。需要说明的是,为模型书写的方便,在预警模型中的子模型中可能重复一些系数,例如 $a_{0}, a_{1}, a_{2}$ 等,但它在不同模型中表达含义并不一样。

\subsubsection{4.3.1 价格模型的相关假设及符号说明}

(1)模型的相关假设

在宏观经济的范畴内,价格主要由供给量与需求量的平衡而形成的。但不可否认的是,整个市场环境错综复杂,人们的偏好、不同地域的经济发展水平、环境因素、政策法规等诸多因素都会影响到价格的形成。我们必须对现实世界的经济运行规律在合理假设的基础上进行简化,提炼出核心要素。这样既有利于模型的计算和运用,又符合经济学理论,从而建立出价格模型。具体假设有:

假设 1:粮食的预期供给量与同期价格密切相关。根据供给理论,在市场经济条件下,如果粮食的预期价格高,生产者就愿意扩大生产,向市场提供产品,供应量就会提高;如果粮食预期市场价格低,有些生产者就会退出粮食生产,转而生产其它利润高的产品,从而导致粮食市场供应量减少。

假设 2:粮食生产者价格不是影响粮食供给变化的唯一因素。根据第 1 问的 spearman 结果可见,除了价格外,不同省的粮食供应量 $GQ_{t}$ 还可能与农业劳动力人口、农民受教育程度、城乡收入差距、家庭负担、市场价格、生产成本、相对收益竞争力、农作物播种面积、受灾面积、净出口、城市化水平等有关。

假设 3:粮食供给量与其影响因素之间是线性相关关系。在经济模型中,柯布-道格拉斯函数一直以弹性稳定、操作简单和解释明晰而得到广泛推广。本文中粮食供应量与其影响因素之间假定是线性关系。这样拟合出来的系数就是供给量对各因素的弹性。同样,粮食消费量与其影响因素之间、价格与其影响因素之间均选择线性模型。

假设 4:粮食供应量中的库存变化量以及进出口量差值可以忽略。因为每年的粮食库存量变化不大,而且进出口量差值相比当年产量是微乎其微的。为简化研究,在此省略库存量和进出口量差值。

假设 5:粮食的销售价格主要受到供需平衡下粮食的均衡产量的影响。价格是由供需双方博弈产生的。在供需平衡条件下,市场达到均衡。此时供给量与需求量相等。由于消费量受到许多模糊信息的影响,例如损耗量、加工量,数据的获取很难精确。因此在这里用均衡条件下的粮食供给量作为均衡产量的替代是可以的。

假设 6:粮食的销售价格还要受到其它因素如最低收购价政策的影响。由问题二的分析我们知道,粮食的最低收购价与粮食的销售价格由很大的关系。

假设 7:粮食市场预警模型不考虑金融危机的影响。

(2)符号说明

\begin{tabular}{ll}
\hline 符号 & 符号说明 \\
\hline $GQ$ & 粮食供应量 \\
$SP_{t}$ & 粮食的收购价格(生产者价格) \\
$CQ_{t}$ & 粮食国内消费量 \\
$Q_{t}$ & 最低收购价政策 \\
$P_{t}$ & 粮食的市场销售价格 \\
$I_{t}$ & 居民的可支配收入 \\
$Z_{t}$ & 主成分 \\
$CQ_{t}^{*}$ & 粮食预期的消费量 \\
$GQ_{t}^{*}$ & 粮食的均衡产量 \\
$PE_{tj}$ & $t$ 年 $j$ 省的农业人口 \\
$O_{tij}$ & $j$ 省单人 1 年食用 $i$ 粮食的总量 \\
$S_{tij}$ & $t$ 年 $i$ 种粮食 $j$ 省的播种面积 \\
$D_{ij}$ & $j$ 省 $i$ 粮食的单产 \\
\hline
\end{tabular}

\subsection*{4.3.2 价格模型的建立}

\subsubsection{4.3.2.1 粮食供给量模型的构建}

粮食供应量是指能够向国内消费者提供的小麦和水稻的量。它包括白小麦、红小麦和混合小麦,以及早籼稻、中晚籼稻和粳稻。

在利益最大化的驱动下,粮食供应主要依赖于粮食的收购价格(也称为生产者价格)$SP_{t}$ 和最低收购价政策 $Q_{t}$。依据假设 1,2,3,4,得到粮食预期供应量的函数为:
\begin{equation}
GQ_{t}^{*} = a_{0} + a_{1}SP_{t} + a_{2}Q_{t} + e_{t}
\tag{3-1}
\end{equation}
其中 $e_{t}$ 是随机误差,表明除了价格之外的其它影响粮食供应量的因素的综合量,且要求 $e_{t} \sim N\left(\mu, \sigma^{2}\right)$。

由于 $GQ_{t}^{*}$ 是目标供应量,不可观测。由于生产管理,生产条件的波动以及其它不可预知的原因,粮食供应量 $GQ_{t}$ 的实际变化量只是预期变化量的一部分。

粮食供应量按照预定水平逐步进行调整,故有如下局部调整假设:
\begin{equation}
GQ_{t} - GQ_{t-1} = \delta \left( GQ_{t}^{*} - GQ_{t-1} \right)
\tag{3-2}
\end{equation}
或写成
\begin{equation}
GQ_{t} = \delta GQ_{t}^{*} + \left( 1 - \delta \right) GQ_{t-1}
\tag{3-3}
\end{equation}
其中 \(\delta\) 为调整系数,\(0 \leq \delta \leq 1\)。通过调整系数可以减少实际供应量与预期供应量的差距。将(3-2)式代入(3-1)式,得到粮食实际供应量为:
\begin{equation}
GQ_{t} = \delta a_{0} + \left( 1 - \delta \right) GQ_{t-1} + \delta a_{1} SP_{t} + \delta a_{2} Q_{t} \delta + e_{t}
\tag{3-4}
\end{equation}

### 4.3.2.2 粮食国内消费量模型的构建

粮食国内消费量 \(CQ_{t}\) 主要是与粮食的市场销售价格 \(P_{t}\)、最低收购价政策 \(Q_{t}\) 和居民的可支配收入 \(I_{t}\) 有关。建立粮食预期的消费量 \(CQ_{t}^{*}\) 表达式为:
\begin{equation}
CQ_{t}^{*} = a_{0} + a_{1} SP_{t} + a_{2} Q_{t} + \sum_{t=0}^{\infty} b_{i} I_{t-i} + \mu_{t}
\tag{3-5}
\end{equation}
假设 \(b_{i}\) 随滞后期 \(i\) 按几何级数衰减,运用科伊克(Koyck)变换,即
\begin{equation}
b_{i} = b_{0} \lambda^{i}
\tag{3-6}
\end{equation}
其中 \(0 < \lambda < 1\),称为分布滞后衰减率,\(1 - \lambda\) 称为调整速率。

将(3-6)式代入(3-5)式得:
\begin{equation}
CQ_{t}^{*} = a_{0} + a_{1} SP_{t} + a_{2} Q_{t} + b_{0} \sum_{t=0}^{\infty} \lambda^{i} I_{t-i} + \mu_{t}
\tag{3-7}
\end{equation}
滞后一期且乘以 \(\lambda\),得:
\begin{equation}
\lambda CQ_{t-1}^{*} = \lambda a_{0} + \lambda a_{1} SP_{t-1} + \lambda a_{2} Q_{t-1} + b_{0} \sum_{t=1}^{\infty} \lambda^{i} I_{t-i} + \lambda \mu_{t-1}
\tag{3-8}
\end{equation}
将(3-7)式减去(3-8)式得科伊克变换模型:
\begin{equation}
CQ_{t}^{*} - \lambda CQ_{t-1}^{*} = \left( 1 - \lambda \right) a_{0} + \left( 1 - \lambda \right) a_{1} SP_{t} + \left( 1 - \lambda \right) a_{2} Q_{t} + b_{0} I_{t} + \mu_{t} - \lambda \mu_{t-1}
\end{equation}
整理得:
\begin{equation}
CQ_{t}^{*} = \lambda CQ_{t-1}^{*} + \left( 1 - \lambda \right) a_{0} + \left( 1 - \lambda \right) a_{1} SP_{t} + \left( 1 - \lambda \right) a_{2} Q_{t} + b_{0} I_{t} + \left( 1 - \lambda \right) \mu_{t-1}
\tag{3-9}
\end{equation}
由于 \(CQ_{t}^{*}\) 与 \(CQ_{t-1}^{*}\) 是预期国内消费量,无法观测。同时由于营销渠道、市场的波动以及国民经济的发展等方面的原因,消费量的实际变化量是预期变化量的一部分,设置如下局部调整假设:
\begin{equation}
CQ_{t} - CQ_{t-1} = \delta \left( CQ_{t}^{*} - CQ_{t-1} \right)
\end{equation}
也可写成:
\begin{equation}
CQ_{t} = \delta CQ_{t}^{*} + \left( 1 - \delta \right) CQ_{t-1}
\tag{3-10}
\end{equation}
滞后一期得:
\begin{equation}
CQ_{t-1} = \delta CQ_{t-1}^{*} + \left( 1 - \delta \right) CQ_{t-2}
\tag{3-11}
\end{equation}
将(3-10)和(3-11)式代入(3-9)式得:
\begin{equation}
CQ_{t} - \left( 1 - \delta \right) CQ_{t-1} = \lambda CQ_{t-1} - \lambda \left( 1 - \delta \right) CQ_{t-2} + \delta \left( 1 - \lambda \right) a_{0} + \delta \left( 1 - \lambda \right) a_{1} SP_{t} +
\end{equation}
\begin{equation}
\delta \left( 1 - \lambda \right) a_{2} Q_{t} + \delta b_{0} I_{t} + \delta \left( 1 - \lambda \right) \mu_{t-1}
\tag{3-12}
\end{equation}
于是整理得到粮食国内消费模型为:
\begin{equation}
CQ_{t} = \left( 1 - \delta + \lambda \right) CQ_{t-1} - \lambda \left( 1 - \delta \right) CQ_{t-2} + \delta \left( 1 - \lambda \right) a_{0} + \delta \left( 1 - \lambda \right) a_{1} SP_{t} +
\end{equation}
\begin{equation}
\delta \left( 1 - \lambda \right) a_{2} Q_{t} + \delta b_{0} I_{t} + \delta \left( 1 - \lambda \right) \mu_{t-1}
\tag{3-13}
\end{equation}

\subsection{4.3.2.3 粮食销售价格模型的建立}

中国粮食市场预警辅助模型有 1 个子模型,即销售价格模型。下面为粮食销售价格模型的构建。

粮食销售价格的高低主要是由生产者提供的数量、消费者的购买数量和最低收购价所决定。作为市场经济操作手,价格对供给和需求起着调节作用。粮食销售价格与供给关系如下:
\begin{equation}
SP_t = a_0 + a_1 GQ_t^* + a_2 Q_t + e_t
\tag{3-14}
\end{equation}
其中 $GQ_t^*$ 为粮食供求平衡下粮食的均衡产量,$e_t$ 为误差项,$e_t \sim N\left(\mu, \sigma^2\right)$。

由于 $GQ_t^*$ 是一个无法观测的数据,因此需要进一步利用预期形式的假设,即自适应预期假设有:
\begin{equation}
GQ_t^* - GQ_{t-1}^* = \delta \left(GQ_t - GQ_{t-1}^*\right)
\tag{3-15}
\end{equation}
其中 $0 \leq \delta \leq 1$ 为修正系数,也就是说粮食均衡产量是在现有产量的基础上,通过修正同期误差实现的。即:
\begin{equation}
GQ_t^* = \delta GQ_t + (1-\delta) GQ_{t-1}^*
\tag{3-16}
\end{equation}
将(3-26)式代入(3-24)式得:
\begin{equation}
SP_t = a_0 + a_1 \left[\delta GQ_t + (1-\delta) GQ_{t-1}^*\right] + a_2 Q_t + e_t
\end{equation}
上式经过科伊克变换可变为:
\begin{equation}
SP_t = a_0 + a_1 \delta GQ_t + (1-\delta)(SP_{t-1} - a_0 - a_2 Q_{t-1} - e_{t-1}) + a_2 Q_t + e_t
\tag{3-17}
\end{equation}
进一步得到销售价格模型:
\begin{equation}
SP_t = \delta a_0 + a_1 \delta GQ_t + (1-\delta) SP_{t-1} + a_2 Q_t - (1-\delta) a_2 Q_{t-1} + e_t - (1-\delta) e_{t-1}
\tag{3-18}
\end{equation}

\subsection{4.3.3 模型总结}

我们运用经济预期理论、局部调整模型导出了粮食供给模型、消费量模型、销售价格模型 3 个模型。粮食市场预警模型就是供给量模型和需求量模型,我们将两者之间用一个价格模型相联系。

粮食市场预警模型包括以下 3 个子模型:

供给量模型:
\begin{equation}
GQ_t = \delta a_0 + (1-\delta) GQ_{t-1} + \delta a_1 SP_t + \delta a_2 Q_t \varepsilon_t
\end{equation}

消费量模型:
\begin{equation}
CQ_t = (1-\delta + \lambda) CQ_{t-1} - \lambda (1-\delta) CQ_{t-2} + \delta (1-\lambda) a_0 + \delta (1-\lambda) a_1 SP_t +
\end{equation}
\begin{equation}
\delta (1-\lambda) a_2 Q_t + \delta b_0 I_t + \delta (1-\lambda) \mu_{t-1}
\end{equation}

销售价格模型:
\begin{equation}
SP_t = \delta a_0 + a_1 \delta GQ_t + (1-\delta) SP_{t-1} + a_2 Q_t - (1-\delta) a_2 Q_{t-1} + e_t - (1-\delta) e_{t-1}
\end{equation}

这 3 个模型相互联系,模型拟合是整体完成且共同组成一个系统,有别于单方程的单个拟合。

3 个模型 $\mathbf{e}_t$ 或 $\mathbf{e}_{t-1}$,包含了未知因素对模型的被解释变量的影响。在下面的实证分析中,需要在模型拟合中对变量进行经济诊断和统计诊断,因为粮食被解释变量有可能和模型中已有的解释变量以外的因素有关。

任何经济现象都有其特殊性,因此理论模型提供的框架在实证分析中并不一定完全相同。所以在做实证分析时,理论模型需要作进一步的调整。

消费量模型的构建既用到了局部调整模型,又用到了分布滞后模型转化为

自回归模型的理论,而且在供需平衡价格模型的基础上加入了粮食最低收购价指标,模型的构建具有创新性。

\section{价格实证模型的建立}

\subsection{安徽小麦实证模型的建立及总结}

安徽小麦实证模型包括供给量模型、消费量模型和销售价格模型。结合 4.3 节理论模型的解释变量来完成最终实证模型的构建。

\subsubsection{供给量实证模型的建立}

我们对第一问通过 spearman 检验的 8 个指标,分别为 x1 农业劳动人口、x2 农民受教育程度、x4 家庭负担、x5 市场价格、x6 生产成本、x8 农作物播种面积、x9 受灾面积和 x11 城市化水平进行主成分分析,得到主成分的特征值与贡献率。见表 3.7。

表 3.7 的结果显示,第一主成分累计解释方差的比率已经超过了 80.73\%,故可选取第 1 个主成分代替这 8 个影响因素指标。

将安徽小麦供给量对第一个主成分进行相关性检验,得到与第一个主成分的相关系数为 0.50376,其检验 P 值为 $<.0001$。说明小麦与第一主成分有明显的关联性。

因此,供给量模型可以调整为:
\[
GQ_t = \delta a_{10} + (1-\delta) GQ_{t-1} + \delta a_{11} SP_t + \delta a_{12} Q_t \delta + a_{13} Z_{1t} + e_{1t}
\]
其中 $a_{13}$ 为第一主成分的滞后项的偏相关系数。$e_{1t}$ 是随机误差,要求 $e_{1t} \sim N\left(0, \sigma^2\right)$。

\subsubsection{消费量实证模型的建立}

同理我们将消费量模型可以调整为:
\[
CQ_t = (1-\delta + \lambda) CQ_{t-1} - \lambda (1-\delta) CQ_{t-2} + \delta (1-\lambda) a_{20} + \delta (1-\lambda) a_{21} SP_t +
\]
\[
\delta (1-\lambda) a_{22} Q_t + \delta b_{20} I_t + \delta (1-\lambda) a_{23} Z_{2,t-1} + e_{2t}
\]
其中 $\delta (1-\lambda) a_{23}$ 为第一主成分的滞后项的偏相关系数。$e_{2t}$ 是随机误差,要求 $e_{2t} \sim N\left(0, \sigma^2\right)$。

\subsubsection{价格实证模型的建立}

同理我们将消费量模型可以调整为:
\[
SP_t = \delta a_{30} + a_{31} \delta GQ_t + (1-\delta) SP_{t-1} + a_{32} Q_t - (1-\delta) a_{32} Q_{t-1} + a_{33} Z_{3t} - (1-\delta) a_{34} Z_{3,t-1} + e_{3t}
\]
其中 $a_{33}, -(1-\delta) a_{34}$ 为第一主成分的滞后项的偏相关系数。$e_{3t}$ 是随机误差,要求 $e_{3t} \sim N\left(0, \sigma^2\right)$。

\subsubsection{上面三个模型的总结}
\begin{align*}
GQ_t &= \beta_{1,0} + \beta_{1,1} GQ_{t-1} + \beta_{1,2} SP_t + \beta_{1,3} Q_t + \beta_{1,4} Z_{1,t} + e_{1t} \\
CQ_t &= \beta_{2,0} + \beta_{2,1} CQ_{t-1} + \beta_{2,2} CQ_{t-2} + \beta_{2,3} SP_t + \beta_{2,4} Q_t + \beta_{2,5} I_t + \beta_{2,6} Z_{2,t-1} + e_{2t} \\
SP_t &= \beta_{3,0} + \beta_{3,1} GQ_t + \beta_{3,2} SP_{t-1} + \beta_{3,3} Q_t + \beta_{3,4} Q_{t-1} + \beta_{3,5} Z_{3t} + \beta_{3,6} Z_{3,t-1} + e_{3t}
\end{align*}
其中,$GQ_t$ 指粮食供应量,$CQ_t$ 为消费量,$SP_t$ 指销售价格,$P_t$ 指市场价格,$Q_t$ 指最低收购价政策,$Z_t$ 指主成分,$I_t$ 指居民的可支配收入。

\section*{4.4.2 山东小麦实证模型的建立及总结}

同理山东小麦实证模型包括供给量模型、消费量模型和销售价格模型。结合 4.3 节理论模型的解释变量来完成最终实证模型的构建。

(1) 供给量实证模型的建立

我们对第一问通过 spearman 检验的 6 个指标,分别为 x1 农业劳动人口、x2 农民受教育程度、x3 城乡收入差距、x4 家庭负担、x10 净出口、x11 城市化水平进行主成分分析,得到主成分的特征值与贡献率。见表。

表 的结果显示, 第一主成分累计解释方差的比率已经超过了 69.82\%, 故可选取第 1 个主成分代替这 6 个影响因素指标。

将山东小麦供给量对第一个主成分进行相关性检验, 得到与第一个主成分的相关 t 值为-10.67, 其检验 P 值为 $<.0001$。说明小麦与第一主成分有明显的关联性。

因此, 供给量模型可以调整为:
\[
GQ_{t} = \delta a_{10} + (1-\delta) GQ_{t-1} + \delta a_{11} SP_{t} + \delta a_{12} Q_{t} + a_{13} Z_{1t} + e_{1t}
\]
其中 $a_{13}$ 为第一主成分的滞后项的偏相关系数。$e_{1t}$ 是随机误差, 要求 $e_{1t} \sim N\left(0, \sigma^{2}\right)$。

(2) 消费量实证模型的建立

同理我们将消费量模型可以调整为:
\[
\begin{aligned}
CQ_{t} = & (1-\delta + \lambda) CQ_{t-1} - \lambda (1-\delta) CQ_{t-2} + \delta (1-\lambda) a_{20} + \delta (1-\lambda) a_{21} SP_{t} + \\
& \delta (1-\lambda) a_{22} Q_{t} + \delta (1-\lambda) a_{23} Z_{2,t-1} + e_{2t}
\end{aligned}
\]
其中 $\delta (1-\lambda) a_{23}$ 为第一主成分的滞后项的偏相关系数。$e_{2t}$ 是随机误差, 要求 $e_{2t} \sim N\left(0, \sigma^{2}\right)$。

(3) 价格实证模型的建立

同理我们将消费量模型可以调整为:
\[
SP_{t} = \delta a_{30} + a_{31} \delta GQ_{t} + (1-\delta) SP_{t-1} + a_{32} Q_{t} - (1-\delta) a_{33} Z_{3t} - (1-\delta) a_{34} Z_{3,t-1} + e_{3t}
\]
其中 $a_{33}, -(1-\delta) a_{34}$ 为第一主成分的滞后项的偏相关系数。$e_{3t}$ 是随机误差, 要求 $e_{3t} \sim N\left(0, \sigma^{2}\right)$。

4) 上面三个模型的总结
\[
\begin{aligned}
GQ_{t} & = \beta_{4,0} + \beta_{4,1} GQ_{t-1} + \beta_{4,2} SP_{t} + \beta_{4,3} Q_{t} + \beta_{4,4} Z_{1t} + e_{1t} \\
CQ_{t} & = \beta_{5,0} + \beta_{5,1} CQ_{t-1} + \beta_{5,2} CQ_{t-2} + \beta_{5,3} SP_{t} + \beta_{5,4} Q_{t} + \beta_{5,5} Z_{5,t-1} + e_{2t} \\
SP_{t} & = \beta_{6,0} + \beta_{6,1} GQ_{t} + \beta_{6,2} SP_{t-1} + \beta_{6,3} Q_{t} + \beta_{6,4} Q_{t-1} + \beta_{6,5} Z_{3t} + \beta_{6,6} Z_{6,t-1} + e_{3t}
\end{aligned}
\]
其中, $GQ_{t}$ 指粮食供应量, $CQ_{t}$ 为消费量, $SP_{t}$ 指销售价格, $P_{t}$ 指市场价格, $Q_{t}$ 指最低收购价政策, $Z_{t}$ 指主成分。

\section*{4.4.3 湖南水稻实证模型的建立及总结}

同理湖南水稻实证模型包括供给量模型、消费量模型和销售价格模型。结合 4.3 节理论模型的解释变量来完成最终实证模型的构建。

(1) 供给量实证模型的建立

我们对第一问通过 spearman 检验的 6 个指标, 分别为 x2 农民受教育程度、x4 家庭负担、x5 市场价格、x6 生产成本、x8 农作物播种面积、x11 城市化水平进

行主成分分析,得到主成分的特征值与贡献率。见表 。

表 的结果显示, 第一主成分累计解释方差的比率已经超过了 82.35%, 故可选取第 1 个主成分代替这 6 个影响因素指标。

将湖南水稻供给量对第一个主成分进行相关性检验, 得到与第一个主成分的相关 $t$ 值为 3.71, 其检验 $P$ 值为 0.0026 。说明小麦与第一主成分有明显的关联性。

因此, 供给量模型可以调整为:
\[
G Q_{t}=\delta a_{10}+\left(1-\delta\right) G Q_{t-1}+\delta a_{11} S P_{t}+\delta a_{12} Q_{t} \delta+a_{13} Z_{1 t}+e_{1 t}
\]
其中 $a_{13}$ 为第一主成分的滞后项的偏相关系数。 $e_{1 t}$ 是随机误差, 要求 $e_{1 t} \sim N\left(0, \sigma^{2}\right)$ 。

(2) 消费量实证模型的建立

同理我们将消费量模型可以调整为:
\[
\begin{aligned}
C Q_{t}= & (1-\delta+\lambda) C Q_{t-1}-\lambda(1-\delta) C Q_{t-2}+\delta(1-\lambda) a_{20}+\delta(1-\lambda) a_{21} S P_{t}+ \\
& \delta(1-\lambda) a_{22} Q_{t} 4+\delta(1-\lambda) a_{23} Z_{2, t-1}+e_{2 t}
\end{aligned}
\]
其中 $\delta(1-\lambda) a_{23}$ 为第一主成分的滞后项的偏相关系数。 $e_{2 t}$ 是随机误差, 要求 $e_{2 t} \sim N\left(0, \sigma^{2}\right)$ 。

(3) 价格实证模型的建立

同理我们将消费量模型可以调整为:
\[
S P_{t}=\delta a_{30}+a_{31} \delta G Q_{t}+(1-\delta) S P_{t-1}+a_{32} Q_{t}-(1-\delta) a_{32} Q_{t-1}+a_{33} Z_{3 t}-(1-\delta) a_{34} Z_{3, t-1}+e_{3 t}
\]
其中 $a_{33},-(1-\delta) a_{34}$ 为第一主成分的滞后项的偏相关系数。 $e_{3 t}$ 是随机误差, 要求 $e_{3 t} \sim N\left(0, \sigma^{2}\right)$ 。

4) 上面三个模型的总结
\[
\begin{aligned}
G Q_{t}= & \beta_{7,0}+\beta_{7,1} G Q_{t-1}+\beta_{7,2} S P_{t}+\beta_{7,3} Q_{t}+\beta_{7,4} Z_{7 t}+e_{1 t} \\
C Q_{t}= & \beta_{8,0}+\beta_{8,1} C Q_{t-1}+\beta_{8,2} C Q_{t-2}+\beta_{8,3} S P_{t}+\beta_{8,4} Q_{t}+\beta_{8,5} Z_{2, t-1}+e_{2 t} \\
S P_{t}= & \beta_{9,0}+\beta_{9,1} G Q_{t}+\beta_{9,2} S P_{t-1}+\beta_{9,3} Q_{t}+\beta_{9,4} Q_{t-1}+\beta_{9,5} Z_{9 t}+\beta_{9,6} Z_{9, t-1}+e_{3 t}
\end{aligned}
\]
其中, $G Q_{t}$ 指粮食供应量, $C Q_{t}$ 为消费量, $S P_{t}$ 指销售价格, $P_{t}$ 指市场销售价格, $Q_{t}$ 指最低收购价政策, $Z_{t}$ 指主成分。

\subsection*{4.4.4 吉林水稻实证模型的建立及总结}

同理吉林水稻实证模型包括供给量模型、消费量模型和销售价格模型。结合 4.3 节理论模型的解释变量来完成最终实证模型的构建。

(1) 供给量实证模型的建立

我们对第一问通过 spearman 检验的 6 个指标, 分别为 x2 农民受教育程度、x4 家庭负担、x5 市场价格、x6 生产成本、x8 农作物播种面积、x10 净出口、x11 城市化水平进行主成分分析, 得到主成分的特征值与贡献率。见表 。

表 的结果显示, 第一主成分累计解释方差的比率已经超过了 80.53%, 故可选取第 1 个主成分代替这 6 个影响因素指标。

将吉林水稻供给量对第一个主成分进行相关性检验, 得到与第一个主成分的相关 $t$ 值为 10.49, 其检验 $P$ 值为 $<.0001$ 。说明小麦与第一主成分有明显的关联性。

因此, 供给量模型可以调整为:

\begin{align*}
GQ_t &= \delta a_{10} + (1-\delta)GQ_{t-1} + \delta a_{11}SP_t + \delta a_{12}Q_t\delta + a_{13}Z_{1t} + e_{1t}
\end{align*}
其中 $a_{13}$ 为第一主成分的滞后项的偏相关系数。$e_{1t}$ 是随机误差,要求 $e_{1t} \sim N\left(0, \sigma^2\right)$。

(2) 消费量实证模型的建立

同理我们将消费量模型可以调整为:
\begin{align*}
CQ_t &= (1-\delta + \lambda)CQ_{t-1} - \lambda(1-\delta)CQ_{t-2} + \delta(1-\lambda)a_{20} + \delta(1-\lambda)a_{21}SP_t \\
&\quad + \delta(1-\lambda)a_{22}Q_t4 + \delta(1-\lambda)a_{23}Z_{2,t-1} + e_{2t}
\end{align*}
其中 $\delta(1-\lambda)a_{23}$ 为第一主成分的滞后项的偏相关系数。$e_{2t}$ 是随机误差,要求 $e_{2t} \sim N\left(0, \sigma^2\right)$。

(3) 价格实证模型的建立

同理我们将消费量模型可以调整为:
\begin{align*}
SP_t &= \delta a_{30} + a_{31}\delta GQ_t + (1-\delta)SP_{t-1} + a_{32}Q_t - (1-\delta)a_{32}Q_{t-1} + a_{33}Z_{3t} - (1-\delta)a_{34}Z_{3,t-1} + e_{3t}
\end{align*}
其中 $a_{33}, -(1-\delta)a_{34}$ 为第一主成分的滞后项的偏相关系数。$e_{3t}$ 是随机误差,要求 $e_{3t} \sim N\left(0, \sigma^2\right)$。

4) 上面三个模型的总结
\begin{align*}
GQ_t &= \beta_{10,0} + \beta_{10,1}GQ_{t-1} + \beta_{10,2}SP_t + \beta_{10,3}Q_t + \beta_{10,4}Z_{10,t} + e_{1t} \\
CQ_t &= \beta_{11,0} + \beta_{11,1}CQ_{t-1} + \beta_{11,2}CQ_{t-2} + \beta_{11,3}SP_t + \beta_{11,4}Q_t + \beta_{11,5}Z_{11,t-1} + e_{2t} \\
SP_t &= \beta_{12,0} + \beta_{12,1}GQ_t + \beta_{12,2}SP_{t-1} + \beta_{12,3}Q_t + \beta_{12,4}Q_{t-1} + \beta_{12,5}Z_{3t} + \beta_{12,6}Z_{12,t-1} + e_{3t}
\end{align*}
其中,$GQ_t$ 指粮食供应量,$CQ_t$ 为消费量,$SP_t$ 指销售价格,$P_t$ 指市场价格,$Q_t$ 指最低收购价政策,$Z_t$ 指主成分。

\subsection{4.4.5 江苏水稻实证模型的建立及总结}

同理江苏水稻实证模型包括供给量模型、消费量模型和销售价格模型。结合 4.3 节理论模型的解释变量来完成最终实证模型的构建。

(1) 供给量实证模型的建立

我们对第一问通过 spearman 检验的 5 个指标,分别为 x3 城乡收入差距、x6 生产成本、x8 农作物播种面积、x9 受灾面积、x10 净出口进行主成分分析,得到主成分的特征值与贡献率。见表。

表 的结果显示,第一主成分累计解释方差的比率已经超过了 75.71\%,故可选取第 1 个主成分代替这 5 个影响因素指标。

将江苏水稻供给量对第一个主成分进行相关性检验,得到与第一个主成分的相关 t 值为 8.8,其检验 P 值为 0.09369。说明小麦与第一主成分有明显的关联性。

因此,供给量模型可以调整为:
\begin{align*}
GQ_t &= \delta a_{10} + (1-\delta)GQ_{t-1} + \delta a_{11}SP_t + \delta a_{12}Q_t\delta + a_{13}Z_{1t} + e_{1t}
\end{align*}
其中 $a_{13}$ 为第一主成分的滞后项的偏相关系数。$e_{1t}$ 是随机误差,要求 $e_{1t} \sim N\left(0, \sigma^2\right)$。

(2) 消费量实证模型的建立

同理我们将消费量模型可以调整为:

\begin{align*}
CQ_t &= (1-\delta+\lambda)CQ_{t-1} - \lambda(1-\delta)CQ_{t-2} + \delta(1-\lambda)a_{20} + \delta(1-\lambda)a_{21}SP_t \\
&\quad + \delta(1-\lambda)a_{22}Q_t4 + \delta(1-\lambda)a_{23}Z_{2,t-1} + e_{2t}
\end{align*}

其中 $\delta(1-\lambda)a_{23}$ 为第一主成分的滞后项的偏相关系数。$e_{2t}$ 是随机误差,要求 $e_{2t} \sim N\left(0, \sigma^2\right)$。

(3) 价格实证模型的建立

同理我们将消费量模型可以调整为:
\begin{equation}
SP_t = \delta a_{30} + a_{31}\delta GQ_t + (1-\delta)SP_{t-1} + a_{32}Q_t - (1-\delta)a_{32}Q_{t-1} + a_{33}Z_{3t} - (1-\delta)a_{34}Z_{3,t-1} + e_{3t}
\end{equation}

其中 $a_{33}, -(1-\delta)a_{34}$ 为第一主成分的滞后项的偏相关系数。$e_{3t}$ 是随机误差,要求 $e_{3t} \sim N\left(0, \sigma^2\right)$。

4) 上面三个模型的总结
\begin{align*}
GQ_t &= \beta_{13,0} + \beta_{13,1}GQ_{t-1} + \beta_{13,2}SP_t + \beta_{13,3}Q_t + \beta_{13,4}Z_{13,t} + e_{1t} \\
CQ_t &= \beta_{14,0} + \beta_{14,1}CQ_{t-1} + \beta_{14,2}CQ_{t-2} + \beta_{14,3}SP_t + \beta_{14,4}Q_t + \beta_{14,5}Z_{14,t-1} + e_{2t} \\
SP_t &= \beta_{15,0} + \beta_{15,1}GQ_t + \beta_{15,2}SP_{t-1} + \beta_{15,3}Q_t + \beta_{15,4}Q_{t-1} + \beta_{15,5}Z_{3t} + \beta_{15,6}Z_{15,t-1} + e_{3t}
\end{align*}

其中,$GQ_t$ 指粮食供应量,$CQ_t$ 为消费量,$SP_t$ 指销售价格,$P_t$ 指市场销售价格,$Q_t$ 指最低收购价政策,$Z_t$ 指主成分。

\subsection{4.5 价格实证模型的求解}

在上一节,我们建立了两个小麦主产省份的价格实证模型和三个水稻主产省份的价格实证模型,并且对每个模型的变量给出了详细的说明。在本节,我们将在之前搜集的数据的基础上,进一步整理出每个省份的实证模型的变量数据。最终我们使用 SAS 软件编程计算出了每个模型的系数,对每个价格实证模型求出了结果,不仅建立了供需双方通过市场调节与粮食最低收购价一起构成粮食价格体系,而且为问题五的优化模型的建立奠定了基础。

\subsubsection{4.5.1 各指标数据的意义和数据的搜集}

\subsubsection{4.5.1.1 各指标数据的意义}

(1) 供给量:该指标供给模型中充当因变量,在价格模型中充当自变量,是价格实证模型的主要指标。该指标衡量的是我国粮食生产力,我们选取各个省份的粮食产量作为供给量。该指标用符号 $GQ_t$ 表示。

(2) 需求量:该指标是国家在实施粮食最低收购价政策时的购买量,与国家在实行该政策时的财政支出相关。该指标的计算公式为:需求量 = 生产量 - 农民自留量 + 进口量 - 出口量。该指标用 $CQ_t$ 表示。

(3) 销售价格:是市场经济下供给平衡形成的价格,又称为市场价格。该指标用 $SP_t$ 表示。

(4) 粮食最低收购价:该指标是国家为了保护农业生产力给出的一个价格。该指标是从国家在某些省份开始实行粮食最低收购价政策时开始出现的。在搜集该指标时要考虑不同省份,不同年份,不同粮食作物的区别。该指标用 $Q_t$ 表示。

(5) 主成分:该指标是供需平衡价格模型中的外生变量,该指标的实质是一系列影响粮食面积的指标线性组合。该指标用 $Z_t$ 表示。

\subsection*{4.5.1.2 数据的整理}

正如 4.4 节实证模型所呈现的那样,每个模型在进行一系列的推倒演变之后,都含有一系列的变量,有的甚至包含一阶和二阶的滞后变量。有些指标需要很多其他指标的计算才能获得,再加上主成分变量所包含的一系列与粮食面积相关的指标,整个数据的整理过程的繁复以及数据量之庞大不亚于本文开篇数据的搜集,我们在本部分不再展示,具体的整理好的数据我们将以数据包的形式随论文一起上传,我们本节只展示利用该部分数据编程计算所得的模型结果。

\subsection*{4.5.1 安徽小麦实证模型的求解结果}

我们使用 SAS 软件对每个模型进行编程求解,其它省份也是如此,后文不再赘述。程序运行结果如下:

(1) 供给
\[
GQ_{t} = 1821.25 - 0.35GQ_{t-1} + 0.54SP_{t} - 475.1Q_{t} - 0.79X_{1} + 95.38X_{2} + 0.01X_{4} - 0.78X_{6}
\]
由上式可以看出,安徽小麦的供应量主要受到最低收购价政策的影响;前一年的粮食供应量与当年的粮食供应量是负相关的;粮食的市场价格越高,粮食供应量越大,且市场价格提高 1 元,粮食供应量增加 0.54 万吨;农民受教育程度对该省小麦的供应量影响也较大且成正相关;与农业劳动人口、小麦的生产成本成负相关,与家庭负担成正相关。

(2) 需求
\[
CQ_{t} = 4772.21 - 0.52CQ_{t-1} + 0.29CQ_{t-2} + 1.39SP_{t} - 824.28Q_{t} - 3.78X_{1} + 89.3X_{2} - 0.04X_{4}
\]
由上式可以看出,安徽省小麦的需求量主要受到最低收购价的影响,最低收购价越低,需求量越大;前一年的小麦需求量与当年的需求量是负相关的,而前两年的小麦需求量与当年的需求量是正相关的;粮食的市场价格越高,粮食供应量越大,且市场价格提高 1 元,粮食供应量增加 1.39 万吨;农民受教育程度对该省小麦的需求量影响也较大且成正相关;与农业劳动人口与家庭负担成负相关。

(3) 价格
\[
SP_{t} = 5014.42 + 0.09GQ_{t} + 1.09SP_{t-1} + 1.07Q_{t} - 1.57Q_{t-1} - 1.09X_{1} - 6.61X_{2}
\]
由上式可以看出,安徽省小麦的市场价格与小麦供应量、上一年小麦的市场价格、当年最低收购价格成正比,与上一年的最低收购价格、农业劳动人口、农民受教育程度成反比。

\subsection*{4.5.2 山东小麦实证模型的求解结果}

(1) 供给
\[
GQ_{t} = -5638.87 + 2GQ_{t-1} - 33.74SP_{t} - 242.74Q_{t} + 1.02X_{1} + 129.53X_{2} + 103.38X_{3} + 0.04X_{4}
\]
由上式可以看出,粮食最低收购价与市场价格对山东省小麦的供应量影响很大;该省小麦当年的供应量与上一年的供应量成正比,与农民受教育程度和城乡收入差距关系较大,且均成正比。

(2) 需求
\[
CQ_{t} = -3065.23 - 0.71CQ_{t-1} - 0.54CQ_{t-2} + 73.06SP_{t} + 5614.52Q_{t} + 0.49X_{1}
\]
由上式可知,粮食最低收购价与市场价对山东省小麦的需求量影响很大;当年小麦的需求量与前两年的需求量均成反比,与农业劳动力人口成正比。

(3) 价格
\[
SP_{t} = 2.3_{t} + 0.19SP_{t-1} + 1.5Q_{t} - 0.71Q_{t-1} - 0.06X_{2} - 1.34X_{3}
\]
由上式可以看出,山东省小麦的市场价格与小麦上一年的市场价格、当年最低收购价格成正比,与上一年的最低收购价格、农民受教育程度、城乡收入差距成反比。

\subsection*{4.5.3 湖南水稻实证模型的求解结果}

(1) 供给
\[
GQ_{t} = 3974.22 - 0.77GQ_{t-1} + 410.9SP_{t} + 1589.26Q_{t} + 100.99X_{2} - 0.27X_{4} - 0.17X_{6} - 0.12X_{8}
\]
湖南省水稻的供应量与市场价格与最低收购价成正比,与上一年的水稻供应量成反比,且最低收购价格每提高 1 元,水稻供应量提高 1589.2 万吨;该省水稻还与农民受教育程度成正比。

(2) 需求
\[
CQ_{t} = 3964.4 - 1.73CQ_{t-1} - 0.22CQ_{t-2} - 583.17SP_{t} + 3284.27Q_{t} - 21.63X_{2} + 0.07X_{4} - 1.84X_{6}
\]
该省水稻的需求量主要受到粮食最低收购价与市场价的影响,且市场价越高,水稻的需求量越低;当年水稻的需求量与前两年水稻的需求量均成反比。

(3) 价格
\[
SP_{t} = -0.49 - 1.56SP_{t-1} + 3.06Q_{t} - 1.46Q_{t-1} + 0.02X_{2}
\]
湖南省水稻当年的市场价与上一年的最低收购价与市场价成反比,与当年的最低收购价成正比。

\subsection*{4.5.4 吉林水稻实证模型的求解结果}

(1) 供给
\[
GQ_{t} = 11408 - 1.16GQ_{t-1} + 1286.6SP_{t} - 2319.46Q_{t} - 1458.49X_{2} - 0.12X_{4} + 4.14X_{6} + 0.32X_{8}
\]
吉林省水稻的供给量主要受到水稻市场价与最低收购价的影响,且市场价越高,供应量越高,市场价每提高 1 元,供应量提高 1286.6 万吨;当年水稻的供应量与上一年的水稻供应量成反比。

(2) 需求
\[
CQ_{t} = 870.65 - 0.79CQ_{t-1} + 0.45CQ_{t-2} + 106.07SP_{t} + 153.25Q_{t} - 128.67X_{2} + 0.08X_{4} + 0.15X_{6}
\]
吉林省水稻的需求量主要受到当年市场价与最低收购价的影响,且均成正比;当年需求量与一年前需求量成反比,与两年前需求量成正比。

(3) 价格
\[
SP_{t} = 93.07 + 0.02GQ_{t} - 13.11SP_{t-1} + 41.5Q_{t} + 40.83Q_{t-1} - 13.19X_{2} - 0.01X_{4}
\]
吉林省当年水稻的市场价与当年和前一年的最低收购价成正比,以前一年的市场价格成反比。

\subsection*{4.5.5 江苏水稻实证模型的求解结果}

(1) 供给
\[
GQ_{t} = 4849.09 - 0.71GQ_{t-1} + 25.64SP_{t} + 417.05Q_{t} - 281.442X_{3} - 0.01X_{6} - 0.19X_{8} + 0.37X_{9}
\]
江苏省水稻的供应量主要受到当年最低收购价的影响,最低收购价越高,供应量越大,且价格每提高 1 元,供应量提高 417.05 万吨;水稻供应量还与当年的市场价成正比,与一年前的水稻供应成反比。

(2) 需求

\begin{equation}
CQ_{t} = 12111 + 1.85CQ_{t-1} - 1.95CQ_{t-2} + 2507.02SP_{t} - 1494.4Q_{t} + 4996.43X_{3} + 4.47X_{6} - 3.58X_{8}
\end{equation}

江苏省水稻的需求量主要受到当年最低收购价与市场价的影响,且与一年前的水稻需求量成正比,与两年前的需求量成反比。

(3) 价格

\begin{equation}
SP_{t} = -20.98 + 0.01GQ_{t} + 0.6SP_{t-1} - 2.63Q_{t} + 1.31Q_{t-1} + 0.56X_{3}
\end{equation}

江苏省当年的市场价格与一年前的市场价与最低收购价成正比,与当年的最低收购价成反比。

\section*{第 5 章 问题四的解答}

\section*{——粮食最低收购价合理定价的线性规划模型}

在本章,我们利用第四章问题三的求解的价格实证模型作为价格波动约束条件;以各地粮食市场价格的 10\% 作为财政支出约束;考虑到现实世界粮食储备的空间限制,我们建立了库存约束;而且在城镇化日趋加剧的情况下,考虑到种植不会超过以往历年种植面积的最大值,我们建立了面积约束。利用以上约束,我们以供给量最大作为目标建立了关于水稻和小麦的价格模型。最后我们利用此模型以当年粮食最低收购价为基点,采用遍历法思想,在小范围内上调或者下降该价格,在满足其它约束的情况下,当植物产量最高时所对应的价格即为“最合理”的粮食最低收购价。最后我们通过比较优化模型得到的粮食最低收购价与当年的真实实施的粮食最低收购价,对粮食最低收购价的定价进行评价,并对 2017 年的粮食最低收购价的合理范围进行预测。

\subsection*{5.1 模型所涉及的指标变量说明}

(1) $GQ_{tij}$:该指标表示在 $t$ 年份粮食品种 $i$ 在 $j$ 省的粮食供给量。

(2) $W_{ij}$:该指标表示粮食品种 $i$ 在 $j$ 省实行粮食最低收购价政策时的财政支出。当粮食最低收购价小于市场价格时,该指标为 0;当粮食最低价高于市场价格时,该指标的值等于每单位产量粮食最低收购价与市场价格的差值和粮食产量的乘积。

(3) $KQ_{tij}$:该指标表示在 $t$ 年份粮食品种 $i$ 在 $j$ 省的粮食库存量,该指标是由该年份的人口量与每人每年的粮食食用量的乘积确定的。

(4) $S_{tij}$:该指标表示 $t$ 年 $i$ 种粮食 $j$ 省的播种面积,该指标是由该年份的粮食产量与单位面积产量的比值确定的。

(5) $PE_{tij}$:该指标表示在 $t$ 年份针对粮食品种 $i$ 分析时 $j$ 省的人口总量。

(6) $O_{tij}$:该指标表示在 $t$ 年份针对粮食品种 $i$ 分析时 $j$ 省的平均粮食食用量。

(7) $D_{tij}$:该指标表示在 $t$ 年份针对粮食品种 $i$ 分析时 $j$ 省的粮食单位面积产量。

\subsection*{5.2 线性规划理论模型的建立}

过高的粮食最低收购价不仅会提高粮食市场价格从而加重消费者负担,同时也会增加粮食的库存压力和国家财政的支出风险。另一方面,过低的粮食最低收购价会打压粮农种植粮食的积极性,造成粮食种植面积的萎缩,这更不是国家所愿意看到的。所以本节我们构造了四个约束条件,以粮食供应量作为目标函数,建立线性规划理论模型。以期确定合适的粮食最低收购价,是我国的粮食生产力达到一个很高的水平。

\subsubsection{5.2.1 目标函数的建立}

在我国实行粮食最低收购价政策的 $n$ 个省份中,我们的目标是使得这些省份的粮食产量之和达到最大。所以我们建立目标函数如下:

\[
\text{Maximize } \sum_{i=1}^{m} \sum_{j=1}^{n} GQ_{tij}
\]

\section*{5.2.2 约束条件}

(1) 价格波动约束

粮食销售价格的高低主要是由生产者提供的数量及消费者的购买数量所决定。价格作为市场经济操作手,对供给和需求起着调节作用。粮食的销售价格主要受到供需平衡下粮食的均衡产量的影响。价格是由供需双方博弈产生的。在供需平衡条件下,市场达到均衡,且此时供给量与需求量相等。由于消费量受到许多模糊信息的影响,数据的获取很难精确。因此在这里用均衡条件下的粮食供给量作为均衡产量的替代是可以的。基于以上分析,我们使用 4.4 节建立的价格模型作为线性型规划模型的价格波动约束,模型的具体形式如下:

\begin{align*}
GQ_t &= \beta_{1,0} + \beta_{1,1} GQ_{t-1} + \beta_{1,2} SP_t + \beta_{1,3} Q_t + \beta_{1,4} Z_{1,t} + e_{1t} \\
CQ_t &= \beta_{2,0} + \beta_{2,1} CQ_{t-1} + \beta_{2,2} CQ_{t-2} + \beta_{2,3} SP_t + \beta_{2,4} Q_t + \beta_{2,5} Z_{2,t-1} + e_{2t} \\
SP_t &= \beta_{3,0} + \beta_{3,1} GQ_t + \beta_{3,2} SP_{t-1} + \beta_{3,3} Q_t + \beta_{3,4} Q_{t-1} + \beta_{3,5} Z_{3t} + \beta_{3,6} Z_{3,t-1} + e_{3t}
\end{align*}

(2) 财政支出约束

我国自 2005 年起开始对粮食主产区实行了最低收购价政策,同时国家对于粮食的补贴金额也是有限制的,题目中指出在保持合理库存的前提下,一般不会超出各地粮食市场价格的 10%。因此我们确定下面的财政支出约束为:

\[
W_{tij} =
\begin{cases}
0, & Q_{ti} \leq SP_{tij}, \\
(Q_{ti} - SP_{tij}) * CQ_{tij}, & Q_{ti} > SP_{tij}.
\end{cases}
\]

\[
W_{tij} \leq 10\% * SP_{tij} * CQ_{tij}.
\]

该约束的含义是:若小麦或水稻的市场价格大于国家公布的最低收购价,则农民就会将粮食卖向市场,这时国家就不用补贴粮食差价。若小麦或水稻的市场价格小于国家公布的最低收购价,这时国家就不用补贴粮食差价,且差价金额也不会超过各地粮食市场价格的 10%。

(3) 库存约束

中国是一个“以粮为纲”的国家,存储的粮食一般要能够满足全国人民三年的吃饭和需求。由于粮食库存的有限性,国家收购的粮食总量不可能大于人民两年的吃饭需求,且应该大于人民一年的吃饭需求。

粮食库存约束:

\[
\sum_{j=1}^{n} K_{tij} \leq \sum_{j=1}^{n} CQ_{tij} \leq 2 \sum_{j=1}^{n} K_{tij}
\]

其中,$KQ_{tij} = PE_{ij} * O_{tij}$。

(4) 种植面积约束

理论上粮食的种植面积越大越好,但是种植面积会受到土地面积的约束。随着城市化程度加大,我们假设粮食的种植面积不大于历年种植粮食面积的最大值。基于此假设,我们建立粮食种植面积约束如下:

\[
\sum_{j=1}^{n} S_{tij} = \frac{\sum_{j=1}^{n} GQ_{tij}}{D_{ij}} \leq S_m
\]

(5) 线性规划模型总结

基于以上分析,我们建立的线性规划理论模型如下:

\[
\text{Max} \quad \sum_{j=1}^{n} GQ_{tij}
\]

\[
\begin{cases}
GQ_{t} = \beta_{1,0} + \beta_{1,1} GQ_{t-1} + \beta_{1,2} SP_{t} + \beta_{1,3} Q_{t} + \beta_{1,4} Z_{1,t} + e_{1t} \\
CQ_{t} = \beta_{2,0} + \beta_{2,1} CQ_{t-1} + \beta_{2,2} CQ_{t-2} + \beta_{2,3} SP_{t} + \beta_{2,4} Q_{t} + \beta_{2,5} Z_{2,t-1} + e_{2t} \\
SP_{t} = \beta_{3,0} + \beta_{3,1} GQ_{t} + \beta_{3,2} SP_{t-1} + \beta_{3,3} Q_{t} + \beta_{3,4} Q_{t-1} + \beta_{3,5} Z_{3t} + \beta_{3,6} Z_{3,t-1} + e_{3t} \\
W_{tij} \leq 10\% * SP_{tij} * CQ_{tij}, \quad \text{其中 } W_{tij} = 
\begin{cases}
0, & Q_{ti} \leq SP_{tij}, \\
(Q_{ti} - SP_{tij}) * CQ_{tij}, & Q_{ti} > SP_{tij}.
\end{cases} \\
\sum_{j=1}^{n} K_{tij} \leq \sum_{j=1}^{n} CQ_{tij} \leq 2 \sum_{j=1}^{n} K_{tij}, \text{ 其中 } KQ_{tij} = PE_{j} * O_{tij} \\
\sum_{j=1}^{n} S_{tij} = \frac{\sum_{j=1}^{n} GQ_{tij}}{D_{ij}} \leq S_{m}
\end{cases}
\]

\section{线性规划实证模型的建立}

基于 5.2 理论线性模型,我们针对小麦和水稻分别建立了线性规划实证模型。然后我们采用遍历法,以当年粮食最低收购价为基点,在小范围内上调或者下降该价格,在满足其它约束的情况下,当植物产量最高时得出“最合理”的粮食最低收购价。接着我们通过比较优化模型得到的粮食最低收购价与当年的粮食最低收购价,对粮食最低收购价的定价进行评价。最后,我们对 2017 年的粮食最低收购价的合理范围进行了预测。

\subsection{线性规划实证模型的合理假设}

我们知道,在错综复杂、波诡云谲的宏观经济环境中,要想精确预测一些经济指标,这是非常困难的。而本处建立的线性规划实证模型所需要计算的最低粮食收购价更是极其微妙的指标,在我国庞大的经济体量中,真可谓动之毫厘,差之千万!但是我们可以对现实世界的经济规律进行合理的假设,在一定理想化的条件下通过建立模型实现对一些事物进行预测。

(1)假设我们可以利用安徽和山东两个省份代表全国所有施行粮食最低收购价政策的省份的效果。

(2)假设在市场经济活动中,主要影响目标的约束均包含在以上建立的四个约束之中,不存在其它重大经济或者政策变故。

(3)假设平均每位居民每日吃的米饭为 0.35 公斤,折合为稻谷的重量为 0.5 公斤;每位居民吃的小麦的重量为 1 公斤。

(4)假设我们可以用粮食产量与单位面积产量的比值来计算粮食种植面积。

\subsection{小麦线性规划实证模型的建立}

(1)安徽和山东小麦产量目标函数:

\begin{equation}
\text{Max } \sum_{j=1}^{2} GQ_{t1j}
\end{equation}

其中,$i=1$ 表示小麦。

(2)安徽和山东的价格波动约束

根据 4.5.1 和 4.5.2 所建立的安徽和山东的小麦实证模型,我们得到小麦的价格波动约束为:
\begin{align*}
GQ_{t} &= 1821.25 - 0.35GQ_{t-1} + 0.54SP_{t} - 475.1Q_{t} - 0.79X_{1} + 95.38X_{2} + 0.01X_{4} - 0.78X_{6} \\
GQ_{t} &= -5638.87 + 2GQ_{t-1} - 33.74SP_{t} - 242.74Q_{t} + 1.02X_{1} + 129.53X_{2} + 103.38X_{3} + 0.04X_{4} \\
CQ_{t} &= 4772.21 - 0.52CQ_{t-1} + 0.29CQ_{t-2} + 1.39SP_{t} - 824.28Q_{t} - 3.78X_{1} + 89.3X_{2} - 0.04X_{4} \\
CQ_{t} &= -3065.23 - 0.71CQ_{t-1} - 0.54CQ_{t-2} + 73.06SP_{t} + 5614.52Q_{t} + 0.49X_{1} \\
SP_{t} &= 5014.42 + 0.09GQ_{t} + 1.09SP_{t-1} + 1.07Q_{t} - 1.57Q_{t-1} - 1.09X_{1} - 6.61X_{2} \\
SP_{t} &= 2.3 + 0.19SP_{t-1} + 1.5Q_{t} - 0.71Q_{t-1} - 0.06X_{2} - 1.34X_{3}
\end{align*}

(2)财政支出约束
\begin{equation}
W_{t1j} \leq 10\% * SP_{t1j} * CQ_{t1j}.
\end{equation}
其中
\begin{equation}
W_{t1j} =
\begin{cases}
0, & Q_{t1} \leq SP_{t1j}, \\
(Q_{t1} - SP_{t1j}) * CQ_{t1j}, & Q_{t1} > SP_{t1j}.
\end{cases}
\end{equation}

(3)库存约束
\begin{equation}
\sum_{j=1}^{2} K_{t1j} \leq \sum_{j=1}^{2} CQ_{t1j} \leq 2 \sum_{j=1}^{2} K_{t1j}
\end{equation}

其中历年的需求量数据在 4.5.1.2 节中已经整理,在这里可以直接使用。而在整理计算需求量时,我们用到公式:需求量 = 生产量 - 农民自留量 + 进口量 - 出口量,其中农民自留量即本处的 $K_{t1j}$。

(4)种植面积约束
\begin{equation}
\sum_{j=1}^{2} S_{t1j} = \frac{\sum_{j=1}^{2} GQ_{t1j}}{D_{1j}} \leq S_{m}
\end{equation}

(5)小麦线性规划实证模型总结
\begin{equation}
\text{Max } \sum_{j=1}^{2} GQ_{t1j}
\end{equation}

\begin{equation}
\begin{cases}
GQ_{t} = 1821.25 - 0.35GQ_{t-1} + 0.54SP_{t} - 475.1Q_{t} - 0.79X_{1} + 95.38X_{2} + 0.01X_{4} - 0.78X_{6} \\
GQ_{t} = -5638.87 + 2GQ_{t-1} - 33.74SP_{t} - 242.74Q_{t} + 1.02X_{1} + 129.53X_{2} + 103.38X_{3} + 0.04X_{4} \\
CQ_{t} = 4772.21 - 0.52CQ_{t-1} + 0.29CQ_{t-2} + 1.39SP_{t} - 824.28Q_{t} - 3.78X_{1} + 89.3X_{2} - 0.04X_{4} \\
CQ_{t} = -3065.23 - 0.71CQ_{t-1} - 0.54CQ_{t-2} + 73.06SP_{t} + 5614.52Q_{t} + 0.49X_{1} \\
SP_{t} = 5014.42 + 0.09GQ_{t} + 1.09SP_{t-1} + 1.07Q_{t} - 1.57Q_{t-1} - 1.09X_{1} - 6.61X_{2} \\
SP_{t} = 2.3 + 0.19SP_{t-1} + 1.5Q_{t} - 0.71Q_{t-1} - 0.06X_{2} - 1.34X_{3} \\
W_{t1j} \leq 10\% * SP_{t1j} * CQ_{t1j} \\
\text{其中 } W_{t1j} = 
\begin{cases}
0, & Q_{t1} \leq SP_{t1j}, \\
(Q_{t1} - SP_{t1j}) * CQ_{t1j}, & Q_{t1} > SP_{t1j}.
\end{cases} \\
\sum_{j=1}^{2} K_{t1j} \leq \sum_{j=1}^{2} CQ_{t1j} \leq 2 \sum_{j=1}^{2} K_{t1j} \\
\sum_{j=1}^{2} S_{t1j} = \frac{\sum_{j=1}^{2} GQ_{t1j}}{D_{1j}} \leq S_{m}
\end{cases}
\end{equation}

\subsection{5.3.3 水稻线性规划实证模型的建立}

(1) 吉林、江苏和湖南水稻产量目标函数:

\[
\text{Max } \sum_{j=1}^{3} GQ_{t2j}
\]

其中,$i=2$ 表示小麦。

(2) 吉林、江苏和湖南水稻的价格波动约束

根据 4.5.3、4.5.4 和 4.5.5 所建立的吉林、江苏和湖南水稻的价格实证模型,我们得到水稻的价格波动约束为:

\begin{align*}
GQ_{t} &= 3974.22 - 0.77GQ_{t-1} + 410.9SP_{t} + 1589.26Q_{t} + 100.99X_{2} - 0.27X_{4} - 0.17X_{6} - 0.12X_{8} \\
GQ_{t} &= 11408 - 1.16GQ_{t-1} + 1286.6SP_{t} - 2319.46Q_{t} - 1458.49X_{2} - 0.12X_{4} + 4.14X_{6} + 0.32X_{8} \\
GQ_{t} &= 4849.09 - 0.71GQ_{t-1} + 25.64SP_{t} + 417.05Q_{t} - 281.442X_{3} - 0.01X_{6} - 0.19X_{8} + 0.37X_{9} \\
CQ_{t} &= 3964.4 - 1.73CQ_{t-1} - 0.22CQ_{t-2} - 583.17SP_{t} + 3284.27Q_{t} - 21.63X_{2} + 0.07X_{4} - 1.84X_{6} \\
CQ_{t} &= 12111 + 1.85CQ_{t-1} - 1.95CQ_{t-2} + 2507.02SP_{t} - 1494.4Q_{t} + 4996.43X_{3} + 4.47X_{6} - 3.58X_{8} \\
CQ_{t} &= 870.65 - 0.79CQ_{t-1} + 0.45CQ_{t-2} + 106.07SP_{t} + 153.25Q_{t} - 128.67X_{2} + 0.08X_{4} + 0.15X_{6} \\
SP_{t} &= -0.49 - 1.56SP_{t-1} + 3.06Q_{t} - 1.46Q_{t-1} + 0.02X_{2} \\
SP_{t} &= 93.07 + 0.02GQ_{t} - 13.11SP_{t-1} + 41.5Q_{t} + 40.83Q_{t-1} - 13.19X_{2} - 0.01X_{4} \\
SP_{t} &= -20.98 + 0.01GQ_{t} + 0.6SP_{t-1} - 2.63Q_{t} + 1.31Q_{t-1} + 0.56X_{3}
\end{align*}

(2) 财政支出约束

其中 $W_{t2j} = \begin{cases} 0, & Q_{t1} \leq SP_{t2j}, \\ (Q_{t2} - SP_{t2j}) * CQ_{t2j}, & Q_{t1} > SP_{t2j}. \end{cases}$

(3) 库存约束

$$
\sum_{j=1}^{3} K_{t2j} \leq \sum_{j=1}^{3} CQ_{t2j} \leq 2 \sum_{j=1}^{3} K_{t2j}
$$

其中历年的需求量数据在 4.5.1.2 节中已经整理,在这里可以直接使用。而在整理计算需求量时,我们用到公式:需求量 = 生产量 - 农民自留量 + 进口量 - 出口量,其中农民自留量即本处的 $K_{t2j}$.

(4) 种植面积约束

$$
\sum_{j=1}^{3} S_{t2j} = \frac{\sum_{j=1}^{3} GQ_{t2j}}{D_{2j}} \leq S_{m}
$$

(5) 小麦线性规划实证模型总结

$$
\text{Max } \sum_{j=1}^{3} GQ_{t2j}
$$

$$
\begin{cases}
GQ_{t} = 3974.22 - 0.77GQ_{t-1} + 410.9SP_{t} + 1589.26Q_{t} + 100.99X_{2} - 0.27X_{4} - 0.17X_{6} - 0.12X_{8} \\
GQ_{t} = 11408 - 1.16GQ_{t-1} + 1286.6SP_{t} - 2319.46Q_{t} - 1458.49X_{2} - 0.12X_{4} + 4.14X_{6} + 0.32X_{8} \\
GQ_{t} = 4849.09 - 0.71GQ_{t-1} + 25.64SP_{t} + 417.05Q_{t} - 281.442X_{3} - 0.01X_{6} - 0.19X_{8} + 0.37X_{9} \\
CQ_{t} = 3964.4 - 1.73CQ_{t-1} - 0.22CQ_{t-2} - 583.17SP_{t} + 3284.27Q_{t} - 21.63X_{2} + 0.07X_{4} - 1.84X_{6} \\
CQ_{t} = 12111 + 1.85CQ_{t-1} - 1.95CQ_{t-2} + 2507.02SP_{t} - 1494.4Q_{t} + 4996.43X_{3} + 4.47X_{6} - 3.58X_{8} \\
CQ_{t} = 870.65 - 0.79CQ_{t-1} + 0.45CQ_{t-2} + 106.07SP_{t} + 153.25Q_{t} - 128.67X_{2} + 0.08X_{4} + 0.15X_{6} \\
SP_{t} = -0.49 - 1.56SP_{t-1} + 3.06Q_{t} - 1.46Q_{t-1} + 0.02X_{2} \\
SP_{t} = 93.07 + 0.02GQ_{t} - 13.11SP_{t-1} + 41.5Q_{t} + 40.83Q_{t-1} - 13.19X_{2} - 0.01X_{4} \\
SP_{t} = -20.98 + 0.01GQ_{t} + 0.6SP_{t-1} - 2.63Q_{t} + 1.31Q_{t-1} + 0.56X_{3} \\
W_{t2j} \leq 10\% * SP_{t2j} * CQ_{t2j}. \text{其中 } W_{t2j} = \begin{cases} 0, & Q_{t1} \leq SP_{t2j}, \\ (Q_{t2} - SP_{t2j}) * CQ_{t2j}, & Q_{t1} > SP_{t2j}. \end{cases} \\
\sum_{j=1}^{3} K_{t2j} \leq \sum_{j=1}^{3} CQ_{t2j} \leq 2 \sum_{j=1}^{3} K_{t2j} \\
\sum_{j=1}^{3} S_{t2j} = \frac{\sum_{j=1}^{3} GQ_{t2j}}{D_{2j}} \leq S_{m}
\end{cases}
$$

\section*{5.4 线性规划实证模型的求解}

通过 5.3 构建的线性规划实证模型,各个影响产量的因素之间的关系错综复杂,因素之间相互影响。为了找出最低收购价对产量的影响程度,即找出最低收购价政策对各个省的有效性。借助 matlab 程序,我们将各个省的最低收购价按比例调整,来观察各个省的种植面积增长情况。下面表中的调整(调高或

\begin{table}
\centering
\begin{tabular}{l c c c c c}
\hline
年份 & 2011年 & 2012年 & 2013年 \\
\hline
最低收购价 & 小麦产量 & 最低收购价 & 小麦产量 & 最低收购价 & 小麦产量 \\
调高5\% & 增长3.4\% & 调高5\% & 增长2.8\% & 调高5\% & 增长3.8\% \\
调高10\% & 增长7.2\% & 调高10\% & 增长8.5\% & 调高10\% & 增长7.9\% \\
调高15\% & 增长10.1\% & 调高15\% & 增长11.6\% & 调高15\% & 增长10.4\% \\
\hline
年份 & 2011年 & 2012年 & 2013年 \\
\hline
最低收购价 & 小麦产量 & 最低收购价 & 小麦产量 & 最低收购价 & 小麦产量 \\
调高5\% & 增长5.2\% & 调高5\% & 增长6.1\% & 调高5\% & 增长3.8\% \\
调高10\% & 增长8.2\% & 调高10\% & 增长9.5\% & 调高10\% & 增长9.9\% \\
调高15\% & 增长12.1\% & 调高15\% & 增长13.2\% & 调高15\% & 增长12.4\% \\
\hline
\end{tabular}
\caption{安徽和山东小麦最低收购价与小麦产量的关系比例(2011-2013年)}
\end{table}

\begin{table}
\centering
\begin{tabular}{l c c c c c}
\hline
年 & 2011年 & 2012年 & 2013年 \\
\hline
最低收购价 & 水稻产量 & 最低收购价 & 水稻产量 & 最低收购价 & 水稻产量 \\
调高5\% & 增长3.7\% & 调高5\% & 增长2.8\% & 调高5\% & 增长3.8\% \\
调高10\% & 增长6.8\% & 调高10\% & 增长8.5\% & 调高10\% & 增长7.2\% \\
调高15\% & 增长9.9\% & 调高15\% & 增长10.9\% & 调高15\% & 增长9.7\% \\
\hline
年 & 2011年 & 2012年 & 2013年 \\
\hline
最低收购价 & 水稻产量 & 最低收购价 & 水稻产量 & 最低收购价 & 水稻产量 \\
降低5\% & 增长2.3\% & 降低5\% & 增长4.1\% & 降低5\% & 增长4.8\% \\
调高5\% & 增长0.2\% & 调高5\% & 降低1.1\% & 调高5\% & 增长0.8\% \\
调高10\% & 降低1.3\% & 调高10\% & 降低5.5\% & 调高10\% & 降低4.9\% \\
\hline
年 & 2011年 & 2012年 & 2013年 \\
\hline
最低收购价 & 水稻产量 & 最低收购价 & 水稻产量 & 最低收购价 & 水稻产量 \\
调高5\% & 增长1.2\% & 调高5\% & 增长2.1\% & 调高5\% & 增长1.8\% \\
调高10\% & 增长3.2\% & 调高10\% & 增长4.5\% & 调高10\% & 增长3.9\% \\
调高15\% & 增长5.1\% & 调高15\% & 增长5.2\% & 调高15\% & 增长4.4\% \\
\hline
\end{tabular}
\caption{吉林、湖南和江苏水稻最低收购价与水稻产量的关系比例(2011-2013年)}
\end{table}

\begin{table}
\centering
\begin{tabular}{c c c c c}
\hline
安徽2017年 & & 山东2017年 & & \\
\hline
最低收购价 & 小麦产量 & 最低收购价 & 小麦产量 & \\
\hline
调高5\% & 增长5.4\% & 调高5\% & 增长6.3\% & \\
调高10\% & 增长7.2\% & 调高10\% & 增长8.5\% & \\
调高15\% & 增长12.1\% & 调高15\% & 增长13.6\% & \\
\hline
吉林2017年 & & 湖南2017年 & & 江苏2017年 \\
\hline
最低收购价 & 小麦产量 & 最低收购价 & 小麦产量 & 最低收购价 & 小麦产量 \\
\hline
调高5\% & 增长4.5\% & 减少5\% & 增长4.1\% & 调高5\% & 增长1.1\% \\
调高10\% & 增长7.6\% & 调高5\% & 减少3.5\% & 调高10\% & 增长3.5\% \\
调高15\% & 增长11.4\% & 调高10\% & 减少8.5\% & 调高15\% & 增长5.3\% \\
\hline
\end{tabular}
\caption{五省最低收购价与粮食产量的关系比例(2017年)}
\end{table}

在2016年的基础上,得出2017年的各省最低收购价与粮食产量的关系比例。安徽省、山东省、吉林省的最低收购价政策效果显著,江苏省的最低收购价效果不显著,而湖南省的最低收购价政策没有效果。

\begin{table}
\centering
\begin{tabular}{c c c c c}
\hline
年份 & & & & \\
\hline
省份 & 安徽省 & 山东省 & 吉林省 & 湖南省 & 江苏省 \\
\hline
最低收购价 & 0.94 & 0.94 & 1.07 & 1.07 & 1.07 \\
合理定价 & 1.02 & 0.99 & 1.2132 & 1.13 & 1.29 \\
\hline
年份 & & & & \\
\hline
省份 & 安徽省 & 山东省 & 吉林省 & 湖南省 & 江苏省 \\
\hline
最低收购价 & 1.02 & 1.02 & 1.25 & 1.25 & 1.25 \\
合理定价 & 1.10 & 1.07 & 1.37 & 1.26 & 1.48 \\
\hline
年份 & & & & \\
\hline
省份 & 安徽省 & 山东省 & 吉林省 & 湖南省 & 江苏省 \\
\hline
最低收购价 & 1.12 & 1.12 & 1.35 & 1.35 & 1.35 \\
合理定价 & 1.21 & 1.18 & 1.49 & 1.36 & 1.599 \\
\hline
\end{tabular}
\caption{五省最低收购价的合理定价(2011-2013年)}
\end{table}

在一定范围内,最低收购价越高越好,但是如果与上一年的增长太大对于社会发展也不利,有可能引发通货膨胀。因此在上一年的产量上增加5\%为标准抬高最低收购价。从上表可以看出湖南省的合理定价只是微涨或微跌,江苏省的合理定价微涨。

\begin{table}
\centering
\begin{tabular}{c c c c c}
\hline
省份 & 安徽省 & 山东省 & 吉林省 & 湖南省 & 江苏省 \\
\hline
合理定价(元/斤) & 1.32 & 1.28 & 1.42 & 1.37 & 1.38 \\
\hline
\end{tabular}
\caption{五省最低收购价的合理定价(2017年)}
\end{table}

上表在2016年的最低收购价的基础上按照比上一年的产量增加5\%进行了调整。

\section*{5.5 线性规划实证模型的分析}

从表5.1中可以看出,安徽和山东的最低收购价政策效果显著,小麦产量随着最低收购价的提高而增大,因为库存和国家财政的约束,社会的稳定和小麦的产量持续增长,最低收购价不能无限增大。因此我们得出国家的最低收购

价略低。

从表 5.2 中可以看出,吉林省的水稻最低收购价政策效果显著,水稻产量随着最低收购价的提高而提高,同样考虑到库存和国家财政的约束,社会的稳定和水稻的产量持续增长,最低收购价不能无限增大。同样我们认为国家的最低收购价对于吉林水稻略低。江苏省的水稻最低收购价政策效果不是很显著,水稻的产量随着最低收购价的抬高没有显著增大,只是略微抬高,说明国家最低收购价政策相对于江苏水稻来说比较合理。湖南水稻比较特殊,水稻产量随着最低收购价的提高反而减少,最低收购价政策在湖南水稻上没有发挥到正作用。

从上面的分析来看,最低收购价政策对不同的省有不同的效果。为了将最低收购价政策的作用发挥出来,不同的省最好制定不同的最低收购价。

\section*{第 6 章 问题五的解答}

\section*{——粮食最低收购价的可行性分析}

本题需要解答的是能否通过施行粮食最低收购价政策来使得小麦种植面积增大 5\%。基于第五章的建模,我们以安徽和山东两个省份的小麦种植面积之和为分析对象,研究粮食最低收购价政策的可行性。我们首先对第 5 章建立的小麦线性规划实证模型的目标函数做了适当的调整,得到了以最大小麦种植面积为目标的小麦线性规划模型。本题的解答实质上就成为了讨论通过设定合适的小麦最低收购价,在满足约束条件的前提下,能否达到面积较 2000 年增大 5\%。我们基于此思想,展开了本问题的解答。

\subsection{6.1 小麦线性规划实证模型的调整}

我们将原来目标函数的粮食生产量除以单产把目标转化为面积,建立了适用于本题分析求解的小麦线性规划模型,本题的建模过程以及指标都与第 5 章相同,在此我们直接通过调整,建立小麦线性规划模型的目标函数为:

\[
\text{Max } \sum_{j=1}^{2} GQ_{t1j} / D_{1j}
\]

模型的约束条件与 5.3.2 的约束条件一样,我们直接得出调整后的小麦线性规划实证模型如下:

\[
\begin{cases}
\text{Max } \sum_{j=1}^{2} GQ_{t1j} / D_{1j} \\
GQ_{t} = 1821.25 - 0.35GQ_{t-1} + 0.54SP_{t} - 475.1Q_{t} - 0.79X_{1} + 95.38X_{2} + 0.01X_{4} - 0.78X_{6} \\
GQ_{t} = -5638.87 + 2GQ_{t-1} - 33.74SP_{t} - 242.74Q_{t} + 1.02X_{1} + 129.53X_{2} + 103.38X_{3} + 0.04X_{4} \\
CQ_{t} = 4772.21 - 0.52CQ_{t-1} + 0.29CQ_{t-2} + 1.39SP_{t} - 824.28Q_{t} - 3.78X_{1} + 89.3X_{2} - 0.04X_{4} \\
CQ_{t} = -3065.23 - 0.71CQ_{t-1} - 0.54CQ_{t-2} + 73.06SP_{t} + 5614.52Q_{t} + 0.49X_{1} \\
SP_{t} = 5014.42 + 0.09GQ_{t} + 1.09SP_{t-1} + 1.07Q_{t} - 1.57Q_{t-1} - 1.09X_{1} - 6.61X_{2} \\
SP_{t} = 2.3 + 0.19SP_{t-1} + 1.5Q_{t} - 0.71Q_{t-1} - 0.06X_{2} - 1.34X_{3} \\
W_{t1j} \leq 10\% * SP_{t1j} * CQ_{t1j}. \\
\text{其中 } W_{t1j} = 
\begin{cases}
0, & Q_{t1} \leq SP_{t1j}, \\
(Q_{t1} - SP_{t1j}) * CQ_{t1j}, & Q_{t1} > SP_{t1j}.
\end{cases} \\
\sum_{j=1}^{2} K_{t1j} \leq \sum_{j=1}^{2} CQ_{t1j} \leq 2 \sum_{j=1}^{2} K_{t1j} \\
\sum_{j=1}^{2} S_{t1j} = \frac{\sum_{j=1}^{2} GQ_{t1j}}{D_{1j}} \leq S_{m}
\end{cases}
\]

\subsection{6.2 粮食最低收购价的可行性分析}

相似于问题四的分析过程,因为所编写的 matlab 程序是一个不连续的遍历

求解过程,我们依然通过手动不断调整粮食最低收购价的值,观察在满足约束条件的基础上,面积能否达到较 2000 年增大 5%。具体的求解分析过程如下:

(1) 计算目标面积

根据搜集的数据,2000 年的安徽和山东的小麦种植面积之和为:5679.35 千公顷。在面积增大 5% 后,我们的小麦种植面积目标为 5963.3175 千公顷。

(2) 以往数据的实证分析

安徽和山东从 2006 年开始实施粮食最低收购价政策,我们先分析实行粮食最低收购价政策之后安徽和山东小麦种植面积的变化特征,我们计算 2006-2014 年的安徽和山东小麦种植面积的值如下:

\begin{table}[h]
\centering
\caption{历年小麦种植面积}
\begin{tabular}{c c c c}
\hline
年份 & 面积 & 年份 & 面积 \\
\hline
2006 & 5690.911 & 2011 & 5976.53 \\
2007 & 5849.35 & 2012 & 6041.38 \\
2008 & 5871.911 & 2013 & 6106.12 \\
2009 & 5900.467 & 2014 & 6174.73 \\
2010 & 5927.54 & & \\
\hline
\end{tabular}
\end{table}

从以往数据来看,实行粮食最低收购价政策之后 2004-2014 年度安徽和山东的小麦种植面积呈现不断增长的趋势,这说明粮食最低收购价政策的实施是有效的。但是 2015 年的小麦种植面积不增反降,甚至低于 2000 年的 5679.35 千公顷。这是偶然现象还是小麦种植面积不断增大后开始呈现下降趋势,以下将利用小麦种植面积目标优化模型分析进行分析。

(3) 粮食最低收购价政策的可行性分析

据洪泽县粮食局 2015 年小麦最低收购价收购质价政策公告,2015 年全国粮食最低收购价为 1.18 元。我们将此数据代入小麦线性规划实证模型中,满足各个约束条件,预测得到 2015 年的面积为 5627.36 千公顷,比 2000 年的 5679.35 千公顷稍稍低一些,但是远远低于近几年小麦种植面积的平均水平。

为了进一步探讨粮食最低收购价政策对小麦种植面积的影响作用,我们将粮食最低收购价由原来的 1.18 元提高为 1.28 元。粮食最低收购价涨幅为 8.47%,各约束均可满足,此时得出的 2015 年的面积为 5839.57 千公顷,实现了原来面积 3.37% 的增幅。超过了 2000 年的 5679.35 千公顷,相比但是依然低于涨幅 5% 时 5963.3175 千公顷的种植面积。在此次的调整实验中,我们已经看出粮食最低收购价政策对提高小麦种植面积的促进作用。

我们继续调整粮食最低收购价将粮食最低收购价调整为 1.38 元。粮食最低收购价相比未作调整涨幅为 16.9%,此时小麦种植面积增加至 6258.52 千公顷。此时的小麦种植面积相比之前未作调整涨幅为 11.21%。而且远远高于 5963.3175 千公顷,实现了较 2000 年增大超过 5% 的目标。但是我们在检查约束条件时,发现不满足库存限制约束。说明档粮食最低收购价显著提高时,虽然能提高农民的种粮积极性,增大小麦的种植面积,但是却加重了国家的收购负担,甚至出现了爆仓情形。

我们将上次实验的最低收购价下调,调整为 1.33 元粮食最低收购价相比未作调整涨幅为 12.7%,此时小麦种植面积相比未作调整增加至 6003.9 千公顷。此时的小麦种植面积相比之前未作调整涨幅为 6.69%。而且也高于 5963.3175 千公顷,实现了较 2000 年增大超过 5% 的目标。

\section*{第 7 章 问题六的解答}

\section*{——调控粮食种植的优化决策和建议}

\subsection{7.1 主要结论}

本论文经过四天的研究,解决了以下几个问题:

第一,第一问中显示每个省每个粮食品种种植面积的影响指标体系并不一样。因此我国个区域各品种调控种植面积的方法和手段应该也不一样。

第二,第二问构建了粮食最低收购价政策执行是否有效的评价模型。并用此模型得出粮食最低收购价政策对吉林水稻和安徽、山东省小麦都有积极正向的作用。但在江苏、湖南省水稻索取的作用却并不显著。

第三,第三问粮食市场收购价形成有其机制和规律。每个省每个粮食品种的两只最低收购价均对粮食市场收购价产生正向的影响。这说明粮食最低收购价扭曲了市场定价。但为了粮食产业的发展,实施粮食最低收购价又是很有必要的。

第四,第四问建立了最低收购价政策合理范围确定的优化模型。通过计算发现,国家公布的最低收购价政策普遍还可以提高。

\subsection{7.2 政策建议}

结合上述研究结论,建议如下:

第一,完善我国粮食最低收购价定价机制,逐步适度的提高粮食最低收购价;数据资料显示,最近几年,粮食生产成本不断攀升,如种子、化肥、农药等生产资料价格不断提升。另外由于大量青壮年劳动力外出打工,一些繁重的体力活,如收割、脱谷等需要用机械来替代。这不仅提高了成本,而且对粮农的机械操作水平要求也提高了。

第二,建立粮食供需平衡的预警机制,预测粮食市场价格走势,有目的的制定各种农作物的粮食最低收购价;现阶段,政府应继续坚持利用最低收购价政策对市场价格进行宏观调控,以实现市场调节与宏观调控的有机结合,同时也应对最低收购价政策不完善的地方进行改进。由于各地粮食生产效率不一样,建议政府在每个省不同品种实行不一样的粮食最低收购价标准,用以充分发挥各地粮食最低收购价政策的效果。

第三,将粮食最低收购价和粮食直接补贴有效的结合起来,创新补贴方式;加强土地流转,提高土地利用率。提高农户对农业生产的积极性有极大的帮助,坚持优化水稻生产的资源环境和政策环境,制定更多的政策从市场中保护农民利益。同时利用价格保护政策推广和播种优质水稻品种,向农民提供有效的技术支持。强化旱情监测预警能力,完善抗旱保障体系,夯实稻田水利基础设施,提高水稻综合生产能力和抗旱减灾能力。建议政府将补贴多元化,对于种粮的农户实行种子、灌溉、机械等补贴,改善生产条件、提高科技含量。

第四,要努力提高粮农的科技素质和文化素质。为了保证未来粮食安全,建议在制定生产投入决策时,应该同时考虑稳定水稻播种面积,在农民受教育年限相对偏低的情况下,对农业生产进行积极有效的引导和保护。另外,农村土地流转之后,农业机械开始大范围的粮食生产中使用,机械的保养、维护、使用需要有一定文化基础的现代农民。

\section*{第五,从问题二的计算结果来看,粮食最低收购价政策在不同地区不同品种上表现出来的执行效果并不一致。从地区来看,粮食最低收购价政策在山东省、安徽省和吉林省的执行效果较好,所以这几个省所在的地区可以继续实行该政策来促进粮食种植面积的增加;而在湖南省和江苏省粮食最低收购价政策的效果并不明显,这两个省以及其所在的地区以后可以依据具体情况考虑将该政策撤销。从粮食品种来看,粮食最低收购价政策对小麦的种植具有较好的促进作用,所以,对于小麦的种植应该继续实行该政策。而该政策对于水稻种植的执行效果因地区而异,所以对于水稻是否应该实行粮食最低收购价政策,需要具体地区具体对待。}

\section*{第六,目前我国执行的粮食最低收购价是全国统一的价格,然而问题四的计算结果表明,同一个粮食最低收购价,不同地区执行的效果并不相同,有的地区效果好,有的地区效果不明显。所以为了更好地达到国家制定粮食最低收购价政策的初衷,可以因地制宜,针对不同地区实行不同的粮食最低收购价。}

\section*{第七,问题五的计算结果表明调整粮食最低收购价可以达到让小麦种植面积增加 5\% 的目标,证明了最低收购价政策可以有效调整粮食的种植面积。所以国家如果想调控某种粮食的种植面积,可以采取调整粮食最低收购价的方法。}

\section*{粮食生产关系到国家的安全和稳定,关系到老百姓的生活质量,因此实行粮食最低收购政策是一个非常正确的,也是有效的。本论文提出的粮食最低收购价的合理确定模型具有一定的借鉴性。当然,这需要后续的数据资料和技术支撑。}

\section*{第 8 章 模型评价和推广}

\subsection{8.1 模型的优点}

1. 斯皮尔曼(Spearman)相关系数对数据不要求服从正态分布,且可以反应变量间趋同关系。

2. 线性规划可以综合考虑,根据决策者要求和实际情况变化作出相应的决策。

3. 线性规划理论模型不仅适合各省的粮食收购价模型的建立,并且适合全国的粮食收购价的建立。

\subsection{8.2 模型的缺点}

1. 基于前提假设进行研究,可能有些与实际情况不符。

2. 斯皮尔曼(Spearman)相关系数准确度不高。

\subsection{8.3 模型的推广}

1. 该模型可以推广到解决多个变量有关联,而且有滞后性的问题,并应用于实际问题中的预测问题。

2. 本文中主成分-统计控制模型可以应用到比较任意有其它因素影响的变量之间的真实关系。

\section*{参考文献}

[1] 丁伟. 对粮食最低收购价政策的评价与思考[B]. 湖北: 商业银行实务, 2006.

[2] 陈晓玲、产颖. 对实行粮食最低收购价政策的效果评析[B]. 北京: 黑龙江对外经贸, 2011年第4期.

[3] 彭建霞、张蜜. 粮食最低收购价格的困境分析[B]. 国家统计局湖南调查总队, 2014.

[4] 邵长安、周士发、孟庆军. 浅论粮食最低收购价政策在国家宏观调控中的作用及相关建议[B]. 粮经论坛, 2010.

[5] 郭晓慧. 我国粮食价格波动及调控研究[D]. 四川: 西南财经大学, 2010

[6] 高铁梅. 计量经济分析方法与建模[M]. 北京: 清华大学出版社, 2009

[7] 陈广雷. 教学质量评价中协方差分析的应用[J]. 武警学院学报, 2003, 19(4): 63,91.

[8] 任红松. 协方差分析的SAS实现方法 [J]. 石河子大学学报, 2003, 7(1): 45-48.

\section*{附录}

\subsection*{附录一:聚类分析程序:}
\begin{verbatim}
clc;clear;
load sdshengfen.txt %把原始数据保存在纯文本文件 sdshengfen.txt 中
r=corrcoef(sdshengfen) %计算相关系数矩阵
d=1-r; %进行数据变换,把相关系数转化为距离
d=tril(d); %取出矩阵 d 的下三角元素
d=nonzeros(d); %取出非零元素
d=d'; %化成行向量
z=linkage(d,'ward'); %按类平均法聚类
dendrogram(z); %画聚类图
T=cluster(z,'maxclust',3) %把变量划分成 3 类
for i=1:3
tm=find(T==i); %求第 i 类的对象
tm=reshape(tm,1,length(tm)); %变成行向量
fprintf('第%d 类的有%s\n',i,int2str(tm)); %显示分类结果
end
\end{verbatim}

\subsection*{附录二: Spearman 相关系数的 SAS 程序}
\begin{verbatim}
ods html file='G:\RESULT1.xls' style=Threelines;
data ex;
input x1-x12 y@@;
x1=log(x1);
x2=log(x2);
x3=log(x3);
x4=log(x4);
x5=log(x5);
x6=log(x6);
x7=(x7);
x8=log(x8);
x9=log(x9);
x10=(x10);
x11=log(x11);
x12=(x12);
y1=log(y);
cards;
各省份各指标的数据略
;
proc corr data=ex spearman;
\end{verbatim}

\begin{verbatim}
var x1-x12;
with y;
run;
ods html close;

附录三:主成分回归的SAS程序
data ex;
input x1-x12 y@@;
cards;
数据略
run;
proc corr; var x1 x2 x4-x6 x8 x9 x11;
proc princomp out=prin;
var x1 x2 x4-x6 x8 x9 x11;
run;
proc print data=prin;
var prin1-prin8;
proc reg; model y=prin1-prin8/stb;
run;
proc standard m=0 s=1 out=ex2; run;
proc reg data=ex outtest=out3;
model y=x1 x2 x4-x6 x8 x9 x11 / pcomit=1,2,3,4,5,6,7 outvif;
proc reg data=ex outtest=out2;
model y=x1 x2 x4-x6 x8 x9 x11 / vif collinoint;
run;
proc print data=out3;
run;

附录四:三省各变量在主成分中的系数表与主成分得分表
山东省:
表 各变量的系数
\end{verbatim}

\begin{tabular}{c c c c}
 & Prin 1 & Prin 2 & Prin 3 \\
\hline
x1 & -0.4616 & 0.179696 & -0.049248 \\
x2 & 0.475883 & -0.001438 & 0.096718 \\
x3 & 0.372183 & 0.304353 & -0.788684 \\
x4 & 0.42898 & -0.330582 & 0.413963 \\
x10 & 0.101198 & 0.873678 & 0.441366 \\
x11 & 0.477154 & 0.049786 & 0.0053 \\
\end{tabular}

\begin{verbatim}
表 主成分得分
\end{verbatim}

\begin{tabular}{c c c c}
 & Prin 1 & Prin 2 & Prin 3 \\
\hline
年份 & & & \\
1995 & -2.81547 & -2.02592 & -1.38151 \\
1996 & -2.91416 & -1.42527 & -0.39139 \\
\end{tabular}

\begin{tabular}{l l l l}
1997 & -2.83918 & 0.1001 & 0.70816 \\
1998 & -2.94647 & 0.1023 & 1.00758 \\
1999 & -2.58534 & 0.46138 & 0.88546 \\
2000 & -1.70817 & 0.6222 & 0.27548 \\
2001 & -1.09757 & 0.91933 & 0.1341 \\
2002 & -0.50684 & 0.96384 & 0.08729 \\
2003 & -0.13826 & 1.41243 & 0.00573 \\
2004 & -0.02747 & -0.70126 & -1.04146 \\
2005 & 0.41294 & -0.04837 & -0.73513 \\
2006 & 1.0126 & 0.98759 & -0.28998 \\
2007 & 1.47318 & 1.31836 & -0.24488 \\
2008 & 1.5574 & 0.86505 & -0.48766 \\
2009 & 1.70674 & 0.52013 & -0.64339 \\
2010 & 1.70477 & 0.31123 & -0.40112 \\
2011 & 2.1614 & -0.11653 & 0.29399 \\
2012 & 2.45791 & -0.91207 & 0.19309 \\
2013 & 2.36945 & -1.68811 & 0.7162 \\
2014 & 2.72252 & -1.6664 & 1.30943 \\
\end{tabular}

\textbf{湖南省:}

\textbf{表 各变量的系数}

\begin{tabular}{l l l}
 & Prin 1 & Prin 2 \\
x1 & -0.447196 & 0.205596 \\
x3 & -0.082946 & -0.701722 \\
x5 & 0.483703 & -0.032892 \\
x6 & 0.479279 & 0.035098 \\
x7 & 0.169988 & 0.62144 \\
x8 & 0.42314 & 0.052167 \\
x10 & -0.348588 & 0.272203 \\
\end{tabular}

\textbf{表 主成分得分}

\begin{tabular}{l l l}
 & Prin 1 & Prin 2 \\
Obs & & \\
1 & -1.19833 & -1.68284 \\
2 & -0.85941 & 0.47844 \\
3 & -0.78524 & 1.47217 \\
4 & -1.76696 & 1.68193 \\
5 & -1.8592 & 1.91646 \\
6 & -2.5795 & 1.09181 \\
7 & -2.08502 & 0.73738 \\
8 & -2.23928 & -0.4002 \\
\end{tabular}

江苏省:

表 各变量的系数

\begin{tabular}{c c c}
 & Prin 1 & Prin 2 \\
\hline
x3 & -0.568463 & -0.178623 \\
x6 & -0.474468 & 0.244546 \\
x8 & 0.510118 & 0.146289 \\
x9 & 0.220627 & 0.735323 \\
x10 & 0.377933 & -0.58838 \\
\end{tabular}

表 主成分得分

\begin{tabular}{c c c}
 & Prin 1 & Prin 2 \\
\hline
Obs & & \\
1 & 0.75829 & 1.14893 \\
2 & 2.31231 & 3.73456 \\
3 & 1.54365 & 0.35679 \\
4 & 2.4744 & -0.31577 \\
5 & 2.43259 & -1.08553 \\
6 & 2.16413 & -1.29762 \\
7 & 1.28035 & -0.91232 \\
8 & 1.19241 & -0.83612 \\
9 & 0.82585 & -0.88639 \\
10 & 0.08268 & 0.16956 \\
11 & -0.58526 & -0.13565 \\
12 & -1.3797 & -0.48707 \\
13 & -1.57487 & -0.73143 \\
14 & -1.53034 & -0.44858 \\
15 & -1.4894 & -0.18514 \\
16 & -1.34833 & -0.25543 \\
\end{tabular}

\begin{verbatim}
附录五:价格规律 matlab 程序
x=[0.71 0.71 0.72 0.84 0.90.94 1.02 1.12 1.18];%小麦最低价
y1=[0.7 0.70.77 0.90.93 1.02 1.21.32 1.35 ];%早籼稻
y2=[0.72 0.72 0.79 0.92 0.97 1.07 1.25 1.35 1.38];%中晚籼稻
y3=[0.75 0.75 0.82 0.95 1.05 1.28 1.41.51.55];%粳稻
z1=[0.704 0.7127 0.7636 0.8657 0.9371 0.9786 1.0443 1.1293 1.1484];%安徽
z2=[0.7266 0.8031 0.8388 0.9555 1.0232 1.061 1.0443 1.2358 1.2374];%山东
z3=[0.9557 0.8271 0.9385 1.1044 1.4226 1.4586 1.4967 1.4392 1.5218];%吉林
z4=[0.7095 0.7596 0.8973 0.8952 0.9773 1.1788 1.2932 1.2563 1.3154];%湖南
z5=[0.754 0.8298 0.9057 0.9645 1.1262 1.3327 1.2553 1.3146 1.3515];%江苏
t=2006:1:2014;
subplot(2,3,1);
plot(t,y3,'k:',t,z3,'b-')
title('吉林水稻市场价与最低收购价');
xlabel('年份');
ylabel('价格');
legend('最低价','市场价');
subplot(2,3,2);
plot(t,y1,'k:',t,z4,'b-')
title('湖南水稻市场价与最低收购价');
xlabel('年份');
ylabel('价格');
legend('最低价','市场价');
subplot(2,3,3);
plot(t,y2,'k:',t,z5,'b-')
title('江苏水稻市场价与最低收购价');
xlabel('年份');
ylabel('价格');
legend('最低价','市场价');
subplot(2,3,4);
\end{verbatim}

\begin{verbatim}
plot(t,x,'k:',t,z1,'b-')
title('安徽小麦市场价与最低收购价');
xlabel('年份');
ylabel('价格');
legend('最低价','市场价');
subplot(2,3,5);
plot(t,x,'k:',t,z2,'b-')
title('山东小麦市场价与最低收购价');
xlabel('年份');
ylabel('价格');
legend('最低价','市场价');
\end{verbatim}

\section*{附录六:Pearson相关系数的SAS程序}
\begin{verbatim}
ods html file='E:\RESULT1.xls' style=Threelines;
data ex;
input x1-x5 y1-y5 @@;
cards;
0.9557 0.7095 0.754 0.7266 0.704 0.75 0.70 0.72 0.71 0.71
0.8271 0.7596 0.8298 0.8031 0.7127 0.75 0.70 0.72 0.71 0.71
0.9385 0.8973 0.9057 0.8388 0.7636 0.82 0.77 0.79 0.72 0.72
1.1044 0.8952 0.9645 0.9555 0.8657 0.95 0.90 0.92 0.84 0.84
1.4226 0.9773 1.1262 1.0232 0.9371 1.05 0.93 0.97 0.90 0.9
1.4586 1.1788 1.3327 1.061 0.9786 1.28 1.02 1.07 0.94 0.94
1.4967 1.2932 1.2553 1.0443 1.0443 1.41 21.25 1.02 1.02
1.4392 1.2563 1.3146 1.2358 1.1293 1.51 32 1.35 1.12 1.12
1.5218 1.3154 1.3515 1.2374 1.1484 1.55 1.35 1.38 1.18 1.18
;
end;
proc corr data=ex pearson;
var x1-x5;
with y1-y5;
run;
ods html close;
\end{verbatim}