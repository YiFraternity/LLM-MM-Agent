\begin{center}
\textbf{“华为杯”第十三届全国研究生数学建模竞赛}
\end{center}

\begin{table}[h]
\centering
\begin{tabular}{c c c}
\hline
学 & 校 & 天津工业大学 \\
\hline
参赛队号 & 10058009 & \\
\hline
队员姓名 & 1. 王修杰 & \\
 & 2. 王远野 & \\
 & 3. 袭蓓 & \\
\hline
\end{tabular}
\end{table}

\begin{flushright}
参赛密码 \_\_\_\_\_\_\_\_\_\_\_\_\_\_\_\_\_\_\_\_\_\_\_\_\_\_\_\_\_\_\_\_\_\_\_\_\_\_\_\_\_\_\_\_\_\_\_\_\_\_\_\_\_\_\_\_\_\_\_\_\_\_\_\_\_\_\_\_\_\_\_\_\_\_\_\_\_\_\_\_\_\_\_\_\_\_\_\_\_\_\_\_\_\_\_\_\_\_\_\_\_\_\_\_\_\_\_\_\_\_\_\_\_\_\_\_\_\_\_\_\_\_\_\_\_\_\_\_\_\_\_\_\_\_\_\_\_\_\_\_\_\_\_\_\_\_\_\_\_\_\_\_\_\_\_\_\_\_\_\_\_\_\_\_\_\_\_\_\_\_\_\_\_\_\_\_\_\_\_\_\_\_\_\_\_\_\_\_\_\_\_\_\_\_\_\_\_\_\_\_\_\_\_\_\_\_\_\_\_\_\_\_\_\_\_\_\_\_\_\_\_\_\_\_\_\_\_\_\_\_\_\_\_\_\_\_\_\_\_\_\_\_\_\_\_\_\_\_\_\_\_\_\_\_\_\_\_\_\_\_\_\_\_\_\_\_\_\_\_\_\_\_\_\_\_\_\_\_\_\_\_\_\_\_\_\_\_\_\_\_\_\_\_\_\_\_\_\_\_\_\_\_\_\_\_\_\_\_\_\_\_\_\_\_\_\_\_\_\_\_\_\_\_\_\_\_\_\_\_\_\_\_\_\_\_\_\_\_\_\_\_\_\_\_\_\_\_\_\_\_\_\_\_\_\_\_\_\_\_\_\_\_\_\_\_\_\_\_\_\_\_\_\_\_\_\_\_\_\_\_\_\_\_\_\_\_\_\_\_\_\_\_\_\_\_\_\_\_\_\_\_\_\_\_\_\_\_\_\_\_\_\_\_\_\_\_\_\_\_\_\_\_\_\_\_\_\_\_\_\_\_\_\_\_\_\_\_\_\_\_\_\_\_\_\_\_\_\_\_\_\_\_\_\_\_\_\_\_\_\_\_\_\_\_\_\_\_\_\_\_\_\_\_\_\_\_\_\_\_\_\_\_\_\_\_\_\_\_\_\_\_\_\_\_\_\_\_\_\_\_\_\_\_\_\_\_\_\_\_\_\_\_\_\_\_\_\_\_\_\_\_\_\_\_\_\_\_\_\_\_\_\_\_\_\_\_\_\_\_\_\_\_\_\_\_\_\_\_\_\_\_\_\_\_\_\_\_\_\_\_\_\_\_\_\_\_\_\_\_\_\_\_\_\_\_\_\_\_\_\_\_\_\_\_\_\_\_\_\_\_\_\_\_\_\_\_\_\_\_\_\_\_\_\_\_\_\_\_\_\_\_\_\_\_\_\_\_\_\_\_\_\_\_\_\_\_\_\_\_\_\_\_\_\_\_\_\_\_\_\_\_\_\_\_\_\_\_\_\_\_\_\_\_\_\_\_\_\_\_\_\_\_\_\_\_\_\_\_\_\_\_\_\_\_\_\_\_\_\_\_\_\_\_\_\_\_\_\_\_\_\_\_\_\_\_\_\_\_\_\_\_\_\_\_\_\_\_\_\_\_\_\_\_\_\_\_\_\_\_\_\_\_\_\_\_\_\_\_\_\_\_\_\_\_\_\_\_\_\_\_\_\_\_\_\_\_\_\_\_\_\_\_\_\_\_\_\_\_\_\_\_\_\_\_\_\_\_\_\_\_\_\_\_\_\_\_\_\_\_\_\_\_\_\_\_\_\_\_\_\_\_\_\_\_\_\_\_\_\_\_\_\_\_\_\_\_\_\_\_\_\_\_\_\_\_\_\_\_\_\_\_\_\_\_\_\_\_\_\_\_\_\_\_\_\_\_\_\_\_\_\_\_\_\_\_\_\_\_\_\_\_\_\_\_\_\_\_\_\_\_\_\_\_\_\_\_\_\_\_\_\_\_\_\_\_\_\_\_\_\_\_\_\_\_\_\_\_\_\_\_\_\_\_\_\_\_\_\_\_\_\_\_\_\_\_\_\_\_\_\_\_\_\_\_\_\_\_\_\_\_\_\_\_\_\_\_\_\_\_\_\_\_\_\_\_\_\_\_\_\_\_\_\_\_\_\_\_\_\_\_\_\_\_\_\_\_\_\_\_\_\_\_\_\_\_\_\_\_\_\_\_\_\_\_\_\_\_\_\_\_\_\_\_\_\_\_\_\_\_\_\_\_\_\_\_\_\_\_\_\_\_\_\_\_\_\_\_\_\_\_\_\_\_\_\_\_\_\_\_\_\_\_\_\_\_\_\_\_\_\_\_\_\_\_\_\_\_\_\_\_\_\_\_\_\_\_\_\_\_\_\_\_\_\_\_\_\_\_\_\_\_\_\_\_\_\_\_\_\_\_\_\_\_\_\_\_\_\_\_\_\_\_\_\_\_\_\_\_\_\_\_\_\_\_\_\_\_\_\_\_\_\_\_\_\_\_\_\_\_\_\_\_\_\_\_\_\_\_\_\_\_\_\_\_\_\_\_\_\_\_\_\_\_\_\_\_\_\_\_\_\_\_\_\_\_\_\_\_\_\_\_\_\_\_\_\_\_\_\_\_\_\_\_\_\_\_\_\_\_\_\_\_\_\_\_\_\_\_\_\_\_\_\_\_\_\_\_\_\_\_\_\_\_\_\_\_\_\_\_\_\_\_\_\_\_\_\_\_\_\_\_\_\_\_\_\_\_\_\_\_\_\_\_\_\_\_\_\_\_\_\_\_\_\_\_\_\_\_\_\_\_\_\_\_\_\_\_\_\_\_\_\_\_\_\_\_\_\_\_\_\_\_\_\_\_\_\_\_\_\_\_\_\_\_\_\_\_\_\_\_\_\_\_\_\_\_\_\_\_\_\_\_\_\_\_\_\_\_\_\_\_\_\_\_\_\_\_\_\_\_\_\_\_\_\_\_\_\_\_\_\_\_\_\_\_\_\_\_\_\_\_\_\_\_\_\_\_\_\_\_\_\_\_\_\_\_\_\_\_\_\_\_\_\_\_\_\_\_\_\_\_\_\_\_\_\_\_\_\_\_\_\_\_\_\_\_\_\_\_\_\_\_\_\_\_\_\_\_\_\_\_\_\_\_\_\_\_\_\_\_\_\_\_\_\_\_\_\_\_\_\_\_\_\_\_\_\_\_\_\_\_\_\_\_\_\_\_\_\_\_\_\_\_\_\_\_\_\_\_\_\_\_\_\_\_\_\_\_\_\_\_\_\_\_\_\_\_\_\_\_\_\_\_\_\_\_\_\_\_\_\_\_\_\_\_\_\_\_\_\_\_\_\_\_\_\_\_\_\_\_\_\_\_\_\_\_\_\_\_\_\_\_\_\_\_\_\_\_\_\_\_\_\_\_\_\_\_\_\_\_\_\_\_\_\_\_\_\_\_\_\_\_\_\_\_\_\_\_\_\_\_\_\_\_\_\_\_\_\_\_\_\_\_\_\_\_\_\_\_\_\_\_\_\_\_\_\_\_\_\_\_\_\_\_\_\_\_\_\_\_\_\_\_\_\_\_\_\_\_\_\_\_\_\_\_\_\_\_\_\_\_\_\_\_\_\_\_\_\_\_\_\_\_\_\_\_\_\_\_\_\_\_\_\_\_\_\_\_\_\_\_\_\_\_\_\_\_\_\_\_\_\_\_\_\_\_\_\_\_\_\_\_\_\_\_\_\_\_\_\_\_\_\_\_\_\_\_\_\_\_\_\_\_\_\_\_\_\_\_\_\_\_\_\_\_\_\_\_\_\_\_\_\_\_\_\_\_\_\_\_\_\_\_\_\_\_\_\_\_\_\_\_\_\_\_\_\_\_\_\_\_\_\_\_\_\_\_\_\_\_\_\_\_\_\_\_\_\_\_\_\_\_\_\_\_\_\_\_\_\_\_\_\_\_\_\_\_\_\_\_\_\_\_\_\_\_\_\_\_\_\_\_\_\_\_\_\_\_\_\_\_\_\_\_\_\_\_\_\_\_\_\_\_\_\_\_\_\_\_\_\_\_\_\_\_\_\_\_\_\_\_\_\_\_\_\_\_\_\_\_\_\_\_\_\_\_\_\_\_\_\_\_\_\_\_\_\_\_\_\_\_\_\_\_\_\_\_\_\_\_\_\_\_\_\_\_\_\_\_\_\_\_\_\_\_\_\_\_\_\_\_\_\_\_\_\_\_\_\_\_\_\_\_\_\_\_\_\_\_\_\_\_\_\_\_\_\_\_\_\_\_\_\_\_\_\_\_\_\_\_\_\_\_\_\_\_\_\_\_\_\_\_\_\_\_\_\_\_\_\_\_\_\_\_\_\_\_\_\_\_\_\_\_\_\_\_\_\_\_\_\_\_\_\_\_\_\_\_\_\_\_\_\_\_\_\_\_\_\_\_\_\_\_\_\_\_\_\_\_\_\_\_\_\_\_\_\_\_\_\_\_\_\_\_\_\_\_\_\_\_\_\_\_\_\_\_\_\_\_\_\_\_\_\_\_\_\_\_\_\_\_\_\_\_\_\_\_\_\_\_\_\_\_\_\_\_\_\_\_\_\_\_\_\_\_\_\_\_\_\_\_\_\_\_\_\_\_\_\_\_\_\_\_\_\_\_\_\_\_\_\_\_\_\_\_\_\_\_\_\_\_\_\_\_\_\_\_\_\_\_\_\_\_\_\_\_\_\_\_\_\_\_\_\_\_\_\_\_\_\_\_\_\_\_\_\_\_\_\_\_\_\_\_\_\_\_\_\_\_\_\_\_\_\_\_\_\_\_\_\_\_\_\_\_\_\_\_\_\_\_\_\_\_\_\_\_\_\_\_\_\_\_\_\_\_\_\_\_\_\_\_\_\_\_\_\_\_\_\_\_\_\_\_\_\_\_\_\_\_\_\_\_\_\_\_\_\_\_\_\_\_\_\_\_\_\_\_\_\_\_\_\_\_\_\_\_\_\_\_\_\_\_\_\_\_\_\_\_\_\_\_\_\_\_\_\_\_\_\_\_\_\_\_\_\_\_\_\_\_\_\_\_\_\_\_\_\_\_\_\_\_\_\_\_\_\_\_\_\_\_\_\_\_\_\_\_\_\_\_\_\_\_\_\_\_\_\_\_\_\_\_\_\_\_\_\_\_\_\_\_\_\_\_\_\_\_\_\_\_\_\_\_\_\_\_\_\_\_\_\_\_\_\_\_\_\_\_\_\_\_\_\_\_\_\_\_\_\_\_\_\_\_\_\_\_\_\_\_\_\_\_\_\_\_\_\_\_\_\_\_\_\_\_\_\_\_\_\_\_\_\_\_\_\_\_\_\_\_\_\_\_\_\_\_\_\_\_\_\_\_\_\_\_\_\_\_\_\_\_\_\_\_\_\_\_\_\_\_\_\_\_\_\_\_\_\_\_\_\_\_\_\_\_\_\_\_\_\_\_\_\_\_\_\_\_\_\_\_\_\_\_\_\_\_\_\_\_\_\_\_\_\_\_\_\_\_\_\_\_\_\_\_\_\_\_\_\_\_\_\_\_\_\_\_\_\_\_\_\_\_\_\_\_\_\_\_\_\_\_\_\_\_\_\_\_\_\_\_\_\_\_\_\_\_\_\_\_\_\_\_\_\_\_\_\_\_\_\_\_\_\_\_\_\_\_\_\_\_\_\_\_\_\_\_\_\_\_\_\_\_\_\_\_\_\_\_\_\_\_\_\_\_\_\_\_\_\_\_\_\_\_\_\_\_\_\_\_\_\_\_\_\_\_\_\_\_\_\_\_\_\_\_\_\_\_\_\_\_\_\_\_\_\_\_\_\_\_\_\_\_\_\_\_\_\_\_\_\_\_\_\_\_\_\_\_\_\_\_\_\_\_\_\_\_\_\_\_\_\_\_\_\_\_\_\_\_\_\_\_\_\_\_\_\_\_\_\_\_\_\_\_\_\_\_\_\_\_\_\_\_\_\_\_\_\_\_\_\_\_\_\_\_\_\_\_\_\_\_\_\_\_\_\_\_\_\_\_\_\_\_\_\_\_\_\_\_\_\_\_\_\_\_\_\_\_\_\_\_\_\_\_\_\_\_\_\_\_\_\_\_\_\_\_\_\_\_\_\_\_\_\_\_\_\_\_\_\_\_\_\_\_\_\_\_\_\_\_\_\_\_\_\_\_\_\_\_\_\_\_\_\_\_\_\_\_\_\_\_\_\_\_\_\_\_\_\_\_\_\_\_\_\_\_\_\_\_\_\_\_\_\_\_\_\_\_\_\_\_\_\_\_\_\_\_\_\_\_\_\_\_\_\_\_\_\_\_\_\_\_\_\_\_\_\_\_\_\_\_\_\_\_\_\_\_\_\_\_\_\_\_\_\_\_\_\_\_\_\_\_\_\_\_\_\_\_\_\_\_\_\_\_\_\_\_\_\_\_\_\_\_\_\_\_\_\_\_\_\_\_\_\_\_\_\_\_\_\_\_\_\_\_\_\_\_\_\_\_\_\_\_\_\_\_\_\_\_\_\_\_\_\_\_\_\_\_\_\_\_\_\_\_\_\_\_\_\_\_\_\_\_\_\_\_\_\_\_\_\_\_\_\_\_\_\_\_\_\_\_\_\_\_\_\_\_\_\_\_\_\_\_\_\_\_\_\_\_\_\_\_\_\_\_\_\_\_\_\_\_\_\_\_\_\_\_\_\_\_\_\_\_\_\_\_\_\_\_\_\_\_\_\_\_\_\_\_\_\_\_\_\_\_\_\_\_\_\_\_\_\_\_\_\_\_\_\_\_\_\_\_\_\_\_\_\_\_\_\_\_\_\_\_\_\_\_\_\_\_\_\_\_\_\_\_\_\_\_\_\_\_\_\_\_\_\_\_\_\_\_\_\_\_\_\_\_\_\_\_\_\_\_\_\_\_\_\_\_\_\_\_\_\_\_\_\_\_\_\_\_\_\_\_\_\_\_\_\_\_\_\_\_\_\_\_\_\_\_\_\_\_\_\_\_\_\_\_\_\_\_\_\_\_\_\_\_\_\_\_\_\_\_\_\_\_\_\_\_\_\_\_\_\_\_\_\_\_\_\_\_\_\_\_\_\_\_\_\_\_\_\_\_\_\_\_\_\_\_\_\_\_\_\_\_\_\_\_\_\_\_\_\_\_\_\_\_\_\_\_\_\_\_\_\_\_\_\_\_\_\_\_\_\_\_\_\_\_\_\_\_\_\_\_\_\_\_\_\_\_\_\_\_\_\_\_\_\_\_\_\_\_\_\_\_\_\_\_\_\_\_\_\_\_\_\_\_\_\_\_\_\_\_\_\_\_\_\_\_\_\_\_\_\_\_\_\_\_\_\_\_\_\_\_\_\_\_\_\_\_\_\_\_\_\_\_\_\_\_\_\_\_\_\_\_\_\_\_\_\_\_\_\_\_\_\_\_\_\_\_\_\_\_\_\_\_\_\_\_\_\_\_\_\_\_\_\_\_\_\_\_\_\_\_\_\_\_\_\_\_\_\_\_\_\_\_\_\_\_\_\_\_\_\_\_\_\_\_\_\_\_\_\_\_\_\_\_\_\_\_\_\_\_\_\_\_\_\_\_\_\_\_\_\_\_\_\_\_\_\_\_\_\_\_\_\_\_\_\_\_\_\_\_\_\_\_\_\_\_\_\_\_\_\_\_\_\_\_\_\_\_\_\_\_\_\_\_\_\_\_\_\_\_\_\_\_\_\_\_\_\_\_\_\_\_\_\_\_\_\_\_\_\_\_\_\_\_\_\_\_\_\_\_\_\_\_\_\_\_\_\_\_\_\_\_\_\_\_\_\_\_\_\_\_\_\_\_\_\_\_\_\_\_\_\_\_\_\_\_\_\_\_\_\_\_\_\_\_\_\_\_\_\_\_\_\_\_\_\_\_\_\_\_\_\_\_\_\_\_\_\_\_\_\_\_\_\_\_\_\_\_\_\_\_\_\_\_\_\_\_\_\_\_\_\_\_\_\_\_\_\_\_\_\_\_\_\_\_\_\_\_\_\_\_\_\_\_\_\_\_\_\_\_\_\_\_\_\_\_\_\_\_\_\_\_\_\_\_\_\_\_\_\_\_\_\_\_\_\_\_\_\_\_\_\_\_\_\_\_\_\_\_\_\_\_\_\_\_\_\_\_\_\_\_\_\_\_\_\_\_\_\_\_\_\_\_\_\_\_\_\_\_\_\_\_\_\_\_\_\_\_\_\_\_\_\_\_\_\_\_\_\_\_\_\_\_\_\_\_\_\_\_\_\_\_\_\_\_\_\_\_\_\_\_\_\_\_\_\_\_\_\_\_\_\_\_\_\_\_\_\_\_\_\_\_\_\_\_\_\_\_\_\_\_\_\_\_\_\_\_\_\_\_\_\_\_\_\_\_\_\_\_\_\_\_\_\_\_\_\_\_\_\_\_\_\_\_\_\_\_\_\_\_\_\_\_\_\_\_\_\_\_\_\_\_\_\_\_\_\_\_\_\_\_\_\_\_\_\_\_\_\_\_\_\_\_\_\_\_\_\_\_\_\_\_\_\_\_\_\_\_\_\_\_\_\_\_\_\_\_\_\_\_\_\_\_\_\_\_\_\_\_\_\_\_\_\_\_\_\_\_\_\_\_\_\_\_\_\_\_\_\_\_\_\_\_\_\_\_\_\_\_\_\_\_\_\_\_\_\_\_\_\_\_\_\_\_\_\_\_\_\_\_\_\_\_\_\_\_\_\_\_\_\_\_\_\_\_\_\_\_\_\_\_\_\_\_\_\_\_\_\_\_\_\_\_\_\_\_\_\_\_\_\_\_\_\_\_\_\_\_\_\_\_\_\_\_\_\_\_\_\_\_\_\_\_\_\_\_\_\_\_\_\_\_\_\_\_\_\_\_\_\_\_\_\_\_\_\_\_\_\_\_\_\_\_\_\_\_\_\_\_\_\_\_\_\_\_\_\_\_\_\_\_\_\_\_\_\_\_\_\_\_\_\_\_\_\_\_\_\_\_\_\_\_\_\_\_\_\_\_\_\_\_\_\_\_\_\_\_\_\_\_\_\_\_\_\_\_\_\_\_\_\_\_\_\_\_\_\_\_\_\_\_\_\_\_\_\_\_\_\_\_\_\_\_\_\_\_\_\_\_\_\_\_\_\_\_\_\_\_\_\_\_\_\_\_\_\_\_\_\_\_\_\_\_\_\_\_\_\_\_\_\_\_\_\_\_\_\_\_\_\_\_\_\_\_\_\_\_\_\_\_\_\_\_\_\_\_\_\_\_\_\_\_\_\_\_\_\_\_\_\_\_\_\_\_\_\_\_\_\_\_\_\_\_\_\_\_\_\_\_\_\_\_\_\_\_\_\_\_\_\_\_\_\_\_\_\_\_\_\_\_\_\_\_\_\_\_\_\_\_\_\_\_\_\_\_\_\_\_\_\_\_\_\_\_\_\_\_\_\_\_\_\_\_\_\_\_\_\_\_\_\_\_\_\_\_\_\_\_\_\_\_\_\_\_\_\_\_\_\_\_\_\_\_\_\_\_\_\_\_\_\_\_\_\_\_\_\_\_\_\_\_\_\_\_\_\_\_\_\_\_\_\_\_\_\_\_\_\_\_\_\_\_\_\_\_\_\_\_\_\_\_\_\_\_\_\_\_\_\_\_\_\_\_\_\_\_\_\_\_\_\_\_\_\_\_\_\_\_\_\_\_\_\_\_\_\_\_\_\_\_\_\_\_\_\_\_\_\_\_\_\_\_\_\_\_\_\_\_\_\_\_\_\_\_\_\_\_\_\_\_\_\_\_\_\_\_\_\_\_\_\_\_\_\_\_\_\_\_\_\_\_\_\_\_\_\_\_\_\_\_\_\_\_\_\_\_\_\_\_\_\_\_\_\_\_\_\_\_\_\_\_\_\_\_\_\_\_\_\_\_\_\_\_\_\_\_\_\_\_\_\_\_\_\_\_\_\_\_\_\_\_\_\_\_\_\_\_\_\_\_\_\_\_\_\_\_\_\_\_\_\_\_\_\_\_\_\_\_\_\_\_\_\_\_\_\_\_\_\_\_\_\_\_\_\_\_\_\_\_\_\_\_\_\_\_\_\_\_\_\_\_\_\_\_\_\_\_\_\_\_\_\_\_\_\_\_\_\_\_\_\_\_\_\_\_\_\_\_\_\_\_\_\_\_\_\_\_\_\_\_\_\_\_\_\_\_\_\_\_\_\_\_\_\_\_\_\_\_\_\_\_\_\_\_\_\_\_\_\_\_\_\_\_\_\_\_\_\_\_\_\_\_\_\_\_\_\_\_\_\_\_\_\_\_\_\_\_\_\_\_\_\_\_\_\_\_\_\_\_\_\_\_\_\_\_\_\_\_\_\_\_\_\_\_\_\_\_\_\_\_\_\_\_\_\_\_\_\_\_\_\_\_\_\_\_\_\_\_\_\_\_\_\_\_\_\_\_\_\_\_\_\_\_\_\_\_\_\_\_\_\_\_\_\_\_\_\_\_\_\_\_\_\_\_\_\_\_\_\_\_\_\_\_\_\_\_\_\_\_\_\_\_\_\_\_\_\_\_\_\_\_\_\_\_\_\_\_\_\_\_\_\_\_\_\_\_\_\_\_\_\_\_\_\_\_\_\_\_\_\_\_\_\_\_\_\_\_\_\_\_\_\_\_\_\_\_\_\_\_\_\_\_\_\_\_\_\_\_\_\_\_\_\_\_\_\_\_\_\_\_\_\_\_\_\_\_\_\_\_\_\_\_\_\_\_\_\_\_\_\_\_\_\_\_\_\_\_\_\_\_\_\_\_\_\_\_\_\_\_\_\_\_\_\_\_\_\_\_\_\_\_\_\_\_\_\_\_\_\_\_\_\_\_\_\_\_\_\_\_\_\_\_\_\_\_\_\_\_\_\_\_\_\_\_\_\_\_\_\_\_\_\_\_\_\_\_\_\_\_\_\_\_\_\_\_\_\_\_\_\_\_\_\_\_\_\_\_\_\_\_\_\_\_\_\_\_\_\_\_\_\_\_\_\_\_\_\_\_\_\_\_\_\_\_\_\_\_\_\_\_\_\_\_\_\_\_\_\_\_\_\_\_\_\_\_\_\_\_\_\_\_\_\_\_\_\_\_\_\_\_\_\_\_\_\_\_\_\_\_\_\_\_\_\_\_\_\_\_\_\_\_\_\_\_\_\_\_\_\_\_\_\_\_\_\_\_\_\_\_\_\_\_\_\_\_\_\_\_\_\_\_\_\_\_\_\_\_\_\_\_\_\_\_\_\_\_\_\_\_\_\_\_\_\_\_\_\_\_\_\_\_\_\_\_\_\_\_\_\_\_\_\_\_\_\_\_\_\_\_\_\_\_\_\_\_\_\_\_\_\_\_\_\_\_\_\_\_\_\_\_\_\_\_\_\_\_\_\_\_\_\_\_\_\_\_\_\_\_\_\_\_\_\_\_\_\_\_\_\_\_\_\_\_\_\_\_\_\_\_\_\_\_\_\_\_\_\_\_\_\_\_\_\_\_\_\_\_\_\_\_\_\_\_\_\_\_\_\_\_\_\_\_\_\_\_\_\_\_\_\_\_\_\_\_\_\_\_\_\_\_\_\_\_\_\_\_\_\_\_\_\_\_\_\_\_\_\_\_\_\_\_\_\_\_\_\_\_\_\_\_\_\_\_\_\_\_\_\_\_\_\_\_\_\_\_\_\_\_\_\_\_\_\_\_\_\_\_\_\_\_\_\_\_\_\_\_\_\_\_\_\_\_\_\_\_\_\_\_\_\_\_\_\_\_\_\_\_\_\_\_\_\_\_\_\_\_\_\_\_\_\_\_\_\_\_\_\_\_\_\_\_\_\_\_\_\_\_\_\_\_\_\_\_\_\_\_\_\_\_\_\_\_\_\_\_\_\_\_\_\_\_\_\_\_\_\_\_\_\_\_\_\_\_\_\_\_\_\_\_\_\_\_\_\_\_\_\_\_\_\_\_\_\_\_\_\_\_\_\_\_\_\_\_\_\_\_\_\_\_\_\_\_\_\_\_\_\_\_\_\_\_\_\_\_\_\_\_\_\_\_\_\_\_\_\_\_\_\_\_\_\_\_\_\_\_\_\_\_\_\_\_\_\_\_\_\_\_\_\_\_\_\_\_\_\_\_\_\_\_\_\_\_\_\_\_\_\_\_\_\_\_\_\_\_\_\_\_\_\_\_\_\_\_\_\_\_\_\_\_\_\_\_\_\_\_\_\_\_\_\_\_\_\_\_\_\_\_\_\_\_\_\_\_\_\_\_\_\_\_\_\_\_\_\_\_\_\_\_\_\_\_\_\_\_\_\_\_\_\_\_\_\_\_\_\_\_\_\_\_\_\_\_\_\_\_\_\_\_\_\_\_\_\_\_\_\_\_\_\_\_\_\_\_\_\_\_\_\_\_\_\_\_\_\_\_\_\_\_\_\_\_\_\_\_\_\_\_\_\_\_\_\_\_\_\_\_\_\_\_\_\_\_\_\_\_\_\_\_\_\_\_\_\_\_\_\_\_\_\_\_\_\_\_\_\_\_\_\_\_\_\_\_\_\_\_\_\_\_\_\_\_\_\_\_\_\_\_\_\_\_\_\_\_\_\_\_\_\_\_\_\_\_\_\_\_\_\_\_\_\_\_\_\_\_\_\_\_\_\_\_\_\_\_\_\_\_\_\_\_\_\_\_\_\_\_\_\_\_\_\_\_\_\_\_\_\_\_\_\_\_\_\_\_\_\_\_\_\_\_\_\_\_\_\_\_\_\_\_\_\_\_\_\_\_\_\_\_\_\_\_\_\_\_\_\_\_\_\_\_\_\_\_\_\_\_\_\_\_\_\_\_\_\_\_\_\_\_\_\_\_\_\_\_\_\_\_\_\_\_\_\_\_\_\_\_\_\_\_\_\_\_\_\_\_\_\_\_\_\_\_\_\_\_\_\_\_\_\_\_\_\_\_\_\_\_\_\_\_\_\_\_\_\_\_\_\_\_\_\_\_\_\_\_\_\_\_\_\_\_\_\_\_\_\_\_\_\_\_\_\_\_\_\_\_\_\_\_\_\_\_\_\_\_\_\_\_\_\_\_\_\_\_\_\_\_\_\_\_\_\_\_\_\_\_\_\_\_\_\_\_\_\_\_\_\_\_\_\_\_\_\_\_\_\_\_\_\_\_\_\_\_\_\_\_\_\_\_\_\_\_\_\_\_\_\_\_\_\_\_\_\_\_\_\_\_\_\_\_\_\_\_\_\_\_\_\_\_\_\_\_\_\_\_\_\_\_\_\_\_\_\_\_\_\_\_\_\_\_\_\_\_\_\_\_\_\_\_\_\_\_\_\_\_\_\_\_\_\_\_\_\_\_\_\_\_\_\_\_\_\_\_\_\_\_\_\_\_\_\_\_\_\_\_\_\_\_\_\_\_\_\_\_\_\_\_\_\_\_\_\_\_\_\_\_\_\_\_\_\_\_\_\_\_\_\_\_\_\_\_\_\_\_\_\_\_\_\_\_\_\_\_\_\_\_\_\_\_\_\_\_\_\_\_\_\_\_\_\_\_\_\_\_\_\_\_\_\_\_\_\_\_\_\_\_\_\_\_\_\_\_\_\_\_\_\_\_\_\_\_\_\_\_\_\_\_\_\_\_\_\_\_\_\_\_\_\_\_\_\_\_\_\_\_\_\_\_\_\_\_\_\_\_\_\_\_\_\_\_\_\_\_\_\_\_\_\_\_\_\_\_\_\_\_\_\_\_\_\_\_\_\_\_\_\_\_\_\_\_\_\_\_\_\_\_\_\_\_\_\_\_\_\_\_\_\_\_\_\_\_\_\_\_\_\_\_\_\_\_\_\_\_\_\_\_\_\_\_\_\_\_\_\_\_\_\_\_\_\_\_\_\_\_\_\_\_\_\_\_\_\_\_\_\_\_\_\_\_\_\_\_\_\_\_\_\_\_\_\_\_\_\_\_\_\_\_\_\_\_\_\_\_\_\_\_\_\_\_\_\_\_\_\_\_\_\_\_\_\_\_\_\_\_\_\_\_\_\_\_\_\_\_\_\_\_\_\_\_\_\_\_\_\_\_\_\_\_\_\_\_\_\_\_\_\_\_\_\_\_\_\_\_\_\_\_\_\_\_\_\_\_\_\_\_\_\_\_\_\_\_\_\_\_\_\_\_\_\_\_\_\_\_\_\_\_\_\_\_\_\_\_\_\_\_\_\_\_\_\_\_\_\_\_\_\_\_\_\_\_\_\_\_\_\_\_\_\_\_\_\_\_\_\_\_\_\_\_\_\_\_\_\_\_\_\_\_\_\_\_\_\_\_\_\_\_\_\_\_\_\_\_\_\_\_\_\_\_\_\_\_\_\_\_\_\_\_\_\_\_\_\_\_\_\_\_\_\_\_\_\_\_\_\_\_\_\_\_\_\_\_\_\_\_\_\_\_\_\_\_\_\_\_\_\_\_\_\_\_\_\_\_\_\_\_\_\_\_\_\_\_\_\_\_\_\_\_\_\_\_\_\_\_\_\_\_\_\_\_\_\_\_\_\_\_\_\_\_\_\_\_\_\_\_\_\_\_\_\_\_\_\_\_\_\_\_\_\_\_\_\_\_\_\_\_\_\_\_\_\_\_\_\_\_\_\_\_\_\_\_\_\_\_\_\_\_\_\_\_\_\_\_\_\_\_\_\_\_\_\_\_\_\_\_\_\_\_\_\_\_\_\_\_\_\_\_\_\_\_\_\_\_\_\_\_\_\_\_\_\_\_\_\_\_\_\_\_\_\_\_\_\_\_\_\_\_\_\_\_\_\_\_\_\_\_\_\_\_\_\_\_\_\_\_\_\_\_\_\_\_\_\_\_\_\_\_\_\_\_\_\_\_\_\_\_\_\_\_\_\_\_\_\_\_\_\_\_\_\_\_\_\_\_\_\_\_\_\_\_\_\_\_\_\_\_\_\_\_\_\_\_\_\_\_\_\_\_\_\_\_\_\_\_\_\_\_\_\_\_\_\_\_\_\_\_\_\_\_\_\_\_\_\_\_\_\_\_\_\_\_\_\_\_\_\_\_\_\_\_\_\_\_\_\_\_\_\_\_\_\_\_\_\_\_\_\_\_\_\_\_\_\_\_\_\_\_\_\_\_\_\_\_\_\_\_\_\_\_\_\_\_\_\_\_\_\_\_\_\_\_\_\_\_\_\_\_\_\_\_\_\_\_\_\_\_\_\_\_\_\_\_\_\_\_\_\_\_\_\_\_\_\_\_\_\_\_\_\_\_\_\_\_\_\_\_\_\_\_\_\_\_\_\_\_\_\_\_\_\_\_\_\_\_\_\_\_\_\_\_\_\_\_\_\_\_\_\_\_\_\_\_\_\_\_\_\_\_\_\_\_\_\_\_\_\_\_\_\_\_\_\_\_\_\_\_\_\_\_\_\_\_\_\_\_\_\_\_\_\_\_\_\_\_\_\_\_\_\_\_\_\_\_\_\_\_\_\_\_\_\_\_\_\_\_\_\_\_\_\_\_\_\_\_\_\_\_\_\_\_\_\_\_\_\_\_\_\_\_\_\_\_\_\_\_\_\_\_\_\_\_\_\_\_\_\_\_\_\_\_\_\_\_\_\_\_\_\_\_\_\_\_\_\_\_\_\_\_\_\_\_\_\_\_\_\_\_\_\_\_\_\_\_\_\_\_\_\_\_\_\_\_\_\_\_\_\_\_\_\_\_\_\_\_\_\_\_\_\_\_\_\_\_\_\_\_\_\_\_\_\_\_\_\_\_\_\_\_\_\_\_\_\_\_\_\_\_\_\_\_\_\_\_\_\_\_\_\_\_\_\_\_\_\_\_\_\_\_\_\_\_\_\_\_\_\_\_\_\_\_\_\_\_\_\_\_\_\_\_\_\_\_\_\_\_\_\_\_\_\_\_\_\_\_\_\_\_\_\_\_\_\_\_\_\_\_\_\_\_\_\_\_\_\_\_\_\_\_\_\_\_\_\_\_\_\_\_\_\_\_\_\_\_\_\_\_\_\_\_\_\_\_\_\_\_\_\_\_\_\_\_\_\_\_\_\_\_\_\_\_\_\_\_\_\_\_\_\_\_\_\_\_\_\_\_\_\_\_\_\_\_\_\_\_\_\_\_\_\_\_\_\_\_\_\_\_\_\_\_\_\_\_\_\_\_\_\_\_\_\_\_\_\_\_\_\_\_\_\_\_\_\_\_\_\_\_\_\_\_\_\_\_\_\_\_\_\_\_\_\_\_\_\_\_\_\_\_\_\_\_\_\_\_\_\_\_\_\_\_\_\_\_\_\_\_\_\_\_\_\_\_\_\_\_\_\_\_\_\_\_\_\_\_\_\_\_\_\_\_\_\_\_\_\_\_\_\_\_\_\_\_\_\_\_\_\_\_\_\_\_\_\_\_\_\_\_\_\_\_\_\_\_\_\_\_\_\_\_\_\_\_\_\_\_\_\_\_\_\_\_\_\_\_\_\_\_\_\_\_\_\_\_\_\_\_\_\_\_\_\_\_\_\_\_\_\_\_\_\_\_\_\_\_\_\_\_\_\_\_\_\_\_\_\_\_\_\_\_\_\_\_\_\_\_\_\_\_\_\_\_\_\_\_\_\_\_\_\_\_\_\_\_\_\_\_\_\_\_\_\_\_\_\_\_\_\_\_\_\_\_\_\_\_\_\_\_\_\_\_\_\_\_\_\_\_\_\_\_\_\_\_\_\_\_\_\_\_\_\_\_\_\_\_\_\_\_\_\_\_\_\_\_\_\_\_\_\_\_\_\_\_\_\_\_\_\_\_\_\_\_\_\_\_\_\_\_\_\_\_\_\_\_\_\_\_\_\_\_\_\_\_\_\_\_\_\_\_\_\_\_\_\_\_\_\_\_\_\_\_\_\_\_\_\_\_\_\_\_\_\_\_\_\_\_\_\_\_\_\_\_\_\_\_\_\_\_\_\_\_\_\_\_\_\_\_\_\_\_\_\_\_\_\_\_\_\_\_\_\_\_\_\_\_\_\_\_\_\_\_\_\_\_\_\_\_\_\_\_\_\_\_\_\_\_\_\_\_\_\_\_\_\_\_\_\_\_\_\_\_\_\_\_\_\_\_\_\_\_\_\_\_\_\_\_\_\_\_\_\_\_\_\_\_\_\_\_\_\_\_\_\_\_\_\_\_\_\_\_\_\_\_\_\_\_\_\_\_\_\_\_\_\_\_\_\_\_\_\_\_\_\_\_\_\_\_\_\_\_\_\_\_\_\_\_\_\_\_\_\_\_\_\_\_\_\_\_\_\_\_\_\_\_\_\_\_\_\_\_\_\_\_\_\_\_\_\_\_\_\_\_\_\_\_\_\_\_\_\_\_\_\_\_\_\_\_\_\_\_\_\_\_\_\_\_\_\_\_\_\_\_\_\_\_\_\_\_\_\_\_\_\_\_\_\_\_\_\_\_\_\_\_\_\_\_\_\_\_\_\_\_\_\_\_\_\_\_\_\_\_\_\_\_\_\_\_\_\_\_\_\_\_\_\_\_\_\_\_\_\_\_\_\_\_\_\_\_\_\_\_\_\_\_\_\_\_\_\_\_\_\_\_\_\_\_\_\_\_\_\_\_\_\_\_\_\_\_\_\_\_\_\_\_\_\_\_\_\_\_\_\_\_\_\_\_\_\_\_\_\_\_\_\_\_\_\_\_\_\_\_\_\_\_\_\_\_\_\_\_\_\_\_\_\_\_\_\_\_\_\_\_\_\_\_\_\_\_\_\_\_\_\_\_\_\_\_\_\_\_\_\_\_\_\_\_\_\_\_\_\_\_\_\_\_\_\_\_\_\_\_\_\_\_\_\_\_\_\_\_\_\_\_\_\_\_\_\_\_\_\_\_\_\_\_\_\_\_\_\_\_\_\_\_\_\_\_\_\_\_\_\_\_\_\_\_\_\_\_\_\_\_\_\_\_\_\_\_\_\_\_\_\_\_\_\_\_\_\_\_\_\_\_\_\

\begin{center}
参赛密码 \_\_\_\_\_\_\_\_\_\_\_\_\_\_\_\_\_\_\_\_\_\_\_\_\_\_\_\_\_\_\_\_\_\_\_\_\_\_\_\_\_\_\_\_\_\_\_\_\_\_\_\_\_\_\_\_\_\_\_\_\_\_\_\_\_\_\_\_\_\_\_\_\_\_\_\_\_\_\_\_\_\_\_\_\_\_\_\_\_\_\_\_\_\_\_\_\_\_\_\_\_\_\_\_\_\_\_\_\_\_\_\_\_\_\_\_\_\_\_\_\_\_\_\_\_\_\_\_\_\_\_\_\_\_\_\_\_\_\_\_\_\_\_\_\_\_\_\_\_\_\_\_\_\_\_\_\_\_\_\_\_\_\_\_\_\_\_\_\_\_\_\_\_\_\_\_\_\_\_\_\_\_\_\_\_\_\_\_\_\_\_\_\_\_\_\_\_\_\_\_\_\_\_\_\_\_\_\_\_\_\_\_\_\_\_\_\_\_\_\_\_\_\_\_\_\_\_\_\_\_\_\_\_\_\_\_\_\_\_\_\_\_\_\_\_\_\_\_\_\_\_\_\_\_\_\_\_\_\_\_\_\_\_\_\_\_\_\_\_\_\_\_\_\_\_\_\_\_\_\_\_\_\_\_\_\_\_\_\_\_\_\_\_\_\_\_\_\_\_\_\_\_\_\_\_\_\_\_\_\_\_\_\_\_\_\_\_\_\_\_\_\_\_\_\_\_\_\_\_\_\_\_\_\_\_\_\_\_\_\_\_\_\_\_\_\_\_\_\_\_\_\_\_\_\_\_\_\_\_\_\_\_\_\_\_\_\_\_\_\_\_\_\_\_\_\_\_\_\_\_\_\_\_\_\_\_\_\_\_\_\_\_\_\_\_\_\_\_\_\_\_\_\_\_\_\_\_\_\_\_\_\_\_\_\_\_\_\_\_\_\_\_\_\_\_\_\_\_\_\_\_\_\_\_\_\_\_\_\_\_\_\_\_\_\_\_\_\_\_\_\_\_\_\_\_\_\_\_\_\_\_\_\_\_\_\_\_\_\_\_\_\_\_\_\_\_\_\_\_\_\_\_\_\_\_\_\_\_\_\_\_\_\_\_\_\_\_\_\_\_\_\_\_\_\_\_\_\_\_\_\_\_\_\_\_\_\_\_\_\_\_\_\_\_\_\_\_\_\_\_\_\_\_\_\_\_\_\_\_\_\_\_\_\_\_\_\_\_\_\_\_\_\_\_\_\_\_\_\_\_\_\_\_\_\_\_\_\_\_\_\_\_\_\_\_\_\_\_\_\_\_\_\_\_\_\_\_\_\_\_\_\_\_\_\_\_\_\_\_\_\_\_\_\_\_\_\_\_\_\_\_\_\_\_\_\_\_\_\_\_\_\_\_\_\_\_\_\_\_\_\_\_\_\_\_\_\_\_\_\_\_\_\_\_\_\_\_\_\_\_\_\_\_\_\_\_\_\_\_\_\_\_\_\_\_\_\_\_\_\_\_\_\_\_\_\_\_\_\_\_\_\_\_\_\_\_\_\_\_\_\_\_\_\_\_\_\_\_\_\_\_\_\_\_\_\_\_\_\_\_\_\_\_\_\_\_\_\_\_\_\_\_\_\_\_\_\_\_\_\_\_\_\_\_\_\_\_\_\_\_\_\_\_\_\_\_\_\_\_\_\_\_\_\_\_\_\_\_\_\_\_\_\_\_\_\_\_\_\_\_\_\_\_\_\_\_\_\_\_\_\_\_\_\_\_\_\_\_\_\_\_\_\_\_\_\_\_\_\_\_\_\_\_\_\_\_\_\_\_\_\_\_\_\_\_\_\_\_\_\_\_\_\_\_\_\_\_\_\_\_\_\_\_\_\_\_\_\_\_\_\_\_\_\_\_\_\_\_\_\_\_\_\_\_\_\_\_\_\_\_\_\_\_\_\_\_\_\_\_\_\_\_\_\_\_\_\_\_\_\_\_\_\_\_\_\_\_\_\_\_\_\_\_\_\_\_\_\_\_\_\_\_\_\_\_\_\_\_\_\_\_\_\_\_\_\_\_\_\_\_\_\_\_\_\_\_\_\_\_\_\_\_\_\_\_\_\_\_\_\_\_\_\_\_\_\_\_\_\_\_\_\_\_\_\_\_\_\_\_\_\_\_\_\_\_\_\_\_\_\_\_\_\_\_\_\_\_\_\_\_\_\_\_\_\_\_\_\_\_\_\_\_\_\_\_\_\_\_\_\_\_\_\_\_\_\_\_\_\_\_\_\_\_\_\_\_\_\_\_\_\_\_\_\_\_\_\_\_\_\_\_\_\_\_\_\_\_\_\_\_\_\_\_\_\_\_\_\_\_\_\_\_\_\_\_\_\_\_\_\_\_\_\_\_\_\_\_\_\_\_\_\_\_\_\_\_\_\_\_\_\_\_\_\_\_\_\_\_\_\_\_\_\_\_\_\_\_\_\_\_\_\_\_\_\_\_\_\_\_\_\_\_\_\_\_\_\_\_\_\_\_\_\_\_\_\_\_\_\_\_\_\_\_\_\_\_\_\_\_\_\_\_\_\_\_\_\_\_\_\_\_\_\_\_\_\_\_\_\_\_\_\_\_\_\_\_\_\_\_\_\_\_\_\_\_\_\_\_\_\_\_\_\_\_\_\_\_\_\_\_\_\_\_\_\_\_\_\_\_\_\_\_\_\_\_\_\_\_\_\_\_\_\_\_\_\_\_\_\_\_\_\_\_\_\_\_\_\_\_\_\_\_\_\_\_\_\_\_\_\_\_\_\_\_\_\_\_\_\_\_\_\_\_\_\_\_\_\_\_\_\_\_\_\_\_\_\_\_\_\_\_\_\_\_\_\_\_\_\_\_\_\_\_\_\_\_\_\_\_\_\_\_\_\_\_\_\_\_\_\_\_\_\_\_\_\_\_\_\_\_\_\_\_\_\_\_\_\_\_\_\_\_\_\_\_\_\_\_\_\_\_\_\_\_\_\_\_\_\_\_\_\_\_\_\_\_\_\_\_\_\_\_\_\_\_\_\_\_\_\_\_\_\_\_\_\_\_\_\_\_\_\_\_\_\_\_\_\_\_\_\_\_\_\_\_\_\_\_\_\_\_\_\_\_\_\_\_\_\_\_\_\_\_\_\_\_\_\_\_\_\_\_\_\_\_\_\_\_\_\_\_\_\_\_\_\_\_\_\_\_\_\_\_\_\_\_\_\_\_\_\_\_\_\_\_\_\_\_\_\_\_\_\_\_\_\_\_\_\_\_\_\_\_\_\_\_\_\_\_\_\_\_\_\_\_\_\_\_\_\_\_\_\_\_\_\_\_\_\_\_\_\_\_\_\_\_\_\_\_\_\_\_\_\_\_\_\_\_\_\_\_\_\_\_\_\_\_\_\_\_\_\_\_\_\_\_\_\_\_\_\_\_\_\_\_\_\_\_\_\_\_\_\_\_\_\_\_\_\_\_\_\_\_\_\_\_\_\_\_\_\_\_\_\_\_\_\_\_\_\_\_\_\_\_\_\_\_\_\_\_\_\_\_\_\_\_\_\_\_\_\_\_\_\_\_\_\_\_\_\_\_\_\_\_\_\_\_\_\_\_\_\_\_\_\_\_\_\_\_\_\_\_\_\_\_\_\_\_\_\_\_\_\_\_\_\_\_\_\_\_\_\_\_\_\_\_\_\_\_\_\_\_\_\_\_\_\_\_\_\_\_\_\_\_\_\_\_\_\_\_\_\_\_\_\_\_\_\_\_\_\_\_\_\_\_\_\_\_\_\_\_\_\_\_\_\_\_\_\_\_\_\_\_\_\_\_\_\_\_\_\_\_\_\_\_\_\_\_\_\_\_\_\_\_\_\_\_\_\_\_\_\_\_\_\_\_\_\_\_\_\_\_\_\_\_\_\_\_\_\_\_\_\_\_\_\_\_\_\_\_\_\_\_\_\_\_\_\_\_\_\_\_\_\_\_\_\_\_\_\_\_\_\_\_\_\_\_\_\_\_\_\_\_\_\_\_\_\_\_\_\_\_\_\_\_\_\_\_\_\_\_\_\_\_\_\_\_\_\_\_\_\_\_\_\_\_\_\_\_\_\_\_\_\_\_\_\_\_\_\_\_\_\_\_\_\_\_\_\_\_\_\_\_\_\_\_\_\_\_\_\_\_\_\_\_\_\_\_\_\_\_\_\_\_\_\_\_\_\_\_\_\_\_\_\_\_\_\_\_\_\_\_\_\_\_\_\_\_\_\_\_\_\_\_\_\_\_\_\_\_\_\_\_\_\_\_\_\_\_\_\_\_\_\_\_\_\_\_\_\_\_\_\_\_\_\_\_\_\_\_\_\_\_\_\_\_\_\_\_\_\_\_\_\_\_\_\_\_\_\_\_\_\_\_\_\_\_\_\_\_\_\_\_\_\_\_\_\_\_\_\_\_\_\_\_\_\_\_\_\_\_\_\_\_\_\_\_\_\_\_\_\_\_\_\_\_\_\_\_\_\_\_\_\_\_\_\_\_\_\_\_\_\_\_\_\_\_\_\_\_\_\_\_\_\_\_\_\_\_\_\_\_\_\_\_\_\_\_\_\_\_\_\_\_\_\_\_\_\_\_\_\_\_\_\_\_\_\_\_\_\_\_\_\_\_\_\_\_\_\_\_\_\_\_\_\_\_\_\_\_\_\_\_\_\_\_\_\_\_\_\_\_\_\_\_\_\_\_\_\_\_\_\_\_\_\_\_\_\_\_\_\_\_\_\_\_\_\_\_\_\_\_\_\_\_\_\_\_\_\_\_\_\_\_\_\_\_\_\_\_\_\_\_\_\_\_\_\_\_\_\_\_\_\_\_\_\_\_\_\_\_\_\_\_\_\_\_\_\_\_\_\_\_\_\_\_\_\_\_\_\_\_\_\_\_\_\_\_\_\_\_\_\_\_\_\_\_\_\_\_\_\_\_\_\_\_\_\_\_\_\_\_\_\_\_\_\_\_\_\_\_\_\_\_\_\_\_\_\_\_\_\_\_\_\_\_\_\_\_\_\_\_\_\_\_\_\_\_\_\_\_\_\_\_\_\_\_\_\_\_\_\_\_\_\_\_\_\_\_\_\_\_\_\_\_\_\_\_\_\_\_\_\_\_\_\_\_\_\_\_\_\_\_\_\_\_\_\_\_\_\_\_\_\_\_\_\_\_\_\_\_\_\_\_\_\_\_\_\_\_\_\_\_\_\_\_\_\_\_\_\_\_\_\_\_\_\_\_\_\_\_\_\_\_\_\_\_\_\_\_\_\_\_\_\_\_\_\_\_\_\_\_\_\_\_\_\_\_\_\_\_\_\_\_\_\_\_\_\_\_\_\_\_\_\_\_\_\_\_\_\_\_\_\_\_\_\_\_\_\_\_\_\_\_\_\_\_\_\_\_\_\_\_\_\_\_\_\_\_\_\_\_\_\_\_\_\_\_\_\_\_\_\_\_\_\_\_\_\_\_\_\_\_\_\_\_\_\_\_\_\_\_\_\_\_\_\_\_\_\_\_\_\_\_\_\_\_\_\_\_\_\_\_\_\_\_\_\_\_\_\_\_\_\_\_\_\_\_\_\_\_\_\_\_\_\_\_\_\_\_\_\_\_\_\_\_\_\_\_\_\_\_\_\_\_\_\_\_\_\_\_\_\_\_\_\_\_\_\_\_\_\_\_\_\_\_\_\_\_\_\_\_\_\_\_\_\_\_\_\_\_\_\_\_\_\_\_\_\_\_\_\_\_\_\_\_\_\_\_\_\_\_\_\_\_\_\_\_\_\_\_\_\_\_\_\_\_\_\_\_\_\_\_\_\_\_\_\_\_\_\_\_\_\_\_\_\_\_\_\_\_\_\_\_\_\_\_\_\_\_\_\_\_\_\_\_\_\_\_\_\_\_\_\_\_\_\_\_\_\_\_\_\_\_\_\_\_\_\_\_\_\_\_\_\_\_\_\_\_\_\_\_\_\_\_\_\_\_\_\_\_\_\_\_\_\_\_\_\_\_\_\_\_\_\_\_\_\_\_\_\_\_\_\_\_\_\_\_\_\_\_\_\_\_\_\_\_\_\_\_\_\_\_\_\_\_\_\_\_\_\_\_\_\_\_\_\_\_\_\_\_\_\_\_\_\_\_\_\_\_\_\_\_\_\_\_\_\_\_\_\_\_\_\_\_\_\_\_\_\_\_\_\_\_\_\_\_\_\_\_\_\_\_\_\_\_\_\_\_\_\_\_\_\_\_\_\_\_\_\_\_\_\_\_\_\_\_\_\_\_\_\_\_\_\_\_\_\_\_\_\_\_\_\_\_\_\_\_\_\_\_\_\_\_\_\_\_\_\_\_\_\_\_\_\_\_\_\_\_\_\_\_\_\_\_\_\_\_\_\_\_\_\_\_\_\_\_\_\_\_\_\_\_\_\_\_\_\_\_\_\_\_\_\_\_\_\_\_\_\_\_\_\_\_\_\_\_\_\_\_\_\_\_\_\_\_\_\_\_\_\_\_\_\_\_\_\_\_\_\_\_\_\_\_\_\_\_\_\_\_\_\_\_\_\_\_\_\_\_\_\_\_\_\_\_\_\_\_\_\_\_\_\_\_\_\_\_\_\_\_\_\_\_\_\_\_\_\_\_\_\_\_\_\_\_\_\_\_\_\_\_\_\_\_\_\_\_\_\_\_\_\_\_\_\_\_\_\_\_\_\_\_\_\_\_\_\_\_\_\_\_\_\_\_\_\_\_\_\_\_\_\_\_\_\_\_\_\_\_\_\_\_\_\_\_\_\_\_\_\_\_\_\_\_\_\_\_\_\_\_\_\_\_\_\_\_\_\_\_\_\_\_\_\_\_\_\_\_\_\_\_\_\_\_\_\_\_\_\_\_\_\_\_\_\_\_\_\_\_\_\_\_\_\_\_\_\_\_\_\_\_\_\_\_\_\_\_\_\_\_\_\_\_\_\_\_\_\_\_\_\_\_\_\_\_\_\_\_\_\_\_\_\_\_\_\_\_\_\_\_\_\_\_\_\_\_\_\_\_\_\_\_\_\_\_\_\_\_\_\_\_\_\_\_\_\_\_\_\_\_\_\_\_\_\_\_\_\_\_\_\_\_\_\_\_\_\_\_\_\_\_\_\_\_\_\_\_\_\_\_\_\_\_\_\_\_\_\_\_\_\_\_\_\_\_\_\_\_\_\_\_\_\_\_\_\_\_\_\_\_\_\_\_\_\_\_\_\_\_\_\_\_\_\_\_\_\_\_\_\_\_\_\_\_\_\_\_\_\_\_\_\_\_\_\_\_\_\_\_\_\_\_\_\_\_\_\_\_\_\_\_\_\_\_\_\_\_\_\_\_\_\_\_\_\_\_\_\_\_\_\_\_\_\_\_\_\_\_\_\_\_\_\_\_\_\_\_\_\_\_\_\_\_\_\_\_\_\_\_\_\_\_\_\_\_\_\_\_\_\_\_\_\_\_\_\_\_\_\_\_\_\_\_\_\_\_\_\_\_\_\_\_\_\_\_\_\_\_\_\_\_\_\_\_\_\_\_\_\_\_\_\_\_\_\_\_\_\_\_\_\_\_\_\_\_\_\_\_\_\_\_\_\_\_\_\_\_\_\_\_\_\_\_\_\_\_\_\_\_\_\_\_\_\_\_\_\_\_\_\_\_\_\_\_\_\_\_\_\_\_\_\_\_\_\_\_\_\_\_\_\_\_\_\_\_\_\_\_\_\_\_\_\_\_\_\_\_\_\_\_\_\_\_\_\_\_\_\_\_\_\_\_\_\_\_\_\_\_\_\_\_\_\_\_\_\_\_\_\_\_\_\_\_\_\_\_\_\_\_\_\_\_\_\_\_\_\_\_\_\_\_\_\_\_\_\_\_\_\_\_\_\_\_\_\_\_\_\_\_\_\_\_\_\_\_\_\_\_\_\_\_\_\_\_\_\_\_\_\_\_\_\_\_\_\_\_\_\_\_\_\_\_\_\_\_\_\_\_\_\_\_\_\_\_\_\_\_\_\_\_\_\_\_\_\_\_\_\_\_\_\_\_\_\_\_\_\_\_\_\_\_\_\_\_\_\_\_\_\_\_\_\_\_\_\_\_\_\_\_\_\_\_\_\_\_\_\_\_\_\_\_\_\_\_\_\_\_\_\_\_\_\_\_\_\_\_\_\_\_\_\_\_\_\_\_\_\_\_\_\_\_\_\_\_\_\_\_\_\_\_\_\_\_\_\_\_\_\_\_\_\_\_\_\_\_\_\_\_\_\_\_\_\_\_\_\_\_\_\_\_\_\_\_\_\_\_\_\_\_\_\_\_\_\_\_\_\_\_\_\_\_\_\_\_\_\_\_\_\_\_\_\_\_\_\_\_\_\_\_\_\_\_\_\_\_\_\_\_\_\_\_\_\_\_\_\_\_\_\_\_\_\_\_\_\_\_\_\_\_\_\_\_\_\_\_\_\_\_\_\_\_\_\_\_\_\_\_\_\_\_\_\_\_\_\_\_\_\_\_\_\_\_\_\_\_\_\_\_\_\_\_\_\_\_\_\_\_\_\_\_\_\_\_\_\_\_\_\_\_\_\_\_\_\_\_\_\_\_\_\_\_\_\_\_\_\_\_\_\_\_\_\_\_\_\_\_\_\_\_\_\_\_\_\_\_\_\_\_\_\_\_\_\_\_\_\_\_\_\_\_\_\_\_\_\_\_\_\_\_\_\_\_\_\_\_\_\_\_\_\_\_\_\_\_\_\_\_\_\_\_\_\_\_\_\_\_\_\_\_\_\_\_\_\_\_\_\_\_\_\_\_\_\_\_\_\_\_\_\_\_\_\_\_\_\_\_\_\_\_\_\_\_\_\_\_\_\_\_\_\_\_\_\_\_\_\_\_\_\_\_\_\_\_\_\_\_\_\_\_\_\_\_\_\_\_\_\_\_\_\_\_\_\_\_\_\_\_\_\_\_\_\_\_\_\_\_\_\_\_\_\_\_\_\_\_\_\_\_\_\_\_\_\_\_\_\_\_\_\_\_\_\_\_\_\_\_\_\_\_\_\_\_\_\_\_\_\_\_\_\_\_\_\_\_\_\_\_\_\_\_\_\_\_\_\_\_\_\_\_\_\_\_\_\_\_\_\_\_\_\_\_\_\_\_\_\_\_\_\_\_\_\_\_\_\_\_\_\_\_\_\_\_\_\_\_\_\_\_\_\_\_\_\_\_\_\_\_\_\_\_\_\_\_\_\_\_\_\_\_\_\_\_\_\_\_\_\_\_\_\_\_\_\_\_\_\_\_\_\_\_\_\_\_\_\_\_\_\_\_\_\_\_\_\_\_\_\_\_\_\_\_\_\_\_\_\_\_\_\_\_\_\_\_\_\_\_\_\_\_\_\_\_\_\_\_\_\_\_\_\_\_\_\_\_\_\_\_\_\_\_\_\_\_\_\_\_\_\_\_\_\_\_\_\_\_\_\_\_\_\_\_\_\_\_\_\_\_\_\_\_\_\_\_\_\_\_\_\_\_\_\_\_\_\_\_\_\_\_\_\_\_\_\_\_\_\_\_\_\_\_\_\_\_\_\_\_\_\_\_\_\_\_\_\_\_\_\_\_\_\_\_\_\_\_\_\_\_\_\_\_\_\_\_\_\_\_\_\_\_\_\_\_\_\_\_\_\_\_\_\_\_\_\_\_\_\_\_\_\_\_\_\_\_\_\_\_\_\_\_\_\_\_\_\_\_\_\_\_\_\_\_\_\_\_\_\_\_\_\_\_\_\_\_\_\_\_\_\_\_\_\_\_\_\_\_\_\_\_\_\_\_\_\_\_\_\_\_\_\_\_\_\_\_\_\_\_\_\_\_\_\_\_\_\_\_\_\_\_\_\_\_\_\_\_\_\_\_\_\_\_\_\_\_\_\_\_\_\_\_\_\_\_\_\_\_\_\_\_\_\_\_\_\_\_\_\_\_\_\_\_\_\_\_\_\_\_\_\_\_\_\_\_\_\_\_\_\_\_\_\_\_\_\_\_\_\_\_\_\_\_\_\_\_\_\_\_\_\_\_\_\_\_\_\_\_\_\_\_\_\_\_\_\_\_\_\_\_\_\_\_\_\_\_\_\_\_\_\_\_\_\_\_\_\_\_\_\_\_\_\_\_\_\_\_\_\_\_\_\_\_\_\_\_\_\_\_\_\_\_\_\_\_\_\_\_\_\_\_\_\_\_\_\_\_\_\_\_\_\_\_\_\_\_\_\_\_\_\_\_\_\_\_\_\_\_\_\_\_\_\_\_\_\_\_\_\_\_\_\_\_\_\_\_\_\_\_\_\_\_\_\_\_\_\_\_\_\_\_\_\_\_\_\_\_\_\_\_\_\_\_\_\_\_\_\_\_\_\_\_\_\_\_\_\_\_\_\_\_\_\_\_\_\_\_\_\_\_\_\_\_\_\_\_\_\_\_\_\_\_\_\_\_\_\_\_\_\_\_\_\_\_\_\_\_\_\_\_\_\_\_\_\_\_\_\_\_\_\_\_\_\_\_\_\_\_\_\_\_\_\_\_\_\_\_\_\_\_\_\_\_\_\_\_\_\_\_\_\_\_\_\_\_\_\_\_\_\_\_\_\_\_\_\_\_\_\_\_\_\_\_\_\_\_\_\_\_\_\_\_\_\_\_\_\_\_\_\_\_\_\_\_\_\_\_\_\_\_\_\_\_\_\_\_\_\_\_\_\_\_\_\_\_\_\_\_\_\_\_\_\_\_\_\_\_\_\_\_\_\_\_\_\_\_\_\_\_\_\_\_\_\_\_\_\_\_\_\_\_\_\_\_\_\_\_\_\_\_\_\_\_\_\_\_\_\_\_\_\_\_\_\_\_\_\_\_\_\_\_\_\_\_\_\_\_\_\_\_\_\_\_\_\_\_\_\_\_\_\_\_\_\_\_\_\_\_\_\_\_\_\_\_\_\_\_\_\_\_\_\_\_\_\_\_\_\_\_\_\_\_\_\_\_\_\_\_\_\_\_\_\_\_\_\_\_\_\_\_\_\_\_\_\_\_\_\_\_\_\_\_\_\_\_\_\_\_\_\_\_\_\_\_\_\_\_\_\_\_\_\_\_\_\_\_\_\_\_\_\_\_\_\_\_\_\_\_\_\_\_\_\_\_\_\_\_\_\_\_\_\_\_\_\_\_\_\_\_\_\_\_\_\_\_\_\_\_\_\_\_\_\_\_\_\_\_\_\_\_\_\_\_\_\_\_\_\_\_\_\_\_\_\_\_\_\_\_\_\_\_\_\_\_\_\_\_\_\_\_\_\_\_\_\_\_\_\_\_\_\_\_\_\_\_\_\_\_\_\_\_\_\_\_\_\_\_\_\_\_\_\_\_\_\_\_\_\_\_\_\_\_\_\_\_\_\_\_\_\_\_\_\_\_\_\_\_\_\_\_\_\_\_\_\_\_\_\_\_\_\_\_\_\_\_\_\_\_\_\_\_\_\_\_\_\_\_\_\_\_\_\_\_\_\_\_\_\_\_\_\_\_\_\_\_\_\_\_\_\_\_\_\_\_\_\_\_\_\_\_\_\_\_\_\_\_\_\_\_\_\_\_\_\_\_\_\_\_\_\_\_\_\_\_\_\_\_\_\_\_\_\_\_\_\_\_\_\_\_\_\_\_\_\_\_\_\_\_\_\_\_\_\_\_\_\_\_\_\_\_\_\_\_\_\_\_\_\_\_\_\_\_\_\_\_\_\_\_\_\_\_\_\_\_\_\_\_\_\_\_\_\_\_\_\_\_\_\_\_\_\_\_\_\_\_\_\_\_\_\_\_\_\_\_\_\_\_\_\_\_\_\_\_\_\_\_\_\_\_\_\_\_\_\_\_\_\_\_\_\_\_\_\_\_\_\_\_\_\_\_\_\_\_\_\_\_\_\_\_\_\_\_\_\_\_\_\_\_\_\_\_\_\_\_\_\_\_\_\_\_\_\_\_\_\_\_\_\_\_\_\_\_\_\_\_\_\_\_\_\_\_\_\_\_\_\_\_\_\_\_\_\_\_\_\_\_\_\_\_\_\_\_\_\_\_\_\_\_\_\_\_\_\_\_\_\_\_\_\_\_\_\_\_\_\_\_\_\_\_\_\_\_\_\_\_\_\_\_\_\_\_\_\_\_\_\_\_\_\_\_\_\_\_\_\_\_\_\_\_\_\_\_\_\_\_\_\_\_\_\_\_\_\_\_\_\_\_\_\_\_\_\_\_\_\_\_\_\_\_\_\_\_\_\_\_\_\_\_\_\_\_\_\_\_\_\_\_\_\_\_\_\_\_\_\_\_\_\_\_\_\_\_\_\_\_\_\_\_\_\_\_\_\_\_\_\_\_\_\_\_\_\_\_\_\_\_\_\_\_\_\_\_\_\_\_\_\_\_\_\_\_\_\_\_\_\_\_\_\_\_\_\_\_\_\_\_\_\_\_\_\_\_\_\_\_\_\_\_\_\_\_\_\_\_\_\_\_\_\_\_\_\_\_\_\_\_\_\_\_\_\_\_\_\_\_\_\_\_\_\_\_\_\_\_\_\_\_\_\_\_\_\_\_\_\_\_\_\_\_\_\_\_\_\_\_\_\_\_\_\_\_\_\_\_\_\_\_\_\_\_\_\_\_\_\_\_\_\_\_\_\_\_\_\_\_\_\_\_\_\_\_\_\_\_\_\_\_\_\_\_\_\_\_\_\_\_\_\_\_\_\_\_\_\_\_\_\_\_\_\_\_\_\_\_\_\_\_\_\_\_\_\_\_\_\_\_\_\_\_\_\_\_\_\_\_\_\_\_\_\_\_\_\_\_\_\_\_\_\_\_\_\_\_\_\_\_\_\_\_\_\_\_\_\_\_\_\_\_\_\_\_\_\_\_\_\_\_\_\_\_\_\_\_\_\_\_\_\_\_\_\_\_\_\_\_\_\_\_\_\_\_\_\_\_\_\_\_\_\_\_\_\_\_\_\_\_\_\_\_\_\_\_\_\_\_\_\_\_\_\_\_\_\_\_\_\_\_\_\_\_\_\_\_\_\_\_\_\_\_\_\_\_\_\_\_\_\_\_\_\_\_\_\_\_\_\_\_\_\_\_\_\_\_\_\_\_\_\_\_\_\_\_\_\_\_\_\_\_\_\_\_\_\_\_\_\_\_\_\_\_\_\_\_\_\_\_\_\_\_\_\_\_\_\_\_\_\_\_\_\_\_\_\_\_\_\_\_\_\_\_\_\_\_\_\_\_\_\_\_\_\_\_\_\_\_\_\_\_\_\_\_\_\_\_\_\_\_\_\_\_\_\_\_\_\_\_\_\_\_\_\_\_\_\_\_\_\_\_\_\_\_\_\_\_\_\_\_\_\_\_\_\_\_\_\_\_\_\_\_\_\_\_\_\_\_\_\_\_\_\_\_\_\_\_\_\_\_\_\_\_\_\_\_\_\_\_\_\_\_\_\_\_\_\_\_\_\_\_\_\_\_\_\_\_\_\_\_\_\_\_\_\_\_\_\_\_\_\_\_\_\_\_\_\_\_\_\_\_\_\_\_\_\_\_\_\_\_\_\_\_\_\_\_\_\_\_\_\_\_\_\_\_\_\_\_\_\_\_\_\_\_\_\_\_\_\_\_\_\_\_\_\_\_\_\_\_\_\_\_\_\_\_\_\_\_\_\_\_\_\_\_\_\_\_\_\_\_\_\_\_\_\_\_\_\_\_\_\_\_\_\_\_\_\_\_\_\_\_\_\_\_\_\_\_\_\_\_\_\_\_\_\_\_\_\_\_\_\_\_\_\_\_\_\_\_\_\_\_\_\_\_\_\_\_\_\_\_\_\_\_\_\_\_\_\_\_\_\_\_\_\_\_\_\_\_\_\_\_\_\_\_\_\_\_\_\_\_\_\_\_\_\_\_\_\_\_\_\_\_\_\_\_\_\_\_\_\_\_\_\_\_\_\_\_\_\_\_\_\_\_\_\_\_\_\_\_\_\_\_\_\_\_\_\_\_\_\_\_\_\_\_\_\_\_\_\_\_\_\_\_\_\_\_\_\_\_\_\_\_\_\_\_\_\_\_\_\_\_\_\_\_\_\_\_\_\_\_\_\_\_\_\_\_\_\_\_\_\_\_\_\_\_\_\_\_\_\_\_\_\_\_\_\_\_\_\_\_\_\_\_\_\_\_\_\_\_\_\_\_\_\_\_\_\_\_\_\_\_\_\_\_\_\_\_\_\_\_\_\_\_\_\_\_\_\_\_\_\_\_\_\_\_\_\_\_\_\_\_\_\_\_\_\_\_\_\_\_\_\_\_\_\_\_\_\_\_\_\_\_\_\_\_\_\_\_\_\_\_\_\_\_\_\_\_\_\_\_\_\_\_\_\_\_\_\_\_\_\_\_\_\_\_\_\_\_\_\_\_\_\_\_\_\_\_\_\_\_\_\_\_\_\_\_\_\_\_\_\_\_\_\_\_\_\_\_\_\_\_\_\_\_\_\_\_\_\_\_\_\_\_\_\_\_\_\_\_\_\_\_\_\_\_\_\_\_\_\_\_\_\_\_\_\_\_\_\_\_\_\_\_\_\_\_\_\_\_\_\_\_\_\_\_\_\_\_\_\_\_\_\_\_\_\_\_\_\_\_\_\_\_\_\_\_\_\_\_\_\_\_\_\_\_\_\_\_\_\_\_\_\_\_\_\_\_\_\_\_\_\_\_\_\_\_\_\_\_\_\_\_\_\_\_\_\_\_\_\_\_\_\_\_\_\_\_\_\_\_\_\_\_\_\_\_\_\_\_\_\_\_\_\_\_\_\_\_\_\_\_\_\_\_\_\_\_\_\_\_\_\_\_\_\_\_\_\_\_\_\_\_\_\_\_\_\_\_\_\_\_\_\_\_\_\_\_\_\_\_\_\_\_\_\_\_\_\_\_\_\_\_\_\_\_\_\_\_\_\_\_\_\_\_\_\_\_\_\_\_\_\_\_\_\_\_\_\_\_\_\_\_\_\_\_\_\_\_\_\_\_\_\_\_\_\_\_\_\_\_\_\_\_\_\_\_\_\_\_\_\_\_\_\_\_\_\_\_\_\_\_\_\_\_\_\_\_\_\_\_\_\_\_\_\_\_\_\_\_\_\_\_\_\_\_\_\_\_\_\_\_\_\_\_\_\_\_\_\_\_\_\_\_\_\_\_\_\_\_\_\_\_\_\_\_\_\_\_\_\_\_\_\_\_\_\_\_\_\_\_\_\_\_\_\_\_\_\_\_\_\_\_\_\_\_\_\_\_\_\_\_\_\_\_\_\_\_\_\_\_\_\_\_\_\_\_\_\_\_\_\_\_\_\_\_\_\_\_\_\_\_\_\_\_\_\_\_\_\_\_\_\_\_\_\_\_\_\_\_\_\_\_\_\_\_\_\_\_\_\_\_\_\_\_\_\_\_\_\_\_\_\_\_\_\_\_\_\_\_\_\_\_\_\_\_\_\_\_\_\_\_\_\_\_\_\_\_\_\_\_\_\_\_\_\_\_\_\_\_\_\_\_\_\_\_\_\_\_\_\_\_\_\_\_\_\_\_\_\_\_\_\_\_\_\_\_\_\_\_\_\_\_\_\_\_\_\_\_\_\_\_\_\_\_\_\_\_\_\_\_\_\_\_\_\_\_\_\_\_\_\_\_\_\_\_\_\_\_\_\_\_\_\_\_\_\_\_\_\_\_\_\_\_\_\_\_\_\_\_\_\_\_\_\_\_\_\_\_\_\_\_\_\_\_\_\_\_\_\_\_\_\_\_\_\_\_\_\_\_\_\_\_\_\_\_\_\_\_\_\_\_\_\_\_\_\_\_\_\_\_\_\_\_\_\_\_\_\_\_\_\_\_\_\_\_\_\_\_\_\_\_\_\_\_\_\_\_\_\_\_\_\_\_\_\_\_\_\_\_\_\_\_\_\_\_\_\_\_\_\_\_\_\_\_\_\_\_\_\_\_\_\_\_\_\_\_\_\_\_\_\_\_\_\_\_\_\_\_\_\_\_\_\_\_\_\_\_\_\_\_\_\_\_\_\_\_\_\_\_\_\_\_\_\_\_\_\_\_\_\_\_\_\_\_\_\_\_\_\_\_\_\_\_\_\_\_\_\_\_\_\_\_\_\_\_\_\_\_\_\_\_\_\_\_\_\_\_\_\_\_\_\_\_\_\_\_\_\_\_\_\_\_\_\_\_\_\_\_\_\_\_\_\_\_\_\_\_\_\_\_\_\_\_\_\_\_\_\_\_\_\_\_\_\_\_\_\_\_\_\_\_\_\_\_\_\_\_\_\_\_\_\_\_\_\_\_\_\_\_\_\_\_\_\_\_\_\_\_\_\_\_\_\_\_\_\_\_\_\_\_\_\_\_\_\_\_\_\_\_\_\_\_\_\_\_\_\_\_\_\_\_\_\_\_\_\_\_\_\_\_\_\_\_\_\_\_\_\_\_\_\_\_\_\_\_\_\_\_\_\_\_\_\_\_\_\_\_\_\_\_\_\_\_\_\_\_\_\_\_\_\_\_\_\_\_\_\_\_\_\_\_\_\_\_\_\_\_\_\_\_\_\_\_\_\_\_\_\_\_\_\_\_\_\_\_\_\_\_\_\_\_\_\_\_\_\_\_\_\_\_\_\_\_\_\_\_\_\_\_\_\_\_\_\_\_\_\_\_\_\_\_\_\_\_\_\_\_\_\_\_\_\_\_\_\_\_\_\_\_\_\_\_\_\_\_\_\_\_\_\_\_\_\_\_\_\_\_\_\_\_\_\_\_\_\_\_\_\_\_\_\_\_\_\_\_\_\_\_\_\_\_\_\_\_\_\_\_\_\_\_\_\_\_\_\_\_\_\_\_\_\_\_\_\_\_\_\_\_\_\_\_\_\_\_\_\_\_\_\_\_\_\_\_\_\_\_\_\_\_\_\_\_\_\_\_\_\_\_\_\_\_\_\_\_\_\_\_\_\_\_\_\_\_\_\_\_\_\_\_\_\_\_\_\_\_\_\_\_\_\_\_\_\_\_\_\_\_\_\_\_\_\_\_\_\_\_\_\_\_\_\_\_\_\_\_\_\_\_\_\_\_\_\_\_\_\_\_\_\_\_\_\_\_\_\_\_\_\_\_\_\_\_\_\_\_\_\_\_\_\_\_\_\_\_\_\_\_\_\_\_\_\_\_\_\_\_\_\_\_\_\_\_\_\_\_\_\_\_\_\_\_\_\_\_\_\_\_\_\_\_\_\_\_\_\_\_\_\_\_\_\_\_\_\_\_\_\_\_\_\_\_\_\_\_\_\_\_\_\_\_\_\_\_\_\_\_\_\_\_\_\_\_\_\_\_\_\_\_\_\_\_\_\_\_\_\_\_\_\_\_\_\_\_\_\_\_\_\_\_\_\_\_\_\_\_\_\_\_\_\_\_\_\_\_\_\_\_\_\_\_\_\_\_\_\_\_\_\_\_\_\_\_\_\_\_\_\_\_\_\_\_\_\_\_\_\_\_\_\_\_\_\_\_\_\_\_\_\_\_\_\_\_\_\_\_\_\_\_\_\_\_\_\_\_\_\_\_\_\_\_\_\_\_\_\_\_\_\_\_\_\_\_\_\_\_\_\_\_\_\_\_\_\_\_\_\_\_\_\_\_\_\_\_\_\_\_\_\_\_\_\_\_\_\_\_\_\_\_\_\_\_\_\_\_\_\_\_\_\_\_\_\_\_\_\_\_\_\_\_\_\_\_\_\_\_\_\_\_\_\_\_\_\_\_\_\_\_\_\_\_\_\_\_\_\_\_\_\_\_\_\_\_\_\_\_\_\_\_\_\_\_\_\_\_\_\_\_\_\_\_\_\_\_\_\_\_\_\_\_\_\_\_\_\_\_\_\_\_\_\_\_\_\_\_\_\_\_\_\_\_\_\_\_\_\_\_\_\_\_\_\_\_\_\_\_\_\_\_\_\_\_\_\_\_\_\_\_\_\_\_\_\_\_\_\_\_\_\_\_\_\_\_\_\_\_\_\_\_\_\_\_\_\_\_\_\_\_\_\_\_\_\_\_\_\_\_\_\_\_\_\_\_\_\_\_\_\_\_\_\_\_\_\_\_\_\_\_\_\_\_\_\_\_\_\_\_\_\_\_\_\_\_\_\_\_\_\_\_\_\_\_\_\_\_\_\_\_\_\_\_\_\_\_\_\_\_\_\_\_\_\_\_\_\_\_\_\_\_\_\_\_\_\_\_\_\_\_\_\_\_\_\_\_\_\_\_\_\_\_\_\

对于问题三,首先,选取影响我国水稻和小麦价格、供给量及需求量的指标。然后,基于供需理论构建粮食供需及价格联动模型,即水稻和小麦的供给模型、需求模型和价格模型,并运用广义矩估计法通过 \( R \) 软件对模型进行估计。其次,运用所建立的粮食供需及价格联动模型对小麦和水稻的未来几年的价格进行预测,并与运用 ARIMA 模型得到的未来几年的水稻和小麦价格的预测值进行对比,得到粮食供需及价格联动模型的精度和误差率较低。最后,通过建立的模型对我国粮食价格的规律性进行分析可得,粮食供给、价格和需求分别受不同外生变量的影响,同时它们三者之间又是相互影响、相互制约、共同演化的动态关系且粮食收购价格对市场价格有直接影响。

对于问题四,在问题三供需价格联动模型的基础上进行优化,将粮食种植面积最大作为优化模型目标函数,从种植面积、财政支出及库存量三个方面进行约束。分别得到水稻和小麦合理定价模型,应用该模型对国家“十二五”时期公布的粮食最低收购价进行评价。定义误差率做为粮食最低收购价的评价标准,通过 LINGO 软件计算结果可以看出 2011 年-2014 年公布的粮食最低收购价格较为合理。最后应用该模型对 2017 年粮食最低收购价进行预测,得出水稻最低收购价格预测区间为 \([158.65, 175.35]\),小麦最低收购价格预测区间为 \([129.2, 142.8]\)。

对于问题五,在问题四中小麦的合理定价模型的基础上进行优化。在原有模型的基础上,固定其它变量不变,将种植面积提高 \(5\%\),在满足约束条件的基础上,利用 Lingo 软件计算得出调整后的粮食最低收购价为 155 元。并通过 Matlab 软件模拟出小麦种植面积随粮食最低收购价变动曲线。

对于问题六,根据问题一到问题五所得到的结论,我们从影响粮食种植面积因素、粮食最低收购价政策实施、粮食价格规律及粮食最低收购价制定四个方面给出相应的调控粮食种植的优化决策和建议。国家和政府在以后制定关于粮食的相关政策时,要考虑地域及粮食品种差异;有关部门要打造粮食价格信息咨询平台,引导生产者对粮食价格进行合理的预期进而进行合理的种植决策。

关键词:Spearman 偏最小二乘回归 混合线性模型 供需及价格联动模型 线性规划

\section*{目录}

\begin{enumerate}
    \item 前言 \dotfill 5
    \begin{enumerate}
        \item 研究目的与意义 \dotfill 5
        \item 国内外研究动态 \dotfill 5
        \begin{enumerate}
            \item 国内研究动态 \dotfill 5
            \item 国外研究动态 \dotfill 6
        \end{enumerate}
        \item 基本假设 \dotfill 6
        \item 模型的符号说明 \dotfill 6
        \item 本文研究思路 \dotfill 7
        \item 问题分析 \dotfill 7
    \end{enumerate}
    \item 问题一解答——影响粮食种植面积因素分析 \dotfill 9
    \begin{enumerate}
        \item 影响粮食种植面积的因素定性分析 \dotfill 9
        \item 数据来源及说明 \dotfill 10
        \begin{enumerate}
            \item 数据来源 \dotfill 10
            \item 粮食品种及区域划分 \dotfill 13
        \end{enumerate}
        \item 影响水稻种植面积的指标体系模型建立与求解 \dotfill 14
        \begin{enumerate}
            \item 指标选取 \dotfill 14
            \item 模型建立与求解——影响水稻种植面积的指标与水稻种植面积 \dotfill 16
            \item 关于水稻种植面积的模型可信度和可靠性检验与分析 \dotfill 19
        \end{enumerate}
        \item 影响小麦种植面积的因素指标体系模型建立与求解 \dotfill 20
        \begin{enumerate}
            \item 指标选取 \dotfill 20
            \item 模型建立与求解——影响小麦种植面积的指标与小麦种植面积 \dotfill 22
            \item 关于小麦种植面积的模型可信度和可靠性检验与分析 \dotfill 23
        \end{enumerate}
    \end{enumerate}
    \item 问题二解答——粮食最低收购价政策评价模型 \dotfill 24
    \begin{enumerate}
        \item 数据处理 \dotfill 24
        \begin{enumerate}
            \item 水稻数据处理 \dotfill 24
            \item 小麦数据处理 \dotfill 25
        \end{enumerate}
        \item 水稻最低收购价政策评价模型的建立 \dotfill 25
        \begin{enumerate}
            \item 混合线性模型 \dotfill 25
            \item 确定水稻固定效应和随机效应 \dotfill 26
            \item 协方差结构的选择 \dotfill 28
            \item 水稻最低收购价政策评价模型 \dotfill 28
        \end{enumerate}
        \item 水稻最低收购价政策评价模型的求解与检验 \dotfill 28
        \begin{enumerate}
            \item 黑龙江省水稻最低收购价政策评价模型求解与检验 \dotfill 28
            \item 湖南省水稻最低收购价政策评价模型求解与检验 \dotfill 29
            \item 广西壮族自治区水稻最低收购价政策评价模型求解与检验 \dotfill 30
        \end{enumerate}
        \item 各省水稻最低收购价政策效果比较 \dotfill 31
        \item 小麦最低收购价政策评价模型的建立 \dotfill 32
        \begin{enumerate}
            \item 小麦模型固定效应确定 \dotfill 32
            \item 小麦模型随机效应确定 \dotfill 32
            \item 小麦最低收购价政策评价模型 \dotfill 33
        \end{enumerate}
        \item 小麦最低收购价政策评价模型的求解与检验 \dotfill 34
        \begin{enumerate}
            \item 安徽省小麦最低收购价政策评价模型求解与检验 \dotfill 34
        \end{enumerate}
    \end{enumerate}
\end{enumerate}

\begin{itemize}
    \item[3.6.2] 河南省小麦最低收购价政策评价模型求解与检验 \dotfill 35
    \item[3.7] 各省小麦最低收购价政策效果比较 \dotfill 36
    \item[4] 问题三解答——粮食供需及价格联动模型 \dotfill 36
    \begin{itemize}
        \item[4.1] 数据来源及说明 \dotfill 36
        \item[4.2] 粮食供需及价格联动模型建立 \dotfill 38
        \begin{itemize}
            \item[4.2.1] 广义矩估计方法 (GMM) \dotfill 38
            \item[4.2.2] 粮食供需及价格联动模型系统的设计 \dotfill 38
        \end{itemize}
        \item[4.3] 粮食供需平衡价格联动模型求解 \dotfill 39
        \begin{itemize}
            \item[4.3.1] 水稻供需平衡价格联动模型 \dotfill 39
            \item[4.3.2] 小麦供需平衡价格联动模型 \dotfill 40
        \end{itemize}
        \item[4.4] 粮食供需平衡价格联动模型检验 \dotfill 41
        \begin{itemize}
            \item[4.4.1] 水稻供需平衡价格联动模型检验 \dotfill 41
            \item[4.4.2] 小麦供需平衡价格联动模型检验 \dotfill 42
        \end{itemize}
        \item[4.5] 我国粮食价格规律分析 \dotfill 43
    \end{itemize}
    \item[5] 问题四——对粮食最低收购价的评价和预测 \dotfill 44
    \begin{itemize}
        \item[5.1] 粮食最低收购价的合理定价模型的建立 \dotfill 44
        \begin{itemize}
            \item[5.1.1] 水稻最低收购价的合理定价优化模型的建立 \dotfill 44
            \item[5.1.2] 小麦最低收购价的合理定价模型的建立 \dotfill 45
        \end{itemize}
        \item[5.2] “十二五”公布的粮食最低收购价格的合理性评价 \dotfill 46
        \begin{itemize}
            \item[5.2.1] “十二五”公布的水稻最低收购价格的合理性评价 \dotfill 47
            \item[5.2.2] “十二五”公布的小麦最低收购价格的合理性评价 \dotfill 47
        \end{itemize}
        \item[5.3] 对2017年的粮食最低收购价合理范围的预测 \dotfill 47
        \begin{itemize}
            \item[5.3.1] 对2017年的水稻最低收购价合理范围的预测 \dotfill 47
            \item[5.3.2] 对2017年的小麦最低收购价合理范围的预测 \dotfill 48
        \end{itemize}
    \end{itemize}
    \item[6] 问题五——小麦最低收购价合理定价模型的应用 \dotfill 48
    \begin{itemize}
        \item[6.1] 调整后的小麦最低收购价格的合理定价模型建立 \dotfill 48
        \item[6.2] 调整后的小麦最低收购价格的合理定价模型求解 \dotfill 48
    \end{itemize}
    \item[7] 问题六——调控粮食种植的优化决策和建议 \dotfill 49
    \begin{itemize}
        \item[7.1] 从影响粮食种植面积因素方面 \dotfill 49
        \item[7.2] 从粮食最低收购价政策实施方面 \dotfill 49
        \item[7.3] 从粮食价格规律方面 \dotfill 49
        \item[7.4] 从粮食最低收购价制定方面 \dotfill 50
    \end{itemize}
    \item[8] 模型评价与推广 \dotfill 50
    \begin{itemize}
        \item[8.1] 模型的优点 \dotfill 50
        \item[8.2] 模型的缺点 \dotfill 50
        \item[8.3] 模型的推广 \dotfill 50
    \end{itemize}
    \item 参考文献 \dotfill 50
    \item 附录 \dotfill 52
\end{itemize}

\section{前言}

\subsection{研究目的与意义}

粮食是人类最基本的生活消费品,粮食问题是关系到国家的国计民生的头等问题。众所周知,农业是国民经济发展的基础,粮食是基础的基础,因此粮食生产是关系到一个国家发展与生产的一个关键的主题。由于耕地减少、人口增加、水资源短缺、气候变化等问题日益凸显,加之国际粮食市场的冲击,我国粮食产业面临着潜在的风险。因此,研究我国的粮食保护政策具有十分重要的作用和意义。

由于从 2004 年开始,我国粮食生产获得连年丰收,粮食价格面临着较大的下行压力。为了调控粮食市场价格,继续稳定粮食生产,避免重蹈谷贱伤农的老路。我国自 2005 年起开始对粮食主产区实行了最低收购价政策。粮食最低收购价政策,是为保护农民利益、保障粮食市场供应实施的粮食价格调控政策。最低收购价政策不仅可以促进农民种粮的积极性,稳定粮食生产,促进农民收入的提高还可以有效地抵御国际粮价的大幅上涨对我国粮食市场的冲击。但是,部分实行粮食最低收购价的地区,粮食品种种植面积、粮食总产量不增反降,导致部分学者质疑粮食最低收购价政策的效果。

因此,在当前形势下,我们研究粮食最低收购价政策,分析影响粮食种植面积的指标,建立这些指标和粮食最低收购价的数学模型,并给出调控粮食种植的优化决策和建议,具有非常重要的现实意义。本文力求通过定性与定量的研究,科学地给出我国粮食最低收购价政策的有效建议,对指导我国粮食产业的发展具有非常重要的现实意义。

\subsection{国内外研究动态}

\subsubsection{国内研究动态}

粮食最低收购价政策近年来得到国内外学者的广泛关注,其中大多数学者致力于研究粮食生产的有关政策,而专门对粮食价格收购政策的研究并不是很多,部分学者将粮食收购政策放在农业保护政策研究中。我国自 2005 年开始执行粮食最低收购价政策,少数学者对我国的粮食最低收购价政策进行研究,但研究也不是很深入,且对相关问题争议较大。

改革开放后,国内开始出现对农业和粮食保护问题的研究。随着我国市场经济体制逐步建立以及粮食流通体制改革的深入,我国经济的迅速发展,粮食安全面临新形势、新情况,粮食政策研究进入了一个新的发展阶段。我国粮食最低收购价格从 2004 年开始实施,相较于粮食最低收购价,临时收储政策操作起来弹性较大,与粮食最低收购价政策构成了我国现行的粮食价格支持政策。由于粮食政策研究的特殊性,粮食政策研究并没有形成比较系统、规范的方法。如李成贵综合运用社会科学知识,将经济因素和非经济因素纳入科学范围,系统地阐述了农业政策的规范性认识及其政治经济学意义。国内一些学者对价格支持政策的经济学研究基于福利经济学的视角,任军军,王文举等 [1](2010)利用供给需求模型对我国粮食最低收购价启动后对市场供、求曲线产生的影响进行实证研究,验证了最低收购价存在托底粮价和提高市场均衡产量的作用。史长亮等通过研究中国粮食供给和需求状况,对粮食价格走势进行预测分析。随着粮食最低收购价政策的不断加深,国内学者开始运用经济学工具分析政策实施效应。部分学者用粮食生产函数以及粮食生产成本曲线,动态分析粮食收购政策的产量效应。如韩喜平(2007)等,利用福利经济学原理,借用供给—需求曲线分析粮食最低收购价政策对粮食生产者、消费者等各种主体的影响。

此外,关于中国粮食政策的研究大部分是定性分析,对政策效果的正面评价较多,侧重于如何更好地提高政策效果。高峰(2008)、程秀梅(2011)等 [16] 主要侧重于对

粮食农业保护的研究。王士海(2011)[17]分别考察了最低收购价政策对市场价格的托市效应、政策性粮食拍卖对粮食价格的平抑效应、粮食贸易政策对粮食调控的作用。王冰(2001)分析讨论了有关的制度在粮食生产和流通中的作用。叶兴庆(2002)提出,保护价收购存在系列问题,应以直接补贴农民的方式替代保护价收购。

\subsection*{1.2.2 国外研究动态}

国内外在农业保护和粮食价格支持政策方面的研究成果,为我们研究我国粮食最低收购政策,特别是现行的粮食最低收购价提供了有益的借鉴,给我们以启示。但是,我国是人口大国、粮食生产和消费大国,国情复杂,粮情特殊。因此,在分析研究国外的农业保护和粮食价格支持政策,应结合我国的国情、粮情实际,借鉴有益的做法和经验。

国外包含粮食收购政策在内的农产品价格支持政策发展大致可分为三个阶段,在农产品价格支持政策之前,国外大多数学者将粮食收购政策放在农业支持政策体系下进行研究。萌芽阶段,1993年美国首次出台《农业调整法》,对农业支持系列政策作了相关规定,标志着农业支持政策的萌芽。形成阶段,1962年欧洲共同体出台“共同农业政策”,之后越多学者开始研究关于农产品保护政策。Jayne应用简化的向量自回归模型研究了政府对采购和销售玉米价格的决定。AP Barkley使用多变量协整检验方法对印尼市场稻米的每周数据进行评估,结果发现政府的干预对农产品市场产生了一定积极的影响。Mckenzie AM等通过应用标准的 OLS、协整和误差修正模型用于确定无偏性。发展阶段,随着经济社会发展,世界经济进入经济自由化阶段,不少学者认为保护政策扭曲了农产品正常贸易,同时认为要取消农业保护政策,另一部分学者认为农业保护政策不能取消,因为农业保护政策对保护生态环境、保障粮食安全存在着重要的意义。农业补贴政策主要分为两大类:第一类即我国现行的粮食最低收购价制度和粮油临时收储措施,即价格支持政策。第二类是与价格和产量脱钩的直接补贴政策。国外关于粮食最低收购价的研究主要集中在最低收购价的经济学原理、定价机制,以及与其它补贴政策的比较分析等方面。其中最早的粮食收购价的经济学思想来源于20世纪30年代Schultz等提出的“蛛网模型”。蛛网理论的主要思想是理性经济人会根据上一生产周期的市场均衡价格预期当期市场价格,即 $p_t = p_{t-1}$。随着适应性预期和理性预期理论的兴起,预期理论被广泛应用于研究粮食最低收购价问题。

\subsection*{1.3 基本假设}

1. 假设未来10年内技术进步不会对粮食产量产生巨大影响。

2. 假设未来10年内不会发生战争或其它社会动荡。

3. 假设未来10年内粮食补贴政策能及时落实。

4. 假设在各大网站查到的数据都为真实数据。

5. 暂时不考虑本文选取因素之外的因素对问题产生的影响。

\subsection*{1.4 模型的符号说明}

\begin{tabular}{c c c c}
\hline 符号 & 说明 & 符号 & 说明 \\
\hline $r_s$ & 斯皮尔曼相关系数 & $x_{11}$ & 城市化水平 \\
$x_1$ & 农业劳动力人口 & $z_5$ & 小麦市场价格 \\
$x_2$ & 农民人均受教育程度 & $z_6$ & 小麦生产成本 \\
$x_3$ & 城乡收入差距 & $z_7$ & 小麦相对收益竞争力 \\
$x_4$ & 农民家庭负担 & $z_{10}$ & 小麦净出口量 \\
$x_5$ & 水稻的市场价格 & $y_1$ & 黑龙江省水稻面积 \\
$x_6$ & 水稻生产成本 & $y_2$ & 湖南省水稻种植面积 \\
\hline
\end{tabular}

\begin{table}
\centering
\begin{tabular}{c c c c}
\hline
$x_{7}$ & 水稻相对收益竞争力 & $y_{3}$ & 广西水稻种植面积 \\
$x_{8}$ & 农作物播种面积 & $y_{4}$ & 河南省小麦种植面积 \\
$x_{9}$ & 受灾情况 & $y_{5}$ & 安徽省小麦种植面积 \\
$x_{10}$ & 水稻净出口量 & $s$ & 粮食最低收购价政策 \\
\hline
\end{tabular}
\end{table}

\subsection{1.5 本文研究思路}

\begin{figure}[h]
\centering
\includegraphics[width=0.8\textwidth]{research_flowchart.png}
\end{figure}

\subsection{1.6 问题分析}

\subsubsection{1. 问题一分析}

对于问题一,要求建立影响粮食种植面积的指标体系和关于粮食种植面积的数学模型,讨论指标体系的合理性及他们之间的关系,并对得出的相应结果的可信度和可靠性给出检验和分析。

\begin{enumerate}
    \item 第一步,根据相关研究选取影响粮食种植面积的候选因素,并对这些因素与粮食种植面积的关系进行定性分析。
    \item 第二步,由于这些指标与粮食种植面积间存在着粮食品种和区域差异,因此对粮食品种与区域进行划分。小麦和水稻是我国主要的粮食作物,因此选取小麦和水稻为研究的粮食品种。根据其主产区的地域性特点进行区域划分,并从划分的区域中各选出一个代表性的省(区)分别进行研究。
    \item 第三步,对于水稻的代表省份和小麦的代表省份的候选因素与其种植面积用 Spearman 相关检验的方法分别进行相关性检验,最终选取小麦和水稻代表省份中与种植面积相关的因素。
    \item 第四步,以第三步选取的水稻和小麦个各代表省份的指标为自变量,各省份的水稻和小麦种植面积为因变量,运用偏最小二乘回归法建立影响水稻和小麦种植面积的指标体系与种植面积的模型,并对模型的拟合效果与显著性进行检验。
\end{enumerate}

\section*{2. 问题二分析}

对于问题二,要求建立粮食最低收购价政策执行效果的评价模型,并运用所建立的评价模型,结合粮食品种和区域差异,选择几个省份比较研究粮食主产区粮食最低收购价执行的效果。

第一步,由问题一对小麦和水稻各代表省份选取的影响小麦和水稻种植面积的指标,通过主成分分析的方法,对其进行降维处理,并选取合适的主成分的个数。

第二步,以第一步得到的小麦和水稻的主成分为随机效应,小麦和水稻的最低收购价政策为固定效应,水稻和小麦的种植面积为因变量分别建立混合线性模型。所建模型即为粮食最低收购价政策执行效果的评价模型。

第三步,通过对模型进行检验并求解,计算各省份水稻和小麦的影响效果,通过影响效果比较研究几个代表省份粮食最低收购价执行的效果。

\section*{3. 问题三分析}

对于问题三,要求运用数据分析或建立数学模型探讨我国粮食价格所具有的特殊规律性。

第一步,选取影响我国水稻小麦价格、供给量及需求量的指标。

第二步,基于供需理论构建粮食供需及价格联动模型。即对水稻和小麦的供给模型、需求模型和价格模型,并运用广义矩估计法对该模型进行估计。

第三步,运用所建立的粮食供需及价格联动模型对小麦和水稻的未来几年的价格进行预测,并与运用 ARIMA 模型得到的未来几年的水稻和小麦价格的预测值进行对比,得到粮食供需及价格系统的精度和误差率,从而检验模型的正确性。

\section*{4. 问题四分析}

对于问题四,要求建立粮食最低收购价的合理定价模型,进而对“十二五”期间国家发展与改革委员会公布的粮食最低收购价价格的合理性做出评价,并运用所建立的模型对 2017 年的粮食最低收购价的合理范围进行预测。

第一步,在问题三供需价格联动模型的基础上进行优化,将粮食种植面积最大作为优化模型目标函数,从种植面积、财政支出及库存量三个方面进行约束。分别得到水稻和小麦合理定价模型。

第二步,应用所建立的水稻和小麦合理定价模型对国家“十二五”时期公布的粮食最低收购价进行评价。定义误差率做为粮食最低收购价的评价标准,通过 LINGO 软件计算结果判断公布的粮食最低收购价格的合理性。

第三步,应用所建立的模型对 2017 年粮食最低收购价进行预测,得出水稻和小麦最低收购价格预测区间。

\section*{5. 问题五分析}

对于问题五,要求通过调整粮食最低收购价使我国小麦种植面积增加 5\%。

第一步,在问题四中小麦的合理定价模型的基础上进行优化。在原有模型的基础上,固定其它变量不变,将种植面积提高 5\%。

第二步,在满足约束条件的基础上,利用 Lingo 软件计算得出调整后的粮食最低收购价。

第三步,通过 Matlab 软件模拟出小麦种植面积随粮食最低收购价变动曲线。

\section*{6. 问题六分析}

对于问题六,要求根据研究结论,提出调控粮食种植的优化决策和建议。

本问题我们根据问题一到问题五所得到的结论,从影响粮食种植面积因素、粮食最

低收购价政策实施、粮食价格规律及粮食最低收购价制定四个方面给出相应的调控粮食种植的优化决策和建议。

\section*{2. 问题一解答——影响粮食种植面积因素分析}

一般来讲,粮食的种植面积是决定粮食供给的关键因素,也是保障粮食安全的重要前提。衡量粮食最低收购价政策实施的效果,主要是比较政策实施前后粮食种植面积是否有显著性变化。可能影响粮食种植面积的因素有很多,根据国内外相关研究,我们选取了除粮食最低收购价政策外的 11 个可能对粮食种植面积有影响的因素。分别为农业劳动力人口、农民人均受教育程度、城乡收入差距、农民家庭负担、粮食的市场价格、粮食生产成本、粮食相对收益竞争力、农作物播种面积、受灾情况、我国粮食净出口量、城市化水平。本题主要研究这 11 个因素对粮食种植面积的影响。

\subsection*{2.1 影响粮食种植面积的因素定性分析}

1、农业劳动力人口对粮食种植面积的影响

农业劳动力人口,是指社会中符合劳动年龄并有劳动能力的人的数量,和不到劳动年龄或已超过劳动年龄但实际参加劳动的人的数量。农业劳动在农业中具有重要的作用,表现在农业劳动具有能动性,即它是在农业生产力各要素中,唯一具有活力和发展最快的。而且农业劳动力的不断提高,使得农业和国民经济得到迅速的发展。因此,农业劳动力人口对农产品种植产生积极的影响,从而对粮食种植面积也会产生积极影响。

2、农民人均受教育程度对粮食种植面积的影响

农民所接受的教育和掌握的技术程度是他们人力资本的主要构成。尽管较多农民认为种庄稼不需要什么教育和技术,但是作为一种特殊的资本形式,个人受教育程度和技术水平会影响到他们的经济活动效率,而且也关系到乡村居民的个人发展和乡村社会的发展问题。农民受教育程度的高低,会影响农业生产效率、农业现代实现。因此,农民人均受教育程度提高,会提高农产品的产量,增加农民的收入,从而也会调动农民种植的积极性,也会对粮食种植面积产生积极影响。

3、城乡收入差距对粮食种植面积的影响

城乡收入差距是城镇居民人均可支配收入与农村居民人均纯收入的比值。改革开放以来,中国经济社会发展取得了举世瞩目的辉煌成就,综合国力明显增强,但城乡差距居民收入差距也快速扩大。城乡收入差距状况对其流动方式会产生影响。农民所感受到的收入差距越大,则外出打工动机越强;相反,农民相对剥夺感越低则更愿意留在本地。因此,城乡收入差距的大小,影响农民的流动方式及种植的意愿,从而就会使粮食种植面积受到影响。

4、农民家庭负担对粮食种植面积的影响

农民家庭负担指农村居民家庭生活消费支出,即农村常住居民家庭用于日常生活的全部开支,是反映和研究农民家庭实际生活消费水平高低的重要指标。随着我国经济的持续快速增长,农村经济也得到了迅速的发展。国家大力扶持农村各个方面的建设,使得农村居民收入水平持续增加,农村的消费水平也有所提高。农民消费水平高,说明农民的收入高,农民收入的高低影响着农民的种植积极性。因此,从农民的消费水平的高低,影响农民的种植积极性,从而对种植面积产生间接的影响。

5、粮食的市场价格对粮食种植面积的影响

粮食价格的高低,决定了棉农利润的高低。粮食的市场价格越高,农民所得利润就高,从而会调动农民种植粮食的积极性,扩大粮食的种植面积。相反,粮食的市场价格低,农民会减少粮食的种植面积。

\section*{6、粮食生产成本对粮食种植面积的影响}

粮食生产成本包括物质与服务费用及人工成本,生产成本与市场价格一样都直接影响了农民的利润。粮食的生产成本越高,农民所得利润就越低,从而会影响农民种植粮食的积极性,减少粮食的种植面积。相反,粮食生产成本低,农民会增加粮食的种植面积。

\section*{7、粮食相对收益竞争力对粮食种植面积的影响}

粮食相对收益竞争力是指水稻每亩净收益和小麦每亩净收益分别与其竞争农作物每亩净收益的差值。粮食相对收益竞争力反映了粮食与其竞争农作物在收益上的差距。粮食收益比其竞争农作物的收益高,则农民会增加粮食的种植面积,反之则会降低粮食种植面积。

\section*{8、农作物总播种面积与粮食种植面积的影响}

农作物播种面积,指实际播种或移植有农作物的面积。凡是实际种植有农作物的面积,不论种植在耕地上还是种植在非耕地上,均包括在农作物播种面积中。随着改革开放步伐的加快、加深,我国城市化迅速发展起来,乡镇企业大幅增加,新的企业、厂房、公司及住房等建筑正如雨后春笋般地林立起来。新的建筑的建造需要新的地基,因为城市化而占用的土地大多是农村良好的耕地。因此,农作物总播种面积增大,粮食种植面积也会随之增大。

\section*{9、受灾情况与粮食种植面积的影响}

农作物受灾情况,用农作物受灾面积与成灾面积的比值来表示。受灾面积是指因遭受旱灾、水灾、风雹灾、霜冻、病虫害、及其它自然灾害使农作物较正常年景产量减产一成以上的农作物播种面积。成灾面积是指在遭受上述自然灾害的受灾面积中,农作物实际收获量较常年产量减少 3 成以上的播种面积。受灾情况,反映农民的损失情况,受灾情况严重会直接影响农民的收益,从而会影响农民种植粮食的积极性,进而对粮食种植面积产生影响。

\section*{10、我国粮食净出口量与粮食种植面积的影响}

粮食净出口量指粮食出口量与进口量的差值。大量的粮食进口虽然可以满足国内对粮食的需求但同时也对国内的粮食产业受到冲击,会严重影响农民的利益。因此,进口量越多,导致粮食的种植面积越少,即粮食净出口值减小,粮食种植面积也会相应减少。

\section*{11、城市化水平与粮食种植面积的影响}

城市化水平又叫城市化率,是衡量城市化发展程度的数量指标,一般用一定地域内城市人口占总人口比例来表示。随着我国经济的发展我国的城市化进程越来越快,一方面吸引了更多的农村劳动力进城打工,减少了粮食生产的劳动力投入,不利于维持粮食的产出;另一方面城市化进程的加快,又有利于资本的运作,可以进一步提高资本的效率,从而对技术的产出有着积极的影响,进而对粮食的生产效率产生积极影响。因此,城市化水平对粮食产量产生影响,从而间接影响粮食的种植面积。

\subsection*{2.2 数据来源及说明}

\subsubsection{2.2.1 数据来源}

\begin{table}
\centering
\caption{湖南省影响水稻种植面积因素}
\begin{tabular}{c c c c c c c}
\hline
年份 & 农业劳动力人口 & 农民受教育程度 & 城乡收入差距 & 农民家庭负担 & 市场收购价格 & 生产成本 \\
 & (万人) & (年) & (无) & (元) & (元/50公斤) & (元/亩) \\
\hline
1995 & 2071.61 & 6.78 & 3.30 & 1367.3 & 61.82 & 324.13 \\
1996 & 1994.9 & 6.94 & 2.82 & 1736.71 & 61.1 & 352.2 \\
1997 & 1998.59 & 7.22 & 2.56 & 1815.79 & 60.55 & 371.14 \\
1998 & 2002.51 & 7.29 & 2.63 & 1889.17 & 52.47 & 334.71 \\
\hline
2010 & 1690.03 & 8.23 & 3.04 & 4310.4 & 97.73 & 549.46 \\
2011 & 1679.94 & 8.81 & 2.92 & 5179.4 & 117.88 & 643.8 \\
2012 & 1668.99 & 8.72 & 2.90 & 5870.1 & 129.32 & 745.8 \\
2013 & 1656.01 & 8.96 & 2.80 & 6609 & 125.63 & 819.86 \\
2014 & 1651.37 & 9.02 & 2.64 & 6780.8 & 131.54 & 820.4 \\
\hline
\end{tabular}
\end{table}

\begin{table}
\centering
\begin{tabular}{c c c c c c}
\hline
年份 & 水稻相对收益竞争力 (无) & 农作物播种面积 (千公顷) & 受灾情况 (无) & 全国水稻净出口量 (千吨) & 城市化水平 (无) & 水稻的种植面积 (千公顷) \\
\hline
1995 & -120.26 & 7840.4 & 2.3 & -3744 & 0.24 & 4084 \\
1996 & 7.1 & 7927.4 & 4.41 & -958 & 0.25 & 4064.14 \\
1997 & 4.37 & 8009 & 1.87 & 445 & 0.25 & 4075.8 \\
1998 & 4.82 & 7936.30 & 3.60 & 5554 & 0.26 & 0 \\
\hline
2011 & 6.66 & 8402 & 2.53 & -167.56 & 0.45 & 4066.3 \\
2012 & 161.78 & 8512 & 2.31 & -1545.11 & 0.47 & 4095.12 \\
2013 & 318.13 & 8650 & 1.83 & -1786.03 & 0.48 & 4085.03 \\
2014 & 760.35 & 8764.5 & 1.64 & -2063 & 0.49 & 4120.71 \\
\hline
\end{tabular}
\end{table}

数据来源:中华人民共和国国家统计局、中国统计年鉴、中国农村统计年鉴、各省统计年鉴、布瑞克农产品数据库、全国成本调查网

以下数据为河南省1995-2014时间段内影响小麦种植面积的12个因素的数据。其中,农业劳动力人口为第一产业就业人口数、农民受教育程度为农民人均受教育年限、城乡收入差距是城镇居民人均可支配收入与农村居民人均纯收入的比值、农民家庭负担指农村居民家庭生活消费支出、市场收购价为本省小麦市场价格、小麦生产成本为生产小麦产生的物质与服务费用及人工成本、小麦相对收益竞争力是指小麦每亩净收益与其竞争农作物每亩净收益的差值、农作物种植面积为本省实际种植有农作物的面积、农作物受灾情况为农作物受灾面积与成灾面积的比值、全国小麦净出口量指全国小麦出口量与进口量的差值、城市化水平为城镇人口占总人口比重、小麦种植面积为本省实际种植小麦的面积。

\begin{table}
\centering
\begin{tabular}{c c c c c c c}
\hline
年份 & 农业劳动力人口 (万人) & 农民受教育程度 (年) & 城乡收入差距 (无) & 农民家庭负担 (元) & 市场收购价 (元/吨) & 生产成本 (元/亩) \\
\hline
1995 & 2814 & 7.17 & 2.68 & 929.39 & 73.5 & 153.7 \\
1996 & 2822 & 7.19 & 2.38 & 1206.43 & 70.23 & 211.4 \\
1997 & 2902 & 7.2 & 2.36 & 1270.52 & 65.58 & 293.71 \\
1998 & 2947 & 7.23 & 2.26 & 1240.3 & 63.04 & 260.39 \\
\hline
2011 & 2670 & 8.27 & 2.76 & 4319.95 & 101.07 & 542.73 \\
2012 & 2628 & 8.31 & 2.72 & 5032.14 & 99.23 & 657.23 \\
2013 & 2563 & 8.45 & 2.64 & 5627.73 & 110.44 & 746.53 \\
2014 & 2652 & 8.48 & 2.59 & 6438.12 & 115.82 & 759.43 \\
\hline
\end{tabular}
\caption{表2-2 影响小麦种植面积因素}
\end{table}

\begin{table}
\centering
\caption{影响小麦种植面积因素(续)}
\begin{tabular}{c c c c c c c}
\hline
年份 & 小麦的相对收益竞争力 (元/亩) & 农作物播种面积 (千公顷) & 受灾情况 (无) & 小麦的净出口 (千吨) & 城市化水平 (无) & 小麦的种植面积 (千公顷) \\
\hline
1995 & -7.88 & 12136.8 & 2.45 & -12667.42 & 0.17 & 4814.03 \\
1996 & 2.8 & 12257.4 & 3.47 & -9230.52 & 0.18 & 4868.2 \\
1997 & -14.69 & 12276.7 & 1.53 & -2844.24 & 0.2 & 4927.3 \\
1998 & -24.12 & 12567.1 & 4.41 & -2516.88 & 0.21 & 4964 \\
\hline
2011 & 3.72 & 14258.61 & 4.83 & -2572.48 & 0.41 & 5323.33 \\
2012 & 1.26 & 14262.17 & 6.15 & -5057.51 & 0.42 & 5340 \\
2013 & 3.24 & 14323.54 & 3.67 & -6828.12 & 0.44 & 5366.66 \\
2014 & 8.26 & 14378.34 & 6.15 & -5618.12 & 0.45 & 5406.67 \\
\hline
\end{tabular}
\end{table}

数据来源:中华人民共和国国家统计局、中国统计年鉴、各省统计年鉴、中国农村统计年鉴、布瑞克农产品数据库、全国成本调查网、中国农村统计年鉴

\subsubsection{2.2.2 粮食品种及区域划分}

由于影响粮食种植面积的因素比较多,它们之间的关系错综复杂,而且存在着粮食品种和区域差异粮食品种和区域差异。因此我们对粮食品种及其区域进行划分,分别对其进行研究。

\subsubsection{1. 水稻主产区区域划分}

水稻种植区域化特征明显,不同区域品种和产量都差异较大。我国水稻种植主要集中于东北平原、长江流域和东南沿海三大区域,分别占全国水稻种植面积的 12\%、64\% 和 22\%。东北平原水稻种植区域一般全年只种一季水稻,生长周期长,水肥条件好,单产水平高出全国 10\%。长江流域种植面积最大,以籼稻为主,可以种植单季稻(中稻)或者双季稻(早晚稻),北部区域有少量粳稻,平均单产比全国高出 4\%;东南沿海种植的基本都是籼稻,可种植一到三季,由于生长周期较短,平均单产低于全国水平 9\%。因此对水稻 11 个主产区根据其区域特征进行划分,结果如下:

\begin{table}
\centering
\caption{水稻区域划分}
\begin{tabular}{c c c c}
\hline
范围 & 东北平原区 & 长江流域区 & 东南沿海区 \\
\hline
地区 & 黑龙江、吉林、辽宁 & 四川、湖南、湖北、江西、安徽、江苏、河南 & 广西 \\
\hline
\end{tabular}
\end{table}

\subsubsection{2. 小麦主产区区域划分}

中国小麦遍及全国,根据种植季节划分,可分为春小麦区、北方冬小麦区、南方冬小麦区。由于小麦的主产区都在北方冬小麦区和南方冬小麦区,因此我们对小麦的主产区按这两个区域进行划分。北方冬小麦区主要分布在秦岭、淮河以北,这里冬小麦约占全国小麦总量的 56\% 左右。南方冬小麦区主要分布在秦岭、淮河以南,这里是我国水稻主产区,种植冬小麦有利于提高复种指数,增加粮食产量。根据以上区域划分,小麦主产区划分结果如下:

\begin{table}
\centering
\begin{tabular}{c c c}
\hline
范围 & 北方冬小麦区 & 南方冬小麦区 \\
\hline
地区 & 河南、河北、山东 & 安徽、江苏、湖北 \\
\hline
\end{tabular}
\caption{小麦区域划分}
\end{table}

以上是对小麦和水稻两个粮食品种在区域上的划分,由于小麦和水稻的主产区较多,因此为了研究方便,可从划分的区域中选取具有代表性的省(区)进行分析。

\subsection{影响水稻种植面积的指标体系模型建立与求解}

根据 2.2.2 中粮食品种区域划分,水稻的 11 个主产省(区)被分为东北平原、长江流域和东南沿海三大区域。为了研究方便,分别从这三个区域中选取最具代表性的三个省(区)对影响水稻种植面积的因素进行分析。在东北平原地区,由于黑龙江省的水稻种植面积是最大的,约占东北平原地区水稻种植总面积的 60\% 以上,故选取黑龙江省作为代表省。在长江流域地区,由于湖南省近几年的种植面积是最大的,约占长江流域地区水稻种植总面积的 26\% 以上,故选取湖南省为代表省。在东南沿海区域,水稻主产区只有广西壮族自治区,故选取广西壮族自治区为代表省(区)。接下来,用选取的黑龙江省、湖南省、广西壮族自治区三个省(区)的数据研究影响水稻种植面积的因素。

\subsubsection{指标选取}

对 2.1 中通过定性分析选取的影响水稻种植面积的 11 个指标与水稻种植面积进行相关性检验,选取与水稻种植面积相关的指标,并分析影响水稻种植面积的指标与水稻种植面积的趋同性。

\subsubsection{非参数 Spearman 相关检验}

非参数统计是统计学的一个重要分支,它在实践中有着广泛的应用。在实际问题中,人们往往对总体的分布知之甚少,以致于在推断之前不能对总体作任何假设,或仅能作一些非常一般性(例如连续分布、对称分布等)的假设。这时,如果仍然使用参数统计方法其统计推断的结果显然是不可信的,甚至有可能是错的。在对总体的分布不作假设或仅作非常一般性假设条件下的统计方法称为非参数统计方法。对于非参数统计方法的检验问题称为非参数检验法。

斯皮尔曼(Spearman)相关系数是描述两组变量之间是否存在着相同或相反趋同性的一种指标,由于该检验不需要假定服从正态分布,仅需要确定变量在每个点(时期)上的等级即可获得,因此具有较好的性质。它和相关系数 \( r \) 一样,取值在 \(-1\) 到 \(+1\) 之间,所不同的是它是建立在等级的基础上计算的。斯皮尔曼(Spearman)相关系数的公式为:
\[
r_s = \left[ \frac{1}{n} \sum_{i=1}^n r_i R_i - \left( \frac{n+1}{2} \right)^2 \right] / \left( \frac{n^2-1}{12} \right)
\]
其中,\( r_i \) 为由某个二元总体 \((X_i, Y_i)\) 的样本 \(X_1 \cdots X_n\) 产生的秩统计量,\( R_i \) 为由 \(Y_1 \cdots Y_n\) 产生的秩统计量。

当以样本的数据来推测总体时,由于样本带有随机性,在小样本时数据间有相关,但总体之间不一定相关。因此有必要进行假设检验。在此,设定原假设 \( H_0 \) 研究的总体之间无相关,备责假设 \( H_1 \) 为研究的总体之间有相关。用 \( r_s \) 的分布来检验 \( X \) 与 \( Y \) 是否独立。当 \( n \) 较大时 \(\sqrt{n-1} r_s\) 的近似分布为 \( N(0,1) \),由此构造拒绝域和计算相应的 \( P \) 值,当 \( P \) 值小于某一显著水平 \( \alpha \)(本文规定 \( \alpha = 0.1 \))时,则拒绝原假设。本题用此方法研究影响水稻种植面积的指标与水稻种植面积之间的关系,是非常适合的。

\begin{table}
\centering
\caption{黑龙江省指标间相关性检验结果}
\begin{tabular}{l c c l}
\hline
变量 & $r_s$ & $t$检验的$p$值 & 趋同性说明 \\
\hline
农业劳动力人口与水稻种植面积 & 0.88120 & $<$0.0001 & 正向趋同 \\
农民受教育程度与水稻种植面积 & 0.89925 & $<$0.0001 & 正向趋同 \\
城乡收入差距与与水稻种植面积 & 0.07368 & 0.7575 & 没有关联 \\
农民家庭负担与水稻种植面积 & 0.87218 & $<$0.0001 & 正向趋同 \\
水稻的市场价格与水稻种植面积 & 0.75668 & 0.0001 & 正向趋同 \\
水稻的生产成本与水稻种植面积 & 0.76241 & $<$0.0001 & 正向趋同 \\
水稻相对收益竞争力与水稻种植面积 & 0.44812 & 0.0475 & 正向趋同 \\
农作物播种面积与水稻种植面积 & 0.93684 & $<$0.0001 & 正向趋同 \\
受灾情况与水稻种植面积 & -0.20301 & 0.3907 & 没有关联 \\
水稻净出口与水稻种植面积 & -0.34436 & 0.1371 & 没有关联 \\
城市化水平与水稻种植面积 & 0.73383 & 0.0002 & 正向趋同 \\
\hline
\end{tabular}
\end{table}

\begin{table}
\centering
\caption{湖南省指标间相关性检验结果}
\begin{tabular}{l c c l}
\hline
变量 & $r_s$ & $t$检验的$p$值 & 趋同性说明 \\
\hline
农业劳动力人口与水稻种植面积 & -0.52331 & 0.0179 & 负向趋同 \\
农民受教育程度与水稻种植面积 & 0.28271 & 0.2272 & 没有关联 \\
城乡收入差距与与水稻种植面积 & -0.47820 & 0.0330 & 负向关联 \\
农民家庭负担与水稻种植面积 & 0.29925 & 0.1999 & 没有关联 \\
水稻的市场价格与水稻种植面积 & 0.66015 & 0.0015 & 正向关联 \\
水稻的生产成本与水稻种植面积 & 0.50376 & 0.0235 & 正向关联 \\
水稻相对收益竞争力与水稻种植面积 & 0.48421 & 0.0305 & 正向关联 \\
农作物播种面积与水稻种植面积 & 0.44361 & 0.0501 & 正向关联 \\
受灾情况与水稻种植面积 & 0.29624 & 0.2047 & 没有关联 \\
水稻净出口与水稻种植面积 & -0.71278 & 0.0004 & 负向关联 \\
城市化水平与水稻种植面积 & 0.29925 & 0.1999 & 没有关联 \\
\hline
\end{tabular}
\end{table}

由以上斯皮尔曼检验的结果可以看出,湖南省农民受教育程度、家庭负担、受灾情况和城市化水平与水稻种植面积 $t$ 检验的 $P$ 值均大于显著性水平 0.1,故它们之间没有关联。而水稻的市场价格、水稻的生产成本、水稻的相对收益竞争力和农作物播种面与水稻种植面积 $t$ 检验的 $P$ 值均小于于显著性水平 0.1,故们之间存在存在相关关系。且从 $r_{s}$ 的值可以看出,它们之间呈正向趋同关系。农业劳动力人口、城乡收入差距和水稻的净出口与水稻种植面积,故们之间存在存在相关关系。且其 $r_{s}$ 的值为负,故它们之间呈负向趋同。

通过斯皮尔曼检验,我们剔除掉农民受教育程度、家庭负担、受灾情况和城市化水平四个无关联的指标。剩下的 7 个指标分别是农业劳动力人口、城乡收入差距、水稻的市场价格、水稻的生产成本、水稻的相对收益竞争力、农作物播种面积和水稻的净出口。这 7 个指标将作为影响湖南省水稻种植面积的主要因素或指标。

### 3. 相关性检验结果——— 广西省

表 2-7 广西省指标间相关性检验结果

\begin{tabular}{l c c c}
\hline
变量 & $r_{s}$ & $t$ 检验的 $p$ 值 & 趋同性说明 \\
\hline
农业劳动力人口与水稻种植面积 & 0.80843 & $<.0001$ & 正向趋同 \\
农民受教育程度与水稻种植面积 & -0.85897 & $<.0001$ & 负向趋同 \\
城乡收入差距与与水稻种植面积 & -0.44152 & 0.0513 & 负向趋同 \\
农民家庭负担与水稻种植面积 & -0.91463 & $<.0001$ & 负向趋同 \\
水稻市场价格与水稻种植面积 & -0.68447 & 0.0009 & 负向趋同 \\
水稻的生产成本与水稻种植面积 & -0.69049 & 0.0008 & 负向趋同 \\
水稻相对收益竞争力与水稻种植面积 & 0.08800 & 0.7122 & 没有关联 \\
农作物播种面积与水稻种植面积 & 0.23242 & 0.3241 & 没有关联 \\
受灾情况与水稻种植面积 & 0.46785 & 0.0375 & 正向趋同 \\
水稻净出口与水稻种植面积 & 0.38586 & 0.0929 & 正向趋同 \\
城市化水平与水稻种植面积 & -0.90971 & $<.0001$ & 负向趋同 \\
\hline
\end{tabular}

由以上斯皮尔曼检验的结果可以看出,广西省水稻的相对收益竞争力和农作物播种面积 $t$ 检验的 $P$ 值均大于显著性水平 $\alpha=0.1$,故它们之间没有关联。而农民受教育程度、城乡收入差距、家庭负担、水稻的市场价格、水稻的生产成本和城市化水平与水稻种植面积与 $t$ 检验的 $P$ 值均小于显著性水平 $\alpha=0.1$,故他们之间存在相关关系,且从 $r_{s}$ 的值为负值可以得出,它们之间都呈负向趋同关系。农业劳动力人口、受灾情况和水稻的净出口与水稻种植面积 $t$ 检验的 $P$ 值均小于显著性水平 $\alpha=0.1$,故他们之间存在相关关系,且从 $r_{s}$ 的值为正值可以得出,它们之间都呈正向趋同关系呈正向趋同。

通过斯皮尔曼检验,我们剔除掉水稻的相对收益竞争力和农作物播种面积两个无关联的指标。剩下的 9 个指标分别是农业劳动力人口、农民受教育程度、城乡收入差距、家庭负担、水稻的市场价格、水稻的生产成本、受灾情况、水稻的净出口和城市化水平。这 9 个指标将作为影响广西省水稻种植面积的主要因素或指标。

#### 2.3.2 模型建立与求解——影响水稻种植面积的指标与水稻种植面积

##### 2.3.2.1 偏最小二乘回归模型

偏最小二乘回归法是一种新型的多元统计数据分析方法,它主要研究的是多自变量的回归建模,特别当各变量内部高度线性相关时,用偏最小二乘回归法更有效。另外,偏最小二乘法是集主成分分析、典型相关分析和多元线性回归分析 3 种分析方法的优点

于一身。它与主成分分析法都试图提取出反映数据变异的最大信息,但主成分分析法只考虑一个自变量矩阵,而偏最小二乘法还有一个“响应”矩阵,因此具有预测功能。

1. 偏最小二乘回归方程建模步骤:

首先对原始数据 $X$ 与 $Y$ 矩阵进行标准化处理,得到标准化后的 $A$ 与 $B$ 矩阵,分别提取两组变量组的第一对主成分。假设从两组变量分别提出第一对主成分为 $u_1$ 和 $v_1$,$u_1$ 为自变量 $X = [x_1, \cdots x_m]^T$ 的线性组合,$u_1 = \rho^{(1)T} X$;$v_1$ 为因变量 $Y = [y_1, \cdots y_m]^T$ 的线性组合,$v_1 = \gamma^{(1)T} Y$。由两组变量集的标准化后的数据矩阵 $A$ 和 $B$,可计算第一对成分的得分向量记为 $\hat{u}_1$ 和 $\hat{v}_1$。为了回归的需要,要求,

1) $u_1$ 和 $v_1$ 各自尽可能多地提取所在变量组的变异信息;

2) $u_1$ 和 $v_1$ 的相关程度达到最大。

于是对第一对主成分的提取,化为求条件极值问题:
\[
\begin{aligned}
\max (\hat{u}_1 \cdot \hat{v}_1) &= \rho^{(1)T} A^T B \gamma^{(1)} \\
\text{s.t.} & \begin{cases}
\left\| \rho^{(1)} \right\|^2 = 1 \\
\left\| \gamma^{(1)} \right\|^2 = 1
\end{cases}
\end{aligned}
\]

需求出矩阵 $M = A'BB'A$ 的特征值与特征向量,其最大特征值 $\lambda_1$,对应的特征向量即为所求的 $\rho^{(1)}$,目标函数值为 $\sqrt{\lambda_1}$。

然后,建立 $y_1 \cdots y_m$ 对 $u_1$ 的回归及 $x_1 \cdots x_m$ 对 $u_1$ 的回归,
\[
\begin{cases}
A = \hat{u}_1 \sigma^{(1)T} + A_1 \\
B = \hat{u}_1 \tau^{(1)T} + B_1
\end{cases}
\]

其中,$A_1$ 和 $B_1$ 为残差阵,回归系数向量 $\sigma^{(1)T}$,$\tau^{(1)T}$ 由最小二乘法求得。

最后,用残差阵 $A_1$ 和 $B_1$ 代替 $A$ 和 $B$ 重复以上步骤求得第二主成分,依次循环。设矩阵 $A$ 的秩为 $r$,则存在 $r$ 个主成分使得
\[
\begin{cases}
A = \hat{u}_1 \sigma^{(1)T} + \cdots + \hat{u}_r \sigma^{(r)T} + A_r \\
B = \hat{u}_1 \tau^{(1)T} + \cdots + \hat{u}_r \tau^{(r)T} + B_r
\end{cases}
\]

将 $u_k = \alpha_{k1} x_1 + \cdots \alpha_{km} x_m$,$(k = 1, 2, \cdots r)$,代入 $Y = u_1 \tau^{(1)} + \cdots u_r \tau^{(r)}$,即得 $P$ 个因变量的偏最小二乘回归方程式:
\[
y_j = c_{j1} x_1 + \cdots + c_{jm} x_m, \quad j = 1, 2, \cdots p。
\]

2. 交叉验证法选择模型成分个数

若选取成分的个数过多,会很容易出现过度拟合的问题,因此我们需要一个有效的原则来确定成分的个数。采用类似抽样测试的工作方式,把所有样本点分成两部分:第一部分用来重新拟合一个偏最小二乘模型,第二部分的样本点作为测试数据;带入拟合模型中求得预测值误差平方和 $PRESS = \sum (y_i - \hat{y}_i)^2$。再以这种方式重复 $g$ 次,直到所有的样本都被预测了一次,最后把每个样本的预测误差平方和加总,称为 $PRESS$。

\[
PRESS = \sum_{i=1}^g PRESS_i
\]

常见的交叉验证方法有“留一验证”,“K 折交叉验证”,“Holdout 验证”等方法,选取一种方法分别求出第 $1 \sim r$ 个成分对应的 $PRESS$ 值,取 $PRESS$ 最小的或者 $PRESS$ 几乎不再变化的成分个数作为最终模型选取的成分个数 $m$。

3. 回归系数的显著性检验

由于偏最小二乘法不同于一般最小二乘法,它的回归系数方程无法得到准确的无偏估计。Miller 提出了 Quenouille—Tukey jackknife 方法来估计回归系数的方差:先抽出 \( g \) 个样本子集,然后用只去除一个子集的样本做一次偏最小二乘的回归系数估计,记去除第 \( i \) 个样本子集对应的回归系数为 \( \beta_{-i} \),则 jackknife 方差估计为

\[
\operatorname{var}(\beta_{i}) = \frac{g-1}{g} \sum_{i=1}^{g} (\beta_{-i} - \bar{\beta})^2
\]

其中 \( \bar{\beta} \) 是 \( \beta_{-i} \) 的均值,最常见的重抽样法是留一 jackknife 法,即每次选一个样本点,于是共有 \( n \) 个样本子集(即 \( g = n \))。在估计出方差后,类似于普通最小二乘法,求出 \( \beta_{i} \) 对应的 \( t \) 统计量,再进行均值是否为零的假设检验。

在本题中,我们应用以上偏最小二乘回归法建立自变量影响小麦和水稻种植面积的指标体系与因变量小麦种植面积的模型并通过 R 软件求解。

### 2.3.2 模型建立与求解

运用偏最小二乘回归的方法通过 R 软件分别对黑龙江省、湖南省、广西省水稻种植面积的指标与水稻种植面积的 1995-2014 时间段内的相关数据标准化后建立模型,令影响水稻种植面积的农业劳动力人口、农民人均受教育程度、城乡收入差距、农民家庭负担、水稻的市场价格、水稻生产成本、水稻相对收益竞争力、农作物播种面积、受灾情况、我国水稻净出口量、城市化水平 11 个指标数据分别为 \( x_1, x_2, x_3, \ldots, x_{11} \)。黑龙江省影响水稻种植面积的指标与水稻种植面积模型部分结果如下所示:

首先,通过留一交叉验证法的结果及各主成分对因变量的贡献率,选取回归的主成分的个数。如下表所示:

**表 2-8 主成分选取(部分)**

```
VALIDATION: RMSEP
Cross-validated using 20 leave-one-out segments.
          (Intercept) 1 comps 2 comps 3 comps 4 comps
CV            747.1     195     193   180.5   178.8
adjCV         747.1     194.4   192.6   179     176.7
TRAINING: % variance explained
          1 comps 2 comps 3 comps 4 comps 5 comps
X            93.21    99.78   99.87   99.91   100
Y            94.04    94.55   96.77   97.07   97.1
```

其中 CV 即为不同主成分个数对应的 PRESS,adjCV 为调整后的 PRESS,“variance explained” 为主成分对各变量的累积贡献率。由结果可知,主成分个数为 3 个时,模型在经过留一交叉验证后求得的 PRESS 总和较小,随着主成分个数的增加,PRESS 值也没有太大改变。且 3 个主成分对各因变量的累积贡献率达到了 99% 以上。因此选取回归的成分个数为 3。

进一步,由 R 软件计算得出数据标准化后的回归系数,写出因变量对解释变量的回归方程,如下表所示:

**表 2-9 回归系数**

```
          x1          x2          x4          x5
y1  0.379294797 0.084083161 0.111220235 0.032193578
```

\begin{table}[h]
\centering
\caption{偏最小二乘回归方程系数检验}
\begin{tabular}{c c c | c c c}
\hline
各省水稻 & 影响水稻种植 & t检验的P & 各省水稻 & 影响水稻种植 & t检验的P \\
种植面积 & 面积的指标 & 值 & 种植面积 & 面积的指标 & 值 \\
\hline
x1 & 0.03081 & * & x7 & 0.17735 & \\
x2 & 0.92571 & & y2 & x8 & 0.0550885 \\
x4 & 0.10481 & & & x10 & 0.0294102 \\
x5 & 0.73714 & & & x1 & 0.0002207 & *** \\
x6 & 0.68514 & & & x2 & 4.49E-05 & *** \\
x7 & 0.29261 & & & x3 & 0.0001588 & *** \\
x8 & 0.03485 & * & & x4 & 0.0062549 & ** \\
x11 & 0.59422 & & & x5 & 0.4562413 & \\
x1 & 0.0222131 & * & & x6 & 0.3058172 & \\
y2 & x3 & 0.0871383 & & x9 & 0.1239073 & \\
& x5 & 0.0039154 & ** & x10 & 0.3891833 & \\
\hline
\end{tabular}
\end{table}

由上表我们可得影响黑龙江省水稻种植面积的8个指标与水稻种植面积的偏最小二乘回归方程为:
\begin{equation}
y_1 = 0.3793x_1 + 0.0841x_2 + 0.1112x_4 + 0.3219x_5 + 0.0223x_6
\end{equation}
\begin{equation}
+ 0.0449x_7 + 0.2756x_8 + 0.1382x_{11}
\end{equation}

由以上方程我们可得黑龙江省影响水稻种植面积的指标农业劳动力人口$x_1$、农民受教育程度$x_2$、家庭负担$x_4$、水稻的市场价格$x_5$、水稻的生产成本$x_6$、水稻的相对收益竞争力$x_7$、农作物播种面积$x_8$和城市化水平$x_{11}$与黑龙江水稻种植面积$y_1$的关系。

根据以上分析方法我们可得湖南省和广西省影响水稻种植面积的指标与水稻种植面积的偏最小二乘回归方程分别为:
\begin{equation}
y_2 = 0.1094x_1 - 0.1157x_3 + 0.1157x_5 + 0.1303x_6
\end{equation}
湖南省:
\begin{equation}
+ 0.0651x_7 + 0.1241x_8 - 0.1440x_{10}
\end{equation}

其中,$y_2$为湖南省水稻种植面积,以上方程为湖南省水稻种植面积与影响水稻种植面积的指标农业劳动力人口、城乡收入差距、水稻的市场价格、水稻的生产成本、水稻的相对收益竞争力、农作物播种面积和水稻的净出口的回归方程。

\begin{equation}
y_3 = 0.1633x_1 - 0.2228x_2 - 0.2559x_3 - 0.1065x_4 + 0.0404x_5
\end{equation}
广西省:
\begin{equation}
+ 0.0431x_6 - 0.15686x_9 - 0.7765x_{10} - 0.1981x_{11}
\end{equation}

其中,$y_3$为广西省水稻种植面积,以上方程为广西省水稻种植面积与影响水稻种植面积的指标农业劳动力人口、农民受教育程度、城乡收入差距、家庭负担、水稻的市场价格、水稻的生产成本、受灾情况、水稻的净出口和城市化水平的回归方程。

\subsection{关于水稻种植面积的模型可信度和可靠性检验与分析}
\subsubsection{回归方程系数的显著性检验}
运用Jackknife方差估计法通过R软件对回归方程的显著性进行检验,经整理可得如下结果:

\begin{table}
\centering
\begin{tabular}{c c c | c c c}
\hline
x6 & 0.0005713 & *** & x11 & 3.42E-07 & **** \\
\hline
\end{tabular}
\end{table}

显著性符号表示:极其显著 ‘***’ 非常显著 ‘**’ 很显著 ‘*’ 较显著 ‘.’ 

通过显著性检验可知,各省水稻种植面积受哪些自变量(影响水稻种植面积的指标)的影响叫道,及其影响程度。由上表可知,对于黑龙江省水稻种植面积 $y_1$,对其有显著性影响的为农业劳动力人口 $x_1$、农作物播种面积 $x_8$,并且通过回归方程可得均为正的影响,说明农业劳动人口越多、农作物播种面积越大则水稻种植面积越大这与实际相符。对于湖南省水稻种植面积 $y_2$,对其有显著影响的为农业劳动力人口、城乡收入差距、水稻的市场价格、水稻的相对收益竞争力、农作物播种面积和水稻的净出口。对于广西省 $y_3$,对其有显著性影响的为农业劳动力人口、农民受教育程度、城乡收入差距、家庭负担和城市化水平。

\subsubsection{2.3.3.2 模型拟合效果分析}

运用 R 软件画出,三个省份最终模型的拟合效果图,如下所示:

\begin{figure}[h]
\centering
\includegraphics[width=\textwidth]{image.png}
\caption{模型拟合效果图}
\end{figure}

上图从左往右依次为黑龙江省、湖南省、广西省水稻种植面积及其影响因素指标的模型拟合效果图。从三个图可以看出,散点大致都分布在对角线上,说明三个最终模型的拟合效果较好。因此,我们建立的模型可以用来描述水稻种植面积与影响水稻种植面积指标之间的关系是可靠的。

\subsection{2.4 影响小麦种植面积的因素指标体系模型建立与求解}

根据 2.2.2 中粮食品种区域划分,小麦的 6 个主产省(区)被分在北方冬小麦主产区和南方冬小麦主产区两个区域。为了研究方便,分别从这两个区域中各选取一个最具代表性的省(区)进行分析。在北方冬小麦主产区,由于河南省的水稻种植面积是最大的,约占北方冬小麦主产区种植总面积的 40\% 以上,故选取河南省作为代表省进行分析。在南方冬小麦主产区,由于安徽省近几年的种植面积是最大的,约占南方冬小麦主产区小麦种植总面积的 35\% 以上,故选取安徽省为代表省进行分析。接下来,用选取的河南和安徽量个省的相关数据研究影响水稻种植面积的因素。

\subsubsection{2.4.1 指标选取}

对 2.1 中通过定性分析选取的影响粮食种植面积的 11 个指标与小麦种植面积进行相关性检验,选取与小麦种植面积相关的指标,并分析影响小麦种植面积的指标与小麦种植面积的趋同性。与水稻的相关性分析的方法相同,运用非参数 Spearman 相关检验的方法通过 SAS 软件分别对河南省和安徽省小麦种植面积的指标与小麦种植面积的 1995-2014 时间段内的相关数据进行相关性检验,结果如下所示:

\section*{1. 指标相关性分析结果——河南省}

\begin{table}[h]
\centering
\caption{表2-11 河南省指标相关性分析结果}
\begin{tabular}{l c c l}
\hline
变量 & $r_{s}$ & $t$检验的$p$值 & 趋同性说明 \\
\hline
农业劳动力人口与小麦种植面积 & -0.80150 & $<.0001$ & 负向趋同 \\
农民受教育程度与小麦种植面积 & 0.79549 & $<.0001$ & 正向趋同 \\
城乡收入差距与与小麦种植面积 & -0.00150 & 0.9950 & 没有关联 \\
农民家庭负担与小麦种植面积 & 0.81053 & $<.0001$ & 正向趋同 \\
小麦的市场价格与小麦种植面积 & 0.82105 & $<.0001$ & 正向趋同 \\
小麦的生产成本与小麦种植面积 & 0.95789 & $<.0001$ & 正向趋同 \\
小麦相对收益竞争力与小麦种植面积 & 0.57594 & 0.0079 & 正向趋同 \\
农作物播种面积与小麦种植面积 & 0.81353 & $<.0001$ & 正向趋同 \\
受灾情况与小麦种植面积 & 0.41053 & 0.0722 & 正向趋同 \\
小麦净出口与小麦种植面积 & -0.17143 & 0.4699 & 没有关联 \\
城市化水平与小麦种植面积 & 0.80602 & $<.0001$ & 正向趋同 \\
\hline
\end{tabular}
\end{table}

由以上斯皮尔曼检验的结果可以看出,河南省小麦种植面积与城乡收入差距和小麦的净出口$t$检验的$P$值均大于显著性水平$\alpha=0.1$,故它们之间没有关联。而小麦种植面积与农民受教育程度、家庭负担、小麦的价格预期、小麦的生产成本、小麦的相对收益竞争力、农作物播种面积、受灾情况、城市化水平和农业劳动人口$t$检验的$P$值均小于显著性水平$\alpha=0.1$,故他们之间存在相关关系。且从$r_{s}$的值可以看出,除农业劳动人口外,它们之间呈正向趋同关系。

通过斯皮尔曼检验,我们剔除掉城乡收入差距和小麦的净出口两个无关联的指标。剩下的9个指标分别是农业劳动力人口、农民受教育程度、家庭负担、小麦的市场价格、小麦的生产成本、小麦的相对收益竞争力、农作物播种面积、受灾情况和城市化水平。这9个指标将作为影响河南省小麦种植面积的主要指标。

\section*{2. 指标相关性分析结果——安徽省}

由表2-12斯皮尔曼检验的结果可以看出,安徽省小麦种植面积与城乡收入差距、小麦的相对收益竞争力和小麦的净出口$t$检验的$P$值均大于显著性水平$\alpha=0.1$,故他们之间没有关联。而小麦种植面积与农民受教育程度、家庭负担、小麦的价格预期、小麦的生产成本、农作物播种面积、受灾情况、城市化水平、农业劳动力人口与小麦种植面积$t$检验的$P$值均小于显著性水平$\alpha=0.1$,故他们之间存在相关关系,从$r_{s}$的值可以看出,除农业劳动人口与小麦种植面积间负向趋同外,其它指标与小麦种植面积呈正向趋同关系。

\begin{table}[h]
\centering
\caption{表2-12 安徽省指标相关性分析结果}
\begin{tabular}{l c c l}
\hline
变量 & $r_{s}$ & $t$检验的$p$值 & 趋同性说明 \\
\hline
农业劳动力人口与小麦种植面积 & -0.77143 & $<.0001$ & 负向趋同 \\
农民受教育程度与小麦种植面积 & 0.65564 & 0.0017 & 正向趋同 \\
城乡收入差距与与小麦种植面积 & -0.03008 & 0.8998 & 没有关联 \\
农民家庭负担与小麦种植面积 & 0.74135 & 0.0002 & 正向趋同 \\
小麦的市场价格与小麦种植面积 & 0.91880 & $<.0001$ & 正向趋同 \\
小麦的生产成本与小麦种植面积 & 0.91729 & $<.0001$ & 正向趋同 \\
小麦相对收益竞争力与小麦种植面积 & 0.25263 & 0.2826 & 没有关联 \\
农作物播种面积与小麦种植面积 & 0.81053 & $<.0001$ & 正向趋同 \\
\hline
\end{tabular}
\end{table}

\begin{tabular}{l c c c c}
受灾情况与小麦种植面积 & 0.60474 & 0.0047 & 正向趋同 \\
小麦净出口与小麦种植面积 & -0.25714 & 0.2738 & 没有关联 \\
城市化水平与小麦种植面积 & 0.70827 & 0.0005 & 正向趋同 \\
\end{tabular}

通过斯皮尔曼检验,我们剔除掉城乡收入差距、小麦的相对收益竞争力和小麦的净出口三个无关联的指标。剩下的8个指标分别是农业劳动力人口、农民受教育程度、家庭负担、小麦的市场、小麦的生产成本、农作物播种面积、受灾情况和城市化水平。这8个指标将作为影响安徽省小麦种植面积的主要指标。

\subsection*{2.4.2 模型建立与求解——影响小麦种植面积的指标与小麦种植面积}

运用偏最小二乘回归的方法通过R软件分别对河南省和安徽省小麦种植面积的指标与小麦种植面积的1995-2014时间段内的相关数据标准化后建立模型,令影响小麦种植面积的农业劳动力人口、农民人均受教育程度、城乡收入差距、农民家庭负担、小麦的市场价格、小麦生产成本、小麦相对收益竞争力、农作物播种面积、受灾情况、我国小麦净出口量、城市化水平11个指标数据分别为 \(z_1, z_2, z_3, \ldots, z_{11}\) 。河南省影响小麦种植面积的指标与小麦种植面积模型部分结果如下所示:

首先,通过留一交叉验证法的结果及各主成分对因变量的贡献率,选取回归的主成分的个数。如下表所示:

\textbf{表2-13 主成分选取(部分)}

\begin{tabular}{l c c c c}
VALIDATION: RMSEP & & & & \\
Cross-validated using 20 leave-one-out segments. & & & & \\
(Intercept) & 1 comps & 2 comps & 3 comps & 4 comps \\
CV & 1.026 & 0.4804 & 0.4841 & 0.4807 \\
adjCV & 1.026 & 0.4794 & 0.4808 & 0.4781 \\
\end{tabular}

\begin{tabular}{l c c c c c}
TRAINING: \% variance explained & & & & & \\
1 comps & 2 comps & 3 comps & 4 comps & 5 comps \\
Z & 73.51 & 81.69 & 90.37 & 96.6 & 98.06 \\
Y & 80.52 & 85.41 & 85.9 & 86.29 & 87.59 \\
\end{tabular}

由上表结果可知,主成分个数为1个时,模型在经过留一交叉验证后求得的PRESS总和最小,随着主成分个数的增加,PRESS值也没有太大改变。且1个主成分对各因变量的累积贡献率达到了80%以上。因此选取回归的成分个数为1。

进一步,由R软件计算得出数据标准化后的回归系数,写出因变量对解释变量的回归方程,如下表所示:

\textbf{表2-14 回归系数}

\begin{tabular}{c c c c c c}
z1 & z2 & z4 & z5 & z6 \\
y4 & 0.11670931 & 0.11800376 & 0.13230034 & 0.13207702 & 0.1335455 \\
z7 & z8 & z9 & z11 \\
y4 & 0.09778338 & 0.11007903 & 0.05591939 & 0.13164516 \\
\end{tabular}

由上表我们可得,影响河南省小麦种植面积的9个指标与小麦种植面积的偏最小二乘回归方程为:

\[
y_4 = 0.1167z_1 + 0.1180z_2 + 0.1323z_3 + 0.1321z_4 + 0.1335z_5
\]
\[
+ 0.0978z_6 + 0.1100z_7 - 0.0559z_{10} - 0.1316z_{11}
\]

由以上方程我们可得河省影响小麦种植面积的指标农业劳动力人口 \(z_1\)、农民受教育程度 \(z_2\)、家庭负担 \(z_4\)、小麦的市场价格 \(z_5\)、小麦的生产成本 \(z_6\)、受灾情况 \(z_9\)、小麦净出口 \(z_{10}\) 和城市化水平 \(z_{11}\) 与河南小麦种植面积 \(y_3\) 的关系。

根据以上分析方法我们可得安徽省影响小麦种植面积的指标与小麦种植面积的偏最小二乘回归方程为:
\[
\begin{aligned}
y_5 &= -0.2058z_1 - 0.2487z_2 - 0.0978z_4 + 0.1939z_5 \\
&\quad - 0.6194z_6 + 0.6925z_8 + 0.0230z_9 - 0.4673z_{11}
\end{aligned}
\]
其中,\(y_5\) 为安徽省小麦种植面积,以上方程为安徽省小麦种植面积与影响小麦种植面积的指标农业劳动力人口 \(z_1\)、农民受教育程度 \(z_2\)、家庭负担 \(z_4\)、小麦的市场价格 \(z_5\)、小麦的生产成本 \(z_6\)、农作物播种面积 \(z_8\)、受灾情况 \(z_9\) 和城市化水平 \(z_{11}\) 的回归方程。

### 2.4.3 关于小麦种植面积的模型可信度和可靠性检验与分析
#### 2.3.3.1 回归方程系数的显著性检验
运用 Jackknife 方差估计法通过 R 软件对回归方程的显著性进行检验,经整理可得如下结果:

**表 2-15 偏最小二乘回归系数检验**

| 各省小麦种植面积 | 影响小麦种植面积的指标 | \(t\) 检验的 \(P\) 值 | 显著性 | 各省小麦种植面积 | 影响小麦种植面积的指标 | \(t\) 检验的 \(P\) 值 | 显著性 |
|-------------------|--------------------------|---------------------|--------|-------------------|--------------------------|---------------------|--------|
|                   | \(z1\)                   | 0.002198            | **\*** |                   | \(z1\)                   | 0.3959025           |        |
|                   | \(z2\)                   | 1.04E-05            | **\*\*\*** |                   | \(z2\)                   | 0.1291363           |        |
|                   | \(z3\)                   | 1.40E-09            | **\*\*\*** |                   | \(z4\)                   | 0.6524593           |        |
|                   | \(z4\)                   | 1.76E-05            | **\*\*\*** |                   | \(z5\)                   | 0.5857672           |        |
| \(y4\)            | \(z5\)                   | 1.13E-07            | **\*\*\*** | \(y5\)            | \(z6\)                   | 0.047598            | **\*** |
|                   | \(z6\)                   | 0.000512            | **\*\*\*** |                   | \(z8\)                   | 0.0006989           | **\*\*\*** |
|                   | \(z9\)                   | 9.41E-05            | **\*\*\*** |                   | \(z9\)                   | 0.8496121           |        |
|                   | \(z10\)                  | 0.08208             | .       |                   | \(z11\)                  | 0.0532685           | .      |
|                   | \(z11\)                  | 3.37E-07            | **\*\*\*** |                   |                          |                     |        |

显著性符号表示:极其显著 ‘\*\*\*’ 非常显著 ‘\*\*’ 很显著 ‘\*’ 较显著 ‘.’ 

通过显著性检验可知,各省小麦种植面积受哪些自变量(影响小麦种植面积的指标)的影响及其影响程度。由上表可知,对于河南省小麦种植面积 \(y_4\),对其有显著性影响的为农业劳动力人口 \(x_1\)、农民受教育程度 \(x_2\)、家庭负担 \(x_4\)、小麦的市场价格 \(x_5\)、小麦的生产成本 \(x_6\)、小麦的相对收益竞争力 \(x_7\)、农作物播种面积 \(x_8\)、受灾情况 \(x_9\) 和城市化水平 \(x_{11}\)。对于安徽省小麦种植面积 \(y_5\),对其有显著影响的为小麦的生产成本、农作物播种面积和城市化水平。

#### 2.3.3.2 模型拟合效果分析
运用 R 软件画出,两个省份最终模型的拟合效果图,如下所示:

\begin{figure}[h]
    \centering
    \includegraphics[width=\textwidth]{image.png}
    \caption{模型拟合效果图}
    \label{fig:2-2}
\end{figure}

上图从左往右依次为河南省、安徽省小麦种植面积及其影响因素指标的模型拟合效果图。从上图可以看出,两个图的散点大致都分布在对角线上,说明两个最终模型的拟合效果较好。因此,我们建立的模型可以用来描述小麦种植面积与影响小麦种植面积指标之间的关系是可靠的。

\section{问题二解答——粮食最低收购价政策评价模型}

我国自2005年起开始对粮食主产区实行了最低收购价政策。对粮食最低收购价政策的作用,学者们褒贬不一。我们需要通过建立粮食最低收购价政策执行效果的评价模型,结合粮食品种和区域差异,比较研究粮食主产区粮食最低收购价执行的效果。本题由问题一相关性分析选取的各省影响水稻和小麦种植面积的指标及粮食最低收购价政策的执行情况建立水稻和小麦最低收购价政策评价模型。

\subsection{数据处理}

\subsubsection{水稻数据处理}

由问题一相关性分析选取的各省影响水稻种植面积的因素及最低收购价政策作为自变量,水稻种植面积为因变量,建立模型。令影响水稻种植面积的农业劳动力人口、农民人均受教育程度、城乡收入差距、农民家庭负担、水稻的市场价格、水稻生产成本、水稻相对收益竞争力、农作物播种面积、受灾情况、我国水稻净出口量、城市化水平11个指标数据分别为 \(x_1, x_2, x_3, \ldots, x_{11}\);粮食最低收购价政策为 \(s\) 用0-1变量表示,实行政策前为0,实行政策后为1;黑龙江省、湖南省、广西壮族自治区的水稻种植面积分别为 \(y_1, y_2, y_3\)。黑龙江省1995-2014时间段内的部分数据经整理,如下所示:

\begin{table}[h]
    \centering
    \caption{黑龙江省与水稻种植面积相关的指标}
    \label{tab:3-1}
    \begin{tabular}{c c c c c c c c c c c}
        \hline
        年份 & x1 & x2 & x4 & x5 & x6 & x7 & x8 & x11 & s & y1 \\
        \hline
        1995 & 571.8 & 7.529 & 1479.84 & 90.19 & 389.44 & 189.76 & 8647 & 0.537 & 0 & 835.1 \\
        1996 & 566 & 7.734 & 1537.3 & 80.1 & 369.3 & 147.54 & 8883.7 & 0.538 & 0 & 1107.5 \\
        1997 & 653.2 & 7.856 & 1549.1 & 63.43 & 357.01 & 217.16 & 9030.9 & 0.539 & 0 & 1396.9 \\
        1998 & 826.5 & 7.85 & 1464.64 & 64.81 & 275.79 & 110.15 & 9193.78 & 0.54 & 0 & 1566.7 \\
        \hline
    \end{tabular}
\end{table}

\begin{table}
\centering
\caption{安徽省与小麦种植面积相关的指标}
\begin{tabular}{ccccccccccccc}
\hline
年份 & z1 & x2 & z4 & z5 & z6 & z8 & z9 & z11 & s & y5 \\
\hline
1995 & 1945.3 & 5.85 & 1855.83 & 66.03 & 154.14 & 8354.23 & 2.31 & 0.19 & 0 & 1992.66 \\
1996 & 1968.6 & 5.88 & 2073.09 & 70.23 & 249.23 & 8410 & 6.39 & 0.21 & 0 & 2100.11 \\
1997 & 1979.2 & 6.19 & 2097.22 & 67.81 & 297.58 & 8488.93 & 2.32 & 0.22 & 0 & 2137.57 \\
1998 & 2007.8 & 6.34 & 2012.59 & 63.87 & 248.23 & 8564.23 & 3.62 & 0.22 & 0 & 2095.8 \\
\multicolumn{11}{c}{$\cdots$} \\
2010 & 1583.6 & 7.68 & 5909.58 & 93.71 & 424.13 & 9054.87 & 30 & 0.43 & 1 & 2365.67 \\
2011 & 1598.9 & 8.24 & 7602.86 & 97.86 & 504.71 & 9022.99 & 32.71 & 0.45 & 1 & 2383 \\
2012 & 1531.2 & 8.42 & 8562.63 & 104.43 & 544.12 & 8969.6 & 2.23 & 0.47 & 1 & 2415.51 \\
2013 & 1469.7 & 8.54 & 8134.87 & 112.93 & 589.98 & 8945.64 & 4.98 & 0.48 & 1 & 2432.85 \\
2014 & 1415.3 & 8.69 & 14176.35 & 114.84 & 624.11 & 8945.53 & 2.59 & 0.49 & 1 & 2434.5 \\
\hline
\end{tabular}
\end{table}

\section*{混合线性模型的一般形式为:}
\[ Y = X\beta + Z\Gamma + \varepsilon \]
式中 $Z$ 为随机效应变量构造的设计矩阵, 其构造方式与 $X$ 相同。$\Gamma$ 为随机效应参数向量, $\Gamma$ 服从均值向量为 0 、方差协方差矩阵为 $G$ 的正态分布, 表示为 $\Gamma \sim N(0, G)$。$\varepsilon$ 为随机误差向量, 放宽了对 $\varepsilon$ 的限制条件, 其元素不必为独立同分布。

\subsection*{3.2.2 确定水稻固定效应和随机效应}

\subsubsection{3.2.2.1 固定效应确定}

固定效应为设计者所研究的因素, 试验中若因素的水平选取已定好, 该因素的水平效应作为固定参数, 称此为固定效应。本题中, 我们主要研究水稻最低收购价政策对水稻种植面积的影响, 因此将水稻最低收购价政策作为固定效应。

\subsubsection{3.2.2.2 随机效应确定}

若因素的水平选取是随机地从某一总体中抽取, 水平效应值是一随机变量, 称此效应为随机效应。由于除水稻最低收购价政策影响水稻种植面积的因素还有很多, 其它影响水稻种植面积的视为随机效应。由于问题一中经相关性分析选取的影响水稻种植面积的指标较多且有可能存在相关关系, 因此我们用主成分的方法对这些影响因素降维。

以黑龙江省为例, 我们运用主成分的方法对影响黑龙江省水稻种植面积的因素进行降维处理, 经 SAS 软件运行, 可得主成分的结果如下所示:

\begin{table}[h]
\centering
\caption{主成分分析结果}
\begin{tabular}{c c c c c}
\hline
 & Eigenvalue & Difference & Proportion & Cumulative \\
\hline
1 & 6.27257535 & 5.26112706 & 0.7841 & 0.7841 \\
2 & 1.01144829 & 0.61398455 & 0.1264 & 0.9105 \\
3 & 0.39746375 & 0.23863373 & 0.0497 & 0.9602 \\
4 & 0.15883001 & 0.06598368 & 0.0199 & 0.9800 \\
5 & 0.09284633 & 0.05540989 & 0.0116 & 0.9916 \\
6 & 0.03743644 & 0.01760329 & 0.0047 & 0.9963 \\
7 & 0.01983316 & 0.01026648 & 0.0025 & 0.9988 \\
8 & 0.00956667 & & 0.0012 & 1.0000 \\
\hline
\end{tabular}
\end{table}

从上表可以看出, 第一和第二主成分的方差累计贡献率已经超过了 90%, 故我们只选取第一、二主成分, 其它主成分可以舍去达到降维的目的。

\begin{table}[h]
\centering
\caption{主成分对应原始变量的系数(部分)}
\begin{tabular}{c c c c c c}
\hline
 & Prin1 & Prin2 & Prin3 & Prin4 & Prin5 \\
\hline
x1 & 0.348744 & $-$0.256452 & 0.526058 & $-$0.569969 & 0.174683 \\
x2 & 0.371820 & $-$0.231941 & 0.310100 & 0.144963 & $-$0.506873 \\
x4 & 0.393495 & $-$0.018293 & $-$0.127488 & $-$0.155586 & $-$0.206494 \\
x5 & 0.378559 & 0.216952 & $-$0.224572 & 0.333367 & 0.007008 \\
x6 & 0.384964 & 0.085176 & $-$0.317499 & $-$0.002576 & $-$0.427096 \\
x7 & 0.184544 & 0.831928 & 0.457159 & 0.075985 & 0.110183 \\
x8 & 0.353681 & $-$0.350361 & 0.142438 & 0.598011 & 0.516738 \\
x11 & 0.366931 & 0.104500 & $-$0.479923 & $-$0.394186 & 0.456407 \\
\hline
\end{tabular}
\end{table}

从上表可知, 第一主成分对应于原始变量的系数向量和第二主成分对应于原始变

\begin{table}[h]
\centering
\caption{表3-5 黑龙江省水稻主成分分析结果}
\begin{tabular}{c c c}
\hline
年份 & $Z_{1}^{*}$ & $Z_{2}^{*}$ \\
\hline
1995 & 4062.079228 & -2994.367201 \\
1996 & 4147.095335 & -3115.937278 \\
1997 & 4236.062792 & -3136.861657 \\
1998 & 4270.378819 & -3332.468282 \\
\vdots & \vdots & \vdots \\
2011 & 7138.765206 & -4302.010278 \\
2012 & 7356.8303 & -4377.040643 \\
2013 & 7806.595767 & -4366.874347 \\
2014 & 8226.834901 & -4353.216295 \\
\hline
\end{tabular}
\end{table}

\begin{table}[h]
\centering
\caption{表3-6 湖南省和广西省主成分结果(部分)}
\begin{tabular}{c c c c c c}
\hline
\multicolumn{3}{c}{湖南省} & \multicolumn{3}{c}{广西省} \\
\hline
年份 & $Z_{1}^{\prime}$ & $Z_{2}^{\prime}$ & $Z_{3}^{\prime}$ & $Z_{1}^{\prime\prime}$ & $Z_{2}^{\prime\prime}$ & $Z_{3}^{\prime\prime}$ \\
\hline
1995 & 3871.3518 & -310.0483 & 783.2385 & 1018.1389 & -1814.1116 & 1346.2929 \\
1996 & 3010.9871 & 541.8776 & -999.3004 & 414.5565 & -762.8601 & 162.8887 \\
1997 & 2565.3363 & 940.5030 & -1931.6361 & 58.3252 & -220.9531 & -413.2633 \\
1998 & 738.0846 & 2373.5906 & -5214.6223 & -1256.6857 & 1789.3142 & -2482.7710 \\
\vdots & \vdots & \vdots & \vdots & \vdots & \vdots & \vdots \\
2011 & 3245.1993 & 731.5099 & -1758.3783 & 1527.5381 & -919.9870 & -741.1882 \\
2012 & 3856.5541 & 447.1932 & -820.0053 & 2233.7995 & -1576.3106 & -339.5345 \\
2013 & 4065.7360 & 483.5527 & -623.4474 & 2457.8231 & -1731.1293 & -318.6194 \\
2014 & 4293.6807 & 683.3184 & -232.8413 & 2599.6503 & -1864.2069 & -236.8809 \\
\hline
\end{tabular}
\end{table}

接下来, 对所得的随机效应进行分级处理, 黑龙江省变量分级处理的结果如下所示:

\begin{table}[h]
\centering
\caption{表3-7 黑龙江省变量分级表}
\begin{tabular}{c c c c c c}
\hline
变量名 & 区间 & 记号 & 变量名 & 区间 & 记号 \\
\hline
$Z_{1}^{*}$ & $[4062.174, 4895.126]$ & 1 & $Z_{1}^{*}$ & $[7393.979, 8226.835]$ & 5 \\
$Z_{1}^{*}$ & $[4895.126, 5728.077]$ & 2 & $Z_{2}^{*}$ & $[-4482.371, -3738, 369]$ & 1 \\
$Z_{1}^{*}$ & $[5728.077, 6561.028]$ & 3 & $Z_{2}^{*}$ & $[-3738, 369, -2994.367]$ & 2 \\
$Z_{1}^{*}$ & $[6561.028, 7393.979]$ & 4 & & & \\
\hline
\end{tabular}
\end{table}

上表为所选取的黑龙江省随机效应变量的区间分级表, 对于第一主成分根据其区间长度共分 5 级, 第二主成分分为两级。其它两省按以上方法进行分级处理, 分级结果见附件。

\subsection{3.2.3 协方差结构的选择}

模型中具体选用哪种结构矩阵: 在相同模型结构下, 选择几个不同结构的协方差矩阵, 从中选取似然比统计量 ($-2 \log \text{Likelihood}$)、Akaike's Information Criterion (AIC)、Bayesian Information Criterion (BIC) 三个指标均较小的一个, 同时以 AIC 为主要判断指标。依据专业知识和以往文献, 在此模型中选用符合对称结构 [CS]、不规则结构 [UN]、一阶自回归结构 [AR(1)]、空间幂相关结构 [SP(POW)]。

\subsection{3.2.4 水稻最低收购价政策评价模型}

根据混合线性模型的形式, 初步建立黑龙江省混合线性模型的线性模型, 如下:
\[
y_{1} = \beta_{0} + \beta_{1} Z_{1}^{*} + \beta_{2} Z_{2}^{*} + \alpha s + \varepsilon,
\]
其中, $\beta_{0}$ 是常数项, $\beta_{i} (i=1,2)$ 是随机效应变量 $Z_{i}^{*} (i=1,2)$ 的系数, $Z_{i}^{*}$ 是随机效应变量 $\alpha$ 是固定效应变量的系数; $s$ 是固定效应变量水稻的最低收购价政策, 取值为 0 或 1。至此混合线性模型已经建立。

根据以上形式可得湖南省和广西省水稻的混合线性模型分别如下所示:
\[
y_{2} = \beta_{0}' + \beta_{1}' Z_{1}' + \beta_{2}' Z_{2}' + \beta_{3}' Z_{3}' + \alpha' s + \varepsilon'
\]
\[
y_{3} = \beta_{0}'' + \beta_{1}'' Z_{1}'' + \beta_{2}'' Z_{2}'' + \beta_{3}'' Z_{3}'' + \alpha'' s + \varepsilon''
\]
以上所建立的混合线性模型 $y_{1}$、$y_{2}$、$y_{3}$ 为黑龙江省、湖南省、广西水稻最低收购价政策评价模型。

\subsection{3.3 水稻最低收购价政策评价模型的求解与检验}

采用 SAS 的 MIXED 模块对 3.2 中所建立的黑龙江省、湖南省、广西壮族自治区的水稻最低收购价政策评价模型计算, 程序中采用最大似然估计算法, 并用混合线性模型平方和对模型做假设检验。

\subsubsection{3.3.1 黑龙江省水稻最低收购价政策评价模型求解与检验}

采用 SAS 的 MIXED 模块进行计算, 程序中采用最大似然估计算法, 用混合线性模型平方和做假设检验, 得到如下表所示:

\begin{table}[h]
\centering
\caption{表3-8 黑龙江各变量的参数表}
\begin{tabular}{c c c c c}
\hline
Source & DF & Sum of Squares & Mean Square & F Value & Pr > F \\
\hline
Model & 2 & 9279918.57 & 4639959.28 & 99.24 & $<$ .0001 \\
Error & 17 & 794795.56 & 46752.68 & & \\
Corrected Total & 19 & 10074714.13 & & & \\
\hline
\end{tabular}
\end{table}

\begin{table}[h]
\centering
\caption{黑龙江不同协方差结构下的各种检验统计量}
\begin{tabular}{c c}
\hline
统计量 & 统计值 \\
\hline
$-2$ Res Log Likelihood & 253.5 \\
AIC (smaller is better) & 257.5 \\
AICC (smaller is better) & 258.3 \\
BIC (smaller is better) & 253.5 \\
\hline
\end{tabular}
\end{table}

\begin{table}[h]
\centering
\caption{黑龙江固定效应的检验 (Type 3 Tests of Fixed)}
\begin{tabular}{c c c c c}
\hline
Effect & Num DF & Den DF & F Value & Pr $>$ F \\
\hline
a & 1 & 17 & 0.46 & 0.05077 \\
\hline
\end{tabular}
\end{table}

\begin{table}[h]
\centering
\caption{黑龙江固定效应的系数估计结果 (Least Squares Means)}
\begin{tabular}{c c c c c c}
\hline
Effect & a & Estimate & Standard Error & DF & t Value & Pr $>$ |t| \\
\hline
a & 0 & 954.15 & 89.6778 & 17 & 10.64 & $<$ .0001 \\
a & 1 & 1059.32 & 186.41 & 17 & 5.68 & $<$ .0001 \\
\hline
\end{tabular}
\end{table}

\begin{table}[h]
\centering
\caption{湖南各变量的参数表}
\begin{tabular}{c c c c c}
\hline
Source & DF & Sum of Squares & Mean Square & F Value & Pr $>$ F \\
\hline
Model & 3 & 285162.957 & 95054.319 & 3.34 & 0.0460 \\
Error & 16 & 455748.3633 & 28484.2727 & & \\
Corrected Total & 19 & 740911.3203 & & & \\
\hline
\end{tabular}
\end{table}

\begin{table}[h]
\centering
\caption{湖南不同协方差结构下的各种检验统计量}
\begin{tabular}{c c}
\hline
统计量 & 统计值 \\
\hline
$-2$ Res Log Likelihood & 241.6 \\
AIC (smaller is better) & 245.6 \\
AICC (smaller is better) & 246.4 \\
BIC (smaller is better) & 241.6 \\
\hline
\end{tabular}
\end{table}

\begin{table}[h]
\centering
\caption{湖南固定效应的检验 (Type 3 Tests of Fixed)}
\begin{tabular}{c c c c c}
\hline
Effect & Num DF & Den DF & F Value & Pr $>$ F \\
\hline
a & 1 & 16 & 0.11 & 0.07499 \\
\hline
\end{tabular}
\end{table}

\begin{table}[h]
\centering
\caption{湖南固定效应的系数估计结果 (Least Squares Means)}
\begin{tabular}{c c c c c c}
\hline
Effect & a & Estimate & Standard Error & DF & t Value & Pr $>$ |t| \\
\hline
a & 0 & 3675.29 & 89.4755 & 16 & 41.08 & $<$ .0001 \\
a & 1 & 3645.05 & 136.39 & 16 & 26.73 & $<$ .0001 \\
\hline
\end{tabular}
\end{table}

\begin{table}[h]
\centering
\caption{广西各变量的参数表}
\begin{tabular}{c c c c c}
\hline
Source & DF & Sum of Squares & Mean Square & F Value & Pr $>$ F \\
\hline
Model & 3 & 303165.1014 & 101055.0338 & 10.22 & 0.0005 \\
Error & 16 & 158133.3748 & 9883.3359 & & \\
Corrected Total & 19 & 461298.4763 & & & \\
\hline
\end{tabular}
\end{table}

R-Square: 0.857199

由上表可知,$p=0.0005$,小于显著性水平 $\alpha=0.05$,说明黑龙江省水稻最低收购价政策评价模型通过检验,且 $R^2=0.857199$,说明模型的整体拟合效果较好。

混合模型输出有关协方差的各种统计量,如下表所示:

\textbf{表3-17 广西不同协方差结构下的各种检验统计量}

\begin{tabular}{ll}
\hline
统计量 & 统计值 \\
\hline
$-2$ Res Log Likelihood & 220.0 \\
AIC (smaller is better) & 222.0 \\
AICC (smaller is better) & 222.3 \\
BIC (smaller is better) & 220.0 \\
\hline
\end{tabular}

由上表我们可知,$-2$ Res Log Likelihood 和 BIC (smaller is better) 的统计量的值最小,故这里选 $-2$ Res Log Likelihood 作为最后协方差结构。用它的结构计算的固定效应的假设检验结果如下表所示:

\textbf{表3-18 广西固定效应的检验 (Type 3 Tests of Fixed)}

\begin{tabular}{lllll}
\hline
Effect & Num DF & Den DF & F Value & Pr > F \\
\hline
a & 1 & 16 & 31.96 & $<.0001$ \\
\hline
\end{tabular}

由上表可知,固定效应的假设检验结果,在置信水平 $\alpha=0.1$ 下,模型为无效假设,$\chi^2=3.18, p<0.0001$,说明拟合的模型是显著的,即水稻的最低收购价政策对水稻的种植面积有显著作用。

固定效应的系数估计结果如下表所示:

\textbf{表3-19 广西固定效应的系数估计结果 (Least Squares Means)}

\begin{tabular}{lllllll}
\hline
Effect & a & Estimate & Standard & DF & t Value & Pr > |t| \\
& & & Error & & & \\
\hline
a & 0 & 2345.15 & 32.0306 & 16 & 73.22 & $<.0001$ \\
a & 1 & 2100.99 & 28.9728 & 16 & 72.52 & $<.0001$ \\
\hline
\end{tabular}

上表为广西省固定效应的系数估计效果,即系数越大,则效果越好。从上表中我们可以看到水稻的最低收购价政策实行前,水稻的种植面积的估计值为 2345.15 千公顷,水稻的最低收购价政策实行后,水稻的种植面积的估计值为 2100.99 千公顷。由此,我们可以知道水稻的最低收购价政策的实行可以促进广西水稻的种植面积的增加,即水稻的最低收购价政策对广西水稻的种植面积有显著影响。

\subsection*{3.4 各省水稻最低收购价政策效果比较}

经以上水稻最低收购价政策评价模型的建立与求解我们可得水稻最低收购价政策对黑龙江省、湖南省、广西壮族自治区水稻种植面积的影响效果。影响效果用公式为:

\[
\frac{\text{实行政策后水稻种植面积} - \text{实行政策前水稻播种面积}}{\text{实行政策前水稻播种面积}} \times 100\%
\]

经计算结果如下所示:

\textbf{表3-20 三省最低收购价政策效果比较}

\begin{tabular}{llll}
\hline
省份 & 黑龙江省 & 湖南省 & 广西省 \\
\hline
实行政策前 & 954.15 & 3675.29 & 2651.76 \\
实行政策后 & 1059.32 & 3645.05 & 2805.35 \\
影响效果 & 0.110223759 & -0.008227922 & 0.057920023 \\
\hline
\end{tabular}

\begin{table}
\centering
\caption{主成分分析结果}
\begin{tabular}{c c c c c}
\hline
Eigenvalue & Difference & Proportion & Cumulative \\
\hline
1 & 6.46404223 & 5.56926366 & 0.8080 & 0.8080 \\
2 & 0.89477857 & 0.58538608 & 0.1118 & 0.9199 \\
3 & 0.30939248 & 0.12666510 & 0.0387 & 0.9585 \\
4 & 0.18272738 & 0.11414573 & 0.0228 & 0.9814 \\
5 & 0.06858165 & 0.01570319 & 0.0086 & 0.9899 \\
6 & 0.05287846 & 0.03630310 & 0.0066 & 0.9966 \\
7 & 0.01657536 & 0.00555149 & 0.0021 & 0.9986 \\
8 & 0.01102387 & & 0.0014 & 1.0000 \\
\hline
\end{tabular}
\end{table}

\begin{table}
\centering
\caption{主成分对应原始变量的系数(部分)}
\begin{tabular}{c c c c c c}
\hline
 & Prin1 & Prin2 & Prin3 & Prin4 & Prin5 \\
\hline
z1 & -0.384002 & 0.010274 & 0.114854 & 0.218275 & 0.499196 \\
z2 & 0.372041 & -0.105864 & -0.450682 & 0.316818 & 0.350732 \\
z4 & 0.361815 & -0.275293 & 0.224597 & 0.457847 & -0.516469 \\
z5 & 0.362308 & -0.114807 & 0.628809 & -0.080137 & 0.107803 \\
z6 & 0.378146 & -0.180946 & 0.220390 & 0.021738 & 0.589984 \\
z8 & 0.360860 & 0.244544 & -0.030387 & -0.709478 & -0.011744 \\
z9 & 0.196361 & 0.897896 & 0.129515 & 0.363130 & 0.012043 \\
z11 & 0.374404 & -0.030989 & -0.521134 & 0.015555 & -0.032622 \\
\hline
\end{tabular}
\end{table}

\begin{align*}
P_{1}^{*} &= -0.384002 * z_{1} + 0.372041 * z_{2} + 0.361815 * z_{4} + 0.362308 * z_{5} \\
&+ 0.378146 * z_{6} + 0.360860 * z_{8} + 0.196361 * z_{9} + 0.374404 * z_{11} \\
P_{2}^{*} &= 0.010274 * z_{1} - 0.105864 * z_{2} - 0.275293 * z_{4} - 0.114807 * z_{5} \\
&- 0.180946 * z_{6} + 0.244544 * z_{8} + 0.897896 * z_{9} - 0.030989 * z_{11}
\end{align*}

将原始自变量的值代入上述主成分方程,可得主成分 $P_{1}^{*}$、$P_{2}^{*}$ 的值。

运用上述分析方法,经过对河南种植面积影响因素中与小麦种植面积存在相关关系的指标的主成分分析我们也可得其 $P_{1}'$、$P_{2}'$、$P_{3}'$ 三个主成分的值。以下为经整理的安徽省与河南省主成分的值:

\begin{table}[h]
\centering
\caption{表3-23 安徽省与河南省小麦主成分值}
\begin{tabular}{c c c c c c}
\hline
\multirow{2}{*}{年份} & \multicolumn{2}{c}{安徽省} & \multicolumn{3}{c}{河南省} \\
\cline{2-6}
& $P_{1}^{*}$ & $P_{2}^{*}$ & $P_{1}'$ & $P_{2}'$ & $P_{3}'$ \\
\hline
1995 & 3024.086719 & 1518.03896 & 3575.034424 & 7159.587879 & 2623.350566 \\
1996 & 3152.173749 & 1458.081213 & 3726.116641 & 7194.959055 & 2620.281944 \\
1997 & 3202.039437 & 1458.688005 & 3716.426207 & 7227.04509 & 2641.905311 \\
1998 & 3167.834218 & 1511.232836 & 3715.357964 & 7372.982172 & 2709.983059 \\
$\cdot$ & $\cdot$ & $\cdot$ & $\cdot$ & $\cdot$ & $\cdot$ \\
$\cdot$ & $\cdot$ & $\cdot$ & $\cdot$ & $\cdot$ & $\cdot$ \\
2011 & 5628.851261 & 55.85827209 & 5926.821923 & 7864.179976 & 2576.634212 \\
2012 & 5994.209238 & -257.3854001 & 6251.994743 & 7787.087217 & 2503.197859 \\
2013 & 5875.42204 & -152.9388413 & 6655.225299 & 7763.791034 & 2428.059129 \\
2014 & 8095.360107 & -1825.256236 & 7131.642646 & 7837.68603 & 2406.357753 \\
\hline
\end{tabular}
\end{table}

上表为安徽省与河南省小麦种植面积影响因素中与小麦种植面积存在相关关系的指标主成分分析结果。接下来,用主成分分析得到的变量 $P_{1}^{*}$、$P_{2}^{*}$,$P_{1}'$、$P_{2}'$、$P_{3}'$,作为混合线性模型中的随机效应。至此,随机效应已确定。

接下来,对所得的随机效应进行分级处理,安徽省变量分级处理的结果如表3-24所示。表3-24为所选取的安徽省随机效应变量的区间分级表,对于第一主成分根据其区间长度共分5级,第二主成分分为5级。河南省主成分的值按以上方法进行分级处理,分级结果见附件。

\begin{table}[h]
\centering
\caption{表3-24 安徽省变量分级表}
\begin{tabular}{c c c c c c}
\hline
变量名 & 区间 & 记号 & 变量名 & 区间 & 记号 \\
\hline
$P_{1}^{*}$ & $[3024.087, 4038.341]$ & 1 & $P_{2}^{*}$ & $[-1825.256, -1153.6]$ & 1 \\
$P_{1}^{*}$ & $[4038.341, 5052.596]$ & 2 & $P_{2}^{*}$ & $[-1153.6, -481.943]$ & 2 \\
$P_{1}^{*}$ & $[5052.596, 6066.851]$ & 3 & $P_{2}^{*}$ & $[-481.943, 189.714]$ & 3 \\
$P_{1}^{*}$ & $[6066.851, 7081.105]$ & 4 & $P_{2}^{*}$ & $[189.714, 861.371]$ & 4 \\
$P_{1}^{*}$ & $[7081.105, 8095.36]$ & 5 & $P_{2}^{*}$ & $[861.371, 1533.027]$ & 5 \\
\hline
\end{tabular}
\end{table}

\subsection{3.5.3 小麦最低收购价政策评价模型}

根据混合线性模型的形式,初步建立安徽省最低收购价政策评价模型的线性模型,如下:

\begin{equation}
y_5 = \varphi_0 + \varphi_1 P_1^* + \varphi_2 P_2^* + \mu s + e,
\end{equation}
其中,$\varphi_0$ 是常数项,$\varphi_i (i=1,2)$ 是随机效应变量 $P_i^* (i=1,2)$ 的系数;$\mu$ 是固定效应变量的系数;$s$ 是固定效应变量小麦的最低收购价政策,取值为 0 或 1。至此混合线性模型已经建立。

根据以上形式可得河南省小麦的混合线性模型如下所示:
\begin{equation}
y_4 = \varphi_0' + \varphi_1' P_1' + \varphi_2' P_2' + \varphi_3' P_3' + \mu' s + e'
\end{equation}
以上所建立的混合线性模型 $y_4$、$y_5$ 为河南省、安徽省小麦最低收购价政策评价模型。

\subsection{3.6 小麦最低收购价政策评价模型的求解与检验}

采用 SAS 的 MIXED 模块对 3.4 中所建立的安徽省和河南省小麦最低收购价政策评价模型计算,程序中采用最大似然估计算法,并用混合线性模型平方和对模型做假设检验。

\subsubsection{3.6.1 安徽省小麦最低收购价政策评价模型求解与检验}

采用 SAS 的 MIXED 模块进行计算,程序中采用最大似然估计算法,用混合线性模型平方和做假设检验,得到如下表所示:

\begin{table}[h]
\centering
\caption{表 3-25 安徽各变量的参数表}
\begin{tabular}{l c c c c}
\hline
Source & DF & Sum of Squares & Mean Square & F Value & Pr > F \\
\hline
Model & 3 & 727142.905 & 242380.968 & 11.85 & 0.0002 \\
Error & 16 & 327161.369 & 20447.586 & & \\
Corrected Total & 19 & 1054304.274 & & & \\
\hline
R-Square: 0.869690 & & & & & \\
\hline
\end{tabular}
\end{table}

由上表可知,$p=0.0002$,小于显著性水平 $\alpha=0.05$,说明安徽省小麦最低收购价政策评价模型通过检验,且 $R^2=0.869690$,说明模型的整体拟合效果较好。

混合模型输出有关协方差的各种统计量,如下表所示:

\begin{table}[h]
\centering
\caption{表 3-26 黑龙江不同协方差结构下的各种检验统计量}
\begin{tabular}{l c}
\hline
统计量 & 统计值 \\
\hline
$-2 \text{Res Log Likelihood}$ & 236.3 \\
AIC (smaller is better) & 242.3 \\
AICC (smaller is better) & 244.0 \\
BIC (smaller is better) & 236.3 \\
\hline
\end{tabular}
\end{table}

由上表我们可知,$-2 \text{Res Log Likelihood}$ 和 BIC (smaller is better) 的统计量的值为 236.3 是最小的,故这里选 $-2 \text{Res Log Likelihood}$ 作为最后协方差结构。用它的结构计算的固定效应的假设检验结果如下表所示:

\begin{table}[h]
\centering
\caption{表 3-27 安徽省固定效应的检验 (Type 3 Tests of Fixed)}
\begin{tabular}{l c c c}
\hline
Effect & Num DF & Den DF & F Value & Pr > F \\
\hline
a & 1 & 17 & 0.46 & 0.05077 \\
\hline
\end{tabular}
\end{table}

由上表可知,固定效应的假设检验结果,在显著性水平 $\alpha=0.1$ 下,模型为无效假设,$\chi^2=3.18, p=0.05077<0.1$,说明模型拟合效果是显著的,即小麦的最低收购价政策对小麦的种植面积有显著作用。

固定效应的系数估计结果如下表所示:

\begin{table}
\centering
\caption{安徽省固定效应的系数估计结果 (Least Squares Means)}
\begin{tabular}{ccccccc}
\hline
Effect & a & Estimate & Standard Error & DF & t Value & Pr > |t| \\
\hline
a & 0 & 2651.76 & 294.62 & 16 & 9.00 & $<$ .0001 \\
a & 1 & 2805.35 & 262.45 & 16 & 10.69 & $<$ .0001 \\
\hline
\end{tabular}
\end{table}

\section*{3.6.2 河南省小麦最低收购价政策评价模型求解与检验}

采用 SAS 的 MIXED 模块进行计算,程序中采用最大似然估计算法,用混合线性模型平方和做假设检验,得到如下表所示:

\begin{table}
\centering
\caption{河南省各变量的参数表}
\begin{tabular}{l c c c c}
\hline
Source & DF & Sum of Squares & Mean Square & F Value & Pr > F \\
\hline
Model & 2 & 90308.5240 & 45154.2620 & 1.18 & 0.03313 \\
Error & 17 & 650602.7963 & 38270.7527 & & \\
Corrected Total & 19 & 740911.3203 & & & \\
\hline
R-Square: 0.85121 & & & & & \\
\hline
\end{tabular}
\end{table}

由上表可知,$p=0.0460$,小于显著性水平 $\alpha=0.05$,说明黑龙江省小麦最低收购价政策评价模型通过检验,且 $R^2=0.884881$,说明模型的整体拟合效果很好。

混合模型输出有关协方差的各种统计量,如下表所示:

\begin{table}
\centering
\caption{河南省不同协方差结构下的各种检验统计量}
\begin{tabular}{l c}
\hline
统计量 & 统计值 \\
\hline
-2 Res Log Likelihood & 245.2 \\
AIC (smaller is better) & 247.2 \\
AICC (smaller is better) & 247.5 \\
BIC (smaller is better) & 245.2 \\
\hline
\end{tabular}
\end{table}

由上表我们可知,-2 Res Log Likelihood 和 BIC (smaller is better) 的统计量为 245.2,其值最小,故这里选 -2 Res Log Likelihood 作为最后协方差结构。用它的结构计算的固定效应的假设检验结果如下表所示:

\begin{table}
\centering
\caption{河南省固定效应的检验 (Type 3 Tests of Fixed)}
\begin{tabular}{l c c c}
\hline
Effect & Num DF & Den DF & F Value & Pr > F \\
\hline
a & 1 & 17 & 1.78 & 0.01995 \\
\hline
\end{tabular}
\end{table}

由上表可知,固定效应的假设检验结果,在显著性水平 $\alpha=0.1$ 下,模型为无效假设,$\chi^2=3.18, p=0.01995<0.1$,说明拟合的模型是显著的,即小麦的最低收购价政策对小麦的种植面积有显著作用。

固定效应的系数估计结果如下表所示:

\begin{table}
\centering
\caption{河南省固定效应的系数估计结果 (Least Squares Means)}
\begin{tabular}{c c c c c c}
\hline
Effect & a & Estimate & Standard Error & DF & t Value & Pr > |t| \\
\hline
a & 0 &  &  &  &  & \\
a & 1 &  &  &  &  & \\
\hline
\end{tabular}
\end{table}

\begin{table}[h]
\centering
\begin{tabular}{c c c c c c}
\hline
Effect & a & Estimate & Standard Error & DF & $t$ Value & $Pr > |t|$ \\
\hline
a & 0 & 3858.23 & 64.5099 & 17 & 59.81 & $<.0001$ \\
a & 1 & 3974.35 & 58.3514 & 17 & 68.11 & $<.0001$ \\
\hline
\end{tabular}
\end{table}

上表为河南省固定效应的系数估计效果,即系数越大,则效果越好。从上表中我们可以看到小麦的最低收购价政策实行前,小麦的种植面积的估计值为 3858.23 千公顷,小麦的最低收购价政策实行后,小麦的种植面积的估计值为 3974.35 千公顷。由此,我们可以知道小麦的最低收购价政策的实行可以促进河南省小麦的种植面积的增加,即小麦的最低收购价政策对河南省小麦的种植面积的影响是有效的。

\subsection*{3.7 各省小麦最低收购价政策效果比较}

经以上小麦最低收购价政策评价模型的建立与求解我们可得小麦最低收购价政策对安徽省和河南省小麦种植面积的影响效果。影响效果计算公式为:

\[
\frac{\text{实行政策后水稻种植面积} - \text{实行政策前水稻播种面积}}{\text{实行政策前水稻播种面积}} \times 100\%
\]

经计算结果如下所示:

\begin{table}[h]
\centering
\caption{两省小麦最低收购价政策效果比较}
\begin{tabular}{c c c}
\hline
省份 & 河南省 & 安徽省 \\
\hline
实行政策前 & 3858.23 & 2651.76 \\
实行政策后 & 3974.35 & 2805.35 \\
影响效果 & 0.030096702 & 0.057920023 \\
\hline
\end{tabular}
\end{table}

由上表我们可知小麦最低收购价政策对河南省、安徽省小麦种植面积的影响效果的值分别为 0.030096702、0.057920023,即实行政策后比实行政策前河南省、安徽省小麦种植面积分别增长 3\% 和 5.792\%。从结果可以看出两省影响效果为安徽省大于河南省。说明小麦最低收购价政策对于河南省的影响效果比安徽省的影响效果好。

\section*{4. 问题三解答——粮食供需及价格联动模型}

粮食是我国国民生活的必须品,其供给、需求、价格之间的关系几乎完全由市场决定。粮食供给、需求及价格三者本身分别是由多种经济、社会和文化方面因素构成的动态系统,他们之间既具有多层次性又具有动态演化特征。三个子系统之间在市场机制作用下具有高度相关性,因此使得粮食供给、需求及价格系统是个复杂的联动系统。本题通过研究三个子系统间的关系,建立价格联动模型,从而研究粮食价格所具有的特殊规律性。

\subsection*{4.1 数据来源及说明}

本题研究我国粮食价格所具有的规律性,由于粮食具有品种上的差异本题只研究水稻和小麦价格规律性。由于本题建立供给、需求及价格的联动模型,故首先选取影响我国水稻和小麦供给、需求及价格的因素。接下来,给出影响我国小麦和水稻供给、需求量、价格都是全国的数据。以水稻为例给出具体的数据,小麦的数据见附件。

我们所选取的影响水稻供给量的因素为:前一年供给量、水稻价格、进口量、单产水平、种植面积;影响水稻需求量的因素为:价格水平、居民人均消费水平、替代品(棉

\begin{table}
\caption{影响我国水稻供给量的指标(部分数据)}
\begin{tabular}{ccccccc}
\hline
年份 & 前一年供给量 (万吨) & 价格水平 (元/50公斤) & 进口量 (万吨) & 单产水平 (公斤/亩) & 种植面积 (千公顷) & 供给量 (万吨) \\
\hline
1995 & 19748.5 & 82.11 & 47 & 408.2 & 30744 & 18569.6 \\
1996 & 18569.6 & 80.62 & 76 & 415.8 & 31405.6 & 19586.27 \\
1997 & 19586.27 & 69.42 & 33 & 423.1 & 31764.87 & 20106.48 \\
1998 & 20106.48 & 66.92 & 24 & 421.9 & 31213.8 & 19895.3 \\
\hline
2011 & 19614.92 & 134.53 & 59.78 & 464.45 & 30057.04 & 20159.87 \\
2012 & 20159.87 & 138.07 & 236.86 & 478.75 & 30137.11 & 20660.45 \\
2013 & 20660.45 & 136.52 & 227.11 & 471.67 & 30311.75 & 20588.33 \\
2014 & 20588.33 & 140.63 & 257.9 & 484.98 & 30309.87 & 20908.64 \\
\hline
\end{tabular}
\end{table}

数据来源:中华人民共和国国家统计局、全国农产品成本收益资料汇编

\begin{table}
\caption{影响我国水稻供给量的指标(部分数据)}
\begin{tabular}{ccccccc}
\hline
年份 & 价格水平 (元/50公斤) & 居民人均消费水平 (元) & 替代品价格水平 (公斤/亩) & 出口水平金额 (百万美元) & 城镇化率 万吨 & 需求量 (万吨) \\
\hline
1995 & 82.11 & 2330 & 418.35 & 16.24 & 0.2904 & 16338.772 \\
1996 & 80.62 & 2765 & 262.66 & 111.64 & 0.3048 & 17684.061 \\
1997 & 69.42 & 2978 & 302.53 & 264.57 & 0.3191 & 17388.659 \\
1998 & 66.92 & 3126 & 194.15 & 924.03 & 0.3335 & 18303.403 \\
\hline
2011 & 134.53 & 13134 & 202.49 & 426.98 & 0.5127 & 20046.244 \\
2012 & 138.07 & 14699 & 25.26 & 272.13 & 0.5257 & 19740.12 \\
2013 & 136.52 & 16190 & -214.98 & 416.74 & 0.537 & 19640.110 \\
2014 & 140.63 & 17778 & -686.44 & 378.4 & 0.5477 & 19645.012 \\
\hline
\end{tabular}
\end{table}

数据来源:中华人民共和国国家统计局、全国农产品成本收益资料汇编

\begin{table}
\caption{影响我国水稻价格的指标(部分数据)}
\begin{tabular}{cccccc}
\hline
年份 & 供给量 (万吨) & 前一年价格水平 (元/50公斤) & 进口贸易额 (百万美元) & 出口贸易额 (百万美元) & 价格水平 (元/50公斤) \\
\hline
1995 & 18569.6 & 71.17 & 177.1590789 & 16.24 & 82.11 \\
1996 & 19586.27 & 82.11 & 286.47 & 111.64 & 80.62 \\
1997 & 20106.48 & 80.62 & 139.77 & 264.57 & 69.42 \\
\hline
\end{tabular}
\end{table}

\begin{tabular}{c c c c c}
\hline
1998 & 19895.3 & 69.42 & 120.04 & 924.03 & 66.92 \\
\hline
$\cdot$ & $\cdot$ & $\cdot$ & $\cdot$ & $\cdot$ & $\cdot$ \\
$\cdot$ & $\cdot$ & $\cdot$ & $\cdot$ & $\cdot$ & $\cdot$ \\
$\cdot$ & $\cdot$ & $\cdot$ & $\cdot$ & $\cdot$ & $\cdot$ \\
2011 & 20159.87 & 118 & 407.64 & 426.98 & 134.53 \\
2012 & 20660.45 & 134.53 & 1153.35 & 272.13 & 138.07 \\
2013 & 20588.33 & 138.07 & 1083.03 & 416.74 & 136.52 \\
2014 & 20908.64 & 136.52 & 1254.24 & 378.4 & 140.63 \\
\hline
\end{tabular}

数据来源:中华人民共和国国家统计局、全国农产品成本收益资料汇编

以上三个表为影响我国水稻供给量、需求量、价格的指标 1994-2015 时间段内的数据,接下来对水稻的研究都以此数据为基础。

\subsection*{4.2 粮食供需及价格联动模型建立}

\subsubsection{4.2.1 广义矩估计方法(GMM)}

广义矩估计方法是基于模型实际参数满足的一些矩条件而形成的一种参数估计方法,是矩估计方法的一般化。如果模型的设定是正确的,则总能找到该模型实际参数满足的若干矩条件而采用 GMM。由于传统的计量经济学模型估计方法,例如普通最小二乘法、工具变量法、极大似然法等,都有它们的局限性,其参数估计量必须在模型满足某些假设时才具有良好的性质,诸如只有当模型的随机误差项服从正态分布或某一已知分布,极大似然参数估计量才是可靠的估计量;而 GMM 允许随机误差项存在异方差和序列相关,所得到的参数估计量比其它参数估计方法更合乎实际;而且可以证明,GMM 包容了许多常用的估计方法,普通最小二乘法、工具变量法、极大似然法都是它的特例。所以,GMM 具有其优越性而得到应用。

\subsubsection{4.2.2 粮食供需及价格联动模型系统的设计}

\subsubsection{4.2.2.1 粮食供给模型}

在粮食供求系统中,粮食生产者会根据粮食过去的需求和价格来决定当期的供给;粮食消费者会根据粮食过去的价格来决定当期的心理价格,同时粮食当期价格也会受其过去价格的影响,因此构建粮食供需及价格联动模型有必要引入适应性预期理论。

市场经济条件下粮食生产者按照利润最大化原则来决定产品供给,价格是影响利润的关键因素,生产者根据对当期价格的预测来安排生产和供给,而按照适应性预期理论对当期价格的预测取决于对过去价格水平及对过去价格预期的误差。此外,种植面积、粮食进口量、单产水平等因素的影响。因此设计粮食供给模型具体形式如下:

\[
G_{t}^{s} = \lambda_{0} + \lambda_{1} G_{t-1}^{s} + \lambda_{2} P_{t-1} + \lambda_{3} IM_{t} + \lambda_{4} UP_{t} + \lambda_{5} PR_{t} + \varepsilon_{t}
\]

其中 $G_{t}^{s}$、$G_{t-1}^{s}$ 分别为 $t$ 和 $t-1$ 时期的供给量,$P_{t-1}$ 为 $t-1$ 时期的价格水平,$IM_{t}$、$UP_{t}$、$PR_{t}$ 分别为 $t$ 时期的进口量、单产水平以及种植面积,$\varepsilon_{t}$ 为随机误差项。

\subsubsection{4.2.2.2 粮食需求模型}

影响粮食需求的因素有很多,主要分为社会因素和经济因素,其中社会因素包括人口数量及其增长、城镇化水平等;经济因素主要有粮食价格、出口水平、居民人均消费水平以及替代品价格等,设计粮食需求模型具体形式如下:

\[
G_{t}^{d} = \alpha_{0} + \alpha_{1} P_{t} + \alpha_{2} CU_{t} + \alpha_{3} SP_{t} + \alpha_{4} EX_{t} + \alpha_{5} URBAN + \nu_{t}
\]

其中 $CU_{t}$、$URBAN_{t}$ 分别为 $t$ 时期的居民人均消费水平以及城镇化水平,$\nu_{t}$ 为随机误差

\section*{4.2.2.3 粮食价格模型}

粮食供需系统中,供给与需求之间的相互作用通过价格这个中介变量产生。按照价格理论,粮食价格主要由粮食的供需水平决定。然而,除了供需之外,过去的价格水平、进口水平和出口水平、替代品价格、粮食补贴政策也会影响粮食价格形成。(说一下粮食补贴)故设计粮食价格模型具体形式如下:
\[
P_{t} = \beta_{0} + \beta_{1} G_{t}^{s} + \beta_{2} P_{t-1} + \beta_{3} IN_{t} + \beta_{4} EX_{t} + \beta_{5} SP_{t} + \beta_{6} PO_{t} + \mu_{t}
\]
其中 $P_{t}$、$P_{t-1}$ 分别为 $t$ 和 $t-1$ 时期的粮食价格水平,$IN_{t}$、$EX_{t}$、$SP_{t}$、$PO_{t}$ 分别为 $t$ 时期的进口水平、出口水平、替代品价格水平以及最低粮食收购价,$\mu_{t}$ 为随机误差项。

从粮食供需与价格模型的构建可以看出,粮食供给、需求及价格之间相互影响,再由这个三个模型组成的供需及价格联动系统模型中,粮食供给、需求与价格三个变量内生于系统,有多个不同因素影响它们,并且彼此之间也是相互影响和制约的。

\section*{4.3 粮食供需平衡价格联动模型求解}

首先根据影响粮食供给、价格需求的影响因素应用最小二乘回归,得到粮食价格形成的系统模型。然后利用粮食供给、价格、需求组合成的模型系统分别对水稻和小麦供需及价格进行估计,同时对影响水稻、小麦的供给量、需求量及价格演化的影响因素进行分析。进而得出水稻、小麦供给量、需求量及价格之间的相互关系。

\subsection*{4.3.1 水稻供需平衡价格联动模型}

利用广义矩估计法对水稻价格形成系统模型进行估计,得到水稻总体价格形成系统模型估计结果如表 4-4 所示。

\begin{table}[h]
\centering
\caption{表 4-4 水稻价格形成系统模型估计}
\begin{tabular}{l l l l l l}
\hline
\multirow{2}{*}{模型} & \multirow{2}{*}{被解释变量} & 解释变量 & 参数估计 & T 检验值 & P 值 & $R^{2}$ \\
\cline{3-7}
 & & & & & & \\
\hline
\multirow{4}{*}{水稻供给模型} & \multirow{4}{*}{供给量} & 前一期供给量 & 0.16 & 3.99 & 0 & \\
 & & 前一期价格水平 & 1.03 & 0.21 & 0.03 & \\
 & & 进口量 & 0.07 & 0.14 & 0.09 & 0.93 \\
 & & 每亩单产水平 & 25.39 & 4.02 & 0 & \\
 & & 种植面积 & 0.65 & 14.83 & 0 & \\
\hline
\multirow{6}{*}{水稻价格模型} & \multirow{6}{*}{价格} & 供给量 & -0.02 & 0.99 & 0.02 & \\
 & & 前一期价格水平 & 0.68 & 2.92 & 0.003 & \\
 & & 进口贸易额 & -0.009 & -0.59 & 0.55 & \\
 & & 出口贸易额 & -0.017 & -1.33 & 0.18 & 0.96 \\
 & & 替代品价格 & 0.0213 & 2.69 & 0.007 & \\
 & & 最低粮食收购价 & 0.003 & 0.017 & 0.08 & \\
\hline
\multirow{5}{*}{水稻需求模型} & \multirow{5}{*}{需求量} & 价格水平 & -45.09 & 2.789 & 0 & \\
 & & 居民人均消费 & -0.026 & -0.15 & 0.088 & \\
 & & 替代品价格 & 0.44 & -0.82 & 0.041 & 0.94 \\
 & & 出口贸易额 & 2.87 & 6.61 & 0 & \\
 & & 城镇化水平 & 3821 & -0.67 & 0.051 & \\
\hline
\end{tabular}
\end{table}

从表水稻价格形成系统模型参数估计结果可以看到,运用系统估计方程同时对水稻总量供给、价格和需求三个单方程模型进行估计后,三个单方程的 $R^{2}$ 值分别为 0.93、

0.96、0.94,说明整个水稻总量供需及价格系统模型拟合较好. 此外,在显著性水平为 0.1 的条件下,价格模型中水稻进口贸易额、出口贸易额这两个变量对其被解释变量的影响不显著。

在供给模型中,水稻的价格水平对供给量的影响显著且为正,说明水稻生产者倾向于根据水稻价格安排生产计划,如果当前一期价格升高,就会增加产量和供给,反之则会减少产量和供给。进口量及每亩单产水平对供给量的影响也是显著的并且为正相关的,说明水稻的进口量逐年增加,已构成我国水稻供给量的重要组成部分;每亩单产水平逐年提高,已成为我国水稻供给的重要来源。种植面积对水稻供给量的影响显著且为正,说明种植面积的增加势必会促进供给量的增加。

在价格模型中,进口贸易额和出口贸易额对水稻价格影响不显著,这是因为近几年水稻的进口量和出口量占水稻产量的比重比较小,故而不能影响水稻的价格。水稻的供给量对水稻的价格影响显著且为负,说明水稻供给量的增加,自然会导致水稻价格的下降。水稻替代品对水稻的价格影响显著且为正,说明随着水稻替代品价格的增加,商家为获取更高的利润,会相应提高水稻的价格。最低粮食收购价对水稻的价格影响显著且为正,由于水稻的价格一定高于最低粮食收购价格,故随着粮食收购价的提高,水稻的价格也一定会提高。

在需求模型中,价格水平对水稻需求的影响显著且为负,可想而知,水稻的价格水平越高,居民对水稻的需求就会减少,因此价格水平会对水稻需求有负影响。居民人均消费对水稻需求的影响显著且为负,说明随着居民人均消费水平的提高,居民可以有更多的选择,这样对水稻的需求比重就会降低。替代品价格对水稻需求的影响显著且为负,说明替代品的价格越高,人们越倾向于购买价格相对较低的水稻,因而对水稻的需求就会增大。出口贸易额对水稻需求的影响显著且为正,出口贸易额已成为水稻需求的重要组成部分,出口贸易额越大,说明水稻的需求量越大。城镇化水平对水稻需求的影响显著且为正,城镇化水平越高,水稻的种植面积就会越少,水稻的产量就会降低,因而就会刺激水稻的需求量增加。

从以上分析可以看出,水稻供给、价格和需求分别受不同外生变量的影响,同时它们三者之间又是相互影响、相互制约、共同演化的动态关系。

\subsection*{4.3.2 小麦供需平衡价格联动模型}

从表 4-5 可知,小麦价格形成系统模型参数估计,运用系统估计方程同时对小麦总量供给、价格和需求三个单方程模型进行估计后,三个单方程的 \(R^{2}\) 值分别为 0.95、0.98、0.99,说明整个小麦总量供需及价格系统模型拟合较好。此外,在显著性水平为 0.1 的条件下,价格模型中小麦进口贸易额、出口贸易额这两个变量对其被解释变量的影响不显著。

\begin{table}[h]
\centering
\caption{表 4-5 小麦价格形成系统模型估计}
\begin{tabular}{c c c c c c}
\hline
模型 & 被解释变量 & 解释变量 & 参数估计 & T 检验值 & P 值 & \(R^{2}\) \\
\hline
小麦供给 & 供给量 & 前一期供给量 & 0.22 & -1.37 & 0.0172 & \\
模型 & & 价格水平 & 41.05 & 4.63 & 0 & \\
& & 进口量 & 0.24 & -1.12 & 0.026 & 0.95 \\
& & 每亩单产水平 & 9.8 & 3.58 & 0 & \\
& & 种植面积 & 0.41 & 5.61 & 0 & \\
\hline
小麦价格 & 供给量 & -0.0003 & 0.33 & 0.0743 & & 0.98 \\
& 前一期价格水平 & 0.42 & 3.8 & 0 & & \\
\hline
\end{tabular}
\end{table}

\begin{table}
\centering
\begin{tabular}{l l r r r}
\hline
模型 & 进口水平 & -0.009 & 6.60 & 0.36 \\
价格 & 出口水平 & -0.003 & -0.65 & 0.51 \\
 & 替代品价格 & 0.006 & 1.59 & 0.011 \\
 & 最低粮食收购价格 & 0.22682 & 5.4324 & 0 \\
 & 价格水平 & -30.571 & 3.29 & 0 \\
小麦需求 & 居民人均消费水平 & -0.101 & 1.08 & 0.028 \\
模型 & 需求量 & 替代品价格 & -0.39 & -1.3 \\
 & & & 0.019 & 0.99 \\
 & & 出口水平 & 0.16 & 0.22 \\
 & & & 0.082 & \\
 & & 城镇化水平 & 1487.1 & -0.38 \\
 & & & 0.071 & \\
\hline
\end{tabular}
\end{table}

在供给模型中,小麦的价格水平对供给量的影响显著且为正,说明小麦生产者倾向于根据小麦价格安排生产计划,如果当前一期价格升高,就会增加产量和供给,反之则会减少产量和供给。进口量及每亩单产水平对供给量的影响也是显著的并且为正相关的,说明小麦的进口量逐年增加,已构成我国小麦供给量的重要组成部分;每亩单产水平逐年提高,已成为我国小麦供给的重要来源。种植面积对小麦供给量的影响显著且为正,说明种植面积的增加势必会促进供给量的增加。

在价格模型中,进口贸易额和出口贸易额对小麦价格影响不显著,这是因为近几年小麦的进口量和出口量占小麦产量的比重比较小,故而不能影响小麦的价格。小麦的供给量对小麦的价格影响显著且为负,说明小麦供给量的增加,自然会导致小麦价格的下降。小麦替代品对小麦的价格影响显著且为正,说明随着小麦替代品价格的增加,商家为获取更高的利润,会相应提高小麦的价格。最低粮食收购价对小麦的价格影响显著且为正,由于小麦的价格一定高于最低粮食收购价格,故随着粮食收购价的提高,小麦的价格也一定会提高。

在需求模型中,价格水平对小麦需求的影响显著且为负,可想而知,小麦的价格水平越高,居民对小麦的需求就会减少,因此价格水平会对小麦需求有负影响。居民人均消费对小麦需求的影响显著且为负,说明随着居民人均消费水平的提高,居民可以有更多的选择,这样对小麦的需求比重就会降低。替代品价格对小麦需求的影响显著且为负,说明替代品的价格越高,人们越倾向于购买价格相对较低的小麦,因而对小麦的需求就会增大。出口贸易额对小麦需求的影响显著且为正,出口贸易额已成为小麦需求的重要组成部分,出口贸易额越大,说明小麦的需求量越大。城镇化水平对小麦需求的影响显著且为正,城镇化水平越高,小麦的种植面积就会越少,小麦的产量就会降低,因而就会刺激小麦的需求量增加。

从以上分析可以看出,小麦供给、价格和需求分别受不同外生变量的影响,同时它们三者之间又是相互影响、相互制约、共同演化的动态关系。

\subsection*{4.4 粮食供需平衡价格联动模型检验}

\subsubsection{4.4.1 水稻供需平衡价格联动模型检验}

\subsubsection{4.4.1.1 ARIMA 模型预测影响水稻价格的因素}

运用 ARIMA 模型,按照 BIC 最小化准则同时构造 Ljung-Box Q 统计量,对 2015 年至 2020 年影响水稻供给量和需求量的居民人均消费水平、水稻出口贸易额、水稻进口贸易额、水稻单产水平、替代品价格水平、水稻种植面积、进口量和城镇化水平进行合理预测如下表所示。

\begin{table}
\centering
\caption{影响水稻总体价格形成的各因素预测值}
\begin{tabular}{c c c c c c c}
\hline
居民人均 & 水稻进口 & 水稻出口 & 水稻单产 & 替代品价 & 水稻种 & 进口量 \\
 & & & & & & 城镇化 \\
\hline
\end{tabular}
\end{table}

\begin{table}
\centering
\begin{tabular}{c c c c c c c c}
\hline
消费水平 & 贸易额 & 贸易额 & 水平 & 格水平 & 植面积 & (万吨) & 水平 \\
(元) & (百万美元) & (百万美元) & (公斤/亩) & (元/50公斤) & (千公顷) & & \\
\hline
2015 & 17340 & 428.174 & 377.994 & 11970.2 & 361.35 & 29738.3 & 132.77 \\
 & & & & & & & 17340 \\
2016 & 17411.3 & 263.11 & 396.852 & 12069.7 & 321.58 & 29737.4 & 133.95 \\
 & & & & & & & 17411.3 \\
2017 & 17483.2 & 198.283 & 400.806 & 12188.5 & 314.86 & 29727.2 & 135.18 \\
 & & & & & & & 17483.2 \\
2018 & 17555.9 & 67.5365 & 398.646 & 12330.3 & 357.1 & 29719.4 & 136.46 \\
 & & & & & & & 17555.9 \\
2019 & 17629.2 & 316.98 & 400.899 & 12499.5 & 452.69 & 29764.1 & 137.79 \\
 & & & & & & & 17629.2 \\
2020 & 17703.2 & 636.833 & 398.697 & 12701.6 & 540 & 29863.3 & 139.18 \\
 & & & & & & & 17703.2 \\
ARIMA & (1,1,0) & (1,1,2) & (1,0,1) & (1,1,1) & (1,0,1) & (1,1,2) & (1,1,0) \\
LB统计量 & 56.12 & 5.07 & 5.77 & 8.73 & 4.07 & 44.81 & 3.32 \\
计量 & 0.175 & 0.427 & 0.588 & 0.681 & 0.436 & 0.725 & 0.194 \\
 & & & & & & & 0.681 \\
\hline
\end{tabular}
\end{table}

\textbf{注:} ARIMA(1, 1, 1) 表示一阶自回归、一阶差分和一阶移动平均项的 ARIMA 模型;“LB 统计量”行中括号中数值表示相应统计量的显著水平。

\section*{4.4.1.2 水稻供需平衡价格联动系统预测值检验}

价格是联结供给和需求的中间变量,在给定其它因素条件下,通过价格变量可以预测出供给和需求变量的走势,同样,也可以通过供给和需求变量预测出价格变量的走势。价格的预测性能和预测精度可以反映出整个供需系统预测的合理性。因此,我们在给出各因素的预测值的基础上,利用价格模型对价格进行预测,得到新的价格预测值,将通过用 ARIMA 模型预测得到的价格预测值与系统得到的价格预测值进行对比,来验证价格系统的合理性和两种方法性能的一致性。两种方法预测出来的水稻价格如下表所示。

\begin{table}
\centering
\begin{tabular}{c c c c c}
\hline
ARIMA 预测值 & 系统预测值 & 预测误差率 & 预测精度 \\
\hline
2015 & 132.8 & 125.11 & 5.78 & 94.22 \\
2016 & 134.0 & 126.54 & 5.54 & 94.46 \\
2017 & 135.2 & 127.91 & 5.38 & 94.62 \\
2018 & 139.2 & 132.11 & 5.08 & 94.92 \\
2019 & 136.5 & 130.86 & 4.11 & 95.89 \\
2020 & 137.8 & 131.73 & 4.4 & 95.6 \\
\hline
\end{tabular}
\end{table}

从上表可知,通过运用 ARIMA 预测法和系统模型预测法得到的从 2015 年至 2020 年水稻价格的两种预测值序列的差异并不大,各年份两种预测方法的预测误差率均低于 10%,预测精度均高于 90%。

\section*{4.4.2 小麦供需平衡价格联动模型检验}

\subsection*{4.4.2.1 ARIMA 模型预测影响小麦价格的因素}

运用 ARIMA 模型,按照 BIC 最小化准则同时构造 Ljung-Box Q 统计量,对 2015 年至 2020 年影响小麦供给量和需求量的居民人均消费水平、小麦出口贸易额、小麦进口贸易额、小麦单产水平、替代品价格水平、小麦种植面积、进口量和城镇化水平进行合 -42-

\begin{table}
\centering
\caption{影响小麦总体价格形成的各因素预测值}
\begin{tabular}{c c c c c c c c}
\hline
 & 居民人均 & 小麦进口 & 小麦出口 & 小麦单产 & 替代品价格 & 小麦种植 & 城镇化 \\
 & 消费水平 & 贸易额 & 贸易额 & 水平 & 水平 & 面积(千 & 水平 \\
 & (元) & (百万美元) & (百万美元) & (公斤/亩) & (元/50公斤) & 公顷) & \\
\hline
2015 & 17993.8 & 600.317 & 124.63 & 390.46 & 361.35 & 26932.6 & 250.70 & 0.6047 \\
2016 & 18126.8 & 615.63 & 116.77 & 409.78 & 321.58 & 26436.2 & 266.64 & 0.5967 \\
2017 & 18261.9 & 549.817 & 103.4 & 423.01 & 314.86 & 25883.2 & 243.52 & 0.5881 \\
2018 & 18399.2 & 558.765 & 97.3 & 401.25 & 357.1 & 25306.9 & 259.1 & 0.5789 \\
2019 & 18538.8 & 467.8 & 121.1 & 419.17 & 452.69 & 24764.1 & 232.0 & 0.5691 \\
2020 & 18623.3 & 615.33 & 172.5 & 430.07 & 540 & 24321.6 & 267.2 & 0.5587 \\
ARIMA & (1,1,1) & (1,0,1) & (1,1,0) & (2,1,0) & (1,0,1) & (1,1,2) & (1,1,2) & (1,1,0) \\
LB统 & 18.58 & 11.74 & 6.05 & 6.60 & 4.07 & 2.97 & 11.75 & 3.32 \\
计量 & 0.325 & 0.618 & 0.139 & 0.1945 & 0.436 & 0.849 & 0.577 & 0.683 \\
\hline
\end{tabular}
\end{table}

\begin{table}
\centering
\caption{小麦总体价格形成系统模型适用性检验}
\begin{tabular}{c c c c c}
\hline
 & ARIMA预测值 & 系统预测值 & 预测误差率 & 预测精度 \\
\hline
2015 & 113.6 & 103.78 & 8.64 & 91.36 \\
2016 & 114.6 & 104.82 & 8.56 & 91.44 \\
2017 & 115.7 & 105.58 & 8.76 & 91.24 \\
2018 & 116.9 & 107.55 & 7.96 & 92.04 \\
2019 & 118.0 & 109.13 & 7.55 & 92.45 \\
2020 & 119.3 & 113.05 & 5.22 & 94.78 \\
\hline
\end{tabular}
\end{table}

经过以上分析研究我们可得,我国粮食价格、粮食供给量及粮食需求量之间存在相互制约的动态关系,并且它们还同时受多种因素的影响,如前一年粮食价格水平、进口贸易额、出口贸易额、替代品价格水平、最低粮食收购价、城镇化率、居民人均消费水平等。构建供需平衡价格联动模型,并运用广义据估计法对该系统进行估计,经研究表明前一年粮食价格水平、进口贸易额、出口贸易额、替代品价格水平、最低粮食收购会对价格形成直接的影响,而居民人均消费水平、城镇化率会通过需求间接对价格产生影响,单产水平会通过供给间接对价格形成影响。

\section{问题四——对粮食最低收购价的评价和预测}

\subsection{粮食最低收购价的合理定价模型的建立}

国家制定粮食最低收购价政策,旨在保证粮农种植粮食不亏损,提高粮农的种植积极性,从而增加粮食的种植面积;同时还可以平衡粮食的市场价格,保证国家粮食库存充足,因而,建立一个粮食最低收购价的合理定价模型显得尤为重要。在问题三的研究中,我们建立了粮食供需与价格模型的联动系统,粮食供给、需求及价格之间相互影响,根据三者之间的关系,我们可以建立粮食最低收购价格的合理定价模型:

\subsubsection{水稻最低收购价的合理定价优化模型的建立}

\paragraph{水稻优化目标函数确立}

在问题三的研究中,我们得到了水稻供给模型:
\[
G_{t}^{S}=0.16G_{t-1}^{S}+1.03P_{t-1}+0.07IM_{t}+25.39UP_{t}+0.65PR_{t}
\]
研究了水稻的供给量和前期水稻供给量、水稻前期的价格水平、水稻的进口量、水稻的单产量、水稻的种植面积的关系,为了研究水稻最低收购价的合理定价问题,我们将在给定水稻最低收购价时获得的最大水稻种植面积作为目标,对问题三中水稻供给模型变形,得到如下目标函数:
\[
\max PR_{t}=1.538G_{t}^{S}-0.246G_{t-1}^{S}-1.58P_{t-1}-0.107IM_{t}-39.06UP_{t}
\]
其中,$PR_{t}$ 表示水稻的 $t$ 期种植面积,$G_{t}^{S}$ 表示水稻的 $t$ 期供给量,$P_{t-1}$ 表示水稻的 $t-1$ 期价格水平,$IM_{t}$ 表示水稻的 $t$ 期进口量,$UP_{t}$ 表示水稻的 $t$ 期单产量。

\paragraph{水稻优化模型约束条件}

\textbf{约束条件 1:}

在问题三中,我们得到了水稻的价格模型:
\[
P_{t}=-0.006G_{t}^{S}+0.67P_{t-1}+0.0015IN_{t}-0.0087EX_{t}+0.022SP_{t}+0.137PO_{t}
\]
其中,$P_{t}$ 表示水稻的 $t$ 期价格水平,$G_{t}^{S}$ 表示水稻的 $t$ 期供给量,$P_{t-1}$ 表示水稻的 $t-1$ 期价格水平,$IN_{t}$ 表示水稻的 $t$ 期进口贸易额,$EX_{t}$ 表示水稻的 $t$ 期出口贸易额,$SP_{t}$ 表示水稻的 $t$ 期替代品价格,$PQ_{t}$ 表示水稻的 $t$ 期水稻的最低收购价。用此模型可以反映水稻最低收购价与水稻价格水平的关系,因此我们将此模型作为一个约束条件。

\textbf{约束条件 2:}

在问题三中,我们得到了水稻的需求模型:
\[
G_{t}^{d}=65.5P_{t}-1.48CU_{t}-0.488SP_{t}+2.97EX_{t}-111.4URBAN
\]
其中,$G_{t}^{d}$ 表示水稻的 $t$ 期需求量,$P_{t}$ 表示水稻的 $t$ 期价格水平,$CU_{t}$ 表示 $t$ 期居民人均消费,$SP_{t}$ 表示水稻的 $t$ 期替代品价格,$EX_{t}$ 表示水稻的 $t$ 期出口贸易额,$URBAN$ 表示城镇化水平。用此模型可以反映水稻的需求量与水稻价格的关系,因此我们可以将此模型

作为一个约束条件。

约束条件 3:

随着城镇化水平的不断提高,粮食种植面积显著减少,因而水稻的种植面积也显著减少,故水稻的种植面积不会超过历史上某年最大的水稻种植面积,即得到如下约束条件:
\[
1.538G_{t}^{s}-0.246G_{t-1}^{s}-1.58P_{t-1}-0.107IM_{t}-39.06UP_{t} \leq 31764.8
\]
其中,$G_{t}^{s}$ 表示水稻的 $t$ 期供给量,$P_{t-1}$ 表示水稻的 $t-1$ 期价格水平,$IM_{t}$ 表示水稻的 $t$ 期进口量,$UP_{t}$ 表示水稻的 $t$ 期单产量。

约束条件 4:

由于国家对于粮食的补贴金额是有限制的,在保持合理库存的前提下,一般不会超出各地粮食市场价格的 10%,因此可以得到如下约束条件:
\[
P_{t}-PO_{t} < 0.1*P_{t}
\]

约束条件 5:由于各变量的取值均需大于零,故得约束条件:
\[
G_{t}^{s} > 0, P_{t-1} > 0, IM_{t} > 0, UP_{t} > 0, P_{t} > 0, SP_{t} > 0,
\]
\[
PQ_{t} > 0, G_{t}^{d} > 0, CU_{t} > 0, URBAN > 0
\]

从而我们建立了水稻的最低收购价的合理定价模型:
\[
\max PR_{t} = 1.538G_{t}^{s}-0.246G_{t-1}^{s}-1.58P_{t-1}-0.107IM_{t}-39.06UP_{t}
\]
\[
s.t
\begin{cases}
P_{t} = -0.006G_{t}^{s}+0.67P_{t-1}+0.0015IN_{t}-0.0087EX_{t}+0.022SP_{t}+0.137PO_{t} \\
G_{t}^{d} = 65.5P_{t}-1.48CU_{t}-0.488SP_{t}+2.97EX_{t}-111.4URBAN \\
1.538G_{t}^{s}-0.246G_{t-1}^{s}-1.58P_{t-1}-0.107IM_{t}-39.06UP_{t} \leq 31764.8 \\
P_{t}-PO_{t} < 0.1*P_{t} \\
G_{t}^{s} > 0, P_{t-1} > 0, IM_{t} > 0, UP_{t} > 0, P_{t} > 0, SP_{t} > 0, \\
PQ_{t} > 0, G_{t}^{d} > 0, CU_{t} > 0, URBAN > 0
\end{cases}
\]

5.1.2 小麦最低收购价的合理定价模型的建立

5.1.2.1 小麦优化模型目标函数确立

在问题三的研究中,我们得到了小麦供给模型:
\[
G_{t}^{s} = -0.22G_{t-1}^{s}+41.05P_{t-1}-0.24IM_{t}+9.8UP_{t}+0.41PR_{t}
\]
研究了小麦的供给量和前期小麦供给量、小麦前期的价格水平、小麦的进口量、小麦的单产量、小麦的种植面积的关系,为了研究小麦最低收购价的合理定价问题,我们将在给定小麦最低收购价时获得的最大小麦种植面积作为目标,对问题三中小麦供给模型变形,得到如下目标函数:
\[
\max PR_{t} = 2.44G_{t}^{s}+0.54G_{t-1}^{s}+100.12P_{t-1}+0.59IM_{t}-23.9UP_{t}
\]
其中,$PR_{t}$ 表示小麦的 $t$ 期种植面积,$G_{t}^{s}$ 表示小麦的 $t$ 期供给量,$P_{t-1}$ 表示小麦的 $t-1$ 期价格水平,$IM_{t}$ 表示小麦的 $t$ 期进口量,$UP_{t}$ 表示小麦的 $t$ 期单产量。

5.1.2.2 小麦优化模型约束条件

约束条件 1:

在问题三中,我们得到了小麦的价格模型:
\[
P_{t} = -0.0003G_{t}^{s}+0.42P_{t-1}+0.009IN_{t}-0.003EX_{t}+0.006SP_{t}+0.23PO_{t}
\]

其中,$P_{t}$ 表示小麦的 $t$ 期价格水平,$G_{t}^{s}$ 表示小麦的 $t$ 期供给量,$P_{t-1}$ 表示小麦的 $t-1$ 期价格水平,$IN_{t}$ 表示小麦的 $t$ 期进口贸易额,$EX_{t}$ 表示小麦的 $t$ 期出口贸易额,$SP_{t}$ 表示小麦的 $t$ 期替代品价格,$PQ_{t}$ 表示小麦的 $t$ 期小麦的最低收购价。用此模型可以反映小麦最低收购价与小麦价格水平的关系,因此我们将此模型作为一个约束条件。

**约束条件 2:**

在问题三种,我们得到了小麦的需求模型:
\[
G_{t}^{d}=30.57P_{t}+0.101CU_{t}-0.39SP_{t}+0.16EX_{t}-1487.1URBAN
\]
其中,$G_{t}^{d}$ 表示小麦的 $t$ 期需求量,$P_{t}$ 表示小麦的 $t$ 期价格水平,$CU_{t}$ 表示 $t$ 期居民人均消费,$SP_{t}$ 表示小麦的 $t$ 期替代品价格,$EX_{t}$ 表示小麦的 $t$ 期出口贸易额,$URBAN$ 表示城镇化水平。用此模型可以反映小麦的需求量与小麦价格的关系,因此我们可以将此模型作为一个约束条件。

**约束条件 3:**

随着城镇化水平的不断提高,粮食种植面积显著减少,因而小麦的种植面积也显著减少,故小麦的种植面积不会超过历史上某年最大的小麦种植面积,即得到如下约束条件:
\[
2.44G_{t}^{s}+0.54G_{t-1}^{s}+100.12P_{t-1}+0.59IM_{t}-23.9UP_{t}\leq 30056.69
\]
其中,$G_{t}^{s}$ 表示小麦的 $t$ 期供给量,$P_{t-1}$ 表示小麦的 $t-1$ 期价格水平,$IM_{t}$ 表示小麦的 $t$ 期进口量,$UP_{t}$ 表示小麦的 $t$ 期单产量。

**约束条件 4:**

由于国家对于粮食的补贴金额是有限制的,在保持合理库存的前提下,一般不会超出各地粮食市场价格的 $10\%$,因此可以得到如下约束条件:
\[
P_{t}-PO_{t}<0.1*P_{t}
\]

**约束条件 5:**

由于各变量的取值均需大于零,故得约束条件:
\[
G_{t}^{s}>0, P_{t-1}>0, IM_{t}>0, UP_{t}>0, P_{t}>0, SP_{t}>0,
\]
\[
PQ_{t}>0, G_{t}^{d}>0, CU_{t}>0, URBAN>0
\]

### 5.1.2 水稻最低收购价的合理定价优化模型

经以上的目标函数及约束条件而我们建立了小麦的最低收购价的合理定价模型:
\[
\max PR_{t}=2.44G_{t}^{s}+0.54G_{t-1}^{s}+100.12P_{t-1}+0.59IM_{t}-23.9UP_{t}
\]
\[
s.t
\begin{cases}
P_{t}=-0.0003G_{t}^{s}+0.42P_{t-1}+0.009IN_{t}-0.003EX_{t}+0.006SP_{t}+0.23PO_{t} \\
G_{t}^{d}=30.57P_{t}+0.101CU_{t}-0.39SP_{t}+0.16EX_{t}-1487.1URBAN \\
2.44G_{t}^{s}+0.54G_{t-1}^{s}+100.12P_{t-1}+0.59IM_{t}-23.9UP_{t}\leq 30056.69 \\
P_{t}-PO_{t}<0.1*P_{t} \\
G_{t}^{s}>0, P_{t-1}>0, IM_{t}>0, UP_{t}>0, P_{t}>0, SP_{t}>0, \\
PQ_{t}>0, G_{t}^{d}>0, CU_{t}>0, URBAN>0
\end{cases}
\]

### 5.2 “十二五”公布的粮食最低收购价格的合理性评价

运用 Lingo 软件,对粮食最低收购价格的合理定价模型进行运算,得到合理粮食补贴价格。定义误差率:

\begin{table}[h]
\centering
\caption{水稻的最低补贴价格定价模型的运算结果}
\begin{tabular}{c c c c}
\hline
年份 & 实际补贴价格 & 合理补贴价格 & 误差率 \\
\hline
2011 & 128 & 120 & 0.046875 \\
2012 & 140 & 142 & -0.014285714 \\
2013 & 150 & 147 & 0.02 \\
2014 & 155 & 151 & 0.025806452 \\
\hline
\end{tabular}
\end{table}

\begin{table}[h]
\centering
\caption{小麦的最低补贴价格的定价模型的运算结果}
\begin{tabular}{c c c c}
\hline
年份 & 实际补贴价格 & 合理补贴价格 & 误差率 \\
\hline
2011 & 93 & 95 & -0.021505376 \\
2012 & 102 & 107 & -0.049019608 \\
2013 & 112 & 115 & -0.026785714 \\
2014 & 118 & 123 & -0.042372881 \\
\hline
\end{tabular}
\end{table}

\begin{table}[h]
\centering
\caption{水稻2015-2017年合理最低补贴价格的预测区间}
\begin{tabular}{c c c c}
\hline
年份 & 合理补贴价格 & 预测区间 \\
\hline
2015 & 156 & 148.2 - 163.8 \\
2016 & 161 & 152.95 - 169.05 \\
2017 & 167 & 158.65 - 175.35 \\
\hline
\end{tabular}
\end{table}

从上表可知,2017 年水稻的合理最低补贴价格为 167 元,预测区间为 158.65 元 -175.35 元之间。

\subsection*{5.3.2 对 2017 年的小麦最低收购价合理范围的预测}

运用 Lingo 软件,对小麦最低收购价格的合理定价模型进行运算,得到 2015-2017 年的合理小麦最低补贴价格,同时根据 2、2 中误差率及合理性的评判标准,给出了各年的合理小麦最低补贴价格的预测区间,如下表所示:

\begin{table}[h]
\centering
\caption{小麦 2015-2017 年合理最低补贴价格的预测区间}
\begin{tabular}{c c c c}
\hline \hline
 & 合理补贴价格 & 预测区间 & \\
\hline
2015 & 127 & 120.65 & 133.35 \\
2016 & 132 & 125.4 & 138.6 \\
2017 & 136 & 129.2 & 142.8 \\
\hline \hline
\end{tabular}
\end{table}

从上表可知,2017 年小麦的合理最低补贴价格为 136 元,预测区间为 129.2 元 -142.8 元之间。

\section*{6. 问题五——小麦最低收购价合理定价模型的应用}

\subsection*{6.1 调整后的小麦最低收购价格的合理定价模型建立}

若国家想让小麦种植面积增加 5\%,可以通过调整小麦最低收购价来实现。为此,需对问题四中小麦最低收购价格的合理定价模型进行调整,调整后的小麦最低收购价格的合理定价模型如下:

\begin{equation}
\max PR_{t} = \frac{1}{1+5\%} * (2.44G_{t}^{s} + 0.54G_{t-1}^{s} + 100.12P_{t-1} + 0.59IM_{t} - 23.9UP_{t})
\end{equation}

\begin{equation}
s.t
\begin{cases}
P_{t} = -0.0003G_{t}^{s} + 0.42P_{t-1} + 0.009IN_{t} - 0.003EX_{t} + 0.006SP_{t} + 0.23PO_{t} \\
G_{t}^{d} = 30.57P_{t} + 0.101CU_{t} - 0.39SP_{t} + 0.16EX_{t} - 1487.1URBAN \\
2.44G_{t}^{s} + 0.54G_{t-1}^{s} + 100.12P_{t-1} + 0.59IM_{t} - 23.9UP_{t} \leq 30056.69 \\
P_{t} - PO_{t} < 0.1 * P_{t} \\
G_{t}^{s} > 0, P_{t-1} > 0, IM_{t} > 0, UP_{t} > 0, P_{t} > 0, SP_{t} > 0, \\
PQ_{t} > 0, G_{t}^{d} > 0, CU_{t} > 0, URBAN > 0
\end{cases}
\end{equation}

其中,$P_{t}$ 表示小麦的 $t$ 期价格水平,$G_{t}^{s}$ 表示小麦的 $t$ 期供给量,$P_{t-1}$ 表示小麦的 $t-1$ 期价格水平,$IN_{t}$ 表示小麦的 $t$ 期进口贸易额,$EX_{t}$ 表示小麦的 $t$ 期出口贸易额,$SP_{t}$ 表示小麦的 $t$ 期替代品价格,$PQ_{t}$ 表示小麦的 $t$ 期小麦的最低收购价,$G_{t}^{d}$ 表示小麦的 $t$ 期需求量,$CU_{t}$ 表示 $t$ 期居民人均消费,$SP_{t}$ 表示小麦的 $t$ 期替代品价格,$EX_{t}$ 表示小麦的 $t$ 期出口贸易额,$URBAN$ 表示城镇化水平,$IM_{t}$ 表示小麦的 $t$ 期进口量,$UP_{t}$ 表示小麦的 $t$ 期单产量。

\subsection*{6.2 调整后的小麦最低收购价格的合理定价模型求解}

运用 Lingo 软件,对上述小麦最低收购价格的合理定价模型进行运算,可以得到小麦合理最低收购价,即可以通过调整小麦最低收购价来实现小麦种植面积增加 5\% 的问题。依据 Lingo 软件计算结果得出,当小麦种植面积提高 5\% 后,小麦最低收购价格为 155 元。应用 Matlab 软件模拟出小麦种植面积与小麦最低收购价格曲线图像如下图所示。

\begin{figure}[h]
    \centering
    \includegraphics[width=\textwidth]{image.png}
    \caption{拟合曲线图}
    \label{fig:fit_curve}
\end{figure}

由上图可以看出当小麦低收购价格为 155 元时,小麦的种植面积提高了 5%。

\section{问题六——调控粮食种植的优化决策和建议}

根据问题一到问题五所得到的结论,我们将从影响粮食种植面积因素、粮食最低收购价政策实施、粮食价格规律及粮食最低收购价范围四个方面给出相应的调控粮食种植的优化决策和建议。

\subsection{从影响粮食种植面积因素方面}

由问题一的研究结论可得,粮食种植面积受多种因素的影响,并且粮食品种和粮食种植区域的不同对于粮食种植面积的影响因素也是不同的。因此,若想提高粮食种植面积扩大粮食产量,应该根据其粮食品种和地域的不同,找出对粮食种植面积具有显著性影响的因素,制定相应的政策措施。例如对于水稻来说,在其主产区黑龙江省与种植面积相关的因素有农业劳动力人口、农民受教育程度、家庭负担、水稻的市场价格、水稻的生产成本、水稻的相对收益竞争力、农作物播种面积和城市化水平,但是对其具有显著性影响为农业劳动人口及农作物播种面积,因此调控黑龙江省种植面积是注重调整这两个因素对种植面积的影响。国家和政府在以后制定关于粮食的相关政策时要对症下药,因地制宜。

\subsection{从粮食最低收购价政策实施方面}

由问题二的研究可得,粮食最低收购价政策的实施对于不同粮食品种、不同区域的影响效果是不同的。且并不是所有粮食主产省分实施粮食最低收购价对粮食种植面积有积极的影响。以水稻为例,比较粮食最低收购价政策实施前后对各省份水稻种植面积的影响,发现政策实施对黑龙江省和广西省水稻种植面积有积极影响,而对湖南省水稻种植面积的影响是消极的。因此,在实施粮食最低收购价政策时,要因地域不同制定不同的最低收购价政策或补贴政策。

\subsection{从粮食价格规律方面}

由问题三的研究可得,粮食供给、价格和需求分别受不同外生变量的影响,同时它们三者之间又是相互影响、相互制约、共同演化的动态关系。要平衡粮食供需并稳定粮食价格,需要权衡各方面的因素。粮食供给增长最大的一个原因是生产者对预期价格的高估,造成粮食总量供给相对过剩,因此有关部门要打造粮食价格信息咨询平台,引导生产者对粮食价格进行合理的预期进而进行合理的种植决策。

\section*{7.4 从粮食最低收购价制定方面}

最低收购价并不是实际的市场收购价格,是收购粮食的底价。粮农决定是否种植粮食,取决于很多因素,但最主要的还是看种植粮食所获得的纯收益的大小。粮食最低收购价的公布,使得粮农能清楚地算出这笔经济账。因此粮食最低收购价的高低直接影响着当年的粮食生产。过高的粮食最低收购价不仅会提高粮食市场价格从而加重消费者负担,同时也会增加粮食的库存压力和国家财政的支出风险。另一方面,过低的粮食最低收购价会打压粮农种植粮食的积极性,造成粮食种植面积的萎缩,这更不是国家所愿意看到的。国家要积极调整粮食最低收购价,既不能损害消费者分利益也不能使粮农受到损失。经过问题四、五的研究,在合理的粮食最低收购价范围内,我们可以通过适当的调整粮食最低收购价来调控粮食种植面积。

\section*{8. 模型评价与推广}

\subsection*{8.1 模型的优点}

1. 斯皮尔曼(Spearman)相关系数对数据不要求服从正态分布,且可以反应变量间趋同关系。

2. 偏最小二乘回归模型可以很好的解决多变量对多变量相互作用的情况。

3. 混合线性模型准确度高,稳定性好,能较好地克服多变量的共线性问题。

\subsection*{8.2 模型的缺点}

1. 斯皮尔曼(Spearman)相关系数准确度不高。

2. 供需及价格联动模型只能对粮食价格进行中短期预测,长期预测误差较大。

\subsection*{8.3 模型的推广}

1. 混合线性模型可以推广到因变量不能完全满足正态性、独立性、方差齐性的情况。

2. 供需——价格联动模型该模型可以推广到解决实际中,品牌(规格)的市场状态(量价存需)进行监测、分析、评价,识别品牌(规格)调整需求,通过调控投放量、货源衔接和其它常规调控管理手段,影响品牌量、存、需状态,进而达到稳定价格、有效控制市场、提高客户满意度的目的。

\section*{参考文献}

[1] 任军军, 王文举. 我国粮食最低收购价政策发展研究 [J]. 湖北经济学院学报 (人文社会科学版), 2010, 07(6): 29-30.

[2] 史常亮, 金彦平. 中国粮食供给与需求状况变迁: 1978~2010 [J]. 经济研究参考, 2013(56): 51-60.

[3] 冷崇总. 我国粮食价格波动问题研究 [J]. 新疆农垦经济, 2008(5): 4-11.

[3] 罗万纯, 刘锐. 中国粮食价格波动分析: 基于 ARCH 类模型 [J]. 中国农村经济, 2010(4): 30-37.

[4] 罗锋, 牛宝俊. 我国粮食价格波动的主要影响因素与影响程度 [J]. 华南农业大学 学报 (社会科学版), 2010, 09(2): 51-58.

[5] 彭珂珊. 我国粮食安全、粮食生产状况和未来形势分析——兼论 2003 年粮食价格波动 [J]. 粮食问题研究, 2004, 22(1): 29-31.

[5] 冯云. 中国粮食价格波动的实证分析 [J]. 价格月刊, 2008(2): 41-44.

[6] 董智勇, 王双进. 粮食价格波动态势及调控对策 [J]. 宏观经济管理, 2013(7): 53-55.

\begin{enumerate}
    \item[7] 刘喜明. 国际粮食价格波动对中国经济的影响[D]. 浙江大学, 2009.
    \item[8] 龚芳. 全球化背景下中国粮食价格的波动机理和预警机制研究[D]. 复旦大学, 2012.
    \item[9] 胡友. 水果价格形成、波动及调控政策研究[D]. 华中农业大学, 2014.
    \item[10] 方鸿. 中国粮食最低收购价合理确定机制研究[J]. 经济与管理, 2009, 23(4):20-25.
    \item[11] 辛宏艳, 么鹏, 安春明. 国际粮食价格波动对我国粮价影响的实证研究[J]. 农业经济, 2015(10):120-122.
    \item[12] 贺伟. 我国粮食最低收购价政策的现状、问题及完善对策[J]. 宏观经济研究, 2010(10):32-36.
    \item[13] Jayne TS, Myers RJ, Nyoro J. The effects of NCPB marketing policies on maize market prices in Kenya[J]. Agricultural Economics, 2008, 38(3):313-325.
    \item[14] Ismet M, Barkley AP, Llewelyn RV. Government intervention and market integration in Indonesian rice markets[J]. Agricultural Economics, 1998, 19(3):283-295.
    \item[15] Mckenzie AM, Jiang B, Djunaidi H, et al. Unbiasedness and Market Efficiency Tests of the U.S. Rice Futures Market[J]. Review of Agricultural Economics, 2002, 24(2):474-493.
    \item[16] 程秀梅. 中国农业支持政策体系构建研究[D]. 吉林大学, 2011.
    \item[17] 王士海. 中国粮食价格调控政策的经济效应[D]. 中国农业科学院, 2011.
    \item[18] 刘斌. 我国粮食最低收购价政策研究[D]. 中国农业科学院, 2013.
    \item[19] 王淑艳. 我国粮食价格波动因素分析与预测研究[D]. 东北农业大学, 2013.
\end{enumerate}

\section*{附录}

\begin{enumerate}
    \item Spearman 秩相关性检验 SAS 程序
    \begin{verbatim}
data ex;
input x1-x12 y@@;
cards;
/*数据省略*/
;
proc corr spearman;
var x1-x12 y;
run;
    \end{verbatim}
    
    \item 偏最小二乘法 R 程序
    \begin{verbatim}
pe<-read.table("D:/省份.txt",header=FALSE)
pe
pl<-scale(pe)
pl
x1<-pl[,1];x1
x2<-pl[,2];x2
x3<-pl[,3];x3
x4<-pl[,4];x4
x5<-pl[,5];x5
x6<-pl[,6];x6
x7<-pl[,7];x7
x8<-pl[,8];x8
y<-pl[,9];y
X<-cbind(x1,x2,x3,x4,x5,x6,x7,x8)
Y<-cbind(y)
library(pls)
mark=plsr(Y~X,ncomp=8,validation="LOO",jackknife=TRUE)
summary(mark,what="all")
coef(mark)
plot(RMSEP(mark),legendpos="topright")
explvar(mark)
pls2<-plsr(Y~X,ncomp=4,validation="LOO",jackknife=TRUE)
summary(pls2,what="all")
coef(pls2)
predplot(pls2)
jack.test(pls2)
    \end{verbatim}
\end{enumerate}

\begin{verbatim}
3、混合线性模型 SAS 程序
data ex; input a x1-x2 y @@;
if x1<4895.125507 then x1=1; if 4895.125507<=x1<5728.076642 then x1=2;
if 5728.076642<=x1<6561.027777 then x1=3; if
6561.027777<=x1<7393.978911 then x1=4;
if 7393.978911<=x1 then x1=5; if x2<-3738.368941 then x2=1;
if -3738.368941<=x2 then x2=2;
cards;
/*数据省略*/;
proc glm; class a ; model y=a x1-x2;
means a/duncan;
proc mixed; class a ; model y=a;
random x1-x2;
lsmeans a; run;
\end{verbatim}

\begin{verbatim}
4、粮食最低补贴价格合理定价模型 Lingo 程序

model:
title 种植面积程序
sets:
yuefen/1..11/:c,x,d,e;
endsets
data:
c=数据省略;
d=数据省略;
e=数据省略;
a=数据省略;
b=数据省略;
enddata
max=@sum(yuefen:c*x);
@sum(yuefen:c*x)<=a;
@sum(yuefen:x*d)<@sum(yuefen:x*e*0.1);
@for(yuefen:b<x);
end
\end{verbatim}