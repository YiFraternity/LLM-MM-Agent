\begin{center}
\textbf{“华为杯”第十三届全国研究生数学建模竞赛}
\end{center}

\begin{table}[h]
\centering
\begin{tabular}{l l}
\hline
学校 & 湖南科技大学 \\
\hline
参赛队号 & 10534002 \\
\hline
队员姓名 & 1. 赵文玉 \\
 & 2. 伍璇 \\
 & 3. 姜博宇 \\
\hline
\end{tabular}
\end{table}

\begin{flushright}
1
\end{flushright}

\begin{center}
\textbf{参赛密码} \underline{\hspace{5cm}} \\
(由组委会填写)
\end{center}

\begin{center}
\includegraphics[width=0.3\textwidth]{image1.png} \quad
\includegraphics[width=0.3\textwidth]{image2.png} \quad
\includegraphics[width=0.3\textwidth]{image3.png}
\end{center}

\begin{center}
\textbf{“华为杯”第十三届全国研究生数学建模竞赛}
\end{center}

\begin{flushleft}
\textbf{题目} \quad 基于无线通信基站的室内三维定位问题
\end{flushleft}

\section*{摘要}

本文研究通过测量无线电信号的传播时间,将其转化成终端与基站之间的距离,基于无线移动通信网络定位用户手持终端的位置。

针对问题一,首先建立了不考虑 NLOS 影响的 TOA 定位模型,把终端定位问题转换为超定线性方程组的求解问题。然后根据对测量数据进行误差分析,发现延迟(包括因为 NLOS 环境而导致的延迟以及其他误差导致的延迟)与测量 TOA 值有非常明显的线性关系,并且这种线性关系由传播环境确定,从而真实 TOA(不包含 NLOS 以及其他误差导致的延迟的 TOA)与测量 TOA 之间有线性关系。基于这样的发现,在进行终端定位时,不直接用测量 TOA 来定位而是用测量 TOA 的一个线性函数值来定位。对某个终端进行定位时,并不事先确定这个线性关系,而是先根据实际情况确定这个线性关系的两个参数的范围,然后以某个固定的步长在这个二维范围里面遍历,找出遍历过程中使得模型中超定方程组求解残差最小的解作为该终端的定位坐标,因此对给定的终端,同时完成上述线性关系确定和定位,比如竞赛数据 Case1 中第一个终端定位结果为 $(-284.40, 89.73, 3.89)$,相应的真实 TOA 与测量 TOA 之间的线性关系为:$0.735\phi + 1.20 \times 10^{-9}$,距离误差为 $1.285628 \, \text{m}$。

针对问题二,对任何终端,首先对测量 TOA 按从小到大排序,基站也进行相应的排序,然后利用问题一的定位方法,利用所有基站的数据对该终端进行定位,得到终端定位的基本准确解,然后从测量 TOA 最小的四个基站开始,用同样方法对该终端进行定位,如果定位结果与基本准确解相差较小,则对应的基站数就是最少基站数,否则从排序后的基站中再选一个基站添加进来进行定位,依此循环下去,直到找到与基本准确解相差

\begin{table}[h]
\centering
\begin{tabular}{c c c c c}
\hline
Case & 定位终端数 & 连接数 & 连接度数 & 平均误差 \\
\hline
Case 26 & 193 & 981 & 0.891 & 145.42 \\
Case 27 & 757 & 6512 & 5.4267 & 3.3831 \\
Case 28 & 798 & 6624 & 5.0954 & 8.7548 \\
Case 29 & 1035 & 13292 & 9.4943 & 3.9234 \\
Case 30 & 468 & 2710 & 2.71 & 2.1353 \\
\hline
\end{tabular}
\end{table}

\section{问题重述}

\subsection{问题背景}

随着无线通信网络和移动互联网的蓬勃发展,提供基于地理位置信息的服务(Location Based Service,简称 LBS)已经成为最具市场前景和发展潜力的业务之一。从传统的 GPS 导航,到大众点评、微信等基于地理位置的消费信息服务和社交软件,实现其功能的基础就是要通过手机、导航仪等终端设备收发信号,来获得距离、角度等测量信息,并利用定位算法将这些测量信息转换成坐标信息。

基于无线移动通信网络的定位是以获取用户手持终端(包括手机或者平板等设备)的位置为目标。而达成这一目标的手段是通过测量无线电信号的强度、传播时间、到达角等物理指标,并将其转化成终端与基站之间的距离、角度等信息,最终利用定位算法将距离、角度等信息转化成终端的坐标信息。

虽然商用 GPS 已经随着智能手机的发展而得到了广泛的应用,但是,在诸如室内、地下、高楼林立的市区等诸多场景中,GPS 定位性能较差。由于在覆盖广度和深度上,基于无线网络基站的定位系统相比 GPS 存在优势,因此,越来越得到运营商和新兴创业公司的重视。

此外,对于大数据感兴趣的 IT 公司,通过统计大规模匿名用户的连续地理位置信息,可以获得用户的移动轨迹,以及在相应轨迹上的 APP 流量使用情况,甚至在特殊位置搜索和关注的关键词等信息。因此,诸如 Google、百度等搜索引擎公司也开始提供室内定位和室内地图导航的服务。这类服务,一方面可以弥补传统的 GPS 在室内定位性能较差,且不能分辨用户所在楼层等问题,另一方面,也为商场、博物馆等应用场景提供了为用户提供基于室内实时地理位置信息服务的可能。

目前从事室内定位和导航服务的方法,大多基于室内密集分布的 WiFi 设备与手机之间的通信方式。这类方法存在两个明显的劣势:首先,从技术上,WiFi 设备的覆盖范围有限,并且 WiFi 设备收发信号所在的频段容易受到干扰;其次,从业务模型上看,用户对于接入陌生 WiFi 设备的戒备心理,以及 WiFi 设备的投资如何回收等,都存在较大的商业模式上的不确定性。

与之相对的,使用基于运营商无线通信基站的方式对手机进行定位,则可以规避上述问题。商用基站的覆盖范围、信号质量均优于 WiFi,而且,用户也期望自己的手持终端能够随时保持对基站设备的接入。同时,运营商推进定位服务的盈利模式清晰,在基础的数据服务之外,还可以通过为用户提供增值服务而促进运营商的业务发展。总之,基于无线通信基站的定位技术有着广阔的应用前景和巨大的商业价值。

手持终端设备如何基于基站的测量信息,计算或确定终端在三维空间中的位置坐标,也就是三维定位问题,被认为是现代商用通信网络中对于定位系统真正具有技术难度的挑战。而高精度三维定位也预期能为客户提供更大的价值,在智能仓储、智能工厂、固定资产追踪等对于三维坐标信息敏感的垂直行业,以及传统运营商感兴趣的商场、办公楼中基于位置信息的室内导航、人群流量分析,以及基于精确三维地理位置信息的业务推送等服务提供基础性技术。

从技术角度来看,现代商用通信网络对于三维定位的需求,是使用尽可能少的基站完成对终端设备的定位、算法收敛速度快、对于干扰和噪声具有鲁棒性等优点。

相比于 GPS 等商用卫星定位系统,基于通信基站的定位问题,具有如下特殊性:

首先,通信基站的目标区域是 GPS 等卫星定位系统无法实现定位的场景。在高楼林立的城区,建筑物内部、地下停车场等区域,GPS 等系统是无法满足定位需求的。而这些应用场景基站、终端密集,是基站定位可以实现突破的地方。

其次,通信基站所处的电磁信号环境较之 GPS 等系统更加复杂。以室内环境为例,无线电信号的传播过程中会经过墙面的多次反射、室内物体的折射和吸收等。这些物理因素会导致通信基站测量得到的诸如距离、角度等信息存在噪声。如何基于这些有噪声的测量,得到对于位置信息的准确估计,也是通信基站实现对终端定位需要解决的问题。

基于通信基站的定位问题研究,在科研和工业界都吸引了极高的关注。一方面,定位问题与统计信号处理、最优估计理论、优化算法等诸多领域都有密切的联系,诸如数据拟合、最小二乘估计、半正定规划、流形学习等诸多数学工具都能够被用于求解上述问题。另一方面,工业界对于如何高精度地在现有通信设备上完成上述功能也表现出了浓厚的兴趣,我国除了业已广泛部署商用的北斗导航系统之外,也在积极推进基于室内室外融合定位的羲和导航系统。我们相信,基于通信基站的定位系统,将会成为羲和导航系统有力的技术手段。

求解分析基站定位相关问题的有创新性和可实现性强的算法,都将有可能被快速部署到现代商业通信网络中,带来巨大的社会和经济效益。

\subsection{1.2 问题提出}

在本题中,需要解决如下四个方面的问题:

1. 给定 10 组 LOS 或 NLOS 传播环境下从手持终端到基站的 TOA 测量数据和所有基站的三维坐标(对应附录中编号为 case001_input.txt 到 case010_input.txt 的文件),请根据这些测量数据计算出终端的三维坐标。(请给出详细的建模分析,建模过程中建议考虑测量模型、误差分析等内容。)

2. 给定 10 组 TOA 测量数据和所有基站的三维坐标(对应附录中编号为 case011_input.txt 到 case020_input.txt 的文件),请设计算法,使用尽可能少的基站数目,实现近似最优的三维定位精度。

3. 给定 5 组对处于移动过程中的终端采集到的 TOA 数据(对应附录中编号为 case021_input.txt 到 case025_input.txt 的文件),请设计算法计算出终端的运动轨迹。(此时,编号为 case021_input.txt 到 case025_input.txt 的文件中,只记录一个终端的 TOA 数据,并且是这一个终端在运动轨迹中多个位置上的 TOA 数据。)

4. 在前述 3 问中,都是假设给定区域内终端到每一个基站的距离都是可知的,但事实上,基站的通信半径是有限的,因此,只有在基站通信半径覆盖范围内的终端才有可能测到自身到基站的距离。而一个终端只有获得它与足够数目的基站之间的距离测量值,才能完成定位。假设每个基站的通信半径为 200 米(超过范围虽然有测量数据,但无效)。请根据给定的 5 组测量信息数据集(对应附录中编号为 case026_input.txt 到 case030_input.txt 的文件),设计算法寻找出可以被基站定位的所有终端。进一步,回答如下问题:每一个场景中(对应着 case026_input.txt 到 case030_input.txt 五个文件中的一个),定义终端的平均“连接度数”$\lambda$ 为

\[
\lambda = \frac{\text{所有可以被定位终端到基站之间的连接数}}{\text{终端数}}
\]

请建立模型分析连接度数与定位精度之间的关系。

\subsection{1.3 问题分析}

获取 LOS 和 NLOS 不同传播环境下手持终端的三维坐标,即通过对不同传播环境下分析基站到终端的 TOA 测量值以及基站的三维坐标,得到手持终端的三维坐标。

分析该问题具有两个难点,其一是关于无线电信号测量。由于无线电波从基站到终端的传播过程中存在测量噪声,因此对于无线电传播时间的测量也存在一定误差。在目

前实测数据的基础之上,缺乏无线电信号传播时间误差测量的先验知识,必须结合一些常见的误差处理方法对实测数据进行预处理,通过对实测数据的分析处理,计算出无线电信号传播时间的误差。其二是关于 TOA 数据的误差。在实际的场景中,受宽带、信噪比、时钟同步以及 NLOS 传播环境的影响,TOA 测量数据也存在一定的误差。

对于问题一,终端三维坐标的定位可以从所有基站的三维坐标,以及 LOS 或 NLOS 传播环境下终端到基站的 TOA 测量数据入手,并考虑无线电信号测量时产生传播时间、信号折射、信号反射等情况下的误差。在已知的条件的情况下,首先我们不考虑 NLOS 传播环境下 TOA 定位算法,假设所有的传播环境都为 LOS,在该假设的条件下,用它的解作为 NLOS 算法的参考点,再考虑 NLOS 影响的 TOA 算法。

对于问题二,二维空间对一个终端进行定位至少需要三个基站,三维空间中最少则需要四个基站。对于问题二,尽可能少的基站数,实现近似最优的三维定位精度,有以下三种解决方案。方案一,首先利用所有的基站对当前终端定位,问题 1 中的模型可知,该定位是十分准确的。然后按照特定规则确定最小基站数目,可根据误差的大小将基站逐步剔除,同时重新定位终端,假设剔除当前终端定位结果与利用所有基站对终端定位的精确坐标值解较之前更加接近,则说明当前终端需要被剔除的。由上述方法,循环迭代直到与利用所有基站对终端定位结果无限接近为止,即得到最少基站数目。但是,该算法每次都要利用较多的基站对终端定位,计算量巨大,计算程度复杂,成本高,况且不能保障最终结果为与真实所需最小基站数目相近。方案二,按照可按照 TOA 测量值从小到大对相应基站进行排序,从排好序的列表中依次选择 4 个基站,对终端进行第一次定位,然后再按照同样规则依次添加一个基站进行第二次定位,如果前后两次定位的结果极为相近,则认为第一次定位基站总数可实现近似最优的三维定位精度;否则,从排好序的列表中再选择一个基站,直到连续两次定位结果相近或者直到基站排序列表为空。虽然方案二用到的计算量减少,但是由于该方案无法与较精确的定位比较。方案三,首先对所有基站给该进行终端定位,再按 TOA 测量值从小到大对相应基站进行排序,在排序列表中依次选择 4 个基站对终端进行定位,如果定位结果与所有基站地位的结果相近,则说明当前基站数目就可实现近似最优的三维定位,否则,按照同样的规则继续加一个基站重新对终端进行定位,再判断与用所有基站定位的结果是否相近,相近则当前基站数目为最小,否则,继续循环迭代上述过程。方案三结合并扩展了方案一与方案二两种方法的优点,解决了两者存在的问题,既能满足基站数目最少,又能满足实现近似最优的三维定位精度。

对于问题三,该问题要根据给出了移动终端在移动过程中采集的 TOA 数据,确定移动轨迹。也就是说确定在每个采集点的位置,与问题一类似,就是对采集位置进行定位,这样就可以用问题一的办法来求解。只是该问数据没有给出竖坐标 $Z$,说明基站和移动终端始终在同一个平面上。给定的 TOA 测量矩阵记录的是一个终端在移动过程中多个位置上对于不同基站的 TOA 测量。其中第 $i$ 行中第 $j$ 个元素,表示的是终端移动到标号为 $i$ 的位置上时,相对于标号为 $j$ 的基站的 TOA 测量。为了求出运动轨迹,可以认为位置是按照时间顺序排列的,或者说可以认为移动终端持有者先到第 $i$ 行对应的采集位置,再到第 $i+1$ 行对应的采集位置。采用拟合技术,可以得到移动终端的轨迹方程。

首先为了验证问题一的解决方法,能用到这里的二维问题,修改测试数据里面的基站坐标都为

对于问题四,应该从现实网络的角度出发去为这个问题做建模和分析。比如,真实世界中的一个基站,如果它想实现一个算法去解决该问时,它到网络中所有终端的距离是不太可能一一测量得到。那这个时候,该建立一个准则和相应的模型去衡量度数与定位精度的关系。

测距数据小于200米内的,都视为有效接入,终端被认为是可以与该基站建立连接,一个终端可以同时与多个基站建立连接,所以某个终端的连接数就是与该终端建立连接的基站数。给定了通信半径之后,就可以确定了。连接度数定义中的分子部分,就是所有终端连接数的总和,是所有可以定位的终端连接到基站的“边”数的总和,只取决于基站和终端之间的连接数目。计算连接度数1的终端数是指一个网络中出现的所有终端数目,不管他是否能够被基站定位。

\section*{2 模型假设与说符号说明}

\subsection{模型假设}

为了使基于无线通信基站的室内三维定位模型在尽量接近实际的无线通信环境的前提下,能够合理简化分析过程,在建模时提出如下假设:

1) 无线通信基站发射出信号的强度是稳定的,不会产生剧烈的波动。

2) 考虑无线电信号测量的误差时,我们只考虑时钟同步误差,传播时延误差,信号折射,信号反射,信噪比误差,不考虑其它的误差。

3) 假设传播时延误差主要是由NLOS引起的。

4) 无线电信号的传播速度为$3 \times 10^{8} \, \mathrm{m/s}$。

5) 同一组数据是在相同场景环境中测量得来。

6) 数据测量的场景都是位于地面之上。

\subsection{名词解释}

视距传播环境(Line Of Sight propagation,简称LOS)指的是无线电信号在大气中从A点向B点传播时,如果传播过程中存在一个没有遮挡的直达路径,这种传播环境被称为视距传播环境[1]。

非视距传播环境(Non-Line Of Sight,简称NLOS)指的是如果在传播过程中,由于建筑物或树木的遮挡、反射、折射等物理现象,使得从A点到B点之间存在多条无线电信号的传播路径,这种环境被称为非视距传播环境[2]。

无线电信号的到达时间(TimeOfArrival,简称TOA)指的是信号在基站与终端之间的传播时间[3]。

真实TOA:无线电信号的到达时间TOA减去因非视距传播而导致的延迟,以及测量误差和不同步性导致的延迟。

\subsection{符号说明}

\begin{table}[h]
\centering
\caption{部分常用符号说明}
\begin{tabular}{c c}
\hline
符号 & 符号说明 \\
\hline
$N$ & 基站的数量 \\
$M$ & 手持终端的数量 \\
$(x_{j}, y_{j}, z_{j})$ & 第$j$个基站的三维坐标 \\
$(X_{i}, Y_{i}, Z_{i})$ & 第$i$个手持终端的三维坐标 \\
$\phi_{ij}$ & 第$j$个基站到第$i$个手持终端的TOA测量值 \\
$c$ & 无线电信号的传播速度 \\
\hline
\end{tabular}
\end{table}

\section{问题一模型建立与求解}

\subsection{问题的描述及分析}

问题 1 中提供了 10 组 LOS 或者 NLOS 传播环境下从手持终端到基站的 TOA 测量数据和所有基站的三维坐标,并要求我们根据这些测量数据和基站坐标,对所提供的样本加以分析,建立基于无线通信基站室内三维定位的模型。但是由于基站和终端的视距上有障碍物的存在,导致两者之间的信号不能视距传播,产生了 NLOS 误差,而这个误差有很大的不确定性[5]。LOS 与 NLOS 传播方式如图 3.1 所示。

\begin{figure}[h]
    \centering
    \includegraphics[width=\textwidth]{image.png}
    \caption{无线网络通信中的定位模型}
\end{figure}

如图 3.1 中右图所示,在基站(Base Station)和手持终端(Subscriber)的视距上由于有障碍物的存在,会使得基站与手持终端之间的信号不能进行视距传播。手持终端通过中间建筑物的反射收到信号。手持终端收到的信号的传播路径并不是图 1.1 中左图中的直线路径,因此传播时间大于它们之间采用 LOS 方式的直线传播时间。在 NLOS 的传播路径下,若直接使用手持终端的测量值进行定位,那么必然会带来位置估计的误差。在上述背景条件下,我们遇到的难点主要有:

1. 由题目中背景可知,由于在 NLOS 传播环境中,NLOS 传播干扰对定位的影响比较大,因此对于 LOS 传播环境下测得室内三维坐标的方法与在 NLOS 的传播环境下测得室内三维坐标的方法存在一定的差异性,我们需要区分处理。

2. 在无线信号测量的时候,其中影响测量结果的误差因素包括宽带、信噪比、时钟同步影响,信号的反射、折射,传播时间延迟等多个因素。如何对这些误差进行精确的定位也是一个难点。

通过对以上问题分析可知,解决问题的思路如下:首先假设传播环境都是 LOS,在该假设条件下,使用定位算法计算出手持终端的三维坐标,用该解作为考虑 NLOS 算法中的参考点,计算出终端的真实三维坐标。

\subsection{模型的建立}

\subsubsection{不考虑 NLOS 的 TOA 定位算法}

\textbf{1. 模型一极大似然模型}

本文首先假设传播环境都是 LOS,不考虑 NLOS 传播环境,在该假设下,我们使用 TOA 定位算法,计算出手持终端三维坐标的预测值[4]。将 LOS 传播途径模拟到三维空间中,即存在 $N$ 个基站,$M$ 个手持终端,从第 $j$ 个基站 $(x_j, y_j, z_j)$ 发射的信号到达手持终端 $(X, Y, Z)$ 的 TOA 测量值为 $\phi_j$,采用视线(LOS)传播时有

\begin{equation}
(X-x_{j})^{2}+(Y-y_{j})^{2}+(Z-z_{j})^{2}=c^{2}\phi_{j}^{2}
\tag{3-1}
\end{equation}

(3-1)可等价于下式:
\begin{equation}
r_{i}^{2}=K_{i}-2Xx_{j}-2Yy_{j}-2Zz_{j}+x_{j}^{2}+y_{j}^{2}+z_{j}^{2}
\tag{3-2}
\end{equation}
\begin{equation}
j=1, \, 2, \, \cdots, \, N
\end{equation}

其中,
\begin{equation}
K_{i}=X^{2}+Y^{2}+Z^{2}, \, r_{i}=c\phi_{j}
\tag{3-3}
\end{equation}

式(3-3)中 \(c\) 是信号的传播速度。

由于平方项的存在, 使得方程(3-2)为非线性的方程, 但通过令变量 \(R=x_{j}^{2}+y_{j}^{2}+z_{j}^{2}\), 可使方程(3-2)变为线性方程
\begin{equation}
r_{i}^{2}-K_{i}=-2Xx_{j}-2Yy_{j}-2Zz_{j}+R,
\tag{3-4}
\end{equation}
\begin{equation}
j=1, \, 2, \, \cdots, \, N
\end{equation}

由于式(3-4)中的 \(R\) 并不是与 \(x_{j}\), \(y_{j}\), \(z_{j}\) 无关的变量, 因而在解出它们的值后应再将它们代入另一线性方程, 使它们之间的关系得到满足, 这将在后面进一步讨论。

对式(3-4), 令 \(Z_{a}=[x_{j}, \, y_{j}, \, z_{j}, \, R]^{T}\), 误差矢量为
\begin{equation}
\psi=h-G_{a}Z_{a},
\tag{3-5}
\end{equation}

其中
\begin{equation}
h=\begin{bmatrix}
r_{1}^{2}-K_{1} \\
r_{2}^{2}-K_{2} \\
\vdots \\
r_{N}^{2}-K_{N}
\end{bmatrix}
G_{a}=\begin{bmatrix}
-2x_{1} & -2y_{1} & -2z_{1} & 1 \\
-2x_{2} & -2y_{2} & -2z_{2} & 1 \\
\vdots & \vdots & \vdots & 1 \\
-2x_{N} & -2y_{N} & -2z_{N} & 1
\end{bmatrix}
\end{equation}

如果把真实值表示为 \(r_{1}^{0}, \, \cdots, \, r_{N}^{0}\), 测量值为 \(r_{1}, \, \cdots, \, r_{N}\), 则 \(r_{j}\) 可表示为 \(r_{j}=r_{j}^{0}+cn_{i}\), 而误差矢量
\begin{equation}
\psi=2cBn+c^{2}n^{\circ}n,
\tag{3-6}
\end{equation}
\begin{equation}
B=diag\{r_{1}^{0}, \, \cdots, \, r_{N}^{0}\}
\end{equation}

用最大似然(ML)法对式(3-5)进行估计可得
\begin{equation}
Z_{a}=arg \, min \, \{(h-G_{a}Z_{a})^{T}\Psi^{-1}(h-G_{a}Z_{a})\}=(G_{a}^{T}\Psi^{-1}G_{a})^{-1}G_{a}^{T}\Psi^{-1}h
\end{equation}

其中
\begin{equation}
\Psi=E[\psi\psi^{T}]\approx 4c^{2}BQB
\tag{3-7}
\end{equation}

在上式中, \(Q\) 为估计噪声的协方差矩阵, \(B\) 矩阵对角线上的元素为基站到各手持终端的真实距离, 在估计时为未知数, 但我们可以先用测量值 \(r_{1}, \, \cdots, \, r_{N}\) 代替 \(r_{1}^{0}, \, \cdots, \, r_{N}^{0}\),

估计出一个初始解,然后用此初始解算出其对应的 $B$,再将 $B$ 代入式(3-7)求出更精确的解。如上进行重复迭代,我们可以获得与实际值足够逼近的估计值,一般情况下,只要进行一次迭代即可得到足够精确的一组解。

因此在上面对 $Z_a$ 及其协方差矩阵为
\begin{equation}
cov(Z_a) = E\left[\Delta Z_a \Delta Z_a^T\right] = \left(G_a^T \Psi^{-1} G_a\right)^{-1}
\tag{3-8}
\end{equation}

由于在上面对 $Z_a$ 的求解中,假设 $x_j$, $y_j$, $z_j$, $R$ 是相互独立的,但事实上它们之间存在着一定的关系,因此我们应该将这个关系结合进去以得到对它们更好的估计。假设 $x_j$, $y_j$, $z_j$, $R$ 的估计误差分别为 $e_1$, $e_2$, $e_3$, $e_4$,则 $Z_a$ 中的三个元素可依次表示为
\begin{equation}
Z_a(1) = x_j^0 + e_1 \quad Z_a(2) = y_j^0 + e_2
\tag{3-9}
\end{equation}
\begin{equation}
Z_a(3) = z_j^0 + e_3 \quad Z_a(4) = R_j^0 + e_4
\end{equation}

设误差矢量
\begin{equation}
\psi' = h' - G_a' Z_p
\tag{3-10}
\end{equation}
其中
\begin{equation}
h' = \begin{bmatrix} Z_a(1)^2 \\ Z_a(2)^2 \\ Z_a(3)^2 \\ Z_a(4)^2 \end{bmatrix}, \quad G_a' = \begin{bmatrix} 1 & 0 & 0 \\ 0 & 1 & 0 \\ 0 & 0 & 1 \\ 1 & 1 & 1 \end{bmatrix}, \quad Z_p = \begin{bmatrix} x_j^2 \\ y_j^2 \\ z_j^2 \end{bmatrix}
\end{equation}

将式(3-9)代入式(3-7),可得
\begin{equation}
\begin{aligned}
\Psi_1' &= 2x_j^0 e_1 + e_1^2 \approx 2x_j^0 e_1 \\
\Psi_2' &= 2y_j^0 e_2 + e_2^2 \approx 2y_j^0 e_2 \\
\Psi_3' &= 2z_j^0 e_3 + e_3^2 \approx 2z_j^0 e_3 \\
\Psi_4' &= e_4
\end{aligned}
\tag{3-11}
\end{equation}

上式在 $e_1$, $e_2$, $e_3$, $e_4$ 非常小时成立,因此 $\psi'$ 的协方差矩阵
\begin{equation}
\psi' = E\left[\psi' \psi'^T\right] = 4B' cov(Z_a) B'
\tag{3-12}
\end{equation}
\begin{equation}
B' = diag\{x_j^0, y_j^0, z_j^0, 0.5\}
\end{equation}

$B'$ 元素 $x_j^0, y_j^0, z_j^0$ 可用 $Z_a$ 中元素 $x_j$, $y_j$, $z_j$ 的值近似得到,类似地,$Z_p$ 的 ML 估计为
\begin{equation}
Z_p = \left(G_a'^T \Psi'^{-1} G_a'\right)^{-1} G_a'^T \Psi'^{-1} h
\tag{3-13}
\end{equation}

于是最后的位置估计 $Z = \begin{bmatrix} x_j & y_j & z_j \end{bmatrix}^T$ 为
\begin{equation}
Z = \sqrt{Z_p} \quad \text{或} \quad Z = -\sqrt{Z_p}
\tag{3-14}
\end{equation}

在 $Z$ 中,元素 $x_j$ 的正负号的选取与由式(3-7)求出的 $Z_a$ 的第一个元素 $Z_a(1)$ 的符号一致,$y_j$ 的符号与 $Z_a$ 的第二个元素 $Z_a(2)$ 的符号一致,$z_j$ 的符号与 $Z_a$ 的第三个元素 $Z_a(3)$

的符号一致。

\section{模型二优化模型}

该模型也是不考虑 NLOS 传播环境,只考虑 LOS 的传播方式。在不考虑误差和时钟同步的情况下,从第 $j$ 个基站 $(x_{j}, y_{j}, z_{j})$ 发射的信号到达手持终端 $(X, Y, Z)$ 的 TOA 测量值为 $\phi_{j}$,可得下式:
\begin{equation}
\sqrt{(X-x_{j})^{2}+(Y-y_{j})^{2}+(Z-z_{j})^{2}}=c \phi_{j} j=1,2, \cdots, N
\tag{3-15}
\end{equation}
式 (3-15) 也可等价于
\begin{equation}
\left(X_{i}-x_{j}\right)^{2}+\left(Y_{i}-y_{j}\right)^{2}+\left(Z_{i}-z_{j}\right)^{2}=c^{2} \phi_{j}^{2} j=1,2, \cdots, N
\tag{3-16}
\end{equation}

直接求解以上的 $N$ 维非线性方程组是比较困难的,可以通过第 $j=2,3, \cdots, N$ 个方程减去第一个方程,得
\begin{equation}
\begin{aligned}
& 2\left(x_{1}-x_{j}\right) X+2\left(y_{1}-y_{j}\right) Y+2\left(z_{1}-z_{j}\right) Z \\
& \quad=c^{2}\left(\phi_{j}^{2}-\phi_{1}^{2}\right)+\left(x_{1}^{2}+y_{1}^{2}+z_{1}^{2}\right)-\left(x_{j}^{2}+y_{j}^{2}+z_{j}^{2}\right)
\end{aligned}
\tag{3-17}
\end{equation}
\begin{equation}
j=2,3, \cdots, N
\end{equation}

式 (3-17) 是关于 $(X, Y, Z)$ 的 $N-1$ 个方程的线性方程组,表示为如下形式
\begin{equation}
A X=b
\tag{3-18}
\end{equation}
其中 $A \in R^{(N-1) \times 3}, b \in R^{(N-1)}$
\begin{equation}
A=\left[\begin{array}{ccc}
2\left(x_{1}-x_{2}\right) & 2\left(y_{1}-y_{2}\right) & 2\left(z_{1}-z_{2}\right) \\
2\left(x_{1}-x_{3}\right) & 2\left(y_{1}-y_{3}\right) & 2\left(z_{1}-z_{3}\right) \\
\vdots & \vdots & \vdots \\
2\left(x_{1}-x_{N}\right) & 2\left(y_{1}-y_{N}\right) & 2\left(z_{1}-z_{N}\right)
\end{array}\right]
\end{equation}
\begin{equation}
b=\left[\begin{array}{c}
c^{2}\left(\phi_{2}^{2}-\phi_{1}^{2}\right)+\left(x_{1}^{2}+y_{1}^{2}+z_{1}^{2}\right)-\left(x_{2}^{2}+y_{2}^{2}+z_{2}^{2}\right) \\
c^{2}\left(\phi_{3}^{2}-\phi_{1}^{2}\right)+\left(x_{1}^{2}+y_{1}^{2}+z_{1}^{2}\right)-\left(x_{3}^{2}+y_{3}^{2}+z_{3}^{2}\right) \\
\vdots \\
c^{2}\left(\phi_{N}^{2}-\phi_{1}^{2}\right)+\left(x_{1}^{2}+y_{1}^{2}+z_{1}^{2}\right)-\left(x_{N}^{2}+y_{N}^{2}+z_{N}^{2}\right)
\end{array}\right]
\end{equation}

由于在实际的无线信号的传播过程中,既存在 LOS 传播方式也存在 NLOS 的传播方式,但是在 NLOS 传播的情况下,由于存在时钟不同步,时延,信号折射反射的误差,导致基站到手持终端的 TOA 测量值比真实的 TOA 值要大。同时,一般情况下基站和手持终端的位置都是位于地面之上,则基站和手持终端三维坐标中的纵坐标应该都大于 0 。因此,方程组 (3-18) 的求解通常转换为如下优化问题:
\begin{equation}
\min \|A X-b\|^{2}
\tag{3-19}
\end{equation}
\begin{equation}
\text { s.t. }\left(X-x_{j}\right)^{2}+\left(Y-y_{j}\right)^{2}+\left(Z-z_{j}\right)^{2} \leq c^{2} \phi_{j}^{2} j=2,3, \cdots, N
\tag{3-20}
\end{equation}
\begin{equation}
Z \geq 0
\tag{3-21}
\end{equation}

\section{模型三线性方程组模型}

在模型二中,由于对于式 (3-19)-(3-21) 直接求解,存在一定的困难性,很难找到满足

(3-20)的可行解。因此,在模型三种我们改进模型二的方法,如果(3-18)的系数矩阵 \( A \) 是列满秩的话,则其最小二乘解可简单表示为

\[
X = (A^T A)^{-1} A^T b
\tag{3-22}
\]

其中

\[
A = \begin{bmatrix}
2(x_1 - x_2) & 2(y_1 - y_2) & 2(z_1 - z_2) \\
2(x_1 - x_3) & 2(y_1 - y_3) & 2(z_1 - z_3) \\
\vdots & \vdots & \vdots \\
2(x_1 - x_N) & 2(y_1 - y_N) & 2(z_1 - z_N)
\end{bmatrix}
\]

\[
b = \begin{bmatrix}
c^2 \big( \phi_2^2 - \phi_1^2 \big) + (x_1^2 + y_1^2 + z_1^2) - (x_2^2 + y_2^2 + z_2^2) \\
c^2 \big( \phi_3^2 - \phi_1^2 \big) + (x_1^2 + y_1^2 + z_1^2) - (x_3^2 + y_3^2 + z_3^2) \\
\vdots \\
c^2 \big( \phi_N^2 - \phi_1^2 \big) + (x_1^2 + y_1^2 + z_1^2) - (x_N^2 + y_N^2 + z_N^2)
\end{bmatrix}
\]

通过式 (3-22) 可求解出室内手持终端的三维坐标值。

### 3.2.2 NLOS 误差模型

由于 NLOS 的影响使得测量距离(实际测量的为时间,由于时间与距离的线性关系可得到距离)总是大于真实距离,如果不对测量数据进行处理而直接使用一个精度较高的两步最大似然估计方法来估计手持终端的三维坐标位置,那么所得三维坐标位置将是有偏差的,必须采取有效的措施去掉偏差的影响 [5]。在信号传播环境一定的条件下,NLOS 传播引起的附加时延的均值与方差仅由基站和手持终端的真实距离确定。由于存在系统的测量误差和由 NLOS 引起的附加时延误差 [6]。因此,第 \( j \) 个基站和手持终端之间的 TOA 测量值 \( \phi_j \) 可表示为:

\[
\phi_j = \phi_j^0 + \phi_{bj} + \phi_{ej}, \ j = 1, \ 2, \ \dots, \ N
\tag{3-23}
\]

其中 \( \phi_j^0 \) 为 LOS 环境下 TOA 的真实值,\( \phi_{bj} \) 为系统测量误差,服从 0 均值高斯分布;\( \phi_{ej} \) 为由 NLOS 引起的附加时延误差,它是一个大的正均值随机变量,\( \phi_{bj} \) 与 \( \phi_{ej} \) 之间,各 \( \phi_{ej} \) 之间相互独立。本文考虑 \( \phi_{ej} \) 服从指数分布的情况,其条件概率密度函数为:

\[
f(\phi_{ej} | \phi_{j, \text{rms}}) = \frac{1}{\phi_{j, \text{rms}}} \exp \left( -\frac{\phi_{ej}}{\phi_{j, \text{rms}}} \right), \phi_{ej} \geq 0 \ \phi_{j, \text{rms}} = T_1 d_j^\varepsilon \xi
\tag{3-24}
\]

式 (3-24) 中 \( \phi_{j, \text{rms}} \) 为均方根时延扩展 (\( \mu \)s),\( T_1 \) 为 \( \phi_{j, \text{rms}} \) 在 \( d_j = 1 \) km 时的中值,\( d_j \) 为第 \( j \) 个基站到手持终端的真实距离,\( \varepsilon = 0.5 \),\( \xi \) 为 0 均值对数正态分布的随机变量,标准差 \( \sigma_\xi = 4 \) dB。不同信道环境下的参数 \( T_1 \) 的取值不一样,但是在同一组实验数据中,测试场景都是相同的。则第 \( j \) 个基站到手持终端的 TOA 测量值对应的距离为:

\[
r_j = d_j + b_j + e_j, \ j = 1, \ 2, \ \dots, \ N
\tag{3-25}
\]

其中 \( r_j \),\( d_j \),\( b_j \),\( e_j \) 分别为 \( \phi_j \),\( \phi_j^0 \),\( \phi_{bj} \),\( \phi_{ej} \) 对应的距离,\( b_j \) 的均值为 0,方差 \( \sigma_{bj}^2 \);\( e_j \) 的均值为 \( \mu_{ej} \),方差为 \( \sigma_{ej}^2 \),式 (3-17) 可进一步等效为:

\begin{equation}
r_{j}=d_{j}+\mu_{e_{j}}+b_{j}^{'}, \; i=1,2,\ldots,M
\tag{3-26}
\end{equation}

其中 $b_{j}^{'}$ 均值为 0,方差为 $\sigma^{2}=\sigma_{b_{j}}^{2}+\sigma_{e_{j}}^{2}$ 的随机变量。

关于 NLOS 传播环境引起附加时延均值与方差的推导,有比较多的文献进行了详细的介绍,感兴趣的读者可以根据参考文献自行下载阅读。其中根据文献[7]可知,NLOS 传播引起附加时延的均值与方差为:

\begin{equation}
E\left(\phi_{e_{j}}\right)=T_{1}d_{j}^{\varepsilon}e^{m_{z}+\frac{\delta_{z}^{2}}{2}}
\tag{3-27}
\end{equation}

\begin{equation}
D\left(\phi_{e_{j}}\right)=\left(T_{1}d_{i}^{\varepsilon}\right)^{2}e^{2m_{z}+\delta_{z}^{2}}\left(2e^{\delta_{z}^{2}}-1\right)
\tag{3-28}
\end{equation}

式(3-27)以及式(3-28)中 $m_{z}$,$\delta_{z}$ 为高斯随机变量 $z=10log^{\xi}$ 的均值与均方差,均方差以奈培为单位。

在对 NLOS 误差的处理时,我们将测量的基站和手持终端之间的距离作为对真实距离的一种近似,计算出 NLOS 引起附加时延,然后从测量距离中减去 NLOS 引起的附加距离,作为基站和手持终端传播距离的近似估计,最后对处理后的数据采用两步最大似然估计方法估计手持终端的位置,并对不同时刻的手持终端进行平滑,进一步减小 NLOS 的误差影响。

\subsection{3.3 模型的求解}

\subsubsection{3.3.1 误差估计}

由于在考虑 NLOS 传播方式情况下,无线信号在由基站传播到手持终端时,无线信号遇到障碍物会发生信号的折射、反射,信号也会存在时延,再加上“时间坐标”的不同步和测量误差,因此基站到手持终端的 TOA 测量值会存在一定误差,我们需要减少此系列误差的影响。

无线电波在复杂传播时,估计无线电波从发送点到接收点之间真实传播时间,从数学形式上测量值 $\hat{x}$、真实值 $x$、之间有这么一个关系

\begin{equation}
\hat{x}=f(x,\omega)
\end{equation}

这个关系也可以写出真实值 $x$ 关于测量值 $\hat{x}$、误差 $\omega$ 之间的函数:

\begin{equation}
x=f(\hat{x},\omega)
\end{equation}

这个式子表明,如果知道了 $f$,给定了测量值和误差后,可以估算出真实值。因此寻找这个函数 $f$ 至关重要。

根据测试数据确定真实值、测量值、误差之间的关系 $f$。测试数据给出了基站的三维坐标,基站到终端的 TOA 测量值,以及终端的定位结果。这里的定位结果,可以视为真实值。

第一步,根据测试数据提供的所有基站的三维坐标以及所有终端的定位结果,计算出各基站到各终端之间的真实距离值;

第二步:根据无线电信号的传播速度以及测试数据中各基站到各终端之间的测量 TOA 值,计算出测量距离值;

第三步:算出真实距离与测量距离之间的误差;

第四步:将误差拟合成关于 TOA 测量值的函数。图 3.2 是测试 case1 的第一个终端的误差关于测量 TOA 的图像,从图像可以看出,该终端的误差关于测试 TOA 几乎是线性的。图 3.3-图 3.7 分别给出了测试 case1 到 case5 的每个终端误差关于测量 TOA 的图

像。从这些图可以看出同一个 case 的每个终端的误差关于测量 TOA 的图像都是线性的,并几乎是重合的,这说明对同一个 case,真实值、测量值、误差之间的关系 \( f \) 几乎是相同的,并且是线性的关系。把误差拟合成测量 TOA 的线性函数

\[
\omega = (1 - a) \hat{x} + d
\]

所以真实 TOA(本文用真实 TOA 表示无线电信号的到达时间 TOA 减去因非视距传播而导致的延迟,以及测量误差和不同步性导致的延迟):

\[
x = a \hat{x} - d
\tag{3-29}
\]

\begin{figure}[h]
    \centering
    \includegraphics[width=0.8\textwidth]{image1.png}
    \caption{测试 Case1 的第一个终端的误差关于测量 TOA 的拟合}
\end{figure}

\begin{figure}[h]
    \centering
    \includegraphics[width=0.8\textwidth]{image2.png}
    \caption{测试 Case1 各终端误差关于测量 TOA 的拟合}
\end{figure}

\begin{figure}[h]
    \centering
    \includegraphics[width=0.8\textwidth]{image3.png}
    \caption{在测试 Case2 各终端误差关于测量 TOA 的拟合}
\end{figure}

\begin{figure}[h]
    \centering
    \includegraphics[width=\textwidth]{image1.png}
    \caption{在测试 Case3 各终端误差关于测量 TOA 的拟合}
\end{figure}

\begin{figure}[h]
    \centering
    \includegraphics[width=\textwidth]{image2.png}
    \caption{在测试 Case4 情况下,误差关于测量 TOA 的拟合}
\end{figure}

\begin{figure}[h]
    \centering
    \includegraphics[width=\textwidth]{image3.png}
    \caption{在测试 Case5 各终端误差关于测量 TOA 的拟合}
\end{figure}

每个 case 的每个终端的误差与测量 TOA 值的线性拟合函数参数 \(a, b\) 如表 3.1-表 3.2 所示,这里只给出了每个 case 的前面 10 个终端的结果。从表 3.1 可以明显地看出,每个 case 的各个终端的参数 \(a\) 几乎是相同的,参数 \(d\) 有细微差别。这是因为每个 case 表示的是一个场景,这进一步说明,每个场景的误差与测量值之间的关系参数 \(a, d\) 是大致相同的,这也说明真实值、测量值、误差之间的关系 \(f\) 很大程度取决于场景。\(a\) 的大小有 LOS 占的比例决定,LOS 越多,\(a\) 就越大,说明该场景中 LOS 越多,因此可以推断出 case2、case4、case5 是 LOS 环境,而 case1、case3 很有可能是 NLOS 环境。\(b\) 代表了一些其他的延迟,比如计时的不同步以及测量误差等。

\begin{table}[h]
    \centering
    \caption{测试 case1-3 的误差拟合数据}
    \begin{tabular}{c|c c|c c|c c}
        \hline
        \rowcolor{red!20} 终端号 & \multicolumn{2}{c|}{case1} & \multicolumn{2}{c|}{case2} & \multicolumn{2}{c}{case3} \\
        \rowcolor{red!20} & 参数 \(a\) & 参数 \(d\) & 参数 \(a\) & 参数 \(d\) & 参数 \(a\) & 参数 \(d\) \\
        \hline
        1 & 0.707511417 & 2.09E-09 & 0.970855074 & 1.90E-09 & 0.839344831 & 1.41E-09 \\
        2 & 0.706772194 & 1.17E-09 & 0.970809235 & 1.68E-09 & 0.839370965 & 1.32E-09 \\
        3 & 0.706817016 & 8.93E-10 & 0.97069471 & 1.44E-09 & 0.839541269 & 1.81E-09 \\
        4 & 0.706871871 & 1.30E-09 & 0.970547585 & 1.26E-09 & 0.839471435 & 1.68E-09 \\
        5 & 0.706821776 & 1.01E-09 & 0.970597676 & 1.42E-09 & 0.839229605 & 8.30E-10 \\
        \hline
    \end{tabular}
\end{table}

\begin{table}
\centering
\begin{tabular}{c c c c c c c}
6 & 0.706482757 & 6.74E-10 & 0.970419974 & 1.02E-09 & 0.839279621 & 1.21E-09 \\
7 & 0.707040611 & 1.59E-09 & 0.970519682 & 1.30E-09 & 0.839178635 & 1.04E-09 \\
8 & 0.706887346 & 1.24E-09 & 0.970617154 & 1.32E-09 & 0.839429564 & 1.45E-09 \\
9 & 0.706803661 & 1.01E-09 & 0.970596649 & 1.41E-09 & 0.83954727 & 1.67E-09 \\
10 & 0.706792335 & 1.33E-09 & 0.970806299 & 1.73E-09 & 0.839263214 & 1.19E-09 \\
\end{tabular}
\end{table}

\begin{table}
\centering
\caption{表3.2测试case 4-5的误差拟合数据}
\begin{tabular}{c c c c c}
\hline
\multicolumn{1}{c}{终端号} & \multicolumn{2}{c}{case 4} & \multicolumn{2}{c}{case 5} \\
 & 参数$a$ & 参数$d$ & 参数$a$ & 参数$d$ \\
\hline
1 & 0.749772779 & 1.16E-09 & 0.842929826 & 2.21E-09 \\
2 & 0.749663838 & 1.14E-09 & 0.842532259 & 1.40E-09 \\
3 & 0.749938896 & 1.64E-09 & 0.842596998 & 1.18E-09 \\
4 & 0.749692937 & 1.13E-09 & 0.842469338 & 8.22E-10 \\
5 & 0.749935805 & 1.59E-09 & 0.842640285 & 1.84E-09 \\
6 & 0.749730428 & 1.10E-09 & 0.842509254 & 1.44E-09 \\
7 & 0.749811123 & 1.49E-09 & 0.842668659 & 1.82E-09 \\
8 & 0.749812719 & 1.51E-09 & 0.842650273 & 1.61E-09 \\
9 & 0.749775386 & 1.11E-09 & 0.842538246 & 1.48E-09 \\
10 & 0.74964603 & 1.07E-09 & 0.842630006 & 1.68E-09 \\
\hline
\end{tabular}
\end{table}

\begin{table}
\centering
\caption{表3.3 利用表3.1中的拟合结果计算真实TOA后定位测试case1-3时的误差}
\begin{tabular}{c c c c c c c c c c}
\hline
\multicolumn{1}{c}{终端号} & \multicolumn{3}{c}{Case 1} & \multicolumn{3}{c}{Case 2} & \multicolumn{3}{c}{Case 3} \\
 & $x$误差 & $y$误差 & $z$误差 & $x$误差 & $y$误差 & $z$轴误差 & $x$误差 & $y$误差 & $z$误差 \\
\hline
1 & 0.24 & 0.08 & 0.48 & 0.04 & 0.19 & 0.28 & 0.21 & 0.19 & 0.13 \\
2 & 0.08 & 0.12 & 0.58 & 0.31 & 0.27 & 0.93 & 0.34 & 0.31 & 0.43 \\
3 & 0.05 & 0.12 & 0.46 & 0.15 & 0.17 & 0.71 & 0.21 & 0.13 & 0.37 \\
4 & 0.01 & 0.08 & 0.38 & 0.23 & 0.33 & 0.12 & 0.21 & 0.27 & 0.86 \\
5 & 0.22 & 0.03 & 0.3 & 0.03 & 0.11 & 0.18 & 0.03 & 0.01 & 0.74 \\
6 & 0.05 & 0.02 & 0.66 & 0.04 & 0.01 & 0.28 & 0.26 & 0.16 & 0.67 \\
7 & 0.03 & 0.05 & 0.24 & 0.23 & 0.43 & 0.01 & 0.08 & 0.03 & 0.94 \\
8 & 0.31 & 0 & 0.93 & 0.11 & 0.3 & 0.57 & 0 & 0.07 & 0.73 \\
9 & 0.22 & 0.03 & 0.23 & 0.07 & 0.01 & 0.69 & 0.17 & 0.24 & 0.05 \\
10 & 0.04 & 0.02 & 0.75 & 0 & 0.08 & 0.89 & 0.06 & 0.04 & 0.62 \\
\hline
\end{tabular}
\end{table}

\begin{table}
\centering
\caption{利用表 3.2 中的拟合结果计算真实 TOA 后定位测试 case4-5 时的误差}
\begin{tabular}{c c c c c c c}
\hline
\multicolumn{1}{c}{} & \multicolumn{3}{c}{Case4} & \multicolumn{3}{c}{Case5} \\
\cline{2-7}
端号 & $x$轴误差 & $y$轴误差 & $z$轴误差 & $x$轴误差 & $y$轴误差 & $z$轴误差 \\
\hline
1 & 0.08 & 0.1 & 0.96 & 0.14 & 0.22 & 0.1 \\
2 & 0.38 & 0.34 & 0.32 & 0.33 & 0.39 & 0.55 \\
3 & 0.1 & 0.06 & 0.1 & 0.15 & 0.08 & 0.82 \\
4 & 0.24 & 0.12 & 0.15 & 0.1 & 0.2 & 0.49 \\
5 & 0.12 & 0.11 & 0.46 & 0.03 & 0.11 & 0.99 \\
6 & 0.01 & 0.04 & 0.12 & 0.21 & 0.22 & 0.3 \\
7 & 0.23 & 0.38 & 0.81 & 0.04 & 0.02 & 0.97 \\
8 & 0.15 & 0.08 & 0.9 & 0.11 & 0.11 & 0.63 \\
9 & 0.16 & 0.22 & 0.48 & 0.11 & 0.09 & 0.29 \\
10 & 0.04 & 0.16 & 0.38 & 0.05 & 0.04 & 0.84 \\
\hline
\end{tabular}
\end{table}

\subsection{3.3.2 模型的选择}

本文将通过估计真实 TOA,从而采用不考虑 NLOS 的 TOA 定位方法。在 3.2 中,给出了三个问题的三个模型。用模型一时,所的结果产生了虚数,这无法解释。比如用该模型定位测试 case1 的第一个终端时的结果为 $(21.246573589246940, 4.515959701973359, -2.135151153198069i)$,虽然横坐标和纵坐标值比较精确,但竖坐标是一个虚数,以为出现了负数开平方,采用的最大似然估计无法保证非负性。模型二通过表线性方程组的求解转化为一个约束优化问题,虽然保证了非负性,但由于约束比较复杂,找可行解是比较困难的。所以选择模型三,模型三容易求解,直接用 matlab 命令 A\b 就可以求解那个线性方程。

\subsection{3.3.3 终端定位算法}

虽然在前面对测试数据的误差分析中发现,真实 TOA 关于测量 TOA 是一个线性关系,但这个关系是与场景有关的,不是固定的,甚至对每个终端都不一样。因此无法事先估计出真实 TOA。我们根据 $a, d$ 合适范围,比如 $a \in [0.5, 1]$, $b \in [0, 3] \times 10^{-9}$,按某个固定步长遍历搜索出合适的 $a, d$ 使得在此 $a, d$ 下的定位误差最小。整个过程 $a, d$ 和定位同时确定。该算法的详细步骤如下:

\begin{enumerate}
    \item 第一步:计算 (3-18) 中的系数矩阵 $A$,确定参数 $a$ 的范围 $[a_L, a_u]$,参数 $d$ 的范围 $[d_L, d_u]$,令误差 $\varepsilon^* = \inf$ 为一个很大的数。
    \item 第二步:初始化参数 $a = a_L$。
    \item 第三步:初始化参数 $d = d_L$。
    \item 第四步:校正测量数据 $\widetilde{\phi}_j = a \phi_{ij} - d, j = 1, 2, \cdots, N$
    \item 第五步:用 $\widetilde{\phi}_j$ 计算 (3-18) 中的 $b$;定位 $\widetilde{X} = A \backslash b$;计算终端到每个基站 $j$ 的距离 $L_j \ (j = 1, 2, \cdots N)$ 以及误差 $\tilde{\varepsilon} = \sqrt{\sum_{j=1}^N \big(L_j - c \widetilde{\phi}_j\big)^2}$。
    \item 第六步:比较 $\varepsilon^*$ 与 $\tilde{\varepsilon}$ 的大小,若 $\tilde{\varepsilon} < \varepsilon^*$,则 $\mathbf{X}^* = \widetilde{\mathbf{X}}$, $\varepsilon^* = \tilde{\varepsilon}$,再继续往下执行;否则直接往下执行。
\end{enumerate}

第七步:若 $d < d_u$,则令 $d = d + \Delta d$,转第四步;否则转第八步;

第八步:若 $a < a_u$,则令 $a = a + \Delta a$,转第三步,否则继续往下执行。

第九步:输出结果 $\mathbf{X}^*$,$\varepsilon^*$。

终端定位流程图如图 3.8 所示。

\begin{figure}[h]
    \centering
    \includegraphics[width=\textwidth]{flowchart.png}
    \caption{移动终端空间三维定位流程图}
    \label{fig:flowchart}
\end{figure}

\subsection{3.3.4 z坐标非负性的修正}

当定位出的终端的竖坐标 $z$ 为负值时,显然不符合实际意义,有必要对其进行修正。考虑到前面对测试数据进行误差分析时发现,定位出的终端的横坐标和纵坐标对误差不敏感,因此可以认为采用上述方法定位处的位置的横、纵坐标 $(x, y)$ 是准确的,只是高度 $z$ 不准确而已,那就说明正确的定位点应该在过点 $(x, y, 0)$ 垂直地面向上的射线上。因此可以在 $[0, |z|]$ 这个区间上找一个合适的值作为 $z$ 的修正值。实际计算中,可以从 0 开始,以

\section*{3.3.5 算法改进}

为了确保精度,$a, d$ 的遍历步长不能太小,特别是 $d$ 的遍历步长不能太小。考虑到在对测试数据进行误差分析时,发现同一个场景的对应于不同终端的 $a$ 非常接近,因此对同一个 case,先取前面 10 个终端或者 20 个终端,按上述算法在一个比较大的范围内计算各个终端的 $a, d$,然后根据这些计算出的 $a$,可以确定该 case 对应的场景的 $a$,的更小范围,从而大大提高计算速度。

\section*{3.3.6 结果及分析}

具体计算中 $\Delta a = 0.001, \Delta d = 0.1 \times 10^{-9}, \Delta z = 0.01$。采用如上改进算法,对给定的 10 组数据,为其中每个 case 中每个终端根据测量数据进行定位。表 3.5 和表 3.6 分别给出了竞赛 Case1 和 case10 的前 10 个终端三维的那定位坐标、参数取值以及误差值。从中可以看出同一个 case 中的每个终端的测量 TOA 与真实 TOA 的线性关系参数 $a$ 几乎是相同。

表 3.5 和表 3.6 中的误差表示定位的终端到各基站距离与由真实 TOA 计算的距离之间的误差总和。对于竞赛数据 Case1,最小的误差值只有 1.285628,最大的误差值也只有 6.970913;对于竞赛数据 Case10,最小的误差值只有 1.063387,最大的误差值也只有 3.368365,由此可见对于手持终端三维坐标的定位,我们使用的模型方法存在的误差性比较小,手持终端三维坐标值的定位是比较精准的。

\begin{table}[h]
\centering
\caption{表 3.5 竞赛数据 Case1 前 10 个终端三维坐标、参数取值以及误差值}
\begin{tabular}{l c c c c c c}
\hline
\rowcolor[HTML]{C0C0C0} 
竞赛 Case1 & $x$ 轴坐标 & $y$ 轴坐标 & $z$ 轴坐标 & 参数 $a$ & 参数 $d$ & 误差值 \\
\hline
第 1 个终端 & -284.40 & 89.73 & 3.89 & 0.735 & 1.20E-09 & 1.285628 \\
第 2 个终端 & 99.91 & 60.45 & 3.89 & 0.734 & 0.00E+00 & 1.866557 \\
第 3 个终端 & -45.65 & -3.98 & 2.25 & 0.734 & 0.00E+00 & 1.674168 \\
第 4 个终端 & 11.62 & 158.38 & 3.88 & 0.736 & 3.00E-10 & 2.251318 \\
第 5 个终端 & -18.81 & 297.04 & 3.88 & 0.739 & 1.50E-09 & 4.805203 \\
第 6 个终端 & -47.93 & -34.97 & 0.19 & 0.734 & 4.00E-10 & 2.307780 \\
第 7 个终端 & -315.55 & 278.95 & 3.88 & 0.738 & 0.00E+00 & 6.970913 \\
第 8 个终端 & 145.6 & -273.61 & 0.76 & 0.734 & 1.00E-09 & 1.923968 \\
第 9 个终端 & -201.25 & -320.55 & 3.88 & 0.736 & 8.00E-10 & 1.422452 \\
第 10 个终端 & 105.44 & -85.58 & 3.88 & 0.735 & 0.00E+00 & 1.409902 \\
\hline
\end{tabular}
\end{table}

\begin{table}[h]
\centering
\caption{表 3.6 竞赛数据 Case10 前 10 个终端三维坐标、参数取值以及误差值}
\begin{tabular}{l c c c c c c}
\hline
\rowcolor[HTML]{C0C0C0} 
竞赛 Case10 & $x$ 轴坐标 & $y$ 轴坐标 & $z$ 轴坐标 & 参数 $a$ & 参数 $d$ & 误差值 \\
\hline
第 1 个终端 & -269.79 & 230.49 & 3.15 & 0.989 & 0.00E+00 & 3.368365 \\
第 2 个终端 & -20.37 & 231.05 & 3.17 & 0.988 & 0.00E+00 & 3.161159 \\
第 3 个终端 & -265.56 & 74.1 & 3.23 & 0.995 & 2.00E-09 & 1.07929 \\
第 4 个终端 & 12.51 & -40.26 & 0.21 & 0.994 & 1.70E-09 & 1.063387 \\
第 5 个终端 & -153.06 & -131.57 & 3.18 & 0.994 & 1.90E-09 & 1.56944 \\
第 6 个终端 & -121.18 & -180.08 & 0.05 & 0.991 & 8.00E-10 & 1.972674 \\
第 7 个终端 & 238.58 & 254.06 & 3.2 & 0.995 & 3.00E-10 & 2.261931 \\
第 8 个终端 & 121.72 & 61.26 & 1.95 & 0.991 & 0.00E+00 & 2.517375 \\
第 9 个终端 & 191.77 & -201.18 & 3.17 & 0.99 & 1.00E-09 & 3.360141 \\
第 10 个终端 & -25.96 & 120.49 & 0.28 & 0.99 & 5.00E-10 & 3.336864 \\
\hline
\end{tabular}
\end{table}

\subsection*{3.4 模型检验}

为了检验本文提出的三维定位算法的性能,在 LOS 或者 NLOS 环境下,本文采用该模型对给定的 5 组测试数据 (sample\_case001\_input.txt 到 sample\_case005\_input.txt) 进行求解定位,然后把所得的定位结果,与所给定的的 5 组终端正确位置进行对比,误差结果如下表 3.7-3.8 所示。表 3.7-3.8 给出了每组数据中的前 10 个终端三维坐标 \(x, y, z\) 轴上的误差。从表 3.7-3.8 中结果可以看出,定位结果的横、纵坐标的是比较准确的,而竖坐标有所偏大,但也没有超过 1m。综上所述,对相关的数据进行分析后可知,在误差允许的范围内,采用我们的模型对室内手持终端的三维坐标进行定位,结果还是比较精准的。

\begin{table}[h]
\centering
\caption{测试 Case1—Case3 的前 10 个终端的三维坐标的误差值}
\begin{tabular}{c c c | c c c | c c c}
\hline
\multicolumn{3}{c|}{Case 1} & \multicolumn{3}{c|}{Case 2} & \multicolumn{3}{c}{Case 3} \\
\hline
\(x\) 轴误差 & \(y\) 轴误差 & \(z\) 轴误差 & \(x\) 轴误差 & \(y\) 轴误差 & \(z\) 轴误差 & \(x\) 轴误差 & \(y\) 轴误差 & \(z\) 轴误差 \\
\hline
0.22 & 0.1 & 0.48 & 0.06 & 0.55 & 0.29 & 0.08 & 0.28 & 0.13 \\
0.38 & 0.1 & 0.12 & 0.96 & 0.72 & 0.93 & 0.21 & 0.93 & 0.43 \\
0.12 & 0.37 & 0.95 & 1.39 & 1.26 & 0.46 & 0.86 & 0.01 & 0.12 \\
0 & 0.08 & 0.38 & 0.18 & 0.53 & 0.12 & 0.74 & 0.24 & 0.98 \\
0.14 & 0.12 & 0.3 & 1.4 & 0.48 & 0.18 & 0.08 & 0.02 & 0.74 \\
0.09 & 0.18 & 0.74 & 0.18 & 0.03 & 0.57 & 0.38 & 0.35 & 0.47 \\
0.72 & 0.31 & 0.24 & 0.57 & 1.45 & 0.99 & 0.17 & 0.72 & 0.94 \\
0.67 & 0.7 & 0.93 & 0.45 & 0.22 & 0.57 & 0.69 & 0.94 & 0.1 \\
0.15 & 0.4 & 0.23 & 1.0 & 0.76 & 0.69 & 0.42 & 0.1 & 0.05 \\
0.05 & 0.01 & 0.75 & 0.89 & 1.48 & 0.36 & 0.12 & 0.32 & 0.82 \\
\(\cdots\) & \(\cdots\) & \(\cdots\) & \(\cdots\) & \(\cdots\) & \(\cdots\) & \(\cdots\) & \(\cdots\) & \(\cdots\) \\
\hline
\end{tabular}
\end{table}

\begin{table}[h]
\centering
\caption{测试 Case4—Case5 的前 10 个终端的三维坐标的误差值}
\begin{tabular}{c c c | c c c}
\hline
\multicolumn{3}{c|}{Case 4} & \multicolumn{3}{c}{Case 5} \\
\hline
\(x\) 轴误差 & \(y\) 轴误差 & \(z\) 轴误差 & \(x\) 轴误差 & \(y\) 轴误差 & \(z\) 轴误差 \\
\hline
0.04 & 0.13 & 0.96 & 0.05 & 0.2 & 0.1 \\
0.64 & 0.63 & 0.33 & 0.18 & 0.4 & 0.55 \\
0.07 & 0.06 & 0.1 & 0.79 & 0.19 & 0.08 \\
0.28 & 0.1 & 0.15 & 0.65 & 0.68 & 0.49 \\
0.39 & 0.03 & 0.46 & 0.53 & 0.27 & 0.99 \\
0.17 & 0.12 & 0.12 & 0.92 & 0.22 & 0.3 \\
0.28 & 0.37 & 0.81 & 0.49 & 0.83 & 0.97 \\
0.06 & 0.1 & 0.05 & 0.72 & 2.08 & 0.33 \\
0.2 & 0.45 & 0.48 & 0.78 & 0.56 & 0.29 \\
0.85 & 0.12 & 0.38 & 0.01 & 0.23 & 0.84 \\
\(\cdots\) & \(\cdots\) & \(\cdots\) & \(\cdots\) & \(\cdots\) & \(\cdots\) \\
\hline
\end{tabular}
\end{table}

\section{问题二模型建立与求解}

\subsection{问题的描述及分析}

问题 2 中提供了 10 组 TOA 测量数据和所有基站的三维坐标(对应附录中编号为 case011_input.txt 到 case020_input.txt),并且要求我们根据已知数据设计算法,能够用尽可能少的基站数目,实现近似最优的三维定位精度。根据对问题 1 的建模已知,实现一个终端的定位至少需要三个基站,如果采用方法 2,至少需要四个基站。但是,在 NLOS 传播环境中以及宽带、信噪比、时钟同步影响,信号的反射、折射,传播时间延迟等诸多影响因素存在的情况下如果要实现用最少基站数对终端的精准定位,主要存在以下挑战:

1. 根据问题一所述,利用所有基站对该终端进行定位,可知定位是准确的。以此为依据,按某个规则减少基站剔除误差最大的基站,重新定位终端,直到定位状态达到稳定为止,其定位结果与用所有基站定位得到的结果无限接近。但是,这种方法,由于每次计算都要用较多的基站,计算量大,成本高,并且最终被保留的基站总数不能完全保证是最小基站数。

2. 如果按照某种规则,选择 4 个基站对终端进行定位,然后再用同样规则加一个基站进行定位,直到两次定位结果无限相近。虽然该方法虽然满足了基站总数目最少,但无法跟较精确的定位比较。

通过以上问题的分析,解决方法思路如下:结合将上述两种方式结合到一起,克服了上述存在的缺点,先用所有的基站给该终端定位,由问题一可知,该定位应该可以认为是准确的。然后在按照某个规则,比如按测量值从小到大,选择 4 个基站对终端进行定位,如果定位结果与用所有基站地位的结果相近,就说明 4 就是最少基站数,如果两种定位结果不相近,再用同样规则加一个基站进行定位,与用所有基站定位的结果比较,相近则得到最少基站数,依次下去,最终得到某一个终端的最少定位基站数,即所有单个终端定位基站的并集取值最小。

\subsection{模型的求解}

\subsubsection{算法及其流程图}

在考虑 NLOS 存在误差的情况下,该算法的详细步骤如下:

第一步:对测量 TOA 值按从小到大排序,并对基站进行相应的排序。

第二步:应用问题一中的终端定位算法,使用所有基站测量数据对终端进行定位,得到基本准确定位解。

第三步:选择前面测量 TOA 值最小的 4 个基站对终端进行定位,如果定位结果与第二步的基本准确结果相差不大,就说明 4 就是最少基站数,否则转第四步。

第四步:按测量 TOA 值从小到大从剩下的基站中选一个基站,增加该基站站进行定位。

第五步:如果定位结果与第二步的基本准确结果相差不大,则停,此时的基站数就是最少基站数。否则转第四步。

\begin{figure}[h]
    \centering
    \includegraphics[width=\textwidth]{flowchart.png}
    \caption{流程图}
    \label{fig:flowchart}
\end{figure}

\subsection{结果及分析}

在考虑实际情况下,先用所有的基站给该终端定位,然后按测量时间的规则排序,并一次从中选择 4 个基站对终端进行定位,如果两种定位结果不相近,再用同样规则加一个基站进行定位,直至与用所有基站定位的结果无限接近,这样用最少的基站数目求解出 10 组近似最优的三维坐标值。由于篇幅问题,此处我们随机例举了竞赛数据 case11 和 case13 前 10 个终端的结果,结果及真实距离与测量距离之间的误差如表 \ref{tab:results1} 和表 \ref{tab:results2} 所示。

\begin{table}[h]
    \centering
    \caption{结果 1}
    \label{tab:results1}
    \begin{tabular}{|c|c|c|c|}
        \hline
        终端编号 & 真实坐标 & 测量坐标 & 误差 \\
        \hline
        1 & (x1, y1, z1) & (x1', y1', z1') & e1 \\
        2 & (x2, y2, z2) & (x2', y2', z2') & e2 \\
        3 & (x3, y3, z3) & (x3', y3', z3') & e3 \\
        \vdots & \vdots & \vdots & \vdots \\
        10 & (x10, y10, z10) & (x10', y10', z10') & e10 \\
        \hline
    \end{tabular}
\end{table}

\begin{table}[h]
    \centering
    \caption{结果 2}
    \label{tab:results2}
    \begin{tabular}{|c|c|c|c|}
        \hline
        终端编号 & 真实坐标 & 测量坐标 & 误差 \\
        \hline
        1 & (x1, y1, z1) & (x1', y1', z1') & e1 \\
        2 & (x2, y2, z2) & (x2', y2', z2') & e2 \\
        3 & (x3, y3, z3) & (x3', y3', z3') & e3 \\
        \vdots & \vdots & \vdots & \vdots \\
        10 & (x10, y10, z10) & (x10', y10', z10') & e10 \\
        \hline
    \end{tabular}
\end{table}

\begin{table}
\centering
\caption{竞赛数据 case11 前 10 个终端以及对应所需基站数目、基站下标}
\begin{tabular}{l c l}
\hline
\textbf{终端号} & \textbf{所需基站数} & \textbf{基站下标} \\
\hline
第1个终端 & 5 & 21、2、12、6、13 \\
第2个终端 & 5 & 25、19、14、29、8 \\
第3个终端 & 5 & 1、5、7、23、11 \\
第4个终端 & 8 & 21、2、12、6、13、4、18、23 \\
第5个终端 & 4 & 6、12、23、4 \\
第6个终端 & 5 & 22、10、16、30、20 \\
第7个终端 & 7 & 12、21、2、6、4、23、5 \\
第8个终端 & 4 & 28、7、11、1 \\
第9个终端 & 4 & 24、20、3、30 \\
第10个终端 & 7 & 11、28、7、1、5、23、16 \\
\hline
\end{tabular}
\end{table}

\begin{table}
\centering
\caption{竞赛数据 case13 前 10 个终端以及对应所需基站数目、基站下标}
\begin{tabular}{l c l}
\hline
\textbf{终端号} & \textbf{所需基站数} & \textbf{基站下标} \\
\hline
第1个终端 & 4 & 29、49、30、24 \\
第2个终端 & 4 & 4、1、48、19 \\
第3个终端 & 4 & 21、14、16、33 \\
第4个终端 & 7 & 16、14、7、13、6、32、27 \\
第5个终端 & 13 & 30、40、8、12、9、29、36、49、24、31、22、5、34、 \\
第6个终端 & 4 & 13、7、32、19 \\
第7个终端 & 5 & 25、35、37、28、33 \\
第8个终端 & 6 & 42、20、18、2、17、11 \\
第9个终端 & 4 & 26、15、3、50 \\
第10个终端 & 5 & 32、7、33、4、14 \\
\hline
\end{tabular}
\end{table}

从表 4.1 和表 4.2 中可知,对于在同一场景下参数 $a$ 和参数 $b$ 值是不断变化的,但是变化范围基本处在稳定范围内,并且随着参数 $a$ 和参数 $b$ 的不断变化,终端的三维坐标值也在做相应的误差修正,从而得实现精确的三维坐标定位。通过表格可以看出,竞赛提供的数据 case11 前 10 个终端的误差值中最大只有 1.99909,最小的误差值为 0.214166;case11 前 10 个终端的误差值中最大只有 3.207604,最小的误差值为 0.402552。由数据可知,使用我们提出的模型算法,能够很好地实现用尽可能少的基站数目,得到极为精确的三维定位坐标值。由于篇幅问题,此处我们随机例举了竞赛数据 case11 和 case13 对应的终端所需基站数目柱状图如图 4.2,4.3。

\begin{figure}[h]
    \centering
    \includegraphics[width=0.8\textwidth]{image1.png}
    \caption{在测试数据 case11 情况下,基站数目和终端数目效果图}
    \label{fig:case11}
\end{figure}

\begin{figure}[h]
    \centering
    \includegraphics[width=0.8\textwidth]{image2.png}
    \caption{在测试数据 case13 情况下,基站数目和终端数目效果图}
    \label{fig:case13}
\end{figure}

由图 \ref{fig:case11},图 \ref{fig:case13} 可以看出,通过我们的算法仅需要 4 个基站几乎就可以实现对全部数现存的移动终端实现近似最优的三维定位精度,只有极少的终端需要全部的终端数才能实现精确定位。

表 4.3 case11 和 case15 使用最少基站定位时各基站的使用频率频数,从表可以看出,分布是比较均匀的,因为定位时往往采用力终端最近的基站的数据。基站的发布一般成蜂窝状,其实就是这个道理。

\begin{table}[h]
    \centering
    \caption{case11 和 case15 使用最少基站定位时各基站的使用频率频数}
    \label{tab:case11_case15}
    \begin{tabular}{c c c c c c}
        \hline
        \multicolumn{3}{c}{case 11} & \multicolumn{3}{c}{Case 15} \\
        \hline
        基站序号 & 使用频数 & 使用频率 & 基站序号 & 使用频数 & 使用频率 \\
        \hline
        1 & 211 & 0.030993 & 1 & 301 & 0.046659 \\
        2 & 164 & 0.024089 & 2 & 328 & 0.050845 \\
        3 & 216 & 0.031727 & 3 & 247 & 0.038289 \\
        4 & 303 & 0.044506 & 4 & 149 & 0.023097 \\
        5 & 281 & 0.041275 & 5 & 418 & 0.064796 \\
        6 & 319 & 0.046857 & 6 & 309 & 0.0479 \\
        7 & 217 & 0.031874 & 7 & 180 & 0.027903 \\
        8 & 216 & 0.031727 & 8 & 230 & 0.035653 \\
        9 & 250 & 0.036722 & 9 & 374 & 0.057976 \\
        10 & 274 & 0.040247 & 10 & 306 & 0.047435 \\
        11 & 309 & 0.045388 & 11 & 331 & 0.05131 \\
        12 & 264 & 0.038778 & 12 & 444 & 0.068827 \\
        & & & 24 & & \\
        \hline
    \end{tabular}
\end{table}

\begin{table}
\centering
\begin{tabular}{c c c | c c c}
\hline
13 & 204 & 0.029965 & 13 & 539 & 0.083553 \\
14 & 294 & 0.043184 & 14 & 489 & 0.075802 \\
15 & 83 & 0.012192 & 15 & 237 & 0.036738 \\
16 & 315 & 0.046269 & 16 & 172 & 0.026663 \\
17 & 189 & 0.027761 & 17 & 587 & 0.090994 \\
18 & 150 & 0.022033 & 18 & 234 & 0.036273 \\
19 & 225 & 0.033049 & 19 & 220 & 0.034103 \\
20 & 230 & 0.033784 & 20 & 356 & 0.055185 \\
21 & 192 & 0.028202 & & & \\
22 & 244 & 0.03584 & & & \\
23 & 301 & 0.044213 & & & \\
24 & 196 & 0.02879 & & & \\
25 & 149 & 0.021886 & & & \\
26 & 204 & 0.029965 & & & \\
27 & 114 & 0.016745 & & & \\
28 & 180 & 0.026439 & & & \\
29 & 270 & 0.039659 & & & \\
30 & 244 & 0.03584 & & & \\
\hline
\end{tabular}
\end{table}

\section{问题三模型建立与求解}

\subsection{问题的描述及分析}

该问题要求我们根据终端在移动过程中采集到的 TOA 数据,确定终端的移动轨迹。该问题的本质即定位终端的位置,与问题一相似。因此我们可以使用问题一的模型方法对问题三进行求解,对采集位置进行定位。通过分析已知的数据条件,我们发现在已知的数据中并没有给出竖坐标 $Z$,这说明基站和移动终端始终在同一个平面上。给定的 TOA 测量矩阵记录的是一个终端在移动过程中多个位置对于不同基站的 TOA 测量,其中第 $i$ 行中第 $j$ 个元素,表示终端移动到标号为 $i$ 的位置上时,相对于标号为 $j$ 的基站的 TOA 测量值。为了求出运动轨迹,可以认为位置是按照时间顺序排列的,或者说可以认为移动终端持有者先到第 $i$ 行对应的采集位置,再到第 $i+1$ 行对应的采集位置。采用拟合技术,可以得到移动终端的轨迹方程。

\subsection{问题的建模与求解}

该问题是一个二维问题,所以问题一种的模型要做适当修改,主要是方程 (3-18) 应该修改成如下:

\begin{equation}
AX = b
\tag{5-1}
\end{equation}

其中 $A \in R^{(N-1) \times 2}$, $b \in R^{(N-1)}$

\begin{equation}
A = 
\begin{bmatrix}
2(x_1 - x_2) & 2(y_1 - y_2) \\
2(x_1 - x_3) & 2(y_1 - y_3) \\
\vdots & \vdots \\
2(x_1 - x_N) & 2(y_1 - y_N)
\end{bmatrix}
\end{equation}

\begin{equation}
b = 
\begin{bmatrix}
c^2\big(\phi_2^2 - \phi_1^2\big) + (x_1^2 + y_1^2) - (x_2^2 + y_2^2) \\
c^2\big(\phi_3^2 - \phi_1^2\big) + (x_1^2 + y_1^2) - (x_3^2 + y_3^2) \\
\vdots \\
c^2\big(\phi_N^2 - \phi_1^2\big) + (x_1^2 + y_1^2) - (x_N^2 + y_N^2)
\end{bmatrix}
\end{equation}

\subsection{结果及分析}

对于问题三,在已知 5 组数据的条件下,我们采用问题一的模型方法进行求解。以下图 5.1-5.5 对应 5 组数据终端移动的轨迹图及每一组数据中所有终端的坐标散点图。

\begin{figure}[h]
    \centering
    \includegraphics[width=0.45\textwidth]{image1.png}
    \includegraphics[width=0.45\textwidth]{image2.png}
    \caption{竞赛数据 Case21 中前 10 个终端的移动轨迹图及所有终端的坐标散点图}
    \label{fig:5.1}
\end{figure}

\begin{figure}[h]
    \centering
    \includegraphics[width=0.45\textwidth]{image3.png}
    \includegraphics[width=0.45\textwidth]{image4.png}
    \caption{竞赛数据 Case22 中前 10 个终端的移动轨迹图及所有终端的散点图}
    \label{fig:5.2}
\end{figure}

\begin{figure}[h]
    \centering
    \includegraphics[width=0.45\textwidth]{image5.png}
    \includegraphics[width=0.45\textwidth]{image6.png}
    \caption{竞赛数据 Case23 中前 10 个终端的移动轨迹图及所有终端的坐标散点图}
    \label{fig:5.3}
\end{figure}

\begin{figure}[h]
    \centering
    \includegraphics[width=0.45\textwidth]{image1.png}
    \includegraphics[width=0.45\textwidth]{image2.png}
    \caption{竞赛数据 Case24 中前 10 个终端的移动轨迹图及所有终端的坐标散点图}
    \label{fig:case24}
\end{figure}

\begin{figure}[h]
    \centering
    \includegraphics[width=0.45\textwidth]{image3.png}
    \includegraphics[width=0.45\textwidth]{image4.png}
    \caption{竞赛数据 Case25 中前 10 个终端的移动轨迹图及所有终端的坐标散点图}
    \label{fig:case25}
\end{figure}

图 5.1-5.5 对应 5 组数据(竞赛数据中 Case21-25)前 10 个终端移动的轨迹图及每组数据中所有终端的坐标散点图,由于空间有限,我们只例举了每个 Case 中前 10 个终端的移动轨迹图作为代表。从以上 5 组图中可以看出,终端的移动轨迹是按照图中箭头的指向进行移动的。

\subsection{5.3 模型检验}

为了检验本文提出的三维定位优化算法对于求解终端移动轨迹的性能,我们同样采用题中给定的 5 组测试数据 (sample\_case001\_input.txt 到 sample\_case005\_input.txt) 对三维定位优化算法求解终端移动轨迹的性能进行验证,可得到相对应的 5 组实验结果。然后,将所得的移动终端的位置坐标与已知的 5 组终端坐标的信息文件 (sample\_case001\_ans.txt 到 sample\_case005\_ans.txt) 进行比较,计算出使用该模型方法求解移动终端位置信息与已知的终端位置信息之间的误差值,并对该误差值进行分析。移动终端位置信息误差值如表 5.1 所示:

\begin{table}[h]
    \centering
    \caption{移动终端位置信息误差值}
    \label{tab:position_error}
    \begin{tabular}{c c c c c c c c c c}
        \hline
        \rowcolor{cyan!30} Case21 & \multicolumn{2}{c}{Case22} & \multicolumn{2}{c}{Case23} & \multicolumn{2}{c}{Case24} & \multicolumn{2}{c}{Case25} \\
        \cline{2-10}
        \rowcolor{cyan!30} $x$误差 & $y$误差 & $x$误差 & $y$误差 & $x$误差 & $y$误差 & $x$误差 & $y$误差 & $x$误差 & $y$误差 \\
        \hline
        0.22 & 0.12 & 0.13 & 0.09 & 0.02 & 0.26 & 0.09 & 0.33 & 0.05 & 0.2 \\
        0.05 & 0.12 & 0.24 & 0.37 & 0.06 & 0.17 & 0.31 & 0.27 & 0.18 & 0.4 \\
        0.12 & 0.35 & 0.34 & 0.04 & 0.29 & 0 & 0.23 & 0.01 & 0.03 & 0.13 \\
        0.09 & 0.06 & 0.3 & 0.07 & 0.64 & 0.14 & 0.45 & 0.12 & 0.16 & 0.13 \\
        0.21 & 0.02 & 0.42 & 0.3 & 0.37 & 0.18 & 0.14 & 0.11 & 0.13 & 0.28 \\
        0.02 & 0.22 & 0.15 & 0.14 & 0.22 & 0.31 & 0.09 & 0.07 & 0.22 & 0.22 \\
        0.04 & 0.07 & 0.17 & 0.3 & 0.1 & 0.11 & 0.54 & 0.38 & 0.05 & 0.18 \\
        \hline
    \end{tabular}
\end{table}

\begin{tabular}{c c c c c c c c c} 
\hline 
0.69 & 0.28 & 0.12 & 0.36 & 0.38 & 0.16 & 0.01 & 0.06 & 0.07 & 0.1 \\ 
0.24 & 0.12 & 0.12 & 0.14 & 0.6 & 0.01 & 0.09 & 0.29 & 0.13 & 0.31 \\ 
0.06 & 0.02 & 0.02 & 0.05 & 0.06 & 0.24 & 0.75 & 0.65 & 0.02 & 0.16 \\ 
\ldots & \ldots & \ldots & \ldots & \ldots & \ldots & \ldots & \ldots & \ldots & \ldots \\ 
\hline 
\end{tabular}

由于文章篇幅问题,此处只例举了 5 组数据中的前 10 个移动终端 \(x\), \(y\) 坐标值的误差进行分析,其它终端数据见附录。由表 5.1 可知,在这 5 组数据中,移动终端坐标的 \(x\), \(y\) 轴上,误差最小都为 0.01,在 \(x\) 轴上,误差最大的时候也才只有 0.75,在 \(y\) 轴上,误差最大的情况达到了 0.65。但是总体而言,对于 \(x\), \(y\) 轴来说,一般的误差范围都控制在 0-1 之间。综上所述,对相关的数据进行分析后可知,在误差允许的范围内,采用我们的模型对手持终端的移动轨迹进行定位,结果还是比较精准的。

\section*{6 问题四模型的建立与求解}

\subsection*{6.1 问题描述及分析}

对于问题四,在基站通信范围有限的条件下,需要找出所有能够被基站定位的终端,这是一个与现实网络相接近的问题,我们对该问题的求解,需要从现实网络的角度出发。在现实生活中,对于基站与终端之间的距离无法一一测量得到,因为这是一个比较巨大的工程,对于此类问题,往往通过建立一个准则或者相应的模型衡量度数与定位精度的关系,从而可简单便捷的解决问题,节约大量的人力与物力资源。

在题目中已知的条件下,每个基站的通信半径为 200,则测距数据小于 200 米的,我们都视为有效接入,终端可被认为与该基站之间是建立连接的。通常情况下,一个终端可同时与多个基站建立连接,则某个终端的连接数就是与该终端建立连接的基站数。只要通信半径是确定的,我们则可确定终端的连接数。与连接数密切相关的另一个术语是连接度数,它的为定义为所有可以被定位的终端连接到基站的“边”数的总和与终端数的比值,连接度数的大小取决于基站和终端之间的连接数目。

对于该问题的求解,我们从基站的通信半径入手。首先我们需要确保能够被定位到的终端真实距离值必须小于该基站的通信半径。其次,我们需要考虑能够被定位到终端的坐标值的误差必须在可接受的范围内。然后,我们可确定终端的连接数以及连接度数。

\subsection*{6.2 问题的求解}

\subsubsection{6.2.1 考虑通信半径的终端定位算法}

题中的已知条件为基站的坐标值,基站到终端的 TOA 测量值以及基站的通信半径,根据这些已知条件,我们需要确定在基站通信半径的范围内,所有能够被基站定位到的终端及其被定位到的终端数量。对于给定的某个终端,显然不能直接用测量 TOA 来计算距离并和通信半径比较,必须用问题一种的方法求得的真实 TOA 计算的距离。判断一个终端是否可以定位,如果可以定位,求其连接数的步骤如下:第一步:通常情况下,4 个基站就能够定位一个终端坐标值。由于基站的通信半径为 200m,我们需要计算选取 4 个 TOA 测量值最小的基站,使用这 4 个基站定位一个终端的坐标值,再找出这 4 个基站到该终端 TOA 测量值最大的基站,并通过真实 TOA 计算出此基站与该终端之间的距离,判断此距离值是否小于基站的通信半径。如果大于通信半径,则认为该终端不可定位;

若小于基站的通信半径,转第二步;第二步:增加一个测量 TOA 值最小的基站,使用这些基站定位该终端的坐标值,如果跟前一次的定位结果相差比较小,比如各坐标相差 1m 以内,则可以认为该终端可以定位,转第三步,否则转第四步;第三步:利用定位过程中确定的真实 TOA 与测量 TOA 的线性关系,计算真实距离,计算该终端的连接数。第四步:利用定位过程中确定的真实 TOA 与测量 TOA 的线性关系,计算真实距离,如果有真实距离大于通信半径,则说明该终端不能定位。根据上述算法,求出所有定位中端和其连接数后,可以根据

\[
\lambda = \frac{\text{所有可以被定位终端到基站之间的连接数}}{\text{终端数}}
\]

计算出 case 的连接度。

\subsection*{6.2.2 结果及分析}

在不同的竞赛数据 Case 中,考虑基站的通信半径,使用本文的终端定位算法计算出在基站通信半径的有效范围内,基站能够定位到最大的终端数量。结果如表 6.1 所示:

\begin{table}[h]
\centering
\caption{竞赛数据 Case26-Case 30 定位到的终端数、链接度、平均误差}
\begin{tabular}{|c|c|c|c|c|c|c|c|}
\hline
竞赛 & 终端数 & 基站 & 定位 & 定位中端所 & 连接数 & 连接度数 & 平均误差 \\
数据 & 量 & 数量 & 终端 & 占比例\% & & & \\
& & & 数 & & & & \\
\hline
Case 26 & 1100 & 30 & 193 & 17.5455 & 981 & 0.891 & 145.42 \\
Case 27 & 1200 & 40 & 757 & 63.0833 & 6512 & 5.4267 & 3.3831 \\
Case 28 & 1300 & 50 & 798 & 61.3846 & 6624 & 5.0954 & 8.7548 \\
Case 29 & 1400 & 60 & 1035 & 73.9286 & 13292 & 9.4943 & 3.9234 \\
Case 30 & 1000 & 20 & 468 & 46.8 & 2710 & 2.71 & 2.1353 \\
\hline
\end{tabular}
\end{table}

从表 6.1 可知,对于竞赛数据 Case26,我们可以得知,在该场景中,共有 30 个基站,1100 个终端,但是在基站的通信半径 200m 的范围内,最多有 193 个终端能够被通信半径范围内的基站成功定位;对于竞赛数据 Case27,我们可以得知,在该场景中,共有 40 个基站,1200 个终端,但是在基站的通信半径 200m 的范围内,则有 757 个终端可被通信半径范围内的基站成功定位;对于竞赛数据 Case28,我们可以得知,在该场景中,共有 50 个基站,1300 个终端,但是在基站的通信半径 200m 的范围内,有 798 个终端能够被通信半径范围内的基站成功定位;对于竞赛数据 Case29,我们可以得知,在该场景中,共有 60 个基站,1400 个终端,但是在基站的通信半径 200m 的范围内,最多有 1035 个终端能够被通信半径范围内的基站成功定位;对于竞赛数据 Case30,我们可以得知,在该场景中,共有 20 个基站,1000 个终端,但是在基站的通信半径 200m 的范围内,只有 468 个终端能够被通信半径范围内的基站成功定位。对于 Case26 及 Case30 这两个场景出现能够被定位的终端数量比较少,这说明在这个基站的通信半径范围内,手持终端的密度可能比较低即该区域为使用手持终端的人比较少。对于 Case27-Case29 这两个场景,出现能够被定位的终端数量比较度,造成这种现象的原因,可能是因为在基站的通信半径范围内,手持终端的密度比较高,该基站附近使用手持终端的人比较多。

\subsection*{6.2.3 连接度数与定位精度之间的关系}

连接度数越大,定位终端所占比例越大。从表 6.1 可看出,连接度最高的 case29 中,能定位的终端占了 73.9286\%,而连接度最低的 case21 中,能定位的终端只占总终端的 17.5455。这也是可以理解的,因为能连接的基站越多,被定位的可能性越高。

连接度数越大,平均误差越小,定位精度越高。用定位点与各基站的距离与用估计

的真实 AOT 算出来的距离的差,称为定位误差。显然,定位误差越小,定位精度就越高。从表 6.1 可看出,平均误差最大的,恰好是连接度最小的,除了 case30 外,连接度越大,平均误差就越小。

\section*{7 模型的评价与结论}

本文所建立的移动终端三维空间定位模型具有以下优点:

(1) 本文所建立的移动终端三维空间定位模型考虑了 NLOS 传播方式下的误差,例如传播时的延迟误差,TOA 测量误差等。

(2) 本文建立的大部分模型均为矩阵模型,构造简单,扩展性强,便于计算。

本文所建立的移动终端三维空间定位模型存在的不足之处及需要改进的地方:

(1) 本文所建立的模型虽然考虑了 NLOS 的误差影响,但是误差的精度仍然有提高的空间。

在本文中,我们根据 LOS 以及 NLOS 不同的传播方式,考虑了 NLOS 误差影响,通过对误差及 TOA 测量值的分析,拟合出一个关于 TOA 测量值的线性函数,进而建立了考虑误差的移动终端三维空间定位模型。在测试数据的基础之上,使用的我们的方法计算出的结果与题中给定的结果之间的误差较小,验证了本文提出模型的精确性与实用性。该结果说明,本文建立的模型可以对真实场景下基于无线通信基站室内终端三维坐标定位进行较好的拟合和描述。

\section*{参考文献}

[1]. 刘义, 宗晓征, 柯宏发. 视距传播中地面起伏对传播中值衰减的影响[J]. 电讯技术, 2003, 43(5):103-106.

[2]. 赵云轩. 非视距传播环境下移动台定位算法研究[D]. 吉林大学, 2005.

[3]. 王沁, 何杰, 张前雄, 刘冰峰, 于彦伟. 测距误差分级的室内 TOA 定位算法[J]. 仪器仪表学报, 2011,12:2851-2856.

[4]. 王昕, 王宗欣, 刘石. 一种考虑非视线传播影响的 TOA 定位算法[J]. 通信学报, 2001, 22(3):1-8.

[5]. 田孝华. DS-CDMA 蜂窝网中无线定位与参数估计技术[D]. 西安电子科技大学, 2003.

[6]. 李招华, 汪毓铎, 邵青. 基于 Chan 的 TDOA 三维定位算法[J]. 现代电信科技, 2014(11):36-40.

[7]. 段凯宇, 张力军, 高玲, 等. 两种 NLOS 误差消除及 TOA 定位算法[J]. 信号处理, 2008, 24(4):565-568.

[8]. Chan Y T, Ho K C. A simple and efficient estimator for hyperbolic location[J]. IEEE Transactions on Signal Processing, 1915, 42(8):1905-1915.

\section*{附录一}

\section*{主要程序代码}

\textbf{问题 1 程序}

\textbf{Shitanfa1.m}

\begin{verbatim}
tic
clc;
clear;
load case001input
SJ=case001input;
% load case001input
% SJ=case001input;
c=3.0e+8;%光速
n=SJ(1,1);%基站数目
m=SJ(2,1);%终端数目
w=SJ(3,1);%维数
fid=fopen('output_case_001.txt','w+');
fidindex=fopen('data1index.txt','w+');
%for ZD=1:10%第几个终端
%ZD=1142
%a=1;%衰减系数
%diff=0;
%min=inf;
for i=2:n
    xishu(i-1,:)=(SJ(4,1:w)-SJ(3+i,1:w)).*2;
end

% ZD=1
for ZD=1:m
%ZD=1
    min=inf;
    %a=1;%衰减系数

    % for a=0.5:0.01:1
    %a=1-0.292488582926760;
    %d=-2.090872502954080e-09;

    for a=0.73:0.001:0.74

        for d=0:0.1:2
            d1=-d*10^-9;
            changshu=[];
            for i=2:n
\end{verbatim}

\begin{verbatim}
changshu(i-1)=c^2*((a*SJ(3+n+ZD,i)+d1)^2-(a*SJ(3+n+ZD,1)+d1)^2)+(SJ(4,1)^2+SJ(4,2)^2+SJ(4,3)^2)-(SJ(3+i,1)^2+SJ(3+i,2)^2+SJ(3+i,3)^2);
end
changshu=changshu';
X=xishu\changshu;

% x0=[0;0;0];
% X=fmincon(@(x)
0.5*x'*xishu'*xishu*x-changshu'*xishu*x+0.5*changshu'*changshu+...
% (((x(1)-SJ(4,1)).^2+(x(2)-SJ(4,2)).^2+(x(3)-SJ(4,3)).^2)-(c*SJ(3+n+ZD,1)').^2)^2,x0,...
% [0,0,-1],[0],...
% [],[],[],[]);

% diff=0;
% for i=1:n;
%
diff=diff+(sqrt((X(1)-SJ(3+i,1))^2+(X(2)-SJ(3+i,2))^2+(X(3)-SJ(3+i,3))^2)-(a*c*SJ(4+n,i)))^2;
% end

diff=norm(sqrt((X(1)-SJ(4:3+n,1)).^2+(X(2)-SJ(4:3+n,2)).^2+(X(3)-SJ(4:3+n,3)).^2)-(c*(a*SJ(3+n+ZD,:)' + d1)));
if min>diff
    tempmin=diff;
    tempFX=X;
    tempFa=a;
    tempFd=d1;
    if X(3)>0
        min=diff;
        FX=X;
    end
end
end
end
if min==inf
    min=tempmin;
    FX=tempFX;
    FX(3)=abs(FX(3));
    Fa=tempFa;
    Fd=tempFd;
\end{verbatim}

\begin{verbatim}
end
x1=FX;
x2=FX;
error=
sum((sqrt((FX(1)-SJ(4:3+n,1)).^2+(FX(2)-SJ(4:3+n,2)).^2+(FX(3)-SJ(4:3+n,3)).^2)-c*SJ(3+n+ZD,:).^2);
for step=0:0.01:abs(FX(3))
    x1(3)=step;
    tt=
    sum((sqrt((x1(1)-SJ(4:3+n,1)).^2+(x1(2)-SJ(4:3+n,2)).^2+(x1(3)-SJ(4:3+n,3)).^2)-c*SJ(3+n+ZD,:).^2);
    
    if error<tt;
        error=tt;
        x2(3)=step;
    end
end
FX=x2;

JGofMin(ZD)=min;
JGofFXindex(ZD)=FX(3)<0;
JGofFX(ZD,:)=FX;
% JGofFX(ZD,3)=abs(JGofFX(ZD,3));

JGofFa(ZD)=Fa;
JGofFd(ZD)=Fd;
fprintf(fid,'%7.2f %7.2f %7.2f %f %f %e\n',JGofFX(ZD,1),JGofFX(ZD,2),JGofFX(ZD,3),JGofMin(ZD),JGofFa(ZD),JGofFd(ZD));
fprintf(fidindex,'%f\n',JGofFXindex(ZD));
end
sum(JGofFXindex(:)==1)
toc

问题2程序
Shitanfajinjie11.m
tic
clc;
clear;
load case011input
%load ZJGofFX011 %所有基站计算出来的终端
%ZJG=JGofFX;
\end{verbatim}

\begin{verbatim}
SJ=case011input;

c=3.0e+8;%光速
n=SJ(1,1);%基站数目
m=SJ(2,1);%终端数目
w=SJ(3,1);%维数
fid=fopen('output_case_011.txt','w+');
fid1=fopen('output_case_index_011.txt','w+');
fid2=fopen('output_case_30_011.txt','w+');
XJGofindex=zeros(m,n);

for ZD=1:m     %在这里控制是哪个终端
    %ZD=1;
    shunxu=[];
    index=[];
    %开始从寻找四个最小的 TOA 开始
    [shunxu,index]=sort(SJ(3+n+ZD,:));
    %%%%%%%%%%%%%%%%%%%%%%%%%%%%%%%%%%%%%%%%%%%%%
    %%%%%%%%%%%%%%%%%%%%%%%%%%%%%%%%%%%%%%%%%%%%%
    %%首先计算 30 个已经经过排序终端的答案
    sxZB=SJ(3+index(1:n),1:3);
    sxTOA=SJ(3+n+ZD,index(1:n));
    for i=2:n
        xishu(i-1,:)=(sxZB(1,1:w)-sxZB(i,1:w)).*2;
    end

    min=inf;%先把最小误差设为最大值
    for a=0.95:0.001:0.97     %在这里给定 a 的范围
        for d=0:0.1:2         %在这里给定 d 的范围
            d1=-d*10^-9;
            changshu=[];
            for i=2:n

                changshu(i-1)=c^2*((a*sxTOA(1,i)+d1)^2-(a*sxTOA(1,1)+d1)^2)+(sxZB(1,1)^2+sxZB(1,2)^2+sxZB(1,3)^2)-(sxZB(i,1)^2+sxZB(i,2)^2+sxZB(i,3)^2);
            end
            changshu=changshu';
            X=xishu\changshu;

            diff=norm(sqrt((X(1)-sxZB(1:n,1)).^2+(X(2)-sxZB(1:n,2)).^2+(X(3)-sxZB(1:n,3)).^2)-(c*(a*sxTOA(1,:)' + d1)));
            if min>diff
\end{verbatim}

\begin{verbatim}
tempmin=diff;
tempFX=X;
tempFa=a;
tempFd=d1;
if X(3)>0
    min=diff;
    FX=X;
    Fa=a;
    Fd=d1;
end
end
end
if min==inf
    min=tempmin;
    FX=tempFX;
    FX(3)=abs(FX(3));
    Fa=tempFa;
    Fd=tempFd;
end
x1=FX;
x2=FX;
error=
sum((sqrt((FX(1)-sxZB(1:n,1)).^2+(FX(2)-sxZB(1:n,2)).^2+(FX(3)-sxZB(1:n,3)).^2)-c*sxTOA(1,:))'.^2);
for step=0:0.01:abs(FX(3))
    x1(3)=step;
    tt=
    sum((sqrt((x1(1)-sxZB(1:n,1)).^2+(x1(2)-sxZB(1:n,2)).^2+(x1(3)-sxZB(1:n,3)).^2)-c*sxTOA(1,:))'.^2);
    
    if error<tt;
        error=tt;
        x2(3)=step;
    end
end
JGofFX(ZD,:)=x2;
JGofMin(ZD)=min;
JGofFa(ZD)=Fa;;%已输出
JGofFd(ZD)=Fd;%已输出
%%%%%%%%%%%%%%%%%%%%%%%%%%%%%%%%%%%%%%%%%%%%%%%%%%%%%%%%%%%%%%%
%%%%%%%%%%%%%%%%%%%%%%%%%%%%%%%%%%%%%%%%%%%%%%%%%%%%%%%%%%%%%%%
for JZ=4:n %选择基站的个数
    sxTOA=[];
\end{verbatim}

\begin{verbatim}
sxZB=[];
sxTOA=SJ(3+n+ZD,index(1:JZ)); %选择出来的最小的几个时间;
sxZB=SJ(3+index(1:JZ),1:3);    %选择出来与最小时间对应的基站坐标;
xishu=[];
for i=2:JZ
    xishu(i-1,:)=(sxZB(1,1:w)-sxZB(i,1:w)).*2;
end
    min=inf;%先把最小误差设为最大值
    for a=0.95:0.001:0.97      %在这里给定a的范围
        for d=0:0.1:2          %在这里给定d的范围
            d1=-d*10^-9;
            changshu=[];
            for i=2:JZ
                changshu(i-1)=c^2*((a*sxTOA(1,i)+d1)^2-(a*sxTOA(1,1)+d1)^2)+(sxZB(1,1)^2+sxZB(1,2)^2+sxZB(1,3)^2)-(sxZB(i,1)^2+sxZB(i,2)^2+sxZB(i,3)^2);
            end
            changshu=changshu';
            X=xishu\changshu;
            
            diff=norm(sqrt((X(1)-sxZB(1:JZ,1)).^2+(X(2)-sxZB(1:JZ,2)).^2+(X(3)-sxZB(1:JZ,3)).^2)-(c*(a*sxTOA(1,:)+d1)));
            if min>diff
                tempmin=diff;
                tempFX=X;
                tempFa=a;
                tempFd=d1;
                if X(3)>0
                    min=diff;
                    FX=X;
                    Fa=a;
                    Fd=d1;
                end
            end
        end
    end
    if min==inf
        min=tempmin;
        FX=tempFX;
        FX(3)=abs(FX(3));
        Fa=tempFa;
        Fd=tempFd;
    end
\end{verbatim}

\begin{verbatim}
x1=FX;
x2=FX;
error=
sum((sqrt((FX(1)-sxZB(1:JZ,1)).^2+(FX(2)-sxZB(1:JZ,2)).^2+(FX(3)-sxZB(1:JZ,3)).^2)-c*sxTOA(1,:)).^2);
for step=0:0.01:abs(FX(3))
    x1(3)=step;
    tt=
    sum((sqrt((x1(1)-sxZB(1:JZ,1)).^2+(x1(2)-sxZB(1:JZ,2)).^2+(x1(3)-sxZB(1:JZ,3)).^2)-c*sxTOA(1,:)).^2);
    
    if error<tt;
        error=tt;
        x2(3)=step;
    end
end
FX=x2;
if ((FX(1)-JGofFX(ZD,1))^2+(FX(2)-JGofFX(ZD,2))^2)<2
if FX(3)-JGofFX(ZD,3)<2
XJGofMin(ZD)=min; %已输出
XJGofFX(ZD,:)=FX;%已输出
XJGofFa(ZD)=Fa;;%已输出
XJGofFd(ZD)=Fd;%已输出
XJGofJZ(ZD)=JZ;
XJGofindex(ZD,1:JZ)=index(1:1:JZ);
XJGofwucha(ZD)=sqrt((FX(1)-JGofFX(ZD,1))^2+(FX(2)-JGofFX(ZD,2))^2+(FX(3)-JGofFX(ZD,3))^2);
fprintf(fid,'%7.2f %7.2f %7.2f %f %e %f\n',XJGofFX(ZD,1),XJGofFX(ZD,2),XJGofFX(ZD,3),XJGofMin(ZD),XJGofFa(ZD),XJGofFd(ZD),XJGofwucha(ZD));
fprintf(fid2,'%7.2f %7.2f %7.2f %f %f %e\n',JGofFX(ZD,1),JGofFX(ZD,2),JGofFX(ZD,3),JGofMin(ZD),JGofFa(ZD),JGofFd(ZD));
fprintf(fid1,'%d ',XJGofJZ(ZD));
forkk=1:n
fprintf(fid1,'%d,',XJGofindex(ZD,kk));
if kk==n
fprintf(fid1,'%d\n',XJGofindex(ZD,kk));
end
end
break
end
end
\end{verbatim}

\begin{verbatim}
end
end
toc

问题三程序
Shitanfa21.m
tic
clc;
clear;
load case021input;
SJ=case021input;

% load case001input
% SJ=case001input;
c=3.0e+8;%光速
n=SJ(1,1);%基站数目
m=SJ(2,1);%终端数目
w=SJ(3,1);%维数
fid=fopen('output_case_021.txt','w+');
fidindex=fopen('data1index.txt','w+');
%for ZD=1:10%第几个终端
%ZD=1142
%a=1;%衰减系数
% diff=0;
%min=inf;
for i=2:n
xishu(i-1,:)=(SJ(4,1:w)-SJ(3+i,1:w)).*2;
end

% ZD=1
for ZD=1:m
%ZD=1
min=inf;
%a=1;%衰减系数

% for a=0.5:0.01:1
%a=1-0.292488582926760;
%d=-2.090872502954080e-09;

for a=0.60:0.001:0.61

for d=0:0.1:2
\end{verbatim}

\begin{verbatim}
d1 = -d * 10^-9;
changshu = [];
for i = 2:n
    changshu(i-1) = c^2 * ((a * SJ(3+n+ZD,i) + d1)^2 - (a * SJ(3+n+ZD,1) + d1)^2) + (SJ(4,1)^2 + SJ(4,2)^2) - (SJ(3+i,1)^2 + SJ(3+i,2)^2);
end
changshu = changshu';
X = xishu \ changshu;

% x0 = [0; 0; 0];
% X = fmincon(@(x) 0.5 * x' * xishu' * xishu * x - changshu' * xishu * x + 0.5 * changshu' * changshu + ...
%     (((x(1) - SJ(4,1))^2 + (x(2) - SJ(4,2))^2 + (x(3) - SJ(4,3))^2) - (c * SJ(3+n+ZD,1)')^2)^2, x0, ...
%     [0, 0, -1], [0], ...
%     [], [], [], []);

% diff = 0;
% for i = 1:n;
%     diff = diff + (sqrt((X(1) - SJ(3+i,1))^2 + (X(2) - SJ(3+i,2))^2 + (X(3) - SJ(3+i,3))^2) - (a * c * SJ(4+n,i)))^2;
% end
diff = norm(sqrt((X(1) - SJ(4:3+n,1))^2 + (X(2) - SJ(4:3+n,2))^2) - (c * (a * SJ(3+n+ZD,:)'+d1)));
if min > diff
    tempmin = diff;
    tempFX = X;
    tempFa = a;
    tempFd = d1;
    if X(3) > 0
        min = diff;
        FX = X;
        Fa = a;
        Fd = d1;
    end
end
end

% if min == inf
%     min = tempmin;
%     FX = tempFX;
%     FX(3) = abs(FX(3));
\end{verbatim}

\begin{verbatim}
% Fa=tempFa;
% Fd=tempFd;
% end

% x1=FX;
% x2=FX;
% 
error=sum((sqrt((FX(1)-SJ(4:3+n,1)).^2+(FX(2)-SJ(4:3+n,2)).^2+(FX(3)-SJ(4:3+n,3)).^2)-c*SJ(3+n+ZD,:).^2);
% for step=0:0.01:abs(FX(3))
%     x1(3)=step;
% 
tt=sum((sqrt((x1(1)-SJ(4:3+n,1)).^2+(x1(2)-SJ(4:3+n,2)).^2+(x1(3)-SJ(4:3+n,3)).^2)-c*SJ(3+n+ZD,:).^2);
% 
% if error<tt;
%     error=tt;
%     x2(3)=step;
% end
% end
% FX=x2;

JGofMin(ZD)=min;
% JGofFXindex(ZD)=FX(3)<0;
JGofFX(ZD,:)=FX;
% JGofFX(ZD,3)=abs(JGofFX(ZD,3));

JGofFa(ZD)=Fa;
JGofFd(ZD)=Fd;
fprintf(fid,'%7.2f %7.2f %f %f %e\n',JGofFX(ZD,1),JGofFX(ZD,2),JGofMin(ZD),JGofFa(ZD),JGofFd(ZD));
% fprintf(fidindex,'%f\n',JGofFXindex(ZD));
end
% sum(JGofFXindex(:)==1)
JGofFX(:,3)=1:m;
plot3(JGofFX(:,1),JGofFX(:,2),JGofFX(:,3),' m:p');
%patch(JGofFX(:,1),JGofFX(:,2),JGofFX(:,3),JGofFX(:,3),'edgecolor','flat','facecolor','none')
toc
\end{verbatim}

\textbf{问题四程序}

\textbf{Testcode26.m}

\begin{verbatim}
tic
clc;
clear;
load case026input
SJ=case026input;
c=3.0e+8;%光速
n=SJ(1,1);%基站数目
m=SJ(2,1);%终端数目
w=SJ(3,1);%维数
fid=fopen('output_case_026.txt','w+');

%TOA=SJ(4+n:end,:);
%JL=c*TOA;
XJGofindex=zeros(m,n);
for ZD=1:m     %在这里控制是哪个终端
    % ZD=7;
    for JZ=4:n
        shunxu=[];
        index=[];
        %开始从寻找四个最小的 TOA 开始
        [shunxu,index]=sort(SJ(3+n+ZD,:));
        sxZB=SJ(3+index(1:JZ),1:3);
        sxTOA=SJ(3+n+ZD,index(1:JZ));
        sxJL=c*sxTOA;
        xishu=[];
        for i=2:JZ
            xishu(i-1,:)=(sxZB(1,1:w)-sxZB(i,1:w)).*2;
        end
        min=inf;%先把最小误差设为最大值
        for a=0.5:0.01:1         %在这里给定 a 的范围
            for d=0:0.1:2         %在这里给定 d 的范围
                d1=-d*10^-9;
                changshu=[];
                for i=2:JZ

                    changshu(i-1)=c^2*((a*sxTOA(1,i)+d1)^2-(a*sxTOA(1,1)+d1)^2)+(sxZB(1,1)^2+sxZB(1,2)^2+sx
                    ZB(1,3)^2)-(sxZB(i,1)^2+sxZB(i,2)^2+sxZB(i,3)^2);
                end
                changshu=changshu';
                X=xishu\changshu;
                
                diff=norm(sqrt((X(1)-sxZB(1:JZ,1)).^2+(X(2)-sxZB(1:JZ,2)).^2+(X(3)-sxZB(1:JZ,3)).^2)-(c*(a*sxTOA(1,:)' + d1)));
            end
        end
    end
end
\end{verbatim}

\begin{verbatim}
if min>diff
    tempmin=diff;
    tempFX=X;
    tempFa=a;
    tempFd=d1;
    if X(3)>0
        min=diff;
        FX=X;
        Fa=a;
        Fd=d1;
    end
end
end
if min==inf
    min=tempmin;
    FX=tempFX;
    FX(3)=abs(FX(3));
    Fa=tempFa;
    Fd=tempFd;
end
x1=FX;
x2=FX;
error=sum((sqrt((FX(1)-sxZB(1:JZ,1)).^2+(FX(2)-sxZB(1:JZ,2)).^2+(FX(3)-sxZB(1:JZ,3)).^2)-c*sxTOA(1,:)').^2);
for step=0:0.01:abs(FX(3))
    x1(3)=step;
    tt=sum((sqrt((x1(1)-sxZB(1:JZ,1)).^2+(x1(2)-sxZB(1:JZ,2)).^2+(x1(3)-sxZB(1:JZ,3)).^2)-c*sxTOA(1,:)').^2);
    if error<tt
        error=tt;
        x2(3)=step;
    end
end
FX=x2;

JZ_FX(JZ,:)=FX;
JZ_Fa(JZ)=Fa;
JZ_Fd(JZ)=Fd;
JZ_min(JZ)=min;
JZ_num(JZ)=JZ;
\end{verbatim}

\begin{verbatim}
if (c*(sxTOA(1,JZ)*JZ_Fa(JZ)+JZ_Fd(JZ)))>=200;
if JZ==4
    ZD_FX(ZD,:)=[nan nan nan];
    ZD_Fa(ZD)=JZ_Fa(JZ);
    ZD_Fd(ZD)=JZ_Fd(JZ);
    ZD_min(ZD)=JZ_min(JZ);
    ZD_num(ZD)=JZ_num(JZ);
    break
end
    ZD_FX(ZD,:)=JZ_FX(JZ-1,:);
    ZD_Fa(ZD)=JZ_Fa(JZ-1);
    ZD_Fd(ZD)=JZ_Fd(JZ-1);
    ZD_min(ZD)=JZ_min(JZ-1);
    ZD_num(ZD)=JZ_num(JZ-1);

    break;
else
    if JZ==4
        continue
    else
        if abs(JZ_FX(JZ,1)-JZ_FX(JZ-1,1))<1
            if abs(JZ_FX(JZ,2)-JZ_FX(JZ-1,2))<1
                if abs(JZ_FX(JZ,3)-JZ_FX(JZ-1,3))<2
                    ZD_FX(ZD,:)=JZ_FX(JZ-1,:);
                    ZD_Fa(ZD)=JZ_Fa(JZ-1);
                    ZD_Fd(ZD)=JZ_Fd(JZ-1);
                    ZD_min(ZD)=JZ_min(JZ-1);
                    ZD_num(ZD)=JZ_num(JZ-1);
                    break;
                end
            end
        end
    end
end
end
end
\end{verbatim}

\begin{verbatim}
end
for ZD=1:m
    ZD_LJS(ZD)=0;
    if isnan(ZD_FX(ZD,1))
        continue
    else
        for JZ=1:n
            if (c*(ZD_Fa(ZD)*SJ(3+n+ZD,JZ)-ZD_Fd(ZD)))<=200
                ZD_LJS(ZD)=ZD_LJS(ZD)+1;
            end
        end
    end
end

for ZD=1:m
    fprintf(fid,'%7.2f %7.2f %7.2f %f %e %d\n',ZD_FX(ZD,1),ZD_FX(ZD,2),ZD_FX(ZD,3),ZD_min(ZD),ZD_Fa(ZD),ZD_Fd(ZD),ZD_LJS(ZD));
end
toc
\end{verbatim}

\begin{table}
\centering
\begin{tabular}{|c|c|c|c|l|}
\hline
$x$ & $y$ & $z$ & 个数 & 定位基站 \\
\hline
-233.23 & -267.7 & 3.26 & 5 & 21, 2, 12, 6, 13 \\
\hline
311.92 & 302.07 & 2.86 & 5 & 25, 19, 14, 29, 8 \\
\hline
-245.74 & 68.09 & 3.16 & 5 & 1, 5, 7, 23, 11 \\
\hline
-231.97 & -297.51 & 1.6 & 8 & 21, 2, 12, 6, 13, 4, 18, 23 \\
\hline
-223.01 & -101.38 & 3.55 & 4 & 6, 12, 23, 4 \\
\hline
51.35 & 41.06 & 3.64 & 5 & 22, 10, 16, 30, 20 \\
\hline
-269.99 & -168.61 & 0.59 & 7 & 12, 21, 2, 6, 4, 23, 5 \\
\hline
-188.03 & 307.85 & 3.76 & 4 & 28, 7, 11, 1 \\
\hline
253.91 & -76.76 & 2.82 & 4 & 24, 20, 3, 30 \\
\hline
-174.53 & 196.4 & 0.54 & 7 & 11, 28, 7, 1, 5, 23, 16 \\
\hline
-260.67 & -25.98 & 1.11 & 4 & 5, 23, 1, 6 \\
\hline
-266.01 & 79.55 & 0.68 & 6 & 1, 7, 5, 23, 11, 28 \\
\hline
-72.03 & 139.34 & 1.48 & 6 & 11, 16, 5, 23, 29, 7 \\
\hline
-191.62 & -113.47 & 3.98 & 4 & 6, 12, 4, 23 \\
\hline
-108.02 & 177.71 & 1.75 & 4 & 11, 16, 7, 28 \\
\hline
209.8 & -300.07 & 0.9 & 5 & 17, 27, 26, 15, 18 \\
\hline
223.78 & -185.89 & 3.24 & 4 & 24, 3, 27, 15 \\
\hline
28.97 & -92.95 & 1.1 & 5 & 9, 22, 10, 30, 4 \\
\hline
273.45 & 48.03 & 0.73 & 24 & 8, 20, 30, 14, 24, 3, 22, 25, 19, 16, 27, 10, 29, 15, 9, 17, 26, 11, 4, 1 \\
\hline
 & & & & 8, 13, 23, 5, 6 \\
\hline
-166.01 & -144.61 & 3.89 & 4 & 6, 12, 4, 23 \\
\hline
14.5 & 113.45 & 3.14 & 4 & 16, 11, 29, 14 \\
\hline
76.33 & 166.93 & 2.58 & 4 & 16, 14, 29, 19 \\
\hline
179.89 & 143.37 & 2.23 & 4 & 14, 8, 19, 16 \\
\hline
96.2 & 306.66 & 2.7 & 4 & 29, 19, 25, 14 \\
\hline
156.6 & -174.5 & 2.88 & 4 & 3, 24, 26, 17 \\
\hline
318.2 & 235.17 & 2.95 & 4 & 25, 14, 19, 8 \\
\hline
-275.81 & 193.15 & 1.78 & 11 & 7, 28, 1, 11, 5, 23, 16, 6, 4, 29, 10 \\
\hline
-147.69 & -306.58 & 0.29 & 4 & 13, 21, 12, 2 \\
\hline
-121.71 & -294.33 & 2.63 & 4 & 13, 18, 21, 12 \\
\hline
129.35 & -164.06 & 0.33 & 25 & 3, 26, 24, 17, 30, 9, 22, 20, 10, 18, 13, 27, 15, 8, 4, 6, 14, 16, 23, 5, \\
\hline
 & & & & 11, 12, 21, 29, 2 \\
\hline
-57.01 & 125.85 & 0.23 & 6 & 11, 16, 5, 23, 29, 10 \\
\hline
-50.63 & -197.04 & 4.19 & 4 & 13, 9, 4, 18 \\
\hline
-238.68 & 142.87 & 3.97 & 4 & 7, 1, 28, 11 \\
\hline
-68.41 & -209.12 & 0.24 & 10 & 13, 9, 18, 4, 6, 10, 26, 12, 22, 17 \\
\hline
13.41 & -278.26 & 0.55 & 4 & 13, 18, 26, 17 \\
\hline
-91.89 & -221.18 & 0.16 & 7 & 13, 9, 6, 4, 18, 12, 10 \\
\hline
68.16 & 28.35 & 0.32 & 6 & 22, 10, 30, 20, 16, 9 \\
\hline
\end{tabular}
\end{table}

\begin{table}
\centering
\begin{tabular}{|c|c|c|c|c|}
\hline
-30.72 & 6.58 & 3.59 & 4 & 10,4,22,9 \\
\hline
163.84 & 292.64 & 1.33 & 5 & 19,25,29,14,16 \\
\hline
-72 & -239.26 & 0.79 & 5 & 13,18,9,6,26 \\
\hline
39.78 & -66.94 & 3.25 & 4 & 22,10,9,30 \\
\hline
-143.76 & -219.13 & 3.46 & 4 & 12,6,21,13 \\
\hline
-39.21 & -247.41 & 0.2 & 7 & 13,18,9,26,17,10,4 \\
\hline
31.74 & 22.24 & 0.44 & 5 & 10,22,16,9,30 \\
\hline
-303.71 & 209.59 & 0.41 & 23 & 28,7,1,11,5,23,6,16,4,29,12,10,22,9,2,21,14,19,30,20,2 \\
\hline
 & & & & 5,8,13 \\
\hline
-88.34 & -200.86 & 0.21 & 7 & 13,9,6,4,12,10,18 \\
\hline
116.79 & -28.39 & 3.56 & 6 & 30,22,20,3,10,8 \\
\hline
-287.01 & 194.82 & 3.72 & 6 & 7,28,1,11,5,23 \\
\hline
-292.36 & -123.24 & 3.6 & 7 & 12,2,21,6,23,5,4 \\
\hline
310.4 & 255.92 & 2.99 & 4 & 25,19,14,8 \\
\hline
\end{tabular}
\end{table}

\textbf{Case 12 前面若干个终端的结果}

\begin{table}
\centering
\begin{tabular}{|c|c|c|c|c|}
\hline
\(x\) & \(y\) & \(z\) & 个数 & 定位基站 \\
\hline
-182.31 & 230.56 & 0.92 & 5 & 22,4,34,1,20 \\
\hline
182.46 & -136.85 & 3.52 & 33 & 7,25,27,28,5,3,24,33,32,29,19,40,6,26,16,36,23,15,10,1 \\
\hline
 & & & & 8,2,37,12,21,35,39,11,14,9,8,31,30,34 \\
\hline
4.07 & -237.75 & 3.47 & 4 & 6,32,33,16 \\
\hline
198.32 & 288.12 & 0.6 & 17 & 38,8,13,31,39,12,18,20,23,1,37,36,2,40,11,10,22 \\
\hline
241.6 & -17.92 & 2.89 & 4 & 40,36,23,5 \\
\hline
-95.53 & -277.4 & 0.54 & 13 & 30,6,26,9,16,32,17,33,24,15,35,27,21 \\
\hline
-270.24 & -315.47 & 0.37 & 8 & 17,30,9,6,26,35,16,21 \\
\hline
-55.1 & 312.71 & 0.56 & 4 & 1,20,13,22 \\
\hline
-300.25 & 239.96 & 0.21 & 14 & 4,22,34,1,14,11,20,35,21,2,37,9,13,10 \\
\hline
199.86 & 194.35 & 3.21 & 4 & 8,31,18,12 \\
\hline
-159.62 & -71.16 & 3.23 & 4 & 9,35,14,21 \\
\hline
52.52 & -64.87 & 3.43 & 4 & 15,10,26,24 \\
\hline
-204.17 & 51.57 & 2.35 & 5 & 34,14,11,35,21 \\
\hline
219.75 & -135.87 & 0.11 & 4 & 25,7,3,5 \\
\hline
232.8 & -158.91 & 3.91 & 4 & 25,3,7,5 \\
\hline
-216.45 & -244.03 & 3.57 & 4 & 30,17,9,6 \\
\hline
94.06 & -74.57 & 0.43 & 14 & 24,15,10,26,27,6,7,25,33,2,32,37,28,23 \\
\hline
-121.7 & -44.48 & 3.23 & 4 & 35,21,14,9 \\
\hline
-22.39 & 106.14 & 1.26 & 5 & 11,37,2,21,10 \\
\hline
159.67 & -261.33 & 0.06 & 4 & 28,33,16,32 \\
\hline
134.81 & 21.39 & 0.31 & 8 & 23,18,12,37,2,10,40,36 \\
\hline
-209.48 & -93.82 & 3.35 & 5 & 9,14,35,21,17 \\
\hline
33.49 & 235.3 & 3.48 & 40 & 13,20,1,38,12,8,37,2,11,18,22,34,23,31,10,14,21,39,35, \\
\hline
 & & & & 15,4,36,40,26,6,24,7,5,9,25,27,3,33,32,28,29,16,19,30, \\
\hline
 & & & & 17,17 \\
\hline
\end{tabular}
\end{table}

\begin{table}
\centering
\begin{tabular}{|c|c|c|c|c|}
\hline
72.77 & -207.13 & 4.33 & 4 & 32,33,24,27 \\
\hline
-246.84 & -37.29 & 0.78 & 39 & 9,14,34,35,21,11,17,15,10,30,26,22,2,6,37,4,24,32,33,1 \\
 & & & & ,12,16,20,27,18,23,7,13,28,25,38,8,40,5,3,36,19,29,39 \\
\hline
290.9 & 24 & 2.7 & 5 & 36,40,23,39,5 \\
\hline
-267.91 & -277.31 & 3.54 & 4 & 17,30,9,6 \\
\hline
-172.92 & -280.42 & 3.7 & 4 & 30,17,9,6 \\
\hline
225.96 & 246.18 & 1.97 & 6 & 8,31,38,39,13,18 \\
\hline
284.9 & 3.97 & 1.18 & 26 & 36,40,23,5,18,39,3,25,7,12,31,29,27,28,8,24,19,37,2,10 \\
 & & & & ,33,15,32,26,38,6 \\
\hline
-24.63 & 98.92 & 0.77 & 7 & 11,2,37,21,10,14,34 \\
\hline
209.24 & 62.75 & 2.96 & 4 & 23,18,36,40 \\
\hline
-180.41 & 141.48 & 0.57 & 8 & 34,22,14,11,4,35,21,1 \\
\hline
-218.34 & -87.99 & 3.34 & 5 & 9,14,35,21,17 \\
\hline
-186.19 & 143.88 & 3.72 & 6 & 34,22,14,11,4,35 \\
\hline
104.82 & -132.32 & 0.15 & 13 & 24,27,33,32,7,6,26,25,28,15,10,5,16 \\
\hline
-41.93 & -233.96 & 3.54 & 4 & 6,26,30,32 \\
\hline
290.75 & -277.83 & 3.84 & 6 & 19,29,3,28,25,7 \\
\hline
\end{tabular}
\end{table}

\textbf{Case 13 前面若干个终端的结果}

\begin{table}
\centering
\begin{tabular}{|c|c|c|c|c|}
\hline
\(x\) & \(y\) & \(z\) & 个数 & 定位基站 \\
\hline
216.8 & 323.13 & 0.44 & 4 & 29,49,30,24 \\
\hline
-48.14 & -286.36 & 4.32 & 4 & 4,1,48,19 \\
\hline
-363.88 & 108.21 & 3.49 & 4 & 21,14,16,33 \\
\hline
-158.49 & 14.97 & 0.17 & 7 & 16,14,7,13,6,32,27 \\
\hline
167.31 & 334.59 & 1.47 & 13 & 30,40,8,12,9,29,36,49,24,31,22,5,34 \\
\hline
-51.22 & -69.25 & 3.72 & 4 & 13,7,32,19 \\
\hline
-277.27 & -279.67 & 0.11 & 5 & 25,35,37,28,33 \\
\hline
369.14 & -151.63 & 0.09 & 6 & 42,20,18,2,17,11 \\
\hline
-306.42 & 360.29 & 4.07 & 4 & 26,15,3,50 \\
\hline
-195.86 & -166.88 & 0.06 & 5 & 32,7,33,4,14 \\
\hline
175.46 & -157.8 & 1.88 & 4 & 39,18,2,1 \\
\hline
-101.76 & -164.02 & 0.61 & 5 & 32,4,7,19,13 \\
\hline
-41.06 & 203.64 & 1.72 & 7 & 31,41,36,27,5,50,6 \\
\hline
304.5 & 310.69 & 3.79 & 4 & 49,24,29,22 \\
\hline
171.83 & -311.7 & 1.7 & 5 & 10,1,48,17,43 \\
\hline
-205.05 & 334.21 & 2.44 & 4 & 15,26,3,50 \\
\hline
109.24 & -199.91 & 1.84 & 4 & 1,39,19,18 \\
\hline
-115.91 & -142.76 & 0.15 & 6 & 32,7,4,13,19,14 \\
\hline
-335.27 & 170.5 & 0.93 & 4 & 21,26,3,16 \\
\hline
163.38 & -209.7 & 3.34 & 4 & 1,18,17,39 \\
\hline
18.5 & 191.21 & 4.77 & 4 & 31,5,36,9 \\
\hline
-330.62 & 278.23 & 3.79 & 7 & 26,3,15,50,41,16,21 \\
\hline
269.81 & 187.48 & 1.63 & 4 & 22,44,45,49 \\
\hline
\end{tabular}
\end{table}

\begin{table}
\centering
\begin{tabular}{|c|c|c|c|c|}
\hline
16.61 & 23.87 & 0.08 & 4 & 47,13,6,27 \\
\hline
-377.03 & -56.68 & 3.21 & 5 & 21,33,14,16,7 \\
\hline
141.25 & 186.04 & 0.14 & 5 & 8,9,46,5,45 \\
\hline
153.06 & -367.86 & 3.88 & 4 & 48,10,1,23 \\
\hline
-231.37 & -16.73 & 0.56 & 4 & 14,16,33,7 \\
\hline
303.69 & -225.47 & 3.75 & 4 & 42,17,18,43 \\
\hline
220.73 & 95.71 & 3.9 & 4 & 45,44,46,22 \\
\hline
282.33 & 175.87 & 0.83 & 12 & 22,44,45,49,46,8,9,29,24,11,5,40 \\
\hline
60.01 & -117.05 & 0.11 & 4 & 19,39,13,47 \\
\hline
364.84 & -310.53 & 3.23 & 4 & 42,43,38,23 \\
\hline
-316.85 & 336.19 & 3.82 & 5 & 26,15,3,50,41 \\
\hline
244.66 & 206.59 & 1.77 & 4 & 22,44,45,8 \\
\hline
165.33 & -205.31 & 0.79 & 6 & 1,18,39,17,2,10 \\
\hline
-237.74 & 155.09 & 3.96 & 4 & 16,41,3,14 \\
\hline
-13.84 & 107.58 & 4.19 & 4 & 27,6,5,31 \\
\hline
\end{tabular}
\end{table}

\textbf{Case 14 前面若干个终端的结果}

\begin{table}
\centering
\begin{tabular}{|c|c|c|c|c|}
\hline
\(x\) & \(y\) & \(z\) & 个数 & 定位基站 \\
\hline
-348.86 & 121.96 & 0.05 & 50 & 1,6,37,7,53,19,28,44,45,49,59,47,21,34,42,3,54,31,6 \\
 & & & & 0,57,35,51,56,58,18,25,4,48,24,5,12,20,40,33,39,46, \\
 & & & & 26,52,27,15,16,23,30,17,2,55,29,41,43,36 \\
\hline
94.26 & 277.75 & 2.52 & 4 & 12,48,4,51 \\
\hline
301.7 & -26.97 & 0.2 & 6 & 36,11,43,8,32,29 \\
\hline
-209.48 & -325.8 & 2.9 & 4 & 40,33,58,25 \\
\hline
-42.77 & -151.51 & 3.24 & 5 & 18,20,3,31,35 \\
\hline
193.75 & -230.02 & 3.27 & 5 & 14,30,9,17,26 \\
\hline
129.09 & 309.61 & 1.8 & 4 & 12,48,4,41 \\
\hline
237.26 & -56.76 & 3.47 & 5 & 23,55,16,43,27 \\
\hline
-187.02 & 198.92 & 2.72 & 4 & 19,44,7,1 \\
\hline
-222.58 & 137.96 & 3.96 & 4 & 7,37,53,1 \\
\hline
325.75 & 193.44 & 4.2 & 4 & 2,50,10,29 \\
\hline
-286.87 & -88.13 & 0.28 & 6 & 28,47,45,42,6,54 \\
\hline
-195.04 & 41.44 & 0 & 6 & 53,37,7,45,49,44 \\
\hline
-263.95 & 96.61 & 3.99 & 4 & 37,7,53,1 \\
\hline
82.3 & 287.32 & 3.38 & 4 & 12,48,4,51 \\
\hline
131.54 & 206.24 & 0.4 & 6 & 4,12,51,60,48,52 \\
\hline
-260.68 & 300.74 & 0.66 & 48 & 59,19,1,7,44,37,53,34,60,49,51,21,48,6,12,45,4,28,5 \\
 & & & & 6,35,24,47,3,42,31,5,18,52,39,54,57,41,20,15,27,2,4 \\
 & & & & 6,58,26,16,25,10,50,23,29,40,33,55 \\
\hline
-144.36 & 73 & 0.33 & 31 & 49,53,44,37,21,7,34,45,35,19,56,1,60,51,3,28,31,6,4 \\
 & & & & 2,18,24,5,47,4,59,20,39,57,54,48,46 \\
\hline
25.33 & 364.63 & 3.69 & 5 & 48,12,51,60,4 \\
\hline
-75.98 & -17.63 & 3.71 & 5 & 35,21,49,56,18 \\
\hline
\end{tabular}
\end{table}

\begin{table}
\centering
\begin{tabular}{|c|c|c|c|c|}
\hline
-176.79 & -353.34 & 2.62 & 4 & 33,40,58,25 \\
\hline
75.29 & 120.29 & 3.73 & 4 & 24,4,56,51 \\
\hline
-98.73 & -12.6 & 3.82 & 4 & 35,21,49,45 \\
\hline
330.07 & -0.88 & 0.32 & 12 & 8,29,11,36,32,43,38,55,15,23,50,17 \\
\hline
359.42 & 13.23 & 0.18 & 4 & 8,29,11,32 \\
\hline
293.99 & 27.14 & 1.53 & 4 & 29,8,11,36 \\
\hline
-234.43 & -91.53 & 3.71 & 4 & 45,42,28,47 \\
\hline
-27.73 & -144.35 & 3.07 & 6 & 18,20,3,31,26,35 \\
\hline
29.47 & 294.95 & 3.56 & 4 & 48,12,51,60 \\
\hline
325.82 & 296.54 & 4.35 & 4 & 10,41,2,50 \\
\hline
2.5 & -190.51 & 3.1 & 5 & 20,18,26,46,3 \\
\hline
293.37 & -137.12 & 0.26 & 52 & 17,55,43,22,36,9,38,23,32,11,16,8,27,15,14,29,46,3 \\
\hline
 & & & & 9,26,13,5,30,52,20,24,50,18,56,2,35,4,21,3,10,31,34 \\
\hline
 & & & & 49,51,60,12,41,44,45,42,33,48,58,57,53,37,7,40 \\
\hline
247.79 & 233.47 & 4.26 & 4 & 2,50,10,41 \\
\hline
-67.92 & -12.25 & 3.99 & 4 & 35,21,49,56 \\
\hline
136.95 & 167.35 & 3.76 & 4 & 4,52,24,51 \\
\hline
\end{tabular}
\end{table}

\textbf{Case 15 前面若干个终端的结果}

\begin{table}
\centering
\begin{tabular}{|c|c|c|c|c|}
\hline
\(x\) & \(y\) & \(z\) & 个数 & 定位基站 \\
\hline
-353.78 & -84.2 & 0.63 & 4 & 15,3,9,1 \\
\hline
-212.25 & 160.88 & 3.71 & 4 & 11,1,6,9 \\
\hline
-368.09 & 289.58 & 0.24 & 18 & 11,6,1,3,9,15,14,5,12,17,13,10,16,2,19,20,18,8 \\
\hline
274.69 & -99.05 & 0.33 & 5 & 13,20,2,4,17 \\
\hline
-272.95 & -65.18 & 3.15 & 4 & 9,15,3,1 \\
\hline
222.96 & -89.79 & 3.5 & 4 & 13,17,20,2 \\
\hline
-177.84 & -280.05 & 3.69 & 4 & 10,16,5,12 \\
\hline
105.71 & -292.07 & 4.54 & 4 & 20,18,8,19 \\
\hline
10.17 & -47.32 & 0.63 & 18 & 17,13,12,5,14,10,20,9,19,18,11,8,15,1,2,3,7,16 \\
\hline
250.18 & 9.81 & 3.72 & 5 & 2,13,14,17,20 \\
\hline
-72.2 & 287.34 & 0.69 & 4 & 6,14,11,1 \\
\hline
293.02 & 46.17 & 3.68 & 5 & 2,14,13,17,20 \\
\hline
85.51 & 299.5 & 3.59 & 4 & 14,6,2,17 \\
\hline
-237.63 & -279.74 & 3.67 & 5 & 16,10,5,15,12 \\
\hline
265 & -62.67 & 3.49 & 4 & 13,2,17,20 \\
\hline
-122.12 & 167.94 & 3.68 & 4 & 11,6,14,1 \\
\hline
174.01 & 203.04 & 1.12 & 7 & 14,2,17,13,6,12,5 \\
\hline
-256.73 & -84.9 & 1.85 & 4 & 9,15,3,5 \\
\hline
-157.32 & 319.36 & 3.69 & 8 & 6,11,1,14,9,3,17,13 \\
\hline
-359.72 & 20.92 & 0.79 & 4 & 3,1,11,9 \\
\hline
13.23 & -262.17 & 4.68 & 4 & 19,10,20,18 \\
\hline
-136.24 & 0.96 & 4.22 & 4 & 9,12,5,17 \\
\hline
-320.52 & -308.86 & 3.83 & 6 & 16,15,10,5,9,12 \\
\hline
\end{tabular}
\end{table}

\begin{table}
\centering
\begin{tabular}{|c|c|c|c|c|}
\hline
271.8 & 183.47 & 3.39 & 5 & 2,14,13,17,6 \\
\hline
295.04 & 258.69 & 0.61 & 20 & 2,14,13,17,6,12,20,5,4,7,11,18,8,9,10,1,19,3,15,16,16 \\
\hline
-3.17 & 23.26 & 0.44 & 6 & 17,14,13,12,5,9 \\
\hline
-176.96 & -180.7 & 4.2 & 4 & 5,10,12,9 \\
\hline
312.1 & -167.83 & 3.83 & 4 & 4,7,20,8 \\
\hline
22.63 & -128.37 & 0.27 & 7 & 13,17,12,5,10,20,19 \\
\hline
-117.86 & 254.09 & 3 & 4 & 6,11,14,1 \\
\hline
-285.62 & 4.42 & 3.79 & 4 & 9,3,1,11 \\
\hline
126.1 & 53.64 & 0.49 & 10 & 14,13,17,2,12,5,20,10,18,9 \\
\hline
-127.54 & -234.93 & 4.26 & 4 & 10,5,12,19 \\
\hline
-280.99 & -156.97 & 2.49 & 4 & 15,9,3,5 \\
\hline
270.51 & -285.28 & 3.83 & 4 & 7,4,8,20 \\
\hline
-96.81 & 352.63 & 0.19 & 9 & 6,14,11,1,17,9,13,3,2 \\
\hline
-38.2 & -115.55 & 3.9 & 6 & 12,5,17,13,10,9 \\
\hline
-362.89 & 158.73 & 3.84 & 17 & 1,11,3,9,6,15,5,12,14,17,10,13,16,19,20,2,18 \\
\hline
\end{tabular}
\end{table}

\textbf{Case 16前面若干个终端的结果}

\begin{table}
\centering
\begin{tabular}{|c|c|c|c|c|}
\hline
\(x\) & \(y\) & \(z\) & 个数 & 定位基站 \\
\hline
400.6 & 315.13 & 0.5 & 4 & 10,18,14,17 \\
\hline
-273.46 & 6.64 & 3.41 & 4 & 19,13,16,23 \\
\hline
106.91 & -196.86 & 3.17 & 5 & 26,15,25,4,11 \\
\hline
204.99 & 142.1 & 3.36 & 4 & 21,30,17,4 \\
\hline
-393 & -384.9 & 3.39 & 12 & 6,3,1,8,16,19,2,13,23,22,12,25 \\
\hline
69.68 & -154.01 & 0.22 & 4 & 26,25,15,2 \\
\hline
-359.54 & -59.89 & 1.83 & 7 & 19,16,8,23,3,13,1 \\
\hline
-278.37 & 247 & 3.57 & 5 & 27,23,28,24,22 \\
\hline
-257.63 & -26.56 & 3.9 & 4 & 19,16,8,13 \\
\hline
377.05 & -149.19 & 2.84 & 4 & 11,26,15,4 \\
\hline
-74.73 & -365.97 & 1.69 & 5 & 1,3,6,8,2 \\
\hline
-201.16 & -324.3 & 2.64 & 4 & 3,1,6,8 \\
\hline
104.2 & 116.71 & 1.28 & 4 & 30,4,21,25 \\
\hline
-91.84 & 144.62 & 1.76 & 4 & 12,22,24,13 \\
\hline
-29.03 & -291.38 & 3.04 & 4 & 1,3,2,26 \\
\hline
-60.75 & 246.04 & 0.32 & 4 & 24,28,9,7 \\
\hline
-318.29 & 189.83 & 3.24 & 4 & 27,23,22,28 \\
\hline
17 & -198.64 & 3.19 & 10 & 26,2,1,25,8,16,15,4,3,12 \\
\hline
296.8 & 111.33 & 3.32 & 4 & 21,11,18,15 \\
\hline
296.12 & -331.76 & 1.81 & 7 & 26,11,15,4,25,2,1 \\
\hline
209.11 & -159.23 & 0.44 & 18 & 26,15,11,4,25,2,30,21,12,1,29,8,16,17,13,18,14,3 \\
\hline
240.94 & -70.05 & 2.84 & 4 & 11,15,26,4 \\
\hline
135.79 & -46.43 & 3.21 & 4 & 15,4,25,26 \\
\hline
36.75 & -75.79 & 3.02 & 4 & 25,4,2,15 \\
\hline
-105.9 & -341.64 & 0.43 & 24 & 1,3,8,6,2,16,26,19,25,13,12,15,4,22,23,11,30,24,21,2 \\
\hline
\end{tabular}
\end{table}

\begin{table}
\centering
\begin{tabular}{|c|c|c|c|c|}
\hline
 &  &  &  & 8,27,29,17,7 \\
\hline
-71.44 & 315.03 & 4.16 & 4 & 9,24,28,5 \\
\hline
149.35 & 389.62 & 0.06 & 24 & 20,7,14,5,17,29,9,21,30,10,18,24,28,4,12,25,15,22,11 \\
\hline
 &  &  &  & ,13,2,27,23,16 \\
\hline
391.75 & 286.46 & 1.35 & 14 & 18,10,14,17,21,20,29,30,7,11,4,15,5,9 \\
\hline
342.25 & 330.6 & 3.34 & 8 & 10,18,14,17,20,21,29,30 \\
\hline
-395.66 & 265.03 & 3.59 & 5 & 27,23,22,28,24 \\
\hline
60.53 & -7.8 & 2.71 & 4 & 25,4,15,12 \\
\hline
137.45 & 264.97 & 3.56 & 4 & 17,29,21,7 \\
\hline
-264.22 & 2.39 & 3.75 & 4 & 19,13,16,22 \\
\hline
-172.81 & -83.09 & 3.08 & 4 & 8,16,2,1 \\
\hline
-393.9 & 233.26 & 3.41 & 6 & 27,23,22,28,24,19 \\
\hline
-19.74 & -15.3 & 3.1 & 4 & 25,12,2,4 \\
\hline
-19.67 & -54.7 & 3.64 & 4 & 25,2,12,16 \\
\hline
-29.05 & 338.94 & 0.21 & 5 & 9,5,7,24,28 \\
\hline
\end{tabular}
\end{table}

\textbf{Case 17前面若干个终端的结果}

\begin{table}
\centering
\begin{tabular}{|c|c|c|c|c|}
\hline
\(x\) & \(y\) & \(z\) & 个数 & 定位基站 \\
\hline
-339.82 & 309.12 & 3.63 & 4 & 14,37,12,27 \\
\hline
-314.41 & 216.06 & 3.8 & 5 & 14,27,3,37,38 \\
\hline
-284.65 & -246.57 & 3.73 & 5 & 40,33,9,17,11 \\
\hline
267.02 & 390.53 & 0.63 & 4 & 29,1,18,36 \\
\hline
203.84 & -126.97 & 3.24 & 4 & 23,34,4,5 \\
\hline
238.46 & -13.61 & 1.76 & 12 & 34,4,23,2,6,5,7,15,35,36,16,1 \\
\hline
426.78 & -18.97 & 1.4 & 24 & 4,2,34,7,6,5,36,23,20,18,1,21,15,29,35,31,16,22, \\
\hline
 &  &  &  & 8,39,3,30,19,11 \\
\hline
446.18 & -363.8 & 0.91 & 5 & 20,21,7,6,5 \\
\hline
299.62 & -189.51 & 0.54 & 40 & 6,5,7,23,2,34,4,20,21,31,22,8,16,35,15,11,30,19, \\
\hline
 &  &  &  & 26,36,18,1,40,3,29,27,28,39,33,10,38,9,24,17,13 \\
\hline
 &  &  &  & ,32,25,14,37,12,12 \\
\hline
376 & 313.05 & 0.39 & 4 & 36,18,1,29 \\
\hline
-131.6 & 240.52 & 0.68 & 4 & 3,27,39,35 \\
\hline
275.88 & -86.97 & 3.83 & 4 & 34,4,6,23 \\
\hline
291.24 & 206.48 & 2.86 & 4 & 36,18,1,29 \\
\hline
-86.76 & 143.48 & 0.92 & 4 & 35,3,27,15 \\
\hline
-141.38 & 393.94 & 2.25 & 4 & 39,27,3,37 \\
\hline
218.87 & -329.37 & 3.48 & 4 & 5,6,23,22 \\
\hline
125.71 & 169.03 & 3.16 & 9 & 15,1,35,34,39,3,29,4,36 \\
\hline
-150.71 & -95.98 & 0.82 & 4 & 30,19,11,16 \\
\hline
-248.12 & 157.84 & 3.7 & 5 & 27,3,14,35,38 \\
\hline
46.95 & 202.9 & 3.46 & 4 & 15,35,39,3 \\
\hline
410.4 & -263.25 & 0.61 & 4 & 7,6,20,5 \\
\hline
268.98 & -232.03 & 3.55 & 6 & 5,6,23,7,2,34 \\
\hline
\end{tabular}
\end{table}

\begin{table}
\centering
\begin{tabular}{|c|c|c|c|c|}
\hline
-214.1 & 429.73 & 0.29 & 8 & 37,39,27,12,3,14,15,35 \\
\hline
395.3 & 281.13 & 3.1 & 12 & 36,18,1,29,4,34,15,2,39,7,35,3 \\
\hline
-117.64 & 173.77 & 1.95 & 4 & 3,27,35,15 \\
\hline
-216.6 & 89.88 & 3.78 & 4 & 16,35,38,3 \\
\hline
132.57 & -84.47 & 0.5 & 6 & 23,34,4,5,31,6 \\
\hline
113.66 & 246.87 & 3.37 & 5 & 15,1,39,29,3 \\
\hline
162.38 & -178.79 & 3.85 & 5 & 23,5,34,6,31 \\
\hline
-288.87 & 81.51 & 0.3 & 4 & 38,28,10,14 \\
\hline
-395.06 & 378.52 & 0.11 & 40 & 12,37,14,27,3,39,38,35,28,10,15,16,32,19,30,33, \\
 & & & & 9,11,17,29,1,40,8,13,25,18,34,24,31,36,23,4,26, \\
 & & & & 5,22,6,2,7,20,21,21 \\
\hline
-76.63 & -192.47 & 0.94 & 7 & 8,11,31,30,19,16,40 \\
\hline
-359.72 & 282.09 & 3.73 & 4 & 14,37,12,27 \\
\hline
-4.81 & 243.69 & 3.57 & 4 & 15,39,3,27 \\
\hline
-198.64 & -75.48 & 3.63 & 5 & 19,30,11,16,28 \\
\hline
\end{tabular}
\end{table}

\textbf{Case 18 前面若干个终端的结果}

\begin{table}
\centering
\begin{tabular}{|c|c|c|c|c|}
\hline
\(x\) & \(y\) & \(z\) & 个数 & 定位基站 \\
\hline
24.88 & -300.12 & 3.92 & 10 & 37,20,15,3,8,45,4,7,9,21 \\
\hline
-202.64 & -244.62 & 3.35 & 4 & 4,12,22,3 \\
\hline
424.62 & 304.53 & 4.11 & 4 & 11,34,26,44 \\
\hline
-69.31 & -337.97 & 0.93 & 6 & 7,3,15,4,37,22 \\
\hline
233.13 & 5.8 & 1.31 & 4 & 10,5,45,40 \\
\hline
-324.48 & -179.81 & 1.64 & 4 & 42,4,28,2 \\
\hline
348.81 & -197.37 & 3.6 & 38 & 40,36,43,41,48,16,37,8,45,20,10,5,9,15,3,13,11,49,21 \\
 & & & & ,14,4,34,7,26,38,22,12,44,6,27,2,42,23,30,18,32,19,3 \\
 & & & & 9 \\
\hline
-385.36 & 166.42 & 2.98 & 4 & 33,1,35,30 \\
\hline
82.42 & -389.65 & 4.25 & 4 & 20,8,37,15 \\
\hline
-240.84 & 21.5 & 3.32 & 4 & 21,2,30,38 \\
\hline
259.91 & -392.63 & 0.7 & 11 & 8,20,37,48,43,16,41,40,36,45,15 \\
\hline
378.18 & 348.02 & 4.11 & 4 & 11,34,26,44 \\
\hline
-490.14 & 38.05 & 3.39 & 6 & 1,30,2,33,35,38 \\
\hline
-183.62 & -460.14 & 3.52 & 49 & 7,27,22,12,23,4,42,3,28,20,15,37,8,21,45,2,9,49,30,4 \\
 & & & & 0,38,48,43,14,16,41,1,36,10,13,33,5,35,32,19,29,47,2 \\
 & & & & 4,46,31,25,17,18,50,39,11,6,34,26 \\
\hline
244.99 & -388.47 & 4.31 & 4 & 8,20,37,48 \\
\hline
-313.39 & -452.93 & 3.2 & 4 & 27,22,12,7 \\
\hline
-404.39 & -215.88 & 3.53 & 4 & 42,28,23,12 \\
\hline
-391.48 & -245.78 & 3.49 & 41 & 42,28,23,12,22,4,2,27,30,7,21,3,15,49,1,38,9,33,45,3 \\
 & & & & 7,35,14,20,29,32,8,19,13,47,25,46,24,17,31,50,10,40, \\
 & & & & 18,5,39,36 \\
\hline
275.22 & 101.66 & 2.86 & 4 & 10,5,13,11 \\
\hline
\end{tabular}
\end{table}

\begin{table}
\centering
\begin{tabular}{|c|c|c|c|c|}
\hline
-65.94 & -50.77 & 3.26 & 4 & 21,9,15,49 \\
\hline
313.35 & -183.67 & 3 & 5 & 40,36,43,37,41 \\
\hline
-96.21 & 288.39 & 0.47 & 5 & 19,14,32,18,47 \\
\hline
30.43 & -71.82 & 4.17 & 4 & 9,45,15,3 \\
\hline
400.12 & -240.09 & 3.77 & 4 & 36,43,40,41 \\
\hline
-22.06 & -383.97 & 4.25 & 4 & 20,37,7,8 \\
\hline
-401.41 & 110.19 & 0.18 & 6 & 30,1,33,2,35,38 \\
\hline
-259.02 & -483.78 & 3.2 & 4 & 27,7,22,12 \\
\hline
-436.4 & -12.69 & 3.51 & 5 & 2,30,1,28,42 \\
\hline
-204.64 & -147.26 & 3.37 & 4 & 4,21,3,15 \\
\hline
217.24 & -232.91 & 4.31 & 4 & 37,40,20,8 \\
\hline
-426.91 & -86.68 & 1.7 & 5 & 2,30,28,42,4 \\
\hline
301.22 & 474.6 & 3.64 & 4 & 44,6,26,34 \\
\hline
-452.64 & 442.08 & 3.45 & 5 & 25,17,50,29,46 \\
\hline
490.41 & 233.83 & 3.93 & 4 & 11,34,26,5 \\
\hline
444.36 & -267.39 & 3.6 & 5 & 41,43,48,36,16 \\
\hline
354.35 & 420.59 & 3.72 & 5 & 26,44,34,11,6 \\
\hline
452.62 & 468.05 & 4.11 & 4 & 26,34,44,11 \\
\hline
\end{tabular}
\end{table}

\textbf{Case 19 前面若干个终端的结果}

\begin{table}
\centering
\begin{tabular}{|c|c|c|c|c|}
\hline
\(x\) & \(y\) & \(z\) & 个数 & 定位基站 \\
\hline
402.14 & 293.02 & 1.68 & 4 & 4,1,28,43 \\
\hline
349.6 & -86.29 & 0.68 & 7 & 23,59,6,18,28,38,41 \\
\hline
147.24 & 514.93 & 3.04 & 5 & 60,43,51,1,4 \\
\hline
185 & 296.55 & 1.53 & 9 & 1,43,60,41,4,51,6,28,35 \\
\hline
-521.75 & 611.84 & 0.38 & 24 & 3,45,54,52,21,50,40,55,51,57,46,11,39,30,16,60,43,9, \\
& & & & 17,32,49,5,35,12 \\
\hline
-238.22 & 136.95 & 2.52 & 4 & 11,46,16,40 \\
\hline
-411.34 & 596.1 & 2.98 & 4 & 3,45,52,54 \\
\hline
150.63 & -397.86 & 4.48 & 4 & 26,22,36,27 \\
\hline
139.25 & -94.25 & 0.04 & 4 & 59,23,6,41 \\
\hline
521.06 & 502.09 & 3.1 & 5 & 10,4,60,43,1 \\
\hline
-502.17 & -215.92 & 2.36 & 4 & 37,9,32,15 \\
\hline
326.89 & -468.94 & 3.31 & 4 & 36,14,53,44 \\
\hline
-382.94 & -481.21 & 3.8 & 4 & 56,34,7,19 \\
\hline
308.89 & 602.66 & 3.14 & 4 & 60,43,10,4 \\
\hline
361.06 & -516.4 & 2.9 & 4 & 14,36,53,29 \\
\hline
128.19 & 65.53 & 3.99 & 4 & 41,35,6,1 \\
\hline
-420.15 & -252.33 & 1.44 & 4 & 15,37,32,9 \\
\hline
-492.78 & 603.64 & 3.22 & 5 & 3,45,54,52,21 \\
\hline
154.78 & 92.61 & 0.25 & 17 & 41,1,6,35,59,23,28,11,18,43,27,12,4,5,17,16,22 \\
\hline
509.26 & -550.27 & 3.13 & 5 & 29,14,48,8,36 \\
\hline
235.07 & 41.2 & 2.33 & 4 & 41,6,1,59 \\
\hline
\end{tabular}
\end{table}

\begin{table}
\centering
\begin{tabular}{|c|c|c|c|c|}
\hline
-222.58 & -301.32 & 3.84 & 4 & 13,24,15,47 \\
\hline
412.4 & -523.64 & 1.06 & 4 & 14,36,29,53 \\
\hline
312.67 & 391.14 & 0.16 & 4 & 43,4,60,1 \\
\hline
-490.41 & 131.31 & 0.15 & 10 & 57,55,46,39,40,30,50,54,9,32 \\
\hline
-161.87 & 497.83 & 1.71 & 4 & 51,52,45,21 \\
\hline
613.78 & 134.13 & 1.06 & 7 & 28,18,4,6,41,1,10 \\
\hline
117.02 & -280.82 & 4.34 & 4 & 22,27,26,42 \\
\hline
244.11 & 38.23 & 3.77 & 6 & 41,6,59,23,1,28 \\
\hline
-103.7 & 176.02 & 1.09 & 7 & 11,35,16,17,21,52,46 \\
\hline
-292.87 & -256.59 & 1.03 & 17 & 15,37,49,32,13,31,5,56,12,24,17,9,7,34,47,16,42 \\
\hline
-243.67 & -308.75 & 0.03 & 4 & 13,15,24,56 \\
\hline
-604.23 & 214.57 & 4.41 & 4 & 57,55,39,40 \\
\hline
-76.73 & -291.5 & 1.05 & 4 & 24,42,47,13 \\
\hline
293.84 & -291.81 & 4.68 & 4 & 44,53,38,27 \\
\hline
-626.5 & -440.17 & 0.03 & 60 & 25,33,2,19,31,37,15,56,34,7,9,32,39,13,30,58,57,49,2 \\
\hline
 & & & & 4,17,5,12,47,46,16,42,20,55,40,22,26,11,35,50,54,27, \\
\hline
 & & & & 21,36,52,59,23,44,53,6,41,1,38,14,3,45,51,8,28,43,18 \\
\hline
 & & & & ,48,29,60,4,10,10 \\
\hline
-545.05 & 608.97 & 3.79 & 6 & 3,45,54,21,52,50 \\
\hline
-20.82 & 354.72 & 0.59 & 4 & 51,43,52,11 \\
\hline
\end{tabular}
\end{table}

\textbf{Case 20 前面若干个终端的结果}

\begin{table}
\centering
\begin{tabular}{|c|c|c|c|c|}
\hline
\(x\) & \(y\) & \(z\) & 个数 & 定位基站 \\
\hline
-95.61 & 158.65 & 0.02 & 7 & 17,1,13,9,7,20,4 \\
\hline
-164.04 & -22.07 & 4.15 & 4 & 18,3,10,15 \\
\hline
-49.87 & -71.17 & 3.97 & 4 & 3,18,6,16 \\
\hline
131.17 & -29.91 & 0.63 & 7 & 8,7,2,12,13,14,16 \\
\hline
-224.46 & -283.41 & 3.91 & 4 & 6,5,10,16 \\
\hline
-169.76 & 92.37 & 0.32 & 20 & 4,15,17,18,1,20,19,3,13,7,10,9,5,6,8,11,16,14,2,12,12 \\
\hline
-216.12 & -67.2 & 3.92 & 4 & 10,3,5,18 \\
\hline
70.56 & 156.78 & 3.34 & 4 & 7,13,8,11 \\
\hline
215.61 & -95.6 & 3.84 & 4 & 2,12,8,14 \\
\hline
63.43 & 54.02 & 3.48 & 4 & 7,8,13,1 \\
\hline
290.75 & -99.34 & 2.8 & 11 & 12,2,8,14,7,16,13,11,1,18,6 \\
\hline
0.38 & 11.66 & 0.62 & 5 & 7,13,8,1,18 \\
\hline
120.86 & -285.54 & 3.67 & 5 & 16,2,12,6,8 \\
\hline
-36.22 & -4.12 & 3.73 & 4 & 13,7,18,3 \\
\hline
-15.75 & 296.69 & 3.12 & 4 & 9,11,1,17 \\
\hline
-50.39 & -237.36 & 4.04 & 5 & 16,6,10,3,18 \\
\hline
-102.6 & -114.52 & 1.33 & 10 & 6,3,10,18,16,5,15,4,13,7 \\
\hline
117.99 & 271.43 & 0.12 & 14 & 11,9,14,8,7,1,13,17,20,18,4,15,19,12 \\
\hline
-159.51 & 160.02 & 0.66 & 4 & 17,20,1,4 \\
\hline
-94.88 & 0.61 & 3.67 & 4 & 18,3,13,15 \\
\hline
\end{tabular}
\end{table}

\begin{tabular}{|c|c|c|c|c|}
\hline
25.25 & 186.44 & 1.52 & 4 & 9,1,13,7 \\
\hline
280.1 & -4.72 & 3.59 & 7 & 12,2,14,8,7,13,11 \\
\hline
-244.36 & -165.87 & 3.99 & 4 & 5,10,6,3 \\
\hline
-11.25 & -12.24 & 0.7 & 5 & 7,13,18,8,3 \\
\hline
200.56 & -50.28 & 0.89 & 4 & 2,12,8,14 \\
\hline
-47.18 & -274.99 & 3.63 & 10 & 16,6,10,3,5,18,2,12,7,13 \\
\hline
175.58 & -174.41 & 2.01 & 4 & 2,12,16,8 \\
\hline
114.71 & -82.26 & 0.69 & 4 & 2,8,12,7 \\
\hline
-231.65 & 70.93 & 1.65 & 5 & 4,15,19,18,20 \\
\hline
52.44 & 286.72 & 2.85 & 4 & 11,9,1,13 \\
\hline
291.09 & 50.86 & 3.54 & 19 & 14,8,12,2,7,11,13,1,9,17,16,18,3,6,15,4,10,20,19 \\
\hline
164.43 & -35.2 & 3.01 & 4 & 8,2,12,7 \\
\hline
30.55 & 293.91 & 0.84 & 18 & 9,11,1,17,13,7,8,14,20,4,15,18,19,3,10,2,12,6 \\
\hline
-224.96 & -237.78 & 0.5 & 8 & 6,5,10,3,16,18,15,4 \\
\hline
43.77 & 126.47 & 3.49 & 4 & 7,13,8,1 \\
\hline
-14.3 & 213.74 & 3.03 & 4 & 9,1,17,13 \\
\hline
230.95 & -246.67 & 0.81 & 4 & 2,12,16,8 \\
\hline
-287.95 & -170.8 & 1.05 & 4 & 5,10,6,3 \\
\hline
\end{tabular}

\begin{table}
\centering
\begin{tabular}{|c|c|c|}
\hline
\multicolumn{3}{|c|}{case 12} \\
\hline
基站序号 & 使用频数 & 使用频率 \\
\hline
1 & 206 & 0.022692 \\
\hline
2 & 244 & 0.026878 \\
\hline
3 & 159 & 0.017515 \\
\hline
4 & 200 & 0.022031 \\
\hline
5 & 166 & 0.018286 \\
\hline
6 & 318 & 0.03503 \\
\hline
7 & 200 & 0.022031 \\
\hline
8 & 192 & 0.02115 \\
\hline
9 & 317 & 0.03492 \\
\hline
10 & 287 & 0.031615 \\
\hline
11 & 328 & 0.036131 \\
\hline
12 & 225 & 0.024785 \\
\hline
13 & 160 & 0.017625 \\
\hline
14 & 385 & 0.04241 \\
\hline
15 & 289 & 0.031835 \\
\hline
16 & 182 & 0.020048 \\
\hline
17 & 184 & 0.020269 \\
\hline
18 & 245 & 0.026988 \\
\hline
19 & 119 & 0.013109 \\
\hline
20 & 168 & 0.018506 \\
\hline
21 & 324 & 0.035691 \\
\hline
22 & 289 & 0.031835 \\
\hline
23 & 237 & 0.026107 \\
\hline
24 & 239 & 0.026327 \\
\hline
25 & 203 & 0.022362 \\
\hline
26 & 297 & 0.032716 \\
\hline
27 & 208 & 0.022913 \\
\hline
28 & 180 & 0.019828 \\
\hline
29 & 118 & 0.012998 \\
\hline
30 & 231 & 0.025446 \\
\hline
31 & 139 & 0.015312 \\
\hline
32 & 214 & 0.023573 \\
\hline
33 & 214 & 0.023573 \\
\hline
34 & 368 & 0.040538 \\
\hline
35 & 359 & 0.039546 \\
\hline
36 & 158 & 0.017405 \\
\hline
37 & 242 & 0.026658 \\
\hline
38 & 160 & 0.017625 \\
\hline
\end{tabular}
\end{table}

\begin{table}
\centering
\begin{tabular}{|c|c|c|}
\hline
\multicolumn{3}{|c|}{case 13} \\
\hline
基站序号 & 使用频数 & 使用频率 \\
\hline
1 & 216 & 0.024479 \\
\hline
2 & 194 & 0.021985 \\
\hline
3 & 234 & 0.026519 \\
\hline
4 & 256 & 0.029012 \\
\hline
5 & 150 & 0.016999 \\
\hline
6 & 191 & 0.021646 \\
\hline
7 & 258 & 0.029238 \\
\hline
8 & 157 & 0.017792 \\
\hline
9 & 156 & 0.017679 \\
\hline
10 & 167 & 0.018926 \\
\hline
11 & 201 & 0.022779 \\
\hline
12 & 115 & 0.013033 \\
\hline
13 & 211 & 0.023912 \\
\hline
14 & 230 & 0.026065 \\
\hline
15 & 186 & 0.021079 \\
\hline
16 & 244 & 0.027652 \\
\hline
17 & 172 & 0.019492 \\
\hline
18 & 195 & 0.022099 \\
\hline
19 & 245 & 0.027765 \\
\hline
20 & 167 & 0.018926 \\
\hline
21 & 183 & 0.020739 \\
\hline
22 & 203 & 0.023005 \\
\hline
23 & 128 & 0.014506 \\
\hline
24 & 116 & 0.013146 \\
\hline
25 & 166 & 0.018812 \\
\hline
26 & 178 & 0.020172 \\
\hline
27 & 188 & 0.021306 \\
\hline
28 & 132 & 0.014959 \\
\hline
29 & 120 & 0.013599 \\
\hline
30 & 116 & 0.013146 \\
\hline
31 & 162 & 0.018359 \\
\hline
32 & 267 & 0.030258 \\
\hline
33 & 222 & 0.025159 \\
\hline
34 & 107 & 0.012126 \\
\hline
35 & 143 & 0.016206 \\
\hline
36 & 158 & 0.017906 \\
\hline
\end{tabular}
\end{table}

\begin{table}
\centering
\begin{tabular}{|c|c|c|}
\hline
37 & 116 & 0.013146 \\
\hline
38 & 73 & 0.008273 \\
\hline
39 & 201 & 0.022779 \\
\hline
40 & 132 & 0.014959 \\
\hline
41 & 223 & 0.025272 \\
\hline
42 & 152 & 0.017226 \\
\hline
43 & 137 & 0.015526 \\
\hline
44 & 172 & 0.019492 \\
\hline
45 & 213 & 0.024139 \\
\hline
46 & 179 & 0.020286 \\
\hline
47 & 186 & 0.021079 \\
\hline
48 & 166 & 0.018812 \\
\hline
49 & 144 & 0.016319 \\
\hline
50 & 196 & 0.022212 \\
\hline
\end{tabular}
\end{table}

\begin{table}
\centering
\begin{tabular}{|c|c|c|}
\hline
\multicolumn{3}{|c|}{case 14} \\
\hline
基站序号 & 使用频数 & 使用频率 \\
\hline
1 & 260 & 0.016223 \\
\hline
2 & 288 & 0.01797 \\
\hline
3 & 319 & 0.019904 \\
\hline
4 & 326 & 0.020341 \\
\hline
5 & 283 & 0.017658 \\
\hline
6 & 210 & 0.013103 \\
\hline
7 & 320 & 0.019966 \\
\hline
8 & 194 & 0.012105 \\
\hline
9 & 249 & 0.015536 \\
\hline
10 & 219 & 0.013664 \\
\hline
11 & 198 & 0.012354 \\
\hline
12 & 293 & 0.018282 \\
\hline
13 & 186 & 0.011605 \\
\hline
14 & 240 & 0.014975 \\
\hline
15 & 246 & 0.015349 \\
\hline
16 & 265 & 0.016535 \\
\hline
17 & 245 & 0.015287 \\
\hline
18 & 334 & 0.02084 \\
\hline
19 & 291 & 0.018157 \\
\hline
20 & 336 & 0.020965 \\
\hline
21 & 309 & 0.01928 \\
\hline
22 & 194 & 0.012105 \\
\hline
23 & 240 & 0.014975 \\
\hline
24 & 288 & 0.01797 \\
\hline
\end{tabular}
\end{table}

\begin{table}
\centering
\begin{tabular}{|c|c|c|}
\hline
25 & 243 & 0.015162 \\
\hline
26 & 331 & 0.020653 \\
\hline
27 & 256 & 0.015973 \\
\hline
28 & 246 & 0.015349 \\
\hline
29 & 253 & 0.015786 \\
\hline
30 & 276 & 0.017221 \\
\hline
31 & 315 & 0.019654 \\
\hline
32 & 186 & 0.011605 \\
\hline
33 & 237 & 0.014788 \\
\hline
34 & 317 & 0.019779 \\
\hline
35 & 315 & 0.019654 \\
\hline
36 & 209 & 0.01304 \\
\hline
37 & 288 & 0.01797 \\
\hline
38 & 177 & 0.011044 \\
\hline
39 & 271 & 0.016909 \\
\hline
40 & 228 & 0.014226 \\
\hline
41 & 226 & 0.014101 \\
\hline
42 & 290 & 0.018094 \\
\hline
43 & 207 & 0.012916 \\
\hline
44 & 331 & 0.020653 \\
\hline
45 & 303 & 0.018906 \\
\hline
46 & 318 & 0.019842 \\
\hline
47 & 254 & 0.015848 \\
\hline
48 & 265 & 0.016535 \\
\hline
49 & 309 & 0.01928 \\
\hline
50 & 268 & 0.016722 \\
\hline
51 & 312 & 0.019467 \\
\hline
52 & 291 & 0.018157 \\
\hline
53 & 284 & 0.01772 \\
\hline
54 & 260 & 0.016223 \\
\hline
55 & 214 & 0.013352 \\
\hline
56 & 307 & 0.019155 \\
\hline
57 & 282 & 0.017595 \\
\hline
58 & 295 & 0.018406 \\
\hline
59 & 224 & 0.013976 \\
\hline
60 & 306 & 0.019093 \\
\hline
\end{tabular}
\end{table}

\begin{table}
\centering
\begin{tabular}{|c|c|c|}
\hline
\multicolumn{3}{|c|}{case 16} \\
\hline
基站序号 & 使用频数 & 使用频率 \\
\hline
1 & 360 & 0.04878 \\
\hline
2 & 360 & 0.04878 \\
\hline
3 & 282 & 0.038211 \\
\hline
\end{tabular}
\end{table}

\begin{table}
\centering
\begin{tabular}{|c|c|c|}
\hline
4 & 395 & 0.053523 \\
\hline
5 & 155 & 0.021003 \\
\hline
6 & 143 & 0.019377 \\
\hline
7 & 168 & 0.022764 \\
\hline
8 & 362 & 0.049051 \\
\hline
9 & 182 & 0.024661 \\
\hline
10 & 93 & 0.012602 \\
\hline
11 & 273 & 0.036992 \\
\hline
12 & 284 & 0.038482 \\
\hline
13 & 278 & 0.037669 \\
\hline
14 & 188 & 0.025474 \\
\hline
15 & 353 & 0.047832 \\
\hline
16 & 309 & 0.04187 \\
\hline
17 & 206 & 0.027913 \\
\hline
18 & 157 & 0.021274 \\
\hline
19 & 236 & 0.031978 \\
\hline
20 & 139 & 0.018835 \\
\hline
21 & 235 & 0.031843 \\
\hline
22 & 272 & 0.036856 \\
\hline
23 & 205 & 0.027778 \\
\hline
24 & 240 & 0.03252 \\
\hline
25 & 373 & 0.050542 \\
\hline
26 & 313 & 0.042412 \\
\hline
27 & 170 & 0.023035 \\
\hline
28 & 227 & 0.030759 \\
\hline
29 & 193 & 0.026152 \\
\hline
30 & 229 & 0.03103 \\
\hline
\end{tabular}
\end{table}

\begin{table}
\centering
\begin{tabular}{|c|c|c|}
\hline
\multicolumn{3}{|c|}{case 17} \\
\hline
基站序号 & 使用频数 & 使用频率 \\
\hline
1 & 311 & 0.027537 \\
\hline
2 & 242 & 0.021427 \\
\hline
3 & 426 & 0.037719 \\
\hline
4 & 351 & 0.031078 \\
\hline
5 & 377 & 0.033381 \\
\hline
6 & 369 & 0.032672 \\
\hline
7 & 247 & 0.02187 \\
\hline
8 & 287 & 0.025412 \\
\hline
9 & 197 & 0.017443 \\
\hline
10 & 260 & 0.023021 \\
\hline
11 & 299 & 0.026474 \\
\hline
\end{tabular}
\end{table}

\begin{table}
\centering
\begin{tabular}{|c|c|c|}
\hline
12 & 174 & 0.015406 \\
\hline
13 & 170 & 0.015052 \\
\hline
14 & 263 & 0.023287 \\
\hline
15 & 451 & 0.039933 \\
\hline
16 & 369 & 0.032672 \\
\hline
17 & 200 & 0.017709 \\
\hline
18 & 237 & 0.020985 \\
\hline
19 & 318 & 0.028157 \\
\hline
20 & 204 & 0.018063 \\
\hline
21 & 171 & 0.015141 \\
\hline
22 & 244 & 0.021604 \\
\hline
23 & 401 & 0.035506 \\
\hline
24 & 189 & 0.016735 \\
\hline
25 & 141 & 0.012485 \\
\hline
26 & 232 & 0.020542 \\
\hline
27 & 372 & 0.032938 \\
\hline
28 & 290 & 0.025677 \\
\hline
29 & 269 & 0.023818 \\
\hline
30 & 314 & 0.027802 \\
\hline
31 & 317 & 0.028068 \\
\hline
32 & 206 & 0.01824 \\
\hline
33 & 233 & 0.02063 \\
\hline
34 & 403 & 0.035683 \\
\hline
35 & 421 & 0.037276 \\
\hline
36 & 257 & 0.022755 \\
\hline
37 & 228 & 0.020188 \\
\hline
38 & 282 & 0.024969 \\
\hline
39 & 324 & 0.028688 \\
\hline
40 & 248 & 0.021959 \\
\hline
\end{tabular}
\end{table}

\begin{table}
\centering
\begin{tabular}{|c|c|c|}
\hline
\multicolumn{3}{|c|}{case 18} \\
\hline
基站序号 & 使用频数 & 使用频率 \\
\hline
1 & 182 & 0.015811 \\
\hline
2 & 256 & 0.022224 \\
\hline
3 & 281 & 0.024411 \\
\hline
4 & 278 & 0.024151 \\
\hline
5 & 333 & 0.028929 \\
\hline
6 & 206 & 0.017896 \\
\hline
7 & 234 & 0.020328 \\
\hline
8 & 242 & 0.021023 \\
\hline
9 & 326 & 0.028321 \\
\hline
10 & 354 & 0.030753 \\
\hline
\end{tabular}
\end{table}

\begin{tabular}{|c|c|c|}
\hline
11 & 260 & 0.022587 \\
\hline
12 & 268 & 0.023282 \\
\hline
13 & 339 & 0.02945 \\
\hline
14 & 300 & 0.026062 \\
\hline
15 & 303 & 0.026323 \\
\hline
16 & 178 & 0.015463 \\
\hline
17 & 131 & 0.01138 \\
\hline
18 & 190 & 0.016506 \\
\hline
19 & 188 & 0.016332 \\
\hline
20 & 250 & 0.021718 \\
\hline
21 & 291 & 0.02528 \\
\hline
22 & 251 & 0.021805 \\
\hline
23 & 219 & 0.019025 \\
\hline
24 & 146 & 0.012684 \\
\hline
25 & 126 & 0.010946 \\
\hline
26 & 212 & 0.018417 \\
\hline
27 & 189 & 0.016419 \\
\hline
28 & 186 & 0.016158 \\
\hline
29 & 169 & 0.014682 \\
\hline
30 & 262 & 0.022761 \\
\hline
31 & 128 & 0.01112 \\
\hline
32 & 196 & 0.017027 \\
\hline
33 & 235 & 0.020415 \\
\hline
34 & 208 & 0.01807 \\
\hline
35 & 206 & 0.017896 \\
\hline
36 & 254 & 0.022066 \\
\hline
37 & 303 & 0.026323 \\
\hline
38 & 269 & 0.023369 \\
\hline
39 & 160 & 0.0139 \\
\hline
40 & 317 & 0.027539 \\
\hline
41 & 201 & 0.017462 \\
\hline
42 & 234 & 0.020328 \\
\hline
43 & 223 & 0.019373 \\
\hline
44 & 184 & 0.015985 \\
\hline
45 & 348 & 0.030232 \\
\hline
46 & 136 & 0.011815 \\
\hline
47 & 152 & 0.013205 \\
\hline
48 & 198 & 0.017201 \\
\hline
49 & 289 & 0.025106 \\
\hline
50 & 120 & 0.010425 \\
\hline
\end{tabular}

\begin{table}
\centering
\begin{tabular}{|c|c|c|}
\hline
\multicolumn{3}{|c|}{case 19} \\
\hline
基站序号 & 使用频数 & 使用频率 \\
\hline
1 & 304 & 0.027737 \\
\hline
2 & 122 & 0.011131 \\
\hline
3 & 151 & 0.013777 \\
\hline
4 & 259 & 0.023631 \\
\hline
5 & 164 & 0.014964 \\
\hline
6 & 255 & 0.023266 \\
\hline
7 & 155 & 0.014142 \\
\hline
8 & 171 & 0.015602 \\
\hline
9 & 158 & 0.014416 \\
\hline
10 & 158 & 0.014416 \\
\hline
11 & 207 & 0.018887 \\
\hline
12 & 180 & 0.016423 \\
\hline
13 & 167 & 0.015237 \\
\hline
14 & 164 & 0.014964 \\
\hline
15 & 154 & 0.014051 \\
\hline
16 & 173 & 0.015785 \\
\hline
17 & 185 & 0.01688 \\
\hline
18 & 223 & 0.020347 \\
\hline
19 & 129 & 0.01177 \\
\hline
20 & 126 & 0.011496 \\
\hline
21 & 233 & 0.021259 \\
\hline
22 & 181 & 0.016515 \\
\hline
23 & 213 & 0.019434 \\
\hline
24 & 155 & 0.014142 \\
\hline
25 & 94 & 0.008577 \\
\hline
26 & 188 & 0.017153 \\
\hline
27 & 176 & 0.016058 \\
\hline
28 & 246 & 0.022445 \\
\hline
29 & 88 & 0.008029 \\
\hline
30 & 172 & 0.015693 \\
\hline
31 & 147 & 0.013412 \\
\hline
32 & 167 & 0.015237 \\
\hline
33 & 78 & 0.007117 \\
\hline
34 & 147 & 0.013412 \\
\hline
35 & 217 & 0.019799 \\
\hline
36 & 174 & 0.015876 \\
\hline
37 & 146 & 0.013321 \\
\hline
38 & 211 & 0.019252 \\
\hline
39 & 139 & 0.012682 \\
\hline
\end{tabular}
\end{table}

\begin{table}
\centering
\begin{tabular}{|c|c|c|}
\hline
40 & 227 & 0.020712 \\
\hline
41 & 252 & 0.022993 \\
\hline
42 & 174 & 0.015876 \\
\hline
43 & 286 & 0.026095 \\
\hline
44 & 197 & 0.017974 \\
\hline
45 & 165 & 0.015055 \\
\hline
46 & 178 & 0.016241 \\
\hline
47 & 176 & 0.016058 \\
\hline
48 & 143 & 0.013047 \\
\hline
49 & 156 & 0.014234 \\
\hline
50 & 225 & 0.020529 \\
\hline
51 & 226 & 0.02062 \\
\hline
52 & 242 & 0.02208 \\
\hline
53 & 196 & 0.017883 \\
\hline
54 & 210 & 0.019161 \\
\hline
55 & 204 & 0.018613 \\
\hline
56 & 152 & 0.013869 \\
\hline
57 & 181 & 0.016515 \\
\hline
58 & 162 & 0.014781 \\
\hline
59 & 190 & 0.017336 \\
\hline
60 & 241 & 0.021989 \\
\hline
\end{tabular}
\end{table}

\begin{table}
\centering
\begin{tabular}{|c|c|c|}
\hline
\multicolumn{3}{|c|}{case 20} \\
\hline
基站序号 & 使用频数 & 使用频率 \\
\hline
1 & 348 & 0.061658 \\
\hline
2 & 302 & 0.053508 \\
\hline
3 & 343 & 0.060773 \\
\hline
4 & 226 & 0.040043 \\
\hline
5 & 179 & 0.031715 \\
\hline
6 & 289 & 0.051205 \\
\hline
7 & 422 & 0.07477 \\
\hline
8 & 457 & 0.080971 \\
\hline
9 & 222 & 0.039334 \\
\hline
10 & 274 & 0.048547 \\
\hline
11 & 222 & 0.039334 \\
\hline
12 & 273 & 0.04837 \\
\hline
13 & 383 & 0.06786 \\
\hline
14 & 261 & 0.046244 \\
\hline
15 & 218 & 0.038625 \\
\hline
16 & 325 & 0.057583 \\
\hline
17 & 296 & 0.052445 \\
\hline
\end{tabular}
\end{table}

\begin{table}
\begin{tabular}{c c c}
\hline
18 & 285 & 0.050496 \\
\hline
19 & 134 & 0.023742 \\
\hline
20 & 185 & 0.032778 \\
\hline
\end{tabular}
\end{table}