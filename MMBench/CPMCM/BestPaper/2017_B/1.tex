\begin{center}
\textbf{“华为杯”第十四届中国研究生数学建模竞赛}
\end{center}

\begin{center}
参赛密码 \hspace{2cm} (由组委会填写)
\end{center}

\begin{center}
\includegraphics[width=0.3\textwidth]{image1.png} \quad
\includegraphics[width=0.3\textwidth]{image2.png} \quad
\includegraphics[width=0.3\textwidth]{image3.png}
\end{center}

\begin{center}
\textbf{“华为杯”第十四届中国研究生数学建模竞赛}
\end{center}

\begin{tabular}{l l}
学校 & 吉林大学 \\
\hline
参赛队号 & 10183125 \\
\hline
队员姓名 & 1. 熊晓勇 \\
& 2. 冯达 \\
& 3. 张英泽 \\
\hline
\end{tabular}

\begin{center}
\textbf{参赛密码} \underline{\hspace{2cm}} \\
(由组委会填写)
\end{center}

\begin{center}
\includegraphics[width=0.3\textwidth]{image1.png} \quad
\includegraphics[width=0.3\textwidth]{image2.png} \quad
\includegraphics[width=0.3\textwidth]{image3.png}
\end{center}

\begin{center}
\textbf{“华为杯”第十四届中国研究生} \\
\textbf{数学建模竞赛}
\end{center}

\begin{flushleft}
\textbf{题目} \hspace{1cm} 面向下一代光通信的 VCSEL 激光器仿真建模
\end{flushleft}

\begin{center}
\textbf{摘 \hspace{2cm} 要:}
\end{center}

垂直腔面发射激光器(VCSEL)作为光纤通信系统的核心器件,具有使用简单,功耗较低等特点,因此需要准确掌握该器件的特性以保证光纤通信系统仿真模型的精度。本文依据实验数据,建立 VCSEL 激光器 L-I 模型和小信号频率响应模型,并运用自适应遗传算法、粒子群算法等多种算法求解,对比得出一组最优参数使得拟合曲线与原始曲线重合度较高。同时分析各参数对其工作特性的影响,并提出优化方案。

对于问题 1 提出的非线性隐函数参数寻优问题,本文建立了激光器 L-I 参数优化模型,分别采用自适应遗传算法和粒子群算法求解计算。结果对比显示前者优化效果更佳,拟合曲线误差更低。优化后参数为 \((\eta, I_{th0}, R_{th}, a_0, a_1, a_2, a_3, a_4) = (0.5, 0.3E-3, 2.6E3, 1.25E-3, -2.55E-5, 2.91E-7, -2.53E-10, 1.022E-12)\)。另外根据建立的 L-I 模型推测出:VCSEL 激光器工作环境温度不能高 \(42.7^\circ \mathrm{C}\),才能保证网络正常工作。

对于问题 2 提出的模型改进要求,首先分析上述模型的误差和产生原因,即忽略了温度对某些参数的影响,从而建立考虑温度因素的 L-I 改进模型,新增加

6 项温度相关参数并重新进行优化求解。绘制不同温度下的 L-I 曲线表明,激光器正常工作温度修正为 \(38.2^\circ \mathrm{C}\)。

对于问题 3,首先根据文献建立了基于速率方程的 VCSEL 小信号幅频响应参数模型,确定待优化参数构成为 \((\eta_i, \beta, \tau_n, k, G_0, N_0, \tau_p, \varepsilon)\),比较模拟退火(SA)和多目标遗传算法(NSGA-II)求解结果,最终得到最优参数为 \((0.7, 9.59 \mathrm{E}-5, 4.36 \mathrm{E}-9, 1.72 \mathrm{E}-8, 1.04 \mathrm{E}6, 8.69 \mathrm{E}5, 1.15 \mathrm{E}-11, 1.19 \mathrm{E}-7)\)。接着绘制不同环境温度和不同偏置电流下的带宽响应曲线(S21)曲线,结果表明带宽随着偏置电流的增大而增大,随着温度的增大而减小。然后基于上述结论,提出 2 种方案作为激光器优化设计方向。最后,筛选出对带宽曲线走势影响较大的参数——\(\tau_n\)(载流子复合寿命),降低 \(\tau_n\) 既能扩大激光器的工作带宽,又能降低高于 0 频的幅度(使曲线变得平坦),有助于提高激光器工作性能。

对于问题 4,改进模型基于 VCSEL 器件的硬件结构,不仅从电学、热学两方面考虑,又创新性地考虑了器件的光学特性,重新构建 VCSEL 器件稳小信号特性的速率方程模型。基于改进模型重新绘制宽带响应曲线表明,相同环境温度和偏置电流下改进模型的激光器 -10dB 工作带宽更宽。

本文优势在于采用多种算法进行优化参数求解,比较选取最优参数。同时模型时考虑了温度影响等因素,从光学、电学以及热学 3 方面对模型重构,提高了模型的适应性和准确性。

关键词:垂直腔面发射激光器,非线性参数估计,自适应遗传算法,粒子群算法

\section*{目录}
\begin{itemize}
    \item 一、问题重述 \dotfill 4
    \begin{itemize}
        \item 1.1 背景介绍 \dotfill 4
        \item 1.2 问题引入 \dotfill 4
    \end{itemize}
    \item 二、主要变量说明 \dotfill 6
    \item 三、条件假设 \dotfill 7
    \item 四、问题分析 \dotfill 8
    \begin{itemize}
        \item 4.1 问题1的分析 \dotfill 8
        \item 4.2 问题2的分析 \dotfill 8
        \item 4.3 问题3的分析 \dotfill 8
        \item 4.4 问题4的分析 \dotfill 9
    \end{itemize}
    \item 五、模型建立与求解 \dotfill 10
    \begin{itemize}
        \item 5.1 问题1:L-I模型建立与求解 \dotfill 10
        \begin{itemize}
            \item 5.1.1 数据预处理 \dotfill 10
            \item 5.1.2 参数优化模型建立与计算 \dotfill 10
            \item 5.1.3 模型准确性检验——与粒子群算法对比 \dotfill 16
            \item 5.1.4 不同温度环境下的L-I曲线 \dotfill 18
            \item 5.1.5 最高工作环境温度求解 \dotfill 18
        \end{itemize}
        \item 5.2 问题2:L-I模型的改进 \dotfill 19
        \begin{itemize}
            \item 5.2.1 问题1中L-I温度模型误差分析 \dotfill 19
            \item 5.2.2 考虑温度因素L-I优化模型建立 \dotfill 19
            \item 5.2.3 考虑温度因素的参数优化模型求解 \dotfill 20
            \item 5.2.4 绘制改进后模型不同温度下L-I曲线 \dotfill 21
        \end{itemize}
        \item 5.3 问题3:VCSEL的带宽模型(小信号响应模型)建立与求解 \dotfill 22
        \begin{itemize}
            \item 5.3.1 小信号幅频响应参数模型建立 \dotfill 22
            \item 5.3.2 偏置电流、温度对带宽影响 \dotfill 30
            \item 5.3.3 激光器设计优化 \dotfill 32
            \item 5.3.4 参数对带宽曲线的影响性分析 \dotfill 32
        \end{itemize}
        \item 5.4 VCSEL带宽模型的改进 \dotfill 33
        \begin{itemize}
            \item 5.4.1 改进带宽模型建模 \dotfill 33
            \item 5.4.2 VCSEL宽带模型验证 \dotfill 38
        \end{itemize}
    \end{itemize}
    \item 六、模型评价与改进 \dotfill 41
    \begin{itemize}
        \item 6.1 模型的评价 \dotfill 41
        \item 6.2 模型的改进 \dotfill 41
    \end{itemize}
    \item 七、参考文献 \dotfill 42
    \item 八、附录 \dotfill 43
\end{itemize}

\section*{一、问题重述}

\subsection{1.1 背景介绍}

随着互联网技术的快速发展,家庭固定网络速度从原来的 \(2 \mathrm{Mbps} 、 10 \mathrm{Mbps}\), 快速发展到了今天的百兆 \((100 \mathrm{Mbps})\) ,甚至千兆 \((1000 \mathrm{Mbps})\) 光纤宽带入户。“光纤宽带入户”,顾名思义,就是采用光纤来传输信号。光纤中传输的激光信号具有远高于电信号传输速率的特点(激光信号传输带宽远大于电信号传输带宽),更适合于未来高速率的传输网络。工程师们在光纤通信传输系统设计前,往往会通过计算机仿真的方式研究系统设计的指标,以便快速找到最适合的解决方案。因此在进行系统仿真时,需要准确掌握系统中各个器件的特性以保证仿真模型的精度。激光器作为光纤通信系统的核心器件是系统仿真中需要考虑的一个重要因素。

与我们生活息息相关的激光器种类繁多,其中的垂直腔面发射激光器(VCSEL: Vertical Cavity Surface Emitting Laser)具有使用简单,功耗较低等特点,一般 VCSEL 的工作电流在 \(6 \mathrm{~mA} \sim 8 \mathrm{~mA}\)。本题的主要任务,就是得到能准确反映 VCSEL 激光器特性的数学模型。

激光器输出的光功率强度与器件的温度相关,当器件温度(受激光器自身发热和环境温度的共同影响)改变后,激光器输出的光功率强度也会相应发生变化。在进行建模时,我们既要准确反映 VCSEL 激光器特性,还要考虑:

1. 激光器输出的功率强度与温度的关系——即该激光器可以在多大的外界环境温度范围内使用;
2. 如何设计激光器参数可以使激光器具有更大的传输带宽(即 S21 曲线上纵坐标 \(-10 \mathrm{~dB}\) 位置对应的横坐标频率值更大)——即可以实现更快的传输速率。

\subsection{1.2 问题引入}

\subsubsection{1) 问题 1:VCSEL 的 L-I 模型}

国际上很多研究机构对 VCSEL 的 L-I 建模问题做了大量研究,目前有一个 L-I 经验公式获得了大多数人的认可。附录 1 给出了该公式及其一种参数化表达,请你们根据附件提供的文件名为“L-I-20C.mat”的 L-I 实测数据(数据在室温 \(20^{\circ} \mathrm{C}\) 下采集,载入 matlab 后将获得 4 个变量:P: 光功率,I: 实测驱动电流,U: 实测电压,Ta: 实测温度)和附录 1 中的表 1 给出的一组经验值,完成如下工作:

a) 确定模型参数 \(\left(\eta, I_{th0}, R_{th}, a_{0}, a_{1}, a_{2}, a_{3}, a_{4}\right)\),根据模型画出 \(10^{\circ} \mathrm{C}, 20^{\circ} \mathrm{C}, 30^{\circ} \mathrm{C}, \ldots, 90^{\circ} \mathrm{C}\) 等温度下的 L-I 曲线(横坐标是电流强度,纵坐标是光功率)。

b) 假定当电信机房里 VCSEL 激光器在直流输入时输出的平均光功率低于 \(2 \mathrm{~mW}\) 时,用户的光猫无法检测到信号。那么,根据建立的 L-I 模型推测:电信机房里 VCSEL 激光器工作的环境温度最多不能高于多少摄氏度,才能保证用户可以正常使用网络?

\subsubsection{2) 问题 2:L-I 模型的改进}

分析问题 1 中模型的精度和误差产生的原因,提出你们的改进,根据改进模型画出 \(10^{\circ} \mathrm{C}, 20^{\circ} \mathrm{C}, 30^{\circ} \mathrm{C}, \ldots, 90^{\circ} \mathrm{C}\) 等温度下的 L-I 曲线,并与问题 1 中 L-I

曲线进行比较。你们也可以采用完全不同的方法得到更好的 L-I 模型。

3) 问题 3:VCSEL 的带宽模型(小信号响应模型)

VCSEL 的传输带宽越宽,用户上网的网速也就越快。为了适应未来的高速传输网络需求,期望设计出具有更宽带宽的激光器。带宽模型,通常也称为小信号响应模型,就是给器件输入不同频率的幅度非常小的信号(注意与 L-I 模型不同,这里幅度小,而且含多个频率,对应 S21 曲线横坐标是频率,纵坐标是光功率幅度),获得对应频率点输出信号的幅度(功率);这里“不同频率”是从 0 频到指定频率按一定步长进行扫描,例如从 0 开始以 0.5GHz 为步长,扫描到 30GHz。在无线通信中,器件带宽通常采用幅度 3dB 衰减位置的信号带宽来衡量——即 3dB 带宽。在光纤通信中,由于系统可用幅度衰减范围更大,通常采用幅度 10dB 衰减位置的信号带宽进行衡量——10dB 带宽。附件中名为“S21_5.mat”的文件给出了一组 VCSEL 激光器小信号幅频响应曲线数据和相应的驱动电流、输出光功率数据,将数据载入 MATLAB 将获得 4 个变量:I_b=7.5,表示偏置电流为 7.5mA;Ta=20 表示测试的环境温度是 20°C;f 表示 S21 曲线的频率,单位为 Hz;S21 表示 S21 曲线的幅度,单位为 dB,请你们:

a) 建立恰当的激光器小信号幅频响应参数模型,给出参数构成及其确定方法,画出不同环境温度和不同偏置电流下的带宽响应曲线(即 S21 曲线),其中一条必须是 20°C 下 7.5mA 偏置电流的仿真输出曲线。

b) 利用你们的带宽模型,分析激光器的温度和激光器的偏置电流对器件带宽曲线的影响。

c) 假定激光器工作环境温度可以采用某些措施让它固定在 20°C(例如,通信设备商通常的做法是将激光器放到精密的恒温箱中进行测试。若将激光器放在室内环境测试,随着测试时间的变长,激光器本身会发热导致激光器表面及其周边环境温度升高,而恒温箱则可以保证其表面和周边环境温度恒定不变),偏置电流固定在 7.5mA,那么,如果要获得更宽带宽的激光器设计方案,还可以通过什么手段实现?

改变激光器的某些参数可以发现一些有趣的问题,比如激光器在 3dB 范围内的部分频率处幅度可能会高于 0 频位置。请问,改变哪些参数会有这种现象?在实际应用中,我们希望这部分带宽曲线越平坦越好,那么如何设计这些参数可以实现我们的目的?

4) 问题 4:VCSEL 带宽模型的改进

开放性探索问题:是否有更好的带宽模型建模方式,使得模型运算速度更快?或者在相同的温度和偏置电流下,可以获得更宽的 3dB(或 10dB)带宽?如果有,请给出建模方案,包括可能的数学公式,不同温度和偏置电流下的带宽响应曲线,并与问题 3 的模型进行比较。

\section*{二、主要变量说明}

\begin{tabular}{|c|c|c|}
\hline 符号 & 含义说明 & 单位 \\
\hline $\eta$ & L-I曲线的斜率 & - \\
\hline $I_{th0}$ & 阈值电流;激光器电流超过该值则激光发光 & A \\
\hline $R_{th}$ & VCSEL热阻抗 & $^\circ$C/W \\
\hline $\tau_{n}$ & 载流子复合寿命 & S \\
\hline $q$ & 物理常量,电子电量,$1.6 \times 10^{-19}$库伦 & C \\
\hline $I_{off}(T)$ & 与温度相关的偏置电流 & A \\
\hline $\eta_{i}$ & 注入效率;或转换效率; & - \\
\hline $\beta$ & 受激辐射耦合系数 & - \\
\hline $k$ & 比例因子 & W \\
\hline $G_{0}$ & 增益系数,激光产生的阈值条件,增益大于总损耗; & $s^{-1}$ \\
\hline $N_{0}$ & 透明载流子数,当载流子数$N$大于透明载流子数的时候,激光器有源区发生粒子束反转,满足产生激光的其中一个条件 & - \\
\hline $\tau_{p}$ & 光子寿命(p:Photon,光子) & S \\
\hline $\varepsilon$ & 增益压缩因子 & - \\
\hline
\end{tabular}

\section*{三、条件假设}

\begin{enumerate}
    \item 假设没有作为光功率输出的功率,全部转变为器件的散热功率;
    \item 假设转换效率 $\eta(T)$ 受温度影响较小,即 $\eta(T)$ 近似于常数 $\eta$;
    \item 假设小信号引入的 3 个参数都与频率相关:即电流、载流子数、光子数,数学表达可以写成:$i(f)$,$n(f)$,$s(f)$;
    \item 忽略器件内的空间分布特性对带宽的影响;
    \item 假设 VCSFC 激光器工作在理想条件下,即出射端只存在单纵模;
    \item 假设该器件应用于光纤通信,通常采用幅度 10dB 衰减位置的信号带宽进行衡量——10dB 带宽。
\end{enumerate}

\section*{四、问题分析}

\subsection{4.1 问题 1 的分析}

问题一 a 问题要求根据一组 L-I 实测数据(数据在室温 $20^{\circ} \mathrm{C}$ 下采集),建立参数模型以确定一组参数值 $\left(\eta, I_{th0}, R_{th}, a_{0}, a_{1}, a_{2}, a_{3}, a_{4}\right)$,使得由该组参数构成的参数表达式可以最大程度拟合原始数据曲线,并利用该参数表达式依次画出 $10^{\circ} \mathrm{C}$, $20^{\circ} \mathrm{C}$, $30^{\circ} \mathrm{C}$, $\ldots$, $90^{\circ} \mathrm{C}$ 等温度下的 L-I 曲线。

结合题目可知,该问题要求确定一组参数值 $\left(\eta, I_{th0}, R_{th}, a_{0}, a_{1}, a_{2}, a_{3}, a_{4}\right)$。因此,由已知数据可将本问题转化为非线性隐函数参数优化问题,其中目标函数即为所求参数对应曲线与原曲线对应各点间的距离之和最小,即

\[
\min \sum_{i=0}^{n}\left|y_{i}^{\prime}-y_{i}\right|
\tag{4-1}
\]

其中,$sum$:拟合曲线数据点与原始曲线对应点的距离之和;

$y_{i}^{\prime}$:第 $i$ 个拟合曲线点纵坐标值;

$y_{i}$:第 $i$ 个原始曲线点的纵坐标值;

$n$:原始数据点总个数。

同时各参数约束条件必须参考其表示的实际物理意义的范围,采用遗传算法方法可提高搜索效率,同时又可得到相对最优结果。

b 问题要求在上步获得最优参数后,确定 VCSEL 激光器的最高工作环境温度。可以设计温度临界值捕捉程序,即以 $0.1^{\circ} \mathrm{C}$ 为步长,从 $20^{\circ} \mathrm{C}$ 开始逐渐升高环境温度,并计算各环境温度下 VCSEL 激光器光功率峰值,对首个光功率峰值小于等于临界阈值(2mW)的温度进行捕捉,可确定该温度即为 VCSEL 激光器的最高工作环境温度。

\subsection{4.2 问题 2 的分析}

问题二要求在问题一得到的模型参数基础上,继续对模型进行优化,并且重新画出 $10^{\circ} \mathrm{C}$, $20^{\circ} \mathrm{C}$, $30^{\circ} \mathrm{C}$, $\ldots$, $90^{\circ} \mathrm{C}$ 等温度下的 L-I 曲线与问题模型作比较。

题目中附录在模型推导过程中,提出几项假设条件(如转换效率 $\eta(T)$ 受温度影响,将 $\eta(T)$ 近似于常数 $\eta$,虽然使模型推导过程简化,但忽略了温度对参数的影响,降低了模型的准确性。

因此,模型改进的方向就是重新考虑问题 1 的假设条件。参考其他文献,推导出更贴合实际的仿真模型。

\subsection{4.3 问题 3 的分析}

问题三 a 问题要求根据一组 VCSEL 激光器小信号幅频响应曲线数据和相应的驱动电流、输出光功率数据,建立恰当的激光器小信号幅频响应参数模型。

结合题目,首先依据相关理论推导出 VCSEL 的小信号响应模型数学表达式,然后确定待优化参数。拟采用遗传算法计算得出最优参数,并画出不同环境温度和不同偏置电流下的带宽响应曲线。

对于 b 问题,首先从参数表达式方面分析激光器的温度和激光器的偏置电流对器件带宽曲线的理论影响。然后在上述模型基础上绘制激光器的温度和偏置电

流对器件带宽的曲线关系图,直观地分析温度和偏置电流对器件带宽的影响。

c 问题可考虑温度恒定的作用,因此可提出关于温度检测与控制的方法。同时电流过大或者过小都会有消极影响,所以可以考虑控制电流值稳定在一个恒定的位置,使其获得更大的带宽。另外,可相关文献资料经验公式,考虑其物理结构基础上,对其硬件进行优化。

d 问题要求对实验数据中某些特殊现象(如激光器在 $-10 \mathrm{~dB}$ 范围内的部分频率处幅度可能会高于 0 频位置)寻找影响参数,并说明原因。因此,可用正交试验(有实际意义范围内),控制单独变量,寻找重要影响因素参数。参考模型表达式和其它文献,分析影响规律和改进方法。

\section*{4.4 问题 4 的分析}

以上模型未能考虑到 VCSEL 器件的结构特性,同时,该模型只是从热学和电学的角度来分析,未涉及光学。接下来,基于 VCSEL 器件的结构特性,从光学、电学以及热学三个方面重新建立 VCSEL 器件稳小信号特性的速率方程模型。

\section*{五、模型建立与求解}

\subsection{5.1 问题 1: L-I 模型建立与求解}

\subsubsection{5.1.1 数据预处理}

VCSEL 激光器是将电能转换成光能的半导体器件,在能量转换的过程中,电子的电能同时转换为光子的光能和热能的过程。观察实测数据(室温 $20^{\circ} \mathrm{C}$ 下)L-I 曲线可知,曲线初段在驱动电流 I 小于 $0.38 \mathrm{~mA}$ 时,实测光功率 P 的值始终为零。由此,初段 L-I 曲线部分($0<\mathrm{I}<0.38 \mathrm{~mA}$)与后段曲线连接处明显不连续,若直接采用全部数据进行拟合,势必造成初段曲线拟合后误差偏大。

VCSEL 为阈值器件 [1],具有其典型阈值电流 $I_{th}$ 和开启电压(其中开启电压一般为 $1.4 \mathrm{~V}$),当 VCSEL 激光器在驱动电流小于阈值电流或者电压小于开启电压时,不能转换为光子的光能,仅产生热量耗散。在阈值电流以上,VCSEL 的输出光功率与驱动电流有近似线性的关系。

在建立参数模型时,可将阈值电流($I_{th}=38 \mathrm{~mA}$)作为原始数据曲线的分割点,进行分段进行拟合。即:

1) 驱动电流小于阈值电流($I_{th}=38 \mathrm{~mA}$)时,光功率 P(mW) 为 0;
2) 驱动电流大于阈值电流($I_{th}=38 \mathrm{~mA}$)时,采用实际光功率 P(mW) 的值建立下一步的参数模型。

由此,后文中曲线拟合时所采用的原始数据点仅提取当驱动电流大于阈值电流时的数据;若小于驱动电流时,光功率 P 默认为 $0 \mathrm{~mW}$。

说明:由于开启电压首先被满足,驱动电流后满足阈值,因此阈值电流成为光功率开启决定条件。

\subsubsection{5.1.2 参数优化模型建立与计算}

\subsubsection*{1) 光功率 P 参数模型推导}

据相关文献 [2][3],可推导 VCSEL 激光器光功率 P 与驱动电流参数关系表达式,如下:

\begin{equation}
P_{0}=\eta(T)\left(I-I_{th}(N, T)\right)
\tag{5-1}
\end{equation}

其中:

- $P_{0}$:激光器输出的光功率(mW)
- $I$:注入到激光器的外部驱动电流(mA),包含外部加载的偏置电流 $I_{b}$ 和信号电流,在无信号时为偏置电流 $I_{b}$
- $\eta(T)$:L-I 曲线的斜率,斜率越高,相同电流 I 对应的输出光功率越高,相同电能转换为的光能越多,即转换效率越高;
- $I_{th}(N, T)$:阈值电流;激光器电流超过该值则激光发光;与载流子数 N 和温度 T 相关;

假设:

1. 转换效率 $\eta(T)$ 受温度影响较小,即 $\eta(T)$ 近似于常数 $\eta$;
2. $I_{th}(N, T)=I_{th 0}+I_{off}(T)$

其中 $I_{th0}$ 为常数,$I_{off}(T)$ 是与温度相关的经验热偏置电流(即激光器内部的偏置电流,随激光器温度的变化而变化,有别于外部人为加载的激光器偏置电流 $Ib$)。

因此(5-1)式可以简化为
\begin{equation}
P_{0}=\eta\left(I-I_{th0}-I_{off}(T)\right)
\tag{5-2}
\end{equation}
其中 $I_{off}(T)$ 又可表示为:
\begin{equation}
I_{off}(T)=\sum_{n=0}^{\infty} a_{n} T^{n}
\tag{5-3}
\end{equation}
上式中的温度 $T$ 受外界环境温度 $T_{0}$ 和自身的温度影响,自身的温度与器件产生的瞬时功率 $VI$ 相关,即受 V-I 特性(电压-电流特性)影响:
\begin{equation}
T=T_{0}+\left(IV-P_{0}\right) R_{th}-\tau_{th} \frac{dT}{dt}
\tag{5-4}
\end{equation}
其中,$R_{th}$:VCSEL 热阻抗($^{\circ}\mathrm{C} / \mathrm{W}$)  
$\tau_{th}$:热时间常数  
$T_{0}$:环境温度($^{\circ}\mathrm{C}$)  
$I$:偏置电流 $Ib$(输入电流 $\mathrm{mA}$)  
$V$:输入电压($\mathrm{V}$)

在直流情况下,$\mathrm{d}T / \mathrm{d}t$ 项消失,这里假设所有非光输出能量的损耗都转化为热 [4]。综合式(5-1)-(5-4)可推导得出 VCSEL 激光器在假设(1)、(2)条件下的 L-I 模型最终参数表达式,如下
\begin{align}
P_{0} &= \eta\left(I-I_{th0}-\sum_{n=0}^{4} a_{n} T^{n}\right) \\
P_{0} &\approx \eta\left\{I-I_{th0}-\left[a_{0}+a_{1}\left(T_{0}+\left(IV-P_{0}\right) \cdot R_{th}\right)\right.\right. \nonumber \\
&\quad\left.\left.+a_{2}\left(T_{0}+\left(IV-P_{0}\right) \cdot R_{th}\right)^{2}\right.\right. \nonumber \\
&\quad\left.\left.+a_{3}\left(T_{0}+\left(IV-P_{0}\right) \cdot R_{th}\right)^{3}\right.\right. \nonumber \\
&\quad\left.\left.+a_{4}\left(T_{0}+\left(IV-P_{0}\right) \cdot R_{th}\right)^{4}\right]\right\}
\tag{5-5}
\end{align}

2) 参数估计模型建立与求解

由式(5-5)可知,参数优化模型的待求参数为 $\left(\eta, I_{th0}, R_{th}, a_{0}, a_{1}, a_{2}, a_{3}, a_{4}\right)$。对于以上 8 项参数的求解方法,若采用全局搜索优化方法会导致求解程序效率极低,短时间内不能获得满意结果。

遗传算法 [5] 是一种概率化搜索方法,模拟自然演化原理的人工适应系统能自动获取和指导优化的搜索空间,自适应地调整搜索方式。

GA 算法的优化流程图:

\begin{figure}[h]
    \centering
    \includegraphics[width=\textwidth]{image.png}
    \caption{遗传算法的优化流程}
    \label{fig:genetic_algorithm_flow}
\end{figure}

传统遗传算法参数中交叉概率和变异概率的选择是影响遗传算法行为和性能的最关键。若交叉概率过大,遗传模式被破坏几率变大,使得具有高度适应度的个体很快被破坏;若交叉概率过小,会使搜索搜索过程缓慢,一直停滞不前。

美国密歇根大学 Holland 等\cite{holland1975adaptation}提出一种自适应遗传算法,交叉概率与变异概率能够随着个体适应度大小而自动改变。当种群个体适应度趋于一致或者局部最优时,增加交叉概率和变异概率;当群体适应度比较分散时,使得交叉概率和变异概率减小。

本文采用自适应遗传算法对待求参数进行迭代优化,自适应算法的交叉概率 $P_c$ 和变异概率 $P_m$ 的计算方法如下:

\begin{equation}
P_c = 
\begin{cases} 
\frac{k_1(f_{\max} - f)}{f_{\max} - f_{\text{avg}}}, & f \geq f_{\text{avg}} \\
k_2, & f < f_{\text{avg}}
\end{cases}, \quad
P_m = 
\begin{cases} 
\frac{k_3(f_{\max} - f')}{f_{\max} - f_{\text{avg}}}, & f' \geq f_{\text{avg}} \\
k_4, & f' < f_{\text{avg}}
\end{cases}
\tag{5-6}
\end{equation}

其中,$f_{\max}$:群体中的最大适应度值;

$f_{\text{avg}}$:群体中的平均适应度值;

$f$:要交叉的两个个体中较大适应度值;

$f'$:要变异个体的适应度值;

$k_1$、$k_2$、$k_3$ 和 $k_4$ 为常数。

\textbf{Step1. 遗传算法目标函数的确定}

结合本题,本文建立参数求解模型,目的是得到一组参数 $(\eta, I_{th0}, R_{th}, a_0, a_1, a_2, a_3, a_4)$ 使得新拟合曲线与原始数据曲线最大程度重合。因此,该模型的目标函数即为拟合曲线与原始曲线对应点的距离之和最小,即:

\begin{equation}
\min sum = \sum_{i=0}^{n} \left| P_{i}^{\prime} - P_{i} \right|
\tag{5-7}
\end{equation}

其中,$sum$:拟合曲线数据点与原始曲线对应点的距离之和;

$P_{i}^{\prime}$:第 $i$ 个拟合曲线点的光功率值(mW);

$P_{i}$:第 $i$ 个原始曲线点的光功率值(mW);

$n$:原始数据点总数,为 $n=1401$。

\textbf{Step2. 设定初始种群及参数范围}

依据题目给出的一组参考初值作为初始种群,用以多次迭代得出最优参数组合。首先,将参考初值代入公式 (5-5) 可以绘制出条拟合 L-I 曲线。

\begin{figure}[h]
    \centering
    \includegraphics[width=\textwidth]{image.png}
    \caption{初始参数拟合 L-I 曲线与原始曲线对比}
    \label{fig:2}
\end{figure}

观察上图可知,初始参数拟合出的 L-I 曲线与原始曲线重合度较差,仍有较大优化空间。取参考值为初代参数,即 $(\eta, I_{th0}, R_{th}, a_{0}, a_{1}, a_{2}, a_{3}, a_{4}) = (0.5, 0.3E-3, 2.6E3, 1.246E-3, -2.545E-5, 2.908E-7, -2.531E-10, 1.022E-12)$,单位与题目中表一相同。根据式 (5-7) 可计算出初代参数拟合出的 L-I 曲线数据点与原始曲线对应点的距离之和为

\begin{align*}
sum_{1} &= \sum_{i=0}^{1401} \left| P_{i}^{\prime} - P_{i} \right| = 1904 \\
\Delta_{1} &= \frac{\sum_{i=0}^{n} \left| P_{i}^{\prime} - P_{i} \right|}{n} / P_{avg} \\
&= \frac{1904}{1401} / 1.59 = 85.5\%
\end{align*}

平均相对误差为

其中,$P_{i}^{\prime}$:拟合曲线中第 $i$ 个数据点光功率(mW)

\begin{table}[h]
\centering
\caption{自适应遗传算法求解最优参数值}
\begin{tabular}{|c|c|c|c|c|c|}
\hline
参数 & 初始种群 & 最优参数 & 搜索上限 & 搜索下限 & 单位 \\
\hline
$\eta$ & 0.5 & 0.286 & 0.8 & 0.2 & - \\
$I_{th0}$ & 0.3E-3 & 1.074E-4 & 0.5E-3 & 0.1E-5 & A \\
$R_{th}$ & 2.6E3 & 3047 & 3600 & 1600 & $^\circ$C/W \\
$a_{0}$ & 1.246E-3 & 5.028E-4 & 8.246E-2 & 1.246E-4 & A \\
$a_{1}$ & -2.545E-5 & -2.058E-5 & -0.5E-5 & -8.6E-5 & A/K \\
$a_{2}$ & 2.908E-7 & 3.400E-7 & 4.9E-7 & 1.0E-7 & A/K$^2$ \\
$a_{3}$ & -2.531E-10 & -2.139E-10 & -1.5E-10 & -4.5E-10 & A/K$^3$ \\
$a_{4}$ & 1.022E-12 & 4.945E-12 & 5E-12 & 9E-13 & A/K$^4$ \\
\hline
\end{tabular}
\end{table}

根据表1所得的一组最优参数$(\eta, I_{th0}, R_{th}, a_{0}, a_{1}, a_{2}, a_{3}, a_{4})$重新绘制拟合曲线,并与原始曲线对比如下。

\begin{figure}[h]
    \centering
    \includegraphics[width=\textwidth]{image.png}
    \caption{最优参数L-I拟合曲线与原始曲线对比}
    \label{fig:3}
\end{figure}

由计算结果可知,拟合L-I曲线与原始曲线对应点距离和以及平均相对误差为:

\begin{equation}
sum_{500} = \sum_{i=0}^{1401} |P_i' - P_i| = 12.971
\end{equation}

\begin{equation}
\Delta_{500} = \frac{\sum_{i=0}^{n} |P_i' - P_i|}{n} / P_{avg}
\end{equation}

\begin{equation}
= \frac{12.971}{1401} / 1.59 = 0.582\%
\end{equation}

平均相对误差为

结论:由上述计算结果可知,两条曲线的重合度非常高,平均相对误差为0.582\%。在理想范围内,由此确定的一组最优参数 $(\eta, I_{th0}, R_{th}, a_0, a_1, a_2, a_3, a_4) = (0.5, 0.3E-3, 2.6E3, 1.246E-3, -2.545E-5, 2.908E-7, -2.531E-10, 1.022E-12)$。

该遗传算法模型总共设置500次迭代,每代精英数为100。下表为相应迭代次数所得拟合曲线与原曲线的距离和以及平均相对误差见表2。

\begin{table}[h]
    \centering
    \caption{遗传迭代过程中误差变化情况}
    \label{tab:2}
    \begin{tabular}{c c c}
        \hline
        迭代次数 & 曲线偏差距离和 $sum$ & 平均相对误差\% \\
        \hline
        1 & 1904 & 85.5\% \\
        20 & 273 & 12.3\% \\
        50 & 49.3 & 2.20\% \\
        100 & 27.7 & 1.24\% \\
        300 & 16.2 & 0.72\% \\
        500 & 12.9 & 0.58\% \\
        \hline
    \end{tabular}
\end{table}

\subsection{5.1.3 模型准确性检验—与粒子群算法对比}

为了验证上述自适应遗传算法参数估计模型的准确性,本文拟采用另一种典型非线性系统参数估计方法——粒子群优化算法求解。并将两种方法的拟合结果与原始数据对比。

粒子群优化算法 \cite{PSO} (Particle Swarm Optimazition, 简称 PSO) 是由 Kennedy 博士和 Eberhart 博士于 1995 年提出的一种基于群体智能的优化算法,它源于对鸟群群体运动行为的研究,即粒子群优化算法模拟鸟群的捕食行为。

\subsubsection{1) 确定非线性模型优化参数}

结合本题,非线性系统模型可用式 (1) 表示。
\begin{equation}
P(I) = f(I, V, T_0, \theta_i)
\tag{5-8}
\end{equation}
式中,
- $P(I)$:系统输出向量;
- $I, V, T_0$:系统输入向量,具体含义与上文相同;
- $\theta_i$:一组待优化参数,即 $\theta_i = (\eta, I_{th0}, R_{th}, a_0, a_1, a_2, a_3, a_4)^T$

\subsubsection{2) 粒子群算法计算过程}

\textbf{Step. 1 确定适应函数}

该适应函数与遗传算法中的适应函数类似,取 $P$ 曲线与 $P'$ 曲线之间距离为适应值,因此适应函数与式 (5-7) 相同
\begin{equation}
f = \sum_{i=0}^{n} |P_i' - P_i|
\tag{5-9}
\end{equation}

\textbf{Step. 2 随机产生 n 组参数 $\theta_i$。}

\textbf{Step. 3 计算适应值 $f$,根据式 (5-9) 中确定的适应函数计算出各组 $\theta_i$ 对应的适应值 $f$;}

\textbf{Step. 4 对于每组参数粒子,将其适应值与所经历过的最优位置 $\vec{l}_i = (l_{i_1}, l_{i_2}, \dots, l_{i_n})$ 的适应值进行比较,若较好,则将其作为当前的最优位置;}

\textbf{Step. 5 对于每个粒子,将其适应值与全局所经历的最优位置 $\vec{l}_g = (l_{g_1}, l_{g_2}, \dots, l_{g_n})$ 的适应值进行比较,若较好,则将其作为当前的全局最优位置;}

\textbf{Step. 7 根据公式对粒子的速度和位置进行更新;}

\textbf{Step. 8 如未达到结束条件(通常为足够好的适应值)或达到一个预设最大代数 $G_{\text{max}}$,则返回步骤 2 直至算法收敛,即所有个体基本相同,适应值很难进一步提高为止。}

\subsubsection{3) 模型求解计算}

基于 matlab 软件,编写粒子群算法求解程序(见附录 1.1)。参数设置如下:

学习因子 $c_1 = 2$,$c_2 = 2$,惯性权重 $w = 1.4 - t / T$,$T$ 为最大代数,$t$ 为当前进化代数,在这里 $w$ 将随着迭代次数的增加而逐渐减小,当 $w$ 小于 0.4 时,将令 $w = 0.4$,即不再减小,以保证迭代后期粒子能够在一定空间探索更好的解。根据参考初值设置初始位置,它们的群体规模是 50,其他参数不变。在搜索过程中,以 1000 代为上限(实际上,迭代 50-80 次即可得到满意结果)。求得参数优化结果

\[
\theta_{i}=\left(\eta, I_{th0}, R_{th}, a_{0}, a_{1}, a_{2}, a_{3}, a_{4}\right)^{T}
\]
如表 3 所示。

\begin{table}[h]
\centering
\caption{粒子群算法最优参数值}
\begin{tabular}{c c c c c c c c}
\hline
\hline
$\eta$ & $I_{th0}$ & $R_{th}$ & $a_{0}$ & $a_{1}$ & $a_{2}$ & $a_{3}$ & $a_{4}$ \\
\hline
参数 & 0.294 & 1.94 & 3.42 & 8.64 & -2.11 & 3.16 & -3.28 & 3.38 \\
估计值 & & E-4 & E3 & E-4 & E-5 & E-7 & E-10 & E-12 \\
\hline
单位 & - & A & $^{\circ}\mathrm{C}/\mathrm{W}$ & A & A/K & A/K$^{2}$ & A/K$^{3}$ & A/K$^{4}$ \\
\hline
\hline
\end{tabular}
\end{table}

\section*{4) 粒子群算法结果分析}

按照表 3 的一组优化参数重新绘制 VCSEL 激光器在室温 $25^{\circ}\mathrm{C}$ 时 L-I 曲线并与原始曲线对比,如下图所示。

\begin{figure}[h]
\centering
\includegraphics[width=\textwidth]{image.png}
\caption{粒子群算法优化参数拟合曲线与原始曲线对比}
\end{figure}

由此,可求得 VCSEL 激光器 L-I 曲线数据点与原始曲线对应点的距离之和为
\begin{equation}
\begin{aligned}
\text{sum} &= \sum_{i=0}^{1401} \left|P_{i}^{\prime} - P_{i}\right| = 93.2 \\
\Delta &= \frac{\sum_{i=0}^{n} \left|P_{i}^{\prime} - P_{i}\right|}{n} / P_{avg} \\
&= \frac{93.2}{1401} / 1.59 = 4.18\%
\end{aligned}
\end{equation}

平均相对误差为

结论:采用粒子群算法获得的拟合曲线与原始曲线对应点的距离之和以及平均相对误差大于自适应遗传算法的优化结果。从图 4 的对比结果也可看出,采用粒子群算法的曲线拟合度不如遗传算法。

分析原因,PSO 算法本质上是一种随机算法,可能陷入局部最优陷阱,而且

对于多参数优化,粒子群算法的效果并不理想。

对比结果表明:自适应遗传算法的参数优化效果更好,因此后文在绘制不同温度下 L-I 曲线时,皆依据自适应遗传算法的优化参数。

\subsection*{5.1.4 不同温度环境下的 L-I 曲线}

根据上步自适应遗传算法计算得出最优参数 \((\eta, I_{th0}, R_{th}, a_0, a_1, a_2, a_3, a_4)\) 代入到式 (5-5) 中,重写如下

\begin{equation}
P_0 = \eta \left( I - I_{th0} - \sum_{n=0}^{4} a_n T_i^n \right)
\tag{5-10}
\end{equation}

\(T_i\) 表示代入的不同温度,分别为 \(10^\circ \mathrm{C}, 20^\circ \mathrm{C}, 30^\circ \mathrm{C}, \ldots, 90^\circ \mathrm{C}\),绘制不同温度下的 L-I 曲线,如下图所示

\begin{figure}[h]
\centering
\includegraphics[width=\textwidth]{image.png}
\caption{不同温度环境下 VCSEL 激光器 L-I 曲线}
\end{figure}

结论:由上图可知,随着室温逐渐增加,VCSEL 激光器在相同驱动电流下的光功率逐渐降低,即电能转化为光能的效率降低。所以,保持 VCSEL 激光器工作在适宜的温度范围内,有利于提高其光功率转化效率。

\subsection*{5.1.5 最高工作环境温度求解}

根据题目假设,当电信机房里 VCSEL 激光器在直流输入时输出的平均光功率低于 \(2 \mathrm{~mW}\) 时,用户的光猫无法检测到信号。因此以 \(2 \mathrm{~mW}\) 为临界阈值,求解该激光器的最高工作温度。

本文设计了温度临界值捕捉程序,即以 \(0.1^\circ \mathrm{C}\) 为步长,从 \(20^\circ \mathrm{C}\) 开始逐渐升高环境温度,并计算各环境温度下 VCSEL 激光器光功率峰值,对首个光功率峰值小于等于临界阈值(\(2 \mathrm{~mW}\))的温度进行捕捉,可确定该温度即为 VCSEL 激光器的最高工作环境温度,即当温度高于此临界值时,该激光器无法正常工作。(求解程序代码见附录 1.6)

结论:由上述方法得到该 VCSEL 激光器的最高工作环境温度为 \(42.7^\circ \mathrm{C}\),绘制 L-I

曲线如图 5 所示。

\subsection{5.2 问题 2:L-I 模型的改进}

\subsubsection{5.2.1 问题 1 中 L-I 温度模型误差分析}

VCSEL 激光器是电流注入器件,大的阻抗必然产生可观的热量,这使有源区内的温度随注入电流的增大不断升高,因此 VCSEL 的工作特性随注入电流和温度的变化非常明显,建立一个精确简单的考虑温度的模型就显得十分重要。

原始数据仅为室温为 $20^\circ \mathrm{C}$ 下的 P-I-V 的关系,而 L-I 模型推导时也忽略某些参数受温度的影响的情况。题目附录 1 在推导 L-I 模型过程中,为了简化推导过程设置几项假设条件,忽略了温度对某些参数的影响,如下:

1. 转换效率 $\eta(T)$ 受温度影响,将 $\eta(T)$ 近似于常数 $\eta$;
2. 电压 $V(T)$ 受温度影响,原模型未考虑。

问题 1 所建模型仅能较好地反应室温为 $20^\circ \mathrm{C}$ 下的 P-I-V 相互关系,推广到 $10^\circ \mathrm{C}$,$20^\circ \mathrm{C}$,$30^\circ \mathrm{C}$,$\cdots$,$90^\circ \mathrm{C}$ 等温度下,VCSEL 激光器的工作特性可能会发生变化。所以模型的改进方向就是,考虑温度对参数的影响因素,更好地反映其他温度下 VCSEL 激光器的工作特性。

\subsubsection{5.2.2 考虑温度因素 L-I 优化模型建立}

根据 Tucker 经典模型 \cite{tucker1975}, 当注入电流 $I$ 增大时,VCSEL 内部的温度会升高,载流子的能量将增大,于是会有更多的载流子脱离量子阱的束缚,扩散到有源层之外,于是出现输出光功率滚降的现象。可以在 Tucker 模型中引入温度控制的泄漏电流 $I_{\text{leak}}(T)$ 来表征激光器输出特性的温度效应 \cite{tucker1975}:

\begin{equation}
I_{\text{leak}}(T) = \sum_{i=1}^{n} a_i T^{i-1}
\tag{5-11}
\end{equation}

其中,$a_i$ 为优化因子。

半导体激光器中以热的形式耗散的能量可表示为:

\begin{equation}
\Delta \text{power} = VI - P_L
\tag{5-12}
\end{equation}

其中,$V$ 和 $I$ 分别为输入电压和输入电流;$P_L$ 为输出光功率,则芯片内的温度可表示为

\begin{equation}
T = T_0 + \Delta T = T_0 + R_{th} \Delta \text{power} - \tau_{th} \frac{dT}{dt}
\tag{5-13}
\end{equation}

其中,$T_0$ 为环境温度;$R_{th}$ 为热阻;$\tau_{th}$ 为温度时间常数(量级为 $\mathrm{S}$,用来描述器件温度的非零响应时间)。

VCSEL 的 V-I 特性可以借鉴二极管 V-I 特性的模型来表征:

\begin{equation}
V = n V_T \ln \left[ \frac{I}{I_{ss}} + 1 \right] + I R_s
\tag{5-14}
\end{equation}

其中,n 为理想因子;$V_{T}=kT/q$ 为热电压;$I_{ss}$ 为反向饱和电流;k 为玻尔兹曼常数;q 为电荷电量。

当注入电流增大时,VCSEL 内部的温度会升高,$V_{T}$ 便会增大,寄生电阻 $R_{s}$ 也会发生变化。为了表征 VCSEL 的输入特性的温度效应,应使 $R_{s}$ 和 $I_{ss}$ 受温度控制。为简单起见,令 $V_{T}$ 为常数 (26mV) 不随温度变化,令寄生电阻和反向饱和电流随温度变化,为 $R_{s}(T)$ 和 $I_{ss}(T)$。$R_{s}(T)$ 和 $I_{ss}(T)$ 可以写成如下形式:

\begin{equation}
R_{s}(T)=\frac{c_{1}}{T-c_{2}}
\tag{5-15}
\end{equation}

\begin{equation}
I_{ss}(T)=b_{1}I_{ss0}(T-b_{2})
\tag{5-16}
\end{equation}

其中,$c_{1}$,$c_{2}$,$b_{1}$,$b_{2}$ 为优化因子;$I_{ss0}$ 为某一特定温度下的反向饱和电流,本文取环境温度为 $20^{\circ}\mathrm{C}$ 的值。

在确定描述 VCSEL 输入特性的参数 $c_{1}$,$c_{2}$,$b_{1}$,$b_{2}$ 时,我们不能直接用 (5-13) 式,因为该式表示流过寄生电阻 $R_{s}$ 和注入到二极管的电流是相同的,而在 Tucker 模型中注入到二极管中的电流 $I_{sp0n}$ 只是流过 $R_{s}$ 中的电流的一部分,而且 $I_{sp0n}$ 在阈值以上是保持不变的,大小等于阈值电流 [10]。

当 $I \geq I_{th}$ 时,流过 $R_{s}$ 的电流中的一部分注入二极管中,$I_{sp0n}=I_{th}$,此时用公式 (5-17) 来拟合实验测得的不同温度下的 V-I 曲线。

\begin{equation}
\begin{aligned}
V & =nV_{T} \ln \left[\frac{I_{th}}{I_{ss}}+1\right]+IR_{s} \\
& =nV_{T} \ln \left[\frac{I_{th}}{b_{1}I_{ss0}(T-b_{2})}+1\right]+I \frac{c_{1}}{T-c_{2}}
\end{aligned}
\tag{5-17}
\end{equation}

重写公式 (5-5) 为,

\begin{equation}
\begin{aligned}
P_{0} & =\eta\left(I-I_{th0}-\sum_{n=0}^{4} a_{n} T^{n}\right) \\
P_{0} & \approx \eta\left\{I-I_{th0}-\left[a_{0}+a_{1}\left(T_{0}+\left(IV-P_{0}\right) \cdot R_{th}\right)\right.\right. \\
& \left.\left.\quad+a_{2}\left(T_{0}+\left(IV-P_{0}\right) \cdot R_{th}\right)^{2}\right.\right. \\
& \left.\left.\quad+a_{3}\left(T_{0}+\left(IV-P_{0}\right) \cdot R_{th}\right)^{3}\right.\right. \\
& \left.\left.\quad+a_{4}\left(T_{0}+\left(IV-P_{0}\right) \cdot R_{th}\right)^{4}\right]\right\}
\end{aligned}
\tag{5-18}
\end{equation}

其中,$V=nV_{T} \ln \left[\frac{I_{th}}{b_{1}I_{ss0}(T-b_{2})}+1\right]+I \frac{c_{1}}{T-c_{2}}$。

所以待优化参数由问题 1 中的 8 项 $\left(\eta, I_{th0}, R_{th}, a_{0}, a_{1}, a_{2}, a_{3}, a_{4}\right)$ 增加为 14 项,即 $\left(\eta, I_{th}, R_{th}, a_{0}, a_{1}, a_{2}, a_{3}, a_{4}, n, I_{ss0}, b_{1}, b_{2}, c_{1}, c_{2}\right)$。

5.2.3 考虑温度因素的参数优化模型求解

Step. 1 确定目标函数

根据原始P-I-V数据,再次采用自适应遗传算法求解。目标函数仍为式 (5-7),表示拟合曲线对应点与原始曲线的距离和最小,即

\[
\min sum = \sum_{i=0}^{n} \left| P_i' - P_i \right|
\tag{5-19}
\]

其中,上式中各2变量含义与式 (5-7) 含义相同,不再赘述。

Step. 2 设定初始种群及参数范围

参数的初值参考文献[11](见表4),另外为保证遗传算法的运算效率,在考虑各参数实际意义的前提下,扩大或缩小若干数量级作为其搜索范围的上下限。需要注意的是参数搜索范围并非一次确定,需根据迭代后所求最优参数与上下限值比较。若触碰上下限值,则将该参数范围扩大,重新进行遗传迭代计算。

Step. 3 模型迭代求解

依据上述模型和方法,基于Matlab编写自适应遗传算法程序模型(代码见附录2.1),总迭代次数设为500代。最终预算结果如下表所示,

表4 改进模型参数求解结果

\begin{table}[h]
\centering
\begin{tabular}{|c|c|c|c|c|c|}
\hline
参数 & 参考初值 & 最优参数 & 搜索上限 & 搜索下限 & 单位 \\
\hline
$\eta$ & 0.5 & 0.442 & 0.8 & 0.2 & - \\
$I_{th0}$ & 0.3E-3 & 3.497E-4 & 0.5E-3 & 0.1E-5 & A \\
$R_{th}$ & 8.06E3 & 3.165E3 & 26000 & 260 & $^\circ$C/W \\
$a_0$ & -7.92E-2 & 3.389E-4 & 8.246E-3 & 1.246E-4 & A \\
$a_1$ & -2.36E-5 & -2.018E-5 & -0.5E-5 & -5.6E-5 & A/K \\
$a_2$ & 2.33E-7 & 1.00E-07 & 4.9E-7 & 1.0E-7 & A/K$^2$ \\
$a_3$ & -2.98E-10 & -2.12E-10 & -1.5E-10 & -3.5E-10 & A/K$^3$ \\
$a_4$ & 4.45E-12 & 3.00E-12 & 5E-12 & 3E-12 & A/K$^4$ \\
$n$ & 2 & 0.02 & 200 & 0.02 & - \\
$I_{ss0}$ & 1.208E-17 & 1.21E-15 & 1.208E-15 & 1.208E-19 & A \\
$b_1$ & 0.111 & 10.97 & 11.1 & 0.00111 & - \\
$b_2$ & -9.191 & -1.86E4 & -0.09191 & -919.1 & K \\
$c_1$ & 4.73E4 & 5.83E2 & 4727776 & 472.7776 & $\Omega$ \\
$c_2$ & -2.33E2 & -3.97E2 & -53.389 & -23389 & K \\
\hline
\end{tabular}
\end{table}

5.2.4 绘制改进后模型不同温度下L-I曲线

根据上步自适应遗传算法计算得出最优参数 $(\eta, I_{th}, R_{th}, a_0, a_1, a_2, a_3, a_4, n, I_{ss0},$

$b_{1}, b_{2}, c_{1}, c_{2}$) 改进后的参数模型由于考虑了温度的影响,代入的不同温度时,如 $10^{\circ} \mathrm{C}, 20^{\circ} \mathrm{C}, 30^{\circ} \mathrm{C}, \cdots, 90^{\circ} \mathrm{C}$,绘制出的不同温度下的 L-I 曲线,如下图所示

\begin{figure}[h]
    \centering
    \includegraphics[width=\textwidth]{image.png}
    \caption{a. 改进前模型 L-I 曲线 \hspace{2cm} b. 改进后模型 L-I 曲线}
    \label{fig:LI_curves}
\end{figure}

图 6 不同温度环境下激光器 L-I 曲线(模型改进前后对比)

由上图对比可知,改进后的模型 VCSEL 激光器 L-I 曲线表明:随室温的升高,光功率下降速度明显增加,说明 VCSEL 激光器受工作环境温度影响较大。

为保证用户可以正常使用网络(平均光功率不低于 $2 \mathrm{~mW}$)的工作室温由原来 $42.7^{\circ} \mathrm{C}$ 变为 $38.2^{\circ} \mathrm{C}$。后者模型由于更多考虑了参数受温度影响情况,因此描述不同温度下激光器工作特性真实度更高。

\subsection{5.3 问题 3:VCSEL 的带宽模型(小信号响应模型)建立与求解}

\subsubsection{5.3.1 小信号幅频响应参数模型建立}

1) 小信号幅频响应参数模型推导

参考题目中附录 2 的推导过程,简化部分中间参数公式推导,得到小信号幅频响应模型如下。

首先,通信系统里面的信号响通常有如下形式,写成二阶系统的典型传递函数为

\begin{equation}
h(f) = \frac{X}{(j2\pi f)^2 + (j2\pi f)Y + Z}
\tag{5-20}
\end{equation}

其中,

\begin{equation}
Y = \frac{1}{\tau_p} + \frac{1}{\tau_n} + \frac{G_0 S_s}{1 + \varepsilon S_s} - \frac{G_0 (N_s - N_0)}{1 + \varepsilon S_s}
\tag{5-21}
\end{equation}

\begin{equation}
\begin{aligned}
Z &= \left[\frac{1}{\tau_p} - \frac{G_0 (N_s - N_0)}{1 + \varepsilon S_s}\right] \cdot \left[\frac{1}{\tau_n} + \frac{G_0 S_s}{1 + \varepsilon S_s}\right] + \left[\frac{\beta}{\tau_n} + \frac{G_0 S_s}{1 + \varepsilon S_s}\right] \cdot \frac{G_0 (N_s - N_0)}{1 + \varepsilon S_s} \\
&= \frac{1}{\tau_p \tau_n} + \frac{G_0 S_s}{\tau_p (1 + \varepsilon S_s)} - \frac{(1 - \beta) G_0 (N_s - N_0)}{\tau_n (1 + \varepsilon S_s)}
\end{aligned}
\tag{5-22}
\end{equation}

其中,$\tau_n$:载流子复合寿命

\begin{itemize}
    \item $\tau_p$:光子寿命(p:Photon,光子)
    \item $G_0$:增益系数,激光产生的阈值条件,增益大于总损耗;
    \item $\varepsilon$:增益压缩因子
    \item $S_s$:光子数
    \item $N_0$:透明载流子数,当载流子数 $N$ 大于透明载流子数的时候,激光器有源区发生粒子束反转,满足产生激光的其中一个条件
    \item $N_s$:载流子数
\end{itemize}

上述式(5-21)、(5-22)中除参数外,未知变量有 $N_s$、$S_s$,其推导过程如下:

其中 $N_s$ 的推导过程:

当外部驱动电流保持恒定,载流子数 $N$ 与光子数 $S$ 将在一段时间的瞬态后达到稳定状态,此时 $N$ 和 $S$ 对时间求导为 0(稳态下,载流子和光子数变化为 0),因此

\begin{equation}
\frac{\eta_i \left( I - I_{th0} - I_{off}(T) \right)}{q} - \frac{N}{\tau_n} - \frac{G_0 \left( N - N_0 \right) S}{1 + \varepsilon S} = \frac{dN}{dt} = 0
\tag{5-23}
\end{equation}

\begin{equation}
-\frac{S}{\tau_p} + \frac{\beta N}{\tau_n} + \frac{G_0 \left( N - N_0 \right) S}{1 + \varepsilon S} = \frac{dS}{dt} = 0
\tag{5-24}
\end{equation}

现在研究在已知输出光功率 $P_0$ 情况下激光器中载流子的浓度问题。将 $P_o = kS$ 关系式代入等式(5-24),此时求出的 $N$ 为稳态下载流子浓度,记为 $N_s$,等式(5-24)中的第二个等式得到

\begin{equation}
\left.
\begin{aligned}
-\frac{S}{\tau_p} + \frac{\beta N_s}{\tau_n} + \frac{G_0 \left( N_s - N_0 \right) S}{1 + \varepsilon S} &= 0 \\
P_0 &= kS
\end{aligned}
\right\}
\Rightarrow
N_s = \frac{P_0 / \left( k \tau_p \right) + G_0 N_0 P_0 / \left( k + \varepsilon P_0 \right)}{\beta / \tau_n + G_0 P_0 / \left( k + \varepsilon P_0 \right)}
\tag{5-25}
\end{equation}

其中 $S_s$ 的推导过程:

VCSEL 输出的光功率与光子数成正比,假定比例因子为 $k$

\begin{equation}
P_s = k S_s
\tag{5-26}
\end{equation}

将 $P_s = k S_s$ 在通常的 VCSEL 激光器中,$\varepsilon S_s << 1$,根据等式得到理论计算出的激光器稳态工作时的光子数

\begin{equation}
\frac{\eta_i \left( I - I_{th0} - I_{off}(T) \right)}{q} - \frac{N_s}{\tau_n} = G_0 \left( N_s - N_0 \right) S_s
\tag{5-27}
\end{equation}

\begin{equation}
S_s = \frac{\eta_i \left( I - I_{th0} - I_{off}(T) \right) / q - N_s / \tau_n}{G_0 \left( N_s - N_0 \right)}
\end{equation}

最终公式整理:

将 $P_{s}=kS_{s}$ 代入等式 (5-21) 和等式 (5-22),将稳态下的光子数 $S_{s}$ 用光功率 $P_{s}$ 表示

\begin{equation}
Y=\frac{1}{\tau_{p}}+\frac{1}{\tau_{n}}+\frac{G_{0}P_{s}}{k+\varepsilon P_{s}}-\frac{kG_{0}\left(N_{s}-N_{0}\right)}{k+\varepsilon P_{s}}
\tag{5-28}
\end{equation}

\begin{equation}
Z=\frac{1}{\tau_{p}\tau_{n}}+\frac{G_{0}P_{s}}{\tau_{p}\left(k+\varepsilon P_{s}\right)}-\frac{k\left(1-\beta\right)G_{0}\left(N_{s}-N_{0}\right)}{\tau_{n}\left(k+\varepsilon P_{s}\right)}
\tag{5-29}
\end{equation}

等式 (5-28) 和等式 (5-29) 是忽略了 $\varepsilon s\left(f\right)$ 项得到的,为了进一步提高精度,假定只忽略高阶项,而不忽略 $\varepsilon s\left(f\right)$ 项,则可以得到

\begin{equation}
Y=\frac{1}{\tau_{p}}+\frac{1}{\tau_{n}}+\frac{G_{0}P_{s}}{k+\varepsilon P_{s}}-\frac{G_{0}\left(N_{s}-N_{0}\right)}{\left(1+\varepsilon P_{s}/k\right)^{2}}
\tag{5-30}
\end{equation}

\begin{equation}
Z=\frac{1}{\tau_{p}\tau_{n}}+\frac{G_{0}P_{s}}{\tau_{p}\left(k+\varepsilon P_{s}\right)}-\frac{\left(1-\beta\right)G_{0}\left(N_{s}-N_{0}\right)}{\tau_{n}\left(1+\varepsilon P_{s}/k\right)^{2}}
\tag{5-31}
\end{equation}

其中,$N_{s}$、$P_{s}=kS_{s}$ 代入等式 (5-30) 和 (5-31),响应函数需要进行归一化处理,因此不必关心刚才等式里面的 $X$ 表达式是什么,在归一化处理时 $X$ 被抵消掉了。归一化小信号响应为

\begin{equation}
H\left(f\right)\equiv\frac{h\left(f\right)}{h\left(0\right)}=\frac{Z}{\left(j2\pi f\right)^{2}+\left(j2\pi f\right)Y+Z}
\tag{5-32}
\end{equation}

将上述的 $Y$(式)、$Z$(式)代入等式 (5-32),即可获得 VCSEL 的小信号响应数学表达式。

在该表达式中存在许多的参数,待优化参数为 $\left(\eta_{i},\beta,\tau_{n},k,G_{0},N_{0},\tau_{p},\varepsilon\right)$。需要采用非线性最优化获取最佳的参数值,使得在相同的温度和偏置电流影响下,响应 $H\left(f\right)$ 与实测的响应输出一致。

### 2) 参数模型求解

根据已知数据信息,VCSEL 激光器在偏置电流为 I_b=7.5mA,环境温度为 Ta=20°C 条件下;带宽响应曲线数据。属于非线性多参数优化问题,待优化参数为 $\left(\eta_{i},\beta,\tau_{n},k,G_{0},N_{0},\tau_{p},\varepsilon\right)$,为使模型中存在的驱动电流,光功率值以及小信号响应值三个理论值尽可能接近于实验值。本文拟采用多目标遗传算法 NSGA-II 和模拟退火算法两种方式进行求解,并比较两种算法的误差,最终选取最优参数。

#### ① NSGA-II(多目标遗传算法)求解

NSGA-II $^{[12]}$ 的基本原理与传统遗传算法基本相同。区别在于在它的选择过程上,选择过程分 3 个步骤:

1. **快速非支配排序算子设计**

多目标优化问题的设计关在于求取 Pareto 最优解集。其快速非支配是依据个体的非劣解水平对种群分层,作用是指引搜索向 Pareto 最优解集方向进行。它是一个循环的适应值分级过程:首先找出全体中非支配解集,记为第一非支配

层 F1, 将其所有个体富裕非支配序 \(i_{rank} = 1\)(其中 \(i_{rank}\) 是个体 \(i\) 的非支配序值),并从整个群体中出去;然后继续找出余下全体中非支配解集,记为第二个非支配层 F2,个体被赋予非支配序 \(i_{rank} = 2\);以此类推,直到整个种群被分层,同一分层内的个体具有相同的非支配序 \(i_{rank}\)。

2. 个体拥挤距离算子设计

为能够在具有相同 \(i_{rank}\) 的个体间继续进行选择性排序,NSGA-II 提出了个体拥挤距离的概念。个体 \(i\) 的拥挤距离是目标空间上与 \(i\) 相邻的两个个体 \(i+1\) 和 \(i-1\) 之间的距离,其计算步骤为:

Step. 1 对同层的个体初始化距离。令 \(L[i]_d = 0\) 其中 \(L[i]_d\) 表示任意个体 \(i\) 的拥挤距离。

Step. 2 对同层的个体按第 \(m\) 个目标函数值升序排列。

Step. 3 使得排序边缘上的个体具有选择优势。给定一个大数 \(W\),令

\[
L[0]_d = L[i]_d = 0
\]

Step. 4 对排序中间的个体,求拥挤距离:

\[
L[i]_d = L[i]_d + (L[i+1]_d - L[i-1]_d) / (f^{\max} - f^{\min})
\]

(其中,\(L[i+1]_d\) 为第 \(i+1\) 个体的第 \(m\) 目标函数值,\(f^{\max}\) 和 \(f^{\min}\) 分别为集合中第 \(m\) 目标函数的最大和最小值)。

Step. 5 对不同的目标函数,重复 Step. 2-Step. 4 操作得到个体 \(i\) 的拥挤距离 \(L[i]_d\),通过优先选择拥挤距离较大的个体,可使计算结果在目标空间比较均匀地分布,以维持群体的多样性。

3. 精英策略选择算子设计

精英策略即保留父代中的优良个体直接进入子代,以防止获得的 Pareto 最优解丢失。精英策略选择算子按 3 个指标对由父代 \(C_i\) 和子代 \(D_i\) 合成的种群 \(R_i\) 进行优选,以组成父代种群 \(C_{i+1}\),首先淘汰父代中方案校验标志为不可行的方案;其次按照非支配序 \(i_{rank}\) 从低到高排序,将整层种群依次放入 \(C_{i+1}\),直到放入某一层 \(F_i\) 时出现 \(C_{i+1}\) 大小超出种群规模限值 \(N\) 的情况;最后,依据 \(F_i\) 中的个体拥挤距离由大到小的顺序继续填充 \(C_{i+1}\) 直到种群数量达到 \(N\) 时终止。

结合题目可设 NSGA-II 多目标函数为:

\[
\min \left\{
\begin{aligned}
P - P_0 &= \frac{1}{\left[1 - q G_0 \left(N_s - N_0\right)\right] T_n} - P_0 \\
I_{cur} - I &= \left[\frac{P_0}{\eta_i} + I_{th0} + I_{off}(T)\right] - I = \left[\frac{P_0}{\eta_i} + I_{th0} + T_0 + \sum_{n=0}^4 a_n \left[(IV - P_0) R_{th}\right]^n\right] - I \\
\text{sum}_{USGA} &= \sum_{i=0}^n \left|h_i' - h_i\right|
\end{aligned}
\right.
\]

上式 3 个目标函数意义分别为:1) 光功率值;2) 驱动电流理论值与实验值的差值的绝对值;3) 小信号响应值理论值及实验值差值绝对值总和。初始种群的设定按照题目附录 2 的推荐值,如下表

\begin{table}
\centering
\caption{NSGA-II 初始参数值}
\begin{tabular}{c c c c c c c c}
\hline
$\eta_{i}$ & $\beta$ & $\tau_{n}$ & $k$ & $G_{0}$ & $N_{0}$ & $\tau_{p}$ & $\varepsilon$ \\
\hline
参数估计值 & 0.7 & 1E-5 & 9.6E-9 & 1.5E-8 & 1.8E6 & 4.97E5 & 3.8E-12 & 4.7E-8 \\
单位 & - & - & $s$ & $W$ & $s^{-1}$ & - & $s$ & - \\
\hline
\end{tabular}
\end{table}

结论:依据上述数学模型和求解方法,基于 Matlab 编写 NSGA-II 求解程序(代码见附录 3.5)。计算结果得到的一组最优参数为 $(\eta_{i}, \beta, \tau_{n}, k, G_{0}, N_{0}, \tau_{p}, \varepsilon) = (0.7, 9.59E-3, 1.266E-8, 1.45E-5, 1.18E8, 1.04E8, 2.39E-11, 4.08E-9)$,使得拟合宽带响应曲线最大程度贴合原始曲线。如下图所示:

\begin{figure}[h]
\centering
\includegraphics[width=\textwidth]{image.png}
\caption{NSGA-II 最优参数拟合宽带响应曲线与原始曲线对比}
\end{figure}

由上图可以看出,NSGA-II 算法拟合出的 L-I 曲线与原始曲线可以较好程度地重合。在某些位置(原始曲线有局部波动处),2 条曲线的重合度有所下降。可以算出,两条曲线对应点距离之和以及平均相对误差为:
\[
sum_{USGA} = \sum_{i=0}^{2501} \left| h_{i}^{\prime} - h_{i} \right| = 1109
\]

\begin{equation}
\Delta_{USGA} = \frac{\sum_{i=0}^{n} |h_i' - h_i|}{n} / |h_{avg}|
\end{equation}
\begin{equation}
= \frac{1109}{2501} / |-1.56| = 28.4\%
\end{equation}

平均相对误差为

\section*{② 模拟退火算法求解}

模拟退火算法 (Simulated Annealing, SA) 最早的思想是由 N. Metropolis\cite{metropolis1953equation} 等人于 1953 年提出。模拟退火算法来源于固体退火原理,是通过赋予搜索过程一种时变且最终趋于零的概率突跳性,从而可有效避免陷入局部极小并最终趋于全局最优的串行结构的优化算法。

模拟退火算法作为优化算法的一种,可以分解为解空间、目标函数和初始解三部分。模拟退火算法参数寻优过程可以简化为以下几步:

Step. 1 随机挑选一个单元,并给它一个随机的位移 \( k \),求出系统因此而产生的能量变化 \(\Delta E_k\)。

Step. 2 若 \(\Delta E_k \leq 0\),该位移可采纳,而变化后的系统状态可作为下次变化的起点;若 \(\Delta E_k > 0\),位移后的状态可采纳的概率为

\begin{equation}
P_k = \frac{1}{1 + e^{-\Delta E_k / T}}
\tag{5-35}
\end{equation}

式中 \( T \) 为温度,然后从 \((0, 1)\) 区间均匀分布的随机数中挑选一个数 \( R \),若 \( R < P_k \),则将变化后的状态作为下次的起点;否则,将变化前的状态作为下次的起点。

Step. 3 转第 (1) 步继续执行,直到达到平衡状态为止。

\begin{figure}[h]
    \centering
    \includegraphics[width=\textwidth]{image.png}
    \caption{模拟退火算法流程图}
    \label{fig:annealing_flowchart}
\end{figure}

本文程序初始温度设计为 $90^\circ \mathrm{C}$,结束温度为 $20^\circ \mathrm{C}$,温度下降比例为 $0.99$,退火次数为 $1000$,根据参考初值设置初始解,对目标函数进行参数寻优。

结论:依据上述数学模型和求解方法,基于 Matlab 编写模拟退火求解程序(代码见附录 5),模拟退火求解的初始参数与 NSGA-II 取值一致。计算结果得到的一组最优参数为:
\begin{equation}
(\eta_i, \beta, \tau_n, k, G_0, N_0, \tau_p, \varepsilon) = (0.7, 9.59E-5, 4.36E-5, 1.72E-8, 1.04E6, 8.69E5, 1.15E-11, 1.199E-7)
\end{equation}
使得拟合宽带响应曲线最大程度贴合原始曲线。如下图所示:

\begin{figure}[h]
    \centering
    \includegraphics[width=\textwidth]{image.png}
    \caption{模拟退火算法最优参数宽带响应曲线与原始曲线对比}
\end{figure}

同样可以算出,两条曲线对应点距离之和以及平均相对误差为:

\begin{equation}
sum_{USGA} = \sum_{i=0}^{2501} |h_i' - h_i| = 521
\end{equation}

\begin{equation}
\Delta_{USGA} = \frac{\sum_{i=0}^{n} |h_i' - h_i|}{n} / |h_{avg}|
\end{equation}

\begin{equation}
= \frac{521}{2501} / |-1.56| = 13.6\%
\end{equation}

\section*{③两种算法结果对比}

将两种算法计算出的优化参数对应的拟合曲线对比可知,采用模拟退火算法参数优化效果更好,与原始曲线的重合度更高。因此后将采用模拟退火算法的优化参数,即为:

\begin{equation}
(\eta_i, \beta, \tau_n, k, G_0, N_0, \tau_p, \varepsilon) = (0.7, 9.59E-5, 4.36E-5, 1.72E-8, 1.04E6, 8.69E5, 1.15E-11, 1.199E-7)
\end{equation}

\begin{figure}[h]
    \centering
    \includegraphics[width=\textwidth]{image.png}
    \caption{模拟退火算法与 NAGA-II 拟合结果比较}
    \label{fig:10}
\end{figure}

但是最优拟合曲线与原始曲线的平均相对误差仍然到到 13.6\%,重合度并非特别理想。分析原因,原始曲线为实测值,曲线波动幅度较大,导致与拟合曲线(相对光滑)有一定偏差度。

\subsubsection{偏置电流、温度对带宽影响}

依据上文推导的激光器小信号幅频响应参数模型,以及确定的最优参数,可以绘制不同环境温度和不同偏置电流下的带宽响应曲线(即 S21 曲线)。图 11 为 20°C 环境温度下不同偏置电流下的带宽响应曲线,图 12 为偏置电流 I=7.5mA 时不同环境温度下的带宽响应曲线。

\begin{figure}[h]
    \centering
    \includegraphics[width=\textwidth]{image1.png}
    \caption{在 $20^{\circ}\mathrm{C}$ 下的不同偏置电流幅频响应曲线}
    \label{fig:11}
\end{figure}

取幅度为 $-10\mathrm{dB}$ 画出一条水平截止线,不同偏置电流下幅频响应曲线与截止线交点处的频率即为器件带宽。从图 \ref{fig:11} 中可以看到,在 $20^{\circ}\mathrm{C}$ 的环境温度下 $4.5\mathrm{mA}$、$5.5\mathrm{mA}$ 和 $6.5\mathrm{mA}$ 曲线对应的横坐标频率(带宽)分别约为 $25\mathrm{GHz}$、$28\mathrm{GHz}$ 和 $29\mathrm{GHz}$,其它曲线都大于 $30\mathrm{GHz}$,。曲线变化总体趋势是随着偏置电流的增大,器件带宽不断增大。

\begin{figure}[h]
    \centering
    \includegraphics[width=\textwidth]{image2.png}
    \caption{在 $7.5\mathrm{mA}$ 偏置电流下的不同温度幅频响应曲线}
    \label{fig:12}
\end{figure}

从图 \ref{fig:12} 可以看出,偏置电流为 $7.5\mathrm{mA}$ 时 $50^{\circ}\mathrm{C}$、$40^{\circ}\mathrm{C}$ 和 $30^{\circ}\mathrm{C}$ 曲线对应的横坐

标频率(带宽)分别约为 $27\mathrm{GHz}$、$28.5\mathrm{GHz}$ 和 $29.5\mathrm{GHz}$,其它曲线都大于 $30\mathrm{GHz}$,曲线变化总体趋势是随着温度的增大,器件带宽不断减小。

综上分析,偏置电流和温度的对器件带宽有着较大的影响:带宽随着偏置电流的增大而增大,随着温度的增大而减小。

\subsection*{5.3.3 激光器设计优化}

通过对上述模型和计算结果的分析,同时参考专业文献,本文提供了两项优化设计方案。

\textbf{方案一:}

带宽受温度和电流的大小影响很大,外部温度用恒温箱来控制,激光器内部偏置电流的大小受器件自身内部的温度影响很大,可以用 PID 反馈控制,实现控制电流在一个恒值附近。

其基本原理是:通过传感器实时检测电流的大小,将电流的大小反馈给控制单元,控制单元将这一时刻和上一时刻的电流大小做比较,然后由控制单元利用 PID 控制算法反馈调整激光器的电流控制执行器件,从而实现电流值的恒定,保证输出光功率。

\textbf{方案二:} 改善器件的结构,优化制造工艺。VCSEL 的传统电流注入方式主要有顶注入和侧面注入两种方式。电流注入需要经过 DBR 层才能进入有源区,电流通过 DBR 时形成较大的电阻串联电阻问题也更为严重。大的串联电阻会使 VCSEL 在工作过程中温升较大,影响器件的性能,使器件效率大大降低,进而影响到带宽。

在原来基础上至少增加两道重要工艺过程,台面刻蚀和平坦化。为了减小刻蚀过程对有源区的影响,采用刻蚀方向性较好的 ICP 进行干法刻蚀,同时刻蚀速率要稳定,保证比较精确的刻蚀厚度,以利于后续平坦化工艺的进行。这样可以减少器件本身的阻值,减少自身的产热量,从而改善器件的带宽。

\subsection*{5.3.4 参数对带宽曲线的影响性分析}

根据 5.3.2 推导的 VSCEL 激光器小信号相应模型,表达式见(5-32),表达式中的参数变量有 $(\eta_i, \beta, \tau_n, k, G_0, N_0, \tau_p, \varepsilon)$,因此,可通过正交试验挑选其中对带宽影响较大的参数,并分析其影响效果。

采用控制变量法,研究单一参数变化对带宽的影响情况。基于 MATLAB 编写参数查找模型(代码见附录 3.7)。其基本过程如下

Step. 1 全部参数中选取某一参数作为单独变量;

Step. 2 在该参数具有物理意义范围内,使其按一定规律变化,观察器件带宽变化情况;

Step. 3 若带宽有明显变化,则说明激光器对该参数敏感。

通过计算,结果显示有 3 个参数对激光器的带宽影响较大,分别为:受激辐射耦合系数 $\beta$;载流子复合寿命 $\tau_n$;比例因子 $k$。

\begin{figure}[h]
    \centering
    \includegraphics[width=\textwidth]{image1.png}
    \caption{激光器参数对其带宽的影响}
    \label{fig:13}
\end{figure}

\textbf{结论:}从图 13 b 可以看出,减小载流子复合寿命 $\tau_{n}$ 既可以扩大激光器的工作带宽,又能降低高于 0 频的幅度(使曲线变得平坦)。在一定范围内,降低载流子复合寿命 $\tau_{n}$ 有助于提高激光器工作性能。

但是通过图 a、c 可知,受激辐射耦合系数 $\beta$ 增大或者比例因子 $k$ 减小都会扩大激光器的工作带宽,但同时也使得幅频响应曲线在部分频率处幅度可能会高于 0 频的趋势增加。

\subsection{5.4 VSCEL 带宽模型的改进}

\subsubsection{5.4.1 改进带宽模型建模}

上述 VCSEL 的小信号响应建模虽然能够很好的分析激光器幅频响应,得到仿真数据,但是精确度不够,因为此模型未能考虑到 VCSEL 器件的结构特性,例如 DBR 反射镜,半导体材料的发射波长随着温度的升高会发生红移现象,光子经过 DBR 反射镜过程中存在着光损耗,电流流过 DBR 层过程存在着焦耳热损耗。同时,该模型只是从热学和电学的角度来分析,未涉及光学。接下来,基于 VCSEL 器件的结构特性,从光学、电学以及热学三个方面建立 VCSEL 器件稳小信号特

性的速率方程模型。

参考文献 \cite{ref14},我们将研究 VCSEL 的小信号速率方程组模型,关于时间的量不为 0。为了方便求解小信号速率方程组的解析解,假设载流子浓度和光子浓度变化量比较的小。注入电流 I、载流子浓度 N、光子浓度 Np 以及增益 g 作为变量分别对时间 t 求微分,可得式 (5-36) 和式 (5-37)

\begin{equation}
d\left[\frac{dN}{dt}\right] = \frac{\eta_i}{qV} dI - \frac{1}{\tau_{\Delta N}} - v_g g dN_p - N_p v_g dg
\tag{5-36}
\end{equation}

\begin{equation}
d\left[\frac{dN_p}{dt}\right] = \left(\Gamma v_g - \frac{1}{\tau_p}\right) dN_p + N_p \Gamma v_g + \frac{\Gamma}{\tau_{\Delta N}}
\tag{5-37}
\end{equation}

式中,$V, V_p$ 分别为电子和光子的体积,$\Gamma$ 为光限制因子,$\tau_p$ 为光子寿命;其中,辐射进入激光基模的载流子寿命 $\tau_{\Delta N}$ 和 $\tau_{\Delta N}'$ 分别表示为:

\begin{equation}
\frac{1}{\tau_{\Delta N}} = \frac{dR_{sp}}{dN} = 2\beta_{sp} BN + \frac{d\beta_{sp}}{dN} BN^2
\tag{5-38}
\end{equation}

\begin{equation}
\frac{1}{\tau_{\Delta N}'} = \frac{dR_{sp}'}{dN} = 2\beta BN + \frac{d\beta_{sp}}{dN} BN
\tag{5-39}
\end{equation}

式中,$R_{sp}$ 为描述电子和空穴自发辐射复合率,单位为 $1/\text{cm}^3\text{s}$,$R_{sp}'$ 辐射复合中对激射有贡献的复合率,单位为 $1/\text{cm}^3\text{s}$,$\beta$ 为自发辐射复合产生激射光子的光子增加因子,一般为 $10^{-4}$ 数量级。

其中,$\tau_{\Delta N}'$ 在几十微秒范围内取值,大部分情况下可以忽略。增益 g 是关于载流子浓度 N 和光子浓度 Np 的函数,对其求导可得:

\begin{equation}
dg = \alpha dN - \alpha_p dN_p
\tag{5-40}
\end{equation}

\begin{equation}
\alpha = \frac{\partial g}{\partial N} = \frac{g_0}{(N + N_s)(1 + \varepsilon N_p)}
\tag{5-41}
\end{equation}

\begin{equation}
\alpha_p = -\frac{\partial g}{\partial N_p} = \frac{\varepsilon g}{(1 + \varepsilon N_p)}
\tag{5-42}
\end{equation}

其中,g0、Ns 均为常量,$\varepsilon$ 为经验增益压缩因子,也为常量。将式 (5-40) 代入式 (5-36) 和式 (5-37) 并整理,可得式 (5-43) 和式 (5-44),

\begin{equation}
d\left[\frac{dN}{dt}\right] = \frac{\eta_i}{qV} dI - \left(\frac{1}{\tau_{\Delta N}} + N_p v_g \alpha\right) dN - \left(v_g g - N_p v_g \alpha_p\right) dN_p
\tag{5-43}
\end{equation}

\begin{equation}
d\left[\frac{dN_p}{dt}\right] = \left(\frac{\Gamma}{\tau_{\Delta N}'} + N_p \Gamma v_g \alpha\right) dN - \left(\frac{1}{\tau_p} - \Gamma v_g g + N_p \Gamma v_g \alpha_p\right) dN_p
\tag{5-44}
\end{equation}

通过参考资料 \cite{ref14} 可得:

\begin{equation}
\frac{1}{\tau_{p}}-\Gamma v_{g} g=\frac{\Gamma R_{sp}}{N_{p}}
\tag{5-45}
\end{equation}

将式(5-45)代入式(5-43)和式(5-44)中并将其写成矩阵的形式,如下所示,

\begin{equation}
\frac{d}{dt}\left[\begin{array}{l}
d N \\
d P
\end{array}\right]=\left[\begin{array}{cc}
-\gamma_{N N} & -\gamma_{N P} \\
\gamma_{P N} & -\gamma_{P P}
\end{array}\right]\left[\begin{array}{l}
d N \\
d N_{P}
\end{array}\right]+\frac{\eta_{i}}{q V}\left[\begin{array}{l}
d I \\
0
\end{array}\right]
\tag{5-46}
\end{equation}

\begin{equation}
\gamma_{N N}=\frac{1}{\tau_{\Delta N}}+v_{g} \alpha N_{p}
\tag{5-47}
\end{equation}

\begin{equation}
\gamma_{N P}=v_{g} g-v_{g} \alpha_{p} N_{p}=\frac{1}{\tau_{\Delta N}}-\frac{R_{s p}}{N_{p}}-v_{g} \alpha_{p} N_{p}
\tag{5-48}
\end{equation}

\begin{equation}
\gamma_{P N}=\frac{\Gamma}{\tau_{\Delta N}}+\Gamma v_{g} \alpha N_{p}
\tag{5-49}
\end{equation}

\begin{equation}
\gamma_{P P}=\frac{1}{\tau_{p}}-\Gamma v_{g} \alpha_{p}=\frac{R_{s p}^{\prime}}{N_{p}}+\Gamma v_{g} \alpha_{p} N_{p}
\tag{5-50}
\end{equation}

当注入电流 I 远大于阈值电流 Ith 且光子浓度 Np 足够大时,对式(5-46)中的各项进行优化,可得,

\begin{equation}
\gamma_{N N}=\frac{1}{\tau_{\Delta N}}+v_{g} \alpha N_{p}
\tag{5-51}
\end{equation}

\begin{equation}
\gamma_{N P}=\frac{1}{\Gamma \tau_{p}}-v_{g} \alpha_{p} N_{p}
\tag{5-52}
\end{equation}

\begin{equation}
\gamma_{P N}=\Gamma v_{g} \alpha N_{p}
\tag{5-53}
\end{equation}

\begin{equation}
\gamma_{P P}=\Gamma v_{g} \alpha_{p} N_{p}
\tag{5-54}
\end{equation}

通过求解小信号速率方程组式(5-36),可以帮助我们深入了解 VCSEL 小信号调制特性的动态物理机理。虽然,式(5-51~5-54)中的各个系数依赖于载流子浓度 N 和光子浓度 Np,但在特定的注入电流条件下,N 和 Np 均为常量并通过求解上一节中的稳态速率方程组求得。因而,式(5-51~5-54)所表示的各个系数在具体的注入电流下均可看作常数。假设调制电流为正余弦形式,dI、dN 以及 dNp 可分别表示为式(5-55~5-57)。

\begin{equation}
d I(t)=I_{1}(t)=I_{1} e^{j \omega t}
\tag{5-55}
\end{equation}

\begin{equation}
d N(t)=I_{1}(t)=N_{1} e^{j \omega t}
\tag{5-56}
\end{equation}

\begin{equation}
d N_{p}(t)=I_{1}(t)=N_{p 1} e^{j \omega t}
\tag{5-57}
\end{equation}

其中,$I_{1}$、$N_{1}$ 以及 $N_{p1}$ 分别为调制电流的振幅、载流子浓度幅值以及光子浓度幅值。

将式 (5-55~5-57) 代入到式 (5-47) 中并对其进行整理可得关于 $N_{1}$ 和 $N_{p1}$ 的矩阵形式,

\[
\begin{bmatrix}
\gamma_{NN} + j\omega & \gamma_{NP} \\
-\gamma_{PN} & \gamma_{PP} + j\omega
\end{bmatrix}
\begin{bmatrix}
N_{1} \\
N_{P1}
\end{bmatrix}
=
\frac{\eta_{i}I_{1}}{qV}
\begin{bmatrix}
0 \\
1
\end{bmatrix}
\tag{5-58}
\]

\[
\Delta =
\begin{vmatrix}
\gamma_{NN} + j\omega & \gamma_{NP} \\
-\gamma_{PN} & \gamma_{PP} + j\omega
\end{vmatrix}
=
\gamma_{NP}\gamma_{PN} + \gamma_{NN}\gamma_{PP} - \omega^{2} + j\omega\left(\gamma_{NN} + \gamma_{PP}\right)
\tag{5-59}
\]

求解式 (5-60) 可得

\[
N_{1} = \frac{\eta_{i}I_{i}}{q\nu} \cdot \frac{\gamma_{PP} + j\omega}{\omega_{R}^{2}} \cdot H(\omega)
\tag{5-60}
\]

\[
N_{p1} = \frac{\eta_{i}I_{i}}{q\nu} \cdot \frac{\gamma_{PN}}{\omega_{R}^{2}} \cdot H(\omega)
\tag{5-61}
\]

\[
H(\omega) = \frac{\omega^{2}}{\Delta} = \frac{\omega_{R}^{2}}{\omega_{R}^{2} - \omega^{2} + j\omega\gamma}
\tag{5-62}
\]

\[
\omega_{R}^{2} = \gamma_{NP}\gamma_{PN} + \gamma_{NN}\gamma_{PP}
\tag{5-63}
\]

\[
\gamma = \gamma_{NN} + \gamma_{PP}
\tag{5-64}
\]

其中,$H(\omega)$ 为关于参变量 $\omega_{R}$ 和 $\gamma$ 的传递函数,$\omega_{R}$ 为松弛振荡角频率,$\gamma$ 为阻尼因子。

将式 (5-47~5-50) 代入并整理 $\omega_{R}$ 和 $\gamma$ 的表达式可得,

\[
\omega_{R}^{2} = \frac{\nu_{g}\alpha N_{p}}{\tau_{p}} + \left(\frac{\Gamma\nu_{g}\alpha_{p}N_{p}}{\tau_{\Delta N}} - \frac{\Gamma\nu_{g}g}{\tau_{\Delta N}}\right)\left(1 - \frac{\tau_{\Delta N}}{\tau_{\Delta N}^{'}}\right) + \frac{1}{\tau_{\Delta N}\tau_{p}}
\tag{5-65}
\]

\[
\gamma = \left(\frac{1}{\tau_{\Delta N}} + \nu_{g}\alpha N_{p}\right) + \left(\frac{1}{\tau_{p}} + \Gamma\nu_{g}\alpha_{p}N_{p}\right) - \Gamma\nu_{g}g
\tag{5-66}
\]

从式 (5-66) 可以看出:左端第一个括号项表示由于载流子浓度的变化而引起的损耗,第二个括号项表示由于光子浓度的变化引起的损耗,最后一项为光子浓度的增加项。式 (5-65) 包含了与式 (5-66) 相类似的项,不过在形式上除以了 $\tau_{\Delta N}$、$\tau_{\Delta N}^{'}$ 或 $\tau_{p}$。消去式中的增益 $g$,可变为,

\[
\omega_{R}^{2} = \frac{\nu_{g}\alpha N_{p}}{\tau_{p}} + \left(\frac{\Gamma\nu_{g}\alpha_{p}N_{p}}{\tau_{\Delta N}} + \frac{\Gamma R_{sp}^{'}}{\tau_{\Delta N}N_{p}}\right)\left(1 - \frac{\tau_{\Delta N}}{\tau_{\Delta N}^{'}}\right) + \frac{1}{\tau_{\Delta N}\tau_{p}}
\tag{5-67}
\]

\begin{equation}
\gamma = \nu_{g} \alpha N_{p} \left( \frac{\Gamma \alpha_{p}}{\alpha} + 1 \right) + \frac{1}{\tau_{\Delta N}} + \frac{\Gamma R_{sp}^{'}}{N_{p}}
\tag{5-68}
\end{equation}

若注入电流远大于阈值电流,松弛振荡频率 $\omega_{R}$ 可以进一步的简化,此时式 (5-67) 右端第一项远远大于其它项,对 $\omega_{R}$ 起着主要作用。可简化为式 (5-69),相应的 $\gamma$ 可写为式 (5-70) 的形式。

\begin{equation}
\omega_{R}^{2} = \frac{\nu_{g} \alpha N_{p}}{\tau_{p}}
\tag{5-69}
\end{equation}

\begin{equation}
\gamma \approx K f_{R}^{2} + \gamma_{0}
\tag{5-70}
\end{equation}

\begin{equation}
f_{R} = \frac{\omega_{R}}{2 \pi}
\tag{5-71}
\end{equation}

其中,K 因子和阻尼补偿因子 $\gamma_{o}$ 分别表示为

\begin{equation}
K = 4 \pi^{2} \tau_{p} \left( 1 + \Gamma \frac{\alpha_{p}}{\alpha} \right)
\tag{5-72}
\end{equation}

\begin{equation}
\gamma_{o} = \frac{1}{\tau_{\Delta N}} + \frac{\Gamma R_{sp}^{'}}{N_{p}}
\tag{5-73}
\end{equation}

将式 (5-69) 代入式 (5-71),可得

\begin{equation}
f_{R} = \frac{1}{2 \pi} \sqrt{\frac{\nu_{g} \alpha N_{p}}{\tau_{p}}} = \frac{1}{2 \pi} \sqrt{\frac{\nu_{g} \alpha \eta_{i}}{q V_{p}}} \cdot \sqrt{I - I_{th}} = D \sqrt{I - I_{th}}
\tag{5-74}
\end{equation}

其中,D 因子表征着松弛振荡频率 $f_{R}$ 随注入电流 I 的变化快慢,即曲线的斜率。D 因子不是一个常量,它依赖于微分增益 $\alpha$。根据式 (33) 可知,随着载流子浓度 N 和光子浓度 Np 的增大,微分增益 $\alpha$ 减小,即 D 因子减小了,进而使得 $f_{R}$ 随注入电流 I 变化程度变缓。使用 M 因子,我们将 VCSEL 的调制带宽表示为类似的形式,如式 (41)。

\begin{equation}
f_{-10dB} = M \sqrt{I - I_{th}}
\tag{5-75}
\end{equation}

将式 (5-74) 代入式 (5-75),可得,

\begin{equation}
\gamma = K D^{2} (I - I_{th}) + \gamma_{o}
\tag{5-76}
\end{equation}

整理式 (5-63)、式 (5-71) 等,可得注入电流大于阈值电流的 VCSEL 的 $N_{p1}$ 与 $I_{1}$ 的关系表达式 (5-77)。

\begin{equation}
\frac{N_{p1}}{I_{1}} = \frac{\eta_{i}}{q V} \Gamma \tau_{p} H(\omega)
\tag{5-77}
\end{equation}

依据相关资料 [14],可以将式 (5-77) 进一步表示为式 (5-78)。

\begin{equation}
\frac{P_{1}}{I_{1}}=\eta_{i} \eta_{0} \frac{h r}{q} H(\omega)
\tag{5-78}
\end{equation}

其中,$hr$ 为光子能量。从式(43)可以看出,传递函数 $H(\omega)$ 可以通过实验测量的方式求得。若再考虑到电学寄生效应的影响,为了方便研究,通常将电学寄生效应等效为一阶低通滤波器,则传递函数表达式可进一步的完善为式(5-79)。

\begin{equation}
H(f)=const \cdot \frac{f_{R}}{\left(j 2 \pi f\right)^{2} \omega_{R}^{2}+j 2 \pi f \cdot \gamma+\gamma_{o}}
\tag{5-79}
\end{equation}

综上公式,得到小信号相应为:

\begin{equation}
H(f)=const \cdot \frac{D \sqrt{I-I_{th}}}{\left(j 2 \pi f\right)^{2} \frac{v_{g} \alpha N_{p}}{\tau_{p}}+\left(\frac{\Gamma_{v} \alpha_{p} N_{p}}{\tau_{\Delta N}}+\frac{\Gamma R_{s p}^{'}}{\tau_{\Delta N} N_{p}}\right)\left(1-\frac{\tau_{\Delta N}}{\tau_{\Delta N}^{'}}\right)+\frac{1}{\tau_{\Delta N} \tau_{p}}+j \cdot 2 \pi f\left(K D^{2}\left(I-I_{th}\right)+\gamma_{o}\right)}
\tag{5-80}
\end{equation}

\subsection*{5.4.2 VCSEL 宽带模型验证}

利用模拟退火算法进行迭代优化,本文程序(见附录 4.2)初始温度设计为 $90^{\circ} \mathrm{C}$,结束温度为 $20^{\circ} \mathrm{C}$,温度下降比例为 0.99,退火次数为 1000,根据参考处置设置初始解,对目标函数(与问题 3 相同)进行参数寻优,如下表:

\begin{table}[h]
\centering
\caption{模拟退火法最优参数值}
\begin{tabular}{|c|c|c|c|c|c|c|c|}
\hline
参数 & $\Gamma$ 光限制因子 & $\tau_{p}$ 光子寿命 & $\tau_{\Delta N}$ 载流子寿命 & K 因子系数 & D 松弛振动幅度 & $\gamma_{o}$ 阻尼补偿因子 & $\alpha$ 微分增益 \\
\hline
数值 & $4.7 \mathrm{E}-5$ & $5.7 \mathrm{E}-11$ & $4.5 \mathrm{E}-7$ & $0.0087$ & $2.86$ & $1.05 \mathrm{E}-6$ & $1.40$ \\
\hline
\end{tabular}
\end{table}

通过 MATLAB 软件作图如下:

\begin{figure}[h]
\centering
\includegraphics[width=\textwidth]{image.png}
\end{figure}

\begin{figure}[h]
    \centering
    \includegraphics[width=\textwidth]{image.png}
    \caption{幅频响应曲线}
    \label{fig:amplitude_frequency_response}
\end{figure}

从上图可以看出,建立的改进模型可以较好地拟合原始幅频响应曲线,证明该改进模型有较高准确性。在某些位置(原始曲线有局部波动处),2 条曲线的重合度有所下降。可以算出,两条曲线对应点距离之和以及平均相对误差为:

\begin{align*}
    sum_{USGA} &= \sum_{i=0}^{2501} \left| h_i' - h_i \right| = 509 \\
    \Delta_{USGA} &= \frac{\sum_{i=0}^{n} \left| h_i' - h_i \right|}{n} / \left| h_{avg} \right| \\
    &= \frac{509}{2501} / \left| -1.56 \right| = 13.04\%
\end{align*}

平均相对误差为

\begin{figure}[h]
    \centering
    \includegraphics[width=\textwidth]{image1.png}
    \caption{20°C下的不同电流两种模型对比结果}
    \label{fig:15}
\end{figure}

\begin{figure}[h]
    \centering
    \includegraphics[width=\textwidth]{image2.png}
    \caption{7.5mA下的不同温度两种模型对比结果}
    \label{fig:16}
\end{figure}

从图 \ref{fig:15}、\ref{fig:16} 可以明显看出改进后的模型相比原始模型,在同一温度和同一电流下可以获得更大的工作带宽,提升了 VCSEL 的工作性能。

该改进模型是基于 VCSEL 器件的结构特性,不仅从电学、热学两方面考虑,又创新性地考虑了器件的光学特性,重新构建 VCSEL 器件稳小信号特性的速率方程模型。基于改进模型重新绘制宽带响应曲线表明,相同环境温度和偏置电流下改进模型的激光器-10dB 工作带宽更宽。

\section{模型评价与改进}

\subsection{模型的评价}

1) 优点:本文优势在求解参数优化模型时,采用多种算法进行求解(如自适应遗传算法、粒子群算法、模拟退火法等),比较选取最优参数。同在建立改进模型时考虑了温度影响等因素,另外从光学、电学以及热学 3 方面重构模型,提高了模型的适应性和准确性。

2) 不足:问题 1 和问题 3 建立的的优化参数模型,都是基于一组数据,因此得到的最优参数仅能较好地反映当前实验条件下,VCSEL 的工作特性,推广到其他情况时(如不同温度、不同偏置电流),结果准确度可能会下降。

\subsection{模型的改进}

对于问题 1 和问题中的参数优化模型,通过查找文献,获得更广泛的试验数据样本,将其分为拟合数据和验证数据,则可以获得更准确,适应性更强的参数。模型改进过程中,为简化推导过程,设置了一些假设条件,如忽略某些参数受温度的影响,导致模型与激光器的真实工作特性有区别。可采用实验测试与理论推导结合的方式,建立理论-经验模型,从而提高模型的可靠性。

\section*{七、参考文献}

[1] 周元隆, 65nm CMOS 工艺 22Gb/sVCSEL 驱动器电路设计[D]. 南京, 东南大学硕士学位论文, 9-10 页, 2015.

[2] 毛陆虹等, 垂直腔面发射激光器的小信号电路模型和调制特性[J]. 半导体学报, 83-84 页, 2001.

[3] 杨浩, 垂直腔面发射激光器模式模型的建立及其特性分析[D]. 北京, 北京工业大学硕士论文, 51-55, 2008.

[4] 杨宜等, 基于速率方程的垂直腔表面发射半导体激光器温度模型与仿真[J]. 中国激光, 195 页, 2003.

[5] 龚纯, 王正林, 精通 MATLAB 最优化计算[M], 北京: 电子工业出版社, 313-344, 2012.

[6] Holland JH. Adaptation in Natural and Artificial Systems [M]. [S.I.]: MTP Press. 1975.

[7] 王东风等, 粒子群优化算法的性能分析和参数选择[J]. 自动化学报, 1554 页, 2016.

[8] Hagley G R, Lear K L, Warren M E. Comprehensive numerical modeling of vertical cavity surface-emitting lasers. IEEE J Quntum Electron, 1996.

[9] Mena P V, Morikuni J J, Kang S M. A Simple rate-equation-based thermal VCSEL modle. J Lightwave Technol, 1997.

[10] Brusenbach P R, Swirhum S, Uchida T K. Equivalent circuit for vertical cavity top surface emitting lasers. Electron Lett, 1993.

[11] 梁锋等, 垂直腔面发射激光器的温度模型[J]. 半导体学报, 1125-1127, 2007.

[12] 常伟等, 基于 NSGA-II 算法 Vienna 整流器多目标优化[J]. 中国电机工程学报, 179 页, 2016.

[13] 龙柏军等, 利用模拟退火优化快速反射镜控制策略[J]. 光学精密工程, 2233 页, 2016.

[14] 邓果, 垂直腔面发射激光器非线性动态行为的仿真研究[D], 西南交通大学硕士学位论文, 6-12, 2004.

[15] 史峰, 王辉, 郁磊等, MATLAB 智能算法 30 个案例分析[M], 北京: 北京航空航天大学出版社, 57-66, 2007.

\section*{八、附录}

\textbf{第一题:}

\textbf{1.1 基于粒子群算法寻优程序}

\textbf{PSOsolution.m}

\begin{verbatim}
%******************基于粒子群算法寻优程序******************
clear;
clc;
format long;

%初始化
c1=2;          %学习因子1
c2=2;          %学习因子2
w=0.7;         %惯性权重
MaxDT=1000;    %最大迭代次数
D=8;           %搜索空间维数(未知数个数)
M=10;          %初始化群体个体数目
eps=10^(-6);   %设置精度

%初始化随机数发生器
try
    rng('shuffle');
catch
    disp('请使用更高版本matlab')
    return
end

%初始化种群的个体
for i=1:M
    x(i,:)=[0.5 0.3E-3 2.6E3 1.246E-3 -2.545E-5 2.908E-7 -2.531E-10 1.022E-12];%初始化位置
    v(i,:)=unifrnd(-1,1)*[1E-2 1E-4 1E2 1E-4 1E-6 1E-8 1E-11 1E-12];%随机初始化速度
end

%计算各个粒子的适应度,并初始化最优解及最优目标函数
try
    for i=1:M
        p(i)=tar(x(i,:));
        y(i,:)=x(i,:);
    end
catch
    disp('请确认tar.m文件在工作目录下')
    return
end
gbest=x(1,:);   %gbest为全局最优
\end{verbatim}

\begin{verbatim}
for i=2:M
    if tar(x(i,:)) < tar(gbest)
        gbest=x(i,:);
    end
end

%进入主要循环,按照公式依次迭代,直到满足精度要求
for t=1:MaxDT
    for i=1:M
        v(i,:)=w*v(i,:)+c1*rand*(y(i,:)-x(i,:))+c2*rand*(gbest-x(i,:));
        x(i,:)=x(i,:)+v(i,:);
        if tar(x(i,:))<p(i)
            p(i)=tar(x(i,:));
            y(i,:)=x(i,:);
        end
        if p(i)<tar(gbest)
            gbest=y(i,:);
        end
    end
end

%显示计算结果
disp(['函数的最优参数为:' num2str(gbest)])
disp(['函数的目标极值为:' num2str(tar(gbest))])
%**************************************

1.2 基于粒子群算法得到的 LI 模型计算程序
PIPSO.m
%****************粒子群算法得到的 LI 模型**************
function PIOLI=PIPSO(I,T0)

%粒子群算法得到的最优参数 x
x=[0.294924595012759,1.942681698553922e-04,3.423643700430695e+03,8.647011587412536e-04,-2.112838388265916e-05,3.168905726180628e-07,-3.281688783234085e-10,3.386895430754581e-12];

%参数传值
eta=x(1);
Ith0=x(2);
Rth=x(3);
a0=x(4);
a1=x(5);
a2=x(6);
a3=x(7);
\end{verbatim}

\begin{verbatim}
a4=x(8);
%电压公式
V=(-0.000133312257892878)*I.^4+0.00452190993557788*I.^3+(-0.0539536512976461)*I.^2+0.340433300607837*I+1.44745780360203;
%光功率计算
P0(1)=0;
I=I./1000;%单位换算
for i=2:length(I)+1
    P0(i)=eta.*(I(i-1)-Ith0-(a0+a1.*(T0+(I(i-1).*V(i-1)-P0(i-1)).*Rth)+a2.*((T0+(I(i-1).*V(i-1)-P0(i-1)).*Rth).^2)+a3.*((T0+(I(i-1).*V(i-1)-P0(i-1)).*Rth).^3)+a4.*((T0+(I(i-1).*V(i-1)-P0(i-1)).*Rth).^4)));
end
PIOLI=P0(2:end).*1000;%单位换算
end
%**************************************************************
1.3 自适应遗传算法目标函数
tar.m
%**************************目标函数**************************
function tar=tar(x)

eta=x(1);
Ith0=x(2);
Rth=x(3);
a0=x(4);
a1=x(5);
a2=x(6);
a3=x(7);
a4=x(8);

%载入数据
try
    load('L-I-20C.mat')%载入数据
catch
    disp('请确认 L-I-20C.mat 在工作目录下')
    return
end

T0=20;%设置环境温度为20
V=U;%导入电压数据
I=I./1000;%单位换算
\end{verbatim}

\begin{verbatim}
%计算功率P与误差值
P0(1)=0;
erro(1)=0;
for i=2:length(P)-37

P0(i)=eta.*(I(i+37)-Ith0-(a0+a1.*(T0+(I(i+37).*V(i+37)-P0(i-1)).*Rth)+a2.*((T0+(I(i+37).*V(i+37)-P0(i-1)).*Rth).^2)+a3.*((T0+(I(i+37).*V(i+37)-P0(i-1)).*Rth).^3)+a4.*((T0+(I(i+37).*V(i+37)-P0(i-1)).*Rth).^4)));
    erro(i)=abs(P0(i)*1000-P(i+37));
end
tar=sum(erro);%求误差值总和
end
%*************************************

1.4 基于自适应遗传算法的寻优程序
GAsolution.m
%******************基于自适应遗传算法的寻优程序******************
clc;
clear;
try
    load('L-I-20C.mat')%载入数据
catch
    disp('请确认 L-I-20C.mat 在工作目录下')
    return
end

fitnessfcn=@tar;%将目标函数设为适应度函数
nvars=8;%函数变量数目设置
lb=[0.2 0.1E-5 2600 1.246E-4 -5.6E-5 1.0E-7 -3.5E-10 3E-12];%函数变量下限值设置
ub=[0.8 0.5E-3 3600 8.246E-3 -0.5E-5 4.9E-7 -1.5E-10 5E-12];%函数变量上限值设置
options = 
    gaoptimset('PopulationSize',100,'EliteCount',10,'CrossoverFraction',0.75,'Generations',500,'StallGenLimit',50,'TolFun',1e-100,'PlotFcns',{@gaplotbestf,@gaplotbestindiv}); %遗传算法求解器设置
[x_best,fval] = ga(fitnessfcn,nvars,[],[],[],lb,ub,[],options);%调用遗传算法求解
%*************************************

1.5 自适应遗传算法得到的 LI 模型
PIGA.m
%******************自适应遗传算法得到的 LI 模型******************
function PIGA=PIGA(I,T0)

%自适应遗传算法得到的最优参数
\end{verbatim}

\begin{verbatim}
eta=0.286309239126479;
Ith0=0.000107466670163652;
Rth=3047.8089963014;
a0=0.000502822540760799;
a1=-2.05864936222586e-05;
a2=3.40063466499428e-07;
a3=-2.13997518117457e-10;
a4=4.94565448705955e-12;

%电压公式
V=(-0.000133312257892878)*I.^4+0.00452190993557788*I.^3+(-0.0539536512976461)*I.^2+0.340433300607837*I+1.44745780360203;

%功率计算
P0(1)=0;
I=I./1000;%单位换算
for i=2:length(I)+1

P0(i)=eta.*(I(i-1)-Ith0-(a0+a1.*(T0+(I(i-1).*V(i-1)-P0(i-1)).*Rth)+a2.*((T0+(I(i-1).*V(i-1)-P0(i-1)).*Rth).^2)+a3.*((T0+(I(i-1).*V(i-1)-P0(i-1)).*Rth).^3)+a4.*((T0+(I(i-1).*V(i-1)-P0(i-1)).*Rth).^4)));
end
PIGA=P0(2:end).*1000;%单位换算
end
%*************************************

1.6 画 0~90 摄氏度等温度 P_I 图,并求出 2mW 临界温度
PITplot_1.m
%******************画等温度 P_I 图,并求出 2mW 临界温度**************
clc;
clear;
I=linspace(0,14,10000);%定义电流 I
for i=10:10:90%画 0~90 等温度 P_I 图
    P=PIGA(I,i);
    plot(I,P);
    hold on;
    axis([0,14,0,3])
end
for i=10:0.1:90%求出临界温度,并在图中标出
    P=PIGA(I,i);
    if all(P<=2)
        Tlim=i;
        display(Tlim)
        hold on
    end
end
\end{verbatim}

\begin{verbatim}
plot(I,P);
hold on
plot([I(1),I(end)],[2,2]);
break
end
end
%**************************************
1.7 画问题1拟合曲线与实验曲线对比图
PI20plot_1.m
%******************问题1拟合曲线与实验曲线对比图**************
clc;
clear;

%载入数据
try
    load('L-I-20C.mat')%载入数据
catch
    disp('请确认 L-I-20C.mat 在工作目录下')
    return
end

plot(I,P,'r')%实验曲线绘制
hold on
plot(I,PIGA(I,20))%拟合曲线绘制
%**************************************
问题二
2.1 基于自适应遗传算法改进模型参数寻优
GAsolution_2.m
%******************基于自适应遗传算法改进模型参数寻优**************
clc;
clear;
%载入数据
try
    load('L-I-20C.mat')
catch
    disp('请确认 L-I-20C.mat 在工作目录下')
    return
end

fitnessfcn=@tar2;%将目标函数设为适应度函数
nvars=14;%函数变量数目设置
lb=[0.2 0.1E-5 2600 1.246E-4 -5.6E-5 1.0E-7 -3.5E-10 3E-12 0.02 0.00111 -919.1 472.7776
\end{verbatim}

\begin{verbatim}
-23389 1.208E-19];%函数变量下限值设置
ub=[0.8 0.5E-3 3600 8.246E-3 -0.5E-5 4.9E-7 -1.5E-10 5E-12 200 11.1 -0.09191 4727776
-53.389 1.208E-15];%函数变量上限值设置
options =
gaoptimset('PopulationSize',100,'EliteCount',10,'CrossoverFraction',0.75,'Generations',500,'StallGenLimit',50,'TolFun',1e-100,'PlotFcns',{@gaplotbestf,@gaplotbestindiv}); %遗传算法求解器设置
[x_best,fval] = ga(fitnessfcn,nvars,[],[],[],lb,ub,[],options);%调用遗传算法求解
%**********************************模型改进后光功率计算函数******************************

2.2 问题2 模型改进后光功率计算函数
PI.m
%**********************************模型改进后光功率计算函数******************************

function PI=PI(I,T0)

x=[0.442334609267341,0.000349701398936035,3165.14906261061,0.000338908173261320,-2.0
1867952814801e-05,1.00000000000000e-07,-2.12039453487688e-10,3.00000000000000e-12,0.0
200000000000000,10.9741437683009,-397.942925899070,583.027600000000,-18697.07323660
89,1.20800000000000e-15];
%模型常数定义
eta=0.286309239126479;%注入效率
Ith0=0.000107466670163652;%公式常数
Rth=3047.8089963014;%VCSEL 热阻抗
a0=0.000502822540760799;%公式多项式系数
a1=-2.05864936222586e-05;%公式多项式系数
a2=3.40063466499428e-07;%公式多项式系数
a3=-2.13997518117457e-10;%公式多项式系数
a4=4.94565448705955e-12;%公式多项式系数
n=x(9);
b1=x(10);
b2=x(11);
c1=x(12);
c2=x(13);
Iss0=x(14);
%电压公式
V1=n.*26.*log(Ith0./(b1.*Iss0.*(T0-b2)+1))+Ith0.*(c1./(T0-c2));

%功率计算
P0(1)=0;
V=(-0.000133312257892878)*I.^4+0.00452190993557788*I.^3+(-0.0539536512976461 )*I.^2+
0.340433300607837*I+1.44745780360203;
I=I./1000;%单位换算
for i=2:length(I)+1
    P0(i)=eta.*(I(i-1)-Ith0-(a0+a1.*(T0+(I(i-1).*V(i-1)-P0(i-1)).*Rth)+...
\end{verbatim}

\begin{verbatim}
a2.*((T0+(I(i-1).*V(i-1)-P0(i-1)).*Rth).^2) + a3.*((T0+(I(i-1).*V(i-1)-P0(i-1)).*Rth).^3) + a4.*((T0 + (I(i-1).*V(i-1)-P0(i-1)).*Rth).^4));
end
PI = P0(2:end).*1000; % 单位换算
end
% ************************************************************

2.3 问题 2 遗传算法目标函数
tar2.m
% ********************************** 问题 2 目标函数 **********************************
function tar2 = tar2(x)

% 参数传值
eta = x(1);
Ith0 = x(2);
Rth = x(3);
a0 = x(4);
a1 = x(5);
a2 = x(6);
a3 = x(7);
a4 = x(8);
n = x(9);
b1 = x(10);
b2 = x(11);
c1 = x(12);
c2 = x(13);
Iss0 = x(14);

try
    run TCul
catch
    disp('请确定 TCul.m 在工作目录下')
    return
end

% 导入数据
try
    load('L-I-20C.mat')
    load('T.mat')
catch
    disp('请确定 L-I-20C.mat 在工作目录下')
    return
end
\end{verbatim}

\begin{verbatim}
T0=20;%设置环境温度为20
T=T1;
V=n.*26.*log(Ith0./(b1.*Iss0.*(T-b2)+1))+I.*(c1./(T-c2));%电压公式
I=I./1000;%单位换算

%计算功率P与误差值
P0(1)=0;
erro(1)=0;
for i=2:length(P)-37

P0(i)=eta.*(I(i+37)-Ith0-(a0+a1.*(T0+(I(i+37).*V(i+37)-P0(i-1)).*Rth)+a2.*((T0+(I(i+37).*V(i+37)-P0(i-1)).*Rth).^2)+a3.*((T0+(I(i+37).*V(i+37)-P0(i-1)).*Rth).^3)+a4.*((T0+(I(i+37).*V(i+37)-P0(i-1)).*Rth).^4)));
    erro(i)=abs(P0(i)*1000-P(i+37));
end
tar2=sum(erro);%求误差值总和
end
%*************************************

2.4器件温度计算函数
T.m
%************************器件温度计算函数************************
function T=T(T0,I,V,P0)
    T=T0+(I.*V-P0).*3292.5568*1E-3;
end
%*************************************

2.5器件温度估计值计算程序
TCul.m
%************************器件温度估计值计算************************
%载入数据
try
    load('L-I-20C.mat')
catch
    disp('请确认L-I-20C.mat在工作目录下')
    return
end

%计算温度
for i=1:length(P)
    T1(i)=T(20,I(i),U(i),P(i));
end
\end{verbatim}

\begin{verbatim}
save('T.mat', 'T1')
%******************************************************
问题 3
3.1 模拟退火算法参数寻优程序
SA.m
%****************模拟退火算法参数寻优程序**************
clear
clc

%生成初始解
var_new = [1E-5 9.6E-9 1.5E-8 1.8E6 4.97E5 3.8E-12 4.7E-8];

%初始化当前解
var_current = var_new;

%初始化最优解
var_best = var_new;

%初始化目标函数值
Tarfcn_new = inf(3,1);
Tarfcn_Current = Tarfcn_new;
Tarfcn_best = Tarfcn_new;

%载入实验数据
try
    load('S21_5.mat')
catch
    disp('请确认 S21_5.mat 文件在工作目录中')
    return
end

try
    rng('shuffle'); %初始化随机数发生器
catch
    disp('请使用 matlab2016 版本')
    return
end

t = 90; %初始温度
tf = 20; %结束温度
a = 0.99; %温度下降比例

while t >= tf %结束条件
\end{verbatim}

\begin{verbatim}
for r=1:1000 %退火次数
    %产生随机扰动,新解的产生
    var_new(1)=var_best(1)+unifrnd(-1,1)*1E-5;
    var_new(2)=var_best(2)+unifrnd(-1,1)*1E-9;
    var_new(3)=var_best(3)+unifrnd(-1,1)*1E-8;
    var_new(4)=var_best(4)+unifrnd(-1,1)*1E6;
    var_new(5)=var_best(5)+unifrnd(-1,1)*1E5;
    var_new(6)=var_best(6)+unifrnd(-1,1)*1E-12;
    var_new(7)=var_best(7)+unifrnd(-1,1)*1E-8;

    %参数上下限设置
    if all(var_new>=0.0001.*[1E-5 9.6E-9 1.5E-8 1.8E6 4.97E5 3.8E-12 4.7E-8])...
        &&all(var_new<=10000.*[1E-5 9.6E-9 1.5E-8 1.8E6 4.97E5 3.8E-12 4.7E-8])
        else
            continue
        end

    % 退火过程

    % 参数传值
    bet=var_new(1);%受激辐射耦合系数
    taun=var_new(2);%载流子复合寿命
    k=var_new(3);%光功率与光子数比例因子
    G0=var_new(4);%增益系数
    N0=var_new(5);%透明载流子数
    taup=var_new(6);%光子寿命
    eps=var_new(7);%增益压缩因子

    %常数定义
    eta=0.286309239126479;%注入效率
    Ith0=0.000107466670163652;%公式常数
    I=7.5/1000;%偏置电流实验值
    Rth=3047.8089963014;%VCSEL 热阻抗
    a0=0.000502822540760799;%公式多项式系数
    a1=-2.05864936222586e-05;%公式多项式系数
    a2=3.40063466499428e-07;%公式多项式系数
    a3=-2.13997518117457e-10;%公式多项式系数
    a4=4.94565448705955e-12;%公式多项式系数
    V=2.464597008;%输入电压实验值
    T0=20;%环境温度实验值
    P0=1.9042/1000;%光功率实验值
    q=1.6.*(1E-19);%电子电量
\end{verbatim}

\begin{verbatim}
%根据模型定义传递函数表达式
try
    %传递函数分子系数矩阵表达式
    num=1./(taup.*taun)+(G0.*Ps(P0,k,G0,Ns(P0,k,taup,G0,N0,eps,taun,bet),N0,taun,q)./...
        (taup.*(k+eps.*Ps(P0,k,G0,Ns(P0,k,taup,G0,N0,eps,taun,bet),N0,taun,q)))-((1-bet).*G0.*Ns(P0,k,...
        taup,G0,N0,eps,taun,bet)-N0)./(taun.*(1+eps.*Ps(P0,k,G0,Ns(P0,k,taup,G0,N0,eps,taun,bet),N0,taun,q))./k).^2);
    %传递函数分母系数矩阵表达式
    den=[1 (1/taup+1/taun+(G0.*Ps(P0,k,G0,Ns(P0,k,taup,G0,N0,eps,taun,bet),N0,taun,q)./...
        (k+eps*Ps(P0,k,G0,Ns(P0,k,taup,G0,N0,eps,taun,bet),N0,taun,q)))-G0.*(Ns(P0,k,taup,G0,N0,eps,...
        taun,bet)-N0)./...
        ((1+eps.*Ps(P0,k,G0,Ns(P0,k,taup,G0,N0,eps,taun,bet),N0,taun,q)./k).^2) (1./(taup.*taun)+...
        (G0.*Ps(P0,k,G0,Ns(P0,k,taup,G0,N0,eps,taun,bet),N0,taun,q)./(taup.*(k+...
        eps.*Ps(P0,k,G0,Ns(P0,k,taup,G0,N0,eps,taun,bet),N0,taun,q)))-((1-bet).*G0.*(Ns(P0,k,taup,G0,...
        N0,eps,taun,bet)-N0)./...
        (taun.*(1+eps.*Ps(P0,k,G0,Ns(P0,k,taup,G0,N0,eps,taun,bet),N0,taun,q)./k).^2))];
catch
    disp('请确认函数文件 Ps.m,Ns.m 在工作目录中')
    return
end

%模型求解
f=0:100000000:249900000000;%实验频率
w=2.*pi.*f;%频率转换成圆频率
g=freqs(num,den,w);%传递函数求解
y=20*log10(abs(g));%单位转换为分贝

%定义目标函数表达式

P0d=q.*Ns(P0,k,taup,G0,N0,eps,taun,bet)./((1-q.*G0.*(Ns(P0,k,taup,G0,N0,eps,taun,bet)-N0)).*taun);%激光器出光功率理论值

Icur=(q./eta).*(Ns(P0,k,taup,G0,N0,eps,taun,bet)./taun+G0.*(Ns(P0,k,taup,G0,N0,eps,taun,bet)-N0).*P0./...
    (k+eps.*P0))+Ith0+(a0+a1.*(T0+(I.*V-P0).*Rth)+a2.*((T0+(I.*V-P0).*Rth).^2)+a3.*((T0+(I.*V-P0).*Rth).^3)+...
    a4.*((T0+(I.*V-P0).*Rth).^4));%驱动电流理论值

Tarfcn_new(1)=sum(abs(y'-S21));
Tarfcn_new(2) =abs(P0d-P0);
Tarfcn_new(3) =abs(Icur-I);
\end{verbatim}

\begin{verbatim}
if
Tarfcn_new(1)<Tarfcn_Current(1) && Tarfcn_new(2)<Tarfcn_Current(2) && Tarfcn_new(3)<Tarfcn_Current(3) && Tarfcn_new(2)<0.1 && Tarfcn_new(3)<0.01%接受准则
    Tarfcn_Current=Tarfcn_new;
    var_current=var_new;
    if
    Tarfcn_new(1)<Tarfcn_best(1) && Tarfcn_new(2)<Tarfcn_best(2) && Tarfcn_new(3)<Tarfcn_best(3)
        %把冷却过程中最好的解保存下来
        Tarfcn_best=Tarfcn_new;
        var_best=var_new;
    end
else
    if rand<exp(-(Tarfcn_new(1)-Tarfcn_Current(1))/t)%代价函数差
        Tarfcn_Current=Tarfcn_new;
        var_current=var_new;
    else
        var_new=var_current;
    end
end
plot(r,Tarfcn_best(1),'*')
hold on
end
t=t*a;%降温
end

disp('最优解为:')
disp(var_best)
disp('目标表达式的最小值等于:')
disp(Tarfcn_best(1))
%****************************************************

3.2 模拟退火算法得到的响应模型计算程序
S21SA.m
%****************模拟退火算法得到的响应模型**************
function S21SA=S21SA(I,T)

%模拟退火算法得到的最优参数x
x=[9.598815849229156e-05,4.357730648247615e-09,1.717433104183007e-08
1.037965791964260e+06,8.696056332886347e+05,1.149761037095653e-11,1.186155933354519
e-07];

%载入实验数据
try
\end{verbatim}

\begin{verbatim}
load('S21_5.mat')
catch
    disp('请确认 S21_5.mat 文件在工作目录中')
    return
end

% 参数传值
bet = x(1); % 受激辐射耦合系数
taun = x(2); % 载流子复合寿命
k = x(3); % 光功率与光子数比例因子
G0 = x(4); % 增益系数
N0 = x(5); % 透明载流子数
taup = x(6); % 光子寿命
eps = x(7); % 增益压缩因子

% 常数定义
q = 1.6 * (1E-19); % 电子电量

% 调用 PI(I,T).m 计算光功率
try
    P0 = PI(I, T) / 1000;
catch
    disp('请确认函数文件 PI.m 在工作目录中, 并正确输入参数 I, T')
    return
end

% 根据模型定义传递函数表达式
try

    % 传递函数分子系数矩阵表达式
    num = 1. / (taup * taun) + (G0 * Ps(P0, k, G0, Ns(P0, k, taup, G0, N0, eps, taun, bet), N0, taun, q)) / ...
        (taup * (k + eps * Ps(P0, k, G0, Ns(P0, k, taup, G0, N0, eps, taun, bet), N0, taun, q))) - ((1 - bet) * ...
        (Ns(P0, k, taup, G0, N0, eps, taun, bet) - N0) / (taun * (1 + eps * Ps(P0, k, G0, Ns(P0, k, taup, G0, N0, eps, taun, bet), N0, taun, q)) / k)^2);

    % 传递函数分母系数矩阵表达式
    den = [1 (1 / taup + 1 / taun + (G0 * Ps(P0, k, G0, Ns(P0, k, taup, G0, N0, eps, taun, bet), N0, taun, q)) / ...
        (k + eps * Ps(P0, k, G0, Ns(P0, k, taup, G0, N0, eps, taun, bet), N0, taun, q)) - G0 * (Ns(P0, k, taup, G0, N0, eps, taun, bet) - N0) / ...
        ((1 + eps * Ps(P0, k, G0, Ns(P0, k, taup, G0, N0, eps, taun, bet), N0, taun, q)) / k)^2) ...
        (1. / (taup * taun) + ...
\end{verbatim}

\begin{verbatim}
(G0.*Ps(P0,k,G0,Ns(P0,k,taup,G0,N0,eps,taun,bet),N0,taun,q)./(taup.*(k+...
eps.*Ps(P0,k,G0,Ns(P0,k,taup,G0,N0,eps,taun,bet),N0,taun,q)))-((1-bet).*G0.*(Ns(P0,k,taup,G0,...
N0,eps,taun,bet)-N0)./...
(taun.*(1+eps.*Ps(P0,k,G0,Ns(P0,k,taup,G0,N0,eps,taun,bet),N0,taun,q)./k).^2))];

catch
    disp('请确认函数文件 Ps.m,Ns.m 在工作目录中')
    return
end

%计算频率 0-30GHz 的响应特性
f=0:10000000:300000000000;
w=2.*pi.*f;
g=freqs(num,den,w);
y=20*log10(abs(g));
plot(f,y)
hold on;

%画出 0-24.99GHz 的实验响应特性曲线
fl=0:10000000:24990000000;
plot(fl,S21)

%函数值传递
S21SA=y;
end
%*************************************

3.3 稳态光功率计算函数
Ps.m
%******************稳态光功率计算函数******************
function Ps=Ps(P0,k,G0,Ns,N0,taun,q)
    Ps=k.*(((P0./q)-(Ns./taun))./(G0.*(Ns-N0)));
end
%*************************************

3.4 稳态载流子浓度计算函数
Ns.m
%******************稳态载流子浓度计算函数******************
function Ns=Ns(P0,k,taup,G0,N0,eps,taun,bet)
    Ns=(P0./(k.*taup)+G0.*N0.*P0./(k+eps.*P0))./((bet./taun)+(G0.*P0/(k+eps.*P0)));
end
%*************************************
\end{verbatim}

\section*{3.5 多目标优化遗传算法 NSGA 求解程序}
NSGA.m

\begin{verbatim}
%**************多目标优化遗传算法 NSGA 求解程序**************
clc;
clear;

%载入实验数据
try
    load('S21_5.mat')
catch
    disp('请确认 S21_5.mat 文件在工作目录中')
    return
end

try
    FitnessFunction=@Tar3NSGA;%将目标函数设为适应度函数
catch
    disp('请确认 Tar3mul.m 文件在工作目录中')
    return
end

numberOfVariables=7;%函数变量数目设置
lb=[0.001 0.001 0.001 0.001 0.001 0.001 0.001].*[1E-5 9.6E-9 1.5E-8 1.8E6 4.97E5 3.8E-12 4.7E-8];%函数变量下限值设置
ub=[1000 1000 1000 1000 1000 1000 1000].*[1E-5 9.6E-9 1.5E-8 1.8E6 4.97E5 3.8E-12 4.7E-8];%函数变量上限值设置

options = optimoptions(@gamultiobj,'PlotFcn',@gaplotpareto);%遗传算法求解器设置
[x,Fval,exitFlag,Output] =
gamultiobj(FitnessFunction,numberOfVariables,[],[],[],lb,ub,options);%调用遗传算法求解
fprintf('The tar_best: %d\n', min(Fval(:,1)));%输出优化后目标函数1的最优值
%**************************************************************
\end{verbatim}

\section*{3.6 多目标优化遗传算法 NSGA 目标函数}
Tar3NSGA.m

\begin{verbatim}
%**************多目标优化遗传算法 NSGA 目标函数**************
function Tar3NSGA = Tar3NSGA(x)

Tar3NSGA = zeros(3,1);%目标函数矢量内存预分配

%载入实验数据
try
    load('S21_5.mat')
catch
\end{verbatim}

\begin{verbatim}
disp('请确认 S21_5.mat 文件在工作目录中')
return
end

%参数传值
try
    bet=x(1);%受激辐射耦合系数
    taun=x(2);%载流子复合寿命
    k=x(3);%光功率与光子数比例因子
    G0=x(4);%增益系数
    N0=x(5);%透明载流子数
    taup=x(6);%光子寿命
    eps=x(7);%增益压缩因子
catch
    disp('请正确输入参数,输入的参数 x 应为 1 x 7 行向量')
    return
end

%常数定义
eta=0.286309239126479;%注入效率
Ith0=0.000107466670163652;%公式常数
I=7.5/1000;%偏置电流实验值
Rth=3047.8089963014;%VCSEL 热阻抗
a0=0.000502822540760799;%公式多项式系数
a1=-2.05864936222586e-05;%公式多项式系数
a2=3.40063466499428e-07;%公式多项式系数
a3=-2.13997518117457e-10;%公式多项式系数
a4=4.94565448705955e-12;%公式多项式系数
V=2.464597008;%输入电压实验值
T0=20;%环境温度实验值
P0=1.9042/1000;%光功率实验值
q=1.6.*(1E-19);%电子电量

%根据模型定义传递函数表达式
try

    %传递函数分子系数矩阵表达式
    num=1./(taup.*taun)+(G0.*Ps(P0,k,G0,Ns(P0,k,taup,G0,N0,eps,taun,bet),N0,taun,q)./...
    (taup.*(k+eps.*Ps(P0,k,G0,Ns(P0,k,taup,G0,N0,eps,taun,bet),N0,taun,q)))-((1-bet).*...
    (Ns(P0,k,taup,G0,N0,eps,taun,bet)-N0)./(taun.*(1+eps.*Ps(P0,k,G0,Ns(P0,k,taup,G0,N0,eps,taun,...
    bet),N0,taun,q)./k).^2));
\end{verbatim}

\begin{verbatim}
%传递函数分母系数矩阵表达式
den=[1 (1/taup+1/taun+(G0.*Ps(P0,k,G0,Ns(P0,k,taup,G0,N0,eps,taun,bet),N0,taun,q)./...
    (k+eps*Ps(P0,k,G0,Ns(P0,k,taup,G0,N0,eps,taun,bet),N0,taun,q)))-G0.*(Ns(P0,k,taup,G0,N0,eps,taun,bet)-N0)./...
    ((1+eps.*Ps(P0,k,G0,Ns(P0,k,taup,G0,N0,eps,taun,bet),N0,taun,q))./k).^2))
    (1./(taup.*taun)+...
    (G0.*Ps(P0,k,G0,Ns(P0,k,taup,G0,N0,eps,taun,bet),N0,taun,q)./(taup.*(k+...
    eps.*Ps(P0,k,G0,Ns(P0,k,taup,G0,N0,eps,taun,bet),N0,taun,q)))-((1-bet).*G0.*(Ns(P0,k,taup,G0,N0,eps,taun,bet)-N0)./...
    (taun.*(1+eps.*Ps(P0,k,G0,Ns(P0,k,taup,G0,N0,eps,taun,bet),N0,taun,q)./k).^2))];

catch
    disp('请确认函数文件 Ps.m,Ns.m 在工作目录中')
    return
end

%模型求解
f=0:100000000:249900000000;%实验频率
w=2.*pi.*f;%频率转换成圆频率
g=freqs(num,den,w);%传递函数求解
y=20*log10(abs(g));%单位转换为分贝

%定义目标函数表达式
P0d=q.*Ns(P0,k,taup,G0,N0,eps,taun,bet)./((1-q.*G0.*(Ns(P0,k,taup,G0,N0,eps,taun,bet)-N0)).*taun);%激光器出光功率理论值
Icur=(q./eta).*(Ns(P0,k,taup,G0,N0,eps,taun,bet)./taun+G0.*(Ns(P0,k,taup,G0,N0,eps,taun,bet)-N0).*P0./...
    (k+eps.*P0))+Ith0+(a0+a1.*(T0+(I.*V-P0).*Rth)+a2.*((T0+(I.*V-P0).*Rth).^2)+a3.*((T0+(I.*V-P0).*Rth).^3)+...
    a4.*((T0+(I.*V-P0).*Rth).^4));%驱动电流理论值

%目标函数值计算
Tar3NSGA(1)=sum(abs(y'-S21));%目标 1,小信号响应理论值与小信号响应实验值差值绝对值总和和尽量小
Tar3NSGA(2)=abs(P0d-P0);%目标 2,激光器出光功率理论值与激光器出光功率实验值差值绝对值尽量小
Tar3NSGA(3)=abs(Icur-I);%目标 3,驱动电流理论值与驱动电流实验值差值绝对值尽量小
%************************************************************
3.7 多目标优化遗传算法得到的响应模型
\end{verbatim}

\begin{verbatim}
S21GA.m
%**************多目标优化遗传算法得到的响应模型**************
function S21GA=S21GA(I,T)

%多目标优化遗传算法 NSGA 得到的最优参数 x
x=[0.00959019696252237,1.25982951854295e-08,1.45091943723630e-05,...
   118221540.852524,104244612.175646,2.39471750126508e-11,4.08904471746219e-05];

%载入实验数据
try
    load('S21_5.mat')
catch
    disp('请确认 S21_5.mat 文件在工作目录中')
    return
end

%参数传值
bet=x(1);%受激辐射耦合系数
taun=x(2);%载流子复合寿命
k=x(3);%光功率与光子数比例因子
G0=x(4);%增益系数
N0=x(5);%透明载流子数
taup=x(6);%光子寿命
eps=x(7);%增益压缩因子

%常数定义
q=1.6.*(1E-19);%电子电量

%调用 PI(I,T).m 计算光功率
try
    P0=PI(I,T)/1000;
catch
    disp('请确认函数文件 PI.m 在工作目录中,并正确输入参数 I,T')
    return
end

%根据模型定义传递函数表达式
try

    %传递函数分子系数矩阵表达式
    num=1./(taup.*taun)+(G0.*Ps(P0,k,G0,Ns(P0,k,taup,G0,N0,eps,taun,bet),N0,taun,q)./...
    (taup.*(k+eps.*Ps(P0,k,G0,Ns(P0,k,taup,G0,N0,eps,taun,bet),N0,taun,q)))-((1-bet).*...
    (Ns(P0,k,taup,G0,N0,eps,taun,bet)-N0)./(taun.*(1+eps.*Ps(P0,k,G0,Ns(P0,k,taup,G0,N0,eps,taun,
\end{verbatim}

\begin{verbatim}
bet),N0,taun,q)./k).^2));

%传递函数分母系数矩阵表达式
den=[1 (1/taup+1/taun+(G0.*Ps(P0,k,G0,Ns(P0,k,taup,G0,N0,eps,taun,bet),N0,taun,q)./...
(k+eps*Ps(P0,k,G0,Ns(P0,k,taup,G0,N0,eps,taun,bet),N0,taun,q)))-G0.*(Ns(P0,k,taup,G0,N0,eps,...
taun,bet)-N0)./...
((1+eps.*Ps(P0,k,G0,Ns(P0,k,taup,G0,N0,eps,taun,bet),N0,taun,q))./k).^2))
(1./(taup.*taun)+...
(G0.*Ps(P0,k,G0,Ns(P0,k,taup,G0,N0,eps,taun,bet),N0,taun,q)./(taup.*(k+...
eps.*Ps(P0,k,G0,Ns(P0,k,taup,G0,N0,eps,taun,bet),N0,taun,q)))-((1-bet).*G0.*(Ns(P0,k,taup,G0,...
N0,eps,taun,bet)-N0)./...
(taun.*(1+eps.*Ps(P0,k,G0,Ns(P0,k,taup,G0,N0,eps,taun,bet),N0,taun,q)./k).^2)))];
catch
    disp('请确认函数文件 Ps.m,Ns.m 在工作目录中')
    return
end

%计算频率 0-30GHz 的响应特性
f=0:100000000:300000000000;
w=2.*pi.*f;
g=freqs(num,den,w);
y=20*log10(abs(g));
plot(f,y)
hold on;

%画出 0-24.99GHz 的实验响应特性曲线
f1=0:10000000:24990000000;
plot(f1,S21)

%函数值传递
S21GA=y;
End
%****************************************************

第四题:
4.1 针对第四题所建模型的模拟退火算法 SA 目标函数
Tar4SA.m
%****************模拟退火算法 SA 目标函数**************
function Tar4SA=Tar4SA(x)

%载入实验数据
\end{verbatim}

\begin{verbatim}
try
    load('S21_5.mat')
catch
    disp('请确认 S21_5.mat 文件在工作目录中')
    return
end

% 参数传值
try
    D = x(1);
    taun = x(2);
    k = x(3);
    arfa = x(4);
    vg = x(5);
    taup = x(6);
    sgm = x(7);
catch
    disp('请正确输入参数,输入的参数 x 应为 1 x 7 行向量')
    return
end

% 常数定义
eta = 0.286309239126479; % 注入效率
Ith0 = 0.000107466670163652; % 公式常数
I = 7.5/1000; % 偏置电流实验值
Rth = 3047.8089963014; % VCSEL 热阻抗
a0 = 0.000502822540760799; % 公式多项式系数
a1 = -2.05864936222586e-05; % 公式多项式系数
a2 = 3.40063466499428e-07; % 公式多项式系数
a3 = -2.13997518117457e-10; % 公式多项式系数
a4 = 4.94565448705955e-12; % 公式多项式系数
V = 2.464597008; % 输入电压实验值
T0 = 20; % 环境温度实验值
P0 = 1.9042/1000; % 光功率实验值
q = 1.6*(1E-19); % 电子电量

% 根据模型定义传递函数表达式
try

% 传递函数分子系数矩阵表达式

num = D.*((q./eta).*(Ns(P0,k,taup,arfa,vg,sgm,taun,D)./taun + arfa.*(Ns(P0,k,taup,arfa,vg,sgm,taun, D)-vg)).*P0./...
\end{verbatim}

\begin{verbatim}
(k+sgm.*P0))+Ith0+(a0+a1.*(T0+(I.*V-P0.*Rth)+a2.*((T0+(I.*V-P0).*Rth).^2)+a3.*((T0+(I.*V-P0)).*Rth).^3)...
+a4.*((T0+(I.*V-P0).*Rth).^4))).^(0.5);

% 传递函数分母系数矩阵表达式
den=[vg.*arfa.*Ns(P0,k,taup,arfa,vg,sgm,taun,D)./taup
k.*D.^2.*(Ns(P0,k,taup,arfa,vg,sgm,taun,D)./taun+arfa.*...
(Ns(P0,k,taup,arfa,vg,sgm,taun,D)-vg).*P0./(k+sgm.*P0))+Ith0+(a0+a1.*(T0+(I.*V-P0).*Rth)+a2.*
((T0+(I.*V-P0).*Rth).^2)...
+a3.*((T0+(I.*V-P0).*Rth).^3)+a4.*((T0+(I.*V-P0).*Rth).^4))).^(0.5)
sgm.*vg.*arfa.*Ns(P0,k,taup,arfa,vg,sgm,taun,D)./taun+sgm...
./(taun.*(Ns(P0,k,taup,arfa,vg,sgm,taun,D)).*(1-(taun/taup)))];

catch
    disp('请确认函数文件 Ps.m,Ns.m 在工作目录中')
    return
end

% 模型求解
f=0:100000000:249900000000;%实验频率
w=2.*pi.*f;%频率转换成圆频率
g=freqs(num,den,w);%传递函数求解
y=20*log10(abs(g));%单位转换为分贝

% 目标函数值计算
Tar4SA=sum(abs(y'-S21));%小信号响应理论值与小信号响应实验值差值绝对值总和和尽量小
%************************************************************

4.2 针对问题 4 的模拟退火算法寻优程序
SA4.m
%******************问题 4 模拟退火算法寻优程序******************
clear
clc

% 生成初始解
var_new=[1E-5 9.6E-9 1.5E-8 1.8E6 4.97E5 3.8E-12 4.7E-8];

% 初始化当前解
var_current=var_new;

% 初始化最优解
var_best=var_new;
\end{verbatim}

\begin{verbatim}
% 初始化目标函数值
Tarfcn_new = inf;
Tarfcn_Current = Tarfcn_new;
Tarfcn_best = Tarfcn_new;

% 载入实验数据
try
    load('S21_5.mat')
catch
    disp('请确认 S21_5.mat 文件在工作目录中')
    return
end

try
    rng('shuffle'); % 初始化随机数发生器
catch
    disp('请使用 matlab2016 版本')
    return
end

t = 90; % 初始温度
tf = 20; % 结束温度
a = 0.99; % 温度下降比例

while t >= tf % 结束条件
    for r = 1:1000 % 退火次数

        % 产生随机扰动,新解的产生
        var_new(1) = var_best(1) + unifrnd(-1, 1) * 1E-5;
        var_new(2) = var_best(2) + unifrnd(-1, 1) * 1E-9;
        var_new(3) = var_best(3) + unifrnd(-1, 1) * 1E-8;
        var_new(4) = var_best(4) + unifrnd(-1, 1) * 1E6;
        var_new(5) = var_best(5) + unifrnd(-1, 1) * 1E5;
        var_new(6) = var_best(6) + unifrnd(-1, 1) * 1E-12;
        var_new(7) = var_best(7) + unifrnd(-1, 1) * 1E-8;

        % 参数上下限设置
        if all(var_new >= 0.0001 * [1E-5 9.6E-9 1.5E-8 1.8E6 4.97E5 3.8E-12 4.7E-8]) ...
            && all(var_new <= 10000 * [1E-5 9.6E-9 1.5E-8 1.8E6 4.97E5 3.8E-12 4.7E-8])
            else
                continue
        end
    end
end
\end{verbatim}

\begin{verbatim}
% 退火过程

try
    Tarfcn_new = Tar4SA(var_new); % 调用目标函数
catch
    disp('请确认 Tar4SA.m 文件在工作目录中')
    return
end

if Tarfcn_new < Tarfcn_Current % 接受准则
    Tarfcn_Current = Tarfcn_new;
    var_current = var_new;
    if Tarfcn_new < Tarfcn_best % 把冷却过程中最好的解保存下来
        Tarfcn_best = Tarfcn_new;
        var_best = var_new;
    end
else
    if rand < exp(-(Tarfcn_new - Tarfcn_Current) / t) % 代价函数差
        Tarfcn_Current = Tarfcn_new;
        var_current = var_new;
    else
        var_new = var_current;
    end
end
plot(r, Tarfcn_best, '*')
hold on
end
t = t * a; % 降温
end

disp('最优解为:')
disp(var_best)
disp('目标表达式的最小值等于:')
disp(Tarfcn_best)
% ******************************
\end{verbatim}