\begin{center}
\textbf{第九届“华为杯”} \\
\textbf{全国研究生数学建模竞赛}
\end{center}

\begin{table}[h]
\centering
\begin{tabular}{l l}
学校 & 中国石油大学(华东) \\
\hline
参赛队号 & 10425008 \\
\hline
队员姓名 & 王舒华 \\
& 姚征 \\
& 鲁明晶 \\
\hline
\end{tabular}
\end{table}

\begin{flushright}
参赛密码 \underline{\hspace{3cm}} \\
(由组委会填写)
\end{flushright}

\begin{center}
\textbf{参赛密码} \underline{\hspace{3cm}} \\
(由组委会填写)
\end{center}

\begin{center}
\textbf{第九届“华为杯”全国研究生数学建模竞赛}
\end{center}

\begin{center}
\includegraphics[width=0.3\textwidth]{image1.png} \quad
\includegraphics[width=0.3\textwidth]{image2.png} \quad
\includegraphics[width=0.3\textwidth]{image3.png}
\end{center}

\begin{center}
\textbf{题目} \quad 有杆抽油系统的数学建模及诊断
\end{center}

\begin{center}
\textbf{摘 \quad 要:}
\end{center}

本文围绕有杆抽油系统的数学问题,建立了考虑动载荷影响的光杆悬点运动规律数学模型,分析了动载荷对悬点运动规律的影响;利用 Gibbs 一维振动数学模型将悬点示功图转化为泵功图,绘制了油井的地面示功图和井下泵功图;采用 Origin 软件对泵功图的图形进行分析,建立了两个估算油井产液量的数学模型;结合灰色理论和模式识别技术,建立了基于灰色理论的抽油井泵功图诊断模型;通过对 Gibbs 模型的分析和有杆抽油系统的受力分析,建立了考虑抽油杆柱振动和环空流体振动的二维振动数学模型;通过对抽油杆柱阻尼系统的分析,推导了考虑接箍摩擦的粘滞摩擦阻尼系数的计算公式。

本文的创新点在于:第一,对现有的悬点运动规律数学模型进行评价和改进,推导了考虑动载荷影响的光杆悬点运动规律数学模型;第二,建立了基于灰色理论的抽油井泵功图诊断模型;第三,在考虑油管环空内液体振动的条件下,推导了有杆抽油井的二维振动数学模型;第四,对 Gibbs 阻尼系数的计算方法进行了改进,考虑了接箍与环空流体之间的粘滞摩擦阻力对抽油杆柱振动模型的影响。

问题一,本文通过对现有的简谐运动悬点运动模型、曲柄滑块悬点运动模型和四连杆机构悬点运动模型的评价与分析,得出在悬点上死点的位置模型位移与实际位移数据相差最大,而在悬点下死点的位置模型计算位移与悬点实际位移基本吻合,这说明在悬点逐渐达到上死点的位置时,有杆抽油系统的动载荷对悬点的位移影响最大。因此本文针对此情况,建立了考虑动载荷影响的悬点运动规律模型。

问题二,1996 年,Gibbs 给出了悬点示功图转化为地下泵功图的一维粘滞阻尼波动方程模型,本文通过对该模型的详细分析的求解,给出了悬点示功图转化为地下泵功图的详细计算过程,包括:原始数据处理、边界条件和初始条件处理。

以及求解算法。绘出了地面示功图和井下泵功图,研究发现泵功图相对于地面示功图在坐标轴上表现为向下平移,消除了抽油杆柱的变形、粘滞阻尼、振动和惯性载荷等对示功图的影响,使得泵功图真实地反映了泵的工作状况。

问题三,通过对泵功图的研究可以得到有杆抽油系统的工作状况,其中油井产液量的估算就是其应用之一。本文采用 Origin 软件,通过对泵功图的图形研究,建立了两种估算油井产液量的数学模型。并进行了模型的对比分析,发现有效冲程数学模型能够较为准确的反映油井的产液量。同时通过文献调研研究,建立了基于灰色理论的抽油井泵功图诊断模型。

问题四,本文通过研究发现 Gibbs 模型的不足:Gibbs 一维模型只研究了抽油杆柱的纵向振动问题,忽略了油管柱和油管内液柱的振动,液柱的影响且仅限于粘滞阻力。针对于 Gibbs 模型的不足,本文建立了同时考虑抽油杆振动和油管内液柱振动的二维振动模型,分析了抽油杆柱的阻尼影响因素,给出了考虑接箍粘滞阻力的阻尼系数计算方法。

\section*{目录}
\begin{itemize}
    \item[] 一、问题重述 \dotfill 4
    \item[] 二、模型假设 \dotfill 5
    \item[] 三、符号说明 \dotfill 5
    \item[] 四、模型建立与求解 \dotfill 7
    \begin{itemize}
        \item[] 4.1 问题一模型的建立与求解 \dotfill 7
        \begin{itemize}
            \item[] 4.1.1 问题分析 \dotfill 7
            \item[] 4.1.2 模型建立 \dotfill 7
            \item[] 4.1.3 模型求解与分析 \dotfill 11
            \item[] 4.1.4 悬点运动模型的改进 \dotfill 13
        \end{itemize}
        \item[] 4.2 问题二模型的建立与求解 \dotfill 13
        \begin{itemize}
            \item[] 4.2.1 问题分析 \dotfill 13
            \item[] 4.2.2 Gibbs 模型计算泵功图主要步骤 \dotfill 14
            \item[] 4.2.3 泵功图的计算及分析 \dotfill 17
        \end{itemize}
        \item[] 4.3 问题三模型的建立与求解 \dotfill 19
        \begin{itemize}
            \item[] 4.3.1 问题分析 \dotfill 19
            \item[] 4.3.2 模型建立 \dotfill 19
            \item[] 4.3.3 模型求解及分析 \dotfill 19
            \item[] 4.3.4 泵功图诊断模型建立 \dotfill 24
        \end{itemize}
        \item[] 4.4 问题四模型的建立与求解 \dotfill 28
        \begin{itemize}
            \item[] 4.4.1 Gibbs 模型问题分析 \dotfill 28
            \item[] 4.4.2 有杆抽油系统模型假设 \dotfill 29
            \item[] 4.4.3 有杆抽油系统模型建立 \dotfill 29
            \item[] 4.4.4 阻尼系数模型问题分析 \dotfill 33
            \item[] 4.4.5 摩擦功阻尼系数模型假设 \dotfill 33
            \item[] 4.4.6 摩擦功阻尼系数模型建立 \dotfill 33
            \item[] 4.4.7 考虑接箍摩擦的阻尼系数模型 \dotfill 37
        \end{itemize}
    \end{itemize}
    \item[] 五、模型的评价及改进 \dotfill 38
    \item[] 六、附录 \dotfill 39
    \item[] 参考文献 \dotfill 44
\end{itemize}

\section*{一、问题重述}

目前,开采原油广泛使用的是有杆抽油系统。电机旋转运动转化为抽油杆上下往返周期运动,带动设置在杆下端的泵的两个阀的相继开闭,从而将地下上千米深处蕴藏的原油抽到地面上来 \cite{ref1}。

描述抽油杆中任意水平截面处基本信息的通用方法是示功图:它是该点随时间 $t$ 而变化的荷载(合力,向下为正)数据作为纵坐标,以该点垂直方向上随时间 $t$ 而变化的位置相对于 $t=0$ 时刻该点位置的位移数据作为横坐标构成的图形。函数关系表现为位移-荷载关于时间 $t$ 的参数方程。一个冲程中示功图是一条封闭的曲线。抽油杆上端点称为悬点,该点处的示功图称为悬点示功图。

泵中柱塞处的示功图称为泵功图。因为受到诸多因素的影响,在同一时刻 $t$,悬点处的受力(荷载)与柱塞的受力是不相同的;同样,在同一时刻 $t$,悬点处的相对位移与柱塞的相对位移也不相同。因此悬点示功图与泵功图是不同的。示功图包含了很多信息,其中就有有效冲程。工程上一般根据示功图形状与理论示功图进行对比来判断抽油机工作状态。

通过悬点示功图可以初步诊断该井的工作状况,如产量、气体影响、阀门漏液等等。要精确诊断油井的工作状况,最好采用泵功图。然而,泵在地下深处,使用仪器测试其数据实现困难大、成本高。因此,通过数学建模,把悬点示功图转化为杆上任意点的示功图(统称为地下示功图)并最终确定泵功图,以准确诊断该井的工作状况,是一个很有价值的实际问题。

本题主要解决有杆抽油机系统如下几个问题:

(1) 光杆悬点运动规律研究

电机旋转运动通过四连杆机构转变为抽油杆的垂直运动。假设驴头外轮廓线为部分圆弧、电机匀速运动,悬点 E 下只挂光杆。根据已知数据和参数,求出悬点 E 的一个冲程的运动规律:位移函数、速度函数、加速度函数。并与给定悬点位移数据进行比较。

(2) 泵功图计算

1966 年,Gibbs 给出了悬点示功图转化为地下示功图的模型。使用 Gibbs 模型,给出由悬点示功图转化为泵功图的详细计算过程,其计算过程包括:原始数据的处理、边界条件、初始条件、求解算法等;利用一级杆和三级杆两油井的参数以及悬点的位移和载荷数据分别计算出这两口油井的泵功图数据;并分别绘制出两油井的悬点示功图和泵功图。

(3) 泵功图的应用

建立 2 个不同的由泵功图来估计油井产量的模型,其中至少一个要利用“有效冲程”;并利用上述两油田数据分别估算两口油井一天(24 小时)的产液量。同时建立模型或算法,以计算机自动判别某泵功图数据是否属于泵内有气体的情况。并对一级杆和三级杆对应的两口油井的泵功图进行计算机诊断。

(4) 深入研究的问题

首先,对 Gibbs 模型进行原理分析,发现它的不足。在合理的假设下,重新建立抽油系统模型或对现有模型进行改进;并给出由悬点示功图转化为泵功图的详细计算过程,包括:原始数据的处理、边界条件、初始条件、求解算法;利用已知两油井数据重新进行计算;对计算结果与问题 2 的计算结果进行比较,分析模型的优缺点。

其次,Gibbs 模型在数学上可简化为带阻尼的波动方程。其中 $a$ 为已知常数,

$c$ 称为阻尼系数,鉴于大多数的阻尼系数公式是作了诸多假设后推出的,并不能完整地反应实际情况。研究并详细给出计算 $c$ 的理论推导过程并求出 $c$。

\section{模型假设}

(1) 常规游梁式抽油机的驴头外轮廓线为部分圆弧;

(2) 游梁式抽油机的电机匀速运动;

(3) 将抽油杆简化为均质弹性杆,且光杆与第一级抽油杆粗细相同;

(4) 油管与抽油杆环空中仅为液相,忽略气相的影响;

(5) 抽油杆柱在油管内的横向振动和变形很小,忽略不计;

(6) 抽油杆柱轴线与油管轴线重合,忽略抽油杆与油管壁的接触;

(7) 油管锚定,忽略油管柱的振动;

\section{符号说明}

\begin{tabular}{ll}
\hline
\hline 符号 & 符号说明 \\
\hline $\varphi$ & 曲柄转角,rad \\
$\omega$ & 曲柄角速度,rad/s \\
$\rho_{or}$ & 地层原油密度,kg/m$^3$ \\
$\rho_{os}$ & 地面原油密度,kg/m$^3$ \\
$\rho_{m}$ & 油井排出混合液体的密度,kg/m$^3$ \\
$\rho_{w}$ & 水的密度,1000kg/m$^3$ \\
$\rho_{r}$ & 抽油杆柱的密度,本文中为 8456kg/m$^3$ \\
$B_{o}$ & 原油体积系数,1.025 \\
$B_{w}$ & 地层水体积系数,一般取 1.0 \\
$B_{m}$ & 油井排出混合液体体积系数 \\
$f$ & 油井含水率 \\
$L$ & 抽油杆柱的长度,m \\
$L(t)$ & 测量的悬点载荷函数,kN \\
$L_{p}$ & 泵深,即总的抽油杆长度 $L_{p}=L_{1}+L_{2}+\cdots+L_{m}$,m \\
$A_{r}$ & 抽油杆柱的横截面积,m$^2$ \\
$A_{p}$ & 抽油泵柱塞截面积,$A_{p}=\pi d_{p}^{2}/4$,m$^2$ \\
$g$ & 重量加速度,9.81m/s$^2$ \\
$c$ & 阻尼系数,s$^{-1}$ \\
$a$ & 声速,一般取 4983.416m/s \\
\hline
\hline
\end{tabular}

\begin{tabular}{l l}
$c_{d}$ & 无量纲阻尼系数,查图版(2-1) \\
$a_{i}$ & 无量纲阻尼系数拟合多项式中的系数,查表(2-1) \\
$\bar{v}_{r}$ & 光杆平均速度,m/s \\
$n$ & 傅里叶级数所取项数,总项数$\bar{n}=100$ \\
$E$ & 钢材弹性模量,$2.1\times10^{11}$Pa \\
$u(x_{i},t)$ & 在$x_{i}$断面不同时间$t$的位移函数 \\
$F(x_{i},t)$ & 在$x_{i}$断面不同时间$t$的载荷函数 \\
$d_{p}$ & 泵径,m \\
$s_{pe}$ & 柱塞有效冲程,m \\
$n_{c}$ & 冲次,min$^{-1}$ \\
$Q$ & 泵的每天排量,m$^{3}$/d \\
$Q_{t}$ & 泵的每天排量,$t$ \\
$\varepsilon$ & 允许的计算量误差,一般取0.1 \\
$F_{r}$ & 抽油杆柱单位长度上的受到的摩擦力,N/m; \\
$A_{t}$ & 油管内径截面积,m$^{2}$; \\
$F_{f}$ & 环空中流体单位长度上受到的摩擦力,N/m; \\
$P_{f}$ & 流体压力,Pa; \\
$v_{f}$ & 流体的流速,m/s。
\end{tabular}

\section*{四、模型建立与求解}

\subsection{4.1 问题一模型的建立与求解}

\subsubsection{4.1.1 问题分析}

抽油机作为整个抽油泵系统最重要的地面设备之一,其悬点的运动规律在抽油泵系统仿真中作为上边界条件,是不可或缺的。因而,进行抽油泵系统仿真的模型必须建立抽油机仿真模型。本题为常规型游梁式抽油机,常规型游梁式抽油机主要工作原理是电机旋转运动通过四连杆机构转变为抽油杆的垂直运动。它是以游梁中轴承与曲柄中心的连线为固定杆,曲柄、连杆和游梁后臂为三个活动杆,构成为四连杆机构。本题要求推导出悬点E的一个冲程的运动规律:位移函数、速度函数、加速度函数。

对常规游梁抽油机进行悬点的运动分析,首先要建立抽油机的几何模型,利用运动学的知识进行仿真计算。本文通过一定的模型假设,建立了三种不同的悬点运动规律模型,并进行了模型分析。

\subsubsection{4.1.2 模型建立}

\subsubsection*{(1) 简谐运动的悬点运动模型}

\begin{figure}[h]
    \centering
    \includegraphics[width=0.8\textwidth]{image.png}
    \caption{游梁式抽油机四连杆机构示意图}
\end{figure}

本模型假设曲柄的半径比连杆长度和游梁长度小得多,即 $r/l \approx 0$ 和 $r/b \approx 0$。此时,游梁和连杆的连接点B的运动可看作简谐运动,即认为B点的运动规律和D点做圆周运动时在垂直中心线上的投影(C点)的运动规律相同。则B点经过$t$时间(曲柄转过$\varphi$角)时的位移为\cite{ref1}:

\begin{equation}
S_B = r(1 - \cos \varphi) = r(1 - \cos \omega t)
\tag{1-1}
\end{equation}

以悬点 E 的下死点为坐标零点,向上为坐标正方向,则悬点 E 的位移函数:
\begin{equation}
s_{A}=s_{E}=\frac{a}{b}r(1-\cos \omega t)
\tag{1-2}
\end{equation}
则悬点 E 的速度函数:
\begin{equation}
v_{E}=\frac{ds_{E}}{dt}=\frac{a}{b}\omega r\sin \omega t
\tag{1-3}
\end{equation}
悬点 E 点的加速度函数:
\begin{equation}
a_{E}=\frac{dv_{E}}{dt}=\frac{a}{b}\omega^{2}r\cos \omega t
\tag{1-4}
\end{equation}

图 1-2 是由式(1-2)、(1-3)、(1-4)根据附件 1 中一个周期 $T=7.89\,\text{s}$ 的时间计算得到的悬点 E 位移、速度、和加速度随时间 $t$ 的变化曲线。

\begin{figure}[h]
\centering
\includegraphics[width=\textwidth]{image.png}
\caption{简谐运动模型悬点位移、速度及加速度曲线}
\end{figure}

从图中可以看出,悬点 E 在垂直方向作周期性的上下运动,运行到上死点和下死点时,速度为零,而加速度最大,在上下冲程的重点($\varphi=90^{\circ}$ 和 $270^{\circ}$),加速度为零,速度的绝对值达到最大。

\subsubsection{(2) 曲柄滑块运动悬点运动模型}

上述简化为简谐运动的做法,由于实际抽油机的曲柄半径与连杆和游梁后臂长之比不接近与零,而是有一定的值,特别是采用大冲程时,误差会更大。该简化方法就是取曲柄半径与连杆长度之比为有限值,即 $0<r/l<0.25$,并把 B 点的运动看作直线运动,则可把抽油机的运动简化为图 1-3 所示的曲柄滑块运动,其悬点 E 的运动规律计算公式如下公式所示:

悬点 E 的位移函数:
\begin{equation}
s_{E}=r\left(1-\cos \omega t+\frac{\lambda}{2}\sin^{2}\omega t\right)\frac{a}{b}
\tag{1-5}
\end{equation}

\begin{figure}[h]
    \centering
    \includegraphics[width=0.8\textwidth]{image1.png}
    \caption{曲柄滑块机构示意图}
    \label{fig:1-3}
\end{figure}

悬点E的速度函数:
\begin{equation}
v_{E} = \frac{ds_{E}}{dt} = \omega r \left( \sin \omega t + \frac{\lambda}{2} \sin 2\omega t \right) \frac{a}{b}
\tag{1-6}
\end{equation}

悬点E的加速度函数:
\begin{equation}
a_{E} = \frac{dv_{E}}{dt} = \omega^{2} r (\cos \omega t + \lambda \cos 2\omega t) \frac{a}{b}
\tag{1-7}
\end{equation}

图1-4是由式(1-5)、(1-6)、(1-7)根据附件1中一个周期 $T=7.89\,\text{s}$ 的时间计算得到的悬点E位移、速度、和加速度随时间 $t$ 的变化曲线。

\begin{figure}[h]
    \centering
    \includegraphics[width=\textwidth]{image2.png}
    \caption{曲柄滑块模型悬点位移、速度及加速度曲线}
    \label{fig:1-4}
\end{figure}

上述简化的运动规律计算方法只能用于一般计算和分析,但做精确分析计算和抽油机设计时,则必须按四连杆机构来研究抽油机的实际运动规律。

(3) 游梁式抽油机四连杆机构的悬点运动模型

由本题的参数条件可知,$r/l=0.95/3.675=0.2585$。上述两个模型都不满足其相应的使用条件,故为了更好的预测悬点 E 运动情况,本文根据图解法建立了精确描述四连杆机构的位移、速度和加速度模型$^{[4]}$。

\begin{figure}[h]
    \centering
    \includegraphics[width=0.8\textwidth]{image.png}
    \caption{游梁式抽油机运动机构示意图}
\end{figure}

由上图可知在三角形中:
\begin{equation}
\left[(l+r)-r\cos\omega t\right]^2+\left[b+r\sin\omega t\right]^2=L^2
\tag{1-8}
\end{equation}
则有:
\begin{equation}
L=\sqrt{\left[(l+r)-r\cos\omega t\right]^2+\left[b+r\sin\omega t\right]^2}
\tag{1-9}
\end{equation}
由 $\tan\beta=\frac{b+r\sin\omega t}{(l+r)-r\cos\omega t}$ 得:
\begin{equation}
\beta=\arctan\left[\frac{b+r\sin\omega t}{(l+r)-r\cos\omega t}\right]
\tag{1-10}
\end{equation}
由三角函数的余弦定理可知 $l^2=L^2+b^2-2Lb\cos\alpha$ 则有:
\begin{equation}
\alpha=\arccos\left[\frac{L^2+b^2-l^2}{2Lb}\right]
\tag{1-11}
\end{equation}
则游梁式抽油机的游梁与水平方向的夹角为:

\begin{equation}
\theta = \frac{\pi}{2} - \beta - \alpha
\tag{1-12}
\end{equation}

则悬点 E 的位移函数为:
\begin{equation}
s_{E} = \left[ \frac{\pi}{2} - \arctan \left[ \frac{b + r \sin \omega t}{(l + r) - r \cos \omega t} \right] - \arccos \left[ \frac{L^2 + b^2 - l^2}{2 L b} \right] \right] a
\tag{1-13}
\end{equation}

悬点 E 的速度和加速度函数为:
\begin{equation}
v_{E} = \frac{ds_{E}}{dt} = a \frac{d\theta}{dt}
\tag{1-14}
\end{equation}
\begin{equation}
a_{E} = \frac{dv_{E}}{dt} = a \frac{d^2\theta}{dt^2}
\tag{1-15}
\end{equation}

其中悬点 E 的速度和加速度函数表达式 (1-14) 和 (1-15) 可以通过 Mathematics 编程对式 (1-13) 求导得到其表达式形式。

由 Mathematics 软件求得悬点 E 的加速度公式为:
\begin{equation}
\mathbf{v}_{E} = \frac{ds_{E}}{dt} = w \left( -\frac{rw \cos[tw]}{l + r - r \cos[tw]} - \frac{rw \sin[tw] (b + r \sin[tw])}{(l + r - r \cos[tw])^2} + \frac{1 + \frac{(b + r \sin[tw])^2}{(l + r - r \cos[tw])^2}}{2 b \sqrt{(l + r - r \cos[tw])^2 + (b + r \sin[tw])^2}} \cdot \frac{2 rw (l + r - r \cos[tw]) \sin[tw] + 2 rw \cos[tw] (b + r \sin[tw])}{(2 rw (l + r - r \cos[tw]) \sin[tw] + 2 rw \cos[tw] (b + r \sin[tw])) \cdot \left( \frac{(b^2 - l^2 + (l + r - r \cos[tw])^2 + (b + r \sin[tw])^2)}{4 b^2 ((l + r - r \cos[tw])^2 + (b + r \sin[tw])^2)} \right)^{3/2}} \right)
\end{equation}

\subsection*{4.1.3 模型求解与分析}

采用附件 1 中的数据参数,对以上两个简化模型进行理论分析。做出悬点 E 位移随时间的变化关系图,如图 1-6。

求的简谐运动模型悬点位移与实际测得的悬点位移平均相对误差为 17.75\%;曲柄滑块模型悬点位移与实际测得的悬点位移平均相对误差为 26.32\%;

由图 1-7 可知,两种模型都是在上死点位置的位移误差较大,这是因为两种模型的假设都是将 A 点和 B 点的弧线运动近似为直线运行,这样当 A 点或 B 点的位移较小时,弧线运动可以近似用上下方向的直线位移来表示,但当位移较大时,用上下方向的直线位移来近似弧线位移误差就会变大,而且随着悬点位移的增大,误差也增大。

\begin{figure}[h]
    \centering
    \includegraphics[width=\textwidth]{image1.png}
    \caption{两种简化模型悬点位移数据与实际数据对比图}
    \label{fig:1-6}
\end{figure}

\begin{figure}[h]
    \centering
    \includegraphics[width=\textwidth]{image2.png}
    \caption{单个冲程内两种模型悬点位移与实际值的绝对误差图}
    \label{fig:1-7}
\end{figure}

如图 1-8 所示,三种理论模型都在悬点上死点位置与实际悬点位移有着较大的误差,而在下死点位置,与实际悬点位移都拟合的很好。这是因为,实际悬点位移图中考虑了抽油杆的载荷对其的影响,而本文提出的三种模型为悬点未承载载荷的情况下推到出来的理论位移公式。而且悬点在下死点位置处,受到的拉伸载荷最小,故在下死点位置,理论位移模型得到的位移与实际悬点位移拟合很好;同时当悬点 E 运动到上死点位置时,抽油杆柱承载着最大的拉伸载荷,抽油杆被拉伸,使得理论模型的位移图与实际悬点位移误差最大。为了考虑载荷对悬点位移的影响,本文对精确解模型进行了模型改进,加入了悬点载荷的影响。

\begin{figure}[h]
    \centering
    \includegraphics[width=\textwidth]{image.png}
    \caption{单个冲程内三种模型悬点位移与实际值的绝对误差图}
    \label{fig:1-8}
\end{figure}

\subsection{悬点运动模型的改进}

为了考虑载荷对悬点位移的影响,引入悬点的有效位移,悬点处有效位移为实际悬点位移减去因为载荷的影响而产生的误差。

悬点处的有效位移(带冲程损失的位移):

\begin{equation}
S_{E} = S_{A} - \lambda_{g}
\tag{1-16}
\end{equation}

若E点表示钢缆与光杆的连接点时,$\lambda_{g}$表示钢缆的伸长量:

\begin{equation}
\lambda_{g} = \frac{D(t)}{E_{g}}
\tag{1-17}
\end{equation}

$D(t)$为钢缆受到的动载荷随时间的变化关系,$E_{g}$为钢缆的弹性模量。

则悬点E处的有效位移为:

\begin{equation}
S_{E} = \left[ \frac{\pi}{2} - \arctan \left[ \frac{b + r \sin \omega t}{(l + r) - r \cos \omega t} \right] - \arccos \left[ \frac{L^{2} + b^{2} - l^{2}}{2 L b} \right] \right] a - \frac{D(t)}{E_{g}}
\tag{1-18}
\end{equation}

其中 $L = \sqrt{[(l + r) - r \cos \omega t]^{2} + [b + r \sin \omega t]^{2}}$。

\subsection{问题二模型的建立与求解}

\subsubsection{问题分析}

早在60年代,人们就认识到光杆示功图诊断方面的缺陷,国内外做了大量的工作,直到1966年美国S. G. Gibbs建立了抽油杆系统的一维粘滞阻尼波动方程,边界条件利用光杆示功图的载荷时间曲线和位移时间曲线,并用分离变量法求得截断傅里叶级数近似解\cite{ref5,ref11,ref12,ref13}。这个方法可以绘出抽油杆柱任意截面的泵的示功图,克服了光杆示功图诊断的缺陷。Gibbs方法是把抽油杆柱作为一根井下动态的传导线,其下端的抽油泵作为发送器,上端的动力仪作为接收器。井下泵

的工作状况以应力波的形式沿抽油杆柱以声波速度传递到地面。因此在地面记录光杆悬点的动载荷时间曲线和位移时间曲线,就可以根据 Gibbs 方法计算出泵的动载荷和位移时间曲线,即泵功图。本节内容主要采用了 Gibbs 模型计算了附件中给定条件下的泵示功图。

\subsection*{4.2.2 Gibbs 模型计算泵功图主要步骤}

(1) 原始数据的处理

附件中的悬点示功图数据是自动测量的,因此附件中所给数据的第一对并不一定刚好是一个冲程的起点,上行和下行用的时间也不一定完全相等,对附件中的数据,首先找到位移为零的点,然后以此为起点,将位移按照从小到大后从大到小的顺序排列起来,并将悬点载荷数据对应于位移数据排列起来。这样处理数据后,数据的排列顺序为下死点-上死点-下死点,为一个冲程。

(2) 计算悬点动载荷函数 $D(t)$ 和位移函数 $u(t)$

悬点动载荷 $D(t)$ 为总的光杆载荷减去抽油杆柱在油管内液体中的重力。为了计算抽油杆柱在油管内液体中的重力,首先计算地下原油密度,地下原油密度等于地面原油密度除以原油体积系数:

\begin{equation}
\rho_{or} = \frac{\rho_{os}}{B_{o}}
\tag{2-1}
\end{equation}

由附件 1 和 2 中的数据可以查得,两口井中的地面原油密度均为 $864 \, \text{kg/m}^3$,地层体积系数均为 1.025。由表达式 (2-1) 可以计算得两口井的地下原油密度为 $842.9268 \, \text{kg/m}^3$。

然后计算油管内混合流体的密度,根据模型假设条件,混合流体仅为液相,不存在气相的影响,因而环空中的混合流体为地层原油与地层水的混合,根据附件中给的油井含水率,可以计算出油井中的混合流体密度为:

\begin{equation}
\rho_{m} = \rho_{w} f + \rho_{or} (1 - f)
\tag{2-2}
\end{equation}

最后,计算悬点动载荷函数:

\begin{equation}
D(t) = L(t) - (\rho_{r} - \rho_{m}) L A_{r} g
\tag{2-3}
\end{equation}

(3) 边界条件处理

步骤 (2) 中计算出的悬点动载荷函数 $D(t)$ 和测量的位移-时间函数 $u(t)$ 作为 Gibbs 阻尼波动方程在井口的边界条件。Gibbs 在解阻尼波动方程的过程中是将悬点处的动载荷 $D(t)$ 和位移 $u(t)$ 函数进行了截断的傅里叶级数的展开。因此根据截断的傅里叶级数,此步骤主要计算傅里叶级数中的傅里叶系数项。计算时将 $2\pi$ 周期分为 $K$ 份,因此傅里叶系数可用数值积分结果表述如下:

\begin{align*}
\sigma_{n} &= \frac{2}{K} \sum_{p=1}^{K} D(p) \cos \left( \frac{2\pi n}{K} p \right) & n = 0, 1, 2, 3, 4 \cdots \overline{n} \\
\tau_{n} &= \frac{2}{K} \sum_{p=1}^{K} D(p) \sin \left( \frac{2\pi n}{K} p \right) & n = 1, 2, 3, 4 \cdots \overline{n} \\
\gamma_{n} &= \frac{2}{K} \sum_{p=1}^{K} u(p) \cos \left( \frac{2\pi n}{K} p \right) & n = 0, 1, 2, 3, 4 \cdots \overline{n} \\
\delta_{n} &= \frac{2}{K} \sum_{p=1}^{K} u(p) \sin \left( \frac{2\pi n}{K} p \right) & n = 1, 2, 3, 4 \cdots \overline{n}
\end{align*}

由附件 1 和 2 中的测量点可知,附件 1 中 \( K \) 取值为 144,附件 2 中 \( K \) 取值为 143。

(4)计算阻尼系数

阻尼系数的计算方法很多,这里采用 Gibbs 公式:

\begin{equation}
c = \frac{\pi a c_{d}}{2L}
\tag{2-5}
\end{equation}

式(2-5)中无量纲阻尼系数需查图版获得,因而本文通过 Origin 软件对无量纲阻尼系数进行多项式拟合,拟合结果如下。

\begin{figure}[h]
\centering
\includegraphics[width=\textwidth]{image.png}
\caption{无量纲阻尼系数 \( c_{d} \) 拟合曲线}
\end{figure}

无量纲阻尼系数随光杆平均速度变化的拟合多项式为:

\begin{equation}
c_{d} = \sum_{i=0}^{9} a_{i} \overline{v}_{r}^{i}
\tag{2-6}
\end{equation}

本文中的声速计算值为:

\begin{equation}
a = \sqrt{\frac{E}{\rho_r}} = \sqrt{\frac{2.1 \times 10^{11}}{8456}} = 4983.416 \, \text{(m/s)}
\tag{2-7}
\end{equation}

\textbf{表 2-1 无量纲阻尼系数多项式拟合数据参数表}

\begin{table}[h]
\centering
\begin{tabular}{|l|l|l|l|}
\hline
Model & Poly & & \\
\hline
Equation & $y = a0 + a1 \cdot x + a2 \cdot x^2 + \ldots + a9 \cdot x^9$ & & \\
\hline
Reduced Chi-Sqr & $4.36463 \times 10^{-5}$ & & \\
\hline
Adj. R-Square & $0.99922$ & & \\
\hline
& & Value & Standard Error \\
\hline
B & a0 & $2.29595$ & $0.04267$ \\
\hline
B & a1 & $-44.32202$ & $1.70337$ \\
\hline
B & a2 & $459.29115$ & $25.89665$ \\
\hline
B & a3 & $-2790.90409$ & $200.90432$ \\
\hline
B & a4 & $10451.0482$ & $895.61945$ \\
\hline
B & a5 & $-24714.8911$ & $2423.85207$ \\
\hline
B & a6 & $36933.77379$ & $4042.79856$ \\
\hline
B & a7 & $-33779.575$ & $4055.9106$ \\
\hline
B & a8 & $17244.66252$ & $2242.39422$ \\
\hline
B & a9 & $-3761.98482$ & $524.59692$ \\
\hline
\end{tabular}
\end{table}

\textbf{(5) 计算特殊系数}

\begin{align}
\alpha_n &= \frac{n \omega}{a \sqrt{2}} \sqrt{1 + \sqrt{1 + \left( \frac{c}{n \omega} \right)^2}} \quad n = 1, 2, 3, 4 \cdots \overline{n} \\
\beta_n &= \frac{n \omega}{a \sqrt{2}} \sqrt{-1 + \sqrt{1 + \left( \frac{c}{n \omega} \right)^2}} \quad n = 1, 2, 3, 4 \cdots \overline{n}
\tag{2-8}
\end{align}

\textbf{(6) 计算位移函数 $u(t)$ 和载荷函数 $D(t)$ 中的系数}

\begin{align}
\kappa_n &= \frac{\sigma_n \alpha_n + \tau_n \beta_n}{E A_r (\alpha_n^2 + \beta_n^2)} \quad n = 1, 2, 3, 4 \cdots \overline{n} \\
\mu_n &= \frac{\sigma_n \beta_n - \tau_n \alpha_n}{E A_r (\alpha_n^2 + \beta_n^2)} \quad n = 1, 2, 3, 4 \cdots \overline{n}
\tag{2-9}
\end{align}

\begin{align}
O_n(x) &= (\kappa_n \cosh \beta_n x + \delta_n \sinh \beta_n x) \sin \alpha_n x + (\mu_n \sinh \beta_n x + \nu_n \cosh \beta_n x) \cos \alpha_n x \\
P_n(x) &= (\kappa_n \sinh \beta_n x + \delta_n \cosh \beta_n x) \cos \alpha_n x - (\mu_n \cosh \beta_n x + \nu_n \sinh \beta_n x) \sin \alpha_n x
\tag{2-10}
\end{align}

\begin{equation}
O_{n}^{\prime}(x)=\left[\frac{\tau_{n}}{EA_{r}} \sinh \beta_{n} x+\left(\delta_{n} \beta_{n}-\nu_{n} \alpha_{n}\right) \cosh \beta_{n} x\right] \sin \alpha_{n} x+
\end{equation}
\begin{equation}
\left[\frac{\sigma_{n}}{EA_{r}} \cosh \beta_{n} x+\left(\nu_{n} \beta_{n}+\delta_{n} \alpha_{n}\right) \sinh \beta_{n} x\right] \cos \alpha_{n} x
\end{equation}
\begin{equation}
P_{n}^{\prime}(x)=\left[\frac{\tau_{n}}{EA_{r}} \cosh \beta_{n} x+\left(\delta_{n} \beta_{n}-\nu_{n} \alpha_{n}\right) \sinh \beta_{n} x\right] \cos \alpha_{n} x-
\end{equation}
\begin{equation}
\left[\frac{\sigma_{n}}{EA_{r}} \sinh \beta_{n} x+\left(\nu_{n} \beta_{n}+\delta_{n} \alpha_{n}\right) \cosh \beta_{n} x\right] \sin \alpha_{n} x
\end{equation}

(7) 计算位移函数 \(u(t)\) 和载荷函数 \(D(t)\)

\begin{equation}
u(x, t)=\frac{\sigma_{0}}{2 EA_{r}} x+\frac{v_{0}}{2}+\sum_{n=1}^{\infty} O_{n}(x) \cos n \omega t+P_{n}(x) \sin n \omega t
\end{equation}

\begin{equation}
F(x, t)=EA_{r}\left[\frac{\sigma_{0}}{2 EA_{r}}+\sum_{n=1}^{\infty} O_{n}^{\prime}(x) \cos n \omega t+P_{n}^{\prime}(x) \sin n \omega t\right]
\end{equation}

(8) 根据力学的连续性原理计算新的边界条件

\begin{align}
_{2} \sigma_{0} & =_{1} \sigma_{0} \\
_{2} \sigma_{n} & =EA_{r_{11}} O_{n}^{\prime}\left(x_{1}\right) \\
_{2} \tau_{n} & =EA_{r_{11}} P_{n}^{\prime}\left(x_{1}\right) \\
_{2} \gamma_{0} & =\frac{\sigma_{0} x_{1}}{EA_{r_{1}}}+_{1} \gamma_{0} \\
_{2} \gamma_{n} & =_{1} O_{n}\left(x_{1}\right) \\
_{2} \delta_{n} & =_{1} P_{n}\left(x_{1}\right)
\end{align}

(9) 以步骤 (8) 计算的边界条件为新的边界条件继续计算二级杆末端的位移函数 \(u(t)\) 和载荷函数 \(D(t)\),并依次计算直到最后一级抽油杆的末端,即为泵功图。

\subsection*{4.2.3 泵功图的计算及分析}

根据上述 (1) 到 (9) 步骤采用 Matlab 编写计算机程序,本文针对附件 1 中的模型以及附件 2 中的模型编写了两个 Matlab 程序,程序名称分别为 Gibbsjs1.m 和 Gibbsjs2.m,见附录。首先对原始数据进行预先的处理,并计算悬点的位移函数 \(u(t)\) 和载荷函数 \(D(t)\),然后对一个周期的时间进行划分 \(t(t)\),将这三个量分别存入 Excel 表格中分别命名为 UINPUT.xls、DINPUT.xls 和 TINPUT.xls 作为 Matlab 程序的输入参数。然后输入各个关键参数,运行程序,即可得到泵的位移函数和载荷函数。得到的泵的位移数据和载荷数据见附件。

运用得到的一级杆和二级杆的泵功图数据,分别绘制了两油井的悬点示功图和泵功图,并将同一井的悬点示功图和泵功图绘制在同一张图上,如图 2-2、2-3。

\begin{table}
\centering
\begin{tabular}{|c|c|c|}
\hline
参数名称 & 一级杆 & 三级杆 \\ \hline
冲次/$l/min$ & 7.6 & 4 \\ \hline
$K$ & 144 & 143 \\ \hline
$\overline{n}$ & 100 & 100 \\ \hline
混合液密度/$kg/m^{3}$ & 996.8585 & 986.1776 \\ \hline
光杆平均速度/$m/s$ & 0.6633 & 0.6347 \\ \hline
阻尼系数/$s^{-1}$ & 1.4309 & 0.6127 \\ \hline
\end{tabular}
\caption{一级杆和三级杆泵功图计算关键参数对比表}
\end{table}

\begin{figure}[h]
\centering
\includegraphics[width=\textwidth]{image1.png}
\caption{一级杆悬点示功图与计算泵功图比较}
\end{figure}

\begin{figure}[h]
\centering
\includegraphics[width=\textwidth]{image2.png}
\caption{三级杆悬点示功图与计算泵功图比较}
\end{figure}

由以上两图可知,泵功图相对于悬点示功图向下平移的一段距离。光杆悬点示功图经过该模型的处理得到了泵功图,其消除了抽油杆柱的变形、粘滞阻尼、振动和惯性载荷等的影响,真实的反映了泵的工作状况。

\section*{4.3 问题三模型的建立与求解}

\subsection*{4.3.1 问题分析}

泵功图能够较为准确地反映井下泵的工作状态。因此,运用泵功图对抽油机的工作状态进行分析具有较高的可信度。因而对泵功图进行分析,建立油井产量的估算模型具有较高的现实意义 \cite{ref7}。本文通过对理论泵功图的分析,建立了两个不同的由泵功图估算油井产量的模型,分别是有效冲程产量模型和面积法产量模型。

\subsection*{4.3.2 模型建立}

\subsubsection{(1) 有效冲程产量模型}

抽油泵的吸入过程从固定阀打开时开始到固定阀关闭时结束,柱塞在这一段时间走过的距离即为固定阀开闭点之间的距离,记为 $s_x$;排出过程从游动阀打开时开始到游动阀关闭时结束,柱塞在这一段时间走过的距离即为游动阀开闭点之间的距离,记为 $s_p$。$s_x$ 的大小由固定阀的开闭点位置决定,$s_p$ 由游动阀的开闭点位置决定;在 $s_x > s_p$ 的情况下,柱塞有效冲程 $s_{pe} = s_p$;在 $s_x < s_p$ 的情况下,柱塞有效冲程 $s_{pe} = s_x$;所以,柱塞有效冲程总是等于 $s_x$ 与 $s_p$ 其中的较小值,即 $s_{pe} = \min(s_x, s_p)$。在获得柱塞的有效冲程后,结合有杆抽油系统的基础数据及原油的物性参数,可以通过式 (3-1) 计算得到油井的实际产液量。

\begin{equation}
Q = \frac{1440 A_p s_{pe} n_c (1 + \varepsilon)}{B_m}
\tag{3-1}
\end{equation}

\begin{equation}
Q_t = \frac{1440 A_p s_{pe} n_c (1 + \varepsilon)}{B_m} \rho_m
\tag{3-2}
\end{equation}

\subsubsection{(2) 面积法产量模型}

面积法产量模型,依据理论示功图面积和实际泵功图的面积来计算油井的产量:

\begin{equation}
Q_t = \frac{S}{S_p} \times 1440 \times A_p s n \rho_m
\tag{3-3}
\end{equation}

通过计算的泵功图图像的面积与理论示功图的面积的比值来确定油井产量与理论泵排量的比值。此方法是一种实际可行的油井产量分析方法,但此方法缺点是考虑的因素相对较少,与实际油井产量可能相差较大。

\subsection*{4.3.3 模型求解及分析}

\subsubsection{(1) 有效冲程产量模型}

采用有效冲程法计算油井的产量,首先需要对泵功图进行详细的分析,从泵功图上得到柱塞的有效冲程。

首先,对泵功图进行分析,如图 3-1 和 3-2 所示,从泵功图上读出柱塞的有效冲程:一级杆为 $s_{pe} = s_x = 2.0504 \, \text{m}$。三级杆为:$s_{pe} = s_p = 3.3652 \, \text{m}$。

\begin{figure}[h]
    \centering
    \includegraphics[width=\textwidth]{image1.png}
    \caption{一级杆抽油泵有效冲程计算图}
    \label{fig:1}
\end{figure}

\begin{figure}[h]
    \centering
    \includegraphics[width=\textwidth]{image2.png}
    \caption{三级杆抽油泵有效冲程计算图}
    \label{fig:2}
\end{figure}

然后,估算油井一天的产液量,一级杆油井产液量:
\begin{equation}
Q_{t} = \frac{1400A_{p}s_{pe}n_{c}(1+\varepsilon)}{B_{m}}\rho_{m}
\end{equation}
\begin{equation}
= \frac{1440 \times \frac{\pi}{4} \times 0.07^{2} \times 2.0504 \times 7.6 \times (1+0.1)}{1.025 \times (1-0.98) + 1.0 \times 0.98} \times \left[ 842.9268 \times 0.02 + 1000 \times 0.98 \right]
\end{equation}
\begin{equation}
= 94.6476 \text{(吨)}
\end{equation}

同理,三级杆油井产液量:

\begin{equation}
Q_{t}=\frac{1400 A_{p} s_{p e} n_{c}(1+\varepsilon)}{B_{m}} \rho_{m}
\end{equation}
\begin{equation}
=\frac{1440 \times \frac{\pi}{4} \times 0.044^{2} \times 3.3652 \times 4 \times(1+0.1)}{1.025 \times(1-0.912)+1.0 \times 0.912} \times\left[842.9268 \times 0.088+1000 \times 0.912\right]
\end{equation}
\begin{equation}
=31.9023 \text{(吨)}
\end{equation}

(2) 面积法产量模型

面积法产量模型需要由泵功图计算出理论示功图的面积 \( S_{p} \) 和泵功图的面积 \( S \) 的值。本文通过对图像的数值积分法计算泵功图的面积以及理论示功图的面积。将泵功图划分为上冲程和下冲程两部分,然后分别计算出上冲程与横坐标围成的图形面积以及下冲程与横坐标围的图形面积,泵功图面积为上冲程图形面积减去下冲程图形面积,泵功图面积数值计算公式为:
\begin{equation}
S=\int_{\text{下死点}}^{\text{上死点}}\left(D(u)_{\text{上冲程}}-D(u)_{\text{下冲程}}\right) d u=\sum_{i=1}^{n}\left(D(u)_{\text{上冲程} i}-D(u)_{\text{下冲程} i}\right) \Delta u
\tag{3-4}
\end{equation}

\begin{figure}[h]
\centering
\includegraphics[width=\textwidth]{image1.png}
\caption{一级杆油井示功图面积分析图}
\end{figure}

\begin{figure}[h]
\centering
\includegraphics[width=\textwidth]{image2.png}
\caption{Origin 分析一级杆油井泵功图面积}
\end{figure}

\begin{figure}[h]
    \centering
    \includegraphics[width=0.8\textwidth]{image1.png}
    \caption{Origin 分析一级杆油井示功图面积}
    \label{fig:3-5}
\end{figure}

用 Origin 软件对泵功图进行图像分析,运用 Mathematics Integrated 计算得到泵功图的面积和理论示功图面积,如图 3-4 和 3-5 为计算结果。由计算结果可知,泵功图面积为 \( S = 93.1706 \, \text{kN} \cdot \text{m} \),理论示功图面积 \( S_p = 104.2487 \, \text{kN} \cdot \text{m} \)。则由面积法产量估算公式可得一级杆油井的一天内的产液量为:

\begin{equation}
Q_t = \frac{S}{S_p} \times 1440 \times A_p s n \rho_m
\end{equation}

\begin{equation}
= \frac{93.1706}{104.2487} \times 1440 \times \frac{\pi}{4} \times 0.07^2 \times 3.2 \times 7.6 \times [842.9268 \times 0.02 + 1000 \times 0.98]
\end{equation}

\begin{equation}
= 120.0753 \, \text{(吨)}
\end{equation}

同理用 Origin 软件对三级杆的理论示功图和泵功图的面积进行分析,并求出其相应的面积大小。如图,示功图分析图 3-6,面积计算结果图 3-7 和 3-8。

\begin{figure}[h]
    \centering
    \includegraphics[width=0.8\textwidth]{image2.png}
    \caption{三级杆油井示功图面积分析图}
    \label{fig:3-6}
\end{figure}

\begin{figure}[h]
    \centering
    \includegraphics[width=0.8\textwidth]{image1.png}
    \caption{Origin 分析三级杆油井泵功图面积}
    \label{fig:3-7}
\end{figure}

\begin{figure}[h]
    \centering
    \includegraphics[width=0.8\textwidth]{image2.png}
    \caption{Origin 分析三级杆油井示功图面积}
    \label{fig:3-8}
\end{figure}

由计算结果可知,泵功图面积为 \( S = 94.5280 \, \text{kN·m} \),理论示功图面积 \( S_p = 136.6397 \, \text{kN·m} \)。则由面积法产量估算公式可得三级杆油井的一天内的产液量为:

\begin{equation}
Q_t = \frac{S}{S_p} \times 1440 \times A_p s n \rho_m
\end{equation}

\begin{equation}
= \frac{94.5280}{136.6397} \times 1440 \times \frac{\pi}{4} \times 0.044^2 \times 4.2 \times 4 \times [842.9268 \times 0.088 + 1000 \times 0.912]
\end{equation}

\begin{equation}
= 25.0961 \, (\text{吨})
\end{equation}

\subsection*{(3) 模型比较及分析}

针对两种不同的产量估算方法,其计算的油井产量也不同,这主要是模型所基于的假设条件等的不同所造成的。下表 3-1 给出了一级杆和三级杆基本参数。

\begin{table}[h]
    \centering
    \caption{不同油井参数产量分析表}
    \label{tab:3-1}
    \begin{tabular}{|c|c|c|}
        \hline
        油井 & 一级杆 & 二级杆 \\
        \hline
        冲程 (m) & 3.2 & 4.2 \\
        \hline
        冲次 (\( \text{min}^{-1} \)) & 7.6 & 4 \\
        \hline
        泵径 (mm) & 70 & 44 \\
        \hline
        含水率 & 0.98 & 0.912 \\
        \hline
        有效冲程产量模型估算产量 (吨) & 94.65 & 31.90 \\
        \hline
        面积法产量模型估算产量 (吨) & 120.08 & 25.10 \\
        \hline
    \end{tabular}
\end{table}

\begin{figure}[h]
    \centering
    \includegraphics[width=\textwidth]{image.png}
    \caption{不同产量模型预测油井产量对比图}
    \label{fig:3-9}
\end{figure}

由表 3-1 可知,一级杆的冲次较大而且泵径较大,故该油井的产量相对较大。两种产量模型估算的同一口井的产量相差较大,这是因为在面积法估算过程中,需要对理论示功图和泵功图的面积进行求解,这样会引入一定的计算误差,同时有效冲程产量模型中,在确定泵的有效冲程时也需要对泵功图进行图形的分析,该图形分析的过程人为的影响因素较多,泵功图不同的形状代表了泵不同的工作状态,其是一个一图多解释的过程,在解释泵功图和泵的有效冲程及面积的确定过程中容易导致较大的误差,故不同的产量模型估算的产液量并不一定完全相同。

\subsection{泵功图诊断模型建立}

\subsubsection{问题分析}

目前泵功图特征的提取大多采用的是人类专家相面法、面积法、模板匹配法、网格法、矢量法等注重提取泵功图的特征,如几何、形状、力学等特征,普遍存在特征参数偏多的不足,所使用的判别器多为距离分类器,这类分类器实质是满足一定条件的特殊贝叶斯分类器,要求各特征分量线性无关、各特征分量的方差相同、类条件概率密度函数为正态分布,虽然 Bayes 分类器在某种意义上是最优分类器,但针对采油系统工矿分类问题不能满足这一前提条件,导致识别结果误判\cite{ref6}。

本文试图将灰色理论与模式识别技术相结合,根据泵功图的特点通过提取功图归一化无因次灰度统计特征:灰度均值、灰度方差、灰度偏度、灰度峰度、灰度能量、灰度熵,以此构成分类特征向量;建立以灰关联分析取代距离分类器的改进模型,进一步提高了诊断的正确率,实现了关于抽油系统故障的计算机自动识别。

\subsubsection{模型建立}

本文建立了基于灰色理论的抽油井泵功图诊断模型。基于灰色理论的有杆泵采油系统工况智能诊断模型主要分为:数据预处理、特征提取及故障类型分类三个步骤。

\section*{Step1:原始数据的预处理}

基于泵功图形状特征的识别,必须首先避开量纲和坐标尺度的影响,因而要对功图进行无量纲预处理。

假设泵功图为一组离散序列 $\{\overline{s}_{i}, \overline{f}_{i}\}(i=1,2,\cdots,K)$,采用数理统计无量纲化处理方法,将其转化为式 (3-5),其中,$s_{i}$ 为位移,$f_{i}$ 为载荷。

\begin{equation}
\left\{
\begin{aligned}
s_{i} &= \left[ s_{i} - \min_{j=1,2,\cdots,K}(s_{j}) \right] \Big/ \left[ \max_{j=1,2,\cdots,K}(s_{j}) - \min_{j=1,2,\cdots,K}(s_{j}) \right] \\
f_{i} &= \left[ f_{i} - \min_{j=1,2,\cdots,K}(f_{j}) \right] \Big/ \left[ \max_{j=1,2,\cdots,K}(f_{j}) - \min_{j=1,2,\cdots,K}(f_{j}) \right]
\end{aligned}
\right.
\tag{3-5}
\end{equation}

\section*{Step2:特征向量的提取}

经过预处理,满足识别要求的模式要依据识别方法抽取选择特征,由特征参数张成多维空间的坐标轴,特征参数的选择对识别的效果有直接的影响。在本系统中,我们采用泵功图的灰度矩阵为特征矩阵,提取该灰度矩阵的统计特征。

\begin{table}[h]
\centering
\begin{tabular}{|c|c|}
\hline
\textbf{故障类型} & \textbf{泵功图特点} \\
\hline
泵况正常 & \includegraphics[width=0.3\textwidth]{image1.png} \\
\hline
严重液击 & \includegraphics[width=0.3\textwidth]{image2.png} \\
\hline
泵下碰 & \includegraphics[width=0.3\textwidth]{image3.png} \\
\hline
轻度液击 & \includegraphics[width=0.3\textwidth]{image4.png} \\
\hline
气体影响 & \includegraphics[width=0.3\textwidth]{image5.png} \\
\hline
泵磨损、折断或抽喷 & \includegraphics[width=0.3\textwidth]{image6.png} \\
\hline
泵筒磨损或漏 & \includegraphics[width=0.3\textwidth]{image7.png} \\
\hline
固定凡尔漏 & \includegraphics[width=0.3\textwidth]{image8.png} \\
\hline
$\cdots$ & $\cdots$ \\
\hline
\end{tabular}
\caption{典型故障特征泵功图}
\end{table}

灰度矩阵形成:泵功图的灰度矩阵是在其网格矩阵的基础上形成的,其形成过程分为四步。

(1) 功图标准化。为了便于比较油井各类工况相除泵功图量纲对数据的影响,首先将各类功图放到一个 $2 \times 1$ 的长方形中,使示功图与长方形的四边相切。$2 \times$ 矩形的选取,是为了符合石油界的习惯,经标准化后,坐标成为相对的;

(2) 将长方形分为网格,通常采用个网格,将此网格初始化为 “0”;

(3) 令示功图边界穿越的网眼灰度均赋 “1”;

(4) 边界内部每远离边界一格其灰度值增加一级,外部按等高线的方式赋值,只是每远离边界一格其灰度值减少一级;搜索边界的方式是按列进行。最终获得功图网格矩阵。图 3-10 是典型故障特征泵示功图。图 3-11 是按照上述步骤获得的一种有轻度液击泵功图灰度矩阵。

灰度统计特征:依据数理统计原理,取灰度矩阵的 6 个统计特征。分别是灰度的均值、方差、偏度、峰度、能量、熵。

假设泵功图的灰度矩阵为 $\overline{G}(J, K)$,矩阵中元素 $g_{jk}(1 \leq j \leq J, 1 \leq k \leq K)$ 表示泵功图网格上对应的灰度值;功图的灰度级数为 R,某一灰度 r 的元素个数为,则灰度级的概率为:$p(r) = b(r) / (J \times K)$。

\begin{tabular}{cccccccccccccccccccc}
1 & 1 & 1 & 1 & 1 & 1 & 1 & 1 & 1 & 1 & 1 & 1 & 1 & 1 & 1 & 1 & 1 & 1 & 1 & 1 \\
1 & 2 & 2 & 2 & 2 & 2 & 2 & 2 & 2 & 2 & 2 & 2 & 2 & 2 & 2 & 2 & 2 & 2 & 2 & 1 \\
1 & 2 & 3 & 3 & 3 & 3 & 3 & 3 & 3 & 3 & 3 & 3 & 3 & 3 & 3 & 3 & 2 & 1 & 1 & 1 \\
1 & 2 & 3 & 4 & 4 & 4 & 4 & 4 & 4 & 4 & 4 & 4 & 4 & 4 & 4 & 3 & 2 & 1 & 0 & 0 \\
1 & 2 & 3 & 4 & 5 & 5 & 5 & 5 & 5 & 5 & 5 & 5 & 5 & 5 & 4 & 3 & 2 & 1 & 0 & -2 \\
1 & 2 & 3 & 4 & 5 & 5 & 5 & 5 & 5 & 5 & 5 & 5 & 5 & 4 & 3 & 2 & 1 & 0 & -1 & -2 \\
1 & 2 & 3 & 4 & 4 & 4 & 4 & 4 & 4 & 4 & 4 & 4 & 4 & 3 & 2 & 1 & 0 & -1 & -2 \\
1 & 2 & 3 & 3 & 3 & 3 & 3 & 3 & 3 & 3 & 3 & 3 & 3 & 2 & 1 & 0 & -1 & -2 \\
1 & 2 & 2 & 2 & 2 & 2 & 2 & 2 & 2 & 2 & 2 & 2 & 2 & 1 & 0 & -1 & -2 \\
1 & 1 & 1 & 1 & 1 & 1 & 1 & 1 & 1 & 1 & 1 & 1 & 1 & 0 & 0 & 0 & 0 & 0 & 0 & 0
\end{tabular}

图 3-11 有轻度液击泵示功图的灰度矩阵

灰度均值、灰度方差、灰度偏差、灰度峰度、灰度峰度、灰度能量、灰度熵等的计算表达式如下,以 $\{f_1, f_2, f_3, f_4, f_5, f_6\}$ 为构成分类特征向量 $F$。

\begin{align}
\overline{g}: f_1 &= \overline{g} = \sum_{r=1}^{R} r \cdot p(r) \\
\sigma^2: f_2 &= \sigma^2 = \sum_{r=1}^{R} (r - \overline{g})^3 p(r) \\
S: f_3 &= S = \frac{1}{\sigma^3} \sum_{r=1}^{R} (r - \overline{g})^3 p(r) \\
\nu: f_4 &= P = \frac{1}{\sigma^4} \sum_{r=1}^{R} (r - \overline{g})^4 p(r) \\
E: f_5 &= E = \sum_{r=1}^{R} [p(r)]^2 \\
T: f_6 &= T = -\sum_{r=1}^{R} [1 - p(r)] \cdot \log_{10} [1 - p(r)]
\end{align}

\section*{Step3: 建立基于灰关联分析的分类器}

分类准则:用示功图来诊断系统的故障,是通过比较井下实际工矿泵功图与各种参考故障泵功图统计特征量的相似性比较而得出结论的。这种相似性要定量表述并非容易,其准确性决定了分类的正确性。

根据灰色系统理论,灰关联分析能够较真实地揭示比较序列和参考序列间曲线几何形状的接近程度,几何形状越接近,变化趋势也就越接近,关联度就越大。以此为模式相似性判别依据,所得出数据的分类结果更合理。

令待查样本功图统计特征为 $F_{0}=\left\{f_{01}, f_{02}, f_{03}, f_{04}, f_{05}, f_{06}\right\}$,参考故障泵功图统计特征为 $F_{i}=\left\{f_{i1}, f_{i2}, f_{i3}, f_{i4}, f_{i5}, f_{i6}\right\}(i=1,2, \cdots, m)$,$f_{ij}$ 是依据式 (3-6) 建立的灰度统计量。

\begin{equation}
\varepsilon_{i}(k)=\frac{\min _{i} \min _{k}\left|f_{0}(k)-f_{i}(k)\right|+\rho \max _{i} \max _{k}\left|f_{0}(k)-f_{i}(k)\right|}{\left|f_{0}(k)-f_{i}(k)\right|+\rho \max _{i} \max _{k}\left|f_{0}(k)-f_{i}(k)\right|}
\tag{3-7}
\end{equation}

式 (3-7) 为 $F_{0}$ 与 $F_{i}$ 在第 $k$ 特征处的关联系数。综合各特征点的关联系数,则称:

\[
r_{i}=\frac{1}{n} \sum_{k=1}^{n} \varepsilon_{i}(k)
\]

为 $F_{0}$ 与 $F_{i}$ 的关联度。其中 $\left|f_{0}(k)-f_{i}(k)\right|=\Delta_{i}(k)$ 称为 $F_{0}$ 与 $F_{i}$ 在第 $k$ 特征处的绝对差;$\min _{i} \min _{k}\left|f_{0}(k)-f_{i}(k)\right|$ 成为两级最小差,其中 $\min _{k}\left|f_{0}(k)-f_{i}(k)\right|$,表示 $F_{i}$ 与 $F_{0}$ 上找一个差值最小的点,$\min _{i} \min _{k}\left|f_{0}(k)-f_{i}(k)\right|$ 是第二级最小差,表示在 $m$ 个故障曲线中找一条差值最小的曲线。$\max _{i} \max _{k}\left|f_{0}(k)-f_{i}(k)\right|$ 称为二级最大差,其意义与二级最小差相似;$\rho$ 成为分辨系数,是 0 与 1 之间的数,一般取 $\rho=0.5$。最大关联度识别故障原则:$r_{i}$ 是比较序列与参考序列的关联度,是几何接近程度的定量描述,$r_{i}$ 值越大,则 $F_{i}$ 与 $F_{0}$ 的关系越密切,认为统计特征为 $F_{i}$ 的故障。

\section*{Step4:参考特征的建立}

参考特征的实质是知识和经验库,通过对已知类别的样本识别误判率最小为目的。由于油井分布广泛,为进一步及时准确对其出现的故障进行诊断,通过大量的实验,并借鉴专家的丰富实践经验分析诊断不同井下泵况及杆柱情况,建立了参考特征库。表 3-2 给出了参考类别灰度统计特征向量的值,功图网格为 20×10。

\begin{table}[h]
\centering
\caption{参考类别灰度统计特征向量的值}
\begin{tabular}{|c|c|c|c|c|c|c|}
\hline 故障类型 & 均值 $f_{1}$ & 方差 $f_{2}$ & 偏度 $f_{3}$ & 峰度 $f_{4}$ & 能量 $f_{5}$ & 广义熵 $f_{6}$ \\
\hline 泵正常 & 3.113 & 32.043 & 0.128 & 2.082 & 0.058 & 0.421 \\
\hline 液击 & -4.945 & 71.433 & -0.611 & 2.248 & 0.042 & 0.424 \\
\hline
\end{tabular}
\end{table}

\begin{table}
\centering
\begin{tabular}{|l|c|c|c|c|c|c|}
\hline
供液不足 & -0.664 & 67.745 & -0.812 & 3.140 & 0.044 & 0.424 \\
\hline
双凡尔漏失 & 3.978 & 34.300 & -0.0721 & 2.495 & 0.050 & 0.376 \\
\hline
气体影响 & 0.941 & 59.103 & -0.732 & 2.483 & 0.044 & 0.424 \\
\hline
游动凡尔漏失 & 3.399 & 34.912 & -0.490 & 3.423 & 0.052 & 0.422 \\
\hline
抽油杆断脱 & -1.513 & 52.634 & -0.653 & 3.036 & 0.043 & 0.426 \\
\hline
泵上碰 & -5.190 & 54.599 & -0.413 & 2.149 & 0.043 & 0.424 \\
\hline
泵下碰 & -9.400 & 70.846 & -0.503 & 2.245 & 0.036 & 0.426 \\
\hline
固定凡尔漏失 & 3.708 & 35.636 & -0.269 & 2.884 & 0.052 & 0.422 \\
\hline
严重供液不足 & -1.990 & -30.150 & -0.630 & 2.499 & 0.056 & 0.097 \\
\hline
\end{tabular}
\end{table}

\section{问题四模型的建立与求解}

\subsection{Gibbs 模型问题分析}

\subsubsection{Gibbs 模型基本原理分析}

Gibbs 模型主要是把抽油杆的问题简化成一根细长弹性杆悬挂在振源悬点上做受迫振动的问题。悬点周期性的上下运动通过弹性杆向下传递。Gibbs 将周期性的运动向下传递的过程简化成弹性杆的带阻尼的纵向振动。杆的纵向振动可用带阻尼的波动方程来描述,即振动位移随着空间和时间的函数 \({}^{[8-10]}\)。

在运用 Gibbs 模型对有杆抽油系统进行计算机诊断时,其采用的基本原理为把抽油杆柱看作井下动态型号的传导线,抽油泵作为发送器,泵的工作状况以应力波的形式沿抽油杆柱传递到地面,被作为接收器的动力仪所接收。根据这个原理,Gibbs 建立了带阻尼的波动方程作为描述应力波在抽油杆柱中传递过程的基本微分方程。以光杆载荷及位移作为边界条件,用分离变量法求得其截面的傅立叶级数近似解,可以得到抽油杆柱任意截面处的位移和载荷,从而绘出所需的抽油杆柱各截面及泵的示功图。根据这些示功图,可以对整个油井生产系统的工作状况进行分析和判断。

\subsubsection{Gibbs 模型的评价}

Gibbs 模型在建立过程中,仅仅是把抽油杆独立出来建立了一维带阻尼的波动方程。实际的有杆抽油系统由抽油机主体结构、悬绳器、抽油杆柱、油管柱、油管内液柱、抽油泵与油套环空中的液柱等几个子系统所组成,其中抽油杆柱、油管柱与油管内液柱是三个振动的子系统。而 Gibbs 一维模型只研究了抽油杆柱的纵向振动问题,忽略了油管柱和油管内液柱的振动,液柱的影响仅限于粘滞阻力,其应用的范围受到很大的限制。因此当油管柱未锚定时或油管内液柱的影响较大时,Gibbs 模型在应用过程中可能产生较大的误差。其次,Gibbs 模型是在忽略了抽油杆重量的假设条件下建立的,在初始数据输入的过程中,需要对悬点载荷数据处理,减去抽油杆在油管内液柱中的重量。

国内外的学者针对于 Gibbs 模型的不足进行了大量的研究。1969 年 R. M. Knapp 首先提出用差分方程计算井下示功图,但他的方法不适用于不同抽油杆直径的组合杆柱的计算,更不适用于不同材料的混合抽油杆柱。1980 年 T. Nicol 采用有限元法解波动方程,这种方法可计算玻璃钢-钢混合杆,取得了很好的效果。1987 年 D. J. Schafer 和 J. W. Jennings 对等步长差分格式在遇到抽油杆柱材料性质发生变化进行了研究。1988 年 J. F. Lea 提出了变步长差分格式,可用于多级

杆和混合杆。同年,陈家林等人采用快速傅立叶变换求解波动方程,大大加快了运算速度。本世纪初,王金东、蔡东生等人运用精细逐步积分法和隐式差分法求解波动方程,大大提高了其运算精度和运算速度。

我国的有杆抽油系统计算机诊断技术的研究在上个世纪的 90 年代得到迅速的发展。诊断技术的理论基础是 S. G. Gibbs\textsuperscript{[11-12]}建立的一维带阻尼的波动方程。该方程只考虑了抽油杆柱的振动,而忽略了液柱和管柱的振动。1981 年美国塔尔萨大学的 D. R. Doty 和 Z. Schmidt\textsuperscript{[15]}建立了考虑液柱振动的二维的预测数学模型仁。在浅井、大泵和大排量情况下液体惯性的影响是比较大的,这比 S. G. Gibbs 的一维震动数学模型要完善些,但由于该模型需要四个边界条件,而目前的测试手段只能测到两个地面边界条件,且假定油管是锚定的,实质上是用二维振动数学模型来解决实际上是三维的振动问题,因而并未从根本上解决这一问题。1989 年西安石油学院的余国安等人建立了综合考虑抽油杆柱、液柱和油管振动的三维数学模型,并进行了求解。这种三维振动的数学模型比起前两种数学模型来,在理论上显然要更加合理一些。2003 年我国的李冬平等针对定向井的抽油杆系统建立了定向井有杆抽油系统抽油杆柱动力学仿真模型,真实的考虑了抽油杆柱的弯曲等情况的影响。

\subsubsection{有杆抽油系统模型分析}

针对于本文中的有杆抽油系统,油管是锚定的,则不必考虑油管的振动情况。而且抽油井为直井,则可以忽略抽油杆的弯曲、抽油杆与油管之间的偏磨对有杆抽油系统的影响。油管内的液体考虑成单相,即油管内的液柱中没有气泡。通过对本题的分析,本文建立了同时考虑抽油杆振动和油管内液柱振动的二维振动模型。

\subsubsection{有杆抽油系统模型假设}

(1) 将抽油杆简化为均质弹性杆,且为一级杆;

(2) 假设抽油杆柱轴线与油管轴线重合,忽略抽油杆与油管壁的接触;

(3) 抽油杆柱在油管内的横向振动和变形很小,忽略不计;

(4) 油管锚定,忽略油管柱的振动;

(5) 油管柱内的液体为单相,忽略气泡的影响。

\subsubsection{有杆抽油系统模型建立}

(1) 抽油杆柱振动模型的建立

如图 4-1,向下的方向为正方向,对抽油杆进行受力分析,有牛顿第二定律得:

\begin{equation}
\rho_r A_r \frac{\partial^2 u(x,t)}{\partial t^2} = E_r A_r \frac{\partial^2 u(x,t)}{\partial x^2} - F_r + \rho_r A_r g
\tag{4-1}
\end{equation}

式 (4-1) 中,$\rho_r$ 为抽油杆密度,kg/m\textsuperscript{3};$A_r$ 为抽油杆柱截面积,m\textsuperscript{2};$E_r$ 为抽油杆柱的弹性模量,pa;$F_r$ 为抽油杆柱单位长度上的受到的摩擦力,N/m;$u(x,t)$ 为抽油杆任意截面($x$ 处)在任意时刻 $t$ 的位移,m。

\begin{figure}[h]
    \centering
    \includegraphics[width=0.8\textwidth]{image.png}
    \caption{抽油杆单元模型图}
    \label{fig:4-1}
\end{figure}

(2) 环空流体振动模型的建立

若考虑到流体为可压缩流体时,流体的连续性方程为:
\begin{equation}
    \frac{\partial \rho_{f}}{\partial t} + \frac{\partial}{\partial x} \left( \rho_{f} v_{f} \right) = 0
    \tag{4-2}
\end{equation}

流体密度与压力之间的关系为:
\begin{equation}
    \rho_{f} = \rho_{f}(P_{f}) = \rho_{f0} + C_{0} \rho_{f0} \left( P_{f} - \overline{P} \right)
    \tag{4-3}
\end{equation}

式(4-3)中,$\overline{P}$ 为常压,Pa;$\rho_{f0}$ 是 $\overline{P}$ 下的密度,kg/m$^3$;$C_{0}$ 为液体压缩系数。

对油杆环空中的液体进行受力分析:
\begin{equation}
    \rho_{f} \left( A_{t} - A_{r} \right) \left( \frac{\partial v_{f}}{\partial t} + v_{f} \frac{\partial v_{f}}{\partial x} \right) = - \left( A_{t} - A_{r} \right) \frac{\partial P_{f}}{\partial x} + \rho_{f} \left( A_{t} - A_{r} \right) g + F_{f}
    \tag{4-4}
\end{equation}

式(4-4)中,$\rho_{f}$ 为环空流体密度,kg/m$^3$;$A_{t}$ 为油管内径截面积,m$^2$;$F_{f}$ 为环空中流体单位长度上受到的摩擦力,N/m;$P_{f}$ 为流体压力,Pa;$v_{f}$ 为流体的流速,m/s。

若考虑流体为不可压缩流体时,若假设环空中的流体密度随着油井中的压力和温度变化很小,流体密度为常数,则方程(4-4)为线性偏微分方程。流体连续性方程(4-2)变成:
\begin{equation}
    \frac{\partial v_{f}}{\partial x} = 0
    \tag{4-5}
\end{equation}

方程(4-5)说明流体速度不随位移井深发生变化,流体速度只随着时间发生变化,环空中的流体在各个井深处是等速的。

若不考虑液体的压缩性,则液柱就可以看作不可受拉的弹性体,其运动方程:
\begin{equation}
\rho_{f}(A_{t}-A_{r})\frac{\partial v_{f}}{\partial t}=-(A_{t}-A_{r})\frac{\partial P_{f}}{\partial x}+\rho_{f}(A_{t}-A_{r})g+F_{f}
\tag{4-6}
\end{equation}
液柱压力与速度之间的关系为:
\begin{equation}
E_{f}\frac{\partial v_{f}}{\partial x}=-\frac{\partial P_{f}}{\partial t}
\tag{4-7}
\end{equation}

(3)摩擦阻力计算

抽油杆柱上的摩擦阻力由两部分组成,环控中流体对抽油杆柱的阻力和环空中流体对抽油杆柱接箍上的摩擦阻力:
\begin{equation}
F_{r}=0.5\rho_{f}v_{f}\left|v_{f}\right|\pi D_{r}C_{1}+0.5\rho_{f}v_{f}\left|v_{f}\right|(A_{c}-A_{r})C_{2}/l
\tag{4-8}
\end{equation}
式中,$D_{r}$ 为抽油杆柱的直径,m;$C_{1}$ 和 $C_{2}$ 为摩擦阻力系数,由 Valeev 和 Repin 等人通过实验确定;$l$ 为一根抽油杆柱的长度,m。$C_{1}$ 和 $C_{2}$ 通过下式确定:
\begin{equation}
C_{1}=\frac{24}{R_{e}}\left[1\pm\frac{R_{e}^{'}}{R_{e}}(0.2+0.39\frac{D_{r}}{D_{t}})\right]
\tag{4-9}
\end{equation}
\begin{equation}
C_{2}=\frac{5.2\times10^{4}}{R_{e}}(\frac{D_{c}}{D_{t}}-0.381)^{2.57}(2.77\pm1.69\frac{R_{e}^{'}}{R_{e}})
\tag{4-10}
\end{equation}
式中,$D_{c}$ 为接箍直径,m;$D_{t}$ 为套管内径,m;雷洛数由环空流体和抽油杆速度确定:
\begin{equation}
R_{e}=\frac{v_{f}(D_{t}-D_{r})\rho_{f}}{\mu}
\tag{4-11}
\end{equation}
\begin{equation}
R_{e}^{'}=\frac{\partial u}{\partial t}\frac{(D_{t}-D_{r})\rho_{f}}{\mu}
\tag{4-12}
\end{equation}
由上式可得:
\begin{equation}
F_{r}=b_{r}\frac{\partial u}{\partial t}-b_{fr}v_{f}
\tag{4-13}
\end{equation}
式(4-13)中的系数为:
\begin{equation}
\begin{cases}
b_{r}=12\pi\mu[(0.2+0.39\frac{D_{r}}{D_{t}})+915.42(\frac{D_{c}}{D_{t}}-0.381)^{2.57}((\frac{D_{c}}{D_{r}})^{2}-1)\frac{D_{r}}{l}(\frac{D_{t}}{D_{r}}-1)^{-1}] \\
b_{fr}=12\pi\mu[1+1500.42(\frac{D_{c}}{D_{t}}-0.381)^{2.57}((\frac{D_{c}}{D_{r}})^{2}-1)\frac{D_{r}}{l}(\frac{D_{t}}{D_{r}}-1)^{-1}]
\end{cases}
\tag{4-14}
\end{equation}
同理,流体受到的摩擦阻力也为两部分,一部分为套管对流体的阻力,另一部分为抽油杆对流体的阻力:
\begin{equation}
F_{f}=-(F_{r}+F_{t})
\tag{4-15}
\end{equation}
其中,$\eta$ 为流体的粘度,mP·s。式(4-15)中的套管对流体的阻力为:

\begin{equation}
F_{t}=b_{t}v_{t}-b_{ft}v_{f}
\tag{4-16}
\end{equation}

式(4-16)中系数为:
\begin{equation}
\left\{
\begin{aligned}
b_{t} &= 12\pi\mu(0.2+0.39\frac{D_{r}}{D_{t}})(\frac{D_{t}}{D_{r}}-1)^{-1} \\
b_{ft} &= 12\pi\mu(\frac{D_{t}}{D_{r}}-1)^{-1}
\end{aligned}
\right.
\tag{4-17}
\end{equation}

在本模型中油管锚定,故油管振动速度 $v_{t}$ 为零。

(4)边界条件

边界条件分为两部分,井口边界条件和井底边界条件。井口抽油杆位移函数边界条件为:
\begin{equation}
D(0,t)=E_{r}A_{r}\frac{\partial u(0,t)}{\partial x}
\tag{4-18}
\end{equation}

井口流体柱边界条件:
\begin{equation}
P_{f}(0,t)=P_{0}
\tag{4-19}
\end{equation}

井底边界条件中,抽油杆和流体柱根据游动阀和固定阀的开闭情况分为三种情况:

第一种情况:固定阀和游动阀都关闭的情况下,此时柱塞不移动,抽油杆和流体柱的边界条件为:
\begin{equation}
\frac{\partial u(L,t)}{\partial t}=0
\tag{4-20}
\end{equation}
\begin{equation}
v_{f}(L^{*},t)=0
\tag{4-21}
\end{equation}

第二种情况:固定阀打开游动阀关闭,此时柱塞泵在举升液体,这时抽油杆上的应力与柱塞中的流体有如下关系式:
\begin{equation}
E_{r}A_{r}\frac{\partial u(L,t)}{\partial x}-p_{f}(L^{*},t)(A_{p}-A_{r})+p_{c}A_{p}=0
\tag{4-22}
\end{equation}
其中,$p_{c}$ 是柱塞处的套压。

环空流体速度与抽油杆速度之间关系:
\begin{equation}
\frac{\partial u(L,t)}{\partial t}(A_{p}-A_{r})=v_{f}(L^{*},t)(A_{l}-A_{r})
\tag{4-23}
\end{equation}

第三种情况:游动阀开启,固定阀关闭时,抽油杆的应力和柱塞处液体压力间关系表示为:
\begin{equation}
E_{r}A_{r}\frac{\partial u(L,t)}{\partial x}+p_{f}(L^{*},t)A_{r}=0
\tag{4-24}
\end{equation}

环空流体速度与抽油杆速度之间关系:
\begin{equation}
\frac{\partial u\left(L,t\right)}{\partial t}(L,t)A_{r}+v_{f}\left(L^{*},t\right)\left(A_{f}-A_{r}\right)=0
\tag{4-25}
\end{equation}

(5) 初始条件

悬点处于下死点处, 抽油杆速度为 0 , 此时对于任意点 $x$, $0 \leq x \leq L$

\begin{equation}
\frac{\partial u\left(x,0\right)}{\partial t}=0
\tag{4-26}
\end{equation}

\begin{equation}
v_{f}\left(x,0\right)=0
\tag{4-27}
\end{equation}

对于某一段抽油杆

\begin{equation}
E_{r}A_{r}\frac{\partial u\left(x,0\right)}{\partial x}=gA_{r}\left[\rho_{r}\left(L-x\right)-\rho_{f}L\right]-A_{r}p_{th}
\tag{4-28}
\end{equation}

\begin{equation}
p_{f}\left(x,0\right)=p_{th}+xg\rho_{f}
\tag{4-29}
\end{equation}

其中, $p_{th}$ 为油压。

\subsection*{4.4.4 阻尼系数模型问题分析}

根据地面示功图计算井下示功图时, 必须首先确定阻尼系数。抽油杆柱系统的阻尼力包括粘滞阻尼力和非粘滞阻尼力。粘滞阻尼力有抽油杆、接箍与环空液体之间的粘滞摩擦力, 泵阀和阀座内孔的流体压力损失等; 非粘滞阻尼力包括杆柱及接箍与油管之间的非粘滞摩擦力, 盘根盒与光杆间的摩擦力, 泵柱塞与泵筒之间的摩擦损失等。Gibbs 阻尼系数为假设在光杆简谐运动的条件下, 当光杆速度的均方根等于整个抽油杆的平均速度的条件下推到出来的近似计算公式。由此 Gibbs 阻尼系数计算条件较为理想, 实际的情况很难满足此条件。而抽油杆柱系统的阻尼是受各种复杂因素影响的综合结果。因而 Gibbs 阻尼系数计算公式在某些情况下的计算误差可能较大。

针对于 Gibbs 阻尼系数计算方法的不足, 本文提出了一种通过计算抽油杆柱在一个循环中由粘滞阻力引起的摩擦功来确定阻尼系数的方法。

\subsection*{4.4.5 摩擦功阻尼系数模型假设}

(1) 系统中消除等值阻力时, 每一个循环中的能量与消除真实阻尼时相同;

(2) 忽略抽油杆柱的弹性, 把抽油杆柱看作刚体;

(3) 把悬点运动简化为简谐运动;

\subsection*{4.4.6 摩擦功阻尼系数模型建立}

抽油杆柱长度为 $dx$ 的单元体的粘滞阻尼力为:

\begin{equation}
F_{d}=C_{F}\frac{\partial u}{\partial t}dx=c\rho_{r}A_{r}\frac{\partial u}{\partial t}dx
\tag{4-30}
\end{equation}

长度为 $L$ 的抽油杆柱在一个循环中总的粘滞摩擦功为

\begin{equation}
W = \int_{0}^{L} \int_{0}^{T} F_{d} \frac{\partial u}{\partial t} dt dx = c \rho_{r} A_{r} \int_{0}^{L} \left[ \int_{0}^{T} \left( \frac{\partial u}{\partial t} \right)^{2} dt \right] dx
\tag{4-31}
\end{equation}

\begin{equation}
c = \frac{W}{\rho_{r} A_{r} \int_{0}^{L} \left[ \int_{0}^{T} \left( \frac{\partial u}{\partial t} \right)^{2} dt \right] dx}
\tag{4-32}
\end{equation}

利用将活塞提出泵筒(或脱卸器使抽油杆脱开活塞)后测得的示功图,就可以计算出抽油杆柱在粘性流体中运动时的摩擦功。每口井的抽汲条件不同,则摩擦功不同。在实际工作中,又不可能对所分析的油井都能进行这种专门的测试。下面介绍了如何根据油井流体性质及抽汲参数推导出计算摩擦功的公式。

抽油杆柱在液体中运动时,杆周围液体速度的分布可用不可压缩粘滞液体运动的微分方程来描述。

\begin{equation}
\nu \left( \frac{\partial^{2} v}{\partial r^{2}} + \frac{1}{r} \frac{\partial v}{\partial r} \right) + \left( g - \frac{g}{\gamma} \frac{\partial p}{\partial x} \right) = \frac{\partial v}{\partial x}
\tag{4-33}
\end{equation}

所给问题的边界条件

\begin{equation}
v \bigg|_{r=r_{0}} = \frac{\partial u}{\partial t} \text{(抽油杆的运动速度)}
\tag{4-34}
\end{equation}

\begin{equation}
v \bigg|_{r=r_{1}} = 0
\tag{4-35}
\end{equation}

当抽油杆柱向下运动时,从环形空间被排出的液体体积流量为 $\pi r_{0}^{2} \frac{\partial u}{\partial t} \bigg|_{x=L}$,则

\begin{equation}
2 \pi \int_{r_{0}}^{r_{1}} v r dr = - \pi r_{0}^{2} \frac{\partial u}{\partial t} \bigg|_{x=L}
\tag{4-36}
\end{equation}

\begin{equation}
\frac{\partial P}{\partial x} = \frac{P_{1} - P_{2}}{L}
\end{equation}

$P_{1}$ 及 $P_{2}$ 分别为油管内抽油杆柱下端和井口的压力。

将方程 (4-33) 改写为

\begin{equation}
\frac{1}{r} \frac{\partial}{\partial r} \left( r \frac{\partial v}{\partial r} \right) = \frac{1}{\mu} \left[ \frac{\gamma}{g} \frac{\partial u}{\partial t} - \left( \gamma - \frac{\partial P}{\partial x} \right) \right]
\tag{4-37}
\end{equation}

为了简便地求得抽油杆柱在液体中运动时,一个周期的粘滞摩擦功,先研究某一瞬时、某一断面处油管环形空间中的流速分布。忽略流体运动的惯性力影响,则 $\frac{\gamma}{g} \frac{\partial v}{\partial t} = 0$。

令

\begin{equation}
A = - \left( \gamma - \frac{\partial P}{\partial x} \right)
\tag{4-38}
\end{equation}

则方程(4-37)变为
\begin{equation}
\frac{\partial}{\partial r}\left(r\frac{\partial v}{\partial r}\right)=\frac{A}{\mu}r
\tag{4-39}
\end{equation}
积分上式可得其通解
\begin{equation}
v=\frac{A}{4\mu}r^{2}+C_{1}\ln r+C_{2}
\tag{4-40}
\end{equation}
利用边界条件求得 $C_{1}$ 及 $C_{2}$ 后得
\begin{equation}
v=\frac{-A}{4\mu}\left(r_{1}^{2}-r\right)+\frac{\frac{A}{4\mu}\left(r_{1}^{2}-r_{0}^{2}\right)+\frac{\partial u}{\partial t}}{\ln\frac{r_{1}}{r}}
\tag{4-41}
\end{equation}
将(4-41)代入(4-36)后得
\begin{equation}
A=\frac{4\mu}{r_{0}^{2}}\frac{2}{B_{2}}\left(B_{1}\frac{\partial u}{\partial t}+\frac{\partial u}{\partial t}\bigg|_{x=L}\right)
\tag{4-42}
\end{equation}
式中 $B_{1}=\frac{m^{2}-1}{2\ln m}-1$;$B^{2}=m^{4}-1-\frac{\left(m^{2}-1\right)^{2}}{\ln m}$;$m=\frac{r_{1}}{r_{0}}$

抽油杆柱侧表面的粘滞摩擦力梯度为

\begin{equation}
\frac{\partial T}{\partial x}=-2\pi r_{0}\mu\left(\frac{\partial v}{\partial r}\right)_{r=r_{0}}
\end{equation}

将 $A$ 代入(4-41),并对 $r$ 求导后得

\begin{equation}
\frac{\partial v}{\partial r}\bigg|_{r=r_{0}}=\frac{-1}{r_{0}}\left[\left(\frac{1}{\ln m}+\frac{4B_{1}^{2}}{B_{2}}\right)\frac{\partial u}{\partial t}+\frac{4B_{1}}{B_{2}}\frac{\partial u}{\partial t}\bigg|_{x=L}\right]
\end{equation}

则

\begin{equation}
\frac{\partial T}{\partial x}=2\pi\mu\left[\left(\frac{1}{\ln m}+\frac{4B_{1}^{2}}{B_{2}}\right)\frac{\partial u}{\partial t}+\frac{4B_{1}}{B_{2}}\frac{\partial u}{\partial t}\bigg|_{x=L}\right]
\tag{4-43}
\end{equation}

将 $A$ 代入(9)后得到压力梯度

\begin{equation}
\frac{\partial P}{\partial x}=\frac{4\mu}{r_{0}^{2}}\frac{2}{B_{2}}\left(B_{1}\frac{\partial u}{\partial t}+\frac{\partial u}{\partial t}\bigg|_{x=L}\right)+\gamma
\tag{4-44}
\end{equation}

总的水力阻力梯度为

\begin{equation}
\frac{\partial F}{\partial x}=\frac{\partial T}{\partial x}+\pi r_{0}^{2}\frac{\partial P}{\partial x}
\tag{4-45}
\end{equation}

在 $x$ 断面处所取单元体 $dx$ 在一个循环中的粘滞摩擦功为

\begin{equation}
dw=\int_{0}^{T}\frac{\partial F}{\partial x}dx\frac{\partial u}{\partial t}dt
\tag{4-46}
\end{equation}

将(4-43)、(4-44)、(4-45)代入(4-46)中,其中 $\int_{0}^{T} \frac{\partial u}{\partial t} dt = 0$,最后可得
\begin{equation}
dw = 2\pi\mu \left\{ \left[ \frac{1}{\ln m} + \frac{4B_1}{B_2}(B_1 + 1) \right] \int_{0}^{T} \left( \frac{\partial u}{\partial t} \right)^2 dt + \frac{4}{B_2}(B_1 + 1) \int_{0}^{T} \left. \frac{\partial u}{\partial t} \right|_{x=L} \frac{\partial u}{\partial t} dt \right\} dx
\end{equation}

整个抽油杆往在一个循环中总粘滞摩擦功
\begin{equation}
W = 2\pi\mu \left\{ \left[ \frac{1}{\ln m} + \frac{4B_1}{B_2}(B_1 + 1) \right] \int_{0}^{L} \int_{0}^{T} \left( \frac{\partial u}{\partial t} \right)^2 dt dx \right.
\end{equation}
\begin{equation}
\left. + \frac{4}{B_2}(B_1 + 1) \int_{0}^{L} \int_{0}^{T} \left. \frac{\partial u}{\partial t} \right|_{x=L} \frac{\partial u}{\partial t} dt dx \right\}
\tag{4-47}
\end{equation}

将摩擦功公式(4-47)代入(4-32)得
\begin{equation}
c = \frac{2\pi\mu}{\rho_r A_r} \left\{ \frac{1}{\ln m} + \frac{4B_1}{B_2}(B_1 + 1) \left[ B_1 + \frac{\int_{0}^{L} \left[ \int_{0}^{T} \left. \frac{\partial u}{\partial t} \right|_{x=L} \frac{\partial u}{\partial t} dt \right] dx}{\int_{0}^{L} \left[ \int_{0}^{T} \left( \frac{\partial u}{\partial t} \right)^2 dt \right] dx} \right] \right\}
\tag{4-48}
\end{equation}

式(4-48)中的 $\frac{\partial u}{\partial t}$ 是未知的,还不能用来求 $c$ 值。

在计算 $c$ 时,如果忽略抽油杆柱的弹性,即把抽油杆柱暂时看作刚体,则杆柱各断面的运动速度相同,(4-48)式中的两个积分相等。此时可用下式求得 $c$ 的近似值
\begin{equation}
c = \frac{2\pi\mu}{\rho_r A_r} \left[ \frac{1}{\ln m} + \frac{4}{B_2}(B_1 + 1)^2 \right]
\tag{4-49}
\end{equation}

另一种解决办法是把悬点运动看作简谐运动,在确定 $\frac{\partial u}{\partial t}$ 时,先不考虑粘滞阻尼(即先令 $c = 0$),把所研究的问题先归结为上端做简谐运动、下端自由的细长杆的无阻尼纵振问题。
\begin{equation}
\frac{\partial^2 u}{\partial t^2} = a^2 \frac{\partial^2 u}{\partial x^2}
\tag{4-50}
\end{equation}

边界条件
\begin{equation}
u \big|_{x=0} = \frac{S}{2}(1 - \cos \omega t)
\end{equation}
\begin{equation}
-A_r E \left. \frac{\partial u}{\partial x} \right|_{x=L} = 0
\end{equation}

在上述边界条件下解方程(4-50)可得
\begin{equation}
\frac{\partial u}{\partial t} = \frac{S}{2} \omega \left( \cos \frac{\omega}{a} x + \tan \frac{\omega}{a} L \sin \frac{\omega}{a} x \right) \sin \omega t
\tag{4-51}
\end{equation}

\begin{equation}
\left. \frac{\partial u}{\partial t} \right|_{x=L} = \frac{S}{2} \omega \frac{1}{\cos \frac{\omega}{a} L} \sin \omega t
\tag{4-52}
\end{equation}

将(4-51)、(4-52)式(4-48)得:

\begin{equation}
c = \frac{2 \pi \mu}{\rho_r A_r} \left\{ \frac{1}{\ln m} + \frac{4}{B_2} (B_1 + 1) \left[ B_1 + \frac{2}{\frac{\omega L}{a} \frac{1}{\sin \frac{\omega L}{a}} + \cos \frac{\omega L}{a}} \right] \right\}
\tag{4-53}
\end{equation}

计算 \( c \) 值的更精确的方法是由(2-12)得到 \(\frac{\partial u}{\partial t}\) 的一般表达式:

\begin{equation}
\frac{\partial u}{\partial t} = \sum_{n=1}^{\bar{n}} n \omega P_n \cos n \omega t - n \omega O_n(x) \sin n \omega t
\tag{4-54}
\end{equation}

先假定 \( c = 0 \),由(4-54)式求得 \(\frac{\partial u}{\partial t}\) 后代入(4-48)式,初步计算出 \( c \) 的近似值,然后再进行迭代计算,最后便可得到更精确的 \( c \) 值,但计算量很大。

上述计算中只考虑了液体与抽油杆体的摩擦,实际上,抽油杆接箍将会引起附加的摩擦,为此,在具体“诊断”程序中计算粘滞阻尼系数 \( c \) 时,应该考虑接箍引起的摩擦。对于多级杆柱则应分级计算其阻尼系数。

计算 \( c \) 时所用的粘度应该是泵以上油管中流体的平均粘度。为了能得到比较准确的粘度值,应先计算油管内的压力和温度分布,然后计算相应温度和压力分布下的平均粘度。

\subsection*{4.4.7 考虑接箍摩擦的阻尼系数模型}

上述摩擦功阻尼系数模型只适用于单机抽油杆且为考虑附加接箍引起的阻尼。然而实际的抽油井中的抽油杆柱受到的粘滞摩擦力为流体对抽油杆柱的摩擦力加上环空流体对抽油杆接箍的摩擦阻力。本模型主要考虑了这两个方面的摩擦阻力,建立了较为准确的阻尼系数计算模型。

抽油杆柱上受到的粘滞摩擦阻力由两部分组成,环控中流体对抽油杆柱的阻力和环空中流体对抽油杆柱接箍上的摩擦阻力:

\begin{equation}
F_r = 0.5 \rho_f v_f \left| v_f \right| \pi D_r C_1 + 0.5 \rho_f v_f \left| v_f \right| (A_c - A_r) C_2 / l
\tag{4-55}
\end{equation}

由 4.4.3 节的推导可知:

\begin{equation}
F_r = b_r \frac{\partial u}{\partial t} - b_{fr} v_f
\tag{4-56}
\end{equation}

式(4-56)中的系数为:

\begin{equation}
\begin{cases}
b_{r}=12\pi\mu[(0.2+0.39\frac{D_{r}}{D_{t}})+915.42(\frac{D_{c}}{D_{t}}-0.381)^{2.57}((\frac{D_{c}}{D_{r}})^{2}-1)\frac{D_{r}}{l}(\frac{D_{t}}{D_{r}}-1)^{-1}] \\
b_{fr}=12\pi\mu[1+1500.42(\frac{D_{c}}{D_{t}}-0.381)^{2.57}((\frac{D_{c}}{D_{r}})^{2}-1)\frac{D_{r}}{l}(\frac{D_{t}}{D_{r}}-1)^{-1}]
\end{cases}
\tag{4-57}
\end{equation}

又因为抽油杆柱长度为 \(dx\) 的单元体的粘滞阻尼力为:
\begin{equation}
F_{r}dx=c\rho_{r}A_{r}\frac{\partial u}{\partial t}dx
\tag{4-58}
\end{equation}

将(4-56)代入(4-58)中得到:
\begin{equation}
(c\rho_{r}A_{r}-b_{r})\frac{\partial u}{\partial t}+b_{rf}v_{f}=0
\tag{4-59}
\end{equation}

方程中需要知道抽油杆和环空流体的速度项,因而本方法为一个隐式求解阻尼系数的方法,首先假设一个阻尼系数 \(c\),计算出抽油杆的速度和环空流体的速度,然后由方程(4-59)计算出真实 \(c'\) 的阻尼系数,将两个阻尼系数做比较,若相差较大,则把 \(c'\) 作为初始值重新代入到阻尼波动方程中,然后再计算新的阻尼系数直到阻尼系数 \(\left|\frac{c-c'}{c}\right|\leq e\) 足够小,则可求出阻尼系数 \(c\)。

\section*{五、模型的评价及改进}

(1)问题一:本文通过对简谐悬点运动模型和曲柄滑块悬点运动模型的评价,结合抽油杆四连杆几何模型,建立了悬点运动模型,并分析了该模型与实际悬点运动位移的误差,提出了带冲程损失位移的悬点运动模型。由此可以得到,在建立四连杆机构的运动模型时,不仅要考虑四连杆机构的静载荷下的几何特征,还要考虑抽油机在工作过程中所承受的动载荷对悬点运动规律的影响。

(2)问题二:本节主要采用Gibbs提出的带阻尼的波动方程,将地面悬点示功图转化为了泵功图,在计算过程中,其波动方程的阻尼系数采用的多项式拟合的方法求解。因而阻尼系数的求解就会产生一定的误差,较为准确的阻尼系数的求解需要对抽油杆模型进行详细的受力分析后,考虑抽油杆的真实阻力来建立相应的阻尼模型求解阻尼系数。

(3)问题三:本节主要建立了两种通过泵功图计算油井日产液量的计算模型。两种模型主要基于对泵功图的图形处理,本文主要采用Origin软件对其进行图形处理,该图形处理技术在计算过程中会产生较大的误差,进一步的改进图形处理技术将会大大提高油井日产液量的计算精度。同时,文本还建立了基于灰色理论的抽油井泵功图诊断模型。

(4)问题四:本节对Gibbs模型进行了原理分析,通过研究发现,Gibbs模型在建立波动方程时,对抽油机进行了较多的简化假设,而且其仅仅考虑了抽油杆柱在抽油过程中的振动,忽略了油管环空内液柱的惯性及振动,同时也忽略了油管未锚定时,油管的振动。因此,针对于此情况,本文建立了同时考虑液柱和抽油杆柱振动的二维模型。但此模型仍然存在许多的简化假设条件。本文提出的二维模型还可以进行如下改进:同时考虑抽油杆柱振动、环空中液柱振动和油管振动的三维振动数学模型;考虑抽油杆与油管之间存在偏磨情况的抽油系统模型;考虑定向井的振动模型,即考虑抽油井为有一定倾斜角的情况下的振动模型;考虑环空流体为多相流时抽油井的振动模型,即考虑环空流体存在气相的情况。

\section*{六、附录}

\section*{Gibbs 模型计算泵功图 Matlab 程序}

(1) 一级杆泵功图计算 Matlab 程序

\texttt{Gibbsjs1.m}

\begin{verbatim}
clc; clear;
D = xlsread('DINPUT.xls'); u = xlsread('UINPUT.xls');
K = 144; n = 100; pi = 3.1415926; a = 4983.416; L = 792.5; v = 0.6633;
omega = 2 * pi * 7.6 / 60; E = 2.1 * (10^11); d1 = 0.022; Ar1 = pi * d1^2 / 4;
x = 792.5;

sigma = zeros(n, 1); tau = zeros(n, 1); gamma = zeros(n, 1); delta = zeros(n, 1);
sigma0 = (2 / K) * sum(D(1:K));
gamma0 = (2 / K) * sum(u(1:K));

for i = 1:n
    for p = 1:K
        sigma(i) = sigma(i) + (2 / K) * D(p) * cos(2 * pi * i * p / K);
        tau(i) = tau(i) + (2 / K) * D(p) * sin(2 * pi * i * p / K);
        gamma(i) = gamma(i) + (2 / K) * u(p) * cos(2 * pi * i * p / K);
        delta(i) = delta(i) + (2 / K) * u(p) * sin(2 * pi * i * p / K);
    end
end

cd = 2.29595 - 44.32202 * v + 459.29115 * v^2 - 2790.90409 * v^3 + 10451.0482 * v^4 - ...
     24714.8911 * v^5 + 36933.77379 * v^6 - 33779.575 * v^7 + 17244.66252 * v^8 - ...
     3761.98482 * v^9;

c = pi * a * cd / (2 * L);
alpha = zeros(n, 1); beta = zeros(n, 1);

for i = 1:n
    alpha(i) = ((i * omega) / (a * 2^0.5)) * (1 + (1 + (c / (i * omega))^2)^0.5)^0.5;
    beta(i) = ((i * omega) / (a * 2^0.5)) * (-1 + (1 + (c / (i * omega))^2)^0.5)^0.5;
end

kappa = zeros(n, 1); mu = zeros(n, 1); On = zeros(n, 1);
Pn = zeros(n, 1); dOn = zeros(n, 1); dPn = zeros(n, 1);

for i = 1:n
    kappa(i) = (sigma(i) * alpha(i) + tau(i) * beta(i)) / ...
               (E * Ar1 * (alpha(i)^2 + beta(i)^2));
\end{verbatim}

\begin{verbatim}
mu(i) = (sigma(i) * beta(i) - tau(i) * alpha(i)) / ...
        (E * Ar1 * (alpha(i)^2 + beta(i)^2));
On(i) = (kappa(i) * cosh(beta(i) * x) + delta(i) * ...
        sinh(beta(i) * x) * sin(alpha(i) * x) + ...
        mu(i) * sinh(beta(i) * x) + gamma(i) * ...
        cosh(beta(i) * x) * cos(alpha(i) * x));
Pn(i) = (kappa(i) * sinh(beta(i) * x) + delta(i) * ...
        cosh(beta(i) * x) * cos(alpha(i) * x) - ...
        mu(i) * cosh(beta(i) * x) + gamma(i) * ...
        sinh(beta(i) * x) * sin(alpha(i) * x));
dOn(i) = (tau(i) * sinh(beta(i) * x) / (E * Ar1) + ...
        (delta(i) * beta(i) - gamma(i) * alpha(i)) * cosh(beta(i) * x) * ...
        sin(alpha(i) * x) + (sigma(i) * cosh(beta(i) * x) / (E * Ar1) + ...
        (gamma(i) * beta(i) + delta(i) * alpha(i)) * sinh(beta(i) * x) * ...
        cos(alpha(i) * x));
dPn(i) = (tau(i) * cosh(beta(i) * x) / (E * Ar1) + ...
        (delta(i) * beta(i) - gamma(i) * alpha(i)) * sinh(beta(i) * x) * ...
        cos(alpha(i) * x) + (sigma(i) * sinh(beta(i) * x) / (E * Ar1) + ...
        (gamma(i) * beta(i) + delta(i) * alpha(i)) * cosh(beta(i) * x) * ...
        sin(alpha(i) * x));
end
t = xlsread('TINPUT.xls'); zjbl1 = 0; zjbl2 = 0;
ul = zeros(K, 1); Dl = zeros(K, 1);
for p = 1:K
    zjbl1 = 0; zjbl2 = 0;
    for i = 1:n
        zjbl1 = zjbl1 + On(i) * cos(i * omega * t(p)) + Pn(i) * sin(i * omega * t(p));
        zjbl2 = zjbl2 + dOn(i) * cos(i * omega * t(p)) + dPn(i) * sin(i * omega * t(p));
    end
    ul(p) = (sigma0 * x) / (2 * E * Ar1) + gamma0 / 2 + zjbl1;
    Dl(p) = (sigma0 / 2 + E * Ar1 * zjbl2) / 1000;
end
\end{verbatim}

(2) 三级杆泵功图计算Matlab程序

Gibbsjs2.m

\begin{verbatim}
clc; clear;
D = xlsread('DINPUT2.xls'); u = xlsread('UINPUT2.xls'); t = xlsread('TINPUT2.xls');
K = 143; n = 100; pi = 3.1415926; a = 4983.416; T = 3;
omega = 2 * pi * 4 / 60; E = 2.1 * (10^11); v = 0.6347;
x = [523.61 664.32 618.35]; L = 1819.56;
d1 = [0.025 0.022 0.019]; Ar1 = zeros(1, T);
for i = 1:3
    Ar1(i) = pi * ((d1(i))^2) / 4;
end
sigma = zeros(n, 1); tau = zeros(n, 1); gamma = zeros(n, 1); delta = zeros(n, 1);
sigma0 = (2 / K) * sum(D(1:K));
gamma0 = (2 / K) * sum(u(1:K));
for i = 1:n
    for p = 1:K
        sigma(i) = sigma(i) + (2 / K) * D(p) * cos(2 * pi * i * p / K);
        tau(i) = tau(i) + (2 / K) * D(p) * sin(2 * pi * i * p / K);
        gamma(i) = gamma(i) + (2 / K) * u(p) * cos(2 * pi * i * p / K);
        delta(i) = delta(i) + (2 / K) * u(p) * sin(2 * pi * i * p / K);
    end
end
for q = 1:T
    cd = 2.29595 - 44.32202 * v + 459.29115 * v^2 - 2790.90409 * v^3 + 10451.0482 * v^4 - ...
         24714.8911 * v^5 + 36933.77379 * v^6 - 33779.575 * v^7 + 17244.66252 * v^8 - ...
         3761.98482 * v^9;
    c = pi * a * cd / (2 * L);
    alpha = zeros(n, 1); beta = zeros(n, 1);
    for i = 1:n
        alpha(i) = ((i * omega) / (a * 2^0.5)) * (1 + (1 + (c / (i * omega))^2)^0.5)^0.5;
        beta(i) = ((i * omega) / (a * 2^0.5)) * (-1 + (1 + (c / (i * omega))^2)^0.5)^0.5;
    end
    kappa = zeros(n, 1); mu = zeros(n, 1); On = zeros(n, 1);
    Pn = zeros(n, 1); dOn = zeros(n, 1); dPn = zeros(n, 1);
    for i = 1:n
\end{verbatim}

\begin{align*}
\text{kappa}(i) &= (\text{sigma}(i) * \text{alpha}(i) + \text{tau}(i) * \text{beta}(i)) / \dots \\
&\quad (E * \text{Ar1}(q) * (\text{alpha}(i)^2 + \text{beta}(i)^2)); \\
\text{mu}(i) &= (\text{sigma}(i) * \text{beta}(i) - \text{tau}(i) * \text{alpha}(i)) / \dots \\
&\quad (E * \text{Ar1}(q) * (\text{alpha}(i)^2 + \text{beta}(i)^2)); \\
\text{On}(i) &= (\text{kappa}(i) * \cosh(\text{beta}(i) * x(q)) + \text{delta}(i) * \dots \\
&\quad \sinh(\text{beta}(i) * x(q)) * \sin(\text{alpha}(i) * x(q)) + \dots \\
&\quad \text{mu}(i) * \sinh(\text{beta}(i) * x(q)) + \text{gamma}(i) * \dots \\
&\quad \cosh(\text{beta}(i) * x(q)) * \cos(\text{alpha}(i) * x(q))); \\
\text{Pn}(i) &= (\text{kappa}(i) * \sinh(\text{beta}(i) * x(q)) + \text{delta}(i) * \dots \\
&\quad \cosh(\text{beta}(i) * x(q)) * \cos(\text{alpha}(i) * x(q)) - \dots \\
&\quad \text{mu}(i) * \cosh(\text{beta}(i) * x(q)) + \text{gamma}(i) * \dots \\
&\quad \sinh(\text{beta}(i) * x(q)) * \sin(\text{alpha}(i) * x(q))); \\
\text{dOn}(i) &= (\text{tau}(i) * \sinh(\text{beta}(i) * x(q)) / (E * \text{Ar1}(q)) + \dots \\
&\quad (\text{delta}(i) * \text{beta}(i) - \text{gamma}(i) * \text{alpha}(i)) * \cosh(\text{beta}(i) * x(q)) * \dots \\
&\quad \sin(\text{alpha}(i) * x(q)) + (\text{sigma}(i) * \cosh(\text{beta}(i) * x(q)) / (E * \text{Ar1}(q)) + \dots \\
&\quad (\text{gamma}(i) * \text{beta}(i) + \text{delta}(i) * \text{alpha}(i)) * \sinh(\text{beta}(i) * x(q)) * \dots \\
&\quad \cos(\text{alpha}(i) * x(q))); \\
\text{dPn}(i) &= (\text{tau}(i) * \cosh(\text{beta}(i) * x(q)) / (E * \text{Ar1}(q)) + \dots \\
&\quad (\text{delta}(i) * \text{beta}(i) - \text{gamma}(i) * \text{alpha}(i)) * \sinh(\text{beta}(i) * x(q)) * \dots \\
&\quad \cos(\text{alpha}(i) * x(q)) + (\text{sigma}(i) * \sinh(\text{beta}(i) * x(q)) / (E * \text{Ar1}(q)) + \dots \\
&\quad (\text{gamma}(i) * \text{beta}(i) + \text{delta}(i) * \text{alpha}(i)) * \cosh(\text{beta}(i) * x(q)) * \dots \\
&\quad \sin(\text{alpha}(i) * x(q))); \\
\text{end} \\
\text{zjbl1} &= 0; \text{zjbl2} = 0; \\
\text{ul} &= \text{zeros}(K, 1); \text{D1} = \text{zeros}(K, 1); \\
\text{for } p &= 1:K \\
&\text{zjbl1} = 0; \text{zjbl2} = 0; \\
&\text{for } i = 1:n \\
&\quad \text{zjbl1} = \text{zjbl1} + \text{On}(i) * \cos(i * \text{omega} * t(p)) + \text{Pn}(i) * \sin(i * \text{omega} * t(p)); \\
\end{align*}

\begin{verbatim}
zjbl2 = zjbl2 + dOn(i) * cos(i * omega * t(p)) + dPn(i) * sin(i * omega * t(p));
end
ul(p) = (sigma0 * x(q)) / (2 * E * Arl(q)) + gamma0 / 2 + zjbl1;
D1(p) = (sigma0 / 2 + E * Arl(q) * zjbl2) / 1000;
end
gamma0 = sigma0 * x(q) / (E * Arl(q)) + gamma0;
for i = 1:n
    sigma(i) = E * Arl(q) * dOn(i);
    tau(i) = E * Arl(q) * dPn(i);
    gamma(i) = On(i);
    delta(i) = Pn(i);
end
end
\end{verbatim}

\section*{参考文献}

[1] 王鸿勋,张琪,采油工业原理,石油行业出版社,1985 年 4 月,第二章。

[2] 万仁溥,采油工程手册,石油行业出版社,2000 年 8 月,第五章 第二节。

[3] 张琪,采油工程原理与设计,山东东营:中国石油大学出版社,2006 年 12 月,第三章,第六节。

[4] K.E.布朗,升举法采油工艺,北京:石油工业出版社,1987 年 8 月,第二章,第五节。

[5] 万仁溥,采油工程手册(上册),北京:石油工业出版社,2000 年 8 月,第五章 第四节。

[6] 簠瑛,基于灰色理论的抽油井泵功图诊断技术研究,微电子学与计算机,第 22 卷第三期:77-80,2005。

[7] 余金泽,彭永刚,关成尧,张杰,“功图量油”技术的发展,石油工程技术:51-54。

[8] 张琪,吴晓东,抽油机计算机诊断技术及其应用,华东石油学院学报,第二期:144-159,1985。

[9] 刘清友,王海兰,刘富,有杆抽油泵系统工作行为仿真研究,石油学报,第 25 卷 第三期:106-109,2004 年 5 月。

[10] 余国安,郭亦炯,王国源,有杆泵抽油井的三维振动,石油学报,第十卷 第二期:76-83,1989 年 4 月。

[11] Gibbs. S. G, Neely, A. B, Computer Diagnosis of Downhole Condition in Sucker Rod Pumping Wells, J.Pet.Tech., Jan.1966.

[12] Gibbs. S. G, Method of Determining Sucker Rod Pump Performance, United States Patent Office, Sep.1967.

[13] S. G. Gibbs, A. B. Neely, Computer diagnosis of down-hole conditions in sucker rod pumping wells, Journal of petroleum technology, SPE1165: 93-98, January 1966.

[14] G. W. Wang, S. S. Rahman and G. Y. Yang, An improved model for the sucker rod pumping system, 11\textsuperscript{th} Australasian Fluid Mechanics Conference, Hobart, Australia, 9B-6: 1137-1140, December 2012.

[15] Dale Russel Doty, Zelimir Schmidt, An improved model for the sucker rod pumping, SPE10249: 33-41, February 1983.