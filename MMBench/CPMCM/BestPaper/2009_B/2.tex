\begin{center}
\textbf{全国第六届研究生数学建模竞赛}
\end{center}

\begin{center}
\includegraphics[width=0.5\textwidth]{image.png}
\end{center}

\begin{flushleft}
题目 \hspace{1cm} 基于三维测量仪器的子弹自动对比方法的研究
\end{flushleft}

\begin{center}
\textbf{摘 \hspace{3cm} 要}
\end{center}

本文解决了基于三维测量仪器的子弹自动对比中的原始数据处理和自动对比问题。

第一部分,本文将针对弹头测量姿态不一致的误差补偿问题转化为三坐标测量数据的重定位整合问题,并利用一种基于几何中心的三坐标测量数据的重定位整合方法以弥补已测得数据中缺乏固定基准点的不足。在此基础上,本文进一步通过展平弹头消除了弹头柱体弧度对自动对比的干扰。

第二部分,本文通过消除孤立点并结合中值滤波法除去了原始数据中的脉冲噪声和高斯噪声。

第三部分,本文首先将痕迹特征分为个体特征和类别特征,并通过彩图等高线算法分析得知个体特征是弹痕比对中的关键。因此,本文利用小波分析方法对已预处理的数据进行信号分离,获取能够反映枪膛线稳定信息的个体特征,并对代表个体特征的各频段信号特征采用 CCF(Cross-Correlation Function)方法定量表示弹痕在该频段的相似度。最后通过多尺度加权相关分析方法,对弹痕不同频段的相关度系数加权计算得到痕迹的总相关系数。通过使用总相关系数对问题三中的 12 个弹头两两比对进行模型评价。

第四部分,利用第三部分所提模型对 11 支手枪发射的 22 颗子弹进行了相互的匹配度计算,给出了每个弹头按相似度由高到低给出与之相似的前 5 位的弹头文件号。

最后,本文利用已知的 12 颗子弹的配对关系对第三部分所提模型进行了评价,得出该模型的比对准确率为 75%。同时,针对第三问中匹配时采用的特征方法的局限,进一步提出了 MAD 算法和绝对差分和两个特征用于弹痕的匹配,这两个特征为进一步提高匹配精度提供了可行方法,另外为联系各特征提供了基础。

\textbf{关键词:} 展平面,三坐标测量重定位,CCF,小波分析,绝对差分和

\begin{tabular}{l l}
参赛队号 & \underline{1069809} \\
队员姓名 & \underline{倪华,龙毅,牛国林} \\
\end{tabular}
\begin{tabular}{|c|}
\hline
参赛密码 \underline{\hspace{5cm}} \\
(由组委会填写) \\
\hline
\end{tabular}

\section*{1. 问题重述}

在涉枪案件的侦破过程中,检验发射弹头痕迹是判断作案枪种和枪支同一题的主要手段。准确、快速地鉴定和分析枪弹痕迹将对持枪案件的侦破起着重要作用。而弹头痕迹则是在出枪口的过程中与枪管膛线在火药压力作用下产生挤压、磨擦产生的。而由于制造、使用、保管等方面的原因,不可能有两支枪的膛线完全相同的枪。因此,枪弹痕迹鉴定是合理有效的。

在公安实践中,要根据弹头上的痕迹来判断两个弹头是否为同一支枪发射的传统方法是通过显微镜肉眼观察,比对两个弹头上的线条型痕迹,看这些线条的粗细分布是否吻合。但传统做法效率极低,弹头易发生锈蚀损坏等情况。现代高精度数据采集设备为自动比对方法创造了条件。三维光学测量仪器(原理如图 1)所获得的包含痕迹深度信息的子弹物证三维信息与传统的光学显微镜平面照片比对的方式相比有明显的信息量优势,且不接触弹头以免造成磨损。

\begin{figure}[h]
\centering
\includegraphics[width=0.8\textwidth]{image1.png}
\caption{三维光学检测仪原理}
\end{figure}

本文中考虑的枪支与枪弹主要是 77 制式手枪及与之匹配的枪弹。该类弹头上留下的痕迹分为 4 片主棱线的痕迹和 4 片次棱线的痕迹,由于子弹自身的旋转,痕迹与子弹母线之间通常有一个夹角。且根据实践经验,对同一支枪发射的两个弹头,次棱的吻合度通常比主棱高。因此本文中所用的是分 4 次测量的 4 条次棱的数据。

而本文采用的三维光学测量仪的测量过程则如图 2 所示。

\begin{figure}[h]
\centering
\includegraphics[width=0.8\textwidth]{image2.png}
\caption{三维光学检测仪测量过程}
\end{figure}

由图知,基准平面取为固定在测量设备上的空间直角坐标系的 \(xoy\) 平面,且在弹头痕迹的附近,沿 \(x\) 轴方向和沿 \(y\) 轴方向的测量步长均为 2.75 微米,\(z\) 的测量精度为 1 微米。每次装载弹头进行测量时均要进行姿态和位置的调节使得:

(1) 弹头圆柱中心线尽量平行于基准平面;(2) \(y\) 轴尽量平行于擦痕的走向。

枪弹自动对比过程主要分两步:第一步通过光学设备采集弹头上痕迹的 3 维数据;第二步则是采用适当的方法,通过电脑比对,判别一个弹头与样本弹头的相似程度,以便确认发射该弹头的枪支。然而在现实中,自动对比还面临以下 4 个待解决的问题:

(1) 在光学设备上采集弹头上痕迹的 3 维数据时,用于采集数据的光源和数据采集仪器是固定不动的,光源在被测弹头的侧上方,感光器在弹头的正上方;弹头由人工固定在支架上,可以由人工调节该弹头的位置和姿态(姿态是指弹头在空间的姿势或状态,通常与转动有关)。人工调节很难使两个弹头在同样位置和同样姿态情况下被测量,从而会造成测量误差。通常会造成 \(0.03 \, \text{mm}\) 左右的平移误差和 \(0.2^\circ\) 的转动误差。那么如何处理由位置和姿态造成的测量误差,使得两个弹头可以尽量在相同位置和相同姿态的情况下进行比对。

(2) 弹头表面的损伤、锈迹、油斑、杂质物及痕迹生成的随机性,会造成数据误差和噪声。采取怎样的方法去消除这些误差和噪声。

(3) 怎样的特征可以用于比对,并给出提取这些特征的方法,最终给出完整的比对方案、算法;

(4) 用解决问题 (3) 中所提的方法给出已知的 22 个弹头痕迹两两之间的相似程度,用列表的方式给出每个弹头按相似度由高到低给出与之相似的前 5 位的弹头编号。

\section*{2. 模型假设}

(1) 假设子弹中间部分(即留下痕迹的部分)是一个直径为 \(7.90 \, \text{mm}\),长度约为 \(12 \, \text{mm}\) 的规则圆柱体,且痕迹之外的部分不会产生形变;

(2) 假设每次测量时手动调节子弹姿态和位置造成的平移误差在 \(0.03 \, \text{mm}\) 以内,转动误差在 \(0.2^\circ\) 以内;

(3) 假设弹头圆柱中心线平行于基准平面,\(y\) 轴平行于擦痕的走向。

\section*{3. 符号说明}

\begin{table}[h]
\centering
\begin{tabular}{|c|c|}
\hline
\rowcolor[gray]{0.8} 符号表示 & 符号意义 \\
\hline
\(T\) & 测量数据重定位变换矩阵 \\
\hline
\(h\) & 弹头圆柱体部分的高 \\
\hline
\(r\) & 弹头圆柱体部分的半径 \\
\hline
\(\theta\) & 弹头痕迹与弹头母线的夹角 \\
\hline
\end{tabular}
\end{table}

\section{问题分析}

\subsection{问题一分析}

由光学仪器的测量过程可知, 每次装载弹头进行测量时均要进行姿态和位置的调节使得: (1) 弹头圆柱中心线尽量平行于基准平面; (2) $y$ 轴尽量平行于擦痕的走向。而这两个因素同时又是决定测量坐标系 $xoy$ 建立的直接因素。但是每次装载时的调节均为手动完成, 所以不能保证每次测量时的基准平面和 $y$ 轴的两个平行关系完全一致。这就意味着需要比对的两条次棱痕迹在测量时所采用的坐标系不是完全一致的, 因而所测得两条次棱痕迹数据不能用于直接对比的, 而是应将这些不同坐标系下的测得的数据变换到同一坐标系下后再进行比较, 而这个变换过程则称为测量数据的重定位 [1]。

图形变换一般是指对图形的几何信息经过几何变换后产生新的图形。图形既可以看作是坐标系不动而图形变动, 变动后的图形在坐标系中的坐标值发生变化; 也可以看作图形不动而坐标系变动, 变动后, 该图形在新的坐标系具有新的坐标值, 而这两种情况本质上是一样的。本文所讨论的几何变换属于后一种情况。

利用齐次坐标表达法, 可以方便地用变换矩阵实现对图形的变换。令三维图形变换前的一点坐标为 $(x, y, z, 1)$, 变换后为 $(X, Y, Z, 1)$, 变换矩阵 $T$ 表示如下:

\[
T = \begin{bmatrix}
t_{11} & t_{12} & t_{13} & t_{14} \\
t_{21} & t_{22} & t_{23} & t_{24} \\
t_{31} & t_{32} & t_{33} & t_{34} \\
t_{41} & t_{42} & t_{43} & t_{44}
\end{bmatrix}
\]

从变换功能上, 可以将 $T$ 分为 4 个矩阵, 其中

\[
\begin{bmatrix}
t_{11} & t_{12} & t_{13} \\
t_{21} & t_{22} & t_{23} \\
t_{31} & t_{32} & t_{33}
\end{bmatrix}
\]

产生比例, 旋转, 剪切等几何变换;

\[
\begin{bmatrix}
t_{41} & t_{42} & t_{43}
\end{bmatrix}
\]

产生平移变换;

\[
\begin{bmatrix}
t_{14} \\
t_{24} \\
t_{34}
\end{bmatrix}
\]

产生投影变换; $[t_{44}]$ 产生整体比例变换。图形变换的数学表达式可统一表示为:

\[
[X, Y, Z, 1] = [x, y, z, 1] \cdot T
\]

因此, 第一问实际上就是在手动调节的误差范围内 (即 $0.03 \mathrm{~mm}$ 左右的平移误差和 $0.2^{\circ}$ 的转动误差), 寻找合理的变换矩阵 $T$ 使两个定位下的测量数据尽量变换到同一定位下进行比较。而且由题意可知

同时就子弹检测而言, 由于子弹外表面有由于枪支的膛线引起的凸凹不平的痕迹, 导致了弹头横截面边缘线 $C$ (如图 3 所示) 上有一些齿状的曲线。这些齿状曲线就成了判断两枚子弹是否为同一支枪所发射的重要依据。检测人员真正感兴趣的是那些凹凸不平的痕迹, 也就是 $C$ 上的一些小的齿状特。这样看来, 曲线 $C$ 包含了一个冗余信息: $C$ 上圆弧的弧度。若需要比对的两颗的型号一样, 大小就一样, 这个圆的半径大小也就固定了。圆弧并不能成为区颗子弹的依据, 不是比对人员所关心的特征。

\begin{figure}[h]
    \centering
    \includegraphics[width=0.8\textwidth]{image.png}
    \caption{子弹的截面曲线 $C$}
    \label{fig:bullet_cross_section}
\end{figure}

实际上,这个信息不但是多余的,还给比对带来了困难。从几何上来看,封闭曲线 $C$ 实际上是一个半径固定的圆与一些小的齿状曲线的叠加。这个弹头固有的圆叠加给齿状曲线一个弧度,改变了齿状曲线原本的形状,对特征提取的精度是不利的。因为在相同的高度测量弹头,这是准确比对的基本要求,所以如何去掉这个圆弧便成为一个需要解决的问题。

\subsection{问题二分析}

通常通过光学测量装置得到的测量点数据数目相当庞大,并且获得的三维坐标数据由于受测量设备的精度、操作者经验和被测实物表面质量等诸多因素的影响,不可避免地会引入数据误差,即含有噪声数据。因此,对测量点数据进行预处理是实现枪弹痕迹自动比对的关键环节,其处理结果将直接影响枪弹痕迹匹配的准确度。为了保证后续工作的精度,需要通过适当的算法,对测量数据进行去噪、平滑、对齐等预处理工作,目的是获得完整、正确的测量数据。本文数据预处理主要为数据点的噪声滤波。

通过对已测得数据的分析可知,本文数据中的噪声主要由高斯噪声和脉冲噪声组成,因此第二问的解决依赖于如何分别选择合适的除噪模型来除去这两种噪声。

\subsection{问题三分析}

问题三中需要对数据格式进行处理以提取出能够稳定反映枪膛线次棱痕迹的数据特征,并在此基础上合理选择有效数据,进行弹头比对。即要完成如下的过程:特征提取 $\rightarrow$ 合理选取数据 $\rightarrow$ 特征比对。

枪膛的同一性识别一般都和两个特征相关:类别特征和个体特征。因此弹头痕迹也包括了这两类特征。枪膛的类别特征主要能够反映枪的种类的特征,通常和枪支的制造商是相关的。在弹头痕迹的识别中,这类特征可以用来定位枪支的种类,从而减少工作量,将枪支的目标范围从庞大的数据库级别缩小至可以控制的级别。这类特征反应到弹头痕迹中就是明显的凸起和凹槽的纹路及枪管的直径等信息。枪膛的个体特征主要指一些诸如锈迹、油斑、杂质物等的标记等,这类特征可以用以区分同一种类枪支的不同(如图 4)。用以比对弹头时将主要依赖于弹头经过枪膛时,枪膛线在弹头上留下的痕迹。

\begin{figure}[h]
    \centering
    \includegraphics[width=0.8\textwidth]{image.png}
    \caption{弹头表面特征信息}
    \label{fig:bullet_surface}
\end{figure}

枪支的个体特征是进行弹头比对的重要信息。弹头痕迹是类别特征和个体特征组合而成,不能直接进行关联和比对,因此识别枪支和弹头的关键点就是将弹头痕迹中的类别特征和个体特征进行分离。本文对特征的分离主要使用小波分析模型,按照如下的数据处理步骤进行特征的提取:

1) 表面抽象。枪管膛线的主要特征均体现在弹头痕迹的数据信息中。然而该痕迹数据不能直接用于分析,一方面是由于测量数据时,是以痕迹走向作为基准轴,因而造成了与主轴的夹角,该角度不能忽略。另一方面获取的数据是三维空间中的数据,不能直接用于分析首先需要对该表面进行抽象,抽象方法是之前提到的展平面法。

2) 小波过滤。展平面中的信息既包含了类别特征又包括了个体特征。本部分主要使用小波分析法来分离两类特征。弹头痕迹的变换包含了很精细的时域特性和频域特性。传统的标准傅里叶变换不能刻画局部的变化。窗口傅立叶变换虽然在一定程度上克服了标准傅立叶变换不具有局部分析能力的缺陷,但其缺点是时频窗的形状是固定的,因而其实质也是具有单一分辨率的分析;而小波变换作为窗口大小固定,形状可变的时频变换方法。其时频分辨率是可变的:对于信号的高频部分用短时窗口分析,对于信号的低频部分则用长时窗口进行分析,因而增大了分析的灵活性与合理性。由此可见,小波函数的良好时频局部化特性使得它非常适用于进行信号的时域分析。

针对弹头痕迹的信息,本文主要使用小波分析进行滤波。小波函数在频域上的局部化特性表现了它的滤波器特性。利用小波变换可以很方便地实现对信号的滤波。信号经正交小波变换后得到各个尺度上的分解结果,分别对应于信号在各对应信号在各个频段内的信息,因而通过对这些结果进行控制,可以很方便地实现高通、低通、带通或这几种方式的组合滤波另外由于小波变换每一尺度的分解结果仍保留有信号的时间信息,因而还可以对信号的特定时间段进行滤波。

3) 弹头痕迹比对。弹头痕迹的比对主要是基于 2) 中获取的弹头痕迹中的高频信号即:枪膛的个体信息。因此弹头比对的原始问题即转换为对小波分析后的高频信号进行比对。这里主要采用 CCF(Cross-Correlation Function)方法寻找两个不同信号的相似性。CCF 计算所得的最大值可以定量表示两组信号的相似度。

\subsection{问题四分析}

问题四即为问题三所提方法的具体应用。

\section{模型建立与求解}

\subsection{问题一模型建立与求解}

\subsubsection{模型建立}

由于本文中的弹头平移和转动均是刚体变化, 因此测量数据的重定位是一种刚体移动, 对于此类数据的重定位通常是通过在被测物体上选取不共线且在两次定位状态下均可测量的 3 个点 A、B、C 称为重定位基准点。设在第一次定位状态下测得 A、B、C 的坐标值分别为 $(x_{1}, y_{1}, z_{1})$, $(x_{2}, y_{2}, z_{2})$ 和 $(x_{3}, y_{3}, z_{3})$, 在第二次定位状态下测得的坐标值分别为 $(X_{1}, Y_{1}, Z_{1})$, $(X_{2}, Y_{2}, Z_{2})$ 和 $(X_{3}, Y_{3}, Z_{3})$。设变换矩阵为 $T$, 于是可知,
\begin{equation}
\left\{
\begin{aligned}
[X_{1}, Y_{1}, Z_{1}, 1] &= [x_{1}, y_{1}, z_{1}, 1] \cdot T \\
[X_{2}, Y_{2}, Z_{2}, 1] &= [x_{2}, y_{2}, z_{2}, 1] \cdot T \\
[X_{3}, Y_{3}, Z_{3}, 1] &= [x_{3}, y_{3}, z_{3}, 1] \cdot T
\end{aligned}
\right.
\tag{5-1}
\end{equation}

通过求解以上方程组便可取得变换矩阵 $T$。但是这种方法需要不共线且在两次定位状态下均可测量的 3 个点, 而这在弹头数据的测量中各次测量间是没有可固定测量以供参考的三个点的, 考虑到本文中的弹头中有痕迹的部分为一个已知直径和高度的规则圆柱体, 本文采用了一种基于几何中心的三坐标测量数据重定位整合方法。

该方法试图将每次需固定测量的 3 个基准点坐标测量改变为选取多个不需固定测量的点的坐标测量, 这些点可在待测物体上固定 3 个体积较小的几何体并能计算得到相应的 3 个几何中心 $A$, $B$ 和 $C$。

设在第一次定位状态下测得 $A$、$B$、$C$ 的坐标值分别为 $(x_{1}, y_{1}, z_{1})$, $(x_{2}, y_{2}, z_{2})$ 和 $(x_{3}, y_{3}, z_{3})$, 在第二次定位状态下测得的坐标值分别为 $(X_{1}, Y_{1}, Z_{1})$, $(X_{2}, Y_{2}, Z_{2})$ 和 $(X_{3}, Y_{3}, Z_{3})$。由于子弹为刚体可知, 这些几何中心的相对位置是固定不变的, 所以根据公式 (5-1) 可以求得变换矩阵 $T$。

求得变换矩阵 $T$ 后, 可根据 $[X^{*}, Y^{*}, Z^{*}, 1] = [x, y, z, 1] \cdot T$ 将第一次定位下坐标测为 $(x, y, z)$ 的点移到第二次定位坐标下的 $(X^{*}, Y^{*}, Z^{*})$ 与第二次定位测得的数据进行同一坐标下的比较。

现考虑对于圆柱体由不需固定测量的已测点来计算该圆柱体的几何中心地方法。假设圆柱体高为 $h$, 半径为 $r$, 圆柱体的高 $h$ 和半径 $R$ 已知。在圆柱体顶面或者底面上测量不共线的 3 个点, 再在圆柱侧面上测量 3 个点, 共得到 6 个点的坐标。, 则计算其几何中心如下:

(1) 由底面或顶面上的 3 个点确定底面或顶面所在的平面, 得到该面方程及法向量;

(2) 过侧面上的 3 个点, 分别沿着面的法向量即圆柱轴线方向作直线, 作得 3 条直线;

(3) 三条直线与底面或顶面的交点即侧面上的 3 个点在底面或顶面上的投影点 $p_{0}$, $p_{1}$ 和 $p_{2}$, 如图 4 所示, 这 3 个投影点应在底面或顶面圆

的圆周上;

(4) 按下式计算圆心 $\mathbf{O}_{1}$ 到线段 $\mathbf{p}_{0} \mathbf{p}_{1}$ 的距离 $d$:
\begin{equation}
d=\sqrt{r^{2}-D\left(p_{0}, p_{1}\right)^{2}}
\tag{5-2}
\end{equation}
其中 $D\left(p_{0}, p_{1}\right)$ 为线段 $\mathbf{p}_{0} \mathbf{p}_{1}$ 长度的一半;

(5) 把线段 $\mathbf{p}_{0} \mathbf{p}_{1}$ 的中点 $\mathbf{p}_{3}$ 沿着线段中垂线的方向移动距离 $d$,得到一个圆心 $\mathbf{O}$;

(6) 由于与线段 $\mathbf{p}_{0} \mathbf{p}_{1}$ 的矢量方向有两个(如图 4 中的的 $\mathbf{O}_{1}$ 和 $\mathbf{O}_{2}$),因此(2)中求得的圆心 $\mathbf{O}$ 并不一定是真正的圆心,还需要将其与 $\mathbf{p}_{2}$ 的距离与半径进行对比,如果大于半径,则不是真正的圆心。

(7) 把圆心沿步骤 1 求出的法向量方向移动 $h/2$ 可得到圆柱体几何中心。

\begin{figure}[h]
    \centering
    \includegraphics[width=0.8\textwidth]{image.png} % 替换为实际图片路径
    \caption{圆柱几何中心的计算}
    \label{fig:4}
\end{figure}

\subsection{5.1.2 模型求解}

对已测得数据进行分析可知,光学测量仪器每次测量时 $y$ 轴方向的步进次数 756 个数据点(共 2.079mm),$y$ 方向每次步进一次,在 $x$ 轴方向采集 564 次(共 1.551mm)。且所得数据采集点均分布于圆柱的侧面,不能覆盖全部圆柱体,因此不能直接应用上述模型。同时由于已有数据测量时的坐标系的 $y$ 轴与弹头圆柱母线是不平行的(夹角为 $\theta$),即每次测量弹头时的扫描切面是不平行的(夹角也为 $\theta$),因此很难根据现有数据得到圆柱底(顶)面及其平行面。

注意到,在 $x \in [450, 564]$,$y \in [200, 756]$ 这个区间,弹痕已经结束,弹头测量深度呈自然下降。因此在该区间内的等高线可视为与圆柱母线平行,又知弹头圆柱中心线平行于基准平面,$y$ 轴平行于擦痕,所以该等高线与 $y$ 轴的夹角即为擦痕与圆柱母线的夹角 $\theta$(如图 5 所示)。

\begin{figure}[h]
    \centering
    \includegraphics[width=\textwidth]{image.png}
    \caption{弹头某条次棱痕迹的等高线图(单位 mm)}
    \label{fig:5}
\end{figure}

在已知数据中,该夹角通常为 $6^\circ$ 左右,同时可知在每个定位下所测的截面弧线段只是整个圆弧的 $6\%$ 左右,因此可以将每次所测的弧段看作是一个半径为子弹柱体半径的圆的圆曲线的一段,从而求得该圆心并将其作为每段截面弧线的几何中心。

在测量数据中分别选取 $y=200, 400, 600$ 三个截面上的 3 个规整点(一般要求 $x \in [450, 564]$,$y \in [200, 756]$)可唯一确定三个截面的圆心 $O_1$,$O_2$ 和 $O_3$,将 $O_1$,$O_2$ 和 $O_3$ 坐标带入方程(5-1)便可求得变换矩阵 $T$。

由三维空间的坐标变换原理可知:

(1) 若空间平移量为 $(t_x, t_y, t_z)$,则相应的平移变换矩阵为
\begin{equation}
\begin{bmatrix}
1 & 0 & 0 & 0 \\
0 & 1 & 0 & 0 \\
0 & 0 & 1 & 0 \\
t_x & t_y & t_z & 1
\end{bmatrix}
\tag{5-3}
\end{equation}

(2) 若物体绕 $x$ 轴旋转 $\alpha$,则相应的平移变换矩阵为
\begin{equation}
\begin{bmatrix}
1 & 0 & 0 & 0 \\
0 & \cos \alpha & \sin \alpha & 0 \\
0 & -\sin \alpha & \cos \alpha & 0 \\
0 & 0 & 0 & 1
\end{bmatrix}
\tag{5-4}
\end{equation}

(3) 若物体绕 $y$ 轴旋转 $\beta$,则相应的平移变换矩阵为
\begin{equation}
\begin{bmatrix}
\cos \beta & 0 & -\sin \beta & 0 \\
0 & 1 & 0 & 0 \\
\sin \beta & 0 & \cos \beta & 0 \\
0 & 0 & 0 & 1
\end{bmatrix}
\tag{5-5}
\end{equation}

(4) 若物体绕 \( z \) 轴旋转 \( \gamma \),则相应的平移变换矩阵为

\[
\begin{bmatrix}
\cos \gamma & \sin \gamma & 0 & 0 \\
-\sin \gamma & \cos \gamma & 0 & 0 \\
0 & 0 & 1 & 0 \\
0 & 0 & 0 & 1
\end{bmatrix}
\tag{5-6}
\]

考虑到本文中的每次测量时手动调节子弹姿态和位置造成的平移误差在 \( 0.03 \) mm 以内,转动误差在 \( 0.2^\circ \) 以内,即对变换矩阵 \( T \) 是存在约束的。因此总的变换矩阵要求 \( t_x^2 + t_y^2 + t_z^2 \leq 0.03 \),\( \alpha \),\( \beta \),\( \gamma \) 均不大于 \( 0.2^\circ \)。

在进行重定位整合后,同样是考虑到弹头中有痕迹的部分为一个已知直径和高度的规则圆柱体,本文希望建立一个利用圆柱体横截圆来将子弹的截面曲线进行展平的模型(如图 6 所示)。

\begin{figure}[h]
\centering
\includegraphics[width=0.8\textwidth]{image1.png}
\caption{弹头截面曲线的展平}
\end{figure}

根据重定位整合后的数据,利用截面曲线上的规整点进行圆拟合后,求得曲线与拟合所得圆的弧线的 \( Z \) 值相减,便获得了该曲线的展平线(如图 7 和图 8 所示)。

\begin{figure}[h]
\centering
\includegraphics[width=0.8\textwidth]{image2.png}
\caption{弹头截面展平前曲线}
\end{figure}

\begin{figure}[h]
    \centering
    \includegraphics[width=\textwidth]{image.png}
    \caption{弹头截面展平后曲线}
    \label{fig:8}
\end{figure}

\subsection{问题二模型建立与求解}

\subsubsection{模型建立}

1) 脉冲噪声的去除

激光扫描测量的数据是由激光传感器获得,其噪声模型为 \( e = e_1 + e_2 \),其中:\( e_1 \) 是服从正态分布的高斯噪声;\( e_2 \) 为脉冲噪声。

从实际的扫描数据中可以发现,脉冲噪声点一般都是孤立的数据点或数据点列,与相邻的数据点的距离远大于正常的数据点的距离,其周围缺乏必要的与其它数据点的联系。本文采用如下方法消除该类噪声:

每测量得 \( n \) 个点:\( \cdots X_1, X_2, \ldots X_n \),相应的得到每个点的 \( Z \) 坐标,对 \( n \) 个点的 \( Z \) 坐标排序得到序列 \( Z \):\( Z_1 < Z_2 < \ldots < Z_n \),去掉值最大的 \( p \) 个数据和值最小的 \( q \) 个数据,取均值得:

\begin{equation}
\overline{Z} = \frac{\sum_{i=1}^{n} Z_i - \sum_{m=1}^{p} Z_m - \sum_{j=n-p}^{n} Z_j}{n-p-q}
\tag{5-7}
\end{equation}

取得 \( \overline{Z} \) 后求 \( n \) 个点与 \( \overline{Z} \) 的绝对值之差后再除以 \( \overline{Z} \),若大于设定阈值 \( \sigma \),则将改点的值设为 \( \overline{Z} \)。

2) 高斯噪声的去除

常用的高斯噪声滤波方法有高斯滤波、平均滤波、中值滤波算法等。

(1)高斯滤波:它是一类根据高斯函数的形状来选择权值的线性平滑滤波方法。高斯滤波器在指定域内的权重为高斯分布,其平均效果较差,故在滤波的同时能较好地保持原数据的形貌。

(2)平均滤波:其采样点的滤波值为滤波窗口内各数据点的统计平均值。通过一点和周围点的平均来去除突然变化的点,从而滤掉一定的噪声。这种方法简单,计算速度也较快,但是这种滤波方法的代价是在降低噪声的同时使信息有一定的模糊,特别是在边沿和细节处,领域越大,模糊越大。

(3)中值滤波:中值滤波是一种非线性处理方法,是把以某点 \( (x, y) \) 为中心的小窗口内的所有信息值按从大到小的顺序排列,将中间值作为 \( (x, y) \) 处的信息值。常用的窗口有线性、方形、十字形、圆形和环形等。中值滤波用公式表示为:

\[
g(m, n) = Median\{f(m-k, n-l)\}, (k, l) \in w
\]

其中,\( f(x, y) \),\( g(x, y) \) 分别为原始图像和处理后图像,\( W \) 为窗口大小。

应用该理论,找出待处理的坏点之后,利用程序找出该点所属的扫描线,然后对该扫描线上的点进行排序,如果该扫描线上的点数为奇数,则取中间点的坐标替代该坏点;如果为偶数,则取中间两点的平均值来代替。这种方法在一定条件下,可以克服线性滤波器带来的细节模糊,消除数据毛刺的效果较好。

通过以上介绍,三种方法的比较如下:

\begin{table}[h]
\centering
\caption{三种滤波方法}
\begin{tabular}{|c|c|c|c|}
\hline
 & 平滑效果 & 复杂程度 & 原数据保持程度 \\ \hline
高斯滤波 & 较好 & 复杂 & 较好 \\ \hline
均值滤波 & 较差 & 简单 & 较差 \\ \hline
中值滤波 & 较好 & 见到 & 较好 \\ \hline
\end{tabular}
\end{table}

上述的几种噪声去除方法中,其中中值滤波法算法简单,对高斯噪声有较好平滑能力,因此本文将采用中值滤波法。

\subsection{5.2.2 模型求解}

本文采用 \( n=20 \), \( p=2 \), \( q=2 \), \( \delta=0.1 \),对数据进行脉冲噪声滤波;将测量数据点云的 \( Z \) 坐标值作为中值滤波中的灰度值,对测量数据进行采用线形窗口的中值滤波消除高斯噪声,滤波处理前后结果对比如下所示。

\begin{figure}[h]
\centering
\includegraphics[width=\textwidth]{image.png}
\caption{滤波前后图像对比}
\end{figure}

\subsection{5.3 问题三模型建立与求解}

\subsubsection{5.3.1 模型建立}

弹头本身的特性及枪膛线的影响等诸多因素使得弹头表面痕迹中留下了多种信息。基于产生原因的不同,本文将其大体归为三类:个体特征(微观表象);类别特征(宏观表象);中间表象。弹头中的这三类信息时同时存在的,通常痕迹特征便是融合了这三类特征的综合特征。本文以现代数学小波分析为工具,提出综合特征分离的小波模型。模型能客观地分离各种特征。

令 \( f(t) \) 为表面综合特征,滤波的数学过程可描述如下:设 \( h(t) \) 为一滤波器,\( h(t) \) 的作用是滤掉 \( f(t) \) 的某些频率段,从而得出所需频率的信号。其工作原理为:输入 \( f(t) \to h(t) \) 滤波 \(\to g(t) \) 输出,其数学运算为 \( g(t) = f(t) * h(t) \)。

设计三种滤波器:
a. 低通滤波器 \( h_1(t) \),\( h_1(t) \) 只允许低频信息通过。弹头中的类别信息则属低频信息,在 \( h_1(t) \) 的作用下,\( g_1(t) = f(t) * h_1(t) \),则 \( g_1(t) \) 反映了弹头类别特征。
b. 带通滤波器 \( h_2(t) \),\( h_2(t) \) 限制低频信号同时也限制高频信号通过。\( g_2(t) = f(t) * h_2(t) \),\( g_2(t) \) 是滤掉了低频信息和高频信息的带通信息。
c. 高通滤波器 \( h_3(t) \),\( h_3(t) \) 只允许高频信号通过。\( g_3(t) = f(t) * h_3(t) \),\( g_3(t) \) 描述了弹头的个体特征信息。\( f(t) \) 表示痕迹综合特征的函数,则特征分离的理想数学模型为:

\begin{align*}
f(t) &= g_1(t) + g_2(t) + g_3(t) = f(t) * h_1(t) + f(t) * h_2(t) + f(t) * h_3(t),
\end{align*}
式子中 $h_i(t)$ 须满足假设,
\[
H_i(w) =
\begin{cases}
1, & w_{i-1} < |w| < w_i, \\
0, & \text{其他},
\end{cases}
\]
其中, $i = 1, 2, 3; w_0 < w_1 < w_2 < w_3; H_i(w)$ 为 $h_i(t)$ 的傅里叶变换。

小波变换的定义及视频特性:

定义 1 , 设 $\psi \in L^2(R) \cap L^1(R)$ 且 $\widehat{\psi}(0) = 0$ , 则按方式
\[
\psi_{a,b}(t) = |a|^{-1/2} \psi\left(\frac{t-b}{a}\right), b \in R, a \in R
\]
生成的函数族 $\{\psi_{a,b}\}$ 叫分析小波, $\psi$ 叫基本小波或母小波。

定义 2 设 $\psi$ 是基本小波, $\{\psi_{a,b}\}$ 按定义 1 给出的分析小波, 对 $f \in L^2(R), f(t)$ 的连续小波变换 $w_f(b, a)$ 定义为
\[
w_f(b, a) = \langle f, \psi_{a,b} \rangle = |a|^{-1/2} \int_R f(t) \psi\left(\frac{t-b}{a}\right) dt.
\]

小波函数 $\psi$ 是一个时频局部变化函数, 在时域中, $\psi(t)$ 只是在一个小的时间段内有突起的 “波”。在频域中, $|\widehat{\psi}(w)|$ 是一个带状的频窗。当尺度参数 $a$ 发生变化时, 时频窗发生变。小波分析具有 “变焦” 特性: 在时间-频率相平面的低频段, 小波变换的时窗宽频谱窗窄, 这意味着有较高的频率分辨率和低的时域分辨率, 有利于控制和分析宏观几何形状等低频信号。在相平面的高频段, 时窗窄频谱窗宽, 这时时域分辨率高而频域分辨率低, 有利于处理表面粗糙度等高频误差。

令 $\psi_{k,n}(t) = \psi\left(\frac{t-b}{a}\right)$ (其中 $b = n, a = 2^k$), 且要求 $\psi_{k,n}(t)$ 为 $L^2(R)$ 的一组正交基, 则对任意的 $f \in L^2(R)$, 有如下展开式:
\[
f(t) = \sum d_{k,n} \psi_{k,n}(t),
\]
式中 $d_{k,n} = w_f(n, 2^k)$。在实际问题中, 按定义求 $d_{k,n}$ 是不方便的, 有效的算法是运用多分辨分析给出如下离散算法。设基本小波函数 $\psi(t)$ 由两组系数 $\{h_n\}$ 和 $\{g_n\}$ 定义, $\Phi(t)$ 为一多分辨分析的生成元,
\[
\begin{cases}
\Phi(t) = \sum h_n \Phi(2t-n); \\
\psi(t) = \sum g_n \Phi(2t-n);
\end{cases}
\]
另设 $f_0 = f, f = G_{-1} + G_{-2} + \ldots + G_{-N} + f_{-N}$, 且存在 $\{C_n^0\}$ 满足
\[
f(t) = \sum C_n^0 \Phi(t-n),
\]
则
\[
\begin{cases}
f_{-k} = \sum C_n^k \Phi_{-k,n}; \\
G_{-k} = \sum d_n^k \psi_{-k,n}; \quad (k = 1, 2, \ldots, N),
\end{cases}
\]
式中
\[
\begin{cases}
C_n^k = (1/\sqrt{2}) \sum_{j \in \mathbb{Z}} C_j^{k-1} \overline{h}_{j-2n}; \\
d_n^k = (1/\sqrt{2}) \sum_{j \in \mathbb{Z}} C_j^{k-1} \overline{g}_{j-2n} \quad (k = 1, 2, \ldots, N)
\end{cases}
\]
按小波正交基展开的分解算法有
\[
f(t) = \sum_{k=1}^N G_{-k} + f_{-N}.
\]

式子中 $G_{-k}$ 位带通信号,k 的大小反映了频率的高低,k 越小频率越高。设 $k \leq N_{1}$ 时,$G_{-k}$ 为高信号;$N_{1}+1 \leq k \leq N_{2}$ 时,$G_{-k}$ 为带通信号;$N_{2}+1 \leq k \leq N$ 时,$G_{-k}$ 为低通信号。

令
\[
\left\{
\begin{aligned}
g_{1} &= G_{-N_{2}-1} + G_{-N_{2}-2} + \ldots + G_{-N} + f_{-N}; \\
g_{2} &= G_{-N_{1}-1} + G_{-N_{1}-2} + \ldots + G_{-N_{2}}; \\
g_{3} &= G_{-1} + G_{-2} + \ldots + G_{-N_{1}};
\end{aligned}
\right.
\]

于是得出 $f(t) = g_{1} + g_{2} + g_{3}$。因此该模型能由小波分解实现。

\subsection{5.3.2 模型求解}

本文采用的小波分析模型主要按以下步骤进行模型求解。

1) 表面抽象。枪管膛线的主要特征均体现在弹头痕迹的数据信息中。然而该痕迹数据不能直接用于分析,一方面是由于测量数据时,是以痕迹走向作为基准轴,因而造成了与主轴的夹角,该角度不能忽略。另一方面获取的数据是三维空间中的数据,不能直接用于分析。如下图即为使用原始数据绘制的侧棱曲面:

\begin{figure}[h]
\centering
\includegraphics[width=\textwidth]{image1.png}
\caption{次棱曲面图}
\end{figure}

首先需要对该表面使用展平面法进行抽象,抽象后的结果如下图:

\begin{figure}[h]
\centering
\includegraphics[width=\textwidth]{image2.png}
\caption{次棱展平面}
\end{figure}

选取 $X=200$ 可以得到子弹头次棱痕迹的轮廓线。如下图为使用展平面进行表面抽象化后的结果。

\begin{figure}[h]
    \centering
    \includegraphics[width=\textwidth]{image1.png}
    \caption{次棱展平线}
    \label{fig:12}
\end{figure}

2) 小波过滤。展平面中的信息既包含了类别特征又包括了个体特征。本部分主要使用小波分析法来分离两类特征。

如下图为对原始展平线信号进行小波滤波的结果。

\begin{figure}[h]
    \centering
    \includegraphics[width=\textwidth]{image2.png}
    \caption{展平线滤波后的结果}
    \label{fig:13}
\end{figure}

3) 弹头痕迹比对。弹头痕迹的比对主要是基于 2) 中获取的弹头痕迹中的高频信号即:枪膛的个体信息。因此弹头比对的原始问题即转换为对小波分析后的高频信号进行比对。这里主要采用 CCF(Cross-Correlation Function)方法寻找两个不同信号的相似性。CCF 计算所得的最大值可以定量表示两组信号的相似度。

根据上述三个小波模型求解步骤,做出如下求解:首先采用了多项式拟合方法,消除了边界失真,之后在不同尺度上对弹头发射痕迹进行分析,主要采用了多尺度相关分析法,使用加权法综合利用各个频段信息,以期达到更高的识别率。

在弹头痕迹纹理特征的提取中,采用了一种基于多项式拟合预处理的方法来对数据进行拓展。具体步骤如下:a. 原始信号 $f(x_n)$ 用一多项式 $p(x_n)$ 拟合,拟合的阶次由最小二乘均方误差准则确定;b. 将信号 $f(x_n)$ 按多项式 $p(x_n)$ 在两端延拓为 $fp(x_n)$;c. 将 $fp(x_n)$ 用小波分析进行处理;d. 将处理得到的信号取中间 $M$ 个数据作为最后的纹理提取结果。

图 14 是采用本方法对数据进行预处理后,再进行小波分析对各个频率的数据进行提取的效果图,可见该图可以有效提取有用信息。

\begin{figure}[h]
    \centering
    \includegraphics[width=\textwidth]{image3.png}
    \caption{采用本方法对数据进行预处理后,再进行小波分析对各个频率的数据进行提取的效果图}
    \label{fig:14}
\end{figure}

\begin{figure}[h]
    \centering
    \includegraphics[width=\textwidth]{image.png}
    \caption{使用小波工具箱进行小波分解效果图}
    \label{fig:wavelet_decomposition}
\end{figure}

从该图中可知:$s = a5 + d5 + d4 + d3 + d2 + d1$ 是信号分解后的结果。其中 $a5$ 为低频信号;$d1 \sim d5$ 为高频信号。

在本题的具体背景下,即根据四条已知次棱的信息进行比对时,可以得知,实际有用的信息集中在低频段,而高频段和中频段的信息价值量相对较少。基于此,采用多尺度加权相关分析方法,将弹头痕迹分解并重构成从高到低多个频率段上的信息,然后将不同的弹头次棱痕迹信息进行比对。比对时主要采用了互相关分析法。即计算同一频段不同弹头痕迹的互相关性,并将结果进行归一化,得到量化的相关度指数。并对该指数根据不同频段进行加权求和即可得到两条痕迹的总相关系数。

假设痕迹信息被划分为 $N$ 个频段,设 $\rho_1, \rho_2, \ldots, \rho_n$ 为各个频段的相关系数,标准化后的结果为 $\overline{\rho_1}, \overline{\rho_2}, \ldots, \overline{\rho_n}$,设 $n_1, n_2, \ldots, n_N$ 为各个频段的权重,则总的相关系数可表示如下。

\begin{equation}
\rho = \frac{\sum_{i=1}^{N} \rho_i n_i}{\sum_{i=1}^{N} n_i}
\end{equation}

\begin{table}
\caption{22颗子弹的相互匹配距离}
\begin{tabular}{c c c c c c c c c c c c}
距离 & 1 & 2 & 3 & 4 & 5 & 6 & 7 & 8 & 9 & 10 & 11 \\
\hline
1 & 0 & 363.92 & 350.32 & 449.8 & 489.99 & 330.7 & 387.84 & 442.55 & 305.34 & 292.13 & 244.01 \\
2 & 366.54 & 0 & 316.32 & 390.31 & 445.11 & 444.83 & 363.98 & 274.92 & 354.92 & 412.41 & 348.29 \\
3 & 347.89 & 311.15 & 0 & 277.34 & 469.33 & 448.22 & 375.08 & 360.48 & 384.15 & 421.73 & 388.12 \\
4 & 445.91 & 389.57 & 276.88 & 0 & 456.89 & 515.93 & 258.97 & 358.06 & 429.71 & 313.55 & 517.44 \\
5 & 498.62 & 447.18 & 465.93 & 461.04 & 0 & 435.39 & 307.74 & 301.88 & 390.73 & 550.86 & 521.45 \\
6 & 327.49 & 438.29 & 446.36 & 511.17 & 437.68 & 0 & 343.6 & 377.31 & 344.17 & 568.74 & 305.2 \\
7 & 383.24 & 362.85 & 371.39 & 258.41 & 309.11 & 345.68 & 0 & 250.75 & 275.78 & 300.92 & 418.89 \\
8 & 457.18 & 271.49 & 358.9 & 363.45 & 298.46 & 377.13 & 244.6 & 0 & 275.33 & 511.34 & 423.87 \\
9 & 306.02 & 357.56 & 381.83 & 428.89 & 397.24 & 345.87 & 276.67 & 276.44 & 0 & 386.44 & 355.59 \\
10 & 288.55 & 410.7 & 415.6 & 313.99 & 545.42 & 565.2 & 299.33 & 506.82 & 381.22 & 0 & 535.42 \\
11 & 249.47 & 342.42 & 381.53 & 523.2 & 513.24 & 302.9 & 417.1 & 416.54 & 350.87 & 544.84 & 0 \\
12 & 312.48 & 303.99 & 320.65 & 429.2 & 540.14 & 350.14 & 367.99 & 396.25 & 282.97 & 503.61 & 330.02 \\
13 & 283.33 & 383.86 & 242.45 & 266.43 & 578.64 & 450.88 & 367.55 & 428.8 & 337.52 & 285.71 & 384.7 \\
14 & 240.62 & 334.03 & 367.56 & 322.57 & 362.82 & 473.51 & 341.25 & 410.48 & 329.22 & 235.23 & 382.48 \\
15 & 245.79 & 358.62 & 312.68 & 363.29 & 309.06 & 405.77 & 350.3 & 481.92 & 322.37 & 303.5 & 352.16 \\
16 & 188.82 & 326.2 & 253.92 & 324.26 & 387.39 & 404.85 & 298.89 & 343.61 & 340.2 & 289.73 & 300.7 \\
17 & 209.13 & 360.28 & 291.39 & 404.39 & 488.69 & 348.99 & 395.21 & 429.05 & 274.74 & 403.35 & 350.56 \\
18 & 481.65 & 420.04 & 318.21 & 600.54 & 624.19 & 494.95 & 559.62 & 593.03 & 450.63 & 693.7 & 373.57 \\
19 & 289.76 & 379.69 & 354.03 & 491.55 & 518.93 & 348.98 & 444.09 & 467.55 & 407.28 & 563.13 & 307.67 \\
20 & 234.34 & 335.7 & 302.18 & 310.16 & 361.68 & 289.45 & 269.51 & 288.38 & 272.94 & 335.4 & 221.62 \\
21 & 507.52 & 429.54 & 450.31 & 384.24 & 379.38 & 405.86 & 382.17 & 300.64 & 372.84 & 566.12 & 497.91 \\
22 & 338.66 & 352.77 & 226.75 & 269.21 & 511.28 & 444.58 & 356.58 & 478.86 & 403.11 & 369.37 & 368.08 \\
\hline
12 & 13 & 14 & 15 & 16 & 17 & 18 & 19 & 20 & 21 & 22 & \\
310.8 & 285.88 & 242.98 & 245.14 & 191.23 & 209.6 & 479.81 & 287.37 & 238.94 & 499.93 & 340.42 & \\
303.48 & 385.7 & 341.04 & 363.08 & 330.49 & 356.04 & 419.13 & 383.99 & 339.94 & 426.8 & 357.14 & \\
315.36 & 240.22 & 367.92 & 317.93 & 255.33 & 295.34 & 316.62 & 354.3 & 302.02 & 455.92 & 228.32 & \\
431.66 & 263.64 & 320.09 & 359.99 & 320.02 & 412.29 & 602.74 & 490.37 & 309.73 & 384.37 & 270.29 & \\
537.93 & 571.86 & 358.19 & 307.26 & 389.66 & 484.84 & 629.6 & 532.6 & 360.68 & 382.58 & 512.85 & \\
347.84 & 446.17 & 463.49 & 405.06 & 399.43 & 343.07 & 491.48 & 347.85 & 285.43 & 403.02 & 440.03 & \\
367.82 & 368.44 & 343.22 & 350.09 & 299.26 & 393.14 & 558.44 & 446.21 & 269.03 & 390.12 & 350.74 & \\
397.23 & 431.25 & 411.75 & 487.49 & 345.27 & 436.19 & 595.01 & 471.05 & 296.89 & 302.34 & 478.49 & \\
281.08 & 337.77 & 328.63 & 319.93 & 336.94 & 271.88 & 452.82 & 406.27 & 274.3 & 374.48 & 402 & \\
501.5 & 281.99 & 233.71 & 301.65 & 291.38 & 400.25 & 688.85 & 564.49 & 336.02 & 577.41 & 365.57 & \\
336.44 & 388.79 & 384.64 & 355.73 & 302.98 & 346.07 & 371.17 & 300.87 & 224.89 & 489.73 & 372.1 & \\
0 & 338.71 & 376.05 & 328.58 & 369.43 & 261.76 & 334.36 & 263.62 & 279.94 & 421.63 & 384.39 & \\
338.93 & 0 & 194.69 & 270.19 & 304.58 & 273.49 & 434.15 & 377.07 & 310.97 & 493.29 & 263.89 & \\
373.68 & 196.01 & 0 & 223.38 & 269.21 & 327.44 & 442.37 & 385.64 & 281.41 & 521.34 & 277.1 & \\
\end{tabular}
\end{table}

\begin{table}[h]
\centering
\begin{tabular}{c c}
子弹编号 & 匹配结果 \\
\hline
1 & 1617201415 \\
2 & 81231614 \\
3 & 221316417 \\
4 & 71322320 \\
5 & 81572014 \\
6 & 2011179 \\
7 & 8420916 \\
8 & 7292021 \\
9 & 20178712 \\
10 & 14131617 \\
11 & 20116619 \\
12 & 17192092 \\
13 & 14342215 \\
14 & 131510116 \\
15 & 161417120 \\
16 & 12015314 \\
17 & 11512209 \\
18 & 193121117 \\
19 & 121111822 \\
20 & 221611115 \\
21 & 82016229 \\
22 & 20313414 \\
\end{tabular}
\caption{每颗相似度最大的前5颗子弹列表}
\end{table}

根据上表可知,每颗子弹的相似度最大的前5颗子弹如表三所示。

\begin{table}
\centering
\begin{tabular}{c c c c c c c c c c c c}
\hline
1 & 0.1447 & 0.76399 & 0.71866 & 0.7723 & 0.89815 & 0.9401 & 0.73985 & 0.74418 & 0.75524 & 0.92942 & 0.88892 \\
\hline
0.1447 & 1 & 0.10788 & 0.12759 & 0.043714 & 0.066862 & 0.1008 & 0.28842 & 0.073287 & 0.11814 & 0.09204 & 0.1285 \\
\hline
0.76399 & 0.10788 & 1 & 0.82099 & 0.84438 & 0.81975 & 0.79312 & 0.84448 & 0.93711 & 0.82493 & 0.76064 & 0.86896 \\
\hline
0.71866 & 0.12759 & 0.82099 & 1 & 0.83705 & 0.67946 & 0.69047 & 0.74244 & 0.86059 & 0.82821 & 0.73727 & 0.63895 \\
\hline
0.7723 & 0.043714 & 0.84438 & 0.83705 & 1 & 0.85337 & 0.81181 & 0.71591 & 0.91692 & 0.74546 & 0.84156 & 0.72324 \\
\hline
0.89815 & 0.066862 & 0.81975 & 0.67946 & 0.85337 & 1 & 0.92605 & 0.77735 & 0.87778 & 0.73886 & 0.85361 & 0.87904 \\
\hline
0.9401 & 0.1008 & 0.79312 & 0.69047 & 0.81181 & 0.92605 & 1 & 0.77867 & 0.82008 & 0.80309 & 0.90193 & 0.89463 \\
\hline
0.73985 & 0.28842 & 0.84448 & 0.74244 & 0.71591 & 0.77735 & 0.77867 & 1 & 0.84901 & 0.75777 & 0.67189 & 0.67886 \\
\hline
0.74418 & 0.073287 & 0.93711 & 0.86059 & 0.91692 & 0.87778 & 0.82008 & 0.84901 & 1 & 0.84814 & 0.74056 & 0.79527 \\
\hline
0.75524 & 0.11814 & 0.82493 & 0.82821 & 0.74546 & 0.73886 & 0.80309 & 0.75777 & 0.84814 & 1 & 0.77062 & 0.81506 \\
\hline
0.92942 & 0.09204 & 0.76064 & 0.73727 & 0.84156 & 0.85361 & 0.90193 & 0.67189 & 0.74056 & 0.77062 & 1 & 0.88353 \\
\hline
0.88892 & 0.1285 & 0.86896 & 0.63895 & 0.72324 & 0.87904 & 0.89463 & 0.67886 & 0.79527 & 0.81506 & 0.88353 & 1 \\
\hline
\end{tabular}
\caption{六组弹头相互比对信息表}
\end{table}

由已知的信息可知子弹的对应关系如下表所示:

\begin{table}
\centering
\begin{tabular}{c c}
\hline
同一枪膛的弹头 & \\
\hline
编号1 & 编号7 \\
\hline
编号2 & 编号8 \\
\hline
编号3 & 编号9 \\
\hline
编号4 & 编号10 \\
\hline
编号5 & 编号11 \\
\hline
编号6 & 编号12 \\
\hline
\end{tabular}
\caption{已知弹头配对关系}
\end{table}

可以利用上表已知的弹头比对信息来判断本模型所设计的比对算法的正确性。根据表X中的信息,可以得出每个弹头匹配的前3位,及其相似度的量化信息。

\begin{table}
\centering
\begin{tabular}{c c c c c c c c c c c c}
\hline
编号 & 1 & 2 & 3 & 4 & 5 & 6 & 7 & 8 & 9 & 10 & 11 & 12 \\
\hline
匹配前三位 & 7 & 8 & 9 & 9 & 9 & 7 & 1 & 9 & 3 & 4 & 1 & 7 \\
\hline
 & 11 & 1 & 8 & 5 & 3 & 1 & 6 & 3 & 5 & 3 & 7 & 1 \\
\hline
 & 6 & 12 & 10 & 10 & 11 & 12 & 11 & 7 & 6 & 12 & 12 & 11 \\
\hline
\end{tabular}
\caption{本文所提模型配对结果}
\end{table}

上表中标注为加粗的即表示来源同一枪膛的弹头。在此,做如下假设:即若匹配结果的前三位中出现了正确结果则认为匹配成功,可见本模型的比对率非常高。应用于第三题的十二发子弹进行验证,其配对成功率高达75\%。

\subsection{模型改进}

\subsubsection{选取特征的改进}

由于子弹在出弹膛时的物理作用,所以其在y轴上必保持一些几何特性,其中弹痕曲面在x轴方向上的曲线长度是个重要特征,也即弹痕在垂直于y轴的信息。

息将会是在一定程度上稳定且连续的。因此我们在选取比对特征时可以考虑“绝对差分和”特征。

由于在离散点上求导数的困难,下面采用差分求得近似的导数,并且利用绝对差分和代表但很曲面在 \( x \) 轴上的曲线长度。

绝对差分和定义为:则 \( X_{1n} \) 的绝对差分和定义为:
\[
\sum_{i=2}^{n} abs(x_i - x_{i-1})
\]
其中,量 \( X_{1n} = \{x_1, x_2, \ldots, x_n\} \),为一维向量。

下图为一支枪发射两颗子弹 8 个次棱绝对差分和的图像。

\begin{figure}[h]
    \centering
    \includegraphics[width=\textwidth]{image.png}
    \caption{两颗子弹在 \( x \) 轴上次棱绝对差分和在 \( y \) 轴上的分布}
\end{figure}

因此,上面的图像可以看到绝对差分和的确在 \( y \) 轴上保持连续性,而且,两颗子弹的绝对差分和的走向基本相似,分布比较集中。鉴于绝对差分和保持 \( y \) 轴上连续稳定性的原则,可以为采用那段数据用于对比具有实际意义。从上图可以看到当横坐标在 300-600 时,相应的图像分布比较集中,在具体对比时可以参考。

\subsection{对比算法的改进}

基于深度的最小绝对差算法是计算一组待比对弹头展平面的每点深度之差绝对值总和的分布最小值。弹头展平面之间的 MAD 值恰恰反映了弹头痕迹的相似度。因此可以采用基于深度的 MAD (Minimum Absolute Difference) 算法进行弹头模板展平面与弹头完整展平面的自动匹配。该算法描述如下:

弹头次棱展平面的深度数据在二维平面中组成一个 \( 756 \times 564 \) 的矩阵 \( A_{756 \times 564} \),则对该图像数据进行展开,得到展平面的深度矩阵如下:

\begin{figure}[h]
    \centering
    \includegraphics[width=\textwidth]{image2.png}
\end{figure}

经过分析得知,部分数据并不能有效反应枪膛在子弹上留下痕迹的稳定特征,因此这部分数据不应考虑在内。固 A 子弹 C1 次棱能有效反映枪膛痕迹的数据简化为 \(400 \times 500\) 的深度矩阵即

\[
A_{400 \times 500} =
\begin{bmatrix}
a_{1,1} & a_{1,2} & \cdots & a_{1,500} \\
a_{2,1} & a_{2,2} & \cdots & a_{2,500} \\
\cdots & \cdots & \cdots & \cdots \\
a_{400,1} & a_{400,2} & \cdots & a_{400,500}
\end{bmatrix}
\]

同理,A 子弹四个次棱的有效数据均可简化为如上形式的矩阵,将其组合即得到 A 子弹四条次棱的展平面数据。如下

\[
A_{c400 \times 2000} =
\begin{bmatrix}
a_{1,1} & a_{1,2} & \cdots & a_{1,2000} \\
a_{2,1} & a_{2,2} & \cdots & a_{2,2000} \\
\cdots & \cdots & \cdots & \cdots \\
a_{400,1} & a_{400,2} & \cdots & a_{400,2000}
\end{bmatrix}
\]

为了方便进行比对,此处将子弹 B 四个次棱的展平面数据也做如上处理,即其深度数据在二维平面中组成一个 \(400 \times 2000\) 的矩阵 \(B_{c400 \times 2000}\),如下所示:

\[
B_{c400 \times 2000} =
\begin{bmatrix}
b_{1,1} & b_{1,2} & \cdots & b_{1,2000} \\
b_{2,1} & b_{2,2} & \cdots & b_{2,2000} \\
\cdots & \cdots & \cdots & \cdots \\
b_{400,1} & b_{400,2} & \cdots & b_{400,2000}
\end{bmatrix}
\]

下面计算 \(A_{c400 \times 2000}\) 和 \(B_{c400 \times 2000}\) 得 MAD 值。首先将 \(A_{c400 \times 2000}\) 划分为如下四个矩阵,分别代表了四条顺序未知的次棱。

\[
A_{c1} =
\begin{bmatrix}
a_{1,1} & a_{1,2} & \cdots & a_{1,500} \\
a_{2,1} & a_{2,2} & \cdots & a_{2,500} \\
\cdots & \cdots & \cdots & \cdots \\
a_{400,1} & a_{400,2} & \cdots & a_{400,500}
\end{bmatrix},
\]

\[
A_{c2} =
\begin{bmatrix}
a_{1,501} & a_{1,502} & \cdots & a_{1,1000} \\
a_{2,501} & a_{2,502} & \cdots & a_{2,1000} \\
\cdots & \cdots & \cdots & \cdots \\
a_{400,501} & a_{400,502} & \cdots & a_{400,1000}
\end{bmatrix},
\]

\[
A_{c3} =
\begin{bmatrix}
a_{1,1001} & a_{1,1002} & \cdots & a_{1,1500} \\
a_{2,1001} & a_{2,1002} & \cdots & a_{2,1500} \\
\cdots & \cdots & \cdots & \cdots \\
a_{400,1001} & a_{400,1002} & \cdots & a_{400,1500}
\end{bmatrix},
\]

\begin{equation}
A_{c4} = \begin{bmatrix}
a_{1,1501} & a_{1,1502} & \cdots & a_{1,2000} \\
a_{2,1501} & a_{2,1502} & \cdots & a_{2,2000} \\
\cdots & \cdots & \cdots & \cdots \\
a_{400,1501} & a_{400,1502} & \cdots & a_{400,2000}
\end{bmatrix}
\end{equation}

则

\begin{equation}
A_{c400*2000} = \begin{bmatrix}
A_{c1} & A_{c2} & A_{c3} & A_{c4}
\end{bmatrix}
\end{equation}

遍历四条次棱的位置,得到

\begin{equation}
\begin{cases}
\begin{bmatrix}
A_{c1} & A_{c2} & A_{c3} & A_{c4}
\end{bmatrix} - B_{c400*2000} = C_{1} \\
\begin{bmatrix}
A_{c2} & A_{c3} & A_{c4} & A_{c1}
\end{bmatrix} - B_{c400*2000} = C_{2} \\
\begin{bmatrix}
A_{c3} & A_{c4} & A_{c1} & A_{c2}
\end{bmatrix} - B_{c400*2000} = C_{3} \\
\begin{bmatrix}
A_{c4} & A_{c1} & A_{c2} & A_{c3}
\end{bmatrix} - B_{c400*2000} = C_{4}
\end{cases}
\end{equation}

计算矩阵 $|C_{(i)}|$ 的值,如下

\begin{equation}
|C_{(i)}| = \sum_{n=1}^{400} \sum_{m=1}^{2000} |c_{(i)n,m}|
\end{equation}

如此得到 $|C_{(i)}| (i=1,2,3,4)$,最后得到 $A_{c400*2000}$ 和 $B_{c400*2000}$ 的 MAD 值为:

\begin{equation}
|A_{c400*2000} - B_{c400*2000}|_{MAD} = Min |C_{(i)}| (i=1,2,3,4)
\end{equation}

\section{参考文献}

\begin{enumerate}
    \item 金涛, 单岩, 童水光, 实物测量造型中的测量数据重定位方法, 计算机辅助设计与图形学学报, 13(4):315-318, 2001。
    \item 鲁静, 基于三维激光彩色扫描仪的公安子弹综合检测软件系统关键算法研究和实现, 华中科技大学硕士学位论文, 2006。
    \item 张志涌等, 精通 MATLAB 6.5 版, 北京: 北京航空航天大学出版社, 2003
    \item 姜启源等, 数学模型, 北京: 高等教育出版社, 2005
    \item 苏金明等, MATLAB 工程数学, 北京: 电子工业出版社, 2005
    \item 陈庆虎, 周莉萍, 谢铁邦, 李柱, 表面综合形貌的小波分离法, 华中理工大学学报, 25(5): 25-27, 1997
    \item 杨敏, 王纬东, 牟丽, 景畅, 弹头痕迹图像分形动力学研究, 兵工学报, 30(4): 398-402, 2009
    \item 杨敏, 李冬云, 王纬东, 基于局部小波能量工具痕迹识别的研究, 中国人民公安大学学报, 2(56), 73-75, 2008
\end{enumerate}

\section{附件}

\section*{8.1 通过坐标转换后利用 MAD 算法实现子弹的自动匹配}

在通过 matlab 编程具体实现时,由于从数据预处理到最后结果的给出,需要处理大量的数据,鉴于所使用个人计算机处理能力的考虑,将具体计算过程分解为以下步骤,每一小节为整个计算流程的一部分。

\subsection*{8.1.1 数据预处理代码}

代码名字:data2matrix.m

代码内容:

\begin{verbatim}
% 将一个子弹的每一个次棱 426384x3 数据转换为 756*564 的矩阵 DATE: 200909
% 其中 t22_cl 为转换成矩阵形式后的第一个次棱数据,其中 t22 为子弹编号
%%%%%%%%%%%%%%%%%%%%%%%%%%%%%%%%%%%%%%%%%%%%%%%%%%%%%%%%%%%%%%%
load c1.dat %将第一个次棱的原始数据读入
load c2.dat %第二个
load c3.dat %。
load c4.dat %

for i=1:426384
t22_cl(ceil(i/564),i-ceil(i/564)*564+564)=c1(i,3); %第一个次棱的变化
t22_c2(ceil(i/564),i-ceil(i/564)*564+564)=c2(i,3);
t22_c3(ceil(i/564),i-ceil(i/564)*564+564)=c3(i,3);
t22_c4(ceil(i/564),i-ceil(i/564)*564+564)=c4(i,3);
end
save t22.mat t22_cl t22_c2 t22_c3 t22_c4 %转换后的矩阵存储起来
clear all %清理系统内存
\end{verbatim}

\subsection*{8.1.2 数据的坐标转换程序}

代码名字:circledata_22.m

说明:该代码只处理第 22 颗子弹的数据,其他的数据雷同

代码内容:

\begin{verbatim}
% 进行坐标转换,将相对于观测仪器的坐标系转换为基于子弹表面的坐标,求得相应
% 新坐标系下的矩阵。为了对于后面数据的处理,并且减轻计算量,该段代码只计算了
% 原始矩阵中第 201 到第 600 共 400 行、纵坐标从第 1 到第 500 共 500 列的矩阵对应值。
% 其中原始矩阵中的行和列的选取可以根据情况修改。
%%%%%%%%%%%%%%%%%%%%%%%%%%%%%%%%%%%%%%%%%%%%%%%%%%%%%%%%%%%%%%%
\end{verbatim}

\begin{verbatim}
%%% 
clcclcclcclcclcclcclcclcclcclcclcclcclcclcclcclcclcclcclcclcclcclcclcclcclcclcclcclcclcclcclcclcclcclcclcclcclcclcclcclcclcclcclcclcclcclcclcclcclcclcclcclcclcclcclcclcclcclcclcclcclcclcclcclcclcclcclcclcclcclcclcclcclcclcclcclcclcclcclcclcclcclcclcclcclcclcclcclcclcclcclcclcclcclcclcclcclcclcclcclcclcclcclcclcclcclcclcclcclcclcclcclcclcclcclcclcclcclcclcclcclcclcclcclcclcclcclcclcclcclcclcclcclcclcclcclcclcclcclcclcclcclcclcclcclcclcclcclcclcclcclcclcclcclcclcclcclcclcclcclcclcclcclcclcclcclcclcclcclcclcclcclcclcclcclcclcclcclcclcclcclcclcclcclcclcclcclcclcclcclcclcclcclcclcclcclcclcclcclcclcclcclcclcclcclcclcclcclcclcclcclcclcclcclcclcclcclcclcclcclcclcclcclcclcclcclcclcclcclcclcclcclcclcclcclcclcclcclcclcclcclcclcclcclcclcclcclcclcclcclcclcclcclcclcclcclcclcclcclcclcclcclcclcclcclcclcclcclcclcclcclcclcclcclcclcclcclcclcclcclcclcclcclcclcclcclcclcclcclcclcclcclcclcclcclcclcclcclcclcclcclcclcclcclcclcclcclcclcclcclcclcclcclcclcclcclcclcclcclcclcclcclcclcclcclcclcclcclcclcclcclcclcclcclcclcclcclcclcclcclcclcclcclcclcclcclcclcclcclcclcclcclcclcclcclcclcclcclcclcclcclcclcclcclcclcclcclcclcclcclcclcclcclcclcclcclcclcclcclcclcclcclcclcclcclcclcclcclcclcclcclcclcclcclcclcclcclcclcclcclcclcclcclcclcclcclcclcclcclcclcclcclcclcclcclcclcclcclcclcclcclcclcclcclcclcclcclcclcclcclcclcclcclcclcclcclcclcclcclcclcclcclcclcclcclcclcclcclcclcclcclcclcclcclcclcclcclcclcclcclcclcclcclcclcclcclcclcclcclcclcclcclcclcclcclcclcclcclcclcclcclcclcclcclcclcclcclcclcclcclcclcclcclcclcclcclcclcclcclcclcclcclcclcclcclcclcclcclcclcclcclcclcclcclcclcclcclcclcclcclcclcclcclcclcclcclcclcclcclcclcclcclcclcclcclcclcclcclcclcclcclcclcclcclcclcclcclcclcclcclcclcclcclcclcclcclcclcclcclcclcclcclcclcclcclcclcclcclcclcclcclcclcclcclcclcclcclcclcclcclcclcclcclcclcclcclcclcclcclcclcclcclcclcclcclcclcclcclcclcclcclcclcclcclcclcclcclcclcclcclcclcclcclcclcclcclcclcclcclcclcclcclcclcclcclcclcclcclcclcclcclcclcclcclcclcclcclcclcclcclcclcclcclcclcclcclcclcclcclcclcclcclcclcclcclcclcclcclcclcclcclcclcclcclcclcclcclcclcclcclcclcclcclcclcclcclcclcclcclcclcclcclcclcclcclcclcclcclcclcclcclcclcclcclcclcclcclcclcclcclcclcclcclcclcclcclcclcclcclcclcclcclcclcclcclcclcclcclcclcclcclcclcclcclcclcclcclcclcclcclcclcclcclcclcclcclcclcclcclcclcclcclcclcclcclcclcclcclcclcclcclcclcclcclcclcclcclcclcclcclcclcclcclcclcclcclcclcclcclcclcclcclcclcclcclcclcclcclcclcclcclcclcclcclcclcclcclcclcclcclcclcclcclcclcclcclcclcclcclcclcclcclcclcclcclcclcclcclcclcclcclcclcclcclcclcclcclcclcclcclcclcclcclcclcclcclcclcclcclcclcclcclcclcclcclcclcclcclcclcclcclcclcclcclcclcclcclcclcclcclcclcclcclcclcclcclcclcclcclcclcclcclcclcclcclcclcclcclcclcclcclcclcclcclcclcclcclcclcclcclcclcclcclcclcclcclcclcclcclcclcclcclcclcclcclcclcclcclcclcclcclcclcclcclcclcclcclcclcclcclcclcclcclcclcclcclcclcclcclcclcclcclcclcclcclcclcclcclcclcclcclcclcclcclcclcclcclcclcclcclcclcclcclcclcclcclcclcclcclcclcclcclcclcclcclcclcclcclcclcclcclcclcclcclcclcclcclcclcclcclcclcclcclcclcclcclcclcclcclcclcclcclcclcclcclcclcclcclcclcclcclcclcclcclcclcclcclcclcclcclcclcclcclcclcclcclcclcclcclcclcclcclcclcclcclcclcclcclcclcclcclcclcclcclcclcclcclcclcclcclcclcclcclcclcclcclcclcclcclcclcclcclcclcclcclcclcclcclcclcclcclcclcclcclcclcclcclcclcclcclcclcclcclcclcclcclcclcclcclcclcclcclcclcclcclcclcclcclcclcclcclcclcclcclcclcclcclcclcclcclcclcclcclcclcclcclcclcclcclcclcclcclcclcclcclcclcclcclcclcclcclcclcclcclcclcclcclcclcclcclcclcclcclcclcclcclcclcclcclcclcclcclcclcclcclcclcclcclcclcclcclcclcclcclcclcclcclcclcclcclcclcclcclcclcclcclcclcclcclcclcclcclcclcclcclcclcclcclcclcclcclcclcclcclcclcclcclcclcclcclcclcclcclcclcclcclcclcclcclcclcclcclcclcclcclcclcclcclcclcclcclcclcclcclcclcclcclcclcclcclcclcclcclcclcclcclcclcclcclcclcclcclcclcclcclcclcclcclcclcclcclcclcclcclcclcclcclcclcclcclcclcclcclcclcclcclcclcclcclcclcclcclcclcclcclcclcclcclcclcclcclcclcclcclcclcclcclcclcclcclcclcclcclcclcclcclcclcclcclcclcclcclcclcclcclcclcclcclcclcclcclcclcclcclcclcclcclcclcclcclcclcclcclcclcclcclcclcclcclcclcclcclcclcclcclcclcclcclcclcclcclcclcclcclcclcclcclcclcclcclcclcclcclcclcclcclcclcclcclcclcclcclcclcclcclcclcclcclcclcclcclcclcclcclcclcclcclcclcclcclcclcclcclcclcclcclcclcclcclcclcclcclcclcclcclcclcclcclcclcclcclcclcclcclcclcclcclcclcclcclcclcclcclcclcclcclcclcclcclcclcclcclcclcclcclcclcclcclcclcclcclcclcclcclcclcclcclcclcclcclcclcclcclcclcclcclcclcclcclcclcclcclcclcclcclcclcclcclcclcclcclcclcclcclcclcclcclcclcclcclcclcclcclcclcclcclcclcclcclcclcclcclcclcclcclcclcclcclcclcclcclcclcclcclcclcclcclcclcclcclcclcclcclcclcclcclcclcclcclcclcclcclcclcclcclcclcclcclcclcclcclcclcclcclcclcclcclcclcclcclcclcclcclcclcclcclcclcclcclcclcclcclcclcclcclcclcclcclcclcclcclcclcclcclcclcclcclcclcclcclcclcclcclcclcclcclcclcclcclcclcclcclcclcclcclcclcclcclcclcclcclcclcclcclcclcclcclcclcclcclcclcclcclcclcclcclcclcclcclcclcclcclcclcclcclcclcclcclcclcclcclcclcclcclcclcclcclcclcclcclcclcclcclcclcclcclcclcclcclcclcclcclcclcclcclcclcclcclcclcclcclcclcclcclcclcclcclcclcclcclcclcclcclcclcclcclcclcclcclcclcclcclcclcclcclcclcclcclcclcclcclcclcclcclcclcclcclcclcclcclcclcclcclcclcclcclcclcclcclcclcclcclcclcclcclcclcclcclcclcclcclcclcclcclcclcclcclcclcclcclcclcclcclcclcclcclcclcclcclcclcclcclcclcclcclcclcclcclcclcclcclcclcclcclcclcclcclcclcclcclcclcclcclcclcclcclcclcclcclcclcclcclcclcclcclcclcclcclcclcclcclcclcclcclcclcclcclcclcclcclcclcclcclcclcclcclcclcclcclcclcclcclcclcclcclcclcclcclcclcclcclcclcclcclcclcclcclcclcclcclcclcclcclcclcclcclcclcclcclcclcclcclcclcclcclcclcclcclcclcclcclcclcclcclcclcclcclcclcclcclcclcclcclcclcclcclcclcclcclcclcclcclcclcclcclcclcclcclcclcclcclcclcclcclcclcclcclcclcclcclcclcclcclcclcclcclcclcclcclcclcclcclcclcclcclcclcclcclcclcclcclcclcclcclcclcclcclcclcclcclcclcclcclcclcclcclcclcclcclcclcclcclcclcclcclcclcclcclcclcclcclcclcclcclcclcclcclcclcclcclcclcclcclcclcclcclcclcclcclcclcclcclcclcclcclcclcclcclcclcclcclcclcclcclcclcclcclcclcclcclcclcclcclcclcclcclcclcclcclcclcclcclcclcclcclcclcclcclcclcclcclcclcclcclcclcclcclcclcclcclcclcclcclcclcclcclcclcclcclcclcclcclcclcclcclcclcclcclcclcclcclcclcclcclcclcclcclcclcclcclcclcclcclcclcclcclcclcclcclcclcclcclcclcclcclcclcclcclcclcclcclcclcclcclcclcclcclcclcclcclcclcclcclcclcclcclcclcclcclcclcclcclcclcclcclcclcclcclcclcclcclcclcclcclcclcclcclcclcclcclcclcclcclcclcclcclcclcclcclcclcclcclcclcclcclcclcclcclcclcclcclcclcclcclcclcclcclcclcclcclcclcclcclcclcclcclcclcclcclcclcclcclcclcclcclcclcclcclcclcclcclcclcclcclcclcclcclcclcclcclcclcclcclcclcclcclcclcclcclcclcclcclcclcclcclcclcclcclcclcclcclcclcclcclcclcclcclcclcclcclcclcclcclcclcclcclcclcclcclcclcclcclcclcclcclcclcclcclcclcclcclcclcclcclcclcclcclcclcclcclcclcclcclcclcclcclcclcclcclcclcclcclcclcclcclcclcclcclcclcclcclcclcclcclcclcclcclcclcclcclcclcclcclcclcclcclcclcclcclcclcclcclcclcclcclcclcclcclcclcclcclcclcclcclcclcclcclcclcclcclcclcclcclcclcclcclcclcclcclcclcclcclcclcclcclcclcclcclcclcclcclcclcclcclcclcclcclcclcclcclcclcclcclcclcclcclcclcclcclcclcclcclcclcclcclcclcclcclcclcclcclcclcclcclcclcclcclcclcclcclcclcclcclcclcclcclcclcclcclcclcclcclcclcclcclcclcclcclcclcclcclcclcclcclcclcclcclcclcclcclcclcclcclcclcclcclcclcclcclcclcclcclcclcclcclcclcclcclcclcclcclcclcclcclcclcclcclcclcclcclcclcclcclcclcclcclcclcclcclcclcclcclcclcclcclcclcclcclcclcclcclcclcclcclcclcclcclcclcclcclcclcclcclcclcclcclcclcclcclcclcclcclcclcclcclcclcclcclcclcclcclcclcclcclcclcclcclcclcclcclcclcclcclcclcclcclcclcclcclcclcclcclcclcclcclcclcclcclcclcclcclcclcclcclcclcclcclcclcclcclcclcclcclcclcclcclcclcclcclcclcclcclcclcclcclcclcclcclcclcclcclcclcclcclcclcclcclcclcclcclcclcclcclcclcclcclcclcclcclcclcclcclcclcclcclcclcclcclcclcclcclcclcclcclcclcclcclcclcclcclcclcclcclcclcclcclcclcclcclcclcclcclcclcclcclcclcclcclcclcclcclcclcclcclcclcclcclcclcclcclcclcclcclcclcclcclcclcclcclcclcclcclcclcclcclcclcclcclcclcclcclcclcclcclcclcclcclcclcclcclcclcclcclcclcclcclcclcclcclcclcclcclcclcclcclcclcclcclcclcclcclcclcclcclcclcclcclcclcclcclcclcclcclcclcclcclcclcclcclcclcclcclcclcclcclcclcclcclcclcclcclcclcclcclcclcclcclcclcclcclcclcclcclcclcclcclcclcclcclcclcclcclcclcclcclcclcclcclcclcclcclcclcclcclcclcclcclcclcclcclcclcclcclcclcclcclcclcclcclcclcclcclcclcclcclcclcclcclcclcclcclcclcclcclcclcclcclcclcclcclcclcclcclcclcclcclcclcclcclcclcclcclcclcclcclcclcclcclcclcclcclcclcclcclcclcclcclcclcclcclcclcclcclcclcclcclcclcclcclcclcclcclcclcclcclcclcclcclcclcclcclcclcclcclcclcclcclcclcclcclcclcclcclcclcclcclcclcclcclcclcclcclcclcclcclcclcclcclcclcclcclcclcclcclcclcclcclcclcclcclcclcclcclcclcclcclcclcclcclcclcclcclcclcclcclcclcclcclcclcclcclcclcclcclcclcclcclcclcclcclcclcclcclcclcclcclcclcclcclcclcclcclcclcclcclcclcclcclcclcclcclcclcclcclcclcclcclcclcclcclcclcclcclcclcclcclcclcclcclcclcclcclcclcclcclcclcclcclcclcclcclcclcclcclcclcclcclcclcclcclcclcclcclcclcclcclcclcclcclcclcclcclcclcclcclcclcclcclcclcclcclcclcclcclcclcclcclcclcclcclcclcclcclcclcclcclcclcclcclcclcclcclcclcclcclcclcclcclcclcclcclcclcclcclcclcclcclcclcclcclcclcclcclcclcclcclcclcclcclcclcclcclcclcclcclcclcclcclcclcclcclcclcclcclcclcclcclcclcclcclcclcclcclcclcclcclcclcclcclcclcclcclcclcclcclcclcclcclcclcclcclcclcclcclcclcclcclcclcclcclcclcclcclcclcclcclcclcclcclcclcclcclcclcclcclcclcclcclcclcclcclcclcclcclcclcclcclcclcclcclcclcclcclcclcclcclcclcclcclcclcclcclcclcclcclcclcclcclcclcclcclcclcclcclcclcclcclcclcclcclcclcclcclcclcclcclcclcclcclcclcclcclcclcclcclcclcclcclcclcclcclcclcclcclcclcclcclcclcclcclcclcclcclcclcclcclcclcclcclcclcclcclcclcclcclcclcclcclcclcclcclcclcclcclcclcclcclcclcclcclcclcclcclcclcclcclcclcclcclcclcclcclcclcclcclcclcclcclcclcclcclcclcclcclcclcclcclcclcclcclcclcclcclcclcclcclcclcclcclcclcclcclcclcclcclcclcclcclcclcclcclcclcclcclcclcclcclcclcclcclcclcclcclcclcclcclcclcclcclcclcclcclcclcclcclcclcclcclcclcclcclcclcclcclcclcclcclcclcclcclcclcclcclcclcclcclcclcclcclcclcclcclcclcclcclcclcclcclcclcclcclcclcclcclcclcclcclcclcclcclcclcclcclcclcclcclcclcclcclcclcclcclcclcclcclcclcclcclcclcclcclcclcclcclcclcclcclcclcclcclcclcclcclcclcclcclcclcclcclcclcclcclcclcclcclcclcclcclcclcclcclcclcclcclcclcclcclcclcclcclcclcclcclcclcclcclcclcclcclcclcclcclcclcclcclcclcclcclcclcclcclcclcclcclcclcclcclcclcclcclcclcclcclcclcclcclcclcclcclcclcclcclcclcclcclcclcclcclcclcclcclcclcclcclcclcclcclcclcclcclcclcclcclcclcclcclcclcclcclcclcclcclcclcclcclcclcclcclcclcclcclcclcclcclcclcclcclcclcclcclcclcclcclcclcclcclcclcclcclcclcclcclcclcclcclcclcclcclcclcclcclcclcclcclcclcclcclcclcclcclcclcclcclcclcclcclcclcclcclcclcclcclcclcclcclcclcclcclcclcclcclcclcclcclcclcclcclcclcclcclcclcclcclcclcclcclcclcclcclcclcclcclcclcclcclcclcclcclcclcclcclcclcclcclcclcclcclcclcclcclcclcclcclcclcclcclcclcclcclcclcclcclcclcclcclcclcclcclcclcclcclcclcclcclcclcclcclcclcclcclcclcclcclcclcclcclcclcclcclcclcclcclcclcclcclcclcclcclcclcclcclcclcclcclcclcclcclcclcclcclcclcclcclcclcclcclcclcclcclcclcclcclcclcclcclcclcclcclcclcclcclcclcclcclcclcclcclcclcclcclcclcclcclcclcclcclcclcclcclcclcclcclcclcclcclcclcclcclcclcclcclcclcclcclcclcclcclcclcclcclcclcclcclcclcclcclcclcclcclcclcclcclcclcclcclcclcclcclcclcclcclcclcclcclcclcclcclcclcclcclcclcclcclcclcclcclcclcclcclcclcclcclcclcclcclcclcclcclcclcclcclcclcclcclcclcclcclcclcclcclcclcclcclcclcclcclcclcclcclcclcclcclcclcclcclcclcclcclcclcclcclcclcclcclcclcclcclcclcclcclcclcclcclcclcclcclcclcclcclcclcclcclcclcclcclcclcclcclcclcclcclcclcclcclcclcclcclcclcclcclcclcclcclcclcclcclcclcclcclcclcclcclcclcclcclcclcclcclcclcclcclcclcclcclcclcclcclcclcclcclcclcclcclcclcclcclcclcclcclcclcclcclcclcclcclcclcclcclcclcclcclcclcclcclcclcclcclcclcclcclcclcclcclcclcclcclcclcclcclcclcclcclcclcclcclcclcclcclcclcclcclcclcclcclcclcclcclcclcclcclcclcclcclcclcclcclcclcclcclcclcclcclcclcclcclcclcclcclcclcclcclcclcclcclcclcclcclcclcclcclcclcclcclcclcclcclcclcclcclcclcclcclcclcclcclcclcclcclcclcclcclcclcclcclcclcclcclcclcclcclcclcclcclcclcclcclcclcclcclcclcclcclcclcclcclcclcclcclcclcclcclcclcclcclcclcclcclcclcclcclcclcclcclcclcclcclcclcclcclcclcclcclcclcclcclcclcclcclcclcclcclcclcclcclcclcclcclcclcclcclcclcclcclcclcclcclcclcclcclcclcclcclcclcclcclcclcclcclcclcclcclcclcclcclcclcclcclcclcclcclcclcclcclcclcclcclcclcclcclcclcclcclcclcclcclcclcclcclcclcclcclcclcclcclcclcclcclcclcclcclcclcclcclcclcclcclcclcclcclcclcclcclcclcclcclcclcclcclcclcclcclcclcclcclcclcclcclcclcclcclcclcclcclcclcclcclcclcclcclcclcclcclcclcclcclcclcclcclcclcclcclcclcclcclcclcclcclcclcclcclcclcclcclcclcclcclcclcclcclcclcclcclcclcclcclcclcclcclcclcclcclcclcclcclcclcclcclcclcclcclcclcclcclcclcclcclcclcclcclcclcclcclcclcclcclcclcclcclcclcclcclcclcclcclcclcclcclcclcclcclcclcclcclcclcclcclcclcclcclcclcclcclcclcclcclcclcclcclcclcclcclcclcclcclcclcclcclcclcclcclcclcclcclcclcclcclcclcclcclcclcclcclcclcclcclcclcclcclcclcclcclcclcclcclcclcclcclcclcclcclcclcclcclcclcclcclcclcclcclcclcclcclcclcclcclcclcclcclcclcclcclcclcclcclcclcclcclcclcclcclcclcclcclcclcclcclcclcclcclcclcclcclcclcclcclcclcclcclcclcclcclcclcclcclcclcclcclcclcclcclcclcclcclcclcclcclcclcclcclcclcclcclcclcclcclcclcclcclcclcclcclcclcclcclcclcclcclcclcclcclcclcclcclcclcclcclcclcclcclcclcclcclcclcclcclcclcclcclcclcclcclcclcclcclcclcclcclcclcclcclcclcclcclcclcclcclcclcclcclcclcclcclcclcclcclcclcclcclcclcclcclcclcclcclcclcclcclcclcclcclcclcclcclcclcclcclcclcclcclcclcclcclcclcclcclcclcclcclcclcclcclcclcclcclcclcclcclcclcclcclcclcclcclcclcclcclcclcclcclcclcclcclcclcclcclcclcclcclcclcclcclcclcclcclcclcclcclcclcclcclcclcclcclcclcclcclcclcclcclcclcclcclcclcclcclcclcclcclcclcclcclcclcclcclcclcclcclcclcclcclcclcclcclcclcclcclcclcclcclcclcclcclcclcclcclcclcclcclcclcclcclcclcclcclcclcclcclcclcclcclcclcclcclcclcclcclcclcclcclcclcclcclcclcclcclcclcclcclcclcclcclcclcclcclcclcclcclcclcclcclcclcclcclcclcclcclcclcclcclcclcclcclcclcclcclcclcclcclcclcclcclcclcclcclcclcclcclcclcclcclcclcclcclcclcclcclcclcclcclcclcclcclcclcclcclcclcclcclcclcclcclcclcclcclcclcclcclcclcclcclcclcclcclcclcclcclcclcclcclcclcclcclcclcclcclcclcclcclcclcclcclcclcclcclcclcclcclcclcclcclcclcclcclcclcclcclcclcclcclcclcclcclcclcclcclcclcclcclcclcclcclcclcclcclcclcclcclcclcclcclcclcclcclcclcclcclcclcclcclcclcclcclcclcclcclcclcclcclcclcclcclcclcclcclcclcclcclcclcclcclcclcclcclcclcclcclcclcclcclcclcclcclcclcclcclcclcclcclcclcclcclcclcclcclcclcclcclcclcclcclcclcclcclcclcclcclcclcclcclcclcclcclcclcclcclcclcclcclcclcclcclcclcclcclcclcclcclcclcclcclcclcclcclcclcclcclcclcclcclcclcclcclcclcclcclcclcclcclcclcclcclcclcclcclcclcclcclcclcclcclcclcclcclcclcclcclcclcclcclcclcclcclcclcclcclcclcclcclcclcclcclcclcclcclcclcclcclcclcclcclcclcclcclcclcclcclcclcclcclcclcclcclcclcclcclcclcclcclcclcclcclcclcclcclcclcclcclcclcclcclcclcclcclcclcclcclcclcclcclcclcclcclcclcclcclcclcclcclcclcclcclcclcclcclcclcclcclcclcclcclcclcclcclcclcclcclcclcclcclcclcclcclcclcclcclcclcclcclcclcclcclcclcclcclcclcclcclcclcclcclcclcclcclcclcclcclcclcclcclcclcclcclcclcclcclcclcclcclcclcclcclcclcclcclcclcclcclcclcclcclcclcclcclcclcclcclcclcclcclcclcclcclcclcclcclcclcclcclcclcclcclcclcclcclcclcclcclcclcclcclcclcclcclcclcclcclcclcclcclcclcclcclcclcclcclcclcclcclcclcclcclcclcclcclcclcclcclcclcclcclcclcclcclcclcclcclcclcclcclcclcclcclcclcclcclcclcclcclcclcclcclcclcclcclcclcclcclcclcclcclcclcclcclcclcclcclcclcclcclcclcclcclcclcclcclcclcclcclcclcclcclcclcclcclcclcclcclcclcclcclcclcclcclcclcclcclcclcclcclcclcclcclcclcclcclcclcclcclcclcclcclcclcclcclcclcclcclcclcclcclcclcclcclcclcclcclcclcclcclcclcclcclcclcclcclcclcclcclcclcclcclcclcclcclcclcclcclcclcclcclcclcclcclcclcclcclcclcclcclcclcclcclcclcclcclcclcclcclcclcclcclcclcclcclcclcclcclcclcclcclcclcclcclcclcclcclcclcclcclcclcclcclcclcclcclcclcclcclcclcclcclcclcclcclcclcclcclcclcclcclcclcclcclcclcclcclcclcclcclcclcclcclcclcclcclcclcclcclcclcclcclcclcclcclcclcclcclcclcclcclcclcclcclcclcclcclcclcclcclcclcclcclcclcclcclcclcclcclcclcclcclcclcclcclcclcclcclcclcclcclcclcclcclcclcclcclcclcclcclcclcclcclcclcclcclcclcclcclcclcclcclcclcclcclcclcclcclcclcclcclcclcclcclcclcclcclcclcclcclcclcclcclcclcclcclcclcclcclcclcclcclcclcclcclcclcclcclcclcclcclcclcclcclcclcclcclcclcclcclcclcclcclcclcclcclcclcclcclcclcclcclcclcclcclcclcclcclcclcclcclcclcclcclcclcclcclcclcclcclcclcclcclcclcclcclcclcclcclcclcclcclcclcclcclcclcclcclcclcclcclcclcclcclcclcclcclcclcclcclcclcclcclcclcclcclcclcclcclcclcclcclcclcclcclcclcclcclcclcclcclcclcclcclcclcclcclcclcclcclcclcclcclcclcclcclcclcclcclcclcclcclcclcclcclcclcclcclcclcclcclcclcclcclcclcclcclcclcclcclcclcclcclcclcclcclcclcclcclcclcclcclcclcclcclcclcclcclcclcclcclcclcclcclcclcclcclcclcclcclcclcclcclcclcclcclcclcclcclcclcclcclcclcclcclcclcclcclcclcclcclcclcclcclcclcclcclcclcclcclcclcclcclcclcclcclcclcclcclcclcclcclcclcclcclcclcclcclcclcclcclcclcclcclcclcclcclcclcclcclcclcclcclcclcclcclcclcclcclcclcclcclcclcclcclcclcclcclcclcclcclcclcclcclcclcclcclcclcclcclcclcclcclcclcclcclcclcclcclcclcclcclcclcclcclcclcclcclcclcclcclcclcclcclcclcclcclcclcclcclcclcclcclcclcclcclcclcclcclcclcclcclcclcclcclcclcclcclcclcclcclcclcclcclcclcclcclcclcclcclcclcclcclcclcclcclcclcclcclcclcclcclcclcclcclcclcclcclcclcclcclcclcclcclcclcclcclcclcclcclcclcclcclcclcclcclcclcclcclcclcclcclcclcclcclcclcclcclcclcclcclcclcclcclcclcclcclcclcclcclcclcclcclcclcclcclcclcclcclcclcclcclcclcclcclcclcclcclcclcclcclcclcclcclcclcclcclcclcclcclcclcclcclcclcclcclcclcclcclcclcclcclcclcclcclcclcclcclcclcclcclcclcclcclcclcclcclcclcclcclcclcclcclcclcclcclcclcclcclcclcclcclcclcclcclcclcclcclcclcclcclcclcclcclcclcclcclcclcclcclcclcclcclcclcclcclcclcclcclcclcclcclcclcclcclcclcclcclcclcclcclcclcclcclcclcclcclcclcclcclcclcclcclcclcclcclcclcclcclcclcclcclcclcclcclcclcclcclcclcclcclcclcclcclcclcclcclcclcclcclcclcclcclcclcclcclcclcclcclcclcclcclcclcclcclcclcclcclcclcclcclcclcclcclcclcclcclcclcclcclcclcclcclcclcclcclcclcclcclcclcclcclcclcclcclcclcclcclcclcclcclcclcclcclcclcclcclcclcclcclcclcclcclcclcclcclcclcclcclcclcclcclcclcclcclcclcclcclcclcclcclcclcclcclcclcclcclcclcclcclcclcclcclcclcclcclcclcclcclcclcclcclcclcclcclcclcclcclcclcclcclcclcclcclcclcclcclcclcclcclcclcclcclcclcclcclcclcclcclcclcclcclcclcclcclcclcclcclcclcclcclcclcclcclcclcclcclcclcclcclcclcclcclcclcclcclcclcclcclcclcclcclcclcclcclcclcclcclcclcclcclcclcclcclcclcclcclcclcclcclcclcclcclcclcclcclcclcclcclcclcclcclcclcclcclcclcclcclcclcclcclcclcclcclcclcclcclcclcclcclcclcclcclcclcclcclcclcclcclcclcclcclcclcclcclcclcclcclcclcclcclcclcclcclcclcclcclcclcclcclcclcclcclcclcclcclcclcclcclcclcclcclcclcclcclcclcclcclcclcclcclcclcclcclcclcclcclcclcclcclcclcclcclcclcclcclcclcclcclcclcclcclcclcclcclcclcclcclcclcclcclcclcclcclcclcclcclcclcclcclcclcclcclcclcclcclcclcclcclcclcclcclcclcclcclcclcclcclcclcclcclcclcclcclcclcclcclcclcclcclcclcclcclcclcclcclcclcclcclcclcclcclcclcclcclcclcclcclcclcclcclcclcclcclcclcclcclcclcclcclcclcclcclcclcclcclcclcclcclcclcclcclcclcclcclcclcclcclcclcclcclcclcclcclcclcclcclcclcclcclcclcclcclcclcclcclcclcclcclcclcclcclcclcclcclcclcclcclcclcclcclcclcclcclcclcclcclcclcclcclcclcclcclcclcclcclcclcclcclcclcclcclcclcclcclcclcclcclcclcclcclcclcclcclcclcclcclcclcclcclcclcclcclcclcclcclcclcclcclcclcclcclcclcclcclcclcclcclcclcclcclcclcclcclcclcclcclcclcclcclcclcclcclcclcclcclcclcclcclcclcclcclcclcclcclcclcclcclcclcclcclcclcclcclcclcclcclcclcclcclcclcclcclcclcclcclcclcclcclcclcclcclcclcclcclcclcclcclcclcclcclcclcclcclcclcclcclcclcclcclcclcclcclcclcclcclcclcclcclcclcclcclcclcclcclcclcclcclcclcclcclcclcclcclcclcclcclcclcclcclcclcclcclcclcclcclcclcclcclcclcclcclcclcclcclcclcclcclcclcclcclcclcclcclcclcclcclcclcclcclcclcclcclcclcclcclcclcclcclcclcclcclcclcclcclcclcclcclcclcclcclcclcclcclcclcclcclcclcclcclcclcclcclcclcclcclcclcclcclcclcclcclcclcclcclcclcclcclcclcclcclcclcclcclcclcclcclcclcclcclcclcclcclcclcclcclcclcclcclcclcclcclcclcclcclcclcclcclcclcclcclcclcclcclcclcclcclcclcclcclcclcclcclcclcclcclcclcclcclcclcclcclcclcclcclcclcclcclcclcclcclcclcclcclcclcclcclcclcclcclcclcclcclcclcclcclcclcclcclcclcclcclcclcclcclcclcclcclcclcclcclcclcclcclcclcclcclcclcclcclcclcclcclcclcclcclcclcclcclcclcclcclcclcclcclcclcclcclcclcclcclcclcclcclcclcclcclcclcclcclcclcclcclcclcclcclcclcclcclcclcclcclcclcclcclcclcclcclcclcclcclcclcclcclcclcclcclcclcclcclcclcclcclcclcclcclcclcclcclcclcclcclcclcclcclcclcclcclcclcclcclcclcclcclcclcclcclcclcclcclcclcclcclcclcclcclcclcclcclcclcclcclcclcclcclcclcclcclcclcclcclcclcclcclcclcclcclcclcclcclcclcclcclcclcclcclcclcclcclcclcclcclcclcclcclcclcclcclcclcclcclcclcclcclcclcclcclcclcclcclcclcclcclcclcclcclcclcclcclcclcclcclcclcclcclcclcclcclcclcclcclcclcclcclcclcclcclcclcclcclcclcclcclcclcclcclcclcclcclcclcclcclcclcclcclcclcclcclcclcclcclcclcclcclcclcclcclcclcclcclcclcclcclcclcclcclcclcclcclcclcclcclcclcclcclcclcclcclcclcclcclcclcclcclcclcclcclcclcclcclcclcclcclcclcclcclcclcclcclcclcclcclcclcclcclcclcclcclcclcclcclcclcclcclcclcclcclcclcclcclcclcclcclcclcclcclcclcclcclcclcclcclcclcclcclcclcclcclcclcclcclcclcclcclcclcclcclcclcclcclcclcclcclcclcclcclcclcclcclcclcclcclcclcclcclcclcclcclcclcclcclcclcclcclcclcclcclcclcclcclcclcclcclcclcclcclcclcclcclcclcclcclcclcclcclcclcclcclcclcclcclcclcclcclcclcclcclcclcclcclcclcclcclcclcclcclcclcclcclcclcclcclcclcclcclcclcclcclcclcclcclcclcclcclcclcclcclcclcclcclcclcclcclcclcclcclcclcclcclcclcclcclcclcclcclcclcclcclcclcclcclcclcclcclcclcclcclcclcclcclcclcclcclcclcclcclcclcclcclcclcclcclcclcclcclcclcclcclcclcclcclcclcclcclcclcclcclcclcclcclcclcclcclcclcclcclcclcclcclcclcclcclcclcclcclcclcclcclcclcclcclcclcclcclcclcclcclcclcclcclcclcclcclcclcclcclcclcclcclcclcclcclcclcclcclcclcclcclcclcclcclcclcclcclcclcclcclcclcclcclcclcclcclcclcclcclcclcclcclcclcclcclcclcclcclcclcclcclcclcclcclcclcclcclcclcclcclcclcclcclcclcclcclcclcclcclcclcclcclcclcclcclcclcclcclcclcclcclcclcclcclcclcclcclcclcclcclcclcclcclcclcclcclcclcclcclcclcclcclcclcclcclcclcclcclcclcclcclcclcclcclcclcclcclcclcclcclcclcclcclcclcclcclcclcclcclcclcclcclcclcclcclcclcclcclcclcclcclcclcclcclcclcclcclcclcclcclcclcclcclcclcclcclcclcclcclcclcclcclcclcclcclcclcclcclcclcclcclcclcclcclcclcclcclcclcclcclcclcclcclcclcclcclcclcclcclcclcclcclcclcclcclcclcclcclcclcclcclcclcclcclcclcclcclcclcclcclcclcclcclcclcclcclcclcclcclcclcclcclcclcclcclcclcclcclcclcclcclcclcclcclcclcclcclcclcclcclcclcclcclcclcclcclcclcclcclcclcclcclcclcclcclcclcclcclcclcclcclcclcclcclcclcclcclcclcclcclcclcclcclcclcclcclcclcclcclcclcclcclcclcclcclcclcclcclcclcclcclcclcclcclcclcclcclcclcclcclcclcclcclcclcclcclcclcclcclcclcclcclcclcclcclcclcclcclcclcclcclcclcclcclcclcclcclcclcclcclcclcclcclcclcclcclcclcclcclcclcclcclcclcclcclcclcclcclcclcclcclcclcclcclcclcclcclcclcclcclcclcclcclcclcclcclcclcclcclcclcclcclcclcclcclcclcclcclcclcclcclcclcclcclcclcclcclcclcclcclcclcclcclcclcclcclcclcclcclcclcclcclcclcclcclcclcclcclcclcclcclcclcclcclcclcclcclcclcclcclcclcclcclcclcclcclcclcclcclcclcclcclcclcclcclcclcclcclcclcclcclcclcclcclcclcclcclcclcclcclcclcclcclcclcclcclcclcclcclcclcclcclcclcclcclcclcclcclcclcclcclcclcclcclcclcclcclcclcclcclcclcclcclcclcclcclcclcclcclcclcclcclcclcclcclcclcclcclcclcclcclcclcclcclcclcclcclcclcclcclcclcclcclcclcclcclcclcclcclcclcclcclcclcclcclcclcclcclcclcclcclcclcclcclcclcclcclcclcclcclcclcclcclcclcclcclcclcclcclcclcclcclcclcclcclcclcclcclcclcclcclcclcclcclcclcclcclcclcclcclcclcclcclcclcclcclcclcclcclcclcclcclcclcclcclcclcclcclcclcclcclcclcclcclcclcclcclcclcclcclcclcclcclcclcclcclcclcclcclcclcclcclcclcclcclcclcclcclcclcclcclcclcclcclcclcclcclcclcclcclcclcclcclcclcclcclcclcclcclcclcclcclcclcclcclcclcclcclcclcclcclcclcclcclcclcclcclcclcclcclcclcclcclcclcclcclcclcclcclcclcclcclcclcclcclcclcclcclcclcclcclcclcclcclcclcclcclcclcclcclcclcclcclcclcclcclcclcclcclcclcclcclcclcclcclcclcclcclcclcclcclcclcclcclcclcclcclcclcclcclcclcclcclcclcclcclcclcclcclcclcclcclcclcclcclcclcclcclcclcclcclcclcclcclcclcclcclcclcclcclcclcclcclcclcclcclcclcclcclcclcclcclcclcclcclcclcclcclcclcclcclcclcclcclcclcclcclcclcclcclcclcclcclcclcclcclcclcclcclcclcclcclcclcclcclcclcclcclcclcclcclcclcclcclcclcclcclcclcclcclcclcclcclcclcclcclcclcclcclcclcclcclcclcclcclcclcclcclcclcclcclcclcclcclcclcclcclcclcclcclcclcclcclcclcclcclcclcclcclcclcclcclcclcclcclcclcclcclcclcclcclcclcclcclcclcclcclcclcclcclcclcclcclcclcclcclcclcclcclcclcclcclcclcclcclcclcclcclcclcclcclcclcclcclcclcclcclcclcclcclcclcclcclcclcclcclcclcclcclcclcclcclcclcclcclcclcclcclcclcclcclcclcclcclcclcclcclcclcclcclcclcclcclcclcclcclcclcclcclcclcclcclcclcclcclcclcclcclcclcclcclcclcclcclcclcclcclcclcclcclcclcclcclcclcclcclcclcclcclcclcclcclcclcclcclcclcclcclcclcclcclcclcclcclcclcclcclcclcclcclcclcclcclcclcclcclcclcclcclcclcclcclcclcclcclcclcclcclcclcclcclcclcclcclcclcclcclcclcclcclcclcclcclcclcclcclcclcclcclcclcclcclcclcclcclcclcclcclcclcclcclcclcclcclcclcclcclcclcclcclcclcclcclcclcclcclcclcclcclcclcclcclcclcclcclcclcclcclcclcclcclcclcclcclcclcclcclcclcclcclcclcclcclcclcclcclcclcclcclcclcclcclcclcclcclcclcclcclcclcclcclcclcclcclcclcclcclcclcclcclcclcclcclcclcclcclcclcclcclcclcclcclcclcclcclcclcclcclcclcclcclcclcclcclcclcclcclcclcclcclcclcclcclcclcclcclcclcclcclcclcclcclcclcclcclcclcclcclcclcclcclcclcclcclcclcclcclcclcclcclcclcclcclcclcclcclcclcclcclcclcclcclcclcclcclcclcclcclcclcclcclcclcclcclcclcclcclcclcclcclcclcclcclcclcclcclcclcclcclcclcclcclcclcclcclcclcclcclcclcclcclcclcclcclcclcclcclcclcclcclcclcclcclcclcclcclcclcclcclc

\begin{verbatim}
save('ct22_c1.mat', 'ct22_c1'); % 将该子弹第一个次棱转换后的矩阵进行保存
clear all

%%% 
c2c2c2c2c2c2c2c2c2c2c2c2c2c2c2c2c2c2c2c2c2c2c2c2c2c2c2c2c2c2c2c2c2c2c2c2c2c2c2c2c2c2c2c2c2c2c2c2c2c2c2c2c2c2c2c2c2c2c2c2c2c2c2c2c2c2c2c2c2c2c2c2c2c2c2c2c2c2c2c2c2c2c2c2c2c2c2c2c2c2c2c2c2c2c2c2c2c2c2c2c2c2c2c2c2c2c2c2c2c2c2c2c2c2c2c2c2c2c2c2c2c2c2c2c2c2c2c2c2c2c2c2c2c2c2c2c2c2c2c2c2c2c2c2c2c2c2c2c2c2c2c2c2c2c2c2c2c2c2c2c2c2c2c2c2c2c2c2c2c2c2c2c2c2c2c2c2c2c2c2c2c2c2c2c2c2c2c2c2c2c2c2c2c2c2c2c2c2c2c2c2c2c2c2c2c2c2c2c2c2c2c2c2c2c2c2c2c2c2c2c2c2c2c2c2c2c2c2c2c2c2c2c2c2c2c2c2c2c2c2c2c2c2c2c2c2c2c2c2c2c2c2c2c2c2c2c2c2c2c2c2c2c2c2c2c2c2c2c2c2c2c2c2c2c2c2c2c2c2c2c2c2c2c2c2c2c2c2c2c2c2c2c2c2c2c2c2c2c2c2c2c2c2c2c2c2c2c2c2c2c2c2c2c2c2c2c2c2c2c2c2c2c2c2c2c2c2c2c2c2c2c2c2c2c2c2c2c2c2c2c2c2c2c2c2c2c2c2c2c2c2c2c2c2c2c2c2c2c2c2c2c2c2c2c2c2c2c2c2c2c2c2c2c2c2c2c2c2c2c2c2c2c2c2c2c2c2c2c2c2c2c2c2c2c2c2c2c2c2c2c2c2c2c2c2c2c2c2c2c2c2c2c2c2c2c2c2c2c2c2c2c2c2c2c2c2c2c2c2c2c2c2c2c2c2c2c2c2c2c2c2c2c2c2c2c2c2c2c2c2c2c2c2c2c2c2c2c2c2c2c2c2c2c2c2c2c2c2c2c2c2c2c2c2c2c2c2c2c2c2c2c2c2c2c2c2c2c2c2c2c2c2c2c2c2c2c2c2c2c2c2c2c2c2c2c2c2c2c2c2c2c2c2c2c2c2c2c2c2c2c2c2c2c2c2c2c2c2c2c2c2c2c2c2c2c2c2c2c2c2c2c2c2c2c2c2c2c2c2c2c2c2c2c2c2c2c2c2c2c2c2c2c2c2c2c2c2c2c2c2c2c2c2c2c2c2c2c2c2c2c2c2c2c2c2c2c2c2c2c2c2c2c2c2c2c2c2c2c2c2c2c2c2c2c2c2c2c2c2c2c2c2c2c2c2c2c2c2c2c2c2c2c2c2c2c2c2c2c2c2c2c2c2c2c2c2c2c2c2c2c2c2c2c2c2c2c2c2c2c2c2c2c2c2c2c2c2c2c2c2c2c2c2c2c2c2c2c2c2c2c2c2c2c2c2c2c2c2c2c2c2c2c2c2c2c2c2c2c2c2c2c2c2c2c2c2c2c2c2c2c2c2c2c2c2c2c2c2c2c2c2c2c2c2c2c2c2c2c2c2c2c2c2c2c2c2c2c2c2c2c2c2c2c2c2c2c2c2c2c2c2c2c2c2c2c2c2c2c2c2c2c2c2c2c2c2c2c2c2c2c2c2c2c2c2c2c2c2c2c2c2c2c2c2c2c2c2c2c2c2c2c2c2c2c2c2c2c2c2c2c2c2c2c2c2c2c2c2c2c2c2c2c2c2c2c2c2c2c2c2c2c2c2c2c2c2c2c2c2c2c2c2c2c2c2c2c2c2c2c2c2c2c2c2c2c2c2c2c2c2c2c2c2c2c2c2c2c2c2c2c2c2c2c2c2c2c2c2c2c2c2c2c2c2c2c2c2c2c2c2c2c2c2c2c2c2c2c2c2c2c2c2c2c2c2c2c2c2c2c2c2c2c2c2c2c2c2c2c2c2c2c2c2c2c2c2c2c2c2c2c2c2c2c2c2c2c2c2c2c2c2c2c2c2c2c2c2c2c2c2c2c2c2c2c2c2c2c2c2c2c2c2c2c2c2c2c2c2c2c2c2c2c2c2c2c2c2c2c2c2c2c2c2c2c2c2c2c2c2c2c2c2c2c2c2c2c2c2c2c2c2c2c2c2c2c2c2c2c2c2c2c2c2c2c2c2c2c2c2c2c2c2c2c2c2c2c2c2c2c2c2c2c2c2c2c2c2c2c2c2c2c2c2c2c2c2c2c2c2c2c2c2c2c2c2c2c2c2c2c2c2c2c2c2c2c2c2c2c2c2c2c2c2c2c2c2c2c2c2c2c2c2c2c2c2c2c2c2c2c2c2c2c2c2c2c2c2c2c2c2c2c2c2c2c2c2c2c2c2c2c2c2c2c2c2c2c2c2c2c2c2c2c2c2c2c2c2c2c2c2c2c2c2c2c2c2c2c2c2c2c2c2c2c2c2c2c2c2c2c2c2c2c2c2c2c2c2c2c2c2c2c2c2c2c2c2c2c2c2c2c2c2c2c2c2c2c2c2c2c2c2c2c2c2c2c2c2c2c2c2c2c2c2c2c2c2c2c2c2c2c2c2c2c2c2c2c2c2c2c2c2c2c2c2c2c2c2c2c2c2c2c2c2c2c2c2c2c2c2c2c2c2c2c2c2c2c2c2c2c2c2c2c2c2c2c2c2c2c2c2c2c2c2c2c2c2c2c2c2c2c2c2c2c2c2c2c2c2c2c2c2c2c2c2c2c2c2c2c2c2c2c2c2c2c2c2c2c2c2c2c2c2c2c2c2c2c2c2c2c2c2c2c2c2c2c2c2c2c2c2c2c2c2c2c2c2c2c2c2c2c2c2c2c2c2c2c2c2c2c2c2c2c2c2c2c2c2c2c2c2c2c2c2c2c2c2c2c2c2c2c2c2c2c2c2c2c2c2c2c2c2c2c2c2c2c2c2c2c2c2c2c2c2c2c2c2c2c2c2c2c2c2c2c2c2c2c2c2c2c2c2c2c2c2c2c2c2c2c2c2c2c2c2c2c
%%% 2c2c2c2c2c2c2c2c2c2c2
load t22.mat % 载入第22个子弹的数据 下面计算截取的第一个次棱的矩阵转换坐标后的矩阵下面的代码同上省略。
\end{verbatim}

\subsection*{8.1.3 将坐标转换后的数据汇总}
代码名称:collect_data.m

代码内容:

\begin{verbatim}
% 把原始数据处理成一个矩阵存放。
% 每一个500*2000块存放一个子弹的转换坐标后的数据
% 从上到下依次为子弹1-22 ct1-22.mat中存放的是转换坐标后1-22颗子弹转换坐标后的数据

load ct1.mat % 依次载入数据
load ct2.mat
load ct3.mat
load ct4.mat
load ct5.mat
load ct6.mat
load ct7.mat
load ct8.mat
load ct9.mat
load ct10.mat
load ct11.mat
load ct12.mat
load ct13.mat
load ct14.mat
load ct15.mat
load ct16.mat
load ct17.mat
load ct18.mat
load ct19.mat
load ct20.mat
load ct21.mat
load ct22.mat
%% 数据汇总成一个矩阵
\end{verbatim}

\begin{verbatim}
all_data = [ct1_c1(201:600, 1:500) ct1_c2(201:600, 1:500)
           ct1_c3(201:600, 1:500) ct1_c4(201:600, 1:500);
           ct2_c1(201:600, 1:500) ct2_c2(201:600, 1:500) ct2_c3(201:600, 1:500)
           ct2_c4(201:600, 1:500);
           ct3_c1(201:600, 1:500) ct3_c2(201:600, 1:500) ct3_c3(201:600, 1:500)
           ct3_c4(201:600, 1:500);
           ct4_c1(201:600, 1:500) ct4_c2(201:600, 1:500) ct4_c3(201:600, 1:500)
           ct4_c4(201:600, 1:500);
           ct5_c1(201:600, 1:500) ct5_c2(201:600, 1:500) ct5_c3(201:600, 1:500)
           ct5_c4(201:600, 1:500);
           ct6_c1(201:600, 1:500) ct6_c2(201:600, 1:500) ct6_c3(201:600, 1:500)
           ct6_c4(201:600, 1:500);
           ct7_c1(201:600, 1:500) ct7_c2(201:600, 1:500) ct7_c3(201:600, 1:500)
           ct7_c4(201:600, 1:500);
           ct8_c1(201:600, 1:500) ct8_c2(201:600, 1:500) ct8_c3(201:600, 1:500)
           ct8_c4(201:600, 1:500);
           ct9_c1(201:600, 1:500) ct9_c2(201:600, 1:500) ct9_c3(201:600, 1:500)
           ct9_c4(201:600, 1:500);
           ct10_c1(201:600, 1:500) ct10_c2(201:600, 1:500) ct10_c3(201:600, 1:500)
           ct10_c4(201:600, 1:500);
           ct11_c1(201:600, 1:500) ct11_c2(201:600, 1:500) ct11_c3(201:600, 1:500)
           ct11_c4(201:600, 1:500);
           ct12_c1(201:600, 1:500) ct12_c2(201:600, 1:500) ct12_c3(201:600, 1:500)
           ct12_c4(201:600, 1:500);
           ct13_c1(201:600, 1:500) ct13_c2(201:600, 1:500) ct13_c3(201:600, 1:500)
           ct13_c4(201:600, 1:500);
           ct14_c1(201:600, 1:500) ct14_c2(201:600, 1:500) ct14_c3(201:600, 1:500)
           ct14_c4(201:600, 1:500);
           ct15_c1(201:600, 1:500) ct15_c2(201:600, 1:500) ct15_c3(201:600, 1:500)
           ct15_c4(201:600, 1:500);
           ct16_c1(201:600, 1:500) ct16_c2(201:600, 1:500) ct16_c3(201:600, 1:500)
           ct16_c4(201:600, 1:500);
           ct17_c1(201:600, 1:500) ct17_c2(201:600, 1:500) ct17_c3(201:600, 1:500)
           ct17_c4(201:600, 1:500);
           ct18_c1(201:600, 1:500) ct18_c2(201:600, 1:500) ct18_c3(201:600, 1:500)
           ct18_c4(201:600, 1:500);
           ct19_c1(201:600, 1:500) ct19_c2(201:600, 1:500) ct19_c3(201:600, 1:500)
           ct19_c4(201:600, 1:500);
           ct20_c1(201:600, 1:500) ct20_c2(201:600, 1:500) ct20_c3(201:600, 1:500)
           ct20_c4(201:600, 1:500);
           ct21_c1(201:600, 1:500) ct21_c2(201:600, 1:500) ct21_c3(201:600, 1:500)
           ct21_c4(201:600, 1:500);
           ct22_c1(201:600, 1:500) ct22_c2(201:600, 1:500) ct22_c3(201:600, 1:500)
           ct22_c4(201:600, 1:500);];
\end{verbatim}

\begin{verbatim}
%% 保存矩阵
save all_data.mat all_data

8.1.4 匹配结果计算
代码名称:
代码: compare_all_data.m

% 计算每一个子弹与其他所有子弹的距离值
% 得出的结果在变量x中存放。
function x=compare_all_data()
load all_data     % 载入数据
for index1=1:22    % 对每一个子弹分别求其对其他子弹的距离
    for index2=1:22
        a=index1;
        b=index2;
        data_duiying=zeros(1,4);
        for i=1:4    % 每一个对应 都计算相应的结果
            data_cileng=zeros(1,4); %0
            for j=1:4    % 对每一个刺棱求其距离值
                data_pingyi=zeros(1,5);
                for k=-10:5:10    % 每一个平移求其最小的值 减小题目中平移误差的影响
                    data_pingyi(1,(k+10)/5+1)=sum(sum(abs(...
                    all_data_circle((a-1)*500+100:(a-1)*500+200,(j-1)*500+16+k:(j-1)*500+300+15+k)...
                    -all_data_circle((b-1)*500+100:(b-1)*500+200,mod(i+j-2,4)*500+16:mod(i+j-2,4)*500+300+15)...
                    )));
                end
                data_cileng(1,j)=min(data_pingyi);
            end
            data_duiying(1,i)=sum(data_cileng)-max(data_cileng);% 去除四个次棱距离中最大的一个
        end
        x(index1,index2)=min(data_duiying);% 取四个对应结果中最小的一个为连个子弹的距离
    end
end

8.2 “绝对差分和”的计算
8.2.1 原始数据的预处理
处理过程同8.1.1。
\end{verbatim}

\section*{8.2.1 绝对差分和的计算}

代码名称:all\_data\_original\_diff\_sum.m

代码:

\begin{verbatim}
load all_data_original.mat %501*4
%载入数据
%556*12
all_data_original_diff_sum=zeros(12*556,4);
all_data_original_diff=abs(all_data_original(:,2:2004)-all_data_original(:,1:2003));
for i=1:4

all_data_original_diff_sum(:,1)'=sum(all_data_original(:,(i-1)*501+1:(i-1)*501+500));
end
% 保存绝对差分和到相应的文件中
save all_data_original_diff_sum.mat all_data_original_diff_sum
\end{verbatim}