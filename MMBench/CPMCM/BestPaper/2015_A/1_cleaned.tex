\documentclass{article}
\usepackage{amsmath}
\usepackage{amssymb}

\title{水面舰艇编队防空和信息化战争评估模型}
\author{}
\date{}

\begin{document}

\maketitle

\begin{center}
\textbf{第十二届“中关村青联杯”全国研究生数学建模竞赛}
\end{center}

\begin{center}
\textbf{摘 \quad 要}
\end{center}

防空队形是影响水面舰艇编队防空能力的重要因素,以我海军一水面舰艇编队为例,舰队由一艘指挥舰和四艘护卫舰组成,各舰武备充足,采用雷达探测手段,且通过数据链共享战场态势信息。以我指挥舰为原点的 20 度至 220 度扇面内,等可能的有导弹来袭,且来袭导弹速度射程等属性已知。现由指挥舰统一指挥各舰进行防御,各舰进行拦截的准备时间(含发射)均为 7 秒,拦截的路径为最快相遇。各舰在一次拦截任务中,不能接受对另一批来袭导弹的拦截任务,只有在本次拦截任务完成后,才可以执行下一个拦截任务。指挥舰对拦截任务的分配原则是,对每批来袭导弹只使用一艘舰进行拦截,且无论该次拦截成功与否,不对该批来袭导弹进行第二次拦截。根据题设构建数学模型,按要求解算出规定的五个问题。

问题一,在未发现敌方目标时,设计编队最佳队形,应对所有可能的突发事件,最大可能保护指挥舰。首先考察了几种最终常见的舰艇编队,然后简要论证了常规队形在本题条件中的不适用性,最后根据对题设条件的简化,确定舰队的最佳防守阵型为半环形阵型,如图 9 所示,给出编队中各护卫舰与指挥舰之间的距离(km)和角度分别为:$(42.8901, 45^\circ)$、$(42.8901, 95^\circ)$、$(42.8901, 145^\circ)$、$(42.8901, 195^\circ)$。

问题二,保持问题一中的舰队阵型不变,考察该阵型在最危险方向上抗饱和打击能力。首先在半环形阵型中确定了舰队的防守薄弱方向,该方向定为最危险方向,将扇区抗敌方饱和攻击能力简化为该方向上的抗攻击能力;其次根据来袭导弹和防空导弹的距离、航速、射程,建立“最佳防御纵深模型”,利用排队论思想,合理安排防空时间和顺序,使得防空导弹在舰队的安全防守范围内尽可能多的拦截来袭导弹;最后解算模型得出我舰队最大抗饱和能力为 9 批。

问题三,为舰队配备预警机,极大的扩展了舰队的目标探测距离,考察原舰队阵型在此条件下的最大抗饱和攻击能力的变化,仍然使用最佳防御纵深模型进行建模和解算,最终求得在配备预警机的情况下,我舰队最大抗饱和能力为 18 批,防御能力提升了 100%。

问题四,根据所给的雷达探测信息,包括秒数、目标位置经纬度、目标高度、各目标雷达反射面积以及用于判断目标意图的知识和规则样本,对现有的 12 批待测目标信息进行分类研究,判断这 12 批目标的属性和意图。我们首先对所给数据进行了预处理,剔除错误信息;再由修正的数据进行经纬度和距离、方位角之间的换算,生成可用的目标属性数据;建立三层 BP 神经网络对原始已知数据进行训练,再由成熟网络对待测目标属性进行分类计算,最终确定 12 批次待测目标的属性及其意图,分别为:意图为攻击的目标有四个,其 ID 分别为 (41006839、41006842、41006851、41006885);意图为掩护的目标有三个,分别为 (41006830、41006860、41006872);意图为监听的目标有一个 41006891;其他目标意图为“其它”,为我舰队进行下一步防御措施提供必要的信息。

问题五,信息化条件下作战对传统的作战评估模型和作战结果已经产生重要的甚至某种程度上是决定性的影响,以海湾战争为例,采用包括用兰彻斯特战争模型的传统战争理论和战争评估模型进行战争进程预测的结果与战争的实际结果大相径庭,因此需要建立宏观的战略级信息化战争评估模型,从一般意义上反映信息化战争的规律和特点,利用模型分析研究信息系统、指挥对抗、信息优势、信息系统稳定性,以及其它信息化条件下作战致胜因素的相互关系和影响。并通过信息化战争的经典案例,对模型加以验证。针对该问题,我们首先研究了传统兰彻斯特方程,分析其在现代信息化战争中的劣势;其次构建一种符合现代战争要求的优化战争评估模型---非对称信息战争评估模型;最终通过海湾战争实例演算,验证了我们所构建的非对称信息战争评估模型能够较好的预测,作战双方的战争损耗比率,符合现代化信息战争预估和评价的基本要求。

关键词:舰船编队阵型;抗饱和攻击能力;神经网络分类;兰彻斯特模型;战争进程评估

\section{目录}

\begin{itemize}
    \item[] 1 问题重述 \dotfill 1
    \begin{itemize}
        \item[] 1.1 问题背景 \dotfill 1
        \item[] 1.2 问题分析 \dotfill 1
    \end{itemize}
    \item[] 2 模型假设 \dotfill 2
    \item[] 3 符号说明 \dotfill 2
    \item[] 4 问题一模型建立与求解 \dotfill 3
    \begin{itemize}
        \item[] 4.1 问题描述及分析 \dotfill 3
        \item[] 4.2 阵型初步分析 \dotfill 3
        \item[] 4.3 模型建立 \dotfill 5
        \item[] 4.4 模型解算 \dotfill 6
    \end{itemize}
    \item[] 5 问题二模型建立与求解 \dotfill 7
    \begin{itemize}
        \item[] 5.1 问题描述及分析 \dotfill 7
        \item[] 5.2 模型建立 \dotfill 8
        \item[] 5.3 模型评价 \dotfill 12
    \end{itemize}
    \item[] 6 问题三模型建立与求解 \dotfill 12
    \begin{itemize}
        \item[] 6.1 问题描述及分析 \dotfill 12
        \item[] 6.2 模型建立 \dotfill 12
        \item[] 6.3 模型解算及分析 \dotfill 15
        \item[] 6.4 模型评价 \dotfill 16
    \end{itemize}
    \item[] 7 问题四模型建立与求解 \dotfill 16
    \begin{itemize}
        \item[] 7.1 问题描述及分析 \dotfill 16
        \item[] 7.2 模型建立 \dotfill 16
        \item[] 7.3 基于BP神经网络的目标意图分类模型 \dotfill 22
        \item[] 7.4 模型计算 \dotfill 22
        \item[] 7.5 模型评价 \dotfill 23
    \end{itemize}
    \item[] 8 问题五模型建立与求解 \dotfill 24
    \begin{itemize}
        \item[] 8.1 问题描述及分析 \dotfill 24
        \item[] 8.2 模型建立 \dotfill 25
        \item[] 8.3 模型的验证 \dotfill 29
        \item[] 8.4 模型评价 \dotfill 30
    \end{itemize}
    \item[] 9 总结 \dotfill 30
    \item[] 参考文献 \dotfill 30
    \item[] 附录 \dotfill 31
\end{itemize}

\section{问题重述}

\subsection{问题背景}

我海军由 1 艘导弹驱逐舰和 4 艘导弹护卫舰组成水面舰艇编队在我南海某开阔海域巡逻,其中导弹驱逐舰为指挥舰,重要性最大。某一时刻 \( t \) 我指挥舰位置位于北纬 15 度 41 分 7 秒,东经 112 度 42 分 10 秒,编队航向 200 度(以正北为 0 度,顺时针方向),航速 16 节(即每小时 16 海里)。编队各舰上防空导弹型号相同,数量充足,水平最小射程为 10 千米,最大射程为 80 千米,高度影响不必考虑(因敌方导弹超低空来袭),平均速度 2.4 马赫(即音速 340 米/秒的 2.4 倍)。编队仅依靠自身雷达对空中目标进行探测,但有数据链,所以编队中任意一艘舰发现目标,其余舰都可以共享信息,并由指挥舰统一指挥各舰进行防御。

\subsection{问题分析}

以我指挥舰为原点的 20 度至 220 度扇面内,等可能的有导弹来袭。来袭导弹的飞行速度 0.9 马赫,射程 230 千米,航程近似为直线,一般在离目标 30 千米时来袭导弹启动末制导雷达,其探测距离为 30 千米,搜索扇面为 30 度(即来袭导弹飞行方向向左和向右各 15 度的扇面内,若指挥舰在扇形内,则认为来袭导弹自动捕捉的目标就是指挥舰),且具有“二次捕捉”能力(即第一个目标丢失后可继续向前飞行,假设来袭导弹接近舰艇时受到电子干扰丢失目标的概率为 85\%,并搜索和攻击下一个目标,“二次捕捉”的范围是从第一个目标估计位置算起,向前飞行 10 千米,若仍然没有找到目标,则自动坠海)。每批来袭导弹的数量小于等于 4 枚(即由同一架或在一起的一批飞机几乎同时发射,攻击目标和导弹航向都相同的导弹称为一批)。

由于来袭导弹一般采用超低空飞行和地球曲率的原因,各舰发现来袭导弹的随机变量都服从均匀分布,均匀分布的范围是导弹与该舰之间距离在 20-30 千米。可以根据发现来袭导弹时的航向航速推算其不同时刻的位置,故不考虑雷达发现目标后可能的目标“丢失”。编队发现来袭导弹时由指挥舰统一指挥编队内任一舰发射防空导弹进行拦截,进行拦截的准备时间(含发射)均为 7 秒,拦截的路径为最快相遇。各舰在一次拦截任务中,不能接受对另一批来袭导弹的拦截任务,只有在本次拦截任务完成后,才可以执行下一个拦截任务。指挥舰对拦截任务的分配原则是,对每批来袭导弹只使用一艘舰进行拦截,且无论该次拦截成功与否,不对该批来袭导弹进行第二次拦截。不考虑每次拦截使用的防空导弹数量。

需要通过建立数学模型,解决以下几个问题:

问题一:在未发现敌方目标时,设计编队最佳队形,即设计各护卫舰相对指挥舰的方位和距离,以应对所有可能的突发事件,最大可能保护指挥舰,免遭敌导弹攻击。

问题二:当不考虑使用电子干扰和近程火炮(包括密集阵火炮)等拦截手段,仅使用防空导弹拦截来袭导弹,考察问题一中舰艇编队防御敌来袭导弹对我指挥舰攻击时的抗饱和攻击能力,即当指挥舰遭遇多批次导弹几乎同时攻击时,在最危险的方向上,编队能够拦截来袭导弹的最大批数。

问题三:如果编队得到空中预警机的信息支援,对距离我指挥舰 200 千米内的所有来袭导弹都可以准确预警,即通报来袭导弹的位置与速度矢量,考察编队在仍然保持上述设计的队形,且仅使用防空导弹拦截敌来袭导弹的情况下,我指挥舰攻击时的抗饱和攻击能力的提高情况。

问题四:预警机发现前方有 12 批可疑的空中目标,从 \( t \) 时刻起,根据雷达测得的目标位置信息,包括:从 1970 年 1 月 1 日 0 时起到某一时刻的秒数、目标位置经纬度、目标高度、各目标雷达反射面积以及用于判断空中目标的意图的知识和规则的样本。分析识别空中各目标可能的意图。

问题五:如果我方的预警机和水面舰艇编队的雷达和通信系统遭到敌方强烈的电子干扰,无法发现目标,也无法传递信息,这时,后果将是极其严重的,我编队防空导弹的拦截效能几乎降低到零。此类状况的发生在告诫人们,信息化条件下作战对传统的作战评估模型和作战结果已经产生重要的甚至某种程度上是决定性的影响。然而在海湾战争的“沙漠风暴”行动前,采用包括用兰彻斯特战争模型的传统战争理论和战争评估模型进行战争进程预测的结果与战争的实际结果大相径庭。那么信息化战争的结果应该用何种模型来分析或预测已成为一个极具挑战性,又十分有意义的课题。尝试建立宏观的战略级信息化战争评估模型,从一般意义上反映信息化战争的规律和特点,利用模型分析研究信息系统、指挥对抗、信息优势、信息系统稳定性,以及其它信息化条件下作战致胜因素的相互关系和影响。并通过信息化战争的经典案例,例如著名的海湾战争,对模型加以验证。

\section{模型假设}

(1)假设舰艇编队各舰以固定速度航行,不考虑其加速、减速或转弯等机动行为;

(2)假设编队各舰之间数据链畅通,数据可靠,不存在数据链失效,各舰船通讯受阻等情况,预警机探测数据真实有效,不存在数据包丢失、数据延迟等极端情况;

(3)假设各舰防空导弹信号属性一致,弹药充足,即拦截过程中防御效能不受弹药数量影响;

(4)假设敌方导弹超低空来袭,不考虑导弹飞行的高度影响,并且导弹飞行过程为平稳过程,飞行中不考虑风速、磁场等对弹体的影响;

(5)仅以我指挥舰为原点的 $20^{\circ}-220^{\circ}$ 扇面内等可能有导弹来袭,其他方位均不考虑;

(6)假设同一批来袭导弹为同时发射,且每批不超过 4 枚;

(7)假设舰载探测雷达的探测系数最大,即能够有效发现来袭目标,并不考虑雷达虚警或丢失目标的情况;

(8)规定针对每批来袭导弹仅有一艘护卫舰拦截,且每批仅有一次拦截,在本次拦截中,不参与其他拦截任务;

(9)各舰雷达探测方位是全空域的,即在水平面内可探测 $360^{\circ}$ 范围。

\section{符号说明}

\begin{tabular}{|c|l|}
\hline $\theta_{k}$ & 第 $k$ 艘护卫舰相对驱逐舰的方位角 \\
\hline $R_{k}$ & 第 $k$ 艘护卫舰与驱逐舰之间的距离 \\
\hline $R_{d k}$ & 护卫舰防御纵深 \\
\hline $\beta$ & 防卫范围角度 \\
\hline $X_{a i}$ & 来袭导弹飞行距离 \\
\hline $X_{b i}$ & 防空导弹飞行距离 \\
\hline
\end{tabular}

\begin{tabular}{|c|l|}
\hline
$I_{i}$ & 舰艇 $i$ 的最大拦截数量 \\
\hline
$V_{m}$ & 来袭导弹飞行速度 \\
\hline
$V_{l}$ & 拦截导弹飞行速度 \\
\hline
$t_{m}$ & 来袭导弹飞行时间 \\
\hline
$t_{l}$ & 拦截导弹飞行时间 \\
\hline
$\lambda$ & 舰艇被发现的概率 \\
\hline
$R_{a}$ & 雷达探测距离 \\
\hline
$d_{k}$ & 来袭导弹进入舰艇打击范围时舰艇 $k$ 与来袭导弹的距离 \\
\hline
$\alpha_{k}$ & 来袭导弹进入舰艇打击范围时舰艇 $k$ 与来袭导弹形成的方位角 \\
\hline
\end{tabular}

注:在问题五中模型涉及到的众多参量符号未一一列举在上述表格中,而是在每个计算式下方进行解释说明。

\section{问题一模型建立与求解}

\subsection{问题描述及分析}

我水面舰艇编队由 1 艘导弹驱逐舰和 4 艘导弹护卫舰组成,其中导弹驱逐舰为指挥舰,重要性最大,指挥舰位置为北纬 15 度 41 分 7 秒,东经 112 度 42 分 10 秒,航向 200 度,航速 16 节。以我指挥舰为原点的 20 度至 220 度扇面内,等可能的有导弹来袭。来袭导弹的飞行速度 0.9 马赫,射程 230 千米,航程近似为直线,一般在离目标 30 千米时来袭导弹启动末制导雷达,其探测距离为 30 千米,搜索扇面为 30 度,且具有“二次捕捉”能力。各舰上防空导弹水平射程范围 10 千米-80 千米,平均速度 2.4 马赫。编队依靠自身雷达对空中目标进行探测,且由数据链与其余各舰都可以共享信息,由指挥舰统一指挥各舰进行防御。各舰进行拦截的准备时间(含发射)均为 7 秒,拦截的路径为最快相遇。各舰在一次拦截任务中,不能接受对另一批来袭导弹的拦截任务,只有在本次拦截任务完成后,才可以执行下一个拦截任务。指挥舰对拦截任务的分配原则是,对每批来袭导弹只使用一艘舰进行拦截,且无论该次拦截成功与否,不对该批来袭导弹进行第二次拦截。

在上述条件设定下,敌我攻防范围及能力已知,需要根据舰队各舰防空能力设计保护指挥舰的最佳编队阵型,给出各护卫舰距离指挥舰的距离及方位,该阵型能够最大限度地拦截敌方导弹攻击。结合问题假设,我们考虑需要最大程度保护指挥舰,以应对各种突发问题,则应当重点解决两个问题:

(1)最大程度的防御意味着设计护卫舰编排方位角使得舰队防御范围最大,而又不能因防御距离过远而超出有效打击范围;

(2)为简化问题,最佳防御可理解为重点盯防编队中各护卫舰之间以及各护卫舰与指挥舰之间的防御边界最薄弱点或最防御薄弱方向。

\subsection{阵型初步分析}

水面舰队防空编队问题是海战运筹学中的热点问题 \cite{ref1, ref2},一般小规模舰队会采用人字形、菱形、梯形、环形、楔形、横队或纵队,因此我们首先对基本阵型采用枚举和排除法进行最初的合理化阵型的考虑。

通过简单的建模和讨论,我们首先分析了几种现实中常见的舰队阵形,其初步分析对比结果如下:

单边阵型:如横队、纵队、斜梯队(如图 1 所示)在与其队列平行的方向上具有较强的攻防能力,但是并不符合本题的扇形区域防御要求;

\begin{figure}[h]
    \centering
    \includegraphics[width=0.9\textwidth]{image1.png}
    \caption{单边阵型}
    \label{fig:1}
\end{figure}

双边阵型:如人字形和楔形(见图 2)对单边阵型进行了改良,基本满足扇形区域防御,但不可忽略的是,对于五舰艇规模来说,人字形和楔形无论在何方向上布阵,始终会让指挥舰暴露于阵型顶端或边缘,这样对区域防守来说是较为危险的;

\begin{figure}[h]
    \centering
    \includegraphics[width=0.9\textwidth]{image2.png}
    \caption{双边阵型}
    \label{fig:2}
\end{figure}

多边阵型:以四边形阵型为主,主要分析了菱形、其他四边形,如图 3 所示,在常规海战中,四边形是最常用的阵型,护卫舰布防于指挥舰的四周,攻可形成钳形攻势,守可全空域盯防,是一种合理化阵型 \cite{ref3,ref4,ref5},但针对本题而言,要在 $20^\circ$-$220^\circ$ 范围内达到最大防空效能,即要将所有战舰的防御功能发挥到最大,在此情况下采用四边形阵型则暴露出一种劣势,就是距离最危险方向的最远的护卫舰往往会处于“观战状态”,不能发挥防御功能,因此不适于本题所述情况;

\begin{figure}[h]
    \centering
    \includegraphics[width=0.9\textwidth]{image3.png}
    \caption{四边形阵型}
    \label{fig:3}
\end{figure}

环形阵型:环形阵型是以指挥舰为原点,护卫舰以弧形排列的形式,阵型为向外辐射式阵型,该阵型将指挥舰直接布防在防守圈中心,护卫舰沿防守圈边缘进行布防,在来袭导弹和主力舰之间形成防守屏障,有利于防范最危险方向来袭目标,因此我们初步设计采用符合 $20^{\circ}-220^{\circ}$$ 范围要求的半环形阵型,如图 4 所示。

\begin{figure}[h]
    \centering
    \includegraphics[width=0.8\textwidth]{image.png} % 替换为实际图片路径
    \caption{环形阵型}
    \label{fig:ring_array}
\end{figure}

\subsection{模型建立}

设护卫舰与驱逐舰间的距离为 $R_{i},(i=1,2,3,4)$,护卫舰相对于驱逐舰的方位角为 $\theta_{i},(i=1,2,3,4)$。根据模型假设,若要使得各个舰艇足以应对所有可能的突发事件,保护好指挥舰,使其尽可能免遭敌导弹攻击,需要使护卫舰雷达探测到的范围覆盖导弹可能来袭的方位,即防护范围角 $\beta$ 满足 $\beta \geq 200^{\circ}$。此外,护卫舰的防御纵深尽可能大,即护卫舰防御纵深 $R_{d}$ 满足:

\begin{align*}
S_{r} &= \sum_{i=1}^{n} W_{i} V_{i} = 76*1 + 3500*0.005 + 3000*0.005 + 3300*0.03 + 1900*0.02 \\
&\quad + 180*0.02 + 1600*0.005 = 257.1 \text{万} \\
S_{b} &= \sum_{i=1}^{n} W_{i} V_{i} = 120*1 + 5600*0.005 + 7000*0.005 + 793*0.03 \\
&\quad + 300*0.03 + 60*0.02 + 3000*0.005 = 231.99 \text{万} \\
V_{fr} &= 1; \quad V_{fb} = 1; \quad CEV_{r} = 1; \quad CEV_{b} = 1; \\
P_{Er} &= S_{r} * V_{fr} * CEV_{r} = 257.1 \text{万} \\
P_{Eb} &= S_{b} * V_{fr} * CEV_{b} = 231.99 \text{万}
\end{align*}

另外舰艇被发现的概率尽可能小,即 $\min \lambda$。

护卫舰的角度分布:由于导弹在扇形区域内为等可能来袭,且各个护卫舰以及主舰的防御能力均相同,且忽略舰队航速的影响,因此舰队的分布必然存在对称性。

护卫舰与驱逐舰距离的计算:由于受防空导弹射程的限制,因此有 $10\text{km} \leq R_{d} \leq 80\text{km}$。考虑到尽可能减小舰艇被发现及命中的概率,应当尽可能使舰艇之间散布距离大,使得来袭导弹在同一时刻无法锁定两艘舰艇,因此两艘舰艇之间距离满足 $d_{a} > l_{m}$,如图 5 所示。

其中,$l_{m}$ 为来袭导弹的探测横向间隔,$l=30\text{km}$ 为来袭导弹的探测范围,根据几何关系,$l_{m} = 2 \times 30 \times \sin 15^{\circ} = 15.53\text{km}$。

\begin{figure}[h]
    \centering
    \includegraphics[width=0.6\textwidth]{image1.png}
    \caption{两艘舰艇之间距离}
    \label{fig:ship_distance}
\end{figure}

我们取其中一艘护卫舰进行分析,如图 \ref{fig:ship_defense} 所示,根据模型假设,护卫舰雷达的可靠预警范围为 $20km$。故 $a=b=20km$,而在护卫舰的预警范围中,$a$ 方向上的防护纵深最大,$b$ 方向上的防护纵深最小,因此要使舰队能够预防各种突发情况,提高抗饱和打击能力,需要使

\begin{align*}
\frac{dP_{Ib}}{dT} &= -2k_{b}G_{b}P_{Eb}T_{b} = -2*0.1*0.1667*231.99*1 = -7.7345 \\
\frac{dP_{Ir}}{dT} &= -2k_{r}G_{r}P_{Er}T_{r} = -2*0.9*1*257.1*50 = -23139
\end{align*}

\begin{figure}[h]
    \centering
    \includegraphics[width=0.8\textwidth]{image2.png}
    \caption{单艘护卫舰防空范围}
    \label{fig:ship_defense}
\end{figure}

根据余弦定理,有

\begin{align*}
S_{lr} &= \sum_{i=1}^{n} W_i V_i = 0.119 * 1 + 2 * 0.02 = 0.1590 \text{万} \\
S_{lb} &= \sum_{i=1}^{n} W_i V_i = 27.5 * 1 + 3847 * 0.005 + 1450 * 0.005 \\
&\quad + 324 * 0.03 + 57 * 0.02 + 2917 * 0.005 \\
&= 79.4300 \text{万} \\
\frac{S_{lb}}{S_{lr}} &= 499.5597
\end{align*}

根据护卫舰分布的对称性,得到 $e=25^{\circ}$,因此建立如下非线性模型:

\begin{equation}
\begin{aligned}
\max R_{d} \\
s.t. \begin{cases}
\cos e = \frac{R_{d}^{2} + R_{1}^{2} - b^{2}}{2R_{d}R_{1}} \\
a = b = 20km \\
10km \leq R_{d} \leq 80km \\
e = 25^{\circ}
\end{cases}
\tag{4}
\end{aligned}
\end{equation}

\subsection{模型解算}

通过 matlab 2013a 对上述模型进行仿真验证,利用蒙特卡罗法求解得到舰队防守纵深 $R_{d}$ 与护卫舰距离 $R_{1}$ 布局的关系图,由图 7 可以看出,随着 $R_{1}$ 的增加,$R_{d}$ 先增大后减小,在 $R_{1}=42.8901km$ 处取得最大值,此时 $R_{d}=47.324km$。

\begin{figure}[h]
    \centering
    \includegraphics[width=\textwidth]{image1.png}
    \caption{舰队防守纵深与护卫舰距离布局的关系图}
\end{figure}

又根据对称性原则可知,$R_{1}=R_{2}=R_{3}=R_{4}=42.8901km$,$\theta_{1}=45^{\circ}, \theta_{2}=95^{\circ}, \theta_{3}=145^{\circ}, \theta_{4}=195^{\circ}$。

由此得到,舰队最佳布防阵型是以指挥舰为中心,护卫舰分别按$\theta_{1}=45^{\circ}, \theta_{2}=95^{\circ}, \theta_{3}=145^{\circ}, \theta_{4}=195^{\circ}$,$R_{1}=R_{2}=R_{3}=R_{4}=42.8901km$,均匀分布的半环形阵型,阵型结构如图 8 所示。

\begin{figure}[h]
    \centering
    \includegraphics[width=\textwidth]{image2.png}
    \caption{舰队最佳防守阵型}
\end{figure}

\section{问题二模型建立与求解}

\subsection{问题描述及分析}

在上一问的已有舰队阵型的基础上,仅使用防空导弹拦截来袭导弹,考察问题一中舰艇编队防御敌来袭导弹对我指挥舰攻击时的抗饱和攻击能力,即当指挥舰遭遇多批次导弹几乎同时攻击时,在最危险的方向上,编队能够拦截来袭导弹的最大批数。

因此,问题二的解答需要分为两步:

(1)分析问题一中设计的舰队阵形,分别考察各护卫舰之间探测覆盖边界以及护卫舰与指挥舰之间的防守边界,以确定整个舰队防守最弱方向,即所谓的最危险方向;

(2)根据最危险方向,计算编队能够拦截来袭导弹的最大批数。

(2) 根据来袭导弹和防空导弹的距离、航速、射程,利用排队思想,合理安排防空时间和顺序,使得防空导弹在距离舰队较远的地方,即舰队的安全防守范围内尽可能多的拦截来袭导弹。

\subsection{模型建立}

如图 9 所示为舰艇编队最大射程范围,Q1 为护卫舰 1 的最大射程范围,Q2 为护卫舰 2 的最大射程范围,Q3 为 3 号护卫舰的最大射程范围,Q4 为 4 号护卫舰的最大射程范围,Qz 为驱逐舰的最大射程范围。护卫舰的拦截能力受到防御纵深、拦截导弹射程以及方位角的影响,首先我们考虑最危险的情况,即雷达的探测区域 \( R_{a} \) 取 \(\min R_{a} = 20 \, \text{km}\)。拦截能力最薄弱的地方为防御纵深 \( R_{d} \) 最小的地方,即拦截区域的边界和两个护卫舰探测区域的对称线方向,如图 10 所示。

根据问题一的分析,我们得出了半环形舰队阵形的防守薄弱方向为图 10 中蓝色直线方向,该方向上需要舰队进行防御的战略纵深最小,即为最危险方向,因此,在下面的问题分析中,我们将问题简化为舰队在该方向上的抗饱和打击能力求解问题。

\begin{figure}[h]
    \centering
    \includegraphics[width=0.8\textwidth]{image1.png}
    \caption{舰队整体防空区域}
    \label{fig:9}
\end{figure}

\begin{figure}[h]
    \centering
    \includegraphics[width=0.8\textwidth]{image2.png}
    \caption{拦截区域边界及薄弱方向}
    \label{fig:10}
\end{figure}

若要实现最大防御能力,则每一艘护卫舰探测到来袭导弹之后,在其射程范围内不断打击。在舰艇之间不发生干扰的条件下(即任意一艘舰艇发射的拦截导弹不会经过其他舰艇附近 $10\mathrm{~km}$ 范围内),各个舰艇之间的拦截能力是相互独立的。舰艇拦截来袭导弹的流程图如图 11 所示。

\begin{figure}[h]
\centering
\includegraphics[width=0.8\textwidth]{image.png}
\caption{舰艇拦截来袭导弹的流程图}
\end{figure}

对于单个舰艇拦截导弹的示意图如图 12 所示,在 $A_{1}$ 点处来袭导弹到达护卫舰雷达探测范围之内,雷达开始预警,判断来袭导弹是否在拦截导弹射程之内,如果来袭导弹在拦截导弹射程之内则对导弹进行拦截。如果来袭导弹不在射程范围之内则等待。

其中有一个过渡地带,就是来袭导弹在雷达预警范围之内,但是还未到护卫舰拦截导弹的射程之内,但是如果此时计算来袭导弹的轨迹,发射拦截导弹进行拦截的话拦截导弹与来袭导弹相遇的地方在拦截导弹的射程之内,此时我们假设可以对来袭导弹进行拦截。在 $A_{2}, A_{3}, A_{4}$ 点处拦截到来袭导弹,当来袭导弹飞行至 $A_{n}$ 处时,由于到达驱逐舰的最小射程范围之内,再进行拦截有可能会误伤驱逐舰,因此不再对其进行拦截,达到最大拦截数量。

\begin{figure}[h]
    \centering
    \includegraphics[width=0.8\textwidth]{image1.png}
    \caption{单舰导弹拦截示意图}
    \label{fig:missile_interception}
\end{figure}

根据余弦定理有
\begin{equation}
\cos \alpha = \frac{X_{a1}^2 + d^2 - X_{b1}^2}{2X_{a1}d}
\tag{5}
\end{equation}

其中 $d_k$ 为舰艇 k 与来袭导弹的初始距离,$\alpha_k$ 为当来袭导弹进入护卫舰 k 的可打击区域内时,护卫舰 k 与来袭导弹形成的方位角,$\gamma$ 为护卫舰与驱逐舰的方位角,$X_{a1} = V_{m1}t_{m1}$ 为第一次拦截到来袭导弹时来袭导弹所飞行的距离,$X_{b1} = V_{l1}t_{l1}$ 为第一次拦截到来袭导弹时拦截导弹所飞行的距离;$t_{m1} = t_{l1} + 7$ 为导弹飞行时间。此外,需要保证拦截导弹不进入护卫舰 10km 半径的圆内,需满足 $R_{dk} - X_{ai} > d_{mk}$,$d_{mk}$ 为保证拦截导弹不进入其他舰艇 10km 半径的圆内的最小纵深。

\begin{enumerate}
    \item 对于护卫舰 1,如图 \ref{fig:frigate1} 所示,$d_{m1} = 10$,保证拦截导弹不能进入驱逐舰 10km 半径的圆内。
    \begin{figure}[h]
        \centering
        \includegraphics[width=0.8\textwidth]{image2.png}
        \caption{护卫舰 1 最小防御纵深图}
        \label{fig:frigate1}
    \end{figure}
    \item 对于护卫舰 2,如图 \ref{fig:frigate2} 所示,驱逐舰最小纵深 $d_{m2} = 10$
    \begin{figure}[h]
        \centering
        \includegraphics[width=0.8\textwidth]{image3.png}
        \caption{护卫舰 2 最小防御纵深}
        \label{fig:frigate2}
    \end{figure}
    \item 对于护卫舰 3,如图 15 所示,根据几何关系,
\end{enumerate}

\begin{equation}
\left\{
\begin{aligned}
\sin e3 &= \frac{10}{R_3} \\
\frac{\sin e3}{d_{m3}} &= \frac{\sin 125^\circ}{R_x} \\
\cos e3 &= \frac{R_3^2 + R_x^2 - d_{m3}^2}{2R_3R_x}
\end{aligned}
\right.
\tag{6}
\end{equation}

求解得驱逐舰 3 最小纵深 $d_{m3} = 15.087$。

\begin{figure}[h]
    \centering
    \includegraphics[width=\textwidth]{image1.png}
    \caption{护卫舰 3 最小防御纵深}
    \label{fig:15}
\end{figure}

\begin{figure}[h]
    \centering
    \includegraphics[width=\textwidth]{image2.png}
    \caption{护卫舰 4 防御纵深}
    \label{fig:16}
\end{figure}

(4) 对于驱逐舰 4,如图 \ref{fig:16} 所示,护卫舰 4 进行拦截时拦截导弹必然会经过驱逐舰上方 10km 的范围,因此护卫舰 4 不对来袭导弹进行拦截。

(5) 对于驱逐舰,其防御能力和护卫舰相同,来袭导弹的来袭方向指向驱逐舰,因此驱逐舰的最小防御纵深为其最小射程,即 $d_{m5} = 10km$。

进而得到非线性规划模型,我们称之为“最佳防御纵深模型”

\begin{equation}
\max i
\end{equation}

\begin{equation}
s.t.
\left\{
\begin{aligned}
\cos \alpha_k &= \frac{X_{ai}^2 + d_k^2 - X_{bi}^2}{2X_{ai}d_k} \\
X_{ai} &= V_{mi}t_{mi} \\
X_{bi} &= V_{li}t_{li} \\
t_{mi} &= t_{li} + 7 + t_{mi-1} \\
t_{m0} &= 0 \\
d &= 20 \\
\frac{\sin \alpha_k}{R_1} &= \frac{\sin \gamma}{d_k} \\
R_{dk} - X_{ai} &> d_{mk}
\end{aligned}
\right.
\tag{7}
\end{equation}

其中 $i(i=1,2,3,\cdots,n)$ 为舰艇 $k$ 所拦截的导弹的批数,该模型中各个参数求解结果如表 1 所示

\begin{tabular}{l l}
学校 & 南京理工大学 \\
\hline
参赛队号 & 10288020 \\
\hline
队员姓名 & \begin{tabular}{l l}
1. & 邸忆 \\
2. & 秦朝轩 \\
3. & 孙丽 \\
\end{tabular}

利用蒙特卡罗方法求解得到拦截情况如表2所示

\begin{tabular}{|c|l|}
\hline $\theta_{k}$ & 第 $k$ 艘护卫舰相对驱逐舰的方位角 \\
\hline $R_{k}$ & 第 $k$ 艘护卫舰与驱逐舰之间的距离 \\
\hline $R_{d k}$ & 护卫舰防御纵深 \\
\hline $\beta$ & 防卫范围角度 \\
\hline $X_{a i}$ & 来袭导弹飞行距离 \\
\hline $X_{b i}$ & 防空导弹飞行距离 \\
\hline
\end{tabular}

故在最危险的情况下最大拦截9批。其中起到主要拦截作用的是护卫舰1以及驱逐舰,护卫舰4对于来袭导弹未起到作用。

\subsection{模型评价}

(1) 模型的优点

模型的通用性比较强,可以推广到不同的舰艇布局来解算防空能力的问题。

(2) 模型的缺点

模型计算范围略微保守,只考虑了在最危险的情况下舰艇编队的防空能力,并未考虑到舰艇在20km之外的探测能力。

\section{问题三模型建立与求解}

\subsection{问题描述及分析}

由前两个问题的求解得到,舰队在仅依靠自身雷达探测系统、各舰数据链通信和防空导弹的情况下,抗饱和打击的能力为9批。问题三则在此基础上为编队配备空中预警机的信息支援,我舰队的整体警戒区域由原来的80千米向外扩展至200千米,并且预警机对所有来袭导弹都可以准确预警,包括来袭导弹的位置与速度矢量,那么如果编队在仍然保持上述设计的队形,且仅使用防空导弹拦截敌来袭导弹的情况下,我指挥舰攻击时的抗饱和攻击能力必然得到了提高,因此,问题三主要是保持原有半环形阵型以及各舰武备系统不变,计算舰队防空警戒纵深向外延拓至200千米后,抗饱和打击的能力提高情况。

\subsection{模型建立}

该问题中,我们仍然只考虑舰队重点防守最危险方向上的导弹攻击,在预警机探测到来袭目标时,指挥舰即开始采用排队思想,对各护卫舰进行防空拦截的顺序安排,以确保首发防空导弹到达航程临界时恰好遭遇首批来袭导弹,并由此类推,直至防空导弹在来袭导弹进入10千米格斗盲区前完成最后一批拦截。

在有信息支援的情况下,预警距离提高至 200 公里,预警范围大大提高,如图 17 所示,Q 为预警范围 200km,Q1 为护卫舰 1 的最大射程范围,Q2 为护卫舰 2 的最大射程范围,Q3 为 3 号护卫舰的最大射程范围,Q4 为 4 号护卫舰的最大射程范围。计算此时抗饱和攻击能力提高多少,需要计算提高预警范围后的抗饱和打击能力,首先要确定编队此时打击能力最薄弱的地方,然后借用问题 2 的模型进行计算确定此时的抗饱和打击能力。

\begin{figure}[h]
    \centering
    \includegraphics[width=0.8\textwidth]{image1.png}
    \caption{带预警机的舰队目标探测范围}
    \label{fig:17}
\end{figure}

\subsubsection{薄弱环节的确定}

根据图 12,在编队防护最薄弱的方向,可参与打击的舰艇应当最少,因此各个舰艇射程范围所包围的面积总和最小,那么,防护最薄弱的方向仍然在导弹来袭范围的边界,即 $20^\circ$ 与 $220^\circ$ 方向上。

\subsubsection{最大拦截纵深的确定}

\begin{enumerate}
    \item 对于护卫舰 1,如图 18 所示,护卫舰 1 最大拦截纵深由其射程决定,根据几何关系得 $R_{d1} = 116.7911$。
\end{enumerate}

\begin{figure}[h]
    \centering
    \includegraphics[width=0.8\textwidth]{image2.png}
    \caption{护卫舰 1 防御纵深}
    \label{fig:18}
\end{figure}

\begin{enumerate}
    \setcounter{enumi}{1}
    \item 对于护卫舰 2,其防御纵深如图 19 所示。
\end{enumerate}

\begin{figure}[h]
    \centering
    \includegraphics[width=0.8\textwidth]{image1.png}
    \caption{护卫舰 2 的防御纵深图}
    \label{fig:19}
\end{figure}

由于受护卫舰 1 的影响,发射的拦截导弹必然会进入护卫舰 10km 的范围,在来袭导弹到达护卫舰 2 射程范围 Q2 时护卫舰 2 不能马上攻击,必须等到来袭导弹进入护卫舰 2 的可打击范围后才可进行防空拦截。那么,由问题 1 的解答得到:$g2 = 50^\circ, da = 36.2523, R_1 = 42.8901, g1 = 65^\circ$。

再根据正弦定理与余弦定理得

\begin{equation}
\left\{
\begin{aligned}
g3 &= g1 - \arcsin \frac{10}{da} \\
\frac{\sin g3}{R_{d2}} &= \frac{\sin g4}{R_a} \\
\cos g3 &= \frac{R_1^2 + R_a^2 - R_d^2}{2R_1R_a}
\end{aligned}
\right.
\tag{8}
\end{equation}

求解得护卫舰 2 的防御纵深 $R_{d2} = 39.0318$,护卫舰 2 相对来袭目标的方位角 $\alpha_2 = 56.0124^\circ$ 以及护卫舰 2 和来袭目标的初始距离 $d_2 = 49.9647$。护卫舰 2 的抗饱和打击能力模型建立过程与护卫舰 1 相同。

(3) 对于护卫舰 3,由几何关系求解得到护卫舰 3 相对来袭目标的纵深,如图 \ref{fig:20} 所示,角度及距离参数:$R_{d3} = 32.1851$,$\alpha_3 = 26.0509^\circ$,$d_3 = 80$,与护卫舰 1 和护卫舰 2 相同,将参量带入非线性模型,建立护卫舰 3 的抗饱和攻击模型。

\begin{figure}[h]
    \centering
    \includegraphics[width=0.8\textwidth]{image2.png}
    \caption{护卫舰 3 防御纵深}
    \label{fig:20}
\end{figure}

\begin{enumerate}
    \item[(4)] 对于护卫舰4,舰艇所发射的拦截导弹均会经过驱逐舰周围10km的范围之内,因此护卫舰4不参与拦截。
    \item[(5)] 对于驱逐舰,由于来袭导弹指向驱逐舰,因此其最大纵深为其射程。即 \( R_{d5} = 80 \),\( \alpha_5 = 0^\circ \),\( d_5 = 80 \)。
\end{enumerate}

在问题2模型的基础上,我们同样建立非线性规划模型如式所示。在舰队中分别对四艘护卫舰使用该模型,具体如下:

\begin{equation}
\begin{aligned}
\max i_k \\
\text{s.t.} \quad & \cos \alpha_k = \frac{X_{aki}^2 + d_1^2 - X_{bki}^2}{2X_{aki}d_k} \\
& X_{aki} = V_{mi}t_{mi} \\
& X_{bki} = V_{li}t_{li} \\
& t_{mi} = t_{li} + 7 + t_{mi-1} \\
& t_{m0} = 0 \\
& \frac{\sin \alpha_k}{R_k} = \frac{\sin 25^\circ}{d_k} \\
& R_{dk} - X_{aki} > dm_k
\end{aligned}
\tag{9}
\end{equation}

各个舰的模型计算参数如表3所示

\textbf{表3 各护卫舰防御抗饱和攻击模型计算参数统计(距离单位:km;角度单位:°)}

\begin{tabular}{|c|c|c|c|c|c|c|c|c|}
\hline
\( R_{d1} \) & 116.7911 & \( R_{d2} \) & 39.0318 & \( R_{d3} \) & 32.1851 & \( R_{d4} \) & 0 & \( R_{d5} \) & 80 \\
\hline
\( d_1 \) & 80 & \( d_2 \) & 49.9647 & \( d_3 \) & 80 & \( d_4 \) & 80 & \( d_5 \) & 80 \\
\hline
\( \alpha_1 \) & 13.0956 & \( \alpha_2 \) & 56.0124 & \( \alpha_3 \) & 26.0509 & \( \alpha_4 \) & 2 & \( \alpha_5 \) & 0 \\
\hline
\end{tabular}

\subsection{模型解算及分析}

对上述四艘护卫舰抗饱和能力模型分别进行解算,得到各舰在危险方向上能够拦截来袭导弹的批次数,统计如表4。

\textbf{表4 各舰拦截来袭导弹解算结果统计}

\begin{tabular}{|c|c|c|c|c|}
\hline
 & \multicolumn{2}{c|}{拦截导弹} & \multicolumn{2}{c|}{来袭导弹} \\
\hline
 & 飞行时间/s & 飞行距离/m & 飞行时间/s & 飞行距离/m \\
\hline
\multirow{8}{*}{护卫舰1} & 98.039 & 80000 & 0 & 0 \\
\hline
 & 70.16271 & 57252.77136 & 77.16271 & 23611.78926 \\
\hline
 & 50.25751 & 41010.12816 & 134.42022 & 41132.58732 \\
\hline
 & 36.38215 & 29687.8344 & 177.80237 & 54407.52522 \\
\hline
 & 27.3388 & 22308.4608 & 212.14117 & 64915.19802 \\
\hline
 & 22.72399 & 18542.77584 & 241.86516 & 74010.73896 \\
\hline
 & 23.14932 & 18889.84512 & 272.01448 & 83236.43088 \\
\hline
 & 30.20548 & 24647.67168 & 309.21996 & 94621.30776 \\
\hline
\multirow{2}{*}{护卫舰2} & 61.2314 & 49964.7 & 0 & 0 \\
\hline
 & 52.1779 & 42577.1664 & 59.1779 & 18108.4374 \\
\hline
\multirow{2}{*}{护卫舰3} & 98.039 & 80000 & 0 & 0 \\
\hline
 & 72.46036 & 59127.65376 & 79.46036 & 24314.87016 \\
\hline
\end{tabular}

\begin{tabular}{|c|l|}
\hline
$I_{i}$ & 舰艇 $i$ 的最大拦截数量 \\
\hline
$V_{m}$ & 来袭导弹飞行速度 \\
\hline
$V_{l}$ & 拦截导弹飞行速度 \\
\hline
$t_{m}$ & 来袭导弹飞行时间 \\
\hline
$t_{l}$ & 拦截导弹飞行时间 \\
\hline
$\lambda$ & 舰艇被发现的概率 \\
\hline
$R_{a}$ & 雷达探测距离 \\
\hline
$d_{k}$ & 来袭导弹进入舰艇打击范围时舰艇 $k$ 与来袭导弹的距离 \\
\hline
$\alpha_{k}$ & 来袭导弹进入舰艇打击范围时舰艇 $k$ 与来袭导弹形成的方位角 \\
\hline
\end{tabular}

说明:来袭导弹飞行时间以及飞行距离为0,表明来袭导弹虽然被舰艇探测到,但是没有进入舰艇的打击范围,当舰艇计算其轨迹将要进入舰艇的打击范围时便发射导弹,在来袭导弹进入舰艇打击范围时马上被拦截。

最终由模型解算导弹,舰艇编队总共成功拦截来袭导弹批次为18批,与问题2的计算结果相比较发现,舰艇编队在拥有预警机的情况下,抗饱和打击能力提高了100%。

\subsection{模型评价}

(1) 模型的优点

问题三的模型沿用了问题二中的最佳防御纵深模型,可见该模型实用性强,计算简便并且易于推广。

(2) 模型的缺点

在舰艇比较多并且舰艇之间干扰关系比较复杂时解算比较困难。

\section{问题四模型建立与求解}

\subsection{问题描述及分析}

问题四给出了预警机探测到的12批空中可疑目标的位置、高度、雷达反射面等测量信息,同时给出了已知意图和属性的15批探测目标的位置、速度、高度、角度等信息,通过对已知的目标信息进行考察分析,总结规律,并由此推算待估计的12批目标的属性及意图,为后续战场决策做好预估。我们首先需要将包含该15批目标信息的数据进行预处理,剔除不合理项及不可靠数据;其次,需要通过给出的目标经纬度计算出我们所关心的目标速度、航向等信息,并进行单位统一;然后初步研究测量数据的规律性,建立合适的数学模型对已有信息进行知识学习和网络训练,最终使用训练好的学习系统对带估计的12批可疑目标进行属性判断,得出各目标可能的意图。

根据题设的要求,本问题建模求解我们做需要一下工作:

(1) 原始数据预处理,由于原始数据中给出的包含各目标不同状态参量和属性,包括时间节点、距离、航向、方位角、雷达反射面和意图判断属性等,而部分数据并不完整或连续,因此,首先要对原始数据进行预处理,剔除野值并选定用于有用的参量;

(2) 数值计算及量纲统一,在附件中给出的目标经纬度,需要根据经纬度与平面角度以及距离的计算公式进行距离和角度计算,将地球弧形单位转换为平面参量,并统一计算量纲。

(3) 建立BP神经网络系统,对已知数据进行训练,形成可以网络,再用其对未知的12批目标特征量进行预测及分类,最终估算各目标的意图。

\subsection{模型建立}

\subsubsection{数据预处理}

附件1给出了12批次目标从1471427976时间点至147428013时间点的地球空间三维数据,总时长约为 37s,而其中在 1471427998 时间点的数据中混入了少量 1471427976 的数据,可视为野值进行剔除;其次,在数据表中,发现 1471427982 时刻各目标的位置信息有两个,而在 1471427980 时刻、1471427983 时刻目标位置信息的基础上判断,1471427982 时刻的两个位置信息分别为 1471427981 时刻和 1471427982 时刻的位置信息;同理根据数据一致性原则,将 1471427987 时刻的各批目标两个位置信息分别视为 1471427981 时刻和 1471427982 时刻的位置信息;将 1471427993 时刻的各批目标两个位置信息分别视为 1471427992 时刻和 1471427993 时刻的位置信息;将 1471428000 时刻的各批目标两个位置信息分别视为 1471427999 时刻和 1471428000 时刻的位置信息;将 1471428006 时刻的各批目标两个位置信息分别视为 1471428005 时刻和 1471428006 时刻的位置信息;将 1471428008 时刻的各批目标两个位置信息分别视为 1471428007 时刻和 1471428008 时刻的位置信息;将 1471428011 时刻的各批目标两个位置信息分别视为 1471428010 时刻和 1471428011 时刻的位置信息。通过上述预处理可保证数据无野值以及整体数据的连贯性。

\subsubsection{数值计算和量纲统一}

附件 1 中给出了小数型经纬度数据,首先需要将我指挥舰位置由度分秒型数据转换统一的数据类型,转换公式如式 (10):

\begin{equation}
A = d + m / 60 + s / 3600
\tag{10}
\end{equation}

其次,根据经纬度数据统计 12 批目标各时间点的位置变化,位置变化图如图 21 中 (a)-(1) 所示

\begin{figure}[h]
\centering
\includegraphics[width=\textwidth]{image1.png}
\caption{(a) 目标 41006830}
\end{figure}

\begin{figure}[h]
\centering
\includegraphics[width=\textwidth]{image2.png}
\caption{(b) 目标 41006831}
\end{figure}

\begin{figure}[h]
    \centering
    \includegraphics[width=\textwidth]{image1.png}
    \caption{(c) 目标41006836}
\end{figure}

\begin{figure}[h]
    \centering
    \includegraphics[width=\textwidth]{image2.png}
    \caption{(d) 目标41006837}
\end{figure}

\begin{figure}[h]
    \centering
    \includegraphics[width=\textwidth]{image3.png}
    \caption{(e) 目标41006839}
\end{figure}

\begin{figure}[h]
    \centering
    \includegraphics[width=\textwidth]{image1.png}
    \caption{(f) 目标41006842}
\end{figure}

\begin{figure}[h]
    \centering
    \includegraphics[width=\textwidth]{image2.png}
    \caption{(g) 目标41006851}
\end{figure}

\begin{figure}[h]
    \centering
    \includegraphics[width=\textwidth]{image3.png}
    \caption{(h) 目标41006860}
\end{figure}

\begin{center}
    -19-
\end{center}

\begin{figure}[h]
    \centering
    \includegraphics[width=\textwidth]{image1.png}
    \caption{(i) 目标41006872}
\end{figure}

\begin{figure}[h]
    \centering
    \includegraphics[width=\textwidth]{image2.png}
    \caption{(j) 目标41006885}
\end{figure}

\begin{figure}[h]
    \centering
    \includegraphics[width=\textwidth]{image3.png}
    \caption{(k) 目标41006891}
\end{figure}

\begin{tabular}{|c|c|c|c|c|c|c|c|c|}
\hline
\( R_{d1} \) & 116.7911 & \( R_{d2} \) & 39.0318 & \( R_{d3} \) & 32.1851 & \( R_{d4} \) & 0 & \( R_{d5} \) & 80 \\
\hline
\( d_1 \) & 80 & \( d_2 \) & 49.9647 & \( d_3 \) & 80 & \( d_4 \) & 80 & \( d_5 \) & 80 \\
\hline
\( \alpha_1 \) & 13.0956 & \( \alpha_2 \) & 56.0124 & \( \alpha_3 \) & 26.0509 & \( \alpha_4 \) & 2 & \( \alpha_5 \) & 0 \\
\hline
\end{tabular}

\subsubsection{样本选择}

对12批待测目标意图判断前,我们先对五种判断级别进行量化,约定侦察、监视、攻击、掩护和其它分别对应1、2、3、4、5。通过观察可以发现,在15组训练样本中,并非所有数据均可用来当作先验知识,如:意图为攻击的目标可以是小目标、中目标和大目标。因此,目标雷达反射面积只作为目标大小属性评价,不作为意图评价参量之一,目标大小属性如表6所示。

\begin{tabular}{|c|c|c|c|c|}
\hline
 & \multicolumn{2}{c|}{拦截导弹} & \multicolumn{2}{c|}{来袭导弹} \\
\hline
 & 飞行时间/s & 飞行距离/m & 飞行时间/s & 飞行距离/m \\
\hline
\multirow{8}{*}{护卫舰1} & 98.039 & 80000 & 0 & 0 \\
\hline
 & 70.16271 & 57252.77136 & 77.16271 & 23611.78926 \\
\hline
 & 50.25751 & 41010.12816 & 134.42022 & 41132.58732 \\
\hline
 & 36.38215 & 29687.8344 & 177.80237 & 54407.52522 \\
\hline
 & 27.3388 & 22308.4608 & 212.14117 & 64915.19802 \\
\hline
 & 22.72399 & 18542.77584 & 241.86516 & 74010.73896 \\
\hline
 & 23.14932 & 18889.84512 & 272.01448 & 83236.43088 \\
\hline
 & 30.20548 & 24647.67168 & 309.21996 & 94621.30776 \\
\hline
\multirow{2}{*}{护卫舰2} & 61.2314 & 49964.7 & 0 & 0 \\
\hline
 & 52.1779 & 42577.1664 & 59.1779 & 18108.4374 \\
\hline
\multirow{2}{*}{护卫舰3} & 98.039 & 80000 & 0 & 0 \\
\hline
 & 72.46036 & 59127.65376 & 79.46036 & 24314.87016 \\
\hline
\end{tabular}

\begin{table}
\centering
\begin{tabular}{|c|c|c|}
\hline
41006851 & 5.5 & 大目标 \\
\hline
41006860 & 6.2 & 大目标 \\
\hline
41006872 & 1.7 & 小目标 \\
\hline
41006885 & 1.1 & 小目标 \\
\hline
41006891 & 3.6 & 中目标 \\
\hline
41006893 & 3.1 & 中目标 \\
\hline
\end{tabular}
\end{table}

\section{基于BP神经网络的目标意图分类模型}

用matlab2013a设计三层BR神经网络,根据7.2样本筛选后,采用15批已知目标的距离、方位、航向角、高度和水平速度等五个样本作为网络训练知识,网络构建及求解可分为6个模块进行:①原始数据输入;②数据标准化;③网络训练;④原始数据仿真;⑤仿真结果对比;⑥新数据预测,流程图如图22所示。

\begin{figure}[h]
\centering
\includegraphics[width=0.8\textwidth]{image22.png}
\caption{BP网络训练流程图}
\end{figure}

网络设计参量如下:三层神经元节点个数分别为4、8、2;最大训练循环为50000次;每500次循环显示一个结果;均方误差为0.005;学习速率为0.05,3层BP神经网络学习及分类如图23所示。

\begin{figure}[h]
\centering
\includegraphics[width=0.8\textwidth]{image23.png}
\caption{三层BP神经网络训练界面}
\end{figure}

\section{模型计算}

先对神经网络进行原始数据的知识训练并将原始作为输入数据进行数值仿真,仿真结果如表7所示,不难发现,经过训练的分类网络的分类正确率已达100%。再用该网络对待测目标数据进行预测判别,采用蒙特卡罗法计算20次得到12批目标的意图类别如表8所示

\begin{table}
\centering
\begin{tabular}{|c|c|c|c|}
\hline
目标(15批已知) & 原始数据 & 仿真结果 & 是否正确 \\
\hline
\(e_{1}\) & 侦察1 & 1 & 正确 \\
\hline
\end{tabular}
\caption{BP神经网络训练结果对比}
\end{table}

\begin{table}
\centering
\begin{tabular}{|c|c|c|c|}
\hline
$e_{2}$ & 攻击 3 & 3 & 正确 \\
\hline
$e_{3}$ & 侦察 1 & 1 & 正确 \\
\hline
$e_{4}$ & 攻击 3 & 3 & 正确 \\
\hline
$e_{5}$ & 侦察 1 & 1 & 正确 \\
\hline
$e_{6}$ & 侦察 1 & 1 & 正确 \\
\hline
$e_{7}$ & 攻击 3 & 3 & 正确 \\
\hline
$e_{8}$ & 掩护 4 & 4 & 正确 \\
\hline
$e_{9}$ & 监视 2 & 2 & 正确 \\
\hline
$e_{10}$ & 其它 5 & 5 & 正确 \\
\hline
$e_{11}$ & 掩护 4 & 4 & 正确 \\
\hline
$e_{12}$ & 其它 5 & 5 & 正确 \\
\hline
$e_{13}$ & 其它 5 & 5 & 正确 \\
\hline
$e_{14}$ & 其它 5 & 5 & 正确 \\
\hline
$e_{15}$ & 攻击 3 & 3 & 正确 \\
\hline
\end{tabular}
\caption{表812批待测目标的意图类别}
\end{table}

\begin{table}
\centering
\begin{tabular}{|c|c|c|}
\hline
目标ID & 意图级别判断 & 意图判断 \\
\hline
41006830 & 4 & 掩护 \\
\hline
41006831 & 5 & 其他 \\
\hline
41006836 & 5 & 其他 \\
\hline
41006837 & 5 & 其他 \\
\hline
41006839 & 3 & 攻击 \\
\hline
41006842 & 3 & 攻击 \\
\hline
41006851 & 3 & 攻击 \\
\hline
41006860 & 4 & 掩护 \\
\hline
41006872 & 4 & 掩护 \\
\hline
41006885 & 3 & 攻击 \\
\hline
41006891 & 2 & 监视 \\
\hline
41006893 & 3 & 攻击 \\
\hline
\end{tabular}
\end{table}

根据BP神经网络分类算法得出预警机提供的12批目标中,有意图为攻击的目标有四个,其ID分别为(41006839、41006842、41006851、41006885);意图为掩护的目标有三个,分别为(41006830、41006860、41006872);意图为监听的目标有一个41006891;其他目标意图为“其它”,因此,我舰艇编队应对意图为攻击的目标进行更为严密的探测跟踪甚至拦截动作。

\section{模型评价}

通过网络训练结果(见表7)可以发现,BP神经网络训练成功率很高,原始数据仿真结果与原始样本对比,准确率达到了100%。有图24和图25中可以发现,BP 神经网络对新数据进行分类的收敛速率极快,学习失败率为 0,误差远小于 \(10^{-3}\),因此,基于 BP 神经网络的目标意图分类是可靠的,对于多目标意图分类来说,BP 神经网络模型是一种较为实用的模型。

缺陷:神经网络是一种学习型分类模型,其计算结果依赖于原始已知数据的可靠性,如果用于学习的数据真实性较低,则会影响模型判断结果。

因此,在现在战争中,用于敌我判别的识别系统模型需要具备智能化、实时性、自适应性和鲁棒性强的特点。

\begin{figure}[h]
    \centering
    \includegraphics[width=0.48\textwidth]{image1.png}
    \caption{BP 神经网络模型收敛曲线}
    \label{fig:bp_convergence}
\end{figure}
\begin{figure}[h]
    \centering
    \includegraphics[width=0.48\textwidth]{image2.png}
    \caption{网络训练过程权值及参量状态}
    \label{fig:network_training}
\end{figure}

\section{问题五模型建立与求解}

\subsection{问题描述及分析}

由上述几个问题可以发现,随着信息科技迅猛发展,战争形式也处在巨大的演变过程中,无论海战、空战还是陆战,敌我双方的战争形式不再仅仅依靠排兵布阵、饱和攻防的手段,而是由机械化、人力化转向智能化、信息化战争,战场态势发展已经由以往的对称性逐步演化为非对称信息战争模式,这也正是本题中所提到的如果我方电子信息设备受到干扰,信息通信能力遭受破坏,将极有可能导致一场关键战斗的失利,使得胜利的天平直接向对方倾斜。同样,军队的指挥者、战略家对战场态势的评估方法以及作战评估模型也应当由传统迅速向现代化靠拢,需要更加可靠合理的现代化战争理论及信息战评估模型。

针对热兵器时代战争的预估模型最早是由 F. W. Lanchester 于 1916 提出的兰彻斯特模型 \cite{ref6, ref7},该模型起初被用来描述空对空作战效能。兰彻斯特模型是一种确定性解析模型,它把每一方具有相同作战能力的诸战斗成员或战斗单元聚合为一个作战单元,用一组微分方程描述其在火力毁伤过程中,战斗损耗随时间的变化 \cite{ref8, ref9}。但在现代高技术战争的大背景下,信息化战场成了现代战争的代名词,战场上哪一方掌控了更多的战场信息,或信息获取手段,胜利的天平就会加速向这一方倾斜,信息化战争体现了新型战争的非对称特性,完全取缔了以往的兵力对兵力,装备对装备的战争模式,因此传统兰彻斯特方程在信息化战争的评估中已经无法体现其以往的优势,而其基于对称信息评估的理念已成为当今非对称信息化战争的劣势。

因此,我们在这一问中,首先需要结合信息网络的特点,分析战场上军队战斗力的“量子化”特征;其次在传统兰彻斯特模型的基础上,构建信息化军队战斗力系统的变结构数学模型,我们称之为非对称信息评估模型;最后使用海湾战争两军兵力统计数据进行模型验证。

\section{模型建立}

\subsection{军队战斗力“量子化”假说及特征}

考虑到信息化战争的特点,引入军队战斗力“量子化”假说\cite{ref10}:

\begin{enumerate}
    \item 战斗力量子是信息化战场上的基本作战单元,是战斗力要素具体表现形式的一种表达;军队战斗力的形成建立在战斗力量子的系统集成之上,军队战斗力“量子化”现象建立在战斗力量子网络化的基础之上。
    \item 在成熟的网络中心战的理想条件下,战斗力量子的行为具有二重性:对外被“质点化”,内部结构、属性被遮蔽,功能、效能趋同,只是在网络上的一个节点而已;对内来说,只要信息链式运动存在,即信息变化率不等于 0,就可以在必要的时候展开内部结构,形成局部网络,其内部的结构,发生被解蔽。
    \item 建构信息化军队战斗力系统时,作战单元按照战斗力量子→质点→节点链条改变功能,在战场上形成立体之网,产生信息结构力,并服从“梅特卡夫定律”。
    \item 当信息平台、数据链、信息因子遭到破坏,信息的链式运动在不同范围出现断链时,军队战斗力系统会出现不同的反映:
    \begin{enumerate}
        \item 单个战斗力量子遭破坏,本作战单元从高阶系统降至低阶系统,作战效能回到信息化以前的水平。
        \item 战斗力量子损失数量有限,但未达到临界值时(还有冗余战斗力量子时),由于“冗余战斗力量子+网络互补功能”的作用,网络具有一定程度的代偿功能,整个系统的战斗力水平维持基本不变。
        \item 当战斗力量损失数量达到一定值(如临界值),或信息平台遭到结构性破坏时,军队战斗力系统产生相变,战斗力会出现突变,立即从高阶跳跃到阶。
        \item 战斗力量子本身就是一个复杂系统,同样存在相变、混沌现象;一个战斗力量子的突变情形与整体战斗力系统类似,只不过系统大小、复杂程度、层次规模不同而已。
    \end{enumerate}
\end{enumerate}

战斗力量子是信息化军队战斗力系统基本组成单元,它是以网络为中心,以信息为要素,以信息链式为基础建立的,是对传统战斗力要素的系统集成,是从平台中心战到网络中心战转型的系统机制。战斗力量子应具有如下基本特征:
\begin{enumerate}
    \item 具有独立作战功能;
    \item 具有对内对外通信能力;
    \item 在网格下不封闭,可以接受外部信息;
    \item 具有可变结构,当信息链折断时,结构从信息网状切换到机械线性系统;
    \item 在普遍链接的情况下,任何组分形式的战斗力量子对外功能、效能逐趋同并等值化,但取值是不连续的;
    \item 拥有信息优势的一方,战场对它透明,可等效为直瞄射击。
\end{enumerate}

\subsection{军队战斗力计算模型}

人类战争中介系统经历了手足中介、平台中介、和网络中介(信息)三个过程,分别对应于物质系统、能量系统和信息系统。因此,在一场战争或战斗中,军队战斗力系统存在三种系统结构的可能性。当信息要素变化为零时,切换到能量系统,当能量系统的变化为零时,切换到物质系统,当物质系统的变化为零时,军队完全丧失了战斗力。不同的中介系统,其军队战斗力也存在较大的差别,具体的计算公式如下所示:

(1)物质系统相当于冷兵器时代军队战斗力系统,其战斗力单元是 1 个人 + 1 支武器,整个军队战斗力

\begin{equation}
P_0 = N * V * Q
\tag{11}
\end{equation}

其中, \( N \)——某一交战方军队人数;
\( V \)——影响某一交战方作战条件的因素变量;
\( Q \)——某一交战方军队质量的量化值。

(2)能量系统相当于热核发兵器军队的战斗力系统,其战斗单元是 1 个人(若干人)+ 1 个平台 + 语音通信系统,整个军队战斗力
\begin{equation}
P_{E} = S * V_{f} * CEV
\tag{12}
\end{equation}
其中,\( S \)——军队兵力;
\( V_{f} \)——军队兵力变量,由环境变量、作战变量、行为变量确定;
\( CEV \)——相对战斗效能指数。

兵力 \( S \) 是一支军队所拥有的每一类武器的实际杀伤力指数总值乘以特定战场条件对该类武器使用效果的影响因素之后的总和:
\begin{equation}
S = \sum_{i=1}^{n} W_{i} V_{i} \, (i=1,2,\cdots,n)
\tag{13}
\end{equation}
式中,\( W_{i} \) 为某类武器的实际杀伤力相对指数,\( V_{i} \) 为该武器数量。

战斗效能指数 \( CEV \) 由下式求得:
\begin{equation}
EV_{r} = \left( R_{r} / R_{b} \right) / \left( P_{r} / P_{b} \right)
\tag{14}
\end{equation}
式中,记作战双方分别为红方和蓝方,用下标字母 \( r \) 代表红方,字母 \( b \) 代表蓝方; \( CEV_{r} \) 为红方相对战斗效能指数,\( R_{r} / R_{b} \) 为红方对蓝方的实际作战效果,即双方战斗实际战斗力之比; \( P_{r} / P_{b} \) 为红方对蓝方理论作战结果,即双方战斗潜力之比。其中,作战结果或由下列公式求得:
\begin{equation}
R_{r} = MF + E_{sp} + E_{cas}
\tag{15}
\end{equation}
式中,\( MF \) 为任务因子,反映一支部队达成其领受任务或自行领会任务的能力,由军事专家判定; \( E_{sp} \) 为空间效能,衡量一支部队夺取或扼守阵地的能力指标; \( E_{cas} \) 为杀伤效果,它是根据伤亡率衡量一支部队战斗力的指标。

综上可得,战斗力量子基本作战效能指数为
\begin{equation}
P_{m} = \frac{P_{E}}{T}
\tag{16}
\end{equation}
式中,\( T \) 为战场空间的战斗力量子数。

(3)信息系统是信息化军队的结构特征。成熟的信息化军队或成熟的数字化战场,网络普遍链接,信息的链式运动畅通无阻,作战单元质点化后形成战斗力量子并弥散性分布在战场,使军队战斗力出现飞跃并服从“梅特卡夫定律”。因此,信息化军队战斗力应由两个部分组成,一部分是独立存在情况下战斗力量子之和,另一部分是网络化的战斗力量子产生的信息结构力。

根据战斗力“量子化”假说和战斗力量子的基本特征,除了信息化战场可能存在信息不对称之外,信息化战争的情况正是兰彻斯特方程所假设的条件,考虑利用兰彻斯特战斗理论建立战斗力量子数学模型,鉴于平方律战斗模型适合于描述现代战斗直瞄作战的兵力变化,它的损耗系数主要反映了 \( C^{4}ISR \) 系统对直瞄射击的作用,采用经修正后的兰切斯特平方律战斗模型

\begin{equation}
\begin{cases}
\frac{dr}{dt} = -n_{b} \frac{R_{kb}^{2}}{K_{3b} R_{lr}^{2}} \frac{K_{1b}}{K_{2b}} b \\
\frac{db}{dt} = -n_{r} \frac{R_{kr}^{2}}{K_{3r} R_{lr}^{2}} \frac{K_{1r}}{K_{2r}} r
\end{cases}
\tag{17}
\end{equation}

式中,$r$、$b$ 分别为 $t$ 时刻红蓝双方的兵力;$n_{r}$、$n_{b}$ 分别为红蓝双方的战术射速;$R_{kr}$、$R_{kb}$ 分别为红蓝双方武器杀伤的半径;$R_{lr}$、$R_{lb}$ 分别为红蓝双方对对方的目标定位的不确定半径;$K_{1b}$、$K_{1r}$ 为红蓝双方的目标信息分辨率系数,$K_{1} \geq 1$;$K_{2r}$、$K_{2b}$ 为红蓝双方的作战效果评估速度系数,$0 < K_{2} < 1$;$K_{3r}$、$K_{3b}$ 为红蓝双方发现目标延迟系数,$0 < K_{3} < 1$。

设 $t=0$ 时,双方的初始兵力为 $r_{0}$、$b_{0}$,为了简单起见,假设 $C^{4}ISR$ 系统仅影响红方对蓝方的损耗系数,则红方采用新 $C^{4}ISR$ 系统后的兵力倍增系数为

\begin{equation}
G_{r} = \frac{r_{0}}{r_{0c}} \bigg|_{b(0)=b_{0}} = \sqrt{\frac{a_{c}}{a_{0}}}
\tag{18}
\end{equation}

式中,$a_{c}$ 为采用新 $C^{4}ISR$ 系统后损耗系数,$a_{0}$ 为采用新 $C^{4}ISR$ 系统前的损耗系数,联立各式,即可求得 $G_{r}$。同理,可以求得红方对蓝方的兵力倍增数 $G_{b}$。

战斗力量子的公式计算如下:

\begin{equation}
P_{n} = G_{r} P_{M}
\tag{19}
\end{equation}

式中,$G_{r}$ 为兵力倍增系数,$P_{M}$ 为作战单元信息化之前的战斗效能,可以从作战单元构成要素的作战效能指数中求得,也可以取整个军队作战效能指数的平均值。

为了描述信息化战争中信息对战争的影响,特引入信息结构力的概念。信息结构力是一个独立的增量,它的大小服从“梅特卡夫定律”。信息结构力是作战单元充分质点化之后基于信息平台的网格结构形成的。把每一个战斗力量子看作网络上的一个节点,考察节点数量、相互之间关系的不同所引起系统结构力的变化,而并不关心节点内部的结构和运行过程。每一个战斗力量子产生一部分超出 $P_{n}$ 的独立性增量为 $T$

\begin{equation}
P_{s} = k P_{n} T^{2}
\tag{20}
\end{equation}

式中,$P_{n}$ 为战斗力量子的战斗力指数;$T$ 为有效战场空间内的战斗力量子数,即网络节点数,战场空间取立体空间,上限为卫星到达的高度,即距地表 120 千米;$k$ 为网络结构的修正系数(在 $0 \sim 1$ 之间取值),取决于节点之间的结构,反映的是军队信息化水平,在 Internet、Web 和 Grid 等不同网络连接方式下,分别取不同的值,并由“综合集成研讨厅”确定。在普遍互通互联互操作的理想情况下,$k$ 的取值为 1。

整个军队战斗力系统的独立性增量即为信息结构力,其大小为

\begin{equation}
P_{I} = \sum_{j=1}^{T} P_{sj} = \sum_{j=1}^{T} k P_{nj} T^{2}
\tag{21}
\end{equation}

综上可得信息系统产生的信息化军队战斗力可以表示为

\begin{equation}
P_{I} = P_{E} + \Delta P_{I} = P_{E} + \sum_{j=1}^{T} k P_{nj} T^{2} = P_{E} + k G_{r} P_{E} T^{2} = P_{E} \left( 1 + k G_{r} T^{2} \right) (j=1,2,\cdots T)
\tag{22}
\end{equation}

\subsection{信息化军队战斗力系统的变结构数学模型}

由 8.2.2 节的分析可得,信息化了的军队战斗力系统实际是暗含着三种结构:以物质为主导要素的手足中介系统;以能量为主导要素的平台中介系统;以信息为主导要素的网络中介系统。它们分别与人类战争中介系统革命史上的人体中心战、平台中心战和网络中心战一一对应。将式 (11)、(12)、(22) 联立,得到信息化军队战斗力系统的变结构数学模型(这里我们更愿意称之为“非对称信息战争评估模型”)。

\begin{equation}
\begin{cases}
P_{0} = N * V * Q & T = 0, k = 0 \\
P_{E} = S * V_{f} * CEV & T = 1, k = 0 \\
P_{I} = P_{E} \left( 1 + k G_{r} T^{2} \right) & T \geq 2, 0 < k \leq 1
\end{cases}
\tag{23}
\end{equation}

模型表明,在信息化战争过程中,军事系统的结构是可变的,当战斗量子数大于 2、网络化系统 $k \neq 0$ 时,军事系统下于网络中心战结构,存在“梅特卡夫腾跃”,军队战斗力系统产生信息结构力;当战斗力量子数 $T = 1$、网络化系数 $k = 0$ 时,$P_{I} = P_{E}$,军事系统转化为平台中心战结构;当战斗力量子数 $T = 0$、网络化系数 $k = 0$ 时,意味着军事系统失去能量、信息要素支撑,军事系统转化为人体中心战结构,进入以手足为中介的面对面的作战。

如果战斗力量子数 $T$ 足够大,$P_{E}$ 可以忽略不计,可以推得成熟的网络中心战条件下的军队战斗力系统作战效能指数公式

\begin{equation}
P_{I} = k G_{r} P_{E} T^{2}
\tag{24}
\end{equation}

上式表明,在完全信息化、网络化、智能化理想条件下,军队战斗力主要来自于信息结构力,与战斗量子的数量平方成正比。

方程组中 $P_{I}$ 的计算公式,实际上是信息化军队战斗力成长过程完成后的定态公式。在战争过程中,只要战争的时间 $t$ 足够短,且战斗力量子数 $T$ 足够大,那么就可以将 $P_{I}$ 看成随 $T$ 变化的连续函数。对 (24) 式进行求导,得

\begin{equation}
\frac{dP_{I}}{dT} = -2 k G_{r} P_{E} T
\tag{25}
\end{equation}

上式表明,信息化军队的战斗力损耗将随战斗力量子数线性下降。如果考虑双方都以网络中心战时行对抗,得到方程组

\begin{equation}
\begin{cases}
\frac{dP_{I_{b}}}{dT} = -2 k_{b} G_{b} P_{Eb} T_{b} \\
\frac{dP_{I_{r}}}{dT} = -2 k_{r} G_{r} P_{Er} T_{r}
\end{cases}
\tag{26}
\end{equation}

令

\begin{equation}
\begin{cases}
\xi = 2 k_{b} G_{b} P_{Eb} \\
\theta = 2 k_{r} G_{r} P_{Er}
\end{cases}
\tag{27}
\end{equation}

式 (26) 变为

\begin{equation}
\begin{cases}
\frac{dP_{I_{b}}}{dT} = -\xi T_{b} \\
\frac{dP_{I_{r}}}{dT} = -\theta T_{r}
\end{cases}
\tag{28}
\end{equation}

经过变换可得

\begin{equation}
\xi T_{b}^{2} = \theta T_{r}^{2}
\tag{29}
\end{equation}

式中 \(\xi\)、\(\theta\) 分别为红蓝双方对对方的网络杀伤系数,\(T_{r}\)、\(T_{b}\) 为红蓝双方的战斗力量子数。表明在网络中心战条件下,交战双方每一方的战斗力损耗取决于对方的网络杀伤系统与战斗力量子数的乘积。这反映了信息化战争在网络中心战阶段打结构、打节点的本质特征。同时,与经典兰彻斯特方程中的耗损系数 \(\alpha\),\(\beta\) 的值相比,\(\xi\)、\(\theta\) 的值大得多,说明在信息化战争条件下,军队战斗将随战斗力量子数的变化而迅速变化,这为战斗力“轰塌”现象提供了理论根据。

\subsection{模型的验证}

通过海湾战争对模型进行验证,将美国及多国部队记为红队,用带字母 r 的下标表示,将伊拉克军队记为蓝队,用带字母 b 的下标表示。

为了把军队的实际数量转化为部队兵力,假定每架作战飞机相当于 100 个士兵及其所需的支援兵器(如迫击炮、机关枪和火炮等),每辆美军坦克、装甲车相当于 50 名士兵及其所需支援兵器,每枚巡航导弹相当于 100 名士兵其所需支援兵器,每架攻击直升机相当于 300 名士兵及其所需支援兵器,每架固定翼飞机相当于 300 名士兵及其所需支援兵器,每艘舰船相当于 200 名士兵及其所需支援兵器,每架火炮相当于 50 名士兵及其所需志愿兵力。这种假设方法,尽管随意,但却合理、可靠。根据附件 B 给出的交战双方兵力对比可得双方兵力为:

\begin{align*}
S_{r} &= \sum_{i=1}^{n} W_{i} V_{i} = 76*1 + 3500*0.005 + 3000*0.005 + 3300*0.03 + 1900*0.02 \\
&\quad + 180*0.02 + 1600*0.005 = 257.1 \text{万} \\
S_{b} &= \sum_{i=1}^{n} W_{i} V_{i} = 120*1 + 5600*0.005 + 7000*0.005 + 793*0.03 \\
&\quad + 300*0.03 + 60*0.02 + 3000*0.005 = 231.99 \text{万} \\
V_{fr} &= 1; \quad V_{fb} = 1; \quad CEV_{r} = 1; \quad CEV_{b} = 1; \\
P_{Er} &= S_{r} * V_{fr} * CEV_{r} = 257.1 \text{万} \\
P_{Eb} &= S_{b} * V_{fr} * CEV_{b} = 231.99 \text{万}
\end{align*}

\(T\) 为有效战场空间内的战斗力量子数,即网络节点数,由于海湾战争中美军至少使用了 12 种约 70 颗卫星,8 个电子对抗情报营和 5 个电子对抗情报连,以及大量的雷达侦察系统、通信侦察测向系统、辐射源识别系统、雷达干扰系统、通信干扰系统;相对而言,伊拉克军方的电子战装备数量少,水平低。因此设 \(T_{r} = 50\),\(T_{b} = 1\)。

由于美国及多国部队采用的 M1A1 坦克装备的热成像仪在夜间或烟雾条件下可以识别 1500 米内的目标,而探测距离远达 3000 米;而伊军 T-72M 配备的只是第二代微光夜视仪,最大探测距离 800 米、甚至更短,因此伊军的损耗系数远远大于美军,取 \(G_{r} = 1\),\(G_{b} = 0.1667\)。网络化系数 \(k_{r} = 0.9\),\(k_{b} = 0.1\)。

\begin{align*}
\frac{dP_{Ib}}{dT} &= -2k_{b}G_{b}P_{Eb}T_{b} = -2*0.1*0.1667*231.99*1 = -7.7345 \\
\frac{dP_{Ir}}{dT} &= -2k_{r}G_{r}P_{Er}T_{r} = -2*0.9*1*257.1*50 = -23139
\end{align*}

伊方与多国部队兵力损失率之比为

\[
\frac{dP_{Ib}}{dT} / \frac{dP_{Ir}}{dT} = 2991.6607
\]

通过附录 B 给出的海湾战争损伤情况,可以计算出,多国部队的兵力损失数为:
\begin{align*}
S_{lr} &= \sum_{i=1}^{n} W_i V_i = 0.119 * 1 + 2 * 0.02 = 0.1590 \text{万} \\
S_{lb} &= \sum_{i=1}^{n} W_i V_i = 27.5 * 1 + 3847 * 0.005 + 1450 * 0.005 \\
&\quad + 324 * 0.03 + 57 * 0.02 + 2917 * 0.005 \\
&= 79.4300 \text{万} \\
\frac{S_{lb}}{S_{lr}} &= 499.5597
\end{align*}

\subsection{模型评价}

通过上面的计算可以看出,伊方与多国部队兵力损失率之比远远大于实际数据计算得到的折算兵力损失数之比。可能是由于将坦克、装甲车、直升机等武器转化为部队兵力的局部不合理性造成的。然而通过模型的计算对比可以看出信息化战争条件下,如果战斗力量子数 \( T \) 足够大,\( P_E \) 可以忽略不计,即双方兵力的损失率与初始战斗力相关性不大,而与战斗力量子数 \( T \)、网络化系数 \( k \)、兵力倍增系数 \( G \) 具有强相关性。一定程度上反映信息化战争非接触作战、非对称作战和非线性作战的特点,表明电子战是获取制信息权的利器,而武器装备“质量”优势是打赢信息化战争的重要保证。

\section{总结}

根据题设要求以及所提供的原始数据,我们基本完成了所有五个问题的建模、解算及讨论,针对各问题给出了相应的解答和注释,简要总结如下:

(1) 在未发现敌方目标时,设计编队最佳队形为半环形阵型,给出各护卫舰相对指挥舰的方位和距离,以应对所有可能的突发事件,最大可能保护指挥舰;

(2) 确定了舰队整体防守薄弱方向,建立最佳防御纵深模型,并计算得出半环形舰队的抗饱和攻击能力为 9 批;

(3) 在原有阵型基础上增加预警机条件,采用最佳防御纵深模型计算出,在配备预警机的情况下,舰队抗饱和攻击能力增至 18 批,导弹拦截率提升一倍;

(4) 根据对已知样品数据的处理分类和学习,建立 BP 神经网络学习系统,准确的对 12 批待测目标的属性意图做出判断,发现该批目标中的攻击类;

(5) 分析兰彻斯特模型,确定其在信息化战争下的劣势,建立一种非对称信息战争评估模型,并采用海湾战争数据验证了所提模型的有效性。

\section{参考文献}

[1] 徐培德,余滨,马满好等.军事运筹学基础[M].长沙:国防科技大学出版社,第二版,2007.5.1

[2] 申卯兴,曹泽阳,周林.现代军事运筹[M].北京:国防工业出版社,2014.7.

[3] 李永刚.舰艇编队对空防御队形优化配置模型[J].情报指挥控制系统与仿真技术.2004.26(4):65-67.

[4] 赵建军,王毅,王星博登.舰艇编队防空拦截模型及队形研究[J].舰船电子工程.2012.32(6):20-23.

[5] 姜鲁东,余家祥,斗计华.水面舰艇编队区域防空导弹目标分配模型研究

[J]. 现代防御技术. 2012.40(3):29-33.

[6] 张啸天等. 多维战争中兰彻斯特方程探讨[J]. 火力与指挥控制. 2008. 33(2).

[7] 孔红山,张明清,唐俊. 兰彻斯特方程的系统动力学模型研究[J]. 计算机工程与设计. 2011. 32(8):2789-2792.

[8] 史彦斌,高宪军,张安. 基于兰彻斯特方程的信息支援效能研究[J]. 航空计算技术. 2007. 37(5):21-24.

[9] 唐铁军,徐浩军. 应用兰彻斯特法进行体系对抗效能评估[J]. 火力与指挥控制. 2007. 32(8):52-54.

[10] 张鹏飞,贾易飞. 从杜普伊定量评估模型分析伊战结果错误看信息因素在战争中的作用[J]. 国防科技. 2007. (4):57-59.

\section{附录}

\section{附录一}

\textbf{问题一代码:}

\begin{verbatim}
function [ output_args ] = m2( x,y )
%% 2015 数学建模竞赛 A 题
%% 函数说明:利用余弦定理确定最小纵深 x 与驱逐舰与护卫舰之间的距离之间的函数关系
output_args=x^2+y^2-20^2-2*x*y*cos(25*pi/180);
end
\end{verbatim}

\begin{verbatim}
%% 函数说明:画出最小纵深随驱逐舰与护卫舰之间的距离的变化曲线,其最大值即为最小纵深的最大值
clear;clc;
tic
diert=0.1;
i0=30;i1=47.32;
i=i0:diert:i1;
s=size(i);
y=zeros(s);
for i=1:1:s(2)
    syms x;
    % R=30+(i-1)*diert;
    R=i0+(i-1)*diert;
    f=m2(x,R);
    r=solve(f);
    result=subs(r);
    y(1,i)=max(result);
end
% i=30:diert:47.3;
i=i0:diert:i1;
plot(i,y,'linewidth',3);
xlabel('R1','fontsize',14);
ylabel('Rd','fontsize',14);
set(gca,'FontSize',14);
[m,n]=max(y);
\end{verbatim}

\begin{verbatim}
x=i0+(n-1)*diert;
toc

附录二
问题 2 及问题 3 代码
clear all
clc
vm=306;         %来袭导弹速度
vl=816;         %拦截导弹速度
d=80000         %护卫舰相对来袭导弹的初始距离
alpha=0/180*pi;  %护卫舰相对来袭导弹的初始角度

% t=10:0.0001:50;    %导弹飞行时间
dem=80000;       %危险距离
i=1;             %拦截批数
tf=zeros(20,1);   %拦截导弹飞行时间
det=zeros(20,1);  %来袭导弹飞行时间
lda=1;

while lda<=dem
    t=1:0.00001:100;
    a1=vm.*(det(i)+7+t);         %来袭导弹飞行距离
    a2=vl*t;                      %拦截导弹飞行距离
    yux=(d^2+a1.^2-a2.^2)./(2*a1*d);
    fin=yux-cos(alpha);
    fin=abs(fin);
    b=min(fin);
    x=find(fin==b);
    tf(i)=t(x);
    det(i+1)=tf(i)+7+det(i);
    lda=vm*det(i+1);
    i=i+1;
end

附录三
问题四代码:(基于三层 BP 神经网络的目标意图分类识别代码)
%第四问代码
%BP神经网络修改
clc
clear all;
close all;
%原始数据
%1-距离(单位:km)
sqjl=[281 210 280 215 282 284 150 110 260 110 ...
      120 140 295 210 290];
%2-方位角(单位:mil)
sqfwj=[810 2300 820 2325 830 825 2250 4000 2800 5120 ...
       4020 4800 480 2450 2900];
%3-水平速度(单位:m/s)
sqspd=[250 300 245 320 255 250 300 300 215 210 ...
\end{verbatim}

\begin{verbatim}
280 220 292 230 272];
%4-航向角(单位:°)
sqhxj=[202 310 201 324 200 204 155 50 260 52 ...
       52 18 245 210 350];
%5-高度(单位km)
sqgd=[6.0 4.0 6.5 4.2 4.2 5.0 5.0 3.4 7.7 3.6 ...
       3.6 9.6 9.9 5.0 5.6];

%目标意图类
mbyt1=[1 3 1 3 1 1 3 4 2 5 ...
       4 5 5 5 3];
mbyt2=[1 3 1 3 1 1 3 4 2 5 ...
       4 5 5 5 3];

p=[sqjl;sqfwj;sqpsd;sqhxj;sqgd];    %输入数据矩阵
t=[mbyt1;mbyt2];                    %目标数据矩阵
%利用premnmx函数对数据进行归一化
[pn,minp,maxp,tn,mint,maxt]=premnmx(p,t); %对于输入矩阵p和输出矩阵t进行归一化处理
dx=[-1,1;-1,1;-1,1;-1,1;-1,1];      %归一化处理后最小值为-1,最大值为1

%BP网络训练
net=newff(dx,[4,8,2],{'tansig','tansig','tansig'},'traingdx'); %建立模型
net.trainParam.show=1000;           %1000轮回显示一次结果
net.trainParam.Lr=0.05;             %学习速度0.05
net.trainParam.epochs=50000;        %最大循环次数50000↑
net.trainParam.goal=0.5*10^(-3);    %均方误差
net=train(net,pn,tn);               %开始训练,pn,tn分别为输入输出样本

%利用原始数据对BP网络仿真
an=sim(net,pn);                     %用训练好的模型进行仿真
a=postmnmx(an,mint,maxt);           %把仿真得到的数据还原为原始数量级
%利用训练好的网络进行预测
pnew=[210 202 217 216 230 156 161 210 235 200 225 241 ...
       3990 5183 4373 5976 1268 431 1608 6081 3249 3057 742 1525 ...
       142 269 230 195 195 266 300 283 172 301 327 226 ...
       29 77 80 143 190 220 277 280 92 350 157 137 ...
       7.0 9.2 4.6 5.2 5.2 1.0 1.2 9.4 6.0 1.4 4.8 8.6]; %带估计的12批目标数据
pnewn=tramnmx(pnew,minp,maxp);      %利用原始输入数据的归一化参数对新数据进行归一化
anewn=sim(net,pnewn);               %利用归一化后的数据进行仿真
anew=postmnmx(anewn,mint,maxt);     %把仿真得到的数据还原为原始的数量级
\end{verbatim}

\section{附录四}

经过野值剔除和连续性处理的附件 A 数据,用于目标意图识别及分类计算,数据见附件 1(原始数据修正.xlsx)。

\end{document}