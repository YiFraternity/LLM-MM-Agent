\title{发现空间目标并定位的模型}

\author{
    李成法\textsuperscript{1}, 郭昭辉\textsuperscript{1}, 薛巍立\textsuperscript{2} \\
    \textsuperscript{1} 南京大学计算机科学与技术系,南京 210093 \\
    \textsuperscript{2} 南京大学数学系,南京 210093
}

\maketitle

\begin{abstract}
摘要:本文针对在一个圆柱体内用最少的红球、蓝球发现黄球的问题,采用多个椭圆覆盖圆柱顶面的方法,设计出利用 18 个球覆盖的方案,叙述了该方案的设计思路,并严格证明出方案的可行性。针对用最少的红球、蓝球发现并定位黄球的问题,设计出 36 个球的方案,严格证明出方案的可行性。
\end{abstract}

关键词:目标发现;目标定位;椭圆覆盖

\section{问题简介与分析}

在一个半径为 50m,高为 10m 的圆柱体有红、蓝、黄三种小球,底面上放置若干红色和蓝色的球,在圆柱体内部有黄色的球存在并假设它们是静止不动的。当某只红球到黄球再到某只蓝球的距离之和小于等于 40m 的时候,就认为该黄球能被这对红球、蓝球所发现。首先要求给出使红球和蓝球尽可能少的放置方案,能够发现放在圆柱体内任何位置的黄球。进一步,为了对黄球进行定位,还需要增加红球和蓝球的数量,要求给出可以对黄球定位的红球和蓝球的优化放置方案。

先考虑一对红球和蓝球能发现黄球的区域。在三维空间中,到红、蓝两个球距离之和小于等于 40m 的点形成以这两个球为焦点、长轴为 40m 的椭球体。当黄球处于椭球体内部时,它能被该对红球、蓝球所发现。一对红球、蓝球在某一水平高度上能够探测黄球的几何形状是椭球的一个水平截面即椭圆,而且随着高度的增加,该椭圆逐渐缩小。

\textbf{定义 1} 一对红球、蓝球形成的椭球体与圆柱体顶面的截面称为这对球的覆盖域。

显然,在圆柱体内部以某个覆盖域为顶面的椭圆柱形区域内的黄球都可以被这对红球、蓝球所发现。因此,如果圆柱体顶面上的每一点都属于一对红球、蓝球的覆盖域,则圆柱体内每一个黄球都能被发现。

先考虑底面上只有一对红球和蓝球的情况。我们以圆柱底面圆心为原点,底面为 \(x\)-\(y\) 平面,建立空间坐标系。在 \((p, q, 0)\)、\((-p, q, 0)\) 两点分别放置一个红球和蓝球。当 \(p \neq 0\) 时,这一对红蓝球的覆盖域是一个椭圆,而且该椭圆的长轴平行于 \(x\) 轴,中心在 \(y\) 轴上。容易求得这对红蓝球形成的椭球面的方程为

\[
\frac{x^2}{20^2} + \frac{(y-q)^2}{20^2-p^2} + \frac{z^2}{20^2-p^2} = 1
\]

圆柱体的高为 10m 即顶面 \(z=10\),代入上式得这对球的覆盖域的边界方程为

\begin{equation}
\begin{cases}
\frac{x^2}{20^2} + \frac{(y-q)^2}{20^2-p^2} = 1 - \frac{10^2}{20^2-p^2} \\
z = 10
\end{cases}
\tag{1}
\end{equation}

下面我们分别就发现和定位黄球两个问题给出我们的解决方案。

\section{发现黄球}

本小节我们考虑在这个圆柱体的底面至少要放置多少红球、多少蓝球,又分别放置在什么地方(放置后不能移动),才能使放在圆柱体内任何位置的黄球都能被红球、蓝球发现。

由前文所述,此问题转化为如何用尽可能少的覆盖域来完全覆盖圆柱顶面。由于红球、蓝球可以放在底面上的任意位置,而且每一对间距不同的红球和蓝球的覆盖域的大小不同,因此这是一个复杂的区域覆盖问题。目前这类问题在理论上尚不存在最优解求解的一般方法。下面给出一种启发式的最优放置方案。

我们的思想是利用多个椭圆来覆盖圆柱的顶面。具体过程是先用同心圆将底面分割成若干圆环,各个圆环的直径有待确定。然后在这些圆环的圆周上等间距放置偶数个红蓝相间的球,如图 1 所示。这样同一个圆周上的任意相邻的两个球都能形成一个覆盖域,我们希望所有的覆盖域能完整覆盖圆柱的顶面。需要根据问题的具体情况来确定圆环个数,每个圆环的直径,以及圆环上球的个数。

我们先考虑如何覆盖圆的外圈,因为在相同宽度的圆环中外圈面积比内圈大很多,对圆覆盖的贡献大。在底面以原点为中心,$R_1$ 为半径的圆周上均匀放置红、蓝相隔的球,共 $n$ 个($n$ 是偶数)。相邻的两个球可以生成一个椭球,共得到 $n$ 个椭球,因此在圆柱顶面上产生 $n$ 个覆盖域。这 $n$ 个覆盖域能完整覆盖一个圆环,如图 2 中两条弧线 $\alpha$ 和 $\beta$ 所示。设两个椭圆相交于 $A$、$B$ 两点,则线段 $AB$ 的长度 $L$ 是这个圆环的宽度。我们用这个圆环去覆盖圆柱顶面的外圈部分,因此假设 $B$ 点落在圆柱顶面圆周上,即弧线 $\alpha$ 刚好与顶面圆周重合。两个相邻的球和底面圆心构成的圆心角 $\theta$ 为 $2\pi/n$。假设这两球的连线平行于 $x$ 轴,于是红球、蓝球的坐标分别是 $(-R_1\sin \theta/2, R_1\cos \theta/2)$、$(R_1\sin \theta/2, R_1\cos \theta/2)$。此时 $p = R_1\sin \theta/2$,$q = R_1\cos \theta/2$,代入公式 (1) 得这两个球在顶面的覆盖域的边界方程为(为简单起见,下面均略去 $z=10$)

\begin{equation}
\frac{x^2}{20^2} + \frac{(y - R_1\cos \pi/n)^2}{20^2 - (R_1\sin \pi/n)^2} = 1 - \frac{10^2}{20^2 - (R_1\sin \pi/n)^2}
\tag{2}
\end{equation}

\begin{figure}[h]
\centering
\includegraphics[width=0.4\textwidth]{image1.png}
\caption{图 1}
\end{figure}
\begin{figure}[h]
\centering
\includegraphics[width=0.7\textwidth]{image2.png}
\caption{图 2}
\end{figure}

在图 2 中,这两个椭圆关于 $AB$ 对称。注意到相应的两个椭球共用一个焦点,因此只有当这个焦点在顶面的垂直投影点位于线段 $AB$ 的中点时才能保证对称性。因此点 $B$ 在过底面圆心与此共用焦点且与底面垂直的平面上,即 $B$ 点满足

\begin{equation}
x = y\tan \pi/n
\tag{3}
\end{equation}

另一方面,点 $B$ 位于圆柱的侧面上,因此有

\begin{equation}
x^2 + y^2 = 50^2
\tag{4}
\end{equation}

联立等式 (2)(3)(4),得到一个关于 $R_1$ 与 $n$ 的方程组。

由于 $n$ 只能取偶数,我们依次用可能的 $n$ 值带入上述方程组。利用 MATLAB 计算得出,$n \leq 10$ 时无解;$n \geq 12$ 时有解。当 $n = 12$ 时,圆柱顶面外圆环的覆盖情况如图 3 所示。

为了使所需红蓝球的个数最小,我们就取 $n = 12$。此时 $R_1 = 41.6559$,代入 (2) 得椭圆的方程为

\begin{equation}
\frac{x^2}{259.0371} + \frac{(y - 40.24)^2}{183.7647} = 1
\tag{5}
\end{equation}

下面我们考虑如何覆盖圆柱顶面的其余部分。我们依然是将偶数个球均匀且红蓝相间地放置在一个同心圆的圆周上,假设这个同心圆的半径为 $R_2$,并尝试取球的个数 $m$ 为 $n/2$ 即 6。两个相邻球和下底面圆心构成的圆心角 $\varphi$ 为 $2\pi/m$。仍然假设这两球的连线平行于 $x$ 轴,于是得到红球、蓝球的坐标分别是 $(-R_2\sin \varphi/2, R_2\cos \varphi/2)$、$(R_2\sin \varphi/2, R_2\cos \varphi/2)$。代入公式 (1),得这对红蓝球在顶面生成的覆盖域的边界方程为

\begin{equation}
\frac{x^2}{20^2} + \frac{(y - R_2\cos \pi/m)^2}{20^2 - (R_2\sin \pi/m)^2} = 1 - \frac{10^2}{20^2 - (R_2\sin \pi/m)^2}
\tag{6}
\end{equation}

类似于前面的讨论,相邻两个椭圆的交点在下列平面上

\begin{equation}
x = y\tan \pi/m
\tag{7}
\end{equation}

我们希望此时能够覆盖圆心,即圆心要在所有 $m$ 个覆盖域内,这样我们得到一个关于上面两个椭圆的交点的不等式:

\begin{equation}
(R_{2}\sin\frac{\varphi}{2}-x)^{2}+(R_{2}\cos\frac{\varphi}{2}-y)^{2}\geq R_{2}^{2}
\tag{8}
\end{equation}

与前文类似的做法,由(6)(7)(8)式可以计算出内层同心圆的半径 \(R_{2}\) 最大可以为 17.3205,此时这些覆盖域的边界都经过原点。用此值代入(6)式我们得到椭圆的方程为

\begin{equation}
\frac{x^{2}}{276.9231}+\frac{(y-15)^{2}}{225.0001}=1
\tag{9}
\end{equation}

两个圆环上 18 个红蓝球的布置方案及其覆盖域如图 4 所示。其中球 \(Z\) 的极坐标为 \((R_{1}, \pi/4)\),球 \(X\) 的极坐标为 \((R_{2}, \pi/3)\),其它球在各自的圆环上均匀且红蓝相间地布置。

\begin{figure}[h]
    \centering
    \includegraphics[width=0.45\textwidth]{image1.png}
    \caption{}
    \label{fig:3}
\end{figure}
\begin{figure}[h]
    \centering
    \includegraphics[width=0.45\textwidth]{image2.png}
    \caption{}
    \label{fig:4}
\end{figure}

现在考虑圆柱顶面是否还有未被覆盖的区域。我们来证明不存在这样的区域。由于图形的对称性,我们只考虑椭圆 \(O_{1}、O_{2}、O_{3}、O_{4}、O_{5}\) 组成的区域,如图 4 所示。我们放置内层的球 \(X\) 使得该球位于线段 \(OO_{1}\) 和内层圆环的交点处,这样这一层所有的球就可以按顺序均匀且红蓝相间地布置下去。此时,设外层的两个椭圆相交于 \(A、B\) 两点,则过 \(A、B\) 的直线经过 \(O\) 点,且假设该直线与内层椭圆 \(O_{2}\) 相交于 \(C\)。内层两个相邻的椭圆之间相交于两点,由内层椭圆的方程知道,交点为 \(D\) 和圆心 \(O\)。设过 \(OD\) 的直线与椭圆 \(O_{1}\) 相交于 \(E\) 点。因此,如果点 \(D\) 位于椭圆 \(O_{1}\) 内部,点 \(C\) 位于椭圆 \(O_{4}\) 和 \(O_{1}\) 内部,则区域 \(OO_{1}PO_{4}\) 能够被全部覆盖,然后由图形的对称性就可以知道所有的覆盖域完全覆盖了整个上表面。

下面我们来证明点 \(D\) 位于椭圆 \(O_{1}\) 内部和点 \(C\) 位于椭圆 \(O_{4}\) 和 \(O_{1}\) 内部。为了证明这一点,我们只需证明线段长度 \(OD \geq OE\),\(OC \geq OA\)。由于 \(\angle O_{4}OB = \pi/12\),\(\angle O_{4}OO_{1} = \pi/6\),这样我们可以利用上述的椭圆方程 (5)(9) 和直线 \(y = x \cot(\pi/12)\)、\(y = x \cot(\pi/6)\) 相交的交点到圆心的距离来求 \(OD\)、\(OE\)、\(OA\)、\(OC\)。利用 MATLAB 进行计算得到的数据如下:

\begin{table}[h]
\centering
\begin{tabular}{|c|c|c|c|}
\hline
\(OD\) & \(OE\) & \(OA\) & \(OC\) \\ \hline
27.2585 & 26.6805 & 29.2741 & 29.3464 \\ \hline
\end{tabular}
\end{table}

因此,\(OD > OE\),\(OC > OA\),所以我们证明了我们的结论,即利用 18 个球(9 个红球和 9 个蓝球)就能够使放在圆柱体内任何位置的黄球都有可能被红球、蓝球发现。

\section{定位黄球}

如果增加一个条件:分别以过红球或蓝球(可以将它们看成质点)的两条直线为轴,以红球、蓝球为顶点作两个圆锥,圆锥轴截面的顶角均为 4 度。黄球(直径 2mm)至少有一部分位于上述两个圆锥的交集中(第一问中 40m 的条件仍旧要满足),就认为红球、蓝球发现了黄球并知道了从红球到黄球中心再到蓝球的距离。当然这时还无法给出黄球的准确定位,但是对同一个黄球,如果存在几对符合上述条件的红球、蓝球,就可以为黄球定位。现在要给固定在圆柱体内任意位置的黄球定位,假设以红球、蓝球为顶点的每个圆锥的轴可以取任意位置,即为一只黄球定位时取某个方向,为另一只黄球定位时又可以取另外的方向。为此至少需要红球、蓝球各多少个?红球、蓝球又应如何放置在圆柱体的底面?

黄球的直径是 \(2 \, \text{mm}\),相对圆柱体大小来说可以把它看作一个点,不考虑大小。当黄球位于红球、蓝球生成的两个圆锥的交集中而且满足距离和小于 \(40 \, \text{m}\) 的约束,黄球被红球和蓝球发现。在第 2 节中我们用红球、蓝球生成的椭球体中每一点,都位于以过这个点和红球或蓝球的直线为轴生成的圆锥的交集中。而位于椭球体以外的点,虽然可能处于两圆锥的交集中,但是显然距离红球、蓝球距离之和大于 \(40 \, \text{m}\),因此不能被发现。由上可知,在这样的条件下,红球、蓝球能发现黄球的总的范围并没有变。变化的是当两个圆锥的轴方向确定后,在圆锥交集且距离和小于 \(40 \, \text{m}\) 的区域内的点都能被发现,即每次红球、蓝球能发现位于一个范围内的黄球,而不是仅一个点位置上的黄球。

下面研究如何为黄球定位。当有一对红球、蓝球发现黄球时,它们知道从红球到黄球再到蓝球的距离和,由这个距离和可确定一个椭球体:焦点是红球和蓝球,长轴是这个距离和,而黄球位于椭球体的表面上。当有两对红球、蓝球发现同一个黄球时,它们可以确定黄球位于两个椭球体表面的交集上,即一条封闭曲线。当再增加一对红球、蓝球时,可以确定黄球在这条封闭曲线上位于哪一点,即完成了对黄球的准确定位。注意到,这里需要的三对红球、蓝球不一定指总共需要 6 个球,因为一个球可以同时属于两个或更多的配对中。例如一个红球、三个蓝球或三个红球、一个蓝球就是三对,两个红球、两个蓝球至少可以组成三对,只要同种颜色的球分别处于不同的位置。

我们在第 2 节提出的解决方案的基础上增加一些红球、蓝球,使得圆柱体顶面的每一点都能被三对或更多的红球、蓝球发现,从而整个圆柱体内的点都能被定位。

我们设计了如下的红球、蓝球放置方案。以原点为圆心,\(R_1 = 41.6559 \, \text{m}\) 为半径的圆周上均匀放置 12 个红球和 12 个蓝球,颜色顺序是红-红-蓝-蓝-红-红-蓝……;\(R_2 = 17.3205 \, \text{m}\) 为半径的圆周上均匀放置 6 个红球和 6 个蓝球,颜色顺序也是红-红-蓝-蓝-红-红-蓝……。内圈的放置位置和外圈有一定关系。如图 5 所示,分别将外圈、内圈的球编号,我们的放置关系是:
- 球 1 是红球,球 2、3 是蓝球;
- 球 A 是红球,球 B、C 是蓝球;
- 球 1、球 A 和圆心位于同一条直线上。

\begin{figure}[h]
    \centering
    \includegraphics[width=0.45\textwidth]{image1.png}
    \caption{}
    \label{fig:5}
\end{figure}
\begin{figure}[h]
    \centering
    \includegraphics[width=0.45\textwidth]{image2.png}
    \caption{}
    \label{fig:6}
\end{figure}

下面我们证明这个布局能够保证圆柱体内任意位置的黄球都能被定位。事实上,如果在上顶面上的任一点同时处于3 个椭圆的交集中,则该点能被3 对红球、蓝球发现,从而能被定位。考虑我们给出来的布置方案,显然可以保证空间上的每一点被两重覆盖。原因如下:现在的方案是在发现黄球的布置方案的基础上增加了18 个点,且这18 个点的分布方式跟原来18 个点的分布方式是相同的(只是相对圆心旋转了15度),从而新的18 个点又能够形成一重覆盖,这样就证明了空间每个点已被两重覆盖。然后再考虑由原来的18 个点和增加的点之间能够覆盖的区域。就如图5 中外圈上的球1 和球2,球3 和球4……内圈上的球A 和球B,球C 和球D 等等总共18 对相互之间能覆盖的区域。由于现在每一对蓝球和红球之间的距离变小了,因此由椭圆方程知,每一对在上表面所形成的椭圆的长轴和短轴都相应的变大了,利用第2 节的证明方法,我们容易证明这18 对蓝球和红球能够覆盖的区域可以覆盖圆柱顶面。这样,顶面达到了“三重”覆盖。因此,利用总共36 个红球和蓝球并按照我们上述的布置方式,圆柱体内的任意一点都能被定位。

\section{总结}
本文对发现和定位目标的优化问题进行抽象,建立了用椭圆覆盖圆形区域的数学模型。利用启发式的方式给出了两个问题的优化解决方案,并简要证明了该方法的正确性。

\section{A Model for Detecting and Locating a Space Object}

\textbf{Li Cheng-fa$^1$, Guo Zhao-hui$^1$, Xue Wei-li$^2$}

(1. Department of Computer Science and Technology, Nanjing University, Nanjing 210093)

(2. Department of Mathematics, Nanjing University, Nanjing 210093)

\textbf{Abstract:} To solve the optimization program of detecting a yellow ball using minimum red and blue balls in a cylinder, the paper proposes a way of using multiple ellipses covering the top of the cylinder. We devise a method which only needs 18 balls, describe the design process in detail, and strictly prove its feasibility. Furthermore, to solve the optimization program of locating the yellow ball using minimum red and blue balls, we devise a scheme which needs 36 balls, and prove its correctness too.

\textbf{Keywords:} object detecting; object locating; elliptic overlay