\section{全国首届部分高校研究生数模竞赛}

\begin{center}
\includegraphics[width=0.3\textwidth]{image.png}
\end{center}

\title{目标定位最优布局的研究}

\begin{abstract}

本文对题目的前三问进行了仔细的研究,获得了令人满意的结果;在此基础上分析了后面的问题。本文将第一问转化为一个椭球体的截面对圆的覆盖问题,利用“逐层收缩”方案,给出了一个可计算的较优的结果;通过对逐层收缩方案的调整,获得了最优解:18 个球(9 个红球和 9 个蓝球)。本文将第二问转化为圆的三重覆盖问题,建立“球均定位能力”模型证明了一个红(蓝)球周围有 4 个蓝(红)球这种模式具有最大的球均定位能力,在此基础上给出红、蓝球的一个布局:36 个球(18 个红球和 18 个蓝球)。对第三问,在合理假设的基础上,定义了对运动黄球定位的概率,并将它转化为圆的弦与圆内两个以上“定位圆盘”相交的概率,通过数值模拟计算得到:在不增加球的时候,定位直线运动的黄球的概率是 0.00878946。最后,分析了影响所需红蓝球数目的因素和提高黄球被发现、定位的概率及速度的因素。在建模过程中,对主要的结果在 Mathematica 平台上进行了仿真验证。

\end{abstract}

\section{问题的背景}

经过查找资料和分析,我们认为本题目的背景为多基雷达布站。红球和蓝球分别代表发射站和接收站,黄球是雷达要检测和定位的目标,比如说飞机。本问题是雷达在圆柱形区域中进行飞机等目标的探测、定位的数学模型,将问题中的数据放大 1000 倍后就是真实的问题:要检测、定位的区域是半径为 \(50 \, \text{Km}\)、高度为 \(10 \, \text{Km}\) 的圆柱;飞行的速度范围是 \(150 \, \text{m/s}\)(二战时螺旋桨飞机的速度,0.44 倍音速)- \(1020 \, \text{m/s}\)(目前很先进的战斗机的速度,三倍的音速);双基雷达的检测中目标与发射站和接收站的距离之和的最大值为 \(40 \, \text{Km}\)。本题的目的就是研究如何在这样的条件下进行雷达布站以及雷达运行的设计,并计算定位的概率。

本题是很有难度的,问题也是层进式的。下面就按照题目中的顺序逐个分析求解。

\section{问题一的分析和计算}

设在半径为 \(50 \, \text{m}\)、高为 \(10 \, \text{m}\) 的圆柱体内有红、蓝、黄三种小球,若一只红球到某只黄球再到一只蓝球的距离小于等于 \(40 \, \text{m}\) 才认为该黄球有可能被这对红、蓝球发现,试问在这个圆柱体的底面至少要放置多少红球、多少蓝球,又它们分别放置在什么地方(设放置后不能移动),才能使放在圆柱体内任何位置(距离底面不足 \(0.1 \, \text{m}\) 不予考虑)的黄球都有可能被红球、蓝球发现。

定义 1. 红球 \(R\) 和蓝球 \(B\) 的探测范围:给定红球 \(R\) 和蓝球 \(B\) 后,他们所能探测的所有空间点形成的区域。按题目中所给的要求,是一个以红球 \(R\) 和蓝球 \(B\) 为焦点、以长轴长为 \(L (=40 \, \text{m})\) 的椭球体(实际上是上半个椭球)。

\begin{figure}[h]
    \centering
    \includegraphics[width=0.8\textwidth]{image.png}
    \caption{检测范围}
\end{figure}

当图 2 中圆柱体底面(阴影部分所示)中放置大量的红球和蓝球的时候,它们中的某些红、蓝球对可能形成一个形如椭球体的探测范围。问题一的意思是最少需要多少个红(蓝)球?如何放置?使得它们所形成的椭圆能完全覆盖图 2 中半径 \( R=50 \) 米,高 \( H \) 为 10 米的圆柱体。下面先给出一个定理。

\begin{figure}[h]
    \centering
    \includegraphics[width=0.7\textwidth]{cylinder_diagram.png}
    \caption{圆柱体示意图}
    \label{fig:cylinder}
\end{figure}

\textbf{定理一.} 如图 2 所示,圆柱体底面(如图中阴影部分所示)上给定有限 \( I \) 个红球和 \( J \) 个蓝球(看作质点),任意一个红球 \( R_i, i=1,2,\cdots,I \) 和蓝球 \( B_j, j=1,2,\cdots,J \),如果它们可以形成一个以 \( L=40 \) 米为长轴的椭球,那么这个椭球被高为 \( h \) 的圆柱的上表面截得部分在底面的投影记作 \( S_{i,j}^h \),否则记做 \( S_{i,j}^h = \varnothing \)。设

\[
0.1 = h_0 < h_1 \leq h_2 \leq H
\]

则有

\[
\bigcup_{i,j} S_{i,j}^{h_2} \subseteq \bigcup_{i,j} S_{i,j}^{h_1}
\]

\textbf{证明:}

记 \( O(R) \) 为半径为 \( R \) 的圆盘,即

\[
O(R) = \left\{ (x,y) : x^2 + y^2 \leq R^2 \right\}
\]

设第 $i$ 个红球的座标记做 $(x_{i}^{R}, y_{i}^{R})$,第 $j$ 个蓝球的座标记做 $(x_{i}^{B}, y_{i}^{B})$;第 $i$ 个红球和第 $j$ 个蓝球如上述形成的椭球体与半径的截面在底面上的投影 $S_{i,j}^{h}$ 可以表示为

\[
S_{i,j}^{h} = O(R) \cap \left\{(x, y) : \sqrt{(x - x_{i}^{R})^{2} + (y - y_{i}^{R})^{2} + h^{2}} + \sqrt{(x - x_{j}^{B})^{2} + (y - y_{j}^{B})^{2} + h^{2}} \leq L \right\}
\]

显然

\[
\begin{aligned}
& \left\{(x, y) : \sqrt{(x - x_{i}^{R})^{2} + (y - y_{i}^{R})^{2} + h_{2}^{2}} + \sqrt{(x - x_{j}^{B})^{2} + (y - y_{j}^{B})^{2} + h_{2}^{2}} \leq L \right\} \\
& \subseteq \left\{(x, y) : \sqrt{(x - x_{i}^{R})^{2} + (y - y_{i}^{R})^{2} + h_{1}^{2}} + \sqrt{(x - x_{j}^{B})^{2} + (y - y_{j}^{B})^{2} + h_{1}^{2}} \leq L \right\}
\end{aligned}
\]

所以

\[
S_{i,j}^{h_{2}} \subseteq S_{i,j}^{h_{1}}
\]

由于 I,J 都有限,所以

\[
\bigcup_{i,j} S_{i,j}^{h_{2}} \subseteq \bigcup_{i,j} S_{i,j}^{h_{1}}
\]

证毕。

这个定理说明,这些椭球如果能将圆柱体中高度为 $h$ 的圆面全部覆盖,那么圆柱体中高度小于 $h$ 的面全部覆盖。于是,我们只需要考虑如何覆盖圆柱体的上表面中的点。

当然,这还是一个十分困难的问题。有限点的圆盘覆盖就是一个 NP 完全问题。而本题的本质上是无限个点的椭圆(椭球的截面)覆盖,我们认为很难找出实际上的最优解;即使找出来也很难证明它是最优的。

下面我们利用本题中圆柱的旋转对称性将题目进一步简化。

符号名词说明:

圆柱高 $H=10$ 米;

圆柱底面半径 $R=50$ 米;

黄球被一对红(蓝)球检测到的最大距离和 $L=40$ 米。

\subsection{2.1 问题的第一步简化}

由于圆柱是旋转对称的。直观上分析,我们认为底面上红球和蓝球分布的最优解具有旋转对称性:即,将这个分布旋转任意角度后,分布不变;以下称为问题的对称解。于是,我们有一个猜想:

\textbf{命题 2.} 假设覆盖半径为 $R$ 的圆柱最少需要红球和蓝球共 $N$ 个,这样的覆盖方法不是唯一的,但是一定存在一个对称解。

这个命题我们无法在短期内证明。但是当 $N$ 比较小时,我们验证了结论的正确性。我们假设它是正确的。

在命题 2 的基础上,我们给出一个计算红球和蓝球的最优分布的“逐层收缩”方案:

如图 3 所示,在圆柱底面的同心圆周上(最优半径的计算见后)等间隔放置红球和蓝球各 $k_1$ 个,覆盖圆柱顶面最外围区域;从而将还需要覆盖的区域的缩减为一个半径更小的圆盘的覆盖问题;用类似的方法计算出第二层球的分布以后,考虑到层间红球和蓝球还可以覆盖一部分区域(以下称“耦合效应”)和椭球截面在小圆盘中的部分(以下称“椭圆效应”),第二层球所在的圆周上的数目和半径还可以进一步减小(缩小方案见后);依次类推,层层收缩,直到所有区域都被覆盖。

\begin{figure}[h]
    \centering
    \includegraphics[width=0.7\textwidth]{image1.png}
    \caption{逐层收缩算法}
    \label{fig:1}
\end{figure}

\textbf{【1】 半径为 \( R \) 的圆周覆盖的最优半径和球个数的计算;}

\begin{figure}[h]
    \centering
    \includegraphics[width=\textwidth]{image2.png}
    \caption{第一层覆盖的关系}
    \label{fig:2}
\end{figure}

如图 4 所示,为了减少冗余,相邻两个椭球截面的交点应该落在圆周上(如 \( Q \) 点)。易证,\( P \)、\( B \)、\( M \)、\( O \) 四点共线。设圆桌半径为 \( R \),由 \( 2k \) 个红、蓝球等间距间隔地落在半径为 \( r \) 的圆周上;则圆周上点 \( P \) 满足 \( |AP| + |BP| = L \),于是得到了小球放置的圆周半径 \( r \) 和球的个数所满足的方程:

\begin{equation}
\sqrt{(R-r)^2 + H^2} + \sqrt{(R-r\cos(\frac{Pi}{k}))^2 + (r\sin(\frac{Pi}{k}))^2 + H^2} = L
\tag{3.1}
\end{equation}

当 \(k\) 比较小的时候,方程(1)中 \(r\) 无解,只有当 \(k\) 充分大以后,\(r\) 才有意义的解。不难发现,当 \(k\) 变大后,冗余增加,所以取使得(1)中 \(r\) 有解的最小的 \(k\),同时可求得 \(r\)。

图 5 是在题目所给数据下最外围区域的覆盖,使用 6 各红球和 6 各蓝球间隔放置可以将圆周附近区域覆盖,这样将半径为 \(R^0 = 50\) 的圆盘的覆盖转化为半径为 \(R^1 = 29.28\) 的圆盘的覆盖问题。

\begin{figure}[h]
    \centering
    \includegraphics[width=0.8\textwidth]{image.png}
    \caption{第一层覆盖}
    \label{fig:1st_layer}
\end{figure}

\section{【2】 收缩方案:}

不考虑已经设定位置的红、蓝球的影响;问题转化为小圆盘的覆盖问题。这样只要重复上述步骤就可以对题目所给的数据,经过计算使用两层就可以完全覆盖,如图 6 所示。总共需要 20 个球,红、蓝球各 10 个。这是一个完全可计算的方案。

但是,相邻两层之间红、蓝球之间的耦合作用和前一层覆盖中多出来的部分已经就可以覆盖收缩后的区域了。所以第二层开始,在不考虑耦合计算结果后减去 1 对球;然后将球等间隔放在一定半径的圆周上。如果可行,再减去一对球,…直到用 $(k_{2}-1)$ 对不能覆盖为止;则第二层用 $k_{2}$ 对球覆盖。

\begin{figure}[h]
    \centering
    \includegraphics[width=0.8\textwidth]{image.png}
    \caption{可计算的结果——20 个球的覆盖}
    \label{fig:20_balls_coverage}
\end{figure}

\begin{figure}[h]
    \centering
    \includegraphics[width=0.8\textwidth]{image.png}
    \caption{最优结果——可计算结果的改进}
    \label{fig:optimal_result}
\end{figure}

\section{表1 最优布站点的数据}

\begin{table}[h]
\centering
\begin{tabular}{|c|c|c|c|c|c|}
\hline
n & 坐标 & n & 坐标 & n & 坐标 \\ \hline
1 & (41.6559, 0) & 2 & (36.0751, 20.828) & 3 & (20.828, 36.0751) \\ \hline
4 & (0, 41.6559) & 5 & (-20.828, 36.0751) & 6 & (-36.0751, 20.828) \\ \hline
7 & (-41.6559, 0) & 8 & (-36.0751, -20.828) & 9 & (-20.828, -36.0751) \\ \hline
10 & (0, -41.6559) & 11 & (20.828, -36.0751) & 12 & (36.0751, -20.828) \\ \hline
13 & (16.7298, 4.48275) & 14 & (4.48275, 16.7298) & 15 & (-12.2471, 12.2471) \\ \hline
16 & (-16.7298, -4.48275) & 17 & (-4.48275, -16.7298) & 18 & (12.2471, -12.2471) \\ \hline
\end{tabular}
\end{table}

对题目所给的数据,经过计算使用两层就可以完全覆盖,如图\ref{fig:optimal_result}所示。总共需要18个球,红、蓝球各9个。这就是我们得到的最优方案。表是其布点方式。

已经知道六边形点阵是整个平面上的最优布局。这个事实一直得到人们的公认,但它首次得到证明则是在1892年,由Axel Thue证明的【2】。根据这样一个事实,我们得到了使用19个球覆盖的一种方案。

\begin{figure}[h]
    \centering
    \includegraphics[width=0.8\textwidth]{image.png}
    \caption{19 个球覆盖的一个方案}
    \label{fig:19_balls}
\end{figure}

\section{第二问的分析和计算}

如果增加一个条件。分别以过红球或蓝球(可以将它们看成质点)的两条直线为轴,以红球、蓝球为顶点作两个圆锥,圆锥轴截面的顶角均为 4 度。当黄球(直径 \(2\mathrm{mm}\))至少有一部分位于上述两个圆锥的交集中(第一问中 \(40\mathrm{m}\) 的条件仍旧要满足),就认为红球、蓝球发现了黄球并知道了从红球到黄球中心再到蓝球的距离。当然这时还无法给出黄球的准确定位,但是对同一个黄球,如果存在几对符合上述条件的红球、蓝球,(甚至一个红球、三个蓝球或三个红球、一个蓝球构成的三对)就可以为黄球定位。现在要给固定在圆柱体内任意位置(距离底面不足 \(0.1\mathrm{m}\) 仍不予考虑)的黄球定位,假设以红球、蓝球为顶点的每个圆锥的轴可以取任意位置,即为一只黄球定位时取某个方向,为另一只黄球定位时又可以取另外的方向。为此至少需要红球、蓝球各多少个?红球、蓝球又应如何放置在圆柱体的底面?(仍旧假设红球、蓝球放置后不能移动)

\subsection{对每一个点定位需要 3 个数据}

假设 \( P \) 为空间任意一点,其坐标为 \( (x, y, z) \);我们需要独立的三个方程才能确定 \( (x, y, z) \)。【\( (x_1^B, y_1^B) \), \( (x_1^R, y_1^R) \)】,【\( (x_2^B, y_2^B) \), \( (x_2^R, y_2^R) \)】,【\( (x_3^B, y_3^B) \), \( (x_3^R, y_3^R) \)】为任意三对红、蓝球在圆柱底面上坐标。P 点与每对球的距离的和是已知的,分别为 \( S_1 \)、\( S_2 \)、\( S_3 \);则

\begin{equation}
\begin{cases}
\sqrt{(x - x_1^B)^2 + (y - y_1^B)^2 + z^2} + \sqrt{(x - x_1^R)^2 + (y - y_1^R)^2 + z^2} = S_1 \\
\sqrt{(x - x_2^B)^2 + (y - y_2^B)^2 + z^2} + \sqrt{(x - x_2^R)^2 + (y - y_2^R)^2 + z^2} = S_2 \\
\sqrt{(x - x_3^B)^2 + (y - y_3^B)^2 + z^2} + \sqrt{(x - x_3^R)^2 + (y - y_3^R)^2 + z^2} = S_3
\end{cases}
\tag{3.2}
\end{equation}

如果上述方程是独立的,也就是说只要三对球中没有完全相同的两对或者三对,则上述方程可以解得唯一的位于上半球面的空间点。这个方程等价于求三个椭球体的交点。

第二问相当于将圆用椭球的截面覆盖三次。利用第一问的结论,最多只需要 \( 18 \times 3 = 54 \) 个球就可以实现。这是没有考虑相互之间的耦合,实际上用更少的球就可以实现。于是,问题的关键就是,如何选择好的红、蓝球之间的布站方式。

\subsubsection{3.2 冗余最小的情形}

虽然利用任意三对红、蓝球可以确定一个空间点,但是一个红球对三个蓝球或一个蓝球对三个红球这样的组合效率最高。因此,每个红球周围至少有 3 个蓝球并每个蓝球周围至少有 3 个红球。图 9 中是四种红球和蓝球互相搭配的模式,分别是一个红(蓝)球周围有 3、4、5、6 个蓝(红)球的情形。随着红(蓝)球周围蓝(红)球个数的增加,它们能定位的区域会增加,但是球的个数也变大了。为了衡量每一个球对定位所做的贡献,引入一个概念:

\begin{figure}[h]
    \centering
    \includegraphics[width=\textwidth]{image.png}
    \caption{冗余最小的结构}
    \label{fig:redundancy_min_structure}
\end{figure}

\textbf{定义 1.} 球均定位能力:即红、蓝球集合中平均每一个球对集合中所有球的定位所做的贡献。本文用平均每一个球的平均定位面积来衡量。

假设每一个红(蓝)球周围有 $n$ 个蓝(红)球。为了保证至少有三对红蓝球可以同时定位某个黄球,如图 10 所示由 B、O 所确定的椭球截面必须过 A 点。于是根据题意有:
\[
\sqrt{x^2 + H^2} + \sqrt{\left(2x \sin\left(\frac{\pi}{n}\right)\right)^2 + H^2} = L
\]
其中,$x = |BO|$

\begin{figure}[h]
    \centering
    \includegraphics[width=0.8\textwidth]{image.png}
    \caption{最优结构的选择}
\end{figure}

这样得到这 $n+1$ 个球定位的总为图中的阴影部分的面积。本文用简单的用正多边形 $ABCD\ldots$ 的面积来近似阴影部分的面积。
\[
S_n \doteq \overline{S}_{ABCD\ldots} = nx \cos\left(\frac{\pi}{n}\right) \sin\left(\frac{\pi}{n}\right)
\]

平均每一个球的能力为

\[
E_{n}=\frac{S}{n+1}=\frac{n}{n+1} x \cos \left(\frac{\pi}{n}\right) \sin \left(\frac{\pi}{n}\right)
\]

利用 Mathematica 计算得:

\begin{table}[h]
\centering
\caption{2 球均定位能力}
\begin{tabular}{|c|c|c|c|c|c|c|}
\hline
n & 3 & 4 & 5 & 6 & 7 & 8 \\
\hline
$E_{n}$ & 0.32476 & 0.41 & 0.396274 & 0.371154 & 0.342051 & 0.31427 \\
\hline
\end{tabular}
\end{table}

从表 2 中可见,使用图 9 中 (b) 结构是最优的。所以下面我们采用这种结构去覆盖圆盘。

按照命题,我们寻找对称解。这种结构可以有两种对称覆盖的覆盖方法。就是覆盖中圆盘的中心位于图 11 中 O1 处或者 O2 处(标记 X)。

\begin{figure}[h]
\centering
\includegraphics[width=0.7\textwidth]{image.png}
\caption{最优结构}
\end{figure}

\subsection{3.3 正方形半径的确定}

\begin{figure}[h]
    \centering
    \includegraphics[width=\textwidth]{image.png}
    \caption{正方形}
    \label{fig:12}
\end{figure}

\subsubsection{不考虑圆锥影响时正方形半径的确定}

当正方形半径的取值比较小,满足图 11 所示的情况时,正方形内所有点都至少被覆盖了四层,冗余过多。下面我们讨论能否通过减少冗余,增大正方形的边长,从而减少球的数目。

如图 12(a) 所示,增大正方形的边长,只考虑以(R₁,B₁)、(R₁,B₂)、(R₃,B₁)和(R₃,B₂)为焦点的四个椭球体的投影区域,如图中的四个黑色类椭圆所示。以(R₁,B₂)为焦点的椭球投影将不再包含 B₁ 点和 R₃ 点,而以(B₂,R₃)为焦点的椭球投影也不再包含 B₁ 点和 R₁ 点,这就使得 B₁ 点附近的 A₁ 区域只被所有黑色类椭圆覆盖了 2 层,同理,其他 A 区域也只被黑色类椭圆覆盖了 2 层,但所有 C 区被覆盖 3 层,D 区被覆盖 4 层。

下面考虑要使以(B,R)为焦点的椭球体的投影(绿色类椭圆)能覆盖 A 区,正方形边长所能取得的最大值。显然,边界情况就是绿色椭圆恰能包含 A 区。易证,当绿色类椭圆与正方形的 R₁B₁ 边相交于 M 点(R₁B₁ 和类椭圆 R₂B₂ 的交点)时,也能包含 N 点,所以这时的半径就是正方形半径的上界。事实上,这时(B₁,R₄)类椭圆也能包含 A₁ 区域,从而使 A₁ 区域被覆盖了 4 层。

设正方形边长为 \( x \),\( \mathbf{M R}_{1} \) 为 \( y \),\( H = 10 \) 为圆柱体高度。

由 \( M \) 点在类椭圆 \( (R_{1}, B_{2}) \) 和 \( (R_{2}, B_{1}) \) 上分别可得方程
\[
\sqrt{y^{2} + H^{2}} + \sqrt{y^{2} + x^{2} + H^{2}} = 40
\]
\[
\sqrt{(x - y)^{2} + H^{2}} + \sqrt{(2x - y)^{2} + H^{2}} = 40
\]
联立可解得 \( x = 18.9037 \),\( y = 11.8861 \)。

\subsubsection{3.3.2 考虑圆锥影响时正方形半径的确定}

考虑到是用一个顶角为 \( 4^\circ \) 的圆锥而不是直线进行扫描,所以扫描线落在上底面上的斑点不是点,而是一个类似椭圆的区域,下面考虑它的影响。

当圆锥轴竖直向上时,在上底面的投影是圆形,设其半径为 \( r \)。容易证明,当圆锥以一定角度投向上底面时,其轴线与上底面交点到相交区域外沿的距离大于 \( r \),而且不会大很多。所以为了简化起见,我们在任何角度时将这个距离都取为 \( r \)。且有
\[
r = H \times \tan(2 * \pi / 180) = 10 \times 0.03492 = 0.3492 \, (m)
\]

结合实际情况,发射波投射到飞行物时,将发生漫反射而不是镜面反射,所以,下底面上的球到射线与上表面相交区域任意点的距离都可以看作是球到圆锥轴与上表面交点的距离。

如图 12(b)所示,对于最难覆盖的边界点 \( O \) 取
\[
OM = ON = OP = r
\]

类椭圆 \( (R1, B2) \) 是圆锥轴线与上表面的交线,但圆锥的外沿至少都可以射到 \( O \) 点,从而 \( O \) 点是能被 \( (R_{1}, B_{2}) \) 发现的。可以得到方程:
\[
\sqrt{(MR_{1})^{2} + H^{2}} + \sqrt{(NB_{2})^{2} + H^{2}} = 40
\]

设正方形边长为 \( x \),\( MR1 = y \),则

\begin{equation}
\sqrt{(y)^2 + H^2} + \sqrt{\left(\sqrt{(y + r)^2 + x^2} - r\right)^2 + H^2} = 40
\tag{3.3}
\end{equation}

同理,根据 O 点和类椭圆(B1,R2)的关系,可得方程

\begin{equation}
\sqrt{(PB_1)^2 + H^2} + \sqrt{(PR_2)^2 + H^2} = 40
\end{equation}

即

\begin{equation}
\sqrt{(x - y - 2r)^2 + H^2} + \sqrt{(2x - y - 2r)^2 + H^2} = 40
\tag{3.4}
\end{equation}

联立 (3.4) 和 (3.5),可解得:

\begin{equation}
x = 19.2687, \, y = 11.7796
\end{equation}

\subsection{3.4 边界附近球的分布点的调整}

按照上面的方法布完球以后,第一象限的边界区域如图 13 所示,必须对落在圆外的球 B1,R1 和 B2 进行调整。最简单的方法是将 B1,R1 和 B2 分别沿直线 B1B3,R1R2 和 B2B4 的方向移到圆周上,调整以后三个点的坐标可以通过求直线和圆的交点坐标得到。容易验证这时底面在第一象限内的所有部分至少得被覆盖了三层。根据对称性,其它象限的情况和第一象限类似。

\subsection{3.5 最优的方案}

利用上述结果,可以得到最优的方案为图 14。

\begin{figure}[h]
    \centering
    \includegraphics[width=0.7\textwidth]{image1.png}
    \caption{边界点的调整}
    \label{fig:boundary_adjustment}
\end{figure}

\begin{figure}[h]
    \centering
    \includegraphics[width=0.9\textwidth]{image2.png}
    \caption{用来定位红、蓝球分布图}
    \label{fig:red_blue_distribution}
\end{figure}

\begin{table}
\centering
\caption{用来定位红、蓝球的坐标}
\begin{tabular}{|c|c|c|c|c|c|c|c|}
\hline
n & 坐标 & n & 坐标 & n & 坐标 & n & 坐标 \\
\hline
1 & (9.63, 9.63) & 2 & (-9.63, 9.63) & 3 & (-9.63, -9.63) & 4 & (9.63, -9.63) \\
\hline
5 & (28.89, -9.63) & 6 & (28.89, 9.63) & 7 & (28.89, 28.89) & 8 & (9.63, 28.89) \\
\hline
9 & (-9.63, 28.89) & 10 & (-28.89, 28.89) & 11 & (-28.89, 9.63) & 12 & (-28.89, -9.63) \\
\hline
13 & (-28.89, -28.89) & 14 & (-9.63, -28.89) & 15 & (9.63, -28.89) & 16 & (28.89, -28.89) \\
\hline
17 & (43.648, -24.388) & 18 & (48.15, -9.63) & 19 & (48.15, 9.63) & 20 & (43.648, 24.388) \\
\hline
21 & (35.355, 35.355) & 22 & (24.388, 43.648) & 23 & (9.63, 48.15) & 24 & (-9.63, 48.15) \\
\hline
25 & (-24.388, 43.648) & 26 & (-35.355, 35.355) & 27 & (-43.648, 24.388) & 28 & (-48.15, 9.63) \\
\hline
29 & (-48.15, -9.63) & 30 & (-43.649, -24.389) & 31 & (-35.355, -35.355) & 32 & (-24.389, -43.649) \\
\hline
33 & (-9.63, -48.15) & 34 & (9.63, -48.15) & 35 & (24.389, -43.649) & 36 & (35.355, -35.3553) \\
\hline
\end{tabular}
\end{table}

\section{第三个问题的分析与计算}

假设黄球从圆柱体外任意位置进入圆柱体,并且在圆柱体内可以随意移动(仍不会到达距离底面不足 0.1m 的地方),移动速度的范围是 0.15-1.02m/s。这时给黄球定位显然更困难。现在假设以红球、蓝球为顶点的圆锥的轴也是可以随意旋转的,旋转的角速度最大为 \(60^\circ/s\)。请你们决定为了给作直线运动的黄球定位是否需要增加红球、蓝球的数量,是否需要改变红球、蓝球在圆柱底面的放置位置(仍旧假设红球、蓝球放置后不能移动),并为每个红球、蓝球所在的圆锥轴制定旋转方案。请你们为运动的黄球被定位的概率下个定义,再根据此定义计算你们方案的定位概率。

\subsection{假设}

(1) 任意时刻圆柱体内只有一个做直线运动的黄球;

(2) 黄球从圆柱的侧面进入、并从侧面穿出,且沿直线运动;

(3) 黄球在水平面内运动,因为该圆柱底面半径与高相比非常大(如图 2),为了保证黄球运动所在的直线不穿过圆柱的底面和顶面,黄球应该在几乎平行于水平面的直线上运动。

(4) 根据实际情况,假设黄球飞入点在竖直方向满足以它可能飞入高度(0.1m~10m)的中点(5.05m)为中心的正态分布。

\subsection{4.2 红蓝球的数目及其分布的确定}

我们认为,给作直线运动的黄球定位不需要增加红球和蓝球的数量也不需要改变其在圆柱底面的位置。这是因为在上一问的解答过程中已经保证了圆柱体内任何一个位置都可以被定位并且有一定的冗余。

\subsection{4.3 圆锥旋转方案的确定}

设定所有非边缘的圆锥在旋转时,其轴线与竖直方向的夹角都为 $\theta$。它们的轴线在水平面上投影的旋转角速度都为题设给定的最大值 $60^\circ/s$。

\subsubsection{4.3.1 $\theta$ 的确定}

\paragraph{1、$\theta$ 的最优值应满足的条件}

当 $\theta$ 确定时,如图 16 所示,如果黄球从 MN 范围以外的高度飞入,是不可能被定位的。

\begin{figure}[h]
    \centering
    \includegraphics[width=\textwidth]{image.png}
    \caption{定位圆盘的半径}
\end{figure}

$\theta$ 的取值将决定黄球从 MN 范围内飞入的概率为 $P_a$。根据假定,黄球的飞入位置在竖直方向服从以 P 点为中心的正态分布,根据正态分布的性质,要使 $P_a$ 尽量大,应该让覆盖同一区域的几个圆锥的轴线的交点落在 P 点所在的水平面上。

\paragraph{2、$\theta$ 的最优值的计算}

\begin{figure}[h]
    \centering
    \includegraphics[width=0.6\textwidth]{image.png}
    \caption{最大定位区域示意图}
    \label{fig:16}
\end{figure}

每一个圆锥旋转一周后与任一水平面的相交区域是一个圆环,圆环的半径以及宽度都是水平面高度 \( h \) 和 \( \theta \) 的函数。能定位的区域需要满足的条件是,在某一时刻,同时有至少三对红蓝球的圆锥都包含这个区域。只有当包含同一区域的两对红蓝点在同一水平面上的投影圆环满足如图 \ref{fig:16} 所示的分布时,可能被定位的区域将最大。图 \ref{fig:16} 中的 \( R \) 即为图 15 中的 \( R \),满足

\[
R = \text{红蓝球间距} \times \cos\left(\frac{\pi}{4}\right) \approx 19.2 \times \frac{\sqrt{2}}{2} \approx 13.5765
\]

从而

\[
\theta = \arctan\left(\frac{\text{PQ}}{\text{OA}}\right) = \arctan\left(\frac{h}{R}\right) = \arctan\left(\frac{13.5764}{5.05}\right) = 69.596^\circ = 1.215 \, \text{rad}
\]

\subsubsection{h\(_1\) 和 h\(_2\) 的计算}

ET 和 TF 的长度 \( h_1 \) 和 \( h_2 \) 分别为

\[
h_1 = ET = OE - OT = R \times \tan\left(\frac{\pi}{2} - \theta + \frac{\alpha}{2}\right) - h = 0.548071
\]

\[
h_1 = TF = OT - OF = h - R \times \tan\left(\frac{\pi}{2} - \theta - \frac{\alpha}{2}\right) = 0.534058
\]

其中,\( \alpha = 4^\circ = \frac{\pi}{45} \, \text{rad} \) 为圆锥轴截面的顶角。

\subsubsection{4.3.3 各个圆锥旋转相位的确定}

各个圆锥旋转的相位必须以特定的方式配合,才能使每个球兼顾它周围四个可能定位区域,使它们成为不同的时刻的可定位区域。

根据对称性分析可得,满足上述条件的一种相位分布如图 17 所示,所有以蓝球为顶点的圆锥沿俯视顺时针方向旋转;所有以红球为顶点的圆锥都沿俯视逆时针方向旋转。在 $nT$, $(n+1/4)T$, $(n+1/2)T$ 和 $(n+3/4)T$ ($n \in Z, T = \frac{2\pi}{\omega}$) 时刻,以各球为顶点的圆锥的轴线在底面上的投影分别如橙色、黄色、绿色和紫色线条所示。

\begin{figure}[h]
    \centering
    \includegraphics[width=0.8\textwidth]{image.png}
    \caption{圆锥旋转的安排}
    \label{fig:17}
\end{figure}

\subsubsection{4.3.4 对边界区域内圆锥的讨论}

以上的讨论都是针对内部区域的非边界圆锥,对于边界附近的圆锥,它们不是分布在正方形的一个顶点上,为了和内部其他点配合,它们的旋转轴和竖直方向的夹角 $\theta$ 以及在水平面内的角速度 $\sigma$ 都要发生变化,但总能找到何时的 $\theta$ 和 $\sigma$,使得外围圆锥能和内部圆锥配合形成可定位区域。

\subsection{4.3 直线运动的黄球被定位概率的定义及计算}

根据我们的假设,黄球是在一个同一个高度(也就是平行于圆柱底面的面)上沿直线运动的。

直线运动的黄球被定位的概率是指能够确定黄球运动所在直线的概率。也就是任意高度飞行的黄球沿着圆柱上任意一点、任意方向沿着直线飞入圆柱后至少有两次同时被三对球定位的概率。

在上述的旋转方案下,我们计算运动的球被定位的概率。假设黄球的飞行高度为 $h$,如图 15 可以证明,只有当 $h \in \left[ \frac{H+0.1}{2} - h_2, \frac{H+0.1}{2} - h_1 \right]$ 的时候才可能有三对红(蓝)球在空间中同一区域,此时形成圆盘内对小球能定位的区域(如图 16 中至少有三个圆环交迭的区域),这个区域我们用直径等于圆环的内直径的圆形近似,以下称为定位圆盘。定位圆盘的半径 $r_d$ 与黄球的飞行高度有关

\[
r_d = \frac{\sqrt{2}}{2} \left( Tan(\theta) - Tan\left(\theta + \frac{\pi}{45}\right) \right) h
\]

显然,定位圆盘的圆心对 $h \in \left[ \frac{H+0.1}{2} - h_2, \frac{H+0.1}{2} - h_1 \right]$ 都是相同的,可计算得到定位圆盘的圆心分别为:

\[
(0, 0), (0, 19.26), (0, 38.52), (0, -19.26), (0, -38.52),
\]
\[
(19.26, -38.52), (19.26, -19.26), (19.26, 0), (19.26, 19.26),
\]
\[
(19.26, 38.52), (38.52, 19.26), (38.52, 0), (38.52, -19.26),
\]
\[
(-19.26, -38.52), (-19.26, -19.26), (-19.26, 0), (-19.26, 19.26),
\]
\[
(-19.26, 38.52), (-38.52, 19.26), (-38.52, 0), (-38.52, -19.26)
\]

\begin{figure}[h]
    \centering
    \includegraphics[width=0.8\textwidth]{image.png}
    \caption{可定位区域完全分布图}
    \label{fig:18}
\end{figure}

由于要定位的黄球的运动速度与红(蓝)球的转动速度相比非常慢;而且黄球沿直线运动的速度只有一个范围。所以,我们假设定位直线需要的两个点是完全独立的。

所以,我们可以将得到黄球在高度为 \( h \in \left[ \frac{H+0.1}{2} - h_2, \frac{H+0.1}{2} - h_1 \right] \) 的情况下直线运动被定位的概率计算的数学模型:圆的任意一条弦两次与圆中定位圆盘相交的概率。

这个概率是比较难以计算,我们用数值方法模拟。从上图所知,我们的方案是对称的,我们只需要计算八分之一个圆周即可。如图所示,假设黄球从 P 点进入圆柱 SPM 为圆在 P 点处的切线,PT 和 PU 是黄球运动直线中的两条。令 \(\alpha = \angle POA\),\(\beta = \angle SPT\) 或 \(\beta = \angle SPU\),显然

\[
\alpha \in \left[ 0, \frac{\pi}{4} \right], \quad \beta \in \left[ 0, \pi \right]
\]

由于任意直线都是可能的,我们取有限个 \(\alpha\) 和 \(\beta\) 的值。通过统计其中两次与定位圆盘相交的频率来近似其概率。

\begin{figure}[h]
    \centering
    \includegraphics[width=\textwidth]{image.png}
    \caption{直线的确定}
    \label{fig:line_determination}
\end{figure}

任给 $\bar{\alpha} \in \left[0, \frac{\pi}{4}\right]$, $\beta \in [0, \pi]$, 利用平面几何知识有 P 点坐标是 $(R \cos \alpha, R \sin \alpha)$,

\[
\gamma =
\begin{cases}
\alpha + \beta + \frac{\pi}{2}, & \beta > \frac{\pi}{2} \\
\alpha + \beta - \frac{\pi}{2}, & \beta \leq \frac{\pi}{2}
\end{cases}
\]

而直线的斜率就是 $k = \tan(\gamma)$, 于是利用直线的点斜式方程可以写出方程的解析表达式

\[
y - R \sin \alpha = k(x - R \cos \alpha)
\]

整理得

\[
-kx + y + (kR \cos \alpha - R \sin \alpha) = 0
\]

与直线的标准方程

\[
Ax + By + C = 0
\]

比较,有

\begin{equation}
\begin{cases}
A = -k \\
B = 1 \\
C = kR\cos\alpha - R\sin\alpha
\end{cases}
\end{equation}

利用点 $(x, y)$ 与直线 $Ax + By + C = 0$ 的距离公式

\begin{equation}
d = \frac{|Ax + By + C|}{\sqrt{A^2 + B^2}}
\end{equation}

可以计算得到任意一条直线与圆内的定位圆盘的圆心的距离,如果距离小于定位圆盘的半径,则与直线相交。如果有两个以上的定位圆盘与直线相交,则沿着这条直线运动的黄球是可以被定位的。统计可定位的直线的频数即为当前高度的圆内运动的黄球定位的概率。

如表 4,可以计算得到五个不同高度 $h$ 沿直线运动的黄球被定位的概率 $P_h$。

\begin{table}[h]
\centering
\caption{5 个高度的数值近似概率}
\begin{tabular}{|c|c|c|c|c|c|}
\hline
$h$ & 4.516 & 4.8 & 5.05 & 5.32403 & 5.59807 \\
\hline
$P_h$ & 0 & 0.1281 & 0.1446 & 0.0425 & 0 \\
\hline
\end{tabular}
\end{table}

由于这个概率是连续变化的,将上述数据用三次 B 样条拟合【1】可以得到 $P_h$ 随着 $h$ 连续变化的曲线

\begin{figure}[h]
\centering
\includegraphics[width=0.6\textwidth]{image.png}
\caption{概率数据的样条拟合}
\end{figure}

三次样条的表达式是

\begin{align*}
& 81.3537 - 55.0924x + 12.3155x^2 - 0.909026x^3, \quad x \in [4.516, 4.8] \\
& 60.0893 - 41.8022x + 9.5467x^2 - 0.716749x^3, \quad x \in [4.8, 5.05] \\
& -382.394 + 221.059x - 42.505x^2 + 2.71901x^3, \quad x \in [5.05, 5.32403] \\
& 197.416 - 105.654x + 18.8606x^2 - 1.12304x^3, \quad x \in [5.32403, 5.59807]
\end{align*}

将上述函数在区间上积分得到为,0.0870156

假设黄球在各个可能的高度上出现的概率是相等的,则可以黄球被定位的概率是

\[
\frac{0.0870156}{10 - 0.1} = 0.00878946
\]

即,黄球被定位的概率是 0.00878946

\section{5. 第四、七个问题的分析与计算}

发现目标前,红、蓝球无法预测黄球的飞行轨迹,因此发现目标的概率就是目标被某一对红球、蓝球探测到的概率。

当目标被发现后,我们可以确定当时目标在以发现它的红、蓝球为焦点椭球面上(椭球的长轴长是已知的)。这时候,为了对它定位,计算机应该控制红球和蓝球对尽可能的包围发现它的空间区域。

第四问和第七问的主要区别在于,第四问寻找的是能增大黄球被定位的概率的控制策略,而第七问寻求的是能加快黄球被定位的速度的控制策略。

当一对红球和蓝球发现一个黄球时,这个黄球一定位于以这对红蓝球为焦点,长轴为 40m 的上半椭球体内。由于只有一对红蓝球,而且两个圆锥的轴的倾角信息是未知的,所以,黄球可能在半椭球体内高度高于 0.1m 的任意位置。

对于第四问,只要半椭球未超出圆柱体,则黄球必定会从半椭球面高度高于 0.1 的某一点穿出。如果这时计算机控制其它红蓝球对,使所有红蓝球对在半椭球表面附近区域的总的覆盖面达到最大,则会增加黄球被发现、定位的概率。

当红球和蓝球的总个数增加到 190 或更多时,红球和蓝球之间的平均距离将减小到 4 米以内,每对红球和蓝球所能覆盖的椭球将接近半球,体积增大;而每个圆锥需要扫描的范围也减小,相当于增加了扫描速度;这两方面因素都有益于对黄球的定位。

对于第七问,除了控制位于椭球外的红蓝球对的发现区域分布在半椭球表面附近,以增加黄球被定位的概率以外,还应控制位于半椭球内的红蓝球对在椭球体内快速扫描,以便快速找到更多能发现黄球的红蓝球对,加快对黄球的定位。

\section{6. 最后几个问题的分析和讨论}

\subsection{6.1 圆柱体的底面有起伏时的影响}

在问题的第五问中说:如果圆柱体的底面有一些起伏(位置、高度可自定),而红球、蓝球到黄球的连线穿过底面则无效,即认为不符合第二问中关于黄球被发现、被定位的条件。那么这些起伏对仍需要放置在圆柱体底面的红球、蓝球的个数有无影响?进行适当的讨论。

如果圆柱的底面有起伏,对应于实际情况。相当于地面存在高程起伏,此时的地面雷达在发送或者接受时,就有可能受阻。也就是说,当本来黄球出现后可以定位,但是由于不能穿过底面(相当于电磁波无法穿透障碍物)而导致无法定位,如图所示:

\begin{figure}[h]
    \centering
    \includegraphics[width=0.8\textwidth]{image.png}
    \caption{底面起伏对红蓝球分布的影响示意图}
    \label{fig:21}
\end{figure}

从图上可以看出,当底面有起伏的时候,会影响到定位,并且随着起伏的位置和高度的不同,影响的情况也不同,但是对与红球、蓝球的数目,简单定性分析后有以下结论:(1) 当起伏程度很小,并且分布在底面边界的时候,不会影响红球、蓝球的数目;(2) 起伏比较厉害,分布在边界上的话不影响红球、蓝球数目,但当分布在非边界区域时,由于遮挡的原因,需要增加红、蓝球的数目。

\subsection{增加方向信息和颜色可变的影响}

在问题的第六问中说:如果在第二问中红球、蓝球发现黄球时不但知道从红球到黄球中心再到蓝球的距离,而且同时知道红球、蓝球所在圆锥轴的准确的方向,这一点对黄球的定位有什么影响?如果计算机可以在一毫秒的时间内改变全部红球或蓝球中任意一只或多只球的颜色对于黄球被发现、被定位又有什么影响?

首先分析第一问题,如果知道红球、蓝球发现黄球时从红球到黄球中心再到蓝球的距离以及红球、蓝球所在圆锥轴的准确的方向,则可以得到如下的定位方程:

\begin{equation}
\left\{
\begin{aligned}
\rho &= \sqrt{(x-x_b)^2 + (y-y_b)^2 + (z-z_b)^2} + \sqrt{(x-x_r)^2 + (y-y_r)^2 + (z-z_r)^2} \\
\varphi_r &= arctg\frac{y-y_r}{x-x_r} \\
\varphi_b &= arctg\frac{y-y_b}{x-x_b}
\end{aligned}
\right.
\tag{6.1}
\end{equation}

其中 $\rho$ 为红球到黄球中心再到蓝球的距离,$\varphi_r$、$\varphi_b$ 分别为红球、蓝球所在圆锥轴的准确的方向。$(x, y, z)$, $(x_b, y_b, z_b)$, $(x_r, y_r, z_r)$ 分别代表黄球、蓝球和红球的空间坐标。由上式可解得

\begin{equation}
\left\{
\begin{aligned}
x &= \frac{(y_b - y_r) + x_r \cdot tg\varphi_r - x_b \cdot tg\varphi_b}{tg\varphi_r - tg\varphi_b} \\
y &= y_r + (x - x_r) \cdot tg\varphi_r
\end{aligned}
\right.
\end{equation}

将此式代入 $\rho$ 表达式化简、整理得

\begin{equation}
(4\rho^2 - n^2) \cdot z^2 - 2(m \cdot n + 4\rho^2 \cdot z_r) \cdot z + [4\rho^2 \cdot (d_1^2 + d_2^2) - m^2] = 0
\tag{6.2}
\end{equation}

\begin{equation}
\text{式中}
\begin{cases}
d_{1}^{2} = (x - x_{r})^{2} + (y - y_{r})^{2} \\
d_{2}^{2} = (x - x_{b})^{2} + (y - y_{b})^{2} \\
m = \rho^{2} + (d_{2}^{2} + z_{r}^{2}) \bullet (d_{1}^{2} + z_{b}^{2}) \\
n = 2 \bullet (z_{b} - z_{r})
\end{cases}
\end{equation}

由式(6.2)可解得两个 $z$ 值,因此在垂直方向上存在定位模糊,得到的解只能有如下两种情况:(1)$z_{1}, z_{2} \geq 0$;(2)$z_{1}, z_{2}$ 中有一正一负。对于第(2)种情况直接取 $z > 0$ 即可消除定位模糊,但对于第(1)种情况则无法确定目标的位置,因此需要借助其它辅助信息,才能消除定位模糊。这里不再赘述,有关方法参见文献【4】。

对于第二种情况,计算机可以在一毫秒的时间内改变任意一只蓝球或者红球的颜色,对黄球的发现和定位我们分析如下:

对于如下图所示的红、蓝球分布情况,当颜色可以在一毫秒的时间内改变的话,譬如将左下的蓝球变为红色,那么一毫秒的时间黄球的位置基本上没有发生什么显著变化,但是此时多了一对球可以对其进行定位。同理,在整个的布局中,我们每当改变一次颜色,相当于没有增加球的实际数目,但是却增加了可以用来发现并定位黄球的红蓝球对。

\begin{figure}[h]
\centering
\includegraphics[width=0.8\textwidth]{image.png}
\caption{改变球的颜色的影响示意图}
\end{figure}

对于最后两个问题,也是在雷达领域内较为热门的话题。对于多目标检测,也就是同时有多个黄球同时进入圆柱体,我们可以通过对其编号,然后利用 charp 变换等方法进行检测。为了更好的发现并定位黄球,可以充分利用计算机,对目标进行状态估值,估值方法随着目标状态改变而改变,做到自适应,参看【5】。

\section{参考文献}

\begin{enumerate}
    \item 程正兴,数值逼近和常微分方程数值解,西安交大出版社,2001.
    \item 如何把圆沙丁鱼塞满罐头,\url{http://www.fjsxyz.com/resource}.
    \item 《数学手册》编写组,数学手册,高等教育出版社,1979.
    \item 孙仲康等,单多基地有源无源定位技术,国防工业出版社,1996.5 P88—P95.
    \item 林育涛,自适应雷达定位系统的设计,计算机与数字工程,22(2),1994,P32—P40.
\end{enumerate}