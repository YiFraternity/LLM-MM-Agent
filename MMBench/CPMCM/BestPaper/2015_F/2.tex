\begin{center}
\textbf{第十二届“中关村青联杯”全国研究生数学建模竞赛}
\end{center}

\begin{table}[h]
\centering
\begin{tabular}{c c}
\hline
学校 & 江南大学 \\
\hline
参赛队号 & 10295002 \\
\hline
队员姓名 & \\
1. & 智月明 \\
2. & 周超洁 \\
3. & 范雪峰 \\
\hline
\end{tabular}
\end{table}

\begin{flushright}
参赛密码 \_\_\_\_\_\_\_\_\_\_\_\_\_\_\_\_\_\_\_\_\_\_\_\_\_\_\_\_\_\_\_\_\_\_\_\_\_\_\_\_\_\_\_\_\_\_\_\_\_\_\_\_\_\_\_\_\_\_\_\_\_\_\_\_\_\_\_\_\_\_\_\_\_\_\_\_\_\_\_\_\_\_\_\_\_\_\_\_\_\_\_\_\_\_\_\_\_\_\_\_\_\_\_\_\_\_\_\_\_\_\_\_\_\_\_\_\_\_\_\_\_\_\_\_\_\_\_\_\_\_\_\_\_\_\_\_\_\_\_\_\_\_\_\_\_\_\_\_\_\_\_\_\_\_\_\_\_\_\_\_\_\_\_\_\_\_\_\_\_\_\_\_\_\_\_\_\_\_\_\_\_\_\_\_\_\_\_\_\_\_\_\_\_\_\_\_\_\_\_\_\_\_\_\_\_\_\_\_\_\_\_\_\_\_\_\_\_\_\_\_\_\_\_\_\_\_\_\_\_\_\_\_\_\_\_\_\_\_\_\_\_\_\_\_\_\_\_\_\_\_\_\_\_\_\_\_\_\_\_\_\_\_\_\_\_\_\_\_\_\_\_\_\_\_\_\_\_\_\_\_\_\_\_\_\_\_\_\_\_\_\_\_\_\_\_\_\_\_\_\_\_\_\_\_\_\_\_\_\_\_\_\_\_\_\_\_\_\_\_\_\_\_\_\_\_\_\_\_\_\_\_\_\_\_\_\_\_\_\_\_\_\_\_\_\_\_\_\_\_\_\_\_\_\_\_\_\_\_\_\_\_\_\_\_\_\_\_\_\_\_\_\_\_\_\_\_\_\_\_\_\_\_\_\_\_\_\_\_\_\_\_\_\_\_\_\_\_\_\_\_\_\_\_\_\_\_\_\_\_\_\_\_\_\_\_\_\_\_\_\_\_\_\_\_\_\_\_\_\_\_\_\_\_\_\_\_\_\_\_\_\_\_\_\_\_\_\_\_\_\_\_\_\_\_\_\_\_\_\_\_\_\_\_\_\_\_\_\_\_\_\_\_\_\_\_\_\_\_\_\_\_\_\_\_\_\_\_\_\_\_\_\_\_\_\_\_\_\_\_\_\_\_\_\_\_\_\_\_\_\_\_\_\_\_\_\_\_\_\_\_\_\_\_\_\_\_\_\_\_\_\_\_\_\_\_\_\_\_\_\_\_\_\_\_\_\_\_\_\_\_\_\_\_\_\_\_\_\_\_\_\_\_\_\_\_\_\_\_\_\_\_\_\_\_\_\_\_\_\_\_\_\_\_\_\_\_\_\_\_\_\_\_\_\_\_\_\_\_\_\_\_\_\_\_\_\_\_\_\_\_\_\_\_\_\_\_\_\_\_\_\_\_\_\_\_\_\_\_\_\_\_\_\_\_\_\_\_\_\_\_\_\_\_\_\_\_\_\_\_\_\_\_\_\_\_\_\_\_\_\_\_\_\_\_\_\_\_\_\_\_\_\_\_\_\_\_\_\_\_\_\_\_\_\_\_\_\_\_\_\_\_\_\_\_\_\_\_\_\_\_\_\_\_\_\_\_\_\_\_\_\_\_\_\_\_\_\_\_\_\_\_\_\_\_\_\_\_\_\_\_\_\_\_\_\_\_\_\_\_\_\_\_\_\_\_\_\_\_\_\_\_\_\_\_\_\_\_\_\_\_\_\_\_\_\_\_\_\_\_\_\_\_\_\_\_\_\_\_\_\_\_\_\_\_\_\_\_\_\_\_\_\_\_\_\_\_\_\_\_\_\_\_\_\_\_\_\_\_\_\_\_\_\_\_\_\_\_\_\_\_\_\_\_\_\_\_\_\_\_\_\_\_\_\_\_\_\_\_\_\_\_\_\_\_\_\_\_\_\_\_\_\_\_\_\_\_\_\_\_\_\_\_\_\_\_\_\_\_\_\_\_\_\_\_\_\_\_\_\_\_\_\_\_\_\_\_\_\_\_\_\_\_\_\_\_\_\_\_\_\_\_\_\_\_\_\_\_\_\_\_\_\_\_\_\_\_\_\_\_\_\_\_\_\_\_\_\_\_\_\_\_\_\_\_\_\_\_\_\_\_\_\_\_\_\_\_\_\_\_\_\_\_\_\_\_\_\_\_\_\_\_\_\_\_\_\_\_\_\_\_\_\_\_\_\_\_\_\_\_\_\_\_\_\_\_\_\_\_\_\_\_\_\_\_\_\_\_\_\_\_\_\_\_\_\_\_\_\_\_\_\_\_\_\_\_\_\_\_\_\_\_\_\_\_\_\_\_\_\_\_\_\_\_\_\_\_\_\_\_\_\_\_\_\_\_\_\_\_\_\_\_\_\_\_\_\_\_\_\_\_\_\_\_\_\_\_\_\_\_\_\_\_\_\_\_\_\_\_\_\_\_\_\_\_\_\_\_\_\_\_\_\_\_\_\_\_\_\_\_\_\_\_\_\_\_\_\_\_\_\_\_\_\_\_\_\_\_\_\_\_\_\_\_\_\_\_\_\_\_\_\_\_\_\_\_\_\_\_\_\_\_\_\_\_\_\_\_\_\_\_\_\_\_\_\_\_\_\_\_\_\_\_\_\_\_\_\_\_\_\_\_\_\_\_\_\_\_\_\_\_\_\_\_\_\_\_\_\_\_\_\_\_\_\_\_\_\_\_\_\_\_\_\_\_\_\_\_\_\_\_\_\_\_\_\_\_\_\_\_\_\_\_\_\_\_\_\_\_\_\_\_\_\_\_\_\_\_\_\_\_\_\_\_\_\_\_\_\_\_\_\_\_\_\_\_\_\_\_\_\_\_\_\_\_\_\_\_\_\_\_\_\_\_\_\_\_\_\_\_\_\_\_\_\_\_\_\_\_\_\_\_\_\_\_\_\_\_\_\_\_\_\_\_\_\_\_\_\_\_\_\_\_\_\_\_\_\_\_\_\_\_\_\_\_\_\_\_\_\_\_\_\_\_\_\_\_\_\_\_\_\_\_\_\_\_\_\_\_\_\_\_\_\_\_\_\_\_\_\_\_\_\_\_\_\_\_\_\_\_\_\_\_\_\_\_\_\_\_\_\_\_\_\_\_\_\_\_\_\_\_\_\_\_\_\_\_\_\_\_\_\_\_\_\_\_\_\_\_\_\_\_\_\_\_\_\_\_\_\_\_\_\_\_\_\_\_\_\_\_\_\_\_\_\_\_\_\_\_\_\_\_\_\_\_\_\_\_\_\_\_\_\_\_\_\_\_\_\_\_\_\_\_\_\_\_\_\_\_\_\_\_\_\_\_\_\_\_\_\_\_\_\_\_\_\_\_\_\_\_\_\_\_\_\_\_\_\_\_\_\_\_\_\_\_\_\_\_\_\_\_\_\_\_\_\_\_\_\_\_\_\_\_\_\_\_\_\_\_\_\_\_\_\_\_\_\_\_\_\_\_\_\_\_\_\_\_\_\_\_\_\_\_\_\_\_\_\_\_\_\_\_\_\_\_\_\_\_\_\_\_\_\_\_\_\_\_\_\_\_\_\_\_\_\_\_\_\_\_\_\_\_\_\_\_\_\_\_\_\_\_\_\_\_\_\_\_\_\_\_\_\_\_\_\_\_\_\_\_\_\_\_\_\_\_\_\_\_\_\_\_\_\_\_\_\_\_\_\_\_\_\_\_\_\_\_\_\_\_\_\_\_\_\_\_\_\_\_\_\_\_\_\_\_\_\_\_\_\_\_\_\_\_\_\_\_\_\_\_\_\_\_\_\_\_\_\_\_\_\_\_\_\_\_\_\_\_\_\_\_\_\_\_\_\_\_\_\_\_\_\_\_\_\_\_\_\_\_\_\_\_\_\_\_\_\_\_\_\_\_\_\_\_\_\_\_\_\_\_\_\_\_\_\_\_\_\_\_\_\_\_\_\_\_\_\_\_\_\_\_\_\_\_\_\_\_\_\_\_\_\_\_\_\_\_\_\_\_\_\_\_\_\_\_\_\_\_\_\_\_\_\_\_\_\_\_\_\_\_\_\_\_\_\_\_\_\_\_\_\_\_\_\_\_\_\_\_\_\_\_\_\_\_\_\_\_\_\_\_\_\_\_\_\_\_\_\_\_\_\_\_\_\_\_\_\_\_\_\_\_\_\_\_\_\_\_\_\_\_\_\_\_\_\_\_\_\_\_\_\_\_\_\_\_\_\_\_\_\_\_\_\_\_\_\_\_\_\_\_\_\_\_\_\_\_\_\_\_\_\_\_\_\_\_\_\_\_\_\_\_\_\_\_\_\_\_\_\_\_\_\_\_\_\_\_\_\_\_\_\_\_\_\_\_\_\_\_\_\_\_\_\_\_\_\_\_\_\_\_\_\_\_\_\_\_\_\_\_\_\_\_\_\_\_\_\_\_\_\_\_\_\_\_\_\_\_\_\_\_\_\_\_\_\_\_\_\_\_\_\_\_\_\_\_\_\_\_\_\_\_\_\_\_\_\_\_\_\_\_\_\_\_\_\_\_\_\_\_\_\_\_\_\_\_\_\_\_\_\_\_\_\_\_\_\_\_\_\_\_\_\_\_\_\_\_\_\_\_\_\_\_\_\_\_\_\_\_\_\_\_\_\_\_\_\_\_\_\_\_\_\_\_\_\_\_\_\_\_\_\_\_\_\_\_\_\_\_\_\_\_\_\_\_\_\_\_\_\_\_\_\_\_\_\_\_\_\_\_\_\_\_\_\_\_\_\_\_\_\_\_\_\_\_\_\_\_\_\_\_\_\_\_\_\_\_\_\_\_\_\_\_\_\_\_\_\_\_\_\_\_\_\_\_\_\_\_\_\_\_\_\_\_\_\_\_\_\_\_\_\_\_\_\_\_\_\_\_\_\_\_\_\_\_\_\_\_\_\_\_\_\_\_\_\_\_\_\_\_\_\_\_\_\_\_\_\_\_\_\_\_\_\_\_\_\_\_\_\_\_\_\_\_\_\_\_\_\_\_\_\_\_\_\_\_\_\_\_\_\_\_\_\_\_\_\_\_\_\_\_\_\_\_\_\_\_\_\_\_\_\_\_\_\_\_\_\_\_\_\_\_\_\_\_\_\_\_\_\_\_\_\_\_\_\_\_\_\_\_\_\_\_\_\_\_\_\_\_\_\_\_\_\_\_\_\_\_\_\_\_\_\_\_\_\_\_\_\_\_\_\_\_\_\_\_\_\_\_\_\_\_\_\_\_\_\_\_\_\_\_\_\_\_\_\_\_\_\_\_\_\_\_\_\_\_\_\_\_\_\_\_\_\_\_\_\_\_\_\_\_\_\_\_\_\_\_\_\_\_\_\_\_\_\_\_\_\_\_\_\_\_\_\_\_\_\_\_\_\_\_\_\_\_\_\_\_\_\_\_\_\_\_\_\_\_\_\_\_\_\_\_\_\_\_\_\_\_\_\_\_\_\_\_\_\_\_\_\_\_\_\_\_\_\_\_\_\_\_\_\_\_\_\_\_\_\_\_\_\_\_\_\_\_\_\_\_\_\_\_\_\_\_\_\_\_\_\_\_\_\_\_\_\_\_\_\_\_\_\_\_\_\_\_\_\_\_\_\_\_\_\_\_\_\_\_\_\_\_\_\_\_\_\_\_\_\_\_\_\_\_\_\_\_\_\_\_\_\_\_\_\_\_\_\_\_\_\_\_\_\_\_\_\_\_\_\_\_\_\_\_\_\_\_\_\_\_\_\_\_\_\_\_\_\_\_\_\_\_\_\_\_\_\_\_\_\_\_\_\_\_\_\_\_\_\_\_\_\_\_\_\_\_\_\_\_\_\_\_\_\_\_\_\_\_\_\_\_\_\_\_\_\_\_\_\_\_\_\_\_\_\_\_\_\_\_\_\_\_\_\_\_\_\_\_\_\_\_\_\_\_\_\_\_\_\_\_\_\_\_\_\_\_\_\_\_\_\_\_\_\_\_\_\_\_\_\_\_\_\_\_\_\_\_\_\_\_\_\_\_\_\_\_\_\_\_\_\_\_\_\_\_\_\_\_\_\_\_\_\_\_\_\_\_\_\_\_\_\_\_\_\_\_\_\_\_\_\_\_\_\_\_\_\_\_\_\_\_\_\_\_\_\_\_\_\_\_\_\_\_\_\_\_\_\_\_\_\_\_\_\_\_\_\_\_\_\_\_\_\_\_\_\_\_\_\_\_\_\_\_\_\_\_\_\_\_\_\_\_\_\_\_\_\_\_\_\_\_\_\_\_\_\_\_\_\_\_\_\_\_\_\_\_\_\_\_\_\_\_\_\_\_\_\_\_\_\_\_\_\_\_\_\_\_\_\_\_\_\_\_\_\_\_\_\_\_\_\_\_\_\_\_\_\_\_\_\_\_\_\_\_\_\_\_\_\_\_\_\_\_\_\_\_\_\_\_\_\_\_\_\_\_\_\_\_\_\_\_\_\_\_\_\_\_\_\_\_\_\_\_\_\_\_\_\_\_\_\_\_\_\_\_\_\_\_\_\_\_\_\_\_\_\_\_\_\_\_\_\_\_\_\_\_\_\_\_\_\_\_\_\_\_\_\_\_\_\_\_\_\_\_\_\_\_\_\_\_\_\_\_\_\_\_\_\_\_\_\_\_\_\_\_\_\_\_\_\_\_\_\_\_\_\_\_\_\_\_\_\_\_\_\_\_\_\_\_\_\_\_\_\_\_\_\_\_\_\_\_\_\_\_\_\_\_\_\_\_\_\_\_\_\_\_\_\_\_\_\_\_\_\_\_\_\_\_\_\_\_\_\_\_\_\_\_\_\_\_\_\_\_\_\_\_\_\_\_\_\_\_\_\_\_\_\_\_\_\_\_\_\_\_\_\_\_\_\_\_\_\_\_\_\_\_\_\_\_\_\_\_\_\_\_\_\_\_\_\_\_\_\_\_\_\_\_\_\_\_\_\_\_\_\_\_\_\_\_\_\_\_\_\_\_\_\_\_\_\_\_\_\_\_\_\_\_\_\_\_\_\_\_\_\_\_\_\_\_\_\_\_\_\_\_\_\_\_\_\_\_\_\_\_\_\_\_\_\_\_\_\_\_\_\_\_\_\_\_\_\_\_\_\_\_\_\_\_\_\_\_\_\_\_\_\_\_\_\_\_\_\_\_\_\_\_\_\_\_\_\_\_\_\_\_\_\_\_\_\_\_\_\_\_\_\_\_\_\_\_\_\_\_\_\_\_\_\_\_\_\_\_\_\_\_\_\_\_\_\_\_\_\_\_\_\_\_\_\_\_\_\_\_\_\_\_\_\_\_\_\_\_\_\_\_\_\_\_\_\_\_\_\_\_\_\_\_\_\_\_\_\_\_\_\_\_\_\_\_\_\_\_\_\_\_\_\_\_\_\_\_\_\_\_\_\_\_\_\_\_\_\_\_\_\_\_\_\_\_\_\_\_\_\_\_\_\_\_\_\_\_\_\_\_\_\_\_\_\_\_\_\_\_\_\_\_\_\_\_\_\_\_\_\_\_\_\_\_\_\_\_\_\_\_\_\_\_\_\_\_\_\_\_\_\_\_\_\_\_\_\_\_\_\_\_\_\_\_\_\_\_\_\_\_\_\_\_\_\_\_\_\_\_\_\_\_\_\_\_\_\_\_\_\_\_\_\_\_\_\_\_\_\_\_\_\_\_\_\_\_\_\_\_\_\_\_\_\_\_\_\_\_\_\_\_\_\_\_\_\_\_\_\_\_\_\_\_\_\_\_\_\_\_\_\_\_\_\_\_\_\_\_\_\_\_\_\_\_\_\_\_\_\_\_\_\_\_\_\_\_\_\_\_\_\_\_\_\_\_\_\_\_\_\_\_\_\_\_\_\_\_\_\_\_\_\_\_\_\_\_\_\_\_\_\_\_\_\_\_\_\_\_\_\_\_\_\_\_\_\_\_\_\_\_\_\_\_\_\_\_\_\_\_\_\_\_\_\_\_\_\_\_\_\_\_\_\_\_\_\_\_\_\_\_\_\_\_\_\_\_\_\_\_\_\_\_\_\_\_\_\_\_\_\_\_\_\_\_\_\_\_\_\_\_\_\_\_\_\_\_\_\_\_\_\_\_\_\_\_\_\_\_\_\_\_\_\_\_\_\_\_\_\_\_\_\_\_\_\_\_\_\_\_\_\_\_\_\_\_\_\_\_\_\_\_\_\_\_\_\_\_\_\_\_\_\_\_\_\_\_\_\_\_\_\_\_\_\_\_\_\_\_\_\_\_\_\_\_\_\_\_\_\_\_\_\_\_\_\_\_\_\_\_\_\_\_\_\_\_\_\_\_\_\_\_\_\_\_\_\_\_\_\_\_\_\_\_\_\_\_\_\_\_\_\_\_\_\_\_\_\_\_\_\_\_\_\_\_\_\_\_\_\_\_\_\_\_\_\_\_\_\_\_\_\_\_\_\_\_\_\_\_\_\_\_\_\_\_\_\_\_\_\_\_\_\_\_\_\_\_\_\_\_\_\_\_\_\_\_\_\_\_\_\_\_\_\_\_\_\_\_\_\_\_\_\_\_\_\_\_\_\_\_\_\_\_\_\_\_\_\_\_\_\_\_\_\_\_\_\_\_\_\_\_\_\_\_\_\_\_\_\_\_\_\_\_\_\_\_\_\_\_\_\_\_\_\_\_\_\_\_\_\_\_\_\_\_\_\_\_\_\_\_\_\_\_\_\_\_\_\_\_\_\_\_\_\_\_\_\_\_\_\_\_\_\_\_\_\_\_\_\_\_\_\_\_\_\_\_\_\_\_\_\_\_\_\_\_\_\_\_\_\_\_\_\_\_\_\_\_\_\_\_\_\_\_\_\_\_\_\_\_\_\_\_\_\_\_\_\_\_\_\_\_\_\_\_\_\_\_\_\_\_\_\_\_\_\_\_\_\_\_\_\_\_\_\_\_\_\_\_\_\_\_\_\_\_\_\_\_\_\_\_\_\_\_\_\_\_\_\_\_\_\_\_\_\_\_\_\_\_\_\_\_\_\_\_\_\_\_\_\_\_\_\_\_\_\_\_\_\_\_\_\_\_\_\_\_\_\_\_\_\_\_\_\_\_\_\_\_\_\_\_\_\_\_\_\_\_\_\_\_\_\_\_\_\_\_\_\_\_\_\_\_\_\_\_\_\_\_\_\_\_\_\_\_\_\_\_\_\_\_\_\_\_\_\_\_\_\_\_\_\_\_\_\_\_\_\_\_\_\_\_\_\_\_\_\_\_\_\_\_\_\_\_\_\_\_\_\_\_\_\_\_\_\_\_\_\_\_\_\_\_\_\_\_\_\_\_\_\_\_\_\_\_\_\_\_\_\_\_\_\_\_\_\_\_\_\_\_\_\_\_\_\_\_\_\_\_\_\_\_\_\_\_\_\_\_\_\_\_\_\_\_\_\_\_\_\_\_\_\_\_\_\_\_\_\_\_\_\_\_\_\_\_\_\_\_\_\_\_\_\_\_\_\_\_\_\_\_\_\_\_\_\_\_\_\_\_\_\_\_\_\_\_\_\_\_\_\_\_\_\_\_\_\_\_\_\_\_\_\_\_\_\_\_\_\_\_\_\_\_\_\_\_\_\_\_\_\_\_\_\_\_\_\_\_\_\_\_\_\_\_\_\_\_\_\_\_\_\_\_\_\_\_\_\_\_\_\_\_\_\_\_\_\_\_\_\_\_\_\_\_\_\_\_\_\_\_\_\_\_\_\_\_\_\_\_\_\_\_\_\_\_\_\_\_\_\_\_\_\_\_\_\_\_\_\_\_\_\_\_\_\_\_\_\_\_\_\_\_\_\_\_\_\_\_\_\_\_\_\_\_\_\_\_\_\_\_\_\_\_\_\_\_\_\_\_\_\_\_\_\_\_\_\_\_\_\_\_\_\_\_\_\_\_\_\_\_\_\_\_\_\_\_\_\_\_\_\_\_\_\_\_\_\_\_\_\_\_\_\_\_\_\_\_\_\_\_\_\_\_\_\_\_\_\_\_\_\_\_\_\_\_\_\_\_\_\_\_\_\_\_\_\_\_\_\_\_\_\_\_\_\_\_\_\_\_\_\_\_\_\_\_\_\_\_\_\_\_\_\_\_\_\_\_\_\_\_\_\_\_\_\_\_\_\_\_\_\_\_\_\_\_\_\_\_\_\_\_\_\_\_\_\_\_\_\_\_\_\_\_\_\_\_\_\_\_\_\_\_\_\_\_\_\_\_\_\_\_\_\_\_\_\_\_\_\_\_\_\_\_\_\_\_\_\_\_\_\_\_\_\_\_\_\_\_\_\_\_\_\_\_\_\_\_\_\_\_\_\_\_\_\_\_\_\_\_\_\_\_\_\_\_\_\_\_\_\_\_\_\_\_\_\_\_\_\_\_\_\_\_\_\_\_\_\_\_\_\_\_\_\_\_\_\_\_\_\_\_\_\_\_\_\_\_\_\_\_\_\_\_\_\_\_\_\_\_\_\_\_\_\_\_\_\_\_\_\_\_\_\_\_\_\_\_\_\_\_\_\_\_\_\_\_\_\_\_\_\_\_\_\_\_\_\_\_\_\_\_\_\_\_\_\_\_\_\_\_\_\_\_\_\_\_\_\_\_\_\_\_\_\_\_\_\_\_\_\_\_\_\_\_\_\_\_\_\_\_\_\_\_\_\_\_\_\_\_\_\_\_\_\_\_\_\_\_\_\_\_\_\_\_\_\_\_\_\_\_\_\_\_\_\_\_\_\_\_\_\_\_\_\_\_\_\_\_\_\_\_\_\_\_\_\_\_\_\_\_\_\_\_\_\_\_\_\_\_\_\_\_\_\_\_\_\_\_\_\_\_\_\_\_\_\_\_\_\_\_\_\_\_\_\_\_\_\_\_\_\_\_\_\_\_\_\_\_\_\_\_\_\_\_\_\_\_\_\_\_\_\_\_\_\_\_\_\_\_\_\_\_\_\_\_\_\_\_\_\_\_\_\_\_\_\_\_\_\_\_\_\_\_\_\_\_\_\_\_\_\_\_\_\_\_\_\_\_\_\_\_\_\_\_\_\_\_\_\_\_\_\_\_\_\_\_\_\_\_\_\_\_\_\_\_\_\_\_\_\_\_\_\_\_\_\_\_\_\_\_\_\_\_\_\_\_\_\_\_\_\_\_\_\_\_\_\_\_\_\_\_\_\_\_\_\_\_\_\_\_\_\_\_\_\_\_\_\_\_\_\_\_\_\_\_\_\_\_\_\_\_\_\_\_\_\_\_\_\_\_\_\_\_\_\_\_\_\_\_\_\_\_\_\_\_\_\_\_\_\_\_\_\_\_\_\_\_\_\_\_\_\_\_\_\_\_\_\_\_\_\_\_\_\_\_\_\_\_\_\_\_\_\_\_\_\_\_\_\_\_\_\_\_\_\_\_\_\_\_\_\_\_\_\_\_\_\_\_\_\_\_\_\_\_\_\_\_\_\_\_\_\_\_\_\_\_\_\_\_\_\_\_\_\_\_\_\_\_\_\_\_\_\_\_\_\_\_\_\_\_\_\_\_\_\_\_\_\_\_\_\_\_\_\_\_\_\_\_\_\_\_\_\_\_\_\_\_\_\_\_\_\_\_\_\_\_\_\_\_\_\_\_\_\_\_\_\_\_\_\_\_\_\_\_\_\_\_\_\_\_\_\_\_\_\_\_\_\_\_\_\_\_\_\_\_\_\_\_\_\_\_\_\_\_\_\_\_\_\_\_\_\_\_\_\_\_\_\_\_\_\_\_\_\_\_\_\_\_\_\_\_\_\_\_\_\_\_\_\_\_\_\_\_\_\_\_\_\_\_\_\_\_\_\_\_\_\_\_\_\_\_\_\_\_\_\_\_\_\_\_\_\_\_\_\_\_\_\_\_\_\_\_\_\_\_\_\_\_\_\_\_\_\_\_\_\_\_\_\_\_\_\_\_\_\_\_\_\_\_\_\_\_\_\_\_\_\_\_\_\_\_\_\_\_\_\_\_\_\_\_\_\_\_\_\_\_\_\_\_\_\_\_\_\_\_\_\_\_\_\_\_\_\_\_\_\_\_\_\_\_\_\_\_\_\_\_\_\_\_\_\_\_\_\_\_\_\_\_\_\_\_\_\_\_\_\_\_\_\_\_\_\_\_\_\_\_\_\_\_\_\_\_\_\_\_\_\_\_\_\_\_\_\_\_\_\_\_\_\_\_\_\_\_\_\_\_\_\_\_\_\_\_\_\_\_\_\_\_\_\_\_\_\_\_\_\_\_\_\_\_\_\_\_\_\_\_\_\_\_\_\_\_\_\_\_\_\_\_\_\_\_\_\_\_\_\_\_\_\_\_\_\_\_\_\_\_\_\_\_\_\_\_\_\_\_\_\_\_\_\_\_\_\_\_\_\_\_\_\_\_\_\_\_\_\_\_\_\_\_\_\_\_\_\_\_\_\_\_\_\_\_\_\_\_\_\_\_\_\_\_\_\_\_\_\_\_\_\_\_\_\_\_\_\_\_\_\_\_\_\_\_\_\_\_\_\_\_\_\_\_\_\_\_\_\_\_\_\_\_\_\_\_\_\_\_\_\_\_\_\_\_\_\_\_\_\_\_\_\_\_\_\_\_\_\_\_\_\_\_\_\_\_\_\_\_\_\_\_\_\_\_\_\_\_\_\_\_\_\_\_\_\_\_\_\_\_\_\_\_\_\_\_\_\_\_\_\_\_\_\_\_\_\_\_\_\_\_\_\_\_\_\_\_\_\_\_\_\_\_\_\_\_\_\_\_\_\_\_\_\_\_\_\_\_\_\_\_\_\_\_\_\_\_\_\_\_\_\_\_\_\_\_\_\_\_\_\_\_\_\_\_\_\_\_\_\_\_\_\_\_\_\_\_\_\_\_\_\_\_\_\_\_\_\_\_\_\_\_\_\_\_\_\_\_\_\_\_\_\_\_\_\_\_\_\_\_\_\_\_\_\_\_\_\_\_\_\_\_\_\_\_\_\_\_\_\_\_\_\_\_\_\_\_\_\_\_\_\_\_\_\_\_\_\_\_\_\_\_\_\_\_\_\_\_\_\_\_\_\_\_\_\_\_\_\_\_\_\_\_\_\_\_\_\_\_\_\_\_\_\_\_\_\_\_\_\_\_\_\_\_\_\_\_\_\_\_\_\_\_\_\_\_\_\_\_\_\_\_\_\_\_\_\_\_\_\_\_\_\_\_\_\_\_\_\_\_\_\_\_\_\_\_\_\_\_\_\_\_\_\_\_\_\_\_\_\_\_\_\_\_\_\_\_\_\_\_\_\_\_\_\_\_\_\_\_\_\_\_\_\_\_\_\_\_\_\_\_\_\_\_\_\_\_\_\_\_\_\_\_\_\_\_\_\_\_\_\_\_\_\_\_\_\_\_\_\_\_\_\_\_\_\_\_\_\_\_\_\_\_\_\_\_\_\_\_\_\_\_\_\_\_\_\_\_\_\_\_\_\_\_\_\_\_\_\_\_\_\_\_\_\_\_\_\_\_\_\_\_\_\_\_\_\_\_\_\_\_\_\_\_\_\_\_\_\_\_\_\_\_\_\_\_\_\_\_\_\_\_\_\_\_\_\_\_\_\_\_\_\_\_\_\_\_\_\_\_\_\_\_\_\_\_\_\_\_\_\_\_\_\_\_\_\_\_\_\_\_\_\_\_\_\_\_\_\_\_\_\_\_\_\_\_\_\_\_\_\_\_\_\_\_\_\_\_\_\_\_\_\_\_\_\_\_\_\_\_\_\_\_\_\_\_\_\_\_\_\_\_\_\_\_\_\_\_\_\_\_\_\_\_\_\_\_\_\_\_\_\_\_\_\_\_\_\_\_\_\_\_\_\_\_\_\_\_\_\_\_\_\_\_\_\_\_\_\_\_\_\_\_\_\_\_\_\_\_\_\_\_\_\_\_\_\_\_\_\_\_\_\_\_\_\_\_\_\_\_\_\_\_\_\_\_\_\_\_\_\_\_\_\_\_\_\_\_\_\_\_\_\_\_\_\_\_\_\_\_\_\_\_\_\_\_\_\_\_\_\_\_\_\_\_\_\_\_\_\_\_\_\_\_\_\_\_\_\_\_\_\_\_\_\_\_\_\_\_\_\_\_\_\_\_\_\_\_\_\_\_\_\_\_\_\_\_\_\_\_\_\_\_\_\_\_\_\_\_\_\_\_\_\_\_\_\_\_\_\_\_\_\_\_\_\_\_\_\_\_\_\_\_\_\_\_\_\_\_\_\_\_\_\_\_\_\_\_\_\_\_\_\_\_\_\_\_\_\_\_\_\_\_\_\_\_\_\_\_\_\_\_\_\_\_\_\_\_\_\_\_\_\_\_\_\_\_\_\_\_\_\_\_\_\_\_\_\_\_\_\_\_\_\_\_\_\_\_\_\_\_\_\_\_\_\_\_\_\_\_\_\_\_\_\_\_\_\_\_\_\_\_\_\_\_\_\_\_\_\_\_\_\_\_\_\_\_\_\_\_\_\_\_\_\_\_\_\_\_\_\_\_\_\_\_\_\_\_\_\_\_\_\_\_\_\_\_\_\_\_\_\_\_\_\_\_\_\_\_\_\_\_\_\_\_\_\_\_\_\_\_\_\_\_\_\_\_\_\_\_\_\_\_\_\_\_\_\_\_\_\_\_\_\_\_\_\_\_\_\_\_\_\_\_\_\_\_\_\_\_\_\_\_\_\_\_\_\_\_\_\_\_\_\_\_\_\_\_\_\_\_\_\_\_\_\_\_\_\_\_\_\_\_\_\_\_\_\_\_\_\_\_\_\_\_\_\_\_\_\_\_\_\_\_\_\_\_\_\_\_\_\_\_\_\_\_\_\_\_\_\_\_\_\_\_\_\_\_\_\_\_\_\_\_\_\_\_\_\_\_\_\_\_\_\_\_\_\_\_\_\_\_\_\_\_\_\_\_\_\_\_\_\_\_\_\_\_\_\_\_\_\_\_\_\_\_\_\_\_\_\_\_\_\_\_\_\_\_\_\_\_\_\_\_\_\_\_\_\_\_\_\_\_\_\_\_\_\_\_\_\_\_\_\_\_\_\_\_\_\_\_\_\_\_\_\_\_\_\_\_\_\_\_\_\_\_\_\_\_\_\_\_\_\_\_\_\_\_\_\_\_\_\_\_\_\_\_\_\_\_\_\_\_\_\_\_\_\_\_\_\_\_\_\_\_\_\_\_\_\_\_\_\_\_\_\_\_\_\_\_\_\_\_\_\_\_\_\_\_\_\_\_\_\_\_\_\_\_\_\_\_\_\_\_\_\_\_\_\_\_\_\_\_\_\_\_\_\_\_\_\_\_\_\_\_\_\_\_\_\_\_\_\_\_\_\_\_\_\_\_\_\_\_\_\_\_\_\_\_\_\_\_\_\_\_\_\_\_\_\_\_\_\_\_\_\_\_\_\_\_\_\_\_\_\_\_\_\_\_\_\_\_\_\_\_\_\_\_\_\_\_\_\_\_\_\_\_\_\_\_\_\_\_\_\_\_\_\_\_\_\_\_\_\_\_\_\_\_\_\_\_\_\_\_\_\_\_\_\_\_\_\_\_\_\_\_\_\_\_\_\_\_\_\_\_\_\_\_\_\_\_\_\_\_\_\_\_\_\_\_\_\_\_\_\_\_\_\_\_\_\_\_\_\_\_\_\_\_\_\_\_\_\_\_\_\_\_\_\_\_\_\_\_\_\_\_\_\_\_\_\_\_\_\_\_\_\_\_\_\_\_\_\_\_\_\_\_\_\_\_\_\_\_\_\_\_\_\_\_\_\_\_\_\_\_\_\_\_\_\_\_\_\_\_\_\_\_\_\_\_\_\_\_\_\_\_\_\_\_\_\_\_\_\_\_\_\_\_\_\_\_\_\_\_\_\_\_\_\_\_\_\_\_\_\_\_\_\_\_\_\_\_\_\_\_\_\_\_\_\_\_\_\_\_\_\_\_\_\_\_\_\_\_\_\_\_\_\_\_\_\_\_\_\_\_\_\_\_\_\_\_\_\_\_\_\_\_\_\_\_\_\_\_\_\_\_\_\_\_\_\_\_\_\_\_\_\_\_\_\_\_\_\_\_\_\_\_\_\_\_\_\_\_\_\_\_\_\_\_\_\_\_\_\_\_\_\_\_\_\_\_\_\_\_\_\_\_\_\_\_\_\_\_\_\_\_\_\_\_\_\_\_\_\_\_\_\_\_\_\_\_\_\_\_\_\_\_\_\_\_\_\_\_\_\_\_\_\_\_\_\_\_\_\_\_\_\_\_\_\_\_\_\_\_\_\_\_\_\_\_\_\_\_\_\_\_\_\_\_\_\_\_\_\_\_\_\_\_\_\_\_\_\_\_\_\_\_\_\_\_\_\_\_\_\_\_\_\_\_\_\_\_\_\_\_\_\_\_\_\_\_\_\_\_\_\_\_\_\_\_\_\_\_\_\_\_\_\_\_\_\_\_\_\_\_\_\_\_\_\_\_\_\_\_\_\_\_\_\_\_\_\_\_\_\_\_\_\_\_\_\_\_\_\_\_\_\_\_\_\_\_\_\_\_\_\_\_\_\_\_\_\_\_\_\_\_\_\_\_\_\_\_\_\_\_\_\_\_\_\_\_\_\_\_\_\_\_\_\_\_\_\_\_\_\_\_\_\_\_\_\_\_\_\_\_\_\_\_\_\_\_\_\_\_\_\_\_\_\_\_\_\_\_\_\_\_\_\_\_\_\_\_\_\_\_\_\_\_\_\_\_\_\_\_\_\_\_\_\_\_\_\_\_\_\_\_\_\_\_\_\_\_\_\_\_\_\_\_\_\_\_\_\_\_\_\_\_\_\_\_\_\_\_\_\_\_\_\_\_\_\_\_\_\_\_\_\_\_\_\_\_\_\_\_\_\_\_\_\_\_\_\_\_\_\_\_\_\_\_\_\_\_\_\_\_\_\_\_\_\_\_\_\_\_\_\_\_\_\_\_\_\_\_\_\_\_\_\_\_\_\_\_\_\_\_\_\_\_\_\_\_\_\_\_\_\_\_\_\_\_\_\_\_\_\_\_\_\_\_\_\_\_\_\_\_\_\_\_\_\_\_\_\_\_\_\_\_\_\_\_\_\_\_\_\_\_\_\_\_\_\_\_\_\_\_\_\_\_\_\_\_\_\_\_\_\_\_\_\_\_\_\_\_\_\_\_\_\_\_\_\_\_\_\_\_\_\_\_\_\_\_\_\_\_\_\_\_\_\_\_\_\_\_\_\_\_\_\_\_\_\_\_\_\_\_\_\_\_\_\_\_\_\_\_\_\_\_\_\_\_\_\_\_\_\_\_\_\_\_\_\_\_\_\_\_\_\_\_\_\_\_\_\_\_\_\_\_\_\_\_\_\_\_\_\_\_\_\_\_\_\_\_\_\_\_\_\_\_\_\_\_\_\_\_\_\_\_\_\_\_\_\_\_\_\_\_\_\_\_\_\_\_\_\_\_\_\_\_\_\_\_\_\_\_\_\_\_\_\_\_\_\_\_\_\_\_\_\_\_\_\_\_\_\_\_\_\_\_\_\_\_\_\_\_\_\_\_\_\_\_\_\_\_\_\_\_\_\_\_\_\_\_\_\_\_\_\_\_\_\_\_\_\_\_\_\_\_\_\_\_\_\_\_\_\_\_\_\_\_\_\_\_\_\_\_\_\_\_\_\_\_\_\_\_\_\_\_\_\_\_\_\_\_\_\_\_\_\_\_\_\_\_\_\_\_\_\_\_\_\_\_\_\_\_\_\_\_\_\_\_\_\_\_\_\_\_\_\_\_\_\_\_\_\_\_\_\_\_\_\_\_\_\_\_\_\_\_\_\_\_\_\_\_\_\_\_\_\_\_\_\_\_\_\_\_\_\_\_\_\_\_\_\_\_\_\_\_\_\_\_\_\_\_\_\_\_\_\_\_\_\_\_\_\_\_\_\_\_\_\_\_\_\_\_\_\_\_\_\_\_\_\_\_\_\_\_\_\_\_\_\_\_\_\_\_\_\_\_\_\_\_\_\_\_\

\begin{center}
\includegraphics[width=0.2\textwidth]{image1.png} \quad
\includegraphics[width=0.2\textwidth]{image2.png} \quad
\includegraphics[width=0.3\textwidth]{image3.png} \quad
\includegraphics[width=0.2\textwidth]{image4.png}
\end{center}

\begin{center}
\textbf{第十二届“中关村青联杯”全国研究生数学建模竞赛}
\end{center}

\begin{center}
\textbf{题目} \quad 旅游路线规划问题
\end{center}

\section*{摘要:}

本文主要以自驾游爱好者游遍全国 201 个 5A 级景区的旅游路线安排为研究对象,综合考虑时间约束(景区开放时间、行车时间、年旅行次数、旅行天数等),行车路线(高速优先)和旅行费用(包括住宿花费,行车油耗,飞机机票和高铁车票等)等复杂条件,建立多个模型并通过相应算法求解,以满足不同旅游者对于旅行的差异化需求。

本文的主要创新点:以局部优化简化复杂的全局优化,从而得到旅游路线设计的近似最优解;将自驾游爱好者进行分类,根据不同的旅游偏好设计相应的模型求解出合适的景区游览路线及方式。

本文完成的主要工作总结如下:

\begin{enumerate}
    \item 问题一主要考虑了居住在西安的自驾游爱好者采取全程自驾游遍 201 个 5A 级景区时间最短的旅游路线规划问题。在考虑多个限制条件的情况下,本文建立了基于时间最优的多旅行商(mTSP)数字规划模型,并通过蚁群算法求解出该模型的最优解,安排出该旅行者的时间最优旅行路线。对于本问题,本文考虑先局部优化最后全局优化的主要思路。首先以省为单位对旅游时间进行局部优化:利用解决旅行商问题(TSP)的改良圈(又称 Hamilton 圈)算法分别求得各省旅游时间最优的旅游路线。其次考虑全国范围内的全局优化问题:首先建立 mTSP 时间最优模型,利用蚁群算法求解该模型得到每次旅行的最佳路线,根据所得结果拆分部分省份,然后被拆分省份与邻近省份重新生成一个新的改良圈进行优化,最终得到从西安出发游遍全国各景区需要 10.5 年,并整合出具体旅游路线。
    \item 问题二主要是综合考虑旅行费用和旅游体验度双重指标,设计出一家三口 10
\end{enumerate}

年游遍 201 个 5A 景区的旅游线路。在保证 10 年前提条件下,本问主要采取尽可能提高旅游体验,减少旅游花费的思想。在结合约束条件的前提下,建立了基于多目标优化的规划模型,通过归一化、函数拟合等方法,利用基本粒子群优化算法计算得到接近于最优的次优解,给出该家庭的十年出行路线。首先对时间、花费进行归一化处理,构造体现旅行者对出行要求的 $\varepsilon$ 参数;其次根据已知信息,采用函数拟合方法求得机票、高铁价格与两城距离之间的定量关系;再者针对全国范围内的全局优化,建立花费最少、体验度最优的多目标模型,利用基本粒子群优化算法优化得到出行的最佳路线和合理交通方式,辅以局部修改,得到最终方案。最终,通过对所得出行方案的整理计算,得出家庭十年旅行路线安排、最优花费和最佳旅行体验度,最少花费为 318362 元,最佳时间满意度为 0.82。

三.问题三主要是在问题二所得模型的基础上进行推广,建立起适用于全国范围旅游爱好者旅游路线规划的推广模型。根据问题二中出行方式花费函数,考虑人数和行车路程两个因素对模型的影响。通过分析问题一和问题二中所得结果,定性地分析了在旅游过程中对旅游总时间和旅游费用的影响较大的关键因素。并且本文中还利用问题一中所得模型求解出北京为旅行首末站城市的全程自驾游行车时间最短为 124.5 天,与第二问所得结果(行车时间为 66 天)定量的进行对比,增加约束条件对问题二中的多目标优化模型进行推广。接着以北京作为首末站城市的旅游路线为例,考虑一人出行的情况,同样通过基本粒子群优化算法求解所得多目标优化推广模型,得到十年旅行费用和体验度最优旅游路线且行车时间为 65.5 天(小于 66 天)。最后分别针对费用和旅游人气等向旅游爱好者和旅游部门分别给出相关建议。

四.问题四首先根据景点特色将所有自驾游爱好者分为 5 类,然后基于该旅游者偏好确定在每个 5A 级景区最少的游览时间分析出此自驾游爱好者属于观光休闲类。增加约束条件剔除 5 个离省会较远的 5A 级景点,然后安排适量的 4A 级名胜风景区在该自驾游爱好者的十年旅游计划中,根据 Hamilton 圈方法进行优化,最终得到合理的旅游规划。

最后对所建立的模型进行优缺点分析,并提出改进意见。

关键字:多旅行商;改良圈;蚁群算法;基本粒子群优化算法

\section*{目录}

\begin{itemize}
    \item[] 1. 前言 \dotfill 1
    \begin{itemize}
        \item[] 1.1 问题背景与问题描述 \dotfill 1
        \item[] 1.2 问题分析 \dotfill 1
        \item[] 1.3 基本假设 \dotfill 2
    \end{itemize}
    \item[] 2. 问题一的模型建立与求解 \dotfill 2
    \begin{itemize}
        \item[] 2.1 数据分析 \dotfill 2
        \item[] 2.2 各省时间最优旅游路线求取 \dotfill 3
        \item[] 2.3 全国时间最优旅游路线求取 \dotfill 6
        \begin{itemize}
            \item[] 2.3.1 mTSP 时间最优模型建立 \dotfill 6
            \item[] 2.3.2 基于蚁群算法的模型求解 \dotfill 7
        \end{itemize}
    \end{itemize}
    \item[] 3. 问题二的模型建立与求解 \dotfill 10
    \begin{itemize}
        \item[] 3.1 模型建立 \dotfill 10
        \item[] 3.2 模型求解与结果分析 \dotfill 13
        \item[] 3.3 最终模型方案 \dotfill 17
    \end{itemize}
    \item[] 4. 问题三的模型建立与求解 \dotfill 20
    \begin{itemize}
        \item[] 4.1 数据分析 \dotfill 20
        \item[] 4.2 模型推广 \dotfill 22
        \item[] 4.3 模型求解 \dotfill 23
        \item[] 4.4 旅游建议 \dotfill 25
    \end{itemize}
    \item[] 5. 问题四的模型建立与求解 \dotfill 25
    \begin{itemize}
        \item[] 5.1 数据分析 \dotfill 25
        \item[] 5.2 模型建立 \dotfill 26
        \item[] 5.3 模型求解 \dotfill 27
    \end{itemize}
    \item[] 6. 模型评价与改进 \dotfill 28
    \item[] 参考文献 \dotfill 30
    \item[] 附录 \dotfill 31
\end{itemize}

\section{前言}

\subsection{问题背景与问题描述}

旅游活动正在成为全球经济发展的重要动力之一,它加速国际资金流转和信息、技术管理的传播,创造高效率消费行为模式、需求和价值等。随着我国国民经济的快速发展,人们生活水平得到很大提升,越来越多的人积极参与有益于身心健康的旅游活动。故而为旅游爱好者科学并且合理地规划旅游路线是一个热点问题。本文主要解决以下问题:

\begin{itemize}
    \item 问题一:在行车线路的设计上采用高速优先的策略,考虑多种时间约束,设计出合适的方法,建立数学模型,以该旅游爱好者的常住地在西安市为例,规划设计旅游线路,确定游遍 201 个 5A 级景区至少需要的年数,并且给出每一次旅游的具体行程(每一天的出发地、行车时间、行车里程、游览景区;若有必要,其他更详细表达请另列附件)。
    \item 问题二:考虑多种出行方式,结合题目给出飞机高铁的相关信息,综合考虑前述全程自驾、先乘坐高铁或飞机到达省会城市后再租车自驾到景区等出行方式,建立数学模型设计一个为一家三口十年游遍所有 201 个 5A 景区、费用最优、旅游体验最好的旅游线路,给出每一次旅游的具体线路(含每次具体出行方式;每一天的出发地、费用、路途时间、游览景区、每个景区的游览时间)。
    \item 问题三:在第二问所建立的模型基础上加以推广,可以为全国的自驾游爱好者规划设计类似的旅游线路,进而给出常住地在北京市的自驾游爱好者的十年旅游计划;根据上述三问的结果给旅游爱好者和旅游有关部门提出建议。
    \item 问题四:自 2007 年 3 月 7 日至 2015 年 7 月 13 日,全国旅游景区质量等级评定委员会分 29 批共批准了 201 家景区为国家 5A 级旅游景区。附件 6 是从国家旅游局官网上收集的国家 5A 级旅游景区评定的相关信息,附件 7 给出了国家旅游局官网上收集的国家 4A 级景区名单,请更为合理地规划该旅游爱好者的十年旅游计划。
\end{itemize}

\subsection{问题分析}

针对问题一,主要考虑常住西安的旅游爱好者计划通过自驾游的方式游遍全国 201 个 5A 级景区的时间消耗问题。同时考虑每年总共外出旅游时间不超过 30 天、每年外出旅游的次数不超过 4 次、单次旅游的时间不超过 15 天、旅游者个人偏好在景区的游览时间、行车时间的限制(每天 7:00 至 19:00 之间,每天开车时间不超过 8 小时,全天游览开车时间小于 3 小时内,半天游览开车时间小于 5 小时)、景区开放时间(8:00 至 18:00)等,并综合考虑各地之间高速公路和普通公路的分布,建立数学模型,得出游遍 201 个景区时间消耗最少的最优旅游路线。我们可将全国 201 个景点分布看做一张图,并对各景点编号。利用图论的相关知识进行分析。首先将该景点分布图看做一个 $n$ 阶连通赋权图 $G=(V,E)$,其中顶点集 $V=\{v_1,v_2,\dots,v_n\}$,边集 $E=\{e_1,e_2,\dots,e_n\}$,在本文中边 $e_i,e_j$ 上的权值 $w_{ij}$ 表示距离(或时间或费用)。若 $v_i,v_j$ 不相邻,则令 $w_{ij}=\infty$。由于景点分布在数量上的繁多与地区分布上的差异性,再综合考虑各种约束条件,使得该问题的复杂性急剧增加。因此我们考虑把 201 个景点由小及大的区域化,即按市、省、全国的顺序逐级求解花费时间最优的旅游路线:1). 由于分布在同一个市的景点分布较近且游览时间和行车时间基本固定,所以对于城市直接计算出各市游览时间(包括游览时间和各景点行车时间的总时间)2). 接着考虑游览同一省内的旅游景点消耗时间最少的游览路线,对于各省的多个城市之间如何合理安排合理线路这一问题,可以看做是一个旅行商 (TSP) 问题,即考虑各省的游览安排为从省会城市—全省各市—省会城市的路线。可利用改良圈算法求取各省时间最优路线安排,对各省的最优旅游路线进行局部优化。3). 最后考虑以省为基点如何合理安排旅游路线使时间消耗最少的问题,这可以看做是一个多旅行商

问题(mTSP)。通过合适算法求解 mTSP 模型,即可求解出该旅游爱好者游遍 201 个景区时间消耗与旅游路线。

针对问题二,经过分析得,问题二在问题一的基础上,转化为时间有限、旅游费用最少的规划问题。在十年旅游时间限制下,使用最少的费用游玩 201 个 5A 级景区且旅游者体验度也要最优。即在满足所有的约束条件下,需要建立体验度最优与花费最少的模型。除了问题一中的限制条件外,这里增加了额外的要求:租车费用 300 元/天;在每个景区逗留时间不超过附件中建议时间的 2 倍;高铁乘坐时间不超过 6 小时;乘坐高铁或飞机的当天至多安排半天的游览;需要注意的是,此处是 3 人同行,选择铁路或飞机的时候,需要考虑 3 倍价格;另外,这里提供了住宿费、每公里油耗等额外支出需求。此外,问题中强调了需要设计旅游体验最好的旅游线路。为满足这个要求,本文利用构造时间体验度和旅行总费用的目标函数。附件 3 和附件 5 分别给出了主要城市之间距离和票价信息,所以联想到可以通过函数拟合建立两者之间的定量关系。模型解决拟采用粒子群算法(PSO)得出最终的较优路线。PSO 与遗传算法类似,是一种基于迭代的优化工具。本题需要注意的是,在费用计算时要考虑该旅行是 3 人出行,机票、住宿等需要额外增加费用。

针对问题三,主要是利用问题二所得模型进行推广,从而为全国旅游爱好者设计十年自驾游路线。首先,从全国各地出发游览最主要的差异即是在行车里程上的不同。由于游览时间是一定的,路线的安排对行车里程影响最大,而旅游全程出发地对总行车里程影响很大,因此在针对该问题上要对模型加入对行车里程的限制;其次,由于本问题中并未给出旅游人数,而问题二中所得模型是针对 3 人出行的模型,而人数对于乘飞机或高铁出行的费用以及计算有一定影响,故而在此问题中不能忽略。因此本问题的处理上,我们主要考虑这两方面的因素对问题二中所得模型影响,进而对问题二中的多目标最优模型进行推广。最后,以北京为例,利用该推广模型,同样通过粒子群算法得到以北京为旅游首末站的十年最优旅游路线。

针对问题四,题目给出了 1182 个 4A 级旅游景点,和 201 个 5A 级景点,此问中没有提出必须将 201 个 5A 级旅游景点都游历遍,而且对费用没有要求。该问题的一个难点即是多个旅游景区的筛选,故而在本问中将针对不同旅游者的偏好进行模型改进与设计。本文主要是在模型三的基础上进行推广,同时考虑费用最优和旅游满意度,最后得出合理的旅游规划。

\subsection*{1.3 基本假设}

1. 假设不考虑景区门票以及景区内部消费。

2. 假设不考虑飞机或高铁的候车和延误时间。

3. 假设两城市之间往返机票价格一样。

4. 假设高铁或飞机出行时间安排在每天 7:00 至 19:00 之间,且飞机单次飞行耗费 0.5 天。

\section*{2. 问题一的模型建立与求解}

\subsection*{2.1 数据分析}

首先由题目提供的附件 1 中给出 201 个 5A 级景点,利用地图汇网站(http://c.dituhui.com)导入这些景点,得到图 2-1。另外经过分析得到每个省的 5A 级景点,可以得到图 2-2。

\begin{figure}[h]
    \centering
    \includegraphics[width=\textwidth]{image1.png}
    \caption{5A级景点分布图}
    \label{fig:5a_distribution}
\end{figure}

\begin{figure}[h]
    \centering
    \includegraphics[width=\textwidth]{image2.png}
    \caption{各省5A级景点统计图}
    \label{fig:5a_statistics}
\end{figure}

从分布图可以直观的看出5A级景点在京津唐,华东地区较多,西北地区较少;景点与景点之间的距离也相差较大。因为约束条件很多,如果以景点为单位,求得每次游玩的最佳路径比较复杂,全局最优难以实现。如果以省为单位,将全局最优转换为局部最优,工作量大大减小,因此在本问中先分别求取各省最优路线,最后针对各省最优游览时间,求取游遍全国5A景区的全局最优旅游路线。这样虽然求得的最终解不是最优解,但是近似于最优解。

\subsection{各省时间最优旅游路线求取}

从省会城市—全省各市—省会城市的路线求解可以等同于一个TSP问题。因此我们可以利用求解TSP问题的算法进行求取。

\section{1. TSP 问题}

TSP 问题 \cite{ref1} 是由某地出发,途中恰好不重复的游览完所有的景点,然后回到出发地形成一个闭合的环型的旅游路线。这个问题的特殊之处在于起点与终点重合,中间各景点不能重复游览,各条道路不能重复行走。解决 TSP 问题最简易的解决方法是通过穷举寻找最短路径。其算法复杂度一般取决于顶点个数,这样将导致随着顶点的增大,复杂度成指数形式增长,该方法几乎不可能实现,目前已经证明 TSP 问题是 NP 问题。目前的解法主要有分支定界法、改良圈算法、模拟退火算法、遗传算法、贪心算法、人工神经网络等方法 \cite{ref2,ref3,ref4,ref5,ref6}。但是本次考虑各省中城市安排规模较小,故而在这一问题上利用改良圈算法求解该 TSP 问题。

\section{2. 改良圈算法}

把每个省看做一个整体,即把各省看做图 $G_{1}=(V_{1}, E_{1})$,将该城市看做顶点 $V_{1}=\left\{v_{1}, v_{2}, \ldots, v_{n}\right\}$,各城市间的距离 $d_{ij}$ 可看做边集 $E_{1}$ 中各边的权重。首先给出求近似最优 Hamilton 圈的最邻近算法的基本思想:

\begin{enumerate}
    \item[a)] 任意选一个点 $v_{0}$ 作起点。找一条与 $v_{0}$ 关联且权最小的边 $e_{1}$,$e_{1}$ 的另一个端点记作 $v_{1}$,得到一条路 $v_{0}v_{1}$;
    \item[b)] 设已选出路 $v_{0}v_{1} \ldots v_{i}$,在 $V(G_{1})-\{v_{0}v_{1} \ldots v_{i}\}$ 中取一个与 $v_{i}$ 最邻近的相邻顶点 $v_{i+1}$,得到路 $v_{0}v_{1} \ldots v_{i}v_{i+1}$;
    \item[c)] 若 $i+1<n-1$,则用 $i+1$ 代替 $i+1$ 返回步骤 b);否则,记 $v_{0}v_{1} \ldots v_{n}v_{0}$,停止。
\end{enumerate}

以上算法采用最邻近算法求得的 Hamilton 圈一般不是最优,通过改良可以获得更短的 Hamilton 圈。设 $C_{1}=v_{0}v_{1} \ldots v_{n}v_{1}$ 是图 $G_{1}$ 的一个 Hamilton 圈,对圈 $C_{1}$ 中所有满足 $1<i+1<j<n$ 的 $i, j$,按照以下方法步骤最终可得一条新的 Hamilton 圈 $C_{1}'$:

再检查 $C_{1}$ 是否有 $i \neq j$,使得
\begin{equation}
v_{i}v_{j} \in E_{1}(G), v_{i+1}v_{j+1} \in E_{1}(G)
\end{equation}
且 $w(v_{i}v_{j})+w(v_{i+1}v_{j+1})<w(v_{i}v_{i+1})+w(v_{j}v_{j+1})$,则构成新圈
\begin{equation}
C_{1}'=v_{1}v_{2} \ldots v_{i}v_{j}v_{j-1} \ldots v_{i+1}v_{j+1} \ldots v_{n}v_{1};
\end{equation}
用 $C_{1}'$ 代替 $C_{1}$ 转到步骤 1,直至终止。

利用地图汇导入各个省的旅游景点,然后得到相应的景点的经纬度,其中表 2-1 给出了江苏省各景点的经纬度,以江苏省为例,求得最佳 Hamilton 圈和最小的权值。由于在此考虑的是各省的经纬度,并没有考虑各个景点之间具体高速公路或普通公路的分布,故而该改良圈算法得到是一种近似的结果,结果不一定是最优的,但是比较好的结果。

\begin{table}[h]
\centering
\caption{江苏各景点经纬度表}
\begin{tabular}{c l r}
\hline \hline
No. & 景点 & 经纬度 \\
\hline
1 & 苏州园林(拙政园-留园-虎丘) & 120.636,31.3303 \\
2 & 苏州昆山周庄古镇景区 & 120.856,31.1208 \\
3 & 南京钟山-中山陵风景名胜区(明孝陵-音乐... & 118.836,32.0709 \\
4 & 中央电视台无锡影视基地三国水浒城景区 & 120.24,31.4819 \\
5 & 无锡灵山大佛景区 & 120.112,31.4444 \\
6 & 苏州吴江同里古镇景区 & 120.725,31.1641 \\
7 & 南京夫子庙-秦淮河风光带(江南贡院-白鹭... & 118.899,31.8231 \\
8 & 常州环球恐龙城景区(中华恐龙园-恐龙谷温... & 120.015,31.824 \\
9 & 扬州瘦西湖风景区 & 119.425,32.4148 \\
10 & 南通市濠河风景区 & 120.878,32.0217 \\
\hline \hline
\end{tabular}
\end{table}

\begin{table}
\centering
\begin{tabular}{c l r}
11 & 泰州姜堰区溱湖国家湿地公园 & 120.097,32.6301 \\
12 & 苏州市金鸡湖国家商务旅游示范区 & 120.705,31.3176 \\
13 & 镇江三山风景名胜区(金山-北固山-焦山) & 119.49,32.2453 \\
14 & 无锡鼋头渚景区 & 120.226,31.5336 \\
15 & 苏州吴中太湖旅游区(旺山-穹窿山-东山) & 120.581,31.1743 \\
16 & 苏州常熟沙家浜-虞山尚湖旅游区 & 120.708,31.6758 \\
17 & 常州溧阳市天目湖景区(天目湖-南山竹海-... & 119.447,31.3215 \\
18 & 镇江句容茅山景区 & 119.314,31.7877 \\
19 & 淮安市周恩来故里景区(周恩来纪念馆-周恩... & 119.149,33.5131 \\
\end{tabular}
\end{table}

利用附录1里的MATLAB程序求得最小权值为9.0348。改良后的圈如图2-3所示。各个景点的顺序为1,16,10,14,8,11,19,9,13,3,7,18,17,5,4,15,6,2,12,1。根据计算得到的改良圈得到游玩江苏省最少所用时间为12天。利用同样方式,可以得到游玩其他33个省或者直辖市最少所用时间,如图2-4(具体行程见附录2)。

\begin{figure}[h]
\centering
\includegraphics[width=\textwidth]{image.png}
\caption{改良圈}
\end{figure}

图2-3 江苏各景区改良圈

\begin{figure}[h]
    \centering
    \includegraphics[width=\textwidth]{image.png}
    \caption{旅行各省所需时间}
    \label{fig:travel_time}
\end{figure}

根据得到的结果,可以发现有的省份旅游时间较长,比如新疆的旅行时间为 15 天,但是约束条件中每次出行时间不能大于等于 15 天,所以此类省份必须拆分为 2 次游览;另外,为了出行的时间每次尽可能达到 15 天或者每年的旅行天数尽可能达到 30 天,不能保证每次游玩都能刚好玩遍一个省,所以对于景点分布较分散的省份可以拆分与其他省合并在一起游玩。

\subsection{全国时间最优旅游路线求取}

\subsubsection{mTSP 时间最优模型建立}

在前一小节的基础上,我们已经得到了各省游览时间局部最优路线,最终目标即是要得到全国游览时间全局最优路线。由问题分析我们已了解到该问题可等价为 mTSP 问题。mTSP 比 TSP 更复杂,因此问题的求解更为困难,多常用的算法有 Lin-kernighan 算法、自组织神经网络、遗传算法、蚁群算法等[7]-[9]。本文主要用蚁群算法来求解符合该问题的 mTSP 规划模型。

\subsubsection{目标函数建立}

\begin{itemize}
    \item $k$ ——从西安市出发第 $k$ 次游览,$0 < k \leq m$;
    \item $t_{ij}$ ——从省 $i$ 到省 $j$ 所花费的时间;
    \item $x_{kij}$ ——$x_{kij} = 1$,第 $k$ 次旅游从省 $i$ 到 $j$;否则,$x_{kij} = 0$。
\end{itemize}

经过分析与整合,游遍各省花费时间最少即是该问题的总目标,即计算各次旅行花费在各省之间的行驶时间最少,故得目标函数如下:

\begin{equation}
\min \sum_{k=1}^{m} \sum_{i \neq j} t_{ij} x_{kij}
\tag{2-1}
\end{equation}

\subsubsection{约束条件分析}

由于该旅游者每次旅游从西安出发且必须回到西安,则可得到约束条件如下:

\begin{equation}
\sum_{j=1}^{n} x_{k0j} = \sum_{j=1}^{n} x_{kj0} = m, \forall k
\tag{2-2}
\end{equation}

将每个省看做图上的一个顶点。对于每个顶点来说,只允许最多一条边进入,同样只允许一条边出来,并且只要有一条边进入就要有一条边出去。因此可得约束:

\begin{equation}
\sum_{i \neq j} x_{kij} = 1, \forall k, i, \sum_{i \neq j} x_{kji} = 1, \forall k, i
\tag{2-3}
\end{equation}

为避免出现从省 $i$ 到省 $j$ 和从省 $j$ 到 $i$ 的重复,加入以下约束条件:

\begin{equation}
L_{kj} \geq L_{ki} + 1 + n(x_{kij} - 1), \forall k, i \neq j, j \neq 1
\tag{2-4}
\end{equation}

此外,每次旅行不能超过 15 天,故而得到如下约束条件:

\begin{equation}
\sum_{i \neq j} t_{ij} x_{kij} \leq 15 - t_s, \forall k
\tag{2-5}
\end{equation}

其中,$t_s$ 为单次旅行游览景点的总时间。

\section*{3. 模型建立}

综上所述,可得如下模型:

\begin{equation}
\begin{aligned}
\min & \sum_{k=1}^m \sum_{i \neq j} t_{ij} x_{kij} \\
s.t. & \left\{
\begin{aligned}
\sum_{j=1}^n x_{k0j} & = \sum_{j=1}^n x_{kj0} = m, \forall k \\
\sum_{i \neq j} x_{kij} & = 1, \forall k, i \\
\sum_{i \neq j} x_{kji} & = 1, \forall k, i \\
L_{kj} & \geq L_{ki} + 1 + n(x_{kij} - 1), \forall k, i \neq j, j \neq 1 \\
\sum_{i \neq j} t_{ij} x_{kij} & \leq 15 - a_k, \forall k
\end{aligned}
\right.
\end{aligned}
\end{equation}

\subsection{2.3.2 基于蚁群算法的模型求解}

蚁群算法 [10] 是属于一种求解组合最优化问题的新型通用启发式方法,该方法具有正反馈、分布式计算和富于建设性的贪婪启发式搜索的特点。与其他启发式算法相比,在求解性能上,具有很强的稳定性和搜索较好解的能力。因为蚁群算法是一种基于种群的进化算法,具有本质并行性,易于并行实现,所以它很容易与多种启发式算法结合,以改善算法性能。本文主要利用蚁群算法进行求解。

首先,我们要引入人工蚂蚁 (artificial ants) 的概念。人工蚂蚁具有记忆功能,并具有以下特征:

1) 根据信息素浓度和启发式信息,用相应的转移概率选择下一个城市。

2) 将已经走过的城市放入记忆表中,记忆表里的城市将不再被选择为下一个城市。

3) 完成一次循环后,根据整个路径的长度来释放相应的信息素,并更新走过的路径上的信息素。

其次,我们需要对相关变量进行定义,具体如下:

- $m$ —— 蚂蚁个数;
- $\tau_{ij}$ —— 边 $(i, j)$ 上的轨迹强度;
- $\Delta \tau_{ij}^k$ —— 蚂蚁 $k$ 在边 $(i, j)$ 上留下的轨迹信息素的增量;
- $p_{ij}^k$ —— 蚂蚁 $k$ 从省 $i$ 到省 $j$ 的转移概率;

首先确定游览的天数,本文中每次游览天数都小于等于 15 天,然后对蚂蚁寻找节点的规则进行相关条件的限制,具体步骤如下:

1. 各项参数初始化:$NC = 0$;每条边上的权重记为 $(d_{ij}, w_{ij})$,$w_{ij} = W(i, j)$,$W$ 为 $n \times n$ 的矩阵,$w_{ij}$ 为节点 $i$ 到 $j$ 的最短花费时间;$\tau$ 为 $n \times n$ 的矩阵,并且每个元素都等于 1(初

初始化信息素),$\tau_{ij}=1$,$\tau_{ij}=\tau(i,j)$;$\Delta\tau_{ij}=1$,$\Delta\tau_{ij}=\Delta\tau(i,j)$,$\Delta\tau$ 为 $n \times n$ 的矩阵,并且每个元素均为 0;放置 $m$ 个蚂蚁到 1 号节点上;

2. 把每个蚂蚁的初始城市号码为 1 放置到 travel 中,travel 为 1 个 $m$ 行的矩阵,$m$ 为蚂蚁的个数(矩阵的第 $k$ 只蚂蚁所走过的节点),初始时 travel 中所有的元素均为 0;

3. 删除不符合条件的节点,对于还未被选择的节点 $j$,计算 $T=t+t_{j}+w_{ij}$,如果存在点 $j$,使得 $T \leq 14$(因为还要留一定的时间回到西安,根据需要修改);转入步骤 4,否则转入步骤 5;

4. 第 $k$ 只蚂蚁根据概率 $p_{ij}^{k}(t)$ 来选择下一步应该到达的节点,将第 $k$ 只蚂蚁移到另一个节点。仍然采取轮盘赌的形式,概率大的节点被选中的概率越大,当 $k=1,2 \cdots m$ 时,将选择的节点插入到 travel$(k,:)$ 中;转入步骤 3;

5. $j=1$,$T=0$,转入步骤 3;

6. 所有节点访问完毕,计算第 $k$ 个蚂蚁的总路径长度为 $L$,记录该次迭代中最短路径长度和节点访问次序;

7. 更新每条边上的信息素浓度;NC=NC+1;

8. 如果 $NC \leq NC_{\max}$,清空 travel;

9. 如果 $NC > NC_{\max}$,算法结束。

具体算法流程图如图 2-5 所示:

\begin{figure}[h]
\centering
\includegraphics[width=0.8\textwidth]{image.png}
\caption{旅行路线优化流程图}
\end{figure}

利用蚁群算法确定每次出行的省份,然后根据结果确定需要拆分的省份和邻近省份,可以用改良圈算法再进行计算,得到新的改良圈,使得每次出去的旅行距离最短,从而

\begin{table}
\centering
\begin{tabular}{|c|c|c|c|c|c|c|c|c|c|}
\hline
编号 & 里程/历时 & 第一站/所在省游玩时间 & 里程/历时 & 第二站/所在省游玩时间 & 里程/历时 & 第三站/所在省游玩时间 & 里程/历时 & 总天数 & 年份 \\
\hline
1 & 1090km/11.5h & 江苏/12d & \multicolumn{5}{|c|}{1090km/11.5h} & 15d & 1 \\
\hline
2 & 1090km/11.5h & 江苏/1.5d & 640km/7h & 山东/8d & \multicolumn{3}{|c|}{910km/10h} & 12.5d & \\
\hline
3 & 600km/7h & 山西/6.5d & 223km/3.5h & 河北/6d & \multicolumn{3}{|c|}{790km/9.5h} & 15d & 2 \\
\hline
4 & 790km/9.5h & 河北/1d & 291km/4h & 北京&天津/7d & \multicolumn{3}{|c|}{1100km/12h} & 12.5d & \\
\hline
5 & 2310km/26h & 黑龙江/7d & \multicolumn{5}{|c|}{2310km/26h} & 13d & 3 \\
\hline
6 & 1750km/18h & 辽宁/4d & 380km/4h & 吉林/5d & \multicolumn{3}{|c|}{2040km/22h} & 15d & \\
\hline
8 & 990km/1d & 湖南/6d & \multicolumn{5}{|c|}{990km/1d} & 8d & 4 \\
\hline
9 & 1090km/1d & 江西/7d & \multicolumn{5}{|c|}{1090km/1d} & 9d & \\
\hline
10 & 920km/1d & 安徽/7d & \multicolumn{5}{|c|}{920km/1d} & 9d & \\
\hline
7 & 480km/1d & 河南/12d & \multicolumn{5}{|c|}{480km/1d} & 14d & 5 \\
\hline
11 & 740km/2d & 湖北/10d & \multicolumn{5}{|c|}{740km/2d} & 14d & \\
\hline
12 & 2100km/3d & 海南/3.5d & 478km/0.5d & 广西/5d & \multicolumn{3}{|c|}{1640km/3d} & 15d & 6 \\
\hline
13 & 1330km/2d & 浙江/8d & 175km/2h & 上海/3d & \multicolumn{3}{|c|}{1390km/2d} & 15d & \\
\hline
14 & 1650km/2.5d & 福建/9d & \multicolumn{5}{|c|}{1650km/2.5d} & 14d & 7 \\
\hline
15 & 1650km/2d & 广东/8.5d & \multicolumn{5}{|c|}{1650km/2d} & 12.5d & \\
\hline
16 & 970km/1d3h & 内蒙古/1.5d & 684km/1d & 宁夏/3.5d & 440km/5h & 甘肃/3.5d & 650km/7h & 13.5d & 8 \\
\hline
17 & 2540k & 新疆 & \multicolumn{5}{|c|}{2540km/3.5d} & 13.5d & \\
\hline
\end{tabular}
\end{table}

\begin{tabular}{|c|c|c|c|c|c|c|}
\hline
 & m/3.5 & /7.5d & & & & \\
 & d & & & & & \\
\hline
18 & 2540k & 新疆乌 & 1463 & 新疆喀 & 3733km/5d & 14d \\
 & m/3.5 & 乌鲁木齐 & km/ & 什/1.5d & & \\
 & d & /2d & 2d & & & 9 \\
\hline
19 & 890k & 青海 & 1958 & 西藏 & 2840km/4d & 13d \\
 & m/1d2 & & km/ & /2d & & \\
 & h & /2.5d & 3d & & & \\
\hline
20 & 750k & 四川/11d & 712k & 陕西 & / & 15d \\
 & m/1d & & m/1 & /2d & & 10 \\
 & & & d & & & \\
\hline
21 & 1070k & 贵州 & 506k & 云南 & 1590km/2d2h & 15d \\
 & m/1.5 & /4.5d & m/1 & /6d & & \\
 & d & & d & & & \\
\hline
22 & 700k & 重庆 & 685k & 陕西 & / & 14d \\
 & m/1d & /9.5d & m/1 & /2.5d & & 11 \\
 & & & d & & & \\
\hline
\end{tabular}

从上表可以看出,该自驾游旅行者确定玩遍201个景点至少需要10.5年。具体旅游路线安排见附录2。

\section{问题二的模型建立与求解}

\subsection{模型建立}

\subsubsection{目标函数建立}

经过对题目分析,我们可以知道本题所要实现的目标是,使该旅游爱好者在10年内花费最少的钱游览所有201各5A级景区。首先,所到地方肯定要覆盖全部景点。同时,还需要考虑该游客的体验度。因此,我们的做法是,在问题一的基础上,保证花费时间满足10年的前提下,利用基于粒子群算法的多目标优化方法,对所需时间和花费金钱两个目标进行优化计算,得到一个接近与最优解的次优方案。

\subsubsection{时间满意度}

显然,出游的体验指数受旅游时间和出行总时间的影响。时间因素的增加,增加了出游旅行规划问题的复杂性。所以我们引入时间满意度的定义:

$E$ ——时间满意度;

$t_k$ ——第$k$次旅游时间,$m$次旅游总时间为$t_s$;

$t_a$ ——$k$次旅游时间总时间,即为游览时间和行车时间的总和;

马云峰等[11]介绍了线性时间满意度函数、凹凸分布曲线、岭型分布曲线、降对数Sigmoid函数和降半哥西分布函数5种函数。为方便计算,这里构造线性时间满意度为:

\begin{equation}
\text{Max } E = \frac{t_s}{t_a} \tag{3-1}
\end{equation}

其中,$t_a = t_s + \sum_{k=1}^{m} \sum_{i \neq j} t_{ij} x_{kij} = \sum_{k=1}^{m} t_k + \sum_{k=1}^{m} \sum_{i \neq j} t_{ij} x_{kij}$

\subsubsection{第$k$次旅行总费用}

除时间满意度外,游览总费用我们定义:

$p_{k1}$ ——第$k$次交通总花费;

$p_{k2}$ ——第$k$次住宿总花费;

从而得到目标函数为:
\begin{equation}
\min P_k = p_{k1} + p_{k2}
\tag{3-2}
\end{equation}

1) 第 \( k \) 次交通总花费

根据题目条件可知,共有三种交通方式可以选择:飞机、高铁、汽车(包括租车和自驾),\( p_a, p_t, p_c \) 分别表示上述三种方式的开销,则有:
\begin{equation}
p_{k1} = \alpha_a p_a + \alpha_t p_t + \alpha_c p_c
\tag{3-3}
\end{equation}
其中:
\[
\alpha =
\begin{cases}
1, \text{使用该交通方式} \\
0, \text{未使用}
\end{cases}
\]

a) 汽车

由题目中给出条件可知,高速公路的油耗加过路费平均为 1.00 元/公里,普通公路上油耗平均为 0.60 元/公里。设:
- \( L_h \) ——高速公路里程;
- \( L_l \) ——普通公路里程;
- \( L \) ——普通公路里程,即 \( L = L_h + L_l \);

则有 \( p_c = 1.00 \times L_h + 0.6 L_l \)。然而在实际情况下,大部分路程都是在省市之间行驶,以高速公路居多,为了简便计算,我们假设所有公路都为高速公路,即 \( p_c = L \)。

另外,若选择租车出行,则还需额外增加 300 元/天的租金,所以上式表示为:
\begin{equation}
p_c = L + r \times 300 t_c
\tag{3-4}
\end{equation}
其中,\( t_c \) 表示租车天数;\( r =
\begin{cases}
1, \text{租车} \\
0, \text{自驾}
\end{cases}
\)。

b) 高铁

题目中所提供的附件 4 给出了若干省会城市之间的高铁票价和时间信息。根据网页查询得到,高铁的费用基本上与里程成正比 \( p_t \propto L \),于是可以得到乘坐高铁所需要的费用为:\( p_t = k_t \times L \)。其中,\( k_t \) 表示高铁每公里花费(元/公里)。需要注意的是,题目中给出,旅行者是 3 人同时出行,所以在上式基础还需要乘以 3 倍表示 3 人总费用,即最终:
\begin{equation}
p_t = 3 k_t \times L
\tag{3-5}
\end{equation}

c) 飞机

题目中所提供的给出了若干省会城市之间的机票全价信息。观察得到机票价钱与里程之间存在某种关系,设定 \( p_a = f_a(L) \)。同理,此处我们也需要对结果乘以 3 表示 3 人的费用,得:
\begin{equation}
p_a = 3 f_a(L)
\tag{3-6}
\end{equation}

从而,我们可以得到交通总花费为:
\begin{align}
p_{k1} &= \alpha_a p_a + \alpha_t p_t + \alpha_c p_c \\
&= 3 \alpha_a f_a(L) + 3 \alpha_t k_t L + \alpha_c (L + r \times 300 t_c)
\tag{3-7}
\end{align}

2) 第 \( k \) 次住宿总花费

问题二中提供了,住宿费简化为省会城市和旅游景区 200 元/人·天,地级市 150 元/人·天,县城 100 元/人·天,所以 \( p_{k2} = 200 \times t_1 + 150 \times t_2 + 100 \times t_3 \),\( t_1, t_2, t_3 \) 分别表示住宿省会城市、地级市、县城的天数。然而为了简化计算,结合实际情况,我们以地级市花费表示整个旅行的平均价格,则可以近似表示 \( p_{k2} = 150 \times t \)。需要注意 3 人同时出行的要求,所以要对结果乘以 3 表示总费用,即
\begin{equation}
p_{k2} = 3 \times 150 \times t_k
\tag{3-8}
\end{equation}

3) 旅行总花费

综上所述,可以得到旅游总费用为:
\begin{equation}
\min \sum_{k=1}^{m} P_{k} = \sum_{k=1}^{m} \left( p_{k1} + p_{k2} \right)
\tag{3-9}
\end{equation}

2. 约束条件分析

(1) 时间约束

由题目可知,该旅游爱好者每次出行旅行不能超过 15 天(此处表示 24h/天),而这些时间包括在路途中的时间和在旅游景点逗留的时间。令 $t_{ij}$ 表示从第 $i$ 个景点到第 $j$ 个景点路途中所需时间(已折合成“天”为单位),所以路途中所需总时间为 $\sum_{i=1}^{n} r_{ij} \times t_{ij}$;$t_{i}$ 表示会议代表们在第 $i$ 个景点的逗留时间,故代表们在旅游景点的总逗留时间为 $\sum_{i \neq j} t_{ij} x_{kij}$。故而得到如下约束条件:
\begin{equation}
\sum_{i \neq j} t_{ij} x_{kij} + \sum r_{ij} \times t_{ij} \leq 15 - a_{k}, \forall k
\tag{3-10}
\end{equation}
其中,$a_{k}$ 表示该天折合成以天为单位的住宿时间;$x_{ij} = \begin{cases} 1, & \text{从第 } i \text{ 个景点到达第 } j \text{ 个景点} \\ 0, & \text{其他} \end{cases}$。

(2) 旅游景点约束

根据假设,整个旅游路线是环形,即最终都要回到西安,因此 $\sum_{k} \sum_{m} x_{ij}$ 即表示旅游景点总数。又规定 10 年内需要游完 201 个景点,不难建立:
\begin{equation}
\sum_{k} \sum_{m} x_{ij} = 201
\tag{3-11}
\end{equation}

(3) 时间体验度约束

实际上,为保证出行旅游的满意度,假设时间体验度一般不能低于 0.6,所以此处需要对 $E$ 添加约束:
\begin{equation}
E = \frac{t_{s}}{t_{a}} \geq 0.6
\tag{3-12}
\end{equation}
即表示,景区旅行时间与乘坐交通工具时间的比为 6:4,否则将是一段及其不愉快的出行。

(4) 0—1 变量约束

将每个省看做图上的一个顶点,我们可以把所有的省连成一个圈。对于每个点来说,只允许最多一条边进入,同样只允许最多一条边出来,并且只要有一条边进入就要有一条边出去。因此可得约束:
\begin{equation}
\sum_{i \neq j} x_{kij} = 1, \forall k, i, \quad \sum_{i \neq j} x_{kji} = 1, \forall k, i
\end{equation}
综合以上可知:
\begin{equation}
\sum_{i} x_{kij} = \sum_{j} x_{kij} \leq 1
\tag{3-13}
\end{equation}
由于该旅游者每次旅游从西安出发且必须回到西安,则可得到约束条件如下:
\begin{equation}
\sum_{j=1}^{n} x_{k0j} = \sum_{j=1}^{n} x_{kj0} = m, \forall k
\tag{3-14}
\end{equation}
为避免出现从省 $i$ 到省 $j$ 和省 $j$ 到 $i$ 的重复,加入以下约束条件:

\begin{equation}
L_{kj} \geq L_{ki} + 1 + n(x_{kij} - 1), \forall k, i \neq j, j \neq 1
\tag{3-15}
\end{equation}

同样,当 \(i, j \geq 2\) 时,根据题意不可能出现 \(x_{ij} = x_{ji} = 1\),即不可能出现游客在两地间往返旅游,因为这样显然不满足游览景点尽量多的原则。因此我们可得约束:
\[
x_{ij} \times x_{ji} = 0
\]

\section{模型建立}

经过上述对目标函数和限制条件的讨论,我们可以建立如下模型:
\begin{align*}
\max \quad & E = \frac{t_s}{t_a} \\
\min \quad & \sum_{k=1}^m P_k = \sum_{k=1}^m (p_{k1} + p_{k2})
\end{align*}

约束条件如下:
\begin{equation}
\begin{aligned}
st. \quad & \left\{
\begin{aligned}
& \sum_{i \neq j} t_{ij} x_{kij} + \sum r_{ij} \times t_{ij} \leq 15 - a_k, \forall k \\
& \sum_{i \neq j} x_{kij} = 1, \forall k, i, \sum_{i \neq j} x_{kji} = 1, \forall k, i \\
& E = \frac{t_s}{t_a} \geq 0.6 \\
& \sum_i x_{kij} = \sum_j x_{kij} \leq 1 \\
& \sum_{j=1}^n x_{k0j} = \sum_{j=1}^n x_{kj0} = m, \forall k \\
& L_{kj} \geq L_{ki} + 1 + n(x_{kij} - 1), \forall k, i \neq j, j \neq 1 \\
& x_{ij} \times x_{ji} = 0
\end{aligned}
\right.
\end{aligned}
\end{equation}

\subsection{模型求解与结果分析}

\subsubsection{归一化处理}

经过上述对时间满意度和旅行总费用目标函数的讨论,我们分别可以初步得到两者的函数表达。但是,仔细观察可以发现,两者之间并不是绝对独立的。提高时间体验度,需要增加在景区游览时间,由于总体时间限制,所以势必要缩短在路程上的时间,则要改变更快的交通方式,于是交通花费就会相应提高,根据实际情况可得两者呈对立的关系。但是,时间满意度与花费具有不同的纲量,所以首先我们需要对费用进行归一化处理。其实观察可发现,事实上,对时间满意度 \(E\) 的求解也是一个归一化的过程,所以最终可以假设两者关系为:
\begin{equation}
\begin{aligned}
\varepsilon &= a(1 - E) + bP' \\
&= a\left(1 - \frac{t_s}{t_a}\right) + b\frac{P}{P_{\max}}
\tag{3-16}
\end{aligned}
\end{equation}

其中:
- \(P_{\max}\) 表示旅行总花费最大值,其意义表示在条件允许下以尽可能快的交通方式出行,即首选飞机的方式,其次高铁,最次选择汽车。
- \(a, b\) 取值因人而异,表示个人对旅行的要求,一般情况下都取 0.5。若个人追求旅行舒适度、良好体验的则适当增加 \(b\) 值,同时减小 \(a\);反之,亦然。

\begin{table}[h]
\centering
\caption{从北京出发到各省会城市距离\&全价机票}
\begin{tabular}{c c c c c c c c c}
\hline
到达 & 山西省 & 内蒙古 & 辽宁省 & 吉林省长 & 黑龙江 & 江苏省 & 上海市 & 南京市 \\
城市 & 太原市 & 呼和浩 & 沈阳市 & 春市 & 省哈尔滨 & & & \\
& & 特市 & & & 市 & & & \\
\hline
里程 & 0 & 1070 & 550 & 1470 & 1050 & 1090 & 1240 & 1780 \\
票价 & 0 & 510 & 490 & 700 & 980 & 1240 & 1210 & 1060 \\
\hline
到达 & 浙江省 & 安徽省 & 福建省 & 江西省 & 山东省济南 & 河南省 & 湖南省 & 湖北省 \\
城市 & 杭州市 & 合肥市 & 福州市 & 南昌市 & 市 & 郑州市 & 长沙市 & 武汉市 \\
\hline
里程 & 1860 & 1710 & 1680 & 1430 & 630 & 1160 & 1450 & 1850 \\
票价 & 1300 & 1040 & 1900 & 1430 & 440 & 700 & 1470 & 1200 \\
\hline
到达 & 广东省 & 广西南 & 海南省 & 重庆市 & 四川省成都市 & 贵州省 & 云南省 & 西藏拉萨市 \\
城市 & 广州市 & 宁市 & 海口市 & & & 贵阳市 & 昆明市 & \\
\hline
里程 & 1910 & 2150 & 2350 & 1640 & 1690 & 1980 & 2180 & 2930 \\
票价 & 2120 & 2340 & 2700 & 1750 & 1800 & 2130 & 2670 & 3650 \\
\hline
到达 & 陕西省 & 甘肃省 & 青海省 & 宁夏银川 & 新疆乌鲁木齐 & & & \\
城市 & 西安市 & 兰州市 & 西宁市 & 市 & & & & \\
\hline
里程 & 1850 & 1390 & 1740 & 1180 & 2630 & & & \\
票价 & 1080 & 1500 & 1700 & 1150 & 3170 & & & \\
\hline
\end{tabular}
\end{table}

\begin{figure}[h]
    \centering
    \includegraphics[width=\textwidth]{image.png}
    \caption{飞机花费与里程数的拟合关系}
    \label{fig:airfare}
\end{figure}

\section{高铁票价与里程的关系}

根据题目中所提供的附件4和12306铁路网的信息,对全国主要城市的距离和高铁价格进行观察,我们初步得出高铁票价与里程之间呈线性关系,具体详见目标函数部分高铁票价建模:

\begin{equation}
p_{t} = k_{t} \times L
\end{equation}

再加上通过网络查询进行验证,我们可以得到 $k_{t} \approx 0.5$。考虑3人同时出行,则最终可得高铁票价与里程之间的关系为

\begin{equation}
p_{t} = 1.5L
\tag{3-18}
\end{equation}

\section{飞机、高铁、汽车里程数与花费的关系}

根据上述分别可得3人出行的飞机、高铁、自驾(汽车)的花费-里程模型:

\begin{align}
p_{a} &= 0.0009L^{2} + 0.8655L + 469.74 \\
p_{t} &= 1.5L \\
p_{c} &= L + r \times 300t_{c}
\end{align}

通过对题目所提供的附件3省会之间距离的观察发现,西安与其他城市之间的最大距离为西安-拉萨2800km。又根据网络查询可知,飞机的平均速度在900km/h左右,所以若选择乘坐飞机出行,基本都可以1天以内到达目的地,不需要额外的住宿费开销。

根据题意要求,选择高铁出行当天乘坐高铁的时间不超过6个小时,按照250km/h的平均速度计算,可得乘坐高铁出行一天可到最远距离为1500km。根据实际情况,部分城市之间距离超过1500km,所以此处就需要考虑住宿问题。

同理,若选择汽车出行,则更需要住宿问题。根据假设所有路段都按高速公路处理,则汽车平均速度为90km/h,且一天最大行车时间为10h,所以每天最大行驶距离为900km。

\begin{align}
p_{a} &= 0.0009L^{2} + 0.8655L + 469.74 \\
p_{t} &= 1.5L + \left\lfloor \frac{L}{250 \times 6} \right\rfloor \times 450 \\
p_{c} &= L + r \times 300t_{c} + \left\lfloor \frac{L}{90 \times 10} \right\rfloor \times 450
\end{align}

考虑完全自驾情况下,即 $r = 0$。利用Matlab中取整函数和绘图函数画出在 $[0, 3000]$ 范围内的花费-里程关系。

的定义域上的上述三个函数图像,如图 3-2 所示。从图中可以清晰看出,随着里程数的增加,机票全价呈指数方式增加;而高铁和自驾费用则呈现阶梯型上升,上升趋势相对缓慢。比较高铁和自驾方式,两者在 $L \approx 700 \backslash L \approx 1400$ 处有重叠,但是从整体上看,高铁费用都要超出自驾费用,而且随着距离的增加,两者的差距将逐渐增大。

\begin{figure}[h]
    \centering
    \includegraphics[width=\textwidth]{image.png}
    \caption{飞机\&高铁\&自驾费用比较}
    \label{fig:3-2}
\end{figure}

此外,图中还不包括,乘坐飞机、高铁到达目的地后租车的费用,所以总体考虑三种方式,自驾无疑是最经济实惠的出行选择。

综上所述,针对交通工具的选择,我们可以得到以下初步结论:
\begin{enumerate}[label=\alph*)]
    \item 在该模型中,本文的思想是,在满足 10 年的前提条件下,尽可能提高旅游体验,减少旅行花费;
    \item 满足 10 年条件下,首选自驾出行,只对超长距离的城市采取飞机出行;
    \item 由于高铁的特殊性,全程基本不考虑乘坐高铁;
    \item 若从西安出发采取非自驾出行,则在后续旅程中需要额外租用汽车,且返回西安也只能采用非自驾方式。
\end{enumerate}

\section{粒子群算法多目标优化}

\subsection{粒子群优化算法}

粒子群优化算法(PSO 算法)是一种进化计算技术,最早是由 Kennedy 等于 1995 年提出的 \cite{Kennedy1995}. 系统初始化为一组随机解,通过迭代搜寻最优值,但没有遗传算法采用的交叉和变异,而是粒子在解空间追随最优粒子进行搜索。与遗传算法相比,PSO 的优势在于简单、易实现,而且没有更多的参数需要调整。

在求解单目标优化时,可以找到最优解。但在求解多目标优化问题时,由于有多个目标且存在目标之间的无法比较和冲突现象,要使所有的目标函数同时达到最优是不可能的,一个解可能在其中某个上是最好的,但在其他目标上则达不到最优,但是,可以存在这样的解:对一个或几个目标函数不可能进一步优化,对其他目标函数不至于劣化,这样的解称之为非劣最优解(Pareto optimal)。一般情况下,最优解不只一个,而是一个最优解集。多目标算法的工作就是,构造非支配集,并使非支配集不断逼近 Pareto 最优解集,最终达到最优。

\subsection{算法定义}

粒子群体:群体中有 $n$ 个景点构成的可行路径,这 $n$ 个景点都属于实际的全国省份节点位置数据,每个节点都有自己唯一的标示并且相邻景点间距离可知。而在这 $n$ 个景点中任意两个景点间都具有路径距离值。每个粒子都代表了一条走遍所有景点且每个景

点只走一次的可行路径,每个路径有景点序列组成,这些粒子构成了可行解空间。

粒子位置:记录了粒子的飞行轨迹路径,保存着所找路径中的景点信息。

粒子速度:景点间的交换集,即 \( v = \{(i, j) | i, j \in (1, 2, \cdots n), i \neq j\} \),表示了一个粒子中的第 \( i \) 个节点与第 \( j \) 个节点间进行排列的交换,粒子速度包含了任意两个景点间交换的可能的子交换且粒子速度具有次序性,速度中的子交换间是不能随意更改的。

适应度计算:粒子位置在当前迭代的路径长度的总权重。粒子所代表的目标函数如下:

\[
F(x) = \min \{f(x_i)\}
\tag{3-19}
\]

全局目标函数:是适应度计算的具体化,粒子 \( i \) 的当前最好位置设为 \( pbest_i \),\( k \) 代表了迭代次数,则满足下式:

\[
pbest_i^{k+1} =
\begin{cases}
pbest_i^k, & \text{if } f(x_i^{k+1}) \geq f(pbest_i^k) \\
x_i^{k+1}, & \text{otherwise}
\end{cases}
\tag{3-20}
\]

那么全局最优解 \( pbest_i^k \in \{pbest_0^k, pbest_1^k, pbest_2^k, \cdots pbest_N^k\} \),全局最优解相当于求解目标函数的具体化:

\[
f(pbest^k) = \min \{f(pbest_1^k), f(pbest_2^k), \cdots, f(pbest_N^k)\}
\tag{3-21}
\]

在本文中,我们使用的粒子群算法具体步骤如下:

1. 初始化。初始化 \( N \) 个粒子,粒子迭代次数,粒子群适应度值,并随机初始化每个粒子的位置 \( x_i^0 \) 及速度 \( v_i^0 \)。

2. 判断是否已达算法结束条件。如果是则转到步骤 6,否则进入步骤 3。

3. 根据公式 (3-19) 和 (3-20) 评价每个粒子适应度值。将最初的适应度值作为粒子的当前最优值 \( pbest_i \),\( i \) 属于 1 到 \( N \) 区间上的整数。按照公示 (3-21) 求解当前这代的全局最优解 \( pbest \)。

4. 更新粒子的速度和位置。按照

\[
\begin{cases}
v_{id}^{k+1} = w v_{id}^k \oplus c_1 r_1^k (pbest_{id}^k - x_{id}^k) \oplus c_2 r_2^k (gbest_d^k - x_{id}^k) \\
x_{id}^{k+1} = x_{id}^k + v_{id}^{k+1}
\end{cases}
\]

更新粒子速度和位置,产生新一代的粒子速度和位置。

5. 评价每个粒子适应值。根据粒子目标函数和步骤 3 中的 \( pbest_i \)、\( gbest \) 的值作值比较,再按照,具体化目标函数和全局目标函数对 \( pbest_i \)、\( gbest \) 的值进行替换。

6. 判断是否达到了参数中的迭代次数或规定的适应度值。如果是则算法运行结束,输出全局最优解及粒子群最优路径,否则转到步骤 4。

(3) 数据处理

本节算法主要在 matlab 环境下用 C 语言实现。采用全国各省份(以省会作为节点)之间距离作为数据,并以此作为两点之间的权值再进行下一步的计算。根据粒子群算法,辅以相应的修改,得到各省会的出行顺序和合理的出行方式,且将其消耗时间控制在 10 年以内。问题一中已经提供了各个省份内景点的游览顺序,将上述运算得到结果结合问题一的答案,便可以得到问题二所需要的旅游方案。

\subsection*{3.3 最终模型方案}

观察最终模型方案发现,其实在算法主要在问题一的基础上,处理了超远距离出性

\begin{table}
\centering
\caption{旅行出行规划}
\begin{tabular}{c|c|c|c|c|c|c|c|c|c|c|c|c}
\hline
N & 起点 & 里程/历时 & 第一站/所在省游玩时间(天) & 里程/历时 & 第二站/所在省游玩时间 & 里程/历时 & 第三站/所在省游玩时间 & 里程/历时 & 终点 & 总天数/天 & 时间满意度 & 年份 & 花费/元 \\
\hline
1 & 西安 & 1090km/11.5h & 江苏/12 & \multicolumn{4}{c|}{1090km/11.5h} & \multirow{10}{*}{西安} & 15 & 0.800 & 1 & 2080 \\
\hline
2 & & 1090km/11.5h & 江苏/1.5+1.5 & 640km/7h & 山东/8d & \multicolumn{2}{c|}{910km/10h} & 13.5+1.5 & 0.81 & & 2640 \\
\hline
3 & & 600km/7h & 山西/6.5 & 223km/3.5h & 河北/6d & \multicolumn{2}{c|}{790km/9.5h} & 15 & 0.833 & 2 & 1613 \\
\hline
4 & & 790km/9.5h & 河北/1 & 291km/4h & 北京&天津/7+2.5 & \multicolumn{2}{c|}{1100km/12h} & 12.5+2.5 & 0.84 & & 2181 \\
\hline
5 & & 1750km/飞机 & 辽宁/4+1 & \multicolumn{4}{c|}{1750km/飞机} & 5+1 & 0.83 & 3 & 9000 \\
\hline
6 & & 1090km/1d & 江西/7+3 & \multicolumn{4}{c|}{1090km/1d} & 9+3 & 0.83 & & 2180 \\
\hline
7 & & 920km/1d & 安徽/7+3 & \multicolumn{4}{c|}{920km/1d} & 9+3 & 0.83 & & 1840 \\
\hline
8 & & 2310km/飞机 & 黑龙江/7+2 & 272km/3h飞机 & 吉林5d & \multicolumn{2}{c|}{2040km飞机} & 13+2 & 0.93 & 4 & 10950 \\
\hline
9 & & 480km/1d & 河南/12+1 & \multicolumn{4}{c|}{480km/1d} & 14+1 & 0.87 & & 920 \\
\hline
10 & & 740km/2d & 湖北/10+1 & \multicolumn{4}{c|}{740km/2d} & 14+1 & 0.73 & 5 & 1480 \\
\hline
11 & & 2100km/飞机 & 海南/3.5 & 478km/飞机 & 广西/5+5 & \multicolumn{2}{c|}{1640km/飞机} & 10+5 & 0.90 & & 12240 \\
\hline
\end{tabular}
\end{table}

\begin{table}
\centering
\begin{tabular}{c|c|c|c|c|c|c|c|c|c}
\hline
 &  &  &  &  &  &  &  &  &  \\
\hline
1 & 1330k & 浙江/8 & 175k & 上海 &  & 1390km/2d & 15 & 0.733 & 2895 \\
2 & m/2d &  & m/2 & /3d &  &  &  &  &  \\
\hline
1 & 1650k & 福建 &  &  & 1650km/飞机 &  & 15 & 0.93 & 9180 \\
3 & m/飞机 & /9+5 &  &  &  &  &  &  &  \\
\hline
1 & 1650k & 广东 &  &  & 1650km/2d &  & 12. & 0.73 & 3300 \\
4 & m/2d & /8.5+2. &  &  &  &  & 5+2 &  &  \\
 &  & 5 &  &  &  &  & .5 &  &  \\
\hline
1 & 970k & 内蒙古 & 684k & 宁夏 & 44 & 甘肃 & 65 & 13. & 2744 \\
5 & m/1d3 & /1.5+1. & m/1 & /3.5d & 0k & /3.5d & 0k & 5+1 &  \\
 & h & 5 & d &  & m/ &  & m/ & .5 &  \\
 &  &  &  &  & 5h &  & 7h &  &  \\
\hline
1 & 2540k & 新疆 &  &  & 2540km/3.5d飞机 &  & 10. & 0.93 & 16944 \\
6 & m/飞机 & /9.5+4. &  &  &  &  & 5+4 &  &  \\
 &  & 5 &  &  &  &  & .5 &  &  \\
\hline
1 & 750k & 四川 & 712k & 陕西 &  & / & 15 & 0.87 & 1462 \\
7 & m/1d & /11 & m/1 & /2d &  &  &  &  &  \\
 &  &  & d &  &  &  &  &  &  \\
\hline
1 & 1070k & 贵州 & 506k & 云南 & 1590km/2d2h &  & 15 & 0.70 & 3166 \\
8 & m/1.5 & /4.5 & m/1 & /6d &  &  &  &  &  \\
 & d &  & d &  &  &  &  &  &  \\
\hline
1 & 700k & 重庆 & 685k & 陕西 & / &  & 14+ & 0.87 & 1385 \\
9 & m/1d & /9.5+1 & m/1 & /2.5d &  &  & 1 &  &  \\
 &  &  & d &  &  &  &  &  &  \\
\hline
2 & 2540k & 新疆乌 & 1463 & 新疆 & 3733km/飞机 &  & 6+3 & 0.83 & 21681 \\
0 & m/飞机 & 鲁木齐 & km/ & 喀什 &  &  &  &  &  \\
 &  & /3+3 & 飞机 & /1.5d &  &  &  &  &  \\
\hline
2 & 890k & 青海 & 1958 & 西藏 & 2840km/飞机 &  & 6+3 & 0.83 & 14100 \\
1 & m/飞机 & /2.5+3 & km/ & /2d &  &  &  &  &  \\
 &  &  & 飞机 &  &  &  &  &  &  \\
\hline
2 & 990k & 湖南 &  &  & 990km/1d &  & 8+4 & 0.83 & 1980 \\
2 & m/1d & /6+4 &  &  &  &  &  &  &  \\
\hline
\end{tabular}
\caption{表3-3 省内花费(元)}
\end{table}

\begin{table}
\centering
\begin{tabular}{c|c c c c c c c}
省份 & 江苏 & 山东 & 山西 & 河北 & 北京\&天津 & 黑龙江 \\
省内花费 & 8254.6 & 6240 & 3226 & 4730 & 4603 & 8741 \\
\hline
省份 & 辽宁 & 吉林 & 湖南 & 江西 & 安徽 & 河南 \\
省内花费 & 2795 & 5325 & 4780 & 5484 & 6289 & 9120.3 \\
\hline
省份 & 湖北 & 海南 & 广西 & 浙江 & 上海 & 福建 \\
省内花费 & 8799 & 2400 & 2327 & 6269 & 1920 & 6871 \\
\hline
省份 & 广东 & 内蒙古 & 宁夏 & 甘肃 & 新疆 & 青海 \\
省内花费 & 7127 & 1090 & 1820 & 4956 & 17849.3 & 2010 \\
\hline
省份 & 西藏 & 四川 & 陕西 & 贵州 & 云南 & 重庆 \\
省内花费 & 1800 & 8706 & 2680 & 3194.5 & 4647.4 & 6097 \\
\end{tabular}
\end{table}

根据满意度修改前:通过对表3-2的计算得,省间交通共花费125961元;对表3-3计算可得,省内交通住宿总费用为160151元。

针对时间满意度修改后,修共增加 44.5 天,于是需要增加相应的住宿费用 44.5×600=26700 元和租车费用 44.5×300=5550 元。综上所述,十年旅行共消费 125961+160151+26700+5550=318362 元。

\section*{4. 问题三的模型建立与求解}

\subsection*{4.1 数据分析}

在问题分析中,我们已经了解到行车时间对总旅游时间有一定的影响,故而我们选取了分布比较代表性的五个城市:北京、南京、南昌、西宁、广州和西安。分别计算出各省会城市与这五个城市的总里程之和,得到下图 3-1

\begin{figure}[h]
    \centering
    \includegraphics[width=\textwidth]{image.png}
    \caption{各城市与其他城市总里程汇总}
\end{figure}

从图中我们可以看出各城市与中心城市间的总里程数存在很大差异,由各城市的地理位置我们也可以粗略的分析出:西安地处中国中部,因此与全国各省间距离分布比较均衡,而其余城市分别位于中国的华南,东南等地区,与其他省会城市间的距离分布差异很大,故而总里程数比中部城市要多。

其次,在考虑费用与满意度之前我们首要考虑的是时间耗费问题。所以首先我们以北京为中心考虑该旅游爱好者全程自驾游遍全国的情况,这也可以看做 mTSP 问题,那么利用问题一中改良圈算法同样可以求解得到该旅游爱好者在这种情况下时间最优的旅游路线,见下表 3-1 从表中可以看到,该旅游爱好者若要通过全程自驾游游遍全国需要 12 年的时间,且总共的行车时间需要 124.5 天。显然,远远超过问题二中的行车天数,游遍全国也超过了 10 年。从表中也可以看出从北京出发游览西部、西南部和南部地区行车时间耗费过大,若采取飞机或高铁等将更为便捷的出行方式,可以节省更多的游览时间。

\begin{table}[h]
\centering
\caption{以北京为首末站全程自驾时间最优路线时间统计}
\begin{tabular}{|c|c|c|c|c|c|c|c|c|c|c|}
\hline
No. & 起点 & 历时 & 第一站/所在省游玩时间 & 历时 & 第二站/所在省游玩时间 & 历时 & 第三站/所在省游玩时间 & 历时 & 终点 & 总天数 & 行车天数 & 年份 \\
\hline
1 & 北京 & 1d & 天津/3d & 0.5d & 石家庄/7d & 0.5d & & & 北京 & 15d & 1d & 1 \\
\hline
2 & 北京 & 1d & 沈阳/4d & 0.5d & 长春/5d & 1.5d & & & & 12d & 3d & \\
\hline
\end{tabular}
\end{table}

\begin{table}
\centering
\begin{tabular}{c|c|c|c|c|c|c|c|c|c|c}
\hline
3 &  & 2d & 哈尔滨 & 2d &  &  &  & 11d & 4d & 2 \\
\hline
4 &  & 0.5d & 济南/8d & 0.5d &  &  &  & 10 & 1d & \multirow{2}{*}{3} \\
\hline
5 &  & 1d & 太原 & 1d & 呼和浩 & 1.5d & 西安 & 1.5d & 14.5d & 5d \\
 &  &  & /4.5d &  & 特/1.5d &  & /3.5d &  &  & \\
\hline
6 &  & 1d & 郑州 & 1d &  &  &  & 14d & 2d & \multirow{2}{*}{4} \\
 &  &  & /12d &  &  &  &  &  &  & \\
\hline
7 &  & 1.5d & 南京 & 1.5d &  &  &  & 15d & 3d & \\
 &  &  & /12d &  &  &  &  &  &  & \\
\hline
8 &  & 1.5d & 合肥/8d & 0.5d & 上海/3d & 2d &  & 15d & 4d & \multirow{2}{*}{5} \\
\hline
9 &  & 1.5d & 杭州/8d & 0.5d & 南昌/2d & 2d &  & 15d & 4d & \\
\hline
10 &  & 1.5d & 武汉/6d & 0.5d & 南昌/5d & 2d &  & 15d & 4d & \multirow{2}{*}{6} \\
\hline
11 &  & 1.5d & 武汉/5d & 0.5d & 长沙/6d & 2d &  & 15d & 4d & \\
\hline
12 &  & 3.5d & 南宁/1d & 1.5d & 重庆 & 2.5d &  & 15d & 7.5d & \multirow{2}{*}{7} \\
 &  &  &  &  & /6.5d &  &  &  &  & \\
\hline
13 &  & 2.5d & 福州/9d & 2.5d &  &  &  & 14d & 5d & \\
\hline
14 &  & 3d & 广州 & 1d & 海口 & 4d &  & 15d & 8d & \multirow{2}{*}{8} \\
 &  &  & /3.5d &  & /3.5d &  &  &  &  & \\
\hline
15 &  & 3d & 广州/5d & 1d & 南宁/3d & 3d &  & 15d & 7d & \\
\hline
16 &  & 4d & 昆明 & 1d & 成都/2d & 2.5d &  & 15d & 7.5d & \multirow{2}{*}{9} \\
 &  &  & /5.5d &  &  &  &  &  &  & \\
\hline
17 &  & 4.5d & 拉萨/2d & 3d & 西宁/2d & 2.5d &  & 14d & 10d & \\
\hline
18 &  & 2.5d & 重庆/2d & 0.5d & 贵阳/4d & 1d & 成都 & 2.5d & 15d & 6.5d & \multirow{2}{*}{10} \\
 &  &  &  &  &  &  & /2.5d &  &  & \\
\hline
19 &  & 2.5d & 成都 & 2d & 银川 & 1.5d &  & 15d & 6d & \\
 &  &  & /6.5d &  & /2.5d &  &  &  &  & \\
\hline
20 &  & 4.5d & 乌鲁木齐/6d & 4.5d &  &  &  & 15d & 9d & \multirow{2}{*}{11} \\
\hline
21 &  & 4.5d & 乌鲁木齐/6d & 4.5d &  &  &  & 15d & 9d & \\
\hline
22 &  & 4.5d & 乌鲁木齐/3d & 4.5d &  &  &  & 12d & 9d & \multirow{2}{*}{12} \\
\hline
23 &  & 2.5d & 兰州 & 2.5d & 北京/4d & 0d &  & 14.5d & 5d & \\
 &  &  & /5.5d &  &  &  &  &  &  & \\
\hline
\end{tabular}
\end{table}

为了控制总的旅游时间在十年以内,选取飞机或高铁等更高效的出行方式是必然的选择。然而选取这两种交通方式必然会带来花费的增多。故而如何在这在从行车时间和花费之间做好权衡是关键。以一人旅游为例,我们利用 MATLAB 绘制出三种交通方式在游玩 5 天与游玩 6 天(包括行车时间和游览时间)的不同情况下花费与里程数的关系。如图 4-2,图 4-3。

\begin{figure}[h]
    \centering
    \includegraphics[width=\textwidth]{image1.png}
    \caption{三种交通方式下一人游玩5天花费与里程的关系}
    \label{fig:4-2}
\end{figure}

\begin{figure}[h]
    \centering
    \includegraphics[width=\textwidth]{image2.png}
    \caption{三种交通方式下一人游玩6天花费与里程的关系}
    \label{fig:4-3}
\end{figure}

从对两张图的分析和对比我们可以看出,对比三种方式,在近距离($\leq 2000km$)的情况下,并不用过于考虑交通时间,由于自驾游本神的经济型,选择自驾游出行最为经济;相同条件下,游览时间越多,由于租车费用的增多,自驾游的花费也相对更经济一些。并且此处我们考虑的是一人出行,当多人出行时,飞机与高铁显然花费更多,而自驾游的油费基本固定。故而通过以上交通花费的分析,对于出行方式的选取上,对于短距离的出行主要选择自驾游,而对于长距离、出行时间花费较多的旅行路线出行选择飞机或高铁比较合适,然而由于远距离省会间高铁批次较少且中转不方便,有些甚至不存在高铁线路(比如北京至乌鲁木齐),故而在本问中不考虑选取高铁作为出行方式。因此,选取自驾为主,飞机为辅的出行方式。

\subsection{模型推广}

通过上述分析,接下来我们将针对问题二中所得建立的费用与体验度最优模型进行推广。

\subsubsection{目标函数的确立}

该问题与问题二目标一致,即设计出满足十年旅行时间限制,同时花费与体验度最优的目标。由于本问题未设置旅游人数,故而设旅游人数为 $A$。在目标函数中要加入旅游人数的考虑,故而目标函数为:

\begin{align*}
Max \ E &= \frac{t_s}{t_a} \\
Min \ \sum_{k=1}^{m} P_k' &= \sum_{k=1}^{m} (p_{k1}' + p_{k2}')
\end{align*}

其中: \( p_{k1}' = A\alpha_a f_a(L) + A\alpha_t k_t L + \alpha_c (L + r \times 300t_c) \), \( p_{k2}' = 150A \times t \).

\section*{2. 约束条件分析}

问题二中所得建立费用与体验度最优的旅游路线进行分析,由于在各省游览时间一定,故而我们可以得到十年行车时间的最大值(在本问中近似地设为 66 天)。要尽可能的缩短出行天数,故而得到以下约束条件:

\begin{equation}
\sum_{k=1}^{m} \sum_{i \neq j} t_{ij} x_{kij} \leq 70 \tag{4-1}
\end{equation}

\section*{3. 模型建立}

经过上述对目标函数和限制条件的讨论,我们可以建立如下模型:

\begin{align*}
Max \ E &= \frac{t_s}{t_a} \\
Min \ \sum_{k=1}^{m} P_k &= \sum_{k=1}^{m} (p_{k1}' + p_{k2}')
\end{align*}

约束条件:

\begin{equation}
st.
\begin{cases}
\sum_{i \neq j} t_{ij} x_{kij} + \sum r_{ij} \times t_{ij} \leq 15 - a_k, \forall k \\
\sum_{i \neq j} x_{kij} = 1, \forall k, i, \sum_{i \neq j} x_{kji} = 1, \forall k, i \\
\sum_i x_{kij} = \sum_j x_{kij} \leq 1 \\
E = \frac{t_s}{t_a} \geq 0.6 \\
\sum_{j=1}^{n} x_{k0j} = \sum_{j=1}^{n} x_{kj0} = m, \forall k \\
L_{kj} \geq L_{ki} + 1 + n(x_{kij} - 1), \forall k, i \neq j, j \neq 1 \\
x_{ij} \times x_{ji} = 0 \\
\sum_{k=1}^{m} \sum_{i \neq j} t_{ij} x_{kij} \leq 66
\end{cases}
\end{equation}

\subsection*{4.3 模型求解}

在本问中,考虑以北京市为游览首末站、一人出行的情况,利用问题二中给出的粒子群算法求解上小结得出的推广多目标优化模型,通过整理汇总得到最优路线,如下表 4-2:

\begin{table}[h]
\centering
\caption{以北京市为游览首末站,花费最优体验度最好的旅游路线安排}
\begin{tabular}{|c|c|c|c|c|c|c|c|c|c|c|}
\hline
N & 起 & 历时 & 第一站 & 历时 & 第二站/ & 历时 & 第三站/ & 历时 & 总天 & 行车 & 年 \\
O. & 点 & & 所在省 & & 所在省 & & 所在省 & & 数 & 天数 & 份 \\
& & & 游玩时间 & & 游玩时间 & & 游玩时间 & & & & \\
\hline
1 & 北 & 1d & 天津 & 0.5d & 石家庄 & 0.5d & & & 北 & 15d & 1 \\
\hline
\end{tabular}
\end{table}

\begin{tabular}{c|c|c|c|c|c|c|c|c|c|c}
 &  & /3d &  & /7d &  &  &  &  &  &  \\
\hline
2 & 京 & 1d & 沈阳/4d & 0.5d & 长春/5d & 1.5d &  &  &  &  \\
\hline
3 &  & 2d & 哈尔滨/7d & 2d &  &  &  &  & 12d & 3d \\
\hline
4 &  & 0.5d & 济南/8d & 0.5d &  &  &  &  & 11d & 4d \\
\hline
5 &  & 2.5d & 成都/3d & 2.5d &  &  &  &  & 10 & 1d \\
\hline
6 &  & 1d & 太原/4.5d & 1d & 呼和浩特/1.5d & 1.5d & 西安/3.5d & 1.5d & 8d & 5d \\
\hline
7 &  & 2.5d & 重庆/2d & 0.5d & 贵阳/4d & 1d & 成都/2.5d & 2.5d & 14.5d & 5d \\
\hline
8 &  & 1d & 郑州/12d & 1d &  &  &  &  & 15d & 6.5d \\
\hline
9 &  & 1.5d & 南京/12d & 1.5d &  &  &  &  & 14d & 2d \\
\hline
10 &  & 1.5d & 合肥/8d & 0.5d & 上海/3d & 2d &  &  & 15d & 3d \\
\hline
11 &  & 1.5d & 杭州/8d & 0.5d & 南昌/2d & 2d &  &  & 15d & 4d \\
\hline
12 &  & 1.5d & 武汉/6d & 0.5d & 南昌/5d & 2d &  &  & 15d & 4d \\
\hline
13 &  & 1.5d & 武汉/5d & 0.5d & 长沙/6d & 2d &  &  & 15d & 4d \\
\hline
14 &  & 3.5d & 南宁/1d & 1.5d & 重庆/6.5d & 2.5d &  &  & 15d & 7.5d \\
\hline
15 &  & 2.5d & 福州/9d & 2.5d &  &  &  &  & 14d & 5d \\
\hline
16 &  & 飞机/0.5d & 广州/8.5d & 飞机/0.5d & 海口/3.5d & 飞机/0.5d &  &  & 13.5d & 1.5d \\
\hline
17 &  & 飞机/0.5d & 乌鲁木齐/14d & 飞机/0.5d &  &  &  &  & 15d & 1d \\
\hline
18 &  & 3d & 南宁/3d & 3d &  &  &  &  & 9d & 7d \\
\hline
19 &  & 飞机/0.5d & 拉萨/2d & 飞机/0.5d & 西宁/2d & 飞机/0.5d &  &  & 10d & 1.5d \\
\hline
20 &  & 飞机/0.5d & 乌鲁木齐/1d & 飞机/0.5d & 兰州/5.5d & 飞机/0.5d & 银川/2.5d & 飞机/0.5d & 11d & 2d \\
\hline
21 &  & 飞机/0.5d & 昆明/5.5d & 飞机/0.5d & 成都/5.5d & 飞机/0.5d &  &  & 12.5d & 1.5d \\
\hline
22 &  & 2.5d & 重庆/2d & 0.5d & 贵阳/4d & 1d & 成都/2.5d & 2.5d & 15d & 6.5d \\
\hline
\end{tabular}

\subsection{旅游建议}

对于旅游爱好者,主要有以下几点建议:

1. 对于短途旅行并且人数较多的情况下,可以采取自驾游的方式。既节省花费又方便来回。
2. 对于长途旅行,如新疆、西藏等地,尽量选择飞机为交通工具,从而较少路途时间,提高旅游体验度。
3. 提前制定旅行计划,综合考虑各种因素,可利用本文模型建立最优旅行路线,争取在同样的旅行天数下,游玩的尽兴且花费相对较少。

对于旅游部门,主要有以下几点建议:

1. 对于较近的多个景区,可以捆绑起来出售低价套票,特别是 5A 景区和其他等级较低景区的套票销售。这样不仅可以增加景区的游览人气,而且减少了旅游者买票花费的时间并且降低的游览花费。
2. 在旅游景点较多的城市,开设专门的旅游租车服务。为旅游爱好者谋福利,节省更多租车费用用于旅游其他开销。

\section{问题四的模型建立与求解}

\subsection{数据分析}

问题提供的附件 1 给出了基于个人旅游偏好确定的每个 5A 级景区最少的观光时间。观察 201 个 5A 级景区,根据每个景点本身的特色和其相互之间的相似性,本文可以初步将景点分类:观光休闲型、体育探险型、历史文化型、科学技术型和宗教信仰型。根据对景点的分类和大众对不同景点爱好的差异性,也可以将所有自驾游旅行者依据个人爱好分为类似的 5 类。附件 1 和附件 7 分别罗列了 201 个 5A 景区和 1182 个 4A 景区。

利用问题一中所使用的地图汇制图网站,导入 4A 级景点信息,可以得到如图 5-1 所示的全国分布图。

\begin{figure}[h]
    \centering
    \includegraphics[width=\textwidth]{image.png}
    \caption{1182 个 4A 级景点分布图}
    \label{fig:4a_distribution}
\end{figure}

由图X可以看出,中国东部地区4A级景点较多,西部4A级景点较少。

假设具有某种或多种爱好的旅行者,出行会优先考虑同类型的景点,而综合爱好的复合型旅行者的出行则与上述的问题一、二、三类似了。

针对该问题的解答,本文的思想是:

a) 优先考虑5A级景区;

b) 适当舍弃部分5A级景区;

c) 在问题一、二、三的最优路线的基础上,合理增加该路线附近特征与爱好类似的4A级景区。

在附件1中,该旅行爱好者依据个人旅游偏好(即个人爱好)确定了在每个5A级景区最少的游览时间,其中半天和一天和两天所占景点数如表5-1所示,根据表中信息绘制出如图5-2所示的旅游时间比例饼状图。

\begin{table}[h]
\centering
\caption{表5-1 景区游览时间表}
\begin{tabular}{|c|c|c|}
\hline
天数 & 景点数 & 单个景点数 \\
\hline
半天 & 119 & 111 \\
\hline
一天 & 78 & 67 \\
\hline
两天 & 4 & 1 \\
\hline
\end{tabular}
\end{table}

\begin{figure}[h]
\centering
\includegraphics[width=0.8\textwidth]{image.png}
\caption{旅游时间占总时间比例图}
\end{figure}

观察图5-1和表5-2中的统计信息,不难发现游玩半天的景点数量最多,需要两天的景点数量最少。然而计算具体所占天数可以发现,参观1天的景点数虽然少,但是占总天数的比例最大。在游玩1天的景点里,通过初步观察可以得出,这些景点基本上都属于旅游风景区,所以可以基本上判断该旅行者的爱好为偏向为旅游观光型。

\subsection{5.2 模型建立}

1、目标函数的建立

在该旅行者尽可能玩遍5A级景区的情况下,多余时间可以玩相近的4A级旅游风景区。建立以下模型得到一个新的旅游体验度模型:

\begin{equation}
\max E' = \frac{t_s + 0.8t_f}{t_a}
\end{equation}

\begin{equation}
\varepsilon = a(1 - E) + bP'
\end{equation}

其中,$t_f$表示在4A级景区游览时间。

2、约束条件

考虑到部分 5A 级景区距离省会城市太远,将这些对少部分人具有吸引力的景点纳入旅行线路的制定上,显得太过牵强,甚至不太符合实际。所以此处给出一个距离约束条件以筛选是否去距离较远的 5A 级景区:

\[
\frac{P_{y}}{P_{s}} \geq 0.4
\]

其中,$P_{s}$ 为一个省内旅行的总距离。

剔除部分偏远景点既能节省出行的费用,又能增加时间游玩 4A 级的旅游风景区,整体上提高了出行的时间满意度。

除此之外的其他约束条件与模型三类似,这里就不再赘述。

3、模型建立

经过上述对目标函数和限制条件的讨论,可以建立如下模型为:

\[
\begin{aligned}
\varepsilon & = a(1 - E) + bP' \\
& = a\left(1 - \frac{t_{s} + 0.8t_{f}}{t_{a}}\right) + b\frac{P}{P_{\max}}
\end{aligned}
\]

约束条件可以类比上述问题:

\[
\begin{aligned}
st. \quad \left\{
\begin{array}{l}
\sum_{i \neq j} t_{ij} x_{kij} + \sum r_{ij} \times t_{ij} \leq 15 - a_{k}, \forall k \\
\sum_{i \neq j} x_{kij} = 1, \forall k, i, \sum_{i \neq j} x_{kji} = 1, \forall k, i \\
\frac{P_{y}}{P_{s}} \geq 0.4 \\
\sum_{i} x_{kij} = \sum_{j} x_{kij} \leq 1 \\
\sum_{j=1}^{n} x_{k0j} = \sum_{j=1}^{n} x_{kj0} = m, \forall k \\
L_{kj} \geq L_{ki} + 1 + n(x_{kij} - 1), \forall k, i \neq j, j \neq 1 \\
x_{ij} \times x_{ji} = 0
\end{array}
\right.
\end{aligned}
\]

5.3 模型求解

根据 $\frac{P_{y}}{P_{s}} \geq 0.4$ 的约束条件,可以对 201 个 5A 级景区进行率先,剔除以下景点:

a) 喀什地区噶尔老城景区

b) 喀什地区泽普县金胡杨景区

c) 阿勒泰地区富蕴县可可托海景区

d) 酒泉市敦煌沙山月牙泉景区

e) 大兴安岭地区漠河北极村旅游景区

这样在保证每次出行的路线基本不变的前提下,时间充裕情况下,则选择额外邻近的 4A 级旅游风景区;若没有邻近 4A 级旅游风景区,则把多余时间留给其他线路。通过这种方式,可以得到满意度最佳的旅行。

根据上述模型,类似与问题一的模型解决方法,利用蚁群算法可以求解出近似最优

\section*{图 5-3 福建 4A 级与 5A 级分布图}

根据图中可以看出一些 5A 级景点与 4A 级景点距离十分靠近。另外根据该自驾游爱好者的个人偏好和所建立的模型,除了游玩遍 5A 级景点外,还可以游玩永泰青云山风景名胜区、福清石竹山风景名胜区、平和县三平风景区和福州贵安新天地旅游景区。根据问题一中的 Hamilton 圈算法,求得省内最少路程。最终得到从西安出发旅行福建省部分具体安排如下:

福州市三坊七巷景区—福州贵安新天地旅游景区—体验风土人情,前往宁德市—宁德市福鼎太姥山旅游区—宁德屏南(白水洋·鸳鸯溪)旅游景区—南平武夷山风景名胜区—三明泰宁风景旅游区—福建土楼(永定·南靖)旅游景区—平和县三平风景区—厦门鼓浪屿风景名胜区—泉州市清源山风景名胜区—永泰青云山风景名胜区—福清石竹山风景名胜区—福州市三坊七巷景区

未剔除 5A 级景点的省份按照问题二中方式游玩。

\section*{6. 模型评价与改进}

本文所建模型主要有以下几个优点:

1. 采取更为高效的蚁群算法,解决了路线规划的问题;
2. 对于票价采取了拟合的方法,经过检验,在一定误差允许范围内切实可行,避免了大量数据的查找;
3. 建立的模型求解最短路程时,与实际路程基本保持一致;
4. 通过改变时间和价钱的权重即可得到各种需求的旅行方案,具有较好的可移植性,适用性广。

然而在考虑问题时还是不够全面,故而所建模型存在如下不足:

\begin{enumerate}
    \item 求 Hamilton 圈的时候考虑的是各城市之间的直线距离,并没有考虑实际的行驶路线,只从最优条件考虑,不够精确;
    \item 对于票价的拟合程度仍然有着些许误差;
    \item 飞机票普遍有打折,但是打折程度不好控制,因此对于飞机票价格数据会有出入。
\end{enumerate}

对于模型还存在一定的优化与改进空间,比如对各省时间最优的旅行路线进行改进,考虑各景点间具体高速公路与普通公路的分布,得到更准确的各省最优旅行路线。

\section*{参考文献}

[1] 方冬云, 图论在旅游线路选择中的应用[J], 长春工业大学学报: 自然科学版, 30(5): 582-586, 2009.

[2] 王建忠, 唐红, TSP 问题的一种快速求解算法[J], 微电子学与计算机, 28(1): 7-10, 2011.

[3] 崔之熠, 王茂芝, 刘国涛, 等, 蚂蚁算法在 TSP 问题求解的应用[J], 四川理工学院学报: 自然科学版, 24(3): 334-337, 2011.

[4] 曹文梁, 康岚兰, 基于遗传算法的混合蚁群算法及其在 TSP 中的应用研究[J], 制造业自动化, 33(2): 91-93, 2011.

[5] 邱树伟, 改进的连续 Hopfield 网络求解组合优化问题——以 TSP 求解为例[J], 湖南工业大学学报, 25(3): 42-46, 2011.

[6] 郑宣耀, 基于并行模拟退火算法的 TSP 问题求解[J], 唐山师范学院学报, 27(5): 50-53, 2005.

[7] 周晓蒙, 徐小明, 改进的求解旅行商问题的自组织特征映射网络[J], 计算机应用, 32(07): 1962-1964, 2012.

[8] 周辉仁, 唐万生, 魏颖辉, 基于 GA 的最小旅行时间的多旅行商问题研究[J], 计算机应用研究, 26(7): 2526-2529, 2009.

[9] 吴斌, 史忠植, 一种基于蚁群算法的 TSP 问题分段求解算法[J], 计算机学报, 24(12): 1328-1333, 2001.

[10] 伍文城, 肖建, 基于蚁群算法的中国旅行商问题满意解[J], 计算机与现代化, (8): 6-8, 2002.

[11] 马云峰, 杨超, 张敏, 郝春艳, 基于时间满意的最大覆盖选址问题[J], 中国管理科学, 02:45-51, 2006.

[12] 张利彪, 周春光, 马铭, 等, 基于粒子群算法求解多目标优化问题[J], 计算机研究与发展, 41(7): 1286-1291, 2004.

\section*{附录}

\section*{附录 1: Hamilton 圈 MATLAB 程序}

\begin{verbatim}
function [C d1] = glf(d)
%d表示图的权值矩阵
%C表示算法改良后的hamilton圈
n = size(d, 2);
C = [linspace(1, n, n) 1];
d2 = 0;
for l = 1:100
    %C = [i 57 1:57]
    C1 = C;
    if n > 3
        for v = 4:n+1
            for i = 1:(v-3)
                for j = (i+2):(v-1)
                    if (d(C(i), C(j)) + d(C(i+1), C(j+1)) < d(C(i), C(i+1)) + d(C(j), C(j+1)))
                        C1(1:i) = C(1:i);
                        for k = (i+1):j
                            C1(k) = C(j+i+1-k);
                        end
                        C1((j+1):v) = C((j+1):v);
                    end
                end
            end
        end
    elseif n <= 3
        if n <= 2
            fprintf('it does not exist hamilton circle');
        else
            fprintf('any circle is the right answer');
        end
    end
    C = C1;
    d1 = 0;
    for i = 1:n
        d1 = d1 + d(C(i), C(i+1));
    end
    d1
    C
end
\end{verbatim}

\%经纬度求直线距离

\begin{verbatim}
[i] = xlsread('D:\work\shumo\lvx1.xlsx');
[j] = xlsread('D:\work\shumo\lvx2.xlsx');
v = [i, j];
for i = 1:19
    for j = 1:19
\end{verbatim}

\begin{verbatim}
W(i,j)=sqrt((v(i,1)-v(j,1))^2+(v(i,2)-v(j,2))^2);
end
end
\end{verbatim}

附录2 蚁群算法程序

\textcolor{green}{\%路线规划}

\textcolor{blue}{while} NC <= NC_max

t=zeros(m,1);

\textcolor{blue}{for} j=2:(n+3)

\textcolor{blue}{for} i=1:m

visited=travle(i,1:(j-1));

J=zeros(1,(n-j+1+t(i)));

p=zeros(1,(n-j+1+t(i)));

p_cum=zeros(1,(n-j+1+t(i)));

Jc=1;

sumload1=zeros(m,(n-j+1+t(i)));

\textcolor{blue}{for} k=1:n

\textcolor{blue}{if} length(find(visited==k)~=0)

\textcolor{blue}{continue}

\textcolor{blue}{else}

J(Jc)=k;

Jc=Jc+1;

\textcolor{blue}{end}

\textcolor{blue}{end}

a3=J

\textcolor{blue}{for} k=1:length(J)

sumload1(i,k)=sumload(i)+C(J(k),3);

\textcolor{blue}{if} sumload1(i,k)<=9

p(k)=(Tao(visited(end),J(k))^Alpha)*(Yita(visited(end),J(k)^Beta);

\textcolor{blue}{else}

p(k)=0;

\textcolor{blue}{end}

\textcolor{blue}{end}

a7=sumload1(i,:);

a8=p;

p_sum=sum(p);

sumload1=zeros(m,(n-j+1+t(i)));

\textcolor{blue}{if} sum(p)~=0

\textcolor{blue}{for} k=1:length(J)

p_cum(k)=p(k)/p_sum;

\textcolor{blue}{end}

a4=p_cum;

h=pchioce(p_cum);

visit=J(h);

sumload(i)=sumload(i)+g(visit);

travel(i,j)=visit;

travelk(i,j)=visit;

R=travle(i,:);

t(i)=t(i)

\textcolor{blue}{else}

travelk(i,j)=1;

\end{verbatim}

\begin{verbatim}
sumload(i) = 0;
t(i) = t(i) + 1
end
b = find(travelk(i, :))
c = length(b)
if c >= n + 3
    d = travelk
    if NC >= 2
        travel(1, :) = R_best(NC - 1, :);
    end
    for k = 1:(n + 2)
        L(i) = L(i) + D(travelk(i, k), travelk(i, k + 1));
    end
    L(i) = L(i) + D(travelk(i, n + 3), 1)
else
    continue
end
end
end
\end{verbatim}

\textbf{附录2: 各省时间最优旅游路线}

\begin{table}[h]
\centering
\begin{tabular}{|c|c|c|c|c|c|c|c|}
\hline
省份 & 旅游 & 出发地 & 行车 & 行车 & 目的地(景点) & 总计 & 各省费用/元 \\
 & 天数 & & 里程 & 时间 & & 天数 & \\
 & & & /km & /h & & & \\
\hline
\multirow{12}{*}{江苏} & 1 & 苏州市区 & 50 & 1 & 苏州吴中太湖旅游区(旺山 - 穹窿山 - 东山) & \multirow{12}{*}{12} & \multirow{12}{*}{8254.6} \\
\cline{2-6}
 & 2 & 常熟市区 & 130 & 3 & 苏州常熟沙家浜 - 虞山尚湖旅游区 & & \\
\cline{2-6}
 & 3 & 苏州市区 & 0 & 0 & 苏州园林(拙政园 - 留园 - 虎丘) & & \\
\cline{2-6}
 & 3 & 苏州市区 & 90 & 2 & 苏州昆山周庄古镇景区 & & \\
\cline{2-6}
 & 4 & 苏州市区 & 0 & 0 & 苏州市金鸡湖国家商务旅游示范区 & & \\
\cline{2-6}
 & 4 & 苏州市区 & 110 & 2.5 & 苏州吴江同里古镇景区 & & \\
\cline{2-6}
 & 5 & 无锡市区 & 40 & 1 & 无锡灵山大佛景区 & & \\
\cline{2-6}
 & 5 & 无锡灵山大佛景区 & 40 & 1 & 中央电视台无锡影视基地三国水浒城景区 & & \\
\cline{2-6}
 & 6 & 无锡市区 & & & 无锡鼋头渚景区 & & \\
\cline{2-6}
 & 6 & 无锡鼋头渚景区 & 60 & 1 & 常州环球恐龙城景区(中华恐龙园 - 恐龙谷温泉 - 恐龙城大剧院) & & \\
\cline{2-6}
 & 7 & 常州市区 & 0 & 0 & 常州溧阳市天目湖景区(天目湖 - 南山竹海 - 御水温泉) & & \\
\cline{2-6}
 & 8 & 常州市区 & 77 & 1 & 镇江三山风景名胜区(金山 - 北固山 - 焦山) & & \\
\cline{2-6}
 & 9 & 镇江市区 & 130 & 2.5 & 镇江句容茅山景区,结束后 & & \\
\hline
\end{tabular}
\end{table}

\begin{table}
\begin{tabular}{|c|c|c|c|c|c|c|c|}
\hline
 &  &  &  &  &  &  &  \\
\hline
 &  &  &  & 前往南京体验风土人情 &  &  &  \\
\hline
 & 10 & 南京市区 & 0 & 0 & 南京夫子庙一秦淮河风光带 (江南贡院-白鹭洲-中华门- 瞻园-王谢故居), &  &  \\
\hline
 & 10 & 南京市区 & 0 & 0 & 南京钟山一中山陵风景名胜 区(明孝陵-音乐台-灵谷寺- 梅花山-紫金山天文台),结束后体验风土人情 &  &  \\
\hline
 & 11 & 南京市区 & 100 & 1.1 & 扬州瘦西湖风景区 &  &  \\
\hline
 & 11 & 扬州市区 & 100 & 2 & 泰州姜堰区溱湖国家湿地公 园 &  &  \\
\hline
 & 12 & 泰州市区 & 127.6 & 1.4 & 南通市濠河风景区 &  &  \\
\hline
\multirow{9}{*}{ 山东 } & 1 & 烟台市区 & 0 & 0 & 烟台蓬莱阁一三仙山一八仙 过海旅游区 & \multirow{9}{*}{ 8 } & \multirow{9}{*}{ 6240 } \\
\hline
 & 1 & 烟台蓬莱 阁一三仙 山一八仙 过海旅游 区 & 220 & 3 & 烟台龙口南山景区 &  &  \\
\hline
 & 2 & 烟台市区 & 320 & 4 & 威海刘公岛景区 &  &  \\
\hline
 & 3 & 青岛市区 & 160 & 1.5 & 青岛崂山景区 &  &  \\
\hline
 & 4 & 潍坊市区 & 180 & 3 & 山东沂蒙山旅游区(沂山景 区一龟蒙景区一云蒙景区) &  &  \\
\hline
 & 5 & 潍坊市区 & 200 & 2.5 & 济南天下第一泉景区(趵突 泉一大明湖一五龙潭一环城 公园-黑虎泉) &  &  \\
\hline
 & 6 & 济南市区 & 80 & 1.5 & 体验风土人情,去泰安 &  &  \\
\hline
 & 7 & 济南市区 & 220 & 3 & 泰安泰山景区 &  &  \\
\hline
 & 8 & 枣庄市区 & 60 & 1.5 & 枣庄台儿庄古城景区 &  &  \\
\hline
\multirow{7}{*}{ 山西 } & 1 & 大同市区 & / & / & 大同云岗石窟 & \multirow{7}{*}{ 4.5 } & \multirow{7}{*}{ 3226 } \\
\hline
 & 1 & 大同云岗 石窟 & 128 & 3 & 忻州五台山 &  &  \\
\hline
 & 2 & 忻州五台 山 & 130 & 3.5 & 晋中乔家大院 &  &  \\
\hline
 & 3 & 晋中乔家 大院 & 28 & 1 & 晋中市介休市区 &  &  \\
\hline
 & 3 & 晋中市介 休市区 & 30 & 1 & 晋中市介休市绵山风景名胜 区 &  &  \\
\hline
 & 3 & 晋中市介 休市绵山 风景名胜 区 & 30 & 1 & 平遥古城 &  &  \\
\hline
 & 4 & 平遥古城 & 180 & 4 & 皇城相府 &  &  \\
\hline
 & 5 & 皇城相府 &  &  & 0.5d &  &  \\
\hline
\end{tabular}
\end{table}

\begin{table}
\begin{tabular}{|c|c|c|c|c|c|c|c|}
\hline
 & 1 & 石家庄市区 & 90 & 2 & 平山县西柏坡景区 & \multirow{7}{*}{7} & \multirow{7}{*}{4730} \\
\hline
 & 2 & \multicolumn{4}{c|}{石家庄-游览特色建筑&体验风土人情} & & \\
\hline
 & 3 & 石家庄市区 & 120 & 3 & 白洋淀 & & \\
\hline
 & 4 & 白洋淀 & 80 & 2 & 涞水县 & & \\
\hline
 & 5 & 涞水县 & 200 & 4 & 承德避暑山庄 & & \\
\hline
 & 6 & 承德避暑山庄 & 130 & 2.5 & 山海关 & & \\
\hline
 & 7 & 山海关 & \multicolumn{3}{c|}{1d} & & \\
\hline
\multirow{11}{*}{北京&天津} & 1 & 北京市区 & 20 & 0.5 & 故宫博物院 & \multirow{11}{*}{7} & \multirow{11}{*}{4603} \\
\hline
 & 1 & 故宫博物院 & 13 & 0.5 & 天坛 & & \\
\hline
 & 2 & 北京市区 & 40 & 1 & 颐和园 & & \\
\hline
 & 2 & 颐和园 & 40 & 1 & 恭王府 & & \\
\hline
 & 3 & \multicolumn{4}{c|}{北京-游览特色建筑&体验风土人情} & & \\
\hline
 & 4 & 北京市区 & 60 & 1.5 & 明十三陵 & & \\
\hline
 & 4 & 明十三陵 & 10 & 0.5 & 北京奥林匹克公园 & & \\
\hline
 & 5 & 北京市区 & 70 & 1.5 & 八达岭 & & \\
\hline
 & 5 & 八达岭 & 70 & 1.5 & 天津蓟县 & & \\
\hline
 & 6 & 天津蓟县 & 80 & 1.5 & 古文化旅游区 & & \\
\hline
 & 7 & \multicolumn{4}{c|}{天津-游览特色建筑&体验风土人情} & & \\
\hline
\multirow{9}{*}{黑龙江} & 1 & 牡丹江宁安市镜泊湖景区 & 410 & 5 & 哈尔滨市区 & \multirow{9}{*}{7} & \multirow{9}{*}{8741} \\
\hline
 & 2 & 哈尔滨市区 & / & / & 哈尔滨太阳岛景区 & & \\
\hline
 & 3 & \multicolumn{4}{c|}{哈尔滨-游览特色建筑&体验风土人情} & & \\
\hline
 & 4 & 哈尔滨市区 & 180 & 2 & 伊春市区 & & \\
\hline
 & 4 & 伊春市区 & 120 & 2.5 & 伊春市汤旺河林海奇石景区 & & \\
\hline
 & 5 & 伊春市区 & 330 & 5 & 黑河五大连池景区 & & \\
\hline
 & 5 & 黑河五大连池景区 & \multicolumn{3}{c|}{0.5d} & & \\
\hline
 & 5 & 黑河五大连池景区 & 700 & 15.5 & 大兴安岭地区漠河北极村旅游景区 & & \\
\hline
 & 6 & 黑河五大连池景区 & 701 & 15.6 & 大兴安岭地区漠河北极村旅游景区 & & \\
\hline
 & 7 & 大兴安岭地区漠河北极村旅游景区 & \multicolumn{3}{c|}{1d} & & \\
\hline
辽 & 1 & \multicolumn{4}{c|}{沈阳-游览特色建筑&体验风土人情} & 4 & 2795 \\
\hline
\end{tabular}
\end{table}

\begin{table}
\begin{tabular}{|c|c|c|c|c|c|c|c|}
\hline
\multirow{4}{*}{ 宁 } & 2 & 沈阳市区 & 25 & 0.6 & 沈阳植物园 & \multirow{4}{*}{} & \multirow{4}{*}{} \\
\hline
& 2 & 沈阳植物园 & 100 & 1.5 & 本溪水洞 & & \\
\hline
& 3 & 本溪水洞 & 220 & 4.5 & 大连金石滩 & & \\
\hline
& 4 & 大连金石滩 & 50 & 1 & 老虎滩海洋公园 & & \\
\hline
\multirow{6}{*}{ 吉林 } & 1 & 长春市区 & / & / & 长春伪满皇宫博物馆 & \multirow{6}{*}{ 5 } & \multirow{6}{*}{ 5325 } \\
\hline
& 1 & 长春伪满皇宫博物馆 & 10 & 0.5 & 长春净月潭景区 & & \\
\hline
& 2 & 长春净月潭景区 & / & / & 长春市长影世纪城景区 & & \\
\hline
& 3 & & & & 长春-游览特色建筑&体验风土人情 & & \\
\hline
& 4 & 长春市区 & 420 & 7 & 长白山景区 & & \\
\hline
& 5 & 长白山景区 & & & 1d & & \\
\hline
\multirow{7}{*}{ 湖南 } & 1 & 长沙 & 0 & 0 & 长沙岳麓山一橘子洲旅游区 & \multirow{7}{*}{ 6 } & \multirow{7}{*}{ 4780 } \\
\hline
& 1 & 长沙岳麓山一橘子洲旅游区 & 60 & 1 & 长沙市宁乡县花明楼景区 & & \\
\hline
& 2 & 长沙市宁乡县花明楼景区 & 80 & 1.5 & 湘潭韶山旅游区 & & \\
\hline
& 2 & 湘潭韶山旅游区 & 80 & 1.5 & 长沙 & & \\
\hline
& 3 & 长沙 & 322 & 3.75 & 张家界武陵源一天门山旅游区 & & \\
\hline
& 5 & 张家界武陵源一天门山旅游区 & 444 & 6.5 & 衡阳南岳衡山旅游区 & & \\
\hline
& 6 & 衡阳南岳衡山旅游区 & 194 & 2.75 & 郴州市东江湖旅游区 & & \\
\hline
\multirow{7}{*}{ 江西 } & 1 & 吉安市区 & 190 & 3 & 吉安井冈山风景旅游区 & \multirow{7}{*}{ 7 } & \multirow{7}{*}{ 5484 } \\
\hline
& 2 & 赣州市区 & 430 & 4.8 & 赣州市瑞金市共和国摇篮景区 & & \\
\hline
& 3 & 鹰潭市区 & 100 & 1.5 & 鹰潭市贵溪龙虎山风景名胜区 & & \\
\hline
& 4 & 上饶市区 & 0h & 0 & 上饶三清山旅游景区 & & \\
\hline
& 5 & 上饶市区 & 180 & 3 & 上饶婺源县江湾景区 & & \\
\hline
& 6 & 上饶市区 & 200 & 3 & 景德镇古窑民俗博览区 & & \\
\hline
& 6 & 景德镇古窑民俗博 & 144 & 2 & 去九江 & & \\
\hline
\end{tabular}
\end{table}

\begin{table}
\begin{tabular}{|c|c|c|c|c|c|c|c|}
\hline
 &  & 览区 &  &  &  &  &  \\
\hline
 & 7 & 九江市区 & 40 & 1 & 九江庐山风景名胜区 &  &  \\
\hline
\multirow{8}{*}{安徽} & 1 & 黄山市区 & 120 & 2 & 黄山市黄山风景区 & \multirow{8}{*}{8} & \multirow{8}{*}{6289} \\
\hline
 & 2 & 黄山市区 & 130 & 3 & 黄山市黟县皖南古村落一西递宏村 &  &  \\
\hline
 & 3 & 黄山市区 & 320 & 4 & 黄山市古徽州文化旅游区 (徽州古城一牌坊群鲍家花园一唐模一潜口民宅一呈坎) &  &  \\
\hline
 & 4 & 池州市 & 159 & 3 & 池州青阳县九华山风景区 &  &  \\
\hline
 & 5 & 安庆市 & 110 & 2 & 安庆潜山县天柱山风景区 &  &  \\
\hline
 & 6 & 宣城市 & 200 & 3 & 宣城市绩溪县龙川景区 &  &  \\
\hline
 & 7 & 六安市 & 110 & 2.5 & 六安市金寨县天堂寨旅游区 &  &  \\
\hline
 & 8 & 六安市 & 340 & 3.5 & 阜阳市颍上县八里河风景区 &  &  \\
\hline
\multirow{14}{*}{河南} & 1 & 安阳市区 & 186.2 & 2 & 安阳殷墟景区 & \multirow{14}{*}{12} & \multirow{14}{*}{9120.3} \\
\hline
 & 2 & 焦作市区 & 0 & 0 & 焦作(云台山一神农山一青天河)风景区 &  &  \\
\hline
 & 3 & 焦作市区 & 133 & 1.5 & 焦作(云台山一神农山一青天河)风景区 &  &  \\
\hline
 & 4 & 开封市区 & 76,6 & 1 & 开封清明上河园景区 &  &  \\
\hline
 & 5 & 郑州市区 & 0 & 0 & 郑州登封嵩山少林景区 &  &  \\
\hline
 & 6 & 郑州市区 & 137.7 & 1.5 & 体验风土人情,前往洛阳市 &  &  \\
\hline
 & 6 & 洛阳市区 & 0 & 0 & 洛阳龙门石窟景区 &  &  \\
\hline
 & 7 & 洛阳龙门石窟景区 & 170 & 3 & 洛阳嵩县白云山景区 &  &  \\
\hline
 & 8 & 洛阳市区 & 223.4 & 2.5 & 洛阳新安县龙潭大峡谷景区 &  &  \\
\hline
 & 9 & 平顶山市区 & 150 & 2.5 & 平顶山鲁山县尧山一中原大佛景区 &  &  \\
\hline
 & 10 & 汝州市区 & 340 & 4.5 & 洛阳栾川县老君山一鸡冠洞旅游区 &  &  \\
\hline
 & 11 & 洛阳市区 & 150 & 2.5 & 洛阳栾川县老君山一鸡冠洞旅游区 &  &  \\
\hline
 & 11 & 洛阳栾川县老君山一鸡冠洞旅游区 & 220 & 2.3 & 去南阳市 &  &  \\
\hline
 & 12 & 南阳市 & 210 & 3 & 南阳西峡伏牛山老界岭·恐龙遗址园旅游区 &  &  \\
\hline
\multirow{4}{*}{湖北} & 1 & 武汉市区 & 0 & 0 & 武汉黄鹤楼公园 & \multirow{4}{*}{11} & \multirow{4}{*}{8799} \\
\hline
 & 1 & 武汉市区 & 140 & 3 & 武汉市黄陂木兰文化生态旅游区 &  &  \\
\hline
 & 2 & 武汉市区 & 409 & 3.5 & 武汉市东湖景区 &  &  \\
\hline
 & 4 & 宜昌市 & 70 & 1.5 & 宜昌长阳县清江画廊景区 &  &  \\
\hline
\end{tabular}
\end{table}

\begin{table}
\begin{tabular}{|c|c|c|c|c|c|c|c|}
\hline
 & 4 & 宜昌长阳县清江画廊景区 & 100 & 2 & 宜昌秭归县屈原故里文化旅游区 & & \\
\hline
 & 6 & 宜昌市区 & 40 & 1 & 宜昌三峡大坝旅游区 & & \\
\hline
 & 7 & 宜昌三峡大坝旅游区 & 360 & 2 & 宜昌三峡人家风景区 & & \\
\hline
 & 8 & 宜昌市区 & 170 & 3 & 恩施土家族苗族自治州恩施大峡谷景区 & & \\
\hline
 & 8 & 恩施市 & 240 & 5 & 恩施土家族苗族自治州巴东神龙溪纤夫文化旅游区 & & \\
\hline
 & 8 & 宜昌市 & 230 & 4.5 & & & \\
\hline
 & 8 & 神农架林区 & 0 & 0 & 神农架生态旅游区 & & \\
\hline
 & 9 & 神农架林区 & 200 & 2 & 神农架生态旅游区 & & \\
\hline
 & 9 & 神农架生态旅游区 & 200 & 2 & 十堰丹江口市武当山风景区 & & \\
\hline
 & 11 & 十堰市区 & 40 & 1 & 十堰丹江口市武当山风景区 & & \\
\hline
\hline
 & 1 & 海南 & 40 & 1 & 三亚南山文化旅游区 & \multirow{7}{*}{3.5} & \multirow{7}{*}{2400} \\
\hline
 & 1 & 三亚南山文化旅游区 & 40 & 1 & 三亚南山大小洞天旅游区 & & \\
\hline
 & 2 & 三亚南山大小洞天旅游区 & 40 & 1 & 三亚 & & \\
\hline
 & 3 & 三亚 & 40 & 1 & 保亭县呀诺达雨林文化旅游区 & & \\
\hline
 & 3 & 保亭县呀诺达雨林文化旅游区 & 20 & 0.5 & 保亭县海南槟榔谷黎苗文化旅游区 & & \\
\hline
 & 3 & 保亭县海南槟榔谷黎苗文化旅游区 & 30 & 0.67 & 三亚 & & \\
\hline
 & 4 & 三亚 & 90 & 2 & 陵水县分界洲岛旅游区 & & \\
\hline
\hline
 & 1 & 桂林 & 0 & 0 & 桂林漓江风景区 & \multirow{4}{*}{4} & \multirow{4}{*}{2327} \\
\hline
 & 1 & 桂林漓江风景区 & 70 & 1.5 & 桂林兴安县乐满地度假世界 & & \\
\hline
 & 2 & 桂林兴安县乐满地度假世界 & 70 & 1.5 & 桂林独秀峰·靖江王城景区 & & \\
\hline
 & 3 & 桂林独秀 & 0 & 0 & 桂林 & & \\
\hline
\end{tabular}
\end{table}

\begin{table}
\begin{tabular}{|c|c|c|c|c|c|c|c|}
\hline
 &  & 峰·靖江王 &  &  &  &  &  \\
 &  & 城景区 &  &  &  &  &  \\
\hline
 & 3 & 桂林 & 200 & 5 & 南宁市青秀山旅游区 &  &  \\
\hline
 & 4 & 南宁市青 & 187 & 4.5 & 南宁市青秀山旅游区 &  &  \\
 &  & 秀山旅游 &  &  &  &  &  \\
 &  & 区 &  &  &  &  &  \\
\hline
\multirow{12}{*}{ 浙江 } & 1 & 湖州市区 & 95 & 2 & 湖州市南浔区南浔古镇景区 & \multirow{12}{*}{ 8 } & \multirow{12}{*}{ 6269 } \\
\hline
 & 2 & 嘉兴市区 & 0 & 0 & 嘉兴桐乡乌镇古镇旅游区 &  &  \\
\hline
 & 2 & 嘉兴桐乡 & 80 & 1.5 & 杭州西溪湿地旅游区 &  &  \\
 &  & 乌镇古镇 &  &  &  &  &  \\
 &  & 旅游区 &  &  &  &  &  \\
\hline
 & 3 & 杭州市区 & 170 & 2.5 & 杭州淳安千岛湖风景区 &  &  \\
\hline
 & 4 & 杭州市区 &  &  &  &  &  \\
\hline
 & 5 & 淳安市市区 & 162 & 1.5 & 衢州市开化根宫佛国文化旅游区 &  &  \\
\hline
 & 5 & 衢州市开 & 162 & 1.5 & 金华东阳横店影视城景区 &  &  \\
 &  & 化根宫佛 &  &  &  &  &  \\
 &  & 国文化旅游区 &  &  &  &  &  \\
\hline
 & 6 & 金华市 & 181 & 2 & 绍兴市鲁迅故里一沈园景区 &  &  \\
\hline
 & 6 & 绍兴市鲁 & 238 & 3.1 & 宁波奉化溪口一滕头旅游景区 &  &  \\
 &  & 迅故里一 &  &  &  &  &  \\
 &  & 沈园景区 &  &  &  &  &  \\
\hline
 & 7 & 宁波市区 & 168 & 2 & 舟山普陀山风景区 &  &  \\
\hline
 & 8 & 宁波市区 & 213 & 2.3 & 温州乐清市雁荡山风景区 &  &  \\
\hline
\multirow{4}{*}{ 上海 } & 1 & 上海市区 & 0 & 0 & 上海野生动物园 & \multirow{4}{*}{ 3 } & \multirow{4}{*}{ 1920 } \\
\hline
 & 2 & 上海市区 & 0 & 0 & 东方明珠广播电视塔 &  &  \\
\hline
 & 2 & 上海市区 & 0 & 0 & 上海科技馆 &  &  \\
\hline
 & 3 & 上海市区 & 120 & 1.5 & 体验特色风土人情,前往湖州市区 &  &  \\
\hline
\multirow{8}{*}{ 福建 } & 1 & 三明市市区 & 272 & 4 & 三明泰宁风景旅游区 & \multirow{8}{*}{ 9 } & \multirow{8}{*}{ 6871 } \\
\hline
 & 2 & 三明市沙县 & 210 & 2.3 & 南平武夷山风景名胜区 &  &  \\
\hline
 & 3 & 南平市市区 & 90 & 1 & 福州市三坊七巷景区 &  &  \\
\hline
 & 4 & 福州市 &  &  & 体验风土人情,前往宁德市 &  &  \\
\hline
 & 5 & 宁德市 & 100 & 1.5 & 宁德屏南(白水洋·鸳鸯溪)旅游景区 &  &  \\
\hline
 & 6 & 屏南县 & 210 & 2.1 & 宁德市福鼎太姥山旅游区 &  &  \\
\hline
 & 7 & 宁德市市区 & 269 & 3 & 泉州市清源山风景名胜区 &  &  \\
\hline
 & 8 & 泉州市市区 & 150 & 2 & 厦门鼓浪屿风景名胜区 &  &  \\
\hline
\end{tabular}
\end{table}

\begin{table}
\begin{tabular}{|c|c|c|c|c|c|c|c|}
\hline
 & 9 & 龙岩市区 & 170 & 3 & 福建土楼(永定·南靖)旅游景区 & & \\
\hline
\multirow{16}{*}{广东} & 1 & 清远市区 & 220 & 3 & 清远连州地下河旅游景区 & \multirow{16}{*}{8.5} & \multirow{16}{*}{7127} \\
\cline{2-6}
 & 1 & 清远连州地下河旅游景区 & & & 0.5d & & \\
\cline{2-6}
 & 1 & 清远连州地下河旅游景区 & 140 & 3.5 & 韶关市区 & & \\
\cline{2-6}
 & 2 & 韶关市区 & 50 & 1 & 韶关仁化丹霞山景区 & & \\
\cline{2-6}
 & 2 & 韶关仁化丹霞山景区 & & & 0.5d & & \\
\cline{2-6}
 & 2 & 韶关仁化丹霞山景区 & 542 & 8 & 梅州市区 & & \\
\cline{2-6}
 & 3 & 韶关仁化丹霞山景区 & 542 & 8 & 梅州市区 & & \\
\cline{2-6}
 & 3 & 梅州市区 & 40 & 1 & 梅州市梅县区雁南飞茶田景区 & & \\
\cline{2-6}
 & 3 & 梅州市梅县区雁南飞茶田景区 & & & 0.5d & & \\
\cline{2-6}
 & 4 & 梅州市区 & 230 & 4 & 惠州市区 & & \\
\cline{2-6}
 & 4 & 惠州市区 & 50 & 1 & 惠州市罗浮山景区 & & \\
\cline{2-6}
 & 4 & 惠州市罗浮山景区 & & & 1d & & \\
\cline{2-6}
 & 5 & 惠州市区 & 60 & 1.5 & 深圳市区 & & \\
\cline{2-6}
 & 5 & 深圳市区 & / & / & 深圳观澜湖休闲旅游区 & & \\
\cline{2-6}
 & 5 & 深圳观澜湖休闲旅游区 & 30 & 0.5 & 深圳华侨城旅游度假区 & & \\
\cline{2-6}
 & 6 & 深圳市区 & 60 & 1.5 & 佛山市德顺区长鹿旅游休博园 & & \\
\cline{2-6}
 & 6 & 佛山市德顺区长鹿旅游休博园 & & & 0.5d & & \\
\cline{2-6}
 & 6 & 佛山市德顺区长鹿旅游休博园 & 23 & 0.5 & 佛山西樵山景区 & & \\
\hline
\end{tabular}
\end{table}

\begin{table}
\begin{tabular}{|c|c|c|c|c|c|c|c|}
\hline
 & 6 & 佛山西樵山景区 & \multicolumn{3}{c|}{0.5d} & & \\
\hline
 & 7 & 佛山市区 & 40 & 1 & 广州市区 & & \\
\hline
 & 7 & 广州市区 & / & / & 广州长隆旅游度假区 & & \\
\hline
 & 8 & & \multicolumn{3}{c|}{广州-游览特色建筑&体验风土人情} & & \\
\hline
 & 9 & 广州市区 & / & / & 广州白云山景区 & & \\
\hline
 & 9 & 广州白云山景区 & \multicolumn{3}{c|}{0.5d} & & \\
\hline
\multirow{4}{*}{内蒙古} & 1 & 鄂尔多斯市区 & 90 & 1.5 & 鄂尔多斯达拉特旗响沙湾旅游景区 & \multirow{4}{*}{1.5} & \multirow{4}{*}{1090} \\
\hline
 & 1 & 鄂尔多斯达拉特旗响沙湾旅游景区 & 60 & 1.5 & 鄂尔多斯市区 & & \\
\hline
 & 2 & 鄂尔多斯市区 & 40 & 1 & 鄂尔多斯伊金霍洛旗成吉思汗陵旅游区 & & \\
\hline
 & 2 & 鄂尔多斯伊金霍洛旗成吉思汗陵旅游区 & \multicolumn{3}{c|}{0.5d} & & \\
\hline
\multirow{3}{*}{宁夏} & 1 & 银川镇北堡西部影视城 & 30 & 1 & 银川市灵武水洞沟旅游区 & \multirow{3}{*}{2.5} & \multirow{3}{*}{1820} \\
\hline
 & 1 & 银川市灵武水洞沟旅游区 & 60 & 1 & 石嘴山平罗县沙湖旅游景区 & & \\
\hline
 & 3 & 石嘴山平罗县沙湖旅游景区 & 230 & 3 & 中卫沙坡头旅游景区 & & \\
\hline
\multirow{3}{*}{甘肃} & 1 & 天水麦积山景区 & 240 & 5 & 平凉崆峒山风景名胜区 & \multirow{3}{*}{5.5} & \multirow{3}{*}{4956} \\
\hline
 & 3 & 平凉崆峒山风景名胜区 & 1044 & 12 & 嘉峪关文物景区 & & \\
\hline
 & 4 & 嘉峪关文物景区 & 372 & 4 & 酒泉市敦煌沙山月牙泉景区 & & \\
\hline
\multirow{3}{*}{新疆} & 1 & 乌鲁木齐 & 48 & 1 & 乌鲁木齐天山大峡谷 & \multirow{3}{*}{15} & \multirow{3}{*}{17849.3} \\
\hline
 & 2 & 乌鲁木齐天山大峡谷 & 50 & 1 & 昌吉州阜康市天山天池风景名胜区 & & \\
\hline
 & 3 & 昌吉州阜康市天山天池风景 & 200 & 3 & 吐鲁番葡萄沟风景区 & & \\
\hline
\end{tabular}
\end{table}

\begin{table}
\centering
\begin{tabular}{c c c c c c c}
\hline
 & & 名胜区 & & & & \\
\hline
 & 3 & 吐鲁番葡萄沟风景区 & 336.6 & 4.3 & 巴音郭楞蒙古自治州博湖县博斯腾湖景区 & \\
\hline
 & 5 & 巴音郭楞蒙古自治州博湖县博斯腾湖景区 & 1084.3 & 11 & 喀什地区泽普县金胡杨景区 & \\
\hline
 & 7 & 喀什地区泽普县金胡杨景区 & 200 & 3 & 喀什地区噶尔老城景区 & \\
\hline
 & 8 & 喀什地区噶尔老城景区 & 1109.4 & 13.6 & 伊犁地区新源县那拉提旅游风景区 & \\
\hline
 & 10 & 伊犁地区新源县那拉提旅游风景区 & 870 & 15 & 阿勒泰地区布尔津县喀纳斯景区 & \\
\hline
 & 13 & 阿勒泰地区布尔津县喀纳斯景区 & 451 & 10 & 阿勒泰地区富蕴县可可托海景区 & \\
\hline
\multirow{2}{*}{青海} & 1 & 西宁 & 30 & 0.75 & 西宁市湟中县塔尔寺景区 & \multirow{2}{*}{2} & \multirow{2}{*}{2010} \\
\cline{2-6}
 & 2 & 西宁市湟中县塔尔寺景区 & 180 & 2.5 & 青海湖风景区 & & \\
\hline
\multirow{3}{*}{西藏} & 1 & 拉萨 & 0 & 0 & 拉萨 & \multirow{3}{*}{2} & \multirow{3}{*}{1800} \\
\cline{2-6}
 & 2 & 拉萨 & 0 & 0 & 拉萨布达拉宫景区 & & \\
\cline{2-6}
 & 2 & 拉萨布达拉宫景区 & 0 & 0 & 拉萨大昭寺景区 & & \\
\hline
\multirow{4}{*}{四川} & 1 & 成都 & 65 & 1.5 & 成都青城山-都江堰旅游景区 & \multirow{4}{*}{11} & \multirow{4}{*}{8706} \\
\cline{2-6}
 & 2 & 成都青城山-都江堰旅游景区 & 420 & 8 & 阿坝藏族羌族自治州九寨沟景区 & & \\
\cline{2-6}
 & 4 & 阿坝藏族羌族自治州九寨沟景区 & 130 & 3 & 阿坝藏族羌族自治州松潘县黄龙风景名胜区 & & \\
\cline{2-6}
 & 5 & 阿坝藏族羌族自治州松潘县 & 496 & 8 & 乐山乐山大佛景区 & & \\
\hline
\end{tabular}
\end{table}

\begin{table}
\begin{tabular}{|c|c|c|c|c|c|c|c|}
\hline
 &  & 黄龙风景 &  &  &  &  &  \\
 &  & 名胜区 &  &  &  &  &  \\
\hline
 & 6 & 乐山乐山 & 80 & 2 & 乐山峨眉山景区 &  &  \\
 &  & 大佛景区 &  &  &  &  &  \\
\hline
 & 7 & 乐山峨眉 & 282 & 3.5 & 绵阳北川羌城旅游区(中国 &  &  \\
 &  & 山景区 &  &  & 羌城一老县城地震遗址一 &  &  \\
 &  &  &  &  & “5·12”特大地震纪念馆一 &  &  \\
 &  &  &  &  & 北川羌族民俗博物馆一北川 &  &  \\
 &  &  &  &  & 新县城一吉娜羌寨) &  &  \\
\hline
 & 8 & 绵阳北川 & 209 & 3 & 南充市阆中古城旅游景区 &  &  \\
 &  & 羌城旅游 &  &  &  &  &  \\
 &  & 区(中国羌 &  &  &  &  &  \\
 &  & 城一老县 &  &  &  &  &  \\
 &  & 城地震遗 &  &  &  &  &  \\
 &  & 址一 &  &  &  &  &  \\
 &  & “5·12” &  &  &  &  &  \\
 &  & 特大地震 &  &  &  &  &  \\
 &  & 纪念馆一 &  &  &  &  &  \\
 &  & 北川羌族 &  &  &  &  &  \\
 &  & 民俗博物 &  &  &  &  &  \\
 &  & 馆一北川 &  &  &  &  &  \\
 &  & 新县城一 &  &  &  &  &  \\
 &  & 吉娜羌寨) &  &  &  &  &  \\
\hline
 & 9 & 南充市阆 & 126 & 2 & 广元市剑门蜀道剑门关旅游景区 &  &  \\
 &  & 中古城旅 &  &  &  &  &  \\
 &  & 游景区 &  &  &  &  &  \\
\hline
 & 10 & 广元市剑 & 298 & 4 & 广安市邓小平故里旅游区 &  &  \\
 &  & 门蜀道剑 &  &  &  &  &  \\
 &  & 门关旅游 &  &  &  &  &  \\
 &  & 景区 &  &  &  &  &  \\
\hline
\multirow{6}{*}{陕西} & 1 & 西安 & 40 & 1 & 西安秦始皇兵马俑博物馆 & \multirow{6}{*}{3.5} & \multirow{6}{*}{2680} \\
\hline
 & 1 & 西安秦始 & 10 & 10 & 西安华清池景区 &  &  \\
 &  & 皇兵马俑 &  &  &  &  &  \\
 &  & 博物馆 &  &  &  &  &  \\
\hline
 & 2 & 西安华清 & 120 & 2 & 渭南华阴市华山风景区 &  &  \\
 &  & 池景区 &  &  &  &  &  \\
\hline
 & 3 & 渭南华阴 & 120 & 2 & 宝鸡扶风县法门寺佛文化景区 &  &  \\
 &  & 市华山风 &  &  &  &  &  \\
 &  & 景区 &  &  &  &  &  \\
\hline
 & 3 & 宝鸡扶风 & 120 & 2 & 西安大雁塔一大唐芙蓉园景区 &  &  \\
 &  & 县法门寺 &  &  &  &  &  \\
 &  & 佛文化景 &  &  &  &  &  \\
 &  & 区 &  &  &  &  &  \\
\hline
 & 4 & 西安大雁 & 170 & 2.5 & 延安黄陵县黄帝陵景区 &  &  \\
\hline
\end{tabular}
\end{table}

\begin{table}
\begin{tabular}{|c|c|c|c|c|c|c|}
\hline
 &  & 塔一大唐 &  &  &  & \\
 &  & 芙蓉园景区 &  &  &  & \\
\hline
\multirow{4}{*}{贵州} & 1 & 安顺 & 60 & 1 & 安顺镇宁县黄果树瀑布景区 & \multirow{4}{*}{4} \\
\cline{2-6}
 & 1 & 安顺镇宁县黄果树瀑布景区 & 30 & 1 & 安顺龙宫景区 & \\
\cline{2-6}
 & 2 & 安顺龙宫景区 & 220 & 3 & 毕节市百里杜鹃景区 & \\
\cline{2-6}
 & 4 & 毕节市百里杜鹃景区 & 484.5 & 7 & 黔南布依族苗族自治州荔波樟江景区 & \\
\hline
\multirow{6}{*}{云南} & 1 & 昆明 & 80 & 1 & 云南省昆明市石林彝族自治县 & \multirow{6}{*}{5.5} \\
\cline{2-6}
 & 1 & 云南省昆明市石林彝族自治县 & 319 & 4 & 中科院西双版纳热带植物园 & \\
\cline{2-6}
 & 2 & 中科院西双版纳热带植物园 & 559.4 & 8.5 & 大理崇圣寺三塔文化旅游区 & \\
\cline{2-6}
 & 4 & 大理崇圣寺三塔文化旅游区 & 149 & 3 & 丽江古城景区 & \\
\cline{2-6}
 & 4 & 丽江古城景区 & 40 & 1 & 丽江玉龙雪山景区 & \\
\cline{2-6}
 & 5 & 丽江玉龙雪山景区 & 200 & 5 & 云南省迪庆藏族自治州香格里拉市 & \\
\hline
\multirow{6}{*}{重庆} & 1 & 重庆 & 100 & 1.5 & 大足石刻景区 & \multirow{6}{*}{8.5} \\
\cline{2-6}
 & 2 & 大足石刻景区 & 190 & 3 & 武隆喀斯特旅游区(天生三硚、仙女山、芙蓉洞) & \\
\cline{2-6}
 & 4 & 武隆喀斯特旅游区(天生三硚、仙女山、芙蓉洞) & 140 & 2 & 万盛黑山谷-龙鳞石海风景区 & \\
\cline{2-6}
 & 5 & 万盛黑山谷-龙鳞石海风景区 & 40 & 1 & 南川金佛山一神龙峡风景区 & \\
\cline{2-6}
 & 6 & 南川金佛山一神龙峡风景区 & 100 & 1.5 & 酉阳桃花源旅游景区 & \\
\cline{2-6}
 & 7 & 酉阳桃花 & 427 & 8 & 巫山小三峡一小小三峡旅游 & \\
\hline
\end{tabular}
\end{table}

\begin{tabular}{c c c c c c c c}
\hline
 & & 源旅游景区 & & & 区 & & \\
\hline
 & 8 & 巫山小三峡一小小三峡旅游区 & 0 & 0 & 巫山小三峡一小小三峡旅游区 & & \\
\hline
\end{tabular}