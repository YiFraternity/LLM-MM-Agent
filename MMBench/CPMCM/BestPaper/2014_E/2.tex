\documentclass{article}
\usepackage{amsmath}
\usepackage{amssymb}

\title{第十一届华为杯全国研究生数学建模竞赛\\乘用车物流运输计划问题}
\author{}
\date{}

\begin{abstract}
整车物流指的是按照客户订单对整车快速配送的全过程。随着我国汽车产业的高速发展,整车物流量,特别是乘用车的整车物流量迅速增长。

针对问题一,我们提出整数线性规划模型,并对模型进行逐层优化计算,分析装载要求,根据轿运车的长度、数量以及乘用车的需求量的约束建立模型并求解,得出运输Ⅰ车型的乘用车 100 辆及Ⅱ车型的乘用车 68 辆需要 1-1 型轿运车 16 辆,1-2 型轿运车 2 辆,并且得出每辆轿运车的乘用车装载方案。

针对问题二,根据约束条件同样提出整数线性规划模型,考虑Ⅲ车型的乘用车的高度及适合放置位置为 1-1 型轿运车和 1-2 型轿运车的下层,优先安排Ⅲ车型的乘用车的运输,并对模型进行逐层优化改进并求解,得出运输Ⅱ车型的乘用车 72 辆及Ⅲ车型的乘用车 52 辆需要 1-1 型轿运车 12 辆,1-2 型轿运车 1 辆,并且得出每辆轿运车的乘用车装载方案。

针对问题三,根据约束条件同样提出整数线性规划模型,提出通用程序对模型进行逐层优化求解,得出运输Ⅰ车型的乘用车 156 辆、Ⅱ车型的乘用车 102 辆及Ⅲ车型的乘用车 39 辆需要 1-1 型轿运车 25 辆,1-2 型轿运车 5 辆,并且得出每辆轿运车的乘用车装载方案,同时利用通用程序求解问题一、问题二,与问题一、问题二的求解结果进行比较,分析通用程序的适用性。

针对问题四,考虑约束条件及目的地的不同,提出启发式逐层优化算法。首先,利用问题三的通用程序确定当目的地唯一时,运输Ⅰ车型的乘用车 166 辆及Ⅱ车型的乘用车 78 辆所需的轿运车的数量;其次考虑优先运输远距离目的地的乘用车,由于 1-2 型轿运车的运输量较大,故对于远距离目的地优先使用 1-2 型轿运车运输;然后考虑次远距离目的地的运输方案,最后得出 A、B、C、D

目的地轿运车辆的数目分别为:A 地:4 辆 1-2 型轿运车,1 辆 1-1 型轿运车;B 地:6 辆 1-1 型轿运车;C 地:9 辆 1-1 型轿运车;D 地:5 辆 1-1 型轿运车。完成运输计划共需要 21 辆 1-1 型轿运车,4 辆 1-2 型轿运车,并得出每辆轿运车的乘用车装载方案及运输方案所行驶的里程为 6404 千米。

针对问题五,由于物流公司需要用 10 种轿运车运输 45 种乘用车送往 $A$、$B$、$C$、$D$、$E$ 这 5 个目的地,每种轿运车的拥有量有限,不同目的地的不同乘用车的需求量也不尽相同,其装载、运输方案太多,情况纷繁复杂,得到最优解是不切实际的。建立分层划分模型,提出构造型分层划分启发式算法,对问题进行求解,得出完成运输任务需要轿运车总量为 119 辆,同时得出不同目的地不同轿运车的运输方式以及每辆轿运车的装载方式(见附录 4)。

关键词:物流运输 整数线性规划 启发式逐层优化 构造型分层划分启发式算法
\end{abstract}

\tableofcontents

\section{问题重述}

整车物流指的是按照客户订单对整车快速配送的全过程。随着我国汽车工业的高速发展,整车物流量,特别是乘用车的整车物流量迅速增长。乘用车生产厂家根据全国客户的购车订单,向物流公司下达运输乘用车到全国各地的任务,物流公司则根据下达的任务制定运输计划并配送这批乘用车。

“轿运车”是通过公路来运输乘用车整车的专用运输车,根据型号的不同有单层和双层两种类型。物流公司首先要从他们当时可以调用的“轿运车”中选择出若干辆轿运车,进而给出其中每一辆轿运车上乘用车的装载方案和目的地,以保证运输任务的完成。

装载具体要求如下:每种轿运车上、下层装载区域均可等价看成长方形,各列乘用车均纵向摆放,相邻乘用车之间纵向及横向的安全车距均至少为 0.1 米,下层力争装满,上层两列力求对称,以保证轿运车行驶平稳。受层高限制,高度超过 1.7 米的乘用车只能装在 1-1、1-2 型下层。影响成本高低的首先是轿运车使用数量;其次,在轿运车使用数量相同情况下,1-1 型轿运车的使用成本较低,2-2 型较高,1-2 型略低于前两者的平均值,但物流公司 1-2 型轿运车拥有量小,为方便后续任务安排,每次 1-2 型轿运车使用量不超过 1-1 型轿运车使用量的 20%;再次,在轿运车使用数量及型号均相同情况下,行驶里程短的成本低,注意因为该物流公司是全国性公司,在各地均会有整车物流业务,所以轿运车到达目的地后原地待命,无须放空返回。最后每次卸车成本几乎可以忽略。

请为物流公司安排以下五次运输,制定详细计划,含所需要各种类型轿运车的数量、每辆轿运车的乘用车装载方案、行车路线:

\begin{enumerate}
    \item 物流公司要运输Ⅰ车型的乘用车 100 辆及Ⅱ车型的乘用车 68 辆。
    \item 物流公司要运输Ⅱ车型的乘用车 72 辆及Ⅲ车型的乘用车 52 辆。
    \item 物流公司要运输Ⅰ车型的乘用车 156 辆、Ⅱ车型的乘用车 102 辆及Ⅲ车型的乘用车 39 辆。
    \item 物流公司要运输 166 辆Ⅰ车型的乘用车(其中目的地是 A、B、C、D 的分别为 42、50、33、41 辆)和 78 辆Ⅱ车型的乘用车(其中目的地是 A、C 的,分别为 31、47 辆),具体路线见图 4,各段长度:$OD=160$,$DC=76$,$DA=200$,$DB=120$,$BE=104$,$AE=60$。
    \item 附件的表 1 给出了物流公司需要运输的乘用车类型(含序号)、尺寸大小、数量和目的地,附件的表 2 给出可以调用的轿运车类型(含序号)、数量和装载区域大小(表里数据是下层装载区域的长和宽,1-1 型及 2-2 型轿运车上、下层装载区域相同;1-2 型轿运车上、下层装载区域长度相同,但上层比下层宽 0.8 米。此外 2-2 型轿运车因为层高较低,上、下层均不能装载高度超过 1.7 米的乘用车。由此设计运输方案的表达。
\end{enumerate}

\section{问题假设}

\begin{enumerate}
    \item 在轿运车使用数量及型号相同的情况下,运输成本仅与行驶里程的长短有关,不考虑其他因素;
    \item 每次卸车成本几乎忽略不计;
    3.轿运车到达目的地之前可以在中途停靠卸载部分乘用车,即每次运输目
的地只有一个; 
    4. 轿运车到达目的地后原地待命,无须放空返回。
\end{enumerate}

\section{符号说明}

\begin{table}[h]
\centering
\begin{tabular}{c l}
\hline
\textbf{符号} & \textbf{意义} \\
\hline
$m_k$ & I、II及III车型乘用车的数量 ($k=1,2,3$) \\
$m_k'$ & 未装车的I、II及III车型乘用车的数量 ($k=1,2,3$) \\
$a_k$ & I、II及III车型乘用车的长度 ($k=1,2,3$) \\
$b_k$ & I、II及III车型乘用车的宽度 ($k=1,2,3$) \\
$c_k$ & I、II及III车型乘用车的高度 ($k=1,2,3$) \\
$L_i$ & 1-1、1-2型轿运车的长度 ($i=1,2$) \\
$B_{in}$ & 1-1、1-2型轿运车的上、下层的宽度 ($i=1,2$ $n=1,2$) \\
& $n=1$表示下层,$n=2$表示下层 \\
$S_1$ & 1-1型轿运车的数量 \\
$S_2$ & 1-2型轿运车的数量 \\
$c_{ijk}$ & 第$i$辆1-1型轿运车第$j$列所容纳的$k$型乘用车的数目 \\
& ($i=1,2,\cdots s_1$ $j=1,2$) \\
$d_{ijk}$ & 第$i$辆1-2型轿运车第$j$列所容纳的$k$型乘用车的数目 \\
& ($i=1,2,\cdots s_2$ $j=1,2,3$) \\
$f_{tijk}$ & 地第$i$辆1-1型轿运车第$j$列所容纳的$k$型乘用车的数目 \\
& ($t=1,2,3,4$ $i=1,2\cdots s_2$ $j=1,2$) \\
$g_{tijk}$ & 地第$i$辆1-2型轿运车第$j$列所容纳的$k$型乘用车的数目 \\
& ($t=1,2,3,4$ $i=1,2\cdots s_2$ $j=1,2,3$) \\
$\sigma$ & 乘用车横向、纵向车间距 \\
$w_i$ & 起始点到目的地A、B、C、D、E的距离 ($i=1,2,\cdots 5$) \\
\hline
\end{tabular}
\end{table}

\section{模型的建立与求解}

\subsection{问题一的模型建立与求解}

\subsubsection{问题一分析}

问题一是对于物流公司制定运输方案的问题,要运输I车型的乘用车100辆及II车型的乘用车68辆,针对问题中所提出的装载要求及运输成本要求以及问题一所要满足的乘用车运输数量,乘用车、轿运车规格如下表1、表2所示:

\documentclass{article}
\usepackage{array}

\begin{document}

\begin{table}[htbp]
\centering
\caption{乘用车规格}
\begin{tabular}{|c|c|c|c|}
\hline
乘用车型号 & 长度(米) & 宽度(米) & 高度(米) \\ \hline
I & 4.61 & 1.7 & 1.51 \\ \hline
II & 3.615 & 1.605 & 1.394 \\ \hline
III & 4.63 & 1.785 & 1.77 \\ \hline
\end{tabular}
\end{table}

\begin{table}[htbp]
\centering
\caption{轿运车规格}
\begin{tabular}{|c|c|c|c|}
\hline
轿运车类型 & 上下层长度(米) & 上层宽度(米) & 下层宽度(米) \\ \hline
1-1 & 19 & 2.7 & 2.7 \\ \hline
1-2 & 24.3 & 3.5 & 2.7 \\ \hline
\end{tabular}
\end{table}

\end{document}

此类问题属于 $NP$-Hard 问题 \cite{ref1,ref2},对于此类问题的求解建立整数线性规划模型。首先对于装载要求进行分析:装载时要求相邻乘用车之间纵向及横向的安全车距均至少为 0.1 米,下层力争装满,上层两列力求对称,以保证轿运车行驶平稳,故首先考虑 I、II 车型的乘用车的宽度,由表 1 可知乘用车 I、II 的宽度均小于 1-1 型轿运车上层、下层和 1-2 型轿运车下层的宽度,1-2 型轿运车上层宽度为 3.5 米,为了保证横向安全距离为 0.1 米,所以 1-2 型轿运车上层装载车辆的宽度必须小于等于 $(B_{22}-0.1)/2$ 米,即:
\begin{equation}
b_k \leq \frac{(B_{22}-0.1)}{2} = \frac{(3.5-0.1)}{2} = 1.7 \quad (k=1,2)
\tag{1}
\end{equation}
由于 I、II 型乘用车的宽度均小于等于 1.7 米,所以车辆的装载不受宽度的约束。在车辆装载时必须保证纵向安全距离为 0.1 米以及装载乘用车车辆的长度之和小于等于 1-1 型、1-2 型轿运车的长度。乘用车 I、II 的高度分别为:1.51 米、1.394 米,均小于 1.7 米,故乘用车 I、II 既可以装在 1-1、1-2 型上层也可以装在 1-1、1-2 型下层。因此在安排装载时不受高度的约束。同时为了保证轿运车行驶平稳,下层力争装满,上层两列力求对称,故装载时应该首先保证下层装满,其次上层满足对称。

其次针对运输成本分析:影响成本高低的首先是轿运车使用数量;其次由于在轿运车使用数量相同情况下,1-1 型轿运车的使用成本较低,2-2 型较高,1-2 型略低于前两者的平均值,但物流公司 1-2 型轿运车拥有量小,为方便后续任务安排,每次 1-2 型轿运车使用量不超过 1-1 型轿运车使用量的 20%,由于 1-1 型轿运车、1-2 型轿运车的使用成本不同,使用 1-2 型轿运车能够减少 1-1 型轿运车的使用数量,所以在满足 1-2 型轿运车使用量不超过 1-1 型轿运车使用量的 20% 的前提下,可以充分调动能够有效使用的 1-2 型轿运车。再次,在轿运车使用数量及型号均相同情况下,行驶里程短的成本低。故建立整数线性规划模型,轿运车的使用数量最少作为优化模型的目标函数,满足装载要求及运输成本的约束条件,求解结果并进行分析。

\subsubsection{问题一模型建立}

综合装载要求及运输成本要求,建立逐层优化的整数线性规划 \cite{ref3,ref4} 模型:

\begin{equation}
Obj \quad \min s = \sum_{i=1}^{2} s_{i}
\tag{2}
\end{equation}

\begin{equation}
\begin{aligned}
St. \quad & \left\{
\begin{aligned}
& \sum_{k=1}^{2} (a_{k} + \sigma)c_{ijk} - 0.1 \leq L_{1}, \quad i=1,2,\cdots,s_{1}, j=1,2 \\
& \sum_{k=1}^{2} (a_{k} + \sigma)d_{ijk} - 0.1 \leq L_{2}, \quad i=1,2,\cdots,s_{2}, j=1,2,3 \\
& \sum_{i=1}^{s_{1}} \sum_{j=1}^{2} c_{ij1} + \sum_{i=1}^{s_{2}} \sum_{j=1}^{3} d_{ij1} = 100 \\
& \sum_{i=1}^{s_{1}} \sum_{j=1}^{2} c_{ij2} + \sum_{i=1}^{s_{2}} \sum_{j=1}^{3} d_{ij2} = 68 \\
& \sum_{j=1}^{2} \sum_{k=1}^{3} c_{ijk} \leq 27, i=1,2,\cdots,s_{1} \\
& \sum_{j=1}^{3} \sum_{k=1}^{3} d_{ijk} \leq 27, i=1,2,\cdots,s_{2} \\
& s_{2} \leq 0.2 \times s_{1}
\end{aligned}
\right.
\tag{3}
\end{aligned}
\end{equation}

模型中,式(2)为目标函数,即完成运输任务轿运车的数量达到最少。式(3)为约束条件,前两个约束条件为轿运车的长度约束,即满足乘用车纵向间距不小于 0.1m 的条件下,乘用车的纵向排列长度之和小于等于轿运车的长度。第三、第四个约束条件要求 I 型、II 型乘用车的运输数量满足运输任务。第五、第六个约束条件要求每辆轿运车可以装载乘用车的最大数量是 6 至 27 辆之间。第七个约束条件要求满足每次使用 1-2 型轿运车的数量不超过 1-1 型轿运车的 20%。

\subsubsection{问题一模型求解}

由于求解全局最优解过于复杂,且程序运行的实效性并不是很强,因此采用模型采用分层求解局部最优值,然后针对部分车辆的装载进行优化,使求解结果较优,满足轿运车辆数目最少且在满足 1-1 型、1-2 型轿运车车辆数目总数不变的情况下,1-2 型轿运车的使用量最少,使得运输成本减少。

由于 1-2 型轿运车的上层可装载乘用车两列,1-1 型轿运车的上层可装载一列,所以 1-2 型轿运车的运输能力强于 1-1 型轿运车。由于 1-2 型轿运车的数量不能超过 1-1 型轿运车型数量的 20%,所以我们首先分配 1-1 型车,当它的辆数达到 5 的倍数的时候,可以优先考虑选择 1-2 型轿运车进行运输。

在安排乘用车的装载方式时,建立整数线性规划模型,选择使当前轿运车剩余空间长度最小的装载方式。对于每一辆车的装载方式,都进行一次线性规划。

针对 1-1 型轿运车的最优装载方式,建立模型如下:

\begin{equation}
Obj \quad \max a_{1}c_{i11} + a_{2}c_{i12} + a_{1}c_{i21} + a_{2}c_{i22}
\tag{4}
\end{equation}

\begin{equation}
S.t.
\begin{cases}
(a_{1}+\sigma)c_{i11}+(a_{2}+\sigma)c_{i12}-\sigma\leq L_{1} \\
(a_{1}+\sigma)c_{i21}+(a_{2}+\sigma)c_{i22}-\sigma\leq L_{1} \\
c_{i11}+c_{i21}\leq m_{1}^{'} \\
c_{i12}+c_{i22}\leq m_{2}^{'} \\
c_{i11},c_{i12},c_{i21},c_{i22}\in N
\end{cases}
\tag{5}
\end{equation}

针对 1-2 轿运车,最优装载方式建立模型如下:
\begin{equation}
Obj \quad \max a_{1}d_{i11}+a_{2}d_{i12}+a_{1}d_{i21}+a_{2}d_{i22}+a_{1}d_{i31}+a_{2}+d_{i32}
\tag{6}
\end{equation}

\begin{equation}
S.t.
\begin{cases}
(a_{1}+\sigma)d_{i11}+(a_{2}+\sigma)d_{i12}-\sigma\leq L_{2} \\
(a_{1}+\sigma)d_{i21}+(a_{2}+\sigma)d_{i22}-\sigma\leq L_{2} \\
(a_{1}+\sigma)d_{i31}+(a_{2}+\sigma)d_{i32}-\sigma\leq L_{2} \\
\sum_{j=1}^{3}d_{ijk}\leq m_{k}^{'} k=1,2 \\
d_{i11},d_{i12},d_{i21},d_{i22},d_{i31},d_{i32}\in N
\end{cases}
\tag{7}
\end{equation}

针对以上两个模型,利用 MATLAB 中改进的 $Linprog$ 函数 $^{[5][6]}$ 计算出每辆轿运车的最优装载方式。根据每安排 5 辆 1-1 型轿运车后就安排 1 辆 1-2 型轿运车,建立循环程序。由于 1-1 型轿运车、1-2 型轿运车的使用成本不同,使用 1-2 型轿运车能够减少 1-1 型轿运车的使用数量,所以在满足 1-2 型轿运车使用量不超过 1-1 型轿运车使用量的 20% 的前提下,可以充分调动能够有效使用的 1-2 型轿运车。

为了满足下层力争装满,上层两列力求对称,以保证轿运车行驶平稳,对最后一辆轿运车的装载方式进行优化,由于其余车辆均是满载情形,无需进行优化。模型中均是直接对整车的装载方案提出优化方案,因此对于最后装载的车辆可能出现下层没有装满,上层还有车辆的情况。故对轿运车的下层提出优化方案,确保下层首先被装满,剩余车辆尽量对称装载在上层即可。对下层装载方式进行优化,模型如下:
\begin{equation}
Obj \quad \max a_{1}c_{i11}+a_{2}c_{i12}+a_{1}c_{i21}+a_{2}c_{i22}
\tag{8}
\end{equation}

\begin{equation}
S.t.
\begin{cases}
(a_{1}+0.1)c_{i11}+(a_{2}+0.1)c_{i12}-0.1\leq L_{1} \\
c_{i11}\leq m_{k}^{'} \\
c_{i12}\leq m_{k}^{'} \\
c_{i11},c_{i12}\in N
\end{cases}
\tag{9}
\end{equation}

利用 MATLAB 编程求解,得出总的车辆数为:18 辆轿运车,其中 1-1 型轿运车 16 辆,1-2 型轿运车 2 辆。各类型轿运车的使用数量如下表 3,轿运车的乘用车装载方案如下表 4:

\begin{table}
\centering
\caption{问题一各类型轿运车的使用数量}
\begin{tabular}{|c|c|}
\hline
轿运车类型 & 总使用量 \\
\hline
1-1型轿运车 & 16 \\
\hline
1-2型轿运车 & 2 \\
\hline
\end{tabular}
\end{table}

\begin{table}
\centering
\caption{问题一轿运车的乘用车装载方案}
\begin{tabular}{|c|c|c|c|c|c|c|}
\hline
轿运车类型 & 相同类型、 & 装在上 & 装在上 & 装在上 & 装在下 & 装在下 \\
 & 相同装载 & 层Ⅰ型 & 层Ⅱ型 & 层Ⅲ型 & 层Ⅰ型 & 层Ⅱ型 \\
 & 方式的车辆数 & 乘用车 & 乘用车 & 乘用车 & 乘用车 & 乘用车 \\
 & & 的数量 & 的数量 & 的数量 & 的数量 & 的数量 \\
\hline
1 & 11 & 4 & 0 & 0 & 4 & 0 \\
\hline
2 & 2 & 4 & 8 & 0 & 2 & 4 \\
\hline
1 & 4 & 0 & 5 & 0 & 0 & 5 \\
\hline
1 & 1 & 0 & 0 & 0 & 0 & 4 \\
\hline
\end{tabular}
\end{table}

注:轿运车类型中:1代表1-1型轿运车,2代表1-2型轿运车。

\subsubsection{问题一结果分析}

由上表3、表4可知:1-1型轿运车的使用量为16辆,其中装在上层Ⅰ型乘用车的数量为4辆,装在下层Ⅰ型乘用车的数量为4辆的装载方式相同的车辆数为11辆;装在上层Ⅱ型乘用车的数量为5辆,装在下层Ⅱ型乘用车的数量为5辆的装载方式相同的车辆数为4辆;最后一辆轿运车的装载方式为:4辆Ⅱ型乘用车装在下层。1-2型轿运车相同装载方式的车辆数为2辆,其中装在上层Ⅰ型乘用车的数量为4辆,装装在上层Ⅱ型乘用车的数量为8辆,在下层Ⅰ型乘用车的数量为2辆,装在下层Ⅱ型乘用车的数量为4辆。

通过以上结果进行分析,可以看出通过将整数线性规划模型逐层优化求解既可以简化模型,方便计算,同时可以得出各种类型轿运车的数量以及详细的每辆轿运车的乘用车装载方案。

\subsection{问题二的模型建立与求解}

\subsubsection{问题二分析}

问题二要求完成Ⅱ车型的乘用车72辆及Ⅲ车型的乘用车52辆的运输目标,针对问题中所提出的装载要求及运输成本要求以及问题二所要求的结果进行分析,与问题一相同,需要考虑装载要求及运输成本。根据表1、表2中所给的数据可以得出乘用车、轿运车长、宽、高的数值,由于受层高限制,高度超过1.7米的乘用车只能装在1-1、1-2型下层,乘用车Ⅱ、Ⅲ的高度分别为:1.394米、1.77米,故乘用车Ⅱ既可以装在1-1、1-2型下层也可以装在1-1、1-2型下层,但是乘用车Ⅲ的高度1.77米$>$1.7米,且乘用车Ⅲ的宽度1.785米小于1-1、1-2型轿运车的宽度,所以乘用车Ⅲ只能装在1-1、1-2型轿运车的下层。由问题一可知乘用车Ⅱ既可以装在1-1、1-2型上层也可以装在1-1、1-2型下层。故对于问题二中运输Ⅲ车型的乘用车只能装在1-1、1-2型下层,所以在问题二的求解中需要优先考虑Ⅲ车型的乘用车的运输安排,同样建立逐层优化的整数线性规划模型。

\subsubsection{问题二模型建立}

结合装载要求及运输成本要求,建立逐层优化的整数线性规划模型:

\begin{equation}
Obj \quad \min s = \sum_{i=1}^{2} s_i
\tag{10}
\end{equation}

\begin{equation}
S.t.
\begin{cases}
\sum_{k=2}^{3} (a_k + \sigma)c_{ijk} - \sigma \leq L_1, i=1,2,\cdots,s_1, j=1,2 \\
\sum_{k=2}^{3} (a_k + \sigma)d_{ijk} - \sigma \leq L_2, i=1,2,\cdots,s_1, j=1,2,3 \\
\sum_{i=1}^{s_1} \sum_{j=1}^{2} c_{ij2} + \sum_{i=1}^{s_2} \sum_{j=1}^{3} d_{ij2} = 72 \\
\sum_{i=1}^{s_1} \sum_{j=1}^{2} c_{ij3} + \sum_{i=1}^{s_2} \sum_{j=1}^{3} d_{ij3} = 52 \\
c_{i23} = 0 \\
d_{j23} = 0 \\
d_{j33} = 0 \\
s_2 \leq 0.2 \times s_1
\end{cases}
\tag{11}
\end{equation}

\subsubsection{问题二模型求解}

问题二与问题一相比,问题二有运输Ⅲ型乘用车的运输任务。由于Ⅲ型乘用车不能放在轿运车的上层,所以问题二的求解是在问题一的基础添加相关的上建立的约束建立的。

针对 1-1 型轿运车的最优装载方式模型如下:

\begin{equation}
Obj \quad \max a_1c_{i12} + a_2c_{i13} + a_1c_{i22} + a_2c_{i23}
\tag{12}
\end{equation}

\begin{equation}
S.t.
\begin{cases}
(a_1 + \sigma)c_{i12} + (a_2 + \sigma)c_{i13} - \sigma \leq L_1 \\
(a_1 + \sigma)c_{i22} + (a_2 + \sigma)c_{i23} - \sigma \leq L_1 \\
c_{i12} + c_{i22} \leq m_2' \\
c_{i13} + c_{i23} \leq m_2' \\
c_{i12}, c_{i13}, c_{i22}, c_{i23} \in N
\end{cases}
\tag{13}
\end{equation}

针对 1-2 型轿运车的最优装载方式模型如下:

\begin{equation}
Obj \quad \max a_2d_{i12} + a_3d_{i13} + a_2d_{i22} + a_3d_{i23} + a_2d_{i32} + a_3d_{i33}
\tag{14}
\end{equation}

\begin{equation}
\begin{aligned}
S.t. \left\{
\begin{aligned}
(a_2 + \sigma)d_{i12} + (a_3 + \sigma)d_{i13} - \sigma &\leq L_2 \\
(a_2 + \sigma)d_{i22} + (a_3 + \sigma)d_{i23} - \sigma &\leq L_2 \\
(a_2 + \sigma)d_{i32} + (a_3 + \sigma)d_{i33} - \sigma &\leq L_2 \\
d_{i22} + d_{i32} &\leq m_1' \\
d_{i23} + d_{i33} &\leq m_2' \\
d_{i23} &= 0 \\
d_{i33} &= 0 \\
d_{i12}, d_{i13}, d_{i22}, d_{i23}, d_{i32}, d_{i33} &\in \mathbb{N}
\end{aligned}
\right.
\end{aligned}
\tag{15}
\end{equation}

利用 MATLAB 编程求解,得出总车辆数为:轿运车总车辆数为 13 辆,其中 1-1 型轿运车 11 辆,1-2 型轿运车 2 辆。各类型轿运车的使用数量如下表 5,轿运车的乘用车装载方案如下表 6:

\begin{table}[htbp]
\centering
\caption{问题二各类型轿运车的使用数量}
\begin{tabular}{|c|c|}
\hline
轿运车类型 & 总使用量 \\ \hline
1-1 型轿运车 & 11 \\ \hline
1-2 型轿运车 & 2 \\ \hline
\end{tabular}
\end{table}

\textbf{表 6 问题二轿运车的乘用车装载方案}

\begin{table}[h]
\centering
\begin{tabular}{c l}
\hline
\textbf{符号} & \textbf{意义} \\
\hline
$m_k$ & I、II及III车型乘用车的数量 ($k=1,2,3$) \\
$m_k'$ & 未装车的I、II及III车型乘用车的数量 ($k=1,2,3$) \\
$a_k$ & I、II及III车型乘用车的长度 ($k=1,2,3$) \\
$b_k$ & I、II及III车型乘用车的宽度 ($k=1,2,3$) \\
$c_k$ & I、II及III车型乘用车的高度 ($k=1,2,3$) \\
$L_i$ & 1-1、1-2型轿运车的长度 ($i=1,2$) \\
$B_{in}$ & 1-1、1-2型轿运车的上、下层的宽度 ($i=1,2$ $n=1,2$) \\
& $n=1$表示下层,$n=2$表示下层 \\
$S_1$ & 1-1型轿运车的数量 \\
$S_2$ & 1-2型轿运车的数量 \\
$c_{ijk}$ & 第$i$辆1-1型轿运车第$j$列所容纳的$k$型乘用车的数目 \\
& ($i=1,2,\cdots s_1$ $j=1,2$) \\
$d_{ijk}$ & 第$i$辆1-2型轿运车第$j$列所容纳的$k$型乘用车的数目 \\
& ($i=1,2,\cdots s_2$ $j=1,2,3$) \\
$f_{tijk}$ & 地第$i$辆1-1型轿运车第$j$列所容纳的$k$型乘用车的数目 \\
& ($t=1,2,3,4$ $i=1,2\cdots s_2$ $j=1,2$) \\
$g_{tijk}$ & 地第$i$辆1-2型轿运车第$j$列所容纳的$k$型乘用车的数目 \\
& ($t=1,2,3,4$ $i=1,2\cdots s_2$ $j=1,2,3$) \\
$\sigma$ & 乘用车横向、纵向车间距 \\
$w_i$ & 起始点到目的地A、B、C、D、E的距离 ($i=1,2,\cdots 5$) \\
\hline
\end{tabular}
\end{table}

\subsubsection{问题二模型改进}

为了保证 1-1 型、1-2 型轿运车的最大利用率,讨论最后一辆 1-2 型轿运车与其之后的 1-1 型轿运车的运输量是否可以用相同数量的 1-1 型轿运车来代替,如果可以,则优化方案,重新记录末尾辆的轿运车的装载方式,否则原计算结果不变即可计算出一种运输方式的结果,随后对所得结果进行优化。详细模型如下所示:

首先计算最后 1-2 型轿运车出现后的Ⅱ、Ⅲ车辆总数 $m_{1-2}$:

\begin{equation}
m_{1-2} = \sum_{i=n1}^{n2} \sum_{j=1}^{2} \sum_{k=1}^{3} c_{ijk} + \sum_{i=n1}^{n2} \sum_{j=1}^{3} \sum_{k=1}^{3} d_{ijk}
\tag{16}
\end{equation}

其中第 $n_1$ 为最后一辆 1-2 出现时的车辆数,第 $n_2$ 为总车辆数。

随后调用 1-1 型轿运车的最优装载方式,计算出全部使用 1-1 轿运车的数量 $n_3$。如果 $n_3 = n_2$,虽然总车数不变,1-2 型被 1-1 型替换之后,成本降低,结果得到优化。

利用 MATLAB 编程求解,得出总的车辆数为:13 辆轿运车,其中 1-1 型轿运车 12 辆,1-2 型轿运车 1 辆。各类型轿运车的使用数量如下表7,轿运车的乘用车装载方案如下表8:

\begin{table}[h]
\centering
\caption{各类型轿运车的使用数量}
\begin{tabular}{|c|c|}
\hline
轿运车类型 & 总使用量 \\
\hline
1-1型轿运车 & 12 \\
\hline
1-2型轿运车 & 1 \\
\hline
\end{tabular}
\end{table}

\begin{table}[h]
\centering
\caption{问题二轿运车的乘用车装载方案}
\begin{tabular}{|c|c|c|c|c|c|c|}
\hline
轿运车 & 相同类型、 & 装在上 & 装在上 & 装在上 & 装在下 & 装在下 \\
类型 & 相同装载 & 层Ⅰ型 & 层Ⅱ型 & 层Ⅲ型 & 层Ⅰ型 & 层Ⅱ型 \\
 & 方式的车辆数 & 乘用车 & 乘用车 & 乘用车 & 乘用车 & 乘用车 \\
 & & 的数量 & 的数量 & 的数量 & 的数量 & 的数量 \\
\hline
1 & 11 & 0 & 5 & 0 & 0 & 4 \\
\hline
2 & 1 & 0 & 12 & 0 & 0 & 5 \\
\hline
1 & 1 & 0 & 4 & 0 & 1 & 3 \\
\hline
\end{tabular}
\end{table}

注:轿运车类型中:1代表1-1型轿运车,2代表1-2型轿运车。

\subsubsection{问题二结果分析}

从上述表5、表6、表7、表8的结果可以看出,优化以后的轿运车的数量总和不变为13辆,但是表7的结果中1-1型轿运车的调用为12辆,1-2的轿运车调用为1辆与表5的结果相比1-2型轿运车的调用减少1辆,1-2型轿运车的调用增加1辆。影响成本高低的首先是轿运车使用数量;其次由于在轿运车使用数量相同情况下,1-1型轿运车的使用成本较低,2-2型较高,1-2型略低于前两者的平均值。故在保证轿运车数量总和不变的情况下,考虑运输成本设1-1型轿运车的使用成本为$P$,1-2型轿运车的使用成本为$Q$,则$P<Q$。故对于表7结果的使用成本为$11P+2Q$,表5结果的使用成本为$12P+Q$,两者的使用成本进行比较得:

\begin{equation}
11P+2Q < 12P+Q
\tag{17}
\end{equation}

由此可得,在车辆数不变的,目的地只有一个,不考虑行驶里程的情况下,表7的运输成本小于表5的运输成本,故表7的求解结果优于表5的结果。

\subsection{问题三的模型建立与求解}

\subsubsection{问题三分析}

问题三同样是对于物流公司制定运输方案的问题,建立逐层优化整数规划模型。问题三要求运输Ⅰ车型的乘用车156辆、Ⅱ车型的乘用车102辆及Ⅲ车型的乘用车39辆,且Ⅲ车型的乘用车只能装在1-1、1-2型下层。在求解中需要优先考虑Ⅲ车型的乘用车的运输安排。相比问题一、二,问题三同时有Ⅰ、Ⅱ与Ⅲ型乘用车,需要在此基础上同时考虑3种车,是问题一、二的提升。在求解问题三时,编写一个了通用程序,求解出问题三,同时也可以求解问题一和二。

\subsubsection{问题三模型建立}

根据问题一、问题二的求解过程及轿运车的装载要求,建立逐层优化的整数规划模型。

\begin{equation}
\text{Obj} \quad \min s = \sum_{i=1}^{2} s_i
\tag{18}
\end{equation}

\begin{equation}
\begin{aligned}
\text{St.} \quad
\begin{cases}
\sum_{k=2}^{3} (a_k + \sigma)c_{ijk} - \sigma \leq L_1, \forall i, \forall j \\
\sum_{k=2}^{3} (a_k + \sigma)d_{ijk} - \sigma \leq L_2, \forall i, \forall j \\
\sum_{i=1}^{s_1} \sum_{j=1}^{2} c_{ij1} + \sum_{i=1}^{s_2} \sum_{j=1}^{3} d_{ij} = 156 \\
\sum_{i=1}^{s_1} \sum_{j=1}^{2} c_{ij2} + \sum_{i=1}^{s_2} \sum_{j=1}^{3} d_{ij2} = 102 \\
\sum_{i=1}^{s_1} \sum_{j=1}^{2} c_{ij3} + \sum_{i=1}^{s_2} \sum_{j=1}^{3} d_{ij3} = 39 \\
c_{i23} = 0 \\
d_{ij3} = 0, j = 2, 3 \\
s_2 \leq 0.2 \times s_1
\end{cases}
\tag{19}
\end{aligned}
\end{equation}

\subsubsection{问题三模型求解}

在问题三中 1-2 型乘用车最优装载方式原理相同,只是增加了 I 型乘用车的运输任务。I 型乘用车可以放在 1-1、1-2 型轿运车的上下层,所以只需在问题二求解过程的基础上,对轿运车装载方式进行优化。

针对 1-1 型轿运车的最优装载方式模型如下:

\begin{equation}
\text{Obj} \quad \max a_1c_{i11} + a_2c_{i12} + a_2c_{i13} + a_1c_{i21} + a_2c_{i22} + a_2c_{i23} + a_1c_{i31} + a_2c_{i32} + a_2c_{i33}
\tag{20}
\end{equation}

\begin{equation}
\begin{aligned}
\text{S.t.} \quad
\begin{cases}
(a_1 + \sigma)c_{i11} + (a_2 + \sigma)c_{i12} + (a_2 + \sigma)c_{i13} - \sigma \leq L_2 \\
(a_1 + \sigma)c_{i21} + (a_2 + \sigma)c_{i22} + (a_2 + \sigma)c_{i23} - \sigma \leq L_2 \\
(a_1 + \sigma)c_{i31} + (a_2 + \sigma)c_{i32} + (a_2 + \sigma)c_{i33} - \sigma \leq L_2 \\
\sum_{j=1}^{3} c_{ijk} \leq m_k', k = 1, 2, 3 \\
c_{i11}, c_{i12}, c_{i21}, c_{i22}, c_{i31}, c_{i32} \in \mathbb{N}
\end{cases}
\tag{21}
\end{aligned}
\end{equation}

针对 1-2 型轿运车的最优装载方式模型如下:
\begin{equation}
\text{Obj } \max a_{1}d_{i11} + a_{2}d_{i12} + a_{2}d_{i13} + a_{1}d_{i21} + a_{2}d_{i22} + a_{2}d_{i23} + a_{1}d_{i31} + a_{2}d_{i32} + a_{2}d_{i33}
\tag{22}
\end{equation}

\begin{equation}
\text{S.t. } \left\{
\begin{aligned}
(a_{1} + \sigma)d_{i11} + (a_{2} + \sigma)d_{i12} + (a_{2} + \sigma)d_{i13} - \sigma & \leq L_{2} \\
(a_{1} + \sigma)d_{i21} + (a_{2} + \sigma)d_{i22} + (a_{2} + \sigma)d_{i23} - \sigma & \leq L_{2} \\
(a_{1} + \sigma)d_{i31} + (a_{2} + \sigma)d_{i32} + (a_{2} + \sigma)d_{i33} - \sigma & \leq L_{2} \\
\sum_{j=1}^{3} d_{i3k} & \leq m_{k}^{'} \, k = 1, 2, 3 \\
d_{i22} & = 0 \\
d_{i32} & = 0 \\
d_{i33} & = 0 \\
d_{i11}, d_{i12}, d_{i21}, d_{i22}, d_{i31}, d_{i32} & \in \mathbb{N}
\end{aligned}
\right.
\tag{23}
\end{equation}

通过对问题一、问题二的求解过程的综合分析,得出针对问题一、问题二、问题三的通用优化程序,具体算法步骤如下:

\begin{itemize}
    \item Step1:输入 I、II 与 III 型乘用车的数量;
    \item Step2:安排 1-1 型轿运车进行运输,对每一辆轿运车均进行一次优化计算,优化模型如问题一所示;
    \item Step3:判断车辆是否运输完成,如果未运完,跳回 Step2;
    \item Step4:统计当前已经安排的 1-1 型轿运车的数量,当它的数量是 5 的倍数时,就考虑能否使用 1-2 型车,如果 1-1 型轿运车的数量不是 5 的倍数,则转至 Step2;
    \item Step5:如果剩下的车辆能被 1 辆 1-1 型轿运车全部运完,选择调用 1-1 型轿运车,否则使用 1-2 型车;
    \item Step6:判断车辆是否运输完成,如果未运完,则转至 Step2;
    \item Step7:经过以上计算,可以得出初步的结果;
    \item Step8:利用问题一与问题二的优化程序,对初步结果进行优化,得出最终结果,并输出至 Excel 文件中。
\end{itemize}

逐层整数规划求解算法程序框图如下图 1:

\begin{figure}[h]
    \centering
    \includegraphics[width=\textwidth]{image.png}
    \caption{逐层优化整数规划算法框图}
    \label{fig:algorithm_flowchart}
\end{figure}

利用 MATLAB 编程求解(程序见附录 1、附录 2),得出总的车辆数为:30 辆轿运车,其中 1-1 型轿运车 25 辆,1-2 型轿运车 5 辆。各类型轿运车的使用数量如下表 \ref{tab:car_usage},轿运车的乘用车装载方案如下表 \ref{tab:car_loading}:

\begin{table}[h]
    \centering
    \caption{问题三各类型轿运车的使用数量}
    \label{tab:car_usage}
    \begin{tabular}{|c|c|}
        \hline
        轿运车类型 & 总使用量 \\
        \hline
        1-1 型轿运车 & 25 \\
        \hline
        1-2 型轿运车 & 5 \\
        \hline
    \end{tabular}
\end{table}

\begin{table}
\centering
\begin{tabular}{|c|c|c|c|c|c|c|c|}
\hline
轿运车 & 相同类型、 & 装在上 & 装在上 & 装在上 & 装在下 & 装在下 & 装在下 \\
类型 & 相同装载 & 层Ⅰ型 & 层Ⅱ型 & 层Ⅲ型 & 层Ⅰ型 & 层Ⅱ型 & 层Ⅲ型 \\
 & 方式的车辆数 & 乘用车 & 乘用车 & 乘用车 & 乘用车 & 乘用车 & 乘用车 \\
 & & 的数量 & 的数量 & 的数量 & 的数量 & 的数量 & 的数量 \\
\hline
1 & 8 & 4 & 0 & 0 & 0 & 0 & 4 \\
\hline
2 & 1 & 4 & 8 & 0 & 0 & 0 & 5 \\
\hline
1 & 1 & 4 & 0 & 0 & 2 & 0 & 2 \\
\hline
1 & 12 & 4 & 0 & 0 & 4 & 0 & 0 \\
\hline
2 & 3 & 4 & 8 & 0 & 2 & 4 & 0 \\
\hline
1 & 4 & 0 & 5 & 0 & 0 & 5 & 0 \\
\hline
2 & 1 & 0 & 12 & 0 & 0 & 6 & 0 \\
\hline
\end{tabular}
\caption{问题三轿运车的乘用车装载方案}
\end{table}

注:轿运车类型中:1 代表 1-1 型轿运车,2 代表 1-2 型轿运车。

\subsubsection{问题三结果分析}

由上表 9、表 10 可以得出问题三中运输所需要的各类型轿运车的数量以及每辆轿运车的乘用车装载方案。针对问题三求解过程中所提出的通用程序对问题一、问题二进行求解,得到问题一、问题二的求解结果与上述问题一、问题二的求解结果相同。由此可见,对于问题三中的通用程序适用于问题一、问题二的求解,即问题三中的通用程序具有适用性。

\subsection{问题四的模型建立与求解}

\subsubsection{问题四分析}

对于问题四,要求运输 166 辆Ⅰ车型的乘用车(其中目的地是 A、B、C、D 的分别为 42、50、33、41 辆)和 78 辆Ⅱ车型的乘用车(其中目的地是 A、C 的分别为 31、47 辆)。各目的地之间的距离为 OD=160,OC=236,OB=280,OA=360,如下图 2 所示

\begin{figure}[h]
\centering
\begin{tikzpicture}[scale=0.8]
    \node[circle, draw, inner sep=2pt] (O) at (0,0) {O};
    \node[circle, draw, inner sep=2pt] (D) at (0,4) {D};
    \node[circle, draw, inner sep=2pt] (B) at (3,6) {B};
    \node[circle, draw, inner sep=2pt] (A) at (4,8) {A};
    \node[circle, draw, inner sep=2pt] (C) at (1,2) {C};
    
    \draw (O) -- (D) node[midway, left] {160};
    \draw (D) -- (B) node[midway, right] {120};
    \draw (B) -- (A) node[midway, right] {80};
    \draw (D) -- (C) node[midway, left] {76};
\end{tikzpicture}
\caption{问题四运输线路}
\end{figure}

相比较于前三问,问题四的目的地不同,多了路径的选择。行驶的里程短可以减少一定量的成本,因此必须考虑对路径的优化。为了实现轿运车数量、型号和行驶里程的最优解,必须多目标逐层优化,寻找合适的配送方式,提出启发式逐层优化算法。首先,可以利用问题三的通用程序估算所需的轿运车的数量;其次考虑优先运输远距离目的地的乘用车,由于 1-2 型轿运车的运输量较大,故对于远距离目的地优先使用 1-2 型轿运车运输;然后考虑次远距离目的地的运输方案,最后得出 A、B、C、D 目的地轿运车辆的数目及运输方案所行驶的里程。

\subsubsection{问题四模型建立}

针对问题的约束建立多层次逐层优化 \cite{ref7} 模型,第一层对 1-1、1-2 车型的选择进行优化,第二层对路径选择进行优化,第三层计算对轿运车的数量的优化解,也即对轿运车的装载方式进行优化。

\textbf{第一层的模型:}

\begin{equation}
Obj: \max \sum_{t=1}^{4} \sum_{j=1}^{s_{2}} \sum_{j=1}^{3} \sum_{k=1}^{2} g_{tijk}
\tag{24}
\end{equation}

\begin{equation}
S.t. \sum_{t=1}^{4} \sum_{j=1}^{s_{2}} \sum_{j=1}^{3} \sum_{k=1}^{2} g_{tijk} \leq 0.2 \times \sum_{t=1}^{4} \sum_{i=1}^{s_{1}} \sum_{j=1}^{2} \sum_{k=1}^{2} f_{tijk}
\tag{25}
\end{equation}

\textbf{第二层的模型:}

\begin{equation}
Obj \quad \min \sum_{j=1}^{4} \sum_{i=1}^{2} w_{j} s_{i1}
\tag{26}
\end{equation}

\textbf{第三层的模型如下所示:}

\begin{equation}
Obj \quad \min s = \sum_{j=1}^{4} \sum_{i=1}^{2} s_{ij}
\tag{27}
\end{equation}

\begin{equation}
\begin{cases}
\sum_{k=1}^{2} \left( a_{1} + \sigma \right) f_{tijk} - \sigma \leq L_{1} t = 1, 2, 3, 4, i = 1, 2 \cdots s_{1}, j = 1, 2 \\
\sum_{k=1}^{2} \left( a_{2} + \sigma \right) g_{tijk} - \sigma \leq L_{2} t = 1, 2, 3, 4, i = 1, 2 \cdots s_{2}, j = 1, 2, 3 \\
\sum_{i=1}^{s_{1}} \sum_{j=1}^{2} f_{1ij1} + \sum_{i=1}^{s_{1}} \sum_{j=1}^{3} g_{1ij1} = 42 \\
\sum_{i=1}^{s_{1}} \sum_{j=1}^{2} f_{2ij1} + \sum_{i=1}^{s_{1}} \sum_{j=1}^{3} g_{2ij1} = 50 \\
\sum_{i=1}^{s_{1}} \sum_{j=1}^{2} f_{3ij1} + \sum_{i=1}^{s_{1}} \sum_{j=1}^{3} g_{3ij1} = 33 \\
\sum_{i=1}^{s_{1}} \sum_{j=1}^{2} f_{4ij1} + \sum_{i=1}^{s_{1}} \sum_{j=1}^{3} g_{4ij1} = 41 \\
\sum_{i=1}^{s_{1}} \sum_{j=1}^{2} f_{1ij2} + \sum_{i=1}^{s_{1}} \sum_{j=1}^{3} g_{1ij2} = 31 \\
\sum_{i=1}^{s_{1}} \sum_{j=1}^{2} f_{3ij2} + \sum_{i=1}^{s_{1}} \sum_{j=1}^{3} g_{3ij2} = 47 \\
\sum_{t=1}^{4} \sum_{j=1}^{s_{2}} \sum_{j=1}^{3} \sum_{k=1}^{2} g_{tijk} \leq 0.2 \times \sum_{t=1}^{4} \sum_{i=1}^{s_{1}} \sum_{j=1}^{2} \sum_{k=1}^{2} f_{tijk} \\
f_{tijk}, g_{tijk} \in N
\end{cases}
\tag{28}
\end{equation}

\subsubsection{问题四模型求解}

考虑不同目的地所须运输的车辆数目不同,提出启发式逐层优化算法 \cite{ref8}。

首先针对层次一的车型优化,利用问题三的通用程序,求解出无路径要求的车型优化解,作为估算值。在此基础上基本可以确定 1-2 型车的数量,即使 1-2 型车数量估算有误,有微小增加,也通过可以调整程序参数快速重新计算。

其次考虑层次二路径优化,尽量到把运载量大的车优先派往远距离目的地,即可实现总路径最短。由于 1-2 型轿运车的运输量较大,故对于远距离目的地优先使用 1-2 型轿运车运输。首先考虑距离最远的目的地,尽量派运载量最大的车型,然后考虑次远距离目的地的运输方案,依次计算,得出 A、B、C、D 四目的地轿运车辆数目及运载方式。

最后层次三对车辆的装载方式进行优化。优化模型如下:

针对 1-2 型轿运车的最优装载方式模型如下:
\begin{equation}
\begin{aligned}
\text{Obj} \quad & \max a_1 g_{ti11} + a_2 g_{ti12} + a_1 g_{ti21} + a_2 g_{ti22} + a_1 g_{ti31} + a_2 g_{ti32} \\
\text{S.t.} \quad & \begin{cases}
(a_1 + \sigma) g_{ti11} + (a_2 + \sigma) g_{ti12} - 0.1 \leq L_2 \\
(a_1 + \sigma) g_{ti21} + (a_2 + \sigma) g_{ti22} - 0.1 \leq L_2 \\
(a_1 + \sigma) g_{ti31} + (a_2 + \sigma) g_{ti32} - 0.1 \leq L_2 \\
\sum_{j=1}^3 g_{tij1} \leq m_1' \\
\sum_{j=1}^3 g_{tij2} \leq m_2' \\
g_{tijk} \in \mathbb{N}, j = 1, 2, 3, k = 1, 2
\end{cases}
\end{aligned}
\tag{29}
\end{equation}

针对 1-1 型轿运车的最优装载方式模型如下:
\begin{equation}
\begin{aligned}
\text{Obj} \quad & \max a_1 g_{ti11} + a_2 g_{ti12} + a_1 g_{ti21} + a_2 g_{ti22} + a_1 g_{ti31} + a_2 g_{ti32} \\
\text{S.t.} \quad & \begin{cases}
(a_1 + \sigma) g_{ti11} + (a_2 + \sigma) g_{ti12} - \sigma \leq L_2 \\
(a_1 + \sigma) g_{ti21} + (a_2 + \sigma) g_{ti22} - \sigma \leq L_2 \\
(a_1 + \sigma) g_{ti31} + (a_2 + \sigma) g_{ti32} - \sigma \leq L_2 \\
\sum_{j=1}^3 g_{tij1} \leq m_1' \\
\sum_{j=1}^3 g_{tij2} \leq m_2' \\
g_{tijk} \in \mathbb{N}, j = 1, 2, 3, k = 1, 2
\end{cases}
\end{aligned}
\tag{31}
\end{equation}

在对装载方式进行优化时,两目的地交替时的车辆安排,有可能出现前一目的地装车不满的情况,这是应尽量给沿途较远的下一目的地捎带乘用车,避免有空位,造成浪费。

针对问题四建立的启发式多层次逐层优化方法的计算步骤如下:

\begin{itemize}
    \item Step1:利用问题三的通用程序,计算出无路径限制的轿运车使用数量,计算得使用 1-1 型轿运车 21 辆,1-2 型轿运车 4 辆。
    \item Step2:对最远的 A 地优先使用运载量大的 4 辆 1-2 型轿运车,随后使用 1-1型轿运车,每辆车均进行装载方式最优的计算。
    \item Step3:A地使用的最后一辆1-1型轿运车是装不满的,可以给去往A地沿途中,距离起点次远B点捎部分车。
    \item Step4:随后依次对B、C、D点进行分配,循环上述2-3步,直到所有目的地的车辆都运输完成。
\end{itemize}

利用MATLAB编程求解(见附录3),得出总的车辆数为:25辆轿运车,其中1-1型轿运车21辆,1-2型轿运车4辆。各类型轿运车的使用数量如下表11,轿运车的乘用车装载方案如下表12:

\textbf{表11 第四问各类型轿运车的使用数量}

\begin{tabular}{|c|c|}
\hline 轿运车类型 & 总使用量 \\
\hline 1-1型轿运车 & 21 \\
\hline 1-2型轿运车 & 4 \\
\hline
\end{tabular}

\textbf{表12 问题三轿运车的乘用车装载方案}

\begin{tabular}{|c|c|c|c|c|c|c|c|c|c|}
\hline 轿 & 相同类型、 & 装在 & 装在 & 装在 & 装在 & 装在 & 装在 & 中 & 目的 \\
运 & 相同装载 & 上层 & 上层 & 下层 & 下层 & 下层 & 下层 & 间 & 地 \\
车 & 方式的 & Ⅰ型 & Ⅱ型 & Ⅰ型 & Ⅱ型 & Ⅲ型 & Ⅲ型 & 停 & \\
型 & 车辆数 & 乘用车 & 乘用车 & 乘用车 & 乘用车 & 乘用车 & 乘用车 & 地 & \\
 & & 的数量 & 的数量 & 的数量 & 的数量 & 的数量 & 的数量 & & \\
\hline 2 & 2 & 4 & 8 & 0 & 2 & 4 & 0 & 0 & 1 \\
\hline 2 & 1 & 7 & 4 & 0 & 5 & 0 & 0 & 0 & 1 \\
\hline 2 & 1 & 10 & 0 & 0 & 5 & 0 & 0 & 0 & 1 \\
\hline 1 & 1 & 1 & 3 & 0 & 4 & 0 & 0 & 2 & 1 \\
\hline 1 & 6 & 4 & 0 & 0 & 4 & 0 & 0 & 0 & 2 \\
\hline 1 & 4 & 4 & 0 & 0 & 4 & 0 & 0 & 0 & 3 \\
\hline 1 & 4 & 0 & 5 & 0 & 0 & 5 & 0 & 0 & 3 \\
\hline 1 & 1 & 2 & 2 & 0 & 0 & 5 & 0 & 4 & 3 \\
\hline 1 & 5 & 4 & 0 & 0 & 4 & 0 & 0 & 0 & 4 \\
\hline
\end{tabular}

注:轿运车类型中:1代表1-1型轿运车,2代表1-2型轿运车。中间停靠地和目的地中:0代表不停靠,1代表A,2代表B,3代表C,4代表D。

根据表12的运输方式可得各个目的地运输的车辆数目,由此可得完成运输任务的总运输里程为:

\[
\sum_{j=1}^{4} \sum_{i=1}^{2} w_{j} S_{i_{1}} = 360 \times 5 + 6 \times 280 + 9 \times 236 + 5 \times 160 = 6404 \, \text{km}
\]

\subsubsection{问题四结果分析}

根据表11、12可知A、B、C、D目的地轿运车辆的数目分别为:A地:4辆1-2型轿运车,1辆1-1型轿运车;B地:6辆1-1型轿运车;C地:9辆1-1型轿运车;D地:5辆1-1型轿运车。完成运输计划共需要21辆1-1型轿运车,4 辆 1-2 型轿运车,并得出每辆轿运车的乘用车装载方案及运输方案所行驶的里程为 6404km。

\subsection{问题五的模型建立与求解}

\subsubsection{问题五分析}

对于问题五,物流公司需要用 10 种轿运车运输 45 种乘用车送往 \(A, B, C, D, E\) 这 5 个目的地,每种轿运车的拥有量有限,不同目的地的不同乘用车的需求量也不尽相同,其装载、运输方案太多,情况纷繁复杂,想要得到最优解是不切实际的,因此,考虑采用启发式算法对问题进行简化处理。

10 种轿运车的运输能力不同,行驶同样里程花费的成本是相同的,因此考虑将运输能力强的车派往远的目的地。由于情况复杂,考虑将问题简化成单个目的地依次配车。

\subsubsection{问题五模型建立}

针对该乘用车物流运输计划问题建立构造型分层划分模型 \({ }^{[9]}\),第一层对轿运车的数量进行建模,也即对轿运车的装载方式进行配置,第二层对轿运车车型的选择进行配置,第三层对路径选择进行配置。

\textbf{第一层的模型:}

\begin{equation}
\text{Obj} \quad \min s = \sum_{j=1}^{5} \sum_{i=1}^{10} s_{ij}
\tag{34}
\end{equation}

\begin{equation}
\begin{aligned}
S.t. \left\{
\begin{aligned}
& \sum_{k=1}^{45} (a_k + \sigma) R_{tijkm} - \sigma \leq L_m, t=1,2,\cdots,5, i=1,2,\cdots,s_1, j=1,2, m=1,2,3,4,6,7,8 \\
& \sum_{k=1}^{45} (a_k + \sigma) R_{tijkm} - \sigma \leq L_m, t=1,2,\cdots,5, i=1,2,\cdots,s_1, j=1,2,3, m=1,10 \\
& \sum_{k=1}^{45} (a_k + \sigma) R_{tijkm} - \sigma \leq L_m, t=1,2,\cdots,5, i=1,2,\cdots,s_1, j=1,2,3,4, m=9 \\
& \sum_{i=1}^{s_1} \sum_{j=1}^{4} \sum_{m=1}^{10} R_{tijkm} = q_{tk}, t=1,2,\cdots,5, k=1,2,\cdots,45 \\
& h_5 + h_9 \leq 0.2 \times (\sum_{m=1}^{4} h_m + \sum_{m=6}^{8} h_m) \\
& R_{tijk}, h_m \in N
\end{aligned}
\right.
\tag{35}
\end{aligned}
\end{equation}

\textbf{第二层的模型:}

\begin{equation}
\text{Obj} \quad \max h_5 + h_{10}
\tag{36}
\end{equation}

\begin{equation}
S.t. \quad h_5 + h_{10} \leq 0.2 \times (\sum_{m=1}^{4} h_m + \sum_{m=6}^{8} h_m)
\tag{37}
\end{equation}

\textbf{第三层的模型:}

\begin{equation}
\text{Obj} \quad \min w_1 \times \sum_{i=1}^{45} s_{i1} + w_2 \times \sum_{i=1}^{45} s_{i2} + w_3 \times \sum_{i=1}^{45} s_{i3} + w_4 \times \sum_{i=1}^{45} s_{i4} + w_5 \times \sum_{i=1}^{45} s_{i5}
\tag{38}
\end{equation}

式(34)-(38)中:$R$ 代表乘用车的辆数;$h$ 代表轿运车的辆数;$m$ 代表轿运车的编号,$1, 2, 3, 4, 6, 7, 8$ 代表 1-1 型轿运车的编号,$5, 10$ 代表 1-2 型轿运车的编号,$9$ 代表 2-2 型轿运车的编号。

\subsubsection{问题五模型求解}

经分析可知,$E$ 地和 $A$ 地高度小于 $1.7\mathrm{~m}$、宽度小于等于 $1.7\mathrm{~m}$ 的乘用车占总乘用车数的比重较小,如果将 2-2 型轿运车派往 $E$ 地或者 $A$ 地,就会使高度小于 $1.7\mathrm{~m}$、宽度小于等于 $1.7\mathrm{~m}$ 的乘用车数量骤减,继续配 1-2 型轿运车会使其上层空缺,所以只能 2-2 型轿运车和 1-1 型轿运车搭配使用。

由于 2-2 型轿运车的车长和 1-1 型轿运车的平均车长都比 1-2 型轿运车的平均车长短,所以一辆 2-2 型轿运车和一辆 1-1 型轿运车的组合不如两辆 1-2 型轿运车的运输能力强。基于 $E$ 地和 $A$ 地(目的地最远)高度小于 $1.7\mathrm{~m}$、宽度小于等于 $1.7\mathrm{~m}$ 的乘用车占总乘用车数的比重小的前提,考虑给 $E$ 地和 $A$ 地优先配送 1-2 型轿运车。

由于 1-2 型轿运车的使用数量不能超过 1-1 型轿运车的使用数量的 $20\%$,估算五个目的地的 1-2 型轿运车的用量上限刚好能满足 $E$ 地和 $A$ 地的需求,其他目的地只能配送 2-2 型轿运车和 1-1 型轿运车。又知道运输能力强的车派往远的目的地可以节省路程,考虑给剩余三个目的地中最远的目的地配送 2-2 型轿运车。拟定给 $B$ 地和 $C$ 地配送 2-2 型轿运车和 1-1 型轿运车,$D$ 地配送 1-1 型轿运车。

由于其他目的地的配置情况不能确定,考虑首先给 $E$ 地配置 1-2 型轿运车。由于 1-2 型轿运车上层只能放置高度小于 $1.75\mathrm{~m}$,宽度小于等于 $1.7\mathrm{~m}$ 的乘用车,所以高度小于 $1.75\mathrm{~m}$,宽度小于等于 $1.7\mathrm{~m}$ 的乘用车的数量决定了 1-2 型轿运车的数量。在 $E$ 地高度小于 $1.75\mathrm{~m}$,宽度小于等于 $1.7\mathrm{~m}$ 的乘用车用完之后,考虑给 $E$ 地配置 1-1 型轿运车,最后有空地再给 $B$ 地捎带乘用车。然后考虑给 $B$ 地配送 2-2 型轿运车,由 $B$ 地的高度小于 $1.75\mathrm{~m}$,宽度小于等于 $1.7\mathrm{~m}$ 的乘用车数量决定 2-2 型轿运车的数量为 3 辆,剩余乘用车用 1-1 型轿运车运输,不满一辆的就先不配轿运车,留一少部分给 $A$ 地不满的轿运车补满。接着考虑给 $C$ 地配送两辆 2-2 型轿运车,剩余乘用车用 1-1 型轿运车运输,最后有空地再给 $D$ 地捎带乘用车。然后给 $D$ 地配送 1-1 型轿运车,不满一辆的就先不配轿运车,留一少部分给 $A$ 地和 $B$ 地不满的轿运车补满。最后考虑给 $A$ 地配送 1-2 型轿运车,此时其它四个目的地的 1-1 型轿运车和 1-2 型轿运车的数量都已经确定,可以算出 $A$ 地首先能配送 5 辆 2-2 型轿运车,再按 5 个 1-1 型轿运车和 1 个 1-2 型轿运车的配置方式循环配置,最后有空地再给 $B$ 地和 $D$ 地捎带乘用车。然后依次给 $B$ 地和 $D$ 地剩余的乘用车配置轿运车。

考虑不同目的地所须的乘用车车辆数目不同,提出构造型分层划分启发式算法。在求解过程中采用单个目的地依次配车,模型如下所示:

\begin{equation}
\text{Obj} \quad \max \sum_{i=1}^{45} a_i c_i
\tag{39}
\end{equation}

\begin{equation}
S.t.
\begin{cases}
\sum_{i=1}^{45} (a_i + \sigma)c_i - \sigma \leq L_j \\
\sum_{i=1}^{45} c_i \leq m_i' \\
c_i \in N, i = 1, 2, \dots, 45
\end{cases}
\tag{40}
\end{equation}

五个目的地的配车顺序和配车类型分别为:$E$ 地配 1-2 型轿运车和 1-1 型轿运车;$B$ 地配 2-2 型轿运车和 1-1 型轿运车;$C$ 地配 2-2 型轿运车和 1-1 型轿运车;$D$ 地配 1-1 型轿运车;$A$ 地配 1-2 型轿运车和 1-1 型轿运车。

由于高度大于 1.7m 的乘用车只能放在 1-1 型轿运车和 1-2 型轿运车的下层,2-2 型轿运车和 10 号 1-2 型轿运车的上层只能装载宽度小于 1.7m 的乘用车,5 号 1-2 型轿运车的上层只能装载宽度小于 1.75m 的乘用车,所以考虑将乘用车分为四类:高度大于 1.7m 的乘用车,高度小于 1.7m、宽度大于 1.75m 的乘用车,高度小于 1.7m、宽度大于 1.7m 小于 1.75m 的乘用车和高度小于 1.7m、宽度小于 1.7m 的乘用车。按此分类原则将五个目的地的乘用车逐次分类,然后进行装载方案的配置。

求解结果如表 13 所示:

\begin{table}[h]
\centering
\caption{问题五不同目的地不同轿运车的分配方案}
\begin{tabular}{|c|c|c|c|c|c|c|}
\hline
序号 & \diagbox{目的地}{车辆类型} & A & B & C & D & E \\ \hline
1 & 1-1 型 & 4 & 0 & 0 & 17 & 0 \\ \hline
2 & 1-1 型 & 17 & 1 & 0 & 0 & 0 \\ \hline
3 & 1-1 型 & 0 & 14 & 0 & 0 & 8 \\ \hline
4 & 1-1 型 & 0 & 0 & 15 & 0 & 0 \\ \hline
5 & 1-2 型 & 1 & 0 & 0 & 0 & 9 \\ \hline
6 & 1-1 型 & 0 & 1 & 0 & 0 & 0 \\ \hline
7 & 1-1 型 & 0 & 0 & 4 & 0 & 0 \\ \hline
8 & 1-1 型 & 3 & 0 & 4 & 9 & 0 \\ \hline
9 & 2-2 型 & 0 & 3 & 2 & 0 & 0 \\ \hline
10 & 1-2 型 & 7 & 0 & 0 & 0 & 0 \\ \hline
\end{tabular}
\end{table}

\subsubsection{问题五结果分析}

由表 13 可得不同目的地不同轿运车的安排方式为:$E$ 地安排 17 辆轿运车;$B$ 地安排 19 辆轿运车;$C$ 地安排 25 辆轿运车;$D$ 地安排 26 辆轿运车;$A$ 地安排 32 辆轿运车。由于问题五的装载、运输方案太多,所以求解结果在满足轿运车数量约束的前提下,得到问题优化的可行解,即完成运输任务时调运 97 辆 1-1 型轿运车,17 辆 1-2 型轿运车,5 辆 2-2 型轿运车,共计 119 辆轿运车。

\section{模型的评价}

\textbf{模型的优点:}
\begin{enumerate}
    \item 采用的整数线性规划模型简单,能够较优的安排运输车辆的数目及装载方案;
    \item 问题二的求解过程中提出有效的优化方案充分考虑到轿运车辆数目及运输成本的耗费,得出优化的可行方案;
    \item 问题三综合了问题一、问题二的优化模型给出通用程序,并检验问题一、问题二的求解结果,通用程序具有适用性;
    \item 问题四中充分利用问题三的通用模型,在保证轿运车数量较少的情况下,分层安排远距离运输点,有效减少运输成本;
    \item 问题五的求解中分层优化考虑,得出优化的可行解即运输方案及不同类型轿运车,不同类型乘用车的用书方案。
\end{enumerate}

\textbf{模型的缺点:}
\begin{enumerate}
    \item 对于问题三所求解的通用模型,只能满足3种不同乘用车的装载方式,当乘用车类型较多时,不能得出最优装载方案;
    \item 问题五的构造型分层划分启发式算法的实现较复杂。
\end{enumerate}

[REFERENCES:1]

\end{document}

% Missing placeholders restored
\includegraphics[width=0.3\textwidth]{image1.png}
\includegraphics[width=0.3\textwidth]{image2.png}
\includegraphics[width=0.3\textwidth]{image3.png}