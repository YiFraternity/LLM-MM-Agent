\begin{center}
\textbf{第十一届华为杯全国研究生数学建模竞赛}
\end{center}

\begin{table}[h]
\centering
\begin{tabular}{c c}
\hline
学校 & 信阳师范学院 \\
\hline
参赛队号 & 10477002 \\
\hline
队员姓名 & 1. 马东东 \\
 & 2. 郑桂荣 \\
 & 3. 马文娟 \\
\hline
\end{tabular}
\end{table}

\begin{flushright}
参赛密码 \underline{\hspace{2cm}} \\
(由组委会填写)
\end{flushright}

\begin{center}
\textbf{参赛密码} \underline{\hspace{2cm}} \\
(由组委会填写)
\end{center}

\begin{center}
\includegraphics[width=0.3\textwidth]{image1.png} \quad
\includegraphics[width=0.3\textwidth]{image2.png} \quad
\includegraphics[width=0.3\textwidth]{image3.png}
\end{center}

\begin{center}
\textbf{第十一届华为杯全国研究生数学建模竞赛}
\end{center}

\begin{flushleft}
\textbf{题目} \hspace{1cm} 乘用车物流运输计划问题的探究
\end{flushleft}

\begin{center}
\textbf{摘 \hspace{2cm} 要:}
\end{center}

本文针对我国汽车工业的高速发展,整车物流量,特别是乘用车的整车物流量迅速增长出现的乘用车物流运输计划问题进行了数学建模。

对于问题一,根据乘用车和轿运车的规格,分析 1-1 型、1-2 型轿运车装载 I、II 型乘用车的所有可行方案,建立整数规划模型,利用 lingo 软件编程求得用 16 辆 1-1 型和 2 辆 1-2 型轿运车即可完成运载任务,并给出具体装载方案。

对于问题二,分析 1-1 型、1-2 型轿运车装载 II、III 型乘用车时的所有可能装载方式,建立整数规划模型,利用 lingo 软件求出最优解,即需要 12 辆 1-1 型和 1 辆 1-2 型轿运车,并给出具体装载方案。

对于问题三,乘用车有 I、II、III 型,分别分析 1-1 型、1-2 型轿运车装载各类型乘用车的可行装载方案,建立整数规划模型,利用 lingo 软件求出最优解,可利用 25 辆 1-1 型和 5 辆 1-2 型轿运车完成运输任务,并给出具体装载方案。

对于问题四,考虑运输路线为有向路径,不允许折回。首先将问题简化为单目的地的运输,归结为问题一来求解,然后在尽量不增加轿运车的前提下考虑多个目的地的运输,建立整数规划模型,求出最优装运方案,需要 21 辆 1-1 型和 4 辆 1-2 型轿运车即可完成装载和运输方案,并列出了具体装运方案。

对于第五问,首先根据轿运车对乘用车高度和宽度的装载限制,将 45 类乘用车分成三个部分,然后针对每个部分具体问题具体分析,依据轿运车同类型长优先,装载时长短结合的原则,利用启发式算法,建立三个子规划模型,并利用 lingo 软件编程求得好的可行解,即需要 19 辆 1-2 系列轿运车,98 辆 1-1 系列轿运车即可完成运输任务,且用 EXCEL 表输出具体装载方案。

\textbf{关键词:} 配载优化;整数规划;线性规划;最优解;启发式算法

\begin{center}
- 1 -
\end{center}

\section{问题背景}

近年来,我国汽车市场需求旺盛,从 2003 年到 2006 年,连续四年保持了 20\% 以上的增长速度。2009 年,我国汽车产销分别完成了 1379 万辆和 1364 万辆,以 300 多万辆的优势,首次超越美国,成为世界汽车产销第一大国。2010 年,我国车市在购置税优惠、汽车下乡、以旧换新、节能惠民产品补贴等多种鼓励消费政策叠加效应的作用下,汽车产销量分别为 1826.47 万辆和 1806.19 万辆,同比分别增长 32.44\% 和 32.37\%[1],产销量再创新高。2011 年,我国汽车市场实现了平稳增长,累计生产汽车 1841.89 万辆,同比增长 0.8\%,销售汽车 1850.51 万辆,同比增长 2.5\%[2]。汽车产销量双超 1840 万辆,再次刷新全球历史纪录。

随着社会主义市场经济的迅速发展,人们对于乘用车的需求与日俱增。乘用车运输物流公司的竞争也逐渐升级。为了适应新时期的竞争环境,物流公司需要采用合理的运输方式、有效降低物流成本,从而获取更多的市场占有率和经济效益。

我国汽车产业的蓬勃发展为整车物流业带来了前所未有的发展机遇。作为汽车销售的重要环节,整车物流业的发展速度必须跟上整个行业的发展步伐。在国内,许多物流企业已经开展了汽车整车物流业务,尤其是依托公路运输的物流企业已经占据了大部分市场,并积累了一定的经验,如安吉天地汽车物流有限公司、吉林长久实业集团等。此外,荷兰 TNT、挪威华伦威尔森等许多世界顶级的汽车物流巨头也纷纷而至,准备分割中国汽车物流这块诱人的“蛋糕”[3]。这些物流巨头的到来给我国物流企业尤其是公路运输企业带来一定的挑战和压力。

事实上,进行汽车整车物流配载时,公路汽车运输工具的利用程度仍有较大的提升空间。整车物流配载涉及知识面广,技术要求高,实施难度大,在我国的发展尚处于起步阶段,普遍存在着缺少科学方法指导,劳动工作量繁重的问题。目前,我国公路汽车整车物流配载主要依靠人工完成,致使配载结果差错率高,运输工具运力浪费,装卸过程常有质损事故发生等,进而造成了信息传输变慢,管理效率降低,服务质量得不到保障等诸多弊端。因此,对公路汽车整车物流配载问题的研究就显得尤为重要。

本文将针对物流管理领域中轿运车的选择和乘用车的装载问题进行探究。在满足客户订单要求和乘用车装载要求的前提条件下,以运输成本最低为目标,针对题目中的五个乘用车装载问题,安排最佳物流运输计划。

\section{问题重述}

整车物流是指按照客户订单对整车快速配送的全过程。随着汽车工业的高速发展,整车物流量,特别是乘用车的整车物流量迅速增长。

乘用车生产厂家根据全国客户的购车订单,向物流公司下达运输乘用车到全国各地的任务,物流公司则根据下达的任务制定运输计划并配送这批乘用车。物流公司负责派出轿运车,并指定每一辆轿运车上乘用车的装载方案和目的地,在确保运输任务顺利完成的情况下追求运输成本尽可能低。

转载要求:每种轿运车上、下层装载区均可等价看成长方形,1-1 型车上、下各 1 列乘用车;1-2 型车上层 2 列乘用车,下层 1 列乘用车;2-2 型车上、下各 2 列乘用车。各列乘车均纵向摆放,相邻乘用车之间的纵向及横向的安全车距均至少为 0.1 米,下层力争装满,上层两列力求对称,以保证轿运车行驶平稳,高于 1.7 米的车只能装在 1-1、

\begin{table}[h]
\centering
\caption{乘用车规格}
\begin{tabular}{|c|c|c|c|}
\hline
乘用车型号 & 长度(米) & 宽度(米) & 高度(米) \\ \hline
I & 4.61 & 1.7 & 1.51 \\ \hline
II & 3.615 & 1.605 & 1.394 \\ \hline
III & 4.63 & 1.785 & 1.77 \\ \hline
\end{tabular}
\end{table}

\begin{table}[h]
\centering
\caption{轿运车规格}
\begin{tabular}{|c|c|c|c|}
\hline
轿运车类型 & 上下层长度(米) & 上层宽度(米) & 下层宽度(米) \\ \hline
1-1 & 19 & 2.7 & 2.7 \\ \hline
1-2 & 24.3 & 3.5 & 2.7 \\ \hline
2-2 & 19 & 3.5 & 3.4 \\ \hline
\end{tabular}
\end{table}

成本高低直接影响轿运车的使用数量。在轿运车使用数量相同的情况下,1-1型轿运车的使用成本较低,2-2型较高,1-2型略低于前两者的平均值。由于受实际情况的限制每次1-2型轿运车使用量不超过1-1型轿运车使用量的20%,再次,在轿运车使用数量及型号均相同的情况下,行驶里程短的成本低,注意因为该物流公司是全国性公司,在各地均会有整车物流业务,所以轿运车到达目的地后原地待命,无需放空返回,且每次卸车成本几乎可以忽略不计。

在满足以上要求的前提下为物流公司安排以下五次运输,并制定出详细的计划,含所需要各类型轿运车的数量、每辆轿运车的乘用车装载方案、行车路线。

1、物流公司要运输I型乘用车100辆及II型乘用车68辆。

2、物流公司要运输II型乘用车72辆及III型乘用车52辆。

3、物流公司要运输I型乘用车156辆、II型乘用车102辆及III型乘用车39辆。

4、物流公司要运输166辆I型乘用车(其中目的地是A、B、C、D的分别为42、50、33、41辆)和78辆II型乘用车(其中目的地是A、C,分别为31、47辆),具体路线如下图1,各段长度:OD=160,DC=76,DA=200,DB=120,BE=104,AE=60。

\begin{figure}[h]
\centering
\includegraphics[width=0.5\textwidth]{image.png}
\caption{行车路线}
\end{figure}

5、根据附表1给出的物流公司需要运输的乘用车类型(含序号)、尺寸大小、数量和目的地,附表2给出的可调用的轿车类型(含序号)、数量和装载区域大小(表里数

\section*{三. 问题分析}

对于问题一,物流公司运输的乘用车有 I、II 两种类型,我们可以根据表一中乘用车规格和表二中轿运车规格,分析 1-1 型、1-2 型轿运车装载 I、II 型乘用车时,分别有多少种装载方式,以及各种装载方法的可行性,然后建立整数规划模型,编写 lingo 程序,找出最优解。

对于问题二,分析 1-1 型、1-2 型轿运车装载 II、III 型乘用车时的所有可能装载方式,建立整数规划模型,利用 lingo 软件求出最优解。

对于问题三,乘用车有 I、II、III 型,分别分析 1-1 型、1-2 型轿运车装载 I、II、III 型乘用车的可行装载方案,建立整数规划模型,利用 lingo 软件求出最优解。

对于问题四,考虑运输路线为有向路径,不允许折回,因而仅需考虑乘用车的装载问题。首先将问题四简化为单目的地的运输,归结为问题一来求解,然后在尽量不增加轿运车的前提下考虑多个目的地的运输,建立整数规划模型,求出最佳装运方案。

对于问题五,首先根据轿运车对乘用车高度和宽度的装载限制,将 45 类乘用车分成三个部分,然后针对每个部分具体问题具体分析,利用启发式算法,建立数学模型,在一定条件下寻求最佳装运方案。

\section*{四. 模型的建立和求解}

\subsection{4.1 问题一分析与建模}

\subsubsection{4.1.1 问题分析}

对于问题一,由于考虑到各类轿运车成本(1-1 型轿运车的使用成本较低,2-2 型较高,1-2 型略低于前两者的平均值)和运载能力,我们只用 1-1 型轿运车和 1-2 型轿运车来运载乘用车。

对于装载要求:每种轿运车上、下层装载区均可等价看成长方形,各列乘用车均纵向摆放,相邻乘用车之间的纵向及横向的安全车距均至少为 0.1 米,我们采取把每辆乘用车的长度都加 0.1 米来满足纵向间隔的要求。但我们知道每层的第一辆车不用加 0.1 米,如果此层装载车辆的总长度刚好比运载轿运车此层的长度长 0.1 米以内,我们则考虑把第一辆车前的 0.1 米给去掉,以满足尽可能装满的要求。对于横向的装载要求,由于 I、II 型乘用车在 1-2 型轿运车上层任意纵向摆放,横向间隔均不低于 0.1 米,故我们不用考虑横向的安全车距。

由于 I、II 型乘用车的高度均小于 1.7 米,所以此问不用考虑高于 1.7 米的车只能装在 1-1、1-2 下层的要求。

\subsubsection{4.1.2 可行性装配方案}

由表一和二知 I 型乘用车长度为 4.61m,II 型乘用车长度为 3.615m,1-1 型轿运车长度为 19m,1-2 型轿运车长度为 24.3m,可得 1-1 型轿运车装载 I、II 型乘用车的所有可能方案如表三所示,1-2 型轿运车有 16 种装载 I、II 型乘用车的方案,如表四所示。

\begin{table}
\centering
\caption{1-1型轿运车装载Ⅰ、Ⅱ型乘用车方案}
\begin{tabular}{|c|c|c|c|c|c|c|}
\hline
载车方案序号 & \multicolumn{2}{c|}{总载车辆数} & \multicolumn{2}{c|}{上层摆放车辆数} & \multicolumn{2}{c|}{下层摆放车辆数} \\
\cline{2-7}
 & Ⅰ & Ⅱ & Ⅰ & Ⅱ & Ⅰ & Ⅱ \\
\hline
1 & 8 & 0 & 4 & 0 & 4 & 0 \\
\hline
2 & 7 & 1 & 4 & 0 & 3 & 1 \\
\hline
3 & 6 & 2 & 4 & 0 & 2 & 2 \\
\hline
 & & & 3 & 1 & 3 & 1 \\
\hline
4 & 5 & 3 & 4 & 0 & 1 & 3 \\
\hline
 & & & 3 & 1 & 2 & 2 \\
\hline
5 & 4 & 5 & 4 & 0 & 0 & 5 \\
\hline
6 & 3 & 6 & 3 & 1 & 0 & 5 \\
\hline
7 & 2 & 7 & 2 & 2 & 0 & 5 \\
\hline
8 & 1 & 8 & 1 & 3 & 0 & 5 \\
\hline
9 & 0 & 10 & 0 & 5 & 0 & 5 \\
\hline
\end{tabular}
\end{table}

\begin{table}
\centering
\caption{1-2型轿运车装载Ⅰ、Ⅱ型乘用车方案}
\begin{tabular}{|c|c|c|c|c|c|c|}
\hline
载车方案序号 & \multicolumn{2}{c|}{总载车辆数} & \multicolumn{2}{c|}{上层摆放车辆数} & \multicolumn{2}{c|}{下层摆放车辆数} \\
\cline{2-7}
 & Ⅰ & Ⅱ & Ⅰ & Ⅱ & Ⅰ & Ⅱ \\
\hline
1 & 15 & 0 & 10 & 0 & 5 & 0 \\
\hline
2 & 14 & 1 & 10 & 0 & 4 & 1 \\
\hline
3 & 13 & 2 & 10 & 0 & 3 & 2 \\
\hline
 & & & 8 & 2 & 5 & 0 \\
\hline
4 & 12 & 4 & 10 & 0 & 2 & 4 \\
\hline
5 & 11 & 5 & 10 & 0 & 1 & 5 \\
\hline
6 & 10 & 6 & 10 & 0 & 0 & 6 \\
\hline
 & & & 8 & 2 & 2 & 4 \\
\hline
7 & 9 & 8 & 4 & 8 & 5 & 0 \\
\hline
8 & 8 & 9 & 4 & 8 & 4 & 1 \\
\hline
9 & 7 & 10 & 4 & 8 & 3 & 2 \\
\hline
 & & & 2 & 10 & 5 & 0 \\
\hline
10 & 6 & 12 & 4 & 8 & 2 & 4 \\
\hline
11 & 5 & 13 & 4 & 8 & 1 & 5 \\
\hline
12 & 4 & 14 & 4 & 8 & 0 & 6 \\
\hline
 & & & 2 & 10 & 2 & 4 \\
\hline
13 & 3 & 15 & 2 & 10 & 1 & 5 \\
\hline
14 & 2 & 16 & 2 & 10 & 0 & 6 \\
\hline
 & & & 0 & 12 & 2 & 4 \\
\hline
15 & 1 & 17 & 0 & 12 & 1 & 15 \\
\hline
16 & 0 & 18 & 0 & 12 & 0 & 6 \\
\hline
\end{tabular}
\end{table}

4.1.3 模型的建立与求解

设 $C_{1}$ 表示单辆 1-1 型轿运车的运费,$C_{2}$ 表示单辆 1-2 型轿运车的运费,$C_{1}$ 和 $C_{2}$ 均为固定常数,且 $C_{1} < C_{2}$。根据实际情况,不妨假设 $C_{2} < 2C_{1}$。

(1) 决策变量

设 $x_{i}$ 表示按照第 $i$ 个方案装载的 1-1 型轿运车的数目,$y_{j}$ 表示按照第 $j$ 个方案装载的 1-2 型轿运车的数目,其中 $i = 1, \cdots, 9$;$j = 1, \cdots, 16$。

(2) 目标函数

由物流公司要求运输成本最低,建立目标函数:
\[
\min \ Z = C_{1} \sum_{i=1}^{9} x_{i} + C_{2} \sum_{j=1}^{16} y_{j}
\]

(3) 约束条件

由物流公司要运输 I 型乘用车 100 辆建立约束条件:
\[
\sum_{i=1}^{9} (9 - i) x_{i} + \sum_{j=1}^{16} (16 - j) y_{j} \geq 100
\]

由物流公司要运输 II 型乘用车 68 辆,建立约束条件:
\[
x_{2} + 2x_{3} + 3x_{4} + 5x_{5} + 6x_{6} + 7x_{7} + 8x_{8} + 10x_{9} + y_{2} + 2y_{3} + 4y_{4} + 5y_{5} + 6y_{6}
\]
\[
+ 7y_{7} + 8y_{8} + 10y_{9} + 12y_{10} + 13y_{11} + 14y_{12} + 15y_{13} + 16y_{14} + 17y_{15} + 18y_{16} \geq 68
\]

由于受实际情况的限制,每次 1-2 型轿运车使用量不超过 1-1 型轿运车使用量的 20%,建立约束条件:
\[
5 \sum_{j=1}^{16} y_{j} \leq \sum_{i=1}^{9} x_{i}
\]

考虑到车辆运输实际情况,要求 $x_{i}, y_{j}$ 必须均为非负整数,建立约束条件:
\[
x_{i}, y_{j} \text{ 均为非负整数,} \quad i = 1, \cdots, 9; j = 1, \cdots, 16
\]

(4) 模型建立

根据以上分析,可以建立以下数学模型:
\[
\min \ Z = C_{1} \sum_{i=1}^{9} x_{i} + C_{2} \sum_{j=1}^{16} y_{j}
\]
\[
\text{s.t. } \sum_{i=1}^{9} (9 - i) x_{i} + \sum_{j=1}^{16} (16 - j) y_{j} \geq 100
\]
\[
x_{2} + 2x_{3} + 3x_{4} + 5x_{5} + 6x_{6} + 7x_{7} + 8x_{8} + 10x_{9} + y_{2} + 2y_{3} + 4y_{4} + 5y_{5} + 6y_{6}
\]
\[
+ 7y_{7} + 8y_{8} + 10y_{9} + 12y_{10} + 13y_{11} + 14y_{12} + 15y_{13} + 16y_{14} + 17y_{15} + 18y_{16} \geq 68
\]
\[
5 \sum_{j=1}^{16} y_{j} \leq \sum_{i=1}^{9} x_{i}
\]
\[
x_{i}, y_{j} \text{ 均为非负整数,} \quad i = 1, \cdots, 9; j = 1, \cdots, 16
\]

\begin{align*}
+7y_7 + 8y_8 + 10y_9 + 12y_{10} + 13y_{11} + 14y_{12} + 15y_{13} + 16y_{14} + 17y_{15} + 18y_{16} &\geq 68 \\
\sum_{j=1}^{16} y_j &\leq \frac{1}{5} \sum_{i=1}^{9} x_i
\end{align*}

\(x_i, y_j\) 均为非负整数,\(i=1, \dots, 9; j=1, \dots, 16\)

(5) 结果分析

在 lingo 程序运行中,保证 \(0 < C_1 < C_2 < 2C_1\) 的情况下,可以随机设定 \(C_1\)、\(C_2\),以下是 lingo 运行的 4 种结果:

① \(C_1 = 1\),\(C_2 = 1.1\)、1.2、1.3、1.5、1.7 或 1.8,解得 \(x_1 = 11, x_9 = 5, y_{10} = 2\),其它值均为零,其装载方式如表五所示:

\begin{table}[h]
\centering
\caption{1-1 型和 1-2 型轿运车载车方案}
\begin{tabular}{|c|c|c|c|c|c|c|c|}
\hline
\multirow{2}{*}{载车方案序号} & \multicolumn{2}{c|}{总载车辆数} & \multicolumn{2}{c|}{上层摆放车辆数} & \multicolumn{2}{c|}{下层摆放车辆数} & \multirow{2}{*}{需求轿运车车辆数} \\
\cline{2-7}
 & I & II & I & II & I & II & \\
\hline
1-1 型车装载方案 1 & 8 & 0 & 4 & 0 & 4 & 0 & 11 \\
\hline
1-1 型车装载方案 9 & 0 & 10 & 0 & 5 & 0 & 5 & 5 \\
\hline
1-2 型车装载方案 10 & 6 & 12 & 4 & 8 & 2 & 4 & 2 \\
\hline
\end{tabular}
\end{table}

② \(C_1 = 1\),\(C_2 = 1.4\),解得 \(x_1 = 10, x_8 = 2, x_9 = 4, y_4 = 1, y_9 = 1\),其它值均为零,其装载方式如表六所示:

\begin{table}[h]
\centering
\caption{1-1 型和 1-2 型轿运车载车方案}
\begin{tabular}{|c|c|c|c|c|c|c|c|}
\hline
\multirow{2}{*}{载车方案序号} & \multicolumn{2}{c|}{总载车辆数} & \multicolumn{2}{c|}{上层摆放车辆数} & \multicolumn{2}{c|}{下层摆放车辆数} & \multirow{2}{*}{需求轿运车车辆数} \\
\cline{2-7}
 & I & II & I & II & I & II & \\
\hline
1-1 型车装载方案 1 & 8 & 0 & 4 & 0 & 4 & 0 & 10 \\
\hline
1-1 型车装载方案 8 & 1 & 8 & 1 & 3 & 0 & 5 & 2 \\
\hline
1-1 型车装载方案 9 & 0 & 10 & 0 & 5 & 0 & 5 & 4 \\
\hline
1-2 型车装载方案 4 & 12 & 4 & 10 & 0 & 2 & 4 & 1 \\
\hline
1-2 型车装载方案 9 & 7 & 10 & 4 & 8 & 3 & 2 & 1 \\
\cline{2-7}
 & 2 & 10 & 10 & 0 & 5 & 0 & \\
\hline
\end{tabular}
\end{table}

③ \(C_1 = 1\),\(C_2 = 1.6\),解得 \(x_1 = 11, x_7 = 1, x_9 = 4, y_6 = 1, y_{16} = 1\),其它值均为零,其装载方式如表七所示:

\begin{table}
\centering
\caption{1-1型和1-2型轿运车载车方案}
\begin{tabular}{|c|c|c|c|c|c|c|c|}
\hline
\multirow{2}{*}{载车方案序号} & \multicolumn{2}{c|}{总载车辆数} & \multicolumn{2}{c|}{上层摆放车辆数} & \multicolumn{2}{c|}{下层摆放车辆数} & \multirow{2}{*}{需求轿运车车辆数} \\
\cline{2-7}
 & I & II & I & II & I & II & \\
\hline
1-1型车装载方案1 & 8 & 0 & 4 & 0 & 4 & 0 & 11 \\
\hline
1-1型车装载方案7 & 2 & 7 & 2 & 2 & 0 & 5 & 1 \\
\hline
1-1型车装载方案9 & 0 & 10 & 0 & 5 & 0 & 5 & 4 \\
\hline
1-2型车装载方案6 & \multirow{2}{*}{10} & \multirow{2}{*}{6} & 10 & 0 & 0 & 6 & 2 \\
\cline{4-8}
 & & & 8 & 2 & 2 & 4 & 1 \\
\hline
1-2型车装载方案16 & 0 & 18 & 0 & 12 & 0 & 6 & 1 \\
\hline
\end{tabular}
\end{table}

(4) $C_{1}=1$, $C_{2}=1.9$, 解得 $x_{1}=9$, $x_{2}=1$, $x_{5}=5$, $x_{9}=1$, $y_{3}=1$, $y_{10}=1$, 其它值均为零, 其装载方式如表八所示:

\begin{table}
\centering
\caption{1-1型和1-2型轿运车的载车方案}
\begin{tabular}{|c|c|c|c|c|c|c|c|}
\hline
\multirow{2}{*}{载车方案序号} & \multicolumn{2}{c|}{总载车辆数} & \multicolumn{2}{c|}{上层摆放车辆数} & \multicolumn{2}{c|}{下层摆放车辆数} & \multirow{2}{*}{需求轿运车车辆数} \\
\cline{2-7}
 & I & II & I & II & I & II & \\
\hline
1-1型车装载方案1 & 8 & 0 & 4 & 0 & 4 & 0 & 9 \\
\hline
1-1型车装载方案2 & 7 & 1 & 4 & 0 & 3 & 1 & 1 \\
\hline
1-1型车装载方案5 & 4 & 5 & 4 & 0 & 0 & 5 & 1 \\
\hline
1-1型车装载方案9 & 0 & 10 & 0 & 5 & 0 & 5 & 5 \\
\hline
1-2型车装载方案3 & \multirow{2}{*}{13} & \multirow{2}{*}{2} & 10 & 0 & 3 & 2 & \multirow{2}{*}{1} \\
\cline{4-8}
 & & & 8 & 2 & 5 & 0 & \\
\hline
1-2型车装载方案10 & 6 & 12 & 4 & 8 & 2 & 4 & 1 \\
\hline
\end{tabular}
\end{table}

从以上四种结果分析, 四种结果均使用 16 辆 1-1 型的轿用车, 使用 2 辆 1-2 型轿运车。由于以上四种情况不仅使用的各类型的轿运车车辆数目是一样的, 而且运输的始发地和目的地也是一样的, 那么它们运输 I 型乘用车 100 辆及 II 型乘用车 68 辆所花费的运输成本都是相同的。

考虑到第一、四种空出的空间剩余位置比较多, 那么乘用车在某些轿运车中分配会更加密集, 不利于它们途中的运输。因为在空间资源一定的情况下, 为达到空间资源的最大化利用, 我们选择第二、三种情况进行装配。比如第三种装载方式, 只需随机选择 3 辆按照方案 1 装载的 1-1 型轿运车, 在这 3 辆轿运车的上层各少装一辆。

\subsection*{4.2 问题二分析与建模}

\subsubsection{4.2.1 问题分析}

对于问题二, 在问题一的基础上, 根据表一 III 型乘用车的高度为 1.785m, 大于 1.7m, 它会受层高的限制, 只能装在 1-1、1-2 型下层。

\subsubsection{4.2.2 可行性装配方案}

根据表一和表二, II 型乘用车长度为 3.615m, III 型乘用车长度为 4.63m, 1-1 型轿运车长度为 19m, 1-2 型轿运车长度为 24.3m, 可得 1-1 型轿运车装载 II、III 型车的方

\begin{table}
\centering
\caption{1-1型轿运车装载Ⅱ、Ⅲ型车的方案}
\begin{tabular}{|c|c|c|c|c|c|c|}
\hline
载车方案序号 & \multicolumn{2}{|c|}{总载车辆数} & \multicolumn{2}{|c|}{上层摆放车辆数} & \multicolumn{2}{|c|}{下层摆放车辆数} \\
\cline{2-7}
 & Ⅱ & Ⅲ & Ⅱ & Ⅲ & Ⅱ & Ⅲ \\
\hline
1 & 10 & 0 & 5 & 0 & 5 & 0 \\
\hline
2 & 8 & 1 & 5 & 0 & 3 & 1 \\
\hline
3 & 7 & 2 & 5 & 0 & 2 & 2 \\
\hline
4 & 6 & 3 & 5 & 0 & 1 & 3 \\
\hline
5 & 5 & 4 & 5 & 0 & 0 & 4 \\
\hline
\end{tabular}
\end{table}

\begin{table}
\centering
\caption{1-2型轿运车装载Ⅱ、Ⅲ型车的方案}
\begin{tabular}{|c|c|c|c|c|c|c|}
\hline
载车方案 & \multicolumn{2}{|c|}{总载车辆数} & \multicolumn{2}{|c|}{上层摆放车辆数} & \multicolumn{2}{|c|}{下层摆放车辆数} \\
\cline{2-7}
序号 & Ⅱ & Ⅲ & Ⅱ & Ⅲ & Ⅱ & Ⅲ \\
\hline
1 & 18 & 0 & 12 & 0 & 6 & 0 \\
\hline
2 & 17 & 1 & 12 & 0 & 5 & 1 \\
\hline
3 & 16 & 2 & 12 & 0 & 4 & 2 \\
\hline
4 & 14 & 3 & 12 & 0 & 2 & 3 \\
\hline
5 & 13 & 4 & 12 & 0 & 1 & 4 \\
\hline
6 & 12 & 5 & 12 & 0 & 0 & 5 \\
\hline
\end{tabular}
\end{table}

\subsection*{4.2.3 模型的建立与求解}

设 $C_1$ 表示单辆 1-1 型轿运车的运费,$C_2$ 表示单辆 1-2 型轿运车的运费,$C_1$ 和 $C_2$ 均为固定常数,且 $C_1 < C_2$。根据实际情况,不妨假设 $C_2 < 2C_1$。

\subsubsection{(1)决策变量}

设 $x_i$ 表示按照第 $i$ 个方案装载的 1-1 型轿运车的数目,$y_j$ 表示按照第 $j$ 个方案装载的 1-2 型轿运车的数目,其中 $i = 1, \cdots, 5$;$j = 1, \cdots, 6$。

\subsubsection{(2)目标函数}

由物流公司要求运输成本最低,建立目标函数:
\[
\min Z = C_1 \sum_{i=1}^5 x_i + C_2 \sum_{j=1}^6 y_j
\]

\subsubsection{(3)约束条件}

由物流公司要运输Ⅱ型的乘用车 72 辆,建立约束条件:
\[
10x_1 + 8x_2 + 7x_3 + 6x_4 + 5x_5 + 18y_1 + 17y_2 + 16y_3 + 14y_4 + 13y_5 + 12y_6 \geq 72
\]

由物流公司要运输Ⅲ型的乘用车 52 辆,建立约束条件:

\begin{align*}
x_2 + 2x_3 + 3x_4 + 4x_5 + y_2 + 2y_3 + 3y_4 + 4y_5 + 5y_6 &\geq 52
\end{align*}

由于受实际情况的限制,每次 1-2 型轿运车使用量不超过 1-1 型轿运车使用量的 20%,建立约束条件:
\begin{equation}
5\sum_{j=1}^{6} y_j \leq \sum_{i=1}^{5} x_i
\end{equation}

考虑到车辆运输实际情况,要求 $x_i, y_j$ 必须均为非负整数,建立约束条件:
\begin{equation}
x_i, y_j \text{ 均为非负整数,且 } i=1,\dots,5; j=1,\dots,6
\end{equation}

(4) 模型建立

根据以上分析,可以建立以下数学模型:
\begin{align*}
\min \; Z &= C_1 \sum_{i=1}^{5} x_i + C_2 \sum_{j=1}^{6} y_j \\
\text{s.t. } 10x_1 + 8x_2 + 7x_3 + 6x_4 + 5x_5 + 18y_1 + 17y_2 + 16y_3 + 14y_4 + 13y_5 + 12y_6 &\geq 72 \\
x_2 + 2x_3 + 3x_4 + 4x_5 + y_2 + 2y_3 + 3y_4 + 4y_5 + 5y_6 &\geq 52 \\
5\sum_{j=1}^{6} y_j &\leq \sum_{i=1}^{5} x_i
\end{align*}
\begin{equation}
x_i, y_j \text{ 均为非负整数,且 } i=1,\dots,5; j=1,\dots,6
\end{equation}

(5) 结果分析

在 lingo 程序运行中,保证 $0 < C_1 < C_2 < 2C_1$ 的情况下,可以随机设定 $C_1, C_2$,以下是 lingo 运行的 2 种结果:

① $C_1 = 1, C_2 = 1.1$ 或 1.2,解得 $x_4 = 1, x_5 = 11, y_6 = 1$,其它均为零,其装载方式如表十一所示:

\begin{table}[h]
\centering
\caption{1-1 型和 1-2 型轿运车载车方案}
\begin{tabular}{|c|c|c|c|c|c|c|c|}
\hline
\multirow{2}{*}{载车方案序号} & \multicolumn{2}{c|}{总载车辆数} & \multicolumn{2}{c|}{上层摆放车辆数} & \multicolumn{2}{c|}{下层摆放车辆数} & \multirow{2}{*}{需求轿运车车辆数} \\
\cline{2-7}
 & II & III & II & III & II & III & \\
\hline
1-1 型车装载方案 4 & 6 & 3 & 5 & 0 & 1 & 3 & 1 \\
\hline
1-1 型车装载方案 5 & 5 & 4 & 5 & 0 & 0 & 4 & 11 \\
\hline
1-2 型车装载方案 6 & 12 & 5 & 12 & 0 & 0 & 5 & 1 \\
\hline
\end{tabular}
\end{table}

\begin{table}[h]
\centering
\caption{1-1型和1-2型轿运车载车方案}
\begin{tabular}{|c|c|c|c|c|c|c|c|}
\hline
\multirow{2}{*}{载车方案序号} & \multicolumn{2}{c|}{总载车辆数} & \multicolumn{2}{c|}{上层摆放车辆数} & \multicolumn{2}{c|}{下层摆放车辆数} & \multirow{2}{*}{需求轿运车车辆数} \\
\cline{2-7}
 & II & III & II & III & II & III & \\
\hline
1-1型车装载方案5 & 5 & 4 & 5 & 0 & 0 & 4 & 12 \\
\hline
1-2型车装载方案6 & 12 & 5 & 12 & 0 & 0 & 5 & 1 \\
\hline
\end{tabular}
\end{table}

以上两种结果均使用12辆1-1型的轿用车,使用1辆1-2型轿运车。由于以上两种情况不仅使用的各类型的轿运车车辆数目都是一样的,而且运输的始发地和目的地也是一样的,即它们所花费的运输成本都是相同的。比如按照第一种情况的结果运载,只需在某辆按照方案5装载的1-1型轿运车上层少装一辆II型乘用车即可。

\subsection*{4.3 问题三分析与建模}

问题一需要运输I、II型乘用车,问题二需要运输II、III型轿运车,不难看出问题一、二实际上就是问题三的特殊情况。由于2-2型轿运车装载量是最大的,但其运输成本较高,我们决定在本次计划中仍然不考虑2-2型车的装载情况。

1-2型轿运车的装载量几乎是1-1型轿运车的3/2倍,不难看出1-2型轿运车比1-1型更具备经济价值。但问题的背景中强调物流公司1-2型轿运车拥有量小,为方便后续任务安排,每次1-2型轿运车使用量不超过1-1型轿运车使用量的20\%,所以我们目前能做的就是在满足要求的前提下最大化的使用1-2型轿运车进行装载。

在考虑1-2型轿运车上层严格对称装载的要求后,我们推算了1-1型轿运车共有35种装载方案,1-2型轿运车共有73种装载方案(见附录1)。

\subsubsection{4.3.1 模型的建立与求解}

设$C_{1}$表示单辆1-1型轿运车的运费,$C_{2}$表示单辆1-2型轿运车的运费,$C_{1}$和$C_{2}$均为固定常数,且$C_{1}<C_{2}$。根据实际情况,不妨假设$C_{2}<2C_{1}$。

\subsubsection*{(1) 决策变量}

设$x_{i}$表示按照第$i$个方案装载的1-1型轿运车的数目,$y_{j}$表示按照第$j$个方案装载的1-2型轿运车的数目,其中$i=1,\cdots,35$;$j=1,\cdots,73$。

\subsubsection*{(2) 目标函数的建立}

由物流公司要求运输成本最低,建立目标函数:
\[
\min Z = C_{1}\sum_{i=1}^{35}x_{i} + C_{2}\sum_{j=1}^{73}y_{j}
\]

\subsubsection*{(3) 问题约束条件的建立}

由物流公司要运输156辆I型乘用车,建立约束条件:

\begin{align*}
&8x_1 + 7x_2 + 6x_3 + 5x_4 + 4x_5 + 3x_6 + 2x_7 + 8x_1 + 7x_{10} + 6x_{11} + 5x_{12} + 4x_{13} + 3x_{14} \\
&+ 2x_{15} + x_{16} + 6x_{18} + 5x_{19} + 4x_{20} + 3x_{21} + x_{23} + 5x_{25} + 4x_{26} + 3x_{27} + 2x_{28} + x_{29} \\
&+ 4x_{31} + 3x_{32} + 2x_{33} + x_{34} + 14y_1 + 13y_2 + 12y_3 + 11y_4 + 10y_5 + 9y_6 + 8y_7 + 7y_8 + \\
&6y_9 + 5y_{10} + 4y_{11} + 3y_{12} + y_{14} + 13y_{16} + 12y_{17} + 11y_{18} + 10y_{19} + 8y_{20} + 7y_{21} + 6y_{22} \\
&+ 5y_{23} + 4y_{24} + 2y_{25} + 12y_{27} + 11y_{28} + 10y_{29} + 9y_{30} + 8y_{31} + 7y_{32} + 6y_{33} + 5y_{34} + 4y_{35} + 3y_{36} + 2y_{37} \\
&+ y_{38} + 11y_{40} + 10y_{41} + 9y_{42} + 8y_{43} + 7y_{44} + 15y_{58} + 14y_{59} + 13y_{60} \\
&+ 12y_{61} + 11y_{62} + 10y_{63} + 9y_{64} + 8y_{65} + 7y_{66} + 6y_{67} + 5y_{68} + 4y_{69} \\
&+ 3y_{70} + 2y_{71} + y_{72} \geq 156
\end{align*}

由物流公司要运输 102 辆Ⅱ型乘用车,建立约束条件:
\begin{align*}
&x_2 + 2x_3 + 3x_4 + 5x_5 + 6x_6 + 7x_7 + 8x_8 + 10x_9 + x_{11} + 2x_{12} + 3x_{13} + 5x_{14} + 6x_{15} + \\
&7x_{16} + 8x_{17} + x_{19} + 2x_{20} + 3x_{21} + 5x_{22} + 6x_{23} + 7x_{24} + x_{26} + 2x_{27} + 3x_{28} + 5x_{29} \\
&+ y_2 + 2y_3 + 4y_4 + 5y_5 + 6y_6 + 8y_7 + 9y_8 + 10y_9 + 12y_{10} + 11y_{13} + 14y_{12} + 15y_{13} \\
&+ 16y_{14} + 17y_{15} + y_{17} + 2y_{18} + 4y_{19} + 6y_{20} + 8y_{21} + 9y_{22} + 10y_{23} + 12y_{24} + 14y_{25} \\
&+ y_{28} + 2y_{29} + 3y_{30} + 4y_{31} + 5y_{32} + 8y_{33} + 9y_{34} + 10y_{35} + 11y_{36} + 12y_{37} + 13y_{38} \\
&+ 14y_{39} + y_{41} + 2y_{42} + 3y_{43} + 4y_{44} + 5y_{45} + 8y_{46} + 9y_{47} + 10y_{48} + 11y_{49} + 12y_{50} \\
&+ 13y_{51} + 2y_{53} + 4y_{54} + 8y_{55} + 10y_{65} + 12y_{57} + y_{59} + 2y_{60} + 4y_{61} + 5y_{62} + 6y_{63} + 8y_{64} \\
&+ 9y_{65} + 10y_{66} + 12y_{67} + 13y_{68} + 14y_{69} + 15y_{70} + 16y_{71} + 18y_{73} \geq 102
\end{align*}

由物流公司要运输 39 辆Ⅲ型轿运车,建立约束条件:
\begin{align*}
&x_{11} + x_{12} + x_{13} + x_{14} + x_{15} + x_{16} + x_{17} + 2(x_{18} + x_{19} + x_{20} + x_{21} + x_{22} + x_{23} + x_{24}) + \\
&3(x_{25} + x_{26} + x_{27} + x_{28} + x_{29} + x_{30}) + 4(x_{31} + x_{32} + x_{33} + x_{34} + x_{35}) + (y_1 + y_2 + y_3 \\
&+ y_4 + y_5 + y_6 + y_7 + y_8 + y_9 + y_{10} + y_{11} + y_{12} + y_{13} + y_{14} + y_{15}) + 2(y_{16} + y_{17} + y_{18} \\
&+ y_{19} + y_{20} + y_{21} + y_{22} + y_{23} + y_{24} + y_{25} + y_{26}) + 3(y_{27} + y_{28} + y_{29} + y_{30} + y_{31} + y_{32}
\end{align*}

\begin{align*}
&+y_{33} + y_{34} + y_{35} + y_{36} + y_{37} + y_{38} + y_{39}) + 4(y_{40} + y_{41} + y_{42} + y_{43} + y_{44} + y_{45} + y_{46} \\
&+ y_{47} + y_{48} + y_{49} + y_{50} + y_{51}) + 5(y_{52} + y_{53} + y_{54} + y_{55} + y_{56} + y_{57}) \geq 39
\end{align*}

由于受实际情况限制,每次 1-2 型轿运车使用量不超过 1-1 型轿运车的 20%,建立约束条件:

\[
5\sum_{j=1}^{73} y_j \leq \sum_{i=1}^{35} x_i
\]

考虑的车辆运输的实际情况,要求 \(x_i, \, y_j\) 必须为非负整数:

\[
x_i, y_j \text{ 均为非负整数, } i=1,\cdots,35; j=1,\cdots,73
\]

(3)模型的建立

根据以上分析,可以建立以下数学模型:

\[
\min \, Z = C_1 \sum_{i=1}^{35} x_i + C_2 \sum_{j=1}^{73} y_j
\]

s.t.
\begin{align*}
&8x_1 + 7x_2 + 6x_3 + 5x_4 + 4x_5 + 3x_6 + 2x_7 + 8x_1 + 7x_{10} + 6x_{11} + 5x_{12} + 4x_{13} + 3x_{14} \\
&+ 2x_{15} + x_{16} + 6x_{18} + 5x_{19} + 4x_{20} + 3x_{21} + x_{23} + 5x_{25} + 4x_{26} + 3x_{27} + 2x_{28} + x_{29} + \\
&4x_{31} + 3x_{32} + 2x_{33} + x_{34} + 14y_1 + 13y_2 + 12y_3 + 11y_4 + 10y_5 + 9y_6 + 8y_7 + 7y_8 + \\
&6y_9 + 5y_{10} + 4y_{11} + 3y_{12} + y_{14} + 13y_{16} + 12y_{17} + 11y_{18} + 10y_{19} + 8y_{20} + 7y_{21} + 6y_{22} \\
&+ 5y_{23} + 4y_{24} + 2y_{25} + 12y_{27} + 11y_{28} + 10y_{29} + 9y_{30} + 8y_{31} + 7y_{32} + 6y_{33} + 5y_{34} + 4y_{35} + 3y_{36} + 2y_{37} \\
&+ y_{38} + 11y_{40} + 10y_{41} + 9y_{42} + 8y_{43} + 7y_{44} + 15y_{58} + 14y_{59} + 13y_{60} \\
&+ 12y_{61} + 11y_{62} + 10y_{63} + 9y_{64} + 8y_{65} + 7y_{66} + 6y_{67} + 5y_{68} + 4y_{69} \\
&+ 3y_{70} + 2y_{71} + y_{72} \geq 156
\end{align*}

\begin{align*}
&x_2 + 2x_3 + 3x_4 + 5x_5 + 6x_6 + 7x_7 + 8x_8 + 10x_9 + x_{11} + 2x_{12} + 3x_{13} + 5x_{14} + 6x_{15} + \\
&7x_{16} + 8x_{17} + x_{19} + 2x_{20} + 3x_{21} + 5x_{22} + 6x_{23} + 7x_{24} + x_{26} + 2x_{27} + 3x_{28} + 5x_{29} \\
&+ y_2 + 2y_3 + 4y_4 + 5y_5 + 6y_6 + 8y_7 + 9y_8 + 10y_9 + 12y_{10} + 11y_{13} + 14y_{12} + 15y_{13} \\
&+ 16y_{14} + 17y_{15} + y_{17} + 2y_{18} + 4y_{19} + 6y_{20} + 8y_{21} + 9y_{22} + 10y_{23} + 12y_{24} + 14y_{25}
\end{align*}

\begin{align*}
&+y_{28} + 2y_{29} + 3y_{30} + 4y_{31} + 5y_{32} + 8y_{33} + 9y_{34} + 10y_{35} + 11y_{36} + 12y_{37} + 13y_{38} \\
&+ 14y_{39} + y_{41} + 2y_{42} + 3y_{43} + 4y_{44} + 5y_{45} + 8y_{46} + 9y_{47} + 10y_{48} + 11y_{49} + 12y_{50} \\
&+ 13y_{51} + 2y_{53} + 4y_{54} + 8y_{55} + 10y_{56} + 12y_{57} + y_{59} + 2y_{60} + 4y_{61} + 5y_{62} + 6y_{63} + 8y_{64} \\
&+ 9y_{65} + 10y_{66} + 12y_{67} + 13y_{68} + 14y_{69} + 15y_{70} + 16y_{71} + 18y_{73} \geq 102
\end{align*}

\[
5\sum_{j=1}^{73} y_j \leq \sum_{i=1}^{35} x_i
\]

\(x_i, y_j\) 均为非负整数且 \(i=1, \cdots, 35; j=1, \cdots, 73\)

\subsection*{4.3.2 可行解的分析}

对于上述模型,运用 lingo 编程,在保证 \(0 < C_1 < C_2 < 2C_1\) 的情况下,可以随机设定 \(C_1\)、\(C_2\),出现以下结果:

\(\text{① } C_1 = 1, C_2 = 1.1\),解得 \(x_1 = 9, x_2 = 2, x_9 = 4, x_{25} = 1, x_{31} = 9, y_{67} = 5\),此时两种轿运车的装载方案如下表所示:

\begin{table}[h]
\centering
\caption{1-1 型轿运车载车方案}
\begin{tabular}{|c|c|c|c|c|c|c|c|c|c|}
\hline
\multicolumn{2}{|c|}{ 上层摆放车辆 } & \multicolumn{3}{c|}{ 下层摆放车辆 } & \multicolumn{3}{c|}{ 总载车辆数 } & \multicolumn{2}{c|}{ 载车方案序号(数量) } \\ \hline
I & II & I & II & III & I & II & III & & \\ \hline
4 & 0 & 4 & 0 & 0 & 8 & 0 & 0 & 1 (9) & \\ \hline
3 & 1 & 4 & 0 & 0 & 7 & 1 & 0 & 2 (2) & \\ \hline
4 & 0 & 3 & 1 & 0 & & & & & \\ \hline
0 & 5 & 0 & 5 & 0 & 0 & 10 & 0 & 9 (4) & \\ \hline
4 & 0 & 1 & 0 & 0 & 3 & 5 & 0 & 3 & 25 (1) \\ \hline
4 & 0 & 0 & 0 & 0 & 4 & 4 & 0 & 4 & 31 (9) \\ \hline
\end{tabular}
\end{table}

\begin{table}[h]
\centering
\caption{1-2 型轿运车载车方案}
\begin{tabular}{|c|c|c|c|c|c|c|c|c|c|c|c|}
\hline
\multirow{2}{*}{ 上层左侧 } & \multirow{2}{*}{ 上层右侧 } & \multicolumn{2}{c|}{ 上层摆放车辆数 } & \multicolumn{3}{c|}{ 下层摆放车辆数 } & \multicolumn{3}{c|}{ 总载车辆数 } & \multicolumn{2}{c|}{ 载车方案序号(数量) } \\ \cline{3-12} 
 & & I & II & I & II & III & I & II & III & & \\ \hline
I & II & I & II & I & II & III & & & & & \\ \hline
2 & 4 & 2 & 4 & 4 & 8 & 2 & 4 & 0 & 6 & 12 & 0 & 67 (5) \\ \hline
\end{tabular}
\end{table}

\(\text{② } C_1 = 1, C_2 = 1.2\) 或 \(1.3\),解得 \(x_1 = 11, x_{22} = 10, x_{31} = 4, y_7 = 1, y_{21} = 1, y_{67} = 3\),此时两种轿运车的装载方案如下表所示:

\begin{table}
\centering
\caption{1-1型轿运车载车方案}
\begin{tabular}{c c | c c c | c c c | c}
\hline
\multicolumn{2}{c|}{上层摆放车辆} & \multicolumn{3}{c|}{下层摆放车辆} & \multicolumn{3}{c|}{总载车辆数} & 载车方案序号(数量) \\
\hline
I & II & I & II & III & I & II & III & \\
\hline
4 & 0 & 4 & 0 & 0 & 8 & 0 & 0 & 1(11) \\
0 & 5 & 2 & 0 & 2 & 2 & 5 & 2 & 22(10) \\
4 & 0 & 0 & 0 & 4 & 4 & 0 & 4 & 31(4) \\
\hline
\end{tabular}
\end{table}

\begin{table}
\centering
\caption{1-2型轿运车载车方案}
\begin{tabular}{c c | c c | c c | c c c | c c c | c}
\hline
上层左侧 & \multicolumn{2}{c|}{上层右侧} & \multicolumn{2}{c|}{上层摆放车辆数} & \multicolumn{3}{c|}{下层摆放车辆数} & \multicolumn{3}{c|}{总载车辆数} & 载车方案序号 \\
I & II & I & II & I & II & I & II & III & I & II & III & (数量) \\
\hline
2 & 4 & 2 & 4 & 4 & 8 & 4 & 0 & 1 & 8 & 8 & 1 & 7(1) \\
2 & 4 & 2 & 4 & 4 & 8 & 3 & 0 & 2 & 7 & 8 & 2 & 21(1) \\
2 & 4 & 2 & 4 & 4 & 8 & 2 & 4 & 0 & 6 & 12 & 0 & 67(3) \\
\hline
\end{tabular}
\end{table}

\begin{equation}
③ C_{1}=1, C_{2}=1.4, \text{解得 } x_{1}=2, x_{8}=1, x_{9}=1, x_{10}=15, x_{31}=1, x_{35}=5, y_{67}=5, \text{此时两种轿运车的装载方案如下表所示:}
\end{equation}

\begin{table}
\centering
\caption{1-1型轿运车载车方案}
\begin{tabular}{c c | c c c | c c c | c}
\hline
\multicolumn{2}{c|}{上层摆放车辆} & \multicolumn{3}{c|}{下层摆放车辆} & \multicolumn{3}{c|}{总载车辆数} & 载车方案序号(数量) \\
\hline
I & II & I & II & III & I & II & III & \\
\hline
3 & 1 & 4 & 0 & 0 & 7 & 1 & 0 & 2(2) \\
4 & 0 & 3 & 1 & 0 & & & & \\
0 & 5 & 1 & 3 & 0 & 1 & 8 & 0 & 8(1) \\
1 & 3 & 0 & 5 & 0 & & & & \\
0 & 5 & 0 & 5 & 0 & 0 & 10 & 0 & 9(1) \\
4 & 0 & 3 & 0 & 0 & 1 & 7 & 0 & 10(15) \\
4 & 0 & 0 & 0 & 4 & 4 & 0 & 4 & 31(1) \\
0 & 5 & 0 & 0 & 4 & 0 & 5 & 4 & 35(5) \\
\hline
\end{tabular}
\end{table}

\begin{table}
\centering
\caption{1-2型轿运车载车方案}
\begin{tabular}{c c | c c | c c | c c c | c c c | c}
\hline
上层左侧 & \multicolumn{2}{c|}{上层右侧} & \multicolumn{2}{c|}{上层摆放车辆数} & \multicolumn{3}{c|}{下层摆放车辆数} & \multicolumn{3}{c|}{总载车辆数} & 载车方案序号 \\
I & II & I & II & I & II & III & I & II & III & (数量) \\
\hline
2 & 4 & 2 & 4 & 4 & 8 & 2 & 4 & 0 & 6 & 12 & 0 & 67(5) \\
\hline
\end{tabular}
\end{table}

\begin{equation}
④ C_{1}=1, C_{2}=1.5, \text{解得 } x_{1}=16, x_{4}=1, x_{35}=8, y_{10}=1, y_{24}=3, y_{67}=1, \text{此时两种轿运车的装载方案如下表:}
\end{equation}

\begin{table}
\centering
\caption{1-1型轿运车载车方案}
\begin{tabular}{c}
\hline
-15- \\
\hline
\end{tabular}
\end{table}

\begin{table}
\centering
\begin{tabular}{c c | c c c | c c c | c}
\hline
\multicolumn{2}{c|}{上层摆放车辆} & \multicolumn{3}{c|}{下层摆放车辆} & \multicolumn{3}{c|}{总载车辆数} & \multicolumn{1}{c}{载车方案序号(数量)} \\
\hline
\multirow{2}{*}{\textbf{I}} & \multirow{2}{*}{\textbf{II}} & \multirow{2}{*}{\textbf{I}} & \multirow{2}{*}{\textbf{II}} & \multirow{2}{*}{\textbf{III}} & \multirow{2}{*}{\textbf{I}} & \multirow{2}{*}{\textbf{II}} & \multirow{2}{*}{\textbf{III}} & \multirow{2}{*}{} \\
\cline{1-8}
 & & & & & & & & \\
\hline
4 & 0 & 4 & 0 & 0 & 8 & 0 & 0 & 1 (16) \\
\hline
1 & 3 & 4 & 0 & 0 & \multirow{4}{*}{5} & \multirow{4}{*}{3} & \multirow{4}{*}{0} & \multirow{4}{*}{4 (1)} \\
\cline{1-3} \cline{8-9}
2 & 2 & 3 & 1 & 0 & & & & \\
\cline{1-3} \cline{8-9}
3 & 1 & 2 & 2 & 0 & & & & \\
\cline{1-3} \cline{8-9}
4 & 0 & 1 & 3 & 0 & & & & \\
\hline
0 & 5 & 0 & 0 & 4 & 0 & 5 & 4 & 35 (8) \\
\hline
\end{tabular}
\caption{表二十 1-2型轿运车载车方案}
\end{table}

\begin{table}
\centering
\begin{tabular}{c c | c c | c c | c c c | c c c | c}
\hline
\multirow{2}{*}{上层左侧} & \multirow{2}{*}{上层右侧} & \multicolumn{2}{c|}{上层摆放车辆数} & \multicolumn{3}{c|}{下层摆放车辆数} & \multicolumn{3}{c|}{总载车辆数} & \multirow{2}{*}{载车方案序号(数量)} \\
\cline{3-12}
 & & \textbf{I} & \textbf{II} & \textbf{I} & \textbf{II} & \textbf{III} & \textbf{I} & \textbf{II} & \textbf{III} & \\
\hline
\textbf{I} & \textbf{II} & \textbf{I} & \textbf{II} & \textbf{I} & \textbf{II} & \textbf{III} & \textbf{I} & \textbf{II} & \textbf{III} & \\
\hline
2 & 4 & 2 & 4 & 4 & 8 & 1 & 4 & 1 & 5 & 12 & 1 & 10 (1) \\
\hline
2 & 4 & 2 & 4 & 4 & 8 & 0 & 4 & 2 & 4 & 12 & 2 & 24 (3) \\
\hline
2 & 4 & 2 & 4 & 4 & 8 & 2 & 4 & 0 & 6 & 12 & 0 & 67 (1) \\
\hline
\end{tabular}
\end{table}

\begin{equation}
\text{⑤} C_{1}=1, C_{2}=1.6, \text{解得} x_{1}=10, x_{22}=11, x_{31}=4, y_{7}=1, y_{64}=2, y_{67}=2, \text{此时两种轿运车的装载方案如下表所示:}
\end{equation}

\begin{table}
\centering
\begin{tabular}{c c | c c c | c c c | c}
\hline
\multicolumn{2}{c|}{上层摆放车辆} & \multicolumn{3}{c|}{下层摆放车辆} & \multicolumn{3}{c|}{总载车辆数} & \multicolumn{1}{c}{载车方案序号(数量)} \\
\hline
\multirow{2}{*}{\textbf{I}} & \multirow{2}{*}{\textbf{II}} & \multirow{2}{*}{\textbf{I}} & \multirow{2}{*}{\textbf{II}} & \multirow{2}{*}{\textbf{III}} & \multirow{2}{*}{\textbf{I}} & \multirow{2}{*}{\textbf{II}} & \multirow{2}{*}{\textbf{III}} & \multirow{2}{*}{} \\
\cline{1-8}
 & & & & & & & & \\
\hline
4 & 0 & 3 & 0 & 1 & 7 & 0 & 1 & 10 (10) \\
\hline
0 & 5 & 2 & 0 & 2 & 2 & 5 & 2 & 22 (11) \\
\hline
4 & 0 & 0 & 0 & 4 & 4 & 0 & 4 & 31 (4) \\
\hline
\end{tabular}
\caption{表二十一 1-1型轿运车载车方案}
\end{table}

\begin{table}
\centering
\begin{tabular}{c c | c c | c c | c c c | c c c | c}
\hline
\multirow{2}{*}{上层左侧} & \multirow{2}{*}{上层右侧} & \multicolumn{2}{c|}{上层摆放车辆数} & \multicolumn{3}{c|}{下层摆放车辆数} & \multicolumn{3}{c|}{总载车辆数} & \multirow{2}{*}{载车方案序号(数量)} \\
\cline{3-12}
 & & \textbf{I} & \textbf{II} & \textbf{I} & \textbf{II} & \textbf{III} & \textbf{I} & \textbf{II} & \textbf{III} & \\
\hline
\textbf{I} & \textbf{II} & \textbf{I} & \textbf{II} & \textbf{I} & \textbf{II} & \textbf{III} & \textbf{I} & \textbf{II} & \textbf{III} & \\
\hline
2 & 4 & 2 & 4 & 4 & 8 & 4 & 0 & 1 & 8 & 8 & 1 & 7 (1) \\
\hline
2 & 4 & 2 & 4 & 4 & 8 & 5 & 0 & 0 & 9 & 8 & 0 & 64 (2) \\
\hline
2 & 4 & 2 & 4 & 4 & 8 & 2 & 4 & 0 & 6 & 12 & 0 & 67 (2) \\
\hline
\end{tabular}
\caption{表二十二 1-2型轿运车载车方案}
\end{table}

\begin{equation}
\text{⑥} C_{1}=1, C_{2}=1.7, \text{解得} x_{1}=11, x_{2}=2, x_{9}=4, x_{10}=1, x_{31}=7, y_{24}=5, \text{此时两种轿运车的装载方案如下表所示:}
\end{equation}

\begin{table}
\centering
\caption{表二十三 1-1型轿运车载车方案}
\begin{tabular}{c c | c c c | c c c | c}
\hline
\multicolumn{2}{c|}{上层摆放车辆} & \multicolumn{3}{c|}{下层摆放车辆} & \multicolumn{3}{c|}{总载车辆数} & \multicolumn{1}{c}{载车方案序号(数量)} \\
\hline
\multirow{2}{*}{I} & \multirow{2}{*}{II} & I & II & III & I & II & III & \multirow{2}{*}{} \\
\cline{3-8}
 & & & & & & & & \\
\hline
4 & 0 & 4 & 0 & 0 & 8 & 0 & 0 & 1 (11) \\
\hline
3 & 1 & 4 & 0 & 0 & 7 & 1 & 0 & 2 (2) \\
\hline
4 & 0 & 3 & 1 & 0 & & & & \\
\hline
0 & 5 & 0 & 5 & 0 & 0 & 10 & 0 & 9 (4) \\
\hline
4 & 0 & 3 & 0 & 1 & 7 & 0 & 1 & 10 (1) \\
\hline
4 & 0 & 0 & 0 & 4 & 4 & 0 & 4 & 31 (7) \\
\hline
\end{tabular}
\end{table}

\begin{table}
\centering
\caption{表二十四 1-2型轿运车载车方案}
\begin{tabular}{c c | c c | c | c c c | c c c | c}
\hline
\multirow{2}{*}{上层左侧} & \multirow{2}{*}{上层右侧} & \multicolumn{1}{c|}{\multirow{2}{*}{上层摆放车辆数}} & \multicolumn{3}{c|}{\multirow{2}{*}{下层摆放车辆数}} & \multicolumn{3}{c|}{总载车辆数} & \multicolumn{1}{c}{\multirow{2}{*}{载车方案序号(数量)}} \\
\cline{4-9}
 & & & & & & I & II & III & \\
\hline
I & II & I & II & I & II & III & I & II & III & \\
\hline
2 & 4 & 2 & 4 & 4 & 8 & 0 & 4 & 2 & 4 & 12 & 2 & 24 (5) \\
\hline
\end{tabular}
\end{table}

\begin{equation}
⑦ C_{1}=1, C_{2}=1.8, \text{解得 } x_{1}=10, x_{22}=7, x_{26}=7, x_{31}=1, y_{67}=5, \text{此时两种轿运车的装载方案如下表所示:}
\end{equation}

\begin{table}
\centering
\caption{表二十五 1-1型轿运车载车方案}
\begin{tabular}{c c | c c c | c c c | c}
\hline
\multicolumn{2}{c|}{上层摆放车辆} & \multicolumn{3}{c|}{下层摆放车辆} & \multicolumn{3}{c|}{总载车辆数} & \multicolumn{1}{c}{载车方案序号(数量)} \\
\hline
\multirow{2}{*}{I} & \multirow{2}{*}{II} & I & II & III & I & II & III & \multirow{2}{*}{} \\
\cline{3-8}
 & & & & & & & & \\
\hline
4 & 0 & 4 & 0 & 0 & 8 & 0 & 0 & 1 (10) \\
\hline
0 & 5 & 2 & 0 & 2 & 2 & 5 & 2 & 22 (7) \\
\hline
4 & 0 & 0 & 1 & 3 & 4 & 1 & 3 & 26 (7) \\
\hline
3 & 1 & 1 & 0 & 3 & & & & \\
\hline
4 & 0 & 0 & 0 & 4 & 4 & 0 & 4 & 31 (1) \\
\hline
\end{tabular}
\end{table}

\begin{table}
\centering
\caption{表二十六 1-2型轿运车载车方案}
\begin{tabular}{c c | c c | c | c c c | c c c | c}
\hline
\multirow{2}{*}{上层左侧} & \multirow{2}{*}{上层右侧} & \multicolumn{1}{c|}{\multirow{2}{*}{上层摆放车辆数}} & \multicolumn{3}{c|}{\multirow{2}{*}{下层摆放车辆数}} & \multicolumn{3}{c|}{总载车辆数} & \multicolumn{1}{c}{\multirow{2}{*}{载车方案序号(数量)}} \\
\cline{4-9}
 & & & & & & I & II & III & \\
\hline
I & II & I & II & I & II & III & I & II & III & \\
\hline
2 & 4 & 2 & 4 & 4 & 8 & 2 & 4 & 0 & 6 & 12 & 0 & 67 (5) \\
\hline
\end{tabular}
\end{table}

\begin{equation}
⑧ C_{1}=1, C_{2}=1.9, \text{解得 } x_{1}=11, x_{19}=1, x_{22}=7, x_{30}=1, x_{31}=5, y_{67}=5, \text{此时两种轿运车的装载方案如下表所示:}
\end{equation}

\begin{table}
\centering
\caption{表二十七 1-1型轿运车载车方案}
\begin{tabular}{c c c c c c c c c}
\hline
\multicolumn{2}{c|}{上层摆放车辆} & \multicolumn{3}{c|}{下层摆放车辆} & \multicolumn{3}{c|}{总载车辆数} & \multicolumn{1}{c}{载车方案序号(数量)} \\
\cline{1-8}
I & II & I & II & III & I & II & III & \\
\hline
4 & 0 & 4 & 0 & 0 & 8 & 0 & 0 & 1(11) \\
3 & 1 & 2 & 0 & 2 & 5 & 1 & 2 & 19(1) \\
4 & 0 & 1 & 1 & 2 & & & & \\
0 & 5 & 2 & 0 & 2 & 2 & 5 & 2 & 22(7) \\
0 & 5 & 0 & 1 & 3 & 0 & 6 & 3 & 30(1) \\
4 & 0 & 0 & 0 & 4 & 4 & 0 & 4 & 31(5) \\
\hline
\end{tabular}
\end{table}

\begin{table}
\centering
\caption{表二十八 1-2型轿运车载车方案}
\begin{tabular}{c c c c c c c c c c c}
\hline
\multirow{2}{*}{上层左侧} & \multirow{2}{*}{上层右侧} & \multirow{2}{*}{上层摆放车辆数} & \multicolumn{3}{c|}{下层摆放车辆数} & \multicolumn{3}{c|}{总载车辆数} & \multicolumn{1}{c}{载车方案序号(数量)} \\
\cline{4-11}
 & & & I & II & III & I & II & III & \\
\hline
I & II & I & II & I & II & III & & & \\
\hline
2 & 4 & 2 & 4 & 4 & 8 & 2 & 4 & 0 & 67(5) \\
\hline
\end{tabular}
\end{table}

以上八种情形下的运费相同,均需1-1型轿运车25辆、1-2型轿运车5辆。其一,情况一、二、四、五、六、八都有6种调度方案,情况三有7种调度方案,情况七有5种调度方案,这样就能减少人员的调配,节约人力物力财力,更符合建设节约型社会的标准;其二,情况七的装载正好满足题中I型乘用车156辆,II型乘用车102辆,III型乘用车39辆的订单要求。而其他情况下的运输方案都会多出1个或多个空位,情况七也刚好满足了资源最大化利用的原则。基于以上两条原因,不难看出在以上8种都合理的情况下可以将情况七的解视为最优解。

\subsubsection{4.3.3 对模型的补充说明}

上述的模型我们只考虑了1-2型轿运车上层完全对称的情况,如果允许一个车位的差别也算对称,那我们就在1-2型轿运车的运输方案上加上以下8种方案,即

\begin{table}
\centering
\caption{1-2型轿运车载车方案}
\begin{tabular}{c c c c c c c c c c c c}
\hline
\multirow{2}{*}{上层左侧} & \multirow{2}{*}{上层右侧} & \multicolumn{2}{c|}{上层摆放车辆数} & \multicolumn{3}{c|}{下层摆放车辆数} & \multicolumn{3}{c|}{总载车辆数} & \multirow{2}{*}{载车方案序号} \\
\cline{3-11}
 & & I & II & I & II & III & I & II & III & \\
\hline
I & II & I & II & I & II & III & & & & \\
\hline
5 & 0 & 4 & 1 & 9 & 1 & 1 & 0 & 4 & 2 & 74 \\
2 & 4 & 1 & 5 & 3 & 9 & 9 & 0 & 4 & 2 & 75 \\
1 & 5 & 0 & 6 & 1 & 11 & 11 & 0 & 4 & 2 & 76 \\
5 & 0 & 4 & 1 & 9 & 1 & 1 & 0 & 0 & 0 & 77 \\
4 & 1 & 3 & 2 & 7 & 3 & 3 & 0 & 0 & 0 & 78 \\
5 & 0 & 0 & 6 & 5 & 6 & 6 & 0 & 0 & 0 & 79 \\
2 & 4 & 1 & 5 & 3 & 9 & 9 & 0 & 0 & 0 & 80 \\
1 & 5 & 0 & 6 & 1 & 11 & 11 & 0 & 0 & 0 & 81 \\
\hline
\end{tabular}
\end{table}

把上述模型的 lingo 程序数据段加入这 8 组数据,保证 $C_{1} < C_{2} < 2C_{1}$ 的情况下,可以随机设定 $C_{1}$、$C_{2}$,对输出结果统计分析得知:最优方案仍是需要 1-1 型轿运车 25 辆,1-2 型轿运车 5 辆。

因此,物流公司应优先选用第七种情况的最优解方案,即共需要 1-1 型轿运车 25 辆,1-2 型轿运车 5 辆。其分配方案如下:需要 10 辆 1-1 型轿运车全部用来装载 80 辆 I 型乘用车(每车装载 8 辆,上下层各 4 辆);需要 7 辆 1-1 型轿运车用来装载 14 辆 I 型乘用车、35 辆 II 型乘用车、14 辆 III 型乘用车(每车装载 2 辆 I 型乘用车、5 辆 II 型乘用车、2 辆 III 型乘用车);需要 7 辆 1-1 型轿运车用来装载 28 辆 I 型乘用车、7 辆 II 型乘用车、21 辆 III 型乘用车(每车装载 4 辆 I 型乘用车、1 辆 II 型乘用车、3 辆 III 型乘用车);需要 1 辆 1-1 型轿运车用来装载 4 辆 I 型乘用车和 4 辆 III 型乘用车;需要 5 辆 1-2 型轿运车来装载 30 辆 I 型乘用车和 60 辆 II 型乘用车(每辆装载 6 辆 I 型乘用车、12 辆 II 型乘用车)。

\subsection*{4.4 问题四的求解}

\subsubsection{4.4.1 问题分析}

(1) 问题四中物流公司依旧需要运输 I 车型和 II 车型的乘用车,虽然此问题增加了多目的地,但由路线图考虑实际情况可知:

由 O 到 C,物流公司只会选由 O 出发经过 D 到达 C。

由 O 到 B,物流公司只会选由 O 出发经过 D 到达 B。

由 O 到 A,物流公司只会选由 O 出发经过 D,B 到达 A。

故此问题不涉及运输途径的选择问题;我们可以采取问题一的求解思路。

(2) 考虑到此题装载时类似第一问,但由于目的地不同,为了避免产生不必要的运费,考虑运输路线为有向路径,不允许折回,但可以中途卸载部分乘用车。

(3) 我们可以首先把问题简化为只有一个目的地,进而计算出 1-1 型轿运车需求量和 1-2 型轿运车需求量;再考虑目的地为多个的时候,在尽量不增加任何轿运车辆的前提下,通过调整运输方案,进而使运费尽可能的少。(即在完成运输任务的前提下,到达 D 后不再行驶的轿运车越多越好,继而到达 C,B 后不再行驶的轿运车也是越多越好。)

\subsubsection{4.4.2 模型建立}

设 $C_{1}$ 表示单辆 1-1 型轿运车的运费,$C_{2}$ 表示单辆 1-2 型轿运车的运费,$C_{1}$ 和 $C_{2}$ 均为固定常数,且 $C_{1} < C_{2}$。根据实际情况,不防假设 $C_{2} < 2C_{1}$。

设 $x_{i}$ 表示按照第 $i$ 个方案装载的 1-1 型轿运车的数目,$y_{j}$ 表示按照第 $j$ 个方案装载的 1-2 型轿运车的数目。

(1)假设目的地唯一,物流公司需要运输 I 车型乘用车 166 辆,II 车型乘用车 78 辆。这就与问题一类似,建立模型如下:

\begin{equation}
\begin{aligned}
\text{min} \quad & Z = C_{1} \sum_{i=1}^{9} x_{i} + C_{2} \sum_{j=1}^{16} y_{j} \\
\text{s.t.} \quad & \sum_{i=1}^{9} (9-i)x_{i} + \sum_{j=1}^{16} (16-j)y_{j} \geq 166
\end{aligned}
\end{equation}

\begin{align*}
x_2 + 2x_3 + 3x_4 + 5x_5 + 6x_6 + 7x_7 + 8x_8 + 10x_9 + y_2 + 2y_3 + 4y_4 + 5y_5 + 6y_6 + 7y_7 + 8y_8 + \\
10y_9 + 12y_{10} + 13y_{11} + 14y_{12} + 15y_{13} + 16y_{14} + 17y_{15} + 18y_{16} &\geq 78 \\
\sum_{j=1}^{16} y_j &\leq \frac{1}{5} \sum_{i=1}^9 x_i
\end{align*}

\(x_i, y_j\) 均为非负整数,且 \(i=1, \cdots, 9; j=1, \cdots, 16\)

利用问题一的程序更换约束条件中的数据,通过 lingo 软件编程求解,输出结果如下:

(1) 当 \(C_1\) 取 1,\(C_2\) 取 1.1,1.3,1.5,1.6,1.8 时,\(x_1 = 18\), \(x_9 = 3\), \(y_{10} = 4\)。

(2) 当 \(C_1\) 取 1,\(C_2\) 取 1.2 时,\(x_1 = 17\), \(x_3 = 1\), \(x_7 = 1\), \(x_9 = 2\), \(y_{10} = 3\), \(y_{11} = 1\)。

(3) 当 \(C_1\) 取 1,\(C_2\) 取 1.4,1.7,1.9 时,\(x_1 = 17\), \(x_2 = 1\), \(x_9 = 3\), \(y_{10} = 3\), \(y_{11} = 1\)。

由结果可知,无论 \(C_2\) 为多少(当然 \(C_1 < C_2 < 2C_1\)),上述最优运行方案均是需要 1-1 型轿运车 21 辆,1-2 型轿运车 4 辆。

(2) 我们希望在不多于 21 辆 1-1 型轿运车和 4 辆 1-2 型轿运车的前提下,考察目的地不同的情况。由于分开运输且不允许折回,可能增加轿运车辆,我们分 \(O \to C\) 线路和 \(O \to A\) 线路来考虑。

由于希望到达 D 地后停运的轿运车越多越好,且 D 的运输要求为 I 车型乘用车 41 辆,II 车型乘用车 0 辆;故采用问题一中 1-1 型轿运车的 \(x_1\) 运输方案(装载 I 车型乘用车 8 辆,II 车型乘用车 0 辆)。此方案下用 1-1 型轿运车 5 辆且这 5 辆车到达 D 地后不再行驶。

同理,由于希望到达 B 地后停运的轿运车越多越好且 B 的运输要求为 I 车型乘用车 50 辆,II 车型乘用车 0 辆;故采用问题一中 1-1 型轿运车的 \(x_1\) 运输方案(装载 I 车型乘用车 8 辆,II 车型乘用车 0 辆)。此方案下用 1-1 型轿运车 6 辆且这 6 辆车到达 B 地后不再行驶。

此时 D 地缺少的 1 辆 I 车型乘用车,可由 \(O \to C\) 线路上的轿运车辆中卸载 1 辆 I 车型乘用车,或者由 \(O \to A\) 线路上的轿运车辆中卸载 1 辆 I 车型乘用车。同理,B 地缺少的 2 辆 I 车型乘用车,可由 \(O \to A\) 线路上的轿运车辆中卸载 2 辆 I 车型乘用车。

### 4.4.3 模型的建立

(1) 采用问题一的分析方法,分别对 \(O \to C\) 线路和 \(O \to A\) 线路建立数学模型:

① \(O \to C\) 线路上的数学模型 a1 如下:

\begin{align*}
\min \quad Z &= C_1 \sum_{i=1}^9 x_i + C_2 \sum_{j=1}^{16} y_j \\
\text{s.t.} \quad \sum_{i=1}^9 (9-i)x_i + \sum_{j=1}^{16} (16-j)y_j &\geq 33
\end{align*}

\begin{align*}
x_2 + 2x_3 + 3x_4 + 5x_5 + 6x_6 + 7x_7 + 8x_8 + 10x_9 + y_2 + 2y_3 + 4y_4 + 5y_5 + 6y_6 + 7y_7 + 8y_8 + \\
10y_9 + 12y_{10} + 13y_{11} + 14y_{12} + 15y_{13} + 16y_{14} + 17y_{15} + 18y_{16} &\geq 47 \\
\sum_{j=1}^{16} y_j &\leq \frac{1}{5} \sum_{i=1}^{9} x_i
\end{align*}

\(x_i, y_j\) 均为非负整数,且 \(i=1, \cdots, 9; j=1, \cdots, 16\)

\(\text{② } O \to A\) 线路上的数学模型 b1 如下:

\begin{align*}
\text{min} \quad Z &= C_1 \sum_{i=1}^{9} x_i + C_2 \sum_{j=1}^{16} y_j \\
\text{s.t.} \quad \sum_{i=1}^{9} (9-i)x_i + \sum_{j=1}^{16} (16-j)y_j &\geq 45 \\
x_2 + 2x_3 + 3x_4 + 5x_5 + 6x_6 + 7x_7 + 8x_8 + 10x_9 + y_2 + 2y_3 + 4y_4 + 5y_5 + 6y_6 + 7y_7 + 8y_8 + \\
10y_9 + 12y_{10} + 13y_{11} + 14y_{12} + 15y_{13} + 16y_{14} + 17y_{15} + 18y_{16} &\geq 31 \\
\sum_{j=1}^{16} y_j &\leq \frac{1}{5} \sum_{i=1}^{9} x_i
\end{align*}

\(x_i, y_j\) 均为非负整数,且 \(i=1, \cdots, 9; j=1, \cdots, 16\)

利用问题一的程序更换约束条件中的数据,通过 LINGO 软件编程求解,\(O \to C\) 线路模型 a1 输出结果如下:

(1) 当 \(C_1\) 取 1,\(C_2\) 取 1.1,1.2,1.3,1.4 时,\(x_1 = 3\),\(x_6 = 1\),\(x_8 = 1\),\(x_9 = 2\),\(y_{11} = 1\)。

(2) 当 \(C_1\) 取 1,\(C_2\) 取 1.5 时,\(x_1 = 2\),\(x_4 = 1\),\(x_9 = 4\),\(y_4 = 1\)。

(3) 当 \(C_1\) 取 1,\(C_2\) 取 1.6 时,\(x_1 = 4\),\(x_9 = 3\),\(y_{15} = 1\)。

(4) 当 \(C_1\) 取 1,\(C_2\) 取 1.7,1.8 时,\(x_1 = 3\),\(x_9 = 4\),\(y_7 = 1\)。

(5) 当 \(C_1\) 取 1,\(C_2\) 取 1.9 时,\(x_1 = 3\),\(x_5 = 1\),\(x_9 = 3\),\(y_{10} = 1\)。

由运行结果可知 \(O \to C\) 线路上需要 7 辆 1-1 型轿运车,1 辆 1-2 型轿运车。

利用问题一的程序更换约束条件中的数据,通过 LINGO 软件编程求解,\(O \to A\) 线路模型 b1 输出结果如下:

(1) 当 \(C_1\) 取 1,\(C_2\) 取 1.1,1.2,1.3,1.4,1.5,1.6,1.7,1.8,1.9 时,\(x_1 = 5\),\(x_9 = 2\),\(y_{10} = 1\)。

由运行结果可知 $O \rightarrow A$ 线路上需要 7 辆 1-1 型轿运车,1 辆 1-2 型轿运车。  
综上所述,物流公司总计需要 1-1 型轿运车为 $5+6+7+7=25$ 辆 ($O \rightarrow D$ 为 5 辆,$O \rightarrow B$ 为 6 辆,$O \rightarrow C$ 为 7 辆,$O \rightarrow A$ 为 7 辆);1-2 型轿运车 2 辆 ($O \rightarrow C$ 为 1 辆,$O \rightarrow A$ 为 1 辆)。此时我们比最优运行方案多用 1-1 型轿运车 4 辆,少用 1-2 型轿运车 2 辆。但由于 $C_1 < C_2 < 2C_1$,即 $2C_2 < 4C_1$,运费比最优方案增加,故需要修改上述模型。

(2)考虑到 $O \rightarrow C$ 路线中 1-2 型轿运车占总 1-2 型轿运车的 12.5%,故 $O \rightarrow A$ 的路线中 1-2 型轿运车占总 1-2 型轿运车的比例可以略高于 20%,只要满足总计的运输车辆中 1-2 型所占比例不超过 20% 即可。由此考虑保持修改模型 b,将约束条件 $\sum_{j=1}^{16} y_j \leq \frac{1}{5} \sum_{i=1}^9 x_i$ 去掉,结合最优运输方案,增加 $\sum_{j=1}^{16} y_j \leq 3$ 即可。得到下述模型 b2:

\begin{align*}
\text{min} \quad & Z = C_1 \sum_{i=1}^9 x_i + C_2 \sum_{j=1}^{16} y_j \\
\text{s.t.} \quad & \sum_{i=1}^9 (9-i)x_i + \sum_{j=1}^{16} (16-j)y_j \geq 45; \\
& x_2 + 2x_3 + 3x_4 + 5x_5 + 6x_6 + 7x_7 + 8x_8 + 10x_9 + y_2 + 2y_3 + 4y_4 + 5y_5 + 6y_6 + 7y_7 + 8y_8 + \\
& \quad 10y_9 + 12y_{10} + 13y_{11} + 14y_{12} + 15y_{13} + 16y_{14} + 17y_{15} + 18y_{16} \geq 31; \\
& \sum_{j=1}^{16} y_j \leq 3; \\
& x_i, y_j \text{ 均为非负整数,且 } i=1, \cdots, 9; j=1, \cdots, 16
\end{align*}

利用模型 b1 的程序更换约束条件,通过 lingo 软件编程求解,$O \rightarrow A$ 线路模型 b2 输出结果如下:

(1)当 $C_1$ 取 1,$C_2$ 取 1.1,1.2 时,$x_1 = 3$,$y_7 = 1$,$y_{10} = 2$。

(2)当 $C_1$ 取 1,$C_2$ 取 1.3 时,$x_1 = 1$,$x_3 = 1$,$y_7 = 2$,$y_{10} = 1$。

(3)当 $C_1$ 取 1,$C_2$ 取 1.4, 1.5,1.6,1.7,1.8,1.9 时,$x_1 = 3$,$y_7 = 1$,$y_{10} = 2$。

由运行结果,我们可以得知 $O \rightarrow A$ 线路上需要 3 辆 1-1 型轿运车,3 辆 1-2 型轿运车。改进后物流公司总计需要 1-1 型轿运车为 $5+6+7+3=21$ 辆 ($O \rightarrow D$ 为 5 辆,$O \rightarrow B$ 为 6 辆,$O \rightarrow C$ 为 7 辆,$O \rightarrow A$ 为 3 辆);1-2 型轿运车 4 辆 ($O \rightarrow C$ 为 1 辆,$O \rightarrow A$ 为 3 辆)。由此可知,改进后的方案与最优运输方案吻合。

### 4.4.4 对模型的结果分析

在 lingo 程序运行中,保证 $C_1 < C_2 < 2C_1$ 的情况下,可以随机设定 $C_1$、$C_2$。

1、对于模型 a1,以下是对 lingo 运行的 5 种结果的分析:

① 当 $C_1$ 取 1,$C_2$ 取 1.1,1.2,1.3,1.4 时,$x_1 = 3$,$x_6 = 1$,$x_8 = 1$,$x_9 = 2$,$y_{11} = 1$,其

\begin{table}
\centering
\begin{tabular}{|c|c|c|c|c|c|c|c|}
\hline
\multicolumn{8}{|c|}{轿运车载车方案} \\
\hline
载车方案序号 & \multicolumn{2}{c|}{总载车辆数} & \multicolumn{2}{c|}{上层摆放车辆数} & \multicolumn{2}{c|}{下层摆放车辆数} & 需求轿运车车辆数 \\
\hline
 & I & II & I & II & I & II & \\
\hline
1-1型轿运车方案1 & 8 & 0 & 4 & 0 & 4 & 0 & 3 \\
\hline
1-1型轿运车方案6 & 3 & 6 & 3 & 1 & 0 & 5 & 1 \\
\hline
1-1型轿运车方案8 & 1 & 8 & 1 & 3 & 0 & 5 & 1 \\
\hline
1-1型轿运车方案9 & 0 & 10 & 0 & 5 & 0 & 5 & 2 \\
\hline
1-2型轿运车方案11 & 5 & 13 & 4 & 8 & 1 & 5 & 1 \\
\hline
\end{tabular}
\end{table}

\textbf{②} 当 \(C_1\) 取 1,\(C_2\) 取 1.5 时,\(x_1=2\),\(x_4=1\),\(x_9=4\),\(y_4=1\),其它值均为零。所以:

\begin{table}
\centering
\begin{tabular}{|c|c|c|c|c|c|c|c|}
\hline
\multicolumn{8}{|c|}{轿运车载车方案} \\
\hline
载车方案序号 & \multicolumn{2}{c|}{总载车辆数} & \multicolumn{2}{c|}{上层摆放车辆数} & \multicolumn{2}{c|}{下层摆放车辆数} & 需求轿运车车辆数 \\
\hline
 & I & II & I & II & I & II & \\
\hline
1-1型轿运车方案1 & 8 & 0 & 4 & 0 & 4 & 0 & 2 \\
\hline
1-1型轿运车方案4 & 5 & 3 & 4 & 0 & 1 & 3 & 1 \\
\hline
 & & & 3 & 1 & 2 & 2 & \\
\hline
1-1型轿运车方案9 & 0 & 10 & 0 & 5 & 0 & 5 & 4 \\
\hline
1-2型轿运车方案4 & 12 & 4 & 10 & 0 & 2 & 4 & 1 \\
\hline
\end{tabular}
\end{table}

\textbf{③} 当 \(C_1\) 取 1,\(C_2\) 取 1.6 时,\(x_1=4\),\(x_9=3\),\(y_{15}=1\),其它值均为零。所以:

\begin{table}
\centering
\begin{tabular}{|c|c|c|c|c|c|c|c|}
\hline
\multicolumn{8}{|c|}{轿运车载车方案} \\
\hline
载车方案序号 & \multicolumn{2}{c|}{总载车辆数} & \multicolumn{2}{c|}{上层摆放车辆数} & \multicolumn{2}{c|}{下层摆放车辆数} & 需求轿运车车辆数 \\
\hline
 & I & II & I & II & I & II & \\
\hline
1-1型轿运车方案1 & 8 & 0 & 4 & 0 & 4 & 0 & 4 \\
\hline
1-1型轿运车方案9 & 0 & 10 & 0 & 5 & 0 & 5 & 3 \\
\hline
1-2型轿运车方案15 & 1 & 17 & 0 & 12 & 1 & 15 & 1 \\
\hline
\end{tabular}
\end{table}

\textbf{④} 当 \(C_1\) 取 1,\(C_2\) 取 1.7,1.8 时,\(x_1=3\),\(x_9=4\),\(y_7=1\),其它值均为零。所以:

\begin{table}
\centering
\begin{tabular}{c c c c c c c}
\hline
\multicolumn{7}{c}{轿运车载车方案} \\
\hline
载车方案序号 & \multicolumn{2}{c}{总载车辆数} & \multicolumn{2}{c}{上层摆放车辆数} & \multicolumn{2}{c}{下层摆放车辆数} & 需求轿运车车辆数 \\
\cline{2-7}
 & I & II & I & II & I & II & \\
\hline
 & & & & & & & \\
1-1型轿运车方案1 & 8 & 0 & 4 & 0 & 4 & 0 & 3 \\
1-1型轿运车方案9 & 0 & 10 & 0 & 5 & 0 & 5 & 4 \\
1-2型轿运车方案7 & 9 & 8 & 4 & 8 & 5 & 0 & 1 \\
\hline
\end{tabular}
\end{table}

⑤ 当 $C_1$ 取 1,$C_2$ 取 1.9 时,$x_1=3$,$x_5=1$,$x_9=3$,$y_{11}=1$,其它值均为零。所以:

\begin{table}
\centering
\begin{tabular}{c c c c c c c}
\hline
\multicolumn{7}{c}{轿运车载车方案} \\
\hline
载车方案序号 & \multicolumn{2}{c}{总载车辆数} & \multicolumn{2}{c}{上层摆放车辆数} & \multicolumn{2}{c}{下层摆放车辆数} & 需求轿运车车辆数 \\
\cline{2-7}
 & I & II & I & II & I & II & \\
\hline
 & & & & & & & \\
1-1型轿运车方案1 & 8 & 0 & 4 & 0 & 4 & 0 & 3 \\
1-1型轿运车方案5 & 4 & 5 & 4 & 0 & 0 & 5 & 1 \\
1-1型轿运车方案9 & 0 & 10 & 0 & 5 & 0 & 5 & 3 \\
1-2型轿运车方案11 & 5 & 13 & 4 & 8 & 1 & 5 & 1 \\
\hline
\end{tabular}
\end{table}

综上可知,无论哪种方案,都需要 1-1 型轿运车 7 辆,1-2 型轿运车 1 辆,且第一、二种情形正好装载 I 型乘用车 34 辆,II 型乘用车 47 辆,刚好满足了资源最大化利用的原则,可视为理想的最优方案。

2、对于模型 b1,以下是对 lingo 运行的 3 种结果的分析:

① 当 $C_1$ 取 1,$C_2$ 取 1.1,1.2 时,$x_1=3$,$y_7=1$,$y_{10}=2$,其它值均为零。所以:

\begin{table}
\centering
\begin{tabular}{c c c c c c c}
\hline
\multicolumn{7}{c}{轿运车载车方案} \\
\hline
载车方案序号 & \multicolumn{2}{c}{总载车辆数} & \multicolumn{2}{c}{上层摆放车辆数} & \multicolumn{2}{c}{下层摆放车辆数} & 需求轿运车车辆数 \\
\cline{2-7}
 & I & II & I & II & I & II & \\
\hline
 & & & & & & & \\
1-1型轿运车方案1 & 8 & 0 & 4 & 0 & 4 & 0 & 3 \\
1-2型轿运车方案7 & 9 & 8 & 4 & 8 & 5 & 0 & 1 \\
1-2型轿运车方案10 & 6 & 12 & 4 & 8 & 2 & 4 & 2 \\
\hline
\end{tabular}
\end{table}

综上知,可装载 I 型乘用车 45 辆,可装载 II 型乘用车 32 辆。此种情况满足物流公司往 A 地运输 I 型乘用车 42 辆及 II 型乘用车 31 辆的需求,对于多运的 3 辆 I 型乘用车,可以分别在 D、B 两地卸载 1、2 辆 I 型乘用车给 D、B 两地,解决了 D、B 地少 1、2 辆 I 型乘用车的问题,对于多运的 1 辆 II 型乘用车,这就要求少装 1 辆 II 型乘用车才能解决。

② 当 $C_1$ 取 1,$C_2$ 取 1.3 时,$x_1=1$,$x_3=1$,$y_7=2$,$y_{10}=1$,其它值均为零。所以:

\begin{table}
\centering
\begin{tabular}{|c|c|c|c|c|c|c|c|}
\hline
\multicolumn{8}{|c|}{轿运车载车方案} \\
\hline
载车方案序号 & \multicolumn{2}{c|}{总载车辆数} & \multicolumn{2}{c|}{上层摆放车辆数} & \multicolumn{2}{c|}{下层摆放车辆数} & 需求轿运车车辆数 \\
\hline
 & I & II & I & II & I & II & \\
\hline
1-1型轿运车方案1 & 8 & 0 & 4 & 0 & 4 & 0 & 1 \\
\hline
1-1型轿运车方案2 & 7 & 1 & 4 & 0 & 3 & 1 & 1 \\
\hline
1-1型轿运车方案3 & 6 & 2 & 4 & 0 & 2 & 2 & \multirow{2}{*}{1} \\
\cline{3-7}
 & & & 3 & 1 & 3 & 1 & \\
\hline
1-2型轿运车方案7 & 9 & 8 & 4 & 8 & 5 & 0 & 2 \\
\hline
1-2型轿运车方案10 & 6 & 12 & 4 & 8 & 2 & 4 & 1 \\
\hline
\end{tabular}
\end{table}

综上知,可装载Ⅰ型乘用车45辆,可装载Ⅱ型乘用车31辆。此种情况满足物流公司往A地运输Ⅰ型乘用车42辆及Ⅱ型乘用车31辆的需求,对于多运的3辆Ⅰ型乘用车,可以分别在D、B两地卸载1、2辆Ⅰ型乘用车给D、B两地,解决了D、B地少1、2辆Ⅰ型乘用车的问题,此情况恰好解决A、D、B三地的问题。

③当$C_{1}$取1,$C_{2}$取1.4,1.5,1.6,1.7,1.8,1.9时,$x_{1}=3$,$y_{7}=1$,$y_{10}=2$,其它值均为零。所以:

\begin{table}
\centering
\begin{tabular}{|c|c|c|c|c|c|c|c|}
\hline
\multicolumn{8}{|c|}{轿运车载车方案} \\
\hline
载车方案序号 & \multicolumn{2}{c|}{总载车辆数} & \multicolumn{2}{c|}{上层摆放车辆数} & \multicolumn{2}{c|}{下层摆放车辆数} & 需求轿运车车辆数 \\
\hline
 & I & II & I & II & I & II & \\
\hline
1-1型轿运车方案1 & 8 & 0 & 4 & 0 & 4 & 0 & 3 \\
\hline
1-2型轿运车方案7 & 9 & 8 & 4 & 8 & 5 & 0 & 1 \\
\hline
1-2型轿运车方案10 & 6 & 12 & 4 & 8 & 2 & 4 & 2 \\
\hline
\end{tabular}
\end{table}

综上知,可装载Ⅰ型乘用车45辆,可装载Ⅱ型乘用车32辆。此种情况满足物流公司往A地运输Ⅰ型乘用车42辆及Ⅱ型乘用车31辆的需求,对于多运的3辆Ⅰ型乘用车,可以分别在D、B两地卸载1、2辆Ⅰ型乘用车给D、B两地,解决了D、B地少1、2辆Ⅰ型乘用车的问题,对于多运的1辆Ⅱ型乘用车,这就要求少装1辆Ⅱ型乘用车才能解决。

综上所述,最合理的运输方案为:

对于$O\rightarrow D$线路,采用1-1型轿运车的$x_{1}$运输方案(装载Ⅰ车型乘用车8辆,Ⅱ车型乘用车0辆),用1-1型轿运车5辆且这5辆车到达D地后不再行驶。

对于$O\rightarrow B$线路,采用1-1型轿运车的$x_{1}$运输方案(装载Ⅰ车型乘用车8辆,Ⅱ车型乘用车0辆),用1-1型轿运车6辆且这5辆车到达B地后不再行驶。

对于 $O \rightarrow C$ 线路,采用模型 a1 下的①、②、③三种方案,此三种情况可装载 I 型乘用车 33 辆,可装载 II 型乘用车 47 辆。此种情况恰好满足物流公司往 C 地运输 I 型乘用车 33 辆及 II 型乘用车 47 辆的需求。

对于 $O \rightarrow A$ 线路,采用模型 b2 下的②方案,此种情况满足物流公司往 A 地运输 I 型乘用车 42 辆及 II 型乘用车 31 辆的需求,对于多运的 3 辆 I 型乘用车,可以分别在 D、B 两地卸载 1、2 辆 I 型乘用车给 D、B 两地,解决了 D、B 地少 1、2 辆 I 型乘用车的问题,此情况恰好解决 A、D、B 三地的问题。

对于上述运输方案,从 O 地出发共计需要 1-1 型轿运车 21 辆,1-2 型轿运车 4 辆;恰好满足最优运行方案。

### 4.4.5 对模型的补充说明

考虑到 O 地到 A 地和 C 地的距离不同,1-1 型轿运车和 1-2 型轿运车每公里的耗费不同,那么假设 1-1 型轿运车的耗费为 $d_1$ 元/公里,1-2 型轿运车的耗费为 $d_2$ 元/公里 ($d_1 < d_2 < 2d_1$)。如果考虑到 $O \rightarrow A$ 路线的距离大于 $O \rightarrow C$ 的路线的距离,那就尽可能的让 1-2 型轿运车去 C 地,那就考虑保持 b1,修改模型 a1,将约束条件 $\sum_{j=1}^{16} y_j \leq \frac{1}{5} \sum_{i=1}^9 x_i$ 去掉,结合最优运输方案,增加 $\sum_{j=1}^{16} y_j \leq 3$ 即可。得到下述模型 a2:

\begin{align*}
\text{min} \quad & Z = C_1 \sum_{i=1}^9 x_i + C_2 \sum_{j=1}^{16} y_j \\
\text{s.t.} \quad & \sum_{i=1}^9 (9-i)x_i + \sum_{j=1}^{16} (16-j)y_j \geq 33 \\
& x_2 + 2x_3 + 3x_4 + 5x_5 + 6x_6 + 7x_7 + 8x_8 + 10x_9 + y_2 + 2y_3 + 4y_4 + 5y_5 + 6y_6 + 7y_7 + 8y_8 + \\
& \quad 10y_9 + 12y_{10} + 13y_{11} + 14y_{12} + 15y_{13} + 16y_{14} + 17y_{15} + 18y_{16} \geq 47 \\
& \sum_{j=1}^{16} y_j \leq 3 \\
& x_i, y_j \text{ 均为非负整数,且 } i=1, \cdots, 9; j=1, \cdots, 16
\end{align*}

利用 a1 的程序更换约束条件,通过 LINGO 软件编程求解,$O \rightarrow C$ 线路模型 a2 的运行结果为 $O \rightarrow C$ 线路上需要 3 辆 1-1 型轿运车,3 辆 1-2 型轿运车。

此时的从 O 地出发共计需要 1-1 型轿运车 $7+6+5+3=21$ 辆,1-2 型轿运车 $1+3=4$ 辆恰好满足最优运行方案。但此时的总费用为:

\[
W_{11} = (160*5 + 280*6 + 360*7 + 236*3)d_1 + (236*3 + 360*1)d_2 + 21C_1 + 4C_2 = 5708d_1 + 1068d_2 + 21C_1 + 4C_2
\]

而采用模型 a1 和 b2 构建的最合理的运输方案的总费用为:

\[
W_{12} = (160*5 + 280*6 + 360*3 + 236*7)d_1 + (236*1 + 360*3)d_2 + 21C_1 + 4C_2 = 5212d_1 + 1316d_2 + 21C_1 + 4C_2
\]

两种方案的费用差为:
\[
W_{21} - W_{12} = 496d_1 - 248d_2
\]

由于 $d_1 < d_2 < 2d_1$ 所以 $W_{21} - W_{12} > 0$,即采用模型 a1 和 b2 构建的最合理的运输方案要比采用模型 a2 和 b1 构建的最合理的运输方案更节约成本。

总结上述所有模型的结果,我们最优的方案是:采用模型 a1 和 b2 构建的最合理的运输方案即共计需要 1-1 型轿运车 21 辆,1-2 型轿运车 4 辆,其分配方式为 $O \to D$ 线路 1-1 型轿运车 5 辆,1-2 型轿运车 0 辆;$O \to B$ 线路 1-1 型轿运车 6 辆,1-2 型轿运车 0 辆;$O \to C$ 线路 1-1 型轿运车 7 辆,1-2 型轿运车 1 辆;$O \to A$ 线路 1-1 型轿运车 3 辆,1-2 型轿运车 3 辆。

\subsection*{4.5 问题五分析与建模}

\subsubsection{4.5.1 问题分析}

题目所给附件表 1.xlsx 中有七类 1-1 型轿运车,1-1 型的装载中只考虑所装乘用车的车长即可,因此为了简便运算,可以考虑按长度相近归为一类,将 1-1 型轿运车按如下长度硬性分为 4 类,同理,将两类 1-2 型轿运车归为一类。

\begin{table}[h]
\centering
\begin{tabular}{|c|c|c|c|c|c|}
\hline
 & 1-1-1 型轿运车 & 1-1-2 型轿运车 & 1-1-3 型轿运车 & 1-1-4 型轿运车 & 1-2 型轿运车 \\
\hline
实际长度(m) & 19 & 18.3/18.2 & 24.3 & 21/21/22 & 23.3/23.7 \\
\hline
优化长度(m) & 19 & 18.2 & 24.3 & 21 & 23.3 \\
\hline
\end{tabular}
\end{table}

附表 2.xlsx 中要运输的 1207 辆乘用车宽度分析,其中宽度超过 1.7m(包括 1.7m)的有 960 辆,另外有 37 辆车型编号为 37 的微型车高度超过了 1.7m,这 997 辆乘用车占总体车辆的 80\% 左右,且只能放在轿运车的下层。2-2 型车的宽度为 3.5m,在它的上、下层并排放两辆乘用车可能性不到 20\%;又根据题中受高度限制,高度超过 1.7m 的乘用车只能装在 1-1、1-2 型下层,那么这 37 辆车只能放在 1-1、1-2 型下层,放在 2-2 型车的可能性进一步减少,同时考虑到 2-2 行车的运费比较高,所以本次运输方案仍然不考虑使用 2-2 型轿运车,只选择 1-1、1-2 型轿运车。

\subsubsection{4.5.2 模型简化}

题中需要运输的乘用车种类多数量多,我们把题目中的 45 类乘用车大致分成三大类进行简化,进而建立模型。

\textbf{假设一}

将宽度大于 1.7 米的乘用车归为第一类,且这些车只能放在 1-1 型轿运车上下层或 1-2 轿用车下层(因为宽度都大于 1.7 的乘用车不能同时并排放 1-2 上层,不然太拥挤),这 24 类车需要放在 1-1 型车上下层或 1-2 型下层。在 1-2 型车拥有量较少的实际情况 下,题中要求 1-2 型车的使用不能超过 1-1 型车的 20\%,所以进行人工干预,要求这 24 类车需要全部放在 1-1 型轿运车里面。由于这 24 类乘用车数量较大,我们再次利用简化模型的方式,按照其装载准许的范围内将其强行划分两种类型:其一,将长度不大于 4.61 的车记为 I 型乘用车,由附表 2 知共有 13 类 I 型乘用车;其二,将长度大于 4.61 且不大于 5.015 的车记为 II 型乘用车,则共有 11 类 II 型乘用车。如下表:

\begin{table}[h]
\centering
\begin{tabular}{|c|c|c|}
\hline
 & I 型乘用车 & II 型乘用车 \\
\hline
长度(m) & 长 $\leq 4.61$ & $4.61 < $ 长 $\leq 5.015$ \\
\hline
\end{tabular}
\end{table}

又由附表 1 知,A 地需要 I 型乘用车 108 辆、II 型乘用车 88 辆,B 地需要 I 型乘用车 76 辆、II 型乘用车 61 辆,C 地需要 I 型乘用车 70 辆、II 型乘用车 44 辆,D 地需要 I 型乘用车 56 辆、II 型乘用车 55 辆,E 地需要 I 型乘用车 63 辆、II 型乘用车 43 辆。由于题中要求所有轿运车的行驶路线都是单向的(即不返回),而中途可卸载,为了满足客户的需求,所以每一次运输都先考虑远距离的目的地。

在所有类型的 1-1 类轿运车损耗都是一样的,为了更节约成本每一次远程运输我们都尽可能选择长度长点的车。我们用 1-1-1 型、1-1-3 型及 1-1-4 型的轿运车运输第一类的乘用车。在完成运输任务的前提下,我们要求在运输过程中最大限度的使用长车。从 O 到 A、B、D、C、E 他们之间前一个目的地都会影响后一个目的地的运输方案,由于题中强调每次卸载成本可以忽略不计,因此每类轿运车可以中途卸载。

\textbf{假设二}

将高度大于 1.7 米的乘用车归为第二类,受题中一些限制条件的约束,这些偏高的乘用车只能装在 1-1 型轿运车或 1-2 型轿运车的下层。由附表 2 不难看出这种情形有 8 类车,又因为大型车数量小且不能和其他乘用车并排运输,为了简化模型我们也把大型车归为此类情况,故此类情形共用 9 种类型的乘用车。

同理,在不影响正常装运的前提下,我们可将这类车归结为下表的 3 类:

\begin{table}[h]
\centering
\begin{tabular}{|c|c|c|c|}
\hline
 & I 型乘用车 & II 型乘用车 & III 型乘用车 \\
\hline
长度(m) & 长 $\leqslant 4.61$ & $4.61 < $ 长 $\leqslant 5.16$ & $6.831$(即大型车) \\
\hline
\end{tabular}
\end{table}

由附表 2 知,E 地需要 I 型乘用车 22 辆、II 型乘用车 10 辆、III 型乘用车 1 辆,A 地需要 I 型乘用车 10 辆、II 型乘用车 14 辆、III 型乘用车 2 辆,B 地需要 I 型乘用车 17 辆、II 型乘用车 5 辆;C 地需要 I 型乘用车 31 辆、II 型乘用车 12 辆,D 地需要 I 型乘用车 30 辆、II 型乘用车 5 辆、III 型乘用车 1 辆。此类情形的运输方案仍然遵循远程优先考虑,较长轿运车优先选择的原则。由于情形一已经大量使用了 1-1 系列的轿运车,因此,情形二在确保不 1-2 型轿运车使用量不超过 1-1 型轿运车的 20\% 前提下,优先考虑使用 1-2 的下层,再考虑 1-1 系列的下层。

\textbf{假设三}

除去上述两种情况,其余 12 类乘用车归为一类,这类车可以任意放在各类车的任何位置。为了简便运算,同样将此类乘用车简化成两种类型的乘用车,如下表:

\begin{table}[h]
\centering
\begin{tabular}{|c|c|c|}
\hline
 & I 型乘用车 & II 型乘用车 \\
\hline
长度(m) & 长 $\leqslant 4.0$ & $4.0 < $ 长 $\leqslant 4.4$ \\
\hline
\end{tabular}
\end{table}

由附表 2 知,E 地需要 I 型乘用车 34 辆、II 型乘用车 37 辆,A 地需要 I 型乘用车 18 辆、II 型乘用车 52 辆,B 地需要 I 型乘用车 21 辆、II 型乘用车 31 辆,C 地需要 I 型乘用车 59 辆、II 型乘用车 36 辆,D 地需要 I 型乘用车 44 辆、II 型乘用车 44 辆。由于前两种情形的乘用车装载都比较挑剔,所以此类乘用车可以根据路径来填充之前装载的空缺位置。

\subsection*{4.5.2 模型的建立和求解}

\textbf{模型一}

针对第一类乘用车的运载,考虑所有的装载方式如下表:

\begin{table}
\centering
\begin{tabular}{|c|c||c|c||c|c||c|c|}
\hline
\multicolumn{2}{|c||}{1-1-1 型} & \multicolumn{2}{c||}{1-1-2 型} & \multicolumn{2}{c||}{1-1-3 型} & \multicolumn{2}{c|}{1-1-4 型} \\
\hline
I & II & I & II & I & II & I & II \\
\hline
8 & 0 & 8 & 0 & 12 & 0 & 10 & 0 \\
7 & 1 & 6 & 1 & 9 & 1 & 7 & 1 \\
6 & 2 & 5 & 2 & 8 & 2 & 6 & 2 \\
5 & 3 & 4 & 3 & 7 & 3 & 5 & 3 \\
4 & 4 & 2 & 4 & 6 & 4 & 4 & 4 \\
3 & 5 & 1 & 5 & 5 & 5 & 3 & 5 \\
2 & 6 & & & 4 & 6 & 2 & 6 \\
1 & 7 & & & 3 & 7 & 1 & 7 \\
 & & & & 2 & 8 & 0 & 8 \\
 & & & & 1 & 9 & & \\
 & & & & 0 & 10 & & \\
\hline
\end{tabular}
\end{table}

按照 1-1-3 型、1-1-4 型和 1-1-1 型的轿运车装载方案,可列出情况一中 664 辆乘用车的转载方案矩阵 A。其中,运往 A、B、C、D、E 的车辆数量分别为 196 辆、137 辆、114 辆、111 辆、106 辆。为了简便运算,我可以根据前期的数据分析将此类车归为 373 辆 I 型乘用车和 291 辆 II 型乘用车。

设
\[
A = \begin{pmatrix}
12 & 9 & 8 & 7 & 6 & 5 & 4 & 3 & 2 & 1 & 0 & 10 & 7 & 6 & 5 & 4 & 3 & 2 & 1 & 0 & 8 & 7 & 6 & 5 & 4 & 3 & 2 & 1 \\
0 & 1 & 2 & 3 & 4 & 5 & 6 & 7 & 8 & 9 & 10 & 0 & 1 & 2 & 3 & 4 & 5 & 6 & 7 & 8 & 0 & 1 & 2 & 3 & 4 & 5 & 6 & 7
\end{pmatrix}_{2 \times 28}
\]

\[
X_e = (x_{e_1}, \, x_{e_2}, \, \cdots, \, x_{e_{11}})^{\mathrm{T}}, \, Y_e = (y_{e_1}, \, y_{e_2}, \, \cdots, \, y_{e_9})^{\mathrm{T}}, Z_e = (z_{e_1}, \, z_{e_2}, \, \cdots, \, z_{e_8})^{\mathrm{T}},
\]
\[
X_a = (x_{a_1}, \, x_{a_2}, \, \cdots, \, x_{a_{11}})^{\mathrm{T}}, \, Y_a = (y_{a_1}, \, y_{a_2}, \, \cdots, \, y_{a_9})^{\mathrm{T}}, Z_a = (z_{a_1}, \, z_{a_2}, \, \cdots, \, z_{a_8})^{\mathrm{T}},
\]
\[
X_b = (x_{b_1}, \, x_{b_2}, \, \cdots, \, x_{b_{11}})^{\mathrm{T}}, \, Y_b = (y_{b_1}, \, y_{b_2}, \, \cdots, \, y_{b_9})^{\mathrm{T}}, Z_b = (z_{b_1}, \, z_{b_2}, \, \cdots, \, z_{b_8})^{\mathrm{T}},
\]
\[
X_c = (x_{c_1}, \, x_{c_2}, \, \cdots, \, x_{c_{11}})^{\mathrm{T}}, \, Y_c = (y_{c_1}, \, y_{c_2}, \, \cdots, \, y_{c_9})^{\mathrm{T}}, Z_c = (z_{c_1}, \, z_{c_2}, \, \cdots, \, z_{c_8})^{\mathrm{T}},
\]
\[
X_d = (x_{d_1}, \, x_{d_2}, \, \cdots, \, x_{d_{11}})^{\mathrm{T}}, \, Y_d = (y_{d_1}, \, y_{d_2}, \, \cdots, \, y_{d_9})^{\mathrm{T}}, Z_d = (z_{d_1}, \, z_{d_2}, \, \cdots, \, z_{d_8})^{\mathrm{T}}.
\]

根据以上分析,建立数学模型:
\[
\min \sum_{i=1}^{11} (x_{ai} + x_{bi} + x_{ci} + x_{di} + x_{ei}) + \sum_{i=1}^{9} (y_{ai} + y_{bi} + y_{ci} + y_{di} + y_{ei}) + \sum_{i=1}^{8} (z_{ai} + z_{bi} + z_{ci} + z_{di} + z_{ei})
\]
\[
\text{s.t. } A \times \begin{pmatrix}
X_a & X_b & X_c & X_d & X_e \\
Y_a & Y_b & Y_c & Y_d & Y_e \\
Z_a & Z_b & Z_c & Z_d & Z_e
\end{pmatrix} \geq \begin{pmatrix}
108 & 76 & 70 & 56 & 63 \\
88 & 61 & 44 & 55 & 43
\end{pmatrix}
\]

其中,$X_e, X_a, X_b, X_c, X_d, \, Y_e, Y_a, Y_b, Y_c, Y_d, Z_e, Z_a, Z_b, Z_c, Z_d$ 为非负整数向量。

使用 lingo 软件编程运行得到的结果为

\begin{table}
\centering
\begin{tabular}{|c|c|c|c|}
\hline
 & 1-1-3 & 1-1-4 & 1-1-1 \\ \hline
E & $x_{10}=1$, $x_{11}=2$ & $y_{1}=6$ & $z_{18}=2$ \\ \hline
A & $x_{10}=1$, $x_{11}=3$ & $y_{2}=10$ & $z_{8}=7$ \\ \hline
B & $x_{1}=2$, $x_{11}=4$ & $y_{1}=5$, $y_{9}=1$ & $z_{8}=2$ \\ \hline
C & $x_{11}=2$ & $y_{1}=7$, $y_{9}=3$ & \\ \hline
D & $x_{1}=3$, $x_{11}=4$ & $y_{1}=2$, $y_{9}=1$ & $z_{8}=5$ \\ \hline
总共使用 & 22 & 35 & 12 \\ \hline
剩余 & 0 & 0 & 9 \\ \hline
\end{tabular}
\end{table}

从表中可以得到,使用了22辆1-1-3型轿运车,35辆1-1-4型,12辆1-1-1型。此时1-1-3型和1-1-4型轿运车已经全部被使用;1-1-1车辆还剩余9辆。

\textbf{模型二}

针对第二类乘用车的运载,考虑所有的装载方式如下表:

\begin{table}
\centering
\begin{tabular}{|c|c|c||c|c|c|}
\hline
\multicolumn{3}{|c||}{1-2型下层} & \multicolumn{3}{c|}{1-1-2型下层} \\ \hline
I & II & III & I & II & III \\ \hline
5 & 0 & 0 & 4 & 0 & 0 \\ \hline
4 & 1 & 0 & 3 & 1 & 0 \\ \hline
2 & 2 & 0 & 1 & 2 & 0 \\ \hline
1 & 3 & 0 & 0 & 3 & 0 \\ \hline
0 & 4 & 0 & 2 & 0 & 1 \\ \hline
3 & 0 & 1 & 1 & 1 & 1 \\ \hline
2 & 1 & 1 & 0 & 2 & 1 \\ \hline
1 & 2 & 1 & 1 & 0 & 2 \\ \hline
0 & 3 & 1 & 0 & 1 & 2 \\ \hline
1 & 0 & 2 & & & \\ \hline
0 & 1 & 2 & & & \\ \hline
\end{tabular}
\end{table}

依据情况二中乘用车装在1-2型、1-1-2型轿运车下层的可行性方案,列出其装载矩阵B。其中,运往A、B、C、D、E的车辆数量分别为26辆、22辆、43辆、35辆、33辆。为了方便运算可将此类轿用车简化成115辆I型乘用车,46辆II型乘用车,4辆III型乘用车。

设

\begin{equation}
\mathbf{B} = 
\begin{pmatrix}
5 & 4 & 2 & 1 & 0 & 3 & 2 & 1 & 0 & 1 & 0 & 4 & 3 & 1 & 0 & 2 & 1 & 0 & 1 & 0 \\
0 & 1 & 2 & 3 & 4 & 0 & 1 & 2 & 3 & 0 & 1 & 0 & 1 & 2 & 3 & 0 & 1 & 2 & 0 & 1 \\
0 & 0 & 0 & 0 & 0 & 1 & 1 & 1 & 1 & 2 & 2 & 0 & 0 & 0 & 0 & 1 & 1 & 1 & 2 & 2
\end{pmatrix}
\end{equation}

\begin{align*}
\mathbf{U}_e &= (x_{e_1}, \, x_{e_2}, \ldots, \, x_{e_{11}})^\mathrm{T}, \quad \mathbf{W}_e = (y_{e_1}, \, y_{e_2}, \ldots, \, y_{e_9})^\mathrm{T}, \\
\mathbf{U}_a &= (x_{a_1}, \, x_{a_2}, \ldots, \, x_{a_{11}})^\mathrm{T}, \quad \mathbf{W}_a = (y_{a_1}, \, y_{a_2}, \ldots, \, y_{a_9})^\mathrm{T}, \\
\mathbf{U}_b &= (x_{b_1}, \, x_{b_2}, \ldots, \, x_{b_{11}})^\mathrm{T}, \quad \mathbf{W}_b = (y_{b_1}, \, y_{b_2}, \ldots, \, y_{b_9})^\mathrm{T}, \\
\mathbf{U}_c &= (x_{c_1}, \, x_{c_2}, \ldots, \, x_{c_{11}})^\mathrm{T}, \quad \mathbf{W}_c = (y_{c_1}, \, y_{c_2}, \ldots, \, y_{c_9})^\mathrm{T}, \\
\mathbf{U}_d &= (x_{d_1}, \, x_{d_2}, \ldots, \, x_{d_{11}})^\mathrm{T}, \quad \mathbf{W}_d = (y_{d_1}, \, y_{d_2}, \ldots, \, y_{d_9})^\mathrm{T}.
\end{align*}

根据以上分析,建立数学模型:

\begin{equation}
\min \sum_{i=1}^{11} (u_{ai} + u_{bi} + u_{ci} + u_{di} + u_{ei}) + \sum_{j=1}^{9} (w_{aj} + w_{bj} + w_{cj} + w_{dj} + w_{ej})
\end{equation}

\begin{equation}
\text{s.t.} \quad \mathbf{B} \times 
\begin{pmatrix}
\mathbf{U}_a & \mathbf{U}_b & \mathbf{U}_c & \mathbf{U}_d & \mathbf{U}_e \\
\mathbf{W}_a & \mathbf{W}_b & \mathbf{W}_c & \mathbf{W}_d & \mathbf{W}_e
\end{pmatrix} \geq 
\begin{pmatrix}
10 & 17 & 31 & 30 & 22 \\
14 & 5 & 12 & 5 & 10 \\
2 & 0 & 0 & 1 & 1
\end{pmatrix}
\end{equation}

\begin{equation}
5 \sum_{i=1}^{11} (U_{ai} + U_{bi} + U_{ci} + U_{di} + U_{ei}) \leq \sum_{j=1}^{9} (W_{aj} + W_{bj} + W_{cj} + W_{dj} + W_{ej}) + 78
\end{equation}

\begin{equation}
\mathbf{U}_a, \, \mathbf{U}_b, \, \mathbf{U}_c, \, \mathbf{U}_d, \, \mathbf{U}_e, \, \mathbf{W}_a, \, \mathbf{W}_b, \, \mathbf{W}_c, \, \mathbf{W}_d, \, \mathbf{W}_e \text{为非负整数向量。}
\end{equation}

使用 lingo 软件编程得到的结果为:

\begin{table}[h]
\centering
\begin{tabular}{|c|c|c|}
\hline
 & 1-2 下层 & 1-1-2 下层 \\ \hline
E & $x_2 = 1, \, x_9 = 1$ & $y_2 = 6$ \\ \hline
A & $x_2 = 1, \, x_5 = 1$ & $y_2 = 2, \, y_4 = 1, \, y_7 = 2$ \\ \hline
B & $x_2 = 2$ & $y_2 = 3$ \\ \hline
C & $x_2 = 7, \, x_5 = 1$ & $y_2 = 1$ \\ \hline
D & $x_1 = 4, \, x_2 = 1$ & $y_2 = 2, \, y_7 = 1$ \\ \hline
总共使用 & 19 辆单层 & 18 辆单层 \\ \hline
\end{tabular}
\end{table}

从表中可以得到,此类情况使用了 19 辆 1-2 型轿运车;使用了 18 辆 1-1-2 型轿运车。

\section*{模型三}

针对第三类乘用车的运载,考虑所有的装载方式如下表:

\begin{table}
\centering
\begin{tabular}{|c|c|c|c|c|c|c|c|}
\hline
\multicolumn{2}{|c|}{1-2 型上层} & \multicolumn{2}{c|}{1-1-2 型上层} & \multicolumn{2}{c|}{1-1-1 型} & \multicolumn{2}{c|}{1-1-2 型} \\
\hline
I & II & I & II & I & II & I & II \\
\hline
12 & 0 & 4 & 0 & 10 & 0 & 8 & 0 \\
\hline
10 & 1 & 3 & 1 & 7 & 1 & 7 & 1 \\
\hline
9 & 2 & 2 & 2 & 6 & 2 & 6 & 2 \\
\hline
8 & 3 & 1 & 3 & 5 & 3 & 5 & 3 \\
\hline
7 & 4 & 0 & 4 & 4 & 4 & 4 & 4 \\
\hline
5 & 5 & & & 3 & 5 & 3 & 5 \\
\hline
4 & 6 & & & 2 & 6 & 2 & 6 \\
\hline
3 & 7 & & & 1 & 7 & 1 & 7 \\
\hline
2 & 8 & & & 0 & 8 & 0 & 8 \\
\hline
1 & 9 & & & & & & \\
\hline
0 & 10 & & & & & & \\
\hline
\end{tabular}
\end{table}

因为前两类情形的车受特殊要求限制,在装车的过程中都浪费了很大的空间资源,而此类情形下的乘用车车长较短,装载位置的任意性相对较高,可先填充在上层空置的 1-2 型和 1-1 系列的轿运车上层,然后根据长优先原则选择 1-1-1 型轿运车装载,最后考虑 1-1-2 型轿运车装载。

下面先来定义决策变量,以 OE 途径为例:

放在 1-2 型上层,按照第 \(i\) 个方案的轿运车数量为 \(v_{ei}\);

放在 1-1-2 型上层,按照第 \(j\) 个方案的轿运车为 \(t_{ek}\);

放在 1-1-1 型,按照第 \(k\) 个方案的轿运车数量为 \(T_{ek}\);

放在 1-1-2 型,按照第 \(v\) 个方案的轿运车数量为 \(r_{ev}\);

\[
V_{e} = (v_{e1}, \, v_{e2}, \, \ldots, \, v_{e11})^{\text{T}}, \, S_{e} = (s_{e1}, \, s_{e2}, \, \ldots, \, s_{e5})^{\text{T}},
\]
\[
T_{a} = (t_{e1}, \, t_{e2}, \, \ldots, \, t_{e9})^{\text{T}}, \, R_{e} = (r_{e1}, \, r_{e2}, \, \ldots, \, r_{e9})^{\text{T}},
\]
\[
V_{a}, \ldots, V_{d}, S_{a}, \ldots, S_{d}, T_{a}, \ldots, R_{a}, \ldots, R_{d} \text{ 类似定义}
\]

装载方案可由矩阵 \(C\)、\(D\) 表示:
\[
C = \begin{pmatrix}
12 & 10 & 9 & 8 & 7 & 5 & 4 & 3 & 2 & 1 & 0 & 4 & 3 & 2 & 1 & 0 \\
0 & 1 & 2 & 3 & 4 & 5 & 6 & 7 & 8 & 9 & 10 & 0 & 1 & 2 & 3 & 4
\end{pmatrix}
\]
\[
D = \begin{pmatrix}
10 & 7 & 6 & 5 & 4 & 3 & 2 & 1 & 0 & 8 & 7 & 6 & 5 & 4 & 3 & 2 & 1 & 0 \\
0 & 1 & 2 & 3 & 4 & 5 & 6 & 7 & 8 & 0 & 1 & 2 & 3 & 4 & 5 & 6 & 7 & 8
\end{pmatrix}
\]

因此可建立相应整数规划模型:
\[
\min \sum_{i=1}^{11} (v_{ai} + v_{bi} + v_{ci} + v_{di} + v_{ei}) + \sum_{i=1}^{9} (t_{ai} + t_{bi} + t_{ci} + t_{di} + t_{ei}) + \sum_{i=1}^{5} (s_{ai} + s_{bi} + s_{ci} + s_{di} + s_{ei}) + \sum_{i=1}^{9} (r_{ai} + r_{bi} + r_{ci} + r_{di} + r_{ei})
\]

\begin{equation}
\begin{aligned}
& \left(C \quad D\right)_{2 \times 34} \begin{pmatrix}
V_a & V_b & V_c & V_d & V_e \\
S_a & S_b & S_c & S_d & S_e \\
T_a & T_b & T_c & T_d & T_e \\
R_a & R_b & R_c & R_d & R_e
\end{pmatrix} \geq \begin{pmatrix}
18 & 21 & 59 & 44 & 34 \\
52 & 31 & 36 & 44 & 37
\end{pmatrix} \\
& \sum_{i=1}^{11} \left( \nu_{ai} + \nu_{bi} + \nu_{ci} + \nu_{di} + \nu_{ei} \right) = 19 \\
& \sum_{j=1}^{5} \left( S_{aj} + S_{bj} + S_{cj} + S_{dj} + S_{ej} \right) = 18 \\
& \sum_{k=1}^{9} \left( t_{ak} + t_{bk} + t_{ck} + t_{dk} + t_{ek} \right) \leq 9 \\
& \sum_{l=1}^{9} \left( r_{al} + r_{bl} + r_{cl} + r_{dl} + r_{el} \right) \leq 25 \\
& V_a, \ldots, V_e, S_a, \ldots, S_e, T_a, \ldots, T_e, R_a, \ldots, R_d \text{ 均为非负整数向量}
\end{aligned}
\end{equation}

使用 lingo 软件编程得到的结果为:

\begin{table}[h]
\centering
\begin{tabular}{|c|c|c|c|c|}
\hline
 & 1-2 上层 & 1-1-2 上层 & 1-1-1 型 & 1-1-2 型 \\
\hline
E & $w_4 = 1$, $w_5 = 1$, $w_{11} = 3$ & & $x_1 = 2$ & \\
\hline
A & $w_1 = 1$, $w_4 = 1$, $w_{11} = 5$ & & & \\
\hline
B & $w_{11} = 2$ & $y_5 = 1$ & $x_1 = 2$ & $z_8 = 1$ \\
\hline
C & $w_1 = 2$, $w_{11} = 1$ & $y_5 = 6$ & $x_1 = 3$ & $z_3 = 1$ \\
\hline
D & $w_1 = 2$ & $y_5 = 11$ & $x_1 = 2$ & \\
\hline
总共使用 & 19 辆单层 & 18 辆单层 & 9 辆 & 2 辆 \\
\hline
\end{tabular}
\end{table}

从表中可以看出,第二种情形出现的单层在第三种情形中完全被填充完,即运输 45 各型号的车使用了 19 辆 1-2 型车,22 辆 1-1-3 型车,35 辆 1-1-4 型车,21 辆 1-1-1 型车,使用了 18+2=20 辆 1-1-2 型车,其中仅 1-1-2 型车没有使用完。

\subsection*{4.5.3 结果分析}

通过对上述三个模型的求解,我们得出了最优的可行解。完成这次运输任务,物流公司共需要 117 轿运车辆(21 辆 1-1-1 型轿运车,20 辆 1-1-2 型轿运车,22 辆 1-1-3 型轿运车,35 辆 1-1-4 型轿运车,19 辆 1-2 型轿运车)。在模型一、二、三中,我们定义的所有 I、II、III 型乘用车均是以该类别中车长的上限为该型车的车长,且同一类型乘用车装载时遵循长短结合的原则,因而实际装运过程中必有空缺车位产生。在上述三类

模型最优解对应方案的基础上可结合人工调度,可以得到更好的装运方案。具体装运方案见附件 2 的 EXCEL 文件。

\section*{五、模型评价与拓展}

本文针对题目前四问建立了相应的数学规划模型,利用 lingo 软件编程求解,得出了最优解,且列举了最优装载方案。

对于问题五,基于常规调度思想和调度经验,采用优先考虑远程和优先选择长车的原则,将问题分类讨论,简化问题的同时得到了相对较好的可行解。但由于轿运车类型各异,乘用车种类繁多,我们的简化分类难免会造成轿运车的装载资源的浪费,可以继续考虑其它的启发式算法来进一步优化现有的可行解。

\section*{参考文献}

[1] 俞兵. 汽车维修市场发展模式研究[J]. 汽车零部件,2011,9:78-79

[2] 李晶皎. 模式识别(第四版)[M]. 北京:清华大学出版社,2009.

[3] 司秀丹. 铁路小汽车物流发展研究[D]. 北京邮电人学,2007,27(5):61-62

[4] 温莹莹. 基于智能优化算法的运输及优化问题研究,北京交通大学 2013,9:78-79.

[5] 李炜. 物流过程的建模和优化方法,2010,(1):109-118.

[6] 司秀丹. 我国乘用车运输与物流服务市场分析[D]. 北京交通人学,2007

\begin{table}
\centering
\caption{1-1型轿运车载车方案}
\begin{tabular}{c|c|c|c|c|c|c|c|c}
\hline
上层摆放车辆 & \multicolumn{3}{c|}{下层摆放车辆} & \multicolumn{3}{c|}{总载车辆数} & 载车方案序号 \\
\hline
I & II & I & II & III & I & II & III \\
\hline
4 & 0 & 4 & 0 & 0 & 8 & 0 & 0 & 1 \\
\hline
3 & 1 & 4 & 0 & 0 & 7 & 1 & 0 & 2 \\
\hline
4 & 0 & 3 & 1 & 0 & & & & \\
\hline
2 & 2 & 4 & 0 & 0 & 6 & 2 & 0 & 3 \\
\hline
3 & 1 & 3 & 1 & 0 & & & & \\
\hline
4 & 0 & 2 & 2 & 0 & & & & \\
\hline
1 & 3 & 4 & 0 & 0 & 5 & 3 & 0 & 4 \\
\hline
2 & 2 & 3 & 1 & 0 & & & & \\
\hline
3 & 1 & 2 & 2 & 0 & & & & \\
\hline
4 & 0 & 1 & 3 & 0 & & & & \\
\hline
0 & 5 & 4 & 0 & 0 & 4 & 5 & 0 & 5 \\
\hline
4 & 0 & 0 & 5 & 0 & & & & \\
\hline
0 & 5 & 3 & 1 & 0 & 3 & 6 & 0 & 6 \\
\hline
3 & 1 & 0 & 5 & 0 & & & & \\
\hline
0 & 5 & 2 & 2 & 0 & 2 & 7 & 0 & 7 \\
\hline
2 & 2 & 0 & 5 & 0 & & & & \\
\hline
0 & 5 & 1 & 3 & 0 & 1 & 8 & 0 & 8 \\
\hline
1 & 3 & 0 & 5 & 0 & & & & \\
\hline
0 & 5 & 0 & 5 & 0 & 0 & 10 & 0 & 9 \\
\hline
4 & 0 & 3 & 0 & 1 & 7 & 0 & 1 & 10 \\
\hline
4 & 0 & 2 & 1 & 1 & 6 & 1 & 1 & 11 \\
\hline
3 & 1 & 3 & 0 & 1 & & & & \\
\hline
4 & 0 & 1 & 2 & 1 & 5 & 2 & 1 & 12 \\
\hline
3 & 1 & 2 & 1 & 1 & & & & \\
\hline
2 & 2 & 3 & 0 & 1 & & & & \\
\hline
4 & 0 & 0 & 3 & 1 & 4 & 3 & 1 & 13 \\
\hline
3 & 1 & 1 & 2 & 1 & & & & \\
\hline
2 & 2 & 2 & 1 & 1 & & & & \\
\hline
1 & 3 & 3 & 0 & 1 & & & & \\
\hline
0 & 5 & 3 & 0 & 1 & 3 & 5 & 1 & 14 \\
\hline
0 & 5 & 2 & 1 & 1 & 2 & 6 & 1 & 15 \\
\hline
0 & 5 & 1 & 2 & 1 & 1 & 7 & 1 & 16 \\
\hline
0 & 5 & 0 & 3 & 1 & 0 & 8 & 1 & 17 \\
\hline
4 & 0 & 2 & 0 & 2 & 6 & 0 & 2 & 18 \\
\hline
3 & 1 & 2 & 0 & 2 & 5 & 1 & 2 & 19 \\
\hline
4 & 0 & 1 & 1 & 2 & & & & \\
\hline
4 & 0 & 0 & 2 & 2 & 4 & 2 & 2 & 20 \\
\hline
2 & 2 & 2 & 0 & 2 & & & & \\
\hline
\end{tabular}
\end{table}

\begin{table}
\centering
\begin{tabular}{|c|c|c|c|c|c|c|c|c|c|}
\hline
3 & 1 & 1 & 1 & 2 &  &  &  &  &  \\
\hline
3 & 1 & 0 & 2 & 2 & \multirow{3}{*}{3} & \multirow{3}{*}{3} & \multirow{3}{*}{2} & \multicolumn{2}{c|}{21} \\
\hline
1 & 3 & 2 & 0 & 2 &  &  &  & \multicolumn{2}{c|}{} \\
\hline
2 & 2 & 1 & 1 & 2 &  &  &  & \multicolumn{2}{c|}{} \\
\hline
0 & 5 & 2 & 0 & 2 & 2 & 5 & 2 & 22 &  \\
\hline
0 & 5 & 1 & 1 & 2 & 1 & 6 & 2 & 23 &  \\
\hline
0 & 5 & 0 & 2 & 2 & 0 & 7 & 2 & 24 &  \\
\hline
4 & 0 & 1 & 0 & 3 & 5 & 0 & 3 & 25 &  \\
\hline
4 & 0 & 0 & 1 & 3 & \multirow{2}{*}{4} & \multirow{2}{*}{1} & \multirow{2}{*}{3} & \multicolumn{2}{c|}{26} \\
\hline
3 & 1 & 1 & 0 & 3 &  &  &  & \multicolumn{2}{c|}{} \\
\hline
3 & 1 & 0 & 1 & 3 & \multirow{2}{*}{3} & \multirow{2}{*}{2} & \multirow{2}{*}{3} & \multicolumn{2}{c|}{27} \\
\hline
2 & 2 & 1 & 0 & 3 &  &  &  & \multicolumn{2}{c|}{} \\
\hline
2 & 2 & 0 & 1 & 3 & \multirow{3}{*}{2} & \multirow{3}{*}{3} & \multirow{3}{*}{3} & \multicolumn{2}{c|}{28} \\
\hline
1 & 3 & 1 & 0 & 3 &  &  &  & \multicolumn{2}{c|}{} \\
\hline
0 & 5 & 1 & 0 & 3 & 1 & 5 & 3 & 29 &  \\
\hline
0 & 5 & 0 & 1 & 3 & 0 & 6 & 3 & 30 &  \\
\hline
4 & 0 & 0 & 0 & 4 & 4 & 0 & 4 & 31 &  \\
\hline
3 & 1 & 0 & 0 & 4 & 3 & 1 & 4 & 32 &  \\
\hline
2 & 2 & 0 & 0 & 4 & 2 & 2 & 4 & 33 &  \\
\hline
1 & 3 & 0 & 0 & 4 & 1 & 3 & 4 & 34 &  \\
\hline
0 & 5 & 0 & 0 & 4 & 0 & 5 & 4 & 35 &  \\
\hline
\end{tabular}
\end{table}

\begin{table}
\centering
\begin{tabular}{|c|c|c|c|c|c|c|c|c|}
\hline
\multicolumn{9}{|c|}{1-2型轿运车载车方案} \\
\hline
\multirow{2}{*}{上层摆放车辆数} & \multicolumn{3}{c|}{下层摆放车辆数} & \multicolumn{3}{c|}{总载车辆数} & \multirow{2}{*}{载车方案序号} \\
\cline{2-8}
 & I & II & III & I & II & III &  \\
\hline
10 & 0 & 4 & 0 & 1 & 14 & 0 & 1 & 1 \\
\hline
10 & 0 & 3 & 1 & 1 & 13 & 1 & 1 & 2 \\
\hline
10 & 0 & 2 & 2 & 1 & 12 & 2 & 1 & 3 \\
\hline
8 & 2 & 4 & 0 & 1 & 12 & 2 & 1 & 3 \\
\hline
10 & 0 & 1 & 4 & 1 & 11 & 4 & 1 & 4 \\
\hline
10 & 0 & 0 & 5 & 1 & 10 & 5 & 1 & 5 \\
\hline
8 & 2 & 1 & 4 & 1 & 9 & 6 & 1 & 6 \\
\hline
4 & 8 & 4 & 0 & 1 & 8 & 8 & 1 & 7 \\
\hline
4 & 8 & 3 & 1 & 1 & 7 & 9 & 1 & 8 \\
\hline
4 & 8 & 2 & 2 & 1 & 6 & 10 & 1 & 9 \\
\hline
4 & 8 & 1 & 4 & 1 & 5 & 12 & 1 & 10 \\
\hline
4 & 8 & 0 & 5 & 1 & 4 & 13 & 1 & 11 \\
\hline
2 & 10 & 1 & 4 & 1 & 3 & 14 & 1 & 12 \\
\hline
2 & 10 & 0 & 5 & 1 & 2 & 15 & 1 & 13 \\
\hline
0 & 12 & 1 & 4 & 1 & 1 & 16 & 1 & 14 \\
\hline
0 & 12 & 0 & 5 & 1 & 0 & 17 & 1 & 15 \\
\hline
\end{tabular}
\end{table}

\begin{table}
\centering
\begin{tabular}{|c|c|c|c|c|c|c|c|c|}
\hline
10 & 0 & 3 & 0 & 2 & 13 & 0 & 2 & 16 \\
\hline
10 & 0 & 2 & 1 & 2 & 12 & 1 & 2 & 17 \\
\hline
10 & 0 & 1 & 2 & 2 & 11 & 2 & 2 & 18 \\
\hline
8 & 2 & 3 & 0 & 2 &  &  &  &  \\
\hline
10 & 0 & 0 & 4 & 2 & 10 & 4 & 2 & 19 \\
\hline
8 & 2 & 0 & 4 & 2 & 8 & 6 & 2 & 20 \\
\hline
4 & 8 & 3 & 0 & 2 & 7 & 8 & 2 & 21 \\
\hline
4 & 8 & 2 & 1 & 2 & 6 & 9 & 2 & 22 \\
\hline
4 & 8 & 1 & 2 & 2 & 5 & 10 & 2 & 23 \\
\hline
2 & 10 & 3 & 0 & 2 &  &  &  &  \\
\hline
4 & 8 & 0 & 4 & 2 & 4 & 12 & 2 & 24 \\
\hline
2 & 10 & 0 & 4 & 2 & 2 & 14 & 2 & 25 \\
\hline
0 & 12 & 0 & 4 & 2 & 0 & 16 & 2 & 26 \\
\hline
10 & 0 & 2 & 0 & 3 & 12 & 0 & 3 & 27 \\
\hline
10 & 0 & 1 & 1 & 3 & 11 & 1 & 3 & 28 \\
\hline
10 & 0 & 0 & 2 & 3 & 10 & 2 & 3 & 29 \\
\hline
8 & 2 & 2 & 0 & 3 &  &  &  &  \\
\hline
8 & 2 & 1 & 1 & 3 & 9 & 3 & 3 & 30 \\
\hline
8 & 2 & 0 & 2 & 3 & 8 & 4 & 3 & 31 \\
\hline
6 & 4 & 2 & 0 & 3 &  &  &  &  \\
\hline
6 & 4 & 1 & 1 & 3 & 7 & 5 & 3 & 32 \\
\hline
4 & 8 & 2 & 0 & 3 & 6 & 8 & 3 & 33 \\
\hline
4 & 8 & 1 & 1 & 3 & 5 & 9 & 3 & 34 \\
\hline
4 & 8 & 0 & 2 & 3 & 4 & 10 & 3 & 35 \\
\hline
2 & 10 & 2 & 0 & 3 &  &  &  &  \\
\hline
2 & 10 & 1 & 1 & 3 & 3 & 11 & 3 & 36 \\
\hline
2 & 10 & 0 & 2 & 3 & 2 & 12 & 3 & 37 \\
\hline
0 & 12 & 2 & 0 & 3 &  &  &  &  \\
\hline
0 & 12 & 1 & 1 & 3 & 1 & 13 & 3 & 38 \\
\hline
0 & 12 & 0 & 2 & 3 & 0 & 14 & 3 & 39 \\
\hline
10 & 0 & 1 & 0 & 4 & 11 & 0 & 4 & 40 \\
\hline
10 & 0 & 0 & 1 & 4 & 10 & 1 & 4 & 41 \\
\hline
8 & 2 & 1 & 0 & 4 & 9 & 2 & 4 & 42 \\
\hline
8 & 2 & 0 & 1 & 4 & 8 & 3 & 4 & 43 \\
\hline
6 & 4 & 1 & 0 & 4 & 7 & 4 & 4 & 44 \\
\hline
6 & 4 & 0 & 1 & 4 & 6 & 5 & 4 & 45 \\
\hline
4 & 8 & 1 & 0 & 4 & 5 & 8 & 4 & 46 \\
\hline
4 & 8 & 0 & 1 & 4 & 4 & 9 & 4 & 47 \\
\hline
2 & 10 & 1 & 0 & 4 & 3 & 10 & 4 & 48 \\
\hline
2 & 10 & 0 & 1 & 4 & 2 & 11 & 4 & 49 \\
\hline
0 & 12 & 1 & 0 & 4 & 1 & 12 & 4 & 50 \\
\hline
0 & 12 & 0 & 1 & 4 & 0 & 13 & 4 & 51 \\
\hline
10 & 0 & 0 & 0 & 5 & 10 & 0 & 5 & 52 \\
\hline
\end{tabular}
\end{table}

\begin{table}
\centering
\begin{tabular}{|c|c|c|c|c|c|c|c|c|}
\hline
8 & 2 & 0 & 0 & 5 & 8 & 2 & 5 & 53 \\
\hline
6 & 4 & 0 & 0 & 5 & 6 & 4 & 5 & 54 \\
\hline
4 & 8 & 0 & 0 & 5 & 4 & 8 & 5 & 55 \\
\hline
2 & 10 & 0 & 0 & 5 & 2 & 10 & 5 & 56 \\
\hline
0 & 12 & 0 & 0 & 5 & 0 & 12 & 5 & 57 \\
\hline
10 & 0 & 5 & 0 & 0 & 15 & 0 & 0 & 58 \\
\hline
10 & 0 & 4 & 1 & 0 & 14 & 1 & 0 & 59 \\
\hline
8 & 2 & 5 & 0 & 0 & 13 & 2 & 0 & 60 \\
\hline
10 & 0 & 3 & 2 & 0 & & & & \\
\hline
10 & 0 & 2 & 4 & 0 & 12 & 4 & 0 & 61 \\
\hline
10 & 0 & 1 & 5 & 0 & 11 & 5 & 0 & 62 \\
\hline
8 & 2 & 2 & 4 & 0 & 10 & 6 & 0 & 63 \\
\hline
10 & 0 & 0 & 6 & 0 & & & & \\
\hline
4 & 8 & 5 & 0 & 0 & 9 & 8 & 0 & 64 \\
\hline
4 & 8 & 4 & 1 & 0 & 8 & 9 & 0 & 65 \\
\hline
2 & 10 & 5 & 0 & 0 & 7 & 10 & 0 & 66 \\
\hline
4 & 8 & 3 & 2 & 0 & & & & \\
\hline
4 & 8 & 2 & 4 & 0 & 6 & 12 & 0 & 67 \\
\hline
4 & 8 & 1 & 5 & 0 & 5 & 13 & 0 & 68 \\
\hline
2 & 10 & 2 & 4 & 0 & 4 & 14 & 0 & 69 \\
\hline
2 & 10 & 1 & 5 & 0 & 3 & 15 & 0 & 70 \\
\hline
0 & 12 & 2 & 4 & 0 & 2 & 16 & 0 & 71 \\
\hline
2 & 10 & 0 & 6 & 0 & & & & \\
\hline
0 & 12 & 1 & 5 & 0 & 1 & 17 & 0 & 72 \\
\hline
0 & 12 & 0 & 6 & 0 & 0 & 18 & 0 & 73 \\
\hline
\end{tabular}
\end{table}

\section*{附录 2}

\section*{问题一源代码:}

\begin{verbatim}
model:
!乘用车运输问题1;
sets:
    cyc/1..2/: shumu;
    jyc1/1..9/: a1,b1,x;
    jyc2/1..16/: a2,b2,y;
endsets
!这里是数据;
data:
    c1,c2=?,?;
    shumu=100,68;
    a1 b1=8 0,7 1,6 2,5 3,4 5,3 6,2 7,1 8,0 10;
    a2 b2=15 0,14 1,13 2,12 4,11 5,10 6,9 8,8 9,7 10,6 12,5 13,4 14,3 15,2 16,1 17,0 18;
enddata
!目标函数;
    min=c1*@sum(jyc1: x)+c2*@sum(jyc2: y);
!需求约束;
    @sum(jyc1: x*a1)+@sum(jyc2: y*a2)>shumu(1);
    @sum(jyc1: x*b1)+@sum(jyc2: y*b2)>shumu(2);
!比较约束;
    0.2*@sum(jyc1: x)>@sum(jyc2: y);
!整数约束;
    @for(jyc1(i):@gin(x(i)));
    @for(jyc2(j):@gin(y(j)));
end
\end{verbatim}

\section*{问题二源代码:}

\begin{verbatim}
model:
!乘用车运输问题2;
sets:
    cyc/1..2/: shumu;
    jyc1/1..5/: a1,b1,x;
    jyc2/1..6/: a2,b2,y;
endsets
!这里是数据;
data:
    c1,c2=?,?;
    shumu=72,52;
    a1 b1=10 0,8 1,7 2,6 3,5 4;
\end{verbatim}

\begin{verbatim}
a2 b2=18 0,17 1,16 2,14 3,13 4,12 5;
enddata
!目标函数;
min=c1*@sum(jyc1:x)+c2*@sum(jyc2:y);
!需求约束;
@sum(jyc1:x*a1)+@sum(jyc2:y*a2)>shumu(1);
@sum(jyc1:x*b1)+@sum(jyc2:y*b2)>shumu(2);
!比较约束;
0.2*@sum(jyc1:x)>@sum(jyc2:y);
!整数约束;
@for(jyc1(i):@gin(x(i)));
@for(jyc2(j):@gin(y(j)));
end

问题三源代码:
model:
!乘用车运输问题3;
sets:
cyc/1..3/:shumu;
jyc1/1..35/:a1,b1,c1,x;
jyc2/1..73/:a2,b2,c2,y;
endsets
!这里是数据;
data:
cost1 cost2=?,?;
shumu=156,102,39;
a1 b1 c1=8 0 0,7 1 0,6 2 0,5 3 0,4 5 0,3 6 0,2 7 0,1 8 0,0 10 0,7 0 1,6 1 1,5 2 1,4 3 1,3 5 1,2 6
1,1 7 1,0 8 1,
6 0 2,5 1 2,4 2 2,3 3 2,2 5 2,1 6 2,0 7 2,5 0 3,4 1 3,3 2 3,2 3 3,1 5 3,0 6 3,4 0 4,3 1 4,2 2
4,1 3 4,0 5 4;
a2 b2 c2=14 0 1,13 1 1,12 2 1,11 4 1,10 5 1,9 6 1,8 8 1,7 9 1,6 10 1,5 12 1,4 13 1,3 14 1,2 15
1,1 16 1,0 17 1,
13 0 2,12 1 2,11 2 2,10 4 2,8 6 2,7 8 2,6 9 2,5 10 2,4 12 2,2 14 2,0 16 2,12 0 3,11 1
3,10 2 3,9 3 3,
8 4 3,7 5 3,6 8 3,5 9 3,4 10 3,3 11 3,2 12 3,1 13 3,0 14 3,11 0 4,10 1 4,9 2 4,8 3 4,7 4
4,6 5 4,
5 8 4,4 9 4,3 10 4,2 11 4,1 12 4,0 13 4,10 0 5,8 2 5,6 4 5,4 8 5,2 10 5,0 12 5,15 0 0,14 1
0,13 2 0,
12 4 0,11 5 0,10 6 0,9 8 0,8 9 0,7 10 0,6 12 0,5 13 0,4 14 0,3 15 0,2 16 0,1 17 0,0 18 0;
enddata
!目标函数;
min=cost1*@sum(jyc1:x)+cost2*@sum(jyc2:y);
!需求约束;
\end{verbatim}

\begin{verbatim}
@sum(jyc1:x*a1)+@sum(jyc2: y*a2)>shumu(1);
@sum(jyc1:x*b1)+@sum(jyc2: y*b2)>shumu(2);
@sum(jyc1:x*c1)+@sum(jyc2: y*c2)>shumu(3);
!比较约束;
0.2*@sum(jyc1: x)>@sum(jyc2: y);
!整数约束;
@for(jyc1(i):@gin(x(i)));
@for(jyc2(j):@gin(y(j)));
end
\end{verbatim}

\textbf{问题五源代码:}

\textbf{模型一:}

\textbf{model:}
!乘用车运输问题5.1;

\textbf{sets:}
cyc/1..10/: shumu;
jyc1/1..11/: a1,b1,x1,x2,x3,x4,x5;
jyc2/1..9/: a2,b2,y1,y2,y3,y4,y5;
jyc3/1..8/: a3,b3,z1,z2,z3,z4,z5;
endsets

!这里是数据;
data:
shumu=63,43,108,88,76,61,70,44,56,55;
a1=@ole(aa,'SHU5');
b1=@ole(aa,'SHU6');
a2=@ole(aa,'SHU7');
b2=@ole(aa,'SHU8');
a3=@ole(aa,'SHU1');
b3=@ole(aa,'SHU2');
enddata

!目标函数;
min=@sum(jyc1: x1)+@sum(jyc2: y1)+@sum(jyc3: z1)+@sum(jyc1: x2)+@sum(jyc2: y2)+@sum(jyc3: z2)+@sum(jyc1: x3)+@sum(jyc2: y3)+@sum(jyc3: z3)+@sum(jyc1: x4)+@sum(jyc2: y4)+@sum(jyc3: z4)+@sum(jyc1: x5)+@sum(jyc2: y5)+@sum(jyc3: z5);

!需求约束;
@sum(jyc1:x1*a1)+@sum(jyc2: y1*a2)+@sum(jyc3: z1*a3)>shumu(1);
@sum(jyc1:x1*b1)+@sum(jyc2: y1*b2)+@sum(jyc3: z1*b3)>shumu(2);
@sum(jyc1:x2*a1)+@sum(jyc2: y2*a2)+@sum(jyc3: z2*a3)>shumu(3);
@sum(jyc1:x2*b1)+@sum(jyc2: y2*b2)+@sum(jyc3: z2*b3)>shumu(4);
@sum(jyc1:x3*a1)+@sum(jyc2: y3*a2)+@sum(jyc3: z3*a3)>shumu(5);
@sum(jyc1:x3*b1)+@sum(jyc2: y3*b2)+@sum(jyc3: z3*b3)>shumu(6);
\end{verbatim}

\begin{verbatim}
@sum(jyc1:x4*a1)+@sum(jyc2: y4*a2)+@sum(jyc3: z4*a3)>shumu(7);
@sum(jyc1:x4*b1)+@sum(jyc2: y4*b2)+@sum(jyc3: z4*b3)>shumu(8);
@sum(jyc1:x5*a1)+@sum(jyc2: y5*a2)+@sum(jyc3: z5*a3)>shumu(9);
@sum(jyc1:x5*b1)+@sum(jyc2: y5*b2)+@sum(jyc3: z5*b3)>shumu(10);
!比较约束;
@sum(jyc1: x1)+@sum(jyc1: x2)+@sum(jyc1: x3)+@sum(jyc1: x4)+@sum(jyc1: x5)=22;
@sum(jyc2: y1)+@sum(jyc2: y2)+@sum(jyc2: y3)+@sum(jyc2: y4)+@sum(jyc2: y5)=35;
@sum(jyc3: z1)+@sum(jyc3: z2)+@sum(jyc3: z3)+@sum(jyc3: z4)+@sum(jyc3: z5)<21;
!整数约束;
@for(jyc1(i):@gin(x1(i)));
@for(jyc2(j):@gin(y1(j)));
@for(jyc3(k):@gin(z1(k)));
@for(jyc1(i):@gin(x2(i)));
@for(jyc2(j):@gin(y2(j)));
@for(jyc3(k):@gin(z2(k)));
@for(jyc1(i):@gin(x3(i)));
@for(jyc2(j):@gin(y3(j)));
@for(jyc3(k):@gin(z3(k)));
@for(jyc1(i):@gin(x4(i)));
@for(jyc2(j):@gin(y4(j)));
@for(jyc3(k):@gin(z4(k)));
@for(jyc1(i):@gin(x5(i)));
@for(jyc2(j):@gin(y5(j)));
@for(jyc3(k):@gin(z5(k)));
end

模型二:

model:
!乘用车运输问题5.2;
sets:
    cyc/1..15/: shumu;
    jyc1/1..11/: a1,b1,c1,x1,x2,x3,x4,x5,!1-2型下层;
    jyc2/1..9/: a2,b2,c2,y1,y2,y3,y4,y5,!1-1-2型下层;
endsets
!这里是数据;
data:
shumu=22,10,1,10,14,2,17,5,0,31,12,0,30,5,1;
a1=@ole(aa,'SHU9');
b1=@ole(aa,'SHU10');
c1=@ole(aa,'SHU11');
a2=@ole(aa,'SHU12');
b2=@ole(aa,'SHU13');
\end{verbatim}

\begin{verbatim}
c2=@ole(aa,'SHU14');
enddata
!目标函数;
min=@sum(jyc1: x1)+@sum(jyc2: y1)+@sum(jyc1: x2)+@sum(jyc2: y2)+@sum(jyc1: x3)+@sum(jyc2: y3)+@sum(jyc1: x4)+@sum(jyc2: y4)+@sum(jyc1: x5)+@sum(jyc2: y5);
!需求约束;
@sum(jyc1: x1*a1)+@sum(jyc2: y1*a2) > shumu(1);
@sum(jyc1: x1*b1)+@sum(jyc2: y1*b2) > shumu(2);
@sum(jyc1: x1*c1)+@sum(jyc2: y1*c2) > shumu(3);
@sum(jyc1: x2*a1)+@sum(jyc2: y2*a2) > shumu(4);
@sum(jyc1: x2*b1)+@sum(jyc2: y2*b2) > shumu(5);
@sum(jyc1: x2*c1)+@sum(jyc2: y2*c2) > shumu(6);
@sum(jyc1: x3*a1)+@sum(jyc2: y3*a2) > shumu(7);
@sum(jyc1: x3*b1)+@sum(jyc2: y3*b2) > shumu(8);
@sum(jyc1: x3*c1)+@sum(jyc2: y3*c2) > shumu(9);
@sum(jyc1: x4*a1)+@sum(jyc2: y4*a2) > shumu(10);
@sum(jyc1: x4*b1)+@sum(jyc2: y4*b2) > shumu(11);
@sum(jyc1: x4*c1)+@sum(jyc2: y4*c2) > shumu(12);
@sum(jyc1: x5*a1)+@sum(jyc2: y5*a2) > shumu(13);
@sum(jyc1: x5*b1)+@sum(jyc2: y5*b2) > shumu(14);
@sum(jyc1: x5*c1)+@sum(jyc2: y5*c2) > shumu(15);
!比较约束;
5*(@sum(jyc1: x1)+@sum(jyc1: x2)+@sum(jyc1: x3)+@sum(jyc1: x4)+@sum(jyc1: x5)) < @sum(jyc2: y1)+@sum(jyc2: y2)+@sum(jyc2: y3)+@sum(jyc2: y4)+@sum(jyc2: y5)+78;
!整数约束;
@for(jyc1(i):@gin(x1(i)));
@for(jyc2(j):@gin(y1(j)));
@for(jyc1(i):@gin(x2(i)));
@for(jyc2(j):@gin(y2(j)));
@for(jyc1(i):@gin(x3(i)));
@for(jyc2(j):@gin(y3(j)));
@for(jyc1(i):@gin(x4(i)));
@for(jyc2(j):@gin(y4(j)));
@for(jyc1(i):@gin(x5(i)));
@for(jyc2(j):@gin(y5(j)));
end
\end{verbatim}

模型三:

model:

\begin{verbatim}
!乘用车运输问题5.3;
sets:
    cyc/1..10/: shumu;
    jyc1/1..9/: a1,b1,x1,x2,x3,x4,x5;
    jyc2/1..5/: a2,b2,y1,y2,y3,y4,y5;
    jyc3/1..9/: a3,b3,z1,z2,z3,z4,z5;
    jyc4/1..11/: a4,b4,w1,w2,w3,w4,w5;
endsets
!这里是数据;
data:
    shumu=34,37,18,52,21,31,59,36,44,44;
    a1=@ole(aa,'SHU19');
    b1=@ole(aa,'SHU20');
    a2=@ole(aa,'SHU17');
    b2=@ole(aa,'SHU18');
    a3=@ole(aa,'SHU21');
    b3=@ole(aa,'SHU22');
    a4=@ole(aa,'SHU15');
    b4=@ole(aa,'SHU16');
enddata
!目标函数;
    min=@sum(jyc1: x1)+@sum(jyc2: y1)+@sum(jyc3: z1)+@sum(jyc4: w1)
    +@sum(jyc1: x2)+@sum(jyc2: y2)+@sum(jyc3: z2)+@sum(jyc4: w2)
    +@sum(jyc1: x3)+@sum(jyc2: y3)+@sum(jyc3: z3)+@sum(jyc4: w3)
    +@sum(jyc1: x4)+@sum(jyc2: y4)+@sum(jyc3: z4)+@sum(jyc4: w4)
    +@sum(jyc1: x5)+@sum(jyc2: y5)+@sum(jyc3: z5)+@sum(jyc4: w5);
!需求约束;
    @sum(jyc1:x1*a1)+@sum(jyc2: y1*a2)+@sum(jyc3: z1*a3)+@sum(jyc4: w1*a4)>shumu(1);
    @sum(jyc1:x1*b1)+@sum(jyc2: y1*b2)+@sum(jyc3: z1*b3)+@sum(jyc4:
    w1*b4)>shumu(2);
    @sum(jyc1:x2*a1)+@sum(jyc2: y2*a2)+@sum(jyc3: z2*a3)+@sum(jyc4: w2*a4)>shumu(3);
    @sum(jyc1:x2*b1)+@sum(jyc2: y2*b2)+@sum(jyc3: z2*b3)+@sum(jyc4:
    w2*b4)>shumu(4);
    @sum(jyc1:x3*a1)+@sum(jyc2: y3*a2)+@sum(jyc3: z3*a3)+@sum(jyc4: w3*a4)>shumu(5);
    @sum(jyc1:x3*b1)+@sum(jyc2: y3*b2)+@sum(jyc3: z3*b3)+@sum(jyc4:
    w3*b4)>shumu(6);
    @sum(jyc1:x4*a1)+@sum(jyc2: y4*a2)+@sum(jyc3: z4*a3)+@sum(jyc4: w4*a4)>shumu(7);
    @sum(jyc1:x4*b1)+@sum(jyc2: y4*b2)+@sum(jyc3: z4*b3)+@sum(jyc4:
    w4*b4)>shumu(8);
    @sum(jyc1:x5*a1)+@sum(jyc2: y5*a2)+@sum(jyc3: z5*a3)+@sum(jyc4: w5*a4)>shumu(9);
    @sum(jyc1:x5*b1)+@sum(jyc2: y5*b2)+@sum(jyc3: z5*b3)+@sum(jyc4:
    w5*b4)>shumu(10);
!比较约束;
    @sum(jyc1: x1)+@sum(jyc1: x2)+@sum(jyc1: x3)+@sum(jyc1: x4)+@sum(jyc1:
\end{verbatim}

\begin{verbatim}
x5)<9; !1-1-1型;
@sum(jyc2: y1)+@sum(jyc2: y2)+@sum(jyc2: y3)+@sum(jyc2: y4)+@sum(jyc2: y5)=18; !1-1-2型上层;
@sum(jyc4: w1)+@sum(jyc4: w2)+@sum(jyc4: w3)+@sum(jyc4: w4)+@sum(jyc4: w5)=19; !1-2型上层;
!整数约束;
@for(jyc1(i):@gin(x1(i)));
@for(jyc2(j):@gin(y1(j)));
@for(jyc3(k):@gin(z1(k)));
@for(jyc4(t):@gin(w1(t)));
@for(jyc1(i):@gin(x2(i)));
@for(jyc2(j):@gin(y2(j)));
@for(jyc3(k):@gin(z2(k)));
@for(jyc4(t):@gin(w2(t)));
@for(jyc1(i):@gin(x3(i)));
@for(jyc2(j):@gin(y3(j)));
@for(jyc3(k):@gin(z3(k)));
@for(jyc4(t):@gin(w3(t)));
@for(jyc1(i):@gin(x4(i)));
@for(jyc2(j):@gin(y4(j)));
@for(jyc3(k):@gin(z4(k)));
@for(jyc4(t):@gin(w4(t)));
@for(jyc1(i):@gin(x5(i)));
@for(jyc2(j):@gin(y5(j)));
@for(jyc3(k):@gin(z5(k)));
@for(jyc4(t):@gin(w5(t)));
end
\end{verbatim}