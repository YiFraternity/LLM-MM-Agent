\begin{center}
\textbf{第十一届华为杯全国研究生数学建模竞赛}
\end{center}

\begin{table}[h]
\centering
\begin{tabular}{l l}
学校 & 华东师范大学 \\
\hline
参赛队号 & 10269049 \\
\hline
队员姓名 & 1. 牛姝媛 \\
 & 2. 田欣 \\
 & 3. 侯超 \\
\hline
\end{tabular}
\end{table}

\begin{flushright}
参赛密码 \underline{\hspace{3cm}} \\
(由组委会填写)
\end{flushright}

\begin{center}
\includegraphics[width=0.3\textwidth]{image1.png} \quad
\includegraphics[width=0.3\textwidth]{image2.png} \quad
\includegraphics[width=0.3\textwidth]{image3.png}
\end{center}

\section*{第十一届华为杯全国研究生数学建模竞赛}

\section*{题目 小鼠视觉感受区电位信号(LFP)与视觉刺激之间的关系研究}

\section*{摘 要:}

为了研究视觉感受区局部电位的变化是否与视觉刺激相关,以及其变化规律与呼吸曲线周期性间的关联。本文对小鼠睡眠状态下和清醒状态下(有、无视觉刺激)的脑局部电位信号进行了分析和分离,找出了视觉感受区局部电位的变化与视觉刺激之间的相关性。

依据生物学原理,动作的起始与结束是由神经系统通过电信号来控制的。在这一基础上我们提出了在呼吸动作的转折点(呼气与吸气末)脑电波会出现波动的假设。由此得到了脑电波的两点分布模型:动作中无呼吸相关脑电波,动作末有脑电波信号。然后我们使用小鼠睡眠状态下的脑电波与呼吸功率曲线数据对我们的模型进行的试验验证。[4] 首先通过 python 代码分析出呼吸的各个动作变化的时间点。然后将时间点前后的脑电波提取出来,使用 T 检验对这一区段的脑电波进行差异的显著性检验。从而验证了我们模型的合理性。

问题三中首先通过自相关和互相关模型初步确认了清醒状态下小鼠 LFP 变化具有周期性,并且与呼吸所对应的脑电波周期性相关。然后又通过对数据进行规律化震荡处理后建立了 DFA 模型,经过最小二乘法拟合进一步确认了 LFP 变化的周期性,即 $F(t) \propto t^{0.1041}$。并判断了 LFP 周期性与呼吸所对应的脑电波周期性变化是相关的,但不是线性相关。

问题四的实质是在已知呼吸相关的脑电波模型的情况下,通过对多个位点测量到的脑电波进行分析,研究脑电波中的独立成分,分离出与呼吸相关的脑电波。论文中建立了 ICA 模型,使用 Infomax 的方式,通过对总信息熵以及各成分间的互信息的计算,使用循环迭代的方法得到最优的独立成分分解。并通过对照实验

的方法,比较优化了分解结果。寻求最优的脑电波的分离模型。

问题五运用周期图法的改进 Welch 法对功率谱计算分析,确定了模型的应用可以扩展到视觉刺激相关脑电波信号的分离上。通过对多条件下对照数据的整合分析,结合 ICA 算法分离出了视觉相关的脑电波。在此基础上对视觉相关的脑电波进行了更深一步的研究,探究了脑电波内在周期性的机理以及脑电波与外界刺激的关联性。在探究过程中,我们运用假设检验模型,并且使用 T-test,通过傅立叶变换,自相关性检验以及互相关检验等方法对数据进行分析处理。阐明了小鼠的视觉感受区点位信号与视觉刺激间的关联。

关键词:脑电波、相关性分析、DFA、独立成分分析

\section*{目录}

\begin{itemize}
    \item 问题重述 \dotfill 1
    \item 模型假设 \dotfill 1
    \item 模型建立与求解 \dotfill 1
    \item 问题一 分析与求解 \dotfill 1
    \begin{itemize}
        \item 1.1 睡眠状态下与呼吸相关的脑波建模 \dotfill 1
        \item 1.2 建模结果 \dotfill 2
    \end{itemize}
    \item 问题二分析与求解 \dotfill 3
    \begin{itemize}
        \item 2.1 问题分析 \dotfill 3
        \item 2.2 模型建立与 T 检验 \dotfill 3
        \item 2.3 T 检验结果分析 \dotfill 5
    \end{itemize}
    \item 问题三分析与求解 \dotfill 5
    \begin{itemize}
        \item 3.1 自相关与互相关模型 \dotfill 5
        \item 3.2 DFA 模型 \dotfill 7
    \end{itemize}
    \item 问题四分析与求解 \dotfill 11
    \begin{itemize}
        \item 4.1 问题分析 \dotfill 11
        \item 4.2 ICA 模型 \dotfill 12
    \end{itemize}
    \item 问题五分析与求解 \dotfill 16
    \begin{itemize}
        \item 5.1 视觉刺激下时间曲线及其功率谱、呼吸曲线功率谱 \dotfill 16
        \item 5.2 利用 ICA 模型分离脑电波成分 \dotfill 18
        \item 5.3 与视觉刺激相关脑电波成分与图形形状因素分析 \dotfill 24
    \end{itemize}
    \item 参考文献 \dotfill 26
    \item 附录 \dotfill 27
\end{itemize}

\section*{问题重述}

脑是支配人和高级动物活动的司令部和信息中心,大脑在活动时产生的电信号就是脑电波(Electroencephalogram, EEG)。除作为脑科学基础理论研究的重要指标外,脑电波也是人们思维活动的体现。大量研究都表明脑电波信号与认知等功能活动有关,一旦解析出脑电波中的信息密码就可以实现人机互动,所以进行脑电波分析的基础研究具有重要意义。而局部场电位(Local Field Potential, LFP)则反映来自神经元网络局部神经核团的活动状态,所以 LFP 信号是脑内某局部大量神经元树突电位和的综合反映。LFP 可能与大脑对行为的控制有关。由于人脑的复杂性,所以选取的研究对象为从进化论角度上具有相同脑机理的小鼠做实验。而脑电波信号是无数神经放电的混合,不同行为产生的电信号的强弱差别也很大,所以通常的盲源分离技术误差太大,很难将脑电波较好的分离出来并与脑电波相对应。所以我们需要通过建立数学模型,来分析脑电波与行为之间的联系。

问题一:对于小鼠睡眠状态下的大脑视觉皮层局部电位,通过分析呼吸的机理,建立数学模型反映小鼠在睡眠状态下与呼吸相关联的脑电波。

问题二:在睡眠状态下,通过对所给的视觉感受区的局部电位数据建立模型,反映小鼠脑电波的周期节律是否与呼吸相关联。

问题三:在清醒状态下,分析小鼠视觉感受区的局部电位信号变化是否具有周期性?该周期性的变化是否与小鼠呼吸所对应的脑电波的周期性的变化有关?并分析相关性形式。

问题四:建立脑电波信号的分离模型,从睡眠状态和无视觉刺激的清醒状态的 LFP 信号中分离出与小鼠呼吸相关联的脑电波信号。

问题五:通过 Checkboard 随时间变化的曲线,分析小鼠视觉刺激的时间曲线及其功率谱,并与呼吸曲线的功率谱对比。从有视觉刺激的清醒状态下的脑电波数据中分离出与 Checkboard 刺激相关的脑电波信号成份和可能与呼吸相关的脑电波信号成份。并验证所分离出来的刺激脑电波成份与视觉刺激之间的相关性,以及脑电波成份中包含的图形形状因素。

\section*{模型假设}

为了反映睡眠状态下小鼠视觉区域局部电位与呼吸的关系,并为接下来清醒状态下的建模做对照,我们假设小鼠睡眠状态下脑电波只受呼吸影响。

\section*{模型建立与求解}

\section*{问题一 分析与求解}

\subsection*{1.1 睡眠状态下与呼吸相关的脑波建模}

由于呼吸曲线在采样时,信号是通过热敏电极在小鼠鼻子附近记录的,主要表现是呼气时温度升高(曲线下降),因此无法直接通过曲线的幅值反应相关的信息。我们通过分析呼吸机理,建立模型,从呼吸曲线中提取出吸气相和呼

气相的有用信息。

从生物学角度上看,小鼠的呼吸的过程是在神经系统的植物性神经控制下完成的,在呼吸的过程中其主要行为分为呼气和吸气两个阶段,并且这两个阶段是交替循环的。所以我们分别考虑交替循环中呼和吸的过程。在吸气和呼气行为过程中,身体的主要行为是调动胸腔的多个肌肉群运动,我们可以认为在这过程中神经系统分别发出了两次调控信号,控制身体肌肉的状态发生变化。

根据呼吸的生物特性,我们可以把整个过程的脑电波信号视为只考虑在两种状态对应时刻出现异常突起(下文简称 Breakpoint)。Breakpoint 出现的位置可以看作与呼吸功率曲线中的峰值和谷值所在的点相对应。因此我们可以构建一个算法将呼吸功率曲线中的极值点取出,即脑电波的峰和谷出现在呼吸功率曲线的转折处。

\begin{figure}[h]
    \centering
    \includegraphics[width=\textwidth]{image.png}
    \caption{睡眠状态下的呼吸曲线}
    \label{fig:breathing_curve}
\end{figure}

观察睡眠状态下的呼吸功率曲线如图 \ref{fig:breathing_curve},可以发现在曲线中每个峰值之间的大致距离为 500 个采样点左右。因此对于数据中的每一个点,我们将该点与它所在前后 300 个点(共 599 个点)的范围内做比较,如果该点是其所在 599 个点的范围内的最大或者最小值,那么我们可以认为它是一个转折处,并将该点标注为有生理意义的波峰或波谷点。

\subsection{1.2 建模结果}

我们用 Python 语言编写了峰-谷值点波形处理程序(程序代码见附录),在采样频率为 1000HZ 的数据中提取出了 36 个峰值点和 35 个波谷点。提取出的峰值点和谷值点如下表 \ref{tab:peaks_valleys}:

\begin{table}[h]
    \centering
    \caption{睡眠状态下提取出的呼吸波峰点和波谷点}
    \label{tab:peaks_valleys}
    \begin{tabular}{|c|c|c|c|c|c|c|c|}
    \hline
    峰值点 & 谷值点 & 峰值点 & 谷值点 & 峰值点 & 谷值点 & 峰值点 & 谷值点 \\ \hline
    42 & 467 & 577 & 1000 & 1114 & 1531 & 1660 & 2068 \\ \hline
    2232 & 2612 & 2755 & 3123 & 3272 & 3686 & 3784 & 4214 \\ \hline
    4330 & 4782 & 4993 & 5334 & 5455 & 5886 & 6021 & 6450 \\ \hline
    6601 & 7031 & 7130 & 7574 & 8141 & 8250 & 8250 & 8711 \\ \hline
    8846 & 9279 & 9385 & 9838 & 9945 & 10401 & 10503 & 10950 \\ \hline
    11082 & 11535 & 11635 & 12102 & 12201 & 12690 & 12795 & 13286 \\ \hline
    \end{tabular}
\end{table}

\begin{table}
\begin{tabular}{|c|c|c|c|c|c|c|c|}
\hline 13425 & 13885 & 14023 & 14505 & 14619 & 15088 & 15228 & 15655 \\
\hline 15796 & 16257 & 16404 & 16870 & 16999 & 17477 & 17601 & 18052 \\
\hline 18181 & 18667 & 18798 & 19258 & 19396 & 19834 & 19993 & \\
\hline
\end{tabular}
\end{table}

\section*{问题二分析与求解}

\subsection*{2.1 问题分析}

从生物学原理的角度上来看,呼吸是人体的生理体系中十分重要的一个环节,呼吸过程中血液在肺部通过肺泡与外界的空气进行氧气与二氧化碳的气体的交流。氧气的供应会与体内的各个器官与组织结构有关联。所以呼吸过程必然与产生大范围的关联影响。呼吸的过程中神经系统调用了大量的肌肉群来完成呼吸动作。因此脑电波中必然有周期性电信号调节肌肉参与完成呼吸的动作。[1]

通过观察脑电波曲线,如图 2.1 所示,我们可以明显看出,在总体趋势变化下脑电波会在较小的波动范围内有很多的小波峰和波谷。可以看出这些波峰与波谷的时间间隔有着一定的规律性。而这些波峰与波谷的周期性节律很可能与呼吸相关联。[2] 因此我们建立模型,对脑电波周期性节律与呼吸的关联性进行论证。

\begin{figure}[h]
\centering
\includegraphics[width=\textwidth]{image.png}
\caption{小鼠睡眠状态下的脑电波曲线}
\end{figure}

\subsection*{2.2 模型建立与 T 检验}

由于呼吸有明显的周期性,因此我们假设脑电波中存在周期性分量与小鼠的呼吸周期相对应。在每一次呼吸状态发生变化的时刻,因为神经系统对身体的动作产生神经调节,脑电波的状态会产生变化,即在这个时刻应该会有较明显的脑电波变化可以检测到。

所以我们在脑电波信号中与呼吸的 breakpoint 的位置对应点进行检测。若每个 breakpoint 对应位置处的脑电波与该点周围一定范围内(我们选取 0.01s)

\begin{table}[h]
\centering
\begin{tabular}{c|c|c|c|c|c|c|c|c|c|c}
\hline
 & \multicolumn{5}{c|}{36个峰值点 breakpoint 对应的样本T检验结果} & \multicolumn{5}{c}{35个谷值点 breakpoint 对应的样本T检验结果} \\
\hline
 & Ch11 & Ch12 & Ch13 & Ch14 & Ch15 & Ch11 & Ch12 & Ch13 & Ch14 & Ch15 \\
\hline
1 & 3.41E-07 & 9.02E-09 & 0.57362 & 4.01E-09 & 1.16E-07 & 1.68E-07 & 5.15E-08 & 1.25E-07 & 2.50E-09 & 0.154603 \\
\hline
2 & 5.74E-10 & 3.91E-07 & 3.54E-06 & 2.21E-07 & 6.79E-08 & 0.57025 & 2.13E-09 & 6.36E-07 & 1.00E-07 & 6.09E-06 \\
\hline
3 & 9.48E-07 & 6.28E-08 & 2.13E-05 & 0.26142 & 5.68E-06 & 3.64E-07 & 1.89E-07 & 6.73E-07 & 1.54E-05 & 9.41E-10 \\
\hline
4 & 0.633468 & 6.80E-07 & 5.22E-08 & 1.33E-07 & 0.035642 & 4.93E-07 & 1.76E-07 & 4.87E-06 & 2.03E-06 & 0.051945 \\
\hline
5 & 3.75E-07 & 8.64E-07 & 8.17E-09 & 1.81E-06 & 2.02E-08 & 0.000115 & 4.07E-09 & 0.54938 & 3.80E-10 & 2.36E-05 \\
\hline
6 & 2.32E-08 & 7.65E-07 & 1.34E-07 & 5.84E-07 & 0.089421 & 3.90E-08 & 0.49207 & 4.85E-08 & 4.28E-07 & 4.71E-07 \\
\hline
7 & 2.11E-07 & 2.03E-07 & 3.93E-09 & 7.73E-07 & 3.79E-07 & 1.52E-11 & 2.16E-07 & 1.41E-08 & 1.22E-05 & 6.00E-07 \\
\hline
8 & 3.54E-08 & 0.088532 & 1.78E-06 & 7.66E-09 & 3.64E-08 & 0.512473 & 6.16E-08 & 3.97E-08 & 0.01052 & 3.38E-08 \\
\hline
9 & 1.05E-08 & 2.57E-05 & 2.06E-09 & 1.28E-09 & 3.02E-08 & 1.33E-06 & 1.29E-10 & 3.20E-11 & 0.106534 & 1.02E-05 \\
\hline
10 & 6.94E-06 & 0.000139 & 5.93E-08 & 2.69E-07 & 3.32E-06 & 8.40E-09 & 9.07E-07 & 1.60E-07 & 9.85E-08 & 2.42E-06 \\
\hline
11 & 0.28716 & 7.00E-05 & 1.34E-07 & 1.05563 & 1.65E-06 & 2.58E-08 & 0.79406 & 1.39E-07 & 2.14E-07 & 1.13E-05 \\
\hline
12 & 9.25E-08 & 1.77E-07 & 0.000124 & 1.05E-08 & 2.69E-09 & 3.93E-08 & 2.83E-10 & 2.56E-07 & 2.74E-08 & 5.82E-07 \\
\hline
13 & 6.04E-08 & 5.14E-07 & 4.80E-06 & 2.18E-06 & 3.09E-07 & 7.47E-07 & 5.25E-06 & 0.69451 & 2.30E-07 & 3.09E-08 \\
\hline
14 & 6.08E-07 & 6.81E-08 & 1.33E-08 & 5.90E-07 & 1.79E-06 & 9.90E-09 & 2.96E-08 & 1.69E-05 & 3.17E-09 & 1.91E-06 \\
\hline
15 & 4.25E-07 & 2.15E-09 & 0.63540 & 1.12E-07 & 2.26E-07 & 2.44E-08 & 0.10387 & 7.05E-08 & 3.13E-09 & 7.61E-07 \\
\hline
16 & 0.577952 & 2.87E-09 & 1.77E-06 & 9.60E-08 & 4.24E-08 & 4.66E-09 & 2.63E-07 & 2.16E-07 & 4.29E-11 & 3.33E-06 \\
\hline
17 & 3.98E-05 & 0.054187 & 9.77E-05 & 8.37E-06 & 0.88064 & 7.63E-05 & 8.71E-06 & 8.49E-06 & 2.17E-06 & 1.66E-07 \\
\hline
18 & 1.82E-05 & 8.09E-09 & 2.01E-09 & 1.21E-05 & 7.75E-07 & 2.25E-06 & 6.96E-07 & 0.199302 & 8.78E-08 & 0.30007 \\
\hline
19 & 1.24E-07 & 5.54E-05 & 3.67E-06 & 2.44E-07 & 9.38E-08 & 3.66E-07 & 8.37E-10 & 2.94E-08 & 3.32E-08 & 1.53E-07 \\
\hline
\end{tabular}
\caption{T检验结果}
\end{table}

\begin{table}
\centering
\begin{tabular}{|c|c|c|c|c|c|c|c|c|c|c|}
\hline
20 & 2.44E-07 & 2.84E-06 & 0.298564 & 7.22E-05 & 2.08E-05 & 4.13E-06 & 1.92E-07 & 3.52E-06 & 2.34E-09 & 1.95E-06 \\
\hline
21 & 2.58E-07 & 1.24E-07 & 3.34E-06 & 1.06E-06 & 2.71E-06 & 1.45E-05 & 6.44E-06 & 6.53E-07 & 4.07E-10 & 1.34E-05 \\
\hline
22 & 7.82E-05 & 5.60E-05 & 1.39E-07 & 4.80E-05 & 1.11E-07 & 0.751246 & 2.69E-09 & 1.28E-10 & 6.56E-08 & 4.57E-07 \\
\hline
23 & 1.29E-09 & 3.04E-09 & 9.47E-09 & 2.84E-09 & 0.552103 & 4.77E-06 & 3.91E-09 & 1.14E-08 & 9.77E-08 & 7.91E-07 \\
\hline
24 & 0.559635 & 0.00017 & 1.76E-07 & 1.65E-06 & 1.01E-06 & 1.27E-09 & 1.67E-08 & 5.47E-08 & 0.635298 & 1.70E-07 \\
\hline
25 & 1.71E-06 & 0.090694 & 2.35E-08 & 0.928536 & 1.06E-07 & 7.33E-06 & 4.45E-07 & 3.65E-06 & 1.94E-08 & 3.47E-07 \\
\hline
26 & 3.71E-08 & 4.79E-07 & 1.10E-07 & 5.02E-08 & 4.22E-08 & 2.69E-08 & 0.074486 & 2.34E-08 & 1.06E-10 & 6.69E-11 \\
\hline
27 & 7.16E-07 & 5.72E-07 & 3.58E-07 & 2.76E-05 & 0.809534 & 7.67E-06 & 7.64E-08 & 5.48E-08 & 6.69E-10 & 3.66E-07 \\
\hline
28 & 6.74E-05 & 1.70E-05 & 3.78E-05 & 1.26E-09 & 9.78E-08 & 4.28E-11 & 4.28E-10 & 3.07E-10 & 1.68E-06 & 4.95E-08 \\
\hline
29 & 4.01E-07 & 0.001216 & 0.000106 & 8.99E-07 & 3.21E-08 & 8.44E-07 & 1.30E-06 & 0.536984 & 1.20E-07 & 1.57E-07 \\
\hline
30 & 1.725120 & 0.263412 & 2.15E-07 & 3.77E-07 & 3.31E-07 & 1.09E-06 & 6.77E-08 & 6.30E-07 & 3.59E-08 & 2.04E-07 \\
\hline
31 & 3.45E-06 & 1.22E-06 & 1.42E-06 & 0.555124 & 1.60E-07 & 0.02562 & 6.55E-08 & 5.93E-08 & 0.000228 & 1.64E-08 \\
\hline
32 & 8.38E-07 & 2.39E-06 & 3.06E-05 & 8.84E-09 & 1.38E-09 & 2.91E-07 & 8.05E-07 & 9.02E-07 & 1.95E-07 & 2.40E-08 \\
\hline
33 & 1.97E-07 & 2.27E-06 & 2.35E-05 & 3.94E-06 & 2.43E-09 & 1.61E-10 & 2.31E-07 & 1.07E-06 & 1.31E-07 & 0.394102 \\
\hline
34 & 2.31E-07 & 0.000109 & 0.07528 & 2.50E-07 & 2.70E-06 & 4.99E-09 & 6.29E-09 & 4.16E-08 & 1.45E-06 & 5.26E-06 \\
\hline
35 & 5.60E-09 & 5.09E-06 & 2.14E-05 & 2.94E-06 & 0.098654 & 7.26E-08 & 1.49E-06 & 1.19E-06 & 1.59E-09 & 1.30E-06 \\
\hline
36 & 8.86E-08 & 7.10E-08 & 3.80E-09 & 7.75E-09 & 9.32E-10 & & & & & \\
\hline
\end{tabular}
\end{table}

自相关是对信号自身相关程度的一种度量,由于小鼠视觉感受区的 LFP 信号没有具体的函数形式。因此在衡量信号变化的周期性中,求一个信号的自相关函数(Autocorrelation Function)可以看成是信号与自身的延迟信号进行乘积的卷积运算。

\begin{equation}
\mathbf{R}_{f}(\tau)=f(\tau)f^{*}(-\tau)\int_{-\infty}^{\infty}f(t+\tau)f^{*}(t)\mathrm{d}t
\tag{式 3-1}
\end{equation}

若信号自身变化具有周期性,则其对应的自相关函数也会具有明显的周期性变化且与源信号变化的周期相同。我们利用公式,对源信号进行自相关函数的计算,得到了自相关曲线图如图所示

\begin{figure}[h]
\centering
\includegraphics[width=\textwidth]{image.png}
\caption{小鼠清醒状态下 LFP 源信号的自相关函数曲线图}
\label{fig:3.1}
\end{figure}

\subsection{3.1.2 互相关模型}

互相关(Cross correlation)函数是对两个信号 \(X(t), Y(t)\) 之间相关程度的度量。若两个信号相似性很大,则互相关函数曲线的峰值会很明显。

\begin{equation}
\mathbf{R}_{12}(\tau)=\int_{-\infty}^{\infty}f_{1}^{*}f_{2}(t+\tau)\mathrm{d}t
\tag{式 3-2}
\end{equation}

我们对小鼠视觉 LFP 信号及呼吸所对应的脑电波信号进行了互相关分析,画出互相关函数曲线如图 3.2 所示

\begin{figure}[h]
\centering
\includegraphics[width=\textwidth]{image.png}
\caption{互相关函数曲线图}
\label{fig:3.2}
\end{figure}

\begin{figure}[h]
    \centering
    \includegraphics[width=\textwidth]{image.png}
    \caption{小鼠视觉 LFP 信号及呼吸所对应的脑电波信号进行了互相关曲线图}
    \label{fig:3.2}
\end{figure}

\subsection{结果分析}

从自相关曲线图 3.1 中可以看出,曲线虽能看出较明显的周期性,但是曲线的离散型大,较粗糙,方差较大,周期性效果分辨率较低。因此我们进一步采用 DFA 方法,验证小鼠视觉感受区的局部电位信号变化的周期性。

从互相关曲线图 3.2 中我们可以看出,互相关函数曲线存在较明显的峰值,因此,可以说明小鼠视觉 LFP 信号与呼吸所对应的脑电波信号存在相关性,但相关性的形式不能判断,我们通过 DFA 建模,进一步判断出了两种信号的相关形式。[3, 4]

\subsection{DFA 模型}

\subsubsection{问题分析}

由于脑电波(EEG)是大脑活动时,脑皮质细胞群之间形成的电位差,具有自发性、节律性等特征,并且按照时间标度进行记录,所以脑电波数据可以看做时间序列信号。脑电波数据来源于置于皮下 5 毫米处的电极,得到的信号数据振幅是一组非线性非周期性的复杂时间序列。

由下图 3.3 可以看出脑电波信号具有强烈的不规则性和不稳定性,分析其周期性变化比较复杂。DFA(Detrended Fluctuation Analysis, DFA)方法是一种研究信号涨落的方法,可以用于处理非平稳信号,例如 DNA 序列、气象信号、湍流速度和温度场等非稳态信号的时间长程相关性分析。所以本文引进 DFA 方法用于小鼠脑电波周期性变化研究。

\begin{figure}[h]
    \centering
    \includegraphics[width=\textwidth]{image1.png}
    \caption{清醒状态下脑电波源信号图}
    \label{fig:3.3}
\end{figure}

\subsection{DFA 建模过程}

首先在总时间区间 30000 中按照 $[1:k]$(其中 $k=1, 2, \cdots 30000$)为区间将所有数据点对应到 30000 个新区间从 $t=[1:1]$ 到 $t=[1:10000]$。

\begin{equation}
y(k) = \sum_{i=1}^{k} [B_i - B_{ave}]
\tag{式 3-3}
\end{equation}

式 3-3 中 $B_i$ 为时间点 $i$ 时刻的 LFP 源信号数据,$B_{ave}$ 为时间点 $1 \sim T$ 的数据的算术平均值,最后得到各区间波动规律化震荡处理(数据震荡处理程序见附录)后的数据 $y(k)$。该步骤得到数据集 $y(k)$ 下面画出了 $y(k) \sim k$ 的关系图如图 3.4

\begin{figure}[h]
    \centering
    \includegraphics[width=\textwidth]{image2.png}
    \caption{FP 源信号规律化震荡处理后的信号图}
    \label{fig:3.4}
\end{figure}

将总时间区间 $T$ 以长度 $t$ 为单位分为若干个子区间,对于每个区间的数据进行最小二乘法拟合,求出每个子区间的拟合曲线。由于数据较多,为避免区间长度过大导致曲线欠拟合,这里选取 $t$ 以 3000 为长度单位,将规律化后的源信号共划分出 10 个区间,拟合曲线方程记为 $y_i(k)$。

\begin{table}
\centering
\caption{10个子区间的最小二乘法拟合结果}
\begin{tabular}{|c|c|c|}
\hline
区间 & 对于$t$的拟合系数 & 常数项 \\
\hline
1 & 39.06 & -27660 \\
\hline
2 & -51.179 & 68761.854 \\
\hline
3 & -33.88 & 96446.38 \\
\hline
4 & 54.56 & -89560 \\
\hline
5 & 88.04 & -10320 \\
\hline
6 & -60.75 & 84668.113 \\
\hline
7 & -4.8347 & 61957.5651 \\
\hline
8 & -42.4704 & 84462.4532 \\
\hline
9 & 10.738 & 53560.595 \\
\hline
10 & 74.015 & -47306.503 \\
\hline
\end{tabular}
\end{table}

对于指定单位$t$的每个子区间,
\begin{equation}
F(t) = \sqrt{\frac{1}{T} \sum_{k=1}^{T} [y(k) - y_i(k)]^2}
\tag{式3-4}
\end{equation}
其中$y(k)$与$y_i(k)$的差表示涨落,$t$越大涨落越大。分别计算其$F(t)$,在双对数坐标中画出$\log(F(t)) \sim \log(t)$图。从图中分析$F(t)$与$t$的关系:
\begin{equation}
F(t) \propto t^\alpha
\tag{式3-5}
\end{equation}
其中$\alpha$称为标度指数,它体现了该序列的相关特性,即将$\log(F(t)) \sim \log(t)$图形进行最小二乘法拟合求出直线的斜率。若$\alpha = 1/2$,即
\begin{equation}
\log(F(t)) \sim \log(t)^{(1/2)}
\end{equation}
则说明原始序列不存在时间相关性;若$\alpha \neq \frac{1}{2}$,则说明原始序列具有幂函数性质的时间相关性,并且当$\alpha > 1/2$,说明存在正相关性;若$\alpha < 1/2$,说明存在反相关性。

对于LFP中的Ch11我们得到的$\log(F(t)) \sim \log(t)$曲线图如图3.5,

\begin{figure}[h]
    \centering
    \includegraphics[width=\textwidth]{image.png}
    \caption{LFP-Ch11 脑电波数据 DFA 方法计算结果双对数坐标图}
    \label{fig:3.5}
\end{figure}

\subsection{结果分析及结论}

\subsubsection{周期性判断结果及结论分析}

从图 \ref{fig:3.5} 中可以看出,对于 LFP-Ch11 数据,其 $\log(F(t)) \sim \log(t)$ 在 10 个子区间上都有相同的变化趋势。故子区间之间表现出来相关关系,进一步对数据 $(\log(F(t)), \log(t))$ 进行拟合分析,拟合系数结果如图 \ref{fig:3.6} 所示。得到指数 $\alpha = 0.1041 < 0.5$,说明原数据与时间存在负关联,即源信号时间序列 $x(t) \sim t^{0.1041}$,即清醒状态下脑电波与时间存在周期性关系,但与时间并不是简单线性相关,而是幂函数相关。(DFA 模型程序实现详见附录)。

\begin{table}[h]
    \centering
    \begin{tabular}{l r r r r r}
        \textbf{Coefficients:} & \textbf{Estimate} & \textbf{Std. Error} & \textbf{t value} & \textbf{Pr(>|t|)} \\
        \hline
        (Intercept) & 10.487659 & 0.016905 & 620.37 & $<2e-16$ & *** \\
        $t[1097:3000, 1]$ & 0.104095 & 0.002227 & 46.74 & $<2e-16$ & *** \\
        \hline
    \end{tabular}
    \caption{小鼠 LFP 信号 $(\log(F(t)), \log(t))$ 拟合结果}
    \label{fig:3.6}
\end{table}

我们用 DFA 方法对清醒状态下的其余四条脑电波局部信号序列做了相同的计算,均可以得到类似的结果。这说明了在清醒状态下,小鼠视觉感受区的局部电位信号的变化规律是具有周期性的。

\subsubsection{两种脑电波周期性变化相关性分析结果}

我们对小鼠呼吸所对应的脑电波也用同样方法进行 DFA 建模,得到了十个子区间的 DFA 双对数坐标 $\log(F(t)) \sim \log(t)$ 结果如图 \ref{fig:3.7} 所示。

\begin{figure}[h]
    \centering
    \includegraphics[width=\textwidth]{image2.png}
    \caption{小鼠呼吸脑电波 DFA 双对数坐标图}
    \label{fig:3.7}
\end{figure}

\begin{figure}[h]
    \centering
    \includegraphics[width=\textwidth]{image.png}
    \caption{小鼠呼吸所对应脑电波 DFA 计算结果双对数坐标图}
    \label{fig:3.7}
\end{figure}

从图中我们可以清楚的看到在 10 个区间中,尽管各个子区间数据所在区域相差较大,但各子区间总体的变化趋势相同。对于这 10 个子区间的数据进行线性拟合,得到了拟合系数 $a=0.518904>0.5$,说明呼吸曲线原始数据与时间具有正相关,即 $x(t) \sim t^{0.518904}$。

由 3.2.3.1 中的结论我们已知小鼠 LFP 信号的 $F(t) \sim t$ 关系为:
\begin{equation}
F(t) \propto t^{0.1041}
\end{equation}

而小鼠呼吸所对应的脑电波的 $F(t) \sim t$ 的关系为:
\begin{equation}
F(t) \propto t^{0.5189}
\end{equation}

由于两种信号的采样频率相同,且我们采用 DFA 建模过程中都分成了十个子区间。故两种波的计算结果有比较性。在相同的划分区间内都得到了拟合结果,故两种波是相关的。由结论可以明显看出两种波不是线性相关的,呈幂指数相关。

\section{问题四分析与求解}

\subsection{问题分析}

从脑电波的形成的原理上考虑,大脑中存在着大量的位置、强度和频率各不相同的电波。这些波经过不同程度的衰减后在测量点上互相叠加形成最终测量到的脑电波波谱。在测量的波普中,各种成分的脑电波的大小是发生过改变的,但是其基本的波形属性如周期,峰值的大小比例是不会改变的。因此只要有足够多的变化以及测量的精度,是可以把脑电波的成分分离出来的。因此我们建立 ICA(独立成分分析方法)模型来解决这一问题。[6-8]

\section*{4.2 ICA 模型}

首先我们先对信号的传递假设一个模型:

\begin{equation}
x = As \tag{式 4-1}
\end{equation}

这个模型表示出了观察到的数据是如何由独立成分混合而产生的。其中 \(x\) 是我们观测到的信号,而混合矩阵以及独立成分 \(A\) 和 \(s\) 是假设未知的。我们目的就是要估测出这两个量。ICA 模型假设成分之间是独立且非高斯分布的。由于 ICA 模型有多种实现方式。这里我们选用的是信息最大化算法(InfoMax)方式。InfoMax 就是让分离出来的变量之间的互信息量最小化。[9-12]

\begin{equation}
H(X, Y) = H(X) + H(Y) - I(X, Y) \tag{式 4-2}
\end{equation}

式 4-2 中 \(H(X)\) 是 \(X\) 的信息熵,\(H(X) = -\sum_{x} P(x) \log P(x)\);\(H(Y)\) 是 \(Y\) 的信息熵,\(H(Y) = -\sum_{y} P(y) \log P(y)\);\(I(X, Y)\) 是 \(X\),\(Y\) 的互信息熵,\(I(X, Y) = H(X) - H(X|Y)\) 其中:\(H(X|Y) = H(X, Y) - H(Y)\)。

我们的目标是让分离后的成分之间的 \(H\) 最大而 \(I\) 最小。其具体的步骤为:

1. 首先初始化一个随机矩阵。
2. 根据式 \(W(t+1) = W(t) + \eta(t)(I - f(Y)Y^T)W(t)\) 对矩阵不停的迭代直到收敛。

其中 \(t\) 是一个给定的阈值;\(\eta(t)\) 是一个用来指定矩阵分解过程的步骤的大小的函数,可以设置为常数;\(f(Y)\) 是根据独立成分的性质选择的非线性函数,在 super-Gaussian 情况下,\(f(Y) = Tanh(Y)\);而在 sub-Gaussian 情况下是 \(f(Y) = Y - Tanh(Y)\)。

在前面的问题中我们已经知道了与呼吸相关的脑电波成分的存在。一般情况下,在睡眠状态下以及清醒无刺激的状态下我们能够分离出很多的成分出来。这里就会有呼吸相关联的脑电波。[13, 14] 因此我们首先把睡眠状态下 5 条脑电波进行了独立成分分析,分离后我们得到了四个脑波信号分量,分离出的分量成分效果图如图 4.1 所示:

\begin{figure}[h]
\centering
\includegraphics[width=\textwidth]{image.png}
\caption{分离出的脑波信号分量}
\label{fig:4.1}
\end{figure}

\begin{figure}[h]
    \centering
    \includegraphics[width=\textwidth]{image1.png}
    \caption{睡眠状态下 5 条脑电波独立成分分析结果}
    \label{fig:4.1}
\end{figure}

从图 \ref{fig:4.1} 中可以看出,在独立成分分析的时候,由于其他信号的强度在整个脑电波中占据了更加大的成分比重,分量的频率不能与呼吸的频率相对应,这种情况下呼吸相关的脑电波并不能很好的从其他脑电波分离出来。因此我们把清醒状态下的脑电波也加入进来,通过增加想观测量的数目,增加 ICA 的分离出成分的数目来深化成分的分离。然后我们在这种情况下得到了 9 个脑波信号分量,分离出的分量成分效果图如图 \ref{fig:4.2} 所示:

\begin{figure}[h]
    \centering
    \includegraphics[width=\textwidth]{image2.png}
    \caption{睡眠和清醒状态下共同做 ICA 的成分分离结果}
    \label{fig:4.2}
\end{figure}

我们看到这种情况下呼吸相关的脑电波依然不是主要的成分,因此还是没有被分离出来。仔细观察第二幅图像如图 \ref{fig:4.3} 我们可以发现,其实在大的波形中还是有很多的比较高频的小波没有被分离出来的。我们猜测可能与呼吸相关的脑电波就隐藏在这些小的波形当中。

\begin{figure}[h]
    \centering
    \includegraphics[width=\textwidth]{image3.png}
    \caption{进一步分析图}
    \label{fig:4.3}
\end{figure}

\begin{figure}[h]
    \centering
    \includegraphics[width=\textwidth]{image1.png}
    \caption{图 4.3 睡眠和清醒状态下共同做 ICA 的分离结果第二个成分}
\end{figure}

因此为增强这些小波被筛选出来的水平,我们加入了一个与呼吸的情况高度相关的曲线进来。由于在这条曲线中与呼吸相关的成分是十分明显的主要成分,它必然会被优先考虑,这样这个成分就会把与他相关的脑电波成分在独立成分分析的时候一起分离出来。从而使得与呼吸相关的脑电波能够得到很好的分离。

\begin{figure}[h]
    \centering
    \includegraphics[width=\textwidth]{image2.png}
    \caption{图 4.4 睡眠状态下 ICA 分离成分结果}
\end{figure}

用此方法我们再一次对睡眠状态和清醒无刺激状态分别进行 ICA 分离。得到的睡眠状态下的分离效果如图 4.4,和清醒无刺激状态下的 ICA 分离后的波普图如图 4.5:从图 4.4 中我们可以发现图中的图一在频率上是与呼吸曲线十分一致的。因此我们认为这个就是睡眠状态下的呼吸相关的脑电波的波普图。

\begin{figure}[h]
    \centering
    \includegraphics[width=0.45\textwidth]{image1.png}
    \includegraphics[width=0.45\textwidth]{image2.png}
    \includegraphics[width=0.45\textwidth]{image3.png}
    \includegraphics[width=0.45\textwidth]{image4.png}
    \caption{清醒无刺激状态下的分离成分结果}
    \label{fig:4.5}
\end{figure}

但是,对于图 4.5 我们可以看到清醒状态下分离的效果并不明显。我们认为造成这种分离效果的差异的原因,是由于在清醒状态下脑电波的复杂度有了进一步的提升。仅仅依赖 6 个样本的观测值分成 4 个成分不能很好的满足精确度要求。所以必须要扩大样本量并增加分离的深度。因此接下来我们把睡眠状态下的几个脑电波也加入了进去再一次进行了 ICA 分离,这一次我们得到了脑电波的 9 个成分。ICA 结果如图 4.6 所示:

\begin{figure}[h]
    \centering
    \includegraphics[width=\textwidth]{image5.png}
    \caption{ICA 结果}
    \label{fig:4.6}
\end{figure}

\begin{figure}[h]
    \centering
    \includegraphics[width=\textwidth]{image.png}
    \caption{增加睡眠状态脑电波后清醒状态下 ICA 成分分离结果}
    \label{fig:ica_results}
\end{figure}

从第一幅图我们可以观察到我们成功分离出了与呼吸频率高度吻合的脑电波的波普图。这进一步印证了我们的设想。

\section{问题五分析与求解}

\subsection{视觉刺激下时间曲线及其功率谱、呼吸曲线功率谱}

视皮层是视觉的最高级中枢,各种视觉信息经视觉通路最终都要传导到视皮层,再经皮层的分析与综合产生视觉,视觉刺激会对视觉中枢皮层神经元的场电位等自发电活动产生影响。脑电波来自于大脑内部,一般认为大脑在活动时,脑皮质细胞群之间就会形成电位差,从而在大脑皮质的细胞外产生电流。它是脑神经细胞的电生理活动在大脑皮层或头皮表面的总体反映。而局部场电位(Local Field Potential,LFP)则反映来自神经元网络局部神经核团的活动状态,它也是一种神经集合的协同行为。所以 LFP 信号是脑内某局部大量神经元树突电位和的综合反映。

LFP 可能与大脑对行为的控制有关,如呼吸及视觉刺激等。对应于不同行为或思维的脑电波,我们称之为脑电波成分。事实上,当一个人面对一个物品或需要拿起一个物品时,我们希望知道对应脑电波的反应,即该脑电波成分。该工作具有深远意义,如果能分离出与行为相关联的脑电波,将有助于对大脑疾病的诊断及脑中风病人的生活自理。

本题所有数据来自于视觉刺激实验状态下,信号(LFP:局部电位)分别来自清醒状态下小鼠大脑皮层的视觉感受区,5 个电极(一排)同时记录,每个电极间距为 0.25 毫米。(电极置于皮下 5 毫米处,这样可以减少肌电的影响)。记录视觉感受区局部电位的同时,小鼠的眼睛给了视觉刺激。该刺激是通过配

对出现的 Checkboard,开关的相隔时间是随机的(在小鼠的眼前放置一个 Checkboard,随机地打开或关闭)。

下面根据题目要求分别画出了图 5.1:小鼠视觉刺激的时间曲线,图 5.2:时间曲线的功率谱 [17],图 5.3:呼吸曲线的功率谱。

\begin{figure}[h]
    \centering
    \includegraphics[width=\textwidth]{image1.png}
    \caption{视觉刺激时间曲线}
    \label{fig:5.1}
\end{figure}

\begin{figure}[h]
    \centering
    \includegraphics[width=\textwidth]{image2.png}
    \caption{视觉刺激时间曲线功率谱}
    \label{fig:5.2}
\end{figure}

图 5.1 展示了 checkboard 随时间变化的曲线,可以明显看出 checkboard 打开和关闭的时间点,由于开关的相隔时间是随机的,所以从其时间曲线的功率谱图 5.2 并看不出周期性或者某种规律性变化。

\begin{figure}[h]
    \centering
    \includegraphics[width=\textwidth]{image1.png}
    \caption{视觉刺激实验下呼吸曲线}
    \label{fig:5.3}
\end{figure}

\begin{figure}[h]
    \centering
    \includegraphics[width=\textwidth]{image2.png}
    \caption{视觉刺激实验下呼吸曲线的功率谱}
    \label{fig:5.4}
\end{figure}

图 \ref{fig:5.3} 展示的为视觉刺激实验下呼吸曲线,图 \ref{fig:5.4} 则展示了视觉刺激实验下呼吸曲线的功率谱,从呼吸曲线可以看出其振动存在周期性,所以其功率谱也展现很明显的周期性,由于功率谱的横轴代表了信号的频率 (Hz),图 4 说明呼吸曲线的频率主要集中在小于 10Hz 的低频区。

所以从以上几幅图像中,并不能看出视觉刺激实验下时间曲线与呼吸曲线存在某种相关性。

\subsection{利用 ICA 模型分离脑电成分}

\subsubsection{分离与视觉刺激相关的脑电波成分}

基于问题 4 解决中所建立的模型 ICA,想要分离出与呼吸相关脑电波成分依然需要运用 ICA 方法。为了分离出与呼吸相关脑电波成分,首先我们尝试了把所有的谱图文件都输入来做 ICA,最后输出 15 个成分图如下:

\begin{figure}[h]
    \centering
    \includegraphics[width=0.45\textwidth]{image1.png}
    \includegraphics[width=0.45\textwidth]{image2.png}
    \caption{First row of plots.}
\end{figure}

\begin{figure}[h]
    \centering
    \includegraphics[width=0.45\textwidth]{image3.png}
    \includegraphics[width=0.45\textwidth]{image4.png}
    \caption{Second row of plots.}
\end{figure}

\begin{figure}[h]
    \centering
    \includegraphics[width=0.45\textwidth]{image5.png}
    \includegraphics[width=0.45\textwidth]{image6.png}
    \caption{Third row of plots.}
\end{figure}

\begin{figure}[h]
    \centering
    \includegraphics[width=0.45\textwidth]{image7.png}
    \includegraphics[width=0.45\textwidth]{image8.png}
    \caption{Fourth row of plots.}
\end{figure}

\begin{figure}[h]
    \centering
    \includegraphics[width=0.45\textwidth]{image9.png}
    \includegraphics[width=0.45\textwidth]{image10.png}
    \caption{Fifth row of plots.}
\end{figure}

\begin{figure}[h]
    \centering
    \includegraphics[width=\textwidth]{image1.png}
    \caption{图 5.5 睡眠状态下与清醒状态下 ICA 分析结果}
\end{figure}

通过对以上分离出的 15 个成分比较发现,最后一幅图和原本的 checkboard 的时间曲线图基本没有差别,这说明我们分离出的脑电波中成分中,与 checkboard 的频率相近的的成分是非常不显著的,意味着这次尝试下所分离出的 15 个成分都不能说明与视觉刺激相关。

\begin{figure}[h]
    \centering
    \includegraphics[width=\textwidth]{image1.png}
    \caption{睡眠状态下与清醒+视觉刺激状态下脑电波图谱}
    \label{fig:5.6}
\end{figure}

图 5.6 睡眠状态下与清醒+视觉刺激状态下脑电波图谱

这次尝试说明分析数据的思路并不合理。下面针对以上分析出现的数据方面问题,通过观察睡眠状态下与清醒加视觉刺激状态下脑电波对比图谱(图 5.6),从图中可以十分清楚的看出两者之间的差别十分显著,这说明两个状态下的脑电波图谱所包含成分差别过大,并不适合直接用于 ICA 分析。

\begin{figure}[h]
    \centering
    \includegraphics[width=\textwidth]{image2.png}
    \caption{清醒状态下与清醒+视觉刺激状态下脑电波图谱}
    \label{fig:5.7}
\end{figure}

图 5.7 清醒状态下与清醒+视觉刺激状态下脑电波图谱

然后我们通过对比清醒状态下与清醒加视觉刺激状态下的脑电波图谱(图 5.7),发现两者之间的差别较小,也可以认为这种较小差别便是来自于视觉刺激这个变量,所以将清醒状态下的两次实验记录数据用来进行独立成分分析会更加合适。

第二次尝试将清醒状态下的脑电波数据进行了独立成分分析其结果如下所示:

\begin{figure}[h]
    \centering
    \includegraphics[width=\textwidth]{image3.png}
    \caption{独立成分分析结果}
    \label{fig:5.8}
\end{figure}

\begin{figure}[h]
    \centering
    \includegraphics[width=\textwidth]{image1.png}
    \caption{ICA 分析结果}
\end{figure}

在此次尝试分析中, 一共得到了 5 个主要成分, 将以上各成分图谱与 checkboard 时间曲线(图 5.8)进行对比,不难发现本次分离出的成分中最后一幅成分图都能够体现出 checkboard 时间曲线波峰和波谷出现位置,并且这些位置都表现出了很高的显著性,所以我们可以认定该成分便是与视觉刺激相关的脑电波成分。

\begin{figure}[h]
    \centering
    \includegraphics[width=\textwidth]{image2.png}
    \caption{checkboard 时间曲线}
\end{figure}

\section*{5.2.2 分离与呼吸相关的脑电波成分}

为了分离与呼吸相关的脑电波成分,我们选择清醒状态下脑电波数据、清醒状态下视觉刺激脑电波数据及其呼吸曲线数据,对这三组数据进行 ICA 分析,得到如下 5 个主要成分:

\begin{figure}[h]
    \centering
    \includegraphics[width=\textwidth]{image1.png}
    \caption{分离与呼吸相关脑电波成分 ICA 分析结果}
\end{figure}

在独立成分分析中,一共得到了 5 个主要成分,将以上各成分图谱与呼吸曲线(图 5.10)进行对比,不难发现本次分离出的成分中最后一幅成分图都能够体现出呼吸曲线波峰和波谷出现位置,并且这些位置都表现出了很高的显著性,所以我们可以认定该成分便是与呼吸相关的脑电波成分。

\begin{figure}[h]
    \centering
    \includegraphics[width=\textwidth]{image2.png}
    \caption{呼吸曲线}
\end{figure}

\begin{figure}[h]
    \centering
    \includegraphics[width=\textwidth]{image11.png}
    \caption{视觉刺激试验}
    \label{fig:visual_stimulus}
\end{figure}

\begin{table}[h]
    \centering
    \begin{tabular}{|c|c|c|c|c|c|c|}
    \hline
         & Ch11 & Ch12 & Ch13 & Ch14 & Ch15 & 分离成分 \\ \hline
    A(1) vs B(1) & 4.01E-46 & 3.68E-41 & 3.79E-11 & 8.67E-19 & 5.20E-15 & 3.68E-41 \\ \hline
    A(2) vs B(2) & 4.38E-05 & 1.85E-09 & 7.75E-95 & 1.81E-63 & 7.54E-60 & 1.85E-09 \\ \hline
    A(3) vs B(3) & 3.31E-12 & 0.26043558 & 7.86E-05 & 0.887205099 & 0.0019116 & 0.26043558 \\ \hline
    A(4) vs B(4) & 1.86E-10 & 0.340366742 & 7.86E-15 & 2.75E-45 & 2.90E-57 & 0.340366742 \\ \hline
    A(5) vs B(5) & 0.5851008 & 1.32E-13 & 0.148066632 & 0.059084272 & 2.46E-17 & 1.32E-13 \\ \hline
    \end{tabular}
    \caption{A组与B组脑电波数据的配对t检验结果}
    \label{tab:t_test_results}
\end{table}

\begin{table}
\begin{tabular}{|c|c|c|c|c|c|c|}
\hline A(6) vsB(6) & 1.65E-22 & 8.35E-31 & 0.010713093 & 0.000440133 & 0.0465232 & 8.35E-31 \\
\hline A(7) vsB(7) & 0.0152316 & 5.98E-13 & 4.30E-05 & 0.011940126 & 0.041691114 & 5.98E-13 \\
\hline A(8) vsB(8) & 4.18E-07 & 3.44E-07 & 1.85E-13 & 1.41E-33 & 4.40E-05 & 3.44E-07 \\
\hline A(9) vsB(9) & 3.61E-88 & 9.79E-76 & 1.34E-46 & 9.04E-78 & 3.88E-47 & 9.79E-76 \\
\hline A(10) vsB(10) & 1.59E-14 & 1.39E-05 & 7.80E-21 & 7.67E-42 & 1.77E-32 & 1.39E-05 \\
\hline A(11) vsB(11) & 3.53E-23 & 0.520394457 & 0.156652711 & 4.45E-32 & 8.62E-06 & 0.520394457 \\
\hline A(12) vsB(12) & 1.72E-19 & 8.35E-08 & 7.49E-12 & 1.90E-26 & 1.92E-27 & 8.35E-08 \\
\hline A(13) vsB(13) & 1.59E-08 & 1.28E-17 & 6.76E-30 & 6.28E-44 & 0.017236293 & 1.28E-17 \\
\hline A(14) vsB(14) & 3.25E-21 & 6.45E-26 & 5.20E-17 & 0.005399456 & 0.001739335 & 6.45E-26 \\
\hline A(15) vsB(15) & 8.77E-15 & 2.26E-12 & 1.05E-07 & 2.89E-12 & 0.225637189 & 2.26E-12 \\
\hline A(16) vsB(16) & 9.44E-21 & 1.71E-12 & 1.16E-09 & 0.165875791 & 0.346456433 & 1.71E-12 \\
\hline A(17) vsB(17) & 0.6334685 & 1.98E-16 & 4.01E-25 & 4.71E-16 & 4.64E-12 & 1.98E-16 \\
\hline A(18) vsB(18) & 2.67E-07 & 2.40E-22 & 1.31E-41 & 5.58E-29 & 6.50E-15 & 2.40E-22 \\
\hline A(19) vsB(19) & 1.34E-09 & 8.04E-36 & 3.67E-07 & 1.12E-13 & 0.369068171 & 8.04E-36 \\
\hline A(20) vsB(20) & 4.50E-15 & 7.39E-46 & 2.08E-71 & 1.85E-72 & 3.76E-77 & 7.39E-46 \\
\hline A(21) vsB(21) & 1.01E-05 & 6.24E-11 & 0.38558892 & 3.93E-07 & 1.23E-13 & 6.24E-11 \\
\hline A(22) vsB(22) & 3.60E-16 & 7.32E-08 & 7.04E-16 & 9.60E-20 & 0.229650065 & 7.32E-08 \\
\hline A(23) vsB(23) & 7.57E-91 & 3.67E-67 & 5.44E-29 & 1.09E-23 & 1.07E-21 & 3.67E-67 \\
\hline A(24) vsB(24) & 4.91E-06 & 0.210136874 & 2.26E-12 & 1.95E-103 & 1.23E-52 & 0.210136874 \\
\hline A(25) vsB(25) & 4.15E-05 & 1.56E-24 & 2.04E-05 & 1.26E-06 & 2.41E-13 & 1.56E-24 \\
\hline A(26) vsB(26) & 1.32E-12 & 5.70E-36 & 0.045426886 & 0.002761594 & 1.56E-06 & 5.70E-36 \\
\hline A(27) vsB(27) & 0.0035867 & 0.00019131 & 0.335520456 & 0.542233241 & 0.109866366 & 0.00019131 \\
\hline A(28) vsB(28) & 1.03E-25 & 9.22E-28 & 1.72E-12 & 1.72E-30 & 1.09E-21 & 9.22E-28 \\
\hline A(29) vsB(29) & 3.38E-42 & 2.24E-18 & 2.47E-05 & 5.25E-05 & 0.753319462 & 2.24E-18 \\
\hline A(30) vsB(30) & 1.92E-18 & 1.00E-09 & 3.86E-07 & 0.88063961 & 0.055325167 & 1.00E-09 \\
\hline
\end{tabular}
\end{table}

上述方法能够证明原始数据在受到不同图形A和B视觉刺激下,小鼠LFP信号具有显著差异。下面基于从原始脑电波数据中分离出的可能与呼吸相关的脑电波信号成分(图5.12)。将该分离出的脑电波成分按照时间曲线筛选出的60个区间,一共找到对应配对的脑电波成分60组。

\begin{figure}[h]
\centering
\includegraphics[width=\textwidth]{image.png}
\caption{分离得到的与呼吸相关的脑电波成分}
\end{figure}

\section*{参考文献}

[1] 伍国锋, 张文渊. 脑电波产生的神经生理机制. 《临床脑电学杂志》2000 年 03 期.

[2] 袁蒋伟, 吴伊自, 沈杰. 脑电波的统计特性 浙江大学学报(理学版), 2007 年 01 期

[3] 王晓宏, 刘志峰, 罗海山. 利用方差分析方法分析脑电波. 生物医学工程研究.

[4] 周倩, 张晋昕. 时间序列周期性检验方法研究进展. 中国卫生统计, 2013 年 03 期

[5] J. Ito, S. Roy, Y. Liu, M. Fletcher, L. Lu, J. D. Boughter, S. Grun, D. H. Heck. Whisker barrel cortex delta oscillations and gamma power in the awake mouse are linked to respiration, Nature Communication, 2014, April.

[6] Zhou W1, Zhou J, Zhao H, Ju L. Removing eye movement and power line artifacts from the EEG based on ICA. Conf Proc IEEE Eng Med Biol Soc. 2005;6:6017-20.

[7] Jung TP1, Makeig S, Humphries C, Lee TW, McKeown MJ, Iragui V, Sejnowski TJ. Removing electroencephalographic artifacts by blind source separation. Psychophysiology. 2000

[8] Langlois, Sylvain Chartier, Dominique. An Introduction to Independent Component Analysis: InfoMax and FastICA algorithms Dominic. Tutorials in Quantitative Methods for Psychology Gosselin Mar;37(2):163-78. 2010, Vol. 6(1), p. 31-38.

[9] Delorme A et al. EEGLAB: an open source toolbox for analysis of single-trial EEG dynamics including independent component analysis. J Neurosci Methods. (2004)

[10] Kopal J1, Vyšata O2, Burian J3, Schätz M4, Procházka A5, Vališ M6. Complex continuous wavelet coherence for EEG microstates detection in insight and calm meditation. Conscious Cogn. 2014 Aug 14;30C:13-23. doi: 10.1016/j.concog.2014.07.015. [Epub ahead of print]

[11] Huang CH1, Ju MS, Lin CC. A robust algorithm for removing artifacts in EEG recorded during FMRI/EEG study. Comput Biol Med. 2012 Apr;42(4):458-67. doi: 10.1016/j.compbiomed.2011.12.014. Epub 2012 Jan 25.

[12] L. Noe Oliva Moreno1, Miguel A. Alemán Arce, Jair García Lamont. Implementation of Infomax ICA Algorithm for Blind Source Separation. Electronics, Robotics and Automotive Mechanics Conference 2008.

[13] 黄闯, 侍洪波. 基于独立分量分析 ICA 与小波变换的过程监测方法. 吉林大学学报 (工学版), 2004 年 03 期

[14] Sun Y, Farzan F, Dominguez LG, Barr MS, Giacobbe P, Lozano AM, Wong W, Daskalakis ZJ. A Novel Method for Removal of Deep Brain Stimulation Artifact from Electroencephalography. J Neurosci Methods. 2014 Sep 8.

[15] Xia H, Ruan D, Cohen MS. Front Neurosci. Separation and reconstruction of BCG and EEG

\section*{附录}

\section*{程序一:Python 语言处理睡眠状态下的呼吸数据程序(提取波谷所在时间点,波峰点的提取原理相同)}

\begin{verbatim}
f=open("./shuimianhuxi.txt", "r")
array1=[]
while 1:
    line=f.readline()
    if not line:
        break
    array1.append(int(line.strip()))
f.close()
fout=open("./lowp.txt", "w")
print(len(array1))
startp=0
endp=len(array1)
for i in range(endp):
    atl=i-299
    at2=i+299
    if(atl<0):
        atl=0
    if(at2>endp):
        at2=endp
    ''' print "atl:"+str(atl)
    print "at2:"+str(at2)'''
    lowest=12000
    ''' print 1'''
\end{verbatim}

\begin{verbatim}
for ui in range(at2-at1):
    ''' print "here1"'''
    if(lowest>array1[at1+ui]):
        lowest=array1[at1+ui]
    ''' print "here3"'''
    if (lowest==array1[i]):
        fout.write(str(i)+"\n")
        print i
        print "range: "+str(at1)+"to"+str(at2)
        print "len: "+str(at2-at1+1)
fout.close()
f=open("./lowp.txt","r")
array2=[]
while 1:
    line2=f.readline()
    if not line2:
        break
    array2.append(line2.strip())
f.close()
sa=[]
fout2=open("./lowpafteruniq.txt","w")
for t in range(len(array2)-1):
    if(int(array2[t+1])-int(array2[t])<20):
        sa.append(t)
for uy in range(len(sa)):
    del array2[sa[uy]-uy]
for iu in range(len(array2)):
    fout2.write(str(array2[iu])+"\n")
fout2.close()
\end{verbatim}

\textbf{程序二:数据震荡处理}

\begin{verbatim}
data<-read.csv("data1.csv",header=F)
data[1:300,1]
newdata<-matrix(0,30000,1)
sume<-0
for(j in 1:30000){
    newdata[j,1]<-data[j,1]-mean(data[1:30000,1])}
write.csv(sume,file="sume.csv")
for(i in 1:30000)
sume[i]<-sum(newdata[1:i,1])
\end{verbatim}

\begin{verbatim}
程序三:DFA算法实现
# 各子区间拟合
t <- c(1:3000)
lm.sol <- lm(sume[1:3000] ~ t)
summary(lm.sol)
plot(t, sume[300:600])
lines(mm$fit)
# 求 F(t)
nndata <- 0
F <- matrix(0, 3000, 10)
for (j in 1:10) {
  if (j == 1) {
    for (i in 1:3000) {
      for (m in 1:30000) {
        nndata[m] <- {sume[m] - (39.06 * i - 27660)}^2
      }
      F[i, j] <- sqrt(mean(nndata))
    }
  } else if (j == 2) {
    for (i in 1:3000) {
      for (m in 1:30000) {
        nndata[m] <- {sume[m] - (-51.179 * i + 68761.854)}^2
      }
      F[i, j] <- sqrt(mean(nndata))
    }
  } else if (j == 3) {
    for (i in 1:3000) {
      for (m in 1:30000) {
        nndata[m] <- {sume[m] - (-33.88 * i + 96446.38)}^2
      }
      F[i, j] <- sqrt(mean(nndata))
    }
  } else if (j == 4) {
    for (i in 1:3000) {
      for (m in 1:30000) {
        nndata[m] <- {sume[m] - (54.56 * i - 89560)}^2
      }
      F[i, j] <- sqrt(mean(nndata))
    }
  } else if (j == 5) {
    for (i in 1:3000) {
      for (m in 1:30000) {
        nndata[m] <- {sume[m] - (88.04 * i - 10320)}^2
      }
      F[i, j] <- sqrt(mean(nndata))
    }
  } else if (j == 6) {
    for (i in 1:3000) {
      for (m in 1:30000) {
        nndata[m] <- {sume[m] - (-60.75 * i + 84668.113)}^2
      }
      F[i, j] <- sqrt(mean(nndata))
    }
  }
}
\end{verbatim}

\begin{verbatim}
F[i,j]<-sqrt(mean(nndata))
}}
else if(j==7){
for(i in 1:3000){
for(m in 1:30000){
    nndata[m]<-{sum[m]-(-4.8347*i+61957.5651)}^2}
    F[i,j]<-sqrt(mean(nndata))
}}
else if(j==8){
for(i in 1:3000){
for(m in 1:30000){
    nndata[m]<-{sum[m]-(-42.4704*i+84462.4532)}^2}
    F[i,j]<-sqrt(mean(nndata))
}}
else if(j==9){
for(i in 1:3000){
for(m in 1:30000){
    nndata[m]<-{sum[m]-(10.738*i+53560.595)}^2}
    F[i,j]<-sqrt(mean(nndata))
}}
else {
for(i in 1:3000){
for(m in 1:30000){
    nndata[m]<-{sum[m]-(74.015*i-47306.503)}^2}
    F[i,j]<-sqrt(mean(nndata))
}}
write.csv(F,file="F.csv")
\end{verbatim}

\textbf{程序四:配对 T 检验}

\#图形因素配对 T 检验

\begin{verbatim}
data<-read.csv("2.csv", header=FALSE)
data<-as.matrix(data)
data1<-data[seq(from=1, to=60, by=2), ]
data2<-data[seq(from=2, to=60, by=2), ]
pvalue<-0
for(i in 1:30){
    pvalue[i]<-t.test(na.omit(data1[i,]), na.omit(data2[i,]))$p.value
}
write.csv(pvalue, file="pvalue6.csv")
\end{verbatim}

\begin{center}
- 31 -
\end{center}