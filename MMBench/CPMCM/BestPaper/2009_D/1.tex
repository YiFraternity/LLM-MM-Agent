\documentclass{article}
\usepackage{amsmath}
\usepackage{amssymb}

\title{全国第六届研究生数学建模竞赛}
\author{}
\date{}

\begin{document}

\maketitle

\begin{center}
\includegraphics[width=0.5\textwidth]{image.png}
\end{center}

\begin{table}[h]
\centering
\begin{tabular}{l l}
题目 & 110警车配置及巡逻方案 \\
\hline
\end{tabular}
\end{table}

\begin{abstract}
本文讨论的是某城区110警车配置与巡逻问题,针对不同的条件和要求,对城区道路网络建立模型、求解方案,并提出了相应的评价指标。

对于问题一,建立离散设施选址的k-中心模型,并利用K-均值聚类及Dijkstra算法进行求解,得到最少警车数为13,警车在接警后3分钟赶到现场的比例为91.76\%;

对于问题二,分别从覆盖、辖区范围、巡逻效果三个方面提出了五个指标,为准确评价警车巡逻方案的优劣提供了科学的依据;

对于问题三,建立0-1规划模型,并运用图论的知识,结合具体问题,制定了警车的巡逻方案:布置16辆警车,巡逻路线由3个环路和4个定点组成,动静结合,平均道路覆盖率达93\%以上;

对于问题四,以问题三的方案为基础,采用了中心城区随机选择变换巡逻街道、环线及重点部位随机转向的方案,在不影响方案评价指标的前提下,引入了随机性因素,大大提高了巡逻的隐蔽性;

对于问题五,建立离散设施选址的最大覆盖模型,并选定了警车在覆盖率最大的定点周围做小范围的直线或环线巡逻的方案,同时保证了覆盖率和巡逻效果;

对于问题六,沿用了问题三的0-1规划模型,根据新的条件,优化了巡逻方案:布置12台警车,巡逻路线由4个环路和2个定点组成,在削减警车的同时,提高了巡逻的效果,平均道路覆盖率达95\%以上;

对于问题七,针对目前建模的理想假设,提出了相应的改进方案,为将来模型的改善提供了思路。
\end{abstract}

\begin{table}[h]
\centering
\begin{tabular}{l l}
参赛队号 & \underline{1025106} \\
队员姓名 & \underline{孙漾、王敏、刘珊珊} \\
\end{tabular}
\quad

\end{table}

\tableofcontents

\section*{目录}
\begin{itemize}
    \item[] 1 问题重述 \dotfill 4
    \item[] 2 模型假设 \dotfill 5
    \item[] 3 符号说明 \dotfill 5
    \item[] 4 模型的建立 \dotfill 6
    \item[] 5 问题的分析与求解 \dotfill 7
    \begin{itemize}
        \item[] 5.1 数据的预处理 \dotfill 7
        \item[] 5.2 问题1的求解 \dotfill 10
        \begin{itemize}
            \item[] 5.2.1 问题分析 \dotfill 10
            \item[] 5.2.2 模型分析 \dotfill 10
            \item[] 5.2.3 模型求解 \dotfill 11
            \item[] 5.2.4 计算结果 \dotfill 14
            \item[] 5.2.5 小结 \dotfill 15
        \end{itemize}
        \item[] 5.3 问题2的求解 \dotfill 16
        \begin{itemize}
            \item[] 5.3.1 问题分析 \dotfill 16
            \item[] 5.3.2 具体评价指标 \dotfill 16
            \item[] 5.3.3 小结 \dotfill 18
        \end{itemize}
        \item[] 5.4 问题3的求解 \dotfill 19
        \begin{itemize}
            \item[] 5.4.1 问题分析 \dotfill 19
            \item[] 5.4.2 模型分析 \dotfill 19
            \item[] 5.4.3 模型求解 \dotfill 19
            \item[] 5.4.5 求解结果 \dotfill 21
            \item[] 5.4.6 评价指标的计算和模拟 \dotfill 22
            \item[] 5.4.7 小结 \dotfill 23
        \end{itemize}
        \item[] 5.5 问题4的求解 \dotfill 23
        \begin{itemize}
            \item[] 5.5.1 问题分析 \dotfill 23
            \item[] 5.5.2 问题求解 \dotfill 23
            \item[] 5.5.3 小结 \dotfill 24
        \end{itemize}
        \item[] 5.6 问题5的求解 \dotfill 24
        \begin{itemize}
            \item[] 5.6.1 问题分析 \dotfill 24
            \item[] 5.6.2 模型分析 \dotfill 24
            \item[] 5.6.3 模型求解 \dotfill 25
            \item[] 5.6.4 求解结果 \dotfill 25
            \item[] 5.6.5 评价指标的计算和模拟 \dotfill 26
            \item[] 5.6.7 小结 \dotfill 27
        \end{itemize}
        \item[] 5.7 问题6的求解 \dotfill 27
        \begin{itemize}
            \item[] 5.7.1 问题分析 \dotfill 27
            \item[] 5.7.2 问题模型 \dotfill 27
            \item[] 5.7.3 模型求解 \dotfill 28
            \item[] 5.7.4 求解结果 \dotfill 28
            \item[] 5.7.5 评价指标的计算和模拟 \dotfill 29
        \end{itemize}
    \end{itemize}
\end{itemize}

\begin{itemize}
    \item[5.7.6] 小结 \dotfill 30
    \item[5.8] 问题7的求解 \dotfill 30
    \begin{itemize}
        \item[5.8.1] 问题分析 \dotfill 30
        \item[5.8.2] 问题求解 \dotfill 30
        \item[5.8.3] 小结 \dotfill 31
    \end{itemize}
    \item[] 参考文献 \dotfill 31
\end{itemize}

\section{110 警车配置及巡逻方案}

\section{1 问题重述}

110 警车在街道上巡逻,既能够对违法犯罪分子起到震慑作用,降低犯罪率,又能够增加市民的安全感,同时也加快了接处警(接受报警并赶往现场处理事件)时间,提高了反应时效,为社会和谐提供了有力的保障。但是在整个城区配置多少辆警车进行巡逻,才能使得警车发挥最大作用,又不浪费警车资源,这个问题值得考虑。

某城市拟增加一批配备有 GPS 卫星定位系统及先进通讯设备的 110 警车。设 110 警车的平均巡逻速度为 $20 \mathrm{~km} / \mathrm{h}$,接警后的平均行驶速度为 $40 \mathrm{~km} / \mathrm{h}$。警车配置及巡逻方案要尽量满足以下要求:

D1. 警车在接警后 3 分钟内赶到现场的比例不低于 $90\%$;而赶到重点部位的时间必须在 2 分钟之内。

D2. 使巡逻效果更显著;

D3. 警车巡逻规律应有一定的隐蔽性。

当考虑配置的警车停留在一个固定的地点或巡逻过程中任一时刻所有的警车固定在其路径上的某一位置,接到报案后迅速赶往现场时,警车的配置问题就跟应急设施的选址问题非常类似。应急设施选址问题在很多地方都有讨论,一个应急设施覆盖它周围的一个区域,保证其服务对象及时得到服务。在这个问题中,警车附近的街道上一旦发生犯罪事件,警车必须及时赶到处理。因此,我们先选用选址问题的模型,考虑当警车静止时,需要多少辆警车才能有效地保证全城居民的安全,也就是 $90\%$ 的报警事件发生 3 分钟之内,能得到警车的现场处理。在选址问题中,我们考虑将全城分成若干个区域:认为每辆警车对它所在的辖区负责,$90\%$ 可以理解为警车能在 3 分钟之内到达它所覆盖区域 $90\%$ 的点。全市 3 个重点部位的救援限定在两分钟之内到达。

对于警车的配置和巡逻方案不同,则可能产生的巡逻效果也有所不同。因此我们需要采取一系列统一的评价标准来定量地衡量巡逻效果。

考虑警车沿着街道巡逻,在发生报警事件时及时赶往现场。在这种情况下,为了实现 3 分钟赶到报警现场的目标,全城要合理配置警车,并且制定详细的巡逻计划。巡逻的路线确定后,即可评判警车巡逻时覆盖途经区域的范围。根据该范围的大小,可以有效地制定警车的配置和巡逻方案,即需要警车的数量和警车巡逻的具体时间轨迹。

巡逻的最大作用是震慑犯罪分子。为了不让犯罪分子掌握警车巡逻的规律,巡逻方案必须具有一定的隐蔽性。为此,我们可以先制定优化的巡逻路线,再人为添加随机因素,即制定同一条路线的不同时间表,使警车每天在不同的时间经过同一个地点;或者制定若干条评价指标相当的巡逻路线,警车随机选择其中的一条巡逻。

目前由于资源的限制,我们要充分利用仅有的资源实现警察对市民的安全保护。

考虑仅有 10 辆警车的情况下,尽量做到接警后迅速赶往现场,巡逻时充分震慑犯罪分子。考虑某一特定的时刻,这个问题是资源受限的覆盖问题,即在资源有限的条件下,如何实现最大范围的覆盖需求,可以套用最大覆盖的模型。

警车赶往事发现场的速度对于这个问题具有较大的影响。如果警车在接到报警后赶往现场的速度提高,那么警察在巡逻时的覆盖率将增大,可以使巡逻的效果得到显著的提高。例如题中将接警后的平均时速提高到 \(50 \, \text{km/h}\)(提高 \(12.5\%\)),增大了警车能覆盖的区域,这时根据我们的模型可知使用较少数量的警车依然可以满足覆盖率的需求。

\section{2 模型假设}

1. 区域中各点在各时段发生报警事件的概率相同,即认为报警事件是均匀分布在所有的时间段和所有结点上。

2. 警车抵达靠近报警现场最近的路径,则报警点可看见警车或听到警笛,即警车抵达报警现场所在路径两端的某一道路交叉口时,则认为已抵达报警现场(考虑到区域中某些道路较长,在后续求解过程中将在较长的路段中插入若干结点,以保证该假设的合理性)。

3. 一个区域内,每个时刻不会同时发生 2 起或 2 起以上犯罪事件,即在足够短的时间内(小于等于 3 分钟)只发生一起犯罪事件。当报警发生时,报警点附近的警车处于空闲(即没有在处理其他报警事件),接警后,可立即前往报警现场。

4. 3 个重点部位做如下近似:
   \begin{itemize}
       \item 对重点部位一 \((5112, 4806)\),警车抵达 101、103、110、112 任一交叉口,则认为抵达;
       \item 对重点部位二 \((9126, 4266)\),警车抵达 123 或 141 交叉口,则认为抵达;
       \item 对重点部位三 \((7434, 1332)\),警车抵达 277 号交叉口,则认为抵达。
   \end{itemize}

5. 假设警车在所有道路上的行驶速度都是匀速的,都等于问题里给出的平均速度。

\section{3 符号说明}

\begin{itemize}
    \item \( h_i \):结点 \( i \) 的需求(发生报警事件的次数);
    \item \( X_i \):表示点 \( i \) 是否配置警车的 0-1 变量,等于 1 表示点 \( i \) 处配置一辆警车,等于 0 表示点 \( i \) 处不配置警车;
    \item \( y_i \):表示点 \( i \) 是否被警车覆盖的 0-1 变量;
    \item \( t \):表示时刻,以分钟为单位,\( 0 \leq t \leq 1440 \);
    \item \( X_i(t) \):考虑警车巡逻,则某个点是否有警车跟时间相关。\( X_i(t) = 1 \) 表示 \( t \) 时刻点 \( i \) 处有警车。
\end{itemize}

有警车, $X_{i}(t)=0$ 表示 $t$ 时刻点 $i$ 处没有警车;

$y_{ij}(t)$:$t$ 时刻点 $j$ 是否被点 $i$ 位置的警车覆盖。$y_{ij}(t)=1$,当 $j \in N_{i}$(即警车从 $j$ 点能在 3 分钟之内赶到 $i$)时;否则为 0;

$y_{j}(t)$:表示的是 $t$ 时刻点 $j$ 是否被某个位置的警车覆盖,不计重复覆盖次数。$y_{j}(t)$ 等于对 $y_{ij}(t)$($\forall i$)的逻辑或运算结果;

$d(i,j)$:点 $i$ 和 $j$ 之间的最短路长;

$N_{i}$:点 $i$ 的邻域,满足 $d(i,j) \leq \nu / 20$(单位:km)的点的集合,$\nu$ 表示警车赶往事发地点的行车速度。

$R_{c}$:警车响应覆盖率。

$C_{RT}$:警车交叉响应覆盖率。

$R_{a}$:警车响应平均行车时间。

$L_{c}$:警车资源平均利用水平。

$R_{F}$:警车流动覆盖率。

\section{4 模型的建立}

首先,考虑警车的配置问题。根据问题 1 的假设,只考虑警车在某个点静止不动,在案件发生的时候赶往事发现场,则警车配置问题是个离散设施选址问题。离散设施选址问题中有两个比较类似的模型:其一,覆盖问题;其二,中心问题[1]。

覆盖问题分为完全覆盖问题与最大覆盖问题。完全覆盖问题的目标函数是在给定时间(距离)内满足所有需求的最小设施建设成本。如果每个设施建设费用相同,那么目标函数就是最小化设施的数量。模型具体表示为:

\begin{align*}
\text{max } & \sum_{i} h_i y_i \\
\text{s.t. } & y_i \leq \sum_{j \in N_i} X_j, \forall i \\
& \sum_{j} X_j \leq P \\
& y_i \in \{0, 1\}, \forall i
\end{align*}

其中 $c_{j}$ 为设施建设成本,$X_{j}$ 是表示需求点 $j$ 是否修建设施的参数,如果在 $j$ 点修建设施,$X_{j}$ 为 1,否则为 0。

继续考虑这个模型,它没有考虑需求的规模,不论需求多大都可以被设施点满足。另外,如果边缘的结点有非常小的需求量,那么建造此类设施的投入产出比非常低。用在警车设置问题中,就是有些地方人烟稀少,跟周围区域没有紧密联系,为了保证 3 分钟到达这些地方专门在附近设立警车的代价比较大,警车资源没有发挥最大作用。有资源限制的情况下模型改进为最大覆盖问题:

\begin{align*}
\text{max } & \sum_{i} h_i y_i \\
\text{s.t. } & y_i \leq \sum_{j \in N_i} X_j, \forall i \\
& \sum_{j} X_j \leq P \\
& y_i \in \{0, 1\}, \forall i
\end{align*}

$h_i$ 为需求点 $i$ 的需求;当结点 $i$ 被覆盖时,$y_i$ 为 1,否则为 0。目标函数是求有限资源条件下所能覆盖的最大需求量。约束条件 1 确定设施为哪个在特定距离内的需求点提供服务。约束条件 2 限定设施数量小于等于 $P$。在警车巡逻实例中,我们先假设各个点的需求规模是一样的,这时的目标函数就是最大化需求点数量。如果考虑模型更切合实际,则应讨论不同点的需求不同,这个条件我们放在问题 7 里讨论。

Daskin 与 Stern 将覆盖模型扩展为支援覆盖模型 (Backup coverage) 研究美国 EMS 系统车辆服务。针对已经派出车辆不能对该设施服务区的需求再提供服务这一实际问题,支援模型允许其它在可接受距离内的设施对新增需求提供服务。他们建立了一个多等级目标方程,首先最小化能满足需求的车辆数量,然后最大化这些车辆能多重覆盖的范围。在警车配置问题中,如果一个区域内发生一起以上犯罪事件,则由该区域附近可在 3 分钟内赶到现场的警车提供支援,这是完全可能发生的。这种情况较为复杂,我们在求解时并不考虑支援覆盖,在评价方案的好坏时引入交叉覆盖的指标,在问题 2 中提出。

完全覆盖问题的目标函数是在特定时间(距离)内满足所有需求应修建的最小设施数。与此不同,$k$-中心问题虽然也满足所有需求,但设施数给定,目标是最小化任一需求点到与它最近设施的距离。$k$-中心问题一种常用的解法是,解一系列的集合覆盖问题。每一步选一个覆盖距离的阈值,检查是否所有的点都能在这个距离范围内被最多 $k$ 个点覆盖;若能,则降低阈值,否则增加阈值。这个问题是 NP 难的,目前用得比较多的是贪婪算法,也可以用聚类分析的办法去求解[2]。

\section{5 问题的分析与求解}

\subsection{5.1 数据的预处理}

对于本题,由于牵涉到大量的道路数据信息,因此为了方便后面问题的模型求解,我们先对题目所给出的数据进行了预处理。

首先,我们先对题目中给出地图数据和道路数据表格进行了扩充,主要添加了各条道路的长度信息。在这里,选择欧式距离表达位于同一条道路上的结点的距离,计

\begin{align*}
\text{min } D \\
D \geq \sum_{j} d(i, j) y_{ij}, \forall i
\end{align*}

其次,结合了道路信息和 MATLAB 生成图,通过对道路交叉口数目沿 \(x\) 坐标数目和道路长度分布的统计(分别见图 5.1 和表 5.1),我们得出了该城市的地图布局结构有以下特点:地图左半部分的道路密集而且每段的长度较小,而地图右半部分的道路分布稀疏并且道路较长。

\begin{figure}[h]
    \centering
    \includegraphics[width=\textwidth]{image.png}
    \caption{道路交叉口数目沿 \(x\) 坐标数目统计图}
    \label{fig:5.1}
\end{figure}

\begin{table}[h]
    \centering
    \caption{道路长度分布统计表}
    \label{tab:5.1}
    \begin{tabular}{c c}
        \hline
        道路长度(米) & 道路数目 \\
        \hline
        0~200 & 52 \\
        200~400 & 158 \\
        400~600 & 102 \\
        600~800 & 52 \\
        800~1000 & 34 \\
        1000~1200 & 34 \\
        1200~1400 & 8 \\
        1400~1600 & 4 \\
        1600~1800 & 3 \\
        1800~2000 & 4 \\
        2000~2200 & 4 \\
        2200~2400 & 0 \\
        2400~2600 & 0 \\
        2600~2800 & 1 \\
        2800~3000 & 1 \\
        3000~3200 & 0 \\
        3200~3400 & 0 \\
        3400~3600 & 1 \\
        \hline
    \end{tabular}
\end{table}

最后,根据前面模型假设 2 中,假设警车抵达靠近报警现场最近的路径,则报警点可看见警车或听到警笛,即警车抵达报警现场所在路径两端的某一道路交叉口时,则认为已抵达报警现场。为了保证该假设的合理性。考虑到区域中某些道路较长,我们在几条较长的道路中均匀地插入了 6 个结点,这 6 个结点的位置如图 5.2 所示,坐标信息见表 5.2。表 5.3 给出了插入 6 个结点后的道路信息变化。

\begin{figure}[h]
    \centering
    \includegraphics[width=\textwidth]{image.png}
    \caption{新增道路交叉口位置示意图}
    \label{fig:5.2}
\end{figure}

\begin{table}[h]
    \centering
    \caption{新增道路交叉口信息表}
    \label{tab:5.2}
    \begin{tabular}{c c c}
        \hline
        编号 & X 坐标 & Y 坐标 \\
        \hline
        308 & 1416 & 396 \\
        309 & 2616 & 378 \\
        310 & 9748 & 235 \\
        311 & 13197 & 3297 \\
        312 & 13128 & 4294 \\
        313 & 13018 & 8028 \\
        \hline
    \end{tabular}
\end{table}

\begin{table}
\centering
\caption{道路信息变化一览表}
\begin{tabular}{c c c c c c c}
\hline
序号 & 原道路起点 & 原道路终点 & 原道路长度 & 新道路起点 & 新道路终点 & 道路长度 \\
\hline
\multirow{3}{*}{1} & \multirow{3}{*}{297} & \multirow{3}{*}{298} & \multirow{3}{*}{3582} & 297 & 308 & 1200 \\
 &  &  &  & 308 & 309 & 1200 \\
 &  &  &  & 309 & 298 & 1182 \\
\hline
\multirow{3}{*}{2} & \multirow{3}{*}{79} & \multirow{3}{*}{237} & \multirow{3}{*}{2869} & 79 & 312 & 1000 \\
 &  &  &  & 312 & 311 & 1000 \\
 &  &  &  & 311 & 237 & 869 \\
\hline
\multirow{2}{*}{3} & \multirow{2}{*}{1} & \multirow{2}{*}{2} & \multirow{2}{*}{2754} & 1 & 313 & 1400 \\
 &  &  &  & 313 & 2 & 1354 \\
\hline
\multirow{2}{*}{4} & \multirow{2}{*}{304} & \multirow{2}{*}{305} & \multirow{2}{*}{2124} & 304 & 310 & 1000 \\
 &  &  &  & 310 & 305 & 1124 \\
\hline
\end{tabular}
\end{table}

\subsection{5.2 问题1的求解}

\subsubsection{5.2.1 问题分析}

问题1的要求是任何一个区域警车在接警后3分钟内赶到现场的比例不低于90\%,而赶到重点部位的时间必须在两分钟之内,该区最少需要配置警车的数量。该问题的目标是在满足3个重点部位接警时间不超过2分钟的情况下,用尽量少的警车去保证全城90\%的区域3分钟内可达。取某个特定时刻,警车都可看做静止在某个点,并且覆盖该点的邻域(可在3分钟之内到达该点的邻域内的任意点)。该问题相当于在全城配置若干辆警车,使得警车的总覆盖率达到90\%,并且重点部位在2分钟之内可达。因此,该问题可转化为一个离散设施选址问题。

\subsubsection{5.2.2 模型分析}

根据模型假设2,在添加了6个结点后,我们认为警车到达报警点所在道路的端点就覆盖了该点。考虑用最少的警车达到要求D1时,我们假设警车静止在它所覆盖的区域的中心位置,接警后出动,完成后返回该位置。如果警车只配置在结点处,并且把要求D1简化为警车可在3分钟之内到达全部需求点的90\%,则可以套用离散设施选址的完全覆盖模型。设结点$i$对应变量$X_{i}$,由于题目中不考虑在不同地点设置警车成本的差异,模型为

\begin{align*}
C_i &\neq \phi, \, i = 1, \ldots, m \\
\bigcup_{i=1}^m C_i &= X \\
C_i \cap C_j &= \phi, \, i \neq j, \quad i, j = 1, \ldots, m
\end{align*}

为了求解这个问题,先用直径为2km的圆形去覆盖整个区域,估计出需要的警车在10辆以上。这个模型计算量比较大,改用$k$-中心问题的模型去描述这个问题:

\begin{align*}
\text{min } D \\
D \geq \sum_{j} d(i, j) y_{ij}, \forall i
\end{align*}

$d(i, j)$ 为需求点 $i$ 与设施点 $j$ 的距离。目标函数是使警车与其辖区内最远需求点的距离 $D$ 最大。约束条件限定了任何需求点 $i$ 与最近警车 $j$ 的最大距离。当点 $i$ 需求被设施 $j$ 满足,则 $y_{ij}$ 为 1,否则为 0。

$k$-中心问题的解决思路是首先在需求点集合中任意选取 $k$ 个点并把所有需求点聚合成 $k$ 个簇,然后为每个簇计算中心。再根据新的中心重新聚合所有需求点,重复以上过程直到没有更好的结果产生为止。我们选用 $k$-均值聚类方法,将这个问题转化成 $k$-中心问题。也就相当于把这个问题划分成 $k$ 个区域,每个区域配置一辆警车。

\subsubsection{5.2.3 模型求解}

在该问题求解中,首先我们需要能够计算每个结点到其余各个结点与道路的最短距离,套用图论中的最短路模型即可 [3]。在本题的求解中,我们使用了 Dijkstra 算法,该算法的效率不错。

Dijkstra 算法的基本思想是:如果 $v_i, \ldots, v_j, \ldots, v_n$ 是某图 $G$ 从 $v_i$ 到 $v_n$ 的最短路径,则它的子路 $v_i, \ldots, v_j$ 一定是从 $v_i$ 到 $v_j$ 的最短路径。其算法步骤如下:

1) 假设网络 $G$ 有 $n$ 个顶点,用带权的邻接矩阵 $W$ 来表示,$W(i, j)$ 表示从顶点 $v_i$ 到 $v_j$ 的弧或边上的权值,不存在弧或边的权值用 $\infty$(在 MATLAB 中为 Inf)表示。$S$ 为已求出的从已知始点 $v_i$ 出发的最短路径的终点的集合,它的初始状态为空集,则从 $v_i$ 出发到图上其余各顶点 $v_k$ 可能达到的最短路径长度的初值为

\[
D(k) = \min \{ W(i, k) \mid v_k \in V - \{i\} \}
\]

2) 选择 $v_j$,使得 $D(j) = \min \{ D(k) \mid v_k \in V - S \}$,$v_j$ 就是当前求得的一条从始点 $v_i$ 出发的最短路径的终点,令 $S = S \cup \{j\}$。

3) 修改从 $v_i$ 出发到集合 $V - S$ 上任一点 $v_k$ 可达的最短路径长度。如果 $D(j) + W(j, k) < D(k)$,则修改 $D(k)$ 为

\begin{align*}
\min \sum_i x_i(t) \\
s.t. \sum_j y_j(t) \geq 90\% \times 5142, \forall i, j
\end{align*}

4) 重复操作 2)、3) 共 \( n-1 \) 次,并记录各最短路径经过的所有顶点。由此得到从始点 \( v_i \) 到图上其余各顶点的最短路径是依据路径长度递增的序列。

根据 Dijkstra 算法求得各结点间的最短距离后,我们使用了 \( K \)-均值距离的算法对所有的结点进行聚类,计算所需要的最少警车数量和其辖区范围。本问题中,为了使地图中的 313 个道路结点可以按区域划分,形成相对应警车的辖区,我们在随机生成结点中心的过程中利用了 \( K \)-均值聚类的算法 [4]。

聚类算法是将数据集划分为若干个类(cluster)的过程,完成一个聚类任务,主要取决于特征选择和近邻测度。特征选择就是必须合适地选址特征,尽可能多地包含任务关心的信息。在特征中,使信息冗余减少和最小化是主要目标。因为在有监督分类中,使用之前特征的预处理是必要的。近邻测度是用于定量测量两个特征向量如何 “相似” 或 “不相似”,保证所有选中的特征具有相同的近邻性,并且没有占支配地位的特征。也就是说聚类的目的是使同一个聚类中数据对象(样本点)具有较高的相似度,而不同聚类中的对象相似度较低(聚类相似度是利用各聚类中对象的均值所得到的一个 “中心对象”(引力中心)来进行计算的)。

定义 1 设 \( X \) 是数据集,即 \( X = \{x_1, x_2, \ldots, x_N\} \),定义 \( X \) 的 \( m \) 聚类 \( R \),将 \( X \) 分割成 \( m \) 个集合(聚类) \( C_1, \ldots, C_m \),使其满足下面 3 个条件:

\begin{align*}
C_i &\neq \phi, \, i = 1, \ldots, m \\
\bigcup_{i=1}^m C_i &= X \\
C_i \cap C_j &= \phi, \, i \neq j, \quad i, j = 1, \ldots, m
\end{align*}

而且,在聚类 \( C_i \) 中包含的向量彼此 “更相似”,与其他类中的向量 “不相似”。

聚类的方法主要有:划分方法、分层类方法、基于密度方法、基于网格方法和基于模型方法 [5]。其中,基于划分方法的 \( K \)-均值聚类 (K-means) 算法因其算法复杂度较低 \( O(tkn) \)(\( n \) 为数据对象总数,\( k \) 为聚类个数,\( t \) 为循环次数),效率高、应用领域广、具有一定的可扩展性等优点,在聚类算法中占重要地位。本问题的求解过程中,在路径网络结点划分的过程中就利用到 \( K \)-means 算法的思想。

\( K \)-means 算法过程如下:

首先从 \( n \) 个数据对象任意选择 \( k \) 个对象作为初始聚类中心,而对于所剩下的其它对象,则根据它们与这些聚类中心的相似度(距离),分别将它们配与其最相似的(聚类中心所代表的)聚类,然后再计算每个所获新聚类中心(该聚类中所在对象的均值),

不断重复这一过程,直到标准测度函数开始收敛为止。

相似度通过距离进行计算,常用的距离有相似度通过距离进行计算,常用的距离有 Minkoski 距离、Euclid 距离、Chebyshev 距离、Mahalanois 距离等,在此以欧式距离作为距离计算公式如下:

\[
E = \sum_{i=1}^{t} \sum_{j=1}^{t} \left\| p - m_i \right\|^2
\]

\[
D(x_i, x_j) = \sqrt{\sum_{r=1}^{m} \left\| X_{ir} - X_{jr} \right\|^2}
\]

式中,$P$ 为数据对象(样本点),$m_i$ 为 $C_i$ 类的均值,$D(x_i, x_j)$ 为点 $x_i$ 和 $x_j$ 的距离,为数据对象的维数。

$K$-means 算法步骤如下:

输入:聚类个数 $k$,以及包含 $n$ 个数据对象的数据库。

1) 从 $n$ 个数据对象任意选择 $k$ 个对象作为初始聚类中心;

2) 循环 3) 到 4) 直到每个聚类不再发生变化为止;

3) 根据每个聚类对象的均值(中心对象),计算每个对象与这些中心对象的距离;并根据最小距离重新对相应对象进行划分;

4) 重新计算每个(有变化)聚类的均值(中心对象)。

输出:满足方差最小标准的 $k$ 个聚类。

具体计算过程如图 5.3 所示。

\begin{figure}[h]
    \centering
    \begin{tikzpicture}[node distance=2cm, auto,>=latex']
        \tikzstyle{startstop} = [rectangle, rounded corners, minimum width=3cm, minimum height=1cm,text centered, draw=black]
        \tikzstyle{io} = [trapezium, trapezium left angle=70, trapezium right angle=110, minimum width=3cm, minimum height=1cm, text centered, draw=black]
        \tikzstyle{process} = [rectangle, minimum width=3cm, minimum height=1cm, text centered, draw=black]
        \tikzstyle{decision} = [diamond, minimum width=3cm, minimum height=1cm, text centered, draw=black]
        \tikzstyle{arrow} = [thick,->,>=stealth]

        \node (start) [startstop] {开始};
        \node (init) [process, below of=start] {初始化\\从313个结点中随机选取K个结点作为聚类中心,记作C1};
        \node (dijkstra1) [process, below of=init] {利用Dijkstra算法,计算网络中每个结点到每个中心的最短距离};
        \node (assign) [process, below of=dijkstra1] {将每个结点和距离它最近的中心归为一类};
        \node (dijkstra2) [process, below of=assign] {利用Dijkstra算法,对每一类结点,计算每个结点到类中其它所有结点的最短距离};
        \node (update) [process, below of=dijkstra2] {对每一类结点,确定到类中其它所有结点的最短距离的最大值最小的结点为该类的新的中心。记所有类的新的中心集合为C2};
        \node (decision) [decision, below of=update] {C1 = C2 ?};
        \node (end1) [process, below of=decision, yshift=-1cm] {聚类结束,C1即为K个区域的中心};
        \node (dijkstra3) [process, below of=end1] {利用Dijkstra算法,计算每个结点到聚类中心的距离,统计结点覆盖率和道路覆盖率};
        \node (end2) [startstop, below of=dijkstra3] {算法结束};
        \node (updateC1) [process, right of=update, xshift=4cm] {令C1=C2};

        \path[arrow] (start) -- (init);
        \path[arrow] (init) -- (dijkstra1);
        \path[arrow] (dijkstra1) -- (assign);
        \path[arrow] (assign) -- (dijkstra2);
        \path[arrow] (dijkstra2) -- (update);
        \path[arrow] (update) -- (decision);
        \path[arrow] (decision) -- node [near start] {Y} (end1);
        \path[arrow] (end1) -- (dijkstra3);
        \path[arrow] (dijkstra3) -- (end2);
        \path[arrow] (decision) -- node [near start] {N} (updateC1);
        \path[arrow] (updateC1) |- (dijkstra1);
    \end{tikzpicture}
    \caption{程序流程简图}
    \label{fig:flowchart}
\end{figure}

\subsubsection{5.2.4 计算结果}

分析题意可知,警车接到报警时平均时速为 $40\,\text{km/h}$ 时,3 分钟走过的路程为 $40 \times 3 / 60 = 2\,\text{km}$,而重点部位需要 2 分钟赶到,因此警车距重点部位不能超过

\(40 \times 2 / 60 \approx 1.33 \, \text{km}\) 。设定 \( k = 10, 12, 13, 14 \) 分别进行计算,结果如表 5.4 所示。

\begin{table}[h]
\centering
\caption{问题 1 计算结果}
\begin{tabular}{|c|c|c|l|}
\hline
序号 & 中心 & 道路覆盖率 & \multicolumn{1}{c|}{中心点标号} \\
 & 数目 K & (\%) & \\
\hline
1 & 10 & 69.02 & 308, 59, 211, 192, 304, 243, 183, 5, 312, 88 \\
\hline
2 & 12 & 81.66 & 298, 30, 160, 169, 310, 252, 56, 5, 312, 85, 157, 258 \\
\hline
3 & 13 & 91.76 & 209, 27, 169, 303, 291, 130, 253, 13, 87, 275, 78, 46, 39 \\
\hline
4 & 13 & 91.36 & 27, 169, 303, 276, 131, 253, 23, 107, 275, 85, 36, 39, 160 \\
\hline
5 & 14 & 76.43 & 298, 30, 189, 152, 304, 252, 96, 5, 312, 117, 111, 234, 255, 198 \\
\hline
6 & 14 & 87.09 & 178, 312, 62, 5, 298, 149, 290, 54, 113, 310, 210, 24, 255, 249 \\
\hline
7 & 14 & 90.74 & 276, 208, 27, 169, 303, 270, 130, 253, 18, 118, 275, 84, 46, 21 \\
\hline
\end{tabular}
\end{table}

当 \( k = 13 \) 时,满足 D1 中“警车在接警后 3 分钟内赶到现场的比例不低于 90\%”的要求,因此,依据表中第 3 组计算结果进行区域划分和警车布置。进一步分析可知,重点部位 (5112, 4806), (7434, 1332) 满足接警后两分钟到达的条件,而 (9126, 4266) 距最近的警车(中心点标号 78)1551 km,不能满足条件,因此我们将该布置点微调至 122 号交叉口。最终结果如图 5.4 所示:警车最少应布置 13 辆,位置为:209、27、169、303、291、130、253、13、87、275、122、46、39。

\begin{figure}[h]
\centering
\includegraphics[width=\textwidth]{image.png}
\caption{问题 1 区域划分及警车布置方案}
\end{figure}

\subsubsection{5.2.5 小结}

\( k \)-中心问题的模型比较经典,求解方法多样。我们选用 Dijkstra 算法可以方便的

求出各结点间的最短路径,同时 $k$-均值聚类算法在选址过程中具有一定的随机性,利用该方法进行区域划分和中心点布置,效果良好。

\subsection{5.3 问题 2 的求解}

\subsubsection{5.3.1 问题分析}

问题 2 要求寻找评价 110 警车巡逻效果显著程度的有关指标。问题 1 中我们利用离散设施选址的覆盖模型求解出最少巡逻警车的数量。但是在该问题中,不仅需要满足城市警车需求的经济性,还需要满足警车巡逻的安全性能,因此我们针对警车巡逻提出一套评估指标,对于警车的初始位置主要涉及覆盖分析和辖区划分分析,而对于巡逻过程中的指标主要有警车资源平均利用水平和警车流动覆盖率,为警车巡逻的实际效率求解提供了依据[6]。

\subsubsection{5.3.2 具体评价指标}

\paragraph{1. 覆盖指标}

\subparagraph{1) 警车响应覆盖率 $R_{C}$}

就警车巡逻效果评估而言,首先要考虑的问题就是警车在巡逻时响应报警时间的范围。针对该问题,我们指出评估的第一个指标——警车响应覆盖率。该指标定义:在警车巡逻时,所有警车响应报警路径总长度占城市所有路径总长度的比率。其数学描述如下:

\[
R_{C} = \frac{\sum L_{i}}{L_{T}} \times 100\%
\]

式中,$L_{i}$ 为各警车初始位置时响应报警路径的长度,$L_{T}$ 为该区域所有路径的总长度。

该指标重点反映了城市报警响应安全性能对警车巡逻方案的基本需求。在具体评估时,要求警车的报警响应覆盖率在 90\% 以上、重点区域为 100\% 作为评判依据,其值越大,说明警车巡逻的效果越显著。

\subparagraph{2) 警车交叉响应覆盖率 $C_{RT}$}

上面所求的警车响应覆盖率指标主要是评估警车巡逻方案满足安全响应时间性能基本需求的程度,警车只有满足一定数量才能达到安全需求。然而,并不是警车数量越多越好,还应当考虑警车分配和巡逻的经济性。因此,我们提出了警车交叉响应覆盖评估指标,其值是指在警车巡逻时重复响应的结点总个数占所有警车响应路径总结点个数的比率。其数学描述为:

\begin{align*}
\text{max } & \sum_{i} h_i y_i \\
\text{s.t. } & s.t. y_i \leq \sum_{j \in N_i} X_j, \forall i \\
& \sum_{j} X_j \leq 10 \\
& y_i \in \{0, 1\}, \forall i
\end{align*}

其中,$\sum\limits_{k=1}^{N} N_{eT}(k)$ 表示了各警车在其相关邻域里覆盖到所有路径总个数的和,$\sum N_{i}$ 表示各警车覆盖到所有路径的总个数。

该指标重点反映了警车初始配置的冗余性。在理想配置情况下,警车初始配置的交叉响应覆盖率为零,但是在实际情况下显然不可能发生。因此,根据警车辖区面积及响应范围确定的一般原则,可将警车初始配置的交叉响应覆盖率的评判依据设定为 10~15%,其值越小,说明警车配置的经济性越合理。另外,该指标也能够反映警车配置增援性能,交叉覆盖率越高,表明加强了警察辖区间的增援。

针对警察初始配置而言,以上两个覆盖指标之间关系是互相补充,互相制约。警车响应覆盖率着重于体现警车配置对整个城市街道提供保护的程度,而警车交叉响应覆盖率则着重强调警察第一出动响应范围重叠程度以及警察辖区之间的增援性能。在警车数量确定的情况下,不同的警车配置和巡逻方案的警车响应覆盖率和交叉响应覆盖率这两个指标呈相反趋势,即:如果警车在指定行车时间段内的覆盖率越高,则交叉覆盖率会越低;反之亦然。

\paragraph{2. 辖区范围指标}

由于警车一般是根据属地化原则响应报警事故,而辖区范围作为警车配置的重要组成部分,事实上对 110 指挥中心接警派遣具有重要影响。本指标是针对在已确定消防站数量和位置的情况下,不同辖区范围划分对响应报警点的影响而提出的。

\subparagraph{1) 最优辖区范围划分原理}

如果某条路径 $j$ 到警车 $k$ 初始位置的距离最短,则认为路径 $j$ 包含于警车 $k$ 的辖区,即警车 $k$ 的辖区为距离最短的路径集合。这种最优辖区范围划分原理是朴素的,具有一定 km 特征。

\subparagraph{2) 警车响应平均行车时间 $R_{a}$}

根据假定,我们将警车响应平均行车时间定义为:各警车对辖区内每条路径响应一次所需的时间总和除以城市的路径总数,其数学描述为:

\begin{align*}
\min \sum_{i} x_{i}(t) \\
s.t. \sum_{j} y_{j}(t) \geq 90\% \times 5142, \forall i, j
\end{align*}

式中,$\delta_{ij}=\begin{cases} 0 & \text{如果路径 } j \text{ 不在警车 } i \text{ 的辖区内} \\ 1 & \text{如果路径 } j \text{ 在警车 } i \text{ 的辖区内} \end{cases}$,$S_{P_{ij}}$ 为每条路径响应一次所需的时间,$N$ 为城市的路径总数。

该指标直接反映城市警车响应平均行车时间性能。理论上来说,该响应时间越小,说明各警车的辖区划分越合理。

\paragraph{3. 巡逻效果指标}

\subparagraph{1) 警车资源平均利用水平 $L_{C}$}

警车在巡逻过程中,每辆警车在覆盖区域内可响应一定长度的路径,因此我们引入警车资源平均利用水平 $L_{C}$ 这一指标。该指标定义:在警车巡逻时,所有警车响应报警路径总长度与巡逻警车数量的比值。其数学描述如下:

\[
L_{C}=\frac{\sum L_{i}}{n} \times 100\%
\]

其中,$L_{i}$ 为各警车初始位置时响应报警路径的长度,$n$ 为该巡逻方案中警车的数量。

该指标有效地衡量了警车的利用水平,理论上来说,其值越大,警车的利用水平越高。

\subparagraph{2) 警车流动覆盖率 $R_{F}$}

警车在巡逻过程中,巡逻路线确定后,我们可以计算其流动覆盖率。该指标定义:在警车巡逻路径上所有结点响应报警覆盖的结点个数占城市道路结点总个数的比率。其数学描述如下:

\[
R_{F}=\frac{\sum N_{i}}{N_{T}} \times 100\%
\]

式中,$N_{i}$ 代表了巡逻路径上各结点响应报警的结点个数,$N_{T}$ 表示该区域所有结点的总个数。

该指标体现了巡逻路径的覆盖广度,在有限数量的警车选择巡逻路径时有着重要的意义。

\subsubsection{5.3.3 小结}

通过以上 5 个指标的评定,我们可以很好地分析警车巡逻的效果。根据覆盖指标和辖区范围指标,可以有效地对警车巡逻初始位置的选定进行评估,而通过巡逻效果

指标的计算,则可以帮助我们对比各个巡逻方案的优劣,从而确定使巡逻效果最显著的方案。

\subsection{5.4 问题 3 的求解}

\subsubsection{5.4.1 问题分析}

在问题 1 中,我们求解出满足覆盖要求 D1 情况下的最少警车数量。在本问题中,要求在满足 D1 条件下,给出能尽量满足 D2 条件的警车巡逻方案,并根据问题 2 中定义的评价指标对此巡逻方案进行评估。对于该问题,由于巡逻的警车是动态的,位置是时间的变量,所以我们不能采用问题 1 中假设某警车为静止状态把问题简化为选址问题的方法,而是先确定警车的巡逻路线,然后在巡逻路线上合理地配置警车,最后评估巡逻方案并根据评价指标对巡逻方案进行合理的调整。

\subsubsection{5.4.2 模型分析}

警车在巡逻过程中,位置是时间 \( t \) 的变量,既要最小化警车数,又要考核巡逻效果。当图的结点之间距离足够小时,我们可以认为警车任何时刻都在某个结点上。所以给原图的每条路上均匀插入若干个结点,使得得到的新图每条路径长度不超过 50m,就可以认为警车任何时刻都在某个结点上。得到的新图总共有 5142 个结点,5293 条路径。给每个点对应一个与时间 \( t \) 相关的参数 \( x_i(t) \),\( x_i(t) = 0 \) 表示 \( t \) 时刻点 \( i \) 处没有警车,\( x_i(t) = 1 \) 表示 \( t \) 时刻点 \( i \) 处配置了警车。为了实现要求 D1 和 D2,建立如下模型:

\begin{align*}
\min \sum_i x_i(t) \\
s.t. \sum_j y_j(t) \geq 90\% \times 5142, \forall i, j
\end{align*}

其中 \( y_j(t) \) 等于对 \( y_{ij}(t) \) (\( \forall i \))的逻辑或运算结果。\( y_j(t) \) 表示的是 \( t \) 时刻点 \( j \) 是否被某个位置的警车覆盖,不计重复覆盖次数。约束条件考虑的是被警车覆盖的结点数是否超过 90\%,因为在这个问题里面边长足够小,可以近似认为报警事件只发生在结点上。

\subsubsection{5.4.3 模型求解}

由于插入新道路结点的数目庞大,计算代价过高,上述 0-1 规划模型的求解比较困难。因此,对于该实例,我们用图论的知识分析了地图,先不考虑地图最右边道路比较稀疏并且结点较少的区域,安排一个大环线的巡逻路线,使其在保持线路平滑的前提下,尽量经过问题 1 聚类计算出来的结点,这样可以覆盖大部分周边区域,具体的环形区域如图 5.5 所示。

\begin{figure}[h]
    \centering
    \includegraphics[width=\textwidth]{image.png}
    \caption{问题 3 中大环线巡逻路线}
    \label{fig:patrol_route}
\end{figure}

通过计算可得,该环线的总长度为 26825 米。由于要满足警车在接警后 3 分钟内赶到现场的比例不低于 90\%,即警车在巡逻过程中对周围 3km 道路的覆盖率要达到 90\%,因此我们要确定在环线上布置警车的数量实现对环线周围道路覆盖率最大的目标。通过对警车数量的模拟,发现在环线上配置 9 辆车时(即前后车平均间隔约为 3Km)达到最优。

为了确保重点部位能够 100\% 响应报警事件,即要求重点部位的巡逻警车保证 2 分钟之内能到达,我们单独考虑重点部位警车的安排。由于重点部位(7434,1332)在环线上,而环线上警车间隔为 3km,我们又在模型假设 4 中对其接警范围做了假定,由此可得该点满足要求。而对重点部位(5112,4806)我们考虑分别在其周围半径约为 1.3km 的范围内也寻找较规则的方形区域安排一辆警车做循环巡逻。

接着,考虑道路比较密集的老城区(即左下部位的正方形城区),由于老城区的周边路径比较规范(近似平行于坐标轴),可以安排两辆警车沿着该环线(标号 254-162-159-259)巡逻,而该方案证明满足要求 D1 和 D2。

\begin{figure}[h]
    \centering
    \includegraphics[width=\textwidth]{image.png}
    \caption{问题 3 中小环线巡逻路线}
    \label{fig:small_loop_route}
\end{figure}

最后,对右边区域进行,这块区域与问题 1 中聚类结果中聚类中心为 39 号结点和 253 号结点的周围区域近似相同,因此我们先选择这两个聚类中心为警车的初始位置可以达到最大的道路覆盖率,排除掉这两个聚类中心覆盖的结点后,再考虑周围未被覆盖的结点区域。根据观察,剩下来的结点分为上下两个区域,可以近似地选取这两个区域的几何中心 24 号结点和 116 号结点得到剩余结点的最佳覆盖率。由于在现实情况中,110 警车并不需要全部出动巡逻,而是选择部分巡逻警车留守的策略来达到最好的巡逻目标。同时,在该问题里,由于地图右边区域道路稀疏,结点较少,考虑到警车配置的经济性,我们在该区域中配置四辆分别固定在两个聚类中心和两个几何中心上,方便接警后能迅速地到达该区域中尽可能多的结点。

在巡逻路线确定之后,我们需要计算白天任意连续 4 小时内的各警车的正常巡逻位置数据,由于实际道路坐标复杂,环线道路为多段折线不以计算,因此为了节约计算成本,我们对地图坐标做了近似的处理。近似处理的过程:以大环线巡逻路线为例。大环线 $303 \rightarrow 29 \rightarrow 32 \rightarrow 297 \rightarrow 303$,其顶点实际坐标分别为:303(7434,270)、29(7380,7128)、32(1458,7092)、297(216,414),且其由多条线段曲折构成。现将其近似为直角梯形,每条边的端点坐标为两个端点实际坐标的平均值,则可近似为:303(7407,342)、29(7407,7110)、32(1458,7110)、297(216,342)。其它巡逻路线也均照此法近似。

\subsubsection{5.4.5 求解结果}

该方案共布置 16 辆警车,布置方案和初始位置如下:

1. 大环线上 9 辆警车,坐标为:(216,342)、(3216,342)、(6216,342)、(7407,2151)、(7407,5151)、(6366,7110)、(3366,7110)、(1261,6036)、(719,3085);

2. 重点部位(5112,4806)附近 1 辆巡逻警车,坐标为:(5607,4396);

3. 左下部位的正方形城区 2 辆警车巡逻,坐标为:(3105,2097)、(5157,3771)

4. 右边区域 4 辆警车定点布置,坐标为:(10296,7182)、(14040,6840)、(9162,4662)、(11952,2142)。

警车的巡逻路线和方向如图所 5.7 所示:

\begin{figure}[h]
    \centering
    \includegraphics[width=\textwidth]{image.png}
    \caption{问题 3 警车布置和巡逻方案}
    \label{fig:5.7}
\end{figure}

\subsubsection{5.4.6 评价指标的计算和模拟}

警车响应覆盖率 $R_{c} = \frac{241725}{253685} \times 100\% = 95.29\%$

重点区域警车响应覆盖率 $R_{c} = 100\%$

警车交叉响应覆盖率 $C_{RT} = \left[\frac{361}{279} - 1\right] \times 100\% = 29.4\%$

警察响应平均行车时间 $R_{a} = \frac{241725}{40 \times 313} = 19.31s$

警车资源平均利用水平 $L_{c} = \frac{241725}{16} = 15107$

警车流动覆盖率 $R_{F} = 94.89\%$

经计算可知,区域中有 $94.89\%$ 的结点位于警车巡逻轨迹 $2\mathrm{~km}$ 范围内。随机抽取 $10$ 组警车巡逻过程中的位置坐标,计算其是否满足条件 D1 中“接警后三分钟内赶到现场的比例不低于 $90\%$”,结果如表 5.5 所示:

\begin{table}[h]
\centering
\caption{问题 3 模拟覆盖率结果}
\begin{tabular}{|c|c|}
\hline
序号 & 符合要求的比率(\%) \\
\hline
1 & 94.26 \\
\hline
2 & 93.32 \\
\hline
3 & 92.61 \\
\hline
4 & 94.01 \\
\hline
5 & 92.42 \\
\hline
6 & 93.66 \\
\hline
7 & 93.55 \\
\hline
8 & 93.48 \\
\hline
9 & 92.11 \\
\hline
10 & 95.29 \\
\hline
平均值 & 93.47 \\
\hline
\end{tabular}
\end{table}

\subsubsection{5.4.7 小结}

通过该题的计算,我们在问题 1 求解最少警车数量的基础上,增加了 3 辆警车,并设计了警车动静结合的有效巡逻方案,使巡逻过程中警车的平均道路结点覆盖率达到了 $93\%$ 以上,并且在巡逻过程中在路线平滑的前提下尽可能多地巡逻到结点,使警车起到了很好的震慑作用,显著地提高了巡逻效果。

\subsection{5.5 问题 4 的求解}

\subsubsection{5.5.1 问题分析}

问题 4 要求在问题 3 的基础上,再考虑 D3 条件,给出警车的巡逻方案及其评价指标值。巡逻的最大作用是震慑犯罪分子。为了不让犯罪分子掌握警车巡逻的规律,巡逻方案必须具有一定的隐蔽性。在这里我们对“隐蔽性”的理解为道路上某点在出现警车的时刻是不可预测的。为此,我们可以先制定优化的巡逻路线,再人为添加随机因素,即制定同一条路线的不同时间表,使警车每天在不同的时间经过同一个地点;或者制定若干条评价指标相当的巡逻路线,警车随机选择其中的一条巡逻。

\subsubsection{5.5.2 问题求解}

由于问题 3 中对巡逻方案的设计采取外围大环路,重点区域和中心城区小环路,边缘区域顶点的方案,因此在考虑方案隐蔽性时,我们的重点是放在外围大环路和重点区域以及中心城区的小环路上。

在这里,首先考虑中心城区的正方形区域中,由于我们采用了两辆警车绕外围巡逻,可以在任意时间内覆盖该区域的所有结点,因此我们只需保证这两辆警车在保持一定相对距离的情况下,绕城区中任一道路进行巡逻,都可以满足其对区域道路的覆盖要求。

盖率达标。这样就大大地提高了巡逻规律的隐蔽性。

而剩下的巡逻环线,由于警车在环线上的相对距离恒定,因此在任一时刻这些警车的总覆盖率都能保证基本一致。我们知道警车上都配置有 GPS 卫星定位系统,110 中心能迅速同步的对所有的警车进行调控,为了提高巡逻的隐蔽性,我们可以使这些环线上的警车在任一时刻转向(即可以从顺时针变为逆时针,或者反之),只要保证其相对距离不变,就不会影响到其它的评价指标如覆盖率等等,但是,由于转向的时间是随机因素,因此也可以大大地提高巡逻的隐蔽性。

\subsubsection{5.5.3 小结}

针对该问题中要求提高巡逻的隐蔽性,在与问题 3 巡逻方案保持基本一致的前提下,我们针对中心城区提出了随机选择巡逻街道,而大环线和重点部位的小环线上采取随机转向的方案,可以有效地提高警车巡逻的隐蔽性。

\section{5.6 问题 5 的求解}

\subsection{5.6.1 问题分析}

该问题要求该区域仅配置 10 辆警车,如何在尽量满足要求 D1 和 D2 的情况下制定巡逻方案。在这里我们根据问题 1 的求解结果发现,在仅满足要求 D1 时,该区域也需要配置至少 13 辆警车。因此,仅用 10 辆警车是无法同时满足要求 D1 和 D2 的。那么在该问题中,我们则寻求一个主要目标——即实现巡逻路线的最大覆盖率,在此基础上设计巡逻路线,尽可能达到 D1 和 D2 的要求。

\subsection{5.6.2 模型分析}

在仅有 10 辆警车的情况下,我们重点覆盖道路密度较大的区域,合理安排巡逻路线。选用的模型为离散设施选址的最大覆盖模型。具体描述为:

\begin{align*}
\text{max } & \sum_{i} h_i y_i \\
\text{s.t. } & s.t. y_i \leq \sum_{j \in N_i} X_j, \forall i \\
& \sum_{j} X_j \leq 10 \\
& y_i \in \{0, 1\}, \forall i
\end{align*}

$h_i$ 为需求点 $i$ 的需求;当结点 $i$ 被覆盖时,$y_i$ 为 1,否则为 0。目标函数是求有限资源条件下所能覆盖的最大需求量。约束条件 1 确定设施为哪个在特定距离内的需求点提供服务。约束条件 2 限定设施数量小于等于警车的数量 10。在警车巡逻实例中,我们先假设各个点的需求规模是一样的,这时的目标函数就是最大化需求点数量。如果考虑模型更切合实际,则应讨论不同点的需求不同,这个条件我们放在问题 7 里讨论。

\subsection{5.6.3 模型求解}

通过模型的分析以及前面问题的求解, 我们不难发现数量较少的警车固定时若选取区域的中心可以使得其覆盖率最大。如果采取问题 3 中的巡逻方案, 则在实现巡逻效果显著提高时需要增加警车的数量。该问题中警车的数量不足以实现 D1, 而由前面的结论我们又得出定点的警车覆盖率大, 若我们采取将警车固定在覆盖率最大的定点, 又不能达到 D2 的要求, 因此, 权衡这两个要求, 我们采取了警车在覆盖率最大的定点周围上做小范围的直线或环线巡逻方案, 可以达到较好的效果。

求解的过程中, 我们以聚类个数 $k=10$ 对区域内的结点进行 $K$-均值聚类, 得到 10 辆警车各自辖区的中心位置 (如图 5.8 所示)。接下来, 通过启发式算法, 我们计算得出 10 辆警车的巡逻方案: 其中 3 辆定点待命, 剩余的 7 辆沿着聚类中心周围的直线或环线做小范围的巡逻。

\begin{figure}[h]
    \centering
    \includegraphics[width=\textwidth]{image.png}
    \caption{问题 5 的 K-均值聚类结果}
    \label{fig:5.8}
\end{figure}

\subsection{5.6.4 求解结果}

该方案共布置 10 辆警车, 方案如下:

1、结点 138 至 148 之间布置 1 辆警车巡逻, 初始位置为 (9180,4074), 该车始终保持覆盖重要部位 (9126,4266);

2、结点 303 至 238 之间布置 1 辆警车巡逻, 初始位置为 (7434,1368), 该车始终保持覆盖重要部位 (7434,1332);

3、结点 125 处布置一辆警车, 延方形区域 (125-87-96-124) 巡逻, 初始坐标为:

(5607,4356),该车始终保持覆盖重要部位(5112,4806);

4、结点27至28之间布置1辆警车巡逻,初始位置为(5616,7110);

5、结点36至92之间布置1辆警车巡逻,初始位置为(1236,5814);

6、结点255至260之间布置1辆警车巡逻,初始位置为(432,2061);

7、结点25至39之间布置1辆警车巡逻,初始位置为(9162,6993);

8、结点208、249、253处各定点布置一辆警车。

警车的巡逻路线和方向如图5.9示:

\begin{figure}[h]
    \centering
    \includegraphics[width=\textwidth]{image.png}
    \caption{问题5警车布置和巡逻方案}
    \label{fig:5.9}
\end{figure}

\subsection{5.6.5 评价指标的计算和模拟}

警车响应覆盖率 \( R_{c} = \frac{198296}{253685} \times 100\% = 78.17\% \)

重点区域警车响应覆盖率 \( R_{c} = 100\% \)

警车交叉响应覆盖率 \( C_{RT} = \left[ \frac{254}{226} - 1 \right] \times 100\% = 12.39\% \)

警察响应平均行车时间 \( R_{a} = \frac{198296}{40 \times 313} = 15.84 \, \text{s} \)

警车资源平均利用水平 \( L_{c} = \frac{198296}{10} = 19829.6 \)

警车流动覆盖率 $R_{F}=94.57\%$

经计算可知,区域中有 296 个结点位于警车巡逻轨迹 2km 范围内,占全部结点的 94.57%。随机抽取 10 组警车巡逻过程中的位置坐标,计算区域中满足条件 “接警后三分钟内赶到现场的比例不低于 90%” 的结点的比率,结果如下表:

\begin{table}[h]
\centering
\caption{问题 5 模拟覆盖率结果}
\begin{tabular}{|c|c|}
\hline
序号 & 符合要求的比率(\%) \\
\hline
1 & 76.12 \\
\hline
2 & 75.43 \\
\hline
3 & 79.6 \\
\hline
4 & 80.97 \\
\hline
5 & 76.75 \\
\hline
6 & 75.71 \\
\hline
7 & 74.86 \\
\hline
8 & 77.54 \\
\hline
9 & 78.4 \\
\hline
10 & 78.17 \\
\hline
平均值 & 77.35 \\
\hline
\end{tabular}
\end{table}

\subsubsection{5.6.7 小结}

该问题中,在有限警车数量的情况下,我们采用警车定点和小范围巡逻相结合的方案,通过仿真模拟,验证了该方案达到了较好的评价指标,尽可能地满足了 D1、D2 的要求。

\subsection{5.7 问题 6 的求解}

\subsubsection{5.7.1 问题分析}

在问题 6 中首次将警车接警速率这个变量引入到巡逻效果的考虑中来。在这里,将警车接警后的平均行驶速度提高到 50km/h 后,求解满足要求 D1、D2 的巡逻方案。通过计算,我们得出,平均行驶速度提高后,警车接警响应的覆盖范围达到 $50 \times 3 / 60 = 2.5 \mathrm{~km}$,覆盖范围增大,即可以采用更少数量的警车达到预期的巡逻效果。

\subsubsection{5.7.2 问题模型}

问题 6 的模型跟问题 3 完全一致,只是警车覆盖的邻域改变。在问题 3 的图形和基础上,沿用 0-1 规划的模型:
\begin{align*}
\min \sum_{i} x_{i}(t) \\
s.t. \sum_{j} y_{j}(t) \geq 90\% \times 5142, \forall i, j
\end{align*}

其中 $y_{j}(t)$ 等于对 $y_{ij}(t)$ ($\forall i$) 的逻辑或运算结果。$y_{j}(t)$ 表示的是 $t$ 时刻点 $j$ 是否被某个位置的警车覆盖,不计重复覆盖次数。约束条件考虑的是被警车覆盖的结点数是否超过 90%。

\subsubsection{5.7.3 模型求解}

由于问题 6 与问题 3 采用的模型相同,因此求解过程相似。区别仅为警车覆盖邻域增大,大环线上的警车数量可以相对减少,同时,中心城区不需要再采用两辆警车沿外围巡逻,而是使用一辆警车沿城区内部的较小环线巡逻。

\subsubsection{5.7.4 求解结果}

该方案共布置 12 辆警车,具体如下:

1. 大环线上 7 辆警车,坐标为:(216, 342)、(4043, 342)、(7407, 805)、(7407, 4632)、(6058, 7110)、(2231, 7110)、(907, 4106);

2. 结点 229 附近布置一辆警车巡逻,初始位置为:(5031, 2529),巡逻路线为:229 $\rightarrow$ 209 $\rightarrow$ 205 $\rightarrow$ 179 $\rightarrow$ 174 $\rightarrow$ 231 $\rightarrow$ 229;

3. 重点部位 (5112, 4806) 附近布置 1 辆警车巡逻,初始位置为:(5607, 4396),巡逻路线为:125 $\rightarrow$ 87 $\rightarrow$ 96 $\rightarrow$ 124;

4. 重点部位 (9126, 4266) 附近布置 1 辆警车巡逻,初始位置为:(9990, 5256),巡逻路线为:85 $\rightarrow$ 122 $\rightarrow$ 123 $\rightarrow$ 84;

5. 右边区域 2 辆警车定点布置,坐标为:(11574, 7164)、(11952, 2142)。

警车的巡逻路线和方向如图 5.10 所示:

\begin{figure}[h]
    \centering
    \includegraphics[width=\textwidth]{image.png}
    \caption{问题 6 警车布置和巡逻方案}
    \label{fig:5.10}
\end{figure}

\subsubsection{5.7.5 评价指标的计算和模拟}

警车响应覆盖率 $R_{c} = \frac{243665}{253685} \times 100\% = 96.05\%$

重点区域警车响应覆盖率 $R_{c} = 100\%$

警车交叉响应覆盖率 $C_{RT} = \left[\frac{372}{279} - 1\right] \times 100\% = 33.3\%$

警察响应平均行车时间 $R_{a} = \frac{243665}{50 \times 313} = 15.57 \, \text{s}$

警车资源平均利用水平 $L_{c} = \frac{243665}{12} = 20305$

警车流动覆盖率 $R_{F} = 96.17\%$

经计算可知,区域中有 $96.17\%$ 的结点位于警车巡逻轨迹 $2 \, \text{km}$ 范围内。随机抽取 10 组警车巡逻过程中的位置坐标,计算其是否满足条件 D1 中“接警后三分钟内赶到现场的比例不低于 $90\%$”,结果如表 5.7 所示:

\begin{table}
\centering
\caption{问题6模拟计算结果}
\begin{tabular}{|c|c|}
\hline
序号 & 符合要求的比率(\%) \\
\hline
1 & 92.95 \\
\hline
2 & 97.54 \\
\hline
3 & 94.34 \\
\hline
4 & 95.75 \\
\hline
5 & 93.51 \\
\hline
6 & 97.04 \\
\hline
7 & 93.8 \\
\hline
8 & 94.61 \\
\hline
9 & 96.29 \\
\hline
10 & 96.05 \\
\hline
平均值 & 95.19 \\
\hline
\end{tabular}
\end{table}

\subsubsection{5.7.6 小结}

本题中考虑了警车接警后平均行驶速度提高对巡逻方案的影响。通过计算和仿真模拟可得,速度提升导致警车覆盖邻域增大,因此可以采用较少的警车达到显著的巡逻效果。

\subsection{5.8 问题7的求解}

\subsubsection{5.8.1 问题分析}

由于在前面问题的求解时,为了简化模型,我们采取了一些假设。但在实际情况中,有部分假设是不能完全满足的。因此问题7需要我们找出除上面求解过程出现以外的因素和情况,并给出相应的解决方案。这对于求解问题的实际方案是非常有意义的。

\subsubsection{5.8.2 问题求解}

在以上的求解过程中,我们假设的报警事件以相等的概率发生在一天的各个时间段,在每个点上发生的概率也相等。但在现实问题中,应考虑需求预测,根据各个点的需求安排警车。模型可修改为:

\begin{align*}
\max \sum_{i} h_i y_i \\
\min \sum_{j} X_j \\
s.t. & y_i \leq \sum_{j \in N_i} X_j, \forall i \\
& y_i \in \{0, 1\}, \forall i
\end{align*}

$h_{i}$ 为需求点 $i$ 的需求;当结点 $i$ 被覆盖时,$y_{i}$ 为 1,否则为 0。

同时,在问题的讨论中,我们只考虑了增加警车会增大成本。如果考虑警车配置在不同的点和区域的成本不同,巡逻的路线不同也会产生不同的费用,则不能单单把警车数量最小化作为目标函数,而应该采用复合的成本函数。

对于警车选址,应该采用多目标方法,不仅要考虑节约成本,还要充分考虑公平性和效率性。首先要求警车能够覆盖所有需求区域,在考虑具体目标时,一是从快速反应或公平性考虑,要求需求点的最大距离(或最大加权距离)为最小;二是从超额覆盖和备用设施考虑,要求警车超额覆盖需求区域的总权重为最大;3 是从警车的流动性考虑,警车位置实时变化,覆盖区域也实时变化,要求充分考虑警车之间的配合度。

\subsubsection{5.8.3 小结}

在现实情况中,对于该问题的求解我们不能只做理想的假设,应结合实际的巡逻成本、报警概率分布不均等因素,综合地制定警车的配置及巡逻方案。

\section{参考文献}

[1] 王非,徐渝,李毅学. 离散设施选址问题研究综述[J]. 运筹与管理, 2006(5): 64-69.

[2] 刘显宾,唐常杰,陈瑜,张悦,李川,代术成. p-cluster: 基于聚类的平面 p-中心定位算法[J]. 四川大学学报(自然科学版), 2009 年 1 月, 46 卷 1 期: 80-84.

[3] 周品,赵新芬,MATLAB 数学建模与仿真[M]. 北京,2009 年 4 月: 322-327.

[4] 蔡慧,刘洪波,韩国栋. 基于 K 均值聚类的随机网络拓扑模型,计算机工程与设计[J]. 2009, 30(5):1089-1091.

[5] Khan S S, Ahmad A. Cluster center initialization algorithm for K-means clustering [J]. Pattern Recognition Letters, 2004, 25(11): 1293-1302.

[6] 吴美文,吴军,胡传平. 城市消防站布局评估指标量化分析[J]. 自然灾害学报, 15 卷 5 期: 2006 年 10 月, 162~167.

\end{document}