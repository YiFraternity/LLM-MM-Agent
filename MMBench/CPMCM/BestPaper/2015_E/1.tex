\documentclass{article}
\usepackage{amsmath}
\usepackage{amssymb}

\begin{document}

\begin{center}
\textbf{第十二届“中关村青联杯”全国研究生数学建模竞赛}
\end{center}

\begin{table}[h]
\centering
\begin{tabular}{l l}
学 校 & 中国石油大学(华东) \\
\hline
参赛队号 & 10425010 \\
\hline
队员姓名 & 1. 冯巨龙 \\
 & 2. 易维淋 \\
 & 3. 尹怀强 \\
\hline
\end{tabular}
\end{table}

\title{数控加工刀具运动的优化控制}
\maketitle

\begin{abstract}
数控加工技术正朝着高速高效高精度方向发展,高速加工要求机床各运动轴都能够在极短的时间内达到高速运行状态并实现高速准停,研究开发数控加工刀具运动满足高速、高精度要求的、有效柔性加减速控制方法,已成为现代高性能数控系统研究的重点。本文在深入研究数控加工优化控制的基础上,主要完成以下工作:

\begin{itemize}
    \item 针对问题一:当加工型线为折线时,将折线分为直线加工段与转角加工段,对两段分别建立进行实时加工优化算法,并能保证两段加工优化算法有效衔接。直线段采取一种新的 $S$ 型加减速控制方法,通过选取初速度、末速度计算得到加速区、匀速区和减速区的时间长度,消除一般 $S$ 曲线加减速方法在单一路径中初末速度相同的约束。转角段采用转角多插补周期过渡的方法,且经过前瞻处理后使得整体速度提高,同时该方法满足最大加工误差、最大速度、最大加速度和最大加加速度的要求。此外,分别以相邻两折线段夹角为 $90^\circ$ 和 $135^\circ$ 为例,具体讨论通过折线交点时对应各坐标运动速度的变化。结果表明,在该算法模型下,当折线夹角较大即加工型线较平坦时可以以较大的加工速度通过转角,而当折线夹角较小时,亦可以相对较小的非零速度通过转角,从而在满足加工要求的情况下提高了加工速度。
    \item 针对问题二:当加工型线是由直线段和圆弧段(相切或不相切)组成的连续曲线时,加工过程可以分为直线加工段、直线与圆弧的过渡段和圆弧加工段。直线加工段的优化采用问题一中所述的新型 $S$ 曲线加减速控制算法。直线与圆弧的过渡段采用“圆弧转接法”来处理两相邻运动矢量拐角处的速度,转接小圆弧的半径能随加工精度的变化而自动地调整,通过限制加工小圆弧的最大速度来限制拐角处的速度,在拐点处通过插补小圆弧来提高拐点处的转接速度。圆弧加工段可以视为插补一个半径较大的圆弧。通过理论分析,我们得出,在该优化算法下,圆弧半径越大,插入圆弧的插补速度越大,算法效率越高。最后通过示例对模型
\end{itemize}

进行仿真检验,验证了算法的可行性与高效性。

针对问题三:在问题二的基础上,算法模型考虑瞬时启动速度和瞬时启动加速度。启动瞬时速度与启动瞬时加速度为直线段的初始参数,利用超前分析方法对圆弧过渡阶段与圆弧加工阶段进行分析得到直线段末速度,通过新型 $S$ 曲线加减速控制算法得到该问题下的加速区、匀速区和减速区的时间长度对直线阶段曲线进行加工,其后的加工与问题二一致。仿真结果表明,所建立的优化模型能够在满足加工条件的基础上有效的提高加工速度。

针对问题四:在前文基础上,对 $S$ 型曲线的加减速控制方法的优缺点进行分析。我们得出:$S$ 型曲线的加减速控制方法的加速度都是连续变化的,从而控制机床运动速度平滑变化。这种方法对减小机床冲击有一定作用和效果。但是,其在单一路径加工过程中存在初速度和末速度相等的限制,使得加工过程造成时间增加,电机变速频繁,影响效率;其次,加加速度 $j$ 仍是阶跃变化的,在阶跃变化的上升沿和下降沿,对机床进给系统有振动和冲击,影响机床运行平稳性。基于此,本文提出引入一种加加速度连续的加减速方法,相对 $S$ 曲线加减速方法,该方法使加速度和速度曲线更加平滑,有效提升机床运行平稳性。
\end{abstract}

\tableofcontents


\section{一、问题重述}

数控加工技术正朝着高速高效高精度方向发展,高速加工要求机床各运动轴都能够在极短的时间内达到高速运行状态并实现高速准停,研究开发数控加工刀具运动满足高速、高精度要求的、有效柔性加减速控制方法,已成为现代高性能数控系统研究的重点。

基于计算机的数控系统的工作原理是:首先通过计算机组成的数控编程系统对读入的零件信息进行存储和译码等处理后通过输入装置将它们传输给加工控制系统,然后由数控系统对输入的指令进行信息处理和轨迹插补计算出数控机床各坐标轴方向上刀具运动的控制信息(这个环节就是题目要求完成的任务),进而通过机床驱动以及机床运动将刀具在各坐标轴方向上的运动合成为刀具实际加工轨迹和速度控制,加工出所需的工件。

上述环节的难点在于数控机床对加工刀具在三个坐标轴方向的运动(本题不考虑更复杂的五轴控制等)实行的是分别控制,但显然它们之间必须相互协调;加工刀具行走的路线一定是一系列首尾相接的直线段,因此加工刀具的运动轨迹一般与工件几何形状之间肯定存在误差;每一机床都有对应的分辨率,上述任一直线段对应的坐标增量记为 \(\Delta x, \Delta y, \Delta z\),则 \(\Delta x, \Delta y, \Delta z\) 的长度一定都是分辨率的整数倍,故加工刀具的运动方向受限制,并影响到加工刀具在三个坐标轴方向上的速度、加速度;要求机床运动平稳,速度光滑、加速度连续等。

加工刀具运动的优化控制则是在数控机床所提供的精度、速度、加速度等限制条件下,寻求对机床刀具在各坐标轴方向上的运动进行合理控制,进而优化其加工效率。类似方程式赛车,有经验的车手可根据赛道特点及赛车的性能(如速度、加速度等)对赛车的行进路线及过程进行优化。而对于一般曲线加工,加工控制算法就是在满足误差要求的条件下,通过插补的方法,找出若干小直线段组成加工刀具的运动轨迹,同时计算出刀具对应的运动速度、加速度。

目前,数控加工对单个坐标运动的控制方法有多种,其中较有代表性的是基于 S 型曲线的加减速控制方法。其特点是将加减速过程分为 7 个阶段(每一段对应的加加速度为常量):加加速段、匀加速段、减加速段、匀速段、加减速段、匀减速段、减减速段,从而渐变地控制各段的加速度使机床运动速度按 S 型曲线形式平滑变化,以保证速度光顺,加速度连续,在一定程度上增强机床运行的平稳性。S 型加减速控制曲线如图 1 所示。

\begin{figure}[h]
    \centering
    \includegraphics[width=\textwidth]{image.png}
    \caption{S 型加减速控制曲线的速度、加速度、加加速度随时间变化规律}
\end{figure}

在基于 S 型曲线的运动过程中,对于速度和加速度都有一定的限制性要求,速度 \( V \) 不大于机床最大速度 \( V_{\max} \),加速度 \( a \) 不大于机床最大加速度 \( a_{\max} \),加加速度为常量 \( J_{\text{const}} \)。

在目前采用的 S 型速度控制曲线中,加速度每次都是从 0 增加,最后又降为 0,而在实际运动过程中电机启动时允许有一个瞬时启动加速度 \( a_0 \),即认为加速度可以从 0 瞬间提高到瞬时加速度 \( a_0 \),或瞬间从 \( a_0 \) 下降到 0,速度也有类似功能,这样整个加速过程及速度的变化规律有一些改变。

假设不考虑刀具尺寸大小及刀具磨损,加工刀具抽象为一点。在深入研究数控加工优化控制的基础上,需要完成以下工作:

\begin{enumerate}
    \item 设加工型线为折线,在指定加工误差(指在加工型线的法线方向上加工型线与刀具实际轨迹的差值的最大值)的条件下,建立实时加工优化控制算法,当相邻两折线段夹角为 \( 90^\circ \) 和 \( 135^\circ \) 时,讨论通过折线交点时对应各坐标运动速度的变化;
    \item 设加工型线是由直线段和圆弧段(相切或不相切)组成的连续曲线,在指定加工误差的条件下,不考虑瞬时启动加速度及瞬时启动速度,建立实时加工优化控制算法,讨论圆弧半径的变化对算法效率的影响;并应用所建立的模型对下面的加工路径示例进行检验;
    \item 在第 2 问基础上,考虑瞬时启动加速度及瞬时启动速度,建立相对应的实时加工优化控制算法;并应用所建立的模型对下面的加工路径示例进行检验;
    \item 结合前 3 问,分析 S 型曲线的加减速控制方法的优缺点,在满足精度和速度要求的条件下,建立能提高机床运行平稳性的优化控制运动模型(如刀具在各坐标轴方向上的运动满足加加速度连续变化等)。
\end{enumerate}

\section{二、问题分析}

问题一:当加工型线为折线时,考虑将其分为直线部分与转角部分分别进行控制,再将其结合起来进行实时加工优化控制。折线加工示意图如下图所示。

\begin{figure}[h]
    \centering
    \includegraphics[width=0.8\textwidth]{折线加工示意图.png}
    \caption{折线加工示意图}
    \label{fig:折线加工示意图}
\end{figure}

针对加工型线的直线段部分(图 \ref{fig:折线加工示意图} 中 OA 与 CD 段)可以采取 S 型加减速控制方法,并采取优化算法消除其加减速方法在单一路径中初末速度相同的约束。在满足加工精度和机床各驱动轴最大加速能力的条件下,为提高加工速度,转角部分 (AB 与 BC 段) 可以采用转角多插补周期过渡的方法,以提高转角的通过速度,同时该方法易于满足最大加工误差、最大速度、最大加速度和最大加加速的要求。对转角角度进行分析,通过优化的方法得到进入转角和出转角的起始速度 \( v_1 \) 与终止速度 \( v_2 \)。其中 \( v_1 \) 作为 OA 段的末速度,\( v_2 \) 作为 CD 段的初速度。则整个加工的各个阶段的初末速度均已确定(开始加工与结束加工的速度为 0 或实际加工过程中的一个瞬时速度),则可通过新的 S 型加减速控制和多插补周期过渡的方法对折线进行加工。

问题二:当加工型线是由直线段和圆弧段(相切或不相切)组成的连续曲线时,考虑将加工过程可以分为直线加工段、直线和圆弧的过渡段和圆弧加工段,所建立的实时加工优化控制算法则需要针对三个阶段或其中之一进行优化。直线段与圆弧段组成的曲线加工示意图如下图所示。

\begin{figure}[h]
    \centering
    \includegraphics[width=0.8\textwidth]{直线段与圆弧段组成的曲线加工示意图.png}
    \caption{直线段与圆弧段组成的曲线加工示意图}
    \label{fig:直线段与圆弧段组成的曲线加工示意图}
\end{figure}

在满足最大加工速度的条件下对本问题进行分析:

直线加工段(图 \ref{fig:直线段与圆弧段组成的曲线加工示意图} 中 CA 段)的优化可以采用问题一中所述的新型 S 曲线加减速控制算法,人为的选取初末速度。

直线与圆弧的过渡段(图 \ref{fig:直线段与圆弧段组成的曲线加工示意图} 中 CA 与 AB 段)则可以采用 “圆弧转接法”,在拐点处通过插补小圆弧(弧线 CB)来提高拐点处的转接速度 \cite{参考文献3}。

圆弧加工段通过对法向量方向的最大加速度与半径的限制分析可得到该阶段最大的进给速度。通过对插补圆弧的转接速度与圆弧加工段最大进给速度的分析,选择其中较小的速度作为直线段加工的末速度,同时作为圆弧阶段的加工速度。圆弧加工段也可以视为插补一个半径较大的圆弧,这样直线与圆弧的过渡段圆弧转接插补和加工圆弧段的插补问题可以看做是一个问题,圆弧加工阶段结束的速度作为下一段直线段或其他加工段的初速度。

问题三:在问题二的基础上考虑初始瞬时速度与初始瞬时加速度对算法效率和速度的影响,并对加工过程的直线加工段、直线与圆弧的过渡段和圆弧加工段,进行分析得到该问题下的加速区、匀速区和减速区。并对问题二所利用的算法进行改进。

问题四:通过对 $S$ 型加减速控制曲线的速度、加速度、加加速度随时间变化规律进行分析可以发现,虽然 $S$ 曲线加减速控制算法加速度曲线是连续变化的,但加加速度 $j$ 仍是阶跃变化的,为提高机床运行平稳性,在研究数控机床动力学数学模型的基础上,考虑一种加加速度连续的加减速方法,通过对 $S$ 曲线的分析对其突变的加加速度 $j$ 进行改进,变为沿着一定斜率的线性增加或减小。再考虑数控系统的实时性要求,对加减速方法进行合理简化。

\section{三、问题一模型建立与求解}

\subsection{3.1 转角速度限制条件分析}

在加工过程中,转角处的转接速度常受到以下几个约束:加工速度限制、加速度限制、加工误差限制 \cite{ref5,ref6}。

(1) 加工速度限制:

加工过程中加工速度 $V$ 不大于机床最大速度 $V_{\max }$。

(2) 加速度限制:

如图 3-1 所示,X,Y 平面内,相邻两段加工轨迹为 $l_{1}, l_{2}$,轨迹交点为 $C$,$l_{1}$ 与 X 轴夹角为 $\alpha$,$l_{2}$ 与 X 轴夹角为 $\beta$,则由加速度限制条件决定的转接速度为 \cite{ref4}:

\[
V_{c} \leq \min \left\{\frac{a_{xm} T}{|\cos \beta - \cos \alpha|}, \frac{a_{ym} T}{|\sin \beta - \sin \alpha|}\right\}
\]

其中,$T$ 为插补周期,$a_{xm}$,$a_{ym}$ 为两轴最大允许加速度。

(3) 加工误差限制

如图 3-2 所示,AB 为实际插补轨迹,$h$ 为加工误差,$\theta$ 为两轨迹夹角即 $\beta - \alpha$,若最大加工误差为 $e$,则转接速度为:

\[
V_{c} \leq \frac{2e}{T \tan \frac{\theta}{2}}
\]

\begin{figure}[h]
    \centering
    \includegraphics[width=0.45\textwidth]{fig3-1.png}
    \caption{折线加工示意图}
    \label{fig:3-1}
\end{figure}
\begin{figure}[h]
    \centering
    \includegraphics[width=0.45\textwidth]{fig3-2.png}
    \caption{转接点加工误差}
    \label{fig:3-2}
\end{figure}

为研究折线夹角与转接速度关系,取

\[
T = 5 \, \text{ms}, \, a_{xm} = a_{ym} = 0.6 \, \text{m/s}^2, \, e = 1 \, \text{um}, \, \alpha = 0
\]

改变 $\theta$ 可得转接速度曲线图如下:

\begin{figure}[h]
    \centering
    \includegraphics[width=0.8\textwidth]{fig3-3.png}
    \caption{转接速度与夹角变化关系图}
    \label{fig:3-3}
\end{figure}

由上图可见,若采用一步转接,夹角 \(\theta\) 的变化会引起很大的速度变化。随着两轨迹段转接角的减小,转接速度急剧增大;夹角过大,速度将会很小。故为提高过转角处的速度,转角处采用多插补周期过渡方法实现优化控制。

\subsection{3.2 模型建立与求解}

本文将该问题分为转角与直线段两部分进行建模与求解。

\subsubsection{3.2.1 转角部分优化模型}

连续微小直线段加工中,转角处速度方向存在突变,若高速通过转角会因超出机床最大加速能力引起机床振动,而若低速通过会降低加工速度。在满足加工精度和机床各驱动轴最大加速能力的条件下,为提高加工速度,根据 S 型加减速方式在转角处进行多插补周期过渡,以提高转角的通过速度,且经过前瞻处理后使得整体速度提高,同时该方法满足最大加工误差、最大速度、最大加速度和最大加加速的要求。

\paragraph{(1) S 曲线转角多周期过渡算法}

转角多周期过渡方法 [7] 是将微小直线段插补分为直线上插补和转角处插补,其中转角处插补时通过在多个插补周期内逐步改变转角处的速度方向和大小以提高转角通过速度,插补周期的个数根据最大加工误差确定。为了实现加加速有界的转角多周期过渡,本文基于以下假定: 1: 最大加加速与最大减加速大小相等。 2: 转角处插补与直线上插补的连接处加速度为零。 3 转角处插补时的加速方式为:“加加速和加减速”和“减加速和减减速”两种,该假设基于由于误差限制转角插补周期数一般远远小于加速度从零加速到最大值所需要的周期数。本文转角处插补的方式为“加加速和加减速”方式。

以二轴机床(X, Y 轴为两线性驱动轴)加工为例,介绍加加速有界加减速方式下转角速度的确定方法。根据运动学关系可知转角开始速度和转角结束速度 \(v_1\)、\(v_2\),在 X 和 Y 轴方向的分量满足下面关系式:

\begin{align*}
v_{2x} &= v_2 \cdot \cos \alpha_2 > 0 \\
v_{2y} &= v_2 \cdot \sin \alpha_2 > 0
\end{align*}

其中 \(J_{1x} = -J_{3x}\) 表示机床 X 轴的加加速和减加速,\(J_{1y} = -J_{3y}\) 表示机床 Y 轴的加加速和减加速,\(t_1, t_3\) 表示加加速时间和减加速时间,转角总插补时间 \(t_m = t_1 + t_3\),根据前面假定 1 和 2 可知 \(t_1 = t_3 = t_m / 2\),\(\alpha_1, \alpha_2\) 如图 1 所示,分别是转角开始速度方向和转角结束速度方向与 X 轴的夹角。

\begin{figure}[h]
    \centering
    \includegraphics[width=0.8\textwidth]{image.png}
    \caption{转角多周期过渡示意图}
    \label{fig:3-4}
\end{figure}

\begin{align*}
V_{\text{max}_3} &\leq \frac{2R\cos\theta}{T} = \frac{2 \times 0.5 \times 10^{-2} \cos 45^\circ}{1/60 \times 10^{-3}} = 4.24 \, m/\text{min} \\
V_{\text{max}_4} &\leq \frac{\sqrt{8RE_R}}{T} = \frac{\sqrt{8 \times 0.5 \times 10^{-2} \times 1 \times 10^{-6}}}{1/60 \times 10^{-3}} = 12 \, m/\text{min}
\end{align*}

其中 $\theta=\alpha_{2}-\alpha_{1}$,如图 3-1 所示。若 $\theta=0$,把两个直线段合并为同一直线段处理。

由式(3-2)可知,在固定拐角插补时间时,在机床 X 轴和 Y 轴的最大加加速取值范围内 $(J_{xm},-J_{xm})$, $\left(J_{ym},-J_{ym}\right)$,通过恰当的选取 X、Y 轴的加加速值 $J_{1x}, J_{1y}$ 可以使 $v_{1}, v_{2}$ 值尽量大,即 $v_{1}, v_{2}$ 可以看作关于两个相互独立的变量 $J_{1x}, J_{1y}$ 的函数。根据转角加工的几何和物理性质 $v_{1}, v_{2}$ 的 X 轴和 Y 轴分量或者同时同号,或者只有一个轴的分量异号。据此分别采用以下优化函数优化转角速度 $v_{1}, v_{2}$。

1) 当 $\cos\alpha_{1}\cos\alpha_{2}\geq0$ 且 $\sin\alpha_{1}\sin\alpha_{2}\geq0$ 时,优化函数选为 $f=v_{1}+v_{2}$,此时,优化函数 $f$ 与 $v_{1}, v_{2}$ 的单调性相同,$f$ 达到最大时,转角开始与结束的速度达到最大:

\begin{align*}
a_1(t) &= a_0 + Jt \\
V_1(t) &= V_0 + a_0t + \frac{1}{2}Jt^2 \\
S_1(t) &= V_0t + \frac{1}{2}a_0t^2 + \frac{1}{6}Jt^3
\end{align*}

根据单调性可知,当转角插补时间一定时,满足使上式值最大的 $J_{1x}$ 与 $J_{1y}$ 的取值分别为:当 $\frac{\cos\alpha_{2}+\cos\alpha_{1}}{\sin\theta}>0$ 时取 $J_{1y}=J_{ym}$,否则 $J_{1y}=-J_{ym}$;当 $\frac{\sin\alpha_{2}+\sin\alpha_{1}}{\sin\theta}>0$ 时,取 $J_{1x}=-J_{xm}$,否则 $J_{1x}=J_{xm}$。

2) 当 $\cos \alpha_{1} \cos \alpha_{2} > 0, \sin \alpha_{1} \sin \alpha_{2} < 0$ 或 $\cos \alpha_{1} \cos \alpha_{2} < 0, \sin \alpha_{1} \sin \alpha_{2} > 0$ 时,优化函数选为 $f = \min \left(v_{1}, v_{2}\right)$;此时,$v_{1}, v_{2}$ 的单调性相反,因此转角前后速度相等时满足优化函数 $f = \min \left(v_{1}, v_{2}\right)$:

\[
v_{1} = v_{2} = \min \left(\left|\frac{J_{xm} t_{m}^{2}}{4\left(\cos \alpha_{2} - \cos \alpha_{1}\right)}\right|, \left|\frac{J_{ym} t_{m}^{2}}{4\left(\sin \alpha_{2} - \sin \alpha_{1}\right)}\right|\right)
\]

则此时对应的 X 轴和 Y 轴的加速度为:

\[
J_{1x} = \frac{4\left(v_{2} \cos \alpha_{2} - v_{1} \cos \alpha_{1}\right)}{t_{m}^{2}}
\]

\[
J_{1y} = \frac{4\left(v_{2} \sin \alpha_{2} - v_{1} \sin \alpha_{1}\right)}{t_{m}^{2}}
\]

当 $\cos \alpha_{1} \cdot \cos \alpha_{2} < 0, \sin \alpha_{1} \cdot \sin \alpha_{2} > 0$ 时分析与此类似。

为了使转角开始位置(图 3-1 中 A 点)和转角结束位置(图 3-1 中 C 点)分别位于转角所在的直线段上,需要确定满足式 (3) 和式 (4) 的转角开始距离 $l_{1}$ 和转角结束距离 $l_{2}$,即图 3-1 中线段 AB 和 BC。根据运动学关系,转角开始位置和转角结束位置的 X, Y 方向的位移增量,$L_{x}, L_{y}$ 分别为:

\[
\begin{cases}
L_{x} = v_{1} t_{m} \cos \alpha_{1} + \frac{J_{1x} t_{m}^{3}}{8} \\
L_{y} = v_{1} t_{m} \sin \alpha_{1} + \frac{J_{1y} t_{m}^{3}}{8}
\end{cases}
\]

同时由几何关系可知,转角开始距离 $l_{1}$ 和转角结束距离 $l_{2}$ 在 X, Y 方向的分量为:

\[
\begin{cases}
L_{x} = l_{1} \cos \alpha_{1} + l_{2} \cos \alpha_{2} \\
L_{y} = l_{1} \sin \alpha_{1} + l_{2} \sin \alpha_{2}
\end{cases}
\]

根据式 (3-6)、式 (3-5) 并结合式 (3-2) 可知

\[
\begin{cases}
l_{1} = \frac{L_{x} \sin \alpha_{2} - L_{y} \cos \alpha_{2}}{\sin \theta} = v_{1} t_{m} / 2 \\
l_{2} = \frac{L_{x} \sin \alpha_{1} - L_{y} \cos \alpha_{1}}{-\sin \theta} = v_{2} t_{m} / 2
\end{cases}
\]

因此以拐点为坐标原点(图 3-1 中 B (0, 0) 点),基于 S 型加减速方式的转角多周期过渡方法在转角处插补时的速度和位移矢量计算公式为:

\begin{align*}
a_2(t) &= a_0 - J(t - T_1) + JT_1 \\
V_2(t) &= V_0 + a_0t + \frac{1}{2}Jt^2 + a_0(t - T_1) - \frac{1}{2}J(t - T_1)^2 + 2JT_1(t - T_1) \\
S_2(t) &= V_0t + \frac{1}{2}a_0t^2 + \frac{1}{6}Jt^3 + \int_{T_1}^t V_2(t) dt
\end{align*}

\begin{align*}
a_3 &= 0 \\
V_3 &= V_2
\end{align*}

其中 $e_1, e_2$ 是转角开始速度和转角结束速度单位方向矢量,$J$ 是 $X$ 轴和 $Y$ 轴加速度的矢量和,转角处插补的插补点根据式 (3-9) 进行计算。

\paragraph{(2) 转角多周期过渡插补时间}

上节介绍了在固定转角插补时间时基于 $S$ 型加减速方式的转角多周期过渡方法。由式 (3-2) 和式 (3-7) 可知,转角开始速度和转角结束速度与转角插补时间的平方成正比,而转角开始距离和转角结束距离与转角插补时间的立方成正比,因此可以通过调整转角插补时间统一调整转角通过速度和转角插补范围。下面介绍如何确定每个转角的插补时间。首先需要证明采用转角多周期过渡方法时转角处的插补满足形状误差,即转角处的插补点落在在转角所形成的凸包内。由式 (3-9) 可知,转角处插补的位移曲线是关于时间的三次函数,且

\begin{align*}
a_4 &= -a_0 - J\left[t - (2T_1 + T_3)\right] \\
V_4 &= V_3 + \int_{2T_1 + T_3}^t a_4 dt
\end{align*}

由上式可知,当 $t = 0$, $t = t_m$ 时,即位移函数曲线的拐点在起点和终点处,且曲线内部不存在拐点。因此该曲线的切线即速度方向在区间 $[e_1, e_2]$ 上是单调的,因此拐点处插补点均落在转角所形成的凸包内。

转角处除满足形状误差外,还需满足最大加工误差。转角处由于多周期过渡导致的加工误差为:$\varepsilon = \min \left( d(\tau)^2 \right)$,即计算加工误差时需求解关于时间 $\tau$ 的五次函数。为降低计算复杂度,时即加工误差近似为 $\varepsilon = \left| d \left( \frac{t_m}{2} \right) \right|^2 = \frac{|J| t_m^3}{48}$,因此插补时间可由最大加工误差 $\varepsilon_b$ 计算可以确定转角插补时间为:

\begin{align*}
a_5 &= -a_1 - JT_2 + J\left[t - (2T_1 + T_3 + T_2)\right] \\
V_5 &= V_4 + \int_{2T_1 + T_3 + T_2}^t a_5 dt
\end{align*}

实际加工时要求速度不能超过最大加工速度 $v_m$,故需要判断转角开始速度和转角结束速度是否都不大于 $v_m$。如果否,根据转角开始速度和转角结束速度

与转角插补时间的平方成正比,则转角插补时间需要在式(3-10)基础上调整为:
\begin{align*}
a_{\text{max}} &= a_0 + Jt \\
V_1 &= V_0 + a_0 T_1 + \frac{1}{2} J T_1^2
\end{align*}

\paragraph{(3)基于 S 曲线加减速方式的前瞻处理方法}

转角多周期过渡方法把每个微小直线段的加工分为直线上插补和转角处插补,每个转角根据最大加工误差、最大加工速度以及速度优化函数确定转角开始速度和转角结束速度,因此直线上插补时,起始和终止速度均不为零(分别等于相应转角处的转角结束速度和转角开始速度)。为了使直线段起始速度能够在直线段插补长度范围内以 S 型加减速方式达到另一端速度(简称可达性),整体上需要进行前瞻处理,即若直线段两端速度不满足可达性,通过降低速度较高一端对应的转角插补时间来降低其速度,最终使得任何微小直线段的两端速度都满足可达性。

本文采用的是“加加速和减加速”的加速方式。若待插补直线段长度为 \( L \),且已假设加加速与减加速时间相同,即 \( t_{1}=t_{3}=t^{\prime} \),直线段起始与终止速度为 \( v_{s}, v_{e} \),最大加加速与最大减加速大小相等为 \( J_{m} \),最大加速度故与最大减速度大小相等为 \( a_{m} \),(假设 \( v_{s}<v_{e} \),反之亦然)可达性分析如下:

根据加加速与减加速的速度位移计算公式可得:
\begin{align*}
s &= q_0 t^4 + q_1 t^3 + q_2 t^2 + q_4 \\
v &= 4q_0 t^3 + 3q_1 t^2 + 2q_2 t + q_3 \\
a &= 12q_0 t^2 + 6q_1 t + 2q_2 \\
j &= 24q_0 t^2 + 6q_1
\end{align*}

则由 \( v_{s} \) 到 \( v_{e} \) 能达到的最大加速度为 \( a=\sqrt{\left(v_{e}-v_{s}\right) J_{m}} \),此时 \( t^{\prime}=a / J_{m} \),则最小距离 \( S \):
\begin{align*}
j &= lt \\
a &= \frac{1}{2} lt^2 \\
v &= \frac{1}{6} lt^3 \\
s &= \frac{1}{24} lt^4
\end{align*}

若 \( S<L \),满足可达性,否则需调整 \( v_{e} \) 使之满足:
\begin{align*}
j &= j_{\max} \\
a &= \frac{1}{2} \cdot l t_{1}^{2} + j_{\max} \left(t - t_{1}\right) \\
\nu &= \frac{1}{6} \cdot l t_{1}^{3} + \frac{1}{2} j_{\max} \left(t - t_{1}\right)^{2} \\
s &= \frac{1}{24} \cdot l t_{1}^{4} + \frac{1}{6} j_{\max} \left(t - t_{1}\right)^{3}
\end{align*}

求解上式得到 \( v_{e}^{\prime} \), \( v_{e}^{\prime} \) 对应转角的其它插补参数按相应比例 \( v_{e}^{\prime} / v_{e} \) 进行调整。

\subsubsection{3.2.2 直线部分优化模型}

直线段部分(图 3-5 中 OA 与 CD 段)采用一种新型 S 曲线加减速方法,如图 3-6 所示,分别由加加速、减加速、匀速、加减速、减减速 5 个过程组成 \cite{8}。

\begin{figure}[h]
    \centering
    \includegraphics[width=0.45\textwidth]{image1.png}
    \caption{折线加工示意图}
    \label{fig:3-5}
\end{figure}
\begin{figure}[h]
    \centering
    \includegraphics[width=0.45\textwidth]{image2.png}
    \caption{S 曲线加减速方法}
    \label{fig:3-6}
\end{figure}

设 $S$ 为直线段给定位移,且 $S = S_{up} + S_{w} + S_{down}$ (其中,$S_{up}$ 为升速过程位移,$S_{w}$ 为匀速过程位移,$S_{down}$ 为减速过程位移),$V$ 为速度值,$a$ 为加速度值,$J$ 为加加速值。$V_{s}$ 为初速度值,$V_{e}$ 为末速度值。且加加速与减减速时间相同为 $T_{1}$,减加速与加减速时间相同为 $T_{2}$,匀速时间为 $T_{3}$。

加速度公式如下:
\begin{align*}
j &= j_{\max} - l \left(t - t_{2}\right) \\
a &= j_{\max} \left(t - t_{2}\right) - \frac{1}{2} l \left(t - t_{2}\right)^{2} + j_{\max} \left(t_{2} - t_{1}\right) + \frac{1}{2} l t_{1}^{2} \\
\nu &= \frac{1}{2} j_{\max} \left(t - t_{2}\right)^{2} - \frac{1}{6} l \left(t - t_{2}\right)^{3} + \frac{1}{2} j_{\max} \left(t_{2} - t_{1}\right)^{2} + \frac{1}{6} l t_{1}^{3} \\
s &= \frac{1}{6} j_{\max} \left(t - t_{2}\right)^{3} - \frac{1}{24} l \left(t - t_{2}\right)^{4} + \frac{1}{6} j_{\max} \left(t_{2} - t_{1}\right)^{3} + \frac{1}{24} l t_{1}^{4}
\end{align*}

速度公式如下:
\begin{align*}
j &= 0 \\
a &= a_{\max} \\
\nu &= a_{\max} \left(t - t_{3}\right) + \frac{1}{2} j_{\max} \left(t_{3} - t_{2}\right)^{2} - \frac{1}{6} l \left(t_{3} - t_{2}\right)^{3} + \frac{1}{2} j_{\max} \left(t_{2} - t_{1}\right)^{2} + \frac{1}{6} l t_{1}^{3} \\
s &= \frac{1}{2} a_{\max} \left(t - t_{3}\right)^{2} + \frac{1}{6} j_{\max} \left(t_{3} - t_{2}\right)^{3} - \frac{1}{24} l \left(t_{3} - t_{2}\right)^{4} + \frac{1}{6} j_{\max} \left(t_{2} - t_{1}\right)^{3} + \frac{1}{24} l t_{1}^{4}
\end{align*}

位移公式如下:
\begin{equation}
S(t) =
\begin{cases}
V_{s}t + \frac{1}{6} Jt^{3} & t \in [0, T_{1}) \\
\left(V_{s} + JT_{1}^{2}\right)t - \frac{1}{6} J(t - 2T_{1})^{3} - JT_{1}^{3} & t \in [T_{1}, 2T_{1}) \\
\left(V_{s} + JT_{1}^{2}\right)t - JT_{1}^{3} & t \in [2T_{1}, 2T_{1} + T_{3}) \\
\left(V_{s} + JT_{1}^{2}\right)t - \frac{1}{6} J(t - (2T_{1} + T_{3}))^{3} - JT_{1}^{3} & t \in [2T_{1} + T_{3}, 2T_{1} + T_{3} + T_{2}) \\
\left(V_{s} + J(T_{1}^{2} - T_{2}^{2})\right)t + \frac{1}{6} J(t - (2T_{1} + T_{3} + T_{2}))^{3} & t \in [2T_{1} + T_{3} + T_{2}, 2T_{1} + T_{3} + 2T_{2}) \\
+ JT_{2}^{2}(2T_{1} + T_{3} + T_{2}) - JT_{1}^{3}
\end{cases}
\tag{3-14}
\end{equation}

由上式可知,已知初速度 $V_{s}$、末速度 $V_{e}$ 及给定位移 $S$ 后,设定加速度的变化率 $j$ 和速度最大值 $V_{\max}$,进而确定 $T_{1}, T_{2}, T_{3}$。确定这三个参数也就确定了加速过程。

度、速度、位移三条曲线, 那么问题转变成了求 $T_{1}, T_{2}, T_{3}$ 。

当 $t=2 T_{1}+2 T_{2}+T_{3}$ 时, 得:

\[
V_{e}=V_{s}+J\left(T_{1}^{2}+T_{2}^{2}\right)
\]

即只要求得 $T_{1}, T_{2}$ 中任意一个的值就可以求出另一个的值。以下分两种情况求解 $T_{1}, T_{2}$ 。

(1) 如果 $S_{ev} \geq 0$, 可得

\[
V_{\text {max }}=V_{s}+J T^{2}
\]

则 $T_{1}=\sqrt{\frac{V_{\max }-V_{s}}{J}}$ 代入得 $V_{\max }=V_{e}+J T_{2}^{2}$ 则 $T_{2}=\sqrt{\frac{V_{\max }-V_{e}}{J}}$

带入 (3-15) 得

\[
\begin{cases}
S_{u p}=2 V_{s} T_{1}+J T_{1}^{3} \\
S_{e v}=\left(V_{s}+J T_{1}^{2}\right) T_{3} \\
S_{d o w n}=2 V_{e} T_{2}+J T_{2}^{3}
\end{cases}
\]

(2) 如果 $S_{ev}<0$ 表明没有匀速区, 即 $T_{3}$ 为 0 , 且速度最大值达不到 $V_{\max }$ 。

由上式得:

\[
S(t) \mid_{t=2 T_{1}+2 T_{2}+T_{3}}=2 V_{e} T_{2}+J T_{2}^{3}+\left(V_{e}+V_{s}+J T_{2}^{2}\right) \sqrt{\frac{V_{e}-V_{s}}{J}+T_{2}^{2}}
\]

如果上式中 $T_{2}$ 看作变量, 则 $S(t) \mid_{t=2 T_{1}+2 T_{2}+T_{3}}$ 可以当作关于 $T_{2}$ 的函数, 记作 $\tilde{S}\left(T_{2}\right)$, 即

\[
\tilde{S}\left(T_{2}\right)=2 V_{e} T_{2}+J T_{2}^{3}+\left(V_{e}+V_{s}+J T_{2}^{2}\right) \sqrt{\frac{V_{e}-V_{s}}{J}+T_{2}^{2}}
\]

其中 $\tilde{S}\left(T_{2}\right)$ 相对于 $T_{2} \in[0, \infty)$ 区间是单调递增的。

根据上式得 $T_{2}$ 的取值上限是 $\sqrt{\frac{V_{\max }-V_{e}}{J}}$

1) 当 $V_{e} \geq V_{s}$ 时, $T_{2}$ 的最小值是 0 , 则其可取值范围是 $\left[0, \sqrt{\frac{V_{\max }-V_{e}}{J}}\right]$。因此,

若给定位移 \( S > \left( V_{e} + V_{s} \right) \sqrt{\frac{V_{e} - V_{s}}{J}} \) 则 \( T_{2} \) 存在正数解, 令 \( \tilde{S}(T_{2}) = S \), 采用二分法求解 \( T_{2} \) 否则重新选取初速度 \( V_{s} \) 末速度 \( V_{e} \)。

2) 当 \( V_{e} < V_{s} \) 时, 得 \( \frac{V_{e} - V_{s}}{J} + T_{2}^{2} \geq 0 \) 即 \( T_{2} \) 的最小值是 \( \sqrt{\frac{V_{s} - V_{e}}{J}} \) 则 \( T_{2} \) 可取值范围是 \( \left[ \sqrt{\frac{V_{s} - V_{e}}{J}}, \sqrt{\frac{V_{\max} - V_{e}}{J}} \right] \)。因此, 若给定位移 \( S > \left( V_{e} + V_{s} \right) \sqrt{\frac{V_{s} - V_{e}}{J}} \), 则 \( T_{2} \) 存在正数解, 令 \( \tilde{S}(T_{2}) = S \), 采用二分法求解 \( T_{2} \) 否则重新选取初速度 \( V_{s} \) 末速度 \( V_{e} \)。将求取的 \( T_{2} \), 代入得到 \( T_{1} \)。可求得 \( S_{up}, S_{down}, S_{ev} \) 和 \( T_{3} \)。得到 \( T_{1}, T_{2}, T_{3} \) 后, 各区域的速度也可以得到。

\subsubsection{3.3.3 折线夹角为 \( 90^{\circ} \) 和 \( 135^{\circ} \) 的讨论}

\begin{figure}[h]
    \centering
    \includegraphics[width=0.6\textwidth]{image.png} % 替换为实际图像文件名
    \caption{转角多周期过渡示意图}
    \label{fig:3-7}
\end{figure}

在前瞻性处理的条件下, 假设转角初始速度 \( v_{1} \) 与 \( X \) 轴夹角 \( \alpha_{1} \) (假设其为 \( 15^{\circ} \), 其他夹角类似讨论):

(1) 折线夹角为 \( 90^{\circ} \) (即有 \( \theta = 90^{\circ} \)):

此时 \( \alpha_{2} = \alpha_{1} + 90^{\circ} = 105^{\circ} \), 则有 \( \cos \alpha_{1} > 0 \), \( \cos \alpha_{2} < 0 \), \( \sin \alpha_{1} > 0 \), \( \sin \alpha_{2} > 0 \), 满足 \( \cos \alpha_{1} \cos \alpha_{2} < 0 \), \( \sin \alpha_{1} \sin \alpha_{2} > 0 \), 故此时优化函数选为 \( f = \min \left( v_{1}, v_{2} \right) \); 此时, \( v_{1}, v_{2} \) 的单调性相反, 因此转角前后速度相等时满足优化函数 \( f = \min \left( v_{1}, v_{2} \right) \):

\begin{equation}
v_{1} = v_{2} = \min \left( \left| \frac{J_{xm} t_{m}^{2}}{4 \left( \cos \alpha_{2} - \cos \alpha_{1} \right)} \right|, \left| \frac{J_{ym} t_{m}^{2}}{4 \left( \sin \alpha_{2} - \sin \alpha_{1} \right)} \right| \right)
\tag{3-17}
\end{equation}

则此时对应的 \( X \) 轴和 \( Y \) 轴的加速度为:

\begin{align*}
v_{2x} &= v_2 \cdot \cos \alpha_2 > 0 \\
v_{2y} &= v_2 \cdot \sin \alpha_2 > 0
\end{align*}

且根据前边的介绍可计算得到满足加工精度与最大加速能力的插补时间 \( t_m \) (可根据最大允许加工误差计算和最大加工速度调整)以及满足最大加工速度的加工速度。

而转角起始速度 \( v_1 \) 在 \( X, Y \) 轴上的分量分别为:
\[
\begin{aligned}
v_{1x} &= v_1 \cdot \cos \alpha_1 > 0 \\
v_{1y} &= v_1 \cdot \sin \alpha_1 > 0
\end{aligned}
\]
转角终止速度 \( v_2 \) 在 \( X, Y \) 轴上的分量分别为:
\[
\begin{aligned}
v_{2x} &= v_2 \cdot \cos \alpha_2 < 0 \\
v_{2y} &= v_2 \cdot \sin \alpha_2 > 0
\end{aligned}
\]

由此可见,在转角处多周期插补过程中,速度变化为速度在 \( X \) 轴上的分量先减小至零后逐渐反向增大,而在 \( Y \) 轴上的分量先增大后减小。整个过程中,保证了在转角处只需要速度 \( X \) 轴上的分量降为零而不必停止电机,在满足加工精度和机床各驱动轴最大加速能力的条件下以达到更快的加工速度。

(2) 折线夹角为 \( 135^\circ \)(即有 \( \theta = 45^\circ \)):

此时 \( \alpha_2 = \alpha_1 + 45^\circ = 60^\circ \),则有 \( \cos \alpha_1 > 0 \), \( \cos \alpha_2 > 0 \), \( \sin \alpha_1 > 0 \), \( \sin \alpha_2 > 0 \),满足 \( \cos \alpha_1 \cos \alpha_2 \geq 0 \), \( \sin \alpha_1 \sin \alpha_2 > 0 \),故此时优化函数选为 \( f = v_1 + v_2 \),优化函数 \( f \) 与 \( v_1, v_2 \) 的单调性相同, \( f \) 达到最大时,转角开始与结束的速度达到最大:
\[
f = v_1 + v_2 = \frac{J_{1y} \left( \cos \alpha_2 + \cos \alpha_1 \right) - J_{1x} \left( \sin \alpha_2 + \sin \alpha_1 \right)}{4 \sin \theta} t_m^2 \tag{3-18}
\]

根据单调性可知,当转角插补时间一定时(可根据最大允许加工误差计算和最大加工速度调整),满足使上式值最大的 \( J_{1x} \) 与 \( J_{1y} \) 取值如下:
\[
\frac{\cos \alpha_2 + \cos \alpha_1}{\sin \theta} > 0 \text{,故 } J_{1y} = J_{ym}
\]
\[
\frac{\sin \alpha_2 + \sin \alpha_1}{\sin \theta} > 0 \text{,故 } J_{1x} = -J_{xm}
\]

将 \( J_{1x} \) 与 \( J_{1y} \) 带入速度计算公式即可得到通过转角处的满足要求的最大加工速度。

而转角起始速度 \( v_1 \) 在 \( X, Y \) 轴上的分量分别为:
\[
\begin{aligned}
v_{1x} &= v_1 \cdot \cos \alpha_1 > 0 \\
v_{1y} &= v_1 \cdot \sin \alpha_1 > 0
\end{aligned}
\]
转角终止速度 \( v_2 \) 在 \( X, Y \) 轴上的分量分别为:

\begin{align*}
v_{2x} &= v_2 \cdot \cos \alpha_2 > 0 \\
v_{2y} &= v_2 \cdot \sin \alpha_2 > 0
\end{align*}

由此可见,在转角处多周期插补过程中,速度变化为速度在 \(X\) 轴上的分量逐渐减小但始终大于零,而在 \(Y\) 轴上的分量在逐渐增大。整个过程中,保证了在转角处速度始终大于零而不必停止电机,在满足加工精度和机床各驱动轴最大加速能力的条件下可达到更快的加工速度。

通过以上讨论可知,在本文所述方法下,当折线夹角较大即加工型线较平坦时可以以更大的加工速度通过转角,而当折线夹角较小时,亦可以相对较小的非零速度通过转角,从而在保证要求的情况下提高了加工速度。

\section{四、问题二模型建立与求解}

当加工型线是由直线段和圆弧段(相切或不相切)组成的连续曲线时,加工过程可以分为直线加工段、直线和圆弧的过渡段和圆弧加工段,所建立的实时加工优化控制算法则需要针对三个阶段或其中之一进行优化。

\subsection{4.1 直线段优化模型}

该问题直线段的加工与折线问题中的直线段加工属于一个问题,在此选择与问题一中直线段加工相同的算法——新型 S 型加减法控制算法,具体算法已在前文 3.2.2 节中描述,在此不再赘述。

\subsection{4.2 直线和圆弧的过渡段 “圆弧转接” 优化模型}

\subsubsection{4.2.1 圆弧插入方法}

在圆弧轨迹插补中,由于用小段直线逼近圆弧,所以加工轨迹存在半径误差 $E_{R}^{[9]}$,如图 4-1 所示。

\begin{figure}[h]
    \centering
    \includegraphics[width=0.6\textwidth]{image.png} % 替换为实际图片路径
    \caption{圆弧半径误差}
    \label{fig:4-1}
\end{figure}

用 $E_{rr}$ 表示轨迹段间转接处总允许误差,$E_{R}$ 表示插入圆弧的半径允许误差,如图 4-2(a)所示。图中,B、C 分别为插入圆弧与直线轨迹、圆弧轨迹的切点,$\theta$ 为夹角 $\angle CAB$ 的一半,O 为圆弧的圆心。要用圆弧 CB 代替轨迹段转接处 CAB 部分,需要根据 $E_{rr}$、$E_{R}$ 确定 B、C、O 点的坐标。设 $|AB| = |AC| = d$,$\angle BAO = \theta$,由图中几何关系得:

\[
\left( \frac{d}{\cos \theta} - d \tan \theta \right) + E_{R} \geq E_{rr}
\]

故

\[
d \leq \frac{E_{rr} - E_{R}}{\frac{1}{\cos \theta} - \tan \theta}
\]

圆弧半径 $R = d \tan \theta$

根据上式确定 $d, R, B, C, O$ 点的坐标即可确定。

\begin{figure}[h]
    \centering
    \includegraphics[width=\textwidth]{image.png}
    \caption{圆弧插入示意图}
    \label{fig:arc_insertion}
\end{figure}

\subsubsection{4.2.2 转接速度确定}

根据上述分析可知,采用圆弧转接方法时,轨迹转接速度也就是插入圆弧的插补速度,以 $\nu$ 表示。该速度引入下述三个限制条件。

\paragraph{多步转接条件}

为了保证轨迹段转接处速度矢量的变化分散到多个插补周期内,应有
\[
\nu_{1}T \leq 2R\cos\theta,
\]
即
\[
\nu_{1} \leq \frac{2R\cos\theta}{T}
\]
式中:$T$ 表示插补周期。

\paragraph{圆弧半径误差限制}

在圆弧轨迹插补中,由于用小段直线逼近圆弧,所以加工轨迹存在半径误差 $E_{R}$。如果相邻两段轨迹包含圆弧,则轨迹段间转接速度还将考虑圆弧半径误差 $E_{R}$ 的限制,假设进给速度为 $\nu$,插补步长 $|MN| = \nu T$,圆弧半径为 $R$,圆弧半径允许加工误差为 $E_{R}$,则由圆弧半径允许加工误差限制条件决定的转接速度公式为
\[
\nu_{2} \leq \frac{2R}{T} \sqrt{1 - \left(1 - \frac{E_{R}}{R}\right)^{2}}
\]
当插入圆弧半径较大时,经过化简,舍去二次项后表示如下
\[
\nu_{2} \leq \frac{\sqrt{8RE_{R}}}{T}
\]

\paragraph{加速度限制}

当被插补圆弧半径较大时,因为插补步长比圆弧半径长度小得多,相邻两插补步长的转角可以忽略,不需要考虑加速度限制。但在“圆弧转接法”中,插入圆弧的半径很小,有必要考虑加速度限制。圆弧插补如图 \ref{fig:arc_insertion}(b) 所示。

在圆弧段上,受到机床自身机械特性的约束,机床的最大允许进给法向加速度是有限的,因此在加工曲率半径较小的曲线段时,实际所需要的机床进给法向加速度有可能超过机床的最大允许进给法向加速度 $a_{N_{\text{max}}}$。所以在速度调整中,

应充分考虑机床的机械特性。进给法向加速度 \(a\) 与进给速度 \(v\) 及加工段圆弧半径 \(R\) 之间有如下关系:

\[
a = \frac{V^2}{R}
\]

则最大允许速度为

\[
V_{\text{max}} = \sqrt{a_{N_{\text{max}}} \cdot R}
\tag{4-6}
\]

由上式可得调整速度

\[
V_3 =
\begin{cases}
V_{\text{max}} & \text{若 } V > V_{\text{m}} \\
V & \text{若 } V \leq V_{\text{max}}
\end{cases}
\tag{4-7}
\]

采用“圆弧转接法”的轨迹间转接速度由下式确定:

\[
v = \min \{v_1, v_2, v_3\}
\]

取 \(E_{rr} = 2 \mu m\)、\(E_R = 0.9 \mu m\)、\(a = 600 mm/s^2\)、\(T = 1 ms\),图 4-3 所示的角度 \(\theta \in (0^\circ, 90^\circ)\) 即两相邻轨迹段夹角 \(2\theta \in (0^\circ, 180^\circ)\) 时,根据上式计算转接速度 \(v\),绘制 \(v-2\theta\) 曲线如图 4-3 所示。

\begin{figure}[h]
    \centering
    \includegraphics[width=0.8\textwidth]{image.png}
    \caption{$v-2\theta$ 曲线}
    \label{fig:4-3}
\end{figure}

通过仿真分析,采用“圆弧转接法”能显著提高轨迹段间转接速度,并且当 \(\theta\) 越大时,说明直线与圆弧两段轨迹间转接更平缓,转接速度也越高。

\subsection{4.3 圆弧加工段优化模型}

圆弧加工段插入圆弧的方式和过渡段采用的“圆弧转接法”中插入圆弧的方法是类似的,或者说加工一段圆弧可以等效为在两条轨迹间采用“圆弧转接法”进行圆弧转接过渡,它们都是用一系列首尾相接的直线段逼近圆弧。直接加工圆弧与“圆弧转接法”中插入圆弧在算法上更为简单,加工圆弧的半径、起始点坐标基本都是已知的,所以省去了“圆弧转接法”中求解插入圆弧半径等步骤。直接加工圆弧与“圆弧转接法”中插入圆弧在求解圆弧插补速度时,由于插入圆弧半径的影响,圆弧的插补速度是不同的。在下一小节我们将具体分析圆弧半径对于圆弧插补速度的影响。

\subsection{4.4 圆弧半径对算法效率的影响}

圆弧半径对于算法效率的影响主要表现在圆弧半径对插入圆弧的插补速度的影响两个方面。前文 2.2.2 节中已经分析了插入圆弧的插补速度受多步转接条件、圆弧半径误差、加速度三个限制条件影响。

(1) 多步转接条件限制使得:
\[
v_{1} \leq \frac{2R \cos \theta}{T}
\]

(2) 圆弧半径允许加工误差限制条件决定的转接速度公式为:
\[
v_{2} \leq \frac{2R}{T} \sqrt{1 - \left(1 - \frac{E_{R}}{R}\right)^{2}}
\]
当插入圆弧半径较大时,经过化简,舍去二次项后表示如下
\[
v_{2} \leq \frac{\sqrt{8RE_{R}}}{T}
\]

(3) 在“圆弧转接法”中,插入圆弧的半径很小,有必要考虑加速度限制。插补圆弧半径较大时,因为插补步长比圆弧半径长度小得多,相邻两插补步长的转角可以忽略,不需要考虑加速度限制。
\[
v_{3} \leq \sqrt{aR}
\]
插入圆弧的插补速度 \( v = \min \{v_{1}, v_{2}, v_{3}\} \)

分析以上各式,当 \(\theta\) 一定时,圆弧半径 \(r\) 越大时,\(v_{1}, v_{2}, v_{3}\) 越大,此时 \(v\) 越大,即圆弧半径越大,插入圆弧的插补速度越大,算法效率越高。

\subsection{4.5 优化模型检验}

\subsubsection{4.5.1 圆弧段最大允许速度分析}

圆弧半径为 \(0.5 \times 10^{-2} m\),被插补圆弧与直线段相切故可视为一个半径较大的插补圆弧,故 \(\theta = 45^\circ\),插补周期取 \(T = 1 \, \text{ms} = 1/60 \times 10^{-3} \, \text{min}\)

由表可知,加工过程中最大加工速度为 \(V_{\text{max1}} = 1.26 \, m/\text{min}\);

根据圆弧段最大允许速度与法向量加速度及半径关系可得:
\[
V_{\text{max2}} = \sqrt{a_{N_{\text{max}}} \cdot r} = \sqrt{0.6 \times 60^2 \times 0.5 \times 10^{-2}} = 3.286 \, m/\text{min}
\]

被插补圆弧与直线段相切故可视为一个半径较大的插补圆弧,则由转接速度限制条件可得:
\begin{align*}
V_{\text{max}_3} &\leq \frac{2R\cos\theta}{T} = \frac{2 \times 0.5 \times 10^{-2} \cos 45^\circ}{1/60 \times 10^{-3}} = 4.24 \, m/\text{min} \\
V_{\text{max}_4} &\leq \frac{\sqrt{8RE_R}}{T} = \frac{\sqrt{8 \times 0.5 \times 10^{-2} \times 1 \times 10^{-6}}}{1/60 \times 10^{-3}} = 12 \, m/\text{min}
\end{align*}
故可取得 C 点处(各圆弧处)的最大加工速度为:
\[
V_{\text{max}} = \min\{V_{\text{max}_1}, V_{\text{max}_2}, V_{\text{max}_3}, V_{\text{max}_4}\} = 1.26 \, m/\text{min}
\]

\subsubsection{4.5.2 圆角矩形切割路径的加工速度}

直线段 AB 的加工速度:A 点起始速度加速度均为 0,AB 段最大速度为 \(0.19 \, m/\text{min}\),即 B 点速度为 \(0.19 \, m/\text{min}\);

直线段 BC 的加工速度:B 点起始速度为 \(0.19 \, m/\text{min}\),根据 1) 的分析可知,C 点的加工速度可取 \(1.26 \, m/\text{min}\);

C 点至 J 点的加工速度:圆弧处的加工速度可取 \(1.26 \, m/\text{min}\),而为达到最大加工速度,故从 C 点开始沿路经加工至 J 点均以 \(1.26 \, m/\text{min}\) 的加工速度;

J 到 K 点的加工速度:J 点速度为 \(1.26 \, m/\text{min}\),到达 K 点速度需为 \(0.13 \, m/\text{min}\);

各个加工段的具体加工过程在下一节详细计算。加工示意图如下:

\begin{figure}[h]
\centering
\includegraphics[width=0.8\textwidth]{image.png}
\caption{圆角矩形切割路径的加工速度示意图}
\end{figure}

\subsubsection{4.5.3 圆角矩形切割路径的控制计算}

\[
J = 300 \, mm/s^3 = 0.3 \, m/s^3
\]

AB 段:\(V_A = 0, V_B = 0.19 \, m/\text{min}\),\(V_{\text{max}} = 0.19 \, m/\text{min}\)

根据直线段新 S 曲线加减速算法,假设 AB 段存在匀速阶段,即有 \(S_{ev} \geq 0\),

可得
\[
V_{\max }=V_{A}+JT_{1}^{2}
\]
则
\[
T_{1}=\sqrt{\frac{V_{\max }-V_{A}}{J}}=\sqrt{\frac{0.32}{30}}=0.10s
\]
最大加速度: \( JT_{1}=0.03m/s^{2} \),在最大加速度的允许范围内。

代入得 \( V_{\max }=V_{e}+JT_{2}^{2} \) 则
\[
T_{2}=\sqrt{\frac{V_{\max }-V_{e}}{J}}=0s
\]
匀速阶段位移为:
\[
S_{up}=2V_{A}T_{1}+JT_{1}^{3}=0.03cm
\]
\[
2.292-0.03=2.269cm
\]
BC 段: \( V_{B}=0.19m/\min \), \( V_{\max }=1.26m/\min \)

同理

可得加速时间:
\[
T_{3}=\sqrt{\frac{V_{\max }-V_{B}}{J}}=\sqrt{\frac{1.78}{30}}=0.24s
\]
最大加速度: \( JT_{3}=0.72m/s^{2} \) 在最大加速度的允许范围内。

加速位移:
\[
S_{up}=2V_{B}T_{3}+JT_{3}^{3}=0.416cm
\]
CJ 段:匀速运动。

JK 段: \( V_{k}=0.13m/\min \), \( V_{\max }=1.26m/\min \)

减速时间:
\[
T_{4}=\sqrt{\frac{V_{\max }-V_{B}}{J}}=\sqrt{\frac{1.88}{30}}=0.25s
\]
最大加速度: \( JT_{4}=0.075m/s^{2} \) 在最大加速度的允许范围内。

减速位移:
\[
S_{down}=2V_{k}T_{4}+JT_{4}^{3}=0.470cm
\]
匀速位移: \( 2.292-0.470=1.821cm \)

那么整个运动的匀速位移:
\[
S-S_{up}-S_{down}=160+\pi-0.03-0.47=162.67cm
\]
匀速时间:
\[
T=\frac{162.67}{2.1}=77.46s
\]

\subsubsection{4.5.4 仿真结果分析}

利用所提算法对圆角矩形切割路径在 MATLAB 上进行仿真,整个加工过程的加工速度与加工加速度的仿真结果如图 4-5 与图 4-6 所示。从图上可以看出,总的加工时间为 85.4589 秒,在不考虑初始瞬时速度以及初始加速度的情况下,满足加工要求的基础上,大部分加工过程以要求的最大加工速度进行加工以缩短

\section{加工时间。}

因本问不考虑初始瞬时速度以及初始加速度,故加速度的范围为 $[0, 0.6]$。

\begin{figure}[h]
    \centering
    \includegraphics[width=0.45\textwidth]{image1.png}
    \caption{加工速度曲线}
    \label{fig:4-5}
\end{figure}
\begin{figure}[h]
    \centering
    \includegraphics[width=0.45\textwidth]{image2.png}
    \caption{加工加速度曲线}
    \label{fig:4-6}
\end{figure}

图 4-7 与图 4-8 分别给出了从 A 点开始到达到最大加工速度 $1.26 \, \text{m/min}$ 的加工速度曲线与加工加速度曲线。该加工过程只用到新 S 曲线加减速方法的加加速与减加速两个阶段。

\begin{figure}[h]
    \centering
    \includegraphics[width=0.45\textwidth]{image3.png}
    \caption{AC 段加工速度曲线}
    \label{fig:4-7}
\end{figure}
\begin{figure}[h]
    \centering
    \includegraphics[width=0.45\textwidth]{image4.png}
    \caption{AC 段加工加速度曲线}
    \label{fig:4-8}
\end{figure}

图 4-9 与图 4-10 分别给出了从 J 点以后某点开始减速到 K 点 $0.13 \, \text{m/min}$ 的加工速度曲线与加工加速度曲线。该加工过程用到新 S 曲线加减速方法的加减速与减减速两个阶段。该过程为达到最大加工速度,采用了前瞻性算法,计算得到减速区,之前的区域以最大速度加工,缩短了加工时间。

\begin{figure}[h]
    \centering
    \includegraphics[width=0.45\textwidth]{image5.png}
    \caption{JK 段加工速度曲线}
    \label{fig:4-9}
\end{figure}
\begin{figure}[h]
    \centering
    \includegraphics[width=0.45\textwidth]{image6.png}
    \caption{JK 段加工加速度曲线}
    \label{fig:4-10}
\end{figure}

\section{五、问题三模型建立与求解}

\subsection{5.1 优化模型建立}

在问题二的基础上考虑瞬时启动加速度及瞬时启动速度,即初始加速度和初始速度不再为零。给定初始加速度为 $a_0$ 初始速度为 $V_0$,其速度、加速度计算公式如下:

当 $t \in [0, T_1)$ 时:
\begin{align*}
a_1(t) &= a_0 + Jt \\
V_1(t) &= V_0 + a_0t + \frac{1}{2}Jt^2 \\
S_1(t) &= V_0t + \frac{1}{2}a_0t^2 + \frac{1}{6}Jt^3
\end{align*}

当 $t \in [T_1, 2T_1)$ 时:
\begin{align*}
a_2(t) &= a_0 - J(t - T_1) + JT_1 \\
V_2(t) &= V_0 + a_0t + \frac{1}{2}Jt^2 + a_0(t - T_1) - \frac{1}{2}J(t - T_1)^2 + 2JT_1(t - T_1) \\
S_2(t) &= V_0t + \frac{1}{2}a_0t^2 + \frac{1}{6}Jt^3 + \int_{T_1}^t V_2(t) dt
\end{align*}

当 $t \in [2T_1, 2T_1 + T_3)$ 时:
\begin{align*}
a_3 &= 0 \\
V_3 &= V_2
\end{align*}

当 $t \in [2T_1 + T_3, 2T_1 + T_3 + T_2)$ 时:
\begin{align*}
a_4 &= -a_0 - J\left[t - (2T_1 + T_3)\right] \\
V_4 &= V_3 + \int_{2T_1 + T_3}^t a_4 dt
\end{align*}

当 $t \in [2T_1 + T_3 + T_2, 2T_1 + T_3 + 2T_2)$ 时:
\begin{align*}
a_5 &= -a_1 - JT_2 + J\left[t - (2T_1 + T_3 + T_2)\right] \\
V_5 &= V_4 + \int_{2T_1 + T_3 + T_2}^t a_5 dt
\end{align*}

为求得 $T_1$ 另

\begin{align*}
a_{\text{max}} &= a_0 + Jt \\
V_1 &= V_0 + a_0 T_1 + \frac{1}{2} J T_1^2
\end{align*}

那么

\[
T_1 = \frac{-2a_0 + \sqrt{4a_0^2 + 4J(V_{\text{max}} - V_0)}}{2J}
\]

利用上述公式,根据问题二介绍的方法,可求得 \(T_1, T_2, T_3\)。得到 \(S_{up}, S_{down}, S_{ev}\) 后,各区域的速度也可以得到。

\subsection{5.2 优化模型检验}

利用所提算法对圆角矩形切割路径在 MATLAB 上进行仿真,整个加工过程的加工速度与加工加速度的仿真结果如图 5-1 与图 5-2 所示。从图上可以看出,在考虑初始瞬时速度以及初始加速度的情况下,满足加工要求的基础上,大部分加工过程以要求的最大加工速度进行加工,总的加工时间为 85.3688 s。

因本问考虑初始瞬时速度以及初始加速度,故加速度的范围为 \([0.02, 0.6]\)。

\begin{figure}[h]
    \centering
    \includegraphics[width=0.45\textwidth]{image1.png}
    \caption{加工速度曲线}
    \label{fig:5-1}
\end{figure}
\begin{figure}[h]
    \centering
    \includegraphics[width=0.45\textwidth]{image2.png}
    \caption{加工加速度曲线}
    \label{fig:5-2}
\end{figure}

图 5-3 与图 5-4 分别给出了 AB 段的局部加工速度以及加工加速度曲线。因考虑初始瞬时速度以及初始加速度,故 A 点的初始速度为 \(0.13 \, \text{m/min}\),初始加速度为 \(0.02 \, \text{m/s}^2\),且加速度的范围为 \([0.02, 0.6]\),故由 5-4 可以看出,当达到最大加工速度时,加速度骤降至零,后以 AB 段允许的最大速度进行加工。

\begin{figure}[h]
    \centering
    \includegraphics[width=0.45\textwidth]{image1.png}
    \caption{图 5-3 AB 段局部加工速度曲线}
    \label{fig:5-3}
\end{figure}
\begin{figure}[h]
    \centering
    \includegraphics[width=0.45\textwidth]{image2.png}
    \caption{图 5-4 AB 段局部加工加速度曲线}
    \label{fig:5-4}
\end{figure}

同理,在 BC 段的局部加工速度曲线与加速度曲线如图 5-5 与 5-6 所示。

\begin{figure}[h]
    \centering
    \includegraphics[width=0.45\textwidth]{image3.png}
    \caption{图 5-5 BC 段局部加工速度曲线}
    \label{fig:5-5}
\end{figure}
\begin{figure}[h]
    \centering
    \includegraphics[width=0.45\textwidth]{image4.png}
    \caption{图 5-6 BC 段局部加工加速度曲线}
    \label{fig:5-6}
\end{figure}

而图 5-7 与 5-8 给出了 JK 段减速区域的加工曲线图。

\begin{figure}[h]
    \centering
    \includegraphics[width=0.45\textwidth]{image5.png}
    \caption{图 5-7 JK 段加工速度曲线}
    \label{fig:5-7}
\end{figure}
\begin{figure}[h]
    \centering
    \includegraphics[width=0.45\textwidth]{image6.png}
    \caption{图 5-8 JK 段加工加速度曲线}
    \label{fig:5-8}
\end{figure}

由以上各图可以看出,在考虑启动瞬时加工速度与瞬时加速度的情况下,加工速度能更快的达到最大值,整个加工过程的总加工时间也相应的有所缩短(由问题二中的 85.4589 s 缩短至 85.3688 s)。

\section{六、问题四模型建立与求解}

\subsection{6.1 S 曲线加减速控制方法优缺点分析}

S 曲线加减速方法,其特点是将加减速过程分为 7 个阶段:加加速段、匀加速段、减加速段、匀速段、加减速段、匀减速段、减减速段。在加加速、减加速、加减速、减减速这四个过程加速度变化率 $j$ 的绝对值恒定;匀加速和匀减速过程的加速度恒定,为数控系统的最大加速度;匀速工程加速度为 0。通过对 S 型加减速控制曲线的速度、加速度、加加速度随时间变化规律进行分析可以发现,这种加减速模式在任意位置的加速度都是连续变化的,可避免柔性冲击,速度曲线光滑。但是 S 曲线加减速控制算法单一路径加工过程中存在初速度和末速度相等的限制,这种限制在加工过程造成时间增加,电机变速频繁,影响效率。

接下来,我们对数控机床动力学数学模型进行分析:

数控机床工作时,伺服电机驱动力 $F$、切削力 $F_{c}$、摩擦力 $F_{f}$ 和进给加速度 $a$ 之间的关系为
\[
F - F_{c} - F_{f} \approx ma
\]
式中 $m$ 为运动部件质量。对式(1)两边求导可得
\[
\frac{d\left(F - F_{c} - F_{f}\right)}{dt} \approx m \frac{da}{dt} = mj
\]
式中 $j$ 为加加速度,反映机床响应速度与运行平稳性之间的关系\cite{10,11}。即使是较先进的 S 曲线加减速方法中,$j$ 仍是阶跃变化的。在阶跃变化的上升沿和下降沿,对机床进给系统有振动和冲击,影响运行平稳性,而影响加工质量。

通过以上分析,我们对 S 型曲线加减速控制方法的优缺点作如下结论:

优点:在任意位置的加速度都是连续变化的,可避免柔性冲击,从而控制机床运动速度按 S 型曲线形式平滑变化,以保证速度光顺。这种方法对减小机床冲击有一定作用和效果。

缺点:单一路径加工过程中存在初速度和末速度相等的限制,这种限制在加工过程造成时间增加,电机变速频繁,影响效率;

加速度曲线是连续变化的,而加加速度 $j$ 仍是阶跃变化的,它在加速度曲线加减速阶段的起点和终点处有突变。在阶跃变化的上升沿和下降沿,对机床进给系统有振动和冲击,影响运行平稳性,进而影响加工质量。

\subsection{6.2 一种加加速度连续的加减速方法}

前文研究了数控机床动力学数学模型,从理论上分析造成机床冲击与振动的根源;为保证机床运动的平滑性,减少数控机床冲击与振动的影响,本节提出一种加加速度连续的加减速方法,推导了运动轮廓不同阶段的运动表达式;对该加减速算法进行了合理简化,以满足数控系统实时性要求。

为便于计算,$j$ 采用直线加减速,加减速斜率参数 $l$ 由具体的数控机床通过实验确定。对 $j$ 依次积分可得加速度 $a$、速度 $v$ 和位移 $s$ 表达式。本文采用分段

四次样条曲线来实现机床的平滑运动控制,如图 6-1 所示。其中,
\begin{align*}
s &= q_0 t^4 + q_1 t^3 + q_2 t^2 + q_4 \\
v &= 4q_0 t^3 + 3q_1 t^2 + 2q_2 t + q_3 \\
a &= 12q_0 t^2 + 6q_1 t + 2q_2 \\
j &= 24q_0 t^2 + 6q_1
\end{align*}

\begin{figure}[h]
    \centering
    \includegraphics[width=\textwidth]{image.png}
    \caption{一种加加速度连续的加减速方法}
    \label{fig:6-1}
\end{figure}

一个完整的运动过程包括 7 部分。其中:AB 段是加加速段,BC 段是匀加速段,CD 段是减加速,DE 段是匀速段,EF 段是加减速段,FG 段是匀减速段,GH 段是减减速段。图 6-1 中加加速度 \( j \) 是一阶连续的。

在运动轮廓的不同阶段,常数 \( q_0, q_1, q_2, q_3, q_4 \) 的取值是不同的,即不同阶段 \( s, v, a, j \) 的表达式各不相同。以 AB 段(加加速段)和 BC 段(匀加速段)为例:

(1) 当 \( t \in [0, t_1) \) 时,\( j \) 加速
\begin{align*}
j &= lt \\
a &= \frac{1}{2} lt^2 \\
v &= \frac{1}{6} lt^3 \\
s &= \frac{1}{24} lt^4
\end{align*}

此时 $q_{0}=\frac{1}{24} \times l, q_{1}=0, q_{2}=0, q_{3}=0, q_{4}=0$

(2) 当 $t \in\left[t_{1}, t_{2}\right)$ 时, $j$ 匀速

\begin{align*}
j &= j_{\max} \\
a &= \frac{1}{2} \cdot l t_{1}^{2} + j_{\max} \left(t - t_{1}\right) \\
\nu &= \frac{1}{6} \cdot l t_{1}^{3} + \frac{1}{2} j_{\max} \left(t - t_{1}\right)^{2} \\
s &= \frac{1}{24} \cdot l t_{1}^{4} + \frac{1}{6} j_{\max} \left(t - t_{1}\right)^{3}
\end{align*}

(3) 当 $t \in\left[t_{2}, t_{3}\right)$ 时, $j$ 减速

\begin{align*}
j &= j_{\max} - l \left(t - t_{2}\right) \\
a &= j_{\max} \left(t - t_{2}\right) - \frac{1}{2} l \left(t - t_{2}\right)^{2} + j_{\max} \left(t_{2} - t_{1}\right) + \frac{1}{2} l t_{1}^{2} \\
\nu &= \frac{1}{2} j_{\max} \left(t - t_{2}\right)^{2} - \frac{1}{6} l \left(t - t_{2}\right)^{3} + \frac{1}{2} j_{\max} \left(t_{2} - t_{1}\right)^{2} + \frac{1}{6} l t_{1}^{3} \\
s &= \frac{1}{6} j_{\max} \left(t - t_{2}\right)^{3} - \frac{1}{24} l \left(t - t_{2}\right)^{4} + \frac{1}{6} j_{\max} \left(t_{2} - t_{1}\right)^{3} + \frac{1}{24} l t_{1}^{4}
\end{align*}

(4) 当 $t \in\left[t_{3}, t_{4}\right]$ 时, $a$ 匀速

\begin{align*}
j &= 0 \\
a &= a_{\max} \\
\nu &= a_{\max} \left(t - t_{3}\right) + \frac{1}{2} j_{\max} \left(t_{3} - t_{2}\right)^{2} - \frac{1}{6} l \left(t_{3} - t_{2}\right)^{3} + \frac{1}{2} j_{\max} \left(t_{2} - t_{1}\right)^{2} + \frac{1}{6} l t_{1}^{3} \\
s &= \frac{1}{2} a_{\max} \left(t - t_{3}\right)^{2} + \frac{1}{6} j_{\max} \left(t_{3} - t_{2}\right)^{3} - \frac{1}{24} l \left(t_{3} - t_{2}\right)^{4} + \frac{1}{6} j_{\max} \left(t_{2} - t_{1}\right)^{3} + \frac{1}{24} l t_{1}^{4}
\end{align*}

其他 5 个运动段与之类似。其中, $j_{\max}$ 为最大允许加加速度, $a_{\max}$ 为最大允许加速度, 反映系统的最大加减速能力; $l$ 为 $j$ 的加减速斜率。对于一定结构的机床, $a_{\max}$, $l$ 和 $j_{\max}$ 一般取固定值, 可通过机床动力学分析和试验确定。

这种加减速方法中 $j$ 是线性变化的, 避免了 $j$ 的阶跃脉冲函数造成机床的冲击与振动, 伺服电机可以实现平滑运动, 因此可以提高零件加工的轮廓精度和表面质量。

\subsection{6.3 对加减速方法的合理简化}

图 6-1 中, 加加速度曲线一阶连续, 使运动更加平滑、平稳, 但同时使加减

速处理运算量增大。考虑到数控系统的实时性要求,本文对该加减速方法中加加速度处理作了合理简化。

如图 6-1 所示,以 AB 段(加加速段)为例。由最大允许加加速度 \( j_{max} \) 和 \( j \) 的加减速斜率 \( l \),可得 \( j \) 从 0 加速到 \( j_{max} \) 所用时间

\[
t_{j} = \frac{j_{\text{max}}}{l}
\]

设插补周期为 \( T \),则 \( j \) 从 0 加速到 \( j_{max} \) 所用周期为 \( n_{1} = \frac{t_{i}}{T} \)。实际中为了简化运算,取 \( n_{j} = \text{round}(n_{1}) \),即对 \( n_{1} \) 四舍五入取整。则 \( j \) 从 0 加速到 \( j_{max} \) 的过程中,每个插补周期实际的加加速度

\[
j(k) = \frac{k}{n_{j}} \cdot j_{\text{max}}
\]

式中 \( k = \left[ 0, 1, \ldots, n_{j} \right] \)。同理在 \( t \in \left[ t_{2}, t_{3} \right) \),即 \( j \) 从 \( j_{max} \) 减速到 0 过程中,

\[
j(k) = \frac{n_{i} - k}{n_{j}} \cdot j_{\text{max}}
\]

在 AB 段,\( a \) 从 0 加速到 \( a_{max} \) 所用时间 \( t_{a} \) 为

\[
t_{a} = \frac{a_{\text{max}}}{j_{\text{max}}} + \frac{j_{\text{max}}}{l}
\]

由上式得到 \( j \) 保持为 \( j_{\text{max}} \) 的时间,即 \( t_{2} - t_{1} = t_{a} - 2t_{j} \)。

这样就简化了加速度 \( a \) 和加加速度 \( j \) 的计算,进而简化了 AB 段(加加速段)加减速处理过程的运算量,其他运动段处理过程类似。经过简化处理,该加减速方法完全符合插补周期 4 ms 的需要。

\subsection{6.4 小结}

为提高机床运行平稳性,减少数控机床冲击与振动的影响,在研究数控机床动力学数学模型的基础上,针对高精度数控加工对运动平滑处理的要求,提出了一种加加速度连续的加减速方法,推导了运动轮廓不同阶段的运动表达式,并对该算法进行了一定简化以满足数控系统 4 ms 插补周期的要求。理论结果表明:相对 S 曲线加减速方法,该方法使加速度和速度曲线更加平滑,轮廓运动更加平滑,更具柔性,但运行时间略长。

\section{参考文献}

[1] 张立先,李洪波,高小山,孙瑞勇,高速 CNC 加工 S 型加减速方式连续小直线段插补算法, Chinese Academy of Sciences, Vol. 29, 227-237, September 2010.

[2] 胡磊,林示麟,徐建明等.S 曲线加减速速度控制新方法[J]. 组合机床与自动化加工技术, 2010:22-26.

[3] 张得礼,周来水. 数控加工运动的平滑处理[J]. 航空学报, 2006, 27:125-130.

[4] 赵国勇,赵玉刚,赵庆志. 数控运动中加加速度连续的加减速方法[J]. 计算机集成制造系统, 2011, 17(02):316-320.

[5] 叶佩青,赵慎良,微小直线段的连续插补控制算法研究,中国机械工程, 2014, 15(15):1354-1356.

[6] Han G C, Kim D I, Kim H G, et al. A high speed machining algorithm for CNC machine tools[C]// Industrial Electronics Society, 1999. IECON '99 Proceedings. The 25th Annual Conference of the IEEEIEEE, 1999:1493-1497 vol.3.

[7] 任锟. 高速数控加工的前瞻控制理论及关键技术研究[D]. 浙江大学机械与能源学院, 2008.

[8] 李晓辉,邬义杰,冷洪滨. S 曲线加减速控制新方法的研究[J]. 组合机床与自动化加工技术, 2007:50-53.

[9] 吕强,张辉,杨开明等. 数控连续加工中提高轨迹段转接速度的方法研究[J]. 制造技术与机床, 2008:79-83.

[10] 赵国勇,徐志祥,赵福令. 高速高精度数控加工中 NURBS 曲线插补的研究[J]. 中国机械工程, 2006, 17:291-294.

[11] Nam S H, Yang M Y. A study on a generalized parametric interpolator with real-time jerk-limited acceleration[J]. Computer-Aided Design, 2004, 36(03):27-36.

\section{附 录}

\section{1 问题一相关程序}
\begin{verbatim}
clear all
clc
T=0.005/60;
alpha_x=600*3600;
alpha_y=600*3600;
e=0.001;
alpha=0;
alpha=alpha/180*pi;
for sita=1:180
    beta1=sita+alpha;
    beta=beta1/180*pi;
    VC1(sita)=alpha_x*T/(abs(2*sin((alpha+beta)/2))*abs(sin((alpha-beta)/2)));
    VC2(sita)=alpha_y*T/(abs(2*cos((alpha+beta)/2))*abs(sin((alpha-beta)/2)));
    VC(sita)=min(VC1(sita),VC2(sita))*0.001;
    VC_e(sita)=2*e/(T*tan(sita/180*pi/2))*0.001;
    VC_min(sita)=0.13;
    VC_max(sita)=6;
end
plot(1:180,VC);
xlabel('\theta/(^\circ)')
ylabel('VC/(m/min)')
gtext('--------加速度限制')
hold on
plot(1:180,VC_e,'G');
gtext('----加工误差限制');
% plot(1:180,VC_min,'r');
% gtext('最小加工速度')
plot(1:180,VC_max,'r');
gtext('最大加工速度')
\end{verbatim}

\section{2 问题二相关程序}

\textbf{最大加工速度程序:}
\begin{verbatim}
clear all
clc
T=0.001/60;
ER=0.9*10^(-6);
Err=2*10^(-6);
a=600*3600/1000;
T=0.001/60;
for sita=0.5:0.5:89
\end{verbatim}

\begin{verbatim}
d=abs((Err-ER)/(1/(cos(sita/180*pi)-tan(sita/180*pi)));
R(2*sita)=d*tan(sita/180*pi);
% R=0.5/100;
VC1(2*sita)=2*R(2*sita)*cos(sita/180*pi)/T;
VC2(2*sita)=2*R(2*sita)/T*sqrt(1-(1-ER/R(2*sita))^2);
VC3(2*sita)=sqrt(a*R(2*sita));
VC4(2*sita)=min(VC1(2*sita),VC2(2*sita));
VC(2*sita)=min(VC4(2*sita),VC3(2*sita));
VC_max(2*sita)=6;
end
plot(1:178,VC(1:178),'r');
xlabel('2\theta/(^\circ)')
ylabel('VC/(m/min)')
hold on
% plot(1:180,VC_max,'r');
% gtext('最大加工速度')

模型检验程序:
clear all
clc
%参数 单位,cm,s
J=300*10^(-1);
e=1*10^(-4);
a0=0;VA=0;
a_min=0;a_max=0.6*100;
Vmax0=6*100/60;
Vmin0=0.13*100/60;
R=0.5;
L=20;%直线段
Lab=20-17.708;
sita=pi/4;
T=0.001;
%圆弧处最大速度选取
an=a_max;
Vmax(1)=sqrt(an*R);
Vmax(2)=1.26*100/60;
Vmax(3)=2*R*cos(sita)/T;
Vmax(4)=sqrt(8*R*e)/T;
VC=Vmax0;
for i=1:4
    VC=min(VC,Vmax(i));
end
%%AB 段
Sw1=0;VB=0.19*100/60;
\end{verbatim}

\begin{verbatim}
VAB_max = VB; Lab = 20 - 17.708;

V(1) = VA; a(1) = a0;
if Sw1 >= 0
    Ta1 = sqrt((VAB_max - VA) / J);
    Ta2 = sqrt((VAB_max - VB) / J);
    Sup = 2 * VA * Ta1 + J * Ta1^3;
    if Sup >= 0
        Sw1 = Lab - Sup;
        Ta3 = Sw1 / VAB_max;
    end
    Tab = 2 * Ta1 + Ta2 + Ta3;

for i = 1:round(Ta1 / T)
    a(i + 1) = a0 + J * (i * T);
    % if(a(i))
        V(i + 1) = VA + 0.5 * J * (i * T)^2;
    end
end
for i = round(Ta1 / T):round(2 * Ta1 / T)
    a(i + 1) = a0 - J * (i * T - Ta1) + J * Ta1;
    V(i + 1) = VA + J * Ta1^2 - 0.5 * J * (i * T - 2 * Ta1)^2;
end
for i = round(2 * Ta1 / T):round((2 * Ta1 + Ta3) / T)
    V(i + 1) = VA + J * Ta1^2;
    a(i + 1) = 0;
end
end

% %BC 段
Sw2 = 0; VB = 0.19 * 100 / 60; VC = 1.26 * 100 / 60;
VBC_max = VC; Lbc = 20 + 17.708;
if Sw2 >= 0
    Tb1 = sqrt((VBC_max - VB) / J);
    Tb2 = sqrt((VBC_max - VC) / J);
    Sup = 2 * VB * Tb1 + J * Tb1^3;
    if Sup >= 0
        Sw2 = Lbc - Sup;
        Tb3 = Sw2 / VBC_max;
    end
    Tbc = 2 * Tb1 + Tb2 + Tb3;
    for i = 1:round(Tb1 / T)
        a(i + 1 + round((2 * Ta1 + Ta3) / T)) = J * (i * T);
        V(i + 1 + round((2 * Ta1 + Ta3) / T)) = VB + 0.5 * J * (i * T)^2;
    end
end
\end{verbatim}

\begin{verbatim}
end
for i=round(Tb1/T):round(2*Tb1/T)
    a(i+1+round((2*Ta1+Ta3)/T))=-J*(i*T-Tb1)+J*Tb1;
    V(i+1+round((2*Ta1+Ta3)/T))=VB+J*Tb1^2-0.5*J*(i*T-2*Tb1)^2;
end
for i=round(2*Tb1/T):round((2*Tb1+Tb3)/T)
    V(i+1+round((2*Ta1+Ta3)/T))=VB+J*Tb1^2;
    a(i+1+round((2*Ta1+Ta3)/T))=0;
end
figure(1)
Vac=V*60/100;
plot(Vac(1:12000));
set(gca,'XTick',0:10^3:12000);
set(gca,'XTickLabel',0:1:12);
xlabel('加工时间/{t/s}')
ylabel('加工速度/(m/min)')
gtext('A')
gtext('B')
figure(2)
ac=a/100;
plot(ac(1:12000));
set(gca,'XTick',0:10^3:12000);
set(gca,'XTickLabel',0:1:12);
xlabel('加工时间/{t/s}')
ylabel('加工加速度/(m/s^2)')
gtext('A')
gtext('B')
end
% %JK 段
Sw4=0;VK=0.13*100/60;VJ=1.26*100/60;
Tk=sqrt((VK-VJ)/-J);
Sdown=2*VK*Tk+J*Tk^3;
Sw_JK=Lab-Sdown;
%CK 段总的匀速时间
Sw=pi+40*3+Sw_JK;
Tw=Sw/VJ;
T_all=Tab+Tbc+Tw+2*Tk;
for i=1:round(Tw/T)
    V(i+1+round((2*Tb1+Tb3)/T)+round((2*Ta1+Ta3)/T))=VJ;
    a(i+1+round((2*Tb1+Tb3)/T)+round((2*Ta1+Ta3)/T))=0;
end
for i=1:round(Tk/T)

V(i+1+round((2*Tb1+Tb3)/T)+round((2*Ta1+Ta3)/T)+round(Tw/T))=VJ-1/2*J*(i*T
\end{verbatim}

\begin{verbatim}
)^2;

a(i+1+round((2*Tb1+Tb3)/T)+round((2*Ta1+Ta3)/T)+round(Tw/T))=-J*i*T;
end
for i=round(Tk/T):round(2*Tk/T)

V(i+1+round((2*Tb1+Tb3)/T)+round((2*Ta1+Ta3)/T)+round(Tw/T))=VJ-1/2*J*Tk^
2-1/2*J*(i*T-Tk)^2;

a(i+1+round((2*Tb1+Tb3)/T)+round((2*Ta1+Ta3)/T)+round(Tw/T))=-J*Tk+J*(i*T-
Tk);
end
num=round(2*Tk/T)+round((2*Tb1+Tb3)/T)+round((2*Ta1+Ta3)/T)+round(Tw/T);
figure(3)
Vjk=V(num-1459:num)*60/100;
plot(Vjk);
set(gca,'XTick',0:500:1500);
set(gca,'XTickLabel',84:0.5:86);
xlabel('加工时间/\{t/s\}')
ylabel('加工速度/(m/min)')
gtext('K')
figure(4)
ajk=a(num-round(2*Tk/T)-1459:num)/100;
plot(ajk);
set(gca,'XTick',0:500:1500);
set(gca,'XTickLabel',84:0.5:86);
xlabel('加工时间/\{t/s\}')
ylabel('加工加速度/(m/s^2)')
gtext('K')
% %单位转换
V1=V*60/100;
a1=a/100;
figure(5)
plot(V1);
set(gca,'XTick',0:10^4:10^5);
set(gca,'XTickLabel',0:10:100);
xlabel('加工时间/\{t/s\}')
ylabel('加工速度/(m/min)')
gtext('A')
gtext('B')
gtext('K')
figure(6)
plot(a1);
set(gca,'XTick',0:10^4:10^5);
\end{verbatim}

\begin{verbatim}
set(gca,'XTickLabel',0:10:100);
xlabel('加工时间/{t/s}')
ylabel('加工加速度/(m/s^2)')
gtext('A')
gtext('B')
gtext('K')
\end{verbatim}

\section{3 问题三相关程序}
\begin{verbatim}
clear all
clc
%参数 单位,cm,s
J=300*10^(-1);
e=1*10^(-4);
a0=0.02*100;VA=0.13*100/60;
a_min=0.02;a_max=0.6*100;
Vmax0=6*100/60;
Vmin0=0.13*100/60;
R=0.5;
L=20;%直线段
Lab=20-17.708;
sita=pi/4;
T=0.001;
%圆弧处最大速度选取
an=a_max;
Vmax(1)=sqrt(an*R);
Vmax(2)=1.26*100/60;
Vmax(3)=2*R*cos(sita)/T;
Vmax(4)=sqrt(8*R*e)/T;
VC=Vmax0;
for i=1:4
    VC=min(VC,Vmax(i));
end
% %AB 段
Sw1=0;VB=0.19*100/60;
VAB_max=VB;Lab=20-17.708;
if Sw1>=0
    Ta1=(-2*a0+sqrt(4*a0^2+4*J*(VAB_max-VA)))/(2*J);
    Ta2=sqrt((VAB_max-VB)/J);

    Sup=VA*(2*Ta1-Ta1)+J*Ta1*((2*Ta1)^2-Ta1^2)-1/6*J*((2*Ta1)^3-Ta1^3)-J*Ta1^3
    +VA*Ta1+1/2*a0*Ta1^2+1/6*J*Ta1^3;
    Sw1=Lab-Sup;
    Ta3=Sw1/VAB_max;
    Tab=2*Ta1+Ta2+Ta3;
\end{verbatim}

\begin{verbatim}
V(1)=VA;a(1)=a0;
for i=1:round(Ta1/T)
    a(i+1)=a0+J*(i*T);
    V(i+1)=VA+a0*(i*T)+0.5*J*(i*T)^2;
end
for i=round(Ta1/T):round(2*Ta1/T)
    a(i+1)=a0-J*(i*T-Ta1)+J*Ta1;
    V(i+1)=VA+a0*Ta1+1/2*J*Ta1^2+a0*(i*T)+2*J*Ta1*(i*T)-1/2*J*((i*T)^2)-(a0*Ta1+2*J*Ta1*Ta1-1/2*J*((Ta1)^2));
end
for i=round(2*Ta1/T):round((2*Ta1+Ta3)/T)
    V(i+1)=VAB_max;
    a(i+1)=0;
end
figure(1)
Vab=V*60/100;
plot(Vab(1:100));
set(gca,'XTick',0:10:100);
set(gca,'XTickLabel',0:0.01:0.1);
xlabel('加工时间/{t/s}')
ylabel('加工速度/(m/min)')
gtext('A');
figure(2)
aab=a(1:100)/100;
plot(aab);
set(gca,'XTick',0:10:100);
set(gca,'XTickLabel',0:0.01:0.1);
xlabel('加工时间/{t/s}')
ylabel('加工加速度/(m/s^2)')
gtext('A');
end
%% BC 段
Sw2=0;VB=0.19*100/60;VC=1.26*100/60;
VBC_max=VC;Lbc=20+17.708;
if Sw2>=0
    Tb1=(-2*a0+sqrt(4*a0^2+4*J*(VBC_max-VB)))/(2*J);
    Tb2=sqrt((VBC_max-VC)/J);

    Sup=VB*(2*Tb1-Tb1)+J*Tb1*((2*Tb1)^2-Tb1^2)-1/6*J*((2*Tb1)^3-Tb1^3)-J*Tb1^3+VB*Tb1+1/2*a0*Tb1^2+1/6*J*Tb1^3;

    Sw2=Lbc-Sup;
    Tb3=Sw2/VBC_max;
\end{verbatim}

\begin{verbatim}
Tbc=2*Tb1+Tb2+Tb3;
for i=1:round(Tb1/T)
    a(i+1+round((2*Ta1+Ta3)/T))=a0+J*(i*T);
    V(i+1+round((2*Ta1+Ta3)/T))=VB+a0*(i*T)+0.5*J*(i*T)^2;
end
for i=round(Tb1/T):round(2*Tb1/T)
    a(i+1+round((2*Ta1+Ta3)/T))=a0-J*(i*T-Tb1)+J*Tb1;
    V(i+1+round((2*Ta1+Ta3)/T))=VB+a0*Tb1+1/2*J*Tb1^2+a0*(i*T)+2*J*Tb1*(i*T)
    -1/2*J*((i*T)^2)-(a0*Tb1+2*J*Tb1*Tb1-1/2*J*((Tb1)^2));
end
for i=round(2*Tb1/T):round((2*Tb1+Tb3)/T)
    V(i+1+round((2*Ta1+Ta3)/T))=VBC_max;
    a(i+1+round((2*Ta1+Ta3)/T))=0;
end
figure(3)
Vbc=V(6000:10000)*60/100;
plot(Vbc);
set(gca,'XTick',0:500:4000);
set(gca,'XTickLabel',6:0.5:10);
xlabel('加工时间/{t/s}')
ylabel('加工速度/(m/min)')
gtext('B');
figure(4)
abc=a(6000:10000)/100;
plot(abc);
set(gca,'XTick',0:500:4000);
set(gca,'XTickLabel',6:0.5:10);
xlabel('加工时间/{t/s}')
ylabel('加工加速度/(m/s^2)')
gtext('B');
end
% % JK 段
Sw4=0; VK=0.13*100/60; VJ=1.26*100/60;
Tk=(-2*a0+sqrt(4*a0^2+4*J*(VJ-VK)))/(2*J);
Sdown=VK*(2*Tk-Tk)+J*Tk*((2*Tk)^2-Tk^2)-1/6*J*((2*Tk)^3-Tk^3)-J*Tk^3+VK
    *Tk+1/2*a0*Tk^2+1/6*J*Tk^3;
Sw_JK=Lab-Sdown;
% % CK 段总的匀速时间
Sw=pi+40*3+Sw_JK;
Tw=Sw/VJ;
T_all=Tab+Tbc+Tw+2*Tk;
for i=1:round(Tw/T)
    V(i+1+round((2*Tb1+Tb3)/T)+round((2*Ta1+Ta3)/T))=VJ;
\end{verbatim}

\begin{verbatim}
a(i+1+round((2*Tb1+Tb3)/T)+round((2*Ta1+Ta3)/T))=0;
end
num=round(2*Tk/T)+round((2*Tb1+Tb3)/T)+round((2*Ta1+Ta3)/T)+round(Tw/T);
for i=1:round(Tk/T)

    a(num-i-1)=-a0-J*(i*T);
    V(num-i-1)=VK+a0*(i*T)+0.5*J*(i*T)^2;
end
for i=round(Tk/T):round(2*Tk/T)
    a(num-i-1)=-a0+J*(i*T-Tk)-J*Tk;

    V(num-i-1)=VK+a0*Tk+1/2*J*Tk^2+a0*(i*T)+2*J*Tk*(i*T)-1/2*J*((i*T)^2)-(a0*
    Tk+2*J*Tk*Tk-1/2*J*((Tk)^2));
end
%%单位转换
V1=V*60/100;
a1=a/100;
figure(5)
plot(V1);
set(gca,'XTick',0:10^4:10^5);
set(gca,'XTickLabel',0:10:100);
xlabel('加工时间/{t/s}')
ylabel('加工速度/(m/min)')
gtext('A');
gtext('B'); gtext('K');
gtext('J');
figure(6)
plot(a1);
set(gca,'XTick',0:10^4:10^5);
set(gca,'XTickLabel',0:10:100);
xlabel('加工时间/{t/s}')
ylabel('加工加速度/(m/s^2)')
gtext('A');
gtext('B'); gtext('K');
gtext('J');
figure(7)
plot(V1(80000:num-2));
set(gca,'XTick',0:10^3:6000);
set(gca,'XTickLabel',80:1:86);
xlabel('加工时间/{t/s}')
ylabel('加工速度/(m/min)')
gtext('K');
gtext('J');
figure(8)
\end{verbatim}

\begin{verbatim}
plot(a1(80000:num-2));
set(gca,'XTick',0:10^3:6000);
set(gca,'XTickLabel',80:1:86);
xlabel('加工时间/{t/s}');
ylabel('加工加速度/(m/s^2)');
gtext('K');
gtext('J');
%%
\end{verbatim}

\end{document}