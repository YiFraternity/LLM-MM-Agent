\documentclass[12pt]{article}
\usepackage{ctex} % 中文支持
\usepackage{amsmath, amssymb}

\begin{document}

\title{\textbf{C 题:城市交通管理中的出租车规划}}
\date{}
\maketitle

\begin{abstract}
一个好的出租车规划方案首先应该结合该城市的经济发展和自身的特点,在对资源消耗和环境污染最低的条件下,配合城市的发展目标,最大限度地满足居民的出行需求;同时不让出租车空载率过高,控制好出租车的数量,不给交通带来过多负担。另外要尽可能协调好司机和居民之间的利益关系,既要让出租车司机劳动强度适当,得到满意的净收入,又要使价格能够为居民所接受,愿意去做出租车,有效地提高空载率,活跃出租车市场。

本文问题一首先要建立一个预测该城市居民出行强度 \(A_{\mathrm{p}}\) 和出行总量 \(A_{\mathrm{N}}\) 的数学模型,并以此为基础,进一步建立出居民乘坐出租车人口的预测模型。在此用灰色系统模型对其他城市进行居民出行总量的预测并通过类比得出题中城市居民出行总量的预测。再根据附录中的三年居民数量的数据建立该城市的人口预测模型,居民出行总量除以居民数量便得到居民出行强度的预测模型。最后由交通规划中的相关模型得出该城市乘坐出租车比例的预测模型,用此比例乘以居民出行总量再除以居民出行强度便可得到乘坐出租车人口 \(N\) 的预测模型。

本文问题二是依据问题一中所建立的乘坐出租车人口的预测模型以及附录中提供的一些相关数据建立出该城市出租车最佳数量 \(T_{\mathrm{N}}\) 的预测模型。由于相关模型维数较高,非线性程度大,还受到多方面的约束,本文采用人工神经网络对模型求解。

本文问题三考虑价格是空载率 \(\eta\) 的制约因素,在此建立出租车司机收入以及居民的单次乘车消费与出租车价格的(基价 \(F\)、超过基价公里的每公里价 \(\tau\)、基价公里数 \(L\))关系模型。以居民的单次乘车消费尽可能低,司机的净收入超过其期望值的目标优化模型,协调了双方的利益。

本文问题四对附录中给出的数据,判断其合理性,并结合前面所建模型,进一步提出了合理且可行的数据采集方案。

本文问题五根据前面所建立的所有模型和数据采集方法的基础上,结合所研究的成果提出一套规划出租车市场的方案,提供给公用事业管理部门予以参考,希望能改变现在出租车市场的不景气状况。
\end{abstract}


\section{问题的重述}

“城市交通管理中的出租车规划”是要针对现在出租车行业不景气,提出如何有效地进行出租车的规划,使得出租车能配合城市将来的发展;满足居民的出行需要;同时要协调好司机和乘客的利益关系,最大程度制定出使得出租车司机能够尽可能多的得到收入和乘客满意的价格体系;同时要尽可能的减少对环境的污染和资源的消耗。此类出租车规划不善的问题在现在的许多城市中普遍存在,如不能妥善解决,不仅对国家的经济造成一定的损失,而且会引起司机罢工,顾客抱怨,严重时将会对社会的稳定也造成影响,因此如何有效的对出租车进行规划,是很值得我们做进一步的研究和关注的。

下面我们分步骤来解决出租车规划的问题:

\subsection*{问题一}
首先建立一个预测该城市居民出行强度和出行总量的数学模型,对该城市居民未来几年的出行需求有一个大致的估计,以此为基础,进一步建立出居民乘坐出租车人口的预测模型,并能够对该城市居民对出租车的需求有了基本的了解,为下一步确定出租车数量奠定基础。这个问题的模型建立非常重要,不仅要考虑该城市经济发展和自身的特点,而且还要通过国内外城市的类比来确定出一个比较完善的模型。

\subsection*{问题二}
在对出租车需求有了大致的估计之后,这里就要据此建立一个该城市出租车最佳数量的预测模型。从而既不会对城市交通和环境污染造成影响,也最大程度的满足了市民的需求。

\subsection*{问题三}
在现在出租车管理中的争论比较多的问题就是如何定价,才能够使得司机和顾客双方都能比较满意,特别是在油价调价后(从原来的 3.87 元/升到现在的 4.3 元/升),司机能够赚取的利润更少了,而顾客依然在抱怨现在的价格太高,因此在这里希望能确定一个最优方案,使双方能够得到双赢。

\subsection*{问题四}
在附录中给出了很多的数据给予我们建模提供了帮助,但是数据的采集是否有冲突、遗漏、多余、合理?对此我们应该妥善分析,如不合理,我们则应该建立一套更合理切高效可行的数据采集方法。

\subsection*{问题五}
结合上述的所有问题的分析结果、已经建立的数学模型、制定的数据采集方法等等,在该城市公用事业管理部门的立场上分析我们的研究成果,给予一整套关于出租车规划问题的方案。

\section{问题的假设}

\begin{enumerate}
    \item 所采用的数据具有较好的可信度和典型性。
    \item 城市的出租车数量与其影响因素具有普遍性。
    \item 各城市的出行总量,出行强度具有可比性。
    \item 人口的发展符合 logistic 模型规律。
    \item 油价调整不影响该城市居民的出行总量和出行强度。
    \item 出租车 90\% 的收入来源于白天的运营时间。
    \item 乘坐出租车的数量与价格的关系符合经济学中价格弹性系数规律,且各个城市的弹性系数具有可比性。
    \item 乘客对出租车的满意程度用乘坐出租车单位距离的花费作为指标来评价。
    \item 出租车司机的满意程度用其净收入来衡量。
    \item 载客趟次仅受出租车价格影响。
    \item 选择出租车为出行方式的乘客的乘坐距离满足正态分布 $(\mu, \sigma)$,其均值为 $e$。
\end{enumerate}

\section{符号的约定}

\begin{tabular}{|c|l|}
\hline 符号 & 含义 \\
\hline $R$ & 城市居民人口数量 \\
\hline $A_{c}$ & 城市居民平均日出行次数 \\
\hline $P$ & 城市居民出行方式结构中出租车所占的比例 \\
\hline $A_{p}$ & 城市居民平均日出行强度 \\
\hline $A_{n}$ & 城市居民平均日出行总量 \\
\hline $e$ & 出租车载客的单次出行距离的期望 \\
\hline $h$ & 出租车每天的总行程 \\
\hline $L$ & 出租车起租的基价公里数 \\
\hline $F$ & 白天基价租费 \\
\hline $\tau$ & 白天超过基价公里后每公里车价 \\
\hline $\eta$ & 空载率 \\
\hline $C$ & 出租车每天的成本 \\
\hline $M$ & 乘客做出租车的单次平均花费 \\
\hline $N$ & 每辆出租车每天的乘客总数 \\
\hline $n$ & 每辆出租车每天的乘客趟次 \\
\hline $d_{i}$ & 各乘客的单次出行距离 \\
\hline $I_{1}$ & 驾驶员的每天收入 \\
\hline $I_{d}$ & 驾驶员的每天固定工资 \\
\hline $O_{p}$ & 油价 \\
\hline $O_{N}$ & 出租车每公里耗油量 \\
\hline $T_{N}$ & 出租车数量 \\
\hline $V$ & 出租车平均运营车速 \\
\hline $T$ & 出租车一天平均运营时间 \\
\hline $\theta_{\text {taxi }}$ & 出租车出行比例 \\
\hline
\end{tabular}

\section{模型的分析、建立与计算}

\subsection{问题一:城市居民出行强度和出行总量的预测模型}

\subsubsection{出行总量的预测}

\paragraph{思路分析}

要预测该城市未来几年的出行强度 ($A_{\mathrm{p}}$) 和出行总量 ($A_{\mathrm{n}}$),首先要对它们之间的关系做一个分析。出行总量是指每天该城市的居民用各种交通方式(步行、自行车、公交、出租车等等)出行总的次数,量纲是‘人次’;出行强度是指出行总量与该城市的总的居民人数的比值,量纲是‘次/人·日’,其表示每人日平均出行次数。

在附录中给出了 2004、2010、2020 年这三年的居民数量(附表 1),以及 2004 年该城市居民的出行总量和出行强度(附表 5),要预测该城市未来几年的出行强度和出行总量。考虑到数据的缺乏,在建立模型时,采用其他相似城市的数据,在类比的基础上得出该城市的出行强度和出行总量的预测模型。

\paragraph{模型的建立}

通过查阅资料,考虑数据的完整性,采用上海作为类比对象,找出了其近五年的数据(出行强度、城市居民人口),用灰色系统模型对它未来几年的出行强度进行了预测,得出了它的出行强度预测模型。

\subparagraph{灰色系统模型}

灰色系统(Gray system,简称 G 系统)是指,相对于一定的认识层次,系统内部的信息部分已知,部分未知,即信息不完全。灰色模型(Gray Model)简称 GM 模型,是灰色系统理论的基本模型,它是以灰色模块(所谓模块是时间数列在时间数据平面上的连续曲线或逼近曲线与时间轴所围成的区域)为基础,以微分拟合法而建成的模型。GM 模型有以下特点:

\begin{itemize}
    \item 建模所需信息较少,通常只需 4 个以上数据即可建模(查阅相关资料得到四年数据);
    \item 不必知道原始数据分布的先验特征,对无规或服从任何分布的任意光滑离散的原始序列,通过有限次的生成即可转化成有规序列(出行强度为无规序列);
    \item 建模的精度较高,可保持原系统的特征,较好的反映系统的实际状况。
\end{itemize}

\subparagraph{求解思路}

灰色预测就是对灰色系统问题进行未来的预测,由于灰色系统理论是一种研究包含确知、未确知系统的理论,目前经常被用于交通运输客运量的预测模型,故本文采用该模型,利用 GM(1, 1) 进行建模。预测流程图见下(图 1-1)。

对于给定的一组数据

\begin{figure}[h]
\centering
\includegraphics[width=0.5\textwidth]{image.png}
\caption{灰色系统预测流程图}
\label{fig:1-1}
\end{figure}

\begin{align*}
a &= 0.00014A + 0.6789P + 0.0090G - 0.17004 \\
b &= 0.00013A - 0.2753P + 0.01249G + 0.68791
\end{align*}

1) 做 1-AGO(一次累加生成),即对原始数列中各时刻的数据依次累加,从而形成新的序列,如:对 \(X^{(0)}\) 作一次累加生成,即令

\begin{align*}
I_{1} &= (F - \tau L) \frac{\mathbf{h} - \eta \mathbf{h}}{\mathbf{e}} + \tau (\mathbf{h} - \eta \mathbf{h}) - C \\
\mathbf{M} &= \frac{\sum_{i}^{n} F + \tau(d_{i} - L)}{\sum \mathbf{d}_{i}} = \frac{n}{1 - \eta} (F - \tau L) h + \tau
\end{align*}

\begin{equation}
X^{(1)} = (X^{(1)}(1), X^{(1)}(2), \ldots, X^{(1)}(n))
\end{equation}

\begin{equation}
= (X^{(1)}(1), X^{(1)}(1) + X^{(0)}(2), \ldots, X^{(1)}(n-1) + X^{(0)}(n))
\end{equation}

新生成的数据列为一条单调增长的曲线,增加了原始数据列的规律性,而弱化了波动性。此时就可以把时间序列转化为微分方程,从而建立抽象系统的发展变化动态模型。建立如下微分方程:

\begin{equation}
\frac{dX^{(1)}}{dt} + aX^{(1)} = u \tag{1-1}
\end{equation}

按最小二乘法得到 \(a' = (B^T B)^{-1} B^T Y_1\)

其中:
\[
B = \begin{bmatrix}
-0.5(X^{(1)}(1) + X^{(1)}(2)) & 1 \\
-0.5(X^{(1)}(2) + X^{(1)}(3)) & 1 \\
\vdots & \vdots \\
-0.5(X^{(1)}(n-1) + X^{(1)}(n)) & 1
\end{bmatrix}
\quad
Y_1 = \begin{bmatrix}
X^{(0)}(2) \\
X^{(0)}(3) \\
\vdots \\
X^{(0)}(n)
\end{bmatrix}
\]
\[
a' = (a, u)^T
\]

2) 建立预测模型

易求得上述微分方程的解为:
\begin{equation}
X^{(1)}(k+1) = (X^{(0)}(1) - \frac{u}{a})e^{-ak} + \frac{u}{a} \tag{1-2}
\end{equation}

3) 还原数据
\begin{equation}
X^{(0)}(k+1) = X^{(1)}(k+1) - X^{(1)}(k) \tag{1-3}
\end{equation}

由此得到了城市居民出行强度的预测模型,有了初始年的出行强度值 \(X^{(0)}(1)\) 就可以用上述模型算出未来几年的 \(X^{(1)}(k+1)\),然后通过累减生成(累加操作的逆操作)便可得到未来几年的 \(X^{(0)}(k+1)\),其中 \(a\) 和 \(u\) 是参数。

4) 类比

假定认为题中所提到的城市的 \(a\) 和 \(u\) 两个参数与其它城市存在一定的类比关系,这样在研究其它城市数据的基础上建立类比关系就可得到该城市未来几年居民出行强度的预测值,在以后每一年中可以得到实际的出行强度的值,并用这

\subparagraph{模型求解}

编制 matlab 程序,把查得的上海市综合交通大调查的数据(见表 1-1):

\begin{tabular}{|c|l|}
\hline 符号 & 含义 \\
\hline $R$ & 城市居民人口数量 \\
\hline $A_{c}$ & 城市居民平均日出行次数 \\
\hline $P$ & 城市居民出行方式结构中出租车所占的比例 \\
\hline $A_{p}$ & 城市居民平均日出行强度 \\
\hline $A_{n}$ & 城市居民平均日出行总量 \\
\hline $e$ & 出租车载客的单次出行距离的期望 \\
\hline $h$ & 出租车每天的总行程 \\
\hline $L$ & 出租车起租的基价公里数 \\
\hline $F$ & 白天基价租费 \\
\hline $\tau$ & 白天超过基价公里后每公里车价 \\
\hline $\eta$ & 空载率 \\
\hline $C$ & 出租车每天的成本 \\
\hline $M$ & 乘客做出租车的单次平均花费 \\
\hline $N$ & 每辆出租车每天的乘客总数 \\
\hline $n$ & 每辆出租车每天的乘客趟次 \\
\hline $d_{i}$ & 各乘客的单次出行距离 \\
\hline $I_{1}$ & 驾驶员的每天收入 \\
\hline $I_{d}$ & 驾驶员的每天固定工资 \\
\hline $O_{p}$ & 油价 \\
\hline $O_{N}$ & 出租车每公里耗油量 \\
\hline $T_{N}$ & 出租车数量 \\
\hline $V$ & 出租车平均运营车速 \\
\hline $T$ & 出租车一天平均运营时间 \\
\hline $\theta_{\text {taxi }}$ & 出租车出行比例 \\
\hline
\end{tabular}

先对数据进行插值,获得连续五年的出行强度和城市居民人口数,代入得出:

\[
a' = (B^T B)^{-1} B^T Y_1 = (-0.0443, 3337)
\]

\begin{figure}[h]
\centering
\includegraphics[width=\textwidth]{image1.png}
\caption{上海 2000~2004 年出行强度分布图}
\end{figure}

\begin{figure}[h]
\centering
\includegraphics[width=\textwidth]{image2.png}
\caption{出行强度一次累加曲线}
\end{figure}

\begin{figure}[h]
\centering
\includegraphics[width=\textwidth]{image3.png}
\caption{以后几年出行强度的预测}
\end{figure}

图 1-2 上海出行强度预测图

而经过类比,上海的居民数量是题中所提到城市居民数量的 9.4 倍,分析可知 \(a\) 是一个无量纲的系数,而 \(u\) 是一个有量纲的系数(量纲为:万人次),这就说明 \(u\) 是和人口规模有关的一个系数,因此应该给予调整

\[
\frac{dX^{(1)}}{dt} + aX^{(1)} = u
\]

\[
u' = u / 9.4 = 355.0
\]

从而得到:

\[
\begin{aligned}
X^{(1)}(k+1) & = \left(X^{(0)}(1) - \frac{u}{a}\right)e^{-ak} + \frac{u}{a} \\
& = \left(3561565 + \frac{3550000}{0.043}\right)e^{0.0443k} - \frac{3550000}{0.0443}
\end{aligned}
\]

进而做一次累减生成,得到最后的出行总量预测模型为:

\begin{equation}
X^{(0)}(k+1) = X^{(1)}(k+1) - X^{(1)}(k) \quad \quad (k=0,1,2\cdots)
\tag{1-4}
\end{equation}

其中 $X^{(0)}(0)$ 代表该城市 2004 年出行总量。

以 2010 年为例,预测得其出行总量为
\[
X^{(0)}(6) = X^{(1)}(6) - X^{(1)}(5) = 4657130.141 \text{(人次)}
\]

\subsubsection{出行强度的预测}

在得到居民出行总量的预测模型基础上,便可以进行居民出行强度的预测,两者有如下的关系式:
\[
A_{N} = A_{P} * R
\]

为了对城市居民出行强度进行预测,需要有每年的居民总数,而在附录中也只列出了 2004、2010、2020 这三年的居民总数,因此可以对每年的城市居民总数建立一个人口模型进行预测。

\paragraph{Logistic 人口模型}

假设:单位时间内人口增加的数量 $\frac{dx(t)}{dt}$ 和当时人口数 $x$ 成正比,且比例系数 $r$ 为人口数 $x$ 的减函数(由于资源与环境对人口增长的限制,当 $x$ 达到某一最大允许量 $x_{m}$,应有净增长率 $r(x) = 0$,当人口数 $x$ 超过 $x_{m}$ 时,应当发生负增长)

基于如上想法,可令
\[
r(x) = r(1 - \frac{x}{x_{m}}), \quad r = const.
\]

由此导出的微分方程模型是
\begin{equation}
\frac{dx(t)}{dt} = r(1 - \frac{x}{x_{m}})x(t)
\tag{1-5}
\end{equation}

其中:$x(0) = x_{0}$,$r = const. > 0$

容易解出
\begin{equation}
x(t) = \frac{x_{m}}{[1 + (\frac{x_{m}}{x_{0}} - 1)\exp(-rt)]}.
\tag{1-6}
\end{equation}

用题目中所提供的三个已知值代入上述方程(2004 年的居民总数作为 $x_{0}$,2010 年和 2020 年的居民总数作为 $x(6)$ 和 $x(16)$),便可以解出参数 $x_{m}$ 和 $r$,从而可以预测出任意一年的居民总数。

根据 $x(t) = \frac{x_{m}}{[1 + (\frac{x_{m}}{x_{0}} - 1)\exp(-rt)]}$,其中 $t = 0, 1, 2, \cdots$($t = 0$ 表示 2004 年)

当 $t=0$ 时,$x(0)=185.15$ 万人

当 $t=6$ 时,$x(6)=259$ 万人

当 $t=16$ 时,$x(16)=321$ 万人

求出 $r=0.163 \quad x_{m}=340.85$

代入 (1-6) 得到预测人口模型为

\begin{equation}
\begin{aligned}
x(t) &= x_{m} / \left[1 + \left(x_{m} / x_{0} - 1\right) \exp(-rt)\right] \\
&= 340.85 / \left[1 + (340.85 / 185.15 - 1) \exp(-0.163t)\right]
\end{aligned}
\tag{1-7}
\end{equation}

\paragraph{出行强度预测模型}

在得出以上两个模型的基础上,便可得到出行强度预测的模型:

由 $A_{N} = A_{P} * R$,代入上面两个模型的表达式得到 $A_{P}(k) = \frac{A_{N}}{R} = \frac{X^{(0)}(k)}{x(k)}$

出行强度预测模型:

\begin{equation}
A_{P}(k) = \frac{A_{N}}{R} = \frac{X^{(0)}(k)}{x(k)}
\tag{1-8}
\end{equation}

\subsubsection{乘坐出租车人口的预测模型}

\paragraph{思路分析}

在考虑乘坐出租车人口预测模型之前,先要考虑城市各种出行方式的大致比例分布模型,可得到乘坐出租车人口的比例 $\theta$,最后由关系式:

乘坐出租车人口 $= \frac{\text{出行总量}}{\text{出行强度}} \times \text{出租车乘坐比例}$

就可求出乘坐出租车人口预测模型。

\paragraph{乘坐出租车出行的比例预测模型}

考虑到题目所提到的城市的规模会不断扩大,人口会不断增加,人民生活水平会不断的提高,同时对出租车的需求也会不断变化。但在满足人民群众的出行需要的同时,也要考虑到减少环境污染和资源消耗,实现可持续的发展。

在参考有关资料的基础上,对交通方式的比例采用以下预测模型:

(1) 经济方面考虑:
\[
m = \sum_{j \in C} \frac{T_{j}}{v_{j}}
\]
\tag{1-9}

式中 $m$ 为机动性水平,$m$ 值越小,说明系统的机动性越高;$C$ 为交通方式的选择集;$T_{j}$ 为第 $j$ 种交通方式的分担客运量;$v_{j}$ 为第 $j$ 种交通方式的常规运行速度。

(2) 社会方面考虑:
\[
s=\sum_{j \in C} T_{j} \frac{w_{j}}{a}
\]

式中 $s$ 为可达性, $s$ 值越小, 说明可达性越高; $w_{j}$ 为第 $j$ 种交通方式单位里程的平均费用; $a$ 为城市人均收入。

(3) 生态方面考虑:
\[
ep=\sum_{j \in C} T_{j}\left(e_{j}+\beta \times p_{j}\right)
\]

式中: $ep$ 为能耗及污染性, $ep$ 越小, 说明能耗及污染性越小; $e_{j}$ 为第 $j$ 种交通方式能源消耗因子; $p_{j}$ 为第 $j$ 种交通方式机动车排放因子; $\beta$ 为模型外围参数。

至此我们便可以根据现有的交通容量和需求状况, 综合考虑交通方式的各种效应, 基于可持续发展的理念, 使经济可持续、社会可持续及生态可持续达到量化平衡最优, 从而得到各种交通方式的结构比例。

(4) 模型一: 宏观控制模型

\[
\begin{aligned}
M &=\min \left(\alpha_{1} \times m+\alpha_{2} \times s+\alpha_{3} \times ep\right) \\
&=\min \left(\alpha_{1} \sum_{j \in C} \frac{T_{j}}{v_{j}}+\alpha_{2} \sum_{j \in C} T_{j} \frac{w_{j}}{a}+\alpha_{3} \sum_{j \in C} T_{j}\left(e_{j}+\beta \times p_{j}\right)\right)
\end{aligned}
\]

\[
\begin{aligned}
\text { s.t. } & \left\{\begin{array}{l}
\sum_{j \in C} T_{j}=T \\
T C_{\min j} \leq T_{j} \leq T C_{\max j}(j=1,2, \cdots, n) \\
\sum_{j=1}^{n} T_{j} \times T P_{j} \leq P(j=1,2 \cdots, n) \\
\sum_{j=1}^{n} T_{j} \times E_{j} \leq E(j=1,2 \cdots, n)
\end{array}\right.
\end{aligned}
\]

式中, $T$ 为城市交通需求总量; $C_{\min j}$ 为第 $j$ 种交通方式在容许范围内所能承担的最小比例; $C_{\max j}$ 为第 $j$ 种交通方式在容许范围内所能承担的最大比例; $P$ 为城市各污染物排放限值; $E$ 为城市各能源的交通消耗限值; $\alpha_{1}, \alpha_{2}, \alpha_{3}$ 为参数,这里不妨取为: $\alpha_{1}=\alpha_{2}=\alpha_{3}=1 / 3$ 。

模型说明: $\alpha_{1} \times m+\alpha_{2} \times s+\alpha_{3} \times ep$ 的总和形式反映了在可持续发展的目标下, 交通方式之间在竞争中达到的平衡关系, 而其中的分项形式则反映了城市交通可持续发展的多重目标, 所以, 该模型从宏观上基本能够满足城市交通可持续发展的量化要求。

(5) 模型二: 距离曲线模型

基于本题所考虑要对乘座出租车的比例进行预测, 故在宏观控制的基础上,

需要进一步建立起预测模型。本文通过查阅资料,对查到的不同城市的居民出行方式的结果进行了分析,并借鉴参考资料得出了:根据出行距离来确定方式分担比例的模型。

出租车出行方式的距离曲线模型为:
\begin{equation}
\theta_{taxi} = a \times e^{bx}
\tag{1-13}
\end{equation}
其中:
\begin{align*}
a &= 0.00014A + 0.6789P + 0.0090G - 0.17004 \\
b &= 0.00013A - 0.2753P + 0.01249G + 0.68791
\end{align*}
$x$: 居民出行距离;
$P$: 表示总出行比例;
$A$: 表示城市市区面积,本文中假定与人口规模等比例变化,考虑人口密度的增加,乘以折减系数(折减系数的取法查阅相关资料得出);
$G$: 累计人均可支配收入;
$\theta_{taxi}$: 出租车出行比例。

(6) 模型的求解

以 2010 年为预测目标年:

1) 总出行比例预测:根据 $Logistic$ 人口模型,人口规模为:城市居民人口 259 万人,总出行比例预估为出行总量与人口总数的比值。

2) 城区面积预测:人口 2010 年为 2004 年的 $259/185.15 = 1.4$ 倍,故其城市建成区面积为:$181.77 \times 1.4 \times 85\% = 216.31$ 平方公里。

3) 累计人均可支配收入预测:根据附录(表-11),给出的数据,利用二次拟合出累计人均可支配收入发展曲线从图中看出该曲线的拟合与实际情况符合得并不是很好,但考虑至少为了预估乘坐出租车的比例。故仍采用图中结果,得出 2010 年预测累计人均可支配收入为:2.30(万元)。

\begin{figure}[h]
\centering
\includegraphics[width=0.45\textwidth]{image1.png}
\caption{2002年1-12月份该城市居民累计收入与消费情况拟合}
\end{figure}
\begin{figure}[h]
\centering
\includegraphics[width=0.45\textwidth]{image2.png}
\caption{2003年1-12月份该城市居民累计收入与消费情况拟合}
\end{figure}
\begin{figure}[h]
\centering
\includegraphics[width=0.45\textwidth]{image3.png}
\caption{2004年1-12月份该城市居民累计收入与消费情况拟合}
\end{figure}
\begin{figure}[h]
\centering
\includegraphics[width=0.45\textwidth]{image4.png}
\caption{各年份该城市居民收入拟合与预测}
\end{figure}

图 1-3 该城市居民累计收入拟合图

4) $x$ 的取值,根据类比各大城市,城市居民人口出行距离大约 5.0 公里左右,故认为在预测年份中出行距离不变,采用 2004 年的出行距离,\(x=5.18\) 公里。

带入式 (1-13) 得出,出租车出行比例为:\(\theta_{taxi}=4.6970\%\)。

5) 把所求的结果带入宏观控制模型式 (1-12),综合考虑公车、步行、自行车等交通方式,基本可以得到较优的 \(M\) 值,故认为所求结果比较合理。

\paragraph{乘坐出租车人口的预测模型}

根据乘坐出租车人口与乘坐出租车出行的比例、出行总量、出行强度的关系,乘坐出租车人口 \(=\frac{\text{出行总量}}{\text{出行强度}} \times\) 乘坐出租车的出行比例,即

乘坐出租车人口的预测模型:

\[
N = \frac{A_n}{A_p} \times \theta_{taxi}
\tag{1-14}
\]

结合前面出行强度、出行总量以及人口预测模型,得出 2010 年乘坐出租车的人口为:121631.7148 人/日

问题一的预测结果表:

表 1-2 出行总量、人口、出行强度预测表

\begin{tabular}{|c|c|c|c|}
\hline 年份 & 出行总量的预测 (单位:人次) & 人口的预测 (单位:万人) & 出行强度的预测 (单位:次/人·日) \\
\hline 2006 & 3900869.234 & 212.1039983 & 1.839130457 \\
\hline 2007 & 4077562.603 & 224.8799877 & 1.813217194 \\
\hline 2008 & 4262259.458 & 237.0088553 & 1.798354518 \\
\hline 2009 & 4455322.321 & 248.3907584 & 1.793674753 \\
\hline 2010 & 4657130.141 & 258.9561738 & 1.798424062 \\
\hline 2015 & 5398500.51 & 298.9957212 & 1.805544404 \\
\hline 2020 & 8723980.044 & 320.9625436 & 2.718067955 \\
\hline
\end{tabular}

\subsection{问题二:出租车最佳数量的预测模型}

出租车交通是城市客运交通的一个重要组成部分,是城市常规公共交通的重要补充。出租车作为公共交通的一种特殊方式,由于其快速、便利、舒适、安全等特性,受到越来越多的短途(市内)出行者的青睐,促进了出租车行业的迅速发展,但是同时也给城市综合客运交通体系带来了新的问题和挑战。目前国内许多城市的出租车行业在经历了一段时间的发展之后,都不同程度地出现了总量过剩的现象,其直接的表现特征就是出租车的空驶率高,道路交通资源浪费。许多城市在出现这一局面后,都不约而同地采取了停止发放或限制发放出租车营运证的方式来控制出租车拥有量。因此要想制定出一套有效的出租车规划的方案,应首先对城市出租车拥有量给出一个有效的预测模型,这里介绍两种模型:

\subsubsection{模型一}

是通过对规划城市的居民与流动人口出行调查,结合出租车运营状况调查,获得可以为预测未来出租车拥有量的数据。出租车的空驶率与城市出租车拥有量有密切关系,从出租车所完成的城市居民和流动人口出行周转量入手,结合空驶率的分析, 对城市出租车拥有量进行计算\footnote{1}。

出租车每日总有效里程:
\begin{equation}
L_{\text{有}} = \frac{W_1}{S_1} + \frac{W_2}{S_2} \tag{2-1}
\end{equation}

根据空载率的定义得:
\begin{equation}
\eta = 1 - \frac{L_{\text{有}}}{T \cdot V \cdot N} \tag{2-2}
\end{equation}

由此得出预测出租车数量模型为:
\begin{equation}
N = \frac{\frac{W_1}{S_1} + \frac{W_2}{S_2}}{13(1-K)V} \tag{2-3}
\end{equation}

其中

\begin{itemize}
    \item $W_1$ — 出租车承担的城市居民出行周转量;
    \item $W_2$ — 出租车承担的流动人口出行周转量;
    \item $S_1$ — 城市居民乘坐出租车时平均有效车次载客人数(人);
    \item $S_2$ — 流动人口乘坐出租车时平均有效车次载客人数(人);
    \item $\eta$ — 空驶率;
    \item $V$ — 出租车平均运营车速(km/h);
    \item $T$ — 出租车日平均运营时间,取为 13 小时。
\end{itemize}

因为模型中涉及到了城市居民与流动人口的相关出行特征,只有建立对流动人口和城市居民出行总量的预测模型,才能用于对城市出租车拥有量的模型预测。

\subsubsection{模型二}

通过对城市发展的特性研究可以得到影响城市市区出租车数量的因素主要有以下四个:

\begin{enumerate}
    \item 市区人口:出租车是人的交通工具,所以出租车数量无庸质疑地与人口数量密切相关;
    \item 城区面积:交通需求的前提是距离,因此市区出租车数量与市区城区面积有着直接的联系;
    \item 人均可支配收入:出租车数量的多少,和人们的可支配收入有关。可支配收入越高,人们出门坐出租车的机会也就越多,需要的出租车数量也就越多;
    \item 出租车价格:出租车作为一件商品,其消费量同样受到其价格的影响。我们用 $0.5 \times (\text{起步金额} / \text{起步公里数} + \text{超过起步金额的每公里价格})$ 来计算出租车价格。
\end{enumerate}

假设有两个城市甲和乙,它们的出租车数量是与各自发展状况相符合的,两个城市的客运结构也大体相同。如果甲城市的人口是乙城市的两倍,其它数据完全相同,那么根据我们以往的经验,甲城市的出租车数量一般来说也应该是乙城市的两倍。相同的道理,对于城市面积、出租车消费能力,这个规律也是适用的。但任意两个城市之间都不可能只存在单一因素的差异,对于有两个以上不同因素的两个城市, 它们的出租车数量之间的关系就不那么明显了, 不能单独考虑各种因素。我们必须把城市人口 \(x\)、城市面积 \(y\)、人均可支配收入 \(u\)、出租车价格 \(w\) 这四个因素综合起来考虑, 找出城市之间的出租车数量的规律。

因此可以根据统计的数据得到若干城市的出租车数量以及影响因素的具体数据, 根据这些数据建立从影响因素到出租车数量的映射, 即 \(N = f(x, y, u, w)\)。

\paragraph{模型的建立}

为了求解该模型:
\begin{equation}
N = f(x, y, u, w) \tag{2-4}
\end{equation}
可以借助于现在比较新兴的学科:人工神经网络。

人工神经网络是基于人脑的结构和功能而建立起来的新学科。它之所以引起人们极大的兴趣, 与神经网络本身的特点是分不开的。目前, 神经网络理论和模型研究表明, 神经网络具有以下几个显著特点: (1) 能以任意精度逼近任意复杂的非线性函数;(2) 鲁棒性和容错性;(3) 并行处理;(4) 学习自适应性;(5) 联想能力。

因此我们可以采用前向传递神经网络来获得出租车数量与其影响因素的网络(见图 2-1)。

\begin{figure}[h]
\centering
\includegraphics[width=0.8\textwidth]{neural_network_diagram.png}
\caption{神经网络}
\end{figure}

如果任意设置网络初始权值, 那么对每个输入模式 \(p\), 网络输出与期望输出一般总会有误差定义网络误差:
\begin{equation}
E = \sum_{p} E_{p} \tag{2-5}
\end{equation}
式中, \(E_{p} = \frac{1}{2} \sum_{k} (d_{pj} - O_{pj})^2\), \(d_{pj}\) 表示对第 \(p\) 个输入模式输出单元 \(j\) 的期望输出。

学习规则的实质是利用梯度最速下降法, 使权值沿误差函数的负梯度方向改变。BP 算法权值修正公式可以表示为:
\begin{equation}
W_{ij}(t+1) = W_{ij}(t) + \eta \delta_{pj} O_{pj} \tag{2-6}
\end{equation}

对于输出单元
\begin{equation}
\delta_{pj} = f'(net_{pj})(d_{pj} - O_{pj}) \tag{2-7}
\end{equation}

对于隐单元
\begin{equation}
\delta_{pj} = f'(net_{pj}) \sum_{k} \delta_{pj} W_{kj} \tag{2-8}
\end{equation}

式中,$t$ 为学习次数,$\eta$ 为学习因子,$\eta$ 取值越大,每次权值的改变越激烈,可能导致学习过程发生振荡,因此,为了使学习因子的取值足够大,又不至产生振荡,通常在权值修正公式中加入一个势态项,得

\begin{equation}
W_{j}(t+1)=W_{j}(t)+\eta \delta_{p_{j}} O_{p_{j}}+\alpha\left[W_{j}(t)-W_{j}(t-1)\right]
\tag{2-9}
\end{equation}

式中,$\alpha$ 为一常数,称为势态因子,它决定上一次学习的权值变化对本次权值更新的影响程度。

\paragraph{模型的求解}

神经网络模型求解出租车数量预测:

\begin{enumerate}
    \item 查阅相关文献得到如下数据表:

    表2-1 各地城市2002,2003数据资料

    \begin{tabular}{|c|c|c|c|}
\hline 年份 & 出行总量的预测 (单位:人次) & 人口的预测 (单位:万人) & 出行强度的预测 (单位:次/人·日) \\
\hline 2006 & 3900869.234 & 212.1039983 & 1.839130457 \\
\hline 2007 & 4077562.603 & 224.8799877 & 1.813217194 \\
\hline 2008 & 4262259.458 & 237.0088553 & 1.798354518 \\
\hline 2009 & 4455322.321 & 248.3907584 & 1.793674753 \\
\hline 2010 & 4657130.141 & 258.9561738 & 1.798424062 \\
\hline 2015 & 5398500.51 & 298.9957212 & 1.805544404 \\
\hline 2020 & 8723980.044 & 320.9625436 & 2.718067955 \\
\hline
\end{tabular}

    用这些数据对神经网络进行训练,使它调节好各个参数,渐渐逼近我们的目标函数 $N=f(x,y,u,w)$。

    \item Matlab 模型仿真结果:

    网络结构图为:

    \begin{figure}[h]
\centering
\includegraphics[width=0.8\textwidth]{image.png}
\caption{神经网络结构图}
\end{figure}

    \item 网络训练误差曲线:

    \begin{figure}[h]
    \centering
    \includegraphics[width=\textwidth]{image.png}
    \caption{2-3 神经网络训练误差曲线}
\end{figure}

    得到神经网络后把本问题中城市的相关数据带入到网络进行仿真得到该城市的优化出租车数量为 5866.6。与本城市实际出租车数量比较吻合。

    因此,根据建立好的神经网络模型,要预测城市出租车的最佳数量,只需先对出租车数量的影响因素进行预测。未来规划年的城市人口 \(x\),城市面积 \(y\),以及人均可支配收入可以通过前题模型预测获得。因此只需把得到的影响因素的预测值输入到建立神经网络模型中进行预测得到即可。

    \item 分析说明

    数据的选取对于神经网络预测结果的准确度影响比较大,在本模型中,选取的城市规模都较大,如果能够给出更多的与预测城市规模相当的城市,则预测结果会更好。

    \item 预测结果

    采用问题一的预测模型得到预测出租车最佳数量所需的数据如下:

    \begin{tabular}{|c|c|}
\hline 年份 & 预测出租车数量 \\
\hline 2006 & 6307 \\
\hline 2007 & 7586 \\
\hline 2008 & 8377 \\
\hline
\end{tabular}

    由于2009年后预测数据已大于神经网络训练样本数据阈值,网络无法进行预测。对2006年到2008年的预测结果如下

    表2-3 用神经网络预测的结果

    \begin{tabular}{|c|c|}
\hline 年份 & 预测出租车数量 \\
\hline 2006 & 6307 \\
\hline 2007 & 7586 \\
\hline 2008 & 8377 \\
\hline
\end{tabular}
\end{enumerate}

\subsection{问题三:调整油价前后,出租车价格的调整方案}

\subsubsection{思路分析}

在现在的出租车行业中,司机总是抱怨自己每天工作很多时间但是收入却相对偏低,而顾客却总在抱怨出租车的价格还是偏高,不是很满意。导致出租车空载率不断上升,司机一天干了很多的无用功,对城市的交通也加重了不少的负担。特别是油价调高后,司机的净收入进一步降低,有些地方出现了罢运的情况,长此下去,若不能妥善解决司机和顾客之间的价格问题,不仅在出租车行业不景气,而且也会对国家的经济和社会的稳定造成一定的负面影响。

下表是本题中油价调价前后出租车司机的收入的对比,可以发现有所降低。

表2-4 油价调整前后对比表

\begin{tabular}{|c|c|c|}
\hline 出租车司机 & 油价调整前 & 油价调整后 \\
\hline 年总收入(万元) & 93060.4 & 93060.4 \\
\hline 年固定运营成本(万元) & 69821.92 & 69821.92 \\
\hline 年总油耗(万元) & 29906.1216 & 33229.024 \\
\hline 年盈利(万元) & -6667.6416 & -9990.541 \\
\hline 每辆车盈亏(元) & -10754.2 & -16113.8 \\
\hline 每辆车收入(元) & 25245.8 & 19886.2 \\
\hline 每个司机收入(元) & 12622.9 & 9943.1 \\
\hline
\end{tabular}

然而,对出租车进行价格调整时,需要考虑到价格对市民出行方式的影响,特别是选择出租车为出行方式的比例,从而降低到出租车的空载率。因此通过做出适当的价格调整就能协调满足驾驶员和乘客的满意程度。

\subsubsection{建立模型}

假设出租车定价为:起租基价 \( L \) 公里,基价租费 \( F \),超过基价公里后每车公里价为 \( \tau \),乘客的乘坐距离期望为 \( e \),空载率为 \( \eta \),\( h \) 为出租车每天的总行程,出租车每天的成本为 \( C \)。乘客的平均花费为 \( M \),驾驶员的收入为 \( I_1 \),每辆车每天的乘客总数为 \( N \),每辆车每天乘客趟次为 \( n \),各乘客的出行距离为 \( d_i \)。

对于驾驶员来说:驾驶员的收入

\[
I_1 = (F - \tau L) \frac{h - \eta h}{e} + \tau (h - \eta h) - C
\]

对于乘客来说:用花费的平均值来对其满意程度进行评价

\begin{equation}
M = \frac{\sum_{i}^{n} F + \tau(d_{i} - L)}{\sum \mathbf{d}_{i}} \tag{2-11}
\end{equation}

要使乘客和驾驶员都满意,就需尽可能的使油价调价后在满足 \(I_{1}\) 大于司机所期望的收入的条件下,求 \(M\) 的极小值。

于是我们的目标函数就是:(其中 \(F\)、\(\tau\)、\(L\) 和 \(\eta\) 为变量)

\begin{align*}
I_{1} &= (F - \tau L) \frac{\mathbf{h} - \eta \mathbf{h}}{\mathbf{e}} + \tau (\mathbf{h} - \eta \mathbf{h}) - C \\
\mathbf{M} &= \frac{\sum_{i}^{n} F + \tau(d_{i} - L)}{\sum \mathbf{d}_{i}} = \frac{n}{1 - \eta} (F - \tau L) h + \tau
\end{align*}

于是可以转化成

\begin{equation}
\min M = \frac{n}{1 - \eta} (F - \tau L) h + \tau
\end{equation}

\begin{equation}
s.t \begin{cases}
I_{1} = (F - \tau L) \frac{\mathbf{h} - \eta h}{\mathbf{e}} + \tau (h - \eta h) - C \geq \overline{u} \\
F - \tau L > 0 \\
Q = kP^{E_{d}} \leq n_{\max} \beta
\end{cases} \tag{2-12}
\end{equation}

其中 \(\overline{u}\) 为驾驶员的期望收入。

根据经济学原理,市场对某种商品或服务的需求主要是受到居民收入、该商品或者服务的价格等因素的影响。可以通过价格弹性系数来描述商品价格对需求的影响。需求的价格弹性系数表示的是需求量变化对价格变化反应的灵敏程度。可以表示为需求的弹性系数

\begin{equation}
E_{d} = \frac{\Delta Q}{\Delta P} \cdot \frac{P_{0}}{Q_{0}} \tag{2-13}
\end{equation}

其中 \(Q_{0}\) 为价格变动前需求量,\(P_{0}\) 为变动前价格,\(\Delta Q\) 为需求变化量,\(\Delta P\) 为价格变动量。

对上式变化得到

\begin{equation}
E_{d} = \frac{\Delta Q}{\Delta P} \cdot \frac{P_{0}}{Q_{0}} = \frac{dQ}{dP} \cdot \frac{P}{Q} \tag{2-14}
\end{equation}

求解得到

\begin{equation}
Q = kP^{E_{d}} \tag{2-15}
\end{equation}

因此出租车的乘坐人数与出租车的价格之间的关系用弹性系数法来建立。于是乘坐出租车的人数 \(N\) 可以用需求量 \(Q\) 来替代,空载率

\begin{equation}
\eta = 1 - \frac{ne}{h} = 1 - \frac{e}{h} \cdot kP^{E_d}
\tag{2-16}
\end{equation}

出租车的价格 \( P \) 用问题 2 的出租车价格指数来表示,即 \( 0.5 \times (\text{起步金额} / \text{起步公里数} + \text{超过起步金额的每公里价格}) \) 来计算出租车价格。

\begin{equation}
P = \frac{1}{2} \left( \frac{F}{L} + \tau \right)
\tag{2-17}
\end{equation}

故综合式 (2-15)、(2-16) 得以下表达式

\begin{equation}
I_1 = (F - \tau L) kP^{E_d} + \tau ekP^{E_d} - C \geq \bar{u}
\end{equation}

进一步

\begin{equation}
\min M = \frac{1}{e} (F - \tau L) + \tau
\end{equation}

\begin{equation}
\begin{aligned}
s.t \quad \left\{
\begin{aligned}
I_1 &= 2^{-E_d} ek\tau \left( \frac{F}{L} + \tau \right)^{E_d} + 2^{-E_d} k \left( \frac{F}{L} + \tau \right)^{E_d} (F - \tau L) - C \geq \bar{u} \\
F - \tau L &> 0 \\
Q &= kP^{E_d} \leq n_{\max} \beta
\end{aligned}
\right.
\tag{2-18}
\end{aligned}
\end{equation}

其中:\( n_{\max} \) 为最大载客能力的次数,\( \beta \) 为一实际载客系数,查取相关文献取其值为 0.65。

\subsubsection{模型求解与讨论}

\paragraph{Matlab 编程求解}

首先再来回顾一下一些参数值:

\begin{enumerate}
    \item \( C = 112612 / 365 + h \times O_p \) (\( O_p \) 为油价,\( h \) 为出租车每天的总行程)
    \item \( h = 424 \) 公里
    \item 油价调整前:\( C = 308.5 + 424 \times 0.387 = 472.61 \) 元
    \item 油价调整后:\( C = 308.5 + 424 \times 0.43 = 490.84 \) 元
    \item 出租车乘客的平均乘坐期望 \( e = 210.07 / 40.52 = 5.184 \) 公里
    \item 调价前 \( L = 3 \),\( \tau = 1.6 \),\( F = 8 \) (\( L \) 为出租车起租的基价公里数,\( \tau \) 为白天超过基价后每公里车价,\( F \) 为白天基价租费)
    \item 因为目前的出租车价格情况下出租车司机每天的盈余为一负值,这里不妨先令驾驶员的期望收入 \( \bar{u} \) 取 0 来代如求解。
    \item 假定出租车的价格弹性系数 \( E_d \) 以及参数 \( k \) 各个城市基本不变,因此可根据其他城市的数据得到出租车的价格弹性系数 \( E_d \) 以及参数 \( k \),查阅数据和利用公式 \( Q = kP^{E_d} \),计算得到,\( k = 674.5387 \),\( E_d = -3.5 \)。
\end{enumerate}

经过编程求解上述规划问题,得到以下结果:

油价上涨前:最优解为 \([F \ L \ \tau] = [4.7 \ 1.6 \ 1.2]\);出租车驾驶员收入 \( I_1 = 4.1611 \) 元;居民每公里花费 \( M \) 为 1.7363 元。

油价上涨后:最优解为 \([F \ L \ \tau] = [4.1 \ 1.5 \ 1.4]\);出租车驾驶员收入 \( I_1 = 1.2661 \);

\begin{tabular}{|c|c|c|}
\hline 出租车司机 & 油价调整前 & 油价调整后 \\
\hline 年总收入(万元) & 93060.4 & 93060.4 \\
\hline 年固定运营成本(万元) & 69821.92 & 69821.92 \\
\hline 年总油耗(万元) & 29906.1216 & 33229.024 \\
\hline 年盈利(万元) & -6667.6416 & -9990.541 \\
\hline 每辆车盈亏(元) & -10754.2 & -16113.8 \\
\hline 每辆车收入(元) & 25245.8 & 19886.2 \\
\hline 每个司机收入(元) & 12622.9 & 9943.1 \\
\hline
\end{tabular}

\paragraph{模型分析}

\begin{enumerate}
    \item 这里使用的出租车价格未把夜间出车价格和远程载客回空费用考虑进去,因为一般情况下夜间费用都是在白天价格的基础上增加一个附加费用比例,故只需先考虑白天费用。另外根据居民不同时距出行方式结果表格可以得出结论:远程出行居民采用出租车所占所以乘坐出租车居民数比例很小,故先不考虑这个回空费用得到的模型也能基本正确的反应出租车价格。
    \item 模型中因为根据居民不同时距出行方式结果表格得出结论居民乘坐出租车的平均期望距离大于出租车的起步基价的公里数目,因此为简化模型,考虑居民乘坐出租车的平均公里花费仅考虑 \(d > L\) 的情况。此假设也有一定的实际合理性。
    \item 从最优结果可以看出,价格调整后的出租车驾驶员收入比出租车价格调整前要增加(在价格调整前可以看到出租车驾驶员的每天节余为负,而调整后还有盈余)。虽然驾驶员每天的净收入较低,但是注意到在驾驶员的出租车年固定成本中包含了其固定工资收入,因此该数据也是合理的。而油价调整对居民的每公里花费的增加也是符合实际情况的。另外分析模型的结果可以得到一般性的结论,油价的上涨,可以通过降低出租车的起步价和驾驶员的额外收入 \(F - \tau L\),来吸引更多的乘客而降低空载率,而驾驶员的每天的收入也能有盈余。
\end{enumerate}

\subsection{问题四:数据采集的合理性分析}

\subsubsection{数据的合理性分析}

在本题附录中所给的数据中,大致可以分为以下几类:

\begin{itemize}
    \item 城市规模和道路情况
    \item 城市出租车的主要状况
    \item 城市公交主要状况
    \item 城市公共出行情况
    \item 城市居民累计收入与消费情况
\end{itemize}

城市规模数据可用于人口模型的预测和城市的类比,道路情况可大致确定该城市的交通状况,从而得到各条道路的流量状况,以预测是否需要增宽道路而满足流量需求。

城市出租车的主要状况在本题中是重要的参数:每公里耗油量、空载率、平均行驶速度、日客运量、日载客趟次、日行驶总里程、日营运总收入、出租车固定营运成本、起租基价公里数、基价租费、超过起租基价公里每车公里价,这些参数在乘坐出租车人口的预测模型、出租车最佳数量预测模型、价格调整方案中都起了作用。其它一些给出的参数:每台车日均载客人次、日均载客趟次、每趟载客人次、载客里程、空驶里程、里程利用率、日平均营业里程、平均载客里程、平均空驶里程都可以从前面的参数中得到,有些多余。价格的晚间计费方式、回空费和等候费在模型的改进部分中有所涉及。

城市公交主要状况给出了该城市的详细公交资料,虽在本题中没有涉及到这些参数,但是它对于一个城市的交通规划是非常重要的,如果公交线路的状况不理想,人们势必会多考虑选择出租车去较远的地方,而如果公交线路较发达,则会使出租车的空载率有所增加。在模型改进部分也对此进行了讨论。

城市公共出行情况中有好几张表格,其中出行强度表、出行方式结构表对居民出行强度、出行总量预测模型和乘坐出租车人口预测模型有关系。而居民出行目的结构、不同时距出行方式结构、出行方式平均耗时这几张表格则只有在详细规划该城市各个区时才用到,在本题中的模型中没有涉及。

城市居民累计收入与消费情况和乘坐出租车预测模型、出租车数量预测模型、价格的调整方案都有关系。

综上所述,该题给出的数据大都和题目有着或多或少的关联,然而有些数据是多余的,有些数据给的有些冲突,而有些数据虽然和题目没直接关系,但当用于更详细的分析可以用到。

\subsubsection{优化的数据采集方案}

\begin{itemize}
    \item 关于建立居民出行强度和出行总量的预测模型,需要当年的居民人口数量、出行强度,如果若能够多采集近几年的出行强度,则能直接建立该城市的居民出行强度的预测模型,这样就不需要类比,会使模型更加准确,同时出行总量预测模型也将会有所改善。但考虑到出行强度的调查费时费力,以上海为例:2004年上海交通大调查,花费金额达数千万,故考虑出行强度的调查有周期性,在短期内可以选取具有代表性的城区进行调查。针对乘坐出租车人口的预测模型,如果能采集到居民连续几年出行距离、城市市区面积、累计人均可支配收入等数据,将能对模型进一步进行检验并做出修正,使模型更接近事实。
    \item 关于建立出租车最佳数量预测模型,需要城市人口、城市面积、人均可支配收入、出租车价格这些数据,而出租车价格本来就是一个值得商确的数据量,因此在实际采集中若能给出相同规模的城市(城市人口、城市面积)的出租车数量给予对比,则能对模型做进一步的优化。
    \item 关于价格的调整方案中要用到出租车价格、空载率、成本、日行驶总里程、日运营收入等数据,由于在模型中要用到出租车的价格弹性系数 \( E_d \) 以及参数 \( K \),虽然在各城市中这些参数基本不变,但是若能够对该城市采集相应的价格和其需求的数据,则能得出较为准确的价格弹性系数 \( E_d \) 以及参数 \( K \),变能使模型也更加的趋于实际的情况。
\end{itemize}

\subsection{问题五:出租车规划问题}

\subsubsection{现存问题}

出租车是城市交通工具的重要组成部分。随着经济的发展和城市现代化进程的加快,出租车在我国大中城市迅速发展起来,它关系到人民群众的切身利益,

因而备受社会各界的关注。根据本文上述的分析,在目前的模式下,该城市出租车的运营存在着以下的几个问题:

\begin{itemize}
    \item[A、] 空载率高。其空载率达到了 \(50\%\) 左右,而该数值根据其他城市的数据显示,比较理想的出租车空载率应该为 \(30\%-40\%\) 左右。空载率体现的是资源利用率——空载率越高则能源的利用率越低,这和国家提出的发展集约型社会的目标相背。因此,有效的降低出租车的空载率有利于城市的发展。
    \item[B、] 出租车驾驶员的工作状况不佳。通过统计数据计算得到驾驶员的平均日工作时间超过 13 个小时,而其收入除去固定工资外,盈余为负值。这对出租车驾驶员来说是不合理的。而出租车的运营成本由于油价的上涨而增大,使得出租车司机的收入更少。
    \item[C、] 出租车价格不合理。居民乘坐出租车出行每趟至少花费 8 元,其平均花费约为 11 元左右。这个价格所占居民生活消费比例较大,直接限制了乘坐出租车的人数。由于价格管制,消费者常抱怨价格太高而制约了他们对出租车的需求,认为该行业不能为他们带来或享受更多的社会福利;顾客的稀缺同样导致出租车空载率上升,使车辆运行成本增加。公司与乘客利益对出租车司机形成双重挤压,这种经济利益的非均衡性,不仅使得司机处于一种欲罢不能的困境,而且对城市交通造成很大的压力。
    \item[D、] 调查统计表明,一辆出租车的道路时空资源消耗是一般机动车的 3.6 倍,是别的小汽车的 4-5 倍。也就是说,每增加一辆出租车,给城市道路所造成的压力相当于 4-5 辆小汽车,而出租车的运输效率如同一般小汽车一样是最低的。
\end{itemize}

\subsubsection{规划方案}

\begin{itemize}
    \item[A、] 采用问题四中提出的合理数据采集方式,尽可能多的收集与该城市相关的近几年的数据资料,在此基础上首先用问题一中的方法确定该城市乘坐出租车人口的预测模型,并可在今后几年继续收集资料与预测出来的数据作比较,不断去修正该模型,使之不断完善准确。
    \item[B、] 继续在相关统计资料、数据的基础上,用问题二中的方法建立该城市的最佳出租车数量预测模型,从而有效的控制出租车数量,使之即不会太少而使居民出行需要不能得到满足,又不会太多导致空载率上升,对城市交通加重负担,使资源利用率过低。
    \item[C、] 有了前面两个步骤所建立的模型,并结合该城市居民收入与消费水平的统计数据便可对出租车的价格用问题三中的方法进行模型的建立。从而调节出使司机与顾客双方互相协调,彼此满意的价格,有效降低空载率,缓解城市交通压力。
    \item[D、] 进一步考虑其它出行方式的改善将会对出租车市场造成影响,促进其不断提高服务质量,为居民更好的服务。
\end{itemize}

\section{模型的评价}

本文运用社会经济学以及城市规划原理和知识,采用灰色预测理论建立了城市人口出行强度、出行总量以及乘坐出租车的人口数量的预测模型,并做出预测结果。根据城市交通规划理论,利用神经网络建立了城市出租车数量与其影响因素之间的模型,并根据上述预测模型的预测数据可得到城市出租车的最佳数量预测模型。经验证,结果基本符合。此外,本文综合考虑价格对出租车运营的影响,

通过引入出租车驾驶员和居民的满意指标建立单目标单约束条件的最优化模型。引入价格弹性系数来描述价格对出租车乘客数目以及空载率的影响,通过求解得到一个最优的出租车价格调整方案。最后对本文的数据采集方案进行分析,并提出一个改进的数据采集方案。但是由于城市交通模型的复杂性,目前并无一种能完全有效的模型来加以描述,本文的模型存在着一些值得改进的地方。

总的看来,本文模型具有以下一些优点:

\begin{enumerate}
    \item 引入灰色预测模型做出预测。此方法能较好地对受不确定因素制约的系统进行预测如本文中的居民出行总量。此方法广泛用于交通运输的预测模型。
    \item 以可持续发展作为总体控制目标在宏观上优化各种交通方式的结构比例,引入距离曲线建立城市交通方式分担率模型,并有效建立出居民乘坐出租车人口的预测模型。
    \item 综合考虑各种出租车数量确定方法,采用神经网络建立出租车数量与其影响因素关系来得到预测模型,使得复杂的函数求解变得很方便与可行。
    \item 引入价格弹性系数描述价格,使得价格的确立符合市场供需规律。
\end{enumerate}

尚存在的不足及对改进方法的探讨:

\begin{enumerate}
    \item 使用类比的方法得到居民出行强度和出行总量,而类比对象之间的关系是根据经验得到的,存在着一些局限性和需改进之处。
    \item 本文在建立模型时,较多的参考其他城市的相关规律。
    \item 神经网络的初始权值对预测结果的影响较大,目前对神经网络初始权值的选取没有一套比较有效的方法。而且进行预测时,输入样本必须在训练样本的阈值范围内,这个是神经网络的一个缺陷,可根据增大训练样本的阈值加以克服。
    \item 根据出租车价格优化根据模型假设,本文在考虑出租车价格时,只考虑出租车白天驾驶收费情况,回空费用以及等待费用未考虑进去。可通过采集数据获得这部分费用所占出租车收入的比例,再对模型加以修正。
    \item 价格弹性系数是根据其他城市数据得到的,其精确程度直接影响到模型的最优解。
\end{enumerate}

\section{参考文献:}

\begin{enumerate}
    \item 姜启源. 数学模型(第三版). 北京: 高等教育出版社, 2003.
    \item 雷公炎. 数学模型讲义. 北京: 北京大学出版社, 1999.
    \item 李强. maple 8 基础应用教程. 北京: 中国水利水电出版社, 2004.
    \item 丁大正. mathematica 4 教程. 北京: 电子工业出版社, 2002.
    \item 李海涛. matlab 程序设计教程. 北京: 高等教育出版社, 2002.
    \item 刘长虹等. 客运量预测方法的探讨. 上海工程技术大学学报, 2004.9.
    \item 葛亮、王炜等. 基于可持续发展的城市客运交通方式分担率预测模型研究. 公路交通科技, 2004.8.
    \item 王俊、陈学武. 用经济学理论分析出租汽车服务定价机制. 交通运输工程与信息学报, 2004.12.
    \item 胡坚. 弹性系数法在需求预测中的应用. 知识讲座.
    \item 陆建、王炜. 城市出租车拥有量确定方法. 交通运输工程学报, 2004.3.
\end{enumerate}

\end{document}