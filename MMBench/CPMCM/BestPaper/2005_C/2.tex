\documentclass{article}
\usepackage[UTF8]{ctex}  % 中文支持

\title{出租车预测模型}

\begin{document}

\maketitle

% 插入目录
\tableofcontents
\newpage

\section{问题的提出}

最近几年,出租车经常成为居民、新闻媒体议论的话题。某城市居民普遍反映出租车价格偏高,而另一方面,出租车司机却抱怨劳动强度大,收入相对来说偏低,甚至发生出租车司机罢运的情况,这反映出租车市场管理存在一定问题,整个出租车行业不景气,长此以往将影响社会稳定,值得关注。

我国城市在未来一段时间内,规模会不断扩大,人口会不断增长,人民生活水平将不断提高,对出租车的需求也会不断变化。如何配合城市发展的战略目标,最大限度地满足人民群众的出行需要,减少环境污染和资源消耗,协调各阶层的利益关系,下列问题是值得深入研究的。(附录中给出了某城市的相关数据)。

(1)考虑以上因素,结合该城市经济发展和自身特点,类比国内外城市情况,预测该城市居民出行强度和出行总量,同时进一步给出该城市当前与今后若干年乘坐出租车人口的预测模型。

(2)给出该城市出租车最佳数量预测模型。

(3)按油价调价前后(3.87 元/升与 4.30 元/升),分别讨论是否存在能够使得市民与出租车司机双方都满意的价格调整方案。若存在,给出最优方案。

(4)本题给出的数据的采集是否合理,如有不合理之处,请你给出更合理且实际可行的数据采集方案。

(5)请你们站在市公用事业管理部门的立场上考虑出租车规划问题,并将你们的研究成果写成一篇短文,向市公用事业管理部门概括介绍你们的方案。

\section{模型建立与求解}

\subsection{问题 1:居民出行强度和出行总量预测}

\subsubsection{问题分析}

随着经济的增长,导致了居民的累计人均可支配收入和累计人均生活消费支出的增长。而这两个指标增长,使得居民人均出行强度增大,从而导致居民出行使用出租车的比例将增大。同时居民和外来人口的增长,也将导致出租车使用量的增加。

附录列出的跟据统计分析得到的表格都是基于居民的实际情况得到的,而居民的概念我们认为是常住人口,不包括流动人口。因为根据城市不同区域居民出行强度的表格可以看出,全市人数为 184.325 万人,和城市总体规划人口模型中 2004 年的常住人口

185.15 万人很接近,而不是第一类人口 218.15(185.15+33)万人,更不是 240.15(185.15+55)万人。从统计分析角度来说,居民也应该代表常住人口。因为居民出行强度统计、居民出行目的结构统计、居民出行方式统计、居民不同时距出行方式结构统计、居民出行分方式平均耗时统计、居民出行全方式 OD 分布统计等等这些表格来说,如果居民包括流动人口,那么数据的统计工作将非常困难。相对常住人口来说,流动人口特别是短期及当日进出人口,他们的出行方式、出行强度、出行目的结构、出行方式结构、出行分方式平均耗时和出行全方式 OD 分布变化将非常大,并且在统计上是没有规律可循的。综合上面的分析,我们认为题目中居民的概念只是代表城市的常住人口。

要预测今后若干年平均乘坐出租车的人口,首先必须得到居民、暂住人口和第二类人口出行强度和总量的预测值。而出行强度和居民平均收入预测和消费支出预测有关。因此人口总量和平均收入预测和消费支出预测是首先的,在此基础上才可以做出行强度和总量的预测,最后才是平均乘坐出租车的人口的预测。

\subsubsection{符号约定}

\begin{itemize}
    \item $m$——出租车日客运总量
    \item $m_1$——出租车日客运居民总量
    \item $m_2$——出租车日客运流动人口总量,显然 $m = m_1 + m_2$
    \item $N_1$——居民人口总量
    \item $N_2$——流动人口总量
    \item $N(t)$——$t$ 时刻总人口量
    \item $N_0$——初始时刻(2004 年)总人口量
    \item $C$——选择出租车出行人口总量
    \item $C_1$——居民中选择出租车出行人口总量
    \item $C_2$——流动人口中选择出租车出行人口总量,显然 $C = C_1 + C_2$
    \item $S_{ij}(t)$——$t$ 时刻居民出行全方式 OD 值
    \item $\Pi_i(t)$——$t$ 时刻第 $i$ 小区出行强度
    \item $r_i(t)$——$t$ 时刻第 $i$ 小区人口占总人口比例
    \item $I(t)$——$t$时刻总出行强度
    \item $outgo(t)$——$t$时刻总消费支出占总收入的比例
    \item $\alpha$——生存性出行强度
    \item $P_{i}(t)$——$t$时刻第$i$小区人口数
    \item $i$——城市小区标号,$n=1,\cdots,6$
    \item $Z(t)$——$t$时刻城市总人口
    \item $X$——出租车出行方式占所有出行方式的比例
    \item $K$——出行强度与总消费支出的比例
    \item $d_{ij}$——第$i$小区到第$j$小区的吸引指数
    \item $M(t)$——$t$时刻居民出行总量
\end{itemize}

\subsubsection{居民消费支出预测}

\paragraph{问题的分析}

居民的出行强度变化与居民的生活消费支出关系较为密切,预测城市居民的出行强度规律应首先分析掌握城市居民的人均生活消费支出的变化规律。

根据题目给定的已知城市2002到2004年居民累计收入与消费情况,通过建立时间序列预测模型,可以预测出以后一个时间段内的居民累积收入和消费支出。目前来说常用的时间序列预测模型有$ARMA(p,q)$,$ARIMA(p,q)$和$ARCH$模型等。

附录2中累计人均可支配收入和累积人均生活消费支出是分年累计的。对于缺失的数据项可以根据前后月份的数据差值平滑处理获得。处理后做图如下:

\begin{figure}[h]
    \centering
    \includegraphics[width=\textwidth]{image.png}
    \caption{居民人均可支配收入与生活消费支出波动图}
\end{figure}

使用SPSS软件对居民收入和累计人均生活消费支出进行独立性检验,可以发现两组数据具有一定的相关性(相关系数为0.726)。以每12个月的处理后数据序列,作出序列的自相关函数图和偏自相关函数图,来检验整体时间序列的平稳性和周期性,可以发现:

(1) 自相关函数和偏自相关函数均不截尾。

(2) 数据有明显的周期性,且某一时期其波动剧烈而另一时期又相对平缓,表现出“波动聚集,高峰厚尾,持久记忆”等现象。

$ARMA(p, q)$ 模型比较适合应用于平稳时间序列的偏相关系数 $\phi_k$ 和自相关系数 $r_k$ 均不截尾,但较快收敛到 0 的数据序列。对于如本题中波动起伏较大且某一时期其波动剧烈而另一时期又相对平缓的数据序列,经典 ARMA 模型已不能较好的拟合和预测,Engle(1982)提出自回归条件异方差(ARCH)模型,把方差和条件方差区分开,让条件方差作为过去误差的函数而变化,为解决异方差提供新的途径。Bollerslev(1986)在此基础上提出广义自回归条件异方差(GARCH)模型,让条件方差作为过去误差和滞后条件方差的函数而变化,更好地体现出波动聚集效应。

\paragraph{预测模型的建立}

广义自回归条件异方差(GARCH)模型

\begin{align*}
Y_t &= u + \sum_{i=1}^k \phi_i Y_{t-i} + u_t \\
u_t &= v_t - \sum_{i=1}^m \varphi_i u_{t-i} \\
v_t &= \sqrt{h_t} e_t \\
h_t &= \omega + \sum_{i=1}^q \alpha_i v_{t-i}^2 + \sum_{j=1}^p \gamma_j h_{t-j}
\end{align*}

其中 $h_t = E\left(v_t^2 \mid \Omega_{t-1}\right)$,$\Omega_{t-1}$ 是时刻 $t-1$ 及 $t-1$ 之前的全部信息,$\phi_i$,$\varphi_i$,$\omega$,$\alpha_i$,$\gamma_j$ 为参数 ($\alpha_i \geq 0, i=1, \cdots q; \gamma_j \geq 0, j=1, \cdots, p$,$\sum_{i=1}^q \alpha_i + \sum_{j=1}^p \gamma_j < 1$),$e_t$ 是均值为 0,方差为 1 的白噪声。记为 $AR(k)-AR(m)-GARCH(p, q)$。当 $\gamma_j = 0$,$j=1, \cdots, p$ 时,成为 $AR(k)-AR(m)-ARCH(q)$。

为了比较自回归条件异方差模型和 $ARMA(2,2)$ 模型在预测本题数据的准确性,选用前 30 个月两个数据的时间序列,分别使用两个模型预测后六个月的数据为待检验数据,预测结果和实际结果对比如下:

人均可支配收入预测模型效果:$ARMA(2,2)$ 的平均误差为 59.0393,$GARCH$ 的平均误差为 13.9175。

人均生活消费支出预测模型效果:$ARMA(2,2)$ 的平均误差为 48.5800,$GARCH$ 的平均误差为 17.3997。

\paragraph{预测结果}

使用 $GARCH$ 模型预测利用现有数据预测未来两年人均可支配收入及人均消费支出数据的结果如下图所示:

\begin{figure}[h]
    \centering
    \includegraphics[width=\textwidth]{image1.png}
    \caption{GARCH 模型预测人均可支配收入数据结果}
\end{figure}

\begin{figure}[h]
    \centering
    \includegraphics[width=\textwidth]{image2.png}
    \caption{GARCH 模型预测人均消费支出数据结果}
\end{figure}

预测结果为 2005 年平均

\subsubsection{城市居民人口预测}

题目中缺乏该城市的历史人口统计数据,对于城市未来人口的预测只能根据城市规划人口数和经典的人口 Logistic 数学模型来完成。

\begin{itemize}
    \item \textbf{模型的建立}

    Logistic 模型的描述如下:
    \begin{align*}
Y_t &= u + \sum_{i=1}^k \phi_i Y_{t-i} + u_t \\
u_t &= v_t - \sum_{i=1}^m \varphi_i u_{t-i} \\
v_t &= \sqrt{h_t} e_t \\
h_t &= \omega + \sum_{i=1}^q \alpha_i v_{t-i}^2 + \sum_{j=1}^p \gamma_j h_{t-j}
\end{align*}
    $N(t)$ 为 $t$ 时刻的人口总量,$N(0)$ 为初始人口总量,$a, b$ 为待定参数。

    \item \textbf{模型的求解}

    解此模型得到:
    \begin{align*}
(\lambda + n\mu + (N-n)m\beta) \, P(m,n) - \\
(n+1)\mu P(m,n+1) - \lambda P(m-1,n) - (N-n+1)(m+1)\beta P(m+1,n-1) = 0
\end{align*}
    根据城市总体规划的三组人口数据,可解得:$a = 0.163, b = 0.0004785$。
\end{itemize}

人口预测曲线如下图所示:

\begin{figure}[h]
    \centering
    \includegraphics[width=\textwidth]{population_prediction.png}
    \caption{城市居民人口总量预测图}
\end{figure}

\subsubsection{出行强度预测}

城市中出行主要包括生活性出行和生存性出行。

生活性出行主要包括生活购物、文体娱乐、探亲访友,私有经营等项目的出行。生存性出行主要包括上班、上学、公务出差、看病和回程等项目的出行。

生活性出行占总出行量比例反映了城市的发达程度。综合考察国内外城市发展和居民不同性质出行总量,可以认为居民生存性出行总量在一定城市发展阶段是一定量。而实际消费支出量深刻地影响着生活性出行总量,从而深刻地影响着居民平均出行强度。附录 I 中以图表方式介绍了浙江省关于两种出行方式消费支出的统计数据。

根据相关省市的统计数据可以得到如下结论:

总消费支出的变化对于生存性出行次数的影响较小, 而对生活性出行次数影响较大, 两者呈线性关系。如下式所示:

\[
I(t) = k \cdot outgo(t) + a
\]

其中 $a$ 表示日生存性出行强度, $I(t)$ 日总出行强度, $k$ 为比例系数。

则 $I(t + \Delta t) - I(t) = k \cdot (outgo(t + \Delta t) - outgo(t))$

日出行强度变化率和总消费支出变化率为正比关系。系数可以通过数据拟合得到。日出行强度依赖于总消费支出变化率, 下图为以月为单位的预测日出行强度。

\begin{figure}[h]
    \centering
    \includegraphics[width=\textwidth]{image.png}
    \caption{日出行强度预测数据}
\end{figure}

\paragraph{城市中心区和边缘区的划分与预测模型的检验}

首先我们可以根据分析城市不同区域居民的出行强度和居民出行全方式 OD 分布的两个表格分析得出该城市的 6 个区, 哪几个是中心区, 哪几个又是边缘区。

我们以出行次数(万人次)作为两个表格比较的指标。首先全市总的出行次数可以从两个表格中得出, 一个是 3561565 人次, 另一个是 3561564 人次, 在统计误差允许的范围内可以认为是一致的。在城市不同区域居民的出行强度的表格中分别给出了中心区和边缘区的出行次数, 中心区 2314398 人次, 边缘区 1247167 人次。将这两个数据和居民出行全方式 OD 分布表格中 6 个区的出行数据进行比较分析。我们以每个区出发的人次作为该区的出行人次, 那么 1-6 区出行的人次分别为 635658、741396、720211、524135、860893 和 79269 人次。下面就是要根据这些数据来判断哪几个区是中心区, 哪几个区是边缘区。首先通过第二个表格我们可以判断 6 区肯定是边缘区。因为 6 区出发的人次和其它各区到 6 区的人次相对表格中的其它值低了一个数量级, 显然 6 区位于该城市的边

在此基础上,建立一个中心区和边缘区划分的原则:在其它各区出行人次之间相差不是太大的情况下,采用统计误差最小原则判断哪几个区和 6 区是边缘区,主要判断它们的出行人次相加之和与第一个表格给出的城市边缘区出行人次最接近。下面进行参数的定义和划分原则的描述。

$i$ 为 6 个区的区号,$i=1,\cdots,6$。

$x(i)$ 为第 $i$ 区每日出行的人次。

$\alpha_{i}$ 为第 $i$ 区是否为边缘区的系数,$\alpha_{i}=0$ 表示第 $i$ 区不是边缘区,是中心区;$\alpha_{i}=1$ 表示第 $i$ 区是边缘区。

划分的原则为:
\[
\forall \alpha_{i}, \min \left\{ \text{abs} \left[ \sum_{i=1}^{5} x(i) \cdot \alpha_{i} + x(6) - Y \right] \right\}, \quad i=1,\cdots,5.
\]

得到结果为 1、4 和 6 区为城市边缘区,2、3 和 5 是城市中心区。这样得到的最小统计误差为 8105 人次。与总出行人次数相比,可以认为是统计误差。

根据调查和分析全国主要城市的市区和郊区居民出行强度,中心区和边缘区居民出行强度的比例随着城市和经济的发展基本保持不变。可以利用这一点来检验预测模型的可信度。

在出行总量差分模型计算时间迭代过程中,加入各小区的出行强度的计算。从而可以更新居民出行全方式 $OD$ 图 $S_{ij}(t)$。从而得出各区出行强度:
\[
I_{i}(t) = \frac{\sum_{j} S_{ij}(t)}{r_{i}(t) \cdot N(t)}
\]

计算中心区域和边缘区域出行强度的比例,和基本保持不变的初始值比较,就可以用来检验模型的可行度。

其中 $S_{ij}(t)$ 为 $t$ 时刻的居民出行全方式 $OD$ 图。$r_{i}(t)$ 为 $t$ 时刻第 $i$ 小区人口占总人口的比例,$N(t)$ 为总人口。

\begin{figure}[h]
    \centering
    \includegraphics[width=\textwidth]{image.png}
    \caption{中心区与边缘区出行强度变化}
\end{figure}

可见随着时间的推移,中心和边缘出行强度比例始终维持在 1.40 左右,且最大变化不超过 5\%,验证了模型的可信度。

\subsubsection{出行总量预测}

\paragraph{模型的假设}

(1) 新增人口按各小区现有人口比例平均地分配到各个小区(即该城市没有特别的处理外来人口居住的的措施)。

(2) 随着新增人口和城市的发展,在预测时间范围没有添加新的小区,且城市小区间相对距离保持不变。

(3) 各小区到自身的居民出行人次可以主要取决于该区的人口数量。

\paragraph{模型的建立及求解}

设 $P_{i}(t)$ 为 $t$ 时刻 $i$ 小区的人口,$S_{ij}(t)$ 矩阵为 $t$ 时刻小区之间人次流量 $OD$ 图,$N$ 为小区总数,$t$ 时刻第 $i$ 小区人口占总人口比例,则:

\begin{equation}
\frac{P_{i}(t)}{\sum_{i}P_{i}(t)} = \frac{S_{ii}(t)}{\sum_{i}S_{ii}(t)} = r_{i} \quad (i=1,2...N)
\end{equation}

设 $Z(t)$ 为该城市在时刻 $t$ 时总人口。

出行总量的预测采用基于重力系数预测模型的出行总量差分预测模型,重力模型法根据牛顿的万有引力定律,即两物体间的引力与两物体的质量之积成正比,而与它们之间距离的平方成反比类推而成。该模型于 1955 年由 Casey 最早提出。它可以根据已有的交通 $OD$ 图和各区人口总数,计算得到各小区之间的距离比例。进而得到各小区人口

增加之后的 \(OD\) 图。综合考虑随着消费支出带来的人们出行强度的提高,可以得到出行总量的时间预测。

\[
S_{ij}(t) = \alpha \cdot \frac{P_i(t) \cdot P_j(t)}{d_{ij}^2}
\]

其中 \(P_i\) 表示第 \(i\) 区的总人口,\(d_{ij}\) 表示从 \(i\) 区和 \(j\) 区间的居民平均出行距离,\(S_{ij}\) 为从 \(i\) 区和 \(j\) 区间的居民流量总人次数。 \(\alpha\) 为一比例常数,其大小反映了两个小区之间的道路状况,依据本模型假设(2),可以认为 \(\alpha\) 不随时间变化而变化。

可以根据现有数据得到各各小区之间的吸引指数:

\[
d_{ij} = \sqrt{\alpha \cdot \frac{P_i(t) \cdot P_j(t)}{S_{ij}(t)}}
\]

依据本模型假设(2),认为吸引指数不随时间的变化而变化。

依据本模型假设(1),各小区人口随时间的变化满足

\[
P_i(t + \Delta t) = P_i(t) + (Z(t + \Delta t) - Z(t)) \cdot r_i
\]

\[
S_{ij}(t + \Delta t) = \alpha \cdot \frac{P_i(t + \Delta t) \cdot P_j(t + \Delta t)}{d_{ij}^2}
\]

那么:

通过该模型求解得到的居民流量总人次数没有将消费支出增长带来的出行强度的变化考虑进去,因此有必要对居民流量总人次数做出行强度的变化比例修正。

即:

\[
S_{ij}(t + \Delta t) = S_{ij}(t) \cdot \frac{I(t + \Delta t)}{I(t)}
\]

不考虑居民平均出行强度变化的出行总量差分预测模型如下:

\[
\begin{cases}
M(t + \Delta t) = \sum_i \sum_j S_{ij}(t + \Delta t) = \sum_i \sum_j \alpha \cdot \frac{P_i(t + \Delta t) \cdot P_j(t + \Delta t)}{d_{ij}^2} \\
P_i(t + \Delta t) = P_i(t) + (Z(t + \Delta t) - Z(t)) \cdot r_i \\
\frac{P_i(t)}{\sum_i P_i(t)} = \frac{S_{ii}(t)}{\sum_i S_{ii}(t)} = r_i \quad (i = 1, 2, \ldots, N)
\end{cases}
\]

则考虑居民平均出行强度变化的出行总量差分预测修正模型如下:

\begin{equation}
\left\{
\begin{aligned}
M(t+\Delta t) &= \sum_{i}\sum_{j}S_{ij}(t+\Delta t) = \sum_{i}\sum_{j}\alpha\cdot\frac{P_{i}(t+\Delta t)\cdot P_{j}(t+\Delta t)}{d_{ij}^{2}}\cdot\frac{I(t+\Delta t)}{I(t)} \\
P_{i}(t+\Delta t) &= P_{i}(t) + (Z(t+\Delta t) - Z(t))\cdot r_{i} \\
\frac{P_{i}(t)}{\sum_{i}P_{i}(t)} &= \frac{S_{ii}(t)}{\sum_{i}S_{ii}(t)} = r_{i} \quad (i=1,2...N) \\
I(t+\Delta t) - I(t) &= k\cdot(outgo(t+\Delta t) - outgo(t))
\end{aligned}
\right.
\end{equation}

其中:$I(t)$ 为 $t$ 时刻的居民出行强度预测值(和居民实际消费支出成线形关系)。$Z(t)$ 为 $t$ 时刻的城市居民总量预测值。

$Z(t)$ 根据单位时间居民增长的数据,在每个单位时间递推的过程中(不断有新的外来人口进入该市),不断的更新每个小区的总人数向量和整个城市的 $OD$ 矩阵,可以得到随时间变化的居民总出行次数。

\begin{figure}[h]
\centering
\includegraphics[width=\textwidth]{image1.png}
\caption{不考虑居民出行强度变化的出行总量预测图}
\end{figure}

将实际消费支出的增长对居民出行强度的影响考虑进去,预测后两年以内出行强度总量。

\begin{figure}[h]
\centering
\includegraphics[width=\textwidth]{image2.png}
\caption{考虑居民出行强度变化的出行总量预测图}
\end{figure}

\subsubsection{出租车人口预测模型}

由于第一类人口乘坐出租车的机制不同与第二类人口乘坐出租车的机制,所以将出租车人口 $m$ 分为第一类人口中的乘坐出租车人口 $m_1$ 和第二类人口中的乘坐出租车人口 $m_2$,分别加以考察。那么,$\frac{dm}{dt} = \frac{dm_1}{dt} + \frac{dm_2}{dt}$。出租车人口的变化是第一类人口中的乘坐出租车人口 $m_1$ 的变化和第二类人口中乘的出租车人口 $m_2$ 的变化之和。由此来预测出租车总人口的变化。

\paragraph{居民人口中的乘坐出租车人口计算}

假设居民和暂住乘坐人口乘坐出租车的概率相等。

假定居民总体消费水平只影响出行方式为乘坐出租车人口比例。

定义 $X(t)$ 为 $t$ 时刻第二类人口出租车出行方式占所有出行方式的比例。

则:$outgo = KX$,那么 $\Delta outgo = K\Delta X$。

\begin{equation}
outgo = KX, \quad \Delta outgo = K\Delta X.
\tag{1}
\end{equation}

其中,$outgo$ 为消费支出量比例,$K$ 为消费支出量对以出租车为出行方式的人口比例的影响系数。

则居民选择出租车为出行方式人口总量如下:$C_1(t) = N_1(t) \cdot X(t)$

则

\begin{equation}
\frac{d(C_1(t))}{dt} = \frac{d(N_1(t) \cdot X(t))}{dt} = N_1(t) \cdot \frac{dX(t)}{dt} + X(t) \cdot \frac{dN_1(t)}{dt}
\end{equation}

其中 $N_1(t)$ 为城市居民 $t$ 时刻的总量。

联立公式 (1) 并引入差分结构:

\begin{equation}
C_1(t + \Delta t) = C_1(t) + \frac{1}{K} \cdot N_1(t) \cdot (outgo(t + \Delta t) - 2 \cdot outgo(t)) + \frac{1}{K} \cdot N_1(t + \Delta t) \cdot outgo(t)
\end{equation}

则第一类人口中乘坐出租车的人口和居民消费支出,居民总量有关。

\paragraph{二类人口中的乘坐出租车人口计算}

$C_2 = f(N_2)$

通过调查国内不同城市的第二类人口总量 $N_2$ 和二类人口中的乘坐出租车人口 $C_2$

的关系,可以认为呈正比例关系,即 \( C_2 = f(N_2) = q \cdot N_2 \)。比例常数 \( q \) 可以使用本题所给数据求出:

第二类人口日客运量 = 出租车日客运量 - 居民出租车出行次数

第二类人口 = 流动人口 + 短期及当日进出人口

\[
q = \frac{\text{出租车}}{\text{第二类人口总量}} = \frac{22.5}{77} = 0.2922
\]

只要定出第二类人口总量和时间的关系,就可以确定二类人口中的出租车人口预测值。

第二类人口总量预测的 Logistic 模型

\[
\begin{cases}
\frac{dN_2(t)}{dt} = (a - bN_2(t)) \cdot N_2(t) \\
N_2(t_0) = N_2(0)
\end{cases}
\]

\[
p(t) = \frac{a \cdot N_2(0) \cdot e^{a(t-t_0)}}{a - b \cdot N_2(0) + bN_2(0) \cdot e^{a(t-t_0)}}
\]

将第二类人口总体规划规模代入模型解得:\( a = 0.1765, b = 0.0012 \)

\begin{figure}[h]
    \centering
    \includegraphics[width=\textwidth]{image.png}
    \caption{第二类人口预测图}
\end{figure}

则出租车人口为一二类出租车人口总和:

\[
C = C_1 + C_2
\]

\[
C(t + \Delta t) = q \cdot N_2(t) + C(t) +
\]

\[
\frac{1}{K} \cdot N_1(t) \cdot (outgo(t + \Delta t) - 2 \cdot outgo(t)) + \frac{1}{K} \cdot N_1(t + \Delta t) \cdot outgo(t)
\]

初始值 \( C(0) = \) 全市人口出行人数 \(\times\) 出行方式中出租车方式比例 \(= 73177 \)(人)。因为

需要预测的是坐出租车人口而不是人次数,所以不能用全市人口出行人次数计算。而出行人次数的预测和人口的预测只是初始值不同。

\begin{figure}[h]
    \centering
    \includegraphics[width=\textwidth]{image1.png}
    \caption{日乘坐出租车人口预测图}
\end{figure}

\begin{figure}[h]
    \centering
    \includegraphics[width=\textwidth]{image2.png}
    \caption{日乘坐出租车人次预测图}
\end{figure}

\subsection{问题 2:出租车最佳数量预测}

\subsubsection{问题分析}

出租车和出租车乘客是典型的服务和被服务关系。对于一个典型的城市来说,如果出租车数量过多(出租车资源供大于求),显然能够满足乘客的乘车需要,但较高的空驶率将带来出租车运营成本的提高。如果出租车数量较小,大量顾客得不到服务,出租车的服务作用不能正常发挥。

我们可以使用出租车空驶率和乘客平均等待时间来形象表征上述这种矛盾关系。出租车最佳数量预测的问题既是预测问题又是一个优化问题,如何将空驶率和平均等待时间两个相互制约的指标紧密联系在一起是本模型设计的关键所在。

问题 2 的解决思路为:

(1) 建立描述出租车数量、乘客平均等待时间和出租车空驶率三者关系的系统模型,由于出租车服务系统(包括全部出租车和全部乘客的系统)是一个典型的服务系统,因此可以考虑使用排队论建立模型。

(2) 在服务系统模型的基础上,综合考虑乘客平均等待时间和出租车空驶率两个因素,求取出租车数量最优解。

\subsubsection{符号约定}

\begin{itemize}
    \item $f(t)$ ——每辆车在时间 $t$ 服务完顾客批次数的概率密度函数,服从负指数分布
    \item $u$ ——单位时间服务完的人批次,为 $f(\Delta t)$ 分布的均值,$\frac{1}{u}$ 为平均人次服务时间
    \item $g(t)$ ——等车顾客到达间隔概率密度函数,服从负指数分布,
    \item $\lambda$ ——单位时间内来客人数的期望值,为 $g(t)$ 分布的均值的倒数,$\frac{1}{\lambda}$ 为平均来客时间
    \item $\lambda_1$ ——居民总来客速率
    \item $\lambda_2$ ——流动人口总来客速率
    \item $N$ ——现时刻总出租车数
    \item $n$ ——现时刻正在服务的车辆数
    \item $N-n$ ——表示现时刻空车辆数
    \item $m$ ——现时刻等待出租车的人批次
    \item $\overline{T}(t)$ ——顾客乘出租车不同时刻的平均等待时间
    \item $(m, n) \to (m+1, n)$ ——等待顾客批次增加一个,服务车数没有变化的状态转移
    \item $(m, n) \to (m, n-1)$ ——等待顾客批次不变,正在服务车辆数少一辆的状态转移
    \item $(m, n) \to (m-1, n+1)$ ——等待顾客批次少一批,正在服务车辆数多一辆的状态转移
    \item $(m, n) \to (m, n)$ ——等待顾客批次不变,服务车数不变的状态转移
    \item $\beta$ ——假定该城市在只有一辆空出租车和一批等待顾客情况下,单位时间内发生出租车找到顾客的概率(服务概率)
    \item $B(t)$ ——顾客总体抱怨度,随时间的函数为:$B(t) = f(\overline{T}(t))$
    \item $N(t)$——$t$时刻出租车数量
    \item $P_t(m, n)$——时刻$t$有$m$个等待顾客,$n$辆空车的概率
    \item $\beta_t$——$N(t)$稳态状态下且在一辆空公交车和一批等待顾客情况下,单位时间内发生出租车找到等待顾客的概率(服务速率)
    \item $F(t)$——$t$时刻总体空载比例
    \item $\lambda_1(t)$——不同年月份预测的居民来客速率
    \item $\lambda_2(t)$——不同年月份预测的流动人口来客速率
    \item $\alpha$——表示政府在顾客因素和出租车公司因素之间的折中,$\alpha=1$表示只考虑顾客抱怨程度的预测,$\alpha=0$表示只考虑出租车公司因素的预测
\end{itemize}

\subsubsection{服务系统模型}

\paragraph{模型假设}
\begin{enumerate}
    \item 流动人口以出租车为出行方式的平均耗时和居民以出租车为出行方式的平均耗时相等。
    \item 出租车服务时期单位时间服务完人数服从负指数分布
    \item 服务系统中,时间单元划分能够足够小,使得在一个时间单元中服务系统中的状态(现时刻正在服务的车辆数和现时刻等待出租车的批次)最多只发生一次变化。
    \item 假设$\beta$与$m$和$N-n$成线性关系。
\end{enumerate}

\subsubsubsection{来客速率$\lambda$}

来客速率$\lambda$是指单位时间内顾客增加的数量,假设$\lambda$服从负指数分布,则:
\[
f(\Delta t) =
\begin{cases}
\lambda \cdot e^{-\lambda \cdot \Delta t} & t > 0 \\
0 & t \leq 0
\end{cases}
\]

根据已知的该城市的出租车每日载客趟次可以得到$\lambda = \frac{\text{每日载客趟次}}{24 \times 60 \times 60}$

居民和流动人口的来客速率不同,整类人口的来客速率和该类人口现时刻总数以及性质有关。假定居民和流动人口总来客速率分别为$\lambda_1$和$\lambda_2$,则$\lambda = \lambda_1 + \lambda_2$,

\[
\frac{\lambda_{1}}{\lambda_{2}} = \frac{\text{居民出租车出行人次}}{\text{出租车日客运总量-居民出租车出行人次}}
\]

\subsubsubsection{服务速率 \(\mu\)}

出租车服务速率 \(\mu\) 是指平均每辆车单位时间服务完的批次数,假设 \(\mu\) 服从负指数分布。

\[
f(\Delta t) =
\begin{cases}
u \cdot e^{-u \cdot \Delta t} & \Delta t > 0 \\
0 & \Delta t \leq 0
\end{cases}
\]

\[
\frac{1}{\mu} \text{ 为每辆车服务完成一个批次使用的平均时间。}
\]

\[
\frac{1}{\mu} = \frac{\text{载客里程数}}{\text{每日载客趟次} \times \text{出租车平均速度}}
\]

\subsubsubsection{单车对单人服务速率 \(\beta\)}

\(\beta\) 可以理解为在某一时刻只有一辆空车和一个等待顾客时,该车遇到该顾客所需要的时间的倒数。假设 \(\beta\) 服从负指数分布。 \(\beta\) 参数与城市的道路设施状况和城市的规模相关,如果城市设施良好,则 \(\beta\) 变大;城市规模越大,\(\beta\) 越小。

\(\beta\) 参数由 2004 年该城市的规模和道路情况确定。我们做出一些假设,以便可以通过计算机模拟的方法得到 \(\beta\) 的具体数据。

1. 道路呈均匀田字型网格分布。网格长度为每分钟出租车行驶的距离。不考虑一切路障(包括红绿灯)。
2. 人在田字型网格上固定的一点。
3. 单个出租车在网格上随机游走,计算当人碰到车时的平均耗时(设定不同初始位置,反复计算)。

通过上述模拟机制的运算,可以得到该城市的

\[
\beta = \frac{1}{\text{平均等待时间 (s)}} = 0.000001908 \, (1/s) \, .
\]

\subsubsubsection{状态及状态转移}

设 \(N\) 为现时刻总出租车数,\(n\) 为现时刻正在服务的车辆数,则 \(N-n\) 表示现时刻空车数。

辆数,m 为现时刻等待出租车的批次(某一时刻同时坐上一辆出租车的人为一个批次)。$\overline{T}(t)$ 为顾客乘出租车平均等待时间。

将 n 和 m 组成的一个状态来刻画出租车总服务量和等车顾客总量的系统状态,在下一个时间单位中,状态序列可有如下几种转移:

(1) $(m, n) \rightarrow (m+1, n)$ 等待顾客批次增加一个,服务车数没有变化。这种状态表示在单位时间内,新增一个批次等出租车顾客,等车顾客中没有等到车,乘出租车顾客也没有下车。转移速率为来客速率 $\lambda$。

(2) $(m, n) \rightarrow (m, n-1)$ 等待顾客批次不变,正在服务车辆数少一辆。这种状态表示在单位时间内,即没有新增等车顾客,也没有等到车的顾客,乘车顾客中有一个到达目的地。转移速率为单位时间服务完的批次 $n \cdot u$。

(3) $(m, n) \rightarrow (m-1, n+1)$ 等待顾客批次少一批,正在服务车辆数多一辆。这种状态表示在单位时间内,没有新增等车顾客,也没有下车的顾客,等车顾客中有一个到等到车了。那么由于 $\beta$ 的值很小,根据假设下发生 $(m, n) \rightarrow (m-1, n+1)$ 的转移速率可以近似为:$(N-n) \cdot m \cdot \beta$。在此状态下 $m$ 个顾客总体等待时间为:$m \cdot \frac{1}{(N-n) \cdot m \cdot \beta} = \frac{1}{(N-n) \cdot \beta}$。

设 $P(m, n)$ 为稳定状态时,系统处于 $(m, n)$ 状态的概率。

则
\[
\frac{\sum_{m} \sum_{n} \left( P(m, n) \cdot \frac{1}{(N-n) \cdot \beta} \right)}{\sum_{m} \sum_{n} \left( P(m, n) \cdot m \right)} = \overline{T}(t)。
\]

(4) $(m, n) \rightarrow (m, n)$ 等待顾客批次不变,服务车数不变。这种状态表示在单位时间内,没有新增等车顾客,没有下车的顾客,等车顾客也没有一个等到车了。

服务系统的状态转移图如下图所示:

\begin{center}
\begin{tikzpicture}[node distance=2cm, auto]
    \node (m-1,n+1) at (0,2) {$(m-1,n+1)$};
    \node (m,n+1) at (2,2) {$(m,n+1)$};
    \node (m-1,n) at (0,0) {$(m-1,n)$};
    \node (m,n) at (2,0) {$(m,n)$};
    \node (m+1,n) at (4,0) {$(m+1,n)$};
    \node (m,n-1) at (2,-2) {$(m,n-1)$};
    \node (m+1,n-1) at (4,-2) {$(m+1,n-1)$};

    \path[->] (m-1,n) edge node {$\lambda$} (m,n);
    \path[->] (m,n) edge node {$\lambda$} (m+1,n);
    \path[->] (m-1,n+1) edge node [swap] {$(N-n)\cdot m\cdot\beta$} (m,n);
    \path[->] (m,n+1) edge node {$n\cdot\mu$} (m,n);
    \path[->] (m,n) edge node {$n\cdot\mu$} (m,n-1);
    \path[->] (m+1,n-1) edge node [swap] {$(N-n+1)(m+1)\beta$} (m,n);
\end{tikzpicture}
\end{center}
状态转移图

系统任意时刻正在服务的车辆数 $n$ 的上界为出租车总量 $N$,为了减少问题计算的规模,可以假设在等待顾客总数超过 $N$ 时间,新到的顾客损失(可以认为顾客看到的等待顾客过多而不愿意继续等下去)超过出租车总量 $N$ 的情况认为是小概率时间而不予考虑,则 $(m,n)$ 所决定的系统状态数为 $N \times N$ 个。

根据上图的状态转移图可以得到,系统转移状态方程为:
\begin{align*}
(\lambda + n\mu + (N-n)m\beta) \, P(m,n) - \\
(n+1)\mu P(m,n+1) - \lambda P(m-1,n) - (N-n+1)(m+1)\beta P(m+1,n-1) = 0
\end{align*}

上式的意义表示:稳定状态下,转移到 $(m,n)$ 状态的速率等于从 $(m,n)$ 状态转移出去的速率。

\subsubsubsection{模型建立}

设 $A(N \times N, N \times N)$ 为系统 $N \times N$ 状态到 $N \times N$ 状态的转移速率矩阵,则
\begin{equation}
A(m_1 \times n_1, m_2 \times n_2) =
\begin{cases}
n_1\mu & m_2 = m_1, n_1 - n_2 = 1 \\
\lambda & m_2 - m_1 = 1, n_1 = n_2 \\
(N-n_2)\cdot m_1\cdot\beta & m_2 - m_1 = 1, n_2 - n_1 = 1 \\
-n_2\mu & m_2 = m_1, n_2 - n_1 = 1 \\
-\lambda & m_1 - m_2 = 1, n_1 = n_2 \\
-(N-n_1)\cdot m_2\cdot\beta & m_1 - m_2 = 1, n_1 - n_2 = 1 \\
0 & \text{其他}
\end{cases}
\end{equation}

令 $\overrightarrow{P_{N \times N}}$ 为待求 $N \times N$ 个状态概率,则平稳状态下的服务模型为:

\begin{equation}
\left\{
\begin{aligned}
A(m_1 \times n_1, m_2 \times n_2) \cdot \overrightarrow{P_{N \times N}} &= 0 \\
\sum_{i=1}^{N \times N} P_i &= 1 \\
\overline{T}(t) &= \frac{\sum_m \sum_n (P(m, n) \cdot m)}{\beta \cdot \sum_m \sum_n (P(m, n) \cdot \frac{1}{(N-n)})}
\end{aligned}
\right.
\end{equation}

\subsubsubsection{模型求解}

通过服务系统模型,假设 \(\lambda\),\(\beta\),\(\mu\) 已知,根据出租车数量就可以解得乘客平均等待时间和出租车。

平稳状态方程的总变量数很大,以 2004 年车辆总数为例,共有变量数为 \(6200 \times 6200 = 38440000\),总方程数也为 38440000,状态转移矩阵为 \(38440000 \times 38440000\),通过矩阵运算获取精确解是比较困难的,即使能够解,也将花费大量时间。因此,我们考虑通过数据迭代的方式获取目标方程近似解。

使用数据迭代的方法获取近似解,对状态矩阵的初始设置较为重要,初始值设置不正确,将不能保证迭代收敛。其基本算法如下:

\begin{figure}[h]
    \centering
    \includegraphics[width=\textwidth]{iteration_flowchart.png}
    \caption{迭代算法流程图}
\end{figure}

下图为根据题目已知条件求得 $\lambda$,$\beta$,$\mu$ 后,不同出租车数量情况下空载率和平均等待时间的关系图(实际计算的为多个不同出租车数量情况下离散的点值,图示是通过依次连接各点得到的折线图)。由于迭代过程中实际求取的为近似解,所以曲线显得不是很平滑。

\begin{figure}[h]
    \centering
    \includegraphics[width=\textwidth]{taxi_load_rate.png}
    \caption{出租车数量与空载率的关系}
\end{figure}

\begin{figure}[h]
    \centering
    \includegraphics[width=\textwidth]{image.png}
    \caption{出租车数量与乘客平均等待时间的关系(时间单位:s)}
\end{figure}

特别的,当出租车辆为 6200 辆时,求得空载率为 52.1\%,顾客平均等待时间为 2.02 分钟。

\subsubsection{最优化模型}

\subsubsubsection{模型建立}

顾客总体的抱怨度和顾客的平均等待时间相关,等待时间越大,抱怨越大,等待的时间越小,抱怨越小。定义抱怨函数如下:

\begin{equation}
B(t) = f(\overline{T}(t))
\end{equation}

其中 $\overline{T}(t)$ 为顾客总体平均等待时间,此抱怨度函数可以用分段指数函数来刻画。

\begin{equation}
f(\overline{T}(t)) =
\begin{cases}
e^{a\overline{T}(t)} - 1 & \overline{T}(t) < T_1 \\
1 - e^{-b\overline{T}(t)} & \overline{T}(t) \geq T_1
\end{cases}
\end{equation}

该函数当 $\overline{T}(t) \to 0$ 时,$f(\overline{T}(t)) \to 0$,符合平均等待时间值越小,抱怨度越小的事实;当 $\overline{T}(t) \to \infty$ 时,$f(\overline{T}(t)) \to 1$,符合平均等待时间值越大,抱怨度越大的事实。通过该函数在 $\overline{T}(t) = T_1$ 点的连续性和一阶导数的连续性,可以定出常数 $a$ 和 $b$ 的值。

$T_1$ 的选取可以根据实际情况,通过大量的数据资料分析得到。

出租车公司的利润的本质直接的反映到车辆空载率上,因为车辆空载率越小,单位距离内车辆的收入和车辆本钱的支出之比越小,利润越高。利用排队论模型极为巧妙的计算出各种平均状态下空载比例,从而可以计算得到总体的空载比例。

\begin{equation}
F(t) = \sum_{m} \sum_{n} \left( P_t(m, n) \cdot \frac{N(t) - n}{N(t)} \right)
\end{equation}

其中 \( N(t) \) 为 \( t \) 时刻出租车数量,\( P_t(m, n) \) 为基于出租车数量的各状态概率,\( \beta_t \) 为 \( N(t) \) 稳态状态下且在一辆空车和一批等待顾客情况下,单位时间内发生 \( (m, n) \to (m-1, n+1) \) 的转移速率。

其中 \( F(t) \) 为 \( t \) 时刻总体空载比例,其中 \( N(t) \) 为 \( t \) 时刻出租车数量,\( P_t(m, n) \) 为基于出租车数量的各状态概率。

以最佳出租车数量模型:
\begin{equation}
\min(a \cdot B(t) + b \cdot F(t))
\end{equation}

以最佳出租车数量预测模型如下:
\begin{equation}
\left\{
\begin{aligned}
& \text{对于不同年份的 } \lambda_1(t), \lambda_2(t) \text{ 使得满足 } \min(a \cdot B(t) + b \cdot F(t)) \text{ 的 } N(t) \text{ 为最优} \\
& \text{其中 } F(t) = \sum_{m} \sum_{n} \left( P_t(m, n) \cdot \frac{N(t) - n}{N(t)} \right), \, B(t) = f(\overline{T}(t)) \\
& \text{其中 } a, b \text{ 为政府调控因子} \\
& A(m_1 \times n_1, m_2 \times n_2) \cdot \overrightarrow{P_{N \times N}} = 0 \\
& \sum_{i=1}^{N \times N} P_i = 1 \\
& \overline{T}(t) = \frac{\sum_{m} \sum_{n} (P(m, n) \cdot m)}{\beta \cdot \sum_{m} \sum_{n} \left( P(m, n) \cdot \frac{1}{(N - n)} \right)} \\
& A(m_1 \times n_1, m_2 \times n_2) =
\begin{cases}
n_1 u & m_2 = m_1, n_1 - n_2 = 1 \\
\lambda & m_2 - m_1 = 1, n_1 = n_2 \\
(N - n_2) \cdot m_1 \cdot \beta & m_2 - m_1 = 1, n_2 - n_1 = 1 \\
-n_2 u & m_2 = m_1, n_2 - n_1 = 1 \\
-\lambda & m_1 - m_2 = 1, n_1 = n_2 \\
-(N - n_1) \cdot m_2 \cdot \beta & m_1 - m_2 = 1, n_1 - n_2 = 1 \\
0 & \text{其他}
\end{cases}
\end{aligned}
\right.
\end{equation}

用 \( \alpha \) 来表示政府在顾客因素和出租车公司因素之间的折中,\( \alpha = 1 \) 表示只考虑顾客抱怨程度的预测,\( \alpha = 0 \) 表示只考虑出租车公司因素的预测。

\subsubsubsection{模型求解}

首先根据初始的出租车系统平均服务速率、顾客平均到达速率以及城市道路的基本状况参数 $\beta$,设定初始出租车总量。解出平稳状态时,整个系统处于各个状态的概率。根据概率值计算顾客平均等待时间和出租车空载概率。由此结合设定的政府调控系数,计算需要优化目标的概率。

模型是否稳定的判断条件是 $\max _{m, n}\left|P_{i+1}(m, n)-P_{i}(m, n)\right| \leq \varepsilon$,即前后两步概率矩阵中相差最大的概率值。

最佳出租车数量采用梯度下降算法,设定初始值数量,利用下一年的 $\lambda(t)$ 和迭代算法得到下一年,初始出租车数下的稳态概率,进而得到平均等待时间和空载概率,求得目标函数值。使用梯度下降算法按照目标函数值下降方向不断减少车辆增加数量,求得最佳的出租车数量。

\begin{figure}[h]
    \centering
    \includegraphics[width=\textwidth]{image.png}
    \caption{未来年份的出租车数量预测}
\end{figure}

需要说明的是,预测所得到的各年的最佳出租车数量是基于用户平均等待时间不大于 2 分钟(根据 2004 年的数据实际计算用户平均等待时间为 2 分钟),空载率不高于 $52.1\%$(2004 年的实际计算结果)的情况。

\subsubsubsection{模型的验证}

为了验证模型的合理性,我们采用蒙特卡罗模拟对建立的模型进行模拟。模拟中采用的部分定义如下:

- 模拟时间步进 $\Delta t$
- 当前时间 $t$
- 最大模拟时间 $T_{Max}$

单位时间内需要乘车的顾客批次概率(或来客速率) $p_{\lambda}$;

单位时间服务完的乘客批次概率 $p_{\mu}$;

单位出租车获取某个乘客批次的概率 $p_{\beta}$

算法的描述如下:

(1) 初始化相关参数;

(2) 初始化随机数种子;

(3) 判断 $t < T_{Max}$,如果为真则从(4)向下继续,否则计算统计数据结果,程序运行完成。

(4) 产生随机数 Random,判断是否产生 1 个顾客等待事件,如果产生,则向顾客队列中添加。

(5) 产生随机数 Random,判断是否产生 1 个顾客的乘车事件,如果产生,则改变车辆队列中某辆车的乘坐属性为 false,同时从顾客队列中删除该顾客。

(6) 产生随机数 Random,判断是否产生 1 个顾客的下车事件,如果产生,则改变车辆队列中某辆车的乘坐属性为 True。

(7) 计算出租车空载率和平均等待时间,判断空载率和平均等待时间是否已经趋于稳定,如果稳定开始进行统计记录。跳转到(3)

数据运行结果如下图:

\begin{figure}[h]
    \centering
    \includegraphics[width=\textwidth]{image.png}
    \caption{出租车数量的预测}
\end{figure}

\subsection{问题 3:价格调整方案模型}

\subsubsection{问题分析}

假设在油价调整前后,其它因素都不发生变化,考虑价格调整方案。

题目的要求是在油价发生变化时求解一种价格方案使得市民和出租车司机这两方均满意。这里面我们引入第三方,就是出租车公司。在价格调整之后,显然这三方的某一方或者某几方会发生损失。但是题目并没有提及出租车公司这一方,而只考虑市民和出租车司机这两方的满意程度。那么我们只能这样理解,即在出租车公司满意指数为 100\% 的情况下,考虑使得其它两方均满意的价格函数。理由陈述如下:如果不管出租车公司的利益,那么我们将所有的损失都让出租车公司来承担,也就是说出租车价格方案维持原来不变,同时出租车司机的工资也不变,那么这两方的满意指数都为 100\%,这当然是这两方最理想的价格方案。但是这样的考虑显然是不现实的,出租车公司明显不可能接受价格不变的方案。按经济学的角度,市民是顾客,出租车司机是劳方,而出租车公司是资方,在整个出租车管理和运营的过程中,出租车公司是运作方,其它两方在一定程度上是接受方。因此相对市民和出租车司机这两方来说,出租车公司在价格方案的调整上拥有一定的主动权。因此我们认为题目第 3 小题,是要求我们站在出租车公司的角度来考虑。也就是说油价上升导致的损失只能由市民和出租车司机这两方来承担。当然更合理的价格方案是站在政府部门的角度,同时考虑三方的满意程度。这个我们在模型的扩展中考虑。在这样的分析下,我们首先建立了基于价格函数的考虑市民和出租车司机满意程度的泛函模型。在模型的扩展中,同时考虑三方的满意程度,对前面的泛函模型进行修正。

其次分析市民和出租车司机的满意程度,我们用满意指数来衡量。下面我们来定义这个满意指数。该题由于油价上升导致的损失必须要求市民和出租车司机来承担,哪一方承担的越多,那一方就越不满意。但是这个多少不能用绝对数量来考虑。因为市民是通过多付出租车费用来承担损失的,而出租车司机是通过自己工资的减少来承担损失的。显然这两个承担的损失不是基于同一个标准的。因此我们选用相对承担损失比例来刻画双方的满意度。我们以每天每辆车为单位来考虑,对市民来说,价格调整之前每趟次出行所付出租车费用为 $M_0$,调整之后所付出租车费用为 $M_1$($M_1 > M_0$),那么他的承担损失比例为 $\frac{M_1 - M_0}{M_0}$,显然,这个比例越小,市民越满意。对出租车司机来说,价

格调整之前他的每天工资为 $N_0$,调整之后他的工资为 $N_1$,($N_1 < N_0$)那么他的承担损失比例为 $\frac{N_0 - N_1}{N_0}$,同理,这个比例越小,出租车司机也越满意。我们就认为这个承担损失比例就代表了满意指数,它刻画了双方的满意程度。这里我们将一辆车的正、副驾驶员考虑为一个出租车司机整体。

如何理解使得双方都满意这个概念。可以得到必然存在这样的约束:如果增加一方的满意指数,那么另一方的满意指数肯定会降低。双方都满意的概念是,相互之间找到一个损失额,让他们都能够接受并认可。基于这样的分析,我们直接选择一种损失分配,让所有各方的满意指数尽可能的接近。

再次我们分析一下题目所给的条件和数据。主要分析附录中第 2 点包含的 4 个小点的数据。第(1)小点每辆车每年行驶里程为 124640 公里,那么可以得到平均每车每日行驶为 $124640 / 365 = 341.48$ 公里。第(3)小点又给出所有出租车日行驶总里程为 230.7 万公里。平均到每辆车每日行驶 $230.7 / 0.62 = 372.10$ 公里。显然这两个数据是不吻合的,我们认为是统计误差造成的。我们选择第(1)点中的数据。第(3)点中还给出日平均营业里程为 424.00 公里/车日。这个数值大于前面两个,可以这样理解:每天并不是所有的车都在运营。有些车或者维修,或者司机的问题等等,没有参加运营。而这个数据是基于参加运营的出租车统计的,所以相对偏大。

\subsubsection{符号约定}

\begin{itemize}
    \item $x$ —— 出租撤每此载客里程数
    \item $f(x)$ —— 出租车价格函数,自变量为里程数
    \item $f_0(x)$ —— 初始的出租车价格函数
    \item $f_{0d}(x)$ —— 初始的白天出租车价格函数
    \item $f_{0n}(x)$ —— 初始的晚上出租车价格函数
    \item $f_1(x)$ —— 调整之后出租车价格函数
    \item $f_{1d}(x)$ —— 调整之后白天出租车价格函数
    \item $f_{1n}(x)$ —— 调整之后晚上出租车价格函数
    \item $M(f)$ —— 出租车每次载客出行,市民的平均消费,它的自变量为价格函数 $f$
    \item $M_0$ —— 初始价格函数下出租车每次载客出行,市民的平均消费
    \item $M_1$ —— 调整价格函数之后出租车每次载客出行,市民的平均消费
    \item $m$ —— 市民的满意指数
    \item $N_0$ —— 价格调整之前出租车司机的每天工资
    \item $N_1$ —— 价格调整之后出租车司机的每天工资
    \item $n$ —— 出租车司机的满意指数
    \item $P(x)$ —— 每个市民乘出租车出行里程数小于 $x$ 的概率
    \item $p(x)$ —— $P(x)$ 的概率密度函数
    \item $\sigma_d$ —— 白天出行所占比例
    \item $\sigma_n$ —— 夜晚出行所占比例
    \item $I_0$ —— 价格调整前出租车公司每辆车每天的收益
    \item $I_1$ —— 价格调整之后每辆车每天的收益
    \item $i$ —— 出租车公司的满意指数
\end{itemize}

\subsubsection{基于价格函数的泛函模型}

$m$ 为市民的满意指数,根据前面的分析,
\[ m = 1 - \frac{M_1 - M_0}{M_0} = 2 - \frac{M_1}{M_0}, \quad M_1 \geq M_0, \]
\[ 0 \leq m \leq 1. \]

$n$ 为出租车司机的满意指数,根据前面的分析,
\[ n = 1 - \frac{N_0 - N_1}{N_0} = \frac{N_1}{N_0}, \quad N_1 \leq N_0, \]
\[ 0 \leq n \leq 1. \]

$f(x)$ 为出租车的计价函数。简称价格函数,其中 $x$ 为里程数。一个价格函数就是一种价格方案,当前的价格函数设为 $f_0(x)$,它的具体描述见附录 1 中某城市出租车的收费标准:

\begin{enumerate}
    \item[(1)] 起租基价 3 公里,基价租费:白天 8.00 元,晚上 9.6 元。
    \item[(2)] 超过起租基价公里,每车公里价:白天 1.8 元,晚上 2.16 元。
    \item[(3)] 上日 21 时至次日凌晨 5 时为夜间行车时间。
    \item[(4)] 远程载客从 10 公里开始,计价器将 50\% 回空费输入表内,加收回空费。
    \item[(5)] 行驶中乘客要求临时停车 10 分钟内免费,后每超过 5 分钟按 1 车公里租价收取等候费。
\end{enumerate}

白天:
\[
f_{0d}(x) =
\begin{cases}
8 & 0 < x \leq 3 \\
8 + 1.8 \times (x - 3) & 3 < x \leq 10 \\
8 + 1.8 \times (x - 3) + 2.7 \times (x - 10) & x > 10
\end{cases}
\]

晚上:
\[
f_{0n}(x) =
\begin{cases}
9.6 & 0 < x \leq 3 \\
9.6 + 2.16 \times (x - 3) & 3 < x \leq 10 \\
9.6 + 2.16 \times (x - 3) + 3.24 \times (x - 10) & x > 10
\end{cases}
\]

如果不考虑第五个计价因素,基于以上前四个计价因素,我们可以描绘出出租车收费对于行驶里程数的分段直线。如图所示,其中横坐标为行驶公里数,纵坐标为出租车所收费用。

\begin{figure}[h]
    \centering
    \includegraphics[width=0.8\textwidth]{current_price_analysis.png}
    \caption{当前价格分析}
\end{figure}

调整后的价格函数为 $f_1(x)$,代表调整后的一种价格方案。在本模型的建立和求解中,自变量就是出租车的价格函数,因此我们建立基于这个价格函数的泛函模型。

$P(x)$ 为每个市民出租车出行里程数小于 $x$ 的概率,为概率分布函数。

$p(x)$ 为 $P(x)$ 的概率密度函数。在题目第二小题的解答中,建立了一个排队论模型,其中假设出租车的服务时间是服从负指数分布的。那么假设不考虑出租车速度的变化问题,这里的 $p(x)$ 也服从负指数分布。其中平均载客里程数为 210.07 公里/车日,日均载客趟次为 40.52,那么每次载客行驶里程数为 $210.07 / 40.52 = 5.18$ 公里。这个值就是负指数分布 $p(x)$ 的均值。

$M(f)$ 为出租车每次载客出行,市民的平均消费。它和价格函数 $f(x)$、市民出行里程数的概率密度函数 $p(x)$ 有关。

\[
M(f) = \sigma_d \int_0^\infty x \cdot p(x) \cdot f_d(x) dx + \sigma_n \int_0^\infty x \cdot p(x) \cdot f_n(x) dx
\]

由于白天和晚上由于价格函数不同,并且出行总量所占的比例也不同。所以 $M(f)$ 分两部分考虑。其中前半段代表白天部分,后半段代表晚上部分。$\sigma_d$ 为白天出行所占比例,$\sigma_n$ 为夜晚出行所占比例,$\sigma_d + \sigma_n = 1$,简化认为这个比例和价格无关。

\[
M_0 = M(f_0) = \sigma_d \int_0^\infty x \cdot p(x) \cdot f_{0d}(x) dx + \sigma_n \int_0^\infty x \cdot p(x) \cdot f_{0n}(x) dx
\]

为调整价格之前每个市民平均每次出租车出行的消费。

\[
M_1 = M(f_1) = \sigma_d \int_0^\infty x \cdot p(x) \cdot f_{1d}(x) dx + \sigma_n \int_0^\infty x \cdot p(x) \cdot f_{1n}(x) dx
\]

为调整价格之后每个市民平均每次出租车出行的消费。

那么市民的满意指数可以表示为

\[
m = 1 - \frac{M_1 - M_0}{M_0} = 2 - \frac{\sigma_d \int_0^\infty x \cdot p(x) \cdot f_{1d}(x) dx + \sigma_n \int_0^\infty x \cdot p(x) \cdot f_{1n}(x) dx}{\sigma_d \int_0^\infty x \cdot p(x) \cdot f_{0d}(x) dx + \sigma_n \int_0^\infty x \cdot p(x) \cdot f_{0n}(x) dx}
\]

该凉车每趟次市民承担的损失额为 $M(f_1) - M(f_0)$。

下面定义出租车司机的满意指数,

\[
n = 1 - \frac{N_0 - N_1}{N_0} = \frac{N_1}{N_0}
\]

每辆车的出租车司机所承担的损失额为 $N_0 - N_1$,其中两方共同承担的损失为油价上升导致的损失。

\[
N_0 - N_1 + K \left( \sigma_d \cdot \int_0^\infty x \cdot p(x) \cdot \big(f_{1d}(x) - f_{0d}(x)\big) dx + \sigma_n \cdot \int_0^\infty x \cdot p(x) \cdot \big(f_{1n}(x) - f_{0n}(x)\big) dx \right) = L \cdot (U_1 - U_0)
\]

这个式子基于每天每辆车得出的。其中 $K$ 为每辆车日均载客趟次,$L$ 为平均每辆车每日耗油数,$U_1$ 为上升之后的油价,$U_0$ 为初始油价。

那么建立的泛函模型为
\[
\left\{
\begin{aligned}
& \min_{f} (m-n)^2 \\
& m = 2 - \frac{\sigma_d \int_0^\infty x \cdot p(x) \cdot f_{1d}(x) dx + \sigma_n \int_0^\infty x \cdot p(x) \cdot f_{1n}(x) dx}{\sigma_d \int_0^\infty x \cdot p(x) \cdot f_{0d}(x) dx + \sigma_n \int_0^\infty x \cdot p(x) \cdot f_{0n}(x) dx} \\
& n = \frac{N_1}{N_0}, (0 < n < 1) \\
& N_0 - N_1 + K \left( \sigma_d \cdot \int_0^\infty x \cdot p(x) \cdot \big(f_{1d}(x) - f_{0d}(x)\big) dx \right. \\
& \left. + \sigma_n \cdot \int_0^\infty x \cdot p(x) \cdot \big(f_{1n}(x) - f_{0n}(x)\big) dx \right) = L \cdot (U_1 - U_0)
\end{aligned}
\right.
\]
$\min_{f}(m=n)$ 表示选择一种价格方案,使得 $m=n$。

其中 $L$ 为平均每辆车每日耗油数,$U_1$ 为上升之后的油价,$U_0$ 为初始油价。

\subsubsection{模型求解}

为方便求解,首先分析对价格函数的简化处理。所以就要考虑到计费的实际可操作性和市民对该计费规则的接受程度,参照现有的价格方案。假设新方案是:

① 起租基价 3 公里,基价租费:白天 $A$ 元,晚上 $B$ 元,其中 $A > 8$,$B > 9.6$。

② 超过起租基价公里,每车公里价:白天 $a$ 元,晚上 $b$ 元。其中 $a > 1.8$,$b > 2.16$。

③ 上日 21 时至次日凌晨 5 时为夜间行车时间。

④ 远程载客从 10 公里开始,计价器将 50% 回空费输入表内,加收回空费。

⑤ 行驶中乘客要求临时停车 10 分钟内免费,后每超过 5 分钟按 1 车公里租价收取等候费。当然在实际的计费过程中,作为简化,仍然不考虑第 5 个计费因素。

那么调整之后的价格函数表达式为

白天:
\[
f_{1d}(x) =
\begin{cases}
A & 0 < x \leq 3 \\
A + a \times (x - 3) & 3 < x \leq 10 \\
A + a \times (x - 3) + 1.5a \times (x - 10) & x > 10
\end{cases}
\]

晚上:
\[
f_{1n}(x) =
\begin{cases}
B & 0 < x \leq 3 \\
B + b \times (x - 3) & 3 < x \leq 10 \\
B + b \times (x - 3) + 1.5b \times (x - 10) & x > 10
\end{cases}
\]

出租车每次载客出行里程数,根据前面的分析,服从负指数分布,其均值为 5.18 公

里,得到 \( p(x) = \frac{1}{5.18} \exp \left( -\frac{x}{5.18} \right) \)。

对于白天和晚上出行总量比例的分配。采用简单值:\( \sigma_d = 2/3 \),\( \sigma_n = 1/3 \)。

首先可以求解出满意指数均为 0.67806。

下面就是要求解 \( A \)、\( a \)、\( B \)、\( b \) 的值使得
\[
61.7311 \cdot (a - 1.8) + 5.18 \cdot (A - 8) +
\]
\[
61.7311 \cdot (b - 2.16) + 5.18 \cdot (A - 9.6) = 0.357038
\]

显然这有很多种的 \( (A, a, B, b) \) 方案,使得上面条件得到满足。也就是说,只要满足上面这个表达式的约束条件。使得双方满意指数相等的条件下,会有很多种价格调整方案。

最后模型的解可以表示如下
\[
\begin{cases}
61.7311 \times (a - 1.8) + 5.18 \times (A - 8) + \\
61.7311 \times (b - 2.16) + 5.18 \times (A - 9.6) = 0.357038 \\
\text{其中 } a \geq 1.8, A \geq 8, b \geq 2.16, B \geq 9.6 \\
m = n = 0.67806
\end{cases}
\]

依照上式给出一种调整之后的价格方案:
\[
\begin{cases}
A = 8 \\
B = 9.6 \\
a = 1.8 \\
b = 2.1666
\end{cases}
\]

\subsubsection{模型扩展}

如果站在市公用事业管理部门的角度,出租车公司也应该承担一定的损失,即按照市民、出租车司机和出租车公司这三方共同承担风险的原则,来划分总的损失额。在前面泛函模型的基础上,我们进行修正,得到泛函模型的扩展模型。

与前面模型不同的是,这里还要考虑出租车公司的满意指数。定义 \( I_0 \) 为价格调整前出租车公司每辆车每天的收益,\( I_1 \) 为价格调整之后每辆车每天的收益,\( I_1 < I_0 \),\( I_0 - I_1 \) 就是出租车公司每天在每辆车上损失的利益。那么出租车公司的满意指数定义为
\[
i = 1 - \frac{I_0 - I_1}{I_0} = \frac{I_1}{I_0}
\]

所以得到扩展泛函模型为

\begin{equation}
\left\{
\begin{aligned}
& \min_{f} \left( (m-n)^2 + (m-i)^2 + (n-i)^2 \right) \\
& m = 2 - \frac{\sigma_d \int_{0}^{\infty} x \cdot p(x) \cdot f_{1d}(x) dx + \sigma_n \int_{0}^{\infty} x \cdot p(x) \cdot f_{1n}(x) dx}{\sigma_d \int_{0}^{\infty} x \cdot p(x) \cdot f_{0d}(x) dx + \sigma_n \int_{0}^{\infty} x \cdot p(x) \cdot f_{0n}(x) dx} \\
& n = \frac{N_1}{N_0}, (0 < n < 1) \\
& i = \frac{I_1}{I_0}, (0 < i < 1) \\
& N_0 - N_1 + K \left( \sigma_d \int_{0}^{\infty} x \cdot p(x) \cdot (f_{0d}(x) - f_{1d}(x)) dx + \sigma_n \int_{0}^{\infty} x \cdot p(x) \cdot (f_{0d}(x) - f_{1d}(x)) dx \right) + I_0 - I_1 = L \cdot (U_1 - U_0)
\end{aligned}
\right.
\end{equation}

其它参数的意义如前所述。

\subsubsection{扩展模型求解}

同样也可求出结果:

\begin{equation}
\left\{
\begin{aligned}
& 61.7311 \times (a - 1.8) + 5.18 \times (A - 8) \\
& + 61.7311 \times (b - 2.16) + 5.18 \times (B - 9.6) = 0.351903 \\
& \text{其中 } a \geq 1.8, A \geq 8, b \geq 2.16, B \geq 9.6 \\
& m = n = i = 0.71839
\end{aligned}
\right.
\end{equation}

同样,给出一种调整之后的价格方案:

\begin{equation}
\left\{
\begin{aligned}
& A = 8 \\
& B = 9.6 \\
& a = 1.8 \\
& b = 2.1657
\end{aligned}
\right.
\end{equation}

需要补充的是,这里求出的最优解是基于白天和晚上出行总量为 2:1 这个前提下得到的。

\subsection{问题 4:数据采集的合理问题}

本题主要目的是根据所收集统计来的数据,对当前城市的出租车进行管理和规划。

当然数据的来源和收集的角度是多方面的。数据收集的越多,数据本身反映出来的统计特性也就越完备,对城市出租车的管理规划也就会越有帮助。当然,不同角度收集的数据在所难免会存在不能完全一致的地方,因为统计的口径不同,这是可以理解的。

下面我们从本题求解的角度出发,如果能够有哪些方面的统计数据,对本题的模型和求解更有帮助,对城市出租车的管理规划更有指导意义,我们认为这些数据是需要收集的。

集的。当然这些数据的实际可行性还有待进一步分析。

1、如果能够将整体的贫富层次统计出来更好,收入只有人均的,还不够详细。因为居民按出行坐出租车的比例可以分为三个等级:较富、一般和较贫。较富的居民出行一般选取出租车方式,不考虑步行、自行车、摩托车和公交车出行,即高比例出租车出行居民。一般的居民看实际情况,在多种出行方式中选取,即一般比例出租车出行居民。较贫的居民出行一般不考虑出租车方式,即低出租车出行居民。当然一个地区大概的贫富等级和比例还是可以统计得到的。并且建立档案,将每年的统计结果收集起来,分析这个比例随着经济的发展会发生那些变化。这些变化继而会对城市的出租车管理有哪些影响。

2、新增人口在中心区域和边缘区域分布的比例相对比较重要。因为人口的增长在整个城市不是均匀分布的,它是有区域性的,比如在城市的发展中,中心区的人口发展到一定数量之后,就会达到饱和,不可能继续增长。人口的增长相对偏重于边缘区域。因此如果能够统计得到具体的人口增长分布趋势,对整个城市交通规划有着重要的意义。如果可能的话,该城市各个区的人口分布比例,以及这个比例将来发展的趋势,各个区的发展趋势,是商业区发展模式还是居住区发展模式,或者半商业半居住发展模式。对今后的各方面预测有着重要的作用。

3、城市的交通规划还应该将它和资源利用环境优化结合起来。显然题目所给的数据没有一点有关这两个方面的。这可以说是本题数据不完备或者说不合理的最主要方面。因为城市的交通规划仅仅是城市建设的一个方面,而这个方面和城建的其他方面有着相互制约,相互影响,协调发展的作用和问题。因此,收集出租车在这两个方面的数据,并将这些数据和其他交通方式在这两个方面的数据进行比较,会得到一些更有意义的结果。对城市交通规划有着重要的指导作用。

4、本题在数据收集上还有一个不合理的地方,所有的数据,除了人口的预测数据给出了在时间上的趋势,其它数据在时间的变化趋势都表现得不够充分,这给本题的预测带来的很大的不变,同样使得本题在预测上数据支持和预测数据合理性方面有一定的欠缺。我们认为,将所有的统计数据建立档案,使得它们在时间上的变化趋势体现出来。为预测和其他工作提供更多的支持和帮助。

\subsection{问题 5:出租车规划短文}

\subsubsection{城市出租车规划管理}

我们要分析城市对出租车规划的意义,不能仅仅局限于出租车本身。因为对整个城市交通规划来说,出租车的规划只是其中的一个部分,它和其它交通方式的规划相互影响,相互制约,构成了整个城市交通规划的互动过程。

其次讨论一下城市交通规划对城市各方面建设的意义。特别在环境污染、土地占用、生态破坏和能源消耗等问题上,城市交通规划的优劣会产生截然不同的效果。城市交通规划是城市确定今后交通系统建设与发展方向的关键决策。它必须制定符合可持续发展的战略与管理办法,以引导交通结构向低环境污染和优化利用不可再生资源的合理模式转移,这是城市交通规划正确的发展方向,它决定了今后城市交通规划的措施和方向。

从交通规划的方法来看,目前我们普遍采用的交通规划技术是以交通现状调查与规划模型预测技术为基础。这种方法和技术存在诸多局限:一是没有从资源、能源、环境的角度出发,全面考虑城市交通系统的能源消耗、土地资源占用以及污染物排放等问题;二是没有将实现城市交通系统社会、经济和环境效益最大化作为规划的首要任务,因而难以全面考虑影响城市发展潜力的各种因素,制订出符合可持续发展理念的交通规划。

具体从出租车行业角度出发,我们总结出以下几个方面的规划措施。

1、改善经营环境,规范市场准入,提高服务质量;严格控制出租汽车总量,保持出租车市场供求平衡。具体地说,完善出租车的行业管理和市场准入、退出机制。由于出租车具有公共服务、占有道路资源等特点,应当坚持总量控制原则。其总量增减由市政府根据市场供求、出租车与其他公共交通方式的配比、行业收入和利润等因素,在保障供略大于求的情况下,适时调整出租车总量规模,制定出租车总量调控办法,科学地建立总量调控机制,实施总量控制。

2、严厉打击非法营运,净化市场环境;清理行业收费,规范收费项目和标准,保障经营者和从业人员的合法权益;联合各职能部门建立集中统一的服务窗口,提供一站式服务;提请物价部门调整出租车票价和计价方式;建设一批营业站场,与兄弟城市协调解决长途出租车异地营运问题。

3、逐步实施从业人员资格认证制度,清理一批不合格从业人员;开展培训,提高从业人员的整体素质。在本市出租车经营成本不断上涨的情况下,企业和驾驶员收入水平将会较低。因此,应建立起出租车油价、运价联动的长效机制,实现企业、驾驶员、乘客共同分担市场风险的原则。同时根据社会经济发展状况建立合理的租价调整机制。

4、建设 GPS 全球定位系统和智能调度系统。建立智能调度指挥系统。建立司机运营行为档案,利用科技手段,建立普及智能调度指挥系统,提高出租车有效调度的比例,降低空驶率。

5、发挥行业协会作用是发展市场经济的内在要求。在目前体制下,行业协会应当承担维护司机合法权益、教育司机提高自律能力的责任。同时,应当参与行业规划、行业政策以及相关法律法规制定和决策论证,并协调政府、企业、司机和乘客的利益关系,维护行业内公平竞争。出租车市场要引入竞争机制,淘汰管理差、运营差、服务差的企业。

从资源利用和环境污染的角度,我们来讨论出租车整体规划的方向和原则。现在国内外普遍采用的原则是公交车优先的原则。这对于城市的可持续性发展,交通方式的优化,资源的更高利用,环境污染的优化等方面都是有重要意义的。在这个原则下具体考虑出租车的规划管理问题,使它符合城市整体交通规划的利益,对于城市的战略发展有着重大意义。

总的来说,建立和谐交通的概念,具体原则是:实事求是,因地制宜;势力均衡,协调发展;规则第一,公德至上。

\section{参考文献}

[1] 高自友,任华玲. 城市动态交通流分配模型与算法[M]. 北京:人民交通出版社,2005

[2] 王炜. 城市交通管理规划指南[M]. 北京:人民交通出版社,2003

[3] 苏兆龙. 排队论基础[M]. 成都:成都电子科技大学出版社,1998

[4] 张志涌. 精通 MATLAB6.5 版[M]. 北京:北京航空航天大学出版社,2005

[5] 许天周. 应用泛函分析[M]. 北京:科学出版社,2002

[6] 洪维恩. 数学运算大师 Mathematica4[M]. 北京:人民邮电出版社,2002

\section{附录}

\subsection{附录 I}

资料一:

\section{我国城镇居民消费结构分析}

交通通讯支出持续增长,支出比重变化最大

调查显示,2001 我国城镇居民交通通讯消费的比重为 8.61\%,比 1993 年上升了 4.79

个百分点;其中,2001年城镇居民交通消费的比重为3.31\%,比1993年上升了0.82个百分点;通讯消费的比重为5.30\%,比1993年上升了4.0个百分点。(来自北京统计信息网)

资料二:

浙江城镇居民基本消费结构的变化

\begin{tabular}{l l l l l l l l}
年份 & 基本消费 & 食品 & 衣着 & 设备用品 & 医疗保健 & 交通通讯 & 娱乐教育文化 \\
居住 & 杂项 & & & & & & \\
\hline
1995 & 100.00 & 54.72 & 10.96 & 7.82 & 4.10 & 4.55 & \\
7.07 & 7.11 & 3.67 & & & & & \\
1998 & 100.00 & 53.35 & 7.85 & 3.91 & 4.89 & 5.96 & \\
11.41 & 10.27 & 2.36 & & & & & \\
2001 & 100.00 & 45.54 & 7.08 & 4.79 & 7.00 & 7.10 & \\
13.73 & 12.03 & 2.73 & & & & & \\
\end{tabular}

分析表明,浙江城镇居民生活消费中各项商品(服务)所占比重,近年也发生了明显的变化(见表5)。1995-2001年,尽管食品始终在城镇居民消费结构中占最大份额,但其后的排序已由衣着和设备用品转变成了娱乐教育文化和居住。同期,食品、衣着和设备用品在消费结构中的比重,分别下降了9.18、3.88和3.03个百分点;而娱乐教育文化、居住、医疗保健和交通通讯的比重,则分别上升了6.66、4.92、2.90和2.55个百分点。

资料三:

\begin{figure}[h]
    \centering
    \includegraphics[width=\textwidth]{image1.png}
    \caption{浙江某城市人均统计数据}
\end{figure}

\begin{figure}[h]
    \centering
    \includegraphics[width=\textwidth]{image2.png}
    \caption{出行强度变化图}
\end{figure}

\subsection{附录 2:}

\subsubsection{程序清单:}

(1) 特卡罗模拟程序 \textbackslash moni\textbackslash project1.exe

(2) 排队论实现程序 \textbackslash my\_pro\textbackslash project1.ext

(3) 相关绘图及调试程序 \textbackslash sq.nb(Mathamatic 程序)

(4) matlab 相关预测程序 \textbackslash matlab\textbackslash *.\textbackslash m

\end{document}