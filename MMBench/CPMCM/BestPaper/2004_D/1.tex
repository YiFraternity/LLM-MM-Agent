\title{研究生录取问题的双向选择策略\footnote{本文获2004年首届全国部分高校研究生数学建模竞赛一等奖,并被评为优秀论文}}

\author{操保华 陈焰周 郭兰英 \\ 指导教师:高成修 \\ (武汉大学,湖北 武汉 430072)}

\maketitle

\begin{abstract}
本文根据问题背景和题目要求研究了在各种不同条件下的研究生录取问题。在对笔试、面试成绩以及导师信息进行量化处理基础上设计了对应的研究生录取方案,通过构造选择矩阵和满意度矩阵建立了双向选择策略的 0-1 规划模型,借鉴“八皇后”算法思想,采用回溯法编程求解出了最优解,得到各问题的最优方案;同时采用降阶技巧和创建的定理,快速地求解出实用的较优解,得到对应较优方案。希望本文提出的解决方案对高等教育部门在高校研究生录取工作中起到一定参考作用。
\end{abstract}

\textbf{关键词:} 研究生录取;双向选择策略;0-1 规划模型

\section{问题复述(详见首届全国部分高校研究生数模竞赛 D 题)}

\section{基本假设}

为简化问题,根据实际情况,提出如下假设:

(1) 假设专家对学生的评分公正、合理,且每位专家对学生的成绩评价都是同等重要,并且每个方面的评价也同等重要;

(2) 假设专家对学生专长的复试评分的五个标准是相互独立的,导师的学术水平指标的四项也是相互独立的,并且对总评分的贡献均等;

(3) 如果不特殊指明,研究生初试成绩比重 $\alpha=0.7$,导师对学生客观成绩的偏重系数 $\mu=0.7$,学生对导师学术水平的偏重系数 $\beta=0.7$,学生对导师平均满意度的偏重系数 $w=0.5$。

\section{符号说明(见文中)}

\section{问题分析与数据处理}

\subsection{对笔试成绩与复试成绩的量化处理}

学校在录取学生时,一般首先考虑学生的综合成绩,包括笔试成绩和复试成绩。量化处理综合成绩以 100 制。其中笔试成绩对综合成绩的贡献率为 $\alpha$,面试成绩的贡献率为 $(1-\alpha)$。笔试成绩一般以 500 分作为满分,那么学生 $S_{i}$ 的笔试成绩 $M_{i}$ 可以量化处理为:

\[
m_{i}=\frac{M_{i}}{500} \times 100 \alpha
\]

面试评价中的 A、B、C、D 四个等级分值量化为 4、3、2、1。一个专家给学生 $S_{i}$ 在五个方面的评分为 $f_{i,k}$ ($k=1, \cdots, 5$),假设 $f_{i,k}$ 对复试成绩的权重为 $\nu_{k}$ ($k=1,2,3,4,5$),实际操作中认为 $\nu_{k}=1$,学生 $S_{i}$ 的实际复试成绩量化后:

\[
n_{i}=\sum_{k=1}^{5} \nu_{k} f_{i,k} /(5 \times 4) \times 100 \times (1-\alpha)
\]

那么对于一个专家来说,学生 $S_{i}$ 的笔试和复试的客观综合成绩 $c_{i}=m_{i}+n_{i}$。

\subsection{导师学术水平的量化}

在双向选择时,导师的学术水平是学生考虑导师的一个很重要的因素。问题中给定的导师学术水平指标表现在以下四个方面:发表论文数、论文检索数、编(译)著作数、科研项目数。假定四项指标占有同等重要地位,分别记为 $L_{p,k}(k=1,2,3,4)$,则可以给导师的学术水平指标量化评分,原则如下:

(1) 搜索四个指标的最大值分别记为 $L_{k} \ (k=1,\cdots,4)$,最大分数定为 25;

(2) 导师 $T_{p}$ 在各单项上的得分记为 $l_{p,k}=25L_{p,k}/L_{k} \ (p=1,\cdots,10)$;

(3) 导师 $T_{p}$ 在学术水平上的得分为四个单项分数的总和,本文中记为 $l_{p}$,其中 $l_{p}=25\cdot(\sum_{k=1}^{4}L_{p,k}/L_{k})$;

\subsection{学生自身能力与导师主观因素的符合度的量化}

假设参加复试的学生 $S_{i}$ 在五个专长上得分分别为 $f_{i,k}$,导师 $T_{p}$ 对学生 $S_{i}$ 各项专长的主观因素分数为 $F_{ip,k}$,那么在第 $k$ 项专长上,学生 $S_{i}$ 得分 $f_{i,k}$ 与导师 $T_{p}$ 的主观因素分数 $F_{ip,k}$ 的符合度记为 $\lambda_{ip,k}$,

\[
\lambda_{ip,k}=
\begin{cases}
1 & f_{i,k}\geq F_{ip,k} \\
\frac{f_{i,k}}{F_{ip,k}} & f_{i,k}<F_{ip,k}
\end{cases}
\quad
\begin{pmatrix}
i=1,\cdots,15, \\
p=1,\cdots,10, \\
k=1,\cdots,4
\end{pmatrix}
\]

学生 $S_{i}$ 自身能力与导师 $T_{p}$ 主观因素的符合度为五个专长项上符合度的平均值,记为 $g_{ip}$,其中 $g_{ip}=\frac{1}{5}\cdot\sum_{k=1}^{5}\lambda_{ip,k}$

\section{问题 1 的模型建立与求解}

\subsection{研究生的录取方案}

计算专家组中的专家 $E_{j} \ (j=1,\cdots,8)$ 给学生 $S_{i}$ 的客观综合成绩 $c_{ij}$ 的分数均值 $\overline{c}_{i}=\frac{1}{8}\cdot\sum_{j=1}^{8}c_{ij}$ 作为 $S_{i}$ 的客观成绩。取学生客观成绩的前 10 名作为录取的对象。

\subsection{学生对导师的满意度矩阵 $A=(a_{ip})$}

学生考虑导师时,按照自己专业发展志愿选择导师的原则如下:

(1) 学生 $S_{i}$ 将 10 名导师分成三个考虑等级 $D_{k}(k=1,2,3)$:第 1 志愿、第 2 志愿、不对口专业,考虑的优先顺序为:$D_{1}>D_{2}>D_{3}$。

(2) 等级相同则考虑两方面的因素:a. 导师的学术水平;b. 学生自身能力与导师对学生专长期望要求之间的相符程度。对 a 给定权重 $\beta$,b 的权重为 $(1-\beta)$。

(3) 对这样两个因素加权和,得到该等级下学生 $S_{i}$ 对导师 $T_{p}$ 的评价分,记为 $t_{ip}$,$t_{ip}=\beta\cdot l_{p}+(1-\beta)\cdot g_{ip}$,将属于该等级的导师按照评价分从高到低的顺序排列,就可以得到该等级下学生 $S_{i}$ 对导师的排名。

(4) 三个等级的排名相连接,得到学生 $S_{i}$ 对导师的整体排名 $SN_{ip}$,转化为学生 $S_{i}$ 对导师的满意度 $a_{ip}=0.1(11-SN_{ip})$。构成的满意度矩阵记为 $A=(a_{ip})_{10\times 10}$。

\subsection{5.3 导师对录取后的学生的满意度矩阵 $B=(b_{ip})$}

导师在选择学生时,考虑学生专业发展意愿的原则如下:

(1) 设导师 $T_{p}$ 所在专业为 $P_{k}(k=1,\cdots,4)$,则导师 $T_{p}$ 将优先考虑将该专业作为第 1 志愿和第 2 志愿的学生;最后选择专业不对口的学生。即将 10 名研究生分成三个选择等级 $C_{k}(k=1,2,3)$,顺序为:$C_{1}>C_{2}>C_{3}$。

(2) 在同一选择等级中,导师选择学生的优先次序也要考虑两方面的因素:a. 学生的客观综合成绩;b. 学生自身能力与导师对学生专长主观因素之间的符合度。对 a 给定权重 $\mu$,b 的权重为 $(1-\mu)$。

(3) 将这样的两个因素加权和,得到该等级下导师 $T_{p}$ 对学生 $S_{i}$ 的评价分,记为 $y_{pi}$,$y_{pi}=\mu c_{i}+(\sum_{k=1}^{5}\lambda_{ip,k})\cdot\frac{(1-\mu)}{5}$,按照评价分从高到低的顺序排列,就可以得到该等级下导师 $T_{p}$ 对学生的排名。

(4) 三个等级学生的排名相连接,得到导师 $T_{p}$ 对每个学生的整体排名 $TN_{pi}$,转化为导师 $T_{p}$ 对学生的满意度 $b_{pi}=0.1(11-TN_{pi})$,标准化处理后为 $b_{ip}$。构成的满意度矩阵记为 $B$,$B=(b_{ip})_{10\times 10}$。

\subsection{5.4 定义选择矩阵 $X$}

选择矩阵 $X=(x_{ip})_{10\times 10}\in\{0,1\}$,表示导师 $T_{p}$ 与学生 $S_{i}$ 之间的选择关系。考虑到学生不可能选择两个或两个以上的导师,故要求 $\sum_{p=1}^{10}x_{ip}=1$。

\subsection{5.5 多目标 0-1 规划模型 \footnote{1}}

多目标函数:
\begin{align*}
(1) & \max A_{i}(x)=\sum_{p=1}^{10}a_{ip}\cdot x_{ip} & i=1,\cdots,10 \\
(2) & \max B_{p}(x)=\sum_{i=1}^{10}b_{ip}\cdot x_{ip} & p=1,\cdots,10
\end{align*}

双目标函数:
\begin{align*}
(1) & \max A(x)=\sum_{i=1}^{10}\sum_{p=1}^{10}a_{ip}\cdot x_{ip} & i=1,\cdots,10 \\
(2) & \max B(x)=\sum_{p=1}^{10}\sum_{i=1}^{10}b_{ip}\cdot x_{ip} & p=1,\cdots,10
\end{align*}

\begin{equation}
public \quad s.t.: \left\{
\begin{aligned}
& \sum_{p=1}^{10}x_{ip}=1 & (i=1,\cdots,10; p=1,\cdots,10), \text{记为公共约束条件} \\
& x_{ip}(1-x_{ip})=0
\end{aligned}
\right.
\end{equation}

\subsection{5.6 单目标 0-1 规划模型}

双目标 0-1 规划问题在求解中仍然存在问题,将双目标转化为单目标,即将学生和导师的满意度分别最大转化为考虑综合平均的满意度最大。考虑对学生和老师满意度的偏好程度,对学生和导师分别取偏好系数 $w$,$(1-w)$,则双目标规划模型转化单目标规划模型 \footnote{2}

如下:
\[
\max M(x) = \frac{1}{10} \cdot [\max \{wA(x) + (1-w)B(x)\}] = \frac{1}{10} \left[ w \sum_{i=1}^{10} \sum_{p=1}^{10} a_{ip} \cdot x_{ip} + (1-w) \sum_{i=1}^{10} \sum_{p=1}^{10} b_{ip} \cdot x_{ip} \right]
\]
约束条件: $\quad public \quad s.t$

\subsection{5.7 单目标 0-1 规划模型的求解 \footnote{[3]}}

(1) 选择矩阵 $X = (x_{ip})_{10 \times 10}$ 是不确定 0-1 矩阵, 通过简单计算可知有 $10^{10}$ 种选择矩阵, 若对每一种选择矩阵均进行计算, 求出 $M(x)$ 的 $10^{10}$ 个值, 再在其中找出最大值 $\widetilde{M}$, 计算时间大约需要几十个小时。

(2) 改进算法。在求解最大综合平均满意度 $M(x)$ 的过程中, 当 $a_{ip} = b_{ip} = 1$ 时, 表示学生 $S_i$ 的最大满意度是选择导师 $T_p$, 且此时正好导师 $T_p$ 的最大满意度也是选择学生 $S_i$, 可以证明与 $a_{ip} = b_{ip} = 1$ 对应的 $x_{ip} = 1$。

\textbf{定理 1}: 求解最大平均满意度 $M(x)$, 若满意度矩阵中有 $a_{ip} = b_{ip} = 1$, 则 $x_{ip} = 1$。

\textbf{证明}: 反证法。(证略)

(3) 若满意度矩阵 $A = (a_{ip})_{10 \times 10}$ 和 $B = (b_{ip})_{10 \times 10}$ 中, 没有 $a_{ip} = b_{ip} = 1$ 的项, 但是当 $wa_{ip} + (1-w)b_{ip} > \max(wa_{ik} + (1-w)b_{ik}) (k = 1, \cdots, 10; k \neq p)$ 时, 可以证明在求解最大综合平均满意度 $M(x)$ 时 $x_{ip} = 1$, 即学生 $S_i$ 与导师 $T_p$ 确定双向选择关系。

\textbf{定理 2}: 在求解最大综合平均满意度 $M(x)$ 时, 若满意度矩阵中有 $wa_{ip} + (1-w)b_{ip} > \max(wa_{ik} + (1-w)b_{ik}) (k = 1, \cdots, 10; k \neq p)$ 时, 则 $x_{ip} = 1$。

\textbf{证明}: 反证法。(证略)

(4) 由于每名学生均须选择一名导师, 利用定理 2 十分简便的给出选择矩阵, 确定学生 $S_i$ 对导师的选择, 给出最优解。得到最大综合平均满意度为 $\widetilde{M} = 0.96$。

\section{6 问题 2 的模型建立与求解}

\subsection{6.1 模型的建立}

结合问题 1, 此时可以建立以下单目标 0-1 规划模型:
\[
\max M(x) = wA(x) + (1-w)B(x) = w \sum_{i=1}^{10} \sum_{i=1}^{10} a_{ip} \cdot x_{ip} + (1-w) \sum_{i=1}^{10} \sum_{i=1}^{10} b_{ip} \cdot x_{ip}
\]
约束条件: $\quad public \quad s.t$ 加上 $\sum_{i=1}^{10} x_{ip} = 1 \quad (i = 1, \cdots, 10; p = 1, \cdots, 10)$

\subsection{6.2 模型的求解}

本问题是一个 NP 难题, 除了遍历搜索求法能够得到最优解外, 其它方法得到结果只能是近似最优解。

(1) 近似算法

利用问题 1 中求解的程序。若满意度矩阵 $A = (a_{ip})_{10 \times 10}$ 和 $B = (b_{ip})_{10 \times 10}$ 中, 当 $wa_{ip} + (1-w)b_{ip} > \max(wa_{lk} + (1-w)b_{lk})$ 且 $(k \neq p, l = 1, \cdots, 10; l \neq i)$ 时, 取 $x_{ip} = 1$; 再在满意图矩阵中第 \(i\) 行和第 \(p\) 列以外的元素中找出 \(w a_{mn} + (1-w) b_{mn} > \max(w a_{lk} + (1-w) b_{lk})\) \((k=1,\cdots,10; k \neq n, p, l=1,\cdots,10; l \neq m, i)\)

取 \(x_{mn}=1\);\(\cdots\)。依此方法直至确定出每名导师带一名研究生的方案。

(2) 最优算法

选择矩阵 \(X=(x_{ip})_{10 \times 10}\) 根据约束条件,可知有 \(10!=3628800\) 种满足约束条件的选择矩阵,对每一种选择矩阵均进行计算,求出所有的 \(M(x)\),再在所有值中找出最大值 \(\widetilde{M}\),则一定是最优解,所需计算时间不是太多。

对每一种满足约束条件的选择矩阵,通过目标函数简单计算就能求出。考虑到约束条件的特殊性,这与计算机编程的经典问题“八皇后”问题求解 \({ }^{[4]}\) 十分相似,借鉴其计算思想来记录满足约束条件的选择矩阵。“八皇后”问题的主要思想在于回溯,回溯法的原则是“可行则进,不行则换、换不成则退”。借鉴解决“八皇后”问题的思想,采用回溯法产生满足约束条件的选择矩阵。由于最优解的计算时间稍长,下面设计一种可能是最优的降维近似算法

(3) 可能是最优的降维算法

若在满意度矩阵 \(A=(a_{ip})_{10 \times 10}\) 和 \(B=(b_{ip})_{10 \times 10}\) 中,有 \(j\) 对 \(a_{ip}=b_{ip}=1\) 时,即表示学生 \(S_{i}\) 的最大满意度是选择导师 \(T_{p}\),且此时正好导师 \(T_{p}\) 的最大满意度也是选择学生 \(S_{i}\)。首先将这 \(j\) 对学生 \(S_{i}\) 和导师 \(T_{p}\) 相互选定,选定后就可以将满意度矩阵降阶到 \((10-j) \times (10-j)\),降阶后的满意度矩阵和选择矩阵可以采用上述(2)最优算法,求出降维后的最优方案。最后将降维后的最优方案加上首先相互选定的 \(j\) 对学生 \(S_{i}\) 和导师 \(T_{p}\) 就构成了一个较优选择策略。通过建立转换关系,将高维的 8 一皇后问题转换成低维,有效的降低了时间运行效率,从而得到近似最优的选择方案。

\section{7 问题 3 模型的建立与求解}

与前面录取学生原则不同的是,这里首先对所有参加复试的学生按照客观成绩和主观要求综合评价,得到录取的研究生序列。

\subsection{7.1 导师对 15 名学生的评分}

导师选择学生的先后次序考虑两方面的因素:a. 学生的客观综合成绩(包括初试成绩和面试成绩);b. 学生自身能力与导师对学生专长主观因素之间的相符程度。对 a 给定权重 \(\mu\),b 权重为 \((1-\mu)\)。将 a、b 分数进行量化,采用 100 分制,可以给出导师 \(T_{p}\) 对学生 \(S_{i}\) 满意度评价

\[
y_{ip} = \mu c_{i} + \left( \sum_{k=1}^{5} \lambda_{ip,k} \right) \cdot \frac{100(1-\mu)}{5} \quad (i=1,\cdots,15; p=1,\cdots,10)
\]

每个导师 \(T_{p}\) 对学生 \(S_{i}\) 按照分数的高低对学生排列。按照前面对排名的标准量化得到所有导师对所有学生的满意度矩阵 \(B=(b_{ip})_{10 \times 15}\)。

\subsection{7.2 研究生的新录取方案}

计算 $b_{i}=\frac{1}{10} \cdot \sum_{p=1}^{10} b_{ip}$;对 $b_{i}$ 从大到小进行排序,取前 10 名即为录取研究生。为了表示的简单性,录取的 10 名研究生仍旧记为 $S_{i} \, (i=1,\cdots,10)$,导师 $T_{p}$ 对学生 $S_{i}$ 的满意度为 $b_{ip} \, (i=1,\cdots,10; p=1,\cdots,10)$。

\subsection{7.3 导师对录取的研究生的满意度矩阵 $E=(e_{ip})_{10 \times 10}$}

导师 $T_{p}$ 对学生评价分 $y_{ip}=\mu d_{i}+(\sum_{k=1}^{5} \lambda_{ip,k}) \cdot \frac{100(1-\mu)}{5}$。由于不考虑申报专业志愿,那么导师 $T_{p}$ 对录取的学生按照评价分 $y_{ip}$ 从高到低排列。对这样的学生排列从前到后赋予满意度为:$1.0 > 0.9 > \cdots > 0.1$。则导师 $T_{p}$ 对学生 $S_{i}$ 的满意度记为 $e_{ip}$,由所有的导师对录取后学生的满意度构成的满意度矩阵为 $E=(e_{ip})_{10 \times 10}$。

\subsection{7.4 研究生对导师的满意度矩阵 $Q=(q_{ip})_{10 \times 10}$}

由于不考虑申报专业志愿,那么学生 $S_{i}$ 对导师的选择顺序仅由满意度评价分 $t_{ip}=\beta \cdot l_{p}+(1-\beta) \cdot g_{ip}$ 的大小来确定,则学生 $S_{i}$ 对 10 名导师按照评价分从高到低排列,对这样导师排列从前到后赋予满意度为:$1.0 > 0.9 > \cdots > 0.1$。则学生 $S_{i}$ 对导师 $T_{p}$ 的满意度记为 $q_{ip}$,有所有研究生对导师的满意度构成的满意度矩阵为 $Q=(q_{ip})_{10 \times 10}$。

\subsection{7.5 模型的建立}

(1) 首先找出相互选中的研究生和导师,即找出所有满足 $q_{ip}=e_{ip}=1$ 的导师 $T_{p}$ 和学生 $S_{i}$,假设这样互相选中的研究生有 $j$ 对,则互相选中的导师和学生的下标组成的集合为 $D=\{d_{1}, d_{2}, \cdots d_{j}\}$,剩余的为导师 $T_{p} \, (p \notin D)$ 和学生 $S_{i} \, (i \notin D)$,学生 $S_{i} \, (i \notin D)$ 对导师 $T_{p} \, (p \notin D)$ 的满意度构成的满意度矩阵为 $\overline{Q}=(\overline{q}_{ip})_{(10-j) \times (10-j)}$,导师 $T_{p} \, (p \notin D)$ 对学生 $S_{i} \, (p \notin D)$ 的满意度构成的满意度矩阵为 $\overline{E}=(\overline{e}_{ip})_{(10-j) \times (10-j)}$,且剩余的导师与剩余的学生之间的选择矩阵为 $\overline{x}_{ip}$,其中 $p \notin D, i \notin D$。

(2) 根据上面的讨论,建立类似于问题 2 中的模型,如下
\begin{equation}
\max M(\overline{x})=\frac{1}{10}\left[w \sum_{i=1}^{10-j} \sum_{p=1}^{10-j} \overline{q}_{ip} \cdot \overline{x}_{ip}+(1-w) \sum_{i=1}^{10-j} \sum_{p=1}^{10-j} \overline{e}_{ip} \cdot \overline{x}_{ip}+j\right]
\end{equation}
\begin{equation}
s.t
\left\{
\begin{aligned}
\sum_{i=1}^{10-j} \overline{x}_{ip} &=1 \\
\sum_{p=1}^{10-j} \overline{x}_{ip} &=1 \\
x_{ip}(1-x_{ip}) &=1
\end{aligned}
\right.
\quad
\begin{aligned}
i &=1,\cdots,10-j \\
p &=1,\cdots,10-j
\end{aligned}
\end{equation}

\subsection{7.6 模型的求解}

(1) 记在满意度矩阵中,共有 $j$ 对 $q_{ip}=e_{ip}=1$。首先将这 $j$ 对学生 $S_{i}$ 和导师 $T_{p}$ 相互选定,选定后就可以将满意度矩阵降阶到 $(10-j) \times (10-j)$,降阶后的选择矩阵需要满足的约束条

件与问题 2 中选择矩阵的约束条件相同。

(2) 降阶后选择矩阵、满意度矩阵及约束条件和目标函数与问题 2 是同一类问题。在产生满足约束条件的选择矩阵时,借鉴解决“八皇后”问题的思想,采用回溯法产生选择矩阵。

(3) 在产生的 $(10-j)!$ 种满足约束条件的选择矩阵后,对每一种选择矩阵均进行相应计算,求出所有的 $M(x)$,再在所有值中找出最大值 $\widetilde{M}$,即为最优解。

\section{问题 4 模型的建立与求解}

\subsection{选择 5 名导师的原则}

学校将每位学生对所有导师的满意度进行加权和,满意度排在前 5 名的导师为可以招收研究生的导师。由于考虑申报志愿的影响,故可以参照模型准备中的方法构造学生对导师的满意度矩阵 $A=(a_{ip})_{15\times 10}$。计算 $a_{p}=\sum_{i=1}^{15}a_{ip}/15$,对 $a_{p}$ 排序,取前 5 名为选择的导师。仍然记选出的 5 名导师为 $T_{p}$ ($p=1,\cdots,5$)。

\subsection{选择 10 名研究生的策略}

导师选择学生结合问题 1 中的分析,可以给出导师 $T_{p}$ 对 15 名学生满意度评价 $y_{ip}=\mu c_{i}+(\sum_{k=1}^{5}\lambda_{ip,k})\cdot 100(1-\mu)/5$ ($i=1,\cdots,15; p=1,\cdots,5$) 再根据模型准备中的方法构造导师对学生的满意度矩阵 $B=(b_{ip})_{15\times 5}$。将每位导师对所有学生的满意度进行加权相加,计算 $b_{i}=\sum_{p=1}^{5}b_{ip}/5$;对 $b_{i}$ 从大到小进行排序,取前 10 名即为录取的研究生。仍记录取的 10 名学生为 $S_{i}$ ($i=1,\cdots,10$)。

\subsection{双向选择最佳策略模型}

通过上述分析,在问题 2 中建立的模型基础上进行改进可以得到如下模型:
\begin{equation}
\begin{aligned}
\max M(x) &= w\sum_{i=1}^{10}\sum_{p=1}^{5}a_{ip}\cdot x_{ip}+(1-w)\sum_{i=1}^{10}\sum_{p=1}^{5}b_{ip}\cdot x_{ip} \\
s.t \left\{
\begin{aligned}
\sum_{p=1}^{10}x_{ip} &= 1 \\
\sum_{i=1}^{10}x_{ip} &= 2 \quad (i=1,\cdots,10; p=1,\cdots,5) \\
x_{ip}(1-x_{ip}) &= 0
\end{aligned}
\right.
\end{aligned}
\end{equation}

\subsection{模型的求解}

将每位导师当作两名导师来考虑,则 5 名导师就可看作 10 名导师,若这假想的 10 名导师与 10 名学生之间实现一名导师只带一名研究生,相应的也就实现了实际的 5 名导师每人带两名研究生。那么可以构造矩阵 $\overline{A}=(A,A)_{10\times 10}$,$\overline{B}=(B,B)_{10\times 10}$,对应的将选择矩阵拓展为 $X=(x_{ip})_{10\times 10}$,故模型可以等价转化为:
\begin{equation}
\max M(x) = w\sum_{i=1}^{10}\sum_{p=1}^{10}\overline{a}_{ip}\cdot x_{ip}+(1-w)\sum_{i=1}^{10}\sum_{p=1}^{10}\overline{b}_{ip}\cdot x_{ip}
\end{equation}

约束条件: $public \ s.t$ 加上 $\sum_{i=1}^{10} x_{ip} = 1 \quad (i=1,\cdots,10; p=1,\cdots,10)$

通过对约束条件的巧妙转化,使得转化后的等价模型与问题 2 中模型的形式完全相同,其解法也完全一样。在产生满足约束条件的选择矩阵时,借鉴“八皇后”问题的思想,采用回溯法产生选择矩阵。在产生的 $10!$ 种满足约束条件的选择矩阵后,对每一种选择矩阵均进行相应计算,求出所有的 $M(x)$,再在所有值中找出最大值 $\widetilde{M}$,即为最优解。

\section{问题 5 模型的建立与求解}

\subsection{“双向选择”合理的准则}

双向选择涉及到导师和学生两个群体,一个决策关系到群体中的每个个体的利益,每个个体根据自身所获的利益的大小对决策都有一个满意度,一个绝好的决策是使得群体中的各个个体都获得最大利益。在实际决策中由于各种因素的限制,不可能不牺牲部分个体的利益,而保证整个群体在决策中取得最大效益。因此在涉及到个体与群体时,决策首要目标是实现群体利益最大是合理的。因此对于双向选择的决策,将实现群体综合平均满意度最大作为首要目标是合理的。

\subsection{问题分析}

问题 1 至 4 中所有策略存在的最大不足点在于用分步决策(分层次决策)来代替全局决策,将决策分为两步或三步,都是先录取研究生,再对录取的研究生作出双向选择策略,下一步是在前一步决策的基础上。但由决策论知识可知,分步决策做出的最后结果,常常不是最优解。

解决的方法是改变分步决策的做法,寻找能够实现全局决策的模型。如果我们将录取研究生工作的主要内容分为两个步骤:(1) 录取研究生;(2) 分配研究生。先解决 (1),在确定 (1) 的结果的前提下再对 (2) 作出分配决策,这样很显然是分步决策;但是当将录取研究生的工作放在分配研究生中同时进行,就能实现全局决策。具体做法是:

(1) 分别求出 15 名学生对 10 名导师的满意度矩阵 $A = (a_{ip})_{15 \times 10}$ 和导师对学生的满意度矩阵 $B = (b_{ip})_{15 \times 10}$,满意度矩阵的做法与模型准备中相同,在此不再重述。

(2) 由于最终录取的研究生只有 10 名,故参加分配也只能有 10 名学生,但现在有 15 名学生。为了使得参加分配的学生只有 10 名,只能在 15 名学生中随机的剔除 5 名学生,让剩下的 10 名学生参加分配,这时通过问题 1 中的模型计算可以求得一个在此随机剔除下的一个最优解,记为 $\widetilde{M}_i$。

(3) 为了求出一个最佳的剔除方案,实现全局最优。可以对所有可能的剔除方案 ($C_{15}^5 = 3003$ 种),分别进行计算。取 $\hat{M} = \max \left\{ \widetilde{M}_1, \widetilde{M}_2, \cdots, \widetilde{M}_{3003} \right\}$,则 $\hat{M}$ 是全局最优解,使得师生间的满意度最大,最能体现研究生录取的“双向选择”合理性。

\subsection{问题 5 模型的建立 \footnote{[5]}}

与问题 1 中的模型相比,产生一组剔除 5 名学生的序列 $U_k$,分别记为第$i_{1}, i_{2}, i_{3}, i_{4}, i_{5}$ 名学生,反映在选择矩阵上为 $\sum_{p=1}^{10} x_{i_{1} p}=0, \sum_{p=1}^{10} x_{i_{2} p}=0, \sum_{p=1}^{10} x_{i_{3} p}=0, \sum_{p=1}^{10} x_{i_{4} p}=0, \sum_{p=1}^{10} x_{i_{5} p}=0$。那么在剔除序列 $U_{k}$ 后,可以用下述模型求解 $\widetilde{M}_{k}$:

\[
\begin{aligned}
\widetilde{M}_{k} = & \frac{1}{10} \left[ \max \left\{ w \sum_{i=1}^{15} \sum_{p=1}^{10} a_{i p} \cdot x_{i p} + (1-w) \sum_{i=1}^{15} \sum_{p=1}^{10} b_{i p} \cdot x_{i p} \right\} \right] \\
s.t & \left\{
\begin{array}{ll}
\sum_{p=1}^{10} x_{i_{j} p} = 0 & (i_{j} \in \{1, 2, \cdots, 15\}; j = 1, 2, 3, 4, 5) \\
\sum_{p=1}^{10} x_{i p} = 1 & (i \neq i_{k}) \\
x_{i p}(1 - x_{i p}) = 0 & (i \in \{1, 2, \cdots, 15\}; p \in \{1, 2, \cdots, 10\})
\end{array}
\right.
\end{aligned}
\]

则全局最优解为:$\hat{M} = \max \left\{ \widetilde{M}_{1}, \widetilde{M}_{2}, \ldots, \widetilde{M}_{k}, \ldots, \widetilde{M}_{3003} \right\}$

\subsection{9.4 问题 5 模型的求解}

优化算法。在任取一种剔除 5 名学生序列所对应的选择矩阵下,若满意度矩阵 $A = (a_{i p})_{10 \times 10}$ 和 $B = (b_{i p})_{10 \times 10}$,有 $w a_{i p} + (1-w) b_{i p} > \max(w a_{i k} + (1-w) b_{i k})$ 时,此时 $x_{i p} = 1$,即学生 $S_{i}$ 与导师 $T_{p}$ 确定双向选择关系。

由于每一个学生均须选择一名导师,利用定理 2 的结论可以很容易给出选择矩阵,从而确定学生 $S_{i}$ 对导师的选择,给出最优解。

利用目标函数和最优选择矩阵,求出 $\widetilde{M}_{k}$,再在 3003 个 $\widetilde{M}_{k}$ 中找出的最大值 $\hat{M} = \max \left\{ \widetilde{M}_{1}, \widetilde{M}_{2}, \ldots, \widetilde{M}_{k}, \ldots, \widetilde{M}_{3003} \right\}$ 即为全局最优解。

\section{10 模型结果分析}

从上述问题各模型的最优解中可以得到下表:

\begin{table}[h]
\centering
\begin{tabular}{|c|c|c|c|}
\hline
 & 考虑志愿 & 数量限制 & 满意度 \\
\hline
问题 1 & 是 & 否 & 0.96 \\
\hline
问题 2 & 是 & 是 & 0.89 \\
\hline
问题 3 & 否 & 是 & 0.66 \\
\hline
问题 4 & 是 & 是 & 0.94 \\
\hline
问题 5 & 否 & 否 & 0.97 \\
\hline
\end{tabular}
\end{table}

\begin{figure}[h]
\centering
\includegraphics[width=0.8\textwidth]{image.png}
\caption{问题最优结果比较图}
\end{figure}

从图 1 中,可以得出如下结论:

1) 考虑专业志愿,能够提高综合平均满意度;

\begin{enumerate}
    \item[2)] 允许导师带多名学生,能够提高综合平均满意度;
    \item[3)] 与分步决策相比,对问题进行全局决策,能够提高综合平均满意度。
\end{enumerate}

\section{参考文献:}

\begin{enumerate}
    \item 任善强,雷鸣编著,数学模型(第二版),重庆大学出版社,1998年4月
    \item 叶其孝主编,大学生数学建模竞赛辅导教材,湖南教育出版社,1998年
    \item 运筹学教材编写组,运筹学(修订版),清华大学出版社,1990.3
    \item 张铭,基本算法选讲,北京大学计算机系
    \item 袁亚湘,孙文瑜著,最优化理论与方法,科学出版社,2001年3月
\end{enumerate}

\section{Two-way Selecting Strategy on Postgraduate Recruitment Problem}

\textbf{Cao Bao-hua} \quad \textbf{Chen Yan-zhou} \quad \textbf{Guo Lan-ying}

(Wuhan University, Wuhan Hubei 430072, China)

\textbf{Abstract:} The postgraduate recruitment problem under different conditions is analyzed in this paper according to the background and requirement of this question. Under different conditions, a two-way selecting 0-1 planning model has been put forward as well as the corresponding calculating application program on this problem. The optimum program gained from optimum solution is achieved through calculation of every question. This model has guidance to the practical math calculation and mathematical modeling of the university student.

\textbf{Keywords:} Postgraduate recruitment problem; Two-way Selecting Strategy; 0-1 planning model