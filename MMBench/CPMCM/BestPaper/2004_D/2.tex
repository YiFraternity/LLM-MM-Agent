\title{基于最大满意度的研究生录取问题}

\author{王文娟 张元标 张章华}

\affil{(成都理工大学信息管理学院,四川 成都 610059)}

\begin{abstract}
摘要:本文采用了模糊数学的知识和优化方法讨论解决了研究生的录取问题。首先对相关数据进行了合理的量化,然后定义了导师与学生之间的相互满意度,建立了择优录取和双向选择的优化模型,通过求解得到了理想的结果。
\end{abstract}

关键词:研究生录取问题;优化模型;隶属函数;权重

\section{问题重述(略)}

\section{模型假设}

(1) 8 位专家的地位是平等的,采用统一的标准,打分是公平公正的;

(2) 专家打分的等级(ABCD),各相邻等级差相同;

(3) 学生综合素质的各项指标认为是同等重要的。

(4) 学生和导师在选择中都是理性的,完全依据所给的信息做出选择。

\section{模型建立及求解}

\subsection{问题(1)的分析与建模}

该问题首先要求在综合考虑学生的初试成绩、复试成绩等因素,确定 10 名研究生的录取名单。研究生教育的目的是培养高素质的科研人才,而研究生应该是在综合能力强的学生中选拔,所以严格说来,仅以笔试成绩遴选会有很大的局限性,为此,在录取研究生的过程中需要加入复试的成绩来考察学生的各项综合能力。在此,为了便于对每一个学生综合能素质给出合理评价,需要对学生的各项成绩进行数据的标准化处理。

\paragraph{初试成绩的评价}

这里,学生的初试成绩是以固定的分数给出的,采用常用的数据处理方法——极差变换法,则由公式

\[
r_{j}=\frac{C_{j}-\min _{1 \leq j \leq 15}\left\{C_{j}\right\}}{\max _{1 \leq j \leq 15}\left\{C_{j}\right\}-\min _{1 \leq j \leq 15}\left\{C_{j}\right\}}
\]

得到对各位学生的初试成绩的评价 $r_{j}(j=1,2, \cdots, 15)$。

\paragraph{复试成绩的评价}

由于对学生五项特长指标的评分都具有一定的模糊性,对于评价结果 A,B,C,D 四个等级就构成了模糊集,可以用模糊评语集 $\{$ 很好,好,一般,差 $\}$ 来代替,依据 5 分制评分标准,取其对应的数值为 5,4,3,2。这样用 $P_{ijk}$ 表示第 $i$ 个专家对第 $j$ 个学生的第 $k$ 项指标的评分。根据假设,8 个专家的地位相同,且由于学生报考专业不明确,无法确定各个指标的重要程度,在此认为五项指标具有同等的重要性,于是采用累加求平均的方法:

\[
F_{j}=\frac{1}{8 \times 5} \sum_{i=1}^{8} \sum_{k=1}^{5} P_{ijk} \quad(j=1,2, \cdots, 15)
\]

得到专家组对每个学生的综合评分。这里采用模糊数学中的隶属函数来处理所得数据。在隶属函数的选取上,为了使处理后的数据既能适当的区分所有学生的复试成绩的优劣,又能充分考虑到少量复试成绩突出学生的优势,要求转化之后的分数与原来的分数应存在下面的关系:对于很高分的和很低分的分数,其变化率应较小,而分数在中间那部分其变化率应较大。这样就能较好的区分复试成绩在中间的部分学生的实力。为此,在这里选用偏大型中升岭形分布函数为隶属函数来处理复试成绩。升岭形分布函数公式为

\[
A(x) = \begin{cases}
0, & x \leq a_1 \\
\frac{1}{2} + \frac{1}{2} \sin \frac{\pi}{a_2 - a_1} \left( x - \frac{a_1 + a_2}{2} \right), & a_1 < x \leq a_2 \\
1, & a_2 < x
\end{cases}
\]

求出 \( A(F_j) \) 得到专家对每个学生的评价,其参数 \( a_1, \ a_2 \) 分别取值 \( a_1 = \min_{1 \leq j \leq 15} \{ F_j \} \), \( a_2 = \max_{1 \leq j \leq 15} \{ F_j \} \)。这样就得到专家组在复试中对每个学生的评价 \( f_j = A(F_j) \)。

\paragraph{(3) 学校对学生的综合评价}

学校对学生的录取主要是看学生的综合素质,而学生的综合素质是通过初试成绩和复试成绩两个方面因素决定的,为避免高分学生因心理因素造成面试成绩过低而落选,对高分学生给予适当保护政策,同时又能挑选出部分由于笔试成绩出现失误而本身有着很高综合素质的学生,在此把初试成绩和复试成绩的权重分配确定为 \( W_z = (w_c, w_f) = (0.4, 0.6) \)。在这里,认为对于参加复试的学生,都是达到了录取分数线即达到了研究生的入学条件,故复试成绩的权重应大于初试成绩的权重,加大复试的作用。则由加权求和的方法,利用公式

\[
\lambda_j = (w_c, w_f)(r_j, f_j)^T \quad (j = 1, 2, \cdots, 15)
\]

得到学校对每个学生的综合评价分数 \( \lambda_j \)。于是,确定录取学生的数学模型归纳为

\[
\begin{cases}
\lambda_j = (w_c, w_f)(r_j, f_j)^T \\
r_j = \frac{C_j - \min_{1 \leq j \leq 15} \{ C_j \}}{\max_{1 \leq j \leq 15} \{ C_j \} - \min_{1 \leq j \leq 15} \{ C_j \}} \\
f_j = A(F_j) \\
F_j = \frac{1}{8 \times 5} \sum_{i=1}^8 \sum_{k=1}^5 P_{ijk}
\end{cases}
\]

求解此模型得到 \( \lambda_j \),据此综合评价分数对学生进行排序,即得到学生的排名,从高到低确定出录取的 10 名研究生。计算得到录取的 10 名学生为 \{1, 2, 3, 4, 5, 6, 8, 9, 12, 15\}。

\paragraph{(4) 导师学术水平的评价}

我们知道导师的学术水平是从四个方面体现的,其分别是发表论文数(L)、论文检索数(S)、编(译)著作数(B)、科研项目数(Y)。通过我校研究生院对研究生导师的考核标准的调查分析,对导师的学术水平的衡量,在上面四个方面的看重程度上有所不同,最看重的是导师的论文检索数,其次是导师的科研项目数,对发表论文数和编(译)著作数看得相对较低,故对这四方面的权值确定为

\[
W = (w_1, w_2, w_3, w_4) = (0.15, \ 0.40, \ 0.15, \ 0.3)
\]

对于导师的学术水平各项指标 \( L_i, \ S_i, \ B_i, \ Y_i \),是通过具体数目给出的,在此同样采用极差变换法,即由下面公式得到各导师的相应各项指标量化值:

\[
l_i = \frac{L_i - \min_{1 \leq i \leq 10} \{ L_i \}}{\max_{1 \leq i \leq 10} \{ L_i \} - \min_{1 \leq i \leq 10} \{ L_i \}}, \quad s_i = \frac{S_i - \min_{1 \leq i \leq 10} \{ S_i \}}{\max_{1 \leq i \leq 10} \{ S_i \} - \min_{1 \leq i \leq 10} \{ S_i \}},
\]
\[
b_i = \frac{B_i - \min_{1 \leq i \leq 10} \{ B_i \}}{\max_{1 \leq i \leq 10} \{ B_i \} - \min_{1 \leq i \leq 10} \{ B_i \}}, \quad y_i = \frac{Y_i - \min_{1 \leq i \leq 10} \{ Y_i \}}{\max_{1 \leq i \leq 10} \{ Y_i \} - \min_{1 \leq i \leq 10} \{ Y_i \}}.
\]

记导师学术水平向量为 \( X_i = (l_i, s_i, b_i, y_i) (i = 1, 2, \cdots, 10) \),采用下面公式得到对第 \( i \) 个导师的综合水平的评价 \( D_i \),即

\[
D_i = WX_i^T = (w_1, w_2, w_3, w_4) \cdot (l_i, s_i, b_i, y_i)^T (i = 1, 2, \cdots, 10)
\]

\paragraph{(5) 导师对学生的满意度}

导师对这 10 个学生的选择是根据学生所报专业志愿、专家组对学生的各项特长的评价与导师自

己对学生的期望要求来确定的,综合这些因素,可以得出导师对每个录取学生的满意程度,由满意程度的大小来确定选择学生。

首先根据 8 位专家对每个学生每项特长的评价,用类似前面(3)中的处理方法,先把 A,B,C,D 四个等级,依据 5 分制评分标准,转化为相应的数值 5,4,3,2,然后用求算术平均的方法,得到每个学生的每项特长的综合评分,用这个综合评分与每位导师对每位学生的五项特长的期望要求比较得出差异矩阵 \( H \),依据差异矩阵的数值 \( h_{ij} \) 来划分等级,具体做法为找出最大值和最小值确定长度区间,再除以欲划分的等级数目,得到每个等级之间的等级长度 \( q = \frac{\max\limits_{1 \leq i, j \leq 10} \{h_{ij}\} - \min\limits_{1 \leq i, j \leq 10} \{h_{ij}\}}{4} \)。

用模糊评语 \{很满意,满意,基本满意,不满意\} 来表示,由此得到每位导师对每位学生的各项特长的评价。在此我们考虑划分成四个等级,

- 当数值介于区间 \([ \min\limits_{1 \leq i, j \leq 10} \{h_{ij}\}, \ q + \min\limits_{1 \leq i, j \leq 10} \{h_{ij}\})\) 之间时,认为不满意;
- 当数值介于区间 \([ q + \min\limits_{1 \leq i, j \leq 10} \{h_{ij}\}, \ 2q + \min\limits_{1 \leq i, j \leq 10} \{h_{ij}\})\) 之间时,认为基本满意;
- 当数值介于区间 \([ 2q + \min\limits_{1 \leq i, j \leq 10} \{h_{ij}\}, \ 3q + \min\limits_{1 \leq i, j \leq 10} \{h_{ij}\})\) 之间时,认为满意;
- 当数值介于区间 \([ 3q + \min\limits_{1 \leq i, j \leq 10} \{h_{ij}\}, \ \max\limits_{1 \leq i, j \leq 10} \{h_{ij}\} ]\) 之间时,认为很满意。

这四个等级就构成模糊集,取其对应的数值为 2,3,4,5。这样就得到了每位导师对每位学生的每项特长与自己的期望要求的一个差异评分 \( G_{ij} \)。然后选用偏大型中升岭形分布函数为隶属函数来处理差异评分,得到在不考虑学生所报专业志愿前提下对学生的评价 \( E_{ij} = A(G_{ij}) \)。但由于导师对学生的选择还要看学生的志愿,于是再进行加权处理,对于第一志愿与导师方向相同的赋权值为 1,第二志愿与导师方向相同的赋权值为 0.6,没报与导师方向的赋权值为 0,这样得到第 \( i \) 个导师对第 \( j \) 个录取学生的满意度 \( M_{dij} \)。其计算公式为

\[
M_{dij} = \alpha_{ij} \cdot E_{ij}, \ \text{其中 } \alpha_{ij} =
\begin{cases}
1, & \text{当学生 } j \text{ 第一志愿跟导师 } i \text{ 的方向一样;} \\
0.6, & \text{当学生 } j \text{ 第二志愿跟导师 } i \text{ 的方向一样;} \\
0, & \text{当学生 } j \text{ 的志愿跟导师 } i \text{ 的方向不一样.}
\end{cases}
\]

\paragraph{(6)学生对导师的满意度}

学生对导师的选择主要是根据学生自己的专业发展意愿、导师的基本情况和导师对学生的期望要求。集中体现在专业志愿和导师的学术水平上,导师的学术水平是客观存在的,可以用导师的学术水平评价 \( D_i \) 来衡量,不同的学生对导师的满意程度可以认为是取决于导师所从事的研究方向。类似地,可以通过加权方法来处理,即当导师研究方向与自己的第一志愿相同时,赋权值为 1;当导师研究方向与自己的第二志愿相同时,赋权值为 0.6;当导师研究方向与自己的志愿全不相同时,赋权值为 0。其计算公式为

\[
M_{xji} = (\alpha_{1j}, \alpha_{2j}, \ldots, \alpha_{10,j})^T (D_1, D_2, \ldots, D_{10})
\]

其中志愿权值 \( \alpha_{ij} \) 取法同上。这时我们得到第 \( j \) 个学生对第 \( i \) 个导师的满意度 \( M_{xji} \)。

\paragraph{(7)双方满意度的综合评价}

确定了被录取的学生和导师双方对彼此的满意度,我们认为学生和导师彼此的满意度的和越大、两者的差越小,两者的相互选中可能性越大,又有满意度均为正,所以这里采用几何平均的方法来综合两者的满意度。定义导师和学生两两人之间的相互满意度为

\[
M_{ij} = \sqrt{M_{dij} \cdot M_{xji}}
\]

在各个导师所带的学生数没有限制而学生只能选一个导师的情况下,为了达到一种选择方案使得师生双方的满意度达到最大,我们可以把原问题转化成整数规划问题加以求解。

设决策变量为 \( x_{ij} \)(\( i \) 代表导师,\( j \) 代表录取学生),为此取问题的目标函数为

\begin{equation}
\max Z = \sum_{i=1}^{10} \sum_{j=1}^{10} M_{ij} x_{ij}, \quad \text{约束条件为 } \sum_{i=1}^{10} x_{ij} = 1 (j=1,2,\cdots,10), \text{ 即表示一个学生只能选择一名导师。}
\end{equation}

综合可得到下面的数学模型

\begin{equation}
\begin{aligned}
\max Z &= \sum_{i=1}^{10} \sum_{j=1}^{10} M_{ij} x_{ij} \\
s.t. & \begin{cases}
\sum_{i=1}^{10} x_{ij} = 1 (j=1,2,\cdots,10) \\
x_{ij} = 0 \text{ 或 } 1 (i,j=1,2,\cdots,10)
\end{cases}
\end{aligned}
\end{equation}

其中 $M_{ij} = \sqrt{M_{dij} \cdot M_{xji}}$, $M_{dij} = \alpha_{ij} \cdot E_{ij} = \alpha_{ij} \cdot A(G_{ij})$, $M_{xji} = (a_{1j}, a_{2j}, \ldots, a_{10,j})^T (D_1, D_2, \ldots, D_{10})$, $D_i = WX_i^T = (w_1, \ w_2, \ w_3, \ w_4)(l_i, \ s_i, \ b_i, \ y_i)^T (i,j=1,2,\cdots,10)$。

应用 Lingo 软件对上面整数规划问题求解得到目标值为 7.047270,此时导师与学生的选择方案为学生 1-导师 4,学生 2-导师 6,学生 3-导师 3,学生 4-导师 9,学生 5-导师 6,学生 6-导师 6,学生 8-导师 4,学生 9-导师 3,学生 12-导师 6,学生 15-导师 3。

\subsection{问题 2 的分析与建模}

把问题 (2) 与问题 (1) 做比较,发现问题 (2) 只比问题 (1) 多了一个限制条件,即每个导师只能带一个学生,$\sum_{ji=1}^{10} x_{ij} = 1 (j=1,2,\cdots,10)$,加上原有的每个学生只能选一个导师,这样把问题转化成标准的指派问题,其模型为

\begin{equation}
\begin{aligned}
\max Z &= \sum_{i=1}^{10} \sum_{j=1}^{10} M_{ij} x_{ij} \\
s.t. & \begin{cases}
\sum_{i=1}^{10} x_{ij} = 1 (j=1,2,\cdots,10) \\
\sum_{j=1}^{10} x_{ij} = 1 (i=1,2,\cdots,10) \\
x_{ij} = 0 \text{ 或 } 1 (i,j=1,2,\cdots,10)
\end{cases}
\end{aligned}
\end{equation}

应用 Lingo 软件求解上述指派问题,可以得到一个学生和一个导师一对一选择的方案:学生 1-导师 4,学生 2-导师 6,学生 3-导师 1,学生 4-导师 9,学生 5-导师 7,学生 6-导师 8,学生 8-导师 5,学生 9-导师 3,学生 12-导师 10,学生 15-导师 2。其目标值为 4.8307。

\subsection{问题 3 的分析与建模(思路)}

首先由十位导师根据初试成绩、专家组的面试评价和他们自己对学生的要求条件来确定 10 名研究生。现在需要在假设学生没有申报专业志愿的情况下,给出已确定的 10 名研究生各申报一名导师的策略和导师各选择一名研究生的策略。由于导师和导师、学生和学生、导师和学生他们彼此之间的信息都是公开的,所以任何一个人做出选择的时候,都应考虑到与他有利益冲突的人在做出和他相同的选择时,被选人选他的机会的大小,可用赢得值表示,当赢得值为正时表示他比别人获得更多的机会被选择,否则机会更小。这就可以把问题归结为多人非合作对策问题。由于多人非合作对策问题在求解上存在问题,规模稍大则计算量很大,甚至难以实现。据此依据贪婪算法的思想采用下面折中算法进行求解。以学生选导师为例说明算法思想:

对于学生 $i$ 在导师 $k$ 心中的满意度为 $M_{dk_i}$,做 $Y_{kij} = M_{dk_i} - M_{dk_j}$(表示在对导师 $k$ 上学生 $i$ 比学生 $j$ 赢得值,赢得值为正的表示 $i$ 比 $j$ 在选择 $k$ 时占优),那么学生 $i$ 在选择的时候应选择相对于学生 $j$ 赢得值为正的导师,则计算出学生 $i$ 针对学生 $j$ 选择导师 $k$ 时比其他学生比较占优的总次数

$n_{k_{ij}}(j \neq i, j=1,2,\cdots 10)$(表示学生 $i$ 在选择导师 $k$ 时比另外 9 个学生中占优的学生人数),取 $\{n_{k_{ij}}(j \neq i, j=1,2,\cdots 10)\}$ 中最大的值所对应的导师 $m_{j}^{*}$(如果对于同一个 $j$ 出现两个相同的 $n_{k_{ij}}$ 则在这两个导师中选取学生 $i$ 比学生 $j$ 赢得最大的导师 $m_{j}^{*}$,或者是学生对导师的满意度大的那个导师 $m_{j}^{*}$);学生 $i$ 选取导师过程中,在导师 $m_{j}^{*}$ 中选取学生 $i$ 对导师 $m_{j}^{*}$ 满意度最大的那个导师为 $m_{l}$,这样就确定了选择人 $i$ 选择 $m_{l}$。在导师选择学生和学生选择导师过程中分别采用这个算法确定导师的选择和学生的选择,再确定导师和学生是否互选,互选的则为确定,退出选择系统。余下的继续采用相同方法确定,直到确定每一对导师和学生。

\subsection{问题(4)的分析与建模(思路)}

问题(4)首先要确定导师和研究生的选择(录取)方案。根据 10 名导师和 15 名学生的综合情况来确定 5 名导师招收研究生,对导师的选择主要通过导师的学术水平和学生所填志愿情况来确定。再让这 5 名导师在 15 名学生中择优录取 10 名研究生。这 5 名导师对学生的选择,则根据 15 个学生的初试成绩、学生的条件与 5 名导师对学生的要求这些因素综合得到。其次要确定每一名导师带 2 名研究生的双向选择的最佳策略。可把问题转化为每名导师有一个助手,在其选择时完全按导师的标准做出选择,就把原问题转化成 5 名导师和 5 名导师助手对 10 个学生的选择问题,即把导师 1 的助手看成导师 6,导师 1 和导师 6 的自身条件和对学生的要求是完全相同的,以次类推。这样就得到 10 个导师对 10 个学生的选择。

\section{模型的评价、改进及推广(详略)}

\section{参考文献}

[1] 钱颂迪等. 运筹学 [M]. 北京:清华大学出版社. 1997.7

[2] 汪培庄,李洪兴. 模糊系统理论与模糊计算机 [M]. 上海:科学出版社. 1996.

[3] 李登峰. 模糊多目标多人决策与对策 [M]. 北京:国防工业出版社. 2003.4

[4] 姜启源,谢金星等. 数学模型(第三版)[M]. 北京:高等教育出版社. 2003.8

[5] 李登峰. 模糊多目标多人决策与对策 [M]. 北京:国防工业出版社. 2003.4

\section{The Problem Solution of Recruiting Graduate Students}

\section{For Greatest Satisfaction}

Wang Wen-juan Zhang Yuan-biao Zhang Zhang-hua

(College of Information and Management, Chengdu University of Technology, Chengdu, 6110059, China)

Abstract: This paper discussed the problem of recruiting graduate students with the knowledge of Fuzzy Mathematics and the Optimization Method. At first, relevant data are quantized rationally. Then, the mutual degree of satisfaction between the tutor and student is defined. Finally, the optimal model for recruiting and bi-directional selection is set up, and the ideal result is obtained.

Keywords: the recruiting Problem of Graduate Student; Optimal model; Membership function; Weight