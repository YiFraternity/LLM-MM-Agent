\begin{center}
\textbf{第九届“华为杯”}\\
\textbf{全国研究生数学建模竞赛}
\end{center}

\begin{table}[h]
\centering
\caption{第500行与第500、501、502列交叉处的经纬度}
\begin{tabular}{c c c}
\hline
(行, 列) & 经度 & 纬度 \\
\hline
(500, 500) & 46.1511E & 32.9206N \\
(500, 501) & 46.2371E & 32.9148N \\
(500, 502) & 46.3229E & 32.9090N \\
\hline
\end{tabular}
\end{table}

\begin{abstract}
    摘       要:
    本文针对卫星云图推导云迹风的问题,首先对卫星采集的灰度矩阵数据进行预处理,
    并通过坐标变换将其投影到经纬度坐标上,经过等距节点插值方法,形成新的灰度矩阵,
    并以此数据为依据,分别建立了基于相关系数法和基于灰度梯度以及边缘特征匹配的风
    矢场度量模型,对相应的坐标点进行了风矢量的计算以及等压面的求取。
    针对问题1,首先对地球和卫星建立空间几何模型,求取灰度矩阵行列号对应于地
    球经纬度的推导公式,然后对卫星采集的灰度数据进行降噪滤波处理,并根据海岸线经
    纬度经过公式反推得到其在卫星云图的位置坐标,最后画出带海岸线的卫星云图。
    ## 针对问题2,首先经过坐标变换和节点插值形成投影平面上新的灰度矩阵,然后针
    对窗口大小和搜索范围固定的特征匹配模式,使用相关系数法建立风矢场度量模型,并
    加入图像预处理、云块识别、相关系数阈值检验、风矢质量检测和连续性检测等环节,
    对指定位置上的风矢进行了计算,并进行了各处理环节的有效性分析。
    针对问题3,根据图像窗口匹配和特征匹配两种匹配类型,分别采用基于灰度梯度
    的风矢场模型和基于图像边缘特征提取的风矢场模型进行建模。采用Sobel 算子对图像
    灰度梯度分类。从而自适应确定窗口大小。
    针对问题4,以16×16 像素块为计算单位,对各像素点的温度值进行再处理。对各
    云块进行温度直方图分析,并计算能够表征云块整体云顶特征的加权平均温度,然后对
    云块内所有像素点进行温度调整,在通过对照温度数值预报文件,得出各风矢点的等压
    面。
    关键词:坐标变换,相关系数法,风矢计算,自适应窗口,灰度梯度
\end{abstract}



\tableofcontents

\section{一、问题重述}

大气风场资料的获取对于气象预测等社会生产生活以及灾害性天气学研究等都有着重要意义。在气象卫星投入使用之前,获取风场信息的主要手段是陆地观测站网的建设,而对于海洋、高原、沙漠等测站较稀少的地区,风场资料的获取存在困难。自上世纪 70 年代以来,随着气象卫星的不断发射和相关探测技术的发展,各国逐渐开展了通过卫星云图反演云迹风的工作。由静止气象卫星图像反演出的云迹风范围广泛、质量较好,是对常规观测风场资料的重要补充。

我国的卫星气象部门目前主要通过风云系列卫星传回的红外云图和水汽图像来推导云迹风。在本问题中,主要讨论红外云图推导云迹风的过程,试图利用红外卫星图像数据及其特点建立尽可能准确地描述实际风矢场的度量模型和算法,并推算各云团的高度(以等压面表示),最终画出大气运动矢量图。

气象云图(红外云图)由地球同步卫星上的红外探测仪探测地球上空的温度数据再转换成灰度数据制作而成。红外探测仪扫描采样时,按步进角(南北方向)和行扫描角(东西方向)均为 140 微弧采样。如果天空无云,探测到的温度可以看成是地球表面的温度;在有云层的地方,探测到的温度相对较低,且云层越高越厚温度就越低,探测到的温度可看成云层所在区域的温度。已知同步卫星离地球中心的高度为 42164000 米,星下点在东经 86.5 度,北纬 0 度。题目给出三个不同时刻 $2288 \times 2288$ 的灰度值矩阵,矩阵的每个元素都对应地球上或地球外的一个探测点(或称采样点),还给出了海岸线经纬度坐标数据文件、灰度与温度转换文件以及全球温度数值预报文件。

解决本问题需要进行的主要工作有:

\begin{enumerate}
    \item 换算视场坐标。给出灰度矩阵元素行列号对应于经纬度坐标的换算公式,建立矩阵形式的经纬度坐标文本文件。
    \item 海岸线的绘制。绘制卫星云图并添加海岸线。
    \item 云迹风计算。根据三个相邻时刻的卫星数据,分别用两种方法建立 21:00 时刻风矢场的度量模型,计算给定范围内整数经纬度处的风矢量。
    \item 计算非零风矢所在等压面。根据全球温度数值预报文件计算非零风矢所在的等压面,以该时刻该处大气的压强值表示。
\end{enumerate}

\section{二、问题分析}

本问题主要涉及数据处理、坐标变换、图像处理、度量模型的建立、作图等内容。要根据不同时刻采样点灰度数据计算出风矢并绘制出云导风图,主要需要进行以下几步的工作:

\subsection{1. 预处理}

首先,要对题目所给的数据进行相关处理以便于解题。主要包括数据降噪和坐标变换。由于灰度矩阵是卫星通过旋转固定探测角度采样得到的,而要在地球球面上绘制海岸线以及进行云块的匹配分析,就需要进行视场坐标的转换,确定灰度矩阵中每个元素对应的采样点在地球上的经纬度。为此,可将地球视为理想椭球,这个椭球由地球的一个经过南北极的椭圆截面绕南北极的连线旋转而得。然后建立空间坐标系,对地球及卫星所在空间进行建模,计算出灰度矩阵行列号对应于经纬度的推导公式,将灰度矩阵中的数据投影到以经纬度为坐标的平面上,利用等距节点插值方法得到一定分辨率下各像素点的灰度值,形成该投影平面下的灰度矩阵。

\subsection{2. 风矢的计算}

建立风矢场的度量模型。由于我们计算风矢的唯一依据是卫星云图,因此,要确定某一点在某时刻的风矢,就要确定该点所在云块在该时刻的移动方向和移动速度。而云块的移动可以通过对比该云块在两个不同时刻卫星云图的位置来判断。首先在一张云图上提取目标云块的特征信息,然后在下一时刻云图上在一定范围内搜索匹配该云块特征的区域,记录该位置,即可求出云块在该时段的移动方向和移动速度。在这个过程中,需要进行图像预处理、云块识别、云块特征提取、匹配分析、质量检测、球面距离计算、角度计算等工作。

\subsection{3. 确定风矢所在等压面}

等压面是由对照温度数值预报场数据得到的。先将各点灰度值转换为温度值,然后对风矢点所在云块进行温度的直方图统计,求出该云块的加权平均温度,对云块内所有点进行温度调整,再对照数值预报文件确定等压面。

\subsection{4. 卫星云图海岸线和云导风的绘制}

根据海岸线经纬度数据和各非零风矢的经纬度、速度、角度和等压面数据,通过坐标转换模型,对应海岸线各点和各风矢在云图中的位置,即灰度矩阵中的行列数,再利用 MATLAB 进行绘图。

\section{三、模型假设与符号说明}

假设一:忽略同步卫星轨道的偏移,即忽略轨道高度与卫星静止高度的偏差和轨道平面与赤道平面的交角,假设卫星距星下点距离不变,并始终保持在北纬 \(0^\circ\) 东经 \(86.5^\circ\)。

假设二:假设地球为理想椭球,其由一个经过南北极的椭圆截面绕南北极的连线旋转而得,并且南北极轴没有偏角。

假设三:灰度值小于一定阈值的像素点认为位于晴空区,无云层覆盖,探测到的温度为地表温度;大于该阈值的像素点位于云层覆盖区域,探测到的温度为云层所在区域的温度。

假设四:假设第二问风矢模型中的窗口大小足以区分窗口内外的图像特征,并且在研究时段目标云块内始终不超出搜索范围。

假设五:用标准差表示灰度的均匀程度,如果 \(75\%\) 以上灰度小块的标准差小于某个阈值,认为由这些灰度小块组成的区域不适合云迹风反演。

假设六:对灰度梯度进行分段,认为位于同一梯度段内的像素点都处于云块内部或云块边缘或地表。

\subsection{名词解释}

经度:通过某地的经线面与本初子午面所成的二面角。

纬度:指某点与地球球心的连线和地球赤道面所成的线面角。

\subsection{符号说明}

\begin{itemize}
    \item $a$: 椭球面的半轴,就是椭球面与 $x$ 轴正方向交点。为赤道半径,为 $6378136.5 \, \text{m}$。
    \item $b$: 椭球面的半轴,就是椭球面与 $y$ 轴正方向交点。为赤道半径,为 $6378136.5 \, \text{m}$。
    \item $c$: 椭球面的半轴,就是椭球面与 $z$ 轴正方向交点。为极半径,为 $6356751.8 \, \text{m}$。
\end{itemize}

\section{四、模型的建立与求解}

\subsection{4.1 问题一:视场坐标转换模型的建立与数据处理}

\subsubsection{4.1.1 理论分析}

灰度矩阵每一行和每一列分别是红外探测仪扫描按步进角 $140$ 微弧($1$ 弧度 $= 1000000$ 微弧)分别从北到南或从东到西采样得到的数据,根据地球平面的立体型,灰度矩阵的每一行采样点并不处于相同的纬度上,同理,灰度矩阵的每一列采样点并不处于相同的经度上,为了便于研究工作中需要数字化处理二维云图上的经纬度等等,都需要把卫星形成在二维平面的视经纬坐标转换成平面直角坐标,或者把平面二维直角坐标转换成平面视地理坐标系的经纬度,以方便研究和数字化处理[1]。

本题根据同步气象卫星的视成像原理,导出了同步气象卫星云图像像素平面直角坐标与视地理坐标的转换关系,建立起一种像素地理坐标定位的理论模型。

目标是由灰度矩阵坐标即灰度矩阵行号 $i$ 和列号 $j$ 组成的坐标 $(i, j)$ 按上北下南、左西右东的地图规则换算成相应的经纬度坐标 $(\text{lon}, \text{lat})$。

\subsubsection{4.1.2 像素平面坐标与经纬度坐标转化}

以地球中心为原点,赤道平面为 $x$-$y$ 平面,地球中心与卫星所在直线为 $y$ 轴,南北极点所在直线为 $z$ 轴,建立空间坐标系,如下图:

\begin{figure}[h]
    \centering
    \includegraphics[width=0.8\textwidth]{space_projection_model.png}
    \caption{空间坐标系下探测卫星在地球的投影模型}
\end{figure}

符号说明如下:

\begin{itemize}
    \item $O$ 为地球中心;
    \item $M$ 为卫星所在位置;
    \item $N$ 为采样点所在位置,其坐标为 $(x, y, z)$;
    \item $\theta_i$ 为红外探测仪在南北方向上转过的角度;
    \item $\theta_j$ 为红外探测仪在东西方向上转过的角度;
    \item $\alpha_{ij}$ 灰度矩阵中第 $i$ 行第 $j$ 列对应采样点在地球的纬度;
    \item $\beta_{ij}$ 灰度矩阵中第 $i$ 行第 $j$ 列对应采样点在地球的经度;
\end{itemize}

每一个子午圈平面都是椭圆,长半轴 $a$ 是赤道平面半径,短半轴 $c$ 是南北极轴上的地球半径。

由于星下点所在的灰度坐标为 $(1145, 1145)$,按照左西右东,上北下南的地理规则,$x, y, z$ 正方向围成的空间的灰度矩阵坐标满足 $0 \leq i < 1145$, $0 \leq j < 1145$。

以该四分之一的灰度矩阵坐标平面分析。

由于红外探测仪采样的步进角和行扫描角都为 140 微弧,也即灰度矩阵中相邻的两行和两列之间都是以 140 微弧相隔。同一行对应相同的 $\theta_i$,同一列对应相同的 $\theta_j$。可以得到:

\begin{align*}
\text{lat}(i, 2290 - j) &= \text{lat}(i, j); \\
\text{lon}(i, 2290 - j) &= 86.5 + (86.5 - \text{lon}(i, j));
\end{align*}

设 $(1, m, n)$ 为采样点 $N$ 与球心 $O$ 的连线 $ON$ 的方向向量。方向沿 $x, y, z$ 轴正方向由球心指向球外。

\begin{align*}
\text{lat}(2290 - i, j) &= -\text{lat}(i, j); \\
\text{lon}(2290 - i, j) &= \text{lon}(i, j);
\end{align*}

根据空间直线的点向式方程:

\begin{align*}
lat(2290-i, 2290-j) &= -lat(i, j); \\
lon(2290-i, 2290-j) &= 86.5 + (86.5 - lon(i, j));
\end{align*}

结合椭球标准方程

\begin{align*}
\overline{f} &= \frac{1}{N^2} \sum_i \sum_j f(i, j) \text{ 为像素块的平均灰度,其中取 } N=16, \\
\overline{g} &= \frac{1}{N^2} \sum_i \sum_j g(i+m, j+n) \text{ 为目标搜索像素块的平均灰度}
\end{align*}

列出上述六个方程中,包含 $m, n, x, y, z$ 五个变量。求解步骤如下:

(1) 由公式 (1.2) 解得

\begin{equation}
m = -\sqrt{\frac{1 - \tan^2\theta_j \tan^2\theta_i}{\tan^2\theta_j \tan^2\theta_i + \tan^2\theta_j}}
\tag{4.1.5}
\end{equation}

(2) 将 \( m \) 带入式 (1.2) 解得

\begin{equation}
n = \frac{\tan\theta_i}{\tan\theta_j} \sqrt{\frac{\tan^2\theta_j + 1}{\tan^2\theta_i + 1}}
\tag{4.1.6}
\end{equation}

(3) 将 (1.3) 中的 \( x \) 用 \( y \) 和 \( m \) 表示,\( z \) 用 \( y \) 和 \( n \) 表示,带入方程 (1.3) 可得

\begin{equation}
\frac{(y - H)^2}{m^2 a^2} + \frac{y^2}{b^2} + \frac{n^2}{m^2 c^2} (y - H)^2 = 1
\tag{4.1.7}
\end{equation}

化简

\begin{equation}
\left( \frac{1}{m^2 a^2} + \frac{n^2}{m^2 c^2} \right) (y - H)^2 + \frac{y^2}{b^2} - 1 = 0
\tag{4.1.8}
\end{equation}

得到关于 \( y \) 的一元二次方程

\begin{equation}
\left( \frac{1}{m^2 a^2} + \frac{n^2}{m^2 c^2} + \frac{1}{b^2} \right) y^2 - 2H \left( \frac{1}{m^2 a^2} + \frac{n^2}{m^2 c^2} \right) y + H^2 \left( \frac{1}{m^2 a^2} + \frac{n^2}{m^2 c^2} \right) - 1 = 0
\tag{4.1.9}
\end{equation}

(4) 求解式 (1.9) 中的 \( y \),将 \( m \),\( n \),\( y \) 带入式 (1.3) 求得 \( x \) 与 \( z \)。

(5) 根据空间直角坐标 \((x, y, z)\) 可得到经纬度坐标公式如下:

\begin{equation}
\alpha_{ij} = \arctan\left( \frac{z}{\sqrt{x^2 + y^2}} \right)
\tag{4.1.10}
\end{equation}

\begin{equation}
\beta_{ij} = \arctan\left( \frac{x}{y} \right)
\tag{4.1.11}
\end{equation}

剩余四分之三灰度矩阵对应的经纬度 \((\alpha_{ij}, \beta_{ij})\) 可由该四分之一灰度矩阵的坐标按对称等原则对应得到。同一行上左右四分之一球面的纬度是关于东经 86.5 度经线对称的。经度则按照一定原则以东经 86.5 度为中心,从左向右递增,具体公式如下。

右上四分之一球面的经纬度:

\begin{align*}
\text{lat}(i, 2290 - j) &= \text{lat}(i, j); \\
\text{lon}(i, 2290 - j) &= 86.5 + (86.5 - \text{lon}(i, j));
\end{align*}

左下四分之一球面的经纬度:

\begin{align*}
\text{lat}(2290 - i, j) &= -\text{lat}(i, j); \\
\text{lon}(2290 - i, j) &= \text{lon}(i, j);
\end{align*}

右下四分之一球面的经纬度:

\begin{align*}
lat(2290-i, 2290-j) &= -lat(i, j); \\
lon(2290-i, 2290-j) &= 86.5 + (86.5 - lon(i, j));
\end{align*}

其中 $lat(i, j)$ 表示灰度矩阵中 $(i, j)$ 元素对应的纬度。$lon(i, j)$ 表示灰度矩阵中 $(i, j)$ 元素对应的经度。

\subsubsection{4.1.3 坐标文本输出与交叉点经纬度记录}

MATLAB 采用 fprintf 语句将经纬度数据按格式要求写入文本文件。

\begin{table}[h]
\centering
\caption{第500行与第500、501、502列交叉处的经纬度}
\begin{tabular}{c c c}
\hline
(行, 列) & 经度 & 纬度 \\
\hline
(500, 500) & 46.1511E & 32.9206N \\
(500, 501) & 46.2371E & 32.9148N \\
(500, 502) & 46.3229E & 32.9090N \\
\hline
\end{tabular}
\end{table}

注:E 表示东经,N 表示北纬。

\subsubsection{4.1.4 数据降噪处理并绘制卫星云图及海岸线}

数据一幅卫星云图通常会存在一些干扰和噪声,这些干扰和噪声无疑会降低云图质量,影响云团属性和云团移动的判断。平滑(低通)滤波器在一定程度上能削弱这种干扰。平滑滤波器能减弱或消除傅里叶空间的高频分量,但不影响低频分量。对应图像中的区域边缘等灰度值具有较大较快变化的部器将这些分量滤去可使图像平滑。空间滤波器的工作原理都可借助频域进行分析。它们的基本特图像在傅里叶空间某个范围内的分量受到抑制而让其它分量不受影而改变输出图像的频率分布,达到增强或抑制噪声的目的。

本文采用 9 点均值平滑法对云图进行平滑滤波预处理,旨在消除云图中的噪声干扰。所取均值平滑公式为

\begin{equation}
x_{m,n} = \frac{1}{(2k+1)(2k+1)} \sum_{i=-k}^{k} \sum_{j=-k}^{k} x(i+m, j+n)
\tag{4.1.12}
\end{equation}

式中,k 决定了 $(m, n)$ 点邻域大小,$k=1$ 时为 $3 \times 3$,$k=2$ 时为 $5 \times 5$,k 较大时,会使得图像边缘信息变得模糊,为了尽量保持边缘信息,取 $k=1$。

将灰度矩阵进行归一化处理,化成 $[0, 1]$ 矩阵,调用 MATLAB 中的 imshow 函数进行灰度图绘制。将地球外区域的灰度值化为 1,以白色显示。

由于海岸线给出的是经纬度坐标,因此要将海岸线的经纬度坐标要转换到灰度坐标从而和卫星云图以相同坐标轴绘图。即 4.1.2 节中的反过程,先将经纬度的角度坐标利用空间坐标系转换成 $(x, y, z)$ 坐标,再由 $(x, y, z)$ 坐标转换 $m$ 和 $n$ 的值,由 $m, n$ 再反推得到 $\theta_i$ 和 $\theta_j$,从而得到对应的灰度坐标 $(i, j)$。

添加海岸线的气象云图如图 2 所示。

\begin{figure}[h]
    \centering
    \includegraphics[width=0.8\textwidth]{earth_with_coastlines.png}
    \caption{添加海岸线的卫星云}
    \label{fig:earth_with_coastlines}
\end{figure}

\subsection{4.2 问题二:基于相关系数法的风矢场度量模型的建立与求解}

\subsubsection{4.2.1 理论分析}

风矢场的建模主要分为示踪云选取,示踪云追踪,风矢计算,质量检测几个步骤。其中示踪云的选取要考虑海洋、大陆及云层的划分,采用相关系数等方法进行最大相似度检验,根据球面三角形的几何原理确定风矢的角度和在球面上的实际位移量。在失踪云选取中要合理分辨目标物[2],可以采用分类、聚类等思想,通过设定响应阈值进行分类的判断。为了降低采样干扰,减小随机误差,对图像进行降噪处理。针对本问题提出的固定窗口和搜索范围,假设目标云块在研究时段内始终在搜索范围以内,窗口大小能反映目标块特征。

\subsubsection{4.2.2 投影变换及降噪处理}

在建立了坐标转换模型后,可以将原始灰度矩阵转换为经纬度坐标下的灰度矩阵。

并针对灰度数据中可能出现的干扰和噪音,对数据进行降噪处理。

要对不同时刻卫星云图间进行窗口块特征匹配,首先要明确一个概念:云块是覆盖在地球球面上的,需要在近似相同的球面面积上对云块进行匹配,也就是说,在进行特征匹配前,需要首先将灰度矩阵投影到平面的经纬度坐标下。可以利用问题一中建立的转换模型进行坐标变换。

由于卫星探测仪以固定角度进行扫描,同样角度在星下点扫过的距离小,在地球边缘扫过的距离大,因此,在进行投影后,原始采样点在经纬度坐标下的分布会呈现明显的不均匀状,中间密,四周稀疏,显然不再适合选取作为像素点。本题要求选用的窗口大小为 \(16 \times 16\) 像素,搜索范围为 \(64 \times 64\) 像素,要求计算北纬 40 度到南纬 40 度、东经 46 度到 126 度范围内的风矢,为了保证计算精度以及与灰度矩阵相似的分辨率,选取以 0.05 度间隔的经纬度点为单像素点,选取北纬 45 度到南纬 45 度、东经 40 度到 130 度为计算范围,这样就在经纬度坐标下建立了一个 \(1801 \times 1801\) 的像素矩阵,三个时刻卫星云图分别对应为 \(aa\) 矩阵、\(bb\) 矩阵和 \(cc\) 矩阵。

接下来进行矩阵的灰度赋值。由于矩阵各像素点无法对应原始灰度矩阵中的采样点,这时需要使用等距节点插值方法,对各像素点进行灰度的赋值。具体实现方法为:通过灰度矩阵行列号与经纬度的换算公式,进行反推,求得经纬度对应的行值和列值(为非整数),根据整数行列号上的灰度值,调用 MATLAB 的二维插值函数 interp2,求得该像素点的灰度值。

考虑到实际中 \(1801 \times 1801\) 次调用插值函数耗时很大,也可采用简化办法,即把求得的非整数行列值按四舍五入处理,得到整数行列号,然后进行赋值。由于像素矩阵密度明显大于原始灰度矩阵,故这样处理并不会产生较大误差。

数据降噪处理采用 4.1.4 中采用的均值平滑滤波方法。

数据预处理后,可得对应不同时刻的三个坐标平面灰度矩阵,下一步风矢计算将以这三个矩阵数据为依据。使用 MATLAB 中的 imshow 函数将其中一个灰度矩阵画出云图,如图 3 所示:

\begin{figure}[h]
    \centering
    \includegraphics[width=\textwidth]{image.png}
    \caption{云图示例}
    \label{fig:cloud_map}
\end{figure}

\begin{figure}[h]
    \centering
    \includegraphics[width=\textwidth]{image.png}
    \caption{平面经纬度坐标下所选范围内灰度矩阵生成的云图}
    \label{fig:cloud_map}
\end{figure}

\subsubsection{4.2.3 度量模型的建立与求解}

对两个目标求取相关系数是进行目标匹配简便而有效的方法,故本题选择使用相关系数法进行窗口云块的匹配分析。

求解可使用的数据有 8:30、9:00 和 9:30 三个时刻的灰度矩阵,而要求取的是 9:00 时刻的风矢场,故首先选取 9:00 和 9:30 两个时刻的云图进行相关性匹配,然后利用 8:30 的云图对匹配结果进行连续性检验。

题目要求计算范围为北纬 40 度到南纬 40 度、东经 46 度到 126 度,将在此范围内的 bb 矩阵的行列数按 \(16 \times 16\) 像素进行划分,因此,bb 矩阵中的像素块数为 \(109 \times 109\) 个,再对每个像素块的中心对应位置在 cc 矩阵中划出 \(64 \times 64\) 像素的搜索范围。

海洋面识别:由于海洋与云块有着明显不同的灰度值,因此可以预先剔除无云的海洋面,设灰度值小于 512 的像素点为无云点,处理方法如下:对每个像素块计算灰度值大于 512 的像素个数,若个数小于像素点总个数(\(16 \times 16\))的 25\%,则认为该像素块为无云块,赋该像素块内所有像素点的风矢为零。

相关系数计算:计算 bb 矩阵中各像素块与 cc 矩阵中搜索范围内所有 \(16 \times 16\) 像素块的相关系数,计算公式如下:

\begin{equation}
R(m, n) = \frac{\sum_{i}\sum_{j} \left[ f(i, j) - \overline{f} \right] \left[ g(i+m, j+n) - \overline{g} \right]}{\left[ \sum_{i}\sum_{j} \left[ f(i, j) - \overline{f} \right]^2 \sum_{i}\sum_{j} \left[ g(i+m, j+n) - \overline{g} \right]^2 \right]^{1/2}}
\tag{4.2.1}
\end{equation}

其中,$i, j \in [1, 16]$ 分别为像素块中的行号与列号,
$m, n \in [0, 64]$ 为目标搜索像素块的位移改变量,
$f(i, j)$ 为像素块内对应点的灰度值,
$g(i, j)$ 为目标搜索区内对应点的的灰度值,

\begin{align*}
\overline{f} &= \frac{1}{N^2} \sum_i \sum_j f(i, j) \text{ 为像素块的平均灰度,其中取 } N=16, \\
\overline{g} &= \frac{1}{N^2} \sum_i \sum_j g(i+m, j+n) \text{ 为目标搜索像素块的平均灰度}
\end{align*}

取相关系数最大值对应的像素块为半小时后移动到的位置,并记录位移量(行位移量和列位移量)。此处增加相关系数阈值判断条件:若此相关系数最大值小于 0.5,则认为这两个像素块之间不能匹配,赋该像素块内所有像素点的风矢为零。

匹配面质量检测与连续性检测:在部分情况下,匹配面可能是多峰值并存的状态,无法判断匹配的像素块。这时需要对最大峰值和次大峰值进行匹配面质量检测和连续性检测 [3]。首先根据主峰与次峰的距离 $D$ 及相关系数差值 $R$ 的大小做匹配面检测。若通过检测,再进行连续性检验。设 $bb$ 矩阵中像素块位置为 $B$,$cc$ 矩阵中两个峰值对应像素块位置为 $C1$、$C2$,用向量 $BC1$ 和 $BC2$ 的反向量 $BA1$ 和 $BA2$ 得到在相应的 $aa$ 矩阵(即 $A$ 云图)所在位置处,计算其与 $bb$ 矩阵中目标像素块的相关系数 $R1$ 和 $R2$。计算这 2 个相关系数的差,若大于给定阈值,再判别 $R1$ 和 $R2$ 中最大者是否大于给定阈值。若大于给定阈值,则通过连续性检测,并选定最大相关系数处为最佳匹配像素块[]。

风矢计算:由两个匹配像素块间的位移量,计算它们之间的经纬度之差。然后通过球面三角形的余弦定理,计算出两像素块之间的球面弧长和角度(从正北方顺时针方向量得的角度)[4]。

\begin{figure}[h]
\centering
\includegraphics[width=0.8\textwidth]{image.png}
\caption{地球球面上 A、B 两点之间的位移示意图}
\end{figure}

球面三角形的余弦定理如下:

\begin{equation}
\cos \gamma = \cos(90^\circ - \varphi_A) \cos(90^\circ - \varphi_B) + \sin(90^\circ - \varphi_A) \sin(90^\circ - \varphi_B) \cos N \tag{4.2.2}
\end{equation}

其中,$\gamma$ 为球面角,

\begin{itemize}
    \item $A$ 为目标像素块位置,
    \item $B$ 为匹配像素块位置,
    \item $\varphi_A$ 为 $A$ 点纬度,
    \item $\varphi_B$ 为 $B$ 点纬度,
    \item $N$ 为极点处与 $A$、$B$ 对应球面角,其值为 $\left|\alpha_A - \alpha_B\right|$(经度差)。
\end{itemize}

由公式可得弧长,同样运用球面三角形余弦定理,也可求得角度值。

\textbf{计算结果:}

在本题目建模与求解中,对像素块和相关系数以及匹配块的搜索都进行了相关处理,以求得到更为精确的结果。下面给出采取不同处理措施时求得的给定位置的风矢量。进行相关系数阈值检测、匹配画面质量检测和连续性检测后非零风矢为 1526 个。

\textbf{(1) 以 8 像素为间隔在 $64 \times 64$ 范围内搜索 $16 \times 16$ 像素块,不进行相关系数阈值检测、匹配面质量检测和连续性检测:}

\begin{tabular}{cccc}
纬度 & 经度 & 角度(北顺) & 速度(米/秒) \\
\hline
26 & 52 & 0 & 0 \\
26 & 53 & 0 & 0 \\
26 & 54 & 137.3852 & 54.237 \\
26 & 57 & 226.3088 & 86.575 \\
26 & 58 & 0 & 0 \\
\end{tabular}

\textbf{(2) 以 2 像素为间隔在 $64 \times 64$ 范围内搜索 $16 \times 16$ 像素块,不进行相关系数阈值检测、匹配面质量检测和连续性检测:}

\begin{tabular}{cccc}
纬度 & 经度 & 角度(北顺) & 速度(米/秒) \\
\hline
26 & 52 & 53.0486 & 8.3083 \\
26 & 53 & 0 & 0 \\
26 & 54 & 53.0255 & 16.6135 \\
26 & 57 & 227.6307 & 81.3327 \\
26 & 58 & 53.0486 & 8.3083 \\
\end{tabular}

对搜索进行加密后,得到的风矢结果发生了变化,可以肯定的是,搜索的密度越大,得到的结果的准确性就越高。

\textbf{(3) 以 2 像素为间隔在 $64 \times 64$ 范围内搜索 $16 \times 16$ 像素块,进行相关系数阈值检测,不进行匹配面质量检测和连续性检测:}

\begin{tabular}{cccc}
纬度 & 经度 & 角度(北顺) & 速度(米/秒) \\
\hline
\end{tabular}

\begin{tabular}{cccc}
26 & 52 & 53.0486 & 8.3083 \\
26 & 53 & 0 & 0 \\
26 & 54 & 53.0255 & 16.6135 \\
26 & 57 & 227.6307 & 81.3327 \\
26 & 58 & 53.0486 & 8.3083 \\
\end{tabular}

加入相关系数阈值检测后,能够剔除掉近 100 个相关系数较小的匹配像素块(阈值为 0.5),提高了结果准确度。

\textbf{(4) 以 2 像素为间隔在 $64 \times 64$ 范围内搜索 $16 \times 16$ 像素块,进行相关系数阈值检测、匹配面质量检测和连续性检测:}

\begin{tabular}{cccc}
纬度 & 经度 & 角度(北顺) & 速度(米/秒) \\
26 & 52 & 0 & 0 \\
26 & 53 & 0 & 0 \\
26 & 54 & 53.0255 & 16.6135 \\
26 & 57 & 227.6307 & 81.3327 \\
26 & 58 & 0 & 0 \\
\end{tabular}

加入匹配面质量检测和连续性检测,能够剔除掉相当部分的匹配像素块,提高了风矢场数据的质量。

对窗口大小和搜索范围大小限定的理解:传统图像区域匹配算法采用固定窗口和扫描范围的方法,假设窗口大小足以区分窗口内外的图像特征,并且在研究时段目标云块内始终不超出搜索范围。窗口和搜索范围不能无限制的增大或减小,窗口过小,在低纹理区域无法覆盖足够的灰度变化,会因灰度信息太少使得视差估值不准确;窗口过大,虽然可以包容足够的灰度变化,但包含视差的较大变化,会因为左右图像上不同的投影畸变,使得匹配位置不正确,且计算量大。因此限定合适的窗口和搜索范围具有实际的应用价值。

\subsection{4.3 问题三:基于灰度梯度和边缘特征提取的自适应窗口的风矢场度量模型的建立与求解}

\subsubsection{4.3.1 理论分析}

风矢场建模是图像匹配的过程。图像匹配主要分为基于窗口的匹配和基于特征的匹配两种方法。

图像特征匹配是指通过分别提取两个或多个图像的特征(点、线、面等特征),对特征进行参数描述,然后运用所描述的参数来进行匹配的一种算法。通常使用的特征基元有点特征、边缘特征和区域特征。特征匹配需要用到许多诸如矩阵的运算、梯度的求解、还有傅立叶变换和泰勒展开等数学运算。特征点匹配算法可以分成三步:特征点抽取,特征点匹配和消除不良匹配点。基于对特征点定义的不同,已经出现了不少特征点抽取算法。这些算法大致可以分成三类:(1)使用非线性滤波的方法,如 Smith 提出的 SUSAN 角点检测法,(2) 基于曲率的方法 需要先进行边界抽取然后用曲率信息来找出特征点,(3) 利用图像的灰度变化率来确定角点,如 HarriS 等使用图像的一阶差分,计算每个象素处的平均平方梯度矩阵( average Square gradient matrix) 通过特征值分析给出角点响应。

传统图像区域匹配算法采用固定窗口和扫描范围的方法,假设窗口大小足以区分窗口内外的图像特征,并且在研究时段目标云块内始终不超出搜索范围。但实际进行风矢场建模时,窗口和搜索范围的选定对计算速度和精度都有影响。窗口越小,在视差不连续区域匹配精度越高,但在低纹理区域没有覆盖足够的灰度变化,会因灰度信息太少使得视差估值不准确; 窗口越大,可以包容足够的灰度变化,对于低纹理区域匹配精度越高,但窗口太大且包含视差的较大变化,会因为左右图像上不同的投影畸变,使得匹配位置不正确,且计算量大。因而利用自适应窗口进行匹配是非常有必要的。其中基于图像特征的匹配包括基于灰度特征的匹配和基于几何特征的匹配。

本文提出了基于 Sobel 梯度算子进行梯度分类的自适应窗口确定方法。

本文分别采用这两类图像匹配方法进行风矢场的模型,灰度匹配模型采用基于灰度梯度的风矢模型,几何特征匹配方法采用基于轮廓特征的匹配算法。将自适应窗口应用到这两种模型里。

\subsubsection{4.3.2 基于灰度梯度的风矢度量模型}

因为灰度是通过观测得到的离散数据,传统的是最大交叉相关系数 (Maximum Cross-correlation Coefficient, MCC) 法仅仅利用了图像的灰度信息,没有充分利用图像灰度的梯度信息,而灰度梯度信息会直接影响到图像的相似性。

众所周知灰度的梯度计算 (亦即观测资料求导数) 在数学上是一个不适定的问题。此问题,通过离散点上观测值,要寻找一个函数,使得它与观测点上观测值充分接近,同时其导数与真实函数的导数要有足够的精度。显然,当离散数据不存在数据误差时,这一问题一般用有限差分法就可求得近似导数,并且其收敛性亦得到保证,但当离散数据存在误差时,用有限差分法求近似导数会产生很大误差。本文采用了数值微分方法,该方法所得到的解是一个分片三次样条函数,此解的导数与真实解的导数具有足够高的精度,实施方案如下:

设 \( y = f(x) \) 是定义在 \([0, 1]\) 上的一个函数,记 \(\Delta = \{0 = x_0 < x_1 < \ldots < x_n = 1\}\) 是 \([0, 1]\) 上的一个划分,\( h_j = x_{j+1} - x_j \),\( h = \max(h_j), (j = 1, 2, \ldots, n-1) \),若给定函数在点 \( x_j \) 处的观测值 \( y_i' \),用 \(\delta\) 表示观测数据的观测误差,则

\[
\left| y_j - y(x_j) \right| \leq \delta, j = 1, 2, \ldots, n
\]

数值微分方法是寻找函数 \( f^*(x) \),使得 \( f^*(x) \) 的导数与真实函数 \( y(x) \) 的导数充分接近,于是构造泛函 \( J \) 如下:

\[
J(f) = \sum_{j=1}^{n-1} \frac{h_j + h_{j+1}}{2} (\widetilde{y}_j - f(x_j))^2 + \alpha \left| \frac{\mathrm{d}^2 f}{\mathrm{d}x^2} \right|_{L^2(0,1)}^2 = \min!
\]

这里是正则化参数。则重构图像的函数 \( f^* \) 是一个分段的三次样条函数且一阶微分是连续的,在子区间 \([x_j, x_{j+1}]\) 上其表达式为

\begin{equation}
f^*(x) = a_j + b_j(x - x_j) + c_j(x - x_j)^2 + d_j(x - x_j)^3, \, j = 0, 1, \ldots, n-1.
\tag{4.3.3}
\end{equation}

其一阶导数为

\begin{equation}
f^*(x) = b_j + 2c_j(x - x_j) + 3d_j(x - x_j)^2, \quad x \in [x_j - x_{j+1}],
\tag{4.3.4}
\end{equation}

\begin{equation}
R_2(m, n) = \frac{\sum\limits_i \sum\limits_j \left[ f_x^*(i, j) - \overline{f_x^*} \right] \left[ g_x^*(i+m, j+n) - \overline{g_x^*} \right]}{\left[ \sum\limits_i \sum\limits_j \left[ f_x^*(i, j) - \overline{f_x^*} \right]^2 \sum\limits_i \sum\limits_j \left[ g_x^*(i+m, j+n) - \overline{g_x^*} \right]^2 \right]^{1/2}}
\tag{4.3.5}
\end{equation}

\begin{equation}
R_3(m, n) = \frac{\sum\limits_i \sum\limits_j \left[ f_y^*(i, j) - \overline{f_y^*} \right] \left[ g_y^*(i+m, j+n) - \overline{g_y^*} \right]}{\left[ \sum\limits_i \sum\limits_j \left[ f_y^*(i, j) - \overline{f_y^*} \right]^2 \sum\limits_i \sum\limits_j \left[ g_y^*(i+m, j+n) - \overline{g_y^*} \right]^2 \right]^{1/2}}
\tag{4.3.6}
\end{equation}

其中 $f_x^*(i, j)$ 为示踪云模板在经度方向的梯度,$g_x^*(i, j)$ 为目标搜索区在经度方向的梯度,$f_y^*(i, j)$ 为示踪云模板在纬度方向的梯度,$g_y^*(i, j)$ 为目标搜索区在纬度方向的梯度,$\overline{f_x^*}$,$\overline{f_y^*}$,$\overline{g_x^*}$,$\overline{g_y^*}$ 的定义与 $\overline{f}$,$\overline{g}$ 的定义类似,表示梯度的平均值。计算 $R_1(m, n) + R_2(m, n) + R_3(m, n)$ 最大对应的区域即为目标云区域的位置。

\subsubsection{4.3.3 基于边缘特征的风矢度量模型}

图像的边缘是图像的最基本特征。所谓边缘是指其周围像素灰度有阶跃变化或屋顶变化的那些像素的集合。边缘广泛存在于物体与背景之间、物体与物体之间、基元与基元之间。由于噪声和模糊的存在,检测到的边界可能会变宽或在某些点处发生间断,因此边缘提取的任务首先要检出图像局部特性的不连续性,然后剔出某些边界点或填补边界间断点,并将这些边缘像素连成完备的边界。边缘特性是沿边缘走向的像素变化平缓,而垂直于边缘方向的像素变化剧烈。因此根据像素变化的剧烈程度可以进行特征提取和判别。

图像边缘检测主要分为以下四步:

(1)滤波。边缘检测主要基于导数计算,但受噪声影响。但滤波器在降低噪声的同时也导致边缘强度的损失。

(2)增强。增强算法将领域中灰度有显著变化的点突出显示。一般通过计算梯度幅值完成。

(3)检测。但在有些图像中梯度幅值较大的并不是边缘点。

(4)定位。精确确定边缘的位置。

本文采用 Sobel 算子进行边缘检测。Sobel 算子是图像中的重要算子,只要用作边缘检测,它是一种离散型差分算子,用来运算图像两素函数的梯度的近似值,在图像的任何一点是用此算子,将会产生对应的梯度或是其法矢量的核心公式,如果以 \( A \) 代表原始图像,\( G_x \) 及 \( G_y \) 分别代表经横向及纵向边缘检测的图像,其公式如下:

\[
G_x = \begin{bmatrix}
-1 & 0 & +1 \\
-2 & 0 & +2 \\
-1 & 0 & +1
\end{bmatrix} * A \quad \text{and} \quad G_y = \begin{bmatrix}
+1 & +2 & +1 \\
0 & 0 & 0 \\
-1 & -2 & -1
\end{bmatrix} * A
\tag{4.3.7}
\]

图像的每一个像素的横向及纵向梯度近似值可用以下的公式结合,来计算梯度的大小。

\[
G = \sqrt{G_x^2 + G_y^2}
\tag{4.3.8}
\]

梯度方向为:

\[
\theta = \arctan\left(\frac{G_y}{G_x}\right)
\tag{4.3.9}
\]

采用 Sobel 算子进行边缘检测的图像如图 5 所示。

采用补充边界断点后采用一定的轮廓跟踪算法,对图像的轮廓进行跟踪,通过跟踪找出图像中各个目标的位置和边界。然后,运用一些几何特征对每个对象进行几何特征的提取。然后让它们与模板图像进行特征匹配 [6]。分为如下几步:

(一)区域标记:区域标记就是把连续区域作同一个标记,常见的四邻域标记算法和八邻域标记算法。四邻域标记算法思想:

1) 判断此点四邻域中的最左,最上有没有点,如果都没有点,则表示一个新的区域的开始。

2) 如果此点四邻域中的最左有点,最上没有点,则标记此点为最左点的值;如果此点四邻域中的最左没有点,最上有点,则标记此点为最上点的值。

3) 如果此点四邻域中的最左有点,最上都有点,则标记此点为这两个中的最小的标记点,并修改大标记为小标记。

\begin{figure}[h]
    \centering
    \includegraphics[width=\textwidth]{image.png}
    \caption{边缘检测后的非连续二值矩阵做图}
    \label{fig:5}
\end{figure}

\subsubsection{八邻域标记算法思想}

1) 判断此点八邻域中的最左,左上,最上,上右点的情况。如果都没有点,则表示一个新的区域的开始。

2) 如果此点八邻域中的最左有点,上右都有点,则标记此点为这两个中的最小的标记点,并修改大标记为小标记。

3) 如果此点八邻域中的左上有点,上右都有点,则标记此点为这两个中的最小的标记点,并修改大标记为小标记。

4) 否则按照最左,左上,最上,上右的顺序,标记此点为四个中的一个。

\subsubsection{轮廓提取}

将灰度图像化为二值图像的轮廓提取算法非常简单,就是掏空内部点法,如果原图中有一点为黑点,且它的 8 个相邻点都是黑点时,此时该点为内部点,则将该点删掉。

\subsubsection{轮廓跟踪}

首先找到第一个边界像素点 \( p(x, y) \) 的。按照从右到左的顺序对二值图像进行扫描,找到第一个黑点是最右方的边界点记为 \( A \)。以这个边界点为开始,假设已经沿逆时针方向环绕整个图像一圈找到了所有的边界点。由于边界是连续的,所以每一个边界点都可以用这个边界点对前一个边界点所张的角度来表示。因此可以用以下“跟踪准则”:从第一个边界点开始,定义初始的搜索方向为沿上方;如果上方是黑点,则为边界点,否则搜索方向向逆时针方向旋转 45 度。这样一直找到第一个黑点为止。然后把这个黑点作为新的边界点,在当前搜索方向的基础上逆时针旋转 90 度,继续用同样的方法搜索下一个黑点,直到返回最初的边界点为止。

\subsubsection{4.3.4 基于灰度梯度的窗口及选取方法}

本文采用基于 Sobel 梯度算子的梯度分类进行窗口大小的确定,由上节可以知道,Sobel 梯度算子能够检测灰度梯度的变化幅度,通过不同位置的边缘检测后的二值矩阵进行梯度统计可以看出,在区域内梯度值非常小,多数集中在 50 以下,在只包含一个 1 元素的边缘区域,梯度值之总在 150 左右,而在包含多个 1 元素的边缘区域,梯度值大于 200,因此可以把梯度值划分成 $(0,100)$, $(100,200)$ 及 $>200$ 三个段,第一个段内由于梯度值很小,表明离边界很远,灰度比较均匀,因此可以采用较大的窗口,而在梯度值大于 200 的区域,由于在边界附近,特征变化明显,因此采用小窗口。本文窗口分别定义为 $8 \times 8$, $16 \times 16$, $32 \times 32$ 三个尺度的窗口。

\subsubsection{4.3.5 模型结果比较}

采用自适应窗口及固定 $64 \times 64$ 固定扫描范围,对两种模型进行计算结果如下:

基于灰度梯度的风矢度量模型,以 2 像素为间隔在 $64 \times 64$ 范围内自适应窗口搜索像素块,进行相关系数阈值检测、匹配面质量检测和连续性检测结果如下:

[TABLEENV:7]

基于边缘特征的风矢度量模型,以 2 像素为间隔在 $64 \times 64$ 范围内搜索 $16 \times 16$ 像素块,进行相关系数阈值检测、匹配面质量检测和连续性检测结果如下:

\begin{tabular}{cccc}
纬度 & 经度 & 角度(北顺) & 速度(米/秒) \\
\hline
26 & 52 & 0 & 0 \\
26 & 53 & 0 & 0 \\
26 & 54 & 55.3533 & 17.8386 \\
26 & 57 & 226.5347 & 79.6878 \\
26 & 58 & 0 & 0 \\
\end{tabular}

非零风矢个数比较:采用问题二模型进行相关系数阈值检测、匹配面质量检测和连续性检测计算出的非零风矢个数为 1526,而采用本节两种方法得出的非零风矢个数分别为 1201 和 1314。

虽然上述两种方法与第二问所用方法从风速与角度计算结果来看相差不大,但是非零风矢的个数却相差较大,采用灰度梯度的风矢度量模型所得的非零风速数量最少,说明该模型的判别条件更为明确严格。

模型计算速度比较:问题二采用方法的速度要明显快于本问题的方法,说明在精度和速度上存在必然的矛盾,要选择一种最合适的方法来兼顾计算效率和计算精度。

\subsubsection{4.3.6 云图数据量和云导风风矢场计算研究的建议}

\begin{enumerate}
    \item 数据量

    受光照、地表及云层光滑程度、探测误差等影响,红外探测仪的扫描数据存在一定的随机误差(干扰、噪声),要消除这些误差,可以利用多张近似相同时刻的时间求均值。因此必须提供多张非常接近该时刻的卫星云图。

    本文提供的是三组相邻半小时的灰度数据来计算云导风,对于云导风计算来说时间间隔过长,算出的风矢与实际值存在较大误差,可以缩小相邻图像的时间间隔。

    \item 云导风风矢场计算

    目前大多数的云导风计算都采用最大交叉相关系数法,确定搜索范围与扫描窗口,对目标云的追踪采用最大交叉相关系数法,选定固定窗口在制定搜索范围内滑动平移窗口进行相关系数比较。相关系数最大的一个目标块即为失踪云的终点位置。而相关系数的阈值是认为根据统计或经验制定的,必然存在误差。可以采用优化算法,如遗传算法、粒子群算法等来实现两个目标块的最佳匹配。
\end{enumerate}

\subsection{4.4 问题四:风矢等压面的求取}

\subsubsection{4.4.1 理论分析}

等压面是由对照温度数值预报场数据得到的。首先需要由灰度温度对照矩阵,求取 21:00 时刻投影平面上的温度矩阵。虽然矩阵上每个像素点都对应一个温度值,但单个像素点的温度值可能并不能代表该区域的压强水平,考虑到数据存在噪音以及个别数据的非一致性,在跟温度数值预报场文件进行比对前,需要对各像素点的温度数据进行处理。

\subsubsection{4.4.2 模型方法}

采取的办法是:仍然以 $16 \times 16$ 划分像素块,对每个像素块先将各点灰度值转换为温度值,然后对风矢点所在云块进行温度的直方图统计,求出该云块的加权平均温度,计算公式如下:

\begin{equation}
\bar{x} = \frac{\sum\limits_{i=T_{\min}}^{T_{\max}} f_i \cdot x_i}{\sum\limits_{i=T_{\min}}^{T_{\max}} f_i}
\tag{4.4.1}
\end{equation}

式中,$\bar{x}$ 为加权平均温度,

$T_{\min}$ 为选定像素块的最小亮温,

$T_{\max}$ 为选定像素块的最大亮温,

$x_i$ 为亮温,

$f_i$ 为 $x_i$ 值出现的频率。

该值代表了像素块的整体云顶特征。然后对云块内所有点进行温度调整,使像素块内所有点的温度都为加权平均值。最后对照数值预报文件,确定像素点的等压面。

\subsubsection{4.4.3 计算结果}

在指定位置的计算结果如下:

\begin{tabular}{c c c c c}
纬度 & 经度 & 角度(北顺) & 速度(米/秒) & 压强(毫巴) \\
\hline
26 & 52 & 0 & 0 & 0 \\
26 & 53 & 0 & 0 & 0 \\
26 & 54 & 53.0255 & 16.6135 & 700 \\
26 & 57 & 227.6307 & 81.3327 & 650 \\
26 & 58 & 0 & 0 & 0 \\
\end{tabular}

至此,题目要求范围内所有整数经纬度点的风矢及等压面都已求出。对有无质量检测和连续性检测下平面投影和原始图下的云导风图分别如图 6-9。见下页

\begin{figure}[h]
    \centering
    \includegraphics[width=\textwidth]{image.png}
    \caption{未进行质量检测和连续性检测,低密度搜索(平面投影)}
    \label{fig:6}
\end{figure}

\begin{figure}[h]
    \centering
    \includegraphics[width=\textwidth]{image_of_earth_with_vectors.png}
    \caption{未进行质量检测和连续性检测,低密度搜索(原始图)}
    \label{fig:earth_vectors}
\end{figure}

\begin{figure}[h]
    \centering
    \includegraphics[width=\textwidth]{image.png}
    \caption{进行质量检测和连续性检测,高密度搜索(平面投影)}
    \label{fig:8}
\end{figure}

\newpage
\begin{center}
    24
\end{center}

[FIGUREENV:10]

[REFERENCES:1]