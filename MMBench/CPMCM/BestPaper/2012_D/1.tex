\title{基于卫星云图的风矢场(云导风)度量模型与算法探讨}
\author{}
\date{}

\maketitle

\begin{abstract}
本文首先通过换算视场坐标确定灰度矩阵中每个元素对应的采样点在地球上的经纬度,从而将灰度矩阵转化为卫星云图,并添加海岸线。在此基础上,使用相关匹配法对具有一定时间间隔的两幅相关卫星云图进行模板匹配生成云导风矢量。然后,借助于近年来发展起来的数值微分方法,从图像灰度中提取出图像梯度信息,再利用正则化方法,实现了云导风的反演。对云图中加入灰度梯度信息和未加入灰度梯度信息的风场反演结果进行比较。结果表明,加入图像灰度梯度信息后所实施的新反演方法可有效减小图像干扰的影响,同时也大大提高了风矢量反演的精度,为卫星云图反演云导风探索出一条新路。最后,通过分析温度数据计算出所有非零风矢所在的等压面,不同高度的等压面图组合起来,不仅反映大范围地区高压和低压等天气系统的空间分布状况,而且能反映高空大气的结构及其演化规律。
\end{abstract}

\noindent\textbf{关键词:} 云导风;相关匹配法;灰度梯度信息;等压面;

\section{一、问题的重述}

卫星云图在掌握大气环流、中长期天气预报以及灾害性天气学的研究中有重要作用。它由地球同步卫星上的红外探测仪探测地球上空的温度数据再转换成灰度数据制作而成。红外探测仪扫描采样时,按步进角(南北方向)和行扫描角(东西方向)均为 140 微弧(1 弧度 = 1000000 微弧)采样。在卫星与地球中心的连线和地球表面的交点(称为星下点)处的分辨率大约是 5 公里。本题提供的卫星探测数据文件都是 2288×2288 的灰度值矩阵,矩阵的每个元素都对应地球上或地球外的一个探测点(或称采样点)。同步卫星离地球中心的高度为 42164000 米,星下点在东经 86.5 度,北纬 0 度,星下点对应的矩阵元素位于矩阵的第 1145 行和第 1145 列相交处。为解答本题,首先要确定灰度矩阵中每个元素对应的采样点在地球上的经纬度。据此就可以将灰度矩阵中非负元素的行列号按上北下南、左西右东的地图规则换算成地球上经纬度坐标,此结果既可用于估算各探测点之间的距离,还可用于在云图上依据海岸线经纬度坐标标出海岸线以方便看图。

观测大气环流情况的一个方法是在卫星云图上标出风矢。风矢的大小和方向由云块移动的速度决定。没有云或云块不稳定处的风矢规定为零风矢,这种用云块的移动所定义的风矢被称为云迹风。气象部门已经有一些方法根据变化的卫星云图计算云迹风,这类方法称为云导风方法。本题的主要目的是希望大家充分利用卫星图像数据及其特点建立尽可能准确地描述实际风矢场的度量模型和算法。题目提供了我国风云 2 号卫星获得的三个灰度矩阵,分别表示某天的 20:30, 21:00, 21:30 时刻红外探测仪探测到的地球上空的温度数据对应的灰度值。又给出了海岸线经纬度坐标数据文件,文件的第 1 列为经度(东经),第 2 列是纬度(北纬),每一行 2 个数据对应海岸线上一点,而特大数据(99999.99,99999.99)表示前一曲线已结束,将要开始下一曲线。

具体要求解决如下问题:

1. 换算视场坐标。给出灰度矩阵元素行列号对应于经纬度坐标的换算公式,建立矩阵形式的经纬度坐标文本文件;并在论文中给出第 500 行和第 500、501、502 列三个交叉点处对应的经纬度。进而将卫星灰度矩阵转化为卫星云图,并在这个卫星云图上添加海岸线。

2. 建立风矢场的度量模型。根据三个相邻时刻的卫星数据,用适当的方法建立 21:00 时刻风矢场的度量模型,给出计算方案。利用你们的方法计算 21:00 时刻在各个(整数)经纬度处的二维风矢。在论文中给出上面计算结果中全部非零风矢的个数,并给出在纬度 26 度,经度分别是 52,53,54,57,58(度)处的风矢。给出你们对上述窗口大小和搜索范围的限定的理解或评价。

3. 提高像素块匹配质量。取消问题 2 中窗口大小和搜索范围的限定,利用图像特征匹配等思想,设计能自适应地确定窗口大小和搜索范围的有效方法。进而建立风矢场的度量模型和算法。利用此模型和算法计算问题 2 中在纬度 26 度,经度分别是 52,53,54,57,58(度)处的风矢。分析比较建立的这两种风矢计算方法的优劣。提出关于云图数据量和云导风风矢场计算研究的建议。

4. 计算非零风矢所在的等压面。应用附件中全球温度数值预报文件计算问题 2 中非零风矢所在的等压面。并在论文中给出在纬度 26 度,经度分别是 52,53,54,57,58(度)处附近求出的风矢量所在的等压面。最后在问题 1 中画出的卫星云图上添加所计算出的非零风矢量结果。

\section{二、符号定义}

\begin{itemize}
    \item $R$ ——地球长轴半径(赤道半径)
    \item $r$ ——地球短轴半径(极半径)
    \item $\alpha$ ——步进角和行扫描角
    \item $D$ ——同步卫星离地球中心的高度
    \item $i$ ——偏移步进角的个数
    \item $j$ ——偏移行扫描角的个数
    \item $\psi$ ——经度
    \item $\theta$ ——纬度
    \item $S^{i,j}(m,n)$ ——被探测的目标区
    \item $T(m,n)$ ——样板
    \item $D(i,j)$ ——目标区上每点与样板上对应点的累计差值测度
    \item $L_D(i,j)$ ——阈值
    \item $\varepsilon$ ——误差值
    \item $T_k$ ——阈值
    \item $I(i,j)$ ——SSDA的检测曲面
    \item $R(i,j)$ ——交叉相关系数
    \item $R_{lim}$ ——阈值
    \item $(u,v)$ ——风矢分量
    \item $Vel$ ——风矢大小
    \item $Ang$ ——风向
    \item $\delta$ ——观测误差
    \item $f_x''(i,j)$ ——示踪云模板在经度方向的梯度
    \item $f_y''(i,j)$ ——示踪云模板在纬度方向的梯度
    \item $g_y''(i,j)$ ——目标搜索区在纬度方向的梯度
    \item $\delta^*$ ——观测灰度误差
    \item $\sigma_v$ ——风速均方根误差
    \item $\sigma_\theta$ ——风速均方根误差
    \item $v_1$ ——计算的风速
    \item $\theta_1$ ——计算的风向
    \item $N$ ——相同的风矢量个数
    \item $p$ ——等压面压强
    \item $T$ ——探测温度
\end{itemize}

\section{三、问题一模型的建立与求解}

\subsection{3.1 地球椭球体模型}

为了在地球表面上确定一个准确的位置,必须知道地球本身的形状和大小。因为地球不是一个完美的球体,它的自然表面是一个极其复杂又不规则的曲面,不可能用数学公式来表达,所以为了实用化,就经常用椭球体的形状来描述地球,如图-1所示。其中,长半轴 \( R \)(赤道半径)= 6378136.5m,短半轴 \( r \)(极半径)= 6356751.8m。

\begin{figure}[h]
    \centering
    \includegraphics[width=0.8\textwidth]{earth_ellipsoid.png}
    \caption{地球椭球体简化模型}
    \label{fig:earth_ellipsoid}
\end{figure}

为了对地球表面上每一个点的准确位置进行定位,引入经纬度坐标系统,又称为地理坐标系统。它是一种利用三度空间的球面来定义地球上的空间的球面坐标系统,能够标示地球上的任何一个位置。某地的纬度就是该地的法线与赤道平面之间的夹角。由赤道向南、北度量,向北量值称为北纬度,向南量值称为南纬度。由此可见,一地的纬度是该地对于赤道的方向和角距离。赤道是 \( 0^\circ \) 纬线,北纬度的最大值为 \( 90^\circ \),即北极点;南纬度的最大值为 \( 90^\circ \),即南极点。为了对坐标系统的轴进行定位,必须确定一个本初子午线,通过它可以指定经度值。大多数的地理基准面使用格林威治作为它们的本初子午线。经线则是包括本初子午线在内的地球表面连接南北两极的大圆线上的半圆弧。任两根经线的长度相等,相交于南北两极点。每一根经线都有其相对应的数值,称为经度。经线指示南北方向。

为了解答本题,首先要确定灰度矩阵中每个元素对应的采样点在地球上的经纬度。据此就可以将灰度矩阵中非负元素的行列号按上北下南、左西右东的地图规则换算成地球上经纬度坐标,此结果既可用于估算各探测点之间的距离,还可用于在云图上依据海岸线经纬度坐标标出海岸线以方便看图。

\subsection{3.2 经纬度坐标矩阵}

以赤道圆心为原点,赤道平面为 \( xoy \) 平面,极半径所在直线为 \( z \) 轴(指向北极)建立地心直角坐标系如图-2。则地球上任意一点的坐标可以用 \((x, y, z)\) 坐标系来表示。设同步卫星坐标为 \((0, d, 0)\),星下点在东经 \(86.5^\circ\),北纬 \(0^\circ\)。

\begin{figure}[h]
    \centering
    \includegraphics[width=0.8\textwidth]{image.png}
    \caption{地心直角坐标系}
    \label{fig:earth-coordinate}
\end{figure}

\textbf{地球椭球面方程}

\[
\frac{x^2}{R^2} + \frac{y^2}{R^2} + \frac{z^2}{r^2} = 1
\]

\textbf{步进角截面方程}

\[
\cos(\alpha i) \cdot z + \sin(\alpha i) \cdot (y - d) = 0
\]

\textbf{行扫描角截面方程}

\[
\cos(\alpha j) \cdot x + \sin(\alpha j) \cdot (y - d) = 0
\]

其中,\(R\) 为长轴半径(赤道半径),\(r\) 为短轴半径(极半径),\(d\) 为同步卫星离地球中心的高度,\(i\) 为偏移步进角的个数,\(j\) 为偏移行扫描角的个数 \((1 \leq i \leq 2288; 1 \leq j \leq 2288)\),\(\alpha = \left(140 * 10^{-6} \cdot \frac{180}{\pi}\right)\) 度。

带入数据,三式联立即可以确定椭球面与两截面的交点 \(A\) 的坐标 \((x, y, z)\)。继而,\(A\) 点在 \(xoy\) 平面上的投影 \(B\) 点坐标 \((x, y, 0)\)。则

\[
\tan \phi = \frac{x}{y}
\]

\[
\tan \theta = \frac{AB}{OB} = \frac{z}{\sqrt{x^2 + y^2}}
\]

可以得到经度表达式
\[
\psi = 86.5^\circ + \tan^{-1}\left(\frac{x}{y}\right) \text{(自西向东扫描)}
\]
或
\[
\psi = 86.5^\circ - \tan^{-1}\left(\frac{x}{y}\right) \text{(自东向西扫描)}
\]
纬度表达式
\[
\theta = \tan^{-1}\left(\frac{z}{\sqrt{x^2 + y^2}}\right) \text{(}\theta > 0 \text{北纬,} \theta < 0 \text{南纬)}
\]

由于 \(\psi\)、\(\theta\) 都是关于 \(i\)、\(j\) 的函数,此即灰度矩阵元素行列号 \((i, j)\) 对应于经纬度坐标的换算公式。由此可以得到以 \((\psi_{(i, j)}, \theta_{(i, j)})\) 为元素的 \(2288 \times 2288\) 的经纬度坐标矩阵 \(\mathbf{A}\)。

\[
\mathbf{A} = \begin{bmatrix}
(\psi_{(1, 1)}, \theta_{(1, 1)}) & (\psi_{(1, 2)}, \theta_{(1, 2)}) & \cdots & (\psi_{(i, j)}, \theta_{(i, j)}) \\
(\psi_{(2, 1)}, \theta_{(2, 1)}) & (\psi_{(2, 2)}, \theta_{(2, 2)}) & \cdots & (\psi_{(i, j)}, \theta_{(i, j)}) \\
\vdots & & & \vdots \\
(\psi_{(2288, 1)}, \theta_{(2288, 1)}) & \cdots & & (\psi_{(2288, 2288)}, \theta_{(2288, 2288)})
\end{bmatrix}.
\]

将经纬度坐标矩阵 \(\mathbf{A}\) 按元素位置等分为四部分,由此可以得到任意一点确切的经纬度,如图-3 所示。

\[
\mathbf{A} =
\begin{bmatrix}
\begin{array}{c|c}
\text{东经 } 86.5^\circ - \Phi, \text{ 北纬 } \theta \\ (i \leq 1144, j \leq 1144) & \text{东经 } 86.5^\circ + \Phi, \text{ 北纬 } \theta \\ (i \geq 1144, j \leq 1144)
\end{array} \\
\hline
\begin{array}{c|c}
\text{东经 } 86.5^\circ - \Phi, \text{ 南纬 } \theta \\ (i \leq 1144, j \geq 1144) & \text{东经 } 86.5^\circ + \Phi, \text{ 南纬 } \theta \\ (i \geq 1144, j \geq 1144)
\end{array}
\end{bmatrix}
\]

\begin{figure}[h]
\centering
\includegraphics[width=0.8\textwidth]{image.png}
\caption{经纬度矩阵坐标示意}
\end{figure}

\subsection{3.3 经纬度坐标文本文件}

取灰度矩阵中的第 451 行至 550 行,第 451 列至 550 列建立矩阵 (100×100) 形式的经纬度坐标文本文件,文本文件取名为 jwd.txt,见附录一;表-1 给出灰度矩阵第 500 行和第 500、501、502 列三个交叉点处对应的经纬度。

\begin{table}[h]
\centering
\caption{第 500 行和第 500、501、502 列三个交叉点处对应的经纬度}
\begin{tabular}{c c c}
\hline
交叉点 & 经度 & 纬度 \\
\hline
(500,500) & 东经 46.4851427976136 & 北纬 32.7404592934558 \\
(500,501) & 东经 46.5695660960079 & 北纬 32.7353074185807 \\
(500,502) & 东经 46.6538174766810 & 北纬 32.7301748212274 \\
\hline
\end{tabular}
\end{table}

\subsection{3.4 卫星云图}

卫星云图 (satellite cloud imagery) 由气象卫星自上而下观测到的地球上的云层覆盖和地表面特征的图像。利用卫星云图可以识别不同的天气系统,确定它们的位置,估计其强度和发展趋势,为天气分析和天气预报提供依据。在海洋、沙漠、高原等缺少气象观测台站的地区,卫星云图所提供的资料,弥补了常规探测资料的不足,对提高预报准确率起了重要作用。

红外线卫星云图利用卫星上之红外线仪器,来测量云层之温度。其中,温度低的云层会以亮白色来显示,也就是此处的云层较高,而暗灰色的部分则代表云层高度较低,因为越接近地面的云层温度越高。简单而言,即以云顶的不同温度来判断云层的高度。

红外云图,是气象卫星上的扫描辐射计利用红外辐射通道感测并向地面站发送的云图,其亮度大致反映了云层顶的温度,因而也反映了云顶的高度。一般温度越低,高度越高的云层,图上的色调过白,反之色调越黑。由于红外遥感可以昼夜感测并向地面站发送云图,并可分析高云和云顶温度,提供了可见光云图不能提供的大量信息,但红外云图的分辨率低于可见光云图。实际上要把两者结合起来使用,互相取长补短,从而获得了广泛的应用。[2]

首先将题设附件中卫星灰度矩阵 IR1_2100.mat 用 MATLAB 函数
\[
\text{imshow(IR1_2100.mat, [])}
\]
转化为卫星云图,如图-4 所示。

\begin{figure}[h]
    \centering
    \includegraphics[width=\textwidth]{image.png}
    \caption{某天 21:00 时刻的卫星云图}
    \label{fig:4}
\end{figure}

\section{定义}

\begin{align}
Vel &= \sqrt{u^2 + v^2} \\
Ang &= \tan^{-1}\left(\frac{u}{v}\right) + \pi
\end{align}

其中,$\psi$ 表示经度,$\theta$ 表示纬度。

则

\begin{equation}
x = \sqrt{\frac{1 - \frac{c^2}{a^2 R^2}}{\frac{a^2 + b^2}{R^2 a^2}}}, \quad y = \sqrt{\frac{1 - \frac{c^2}{b^2 r^2}}{\frac{a^2 + b^2}{R^2 a^2}}}, \quad z = \sqrt{\frac{1 - \frac{1}{r^2}}{\frac{a^2 + b^2}{R^2 c^2}}}
\end{equation}

\begin{equation}
i = \frac{\tan^{-1}\left(\frac{z}{d-y}\right)}{\alpha}, \quad j = \frac{\tan^{-1}\left(\frac{x}{d-y}\right)}{\alpha}
\end{equation}

再将题设附件中的海岸线经纬度坐标文件 `coastline0.txt` 中的矩阵 $G_{(24755 \times 2)}$ 按上式进行如下转化:
\[
G_{(24755 \times 2)}(\psi, \theta) \to H_{(24755 \times 3)}(x, y, z) \to J_{(24755 \times 2)}(i, j)
\]

其中, $H_{(24755 \times 3)}$ 是直角坐标矩阵, $J_{(24755 \times 2)}$ 是行列矩阵。

将行列矩阵 $J_{(24755 \times 2)}$ 赋值为 1, 在灰度矩阵形成的卫星云图上将这些点 ($(i, j)$ 表示灰度矩阵第 $i$ 行, 第 $j$ 列的元素) 特别表示出来, 即完成海岸线的添加, 取名为 `spic.pdf` (附件), 如图-5 所示。

\begin{figure}[h]
    \centering
    \includegraphics[width=\textwidth]{image.png}
    \caption{添加海岸线后的卫星云图}
    \label{fig:5}
\end{figure}

\section{四、问题二模型的建立与求解}

\subsection{4.1 云导风方法}

从云的运动推导风的大小和方向,主要是对连续几幅图像上的云进行跟踪,然后计算在这几幅图观测时间内,云移动的距离和方向。根据测量云运动得到的风的定量估计,目前已成为一种重要的卫星产品,称为云迹风或云风矢。它对全球天气和台风分析研究和提供数值预报初始风场资料都是十分重要的。目前,云迹风资料开始应用于台风、暴雨等天气分析中,国内学者也明确指出,云迹风能清楚生动地显示天气系统发展变化的细节,在数值天气分析和预报中具有广泛应用前景,对暴雨落区分析及预测、台风的影响范围以及移向预报等方面有重要的指示意义。气象部门已经有一些方法根据变化的卫星云图计算云迹风,这类方法称为云导风方法。计算云迹风时通常将云块大小限定为 $16 \times 16$ 个像素,搜索范围限定为 $64 \times 64$ 个像素。

在静止卫星云图上,并非所有的云都能作为示踪云。有些云的移动与风之间有较好的关系,而有些云的移动与风并无关系。最好的示踪云是随风而被动运动的云。当云运动受动力学而不是平流输送控制时,云的任何发展和衰亡都将给云迹风场的反演带来误差。因此,在理论上,除随风而被动运动的云外,任何云都不应被选作示踪云。

云导风就是基于多幅相关云图,从云系反演出风场变化的矢量图。它是一种基于灰度的运动检测方法,能反映出台风云系微元的运动趋势。目前国内云导风资料的使用主要集中在数值预报的研究中。

\subsection{4.2 云图的模板匹配方法}

云的运动是一种半流体运动,特点复杂,因此云图的匹配是非精确匹配问题,诸如二维对数搜索算法、三步法和金字塔搜索算法都不宜使用。在图像序列分析中,一类很重要的方法是匹配法。匹配的对象包括特征点、线段和小区域。在不同的应用中,应根据具体情况选用不同的特征来进行匹配。下面介绍的是应用比较广泛的 3 种匹配方法,共同点是允许误差存在,通过设置阈值来检测匹配是否成功。

\subsubsection{4.2.1 绝对值法}

\textbf{定义}

\begin{equation}
D(i, j) = \sum_{m=1}^{M} \sum_{n=1}^{M} \left[ S^{i, j}(m, n) - T(m, n) \right]^2
\end{equation}

式中 $S^{i, j}(m, n)$ 是被探测的目标区,$T(m, n)$ 是样板,$D(i, j)$ 为目标区上每点与样板上对应点的累计差值测度。若 $D(i, j) < L_D(i, j)$(阈值),则认为在坐标 $(i, j)$ 上样板匹配成立。

\subsubsection{4.2.2 序列相似性检测算法(SSDA 算法)}

(1) 定义绝对误差值

\[
\varepsilon(i, j, m_k, n_k) = \left| S^{i, j}(m_k, n_k) - S(i, j) - T(m_k, n_k) + T \right|
\]

(2) 取阈值 $T_k$。在子图 $S^{i, j}(m, n)$ 中随机选取像点,计算它同 T 中对应点的误差值 $\varepsilon$,将所有点对的差值累加起来。若累积误差超过阈值 $T_k$,停止累加,并记下累加次数 r。定义 SSDA 的检测曲面为

\[
I(i, j) = \left\{ r \middle| \min_{1 \leq r \leq m^2} \left[ \sum_{k=1}^r \varepsilon(i, j, m_k, n_k) \geq T_k \right] \right\}
\]

(3) 把 $I(i, j)$ 值大的 $(i, j)$ 点作为匹配点,因为此点需要很多次累加才使总误差 $\sum \varepsilon$ 超过 $T_k$,它最可能是一匹配的像素点。

\subsubsection{4.2.3 红外亮温交叉相关系数法}

文中交叉相关系数定义为:

\[
R(i, j) = \frac{\sum_{m=1}^M \sum_{n=1}^M \left[ S^{i, j}(m, n) - S^{i, j} \right] \times \left[ T(m, n) - T \right]}{\sqrt{\sum_{m=1}^M \sum_{n=1}^M (T(m, n) - T)^2 \times \sum_{m=1}^M \sum_{n=1}^M \left( S^{i, j}(m, n) - S^{i, j} \right)^2}}
\]

式中加入了样板区的均值和个搜索区的均值,若 $R(i, j) > R_{lim}$(阈值),则该区域为最佳匹配区域。定义交叉相关系数有不同的形式,中文方法是是其中的一种,但实质是相同的。

\subsubsection{4.2.4 匹配过程}

对卫星云图上的云进行连续的跟踪,通过计算相邻时段(通常间隔 30~60 min)局部云图的变化情况得到该区域云的运动矢量,对整块云团的各个局部推导出运动矢量就形成了云团的风场图。图像资料的分辨率高低是衡量卫星云图质量高低的一个重要指标。按照卫星上电视照相机的精度和扫描辐射仪观测的瞬时视场大小,分成低分辨率和高分辨率云图两种。分辨率高,云中细小结构能表示出来。在卫星与地球中心的连线和地球表面的交点(称为星下点)处的分辨率大约是 5 公里。

根据三个相邻时刻的卫星数据(20:30, 21:00, 21:30),在第二幅云图(21:00)中选取各个(整数)经纬度的网格点为目标中心点。以此中心点为中心取 $16 \times 16$ 像素元块。再以此中心点为目标在第三幅图像上,取 $64 \times 64$ 像素元的图像块为追踪区。将第二幅图像上的目标(模板)在第三幅图像上的追踪区中进行逐一扫描,如

图-6所示。扫描的方式为每次将模板平移1个像素位置与搜索区相关匹配,如图-7所示。并计算匹配的相关系数, 这样共要计算 $(64-16) \times (64-16)$ 个相关系数, 选择使相关系数 $R(i, j)$ 最大的区域为初始和终了位置, 算出前后两个风矢量。第一副图 (20:30) 用于校验结果的准确性。针对上文中涉及的匹配方法相关理论知识, 在后面风矢计算时将根据具体情况作进一步的阐述。

\begin{figure}[h]
    \centering
    \includegraphics[width=0.9\textwidth]{image1.png}
    \caption{模板匹配示意图}
    \label{fig:template_matching}
\end{figure}

\begin{figure}[h]
    \centering
    \includegraphics[width=0.9\textwidth]{image2.png}
    \caption{风矢量扫描方式}
    \label{fig:vector_scanning}
\end{figure}

\subsection{4.3 风矢的计算}

云导风的计算涉及到运动视觉方面的知识。对于运动视觉中的光流法, 需要满足在不同时刻灰度的一致性及在不同帧间运动的平滑性, 而云图的下面一些特点, 使其无法满足要求, 需选用其它的方法。

1. 成像时受到大气的影响, 并且图像有一定程度的畸变。

2. 序列云图的时间间隔较长。(半小时)

3. 云图中云具有非刚体性, 且其运动情况较复杂, 不仅有平移和旋转运动, 并且随时有局部云的产生和消亡。

国内外云导风的求取通常采用模板匹配法。模板匹配法的原理已在本章第二节做了介绍。用于云导风计算时,主要包括以下的三个步骤:(通常选用三张连续云图进行云导风的反演)

1. 取合适的示踪云目标,并通过红外和水汽云图指定示踪云的高度。

2. 在相邻的云图中,运用相关系数法为每一目标选取最佳匹配结果,作为该目标运动后所在位置,转化为云导风矢量进行存储。

3. 最后,通过自动的质量控制算法对云导风矢量进行检验及调整。在此过程中,需要考察的是云导风矢量是否满足时间连续性,即不同时次的云导风矢量之间的差别是否小于一定的阈值。对不满足条件的矢量剔除。其次是空间连续性,即某一点的云导风矢量与周围的矢量差是否在允许的范围内。如空间连续性不满足,则调整到其它的高度,仍无效则剔除。

在具体计算云导风矢量时,需要考虑的问题有匹配模板大小的选取、云系高度的划分以及匹配算法的选取。这些问题将直接影响结果的准确性和实效性。

由于本课题的研究对象对时间连续性要求不高,因此仅选用了两张云图进行云导风的反演。下面就云导风需要考虑的几个问题分别进行讨论。

\subsubsection{4.3.1 模板大小的选取}

匹配模板越大,需计算的像素点数越多,计算速度越慢,不利于云图实时处理分析;模板太小,则匹配面出现多峰值的可能性增大,匹配精度不够,计算出的矢量误差大。在计算 \((x, y)\) 点的云导风情况时,选取以 \((x, y)\) 为中心 \(16 \times 16\) 模板。相对模板中心,在 \(64 \times 64\) 方形区域内搜索最佳的匹配向量(其中 \((u, v)\) 为相对模板中心移动的距离)。星下点处的分辨率大约 5 公里,每像素大约十几公里。经过示踪云目标追踪,如果找到了目标模块所对应的匹配模块,则两个模块中心间的位移就提供了云的水平运动矢量。

\subsubsection{4.3.2 云系高度的划分}

通常将卫星云图中的全部云系划分为对流层上部(200hpa 左右)、中部(500hpa)、下部(850hpa)三个高度层次。台风云系是图像中灰度值较高的特定云团,它们对应着云图中高度较高的云系。在本文计算云导风矢量时对灰度值较小的云系不予考虑,因此无须对全图中的云系进行高度分类。

\subsubsection{4.3.3 计算过程}

采用绝对值匹配法计算量小,速度快,但是误差也较大,风场一致性也较差。采用 SSDA 和红外最大相关法进行比较,当不变阈值 \(T_k\) 小于 500 时,运算速度较快,但所得风场一致性不如红外最大相关法;当不变阈值 \(T_k\) 大于 1000 时,运算速度明显减慢,风场一致性与红外最大相关法仍略有差距。如图-8 累计误差图所示,选取适当的阈值对系统的运算速度有着非常大的影响。

基于以上结论,本系统采用 SSDA 算法将不变阈值 \(T_k\) 控制在 500,选取 \(16 \times 16\) 的目标模块进行粗匹配;再采取红外亮温相关系数法进行进一步计算。

\begin{figure}[h]
    \centering
    \includegraphics[width=\textwidth]{image.png}
    \caption{SSDA 累积误差增长曲线}
    \label{fig:8}
\end{figure}

假设 21:00 时刻目标模块的参考位置是 \((x_0, y_0)\),匹配模块的参考位置是 \(\left(x_{n_{j-1}}, y_{m_{j-1}}\right)\),云图间间隔时间为 \(dt\),像素点的空间分辨率是 \(res\),令 \(dt = x_{n_{j-1}} - x_0\),\(dy = y_{m_{j-1}} - y_0\),则风矢分量 \((u, v)\) 表示如下:
\begin{equation}
\begin{cases}
u = \frac{res \cdot dx}{dt} \\
v = \frac{res \cdot dy}{dt}
\end{cases}
\end{equation}

可以得到风矢大小 \(Vel\),风向 \(Ang\),表示如下:
\begin{align}
Vel &= \sqrt{u^2 + v^2} \\
Ang &= \tan^{-1}\left(\frac{u}{v}\right) + \pi
\end{align}

由于本系统主要针对低纬度热带气旋,地心角对水平速度的影响带来的偏差较小,因此,近似认为,\(Vel\) 和 \(Ang\) 即为该点风速和风向,如图-9 所示。

\begin{figure}[h]
    \centering
    \includegraphics[width=0.8\textwidth]{image.png}
    \caption{风矢量计算示意图}
    \label{fig:wind_vector}
\end{figure}

\subsubsection{4.3.4 风场随时间变化的质量控制}

云迹风风矢的计算是以图像模块(几十至上百公里尺度)识别为基础的。由于云区的时空变化以及图像处理、计算、识别等各阶段可能存在的错误或限制,使个别风矢不合理,使云迹风风矢群体呈现一定程度的紊乱。利用客观规律或主观判断来调整或删除认为不合理的风矢,最终生成高质量的云迹风风场产品,就是质量控制或编辑。其产生的原因可能是:

\begin{itemize}
    \item[a)] 在风速较大的区域,计算水平速度或垂直速度分量时产生的误差相对较大;
    \item[b)] 随机噪声干扰了频谱分析,使计算谐波速度时产生偏差。但是,取连续 3 幅静止云图求前后两个风矢量的目的是为了检验相邻时刻的风矢量方向有无突变。因此,质量控制也不是特别必要的。
\end{itemize}

\subsection{4.4 结果}

用上文所示的方法,计算 21:00 时刻在各个(整数)经纬度处的二维风矢,计算范围为:南纬 40 度至北纬 40 度,东经 46 度至 126 度。云导风图如图-10 所示:

\begin{figure}[h]
    \centering
    \includegraphics[width=\textwidth]{image.png}
    \caption{云导风图}
    \label{fig:cloud_wind}
\end{figure}

\begin{figure}[h]
    \centering
    \includegraphics[width=\textwidth]{image.png}
    \caption{21:00时刻云导风图}
    \label{fig:10}
\end{figure}

其中,非零风矢的个数为5158个。粉色风矢风速为风速每秒0至20米,绿色风矢为风速20以上至40米,蓝色风矢为风速40米以上。下同。

纬度26度,经度52~58度处的风矢大小和方向如下表所示。

\begin{tabular}{c c c c}
\hline
方法 & 随机扰动值范围 & 风速均方根误差 & 风向均方根误差 \\
 & & (m/s) & (°) \\
\hline
方法1 & -5 ~ 5 & 0.01637 & 5.07380 \\
 & -15 ~ 15 & 0.03803 & 5.27343 \\
方法2 & -5 ~ 5 & 0.00921 & 2.78032 \\
 & -15 ~ 15 & 0.01625 & 3.02057 \\
\hline
\end{tabular}

\subsection{4.5 分析和评价}

(1) 尽管绝对值匹配法计算量小,速度快,但得出的矢量组杂乱无章,风场连续性效果最差,不宜实际应用。这是由于该方法本身原理过于粗糙所致。

(2) SSDA 匹配法计算出的云导风风场连续性效果有明显改善,并且计算速度快,但匹配效果仍差于红外亮温交叉相关匹配法。

(3) 红外亮温交叉相关匹配法算出的云导风连续性效果在 3 种方法中最好,运算速度比 SSDA 法稍慢,但此速度对于云图的实时分析已足够,故被本文采用。

\section{五、问题三的模型的建立与求解}

目前,云导风计算过程中示踪云的追踪用得最有成效的是相关匹配法,并在利用卫星云图反演云导风方面取得了显著的成果,该方法在第四章已做了详细介绍。为了提高像素块匹配质量,取消问题二中窗口大小和搜索范围的限定,利用图像特征匹配等思想,设计能自适应地确定窗口大小和搜索范围的有效方法。进而建立风矢场的度量模型和算法。

\subsection{5.1 云图灰度梯度计算方法}

所谓相关匹配法主要利用了图像间的灰度相似性特征,而没有考虑到灰度梯度特征。而卫星云图中的灰度是通过星载传感器观测转化来得到的,从而存在着观测误差,如果对灰度直接采用数值差分得到灰度梯度特征,加入到相关匹配法中去,会引起较大的风速误差。为此借助于近几年反问题中发展起来的数值微分方法,即离散观测数据求微商,利用一维数值微分方法计算云图各点在经度和纬度方向上灰度的导数(从而得到灰度的梯度),然后利用正则化方法将云图灰度的梯度信息加入相关匹配的要素中去,最后采用仿真和实际试验两种方法对云导风进行反演,数值结果表明这种新的相关匹配方法,即利用了灰度的信息又吸收了图像灰度梯度信息,在提高反演精度的同时,又提高了反演的稳定性,为云导风产品精度的提高及进一步应用于预报走出一条新路。

灰度梯度信息会直接影响到图像的相似性。众所周知灰度的梯度计算(亦即观测资料求导数)在数学上是一个不适定的问题。此问题,通过离散点上观测值,要寻找一个函数,使得它与观测点上观测值充分接近,同时其导数与真实函数的导数要有足够的精度。显然,当离散数据不存在数据误差时,这一问题一般用有限差分法就可求得近似导数,并且其收敛性亦得到保证,但当离散数据存在误差时,用有限差分法求近似导数会产生很大误差。本文采用了数值微分方法,该方法所得到的解是一个分片三次样条函数,此解的导数与真实解的导数具有足够高的精度,实施方案如下:

设 $y = f(x)$ 是定义在 $[0, 1]$ 上的一个函数,记 $\Delta = \{0 = x_0 < x_1 < \cdots < x_n = 1\}$ 是 $[0, 1]$ 上的一个划分,$h_j = x_{j+1} - x_j$, $h = \max(h_j)$, $(j = 1, 2, \ldots, n-1)$,若给定函数在点 $x_j$ 处的观测值 $\widetilde{y}_j$,用 $\delta$ 表示观测数据的观测误差,则

\[
|\widetilde{y}_j - y(x_j)| \leq \delta, \, j = 1, 2, \ldots, n.
\]

数值微分方法是寻找函数 $f''(x)$,使得 $f''(x)$ 的导数与真是函数 $y(x)$ 的导数

充分接近,于是构造泛函数 $\mathbf{J}$ 如下:
\[
J(f) = \sum_{j=1}^{n-1} \frac{h_j + h_{j+1}}{2} \big( \widetilde{y}_i - y(x_j) \big)^2 + \alpha \left| \frac{d^2 f}{dx^2} \right|_{L^2(0,1)}^2 = min!
\]

这里 $\alpha$ 是正则化参数。则重构图像的函数 $f''$ 是一个分段的三次样条函数且一阶微分是连续的,在子区间 $[x_j, \, x_{j+1}]$ 上其表达式为

\[
f''(x) = a_j + b_j (x - x_j) + c_j (x - x_j)^2 + d_j (x - x_j)^3, \, j = 0, \, 1, \, \dots, \, n-1
\]

其一阶导数为

\[
f'''(x) = b_j + 2c_j (x - x_j) + 3d_j (x - x_j)^2, x \in [x_j, x_{j+1}]
\]

根据理论分析,此时正则化参数 $\alpha$ 的选择原则应为 $\alpha = \delta^2$. 对于数字化的静止气象卫星云图,每一个像素代表了约 $20 \, \mathrm{km}^2$ 范围内的平均光谱信息,图像灰度值范围为 $0$—$1023$。

\subsection{5.2 云导风反演的新方法实施}

将图像灰度的梯度信息加入相关系数的计算中,三种信息单独计算出的交叉相关系数分别记为 $R_1$, $R_2$, $R_3$, $R_1$ 由第四章的方法求出,$R_2$, $R_3$ 分别由以下两式求出。

\[
R_2(m, n) = \frac{\sum_i \sum_j \big[ f_x^*(i, \, j) - \overline{f_x^*} \big] \big[ g_x^*(i+m, j+n) - \overline{g_x^*} \big]}{\left[ \sum_i \sum_j \big[ f_x^*(i, \, j) - \overline{f_x^*} \big]^2 \sum_i \sum_j \big[ g_x^*(i+m, j+n) - \overline{g_x^*} \big]^2 \right]^{1/2}}
\]

\[
R_3(m, n) = \frac{\sum_i \sum_j \big[ f_x^*(i, \, j) - \overline{f_x^*} \big] \big[ g_x^*(i+m, j+n) - \overline{g_x^*} \big]}{\left[ \sum_i \sum_j \big[ f_x^*(i, \, j) - \overline{f_x^*} \big]^2 \sum_i \sum_j \big[ g_x^*(i+m, j+n) - \overline{g_x^*} \big]^2 \right]^{1/2}}
\]

其中,$f_x^*(i, \, j)$ 为示踪云模板在经度方向的梯度,$f_y^*(i, \, j)$ 为示踪云模板在纬度方向的梯度,$g_y^*(i, \, j)$ 为目标搜索区在纬度方向的梯度,$\overline{f_x}$, $\overline{g_x}$, $\overline{f_y}$, $\overline{g_y}$ 表示梯度平均值。

构造函数如下:

\[
J[m, \, n] = R_1(m, \, n) + \gamma \big[ R_2(m, \, n) + R_3(m, \, n) \big] = max!
\]

其中$\gamma$表示最大交叉相关系数法中正则化参数,取$\gamma=\delta^{2}$,$\delta$为观测误差值,$\delta=\frac{\delta^{*}}{1024}$,$\delta^{*}$为观测灰度误差,使得上式最大所对应的区域即为目标云区域位置。

\subsection{5.3 云导风仿真试验}

为了进一步提高云导风矢量的精度和计算速度,首先,对卫星云图进行预处理,以此来简化无用的计算量,得到新的卫星云图如图-11所示。

\begin{figure}[h]
    \centering
    \includegraphics[width=\textwidth]{image.png}
    \caption{预处理后得到的卫星云图}
    \label{fig:11}
\end{figure}

为了讨论方便,记方法 1 为只用灰度进行计算的方法,记方法 2 为加入灰度梯度信息的新方法。由于云导风矢量与实际风矢量在高度和速度上具有一定的差异,故能准确代表云导风矢量真值的观测风资料不易获取。由于真实风场无法取得,就无法判别两种方法的优劣,于是我们实施了如下的仿真试验,如图-12所示。仿真数据选取计算范围为:南纬 40 度至北纬 40 度,东经 46 度至 126 度。假定在一个小时内每一点上在经度和纬度方向上运动速度均为 $8m/s$,可以由此

构造出 21:00 和 21:30 时刻的云图,分别加入 5 和 15 以内(观测灰度误差 \(\delta^*\) ) 的随机扰动,分别以方法 1 和方法 2 计算云导风,并将加入与未加入随机扰动的结果进行风速和风向均方根误差统计。

\begin{figure}[h]
\centering
\includegraphics[width=\textwidth]{image.png}
\caption{云导风仿真试验流程图}
\end{figure}

图中
\begin{equation}
\sigma_v = \sqrt{\frac{\sum_{i=0}^n (v - v_1)^2}{n-1}}
\end{equation}
\begin{equation}
\sigma_\theta = \sqrt{\frac{\sum_{i=0}^n (\theta - \theta_1)^2}{n-1}}
\end{equation}

其中 n 代表位置相同的风矢量个数,𝑣,𝑣1代表真值计算出的风速,𝜃,𝜃1代表真值和计算出的风向。

\subsection{5.4 分析、比较与结论}

\begin{table}[h]
\centering
\begin{tabular}{l l}
学校 & 华东交通大学 \\
\hline
参赛队号 & D10404005 \\
\hline
队员姓名 & \begin{tabular}{l l}
1. & 龚正 \\
2. & 宋志成 \\
3. & 杨扬 \\
\end{tabular} \\
\hline
\end{tabular}
\end{table}

\begin{table}[h]
\centering
\caption{第 500 行和第 500、501、502 列三个交叉点处对应的经纬度}
\begin{tabular}{c c c}
\hline
交叉点 & 经度 & 纬度 \\
\hline
(500,500) & 东经 46.4851427976136 & 北纬 32.7404592934558 \\
(500,501) & 东经 46.5695660960079 & 北纬 32.7353074185807 \\
(500,502) & 东经 46.6538174766810 & 北纬 32.7301748212274 \\
\hline
\end{tabular}
\end{table}

\begin{tabular}{c c c c}
\hline
方法 & 随机扰动值范围 & 风速均方根误差 & 风向均方根误差 \\
 & & (m/s) & (°) \\
\hline
方法1 & -5 ~ 5 & 0.01637 & 5.07380 \\
 & -15 ~ 15 & 0.03803 & 5.27343 \\
方法2 & -5 ~ 5 & 0.00921 & 2.78032 \\
 & -15 ~ 15 & 0.01625 & 3.02057 \\
\hline
\end{tabular}

(1) 云图灰度的扰动会引起风矢量计算的误差,且灰度扰动越大风矢量误差也越大;

(2) 加入图像梯度信息后,可有效提高风矢量的反演精度;

(3) 加入图像梯度信息后,可有效抑制灰度扰动的影响,减小灰度扰动引起的风矢量误差。

\begin{figure}[h]
    \centering
    \includegraphics[width=\textwidth]{image1.png}
    \caption{云图灰度梯度计算方法下生成的新风矢云图}
    \label{fig:13}
\end{figure}

本章提出的新方法在总结现有云导风反演方法的基础上,将数值微分方法计算出的图像灰度梯度信息加入到云导风追踪运算中,利用正则化构建了云导风反演新方法。经过对比两种方法计算出的风场结果,加入图像梯度后能够有效降低观测干扰引起的误差,较大地提高了反演的精度,研究表明图像灰度梯度能够有效改善云导风反演的精度。

此方法生成的新风矢云图如图\ref{fig:13}所示。

我们认为这一研究将进一步联合多星源卫星观测资料建立云高计算模型,在风场高度指定的准确性上进行深入研究,并对云导风算法进行优化和长时间试验,力争应用到云导风业务系统中。

\section{六、问题四模型的建立与求解}

等压面是一个气象学词汇,指的是气压相同的面。气压因为海拔高度和地面冷热的不同导致气压的不同,一般来说,在垂直方向上随着高度的升高,等压面的数值在变小。如果地面的性质是均一的,大气的性质也是均一的,那等压面就是互相平行的,否则就是高低起伏。

等压面是空间气压相等的各点所组成的面。由于同一高度,各地气压不相等,等压面在空间不是平面,而是像地形一样起伏不平。采用绘制地形图的方法,用等高线将起伏不平的等压面投影到平面图上,构成等压面图。在等压面图上,用等高线表示气压的空间分布。等压面凸起的地方,表示在同一高度上,气压比四周高。在等压面图上,是一组闭合等高线构成的高值区,即高压区。等压面凹下,表示同一高度上,气压比四周低。在等压面图上,是一组闭合等高线构成的低值区,即低压区。因此,在等压面图上,由等高线的分布可反映等压面的起伏,表示气压的空间分布状况。等压面的高度不是几何高度,而是位势高度。位势高度是以单位质量物体从海平面抬升到空间某高度位置上克服重力所作的功来表示。气象上通常用的等压面图有 850 百帕、700 百帕、500 百帕等多种。不同高度的等压面图组合起来,不仅反映大范围地区高压、低压等天气系统的空间分布状况,而且能反映高空大气的结构及其演化规律,它是大范围天气形势预报的基础。

\subsection{6.1 探测温度和等压面压强}

假设标准大气温度廓线标明了探测温度 $t_1$、$t_2$ 所对应的等压面压强分别为 $p_1$、$p_2$,要求出标准大气温度廓线中未标明的目标模块探测温度 $t$ 所对应的压强 $P$。则可按下述公式进行外推计算:

首先,令对数线性内插(外推)公式为,参看图-14:
\[
T = a + b \ln P
\]

分别将 $(t_1, p_1)$ 和 $(t_2, p_2)$ 代入上式中,得到:
\[
\begin{cases}
t_1 = a + b \ln p_1 \\
t_2 = a + b \ln p_2
\end{cases}
\]

解得:
\[
\begin{cases}
a = \frac{t_1 \ln p_2 - t_2 \ln p_1}{\ln p_2 - \ln p_1} \\
b = \frac{t_2 - t_1}{\ln p_2 - \ln p_1}
\end{cases}
\]

那么,将 $a$、$b$ 代入
\[
t = a + b \ln p
\]

则最终得到:
\[
p = e^{\frac{t-a}{b}}
\]

\begin{figure}[h]
    \centering
    \includegraphics[width=0.8\textwidth]{image.png} % 替换为实际图像文件名
    \caption{探测温度与气压高度示意图}
    \label{fig:temperature_pressure}
\end{figure}

然而,在实际应用中,估计值的精度受很多因素影响:如云顶表面比辐射率;云顶之上大气对云的辐射影响;云层之下地表、大气及较低云层对其上云层的辐射影响等等。在目前,这些问题本身就难以解决;同时,即使有解决办法,实际操作也较为困难。并且,云迹风矢代表云图上指定模块的视运动,由于云具有一定的厚度,其移动受云体所在范围内各高度上气流的控制,所以,即使相同云顶高度的云迹风矢也不一定代表相同高度上的风向风速。

\subsection{6.2 计算非零风矢所在的等压面}

由于本文题设条件中给出了灰度、温度、压强之间的关系,所以计算过程大大简化。首先应用灰度值与温度值的关系文件 k\_temp.txt,将 21:00 时刻各个点的灰度值换算为该点的温度。再应用附件中全球温度数值预报文件 temp3.mat 计算问题 2 中非零风矢所在的等压面(用该时刻该处大气的压强值标志,物理单位为毫巴)。文件 temp3.mat 按三维数据(1:641,1:1280,1:36)存放,具体存放的数据按“等压面高度”从上到下分 36 层,每层大气压相同。同一层是个矩阵,有 641 行,每行纬度相同;有 1280 列,每列经度相同,相邻两行(列)相差 0.28125 度,排列顺序是从北到南,从东到西。矩阵元素是矩阵行列所对应的经纬度坐标点在指定等压面的温度值。只需找到该温度对应“等压面高度”中的第几层,即完成本题的解答。相应程序见附录。

将计算结果保存为电子文档 winsh.txt。纬度 26 度,经度 52~58 度处风矢量所在的等压面如表-6 所示

\begin{table}
\centering
\begin{tabular}{c c c c c}
南纬 & 东经 & 角度 & 速度 & 压强 \\
(度) & (度) & (北顺,度) & (米/秒) & (豪巴) \\
\hline
26 & 52 & 0.00 & 0.00 & 950 \\
26 & 53 & 338.20 & 8.77 & 800 \\
26 & 54 & 258.69 & 9.85 & 750 \\
26 & 57 & 0.00 & 7.06 & 600 \\
26 & 58 & 18.44 & 6.00 & 800 \\
\end{tabular}
\caption{二维风矢量等压面}
\end{table}

用该方法产生21:00时刻等压线图如图-15所示。

\begin{figure}[h]
\centering
\includegraphics[width=\textwidth]{image.png}
\caption{21:00时刻等压线图}
\label{fig:15}
\end{figure}

\section{七、模型评价、总结与展望}

文中首先通过换算视场坐标确定灰度矩阵中每个元素对应的采样点在地球上的经纬度。提出了一种模板匹配生成云导风矢量图的方法。云导风矢量通过对具有一定时间间隔的两幅相关卫星云图进行模板匹配而得出。模板匹配就是根据已知模式的图像在一幅陌生图像中寻找对应于该模式的子图像的过程。然后,借助于近年来发展起来的数值微分方法,从图像灰度中提取出图像梯度信息;然后利用正则化方法,实现了云导风反演;对云图中有扰动时加入灰度梯度信息和未加入灰度梯度信息的风场反演结果进行比较。结果表明,加入图像灰度梯度信息所实施的新反演方法可有效减小图像干扰的影响,同时也大大提高了风矢量反演的精度,为卫星云图反演云导风探索出一条新路。

序列云图可以反映云系的动态信息,在台风的识别及定位中具有单幅云图无法比拟的优势。对已有的云导风算法进行简化和改进,以此作为台风定位的依据,随着实践的深入和进一步的研究,将会在台风的自动识别及定位系统中发挥重要的作用。

[REFERENCES:1]

% Missing placeholders restored
\includegraphics[width=0.3\textwidth]{image1}
\includegraphics[width=0.3\textwidth]{image2}
\includegraphics[width=0.3\textwidth]{image3}